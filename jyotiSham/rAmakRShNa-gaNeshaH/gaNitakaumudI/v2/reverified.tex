\documentclass[11pt, openany]{book}
\usepackage[text={4.65in,7.45in}, centering, includefoot]{geometry}
\usepackage[table, x11names]{xcolor}
\usepackage{fontspec,realscripts}
\usepackage{polyglossia}
\setdefaultlanguage{sanskrit}
\setotherlanguage{english}
\setmainfont[Scale=1]{Shobhika}
\newfontfamily\s[Script=Devanagari, Scale=0.9]{Shobhika}
\newfontfamily\regular{Linux Libertine O}
\newfontfamily\en[Language=English, Script=Latin]{Linux Libertine O}
\newfontfamily\ab[Script=Devanagari, Color=purple]{Shobhika-Bold}
\newfontfamily\qt[Script=Devanagari, Scale=1, Color=violet]{Shobhika-Regular}
\newcommand{\devanagarinumeral}[1]{%
	\devanagaridigits{\number \csname c@#1\endcsname}} % for devanagari page numbers
%\usepackage[Devanagari, Latin]{ucharclasses}
%\setTransitionTo{Devanagari}{\s}
%\setTransitionFrom{Devanagari}{\regular}
\XeTeXgenerateactualtext=1 % for searchable pdf
\usepackage{enumerate}
\pagestyle{plain}
\usepackage{fancyhdr}
\pagestyle{fancy}
\renewcommand{\headrulewidth}{0pt}
\usepackage{afterpage}
\usepackage{multirow}
\usepackage{multicol}
\usepackage{wrapfig}
\usepackage{vwcol}
\usepackage{microtype}
  \usepackage{amsmath,amsthm, amsfonts,amssymb}
\usepackage{mathtools}% <-- new package for rcases
\usepackage{graphicx}
\usepackage{longtable}
\usepackage{setspace}
\usepackage{footnote}
\usepackage{perpage}
\MakePerPage{footnote}
%\usepackage[para]{footmisc}
%\usepackage{dblfnote}
\usepackage{xspace}
\usepackage{array}
\usepackage{emptypage}
\usepackage{hyperref}% Package for hyperlinks
\hypersetup{colorlinks,
citecolor=black,
filecolor=black,
linkcolor=blue,
urlcolor=black}( १६४ )

विषमार्थे जात्ये । आभ्यां जातं विषमम् ।



सूत्रम् ।

१श्रुतिहतिपार्श्वभुजाहति- 

 बधतो लम्बद्वयाहतिविभक्तान् । 

मूलं द्विसमत्रिसमा- 

 ऽसमेषु दलितं भवेद् हृदयम् \textbar\textbar १३१ \textbar\textbar



\begincenter\rule\theta.5\linewidth\theta.5pt\endcenter

(१) अत्रोपपत्तिः । मुखवदने हित्वा शेषभुजौ पार्श्वभुजौ ज्ञेयौ ।
एककर्णैकपार्श्वभुजौ भुजौ चतुर्भुजस्य भूमिर्भूमिस्तत्र त्रिभुजे यो लम्बः
स चतुर्भुजस्यैको लम्बः। एवमन्यकर्णापरपार्श्वभुजौ भुजौ भूमिर्भूमिस्तत्र
त्रिभुजे यो लम्बः सोऽन्यलम्बः ।

अथ यदि चतुर्भुजं वृत्तान्तर्गतं तदा पूर्वोक्ते त्रिभुजे अपि तस्यैव
वृत्तस्यान्तर्गते अतः पूर्वोक्तप्रकारेण वृत्तस्य व्यासः= क१ भु१ /
ल१

एवं व्यासः= क२ भु२ / ल२

द्वयोर्घातः = व्या = क१क२ x भु१भु२/ ल१ ल२

हृदयं नाम चतुर्भुजोपरिगवृत्तस्य व्यासार्धम् । अत उपपन्नम् ।

( १६५ )

उदाहरणम् । 

 १तुल्याक्ष्मा खगुणर्तुभिश्च वदनं 

 खाक्षाक्षिभिश्चादिमो । 

 बाहुर्व्योमसुरैः परोऽङ्गखशरैः 

 ······················· 

 ·············जिनरसैः 

 श्रोत्रं तथैवादिमं 

 व्योमाक्ष्यक्षिमितं··········

 विद्वान्, द्रुतं हृद् वद \textbar\textbar ९३ \textbar\textbar

 अत्रैव श्रवणाधरोर्ध्वशकले 

 लम्बः फलं च श्रवः 

\begincenter\rule\theta.5\linewidth\theta.5pt\endcenter

(१ ) अत्र श्लोके त्रुटिः । नेपालप्राप्तपुस्तकेऽयं पाठ: सोऽपि न समीचीनः ।
तुल्याक्ष्माखगुणर्त्तुभिश्च वदनं खाक्षाक्षिभिश्चादिमो बाहुर्व्योमसुरै:
परोऽङ्गखशरैः पलंबावगौ वह्निभिर्बाणाभ्रब्धिमितः शराशततौ जिनरसैः
श्रोत्रं तथैवादिमं व्योमक्ष्यक्षिमितंविद्वन् द्रुतं दृद्वद ॥

न्यासभावार्थबोधकोऽयं श्लोको निवेशितुं शक्यते ।

तुल्याक्ष्मा खगुणणर्त्तभिश्च वदनं खाक्षाक्षिभिश्चादिमो

बाहुर्व्योमसुरै: परोऽङ्गखशरैः सिन्ध्वष्टबाणेन्दुभिः ।

लम्बोऽन्यो जिनखाक्षिभिः शरहतौ श्रोत्रं जिनैः सर्तुभि---

र्व्योमाक्ष्यक्षमितं परं यदुदितं विद्वन् द्रुतं दृद्वद ॥

( १६६ )

संयोगादधरोर्ध्वलम्बकमिती 

 हृच्चाऽन्यत्दृल्लम्बकान् । 

 इत्यादीन्यपि वक्ष्यमाणगणकैः 

 सूत्रैश्च यद् गण्यते 

 तद् बुद्ध्याऽऽशु तवाऽस्तिभूमिगणित- 

 क्षोदक्षमश्चेच्छ्रमः \textbar\textbar १३२ \textbar\textbar

न्यासः ।

कर्णौ ६२४।५२० लम्बौ १५८४/५ । २०२४/५ लब्धं हृदयम् ३२५

सूत्रम् ।

 १अबधाबधेन हीनो

 लम्बकवर्गोऽवलम्बकविभक्तः । 

\begincenter\rule\theta.5\linewidth\theta.5pt\endcenter

(९) अत्रोपपत्ति: । क्षेत्रदर्शनम् । अ क ग--
त्रिभुजोपरिगतवृत्तपरिधिस्थ-व-बिन्दुपर्यन्तं ग घ-लम्बं संवर्ध्य,
क-विन्दाेश्छ- केन्द्रग्रामिनीं रेखां परिधिस्थ-ज-बिन्दुपर्यन्तं
संवर्ध्य,अ ज-रेखा योज्यो ।

एवं च क अज-कोणः समकोणः स्यात् (रे ३।३१) । तथा च



( १६७ )

तत्कृति भूकृति योगाद् 

 मूलदलं जायते हृदयम् \textbar\textbar १३३ \textbar\textbar

\begincenter\rule\theta.5\linewidth\theta.5pt\endcenter

अक-भूम्युपर्युभयोः गच अज-रेखयोर्लम्बत्वेन समानान्तरत्वं संपन्नम् ( रे
२७-२९ प्र २) ।

अज=गज विधाय, अज, अच, गज, रेखाः कार्याः । तेन अच = गज ( रे ३।२६-२७
प्र१, २९ ) तथा गज = अज ( रे १।३३) अतः

अतः अच =अज ( स्व १) तेन \textless अ च ज =\textless अ ज च ( रे १।५)
तथा \textless अ ज च =\textless अ घ ज ( स्व ११ )

अंतः घ च = घ ज ( रे १।२६) ।

अथ, गज२ =  गघ घज + गघ गज = गघ घच + गघ गज( रे २।२)

∴ गघ .गज = गघ२ - गघ.घच= गघ२- अघ.कघ (रे ३।३५)

तेन, गज = गघ२ - गघ.घच/गघ = अज । कज२ = अक२ + अज२ 

=अक२ + (गघ - गघ.घच/गघ) ( रे १।४७)

( १६८ )

उदाहरणम् \textbar

 एको विश्वमितो दोस्तिथि- 

 सङ्ख्योऽन्यो मही च शक्रमिता । 

\begincenter\rule\theta.5\linewidth\theta.5pt\endcenter

∴ क छ = क ज/२ = \sqrt अक२ + (गघ३- गघ अच/ग घ /२) ।

इत्युपपन्नं यथोक्तम् ।

यद्वा 'त्रिभुजस्य भुजाभ्यासे लम्बविभक्ते प्रजायते व्यासः' ' त्रिभुजे
चतुर्भुजे वा व्यासदलं जायते हृदयम्' इति वचनात् त्रिभुजो
परिगतवृत्तव्यासदलं हृदयाख्यम् = भुभु/२ल । अत्र लम्बावाधावर्गयोगस्य
भुजसमत्वादंशवर्गः = भु२ भु२१ = ( ल२ + आ२ ) ( ल२ + आ२४
) = ल४ +ल२ ( आ२१ + आ२  ) + आ२ आ२४ = ल४ +
ल२  (आ२ + २ आआ१ + आ२१- २ आआ१ ) + आ२
आ२४ = ल४ + ल२ (आ + आ१)२

-- २ आ आ१ ल२ + आ२ आ२१ = ल२ भू२ + ल४ - २ आ
आ१ ल२ + आ२ आ२१ = ल२ भू२ + ( ल२ - आ आ१
)२  

\ = ल२ भू२ + ( ल२ - आ आ१/ल ) \ ।

( यतः आ + आ१ = भू) 

 अतोऽशमानम् = ल \sqrt भू२ + (ल२ -- आआ१/ल )२ ।

( १६९ )

द्वादशलम्बस्त्र्यस्रे 

 खाब्धिगुणे तत्र किं हृदयम् \textbar\textbar ९४
\textbar\textbar



न्यासः \textbar

जातं हृदयम् ३२५ \textbar

अथ फलानयने सूत्रम् \textbar

  १कर्णाश्रितभुजबधयुति-

 गुणिते तस्मिन् श्रवस्यऽपि विभक्तो \textbar 

\begincenter\rule\theta.5\linewidth\theta.5pt\endcenter

अतश्च हृदयमानम्

ल \sqrt भू२ + ( ल२ - आआ१/ल)/२ ल = \sqrt भू२ + ( ल२ - आआ१/ल)२/२
।

इत्युपपन्नम् ।

(१) अत्रोपपत्तिः । यदि चतुर्भुजस्य भुजाः भु१, भु२ ,भु३, भु४ ।
कर्णौ क१, क२ ।

संमुखकोणाभ्यां क१  कर्णोपरि लम्बौ ल१, ल२ हृदयरज्जुः = हृ
\textbar तदा पूर्वसूत्रेण

हृ = भु१, भु२ /२ल ∴ ल१ = भु१, भु२ /२हृ

( १७० )

चतुराहतहृदयेन- 

 द्विसमादिचतुर्भुजे गणितम् । 

उक्तचतुर्भुजस्य गणितार्थ न्यासः \textbar हृदयम् ३२५ गणितम् १६०५१२ ।



अथ त्रिभुजगणितानयने सूत्रम् ।

  १चतुराहतहृदयत्दृतं

 त्रिभुजभुजानां बधं गणितम् ॥ १३४ ॥

\begincenter\rule\theta.5\linewidth\theta.5pt\endcenter

तथा हृ = भु३.भु४/२ल२ ∴ ल२ = भु३.भु४/२ हृ

ल१ + ल२ = भु१.भु२ + भु३.भु४/२ हृ

क्षेत्रफलम् = ( ल१ + ल२ ) क१/२ = क१ ( भु१.भु२ +
भु३.भु४/४हृ ) 

अत उपपद्यते ।

एवं द्वितीयकर्णेन, क्षेत्रफलम् = क२ ( भु१.भु२ + भु३.भु४)/४
हृ  

(१) अत्रोपपत्तिः । पूर्वप्रकारेण

हृ = भु१ भु२ /२ ल = भु१.भु३ भू/२ल.भू = भु१.भु२ भू/ ४ त्रिभुज
फ

∴ त्रिभुज फ = भु१.भु२ भू/ ४हृ । अत उपपन्नम् ।

( १७१ )

उदाहरणम् ।

पूवोक्तत्र्यस्रस्य फलार्थं न्यासः । हृदयम् ३२५ जातं गणितम् १३४४००
\textbar





अथ चतुरस्रयोः कर्णहृदयसाम्ये सूत्रम् \textbar

 १द्विगुणितहृदयकृतेर्भू- 

 मुखभुजवर्गैः पृथग् विहीनायाः । 

\begincenter\rule\theta.5\linewidth\theta.5pt\endcenter

(१) अत्रोपपत्तिः । वृत्तकेन्द्रात् भूमुखभुजोपरि लम्बाः क्रमेण

\sqrt ४ हृ२ - भू२ /२, \sqrt ४ हृ२ - भु२/२,

\sqrt ४ हृ२ - भु२१/२, \sqrt ४ हृ - भु२२/२  

 

( १७२ )

मूलानि स्युर्भूमुख- 

 भुजाः श्रवोहृदयफलसाम्ये ॥ १३५ ॥

पूर्वोक्तचतुर्भुजस्य न्यासः । कर्णौ ५२०\textbar६२४ हृदयम् ३२५ गणितम्
१६०५१२ ।



जातमन्यच्चतुर्भुजम् । कर्णौ ५२०।६२४ हृदयम् ३२५, गणितम् १६०५१२ ।

\begincenter\rule\theta.5\linewidth\theta.5pt\endcenter

एते द्विगुणा अन्यचतुर्भुजस्य भूमुखभुजाः स्युर्यत्र तावेव कर्णौ तदेव
हृदयं च भवति । सर्वं क्षत्रेतः स्फुटम् ।

आ का गा घा-प्रथमं चतुर्भुजम् । यत्र आ का =मुखम् । का गा = भुजः = भु१ 
आ घा= भुजः = भु२ । गा घा= भूमिः= भू । 'द्विगुणितहृदयकृतेर्भू'
इत्यादिना द्वितीयचतुर्भुजे मुखम् = आ का१ । एको बाहुः= गा का ५ ।
द्वितीयो बाहुः = आ आ१ ।

भूमिः = आ१ गा ।

एकः कर्ण = आ गा = प्रथमचतुर्भुजकर्ण एव ।

द्वितीयकर्णः= आ१  का९= का घा ।

अत्र कर्णयोर्हृदययोश्च साम्यम् । वृत्तकेन्द्रात् कोणगतरेखाभि- र्यानि
समद्विबाहुत्रिभुजानि तेषां फलानि द्वयोश्चतुरस्रयोः समानि अतो
द्वयोश्चतुरस्रयोः फलमपि तुल्यम् ।

( १७३ )

तृतीयकर्णानयने सूत्रम् ।

 १चतुराहतहृदयहते 

 गणिते श्रुतिभाजिते भवति । 

 भुजमुखपरिवर्तनजे 

 पराभिधाना श्रुतिर्नियतम् ॥ १३६ ॥

\begincenter\rule\theta.5\linewidth\theta.5pt\endcenter

(१) 'श्रुतिभ्यां भाजिते' इति श्रुतिभाजिते कर्णयोर्वधेन हृते इत्यर्थः ।
भुजमुखपरिवर्त्तनजे मुखस्थाने कमपि भुजं तद् भुजस्थाने मुखं विन्यस्य
यत्तस्यैव वृत्तस्यान्तर्गतं चतुर्भुजं तस्मिन् भुजमुखपरिर्त्तनजे
चतुर्भुजे नियतं पराभिधाना परसंञ्जका श्रुतिर्भवति । अत्रोपत्तिः ।
'कर्णाश्रितभुजवधयुतिगुणिते' इत्यादिना

क्षेत्रफलम् = क२ ( भु१ भु४ + भु२ भु३ )/ ४ हृ ।

यदि क्षेत्रे मुखस्य 'भु४ ' इत्यस्य तथाभुजस्य 'भु२' इत्यस्य च
परिवर्त्तनं कार्यं तदा नूतनक्षेत्रे यदि पूर्वफलं तदा 'क१' मानं तदेव,
कर्णयोर्हतिः = क१ क२ = भु१ भु४ + भु२ . भु३ 

अतः क्षेत्रफलम् = क१ क२ क३ / ४ हृ = फ ∴ क२ = ४ हृ. फ/ क१ क२
।

कर्णयोर्घातस्य, चतुर्गुणहृदयक्षेत्रफलयोर्घातस्य च स्थिरत्वादय
मन्यकर्णश्चतुर्भुजानां स्थिरत्वात् सर्वदा नियतं निश्चितं स्थिरं

( १७४ )

पूर्वाक्तोदाहरणे

हृदयं ३२५ गणितम् १६०५१२ । लब्धस्तृतीयकर्णः पराभिधानः ८३६० /१३

सूत्रं हृदयस्य---

'' श्रुतिहतिपार्श्वभुजाहति- 

 बधतो लम्बद्वयाहतिविभक्तात् । 

 मूलं द्विसमत्रिसमासमेषु 

 दलितं भवेद् हृदयम् " ॥ १३७ ॥

हृदयानयनार्थं न्यासः । कर्णौ ५२०।६२४

लम्बौ १५८४ /५ ।२०२४/५ लब्धं हृदयम् ३२५ \textbar

अथ वा सूत्रम् ।

 १चतुराहतफलविहृते 

 त्रिकर्णघातेऽथवा हृदयम् । 

\begincenter\rule\theta.5\linewidth\theta.5pt\endcenter

भवतीति स्पष्टम् । एवं द्वितीयभुजमुखपरिवर्तनेऽपि अयमेवान्यः कर्णः समायाति
।

(१) अत्रोपपत्तिः । तृतीयकर्णसाधनवैपरीत्येन स्फुटा ।

( १७५ )

चतुरस्रकर्णौ ५२० ।६२४ तृतीयः ८३६०/१३ गणितम् १६०५१२ ।

लब्धं हृदयम् ३२५ ।

ब्रह्मगुप्तलल्लाभ्यां यद् हृदयानयनमुक्तं तन्न ।

तत्र ब्रह्मगुप्तस्य सूत्रम्-

' १हृदयं विषमस्य भुज- 

 प्रतिभुजकृतियोगमूलार्धम्' इति ।

अस्य सूत्रस्य दूषणमव्यापकत्वात् ।

लल्लास्याऽपि सूत्रम् ।

'विषमस्य भुजप्रतिभुज- 

 कृतिसंयुतिपददलं भवेद् हृदयम्' इति \textbar

तथा च श्रीपतेरपि सूत्रम् ।

'अतुल्यबाहोः प्रतिबाहुबाहु- 

 वर्गैक्यमूलस्य दलं हि हृद् वा' ।

एतेऽन्धपरंपरयैवाविचार्य सूत्रणि कृतवन्तः ।

कर्णयोगादधरोर्ध्वकर्णखण्डानयने सूत्रम् ।

  २कर्णाश्रितभुजघातौ 

 स्वयुतिहृतावन्यकर्णसङ्गुणितौ । 

\begincenter\rule\theta.5\linewidth\theta.5pt\endcenter

(१) द्रष्टव्ये सज्जनकसम्पादितब्राह्मस्फुटसिद्धान्तस्य १९०-\/-\/-

१९१ पृष्ठे ।

(२) अत्रोपपत्तिः । 'कर्णाश्रितभुजबधयुति इत्यस्योपपत्तौ पूर्वं
प्रदर्शितम् ।

( १७६ )

श्रुतियोगादधरोर्ध्वे 

 चतुर्भुजे स्तः श्रवः खण्डे ॥ १३८ ॥

न्यासः \textbar

आद्यकर्णाश्रितभुजघातौ ८२५००\textbar ३१८७८० एतौ स्वयुत्या ४०१२८ भक्तौ
१२५/ ६०८ \textbar

४०३/६०८ श्रन्यकर्णेनाऽनेन ६२४ गुणितौ

जाते कर्णयोगादधरोर्ध्वखण्डे १८८३७/ ३८ ।४८७५/३८ एवं द्वितीयस्य

१२२८५/३८ \textbar ७४७५/३८

\begincenter\rule\theta.5\linewidth\theta.5pt\endcenter

लं१ + लं२ = भु१ भु२ + भु३ भु४/ २ हृ । तथा 

लं१  = भु१. भु२ /२ हृ । लं २ = भु३. भु४ /२ हृ 
\textbar

ततोऽनुपातः , लम्बद्वययोगेन अन्यकर्णः ( क२) तदा पृथक् पृथक्
लम्बाभ्यां के जाते अधरोर्ध्वखण्डे क्रमेण---

भु३ . भु४ X क२/ भु१ भु२ + भु३ भु४ । भु१ . भु२ X क२/
भु१ भु२ + भु३ भु४

इत्युपपद्यते ।

( १७७ )

अथ वा सूत्रम् \textbar

  १तार्त्तीयेन श्रवसा 

 कुमुखे भक्ते पृथक् पृथक् ताभ्याम् । 

 बाहू गुणितौ श्रवसो- 

 ऽधरस्थित ऊर्ध्वगे खण्डे ॥ १३९ ॥ 

\begincenter\rule\theta.5\linewidth\theta.5pt\endcenter

(१) तार्त्तीयेन श्रवसा पूर्वसाधितेन तृतीयेन पराख्येन कर्णेन, ताभ्यां
पृथक् पृथक् द्वौ बाहू गुणितौ तदा श्रवसोः कर्णयोरधरस्थिते खण्डे ऊर्ध्वगे
खण्डे च भवतः ।

अत्रोपपत्तिः । 'कर्णााश्रितभुजबधयुति' इत्यादिना वैपरीत्येन

ल१ + ल२ =२फ /क१

तथा 'चतुराहतफलविहृते इत्यादिना हृदयस्योत्थापनेन,

ल१ = भु१ . भु२ / २हृ = २फ . भु१ . भु२/ क१ क२ क३

एवम् ल२ = भु३ . भु४ / २हृ = २फ . भु३ भु४/क१ क२ क३

ततो लम्बयोगेन ( २फ /क१ ) अन्यकर्णः ( क२ ) कर्णो लभ्यते

तदा पृथक् पृथग् लम्बभ्यां के जाते अधरोर्ध्वे खण्डे क्रमेण

२फ . भु१ भु२ क२ . क१/ २फ क१ क२ क३= भु१ भु२/क३=

द्वितीयकर्णस्योर्ध्वखण्डम् ।

२फ . भु३ भु४ क२ . क१/ २फ क१ क२ क३= भु३ भु४/क३=  

द्वितीयकर्णस्यधरखण्डम् ।

१२

( १७८ )

लम्बानयने सूत्रम् \textbar

  १भूहतकर्णविभक्ते 

 स्वाधरखण्डाहते फले द्विगुणे । 

तदेव चतुरस्रम् ।



तृतीयः कर्णः ८३६०/१३ । अत्राऽनेन भूमुखे ६००।२५० भक्ते जाते ८१९/८३६ ।
३२५/८३६ आद्येनाऽनेन ८१९/८३६ भुजौ ३३०।५०६ गुणितौ जाते कर्णयोगादधरखण्डे
१२८५/३८ । ७४७५/३८ । पुनर्द्वितीयेन ३२५/८३६ भुजौ ३३०।५६० गुणितौ जाते
कर्णयोगादूर्ध्वखण्डे ४८७५/३८ \textbar १८८३७ /३८ अन्योन्यखण्डयुक्तौ
जातौ कर्णौ ५२०।६२४ ।

\begincenter\rule\theta.5\linewidth\theta.5pt\endcenter

अत्र भु१= मुखम् । भु३ = भूमिः।

एवमन्यकर्णस्य भु२ भु४ /क३  प्रथमकर्णस्याधरखण्डम् ।

भु३ .भु१/ क२ प्रथमकर्णस्योर्ध्वखण्डम् ।

इत्युपपद्यते ।

( १) अत्रोपपत्तिः । पूर्वसूत्रोपपत्तौ ल१ = २फ . भु१ भु२/क१
क२ क३

( १७९ )

कर्णाग्रस्पृग् लम्बो

 द्विसमादिचतुर्भुजेष्वथ वा ॥ १४० ॥

लम्बज्ञानार्थं न्यासः । गणितं १६०५१२ । कर्णाधरखण्डे \textbar

१२२८५/३८ । १८८३७/३८ । लम्बो १५८४/५ । २०२४/५ ।

अथवा सूत्रम् ।

 १द्विगुणकुगुणहृदयोद्धृृत- 

 तृतीयकर्णाहतौ पृथक्कर्णौ । 

\begincenter\rule\theta.5\linewidth\theta.5pt\endcenter

=२फ/क१ क२ x अखं । ल१ कोटिः । भूमिः कर्णः । क१ कर्णस्य
भूमिलग्नमूलाल्लम्बमूलावधि भुजः । इत्येकं जात्यम् । क १ कर्णः ।

कर्णाग्राद्भूम्युपरि लम्बः कोटिः । कर्णमूलादेतल्लम्बमूलपर्यन्तं
भूमिखण्डं भुजः । इति द्वितीयं जात्यं प्रथमजात्यसजातीयम् । ततोऽनुपातः ।
भूमिकर्णे ल१ कोटिस्तदा क१ कर्णे का जाता कोटि स्वरूपा

लम्बमानम् = ल१.क१/भू २फ. अखं. क१/क१ क२ भू

अतो यत्कर्णस्याग्राल्लम्बोऽपेक्षितस्तदितरकर्णेन तथा तदितरकर्णाधः खण्डेन
चात्र कर्म कर्त्तव्यमिति स्फुटम् ।

(१) अत्रोपपत्तिः । पूर्वसूत्रेण लम्बमानम्

=२ फ.अ ख२ /भू.क२२ 

( १८० )

अन्योन्याधर-( खण्डाभ्यां नि)-हतौ 

 लम्बकावथ वा ॥ १४१ ॥

पुनर्न्यासः । कर्णाधरखण्डे १२२८५/३८ । १८३७/३८ तृतीयकर्णः ८३६० /१३ ।
हृदयम् ३२५ इदं द्विगुणभूगणितम् ४०९५०० अनेन तृतीयकर्णो भक्तः ४१८/२६६१७५
अनेन कर्णौ ५२०।६२४ गुणितौ ३३७५/४०९५ । २००६४/२०४७५
कर्णखण्डाभ्यामाभ्याम् १२२८५/३८ । १८३७/३८ अन्योन्य गुणितौ जातौ लम्बौ
१५८४/५ ।२०२४ /५

कर्णयोगादधरलम्बज्ञानार्थं सूत्रम् ।

 १पार्श्वभुजाहतिगुणितात् 

 कर्णाऽधरखण्डघाततो म् लम् । 

\begincenter\rule\theta.5\linewidth\theta.5pt\endcenter

'चतुराहतहृदयइत्यादिना' फ = क१क२क३ / ४ हृ

अतः फलस्थाने तदुत्थापनेन

लम्बमानम् = २ फ. अ ख२/ भू. क२

= क१ क२ क३ अ२ खं/ २ हृ. भू. क२ X क१ क३ अखं१/२
भू २ हृ

= ( क३/२ भू हृ ) क१ अ ख२  

एवं द्वितीयो लम्बः = ( क३/२ भू हृ ) क२ अ ख१ 

(१) अत्रोपपत्तिः । यदि कर्णेन तदग्रलम्बस्तदा तदधरखण्डेन

( १८१ )

द्विगुणितहृदयविभक्तं 

 श्रुतियुतितो जायते लम्बः ॥ १४२ ॥

\begincenter\rule\theta.5\linewidth\theta.5pt\endcenter

किम्, लब्धः श्रुतियोगादाधारोपरि लम्बः= ल१ अ ख१/क१ । परन्तु
भूहतकर्णविभक्ते इत्यादिना ल१ = २ फ अख२/भु१ क२ 

∴ श्रुतियोगाल्लम्बः = २ फ . अख१ अख२/भु१ क१ क२ । 

'चतुराहतहृदयहते' इत्यादि वैपरीत्येन २ फ = क१ क२ क३ /२ हृ

अतः श्रुतियोगाल्लम्बः = क१ क२ क३ . अख१ अख२/२ हृ भु१ क१
क२

= क१ अख१ अख२/२ हृ भु१  

तद्वर्गः = क२३ अख१ अख२ X अख१ अख२/२ हृ भु२१ X २ हृ =
योल२

अथ 'तार्त्तीयेन श्रवसा' इत्यादिना अख१ = भु१ भु२ 
/क३ 

अख२ = भु१ भु४  /क३  । एकस्थाने एतदुत्थापनेन 

योल२ = क२३ भु२१ भु२ भु४ X अख१. अख२/( २
हृ ) भु२१ क२३

= भु२ भु४ X अख१. अख२/( २ हृ )   

∴ योल = \sqrtभु२ भु४ X अख१. अख२/२ हृ । इत्युपपन्नम् ।

( १८२ )

तदेव क्षेत्रदर्शनम् ।

हृदयम् ३२५ । अत्र पार्श्वभुजकर्णाधरखण्डानि ३३०\textbar५६० ।

१२२८५/३८ एषां घातस्य मूलम् ६२१६२१/३८\textbar एतद् द्विगुणितहृदयेन ६५०
भक्तं जातः कर्णयोगादधरलम्बः ४७८१७/१९०।

अथ वा सूत्रम् ।

 १बाह्वोः कृती विहीने 

 पृथक् पृथग् व्यासवर्गतो मूले । 

 स्वभुजाप्ते शकलाख्ये 

 तद्युतिहृतभूः श्रवो लम्बः ॥ १४२ ॥

\begincenter\rule\theta.5\linewidth\theta.5pt\endcenter

(१) श्रवो लभ्बः श्रवणयोगादाधारोपरि लम्बः । तेन श्रुतियोगागतलम्बेन ।
शेषं स्पष्टम् ।

आ का गा घा चतुर्भुजे का गा = भु१  ,गा घा = भू  । आ घा = भु२ ।
वृत्तकेन्द्रम् = के । आ छा = का चा= वृत्तव्यासः= २हृ । कर्णयोर्योगः= यो,
योगादाधारोपरि लम्बः = योना= ल । छाघा = \sqrtव्य - भु२२ = को२  । गा चा
= \sqrtव्य२ - भु२१ काे१  । अत्र रेखागणिततृतीयाध्यायेन
जात्यत्रिभुजसाजात्यं स्पष्टम् ।

( १८३ )

ते तेन हते शकले 

 श्रुतियुतिलम्बात् कुखण्डे स्तः \textbar

\begincenter\rule\theta.5\linewidth\theta.5pt\endcenter

ततोऽनुपातेन

घना = को१ X ल/भु१ । गाना = को२ ल/भु२

गा घा = ( को१/भु१ + को२/भु२ ) = भू

ल = भू/ को१/भु१ + को२/भु२

अतः घा ना = को१ भु१ ( भु/ को१/भु१ + को२/भु२
) 

गा ना = को२/भु२ ( भू/ को१/भु१ + को२/भु२ )


इत्युपपन्नम् ।

( १८४ )

तदेव क्षेत्रदर्शनम् ।

व्यासः ६५० । अत्र भुजकृती १०८९००।२५६०३६ व्यासकृतितो ४२२५०० पास्य शेषे
३१३६००।१६६४६४ मूले ५६०।४०८ स्वहते स्वभुजभक्ते ५६/३३ । २०४/२५३
अनयोर्योगेनानेन १९००/७५९ भू ६३० र्भक्ता जातः कर्णयोगादधरलम्बः ४७८१७/१९०
। अनेन ते शकलाख्ये २०४/२५३ । ५६/३३ गुणिते जाते कर्णयोगाल्लम्बनिपातखण्डे
- १९२७८/९५ । ४०५७२ /९५

लम्बानयने सूत्रम् ।

 १कुमुखकृतिविवरदलहृत- 

 कुहते गणितेऽथ सूचिकालम्बः ॥ १४३ ॥ 

 तद्गुणितबाहुसन्धो 

 स्वलम्बभक्तौ भुजाववधे । 

\begincenter\rule\theta.5\linewidth\theta.5pt\endcenter

( १ ) तेन सूचीलम्बेन हतौ बाहू तथा सन्धी च द्वौ स्वलम्ब- भक्तौ तदा
बाहुस्थाने लब्धाै भुजौ सन्धिस्थाने च लब्धे सूच्या अवधे भवतः ।

( १८५ )

तदेव क्षेत्रम् । गुणितम् १६०५१२ अत्र भूमुखकृती ३९६९०० ।

६२५०० विवर ३३४४०० दल १६७२०० मनेन भूमि ६३० र्भक्ता

\begincenter\rule\theta.5\linewidth\theta.5pt\endcenter

घा चा = य, चा गा = र \textbar आ घा = भु २। का गा = भु १ ।

आ का = मु । गा घा = भू । आ छा = लं१ । आ गा = क १ । का घा = क२ । का
ना = लं२

चा जा = सूचीलम्बः= सूलं ।

घा जा = सूच्या एकावधा = ब१ । गा जा = सूच्या द्वितीयावधा =

ब२।

घा छा = एक सन्धि = स२ । गा ना = द्वितीया सन्धि = स१ ।  

चा घा = य - भु २ । का चा = र- भु१ ।

चतुर्भुजस्य वृत्तान्तर्गतत्वात् चा घा गा, चा आ का त्रिभुजे

सजातीये अतः र - भु१  = य .मु /भू = का चा,

गा चा = का चा + का गा = य .मु /भू + भु १ = मु.य + भू.भु १/भू = र,

तथा, य ( य - भु२) = र ( र - भु २) = मु.य + भू.भु १/भू x य .मु/भू

∴ य - भु२ = मु२.य + भू.भु१मु /भू२ \textbar समच्छेदेन



( १८६)

६३/१६७२० । गणितेन १६०५१२ हता जातः सूचीलम्बः ३०२४/५ अनेन बाहू ३३०।५०६
गुणितौ १९९५८४ । १५३०१४४/५ एतौ लम्बाभ्यामाभ्यां १५८४/५। २०२४/५ क्रमेण
भक्तौ जातौ सूचीभुजौ, ६३०।७५६ तथा सन्धी ४९२/५ ।१५१८/५ सूचीलम्बेन ३०२४/५
गुणितौ १३९७०८८ /२५ । ४५९०४३२/२५ लम्बाभ्यां क्रमेण भक्तौ जाते सूच्याबाधे
८८२/५।२६६८/५।२६६८/५

\begincenter\rule\theta.5\linewidth\theta.5pt\endcenter

य. भू २ - भु२ भू२ =मु२ य + भू भु१ मु

= य ( भू २ - मु २ ) = भू ( भू भु २ + भु१ मु )

∴ य = भू ( भू भु२ +भु१ मु ) / भू२- मु२

ततोऽनुपातेन सू लं = लं५य /भु२  

'कर्णाश्रितभुजबधयुति' इत्यादि वैपरीत्येन भू भु२+ भु१मु

= ४ फ. हृ/ क१ ।

अतः सू लं  = भू लं१ X ४ फ. ह/( भू२-मु२ ) X भु२. क१
= भू X ४ फ. हृ/( भू२-मु२ ) भु२ क१/लं१ । 

= ४ फ. हृ. भू/(भू२-मु२) X २हृ = २ फ. भू/भू२-मु२ = फ
( भू/ भू२-मु२/२हृ ) 

अन्यवासना त्रैराशिकेन स्फुटा ।

( १८७ )

हृदयलम्बानयने सूत्रम् ।

  १भुजदलकृतिहृत्कृत्य- 

 न्तरतो मूलं भवेद् हृदयलम्बः ॥ १४४ ॥

पूर्ववच्चतुरस्रम्।

हदयम् ३२५ । लब्धं भूमुखपार्श्वभुजानां क्रमेण लम्बाः ८०। ३००।२८०।१६५ ।

कर्णव्यासेभ्यश्चतुरस्रयोः कर्णखण्डानयने सून्त्रम् ।

 २व्यासकृतिकर्णवर्गा- 

 न्तरतो मूलेऽवकाशसञ्ज्ञे स्तः । 

\begincenter\rule\theta.5\linewidth\theta.5pt\endcenter

( १ ) अत्रोपपत्तिः । वृत्तकेन्द्राच्चतुर्भुजस्य भुजानामुपरि यो लम्बः स
हृदयलम्बः कोटिः । भुजदलं भुजः । केन्द्राद्भुजाग्रगामि सूत्रं
वृत्तव्यासार्धं हृदय वा हृत् कर्णः \textbar

अतः हृद्भभुजदलवर्गान्तरतो मूलं कोटिहृर्दयलम्बो भवतीति स्पष्टम् ।

( २ ) अत्र क्षेत्रसंस्थानेन का घा =प्रथमकर्णः = क१ । आ गा =
द्वितीयकर्णः= क२ ।

गा चा = तृतीयकर्णः= क३ ।

( १८८ )

 व्यासतृतीयश्रवसो- 

 र्वर्गान्तरतः पदं गुणाख्यं स्यात् ॥ १४५ ॥

\begincenter\rule\theta.5\linewidth\theta.5pt\endcenter

के केन्द्रात् कर्णोपरिलम्बाः क्रमेण केजा, केता, केना, तत्र केजा =
प्रथमावकाशार्धम्

= व१/२ । केता द्वितीयावकाशार्धम्= व२/२ । के ना= गुणाख्यार्धम् = गु/२
।

\textless ता के जा = आ गा, आ चा चापार्धयोगसमः । तदूनवृत्तार्धम् =
\textless ताकेट । तथा तदूनवृत्तार्धम् = \textless चाकेना । अतः
ताकेट, चाकेना त्रिभुजद्वयं मिथः सजातीयम् ।

ततः केता X केचा/ चाना = व२/२ x व्या/२ ÷ गु/२

= व२ x व्या / २ गु = केट ।

जाट= केट + केजा = व२व्या / २ गु + व१/२ = व२व्या +गु व / २ गु

ततः

यो ट जा त्रिभुजे योजा = योका- योघा/२= केना.जाट/चा ना =



( १८९ )

गुणगुणिताववकाशौ 

 व्यासेन च तौ मिथोऽन्तरितौ । 

 संयुक्तौ च तृतीय- 

 श्रवणाप्तौ कर्णखण्डयोर्विवरौ ॥ १४६ ॥

 अल्पेनाऽल्पमनल्पम - 

 नल्पेन च संक्रमः श्रवसा । 

 चतुरस्रयोद्वर्योश्च 

 क्रमशः श्रुत्योश्च खण्डानि ॥ १४७ ॥

न्यासः ।

चतुरस्रकर्णौ ५२० ।६२४ तृतीयः कर्णः ८३६०/१३ व्यासः ६५०। अत्र करणम् ।
व्यासः ६५० अस्य वर्गात् ४२२५०० कर्णवर्गौ २७०४०० ।३८९३७६ पृथगपास्य
शेषयोरेतयोः १५२१००।३३ १२४ मूले अवकाशाख्ये ३९० । १८२ व्यास: ६५०
तृतीयकर्ण ८३६०/१३ अनयोर्व

\begincenter\rule\theta.5\linewidth\theta.5pt\endcenter

व२ व्या + गु व१/क३ 

वा योका - योघा = व२ व्या + गु व१/क । एतद्वशेन 'द्विगुणित-
हृदयकृतेर्भू ' इत्यादिना यच्चतुर्भुजं तत्रेदं कर्णे खण्डान्तरं व२ व्या
- गु व१/क३ एवं भविष्यति ।

एवं द्वितीयकर्णखण्डान्तरानयनोपपत्तिर्ज्ञेयेति ।

( १९० )



र्गान्तरम् १५१२९००/१६९ अस्य मूलं गुणाख्यः १२३०/१३ अनेनावकाशौ गुणितौ
३६९००।१७२२० पुनरवकाशौ ३९०।१८२ व्यासेन ६५० गुणितौ २५३५००। ११८३० एतौ
पूर्वराशिभ्यामाभ्या ३६९००।१७२२० मन्योन्यान्तरितौ २३६२८०।८१४००
तथैवान्योन्यसंयुतौ २७०७२०।१५५२०० एते सर्वे तृतीयकर्णेन ८३६०/१३ भक्तं
जाते प्रथमक्षेत्रस्य खण्डयोर्विवरे ६९८१/१९।२४०५/१९ कर्णाभ्यामाभ्यां
६२४।५२० संक्रम- णेन जातानि कर्णखण्डानि । लघुकर्णखण्डे ७०७५०/३८
\textbar १२१८५/३८ बृहत्कर्णखण्डे ४७५/३८ । १८८३७/३८ ।
द्वितीयचतुभुर्जस्य श्रवणविवरे ८७९८४ /२०९। ५५४४० /२०९ कर्णाभ्यामाभ्यां
६२४।५२० सङ्क्रमणेन बृहत्कर्ण- खण्डे १०९२००/२०९।२१२१६/२०९ लघुकर्णखण्डे
७९५६०/२०९।२९१२०/२०९

\begincenter\rule\theta.5\linewidth\theta.5pt\endcenter

कर्णखण्डव्यस्राणां पृथक् पृथक् फलानयनाय सूत्रम् ।

 १यस्य त्र्यस्रस्य श्रृति- 

 खण्डाहतिताडिते तृतीये च । 

\begincenter\rule\theta.5\linewidth\theta.5pt\endcenter

( १ ) अत्रोपपत्तिः । 'तार्त्तीयेन श्रवसा ' इत्यादिना

शिरः कोणात् एककर्णोपरि लम्बः = ल१= भु१ भु२/२ हृ

कर्णोर्ध्वखण्डम् = ऊ ख = भु३. भु४ /क३ ।

( १९१ )

चतुराहतहृदयहृते 

 कर्णे तस्यैव गणितं स्यात् ॥ १४८ ॥ 

पूर्वोक्तचतुरस्रयोर्न्यासः । तृतीयः कर्णः ८३६०/१३ हृदयम् ३२५।
मुखादिप्रदक्षिणक्रमेण चतुर्णां त्र्यस्राणां फलानि ४७४३७५/३८ । ७७९६२/
३८। ३०१२४७२/३८ \textbar१८३२५८५/३८ अस्य चतुर्भुजानयनं स्पष्टम् । तत्कथम्
। 'त्रिभुजस्य फले ज्ञाते लम्बज्ञानमिति त्र्यस्रं परिवर्त्य स्वेच्छयैकं
भूमिं परिकल्प्य त्र्यस्रफलं भूभक्तं द्विगुणं मध्यलम्ब इति लम्बमानीय
लम्बवर्गौ भुजवर्गादपास्य मूलमावाधा साऽपि क्वचिदृणगता स्यात्, आवाधोना
भूः पीठलम्बवर्गयोगान्मूलं भुज इति ' अत्र चतुर्भुजे मुखत्र्यस्रदर्शनम् ।
एतत् त्र्यस्रं परिवत्य जातं गणितम् ४७४३७५/३८ अतो लम्बः १६५०/१३ अस्य
वर्गः २७२२५००/१६९

\begincenter\rule\theta.5\linewidth\theta.5pt\endcenter

अनयोर्घातार्धमेककर्णखण्डत्र्यस्रफलम् = भु१ भु२ भु३. भु४ / क३ x
४ हृ

= भु१ भु२ / क३  . भु३. भु४/ क३  . क३ /४ हृ 

= खण्डद्वयघात x क ३ /४ हृ

अत उपपन्नम् ।

( १९२ )



भुजवर्गादस्मात् २३७३५६/१४४४ अपास्य शेषान्मूलमावाधा धनमृणं वा ९२२५/४९४ ।
१६९ इमे भूमेः पृथगपास्य जाते अन्ये अबाधे ४३९७५/४९४ अस्य
वर्गाल्लम्बवर्गयुतान्मूलं लभ्यते सा ग्राह्या नान्या । अत्र तावदियं ९२२५
ग्राह्या इमां भूमेरपास्य शेषमन्यावाधा २८००/१३ अस्य वर्गात् ७८४००००/१६९
लम्बवर्गयुतात् १०५६२५००/१६९ मूलम् २५० । एतदेव चतुरस्रमुखम् । एवं
प्रदक्षिणक्रमेण भुजत्रयम् ५०६।६३०।३३० ।

इति सङ्क्षेपादुक्तं 

 विस्तरभीत्या मयाऽत्र भूगणितम् । 

 तत् क्षन्तव्यं विद्भि- 

 श्चित्तचमत्कारि यन्नोक्तम् ॥

इति श्रीसकलकलानिधिनरसिंहनन्दनगणितविद्याचतुरानन- नारायणपण्डितविरचितायां
गणितपाट्यां कौमुद्याख्यायां क्षेत्र व्यवहारः समाप्तः ।

अथ खातव्यवहारः ।

सूत्रम् ।

 १विस्तारो वा दैर्घ्यं 

 वेधो वा जायते विषमः । 

\begincenter\rule\theta.5\linewidth\theta.5pt\endcenter

( १ ) 'गुणयित्वा विस्तारं बहुषु स्थानेषु ' इत्यादि भास्करोक्तम-
नुरूपमेवेदम् ।

( १९३ )

तद्याेगः पदमित्या 

 भक्तः साम्यत्वमुपयाति ॥ १ ॥

 क्षेत्रफलं वेधगुणं 

 घनहस्तमितिः प्रजायते खाते ।

उदाहरणम् ।

अष्टादशकराऽऽयामा 

 वापी षट्करविस्तरा । 

 वेधे त्रिपञ्चचसप्ताऽत्र 

 वद खातफलं सखे ॥ १ ॥

न्यास: ।

खातम् । जातं समवेधखातम् ।

जातं खातफलम् ५४० ।

अपि च ।

मुखतलतुल्ये खाते 

 चतुष्कहस्ते त्रिहस्तविस्तारे । 

१३

( १९४ )

वेधे हस्तचतुष्के किं 

 गणितं समचतुष्के च ॥ २ ॥

न्यासः ।

जाते घनगणिते ४८\textbar६४

सूत्रम्।

 १मुखतलतद्योगानां 

 क्षेत्रफलैक्यं विभाजितं षड्भिः ॥ २ ॥

 वेधगुणं घनगणितं 

 मुखसदृशतलेऽथवा खाते । 

उदाहरणम् ।

रामाम्बुधी, कृतयुगे, तलविस्तृती ते 

 दृष्टे पृथक् त्रिगुणिते मुखविस्तृती च । 

 बेधश्च षट्, कथय खातफलं तयोर्मे 

 जानासि चेद् गणक खातविधिं समग्रम् ॥ ३ ॥

\begincenter\rule\theta.5\linewidth\theta.5pt\endcenter

( १ ) 'मुखजतलजतद्युतिजक्षेत्रफलैक्यं हृतं षडमिः ' इति
भास्करोक्तानुरूपमेवेदम् ।

( १९५)

न्यासः ।

जाते घनगणिते ३१२\textbar४१६

सूत्रम् ।

१मुखतलतद्योगानां

वर्गसमासेऽष्टभाजिते लब्धम् \textbar\textbar ३ \textbar\textbar

वेधाभिहतं कूपे

घनगणितं जायते स्थूलम् ।

मुखतलसमखातफल-

त्र्यंशः सूचीफलं भवति \textbar\textbar ४ \textbar\textbar

\begincenter\rule\theta.5\linewidth\theta.5pt\endcenter

( १) अत्रोपपत्तिः । मुखव्यासः = व्या१ । तलव्यास: = व्या२ । ततो
'मुखजतलजतद्युतितः' इत्यादिभास्करविधिना, त्रिगुणित- व्याससमं स्थूलं
परिधि प्रकल्प्य

मु फ =३व्या२१/४ । त फ = ३व्या२२ /४

योगफलम् =३(व्या१ + व्या२)१/४

एषां योगः= ३/४ \ व्या२१ +व्या२२ +( व्या१ +व्या२ )\

( १९६ )

उदाहरणम् ।

व्यासस्तु षोडशकरो वदनस्य कूपे 

 व्यासस्तलस्य जलधिप्रमितस्तु वेधः । 

 तिग्मांशुसम्मित इहैव फलं सखे किं 

 सूचीफलं कथय मे यदि वेत्सि मित्र ॥ ४ ॥

जातं स्थूलघनगणितम् १००८ । अतः सूक्ष्मम् १०६२ ५४/१२५ सूच्या न्यासः । जातं
स्थूलं घनगणितम् ८०९ ५९ /१२५ । अतः

सूक्ष्मफलम् १६८ ।

सूत्रम् ।

  १ अङ्गुलसंख्यायां यदि 

 दृषति तदा व्यासदैर्घ्यपिण्डानाम् । 

\begincenter\rule\theta.5\linewidth\theta.5pt\endcenter

षडभिर्हृतः = १/८\व्या२१ +व्या२२ +(व्या१ +व्या२)२\

तता वेधगुणितेन घनफलं भवति ।

यत्र मुखतलयोः समं खातं तस्य फलस्य घनफलस्य त्र्यंशः सूचीफलं भवतीति
'समखातफलत्र्यंशः सूचीखाते फलं भवति'- इति भास्करोक्तानुरूपमेव । अत्र यदि
परिध्यानयनार्थं ३ - स्थाने सूक्ष्मो गुणको गृह्यते तदा सूक्ष्मं कूपघनफलं
भवतीति स्फुटं गणितविदाम् ।

( १ ) अत्र एकपाषाणघनहस्ते घनाङ्गुलानि = ६१४४ कल्पितानि । अस्य ग्रन्थस्य
परिभाषाप्रकरणे द्रष्टव्यो नवमः श्लोकः \textbar

( १९७ )

खातेऽम्बुधिकृतशशिरस- 

 भक्ते पाषाणहस्ताः स्युः ॥ ५ ॥

दाहरणम् ।

दैर्घ्यं त्रिभागसहितं करपञ्जकं च 

 व्यासे दलान्वितकरत्रयमेव पिण्डे । 

 हस्तार्धमार्यवर चेत् पटुताऽस्ति पाट्यां 

 हस्तात्मकं च दृषदे गणिते वदाऽऽशु ॥ ५ ॥ 

न्यासः ।

खातघनगणितम् ३८/३ । 'घनहस्ते तौ च साङ्घ्री स्तः' इत्यनेन ९/४ घनफले गुणिते
जाताः पाषाणहस्ताः २१ ।

अङ्गुलात्मके न्यासः ।

( १९८ )

जातमङ्गुलघनफलम् १२९०२४ एतान्यङ्गुलान्येभिः ६१४४ भक्तानिजाताः
पाषाणहस्तास्त एव २१ । एवं वृत्तत्र्यस्त्रादिक्षेत्रफलमुच्छ्रयहतं
घनफलं स्यात् ।

अपि च ।

समावृत्ते पाषाणे 

 त्रिकरव्यासे तद विस्तारे । 

 पाषाणफले हस्ताः कति 

 गणक, भवन्ति कथयाऽऽशु ॥ ६ ॥

जातं सूक्ष्मं क्षेत्रफेलं १४२२९/२०० एतत् पिण्डेनानेन ३/२ गुणितं १४८७/४००
एतत् स्राङ्घ्रिद्वयगुणितं जाताः पाषाणुहस्ताः ।

सूत्रम् ।

 १गोलव्यासस्य कृति- 

 स्त्रिसङ्गुुणा पृष्ठजं फलं सूक्ष्मम् । 

\begincenter\rule\theta.5\linewidth\theta.5pt\endcenter

( १ ) अत्रोपपत्तिः । अत्र स्थूलत्वात् परिधिः = ३ व्या ।

ततः पृष्ठफलम् - व्या. प = ३ व्या२ ।

तथा घनफलम् =पृ फ x व्या/६ । अत उपपन्नम् ।

( १९९ )

पृष्ठजफलषड्भागा

 व्यासगुणो गोलघनगणितम् ॥ ६ ॥

उदाहरणम् ।

समवृत्तघने गोले 

 दशकरमध्ये वदाशु पृष्ठफलम् । 

 घनगणितं च दृषत्फल- 

 माशु सखे कथय यदि वेत्सि ॥ ७ ॥

न्यासः ।

जातं पृष्ठफलं स्थूलम् ३०० अतः सूक्ष्मम् ३१६ १/४ ।

घनगणितं स्थूलम् ५०० अतः सूक्ष्मम् ५२७ ।

पाषाणफलं स्थूलम् ११२५ अतः सूक्ष्मम् ११८५ अङ्गुलानि ४६०८ ।

सूत्रम् ।

 १इष्टक्षेत्रफलाप्ते 

 घनगणिते स प्रजायते वेधः । 

\begincenter\rule\theta.5\linewidth\theta.5pt\endcenter

( १ ) घनफले इष्टक्षेत्रस्य फलेन भक्ते तदा खाते स वेधः प्रजायते ।
अत्रोपपत्तिः खातघनफलानयनवैपरीत्येन ।

( २०० )

उदाहरणम् ।

पञ्चकरा समवापी 

 नगस्य कस्याप्युपत्त्यकानिकटे । 

 समचतुरस्रा त्र्यङ्गुल- 

 जलधारा तन्नगादधः पतिता ॥ ८ ॥

वाप्यन्तरजलपूर्णा 

 गणक तडागोच्छ्रितिं कथय । 

इति खातव्यवहारः ।

अथ चितिः ।

सूत्रम् ।

 १क्षेत्रफलमुच्छ्रयघ्नं

 चयने गणितं प्रजायते तस्मिन् ।

 सम्भक्तमिष्टकाया 

 गणितेन तदिष्टका संख्या ॥ ७ ॥

\begincenter\rule\theta.5\linewidth\theta.5pt\endcenter

( १ ) 'उच्छ्रयेण गुणितं चितेरपि ' इत्यादि भास्करोक्त्योपपत्तिः

स्फुटा । अत्र गणितशब्देन घनफलमवगम्यम् ।

( २०१ )

उदाहरणम् ।

हस्तायतार्धविस्तृ- 

 त्यङघ्र्युत्सेधाभिरिष्टकाभिश्च । 

 अष्टायतषट्व्यास- 

 त्र्युत्सेधा वेदिका रचिता ॥ ९ ॥

 घनगणितमिष्टकानां 

 संख्या तस्याश्च कथयाऽऽशु । 

न्यासः ।

इष्टकाघनफलम् १/८। वेदिकाघनफलम् १४४ । चयने जाता इष्टकाः ११५२ । अथ वा
सप्तराशिकेन सिध्यति । एवं दृषच्चितेरपि । इति चितिव्यवहारः ।

क्रकचे सूत्रम् ।

 १पिण्डाग्रमूलयुतिदल- 

 हतदैर्घ्यं दारुदारणैर्मार्गैः । 

 फलमङ्गुलात्मकं तत् 

 षडगशराप्तं करात्मकं भवति ॥ ८ ॥

\begincenter\rule\theta.5\linewidth\theta.5pt\endcenter

(१) अत्रोपपत्तिः पिण्डयोगदलमग्रमूलयो-' इत्यादि श्रीभास्करोक्तवज्ज्ञेया ।

( २०२ )

उदाहरणम् ।

मूलाग्रयोर्नखनृपाङ्गुलसम्मिती च 

 दारोश्चतुर्गुणनखाङ्गुलमध्यदैर्घ्यंम् । 

 मार्गेषु षट्सु फलमाशु करात्मकं मे 

 प्रब्रूहि दारुगणिते पटुतास्ति ते चेत् ॥ १० ॥

न्यासः ।

मार्गः ६ पिण्डयोगार्धम् १८ दैर्घ्य ८० गुणम् १४४० मार्गैर्हतम् ८६४० एतत्
षडगशरैः ५७६ हृतं जातं क्रकचगणितं करात्मकम् १५ ।

सूत्रम् ।

 १यदि दारिते तु तिर्यक् 

 विस्तृतिपिण्डाहतेः प्राग्वत् । 

 कर्मकरप्रतिपत्त्या 

 मूल्यं मृदुकर्कशत्वेन ॥ ९ ॥ 

उदाहरणम् ।

यद्विस्तृतिस्त्रिगुणरन्ध्रमिताङ्गुला च 

 पिण्डस्तु षोडश दशस्वपि वर्त्मसु त्वम् । 

\begincenter\rule\theta.5\linewidth\theta.5pt\endcenter

( १ ) अत्रोपपत्तिः । 'छिद्यते तु यदि तिर्यगुक्तवत्-' इत्यादि
ओभास्करोक्तानुरूपमेवेदम् ।

( २०३ )

जानासि चेद् गणितमार्य वदाशु दारो- 

 स्तिर्यक्छिदो गणितमत्र करात्मकं मे ॥ ११ ॥

न्यासः ।

मार्गाः १० जातं क्रकचगणित हस्ताः १५ ।

इति क्रकचव्यवहारः ।

अथ राशिव्यवहारे सूत्रम् ।

 १षड्भक्तपरिधिवर्गोऽभ्यु- 

 दयहतो घनफलं भवेद्राशौ । 

 हस्तात्मके घनफले 

 पञ्चविभक्ते तु खार्यः स्युः ॥ १० ॥ 

उदाहरणम् ।

यस्मिन् राशौ हस्तषष्टिवृर्तिर्भो 

 विद्वन् वेधः षण्मितस्तत्र मे त्वम् । 

 ब्रूहि क्षिप्रं सन्ति खार्यः कियत्यो 

 राशिज्ञाने नैपुणं चाऽस्ति ते चेत् ॥ १२ ॥

\begincenter\rule\theta.5\linewidth\theta.5pt\endcenter

( १ ) अत्रोपपत्तिः । 'परिधिषष्ठे वर्गिते वेधनिघ्ने घनगणितकराः स्युः-'
इति श्रीभास्करोक्तिवत् । उत्तरार्धोपपत्त्यर्थं द्रष्टव्या परिभाषा
तत्रत्या टिप्पणी च । ( श्लोक( १० -११ )

( २०४ )

न्यासः ।

जातं घनगणितम् ६०० । अतो जाता खार्यः १२० । एवं
वृत्तत्र्यस्त्रादिघनहस्तेभ्यः खार्यः स्युः ।

अपि च ।

साष्टाङ्गुलौ करौ वेधे 

 परिधौ हस्तसप्तकम् । 

 त्रिसङ्गुणं सखे तस्मिन् 

 राशौ धान्यमितिं वद ॥ १३ ॥

न्यासः ।

जातानि घनाङ्गुलानि ३९५१३६ एतानि पादिकाघन २१६ हृतानि जाताः पादिकाः १८२९
१/४। अतः खार्यः ५ कुडवाः १४ पादिकाः ५१/३।

सूत्रम्।

 १अन्तःकोणे भित्त्या- 

 श्रिते बहिःकोणके वृत्तिस्त्र्यंशः । 

 स्वघ्नो वेधाभिहतो 

 रूपद्वित्र्युद्धृतो गणितम् ॥ ११ ॥

\begincenter\rule\theta.5\linewidth\theta.5pt\endcenter

(१) अत्रोपपत्तिः । कल्प्यन्तेऽन्तः कोणस्थ-भित्त्याश्रित-बहिः
कोणस्थराशीनां परिधयः क्रमेण प१ ,प२, प३, । अथ-\/-

( २०५ )

उदाहरणम् ।

अभ्यन्तरकोणस्थितराशेः 

 परिधिस्तु पञ्चदशहस्ताः । 

 भित्त्याश्रितस्य त्रिंशत् 

 कोणवहिःस्थस्य पञ्च नवगुणिताः ॥ १४ ॥

 किं घनगणितं विद्वन् 

 षडुच्छ्रयै द्रुततरं कथय । 

\begincenter\rule\theta.5\linewidth\theta.5pt\endcenter

'द्विवेदसत्रिभागैकनिघ्नात्

तु परिधेः फलम् ।

भित्त्यन्तर्बाह्यकोणस्थ

राशेः स्वगुणभाजितम् \textbar\textbar' इति

भास्करोक्तसूत्रानुसारेण क्रमेण घनहस्ताः

घ१ = (४ प१/६)१ वे/४ = १६ प२१. वे/३६.४ = प२१/९
वे/१ = ( प१/३ )१२वे/१ ।

घ२ = ( २ प२/६ )२ वे/२ = ४ प२२. वे/३६.२ = प२२.
वे/९.२ = ( प२/३ )२ वे/२ ।

घ३ = ( ४/३ प३/६ )२ वे/४/३ = १६ प२३. ३वे/९.३६.४ =
प२३.वे/९.६ = ( प३/३ )२ वे/३ । 

इत्युपपन्नं यथोक्तम् ।

( २०६ )

न्यासः

जातानि घनफलानि १५०।३००।४५० अताे जाताः खार्यः ३०।६०।९०

अथ छायाव्यवहारे सूत्रम् ।

 १शङ्कुहृतच्छाया या 

 पौरुष्याख्या प्रभा तयैकयुजा । 

 भक्ते द्युदले द्यगतं 

 शेषमिने पूर्वपश्चिमाशास्थे ॥ १२ ॥

उदाहरणम् ।

शङ्कोः सखेऽर्काङ्गुलसम्मितस्य 

 द्युतिश्चतुर्घ्नाऽपरदिग्विभागे । 

\begincenter\rule\theta.5\linewidth\theta.5pt\endcenter

(१) अत्रोपपत्तिः । मज्जनकमुद्रितत्रिशतिकायां ४५-४६ पृष्ठवोः
'द्विगुणसशङ्कुच्छायाभक्ते' इत्यादि सूत्रोपपस्या स्फुटा ।

तद्यथा दिगशे = इ शं X १/२ ( इ शं + इ शं छा ) = इ शं X १/४/ ( इ शं + इ शं
छा )

= इ शं X दि द/इ शं + इ शं छा = दि द/१ + इ शं छा/इ शं = दि द/१ + पौ छा ।

अत उपपन्नम् ।

( २०७ )

प्राग्वत् प्रदिष्टाऽत्र गतावशेषे 

 दिनस्य के त्वं कथय द्रुतं मे ॥ १५ ॥ 

न्यासः ।

शंकुः १२ छाया ४८ जाता पौरूषी ४ । अतः प्राक् स्थितेऽर्के दिनगतांशः १/४ ।
अपरस्थे दिनशेषम् १/१० अस्मिन्निष्टदिनमान घटिकागुणिते द्युगतशेषघटिकाः
स्युः ।

सूत्रम् ।

 १द्युदलं दिनगतशेषो- 

 ध्दृतं विरूपं च पौरूषी भवति । 

 सा शङ्कुघ्नी छाया 

 भा पौरूष्या हृता शङ्कुः ॥ १३ ॥

उदाहरणम् ।

यातैष्ये दशभागे 

 शङ्कोरर्काङ्गुलस्य च छायाम् । 

\begincenter\rule\theta.5\linewidth\theta.5pt\endcenter

( १ ) अत्रोपपत्तिः । पूर्वसूत्रेण दिगशे = द्यु द/ १ + पौ भा

∴ १ + पौ भा = द्यु द/दि ग शे ∴ पौ भा = द्यु द/दि ग शे -१,

अथ पौ भा = इ छा/ इ शं ∴ इ चा = पौ भा. इ शं वा इ शं = इ चा/पौ भा
अत उपपन्नम् । 

( २०८ )

यातैष्यच्छायाभ्यां 

 शङ्कुं कथयाशु गणितज्ञ ॥ १६ ॥ 

छायानयने न्यासः । शङ्कुः १२ द्युगतशेषम् १/१० जाता छाया ४८ । शङ्कनयने
न्यासः । छाया ४८ द्युगतशेषम् १/१० जातः शङ्कुः १२ । दीपच्छायायां सूत्रम्
।

  १ न्रूपप्रदीपभक्ते 

 नृदीपमध्यान्तरे नृगुणिते भा । 

 नृहते नृदीपमध्ये 

 भाप्ते सनरे प्रदीपः स्यात् ॥ १४ ॥

उदाहरणम् ।

हस्तद्वयं दीपनृमध्यभूमि- 

 र्दीपोच्छ्रयोऽध्यर्धकरत्रयं च । 

 नरस्य वाऽर्काङ्गुलसम्मितस्य 

 तस्य प्रभां मे कियती वदाशु ॥ १७ ॥

( १ ) 'शङ्कुुप्रदीपतलशङ्कुतलान्तरघ्न-

श्छाया भवेद् विनरदीपशिखौच्च्यभक्तः'

'छायाहृते तु नरदीपतलान्तरघ्ने

शङकौ भवेन्नरयुते खलु दीपकौच्च्यम्' इति ।

भास्करोक्तानुरूपमेवैतत् ।

( २०९ )

अपि च ।

प्रदीपकोच्चयं नरभामहीभ्यो 

 नृदोपभाभ्यश्च महीप्रमाणम् । 

 भूदीपभाभ्यो नरमाशु विद्व- 

 न्नाचक्ष्व मे त्वं गणकाग्रणीश्चेत् ॥ १८ ॥

जाता छाया ८ । दीपोऽज्ञाते जातो दीपः ८४ ।

सूत्रम् ।

१न्रूनप्रदीपगुणिता भा नरभक्ता नृदीपमध्यतलम् । 

भागुणदीपो भायुतनृदीपमध्योद्धृतः शङ्कुः ॥ १५ ॥ 

\begincenter\rule\theta.5\linewidth\theta.5pt\endcenter

( १ ) स्रून् = शङ्कुरहितः ।

'विशङ्कुदीपोच्छ्रयसङ्गुणाभा शङ्कूद्धता दीपनरान्तरं स्यात्'-इति
भास्करोक्तानुरूपं पूर्वखण्डम् ।

यतः । दीपनरान्तरम् = ( उ - शं ) छा/शं = दी ।

छेदगमेन उ. छा - शं. छा = शं. दी,

समशोधनेन उछा= शं छा + शं दी= शं ( छा +दी)

∴ शं = उ .छा/छा + दी इत्युपपन्नमुत्तरदलम् ।

१४

( २१० )

प्रागुक्तोदाहरणे जाताः भूः ४८ । नर्यज्ञाते भुव्यविञ्ज्ञातायां च जातौ
शङ्कुभुवौ १२।४८

विशेषसूत्रम् ।

१भान्तरहृतान्तरेण प्रभाहता भृर्नृभूवधो भाप्तः। 

दीपः स्यादनुपाताद् यदविज्ञातं तु तज्ज्ञेयम् ॥ १६ ॥

उदाहरणम् ।

 शङ्कोरर्काङ्गुलस्य द्युतिरपि 

 शरसङ्ख्याङ्गुला स्यात् तदग्रे 

 न्यस्तस्याऽन्यस्य शङ्कोः 

 सदलकरयुगे तत्प्रभार्काङ्गुला च । 

 तद्भूमानं कियद् भोः कथय 

 मम सखे तत्प्रदीपोच्छ्रितिं च 

 ध्वान्तोपध्वंसने चेत् त्वमसि 

 गुणगणापूर्णरत्नः प्रदीपः ॥ १९ ॥ 

न्यासः ।

जाते भूमाने ७५\textbar१३५ उभयतो दीपोच्छ्रायः स एव १८० ।

\begincenter\rule\theta.5\linewidth\theta.5pt\endcenter

( १ ) 'छायाग्रयोरन्तरसङ्गुणाभा '-इति भास्करोक्तानुरूपमेतत् ।

( २११ )

विशेषसूत्रम् ।

  १भान्तरकर्णान्तर- 

 कृत्यन्तरहृतनृकृतितः कृतहतायाः ।

 रूपयुजो मूलं तद् 

 गुणिते श्रुत्योर्भुवोः शेषे ॥ १७ ॥

 क्रमशः प्रभयोः श्रुत्यो- 

 र्योगौ स्यातां ततस्तु सङ्क्रमणात् ।

 छाये श्रवणौ ताभ्यां 

 प्राग्वज्ज्ञेयं प्रदीपौच्यम् ॥ १८ ॥

उदाहरणम् ।

एकं स्तम्भशिरस्यथ प्रणिहितं 

 ज्योतिः परं तत् कियद् 

 देशेऽधो निहितं प्रदीपनरयो- 

 र्मध्यं नभोद्व्यङ्गुलम् । 

 शङ्कोरर्कमिताङ्गुलस्य जनिते- 

 छाये तदग्रान्तरं 

\begincenter\rule\theta.5\linewidth\theta.5pt\endcenter

( १ ) 'छाययोः कर्णयोरन्तरे ये तयो:' -इति भास्करोक्तानुरूपमे

तत् । तत्र द्वादशाङ्गुलः शङ्कुः । अत्रेष्टशङ्कुः । एतावान् विशेष: ।

( २१२ )

व्योमाग्निप्रमिताङ्गुलं जिनमितं 

 श्रुत्योः सखे चान्तरम् ॥ २० ॥

 तत्कर्णौ कथय द्रुतं च सुमते 

 तज्ज्योतिषोरुच्छ्रयौ 

 प्रौढः सद्गणिताम्बुराशितरणे 

 त्वं कर्णधारोऽसि चेत् ॥

न्यास:।

छायान्तरे कर्णान्तरे ३०।२४ अनयोर्वर्गान्तरम् ३२४ अनेन शङ्कुकृतिः १४४
चतुर्गुणा ५७६ भक्ता ९६/९ सैका २५/९ मूलम् ५/३ अनेन छायाकर्णान्तरे
२४\textbar३० गुणिते ४०।५० एतावेव प्रभयोः कर्णयोश्च योगौ । सङ्क्रमणेन
जाते छाये ५।३५ कर्णौ १३। ३७ अधोदीपोच्यम् ३६ । उपरितनदीपोच्यम् १८० ।

इतिच्छायाव्यवहरः ।

( २१३ )

अथ कुट्टक:\textbar

सूत्रम् ।

भाज्यो हारः क्षेपः 

 केनाऽप्यपवर्त्य कुट्टकस्याऽर्थम् ।

 येन विभाज्यच्छेदौ 

 छिन्नौ क्षेपो न तेन खिलम् ॥ १९ ॥

 हरभाज्ययोर्विहृतयो- 

 रन्योन्यं यो भवेद् ययोः शेषः । 

 स तयोरपवर्तनकृत् तौ 

 तेनैवापवर्तितौ तु दृढौ ॥ २० ॥

 दृढभाज्यहरौ विभजेत् 

 परस्परं यावदेकमवशेषम् । 

 विन्यस्याऽधोऽधस्तात् 

 फलानि तदधस्तथा क्षेपम् ॥ २१ ॥

 तदधः खमुपान्त्येना- 

 हते निजोर्ध्वेऽन्तिमेन संयुक्ते ।

 अन्त्यं जह्यादेवं 

 यावद्राशिद्वयं भवति ॥ २२ ॥

 ( २१४) 

 १हरभाज्याभ्यां तष्टा-\\
वधरोर्ध्वौ ते क्रमेण गुणलब्धी ।\\
यदि लब्धयः समाः स्यु-\\
स्तदागुणाप्तो यथागते भवतः ॥ २३ ॥
विषमाश्चेत् ते शोध्ये\\
गुणलब्धी स्वस्वतक्षणाच्छेषे ।\\
योगभवे गुणलब्धी\\
निजतक्षणतो विशोधिते क्षयजे ॥ २४ ॥
इष्टघ्नतक्षणयुते\\
बहुधा भवती गुणाप्ती ते ।\\
सर्वत्र कुट्टकविधौ\\
कार्यं समतक्षणं सुधिया ॥ २५ ॥

  उदाहरणम् । 

 राशिस्त्रिसप्ततियुतेन शतद्वयेन\\
निघ्नो नवोनितशतेन युतश्च कोऽपि ।\\
भागं प्रयच्छति विशुद्धमगाब्धिनेत्रै-\\
र्भक्तः सखे कथय तं च फलं द्रुतं मे ॥ २१ ॥

\begincenter\rule\theta.5\linewidth\theta.5pt\endcenter

 (१ )कुट्टकोपपत्त्यर्थं द्रष्टव्यं मज्जनकमुद्रितभास्करबीजगणितम् ।

 ( २१५) 

 न्यासः । 

 भा २७३ क्षे ९१ हा २४७ । अत्र 'हरभाज्ययोविहृतयोः--- इति\\
भाज्यः २७३ हारेण २४७ भक्तः शेषम् २६ अनेन हारो २४७ भक्तः\\
शेषम् १३ अनेन पूर्वशेषं २६ भक्तं शुध्यति ततोऽपवर्तनराशिः १३ ।\\
अनेन भाज्यहारक्षेपानपवर्त्य जातो दृढकुट्टकः भा २१ क्षे ७ हा १९\\
दृढभाज्यभाजकयोः फलान्यधोऽधस्तदधः क्षेपस्तदधः खमिति\\
जाता वल्ली ---

१ \\\
९\\
७ \ उपान्तिमेन ७ स्वोर्ध्वे ९ हते ६३ अन्त्येन० युते जातम्-\\
०

 १ \

६ ३ \

 ७ \ पुनरुपान्तिमेनानेन ६३ स्वोर्ध्वे १ हते ६३ अन्त्येन ७ युते\\
७० जातं राशिद्वयम् ७०६३ । अधरोर्ध्वौ तौ ६३।७० दृढहारभाज्या-\\
भ्यामाभ्यां १९।२१ तष्टौ जातौ ६।७, सममेव लब्धी यत एते\\
एव गुणाप्ती ६।७, इष्टघ्नतक्षणयुते' इत्येकेनेष्टेन जाते गुणाप्ती\\
२५।२८ द्विकेन ४४।४९ त्रिकेन ६३।७० एवं बहुधा । 

 सूत्रम् । 

 ९हारक्षेपकयोर्वा प्रक्षेपकभाज्ययोस्तदुभयोर्वा ।\\
अपवर्तितयोर्गुणको लब्धिश्च स्वापवर्तहते ॥ २६ ॥

 उदाहरणम् । 

  येनाभिहताशीतिः\\
समन्विता त्रिंशता च वियुता वा । 

\begincenter\rule\theta.5\linewidth\theta.5pt\endcenter

 ( १) 'भवति कुट्टविधेर्युतिभाज्ययोः'--- इति श्रीभास्करोक्तानु-\\
रूपमिदम् ।

 ( २१६) 

 त्रिगुणत्रयोदशाप्ता\\
शुध्यति तं कथय पृथगाप्तिम् ॥ २२ ॥ 

 न्यासः । 

 भा ८० क्षे ३० हा ३९ । प्राग्वज्जाते गुणाप्ती २४।५०\\
अथवा भाज्यक्षेपौ त्रिभिरपवर्तितौ-भा ८० क्षे १० हा १३ ।

प्राग्वज्जाता वल्ली ६ \\\
६\\
५ ० \ गुणाप्ती ५।५० स्वापवर्तनेन त्रिभि-\\
०\\
र्गुणितो गुण इति जाते ते एव गुणाप्ती २४।५० ।\\
अथवा भाज्यक्षेपौ दशभिरपवर्तितौ--- भा ८ क्षे ३ हा ३९ ।\\
प्राग्वज्जाता वल्ली ० \

 ४ \

 १ \ गुणाप्ता १५।३ लब्धयो विषमाः सन्त्यत

 ३ \ 

 ० \

एते स्वतक्षणाभ्यामाभ्यां ३९।८ शोधिते जाते क्षेपजे गुणाप्ती २४।५\\
स्वापवर्तनेन दशभिर्गुणिता लब्धिरिति जाते ते एव गुणाप्ती २४।५०\\
अथवा भाज्यक्षेपौ दशभिरपवर्त्य हारक्षेपौ त्रिभिरपवर्तितौ भा\\
८ क्षे १ हा १३ । प्राग्वज्जातं राशिद्वयम् ३।५ लब्धयो विषमा\\
अतः स्वतक्षणाभ्यामाभ्यां १३।८ शोधिते जाते ५।८ हारक्षेप-भाज्य-\\
क्षेपापवर्तनाभ्यां ३।१० क्रमेण गुणिते ते एव गुणाप्ती २४।५०\\
प्राग्वदेकेनेष्टेन जाते ६३।१३० द्विकेन १०२।२१० एवमनेकधा ।\\
द्वितीयोदाहरणे न्यासः । भा ८० क्षे ३० हा ३९ । जाते योगजे\\
गुणाप्ती २४।५० एते स्वतक्षणाभ्यामाभ्यां ३९।८० शोधिते जाते\\
वियोगजे गुणाप्ती १५।३० प्राग्वदेकेनेष्टेन जाते ५४। ११० द्विकेन\\
९३।१ ९० इष्टवशादनेकधा । 

 अपि च ।

 को राशिः सप्तभिः क्षुण्णः\\
सप्तत्रिंशत्समन्वितः ।

( २१७) 

 वर्जितो वा त्रिभिर्भक्तौ\\
निरग्रः स्याद् वदाशु तम् ॥ २३ ॥ 

 न्यासः । 

 भा ५ क्षे ३७ हा ३ । जाता वल्ली १\\
१ \राशी ७४३७ ।
३ ७\\
० 
अत्राऽधः स्थिते राशौ त्रिभिर्भक्ते द्वादश लभ्यन्ते, ऊर्ध्वस्थितराशौ\\
पञ्चभिर्भक्ते चतुर्दश लभ्यन्ते ते असमानत्वान्न ग्राह्याः 'कार्यं\\
समतक्षणमिति' उभयोर्द्वादशसुगृहीतेषु जाते गुणाप्ती १।१४ चतुर्द-\\
शसु गृहीतेषु जाते गुणाप्ती ५।४ 

 समतक्षणमित्युपचारो यथेष्टघ्नतक्षणयुते बहुधा गुणाप्ती भवत-\\
स्तथेष्टघ्नतक्षणवियुते (राशिद्वये) बहुधा गुणाप्ती भवतः । 

 ऋणक्षेपे द्वादशमितफले गृहीते गुणाप्ती २।९ चतुर्दशमितफले\\
गृहीते गुणाप्ती ८।१ इत्यादि । 

सूत्रम् ।

 १हरतष्टघनक्षेपे\\
लब्धिस्तक्षणफलेन संयुक्ता ।\\
क्षयगे क्षेपे तक्षण-\\
फलोनिते जायते लब्धिः ॥ २७ ॥
हरतष्टभाज्यराशौ\\
फलघ्नगुणसंयुता लब्धिः । 

\begincenter\rule\theta.5\linewidth\theta.5pt\endcenter

 (१) 'हरतष्टे धनक्षेपे' इत्यादि भास्करोक्तानुरूपमेतत् ।

( २१८) 

 उदाहरणम्। 

 को राशिः खाभ्रदिङ्निघ्नो\\
दिगश्विनयनैर्युतः ।\\
हीनो वा त्रीन्द्रसम्भक्तः\\
शुध्यति ब्रूहि तं पृथक् ॥ २४ ॥ 

 न्यासः।

 भा १००० क्षे २२१० हा १४३ अत्र प्रग्वज्जाते गुणाप्ती ६५।४७०।\\
भाज्ये हरेण तष्टे जातः -भा १४२ क्षे २२१०हा१४३ जाते गुणाप्ती
६५।८०\\
अत्र गुणः स एव ६५ । लब्धिस्तु ८० भाज्यतक्षणफल ६ घ्नेन\\
गुणकेन ३९० संयुता जाता ४७० । 

 अथवा हरतष्टे क्षेपे भा १००० क्षे ६५ हा १४३ जाते गुणाप्ती
६५।४५५\\
अत्रापि गुणः स एव । लब्धिः क्षेपतक्षणलब्ध्या १५ युता जाता\\
सैव ४७० । 

 अथवा भाज्यक्षेपयोर्हरतष्टयोर्न्यासः भा १४२ क्षे ६५ हा १४३ जाते\\
गुणाप्ती ६५।६५ भाज्यतक्षणफलं ६ गुणः ६५ अनयोर्हतिः ३९०\\
क्षेपतक्षणफलम् १५ अनयोर्योगः ४०५ अनेन लब्धिः ६५ युता जाता\\
सैव ४७० ।\\
द्वितीय न्यासः भा १००० क्षे २२१० हा १४३ जाते प्राग्वद्गुणाप्ती
७८।५३०\\
हरतष्टे क्षेपे भा १००० क्षे ६५ हा १४३ जाते गुणाप्ती ७८।५४९\\
क्षपतक्षणफलोना जाता लब्धिः सैव ५४५ ।

 ( २१९) 

 सूत्रम् । 

 १क्षयभाज्ये गुणलब्धो\\
धनवत् साध्ये तु भाज्यतः क्षेपे ॥ २८ ॥\\
अल्पे तयोः क्षयं स्या-\\
देकमनल्पे तु ते सकृद्धनगे ॥ २९ ॥  

 उदाहरणम् । 

 त्रयत्रिंशद्धतो राशिस्त्रिभिर्युक्तोऽथवोनितः ।\\
सभक्तो निरग्रः स्यात् तं गुणं वद वेत्सि चेत् ॥ २५ ॥ 

 न्यासः । भा ३० क्षे ३ हा ७ भाज्यं धनं प्रकल्प्य धनभाज्ये धन-\\
क्षेपे गुणाप्ती २।९ एते एव स्वतक्षणाभ्यां शोधिते धनभाज्ये\\
ऋणक्षेपे गुणाप्ती ५।२१ एवमृणभाज्ये धनक्षेपे गुणाप्ती २।९ वा\\
५।२१ एवमेवर्णभाज्यॠणक्षेपे गुणाप्ती २।९ वा ५।२१ । 

 अपि च । 

 क्षयत्रिंशद्धतः सप्तनवत्योनो युतोऽथवा ।\\
सप्ताप्तः शुद्धिमायाति तं गुणं वद मे द्रुतम् ॥ २६ ॥ 

 न्यासः । 

 भा ३० क्षे ९७ हा ७ धनवत् साध्ये इति प्राग्वज्जाते गुणाप्ती
४।३१\\
एतयोरेकमृणमिति लब्धमृणं प्रकल्प्य ऋणभाज्ये धनक्षेपे धनात्मके

\begincenter\rule\theta.5\linewidth\theta.5pt\endcenter

 (१) अत्रालापेन वासना स्फुटा ।

( २२०) 

 गुणाप्ती ३।१ अथवा ऋणगुणके कल्पिते ऋणभाज्ये धनक्षेपे\\
गुणाप्ती ४।३१ इष्टघ्नतक्षणयुते इत्येकेनेष्टेन जाते ते एव ३।१ 

 क्षयगतहारेऽप्येवमूह्यम् । 

 सूत्रम् । 

 १हरतः शुद्धे क्षेपे शून्ये जातेऽथवा गुणः खं स्यात् ।\\
शून्ये तु भाज्यराशौ हारत्दृतः क्षेपको लब्धिः ॥  ३० ॥

उदाहरणम् । 

 को राशिः सप्तहतो\\
नवभिर्युक्तोऽथवोनितः शुद्धिम् ।\\
त्रिभिरुद्धृतः प्रयच्छति\\
भागं तं गुणकमाचक्ष्व ॥ २७ ॥

 न्यासः । भा ७ क्षे ९ हा ३ जाते गुणाप्ती ०।३ एकेनेष्टेन ३।१०\\
द्विकेन ६।१७ नवशुद्धौ गुणाप्ती ३।४ एकेनेष्टेन ६।११ द्विकेन ९।१८।

 अपि च । 

 को राशिर्नवगुणितः\\
शून्ययुतः पञ्चभिर्हृतः शुद्धम् ।\\
भागं यच्छति राशिं\\
तं गणक ब्रूहि यदि वेत्सि ॥ २८ ॥

\begincenter\rule\theta.5\linewidth\theta.5pt\endcenter

 (१) 'क्षेपाभावोऽथवा यत्र' इत्यादि भास्करोक्तानुरूपमेवेदम् ।

( २२१) 

 न्यासः। भा ९ क्षे ० हा ५ जाते गुणाप्ती ०।० एकेनेष्टेन ५।९\\
द्विकेन १०।१८।

 अपि च । 

 को राशिः शून्यहतो\\
द्वादशयुक्तो विवर्जितो वाऽपि ।\\
चतुरुद्धृतो विशुद्धध्यति\\
तं गुणकं गणक मे कथय ॥ २९ ॥ 

 न्यासः । भा ० क्षे १२ हा ४ जाते द्वादशक्षेपे गुणाप्ती ०।३ वा
४।३\\
वा ८।३ द्वादशशुद्धौ जाते ४।३ वा ८।३। 

 भाज्ये शून्ये लब्धिः सर्वत्राविकृतैव (गुणकोऽपि शून्यानन्तवर्जं\\
सर्वोऽप्यभिन्नाङ्कः सम्भवति) । 

 सूत्रम् । 

 १क्षेपं शुद्धिं रूपं\\
परिकल्प्य तयोः पृथग् गुणाप्ती ये।\\
इष्टक्षेपविशुद्ध्या\\
हते स्वहरतक्षिते भवतः ॥ ३१ ॥ 

\begincenter\rule\theta.5\linewidth\theta.5pt\endcenter

 (१) 'रूपं विशुद्धिं परिकल्प्य चैव पृथक् तयोर्ये गुणकार-\\
लब्धी' इत्यादि भास्करोक्तानुरूपमेवेदम् ।

( २२२) 

 प्रथमोदाहरणे दृढाः भा २१ क्षे ७ हा १९ रूपं क्षेपं परिकल्प्य
न्यासः\\
भा २१ क्षे १ हा १९ रूपक्षेपे गुणाप्ती ९।१० इष्टक्षेप ७ गुणिते
६३।७०\\
स्वहारतष्टे ६।७ जाते सप्तक्षेपे । रूपशुद्धौ गुणाप्ती १०।११ इष्ट-\\
शुद्धि ७ गुणिते ७०।७७ स्वहारतष्टे जाते सप्तशुद्धौ १३।१४। 

 सूत्रम् । 

 १आद्यो हारो हारं\\
परो विभाज्यं प्रकल्प्य पूर्वाग्रम् ।\\
त्यक्त्वा पराग्रतस्त-\\
च्छेषं क्षेपं च तल्लब्ध्या ॥ ३२ ॥
गुणितः प्रथमो हारः\\
साग्रोऽग्रं भाज्यताडितस्तु हरः ।\\
सोऽस्याद्यः स्यादेवं\\
तदग्रमपरोऽपि राशिः स्यात् ॥ ३३ ॥ 

\begincenter\rule\theta.5\linewidth\theta.5pt\endcenter

 ( १) अत्रोपपत्तिः । कल्प्यते प्रथमहारः= हा१ । द्वितीयो\\
हारः= हा२ ।\\
प्रथमशेषम्= शे१ । द्वितीय शेषम्= शे२ राशिमानम्= या ।\\
तदा प्रश्नानुसारेण\\
या= क. हा१ । शे१\\
= नी. हा२ + शे२\\
∵  का = नी . हा२ +( शे२ - शे१) / हा१

( २२३)

 उदाहरणम् । 

 द्व्यग्रस्त्रिहृतस्त्र्यग्र-\\
श्चतुराप्तः पञ्चत्दृच्चतुष्काग्रः ।\\
पञ्चाग्रः षड्भक्तो\\
यस्तं कथयाशु मे गणक ॥ ३० ॥

 न्यासः । 

 शे २ शे ३ शे ४ शे ५\\
हा ३ हा ४ हा ५ हा ६ अत्राद्यो हारो हारः ३ परो विभाज्यः\\
४ आद्यशेषं २ परशेषाद् ३ अपास्य शेषम् १ क्षेपः । कुट्टकार्थं\\
न्यास: भा ४ क्षे १ हा ३ जाते गुणप्ती २।३ लब्ध्या ३ प्रथमहारं ३
सङ्गुण्य\\
९ आद्यशेषेण २ युते जातं शेषम् ११ । हरयो ३।४ र्धातो हरः\\
१२ इति जाते हरशेषे शे ११

 हा १२ । पुनः शेषं ११ परशेषादस्माद् ४\\
अपास्य शेषम् ७ प्रागवत् कुट्टकः भा ५ क्षे ७ हा १२ जाते गुणाप्ती
११।४\\
लब्ध्या ४ दृढहरमिमं १२ सङ्गण्य ४८ आद्यशेषेण ११ युते जातं\\
शेषम् ५९ इति हरशेषे शे ५९  

 हा ६० पुनः शेषं परशेषादस्माद् ५ अपास्य 

\begincenter\rule\theta.5\linewidth\theta.5pt\endcenter

 अत्र कुट्टकविधिना लब्धिः= ल= का ।\\
वा का= पी. हा२ ल१ ( 'इष्टाहतस्वस्वहरेण युक्ते' इत्यादिना\\
यदि इ=पी१) 

 उत्थापनेन या= पी. हा१. हा२ +हा१. ल + शे१
अतो नवीन आद्यो हारः= हा१. हा२ तच्छेषं च\\
= हा१. ल +शे१ आभ्यामाद्यहारशेषाभ्यामपरहारशेषाभ्यां च\\
पूर्ववत् क्रिया कर्त्तव्या ।

( २२४)

शेषं क्षेपः ५४ पुनः कुट्टकः भा ६ क्षे ५४ हा ६० अतो दृढाः भा १
क्षे ९ हा १०
जाते गुणाप्ती ९।० पुनर्लब्ध्यानया० दृढहरं १० सङ्गुण्य० आद्यशेषेण\\
५९ युतं जातं शेषम् ५९ हरयो १०।६ र्घातो हर इति जाते हरशेषे\\
शे ५९ हा६० ऊर्ध्वो राशिर्भवति । अधः स्थितः प्रक्षेपो भवति । एवं\\
जातौ क्षेपकराशी क्षे ६० रा ५९ शून्यगुणं प्रक्षेपकं प्रक्षिप्य जातो\\
राशिः ५९ । एकगुणं प्रक्षिप्य जातः ११९ । द्विगुणम् १७९ ।\\
इत्यनेकधा राशिः स्यात् । 

 अपि च । 

 को राशिश्चतुरूनः\\
सप्तविभक्तस्तु शुद्धिमुपयाति ।\\
सप्तयुतो नवभक्त-\\
स्त्र्यूनो दशभाजितः कः स्यात् ॥ ३१ ॥

न्यासः शे ४ । शे ७ । शे ३हा ७ हा ९ हा १० यथोक्तकरणेन जातो
राशिः\\
सक्षेपः क्षे ६३० रा २६३ ।\\
सूत्रम् । 

 १भाज्यं गुणकारोऽग्रं\\
क्षेपं हारो हरं प्रकल्प्याथ । 

\begincenter\rule\theta.5\linewidth\theta.5pt\endcenter

 (१) अत्रोपपत्तिः । कल्प्यते राशिः = या, गुणकाः क्रमेण
गु१,
गु२ ,गु३,... । हाराः क्रमेण हा१ ,हा२ ,हा३,... ।\\
शेषाणि क्रमेण शे१ , शे२ , शे३ ,...।

( २२५) 

 कुट्टकजो यो गुणकः\\
स निजहराग्रं विधिः प्राग्वत् ॥ ३४ ॥

 उदाहरणम् । 

 को राशिर्निधिशैलसायकगुणै-\\
र्निघ्नः पृथग् भाजितो\\
बाणेभेशपुरन्दरैः क्षितिकरा-\\
ग्न्यम्भोधिशेषो भवेत् ।\\
तं राशिं वद कोविदाशु गणका-\\
हङ्कारशैलस्थली-\\
वासिप्रोन्मदकुट्टकज्ञकरिणां\\
जेता नृसिंहोऽसि चेत् ॥ ३२ ॥ 

 न्यासः । शे १ गु ९ हा ५, शे २ गु ७ हा ८, शे ३ गु ५ हा ११,\\
शे ४ गु ३ हा १४ । अत्र गुणकारो भाज्यं, हारो हरमग्रं क्षेपं\\
प्रकल्प्य कुट्टकार्थं न्यासः भा ९ क्षे १ हा ५ । भा ७ क्षे २
 हा ८ । भा ५ क्षे ३ हा ११ ।
भा ३ क्षे ४ ह १४ । अत्र जाता गुणकाः ४।६।५।६ एतान्यग्राणि । एषा-

\begincenter\rule\theta.5\linewidth\theta.5pt\endcenter

 तदा प्रश्नानुसारेण गु१या-शे१ /हा१ अयं निरग्रः । अत्र
गुणको\\
यावत्तावन्मानम् वा य=हा१ र + गु 

 द्वितीयालापे -गु२ हा१र + गु२ गु-शे२/ हा२ अयं
निरग्रः ।\\
अतः द्वितीयगुणकेन हतः प्रथमहारो भाज्यः । इति पूर्व-\\
सूत्रोक्तविधिर्भवतीति स्पष्टम् ।\\
१५

( २२६)

मधो हारान् विन्यस्य जातम् शे ४ हा ५ । शे ६ हा ८ । शे
५ हा ११ । शे ६ हा १४ ।\\
'आद्यो हारो हार-' इत्यादिना जातो राशिः २४१४ क्षे ३०८० । 

 सूत्रम् । 

 १प्राग्वद्राशिः साध्य-\\
स्तच्छेषहरौ समीरितहराप्तौ ।\\
तल्लब्धं प्रथमः स्या-\\
दुद्दिष्टहराग्रगो द्वितीयश्च ॥ ३५ ॥\\
ताभ्यां कुट्टकलब्ध्या\\
राशिहरस्ताडितो निजाग्रयुतः ।\\
परहरगुणितो हारो\\
मुहुर्विधिश्चैवमन्येषु ॥ ३६ ॥

\begincenter\rule\theta.5\linewidth\theta.5pt\endcenter

 ( १ ) अत्रोपपत्तिः । कल्प्यते पूर्वविधिना राशिः= हा. इ + शे ।\\
अयमाद्यहरहृतः प्रथमशेषाग्रः स्यात् । कल्प्यते लब्धिः= हा इ + शे,\\
शेषम्= शे१ अथ हा। इ । शे अयं हा। - हृतः शेषम्= शे१
आद्यहारेण\\
हृतं तदा शेषम्= शे१ । अतोऽस्य प्रथमं शेषम्= शे।, हरः =
हा।,\\
द्वितीयहारः= हा१, द्वितीयशेषम्? शे१ । ततो जातं प्रश्नान्तर को\\
राशिः हा। हृतः शे। - शेषाग्रः, हा१ - हतश्च शे१ -
शेषाग्रः इति ।\\
ततः 'आद्यो हारो हार' इत्यादिना लब्धिः= हा१ इ१ + ल = इ\\
इष्टस्थाने अनेनोत्थापनेन राशिः = इ हा + शे = हा. हा१ इ१ +\\
हा. ल + शे । अतः हा हा१ हारेण हा ल + शे शेषेण च पुनः शेष-\\
हरौ समीरितहराप्तौ तल्लब्धं प्रथमः स्यादित्यादि कर्म द्वितीय-\\
हरशेषाभ्यां कर्त्तव्यम् । एवमसकृद्यावत्सर्वहरसम्बन्धि कर्म भवेत् ।\\
इत्युपपन्नम् ।

( २२७)

 उदाहरणम् । 

 एकाग्रस्त्रिहृतः कः स्यात्\\
त्र्यग्रः पञ्चविभाजितः ।\\
पञ्चाग्रः सप्तभक्तश्च\\
तद्वदेव पृथक् फलम् ॥ ३३ ॥ 

 न्यासः । शे १ हा ३ । शे ३ हा ५ । शे ५ हा ७ ।
'आद्यो हारो हार-' इत्यादिना\\
प्राग्वद्राशिः । शे १०३ हा १०५ । अत्र शेषहरौ समीरितहरेण ३
भक्तौ\\
जातं फलम् । शे ३४ हा ३५ । अयमाद्यः । उद्दिष्टो द्वितीयः । शे
३४ हा ३५ शे १ हा ३ ।\\
'आद्यो हारो हार' इति कुट्टकार्थं न्यासः । भा ३ क्षे ३३ हा ३५ ।\\
गुणाप्ती ११।० लब्ध्यानया ० राशिहरः १०५ ताडितः ० निजा-\\
ग्रेण १०३ युतः १०३ परहरः ३ अनेन हराग्रं १०५ गुणितो जातो\\
हरः ३१५ एवं जातो राशिः शे १०३ हा ३१५ । पुनः पञ्चहृतः फलं शे
२० हा ६३ ।\\
अयमाद्य उद्दिष्टो द्वितीयः शे २० हा ६३ ।शे ३ हा ५
प्राग्वत् कुट्टकः भा ५ क्षे १७ हा ६३\\
जाते गुणाप्ती ४७। १ लब्ध्यानया १ राशिहरोऽयं ३१५ सङ्गुण्य\\
स्वाग्र १०३ युते जातः ४१८ परहरेण ५ हरोऽयं ३१५ गुणितो जातो\\
राशिहरः, १५७५ एवं जातो राशिः शे ४१८ हा १५७५ । एवं तृतीयफलम्\\
शे ५९ हा २२५ । शे ५ हा ७ । अतः कुट्टके न्यासः भा ७ क्षे ५४
हा २२५ । गुणाप्ती ७२।२\\
पूर्वज्जातो राशिः शे ३५६८ हा ११०२२५ । एवं जातो राशिः ३५६८ क्षे
११०२५ ।

( २२८)

 अपि च ।

 कौ रामेषुहतौ शराद्रिविहृता-\\
वेकद्विवेकाग्रौ तयो-\\
र्विश्लेषश्चतुराहतो नवहृतः\\
पञ्चाग्रको जायते ।\\
योगोऽपि त्रिगुणश्च सायकहृतो\\
द्व्यग्रः फलैक्यं दशा-\\
ऽभ्यस्तं रुद्रत्दृतं नग्राग्रकमभू-\\
द्राशी सखे तौ वद ॥ ३४ ॥ 

न्यासः । शे १ हा ३ ।शे २ हा ५ । 'भाज्यं
गुणकारोऽग्रमि'त्यादिना जातौ\\
शे २ हा ५ ।शे ६ हा ७ । एतयोस्त्रिपञ्चगुणयोः
पञ्चसप्तभक्तयोः फले शे १ हा ३ ।शे ४ हा ५ ।\\
पुना राश्योरेतयोरन्तरम् शे ४ हा २ । एतच्चतुर्गुणम् शे १६
हा ८ एतन्नव-\\
हृतं पञ्चाङ्गमिति न्यस्तं जातम् शे १६ हा ८ । 'आद्यो हारो हार'
इत्यादिना\\
कुट्टकः भा ९ क्षे ११ हा ८ गुणः २ लब्धिः ३ अनया गुणितं हारमग्रे\\
प्रक्षिप्य जातौ राशी शे १२ हा ४५ ।शे २० हा ६३ । योगे फले
वा शे ७ हा २७ ।शे १४ हा ४५ ।\\
अन्तरफलम् शे ३ हा ८ । शे ३२ हा १०८ । पुना राश्योरेतयोः
शे १२ हा ४५ । शे २० हा ६३ ।\\
योगः शे ३२ हा १०८ । अयं त्रिगुणः शे ९६ हा ३२४ । पञ्चहृतो
द्व्यग्र इति न्यस्तं\\
जातम् शे ९६ हा ३२४ ।शे २ हा ५ । प्राग्वत् कुट्टकार्थं
न्यासः भा ५ क्षे ९४ हा ३२४ जाते

( २२९)

गुणाप्ती २७८।४ लब्ध्या गुणितं हरमग्रे प्रक्षिप्य प्राग्वज्जातौ राशी,\\
फलानि, योगफलं, सर्वफलैक्यं क्रमेण, शे १९२ हा २२५ । शे २७२
हा ३१५ । शे ११५ हा १३५ ।\\
शे १९४ हा २२५ ।शे ३५ हा ४० ।शे २७८ हा ३२४ ।शे
६२२ हा ७२४ । एतद् दशगुणितमेकादश-\\
भक्तं सप्ताग्रमिति न्यस्तं जातम् शे ६२२० हा ७२४० ।शे ७ हा
११ । प्राग्वत् कुदृ-\\
कार्थं न्यासः भा ११ क्षे ६२११ हा ७२४० । जाते गुणाप्ती १।१२२३
लब्ध्या\\
गुणितं हरमग्रे प्रक्षिप्य जातौ राशी शे ४१७ हा २४७५ ।शे ५८७
हा ३४६५ । फलानि\\
च क्रमात् शे २४० हा १४८५ ।शे ४१९ हा २४७५ ।शे ७५ हा
४४० ।शे ६०२ हा ३४६५ । सर्वत्र हारः\\
प्रक्षेपकः कार्यः । इष्टेन शून्येन गुणितं प्रक्षेपमग्रराशौ प्रक्षिप्य
जातौ\\
राशी ४१७।५८७ एकेनेष्टेन २८९२।४०५२ द्विकेन ३३६७।७५१७ एव-\\
मिष्टवशादनेकधा । 

 सूत्रम् । 

 १तुल्येऽग्रेऽग्रं राशिः\\
प्रक्षेपः कृतसमानहारः स्यात् । 
उदाहरणम् ।\\
 राशिः सखे सागरतर्कनाग-\\
रन्ध्रैर्विभक्तोऽपि निरग्रकः स्यात् । 

\begincenter\rule\theta.5\linewidth\theta.5pt\endcenter

 ( १) अत्रोपपत्तिः । यदाग्राणां साम्यं तदा हराणां समच्छेदः क्षेपः\\
प्रथमो राशिः शेषमेव । अर्थात् तदेष्टवशात् इ. समहा + शे अय-\\
मेव राशिः स्यात् । यतोऽत्र प्रथमखण्डं सर्वहरैर्निःशेषं भवति\\
समच्छेदत्वात् द्वितीयखण्डं शे-समं सर्वत्र शेषमिति स्पष्टम् ।

( २३१) 

 यो गुणकः सैवेच्छा\\
या लब्धिस्तत्फलं भवति ॥ ३९ ॥ 

 उदाहरणम् । 

 पङ्गुर्योजनषष्टिमेकसहिता -\\
मद्बैस्त्रिपञ्चाशता\\
रिङ्गन् क्रामति योजनानि च किय-\\
त्सङ्ख्यानि येनाऽसरत् ।\\
कालेनाशु वदार्य तत्र घटिका-\\
शेषे भवेद् विंशति-\\
स्तत्संवत्सरमासवासरघटी\\
मानानि चेच्छां पृथक् ॥ ३६ ॥ 

 न्यासः ६१।५३ घटिका शेषम् २० अत्र घटिकानां षष्ट्या दिन-\\
मिति षष्टिर्भाज्यः, प्रमाणं हारः, घटिकाशेषं शुद्धिरिति प्रकल्प्य\\
न्यासः भा ६० क्षे २० हा ६१ । जाते गुणाप्ती ४१।४० लब्धिर्घटिका ४०
गुणो\\
दिनशेषम् ४१ । दिनत्रिंशता मास इति त्रिंशद् भाज्यो, दिनशेषं शुद्धि-\\
रिति न्यासः । भा ३० क्षे ४१ हा ६१ जाते गुणाप्ती ४०।१९
लब्धिर्दिनानि\\
१९ गुणो मासशेषम् ४० । द्वादशभिर्मासैर्वर्षमिति द्वादशभाज्यो,\\
माससेषं शुद्धिरिति न्यासः । भा १२ क्षे ४० हा ६१ । गुणाप्ती ४४।८
गुणो\\
वर्षशेषं,लब्धिर्मासाः ८ । त्रिपञ्जाशद् भाज्यो, वर्षशेषं शुद्धिरिति

( २३०) 

 रूपाग्रको वा युगलाग्रको वा\\
राशिं समाचक्ष्व तमाशु मे त्वम् ॥ ३५ ॥

 न्यासः शे ० हा ४ ।शे ० हा ६ ।शे ० हा ८ ।शे
० हा ९ । समहृतहरसङ्गुणिताव-\\
न्योन्यहरौ हताविति जाताः समहाराः ७२।७२।७२ अत्राग्रं राशिः ०\\
प्रक्षेपः ७२ । द्वितीयोदाहरणे राशिः १ प्रक्षेप ७२ । तृतीयोदाहरणे\\
राशिः २ प्रक्षेपः ७२ । इष्टवशादनेकधा । 

 परिभाषितम् । 

 यस्मिन् यस्मिन् कर्मणि\\
यद् यत् परिभाषितं समुदितं च ॥ ३६ ॥
तस्मिँस्तस्मिन् कर्मणि\\
तत् तत् परिभाषितं भवति । 

 सूत्रम् । 

 १त्रैराशिके प्रमाणं\\
हारः परिभाषितोन्मितिर्भाज्यः ॥ ३७ ॥
यो गुणकः सैवेच्छा\\
या लब्धिस्तत्प्रमाणं स्यात् ।\\
गुणकस्तु पूर्वशेषं\\
तत्पूर्वं पूर्वमेवमपि ॥ ३८ ॥
अनुपातेच्छायाम-\\
प्यज्ञातायां च तत्फलं भाज्यः । 

\begincenter\rule\theta.5\linewidth\theta.5pt\endcenter

 (१) इदं 'कल्प्याथ शुद्धिविकलावशेषम्' इत्यादि भास्करप्रकार-\\
वदेव । उदाहरणन्यासविलोकनेन सर्वं स्पष्टम् ।

( २३१) 

 यो गुणकः सैवेच्छा\\
या लब्धिस्तत्फलं भवति ॥ ३९ ॥ 

 उदाहरणम् । 

 पङ्गुर्योजनषष्टिमेकसहिता -\\
मद्बैस्त्रिपञ्चाशता\\
रिङ्गन् क्रामति योजनानि च किय-\\
त्सङ्ख्यानि येनाऽसरत् ।\\
कालेनाशु वदार्य तत्र घटिका-\\
शेषे भवेद् विंशति-\\
स्तत्संवत्सरमासवासरघटी\\
मानानि चेच्छां पृथक् ॥ ३६ ॥ 

 न्यासः ६१।५३ घटिका शेषम् २० अत्र घटिकानां षष्ट्या दिन-\\
मिति षष्टिर्भाज्यः, प्रमाणं हारः, घटिकाशेषं शुद्धिरिति प्रकल्प्य\\
न्यासः भा ६० क्षे २० हा ६१ । जाते गुणाप्ती ४१।४० लब्धिर्घटिका ४०
गुणो\\
दिनशेषम् ४१ । दिनत्रिंशता मास इति त्रिंशद् भाज्यो, दिनशेषं शुद्धि-\\
रिति न्यासः । भा ३० क्षे ४१ हा ६१ जाते गुणाप्ती ४०।१९
लब्धिर्दिनानि\\
१९ गुणो मासशेषम् ४० । द्वादशभिर्मासैर्वर्षमिति द्वादशभाज्यो,\\
माससेषं शुद्धिरिति न्यासः । भा १२ क्षे ४० हा ६१ । गुणाप्ती ४४।८
गुणो\\
वर्षशेषं,लब्धिर्मासाः ८ । त्रिपञ्जाशद् भाज्यो, वर्षशेषं शुद्धिरिति

( २३३) 

 क्षिप्त्योर्घातः क्षेपः\\
स्याद् वज्राभ्यासयोर्विशेषो वा ।\\
ह्रस्वं लघ्वोर्घातः\\
प्रकृतिघ्नो ज्येष्ठयोश्च वधः ॥ ४ ॥
तद्विवरं ज्येष्ठपदं\\
क्षेपः क्षिप्त्योः प्रजायते धातः ।\\
ईप्सितवर्गविभक्तः\\
क्षेपः क्षेपः पदे तदिष्टाप्तौ ॥ ५ ॥
गुणिते वा तन्मूले\\
गुणिते मूले तदा भवतः ।\\
इष्टकृतिगुणकशेषो-\\
द्धृतं तदिष्टं द्विसङ्गुणं भवति ॥ ६ ॥
१ह्रस्वं मूलं च ततो\\
रूपं क्षेपेण साधयेज्ज्येष्ठम् ।\\
तुल्यातुल्यपदानां\\
भावनयाऽनन्तमूलानि ॥ ७ ॥ 

 उदाहरणम् । 

 अष्टाहता यस्य कृतिः सरूपा\\
स्यान्मूलदा ब्रूहि सखे ममाशु । 

\begincenter\rule\theta.5\linewidth\theta.5pt\endcenter

 (१) एतत् सर्वं श्रीभास्करोक्तानुरूपमेव ।

( २३४)

 एकादशघ्नी यदि वा कृतिः का\\
वर्गत्वमेत्येकयुता सुचिन्त्य ॥ १ ॥ 

 न्यासः प्रकृतिः ८ क्षेपः ९ । अत्राभिष्टह्रस्वं मूलं रूपं कल्पितम्\\
१ अस्य वर्गः १ प्रकृतिगुणः ८ रूपयुतः ९ अस्य मूलम् ३ एतज्ज्ये-\\
ष्ठमूलम् । क्रमेण न्यासः क १ ज्ये ३ क्षे १ एषामधस्तान्न्यसेदिति\\
भावनार्थं न्यासः । प्र ८ क १ ज्ये ३ क्षे १

 क १ ज्ये ३ क्षे १ \ 'वज्राभ्यासौ ह्रस्वज्येष्ठ-\\
कयोः' --- इति प्रथमकनिष्ठद्वितीयज्येष्ठयोरभ्यासः ३
प्रथमज्येष्ठद्वितीय-\\
कनिष्ठयोरभ्यासः ३ अनयोः संयुतिः ६ ह्रस्वं भवेत् । लघु १।१ घातः\\
१ प्रकृतिहतः ८ ज्येष्ठवधेन ९ युतो ज्येष्ठपदं भवेत् । क्षिप्त्योर्घातः-\\
क्षेपः १ । क्रमेण न्यासः क ६ ज्ये १७ क्षे १ । 'तुल्यातुल्यापदानां भाव-\\
नयाऽनन्तमूलानि' इत्यसमभावनार्थं न्यासः प्र ८ क १ ज्ये ३ क्षे १ क ६
ज्ये १७ क्षे १\\\
समासभावनया जाते मूले-\/-क ३५ ज्ये ९९ क्षे १ । पुनर्भावनार्थं\\
न्यासः- प्र ८ क १ ज्ये ३ क्षे १ क ३५ ज्ये ९९ क्षे १ \ समासभावनया
जाते मूले\\
क २०४ ज्ये ५७७ क्षे १ । एवमनन्तमूलानि । 

 अथवा कनिष्ठमूलं रूपद्वयं कल्पितं क २ । अस्य वर्गः ४ प्रकृति\\
८ हतः ३२ चतुः क्षेपयुतो ३६ मूलं ६ ज्येष्ठम् । क्रमेण न्यासः\\
क २ ज्ये ६ क्षे ४ । 'ईप्सितवर्गविभक्तः क्षेप' इति रूपक्षेपार्थं
कल्पितमिष्टं\\
रूपद्वयं २ अस्य वर्गः ४ अनेन हृतः क्षेपो ४ लब्धं क्षेपः १ । इष्ट-\\
द्वयेन २ हृते मूले रूपक्षेपमूले । क १ ज्ये ३ क्षे १ एभ्यो भावनया\\
तान्येव मूलानि भवन्ति । 

 द्वितीयोदाहरणे न्यासः । प्र ११ क्षे १ रूपमिष्टं कनिष्ठं १ तद्वर्गः\\
प्रकृतिगुणो द्यूनो मूलं ज्येष्ठम् ३ न्यासः प्र ११ क १ ज्ये ३ क्षे २
क १ ज्ये ३ क्षे २ \

( २३५)

समासभावनया जाते मूले-क ६ ज्ये० २० क्षे ४ । 'ईप्सितवर्गहृत-\\
इति रूपक्षेपमूले-क ३ ज्ये १० क्षे १ । अतः समासभावनया जाते\\
मूले--- क ६० ज्ये १९९ क्षे १ । अथवा रूप-पञ्चकक्षेपमूले -\\
क १ ज्ये ४ क्षे ५ । समासभावनया जाते पञ्चविंशतिक्षेपमूले-\/-\\
क ८ ज्ये २७ क्षे २५ । अतो रूपक्षेपमूले -क८/५ ज्ये २७/५ क्षे १ ।\\
अनयोः पूर्वकल्पिताभ्यामाभ्यां-क ३ ज्ये १० क्षे १ समासभावनया\\
जाते मूले क १६१/५ ज्ये ५३४/५ क्षे १ । एवमनन्तमूलानि । अथवा

न्यासः । प्र ११ क ३ ज्ये १० क्षे १ क ८/५ ज्ये २७/५ क्षे १ \
अन्तरभावनया जाते मूले---
क १/५ ज्ये ६/५ क्षे १ । एवमनन्तमूलानि । 

 'इष्टकृतिगुणकशेषोद्धृत'-मिति रूपक्षेपपदाभ्यां पुनः पुनः\\
समासविशेषभावनाभिर्मूलान्यनन्तानि भवन्ति । तद्यथा । प्रथमो-\\
दाहरणे रूपत्रयमिष्टं प्रकल्प्य यथोक्तकरणेन जातं कनिष्ठम् ६,\\
अस्य वर्गात् ३६ प्रकृतिगुणाद् २८८ रूपयुताद् २८९ मूलं ज्येष्ठम्\\
१७ । रूपपञ्चकेष्टेन जातं कनिष्ठम् १०/१७ । अतो ज्येष्ठम् -३३/१७ । अनयोः\\
पूर्वमूलाभ्यामाभ्यां- क ६ ज्ये १७ । समासभावनया जाते मूले-\\
क ३६८/१७ ज्ये १०४१/१७ । अथ वा विशेषभावनया जाते मूले- 

क २८/१७ ज्ये ८१/१७ क्षे १ । एवं द्वितीयोदाहरणे रूपत्रयेणेष्टेन जाते
मूले-

क ३ ज्ये १० । पञ्चकेन-\/-क५७ ज्ये १८/७ । अनयोः पूर्वमूलाभ्यां
समास-

( २३६) 

भावनयाजाते मूले-क - १०४/७ ज्ये ३४५/७ । अन्तरभावनया
मूले---
क ४/७ ज्ये १५/७ क्षे १ । एवमनन्तमूलानि ।

 एकद्विचतुष्कक्षेपसाधनाय चक्रवाले करणसूत्रमार्याचतुष्टयम् ।

 १ह्रस्वबृहत्प्रक्षेपान्\\
भाज्यप्रक्षेपभाजकान् कृत्वा ।\\
कल्प्यो गुणो यथा त-\\
द्वर्गात् संशोधयेत् प्रकृतिम् ॥ ८ ॥

 प्रकृतेर्गुणवर्गे वा\\
विशोधिते जायते तु यच्छेषम् ।\\
तत क्षेपत्दृतं क्षेपो\\
गुणवर्गविशोधिते व्यस्तम् ॥ ९ ॥
लब्धिः कनिष्ठमूलं\\
तन्निजगुणकाहतं वियुक्तं च ।\\
पूर्वाल्पपदपरप्रक्षि-\\
प्त्योर्घातेन जायते ज्येष्ठम् ॥ १० ॥
प्रक्षेपशोधनेष्व-\\
प्येकद्विचतुर्ष्वभिन्नमूले स्तः ।\\
द्विचतुः क्षेपपदाभ्यां\\
रूपक्षेपाय भावना कार्या ॥ ११ ॥

\begincenter\rule\theta.5\linewidth\theta.5pt\endcenter

 (१) अत्रोपपत्तिः। मज्जनकमुद्रितश्रीभास्करबीजगणितस्य पृष्ठानि\\
५६-५९ द्रष्टव्यानि ।

( २३७)

उदाहरणम् । 

 कस्त्र्युत्तरेण गुणितोऽत्र शतेन वर्गः\\
सैकः कृतित्वमुपयाति वदाऽऽशु तं मे ।\\
को वा त्रिवर्जितशतेन हतस्तु वर्गो\\
रूपान्वितः कृतिगतो भवति प्रचक्ष्व ॥ २ ॥

 न्यासः । प्रकृतिः १०३ क्षेपः १ । प्राग्वद् रूपत्रयशुद्धौ मूले\\
क १ ज्ये १० क्षे ३ अत्र ह्रस्वपदं भाज्यं ज्येष्ठपदं क्षेपं क्षेपं हारं
प्रकल्प्य\\
कुट्टकार्थं न्यासः ।भा १ क्षे १० हा ३ । कुट्टककरणेन जातो गुणः २
इष्ट-\\
रूपेण त्रयेण जातोऽपरो गुणः ११ । अस्य वर्गात् १२१ प्रकृति- १०३\\
मपास्य शेषं १८ क्षेपेण ३ हृतं जातः क्षेपः ६ । लब्धिः ७ कनिष्ठ-\\
मूलम् । एतत् ७ निजगुणकेन ११ हतं ७७ पूर्वह्रस्वपदं १ परक्षेपः ६\\
अनयोर्घातेन ६ वियुक्तं जातं ज्येष्ठम् ७१ । ऋणधनमूलयोरुत्तर-\\
कर्मणि क्रियमाणे न विशेषः । तस्मादृणमूलयोर्धनत्वं प्रकल्प्य षट्-\\
शोधने-प्र १०३ क ७ ज्ये ७१ क्षे ६ । पुनः कुट्टकार्थं न्यासः\\
भा  ७ क्षे ७१ हा ६ । जातो गुणः सक्षेपः गु १ जे क्षे ६
ऋणरूपेष्टेन जातो-\\
ऽपरो गुणः ७ । अस्य वर्गं प्रकृतेरपास्य शेषं ५४ गुणवर्गविशोधिते\\
व्यस्तमिति जातमृणम् ५४ । क्षेपेण ६ हृतं जातः क्षेपः ९ । लब्धिः

\begincenter\rule\theta.5\linewidth\theta.5pt\endcenter

नूतनज्येष्ठम् = प्र.क+इ.ज्ये/क्षे = प्र.क+इ.ज्ये +
इ२क-इ२क/क्षे\\
= इ(इक + ज्ये ) - क( इ२-प्र)/क्षे = इ(इक + ज्ये )/क्षे-क(
इ२-प्र)/क्षे\\
= इ.नूक-क. नूक्षे । इत्युपपन्नं नूतनज्येष्ठानयनम् । शेषं\\
श्रीभास्करोक्तिवज्ज्ञेयमिति ।

( २३८) 

कनिष्ठमूलम् २० । एतन्निजगुणकाहतं १० पूर्वह्रस्वपरक्षेपघातः ६३,\\
अनेन वियुक्तं जातं ज्येष्ठम् २०३ । पूर्ववत् प्र १०३ क २० ज्ये २०३\\
क्षे ९ । कुट्टकः । भा २० क्षे २०३ हा ९ । जातो गुणः २ एकेनेष्टेन\\
जातोऽपरो गुणः ११ । अस्य वर्गात् प्रकृतिमपास्य शेषम् १८ । क्षेपेण\\
हृतं क्षेपः २ । लब्धिः कनिष्ठम् ४७ । एतन्निजगुणकहतम् ५१७ ।\\
पूर्वपद २० परक्षेप २ घातेनाऽनेन ४० वियुक्तं ४७७ जातं ज्येष्ठम् ।\\
प्रकृतिः १०३ क ४७ ज्ये ४७७ क्षे २ 'प्रक्षेपशोधनेष्वप्येक-\\
द्विचतुर्ष्वभिन्नमूले स्तः' इत्यादिना समासभावनार्थंन्यासः

प्र १०३ क ४७ ज्ये ४७७ क्षे २ \ समासभावनया चतुःक्षेपमूले\\
क ४७ ज्ये ४७७ क्षे २ क ४४८३८ ज्ये ४५५०५६ क्षे ४

 अतो रूपक्षेपमूले क २२४१९ ज्ये २२७५२८ क्षे १ ॥ 

 द्वितीयोदाहरणे । प्रकृतिः ९७ क १ ज्ये १० क्षे ३ । प्राग्वत् कुट्टकः\\
भा १ क्षे १० हा ३ । जातो गुणः २ । धनरूपत्रयेणेष्टेन जातोऽपरो\\
गुणः ११ । अस्य वर्गात् प्रकृतिमपास्य शेषं २४ क्षेपहृतं क्षेपः ८ ।\\
लब्धिः कनिष्ठमूलम् ७ । अतो ज्येष्ठम् ६९ । एवम्-प्र ९७ क ७\\
ज्ये ६९ क्षे ८ । पुनः । भा ७ क्ष ६९ हा ८ । जातो गुणः ५ धन-\\
रूपेणैकेनेष्टेन जातोऽपरो गुणः १३ । अस्य वर्गात् प्रकृतिमपास्य\\
शेषं ७२ क्षेपहृतं क्षेपः ९ । लब्धिः कनिष्ठपदम् २० । अतो ज्येष्ठम्\\
१९७ । प्र ९७ क २० ज्ये १९७ क्षे ९ । कुट्टकेन लब्धो गुणः ५ ।\\
धनरूपेण जातोऽपरः १४ । अस्य वर्गात् १९६ प्रकृतिमपास्य शेषं\\
९९ क्षेपहृतं क्षेपः ११ । लब्धिः कनिष्ठपदम् ५३ अतो ज्येष्ठम्\\
५२२ । प्र ९७ क ५३ ज्ये ५२२ क्षे ११ । कुट्टकेन जातो गुणः ८ ।\\
अस्य वर्गं प्रकृतेरपास्य शेषम् ३३ । 'गुणवर्गविशोधिते व्यस्तम्'\\
इति जातमृणम् ३३ क्षेपहृतं क्षेपः ३ं । लब्धिः कनिष्ठम् ८६\\
अतो ज्येष्ठम् ८४७ । प्र ९७ क ८६ ज्ये ८४७ क्षे ३ । कुट्टकेन जातो

( २३९) 

गुणः १ । ऋणरूपत्रयेण जातोऽपरो गुणः १० । अस्य वर्गात्\\
प्रकृतिमपास्य शेषं क्षेपेण हृतं क्षेपः १ । लब्धिः कनिष्ठम् ५६९ ।\\
अतो ज्येष्ठम् ५६०४ । धनत्वऋणत्वे चोत्तरकर्मणि क्रियमाणे न\\
विशेष इति जाते धनगते रूपशुद्धिमूले । क ५६९ ज्ये ५६०४\\
क्षे १ । समासभावनया जाते रूपक्षेपमूले । क ६३७७३५२ ।\\
ज्ये ६२८०९६३३ क्षे १ । 

 सूत्रम् । 

 १रूपविशुद्धौ प्रकृतिः\\
कृतियोगः स्यान्नचेत् खिलं तु तदा ।\\
अखिलप्रकृतौ प्राग्वत्\\
साध्ये मूले बृहत्स्वल्पे ॥ १२ ॥ 

 उदाहरणम् । 

 कस्त्रयोदशनिघ्नश्च\\
वर्गो व्येकः पदप्रदः ।\\
को वर्ग एकषष्ठिघ्नो\\
निरेको मूलदो वद ॥ ३ ॥

 प्रथमोदाहरणे द्विकत्रिकयोर्वर्गयोगः । रूपशुद्धौ मूले १/३ ।२/३ ।\\
चक्रवालेनाभिन्ने ५।१८ 

 द्वितीयोदाहरणे षट्कपञ्चकयोर्वर्गयोगः प्रकृतिः ६१ । प्राग्वत्\\
पञ्चविंशतिशुद्धौ मूले क १ ज्ये ६ क्षे २५ । अतो रूपशुद्धौ- १/५ ।

 ( १) 'रूपशुद्धौ खिलोद्दिष्टं-' इति भास्करोक्तानुरूपमिदम् ।

( २४०) 

६/५ । अथ वा षट्त्रिंशतिशुध्दौ मूले । क १ ज्ये ५ क्षे ३६ । अतो रूप-\\
शुद्धौ १/६ । ५/६ । चक्रवालेनाऽभिन्ने क ३८०५ ज्ये २९७१८ क्षे १ ।\\
एवमनन्तमूलानि । 

 अपि च । 

 वर्गः पञ्चगुणः कश्चि-\\
च्चतुर्भिः संयुत कृतिः ।\\
षट्त्रिंशताऽथ वा युक्तः\\
शतयुक्तोऽथवा भवेत् ॥ ४ ॥ 

 प्रकृतिः ५ क १ ज्ये ३ क्षे ४ । 'गुणिते मूले तदा भवतः' इति\\
त्रिभिर्गुणिते जाते षट्त्रिंशत्क्षेपमूले । क ३ ज्ये ९ क्षे ३६ पञ्च-\\
भिर्गुणिते शतक्षेपे मूले क ५ ज्ये १५ क्षे १०० । एवं बुद्धिमता\\
विशोधने मूले ज्ञेये 

सूत्रम्।

 १प्रकृत्रिभीप्सितवर्गो-\\
द्धृता यथा शुद्धिमेति यल्लब्धम् ।\\
कल्प्यो गुणः कनिष्ठं\\
छेदनमूलोद्धृतं भवति ॥ १३ ॥ 

 उदाहरणम् । 

 द्वासप्ततिप्रगुणिता कृतिरेकयुक्ता\\
मूलप्रदा भवति मे वद मित्र शीघ्रम् । 

\begincenter\rule\theta.5\linewidth\theta.5pt\endcenter

 (१) 'वर्गच्छिन्ने गुणे ह्रस्वं तत्पदेन विभाजितम् ।' इति भास्क-\\
रोक्तानुरूपमेवेदम् ।

( २४१)

पञ्चांशकेन गुणितोऽप्यथवा सरूपो\\
वर्गः कृतित्वमुपयाति सखे विचिन्त्य ॥ ५ ॥

 प्रथमोदाहरणे प्रकृतिः ७२ ईप्सितवर्गेण ९ विहृता शुद्धा लब्ध-\\
मियं प्रकृतिः ८ । क १ ज्ये ३ क्षे १ । अत्र कनिष्ठं छेदनमूलेनानेन ३\\
लब्धं कनिष्टम् १/३ । एवं जाते ह्रस्वज्येष्ठे १/३।३ 

 द्वितीयोदाहरणे प्रकृतिः १/५ । इयं पञ्चांशवर्गेण १/२५ हृता विशुद्धा\\
लब्धमियं प्रकृतिः ५ । प्राग्वद्रूपक्षेपे मूले । क ४ ज्ये ९ क्षे १ ।\\
कनिष्ठं छेदनमूलेनाऽनेन १/५ हृतं जातं कनिष्ठम् २० । एवं जाते\\
ह्रस्वज्येष्ठे -२०।९ 'तुल्यातुल्यपदानां भावनयाऽनन्तमूलानि' 

 वर्गगतायां प्रकृतौ सूत्रम् । 

 १क्षिप्तिरभीष्टविभक्ता\\
द्विधा तदिष्टोनरसंयुता दलिता ।\\
आद्या प्रकृतिपदाऽऽप्ता\\
क्रमशोऽल्पाऽनल्पमूले ते ॥ १४ ॥

 उदाहरणम् । 

 वर्गो नवहतः कश्चिद्\\
दशाढ्यो वा दशोनितः ।\\
मूलदो जायते तं मे\\
गणितज्ञ वद द्रुतम् ॥ ६ ॥ 

\begincenter\rule\theta.5\linewidth\theta.5pt\endcenter

 ( १) 'इष्टभक्तो द्विधा क्षेपः' इत्यादि भास्करोक्तानुरूपमेवेदम् ।\\
१६

( २४२) 

 प्र ९ क्षे १० । अत्र क्षिप्तिः १० द्विधैकेनेष्टेन हृता तदिष्टोनयुता\\
दलिता ९/२ । ११/२ अनयोराद्या प्रकृतिपदेनाऽनेन ३ हृता जाते मूले\\
३/२ । ११/२ । द्विकेनेष्टेन मूले १/२ । ७/२ पञ्चकेन १/२ । ७/२ 

 द्वितीयोदाहरणे प्रकृतिः ९ । प्राग्वदेकेनेष्टेन मूले ११/६ । ९/२
द्विकेन\\
७/२ । ३/२ एते धनमूले वा भवतः । एवमनन्तमूलानि । रूपक्षपप-\\
दाभ्यां समासान्तरभावनाभिर्मूलान्यनन्तात्युत्पद्यन्ते ।\\
प्रकृतिसमक्षेपविशुद्धावुदाहरणम् । 

 का कृतिर्दशभिः क्षुण्णा\\
दशाढ्या वा दशोनिता ।\\
मूलदा जायते विद्वँ -\\
स्तान् द्रुतं वद वेत्सि चेत् ॥ ७ ॥ 

 प्रकृतिः १० क्षे १० । अत्र दशशुद्धौ मूले १।० 'इष्टकृतिगुणकशे-\\
षोद्धृते' इति त्रिकेनेष्टेन रूपक्षेपमूले ६।१९ आभ्यां सह समास-\\
भावनया जाते क १९ ज्ये ६० क्षे १ । अन्तरभावनया जाते मूले\\
ते एव १९।६० । द्वितीयोदाहरणे प्रकृतिः १० क्षे १० । प्राग्वद्दशशुद्धे\\
मूले १।० रूपशुद्धिपदाभ्यामाभ्यां-क १ ज्ये हे क्षे १ । समासभाव-\\
नयाऽन्तरभावनया च जाते मूले, क ३ ज्ये १० क्षे १० । 

 आप च । 

 ऋणपञ्चहतो वर्गो\\
विंशत्या सैकया युतः ।

( २४३) 

 कृतित्वं याति तं ब्रूहि\\
जानासि प्रकृतिं यदि ॥ ८ ॥ 

 प्र ५ क्षे २१ । अत्र जाते ह्रस्वज्येष्ठे १।४ वा २।१ 

 सूत्रम् । 

 प्रक्षेपेषु बहुषु वा\\
शुद्धेषु च निजधिया पदे ज्ञेये ।\\
रूपक्षेपाय तयो-\\
र्भावनयाऽनन्तमूलानि ॥ १५ ॥
यस्य न बुद्धिः स्वान्ते\\
न गणितलेशोऽपि तस्य स्यात् ।\\
तस्मान्निजया बुद्ध्या\\
समूह्यमखिलं तु गणितमिदम् ॥ १६ ॥

 उदाहरणम् 

 कस्त्रयोदशसंनिघ्नो\\
वर्गः सप्तदशाधिकः ।\\
वर्जितो वा पृथङ्मूल-\\
प्रदः स्याद्वद मित्र तम् ॥ ९ ॥

 प्र १३ के १७ । अत्र रूपत्रयक्षेपमूले क १ ज्ये ४ क्षे ३ । अत्र\\
बुद्धिः । क्षेपगुणं क्षेप प्रकल्प्य प्रकृति १३ क्षे ५१ । अत्रैपञ्चाशत्\\
क्षेपमूले, क १ ज्ये ८ क्षे ५१ । अनयोः पूर्वमूलाभ्यां समासभावनया\\
त्रिपञ्चाशदधिकशतक्षेपे मूले, क १२ ज्ये ४५ क्षे १५३ । 'ईप्सित-

( २४४)

वर्गविहृतः क्षेपः' इति येन सप्तदशसङ्ख्यः क्षेपो भवति तथा\\
कल्पित इष्टरूपत्रितयवर्गः ९ । अनेन हृतः क्षेपः १७ । यदेतदिष्टाप्ते\\
इति त्रिभक्ते सप्तदशक्षेपमूले । क ४ ज्ये १५ क्षे १७ । अन्तरभाव-\\
नया प्राग्वज्जाते सप्तदशक्षेपमूले क ४/३ ज्ये १९/३ क्षे १७ । 

 द्वितीयोदाहरणे न्यासः प्र १३ क्षे १७ । प्राग्वज्जाते सप्तदशक्षेपे\\
मूले । क ४ ज्ये १५ क्षे १७ । रूपशुद्धिमूलाभ्यामाभ्यां- क ५ ज्ये १८\\
क्षे १ । समासभावनया जाते मूले, क १४७ ज्ये ५३० । अन्तरभाव-\\
नया जाते क ३ ज्ये १० जे १७ एवमनन्तमूलानि । 

 अमूल्यराशेरासन्नमूलानयनार्थं सूत्रम् । 

 १मूलं ग्राह्यं यस्य च\\
तद्रुपक्षेपजे पदे तत्र ।\\
ज्येष्ठं ह्रस्वपदेन च\\
समुद्धरेन्मूलमासन्नम् ॥ १७ ॥ 

 उदाहरणम् । 

 दशानामपि रूपाणां\\
पञ्चमांशस्य वा वद । 

\begincenter\rule\theta.5\linewidth\theta.5pt\endcenter

 ( १) द्रष्टव्या भास्कराचार्यबीजोपरि मज्जनककृता टिप्पणी ।\\
एतादृशं सूत्रं नारायणीबीजेऽपि । गणकतरङ्गिण्यां भ्रमात्\\
मुनीश्वरगुरुनारायणकृतं बीजगणितं लिखितं वस्तुतः काशिक-\\
राजकीयपुस्तकालये यत्खण्डितं बीजपुस्तकमस्ति तदस्यैव नारा-\\
यणस्य तत्रापि अस्य सूत्रस्य सत्त्वात् ।

( २४५) 

 आसन्नमूलं जानासि\\
चेत् क्रियां प्रकृतेः सखे ॥ १० ॥ 
अत्र रूपक्षेपमूले, क ६ ज्ये १६ क्षे १.वा २२८।७२१ वा ८६५८।\\
२७३७९ अल्पेनानल्पमुध्दरेदिति मूलमासन्नम् १९/६ वा ७२१/२२८ वा
२७३७९/८६५८

 द्वितीयन्यासः । प्र १५ । अत्र रूपक्षेपमूले २७।९ वा १६१।३६०

अत्रासन्नमूलम् १/३ । १६१/३६० इत्यादि ।

 इति श्रीसकलकलानिधिनरसिंहनन्दनगणितविद्याचतुरानन-\\
नारायणपण्डितविरचितायां गणितपाट्यां' कौमुद्याख्यायां वर्गप्रकृ-\\
तिर्नाम दशमोऽध्यायः समाप्तः । 

  अथ भागादानविधिः प्रारभ्यते । 

 अथ गणकानन्दकरं\\
भागादानस्य कौतुकं वक्ष्ये ।\\
ज्ञाते यस्मिन् सपदि\\
सामान्यो जायते गणकः ॥ १ ॥
असकृद् विभजेद् द्वाभ्यां\\
समराशिं यावदेति वैषम्यम् ।\\
सत्सु प्रथमस्थाने\\
पञ्चसु भाज्ये च पञ्चभिश्छिन्द्यात् ॥ २ ॥

( २४५) 

 आसन्नमूलं जानासि\\
चेत् क्रियां प्रकृतेः सखे ॥ १० ॥ 
अत्र रूपक्षेपमूले, क ६ ज्ये १९ क्षे १.वा २२८।७२१ वा ८६५८।\\
२७३७९ अल्पेनानल्पमुद्धरेदिति मूलमासन्नम् १९/६ वा ७२१/२२८ वा
२७३७३/८३५८

 द्वितीयन्यासः । प्र १५ । अत्र रूपक्षेपमूले २७।९ वा १६१।३६०

अत्रासन्नमूलम् १/३ । १६१/३६० इत्यादि ।

 इति श्रीसकलकलानिधिनरसिंहनन्दनगणितविद्याचतुरानन-\\
नारायणपण्डितविरचितायां गणितपाट्यां' कौमुद्याख्यायां वर्गप्रकृ-\\
तिर्नाम दशमोऽध्यायः समाप्तः । 

  अथ भागादानविधिः प्रारभ्यते । 

 अथ गणकानन्दकरं\\
भागादानस्य कौतुकं वक्ष्ये ।\\
ज्ञाते यस्मिन् सपदि\\
सामान्यो जायते गणकः ॥ १ ॥
असकृद् विभजेद् द्वाभ्यां\\
समराशिं यावदेति वैषम्यम् ।\\
सत्सु प्रथमस्थाने\\
पञ्चसु भाज्ये च पञ्चभिश्छिन्द्यात् ॥ २ ॥

( २४६) 

 न समो भाज्यः प्रथमः\\
तस्मिन् यदि पञ्चकं स्थाने ।\\
अच्छेद्याः कल्प्यन्ते\\
त्रिसप्तकैकादशादयश्छेदाः ॥ ३ ॥
यावच्छेदप्राप्ति-\\
स्तावद् हरसाधनं क्रियते ।\\
भाज्यो वर्गश्चेत् त-\\
न्मूलं छेदो द्विधा भवति ॥ ४ ॥
अपदप्रदस्तु भाज्यः\\
कयेष्टकृत्या युतात् पदं भाज्यात् ।\\
पदयोः संयुतिवियुती\\
हारौ परिकल्पितौ भाज्यौ ॥ ५ ॥
राश्योस्तु तयोः प्राग्वत्\\
कुर्वीतच्छेदशोधनं सुधिया ।\\
अपदप्रदस्य राशेः\\
पदमासन्नं द्विसङ्गुणं सैकम् ॥ ६ ॥\\
मूलावशेषहीनं\\
वर्गश्चेत् क्षेपकश्च कृतिसिद्ध्यै ।\\
वर्गो न भवेत् पूर्वा-\\
सन्नपदं द्विगुणितं त्रिसंयुक्तम् ॥ ७ ॥

( २४७)

 आद्याद्युत्तरवृद्ध्या\\
तावद् यावद् भवेद् वर्गः ।\\
असमानां पूर्वहताः\\
परे पुरःस्थास्तथा चाऽन्ये ॥ ८ ॥
तुल्यानां पूर्वघ्नः\\
परः पृथक् तेऽन्यहरनिघ्नाः ॥८ ऽऽ ॥

\begincenter\rule\theta.5\linewidth\theta.5pt\endcenter

 अत्रासकृत्कर्मणि कृते कस्यापि भाज्यमानम्\\
= रा=२.न१ ३.न२ ५. न३  एवं भवति । 

 अतस्तस्य निःशेषकरा हराः= २,२२,... ,३,३२,.. .,२.३,
२.३२...\\
यस्य राशेः प्रथमस्थानीयोऽङ्कः पञ्चसमः स राशिः पञ्चभि-\\
र्निःशेषो भवतीति स्पष्टम् । यदि प्रथमो भाज्यो राशिः समो न\\
तथा स्थाने प्रथमस्थाने पञ्चकमपि यदि न तदा त्रिसप्तैका-\\
दश-इत्यादयोऽच्छेद्या दृढा राशयो भाज्यस्य छेदा हराः कल्प्यन्ते ।\\
मूलं छेदो द्विधा भवतीति स्फुटम् । कल्प्यते भाज्य + इ२
=आ२
तथा भाज्य= आ२ - इ२ =( आ + इ) ( आ - इ) । 

 अत एको हारः= अ२ + इ । द्वितीयश्च= आ - इ ।\\
अतः आ + इ, आ - इ, एतौ भाज्यौ परिकल्प्य अनयोर्हाराः\\
पूर्ववद्विचार्याः ।\\
कल्प्यते अपदप्रदभाज्यराशेरासन्नं पदम्= प, शेषम्= शे ।\\
तदा भा= प२ + शे\\
अथ यदि इ२ = २प + १ - शे\\
तदा द्वयोर्योगेन भा + इ२= ( प + १)२ = आ२
अतस्तदा वर्गकरणार्थम् इ२=२प + १ - शे अयं क्षेपः ।

( २४८)

 उदाहरणम्। 

 स्तम्बेरमाम्बुधिवियत्करसम्मितोऽयं\\
राशिर्विशुद्धिमुपयाति विभाजितो यैः । 

\begincenter\rule\theta.5\linewidth\theta.5pt\endcenter

यदि २ प+ १ - शे अयं वर्गो न तदा यदि\\
२ प +( गु + १ ) + ( गु + १)२ - शे\\
= २ प + २ प गु + गु२ + २ गु + १ - शे\\
= गु ( २ प + गु + २) + २ प + १ - शे\\
= गु (४ प+२गु + ४/२) +२ प + १ - शे\\
= गु \२प+३+२गु-२+२ प +३/२\ + २प +१ -शे\\
= गु \२प+३+२प+३+२(गु-१)/२\\\
+२प+१-शे । अयं वर्गस्तदा\\
भा=प२ + शे\\
इ२= २प(गु +१) + (गु + १)२ -शे

= गु \मु+मु+२ ( गु - १) \ + २ प + १ - शे\\
(यदि २प +३ = मुखम् वा आदिः\\
२=चयः वा वृद्धिः)\\
∴भा+इ२=प२ +२प( गु + १ ) + ( गु + १ )२
=(प+गु+१)२
अत उपपन्नम् ।

( २४९)

तान् ब्रूहि मे गणक मङ्क्षु∗ शराक्षिचन्द्र-\\
रामोन्मितः कथय तान् विहृतोऽथवा यैः ॥ १ ॥
प्रथमोदाहरणे राशिः २०४८ अत्र 'असकृद् विभजेद् द्वाभ्यां सम-\\
राशिंः इति द्वाभ्यां विभज्य जातो राशिः १०२४ । पुनर्द्वाभ्यां विभज्य\\
जातः ५१२ । पुनः २५६, १२८, ६४,३२० १६,८,४,२, १ अयं विष-\\
मोऽच्छेद्यः । लब्धहराणं यथाक्रमं न्यासः २।२।२।२।२।२।२।२।२।२।२\\
'तुल्यानां पूर्वघ्नः परः' इति जाता हराः २।४।८।१६।३२।६४।१२८।२५६।\\
५१२।१०२४।२०४८ 

 द्वितीयोदाहरणे न्यासः । ३१२५ अत्र प्रथमस्थाने पञ्चकं वर्तते ।\\
'पञ्चभिश्च्छिन्द्यात्' इति पञ्चभिर्विभक्तो राशिः ६२५ । पुनः १२५,\\
२५,५, १ अयमच्छेद्यः । लब्धहराणां यथाक्रमं न्यासः ५।५।५।५।५\\
'तुल्यानां पूर्वघ्नः परः' इति जाता हराः ५।२५।१२५।६२५।३१२५ 

 अपि च । 

 व्योमाक्षिबाणशैलास्ते\\
यैः शुद्ध्यन्ति विभाजिताः ।\\
तान् वदेन्द्वभ्रयुग्माभ्र-\\
चन्द्रा यैस्तान् प्रवेत्सि चेत् ॥ २ ॥

 प्रथमन्यासः । ७५२० अयं समरूपो वर्तत इति द्वाभ्यां विभज्य\\
जातं ३७६० पुनः १८८०,९४०,४७०,२३५, अस्य प्रथमस्थाने पञ्चकं\\
वर्ततेऽतः पञ्चभिर्विभज्य लब्धिः ४७ । लब्धहराणां यथाक्रमं न्यासः 

\begincenter\rule\theta.5\linewidth\theta.5pt\endcenter

 ∗ मंक्षु=शीघ्रम्, मंक्षु सपदि द्रुते इत्यमरः ।

( २५०)

२।२।२।२।२, ५ । छिन्नशेषम् ४७ । अयं न समः । नचाऽस्य प्रथम-\\
स्थाने पञ्च । अतः\\
'अच्छेद्याः कल्प्यन्ते त्रिसप्तकैकादशादयश्छेदाः' इति तेषा-\\
मच्छेद्यानां दर्शनम् । न्यासः ३।७।११।१३।१७।१९।२३।२९।३१।३७।४१।\\
४७।५३।५९।६१।६७।७१।७३।७९।८३।८९।१०१।१०३।१०७।१०९।११३।१२७।\\
१३१। इत्यादिषु छिन्नशेषेणु राशिं विचार्य ज्ञेयश्छेदः । लब्धहराणां\\
यथाक्रमं न्यासः २।२।२।२।२।५।४७ असमहरयोरेतयोः ५।४७ पूर्वघ्नः\\
पर इति जाताश्छेदाः ५।४७।२३५, तुल्यानामेषां २।२।२।२।२, पूर्वघ्नः\\
पर इति जाताश्छेदाः २।४।८।१६।३२, पृथगन्यहरगुणिता इति अनेन\\
५ गुणिताश्छेदाः १०।२०।४०।८०।१६०, पुनरनेन ४७ गुणिता जाता\\
हराः ९४।१८८।३७६।७५२। १५०४, पुनरनेन २३५ गुणिता जाता\\
हराः ४७०।९४०।१८८०।३७६०।७५२०; लब्धहराणां यथाक्रमं न्यासः\\
२।४।५।८।१०।१६।२०।३२।४०।४७।८०।९४।१६०।१८८।२३५।३७६।४७०।७९०।\\
९४०।१५०४।३७६०।७५९० 

 द्वितीयोदाहरणे न्यासः । १०२०१ । अयं वर्गो वर्तत इत्यस्य\\
मूलं द्विधा हरौ १०१।१०१ एतौ भाज्यौ प्रकल्प्य पुनर्हरसाधनं\\
प्राग्वत्कुर्यादित्येतावच्छेद्यौ । तयोः सदृशत्वात् पूर्वघ्नः पर इति\\
जातौ छेदौ १०१।१०२०१ 

 अपि च । 

 चन्द्राङ्गभूभुवो भक्ता\\
यैर्विशुद्धिं प्रयान्ति तान् ।\\
ब्रूहि त्वं वेत्सि चेद् भा-\\
गादानं गणितकोविद ॥ ३ ॥ 

 न्यासः ११६१ । अस्याऽऽसन्नमूलम् ३४, एतद् द्विगुणं सैकम्\\
६९, वर्गशेषेणानेन ५ ऊनमयं ६४ वर्गो वर्तत इत्यनेन

( २५१)

भाज्यराशिः ११६१ युतो जातो वर्गः १२२५ । वर्गयोर्मूले ८।३५\\
अनयोः संयुतिवियुती छेदाविति जातौ छेदौ ४३।२७ । एतावेव\\
भाज्यौ प्रकल्प्य पुनर्हरसाधनं क्रियते । त्रिचत्वारिंशतेस्त्रिचत्वा-\\
रिंशदेव हरः ४३ । सप्तविंशतैरासन्नमूलं ५ द्विगुणं सैकं ११ मूला-\\
वशेषेणानेन २ ऊनं जातो वर्गः ९ । एतद्भाज्ये प्रक्षिप्य जातो वर्गः\\
३६ । वर्गयोर्मूले ३।६ अनयोः संयुतिवियुती छेदाविति जातौ ९।३\\
एतौ भाज्यौ परिकल्प्यौ । त्रयाणां त्रय एव हरः । नवानां मूलं\\
द्विधा ३।३ लब्धहराणां यथाक्रमं न्यासः । ३।३।३।४३ तुल्यानां पूर्वघ्नः\\
पर इति जाता हराः ३।९।२७ एतेऽन्यहारगुणिताः १२९।३८७।११६१\\
एषां यथाक्रमं न्यासः ३।९।२७।४३।१२९।३८७।११६१ 

 अपि च । 

 सहस्रं रूपसंयुक्तं\\
यौर्विभक्तं विशुद्ध्यति ।\\
तान् वदाऽऽशु तवाऽलं चेद्\\
भागादानेऽस्ति पाटवम् ॥ ४ ॥

 न्यासः १००१ । अस्यासन्नमूलं ३१ द्विगुणं सैकं ६३ वर्गशेषेणा-\\
नेन ४० ऊन-२३ मेतद् वर्गो न स्यात् । वर्गसाधनायाऽस्मिन् २३\\
पूर्वासन्नपदं ३१ द्विसङ्गुणं ६२ त्रिसंयुक्तम् ६५ । 'आद्याद् द्युत्तर-\\
वृद्ध्या तावद् यावद् भवेद् वर्गः' इति न्यस्ते जातम् ६३।६५।६७।६९।\\
७१।७३।७५।७७।७९।८१।८३।८५।८७।८९ एषां योगे जातो वर्गः १०२४ ।\\
अनेन भाज्यराशिः १००१ युतो जातो वर्गः २०२५ । वर्गयोर्मूले\\
३२।४५ । अनयोः संयुतिवियुती ७७।१३ सप्ततेरासन्नमूलं ८\\
द्विसंगुणं १६ सैकं १७ वर्गशेषेणानेन १३ ऊनम् ४ अयं वर्गः । अनेन\\
भाज्यो ७७ युतो वर्गः ८१ । वर्गयोर्मूले २।९ संयुतिवियुती ११।७\\
लब्धहराणां यथाक्रमं न्यासः ७।११।१३ प्रथमो द्वितीयतृतीयाभ्यां

( २५१)

गुणितः ७७।९१ द्वितीयस्तृतीयेन गुणितः १४३ प्रथमद्वितीयतृतीय-\\
हराणां बधः १००१ लब्धहराणां यथाक्रमं न्यासः ७।११।१३।७७।९१।\\
१४३।१०००१ 

अपि च ।

 व्योमलोचनरसाब्धयः सखे\\
यैर्हृताः समुपयान्ति शुद्धताम् ।\\
तान् वदाऽऽशु यदि विद्यते तव\\
प्रौढिरत्र गणिते निराकुला ॥ ५ ॥

 न्यासः । ४६२० अयं समरूपो द्वाभ्यामसकृद्विभज्य जातः ११५५\\
पञ्चहृतः २३१ । लब्धहराणां यथाक्रमं न्यासः २।२।५।२३१ अथास्या-\\
तन्नमूलम् १५ द्विगुणं ३० सैकं ३१ वर्गशेषेणानेन ६ ऊनं जातो वर्गः\\
२५ अमुं भाज्ये प्रक्षिप्य जातो वर्गः २५६ वर्गयोर्मूले ५।१६ संयुति-\\
वियुती २१।११ एतौ भाज्यौ प्रकल्प्यैकादशानामेकादशैव हरः ।\\
एकविंशतौ रूपद्वयवर्गं प्रक्षिप्य २५ जातो वर्गः । मूले २।५ संयुति-\\
वियुती ७।३ जातौ छेदौ लब्धहराणां यथाक्रमं न्यासः २।२।३।५।७।११\\
तुल्यानां पूर्वघ्नः पर इति जातौ २।४ असमाः ३।५।७।११ एषां प्रथमं\\
द्वितीयादिभिः संगुण्य जाताः १५।२१।३३ द्वितीयं तृतीयचतुर्थाभ्यां\\
३५।५५ तृतीयं चतुर्थेन ७७ असमानां सर्वेषां बधश्च ११५५ लब्ध-\\
हराणां यथाक्रमं न्यासः ३।५।७।११।१५।२१।३३।३५।५५।७७।१०५।\\
१६५।२३१।३८५ ।११५५ एतान् पृथक्पृथक्स्थान् पूर्वहराभ्यां २।४\\
गुणयेदिति द्विगुणिताः ६।१०।१४।२२।३०।४२।६६।७०।११०।१५४।२१०\\
।३३०।४६२।७७०।२३१० चतुर्गुणा जाताः १२।२०।२८।४४।६०।८४।१३२\\
।१४०।२२०।३०८।४२०।६६०।९२४।१५४०।४६२० क्रमेण न्यस्ता जाताः\\
२।३।४।५।७।१०।११।१२।१४।१५।२०।२१।२२।२८।३०।३३।३५।४२।४५।५५।

( २५३)

६०।६६।७०।७७।८४।१०५।११०।१३२।१४०।१५४।१६५।२१०।२२०।२३१।\\
३०८।३३०।३८५।४२०।४६२।६६०।७७०।९२४।१११५।१५४०।२३१०।४६२०

 अपि च । 

 शैलाक्षिनन्दरामायै-\\
र्भाजिताः स्युर्निरग्रकाः ।\\
तानञ्जसा मम ब्रूहि\\
गणितज्ञोऽसि चेत् सखे ॥ ६ ॥ 

 न्यासः । ३९२७ सर्वत्रेष्टकृत्या युतात् पदं भाज्यात्, पदयोः\\
संयुतिवियुती छेदाविति सिद्धम्, यस्य वर्गेण भाज्यो युतो मूलप्रदः\\
स्यात् तथा कल्पितानीष्टानि १३।४७।८३।१०७।१७३।२७७।६५३, प्रथमे-\\
ष्टवर्गादस्मात् १६९ जातौ छेदौ ३।१३०९ अत्र त्रयमच्छेद्यः ३ पुनरिमं\\
१३०९ भाज्यं प्रकल्प्य हरसाधनं क्रियते । अत्र कल्पितानीष्टानि\\
३०।५४।९० प्रथमेष्टाज्जातौ छेदौ ११।७ लब्धहराणां यथाक्रमं न्यासः\\
३।७।११।१७ प्राग्वज्जाता हराः ३।७।११।१७।२१।३३।५१।७७।११९।१८७।\\
२३१।३५७।५९१।१३०९।३९२७ एवमितरैरिष्टैरप्येत एव हराः संभवन्ति । 

 अथाऽन्यथा लघूपायेन हरसाधनाय सूत्रम् । 

 १इष्टोनासन्नपदं\\
हारः स्यादिष्टवर्गशेषयुतिः ॥ ९ ॥

\begincenter\rule\theta.5\linewidth\theta.5pt\endcenter

 ( १) अत्रोपपत्तिः ।\\
भा = प२ + शे = प२ - इ२ + इ२ + शे\\
= (प + इ ) (प - इ) + इ२ + शे\\
∴ भा/प-इ = प+इ + इ२+शे / प-इ ।

( २५४) 

 हारत्दृता चेच्छुद्ध्यति\\
तेनाऽवश्यं त्दृतो भाज्यः ।\\
न विशुद्ध्यति चेदिष्टं\\
स्वधिया परिकल्पयेदन्यत् ॥ १० ॥

 उदाहरणम् । 

 यैः खनेत्रेन्दवो भक्ता\\
यान्ति शुद्धिं वदाशु तान् ।\\
शशिपावकनेत्राणि\\
यैस्तानपि च कोविद ॥ ७ ॥ 

 प्रथमोदाहरणे न्यासः । १२० । अस्यासन्नमूलम् १० इष्टम् २\\
अनेनोनं हारः ८ । इष्टवर्गः ४ मूलशेषम् २० अनयोर्युतिः २४ इयं\\
हारहृता शुद्ध्यति तेन हारेण हृते भाज्येऽवश्यं शुद्धिः स्यात् ।\\
चतुष्केण जातो हरः ६ । पञ्चकेन ५ । षट्केन ४ । अष्टकेन २ ।\\
नवकेन १ । अथवेष्टम् ३ अतो हरः ७ इष्टवर्गः ९ मूलशेषः २०\\
अनयोर्युतिः २९ इयं हारेण हृता न शुद्ध्यत्यतोऽयं हरो न स्यात् ।\\
द्वितीयोदाहरणे राशिः २३१ आसन्नपदम् १५ मूलशेषः ६\\
कल्पितानीष्टानि ४।८।१२ एभिर्जाता हराः ३।७।११ 

 सूत्रम् । 

 इष्टत्दृतगुण्यगुणका -\\
वशेषघातस्तथेष्टहृच्छेषम् । 

\begincenter\rule\theta.5\linewidth\theta.5pt\endcenter

 अतो यदि प - इ अनेन यदि इ२+शे अस्य शुद्धिस्तदा 'भा'\\
अस्यापि प-इ अनेन शुद्धिरिति ।\\
अत्रेष्टं तथा कल्प्यं येनेष्टवर्गयुतशेषस्य प - इ अनेन शुद्धिर्भवेत् ।

( २५५) 

 १तुल्यं चेदिष्टोध्दृति-\\
शेषेण स्यात् स्फुटाऽत्र हतिः ॥ ११ ॥

 उदाहरणम् । 

 एकोनत्रिंशता सप्त-\\
दश सङ्गुणिताः सखे ।\\
इष्टाहतिस्त्रिनन्दाब्धि-\\
तुल्या सा किं स्फुटा वद ॥ ८ ॥ 

 गुण्यगुणकौ २९।१७ त्रिकेनेष्टेन ३ हृतौ शेषे २।२ अनयोर्बधे\\
४ त्रिहृते शेषम् १ । हतिः ४९३ त्रिष्टता शेषम् १। एतत् पूर्वशेषणे\\
सममतो हतिः स्कुटा स्यात् । पञ्चकेन शेषे समे ३।३। अष्टकेन\\
५।५ इत्यादि । 

 इति श्रीसकलकलानिधिनरसिंहनन्दनगणितविद्याचतुरानन\\
नारायणपण्डितविरचितायां गणितपाट्यां कौमुद्याख्यायां भागादानं\\
नामैकादशो व्यवहारः समाप्तः ।

\begincenter\rule\theta.5\linewidth\theta.5pt\endcenter

 ( १) अत्रोपपत्तिः । कल्प्यते गुण्यः= इ.ल१ + शे१\\
गुणकः= इ.ल२  + शे२\\
गुणनफलम्= इ.ल३ + शे३
तदो इ.ल३  + शे३ = (इ.ल१ + शे१) ( इ.ल२  + शे२)\\
=इ२ल\_ल२ + इ ( ल\_शे२  + ल२शे\_)+
शे\_शे२\\
इष्टतष्टे शे३ ( शे- शे२/ इ) एतच्छेषेणे समम् । 

इत्युपपन्नम्

( २५६) 

 अथांशावतारः । तत्र भागप्रभागभागानुबन्धभागाप्रवाहस्वाशा-\\
नुबन्धस्वांशापवाहः षड् जातयः । प्रथमं तावद्भागजातिरुच्यते ।\\
सूत्रम् । 

 १एकाद्येकचयानां\\
द्वयोर्द्वयोर्निकटयोर्बधाश्छेदा ।\\
योऽन्त्यः सोऽन्त्यहरः स्याद्\\
योगे रूपं तदिष्टफलगुणितम् ॥ १ ॥
उदाहरणम् । 

 अंशेन चैकैकमितेषु षट्सु\\
पदेषु हारा वद केऽत्र तेषाम् ।\\
योगे च रूपं परिजायते वा\\
फलं च रूपार्धमपि प्रचक्ष्व ॥ १ ॥

 प्रथमन्यासः १ १ १ १ १ १ फलम् १ । अत्रैकादयः
षट्सु\\
पदेषु कल्पिताः १।२।३।४।५।६ एषां द्वयोर्द्वयोर्निकटयोर्घातजाता-

\begincenter\rule\theta.5\linewidth\theta.5pt\endcenter

 (१) अत्रोपपत्तिः ।\\
यो=१/२+१/२.३+१/३.४+१/४.५+...+१/न(न-१)+१/न\\
=१/१-१/२+१/२-१/३+१/३-१/४+...-१/न+१/न\\
=१/१\\
अतः इ=इ/२+इ/२.३+...+इ/न(न-१) + इ/न

इत्युपपन्नम् ।

( २५७) 

श्छेदाः २।६।१२।२०।३० अन्त्याऽङ्कः ६ अयमन्त्यश्छेदः ९ । एवं रूपफल-\\
भागानां दर्शनम् १/२।१/६।१/१२।१/२०।१/३०।१/६ फलम् १ ।\\
एत एवेष्टफलेनार्धेन गुणिता जाता रूपार्धफलभागाः । दर्शनम्\\
१/४।१/१२।१/२४।१/४०।१/६०।१/१२ फलम् १/२ ।

 अथवा सूत्रम् । 

 ९एकादित्रिगुणोत्तर-\\
वृद्ध्याङ्कस्थानसम्मिताश्छेदाः । 

\begincenter\rule\theta.5\linewidth\theta.5pt\endcenter

 (१) अत्रोपपत्तिः । कल्प्यते\\
योगः= १= अ+१/३ +१/३२+...१/३न-२+क 

 = अ + क + १-१/३न-२x१/३ १-१/३\\
=अ + क + १-१/३न-२/२ = अ + क + 1/2 - १/२x३न-२ 
अत्र यदि क = १/२.३न-२= ३/२.३न-१

 तदा यो = १ = अ +१/२ ∴ अ= १/२

 ततो यो = १ = १/२ + १/३ + १/३२ + १/३२ +... ३ x१/ २x३न-१

 अत उपपन्नम् ।\\
१७

( २५८)

 आद्यन्तौ च द्विगुणा-\\
वन्त्यस्त्रिहतोंऽशके रूपम् ॥ २ ॥

 द्वितीयप्रकारेण रूपफलभागानां दर्शनम् ।१/२ । १/३ । १/९ ।\\
१/२७। १/८१।१/१६२ । फलम् १ । 

 अथवाऽर्धफलभागाः १/४ । १/६ । १/१८ । १/५४ । १/१६२ । १/३२४ ।\\
सूत्रम् ।\\
 १फलहारोऽभीष्टयुतः\\
फलांशभक्तो यथा भवेच्छुद्धिः । 

\begincenter\rule\theta.5\linewidth\theta.5pt\endcenter

 (१) अत्रोपपत्तिः । यदि रूपांशानां भिन्नानां योगः 

 फलेन (=अं/हा) समः स्यादित्यपेक्षितं\\
तदा अं / हा + इ अयं चेद्रूपांशो भिन्नस्तदा\\
कल्प्यते हा+इ/अं=लब्धिः=ल ।\\
अतः अं / हा + इ =१/ल\\
अं/हा = अं/हा\\
फ-अं / हा + इ फ-१/ल=शे\\
∴फ = १/ल + शे ।

 शेषं पुनर्नवीनं फलं प्रकल्प्य 'फलहारोऽभीष्टयुतः' इत्यादिना-\\
ऽस्य खण्डद्वयं १/ल१ + शे१ एतादृशं कार्यम् । पुनरग्रे तथैव कर्म
कर्त्त-\\
व्यम् । एवमभीष्टफलं रूपांशभिन्नानां योगेन समं भवतीति स्पष्टम् ।

( २५९)

 लब्धिश्छेदो भागं\\
फलतः संशोधयेच्च तच्छेषम् ॥ ३ ॥
तस्मादुत्पाद्याऽन्यं\\
शेषमुपान्त्याङ्कशेषं च ।\\
एकैकेष्वंशेषु\\
क्रमोऽयमार्योदितः स्पष्टः ॥ ४ ॥ 

 पूर्वोक्तोदाहरणम् १/० १/० १/० १/० १/० १/० फलम १ । अत्र कल्पितं\\
रुपमिष्टम् १ फलहारः १ इष्टयुतः २ फलांशेन १ हृतो जातः प्रथमः\\
परिच्छेदः १/२ । इमं फलादस्माद् १ अपास्य शेषम् १/२ द्वितीय-\\
मिष्टम् १ फलहारयुतं फलांशभक्तं जातो द्वितीयः परिच्छेदः १/३ ।\\
इमं फलतोऽस्माद् १/२ आपास्य शेषम् १/६ पुनरेकेनेष्टेन जातश्छेदः १/७\\
इमं फलाद् १/६ अपास्य शेषम् १/४२ एकेनेष्टेन जातश्छेदः १/४३ ।\\
इमं फलाद् १/४२ अपास्य शेषम् १/१८०६ एकेनेष्टेन जातश्छेदः १/१८०७ ।\\
इमं फलाद् १/१८०६ अपास्य शेषम् १/३२६३४४२ अयमन्त्यश्छेदः । यथा-\\
कर्मं लब्धछेदानां दर्शनम् 1/२ । १/३ । १/७ । १/४३ । १/१८०७ ।\\
१/३२६३४४२ फलं रूपमेव । 

 उदाहरणम् 

 षडंशकः पञ्चहतो युतिः स्या-\\
च्छेदाश्च ये रूपमितैस्तदंशैः ।

( २६०)

 तच्छेदसंख्याश्च चतुर्ष काः स्यु-\\
र्नवांशकः सप्तहतः फलं वा ॥ २ ॥ 

 न्यासः १/० १/० १/० १/० फलम् ५/६ । इष्टानि ४ । १ । १ एभिर्जातानां
छेदानां दर्शनम् १/२ । १/४ । १/१३ । १/१५६ ।। अथवेष्टेन ४ अनेन जाता-
श्छेदाः १/२ । १/५ । १/८ । १/१२० अथवेष्टानि ९ । ३ । २ एभिर्जाता-\\
श्छेदाः १/३ । १/५ । १/४ । १/२० एवमिष्टवशादानन्त्यम् । 

 द्वितीयोदाहरणे न्यासः १/० १/० १/० १/० फलम् १/९ । इष्टानि ५ । २ ।\\
१ एभिश्छेदाः १/२ । १/४ । १/३७ । १/१३३२ ॥अथवेष्टानि १९।२।२\\
एभिर्जाताश्छेदाः १/४ । १/२। १/३८ । १/१६८४ ॥ एवमिष्टवशाद् बहुधा । 

 सूत्रम् । 

 १परिकल्प्येष्टानङ्का-\\
नाद्यः कन्दाभिधोऽन्तिमोऽग्राख्यः । 

\begincenter\rule\theta.5\linewidth\theta.5pt\endcenter

 ( १) अत्रोपपत्तिः । कल्प्यन्ते इष्ट्राङ्काः क, क१,
क२,...कन
तदोत्क्रमेण, कन, कन-१, कन-२,...क भिन्नाङ्कानां योगः\\
=१/कन + कन -कन-१ /कन -कन-१+कन-१-
कन-२/कन-१कन-२+...+१/क२ क१
अत्र १/कन + कन -कन-१ /कन -कन-१ =१/कन-१

( २६१) 

 निजपूर्वघ्नो हि परोऽ-\\
न्तरं हरांशौ क्रमात् स्याताम् ॥ ५ ॥
अन्त्याग्रच्छेदः स्या-\\
द्रूपं चांशोऽथ तेंऽशकाः सर्वे ।\\
कन्दविनिघ्नास्तेषां 
संयोगो जायते रूपम् ॥ ६ ॥ 

 उदाहरणम् । 

 पदेषु षट्सु संस्थाना-\\
मंशानां जायते युतौ ।\\
रूपं तानाशु मे ब्रूहि\\
यदि वेत्सि सखे द्रुतम् ॥ ३ ॥ 

 अत्र कल्पिता इष्टाङ्काः १।२।३।४।५।6 छेदानां दर्शनम्\\
१/२। १/६ ।१/१२।१/२०।१/३०।१/६ फलम् १ । अथवेष्टा द्व्यादयः २।\\
३। ४। ५।६।७ एभिर्जाता हराः १/३ । १/६ । १/१० । १/१५ । २/७ । 

\begincenter\rule\theta.5\linewidth\theta.5pt\endcenter

 १/ कन-१+ कन-१ -कन-२ /कन-१ -कन-२ =१/कन-२

एवमन्त्ये योगः १/क१ 

 अतो भिन्नाङ्कानां योगः १/क१ । अतस्ते भिन्नांशाः 'क१'\\
अनेन कन्दाख्येन गुणितो योगो रूपसमः स्यादिति ।

( २६२) 

१/२१ फलम् १ । अथवा त्र्यादयः ३ । ४ । ५ । ६ । ७ । ८ एभिर्जातः 

१/४ । ३/२० । १/१० । १/१४ । ३/१४ । ३/८ फलम् १ । एकाद्युत्तरैर्जाता\\
२/१५ । २/३५ ।२/ ६३ ।२/९९ । २/११ फलम् १ । अथवेष्टानि १ । ३\\
२ । ९८/८३ एभिर्जाताः २/३ । ५/२४ । ३/४० । ३/१० । ५१/१४७ धनर्ण\\
वियोग इति फलम् १ ।\\
सूत्रम्। 

 १परिकल्प्यादौ रूपं\\
सांशं परतः परं तदेव स्यात् ।\\
निकटवधस्तुच्छेदाः\\
प्रान्त्यो योऽङ्कः स एव तच्छेदः
उदाहरणम्, । 

 अंशा त्रिकादि द्विचया-\\
श्चतुर्षु स्थानेषु तच्छेदनकाश्च कैश्चित् ।

\begincenter\rule\theta.5\linewidth\theta.5pt\endcenter

 (१) अत्रोपपत्तिः । कल्प्यन्ते अंशाः= अ१, अ२, अ३, अन,\\
कल्प्येष्टानङ्कान्' प्रत्यादिना यदि प्रथममिष्टम्= १, द्वितीया\\
=इ२, इ३ , इ४

 तदा अ१= इ२-१ ∴ इ२=अ१+१,\\
अ = इ३ - इ२ ∴ इ३= अ२ + इ२ ,

एवमंशयोजनेन सर्वाणीष्टानि व्यक्तीभवन्ति इति । ततः\\
कल्प्येष्टानङ्कान् इत्यादिना हरानयनं सुगममिति ।

( २६३)

 संयोजिता येन लवे-\\
न रूपं भवेद्धि तत्राऽथ हरान् वदाशु ॥ ४ ॥
न्यासः ३/० ५/० ७/० ९/० १/० फलम् १ । अत्र 'परिकल्प्यादौ रूपं-'\\
इति कल्पितं रूपम् १ । सांशा जाताः १। ४।९।१६।२५ एषां निकटयो-\\
र्बधाज्जाताश्छेदाः ४।३६।१४४।४०० अन्त्याङ्कः २५ अयमन्त्यश्छेदः ।\\
दर्शनम् ३/४ । ५/३६ ।७/४४।९/४००। १/२५ फलम् १ ॥ एवमेकैकांशकेषु\\
सूत्रम् । 
१उत्पादयोश्च भागाम्\\
युग्ममिते तद्युतौ यथा रूपम् ।\\
तच्छेदहतोद्दिष्टां-\\
शकः परांशाधिकस्तु पूर्वहरः ॥ ८ ॥ 

\begincenter\rule\theta.5\linewidth\theta.5pt\endcenter

(१) अत्रोपपत्तिः । कल्पयन्ते अंशाः= अ१,अ२, अ३, - - -
अ२न
अत्र न-संख्यकं युग्ममानम् ।\\
अतो न-संख्यका रूपांशभिन्नाः पूर्वप्रकारेण उत्पादिताः\\
१/क१, १/क२, १/क३,... १/कन 

आचार्यरीत्याऽभीष्टहरौ अ१क१ + अ२ । क१ (अ१क१
+ अ२)\\
ततो द्वौ भिन्नौ जातौ अ१/अ१क१ + अ२ । अ२/क१
(अ१क१ + अ२)\\
अनयोर्योगः= अ१/अ१क१ + अ२ । अ२/क१ (अ१क१
+ अ२)\\
= अ१क१ + अ२ /क१ (अ१क१ + अ२) =
१/क१

ततः क२ हरेण अ३,अ४ अंशवशेन च द्वौ भिन्नौ भवतो

( २६४)

 सोऽपि हरघ्नस्तु परो\\
हर एवं निखिलयुग्मेषु । 

\begincenter\rule\theta.5\linewidth\theta.5pt\endcenter

ययोर्योगः= १/क२ । त्वमुत्पन्नयोर्द्वयोर्द्वयोर्भिन्नयोर्योगः=
१/क१ +\\
१/क२ +... १/कन = १। एवं समेषु भिन्नांशमानेषु हराणां ज्ञानं\\
भवति । विषमपदेषु विषमस्थानेषु भिन्नभागेषु च यथा भागाः अ१,\\
,अ२, अ३, - - - अ२ न + १ अत्र (न + १) संख्यकं युग्ममानं
प्रकल्प्य\\
पूर्वप्रकारेण उत्पादिता भिन्ना रूपांशाः १/क१, १/क२, १/क३,...
१/कन+१\\
 अत्र १/क१, १/क२, १/क३,... १/कन , एतद्वशेन ये
भिन्नास्तेषां योगः 

= १/क१+१/क२+ ... +१/कन अत्र यदि १/कन+१ अयं वा\\
अ२न+१/अ२न+१(कन+१) अयं योज्यते तदा योगः= १ । अतः साधित-\\
भिन्नेष्वन्तिमो भिन्नोऽयमेव । 

 यद्युत्पादिताभिन्नानां रूपाणि ल१/क१, ल२/क२,
ल३,क३ ....एवं\\
स्युस्तदा साधितच्छेदाः क्रमेण ल१, ल२,...भक्ता अभीष्टच्छेदाः\\
स्युरिति स्फुटम् । यतस्तादृशच्छेदयोर्द्वयोर्द्वयोर्भियोर्योगे\\
ल१/क१, ल२/क२,... एवं भविष्यन्तीति । येषां योगः\\
ल१/क१ + ल२/क२+ ....तु रूपमितो भविष्यतीति ।

( २६५)

 विषमपदेषु तथा प्रां-\\
त्यहरघ्नोद्दिष्टभागश्च ॥ ९ ॥\\
छेदः स्यादन्त्यस्थो\\
निजयुग्मलवैर्हृताश्छेदाः । 

 उदाहरणम् । 

 पृथग् लवास्त्रिप्रमुखा द्विकाधिका-\\
स्तेषां हराः केऽपि पदेषु षट्सु च ।\\
युतौ च रूपं परिजायते कथं\\
पदेषु सप्तस्वपि तत्क्रमेण च ॥ ५ ॥ 

 न्यासः ३/० ५/० ७/० ९/० ११/० १३/० फलम् १ । षट्सु पदेषु युग्मत्रयं\\
वर्तते, युग्ममिते रूपोत्पन्नभागाः १/२ । १/६ । १/३ अत्र प्रथमच्छिदा-\\
ऽनेन २ उद्दिष्टप्रथमयुग्मे प्रथमांशः ३ हतः ६ परांशकेनाऽनेन ५ युतो\\
जातः ११ प्रथमयुग्मे प्रथमच्छेदोऽयम् । अयमुत्पन्नच्छेदेनाऽनेन २\\
हतो द्वितीयः २२ । एवमन्ययोर्युग्मयोर्जाताश्छेदाः ५१ । ३०६ ।\\
४६ । १३८ दर्शनम् ३/११ । ५/२२ । ७/५१ । ९/३०६ । ११/४६ । १३/१३८

 (स्वयुग्मभागैर्लवान् गुणयेत्-इति युग्मप्रथम् १/२ । अस्यांशः\\
१ अनेन प्रथमयुग्मांशाविमौ ३।५ गुणयेत् । एवं सर्वत्राऽन्येषां\\
युग्मानामंशान् गुणयेत् ।) 

 अथ वांऽशत्रययोगो रूपमिति कल्पितास्त्र्यंशाः १/३ । १/३ । १/३\\
एभिः प्राग्वज्जातानां छेदानां दर्शनम् ३/१४ । ५/४२ । ७/३० । १/१० । ११/४६
।\\
१३/१३८ ।

( २६६)

अथवा भागाः २।२।१ कल्पिता इष्टाः १।३।५ एभिर्जाता भागा\\
रूपफलस्य प्राग्वत् स्वभागैर्गुणयेत्-इत्येभिः २।२।१ गुणितेऽपवर्तिते\\
जातम् ३/७ । ५/२१ । ७/५७ । ९/८५५ । ११/६८ । १३/३४ अथवेष्टाः १।५।९ एभि-\\
र्जाताः २/३ । २/१५ । १/५ भागाः ३/५ ३/४५ एभिरपि- ३/५। ५/२५। ७/८१ । ९/३६४५
।\\
११/११२ । १३/१००८ एवमिष्टवशादानन्त्यम् । यद्युद्दिष्टांशछेदयोरपवर्तने\\
कृते तदुद्दिष्टानां विकृतिर्भवति तदा तयोरपवर्तनं न देयम् । 

 द्वितीयोदाहरणे न्यासः ३/० । ५/० । ७/० । ९/० । ११/० । १३/० । १५/०\\
अत्र सप्तसु पदेषु युग्मचतुष्टयं प्रकल्प्य युग्ममिते रूपफले भागाः\\
१/२ । १/६ । १/१२ । १/४ प्राग्वज्जाताश्छेदाः ११,२२,५१,३०६,१४४।१७४०
विषमपदेष्वन्त्यहरेणानेन ४ उद्दिष्टभागो १५ गुणितो जातोऽन्त्य-\\
श्छेदः ६० । अथवा चतुर्थांशानां योगे रूपमिति कल्पिता अंशाः\\
१/४ । १/४ । १/४ । १/४ । एभिर्जाताः
३/२७।५/६८।९/१४८।११/५७।१३/२२८।१५/६०।अथवा युग्मचतुष्टये कल्पिता दृष्ट्यः
१।३।५।७ प्राग्वद् रूपफलभागाः\\
२/३ । २/१५ । २/३५ । १/७ एभिर्येशा जातास्तेषां दशर्नम्३/७ । ५/२१ । ७/५७
।\\
९/५८५ । ११/१९९। १३/६९६५ । १५/१०५ । एवमिष्टवशादानन्त्यम् । 

 सूत्रम्।\\
 १उद्दिष्टांशे प्रथमे\\
फलहारघ्ने परांशसंयुक्ते ।

\begincenter\rule\theta.5\linewidth\theta.5pt\endcenter

 (१) अत्रोपपत्तिः । कल्प्यन्ते उद्दिष्टांशा = अ१, अ२,
भिन्नयोर्योगः\\
=फ=अं/ह

( २६७) 

 फलभागाप्ते व्यग्रे\\
हारः स्यात् फलहरघ्नोऽन्त्यः ॥ १० ॥
शुद्धिर्न भवेद् यदि वा-\\
ऽल्पोंऽशो भाज्ये तथेतरः क्षेपम् ।\\
हारः फलांश इति वा\\
कुट्टकेन सक्षेपका लब्धिः ॥ ११ ॥
छेदः स्यात् फलहारा-\\
दल्पोऽनल्पः फलच्छेदम् ।\\
क्रमशो विभजेद् गुणयेद्\\
यत्र न शुद्धिस्तदेव खिलम् ॥ १२ ॥

\begincenter\rule\theta.5\linewidth\theta.5pt\endcenter

 अ१ह+अं२/अं=प्रथमहारः\\
ह-\/-(अ१ह+अं२/अं)= द्वितीयहरः\\
जातौ भिन्नौ अ१अं/अ१ह+अं२। अं२अं/ह(अ१ह+अं२)\\
योगः= अ१अं ह+अ२अं/ह(अ१ह+अ२)= अं/ह
(अ१ह+अ२/अ१ह+अ२)=अं/ह\\
अत्र यदि अ१ह+अं२/अ अयमभिन्नस्तदैवोद्दिष्टेंऽशे हारमानम् ।\\
कल्प्यते प्रथमहारः अ१इह+अ२/अ अभिन्नस्तदा द्वितीयो

( २६८) 

 उदाहरणम् । 

 ययोरेकांशयोर्योगे\\
विंशांशो जायते सखे ।\\
तच्छेदौ ब्रूहि मे शीघ्र\\
वेत्सि चेदंशकौतुकम् ॥ ६ ॥ 

न्यासः १/०।१/० फलम् १/२० । अत्रोद्दिष्टांशः प्रथमः १ फलहारेणाऽ-\\
नेन २० हतः २० परांशेन १ युतः २१ फलांशेन १ हृतो जातः\\
प्रथमश्छेदः २१ फलच्छेदहतो द्वितीयः ४२० दर्शनम् १/२१ । १/४२० 

\begincenter\rule\theta.5\linewidth\theta.5pt\endcenter

हारः=इह(अ१इ ह+अ२ /अ) आभ्यां भिन्नौ अ१अ/अ१इ ह+अ२।\\
अ२अ/इह(अ१इ ह+अ२ )\\
द्वयोर्योगः= अ(अ१इह+अ२ )/ह(अ१इह+अ२ ) = अ/ह ।\\
अ१ स्थाने अ२ प्रकल्प्यापि तथैव क्रिया भवति । 

अतः अ१ ,अ२ अनयोरल्पं भाज्यमितरं क्षेपं फलांशे हारं\\
प्रकल्प्य कुट्टकेन संक्षेपा लब्धिश्छेदः स्यादिति । एवं यदि लब्धिः\\
\textless ह तदा भिन्नयोर्हरौ ल, हा/ल=ल१ । यदि लब्ध्वा
हारशुद्धिर्न\\
तदोद्दिष्टं खिलमिति । वस्तुतो लब्धिसम्बन्धिगुणको यदा\\
फलहारक्तः शुध्यति तदैव प्रश्नोऽखिलः ।

( २६९) 

अपि च । 

 त्रिसप्तप्रमितावंशौ\\
तद्युतौ सप्तमांशकौ ।\\
तयोश्छेदमितं ब्रूहि\\
जानासि गणितं यदि ॥ ७ ॥

 न्यासः ३/० । ७/० फलम् २/५ । यथोक्तकरणेन जातयोश्छेद\\
योर्दर्शनम् ३/११ । ७/५५ । 

 अपि च । 

 त्रिपञ्चकमितावंशौ\\
तद्युतावेकसप्ततिः ।\\
सप्ततिच्छेदिता शीघ्रं\\
तयोश्छेदौ सखे वद ॥ ८ ॥

 न्यासः ३/० । ५/० फलम् ७१/७० । अत्रोद्दिष्टांशः प्रथमः ३ फलहार\\
७० हतः २१० परांश ५ युतः २१५ फलांशेन ७१ भागे हृते शुद्धिर्न\\
स्यादतः कुट्टकः कार्यः । उद्दिष्टांशयोरल्पो भाज्यः- ३ परः क्षेपः ५\\
फलांशको हारः ७१ इत्थं प्रकल्प्य कुट्टकार्थं न्यासः भा ३ क्षे ५ हा ७१ ।
अतो\\
लब्धिः सक्षेपा क्षे ३ ल १ । त्रिकेनेष्टेन जाता लब्धिः १० अयमेको\\
हरः । फलच्छेदादल्पोऽयमतः फलच्छेदमिम ७० मनेन विभाज्य\\
जातोऽपरच्छेदः ७ । दर्शनम् ३/१० । ५/७ क्वचिदृणक्षेपं प्रकल्प्यछे-\\
दावुत्पद्येते ।

( २७०) 

सूत्रम् । 

 १अज्ञातेष्वंशेषु\\
प्रकल्प्य रूपं पृथक्पृथक् चांऽशान् ।\\
कृत्वा तुल्यच्छेदान्\\
फलहारेणच्छिदो लोप्याः ॥ १३ ॥
तेषु द्वयोः कयोश्चिद्\\
हारस्त्वेकः परश्च ऋणभाज्यः ।\\
इष्टांशहतान्योनिते-\\
फलं भवेत् क्षेपकोऽथ दृढकुट्टात् ॥ १४ ॥

\begincenter\rule\theta.5\linewidth\theta.5pt\endcenter

 (१) अत्रोपपत्तिः । कल्प्यन्ते अंशाः अव्यक्ताः अ१,अ२,
अ३,...\\
तदा अ१/ह१ + अ२/ह२ + अ३/ह३ + ....= फ=अं/ह

 अत्र समच्छेदेन कल्प्यन्ते गुणकाः = गु१, गु२, गु३, ......\\
अतः अ१गु१+ अ२गु२ + अ३गु३ + .... /सछे =
अं.गु/सछे

 छेदगमे, अ१गु१+ अ२गु२ + अ३गु३ + ....  =
अं.गु

पक्षान्तरेण अं.गु-अ२गु२-अ३गु३ /गु१  =अ१

 अत्र अ३,अ४,...इत्यादीनां मानानि इष्टानि प्रकल्प्य
तदुत्थापनेन\\
व्यक्तराशिसंस्कारं अ. गु अस्मिन् कृत्वा क्षेपः कल्प्यः । ततः

क्षे-गु२अ१/गु१=अ१ अथ गु२ ऋणभाज्यं गु१ हारं च
प्रकल्प्य\\
कुट्टकेन अ१,अ२ मानं सुगमम् । अत उपपन्नम् ।

( २७१) 

 गुणलब्धी सक्षेपे\\
विभाज्य हरयोर्लवौ स्याताम् ।\\
हरभाज्यक्षेपाणां यथाऽपवर्त-\\
स्तथांऽशका कल्प्याः ॥ १५ ॥ 

 उदाहरणम् । 

 छेदा बाणगजाङ्कसूर्यमितयो\\
नष्टाश्च तेषां लवा ।\\
स्वाब्ध्यंशेन समन्वितं युतिरभू-\\
देकस्य रूपत्रयम् ।\\
तानंशान् बहुधा वदाऽऽशु गणिता-\\
हंकारमत्तद्विप-\\
स्तोमं क्षोभयितुं क्षमोऽतिकठिना-\\
रावोऽसि कण्ठीरवः ॥ ९ ॥ 

 न्यासः ०/५। ०/८ । ०/९ । १/२ फलम्
३१/४० । अत्राज्ञातेष्वंशेषु रूप-\\
मेकैकमंशं प्रकल्प्य न्यासः १/५ । १/८ । १/९ । १/१२ फलम् १२१/४० फलेन\\
सह कृतसमच्छेदाः ७२/३६० । ४५/३६० । ४०/३६० । ३०/३६० । १०८९/३६० छिदो
लोप्या\\
इतिच्छेदापनयने कृते जातम् ७२ । ४५ । ४० । ३० फल १०८९ म्

( २७२) 

∗अपास्य शेषम् ९०० इतरयोरेतयोः ४०।३० एको भाज्यः परो\\
हरः फलशेषं क्षेपः । कुट्टकार्थं न्यासः भा ४० क्षे ९००/हा ३० ।
दशभिरपवर्त्य\\
जाता दृढाः भा ४ क्षे ९०/हा ३ । जातौ लब्धिगुणौ सक्षेपौ । लब्धिः\\
क्षे ४ रू ३० । गुणः क्षे ३ रू ० । प्रथमावंशौ २।१ एकादिसप्तान्तैः\\
क्षेपं संगुण्य रूपेषु प्रक्षिप्य जाताश्छेदाः । 



\beginlongtable[]@
  >$\raggedright\arraybackslashp(\columnwidth - 6\tabcolsep) * \real\theta.25
  >$\raggedright\arraybackslashp(\columnwidth - 6\tabcolsep) * \real\theta.25
  >$\raggedright\arraybackslashp(\columnwidth - 6\tabcolsep) * \real\theta.25
  >$\raggedright\arraybackslashp(\columnwidth - 6\tabcolsep) * \real\theta.25@
\toprule
\endhead
२ 

२ 

२ 

२ 

२ 

२ 

२  \& १ 

१ 

१ 

१ 

१ 

१ 

१  \& ३ 

६ 

९ 

१२ 

१५ 

१८ 

२१ \& २६ 

२२ 

१८ 

१४ 

१० 

६ 

२  \\
\bottomrule
\endlongtable

 अथवा प्रथमावंशौ २।३ एकादिषडन्तैः संगुणितौ-



\beginlongtable[]@
  >$\raggedright\arraybackslashp(\columnwidth - 6\tabcolsep) * \real\theta.25
  >$\raggedright\arraybackslashp(\columnwidth - 6\tabcolsep) * \real\theta.25
  >$\raggedright\arraybackslashp(\columnwidth - 6\tabcolsep) * \real\theta.25
  >$\raggedright\arraybackslashp(\columnwidth - 6\tabcolsep) * \real\theta.25@
\toprule
\endhead
२ 

२ 

२ 

२ 

२

२ \& ३ 

३ 

३ 

३ 

३ 

३ \& ३ 

६ 

९ 

१२

१५ 

१८  \& २३

१९

१५

११

७

३ \\
\bottomrule
\endlongtable

अथवा प्रथमावंशौ २।५ एकादिपञ्चान्तैः । 



\beginlongtable[]@
  >$\raggedright\arraybackslashp(\columnwidth - 4\tabcolsep) * \real\theta.\ni\ni
  >$\raggedright\arraybackslashp(\columnwidth - 4\tabcolsep) * \real\theta.\ni\ni
  >$\raggedright\arraybackslashp(\columnwidth - 4\tabcolsep) * \real\theta.\ni\ni@
\toprule
\endhead
२ 

२ 

२ 

२ 

२  \& ५ 

५ 

५ 

५ 

५  \& ३ २०

६ १६

९ १२

१२ ८

१५ ४ \\
\bottomrule
\endlongtable



 ∗ अत्र त्रुटिरस्ति पुस्तकद्वयेऽपि ।\\
सा च 'अत्र प्रथमद्वितीयांशमाने च क्रमेण २।१ परिकल्प्य' इति\\
भवितुमर्हतीति ।

( २७३)

 अथवा प्रथमावंशौ २।७ एकादिपञ्चान्तैः- 



\beginlongtable[]@
  >$\raggedright\arraybackslashp(\columnwidth - 6\tabcolsep) * \real\theta.25
  >$\raggedright\arraybackslashp(\columnwidth - 6\tabcolsep) * \real\theta.25
  >$\raggedright\arraybackslashp(\columnwidth - 6\tabcolsep) * \real\theta.25
  >$\raggedright\arraybackslashp(\columnwidth - 6\tabcolsep) * \real\theta.25@
\toprule
\endhead
२ 

२ 

२ 

२ 

२  \& ७ 

७ 

७ 

७ 

७  \& ३ 

६ 

९ 

१२ 

१५  \& १७ 

१३ 

९ 

५

१ \\
\bottomrule
\endlongtable

अथवा प्रथमौ २।९ एकादिचतुरन्तैः-



\beginlongtable[]@
  >$\raggedright\arraybackslashp(\columnwidth - 6\tabcolsep) * \real\theta.25
  >$\raggedright\arraybackslashp(\columnwidth - 6\tabcolsep) * \real\theta.25
  >$\raggedright\arraybackslashp(\columnwidth - 6\tabcolsep) * \real\theta.25
  >$\raggedright\arraybackslashp(\columnwidth - 6\tabcolsep) * \real\theta.25@
\toprule
\endhead
२ 

२ 

२ 

२  \& ९ 

९ 

९ 

९  \& ३ 

६ 

९

१२  \& १४ 

१०

६

२ \\
\bottomrule
\endlongtable

अथवा प्रथमौ २। ११ एकादित्र्यन्तैः- 



\beginlongtable[]@
  >$\raggedright\arraybackslashp(\columnwidth - 6\tabcolsep) * \real\theta.25
  >$\raggedright\arraybackslashp(\columnwidth - 6\tabcolsep) * \real\theta.25
  >$\raggedright\arraybackslashp(\columnwidth - 6\tabcolsep) * \real\theta.25
  >$\raggedright\arraybackslashp(\columnwidth - 6\tabcolsep) * \real\theta.25@
\toprule
\endhead
२ 

२ 

२  \& ११ 

११ 

११  \& ३ 

६ 

९ \& ११ 

७ 

३ \\
\bottomrule
\endlongtable

अथवा प्रथमौ २।१३ एकेन द्वाभ्यां च 



\beginlongtable[]@
  >$\raggedright\arraybackslashp(\columnwidth - 6\tabcolsep) * \real\theta.25
  >$\raggedright\arraybackslashp(\columnwidth - 6\tabcolsep) * \real\theta.25
  >$\raggedright\arraybackslashp(\columnwidth - 6\tabcolsep) * \real\theta.25
  >$\raggedright\arraybackslashp(\columnwidth - 6\tabcolsep) * \real\theta.25@
\toprule
\endhead
२ 

२  \& १३ 

१३ \& ३ 

६ \& ८ 

४ \\
\bottomrule
\endlongtable

अथवा प्रथमौ २। १५ एकेन द्वाभ्यां च



\beginlongtable[]@
  >$\raggedright\arraybackslashp(\columnwidth - 6\tabcolsep) * \real\theta.25
  >$\raggedright\arraybackslashp(\columnwidth - 6\tabcolsep) * \real\theta.25
  >$\raggedright\arraybackslashp(\columnwidth - 6\tabcolsep) * \real\theta.25
  >$\raggedright\arraybackslashp(\columnwidth - 6\tabcolsep) * \real\theta.25@
\toprule
\endhead
२ 

२  \& १५ 

१५  \& ३ 

६ \& ५ 

१ \\
\bottomrule
\endlongtable

अथवा प्रथमौ २।१७ एकेन जाताश्छेदाः २।१७।३।२

 

अथवा प्रथमौ ७।१ एकादिचतुरन्तैर्जाताश्छेदाः- 



\beginlongtable[]@
  >$\raggedright\arraybackslashp(\columnwidth - 6\tabcolsep) * \real\theta.25
  >$\raggedright\arraybackslashp(\columnwidth - 6\tabcolsep) * \real\theta.25
  >$\raggedright\arraybackslashp(\columnwidth - 6\tabcolsep) * \real\theta.25
  >$\raggedright\arraybackslashp(\columnwidth - 6\tabcolsep) * \real\theta.25@
\toprule
\endhead
७ 

७ 

७ 

७ \& १ 

१

१

१ \& ३ 

६

९

१२  \& १४ 

१०

६

२ \\
\bottomrule
\endlongtable

अथवा प्रथमौ ७।२ एकादित्र्यन्तैः- 

\beginlongtable[]@
  >$\raggedright\arraybackslashp(\columnwidth - 6\tabcolsep) * \real\theta.25
  >$\raggedright\arraybackslashp(\columnwidth - 6\tabcolsep) * \real\theta.25
  >$\raggedright\arraybackslashp(\columnwidth - 6\tabcolsep) * \real\theta.25
  >$\raggedright\arraybackslashp(\columnwidth - 6\tabcolsep) * \real\theta.25@
\toprule
\endhead
७ 

७

७ \& २ 

२

२ \& ३ 

६

९ \& ११ 

७

३ \\
\bottomrule
\endlongtable

 वा प्रथमौ ७।५\\
एकेन द्वाभ्यां च 



\beginlongtable[]@
  >$\raggedright\arraybackslashp(\columnwidth - 6\tabcolsep) * \real\theta.25
  >$\raggedright\arraybackslashp(\columnwidth - 6\tabcolsep) * \real\theta.25
  >$\raggedright\arraybackslashp(\columnwidth - 6\tabcolsep) * \real\theta.25
  >$\raggedright\arraybackslashp(\columnwidth - 6\tabcolsep) * \real\theta.25@
\toprule
\endhead
७ 

७ \& ५ 

५ \& ३ 

६ \& ८ 

४ \\
\bottomrule
\endlongtable

वा प्रथमौ ७।७ एकेन द्वाभ्यां च 



\beginlongtable[]@
  >$\raggedright\arraybackslashp(\columnwidth - 6\tabcolsep) * \real\theta.25
  >$\raggedright\arraybackslashp(\columnwidth - 6\tabcolsep) * \real\theta.25
  >$\raggedright\arraybackslashp(\columnwidth - 6\tabcolsep) * \real\theta.25
  >$\raggedright\arraybackslashp(\columnwidth - 6\tabcolsep) * \real\theta.25@
\toprule
\endhead
७ 

७ \& ७ 

७ \& ३ 

६ \& ५ 

१ \\
\bottomrule
\endlongtable

 वा प्रथमौ ७।९\\
एकेन ७।९।३।२ वा प्रथमौ १२।१ एकेन १२।१।३।२

 १८

( २७४) 

 एवं प्रथमद्वितीयौ, प्रथमचतुर्थौ, द्वितीयतृतीयौ वा, इष्टावंशौ\\
प्रकल्प्योक्तवत् करणेनांऽशा भवन्ति । एवमनेकधा । 

 इति भागजातिः । 

 अथ प्रभागजातिः ॥ 

 सूत्रम् । 

 १अंशानिष्टफलोत्था-\\
नुद्दिष्टैः संभजेद् भवन्त्यंशाः ।\\
बहुषु पदेषूद्दिष्टे-\\
ष्टानां घातैर्भजेदेवम् ॥ १ ॥ 

 उदाहरणम् । 

 यस्यां यस्यङ्घ्रित्रयं यस्य\\
पञ्चाशाश्चत्वारो यस्य पञ्चाशकाः षट् ।\\
योगे जातं रूपमेकं वदाऽऽशु\\
जानासि त्वं चेत् प्रभागानुमार्गम् ॥ १० ॥

\begincenter\rule\theta.5\linewidth\theta.5pt\endcenter

 (१) अत्रोपपत्तिः । कल्प्यन्तेऽभीष्टफलभागाः भ१/क१,
भ२/क२, भ३/क३...\\
भा१/हा१, भा२/हा२, भा३/हा३...\\
तथा उदिष्टांशास्तदा विलोमविधिना 

राशयः = हा१अ१/क१भा१, हा२अ२/क२/हा२,
हा३अ३/क३भा३..... 

 एवं बहुषु पदेषु इष्टानामंशानामुद्दिष्टानां घातैरिष्टफलभागा\\
भक्ता राशयः स्युः ।

( २७५) 

न्यासः ०/०। ३/४ । ४/५ । ६/५ फलम् १ । अत्र रूपफलभागाः\\
१/२। १/६ । १/३ । एतानुद्दिष्टैर्भक्त्वा जाता अंशाः २/३ । ५/२४ । ५/१८\\
दर्शनम् २/३। ३/४ । ५४/२४५ । ५/१८ । ६/५ 

 अन्यै रूपफलभागैरन्येंऽशाः संभवन्ति । 

अपि च । 

 यस्यांऽशस्य च योंऽशकस्त्वपि च\\
तद्भागश्च यस्यांऽशक-\\
स्तत्सप्तांशकषट्कमेव धनिना\\
केनाऽपि दत्तं धनम् ।\\
अन्येद्युश्च तथा नवांशकयुगो-\\
ऽन्यस्मिन् दशांशत्रयं\\
तस्मै विप्रवराय रूपमभवत्\\
केभ्योंऽशकेभ्यः सखे ॥ ११ ॥

न्यासः - ०/० । ०/० । ०/० । ६/७ ॥ ०/० । ०/० । ०/० । २/९ ।। ०/० ।\\
०/० । ०/० । ३/१० । फलम् १ । रूपभागाः १/२ । १/६ । १/३ प्रथम दिन
उद्दिष्ट-\\
भागाः ६/७ इष्टकल्पितौ भागो २/३ । ३/४ तु उद्दिष्टेष्टानां घातः २/७ अनेन\\
आद्यांशः २ इष्टौ २/३। ३/४ प्राग्वज्जाता अंशाः ३/२ । २/३ । ३/४ । २/९ ।\\
तृतीयि उद्दिष्टांशः ३/१० इष्टौ १/२ । ५/३ प्राग्वज्जाता भागाः ३। ४/२ ।
१/३ ।

( २७६) 

५/१ । ३/१० । ७/४ । १/२ । २/३ । ६/७ । ३२३/२३४ । २/९ । ४/३ । ५/३ । ३/१०

 इष्टांशकल्पनावशादनेकधा । इति भागप्रभागजातिः । 

 अथ भागानुबन्धभागापवाहयोरुत्पत्तौ सूत्रम् । 

 १रूपाणीष्टानि पृथक्\\
स्थाने विन्यस्य तद्युतिं फलतः ।\\
त्यक्त्वा शेषं स्वमृणं\\
तदुत्थभागा अधस्तेषाम् ॥ १ ॥

 उदाहरणम्। 

 चतुःस्थानस्थितान्यंशै\\
रूपाणि कतिचित् सखे । 

 (१) अत्रोपपत्तिः । कल्प्यते योगः यो । इष्टानि इ१, इ२,
इ३,...\\
ततः यो- ( इ१+ इ२+इ३....)= शे, 

 अथ पूर्वविधिना अ१/क१,अ२/क२,अ३/क३....तथा 

 ज्ञेया यथा अ१/क१+अ२/क२+अ३/क३+....शे\\
तदा इ१ अ१/क१+ इ२अ२/क२+
इ३अ३/क३+....यो।\\
एवं भागापवाहे इष्टानां योग उद्दिष्टयोगाथिकः कल्प्यः ।तदा\\
इ१+इ२+इ३+....यो=शे\\
ततः इ१ अ१/क१+इ२ अ२/क२+इ३
अ३/क३+....इत्युपपद्यते ।

( २७७)

 कैश्चिद् युक्तानि हीनानि 

 द्वादश स्युर्युतौ कथम् ॥ १ ॥ 

 भागानुबन्धे फलम् १२ । कल्पितानीष्टानि १।२।३।४ योगः १०\\
फलतोऽस्मा १२ दपास्य शेषम् २ द्व्यादिरिष्टै रूपफलभागाः\\
२/३ । १/३ । १/५ । ४/५ कल्पितरूपाणामधो विन्यस्य जाता भागानुबन्धाः\\
१ २ ३ ४\\
२ १ १ ४ \ फलम् १२ । अथवेष्टानि १।२।३।५ एकादिभिरिष्टै\\
३ ३ ५ ५\\
रूपफलभागाः १/२ । १/६ । १/१२ । १/४ । एभ्यो भागानुबन्धाः\\
१ २ ३ ५\\
१ १ १ १ \ फलम १२\\
२ ६ १२ ४ 

 अथ भागापवाहेऽपि फलम् १२ । फलाधिकयोगो यथा स्यात्\\
तथा कल्पितानीष्टानि २/३।४।५ योगं १४ फलादपास्य १२ शेषं २\\
द्व्यादिभिरिष्टैर्द्विरूपफलभागाः २/३ । १/३ । १/५ । ४/५ एभ्यो भागापवाहाः


२ ३ ४ ५ 

२ १ १ ४ \ फलम् १२ अथवेष्टानि १।३।४।५ एकादिरूपैः 

३ ३ ५ ५ १ ३ ४ ५

फलभागाः १/२ । १/६ । १/१२ । १/४ एभ्यो भागापवाहोः । १ १ १ १\\
२ ६ १२ ४\\
फलम् १२\\
इति भागानुबन्धापवाहौ ।

( २७८) 

 अथ स्वांशानुबन्धोत्पत्तौ सूत्रम् । 

 १यदि सन्त्यधःस्थितांशा-\\
स्तदुपरि रूपं पृथक् च विन्यस्य ।\\
स्वांशानबन्धविधिना\\
सवर्ण्य तैरंशकैर्विभेजेत् ॥ १ ॥
रूपफलोत्थानंशान्\\
भवन्ति भागास्तदूर्ध्वस्थाः । 

 उदाहरणम् 

 नेत्राब्धिषट्तुरगनागलवैः स्वकीयै-\\
रंशाश्च ये पुथगपि क्रमशोऽनुबन्धाः ।\\
तत्संयुतावभवदेकमिहास्ति ते चे-\\
न्मात्सर्यमार्य वद मे द्रतमूर्ध्वभागान् ॥ १ ॥\\
न्यासः फलम् १ । अत्रातांशस्थानेषु पृथग्रूपं विनस्य जातम्-\\
१ १ १ १ १\\
१ १ १ १ १ \ स्वांशानुबन्धविधिना सवर्ण्य जातम् ३/२ । ५/४\\
१ १ १ १ १\\
२ ४ ६ ७ ८ ७/६ । ८/७ । ९/८ एभी रूपफलभागान् १/२ । १/६ । 

१/१२ । १/२० । १/५ विभजेदिति भक्ता जाता ऊर्ध्वस्थाः १/३ । २/१५ । १/१४ ।


\begincenter\rule\theta.5\linewidth\theta.5pt\endcenter

 (१) अत्रोपपत्तिः । ऊर्ध्वराशिं रुपं प्रकल्प्य स्वांशानुबन्धविधिना\\
ये भिन्नास्तै रूपफलभागा भक्ता ऊर्ध्वस्था भागा भवन्ति यतस्ते\\
भिन्नगुणिता रूपफलभागा भवन्ति यद्योगे रुपं भवति ।

( २७९) 

७/१६० । ८/४५ । दर्शनम्। 

 १ २ १ ७ ८\\
३ १५ १४ १३० ४५ \ अन्यै रूपफलभागैरन्येंऽशाः संभवन्ति ।\\
१ १ १ १ १\\
२ ४ ६ ७ ८

सूत्रम् 

 १ऊर्ध्वस्थितैस्तु भागैः\\
पृथग् भजेद् रूपफलभवानंशान् ॥ २ ॥
पृथगेकैकं तेभ्यः\\
शोध्यमधःस्थो भवन्त्यंशाः । 

 उदाहरणम् । 

 पञ्चेभभूपाङ्कलवाः स्वकीयै-\\
र्यैः कैश्चदार्य क्रमशोऽनुबन्धाः ।\\
आचक्ष्व तानाशु लवानधःस्था-\\
नंशावतारे पटुताऽस्ति ते चेत् ॥ २ ॥

 न्यासः फलम् १ । अत्र रूपफलभागाः १/२ । १/६ । १/१२ । १/४ । ऊध्वे-\\
स्थितैरेभिः १/५ । १/८ । १/१६ । १/९ । भक्ताः ५/२ । ४/३ । ४/३ । ९/४
एकवि-\\
हीनाः ३/२ । १/३ । १/३ । ५/४ एतेऽधःस्थिता भागाः । दर्शनम्-\\
१ १ १ १ 

५ ८ १६ ९ 

३ १ १ ५ \

२ ३ ३ ४ 

\begincenter\rule\theta.5\linewidth\theta.5pt\endcenter

 (१) अत्रोपपत्तिः पूर्वप्रकारवैपरीत्येन स्फुटा ।

( २८०) 

 सूत्रम् ।

 १प्रागंशविधानेन च जाता\\
येऽङ्का विवर्जिताश्चोर्ध्वैः ॥ ३ ॥ 

\begincenter\rule\theta.5\linewidth\theta.5pt\endcenter

 (१) अत्रोपपत्तिः । यद्यूर्ध्वभागाः क्रमेण ऊ१/हा१,
ऊ२/हा२, ऊ३/हा३ 

 अधोभागाः अ१/क१, अ२/क२, अ३/क३,...। 

 मध्यभागाच्च म१/भा१, म२/भा२, म३/भा३, । 

 तदांशानुबन्धविधिना 

भिन्नाः ऊ१/हा१ ( अ१+क१)(म१+भा१)/क१भा१
= १/क....(१)\\
∴ म१+भा१/भा१= १/क / ऊ१/हा१ ( अ१+क१)/क१
∴म१/भा१= १/क / ऊ१/हा१ ( अ+क)/क१ ....(१)

 एतेन 'अथवा मध्यभागं विना सवर्ण्य रूपफलभागान्\\
विभज्य पृथगेकं रूपं विशोध्य शेषाणि मध्यभागा भवन्ति' ।\\
इत्युपपद्यते । 

 अथ ( १) एतद्रुपान्तरेण 

 ऊ१/हा१(१+म१/भा१)=१/क / अ१+क१ / क१\\
∴म१/भा१=१/क / अ१+क१ /क१ --- ऊ१/हा१/
ऊ१/हा१ अनेनेदं सूत्रमुपपद्यते । 

( २८१)

 भागैस्तैरेव पुन-\\
र्विभाजिता मध्यभागाः स्युः ।
उदाहरणम् । 

 निजैश्च पञ्चाष्टषडंशका यैः\\
कैश्चिच्च भागैः सहिताः पुनस्ते ।\\
स्वीयैः षडंशांघ्रिदलैः समेता\\
रूपं फलं स्याद् वद तान् द्रतं मे ॥ ३ ॥

न्यासः १ १ १ फलम् १ । प्रागंशविधानम् । यदि\\
५ ८ ६ सन्त्यधःस्थितांशास्तदुपरि रूपमिति\\
० ० ० कृते जातम् १ १ १ सवर्ण्य जातम्\\
० ० ० \ १ १ १ \ ७/६ । ५/४ । ३/२\\
१ १ १ १ १ १\\
६ ४ २ ६ ४ २ 

 एभी रूपफलभागाः १/२ । १/६ । १/३ भक्ता जाताः ३/७ २/१५ । २/९

 ऊर्ध्वैरुद्दिष्टैर्भागैरेभिः १/५ । १/८ । १/६ विवर्जिताः ८/३५ । १/१२० ।
१/१८\\
तैरेव विभाजिताः - ८/७ । १/१५ । १/३ जाता मध्यभागाः । दर्शनम्

१/५ १/८ १/६\\
८/७ १/१५ १/३ \\\
१/६ १/४ १/२

 अथवा मध्यभागं विना सवर्ण्य रूपफलभागान् विभज्य पृथगेकं\\
रूपं विशोध्य शेषाणि मध्यभागा भवन्ति ।

( २८२) 

सूत्रम् ।

१इष्टानंशानूर्ध्वाज्ञातस्थानेषु विन्यस्य ॥ ४ ॥ 

पूर्वविधानेनाऽधोऽज्ञातस्थानस्थिताः साध्याः । 

उदाहरणम् 

 त्र्यंशो दलं च चरणः स्वलवैश्च कैश्चिद\\
युक्ताश्च पादशरभागषडंशकैः स्वैः ।\\
 अंशैश्च कैश्चिदपि ते सहिताः स्वकीयै-\\
स्तेषां युतौ गणक रूपचतुष्टयं स्यात् ॥ ४ ॥ 

न्यासः । १ १ १ फलम् ४ । अत्रोर्ध्वस्थानेष्विष्टानंशान्\\
३ २ ४\\
० ० ०\\
० ० ० \ प्रकल्प्येति कल्पितानीष्टानि १/२ । १/३ ।\\
१ १ १\\
४ ५ ६\\
० ० ० १५ एत उपरि विन्यस्ता जाताः-\\
० ० ०

१ १ २ ५ ४ ७\\
३ २ ४ ततः पूर्वविधिनाऽज्ञाऽधःस्थिताः साध्या ८ ५ २०\\
१ १ १ ० ० ० \\\
२ ३ ५ \ इति तावदूर्ध्वस्थाः सवर्णिता जाता ० ० ०\\
१ १ १\\
४ ५ ६\\
० ० ० अधुना पूर्वविधिः । 'ऊर्ध्वस्थितैस्तु भागैः पृथग्\\
० ० ० भजेद् । रूपफलभागान्' इति रूपफलभागाः

\begincenter\rule\theta.5\linewidth\theta.5pt\endcenter

 (१) अत्रोपपत्तिः । अत्रोर्ध्वा भागा इष्टाः कल्पितास्ततः पूर्वसूत्र-\\
विधिनाऽधोभागाः साधिता इति ।

( २८३) 

१/२ । १/९ । १/३ योगे रूपचतुष्टयं वर्तत इति चतुर्गुणिताः २/१ । २/३ ।\\
-४/३ पूर्वसवर्णितैर्भागैरेभिः ५/८ । ४/५ । ७/२० भक्ता रूपोना जाता अधः\\
स्थिता भागाः ११/५ । २/३ । १९/२१ । दर्शनम् । 

 १ १ १

 ३ २ ४

 १ १ १

 २ ३ ५ \ 

 १ १ १ 

 ४ ५ ६ 

 ११ २ १९ 

 ५ ३ २१ 

 अत्रेष्टाङ्ककल्पनादनेकधा भागा उत्पद्यन्ते ।

 इति स्वांशानुबन्धजातिः । 

 अथ स्वांशापवाहोत्पत्तौ सूत्रम् । 

 १यदि सन्त्यधः स्थितांशा-\\
स्तदुपरि रूपं पृथक् पृथग् न्यस्य ॥ ५ ॥\\
स्वांशापवाहविधिना\\
सवर्ण्य तैरंशकैर्विभजेत् ।\\
रूपफलोत्थानंशान्\\
भवन्ति भागास्तदूर्ध्वस्थाः ॥ ६ ॥

\begincenter\rule\theta.5\linewidth\theta.5pt\endcenter

 (१) अत्रोपपत्तिः । स्वांशानुबन्धवत् ।

( २८४)

 उदाहरणम् ।

 स्वैरष्टसप्ताङ्गकृताक्षिभागै-\\
विवर्जिताः केऽपि लवाश्च तेषाम् ।\\
रूपं युतौ तत् कथयैवमत्र\\
गर्वोऽस्ति ते चेद् गणितप्रवादे ॥ ५ ॥
न्यासः । ० ० ० ० ० फलम् १ । अत्राऽक्षातांशस्थाने\\
० ० ० ० ० \ पृथग्रूपं विन्यस्य जातम्\\
१/८ १/७ १/६ १/४ १/२\\
१/१ १/१ १/१ १/१ १/१ \ स्वांशांपवाहविधिना सवर्ण्य

 १/८ १/७ १/६ १/४ १/२ जातम् ७/८ । ६/७ । ५/६ । ३/४ ।\\
१/२ एभी रूपफलभागाः १/५ ।

१/२ । १/६ । १/१२ । १/१० विभजेदिति भक्ता जाता ऊर्ध्वस्था भागाः\\
४/७ । ७/३६ । १/१० । १/१५ । २/५ दर्शनम् ४/७ ७/३६ १/१० १/१५ २/५ \ अन्यै
रूप-

 १/८ १/७ १/६ १/१२ १/१०\\
फलभागैरन्येंशा उत्पद्यन्ते । 

सूत्रम् ।
१ऊर्ध्वस्थितैस्तु भागैः\\
पृथग्भजेद् रूपफलभवानंशान् । 

\begincenter\rule\theta.5\linewidth\theta.5pt\endcenter

 (१) अत्रोपपत्तिः । स्वांशानुबन्धविधिनाऽत्र ऊर्ध्वस्थितैर्भागै रूप-\\
फलभवांशेषु विहृतेषु फलानि = फ = क१-क१/क१ = १-अ१/क१
अतः\\
अ१/क१ = १ - फ । अत उपपन्नम् ।

( २८५)

 रूपात् पृथग् विशोध्याः\\
शेषाः स्युरधःस्थिता भागाः ॥ ७ ॥ 

उदाहरणम् । 

 दलं शरांशश्चरणस्त्रिभागः\\
कैश्चिन्निजांशैश्च विवर्जितास्ते ।\\
योगे वद स्यात् कथमेकरूपं\\
दक्षोऽसि चेत् त्वं हि लवावतारे ॥ ६ ॥

न्यासः १ १ १ १ फलम् १ । अत्र रूपफलभागार्थे\\
२ ५ ४ ३ कल्पिता इष्टलवाः ।३०।१०।१०।३०।\\
० ० ० ० \ 'उत्पादयेच्च भागान् युग्ममित इत्या-\\
० ० ० ०

दिना जाता रूपफलभागाः ।\\
३/२ । १/१४ ।१/५ । ३/१० एते उद्दिष्टैरेभिर्भक्ता रूपाद् विशोधिता\\
अधःस्थिता भागाः- १/७ । १/१४ । १/५ । १/१०। दर्शनम् १/२ १/५ १/४ १/३\\
१/७ ९/१४ १/५ १/१० \

 अथ पूर्वसूत्रोक्तं तत्पुरस्करणेनाह । उदाहरणम् । 

 अर्धत्र्यंशचतुर्थभागगुणितं\\
सैकं शतं तु त्रिधा\\
भागैः कैश्च निजैर्विवर्जितमथ\\
स्वार्धाङ्घ्रिपञ्चांशकैः ।

( २८६) 

 हीनं चैव पुनश्च कैर्निजलवैः\\
संवर्जितं तद्युतौ\\
रूपार्धं कथयाशु कोविद, वदा-\\
ऽऽर्य, त्वं प्रगल्भोऽसि चेत् ॥ ७ ॥



न्यासः । १०१ १०१ १०१\\
२ ३ ४ फलम् १/२ पूर्वोक्तस्य करणम् ।इष्टा-\\
० ० ० नंशानूर्ध्वोशातस्थानेषुविन्यसेदित्ति\\
० ० ० कल्पिता इष्टांशा १/३ ।१/४। १/५\\
१ १ १ \ ऊर्ध्वरथा जाताः । ततःस्वांशा-\\
२ ४ ५ पवाहविधिना सवर्णिता जाताः\\
० ० ० १०१ । १०१ । १०१ एभी रूपफल-\\
० ० ० ६ १६ २५



भागाः १/६ । १/३ । १/२ फलं रूपार्धं वर्तते।∗ 

 इति श्रीसकलकलानिधिनरसिंहनन्दनगणितविद्याचतुरानन-\\
नारायणपण्डितविरचितायां गणितपाट्यां कौमुद्याख्यायां रूपाद्यंशा-\\
वतारो नाम द्वादशो व्यवहारः । 

 अथाऽङ्कपाशे सूत्राणि । 

 अथ गणकानन्दकरं\\
संक्षेपादङ्कपाशकं वक्ष्ये ।\\
निपतन्ति यत्र मत्सरवन्तो\\
दुष्टाः कुगणका ये ॥ १ ॥

\begincenter\rule\theta.5\linewidth\theta.5pt\endcenter

 
∗अत्रोभयत्र त्रुटिः ।

( २८७) 

 १भरते छन्दश्शास्त्रे वैद्ये\\
माल्यक्रियासु गणिते च ।\\
शिल्पेऽप्यस्त्युपयोगोऽ-\\
तस्तस्य ज्ञानमङ्कपाशेन ॥  २ ॥
चयपङ्क्तिश्च व्यन्तर-\\
पङ्क्तिर्वैश्लेषिणी च सार्पिणिका ।\\
पङ्क्तिर्जलौकिकाख्या\\
ततश्च सामासिका पङ्क्तिः ॥ ३ ॥
पातालाख्या पङ्क्तिः\\
पङ्क्तिर्गुणकोत्तराभिधाना च ।\\
अभ्यासिका च पङ्क्तिः\\
सूचीपङ्क्तिश्च खण्डसूची च ॥ ४ ॥
यौगिकसंज्ञा पङ्क्तिः\\
खण्डितमेरुस्ततः पताका च ।\\
मेरुस्तिमिमेरुरथो\\
लड्डुक इत्यादिकरणानि ॥ ५ ॥
संख्या प्रत्यय आवृत्ति-\\
स्ततश्चोर्ध्वाङ्कसंयुतिः ।

\begincenter\rule\theta.5\linewidth\theta.5pt\endcenter

  (१) नृत्यशास्त्रे ।

( २८८) 

 सर्वयोगाङ्कपातश्च\\
प्रस्तारप्रत्ययस्ततः ॥ ६ ॥
नष्टोद्दिष्टैस्तथा स्थान-\\
भेदसंख्याविचारणम् ।\\
अन्तिमाद्यङ्कवृद्ध्यङ्क-\\
योगभेदप्रसाधनम् ॥ ७ ॥
निरेककैककद्व येक-\\
त्र्येकादीनां च साधनम् ।\\
एकान्तद्व्यन्तकत्र्यन्त-\\
चतुरन्तादिसाधनम् ॥ ८ ॥
इत्यादिप्रत्यया येऽपि\\
प्रत्येकं ते त्वनेकधा ।\\
स्वस्वोपयोगिसूत्रैस्तान्\\
वक्ष्ये स्फुटतरं यथा ॥ ९ ॥ 

 इति प्रत्ययः । 

 तत्रादौ चयपङ्क्तिव्यन्तरपङ्क्तिवैश्लेषिणीसार्पिणिकाजलौकि-\\
कापङ्क्तिषु सूत्रम्। 

 ∗एकाद्येकचयाङ्कैः\\
स्थानान्तं प्रचयसंज्ञिका पंक्तिः । 

\begincenter\rule\theta.5\linewidth\theta.5pt\endcenter

 ∗अन्त्याङ्क' त्यक्त्वा मूलक्रमे यावत्स्थानेषु अङ्काः समास्ताव-\\
त्सार्पिण्यां पङ्क्तावुपान्तिमाङ्कानां योगः कार्यः । एवं जलौकापंक्तिः

( २८९) 

 अपरिच्छिन्नैकाङ्कैः\\
पंक्तिः सा व्यन्तरारथा स्यात् ॥ १० ॥
साऽपि परिच्छिन्ना यदि\\
पंक्तिर्वैश्लेषिणीति विज्ञेया ।\\
अधिकैकस्थाना सा\\
पंक्तिः स्यात् सर्पिणीतीह ॥ ११ ॥
सार्पिणयन्तं मुक्त्त्वा\\
यावन्ति स्थानकानि तुल्यानि ।\\
तत्संयोगः पंक्ति-\\
र्विज्ञेया सा जलौकिकाख्येति ॥ १२ ॥

 उदाहरणम् । 

 चतुःस्थानस्थितापंक्ति-\\
श्चयाख्या कीदृशी भवेत् । 

\begincenter\rule\theta.5\linewidth\theta.5pt\endcenter

स्यात् यथा 'यावत्स्थानेष्वङ्कास्तुल्यास्तज्जैः' इत्याद वक्ष्यमाण-\\
सूत्रोदाहरणे ५४५४५ अस्मिन् मृलक्रमः= ४४५५५ । अत्र सार्पिणी\\
पंक्तिः= १।१।१।१।१।१ 

 मूलक्रमस्थस्थानद्वये समावङ्कौ ततः स्थानत्रये समा अङ्काः।\\
अतः सार्पिण्यां पंक्तौ अन्त्यं त्यक्त्वा उपान्तिमाङ्कद्वययोगेन ततोऽ-\\
ङ्कत्रययोगेन जाता जलौका पंङ्क्ति= ३।२।१॥ 

 एवं तत्र तृतीयोदाहरणे यत्र मूलक्रमः - ३३३३९\\
सार्पिणी पंक्तिः= १।१।१।१।१।१\\
जलौका पक्तिंः= १।४।१\\
१९

( २९०) 

 व्यन्तरा चैव वैश्लेषी\\
सार्पिणी च, वद द्रुतम् ॥ १ ॥
स्थानकेषु चतुर्ष्वत्र\\
लघ्वाङ्काद्रुत्क्रमासमौ ।\\
पंक्तिर्जलौकिकानाम्नी\\
वेत्सि चेदङ्कपाशकम् ॥ २ ॥ 

 न्यासः अत्र साथनानि ४ । एकाद्येकोत्तरा जाता चयपङ्क्तिः\\
१।२।३।४ 

 अत्र चतुःस्थानगता एकाङ्का जाता व्यन्तरा नाम पक्तिंः १।१।१।१\\
अथ चतुःस्थानगताः पुथगेकाङ्का जाता वैश्लेषिणी पंक्तिः\\
१।१।१।१ 

 इयमपि स्थानैकाधिका जाता सार्पिणी पक्तिंः १।१।१।१\\
लब्धाङ्कान् समान् क्रमादित्यालापे कृते योगं कृत्वा जाता\\
जलौकिकाभिधा पक्तिंः १।१।२।९ 

 सामासिकपङ्क्तौ सूत्रम् । 

 १एकाङ्केनौ विन्यस्य प्रथमं\\
तत्संयुतिं पुरो विलिखेत् ।\\
उत्क्रमतोऽन्तिमतुल्य-\\
स्थानाङ्कयुतिं पुरो विलिखेत् ॥ १३ ॥

\begincenter\rule\theta.5\linewidth\theta.5pt\endcenter

 (१) अन्तिमाङ्कतुल्यस्थानाभावे सति पङ्क्तौ यावन्तोऽङ्कास्तेषां\\
युतिरेव तत्पुरः स्थाप्या ।

( २९१) 

 उत्क्रमतोऽन्तिमतुल्य\\
स्थानयुतिं∗ तत्पुरस्ताच्च ।\\
अन्तिमतुल्यस्थाना-\\
भावे तत्संयुतिं पुरस्ताच्च ॥ १४ ॥
एवं सैकसमास-\\
१स्थानासामासिकीयं स्यात् । 

 उदाहरणम् । 
समासे यत्र सप्त स्यु-\\
रन्तिमस्त्रिमितः सखे ।\\
कीदृशी तत्र कथय\\
पङ्क्तिः सामासिकी द्रुतम् ॥ ३ ॥

 अत्र समासः ७ अन्तिमाङ्कः ३ । सैकसमासस्थानमिता\\
यथोक्तकरणेन जाता सामासिकी पङ्किः १।१।२।४।७।१३।२४।४४ 

\begincenter\rule\theta.5\linewidth\theta.5pt\endcenter

 ∗'तत् सर्वसंयुतिं पुरतः' इति पाठोऽनुमीयते ॥ 

 (१) प्रथमं एकाङ्कौ १।१ अनयोर्योगः= २ तत्परोऽङ्कः । तत उत्क्र-
मतोऽन्तिमाङ्कस्थानपर्यन्तमङ्कानां युतिः = २ + १ + १ = ४, अयं\\
तत्पुरोऽङ्कः । पुनरुक्तक्रमतोऽन्तिमाङ्कस्थानपर्यन्तमङ्कानां युतिः=\\
४+२+१=७ एवमग्रेऽपि सैकसमासस्थानपर्यन्तमङ्काः १।१।२।४।\\
७।१३।२४।४४ इयं सामासिकी पङ्क्तिः ।

( २९२) 

पातालपङ्क्तौ सूत्रम् । 

 १सामासिकाख्यपङ्क्ते-\\
रधः खमेकाङ्कमालिखेच्च ततः ॥ १५ ॥
उत्क्रमतोऽन्तिमतुल्य-\\
स्थानाङ्कैक्येन संयुतोऽन्त्योर्द्ध्वः ।\\
तत्तत्पुरतो विलिखे-\\
देवं सर्वेष्वपि पदेषु ॥ १६ ॥
अन्तिमतुल्यस्थानाऽभावे\\
सति संभवे यथायोगः । 

उदाहरणम् । 

 समासे यत्र सप्त स्यु-\\
रन्तिमस्त्रिमितः सखे ।\\
कीदृशी तत्र पाताल-\\
पङ्क्तिका वद वेत्सि चेत् ॥ ४ ॥ 

\begincenter\rule\theta.5\linewidth\theta.5pt\endcenter

 (१) सामासाख्यपङ्क्तरेधः प्रथमाङ्काधः खं शून्यं लिखेत् ततस्त-\\
दग्रे एकाङ्कमालिखेत् । तत उत्क्रमतोऽन्तिमाङ्कतुल्यस्थानाङ्कानामैक्येन\\
ऊर्ध्वः- पातालपङ्क्तिस्थोऽन्त्योऽङ्कः संयुतोऽधःपङ्क्तौ तत्पुरतस्तं\\
योगाङ्क विलिखेदेवं सर्वपदेषु सर्वस्थानेषु विलिखेत् । अन्तिम-\\
तुल्यस्थानाभावे यथासंभवः स्यात् तथा योगः कार्यः । उदाहरणं\\
विलोक्यम् ।

( २९३) 

 अत्र समासः ७ अन्तिमाङ्कः ३ । अतः सामासिका पङ्क्तिः\\
१।१।२।४।७।१३।२४।४४ 

 यथोक्तकरणेन जाता पातालपङ्क्तिः ०।१।२।५।१२।२६।५९।११८\\
गुणोत्तरपङ्क्तौ सूत्रम् । 

 १आदौ रूपं विलिखे-\\
दन्तिमगुणितं पुरः पुनस्तद्वत् ॥ १७ ॥\\
स्थानाधिकं तु यावत् 

 पङ्क्तिर्गुणकोत्तराख्येयम् । 

 उदाहरणम् । 

 अन्तिमाङ्कस्त्रयं यत्र\\
स्थानानि त्रीणि मे सखे ।\\
गुणोत्तराभिधा पङ्क्तिः\\
कीदृग्रूपा वद द्रुतम् ॥ ५ ॥ 

 अत्रान्तिमाङ्कः ३ स्थानानि ३ । यथोक्तकरणेन जाता गुणो-\\
त्तरा पङ्क्तिः १।३।९।२७ 

 आभ्यासिकपङ्क्तौ सूत्रम् । 

 २स्थानाहतोन्तिमाङ्कः 

 सैकः स्थानोनितश्च तच्छेषम् ॥ १८ ॥

\begincenter\rule\theta.5\linewidth\theta.5pt\endcenter

 (१) अन्तिमेनान्तिमाङ्केन गुणितं पुरः अग्रे पुनरन्तिमगुणितं\\
तत्पुरः पुनस्तद्वत् स्थानाधिकं लिखेत् । 

 (२) यथाचार्योक्तोदाहरणे अन्तिमाङ्कः=३, स्थानानि = ३ । स्था-\\
नाहतान्तिमाङ्कः= ३ x ३ = ९ अयं सैकः= १० स्थानसंख्योनितः=

( २९४)

 आभ्यासिक्यां पङ्क्तौ\\
प्रजायते स्थानमानमिह ।\\
अन्तिममितचयपङ्क्ति-\\
स्तदादिमाङ्कं विहाय चाऽन्येऽङ्काः।।१ ९।।\\
अन्तिमहता पुरस्ताद्\\
विन्यस्य पुनःपुनश्चैवम् ।\\
तानेवान्तिमनिघ्नान्\\
यावत् स्थानाङ्कसम्मितिर्भवति।।२ ० ।।\\
पङक्तिरियं गणकाग्र्यैः\\
समीरिताऽऽभ्यासिकी पूर्वैः । 

\begincenter\rule\theta.5\linewidth\theta.5pt\endcenter

१० - ३ = ७ जातं स्थानमानम् । अन्तिमाङ्कमितचयपङ्क्तिः=\\
१।२।३ 

 अस्या आदिमाङ्कं रूपं विहाय परौ २।३ अन्तिमाङ्कहतौ २ x ३\\
=६, ३x३=९, जातौ पङ्क्तौ तत्पुरोऽङ्कौ एवं पङ्क्तिः=\\
१।२।३।६।९ 

 पुनरन्तिमाङ्कमितचयपङ्क्तिः= ३।६।९, अन्तादिमाङ्कं वयं\\
विहाय परौ ६।९ अङ्कौ अन्तिम ३ हतौ १८।२७ तत्पुरो निवेशितौ\\
जाता पङ्क्तिः= १।२।३।६।९।१८।२७ स्थानसंख्यामिता अत्र अङ्काः ।

( २९५) 

 उदाहरणम् । 

 सखेऽन्तिमस्त्रयं यत्र\\
त्रीणि स्थानानि तत्र मे ।\\
कथयाभ्यासिकी पङ्क्ति-\\
रङ्कपाशं प्रवेत्सि चेत् ॥ ६ ॥ 

 अत्रान्तिमाङ्कः ३ स्थानानि ३ । लब्धा स्थानसंख्या ७ अत्र\\
स्थानगाभ्यासिकी पङ्क्तिः १।२।३।६।९।१८।२७ 

 सूचीपङ्क्तौ सूत्रम् । 

  अन्तिममितवैश्लेष-\\
स्थानाङ्कमिताश्च ताः पृथक् स्थाप्याः ॥ २१ ॥
तासां घातः सूची-\\
पङ्क्तिर्नाराचिका वा स्यात् । 

 उदाहरणम् । 

 अन्तिमाङ्कस्त्रयं यत्र 

 स्थानानि त्रीणि कोविद ।\\
तत्र नाराचिका पङ्क्तिः\\
कीदृशी वद वेत्सि चेत् ॥ ७ ॥ 

 अन्तिमाङ्कः ३ स्थानानि ३ । अत्रान्तिमाङ्कमिता वैश्लोषिकी\\
पङ्क्तिः १।१।१ स्थानानि त्रीणीति त्रिधा १।१।१।१।१।१।१।१।१ तासां\\
घात इति कपाटसन्धिविधिना गुणिता जाता सूचीपङ्क्तिः १।३।६।\\
७।६।३।१

( २९६)

यौगिकपङ्क्तौ सूत्रम् । 

 १स्थानाहतोन्तिमाङ्को\\
योगः प्रथमस्तदूनितैकैकः ॥ २२ ॥
यावत्स्थानाङ्कमितः\\
पङ्क्तिरियं यौगिकाख्या स्यात् । 

उदाहरणम् । 

 त्रिसंख्याकोऽन्तिमो यत्र\\
त्रीणि स्थानानि कोविद ।\\
यौगिकाख्या पङ्क्तिराशु\\
कीदृशी वद वेत्सि चेत् ॥ ८ ॥ 

 अत्रान्तिमाङ्कः ३ स्थानानि ३ । स्थानान्तिमाङ्कघातः ९ अयं\\
प्रथमो योगः । एकैकापचितो यावत्स्थानसमाङ्कः स्यात् तावत्\\
कृते जाता यौगिका पङ्क्तिः ९।८।७।६।५।४।३ 

 खण्डसूचीपङ्क्तौ सूत्रम् । 

 रूपोनस्थानोत्थां\\
सूचीं विलिखेच्च यौगिकाऽधस्तात् ॥ २३ ॥

\begincenter\rule\theta.5\linewidth\theta.5pt\endcenter

 (१) स्थानाङ्कमितिः= स्था x अं + १ - स्था = ३x३+१- ३\\
९+१ - ३= १० - ३= ७ । ('स्थानाहतोऽन्तिमाङ्कः सैकः स्थानोनि-\\
तश्च तच्छेषम् ।' इत्यादिना)

( २९७) 

 अङ्काभावे शून्यं\\
समुक्तयोगादधःस्थितानङ्कात् ।\\
उत्क्रमतोऽन्तिमतुल्य\\
स्थानस्थाच्छेषयेद् विलोप्यान्यान् ॥ २४ ॥
खण्डितनाराचोयं\\
पङ्क्तिर्गणकैरिह प्रोक्ता । 

 उदाहरणम् । 

 त्रीणि स्थानान्यन्तिमाङ्क-\\
स्त्रयं योगे तु षड् भवेत् ।\\
खण्डनाराचिका पङ्क्तिः\\
कीदृग्रूपा वदाशु मे ॥ ९ ॥ 

 अन्तिमाङ्कः ३ स्थानानि ३ । योगः ६ अतः कृता यौगिका\\
पङ्क्तिः ९।८।७।६।५।४।३ विरूपस्थाना नाराचपङ्क्तिः १।२।३।२।१\\
पूर्वपङ्क्तेरधो विन्यस्य जातम् ।९१ ८२ ७३ ६२
५१ ४० ३० । अस्मिन् योगः\\
षट् तदधःस्थितादङ्कादुत्क्रमादन्तिमसमानङ्काञ्छेषान् संलोप्य जाता\\
खण्डनाराचिका पङ्क्तिः २।३।३ 

 खण्डमेरौ सूत्रम् । 

 १स्थानमितकोष्ठकाना-\\
मेकान्तानामधोधराश्च यावन्तः ॥ २५ ॥

\begincenter\rule\theta.5\linewidth\theta.5pt\endcenter

 (१०) चयपङ्क्तिः ('एकाद्येकचयाङ्कैः') इत्यादिना श्रेया । यथाचा-\\
र्योक्तोदाहरणे तृतीयोर्ध्वपङ्क्तौ प्रथमं स्थापिता चयपङिक्तः १।२।३।४।

( २९८) 

 तिर्यक्-श्रेण्यः कार्या\\
भवन्ति यावन्त्य ऊर्ध्वाश्च ।\\
तिर्यक्स्थायां पङक्ता-\\
वाद्यायामाद्यकोष्ठके रूपम् ॥ २६ ॥
विलिखेत् परेषु शून्यं\\
तदधःपङ्क्तिष्वथोर्द्ध्वस्थाः ॥
विलिखेच्चयाख्यपङ्क्तीः\\
स्वपङ्क्तिघातेन तानङ्कान् ॥ २७ ॥
गुणयेदेवं गुणिभिः\\
समीरितः खण्डमेरुरयम् ।\\
श्रुतिकोष्ठाङ्कसमासात्\\
सांख्यत्वं जायते नियतम् ॥ २८ ॥ 

\begincenter\rule\theta.5\linewidth\theta.5pt\endcenter

अत्रस्था अङ्काः स्वपङ्क्तिघातेन स्वपङ्क्तिस्थितानामङ्कानां घातेन\\
१.२.३.४=२४ अनेन गुणिता जाताः -२४।४८।७२।९६ अभीष्टा अङ्काः ।\\
एवं सर्वत्र । 

 अत्र कर्णकोष्ठाङ्कसमासात् कर्णकोष्ठगताङ्कयोगात् नियतं\\
सांख्यत्वं भेदप्रमाणं जायते । यथा चतुर्षु स्थानेषु भेदाश्चतुः-\\
कर्णकोष्ठगताङ्कयोगसमा २४ भवन्तीति । उदाहरणेन सर्वं स्फुटम् ।

( २९९) 

 उदाहरणम्। 

 षट्स्थानकः खण्डमेरुः\\
साङ्कः कोष्ठश्च कीदृशः ।\\
अङ्कपाशविधि वेत्सि\\
चेद् दर्शय सखे द्रुतम् ॥ १० ॥

 अत्र स्थानानि षट्- । यथोक्तकरणेन जातः खण्डमेरुः । 



\beginlongtable[]@llllll@
\toprule
\endhead
१  \& ०  \& ०  \& ०  \& ०  \& ०  \\
\& १  \& २  \& ६ \& २४  \& १२०  \\
\& \& ४ \& १२  \& ४८ \& २४० \\
\& \& \& १८ \& ७२ \& ३६० \\
\& \& \& \& ९६ \& ४८० \\
\& \& \& \& \& ६०० \\
\bottomrule
\endlongtable

 अथ पताकासूत्रम् ।

 १नाराचपंक्त्यङ्कमिताः\\
कोष्ठानामूर्ध्वपंक्तयः ।\\
तिर्यग्गामी च सर्वासां\\
स्वस्वखण्डावसानमा ॥ २९ ॥

\begincenter\rule\theta.5\linewidth\theta.5pt\endcenter

 (१) अन्तिमाङ्कस्थानवशेन प्रथमं नाराचा पङ्क्तिः कर्त्तव्या । तत्र\\
येऽङ्कास्तन्मिताः क्रमेणोर्ध्वकोष्ठकाः कार्याः । एवमूर्ध्वपङ्क्तयः\\
स्युः । एवं स्वस्वखण्डावसानमा स्वस्वखण्डाङ्कमिता पङ्क्ति-\\
र्भवति ।

( ३००)

 पंक्तिस्तदाद्यकोष्ठो\\
यः पञ्चवोऽथाङ्कयोजनाः ।\\
तिर्यक्-स्थितायामाद्या-\\
यां पंक्तिमाभ्यासिकीं लिखेत् ॥ ३० ॥
तदन्तिमाङ्कः क्षेपाख्यः\\
पुरःस्थः साध्यनामकः ।\\
क्षेपं पुरातनैरङ्कैः\\
क्रमात् संयोजयेत् पृथक् ॥ ३९ ॥
तानधस्तिर्यगायां च\\
कोष्ठपंक्त्यां विनिक्षिपेत् ।\\
साध्याङ्कस्य पताका स्यात्\\
साध्ये क्षेपं प्रकल्पयेत् ॥ ३२ ॥
साध्यं पुरःस्थितं कृत्वा\\
क्षेपं प्राग्वत् पुरातनैः ।\\
अङ्कैराद्यद्वितोयादि-\\
कोष्ठपंक्तिगतैर्युतम् ॥ ३३ ॥

( ३०१) 

 तिर्यङ्निरङ्ककोष्ठेषु\\
साङ्काऽस्तेषु विन्यसेत् ।\\
येनाऽङ्केन युतः क्षेपः\\
साध्याङ्कास्तदधो यदा ॥ ३४ ॥
तदा मुक्त्वा तमङ्कं तु\\
योजयेदितराँस्ततः ।\\
गुणोत्तराङ्के साध्ये तु\\
यदा पल्लवपूर्वकान् ॥ ३५ ॥
कोष्ठान् साङ्कान् पुनः कृत्वा\\
पताकानिर्णयोऽप्ययम् । 
उदाहरणम् ।\\
 अन्तिमाङ्कस्त्रयं यत्र\\
स्थानानि त्रीणि मे सखे ।\\
पताका कीदृशी तत्र\\
दर्शयाशु प्रवेत्सि चेत् ॥ ११ ॥ 
अत्रान्तिमाङ्कः ३ स्थानानि ३ । अतो नाराची १।३।६।७।६।३।१\\
आभ्यासिकी १।२।३।६।९।१८।२७ गुणोत्तरा च १।३।९।२७ यथोक्त-\\
करणेन जाता पताका ।

( ३०२)



\beginlongtable[]@lllllll@
\toprule
\endhead
१  \& २  \& ३  \& ६  \& ९  \& १८  \& २७ \\
\& ४ \& ५ \& ८ \& १५  \& २४ \& \\
\& १०  \& ७ \& १२  \& १७ \& २६ \& \\
\& \& ११  \& १४  \& २१ \& \& \\
\& \& १३ \& १६ \& २३ \& \& \\
\& \& १९ \& २० \& २५ \& \& \\
\& \& \& २२ \& \& \& \\
\bottomrule
\endlongtable

 

 सुमेरौ सूत्रम्।

 एकाद्येकोत्तराः कार्या\\
अधोऽधः कोष्ठपंक्तयः ।\\
सरूपस्थानसंख्याश्च\\
तास्वाद्यायां च रूपकम् ॥ ३६ ॥
पंक्तौ लिखेद् द्वितीयायां\\
मेरोरस्य तदादिमे ।\\
कोष्ठेऽन्तिमं विरूपं च\\
लिखेच्छृङ्गाभिधं भवेत् ॥ ३७ ॥
परस्मिन् कोष्ठके रूपं\\
स्वकोष्ठोर्द्ध्वस्थितश्च यः ।\\
शृङ्गघ्नस्तमधो न्यस्य\\
वामकर्णाङ्ककोष्ठयुक् ॥ ३८ ॥

( ३०३) 

 १क्रमादेवं तिर्यगासु\\
कोष्ठपंक्तिष्वयं विधिः ।\\
सुमेरुकरणे प्राज्ञैः\\
प्रोक्तं गणितवेदिभिः ॥ ३९ ॥

 उदाहरणम् । 

 अन्तिमाङ्कस्त्रयं यत्र\\
स्थानानि त्रीणि कोविद ।\\
सुमेरुः कीदृशश्चाऽत्र\\
यदि वेत्सि निगद्यताम ॥ १२ ॥ 

\begincenter\rule\theta.5\linewidth\theta.5pt\endcenter

 (१) अस्य मेरोर्द्वितीयायां पङ्क्तावादिमे कोष्ठेऽन्तिमाङ्कमन्तिम-\\
मङ्कं विरूपमेकोनं लिखेत् । एतदङ्कस्य शृङ्गं नाम ज्ञेयम् । द्वितीयायां\\
पङ्क्तावपरस्मिन् कोष्ठे रूपं लिखेत् । अथ तृतीयपङक्तौ कोष्ठकाङ्क-\\
निरूपणम् । यस्य कोष्ठस्याङ्कज्ञानमपेक्षितं तदूर्ध्वकोष्ठकाङ्कः शृङ्ग-\\
घ्नस्तद्वामकर्णाङ्केन युक्तः कार्यः । एवं तत्कोष्ठशानं भवति ।\\
यथाचार्योक्तोदाहरणे तिर्यक् पङ्क्तित्रये प्रथमकोष्ठकस्योपरि\\
स्थितोऽङ्कः २ शृङ्गेण २ निघ्नः ४ । वामकर्णाभावादयमेवाङ्कस्तत्र\\
स्थाप्यः । द्वितीयकोष्ठोपरिष्ठोऽङ्कः १ अयं शृङ्गा २ घ्नः २ द्वितीय-\\
कोष्ठवामकर्णाङ्केन २ युक्तो जातस्तत्कोष्ठकाङ्कः ४ । एवं तृतीय-\\
कोष्ठकोपर्यङ्कस्याभावात् शृङ्गघ्नफलं शून्यं तत्तद्वामकर्णाङ्केन रूपेण\\
युक्तं जातस्तत्कोष्ठाङ्कः रूपम् । एवं सर्वासु तिर्यक्पङ्क्तिषु अङ्क-\\
स्थापनं भवति । तत्र सर्वान्तिमकोष्ठेषु पूर्ववर्णितनियमानुसारेण\\
रूपमेव भवति-इति सर्वं क्षेत्रतः स्फुटमेव ।

( ३०४) 

 न्यासः । अत्रान्तिमाङ्कः ३ स्थानानि ३ । यथोक्तकरणेन\\
जातः सुमेरुः

 



\beginlongtable[]@llll@
\toprule
\endhead
१  \& \& \& \\
२ \& १ \& \& \\
४ \& ४  \& १  \& \\
८ \& १२  \& ६ \& १  \\
\bottomrule
\endlongtable



 मत्स्यमेरौ सूत्रम् । 

 रूपमादिनिरेकान्ति-\\
माङ्कवृद्ध्यङ्कपंक्तिका ।\\
स्थानमेकाधिकं यावत्\\
तन्मिताः कोष्ठपंक्तयः ॥ ४० ॥
मुक्त्वा स्वोर्द्ध्वादिमं कोष्ठं\\
द्वितीयस्याऽधरोधराः ।\\
पंक्तयस्तिर्यगाः कार्याः\\
अङ्कैक्येन समा अथ ॥ ४१ ॥
अङ्कविन्यस्यमाद्यायां\\
पंक्तौ रूपं च विन्यसेत् ।\\
उत्क्रमादन्तिमसमः\\
स्वोर्ध्वकोष्ठाङ्कसंयुतिः ॥ ४२ ॥
अधो लिखेदन्तिमाङ्के\\
समाभावो भवेद् यदि ।

( ३०५) 

 तथा यथासम्भवाङ्क-\\
योगः कार्यः क्रमेण च ॥ ४३ ॥
नाराच्यस्तिर्यगास्थान-\\
सम्मितास्तद्युतिः पृथक् ।\\
गुणोत्तरा भवेत् पंक्ति-\\
रूर्ध्वा अङ्कैक्यसम्मिताः ॥ ४४ ॥
पृथक् तदूर्ध्वकोष्ठांकयोगात्\\
सामासिका भवेत् ॥ 

 उदाहरणम्। 

 अन्तिमाङ्कस्त्रयं यत्र\\
स्थानानि त्रीणि कोविद । 

 अङ्कैक्ये तु भवेत् सप्त\\
तत्र मत्स्यगिरिः कथम् ॥ १३ ॥ 

 अन्तिमाङ्कः ३ । स्थानानि ३ । अङ्कैक्यम् ७ । अत्रापि\\
रूपादिनिरेकान्तिमाङ्कवृद्ध्या जाता कोष्ठपंक्तिः १।३।५।७ उक्तवत्\\
कृतो मत्स्यमेरुः । अस्मिन् मत्स्यमेरौ नाराच्यः खण्डनाराच्यः\\
पक्तयः सम्भवन्ति। इत्यङ्कपाशे साधनसूत्राणि । 

दर्शनम् 

\beginlongtable[]@llllllll@
\toprule
\endhead
१  \& \& \& \& \& \& \& \\
१ \& १  \& १  \& \& \& \& \& \\
\& २  \& २  \& ३  \& २  \& १ \& \& \\
\& १ \& ३ \& ६ \& ७ \& ६  \& ३  \& १  \\
\& \& १ \& ४ \& १०  \& १६  \& \& \\
\& \& \& १ \& ५ \& १५ \& \& \\
\& \& \& \& १ \& ६ \& \& \\
\& \& \& \& \& १०  \& \& \\
\bottomrule
\endlongtable



 २०

( ३०६)

 अथ नियतस्थानगैर्नियताङ्कैर्भेदावृत्त्यूर्ध्वसर्वयोगाङ्केषु सूत्रम् ।

 १अङ्कष्वसमेषु चया-\\
ऽङ्केपंक्तिघातो भिदां मितिर्भवति । 

\begincenter\rule\theta.5\linewidth\theta.5pt\endcenter

 (१) असमेष्वङ्केषु अङ्कस्थानपर्यन्तं चयाङ्कपंक्तिघातः कार्यः ।\\
चयपंक्तिश्च 'एकाद्येकचयाङ्कैः स्थानान्तम्' इत्यादिविधिना । स\\
घातो भिदां भेदानां मितिर्भवति । 

 सार्वश्रेण्यङ्काः सार्पिणीपंक्तिस्था अंकाः संख्याभेदैस्ताडिताः ।\\
तत्रावसानमत्यघातफलं मुक्त्वा विहायान्ये गुणिताः स्थानसंख्यया\\
भक्ता लब्धा उत्क्रमतोऽल्पादिकान्ता भेदाः स्युः । अन्तिमफल-\\
मल्पाङ्कान्तभेदमितिः । यत्राल्पाङ्कोऽन्त्ये तिष्ठति तेषां भेदानां
मिति-\\
र्भवति । उपान्तिमफलं तदल्पाधिकाङ्कान्तभेदमिति । एवमग्रे च\\
ज्ञेयमित्यर्थः । एवमुत्क्रमतो येऽल्पादिकान्तभेदास्ते निजैर्निजैरङ्कै-\\
र्गुणिताः पृथक् पृथक् स्वस्वभेदाः स्युः । एवं निजैरङ्कैर्हतानां\\
स्वभिदां योगस्तदूर्ध्वस्थो योगो भवति । ऊर्ध्वाधरस्थापितानां\\
सर्वभेदानामूर्ध्वाधरपंक्तिस्थितानामङ्कानां योग ऊर्ध्वस्थो योगः\\
कथ्यते । स योगो व्यन्तरपंक्तिस्थाङ्केन गुणितः सर्वभेदानां योगो\\
भवति । अन्त्यादिभेदमानं अल्पाङ्कान्तादिभेदमानं स्थानकैः स्था-\\
नाङ्कैर्हतं तदङ्काः स्युः । तेषां योगश्चाङ्कनिपातो यावत्सु स्थानेषु\\
अङ्कानां निपातः स्थितिरस्ति तेषां मानं भवेत् ।\\
अत्रोपपत्त्यर्थं श्रीमज्जनकशोधितभास्करलीलावत्यां तट्टिप्पणी\\
विलोक्या । प्रस्तारक्रमतोऽल्पाङ्कान्तादिभेदमानं स्फुटम् । ऊर्ध्वस्थ-\\
योगाद्यानयनोपपत्तिरतिसुगमा ।\\
एवं यदाऽसमाङ्कास्तदा जलौकया पंक्तयाऽयं विधिः । यदाऽङ्काः\\
समास्तदापि जलौकयैव सर्वमूर्ध्वयोगादि भवति इत्यग्रे वक्ष्यते\\
चाचार्येण ।

( ३०७) 

 संख्या मूर्त्तीनामपि\\
शस्त्रैरसमाननामभिर्ज्ञेया ॥ ४५ ॥
संख्याभेदैः सार्प-\\
श्रेण्यङ्कास्ताडितास्तदवसानम् ।\\
मुक्त्वाऽन्ये स्थानाप्ता\\
उत्क्रमतोऽल्पादिकान्तभेदाः स्युः ॥ ४६ ॥
अङ्कैर्निजैर्हतास्ते\\
उत्क्रमतोऽल्पादिकान्तभेदाः स्युः ।\\
अङ्कैर्निजैर्हतानां\\
स्वभिदां योगो भवेत् तदूर्ध्वस्थः ॥
सा व्यन्तरपंक्तिघ्नो\\
सर्वभिदां जायते योगः ॥ ४७ ॥
अन्त्यादिभेदमानं\\
पृथग्घतं स्थानकैस्तदङ्काः स्युः ।\\
तद्योगोङ्कनिपातो\\
जलौकपंक्तौ विधिश्चायम् ॥ ४८ ॥

 उदाहरणम् । 

 नागाग्निरन्ध्रैर्द्विगुणोङ्गचन्द्रै-\\
र्वदाशु रूपादि नवावसानैः ।

( ३०८)

 भेदाँश्च लब्ध्यङ्कमुखान्त्यभेदा-\\
नूर्ध्वाङ्कयोगं सकलाङ्कयोगम् ॥ १ ४ ॥
अङ्कप्रपातं च सखे पृथक् ते\\
वदाऽङ्कपाशेऽस्ति परिश्रमश्चेत् । 

 प्रथमोदाहरणे न्यासः । ७,३,९ । अत्र त्रिस्थानचयपंक्तिः\\
१।२।३ एषां घाते जाताः संख्याभेदाः ६ । एते त्रिस्थानसार्पश्रेण्या\\
१।१।१।१ हताः ६।६।६।६ एषामवसानाङ्कं त्यक्त्वाऽन्येऽङ्काः स्थानकै-\\
स्थिभिर्भकाः २ । उत्क्रमाज्जाता भेदास्त्र्यन्ताः सप्तान्ता नवान्ता वा\\
९७३२२२ । ९३७२२२ स्वभेदाः स्वाङ्कगुणिताः १८।१४।६ एषां योगः ३८-\\
ऊर्ध्वपंक्तियुतिः । इयं व्यन्तरया १११ हता जातः सर्वयोगः ४२१८ ।\\
भेदाः ६ स्थानैः ३ गुणिना जातोऽङ्कपातः १८ । 

 द्वितीयोदाहरणे न्यासः । १,६,३,२ । अत्र चयपंक्तिः\\
१।२।३।४ एषां घाते जाताः संख्याभेदाः २४ । एते चतुःस्थानसार्प-\\
श्रेण्या १।१।१।१।१ हताः २४।२४।२४।२४।२४ एषामवसानाङ्कं त्यक्त्वा-\\
ऽन्येऽङ्काः स्थानैः ४ भक्ता जाता उत्क्रमाद् भेदा एकान्त-द्व्यन्त-\\
त्र्यन्त-षडन्ताः- ६३२१६६६६।भे२४\ स्वभेदाः स्वाङ्कगुणिताः
३६।१८।\\
१२।६ प्रषां योगे जातोर्ध्वयुतिः ७२ । व्यन्तरया ११११ हतो जातः\\
सर्वयोगः ७९९९२ । भेदस्थानघातोऽङ्कपातः ९६ । 

 तृतीयोदाहरणे न्यासः । ९।८।७।६।५।४।३।२। १ यथोक्तकरणेन\\
जाताः संख्याभेदाः ३६२८८० । उत्क्रमभेदाः एकान्तादयश्च ।\\
९ ८ ७ ६ ५ ४ ३ 

४०३२० । ४०३२० । ४०३२० । ४०३२० । ४०३२० । ४०३२० । ४०३२० ।\\
२ १\\
४०३२० । ४०३२० । ऊर्ध्वयुतिः १८१४४०० । सर्वयोगः

( ३०९) 

२०१५९९९९७९८४०० । अङ्कपातः ३२६५९२० एवमसमानामङ्कानां\\
सर्वकरणम् । 

 उदाहरणम् । 

 चापेषु खङ्गडमरूककपालपाशैः\\
खट्वाङ्गशूलफणिशक्तियुतैर्भवन्ति ।\\
अन्योन्यहस्तकलितैः कति मूर्तिभेदाः\\
शम्भो हरेरिव गदारिसरोजशंखैः ॥ १५ ॥

 प्रथमोदाहरणे शम्भोः शस्त्राणि १० 'मूर्तयः शस्त्रैः' इति\\
दशस्थान चयपंक्तिघाते जाताः शम्भोर्मूर्तिभेदाः ३६२८८०० ।\\
द्वितीयोदाहरणे हरेः शस्त्राणि ४ । प्राग्वज्जाता मूर्तिभेदाः २४ ।\\
असमानि शस्त्राणि यतस्ता एव मूर्तयो भवन्ति ।\\
इति नियतस्थानाङ्कसंख्यालब्धादिभेदोर्ध्वयोगाङ्कपातप्रत्ययाः ।\\
प्रस्तारप्रत्यये सूत्रम् ।



 लघूपूर्वोद्दिष्टाङ्क- 

 न्यासो यः स क्रमाह्वयो ज्ञेयः ।\\
न्यस्ताऽल्पमाद्यान्महतो-\\
ऽधस्ताच्छेषंयथोपरि तथास्यात् ॥ ४९ ॥\\
मूलक्रमं तदूने\\
यावत् क्रममुत्क्रमं याति ।\\
अङ्कप्रस्तारविधि-\\
श्चैवं मूर्तिप्रभेदानाम् ॥ ५० ॥

( ३१०) 

 १सरिगमपधनीत्येषां 

 वीणाया निक्वणानां च । 

 इति प्रस्तारविधिः प्रदर्शितः ।

\begincenter\rule\theta.5\linewidth\theta.5pt\endcenter

 (१) ये उद्दिष्टाङ्कास्तेषां लघुपूर्वाणां न्यासो यः स क्रमसंज्ञको\\
ज्ञेयो मूलक्रमो वा । अथैकस्माद्भेदादन्यो यद्यपेक्षितस्तदा तद्भेदे\\
ह्याद्याद्योऽल्पो लघ्वङ्कस्तमाव्यवहितस्य तद्भेदस्थस्य महतो\\
बृहदङ्कस्याधस्तादधो न्यस्य शेषं यथोपरि स्यात् । अत्रैतदुक्तं\\
भवति । शेषान् दक्षिणभागस्थितान् तद्भेदाङ्कानधो न्यस्तलघ्वङ्क-\\
दक्षिणभागे स्थापयेत् । तदूने वामभागस्थाङ्काभावस्थानेषु मूलक्रमं\\
मूलस्थानावशिष्टाङ्कान् क्रमेण स्थापयेत् । एवं तावत् कर्म\\
कर्त्तव्यं यावत् मूलक्रम उत्क्रमं याति । मूलक्रमस्थिताङ्का यस्मिन्\\
भेदे उत्क्रमेण भवन्ति सोऽन्तिमो भेदो भवतीत्यर्थः । यथाचा-\\
र्योक्तद्वितीयोदाहरणे १।६।३।२ मूलक्रमः= १२३६ अयं प्रथमो भेदः ।\\
द्वितीयभेदार्थे अस्य महतोऽङ्कस्य '२' अस्याधो लघुं रूपं न्यस्याग्रे\\
दक्षिणभागे उपरि स्थितौ '३६' अङ्कौ स्थापितौ वामभागे च मूलक्र-\\
मावशिष्टाङ्कः '२' स्थापितः । एवं जातो द्वितीयभेदः= २१३६ ।\\
अस्मादद्वितीयभेदात् तृतीयभेदानयनार्थम् । 

 आद्यो लघ्वङ्कः '२' तद्भेदस्थस्य बृहदङ्काव्यवहितस्य '३' अस्याधः\\
स्थापितस्तदग्रे दक्षिणभागे उपरि स्थितोङ्कः '६' वामभागे च मूलक्र-\\
मावशिष्टाङ्कौ क्रमागतौ '१।३' स्थापितौ । एवं जातस्तृतीयो\\
भेदः= १३२६ । एवं तृतीयाच्चतुर्थश्चतुर्थात् पञ्चमः । इत्यादयो\\
भेदाः साधनीयाः । अन्तिमभेदस्तु मूलक्रमोत्क्रमः= ६३२१ भवि-\\
ष्यतीति । चतुर्थभेदात् ३१२६ अस्मात् पञ्चमभेदानयने च आद्याद्\\
द्वितीयं रूपं लघु ग्राह्यम् । यतस्तदव्यवहिते दक्षिणभागे मूलक्रमे\\
तदीयो महान् ।

( ३११)

 उदाहरणम् । 

 आद्यद्वितीययोर्ब्रूहि\\
प्रस्तारं प्रश्नयोः सखे ।\\
अङ्कपाशाभिधे त्वं चेत्\\
प्रौढतां प्राप्तवानसि ।।। १६ ॥ 

 प्रथमोदाहरणे न्यासः ७।३।९ एते लघुपूर्वकाः स्थापिताः ।\\
जातो मूलक्रमः । 'न्यस्याल्पमाद्यान्महत' इत्यादिना जातः\\
प्रस्तारः । आवृत्तिः २ । आवृत्तिरिति द्विवारं सर्वाङ्कानामाव-\\
र्तनम् । ऊर्ध्वयोगः ३८ सर्वयोगः ४२१८ अङ्कपातः १८ प्रस्तार-\\
दर्शनम्।

 ३७९\\
७३९\\
३९७\\
९३७ \ द्वितीयोदाहरणे न्यासः १।६।३।२ अतो मूलकमः १२३६\\
७९३ प्रस्तारदर्शनम्।\\
९७३





\beginlongtable[]@llllllll@
\toprule
\endhead
१  \& १२३६  \& ७  \& १२६३ \& १३  \& १३६२ \& १९ \& २३६१  \\
२ \& २१३६ \& ८ \& २१६३ \& १४ \& ३१६२  \& २० \& ३२६१ \\
३ \& १३२६ \& ९ \& १६२३ \& १५ \& १६३२ \& २१ \& २६३१ \\
४ \& ३१२६ \& १० \& ६१२३ \& १६ \& ६१३२ \& २२ \& ६२३१ \\
५ \& २३१६ \& ११  \& २६१३ \& १७ \& ३६१२ \& २३ \& ३६२१ \\
६ \& ३२१६ \& १२ \& ६२१३  \& १८ \& ६३१२ \& २४  \& ६३२१ \\
\bottomrule
\endlongtable

  आवृत्तिः ६ । ऊर्ध्वयोगः ७२ सर्वयोगः ७९९९२ अङ्कपातः ९६ । 

\begincenter\rule\theta.5\linewidth\theta.5pt\endcenter

 भास्करलीलावतीटीकायां मुनीश्वरकृतायां निसृष्टार्थदूत्यभिधायां\\
उद्दिष्टाङ्कान् क्रमान्त्यस्य स्थाप्यः पूर्वः परादधः ।\\
स चेदुपरि तत्पूर्वः परस्तूपरिवर्त्तिनः ॥\\
उद्दिष्टाङ्कक्रमात् पृष्ठे शेषाः प्रस्तार ईदृशः ।

( ३१२) 

 अपि च । 

 मुरारेर्मूर्तिभेदानां\\
प्रस्तारः कीदृशः सखे ।\\
अङ्कपाशाभिधं वारि-\\
निधिं तर्तुं क्षमोऽसि चेत् ॥ १७ ॥

 



\beginlongtable[]@llllllll@
\toprule
\endhead
१  \& प.ग.च.श.  \& ७  \& ग.च.श.प.  \& १३  \& च.श.प.ग.  \& १९ \&
श.प.ग.च.  \\
२ \& प.ग.श.च.  \& ८ \& ग.च.प.श.  \& १४ \& च.श.ग.प.  \& २० \&
श.प.च.ग.  \\
३ \& प.च.श.ग.  \& ९ \& ग.श.प.च.  \& १५ \& च.प.ग.श.  \& २१ \&
श.ग.च.प.  \\
४ \& प.च.ग.श.  \& १० \& ग.श.च.प.  \& १६ \& च.प.श.ग.  \& २२ \&
श.ग.प.च.  \\
५ \& प.श.ग.च.  \& ११  \& ग.प.च.श.  \& १७ \& च.ग.श.प.  \& २३ \&
श.च.प.ग.  \\
६ \& प.श.च.ग.  \& १२ \& ग.प.श.च.  \& १८ \& च.ग.प.श. \& २४  \&
श.च.ग.प.  \\
\bottomrule
\endlongtable



 अत्र मुरारेः शास्त्राणां पद्मगदाशङ्खचक्राणांनामाद्याक्षराणि
प्रस्तारे\\
लिखितानि । एवं शम्भोर्मूर्तीनां प्रस्तारेः ॥

 उद्दिष्टे सूत्रम् । 

 स्थानमितखण्डमेरो-\\
र्निरङ्ककोटेषु लोष्टकाः स्थाप्याः ।\\
उद्दिष्टाङ्के योऽन्त्यः\\
सोऽन्त्यान्मूलस्य यावतिथः ॥ ५१ ॥\\
तावतिथेऽधः कोष्ठे\\
परिक्षिपेल्लोष्टकं च दलमेरोः ।\\
मूलक्रम उद्दिष्टे 

 लोपस्तस्योभयोः पुनर्यावत् ॥ ५२ ॥



 \end{document}