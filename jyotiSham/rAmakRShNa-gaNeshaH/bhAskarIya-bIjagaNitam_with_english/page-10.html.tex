
\documentclass[]{article}
\usepackage{lmodern}
\usepackage{amssymb,amsmath}
\usepackage{ifxetex,ifluatex}
\usepackage{fixltx2e} % provides \textsubscript
\ifnum 0\ifxetex 1\fi\ifluatex 1\fi=0 % if pdftex
  \usepackage[T1]{fontenc}
  \usepackage[utf8]{inputenc}
\else % if luatex or xelatex
  \ifxetex
    \usepackage{mathspec}
  \else
    \usepackage{fontspec}
  \fi
  \defaultfontfeatures{Ligatures=TeX,Scale=MatchLowercase}
\fi
% use upquote if available, for straight quotes in verbatim environments
\IfFileExists{upquote.sty}{\usepackage{upquote}}{}
% use microtype if available
\IfFileExists{microtype.sty}{%
\usepackage[]{microtype}
\UseMicrotypeSet[protrusion]{basicmath} % disable protrusion for tt fonts
}{}
\PassOptionsToPackage{hyphens}{url} % url is loaded by hyperref
\usepackage[unicode=true]{hyperref}
\hypersetup{
            pdfborder={0 0 0},
            breaklinks=true}
\urlstyle{same}  % don't use monospace font for urls
\IfFileExists{parskip.sty}{%
\usepackage{parskip}
}{% else
\setlength{\parindent}{0pt}
\setlength{\parskip}{6pt plus 2pt minus 1pt}
}
\setlength{\emergencystretch}{3em}  % prevent overfull lines
\providecommand{\tightlist}{%
  \setlength{\itemsep}{0pt}\setlength{\parskip}{0pt}}
\setcounter{secnumdepth}{0}
% Redefines (sub)paragraphs to behave more like sections
\ifx\paragraph\undefined\else
\let\oldparagraph\paragraph
\renewcommand{\paragraph}[1]{\oldparagraph{#1}\mbox{}}
\fi
\ifx\subparagraph\undefined\else
\let\oldsubparagraph\subparagraph
\renewcommand{\subparagraph}[1]{\oldsubparagraph{#1}\mbox{}}
\fi

% set default figure placement to htbp
\makeatletter
\def\fps@figure{htbp}
\makeatother


\date{}

\begin{document}

{\ldots{}.८\ldots{}.}

{If you know how to add positive and negative numbers given the sum of
(1) minus 3 and minus 4 (2) plus 3 and plus 4 (3) plus 3 and minus 4 (4)
minus 3 and plus 4 separately.}

{अत्र रूपाणामव्यक्तानां चाऽऽद्याक्षराणि उपलक्षणार्थं लेख्यानि । }

{तथा यान्यूणगतानि तानि ऊर्ध्वबिन्दूनि चेति।।५।। }

{Here numbers and unknowns are symbolised by letters of the alphabet.
And negative numbers are shown by a dot on them. }

{एवं भिन्नेष्वपीति ।।६ ।। }

{Fractions are added in a similar way.}

{संशोध्यमानं स्वमृणत्वमेति स्वत्वं क्षयस्तद्युतिरूक्तवच्च ।। ७ ।। }

{The number to be subtracted if positive is made negative and if
negative is made positive and then added as per rule.}

{त्रयाद् द्वयं स्वात् स्वमृणादृणं च व्यस्तं च संशोध्य वदाऽऽशु शेषम् ।। ८
।। }

{Subtract (1) + 2 from +3 (2) }{2 }{from }{3}{ (3) }{2 }{from +3 (4) +2
from}{ 3}{ and give the remainders quickly.}

{स्वयोरस्वयोः स्वं वधः स्वर्णघाते क्षयः ।। ९ ।। }

{Product of two positive quantities or of two negative quantities is
positive. The product of positive and negative quantities is negative.}

{धनं धनेनर्णमृणेन निघ्नं द्वयं त्रयेण स्वमृणेन किं स्या}{त्}{ ।। १०।। }

{What is the product of (1) + 2 and + 3 (2) -2 and -3 (3) +2 and -3?}

{भागहारेऽपि चैवं निरुक्तम् ।। ११ ।। }

{Similar rule applies to division.}

{रूपाष्टकं रूपचतुष्टयेन धनं धनेनर्णमृणेन भक्तम् । }

{ऋणं धनेन स्वमृणेन किं स्याद् द्रतं वदेदं यदि बोबधीषि ।। १२ ।।}{\\
}

\end{document}
