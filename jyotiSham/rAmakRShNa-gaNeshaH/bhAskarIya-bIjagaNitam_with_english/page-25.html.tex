\documentclass[]{article}
\usepackage{lmodern}
\usepackage{amssymb,amsmath}
\usepackage{ifxetex,ifluatex}
\usepackage{fixltx2e} % provides \textsubscript
\ifnum 0\ifxetex 1\fi\ifluatex 1\fi=0 % if pdftex
  \usepackage[T1]{fontenc}
  \usepackage[utf8]{inputenc}
\else % if luatex or xelatex
  \ifxetex
    \usepackage{mathspec}
  \else
    \usepackage{fontspec}
  \fi
  \defaultfontfeatures{Ligatures=TeX,Scale=MatchLowercase}
\fi
% use upquote if available, for straight quotes in verbatim environments
\IfFileExists{upquote.sty}{\usepackage{upquote}}{}
% use microtype if available
\IfFileExists{microtype.sty}{%
\usepackage[]{microtype}
\UseMicrotypeSet[protrusion]{basicmath} % disable protrusion for tt fonts
}{}
\PassOptionsToPackage{hyphens}{url} % url is loaded by hyperref
\usepackage[unicode=true]{hyperref}
\hypersetup{
            pdfborder={0 0 0},
            breaklinks=true}
\urlstyle{same}  % don't use monospace font for urls
\IfFileExists{parskip.sty}{%
\usepackage{parskip}
}{% else
\setlength{\parindent}{0pt}
\setlength{\parskip}{6pt plus 2pt minus 1pt}
}
\setlength{\emergencystretch}{3em}  % prevent overfull lines
\providecommand{\tightlist}{%
  \setlength{\itemsep}{0pt}\setlength{\parskip}{0pt}}
\setcounter{secnumdepth}{0}
% Redefines (sub)paragraphs to behave more like sections
\ifx\paragraph\undefined\else
\let\oldparagraph\paragraph
\renewcommand{\paragraph}[1]{\oldparagraph{#1}\mbox{}}
\fi
\ifx\subparagraph\undefined\else
\let\oldsubparagraph\subparagraph
\renewcommand{\subparagraph}[1]{\oldsubparagraph{#1}\mbox{}}
\fi

% set default figure placement to htbp
\makeatletter
\def\fps@figure{htbp}
\makeatother


\date{}

\begin{document}

{\ldots{}.23\ldots{}.}

{adding two products of x and y we get a new ह्रस्व. By multiplying the
product of two first variables with प्रकृति and adding to it the product
of the second variables we get new y. Product of two augments (i. e.
क्षेप) gives new value for b.}

{another method. Take x}{1}{y}{2}{ - x}{2}{y}{1}{ as new x;
ax}{1}{x}{2}{ - y}{1}{y}{2}{ as new y and b}{1}{b}{2}{ as new b.}

{इष्टवर्गप्रकृत्योर् यद् विवरं तेन वा भजेत् }

{द्विघ्नमिष्टं कनि}{ष्ठ}{ तत्पदं स्याद् एकसंयुतौ । }

{ततो ज्येष्ठीमिहानन्त्यं भावनातस्तथेष्टतः ।। ७३ ।। }

{If the absolute number is 1, then 2x/a-x}{2}{ may be taken as new x and
from that new y can be obtained. From these values we can get infinite
values for the triad (x, y, b) by the application of bhavana process and
इष्ट as given in stanzas 71 and 72.}

{को वर्गोऽष्टहतः सैकः कृतिः स्याद् गणकोच्यताम् । }

{एकादशगुणाः को वा वर्गः सैकः कृतिः सखे ।। ७४ ।। }

{Give the rational solutions for}

{(1) 8x}{2}{ + 1 = y}{2}{ and (2) 11x}{2}{ + 1 = y}{2}

{ह्रस्वज्येष्ठपदक्षेपान् भाज्यप्रक्षेपभाजकान् । }

{कृत्वा कल्प्यो गुणस्तत्र तथा प्रकृतितश्च्युते । }

{गुणवर्गे प्रकृत्योनेऽथवाऽल्पं शेषकं यथा । }

{तत्तु क्षेपहृतं क्षेपो व्यस्तः प्रकृज्यतितश्च्युते । }

{गुणलब्धिः पदं ह्रस्वं ततो ज्येष्ठमतोऽसकृत् । }

{त्यक्त्वा पूर्वपदक्षेपांश्चक्रवालम् इदं जगुः । }

{चतुर्द्व्येकयुतावेवमभिन्ने भवतः पदे- । }

{चतुर्द्विक्षेपमूलाभ्यां रूपक्षेपार्थभावना ।। ७५ ।। }

{For a given prakriti, a let us assume x, y and b satisfying, ax}{2}{ +
b =y}{2}{. We now have knttak where these x, y, b are respectively
dividend, aug-\\
}

\end{document}
