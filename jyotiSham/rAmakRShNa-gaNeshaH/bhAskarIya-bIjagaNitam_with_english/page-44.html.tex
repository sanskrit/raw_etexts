\documentclass[]{article}
\usepackage{lmodern}
\usepackage{amssymb,amsmath}
\usepackage{ifxetex,ifluatex}
\usepackage{fixltx2e} % provides \textsubscript
\ifnum 0\ifxetex 1\fi\ifluatex 1\fi=0 % if pdftex
  \usepackage[T1]{fontenc}
  \usepackage[utf8]{inputenc}
\else % if luatex or xelatex
  \ifxetex
    \usepackage{mathspec}
  \else
    \usepackage{fontspec}
  \fi
  \defaultfontfeatures{Ligatures=TeX,Scale=MatchLowercase}
\fi
% use upquote if available, for straight quotes in verbatim environments
\IfFileExists{upquote.sty}{\usepackage{upquote}}{}
% use microtype if available
\IfFileExists{microtype.sty}{%
\usepackage[]{microtype}
\UseMicrotypeSet[protrusion]{basicmath} % disable protrusion for tt fonts
}{}
\PassOptionsToPackage{hyphens}{url} % url is loaded by hyperref
\usepackage[unicode=true]{hyperref}
\hypersetup{
            pdfborder={0 0 0},
            breaklinks=true}
\urlstyle{same}  % don't use monospace font for urls
\IfFileExists{parskip.sty}{%
\usepackage{parskip}
}{% else
\setlength{\parindent}{0pt}
\setlength{\parskip}{6pt plus 2pt minus 1pt}
}
\setlength{\emergencystretch}{3em}  % prevent overfull lines
\providecommand{\tightlist}{%
  \setlength{\itemsep}{0pt}\setlength{\parskip}{0pt}}
\setcounter{secnumdepth}{0}
% Redefines (sub)paragraphs to behave more like sections
\ifx\paragraph\undefined\else
\let\oldparagraph\paragraph
\renewcommand{\paragraph}[1]{\oldparagraph{#1}\mbox{}}
\fi
\ifx\subparagraph\undefined\else
\let\oldsubparagraph\subparagraph
\renewcommand{\subparagraph}[1]{\oldsubparagraph{#1}\mbox{}}
\fi

% set default figure placement to htbp
\makeatletter
\def\fps@figure{htbp}
\makeatother


\date{}

\begin{document}

{\ldots{}.42\ldots{}.}

{There were three fruiterers. They had with them 6, 8 and 100 panas.
With these they bought fruit at the same rate. They sold some, all at
one rate. Remaining they sold at the rate of 5 panas for one. Remaining
the sold at the rate of 5 panas for one. After the transaction the three
had equal panas. Find the rate of purchase and the rate of selling.}

{१० अनेकवर्णसमीकरणान्तर्गतं मध्यमाहरणम् । }

{Device for solving equations with more than one unknown.}

{'वर्गाद्यं चेत्तुल्यशुद्धौ कृतायां पक्षस्यैकस्योक्तवद्वर्गमूलम्' । }

{वर्गप्रकृत्या परपक्षमूलं तयोः समीकारविधिः पुनश्च । }

{वर्गप्रकृत्या विषयो न चेत्स्यात्तदाऽन्यवर्णस्य कृतेः समं तम् । }

{कृत्वाऽपरं पक्षमथान्यमानं कृतिप्रकृत्याऽऽद्यमितिस्तथा च । }

{वर्गप्रकृत्या विषयो यथास्यात्तथा सुधीभिर्बहुधा विचिन्त्यम ।। १४९ ।। }

{If the equation containd the square of the unknown, by the device given
before we can find the square root of one side. The square root of the
other side can be found by the method of stanza 70. And then we should
equate the two sides. If the other side does not come under that method,
it can be assumed equal to z}{2}{ and then we can get proper value to
make it a perfect square. In short if the second side does not yield to
वर्गप्रकृति method one should think over as to how that can be done.}

{बीजं मतिर्विविधवर्णसहायिनी हि मन्दावबोधविधये विबुधैर्निजाद्यैः । }

{विस्तारिता गणकतामरसांशुमद्भिर्या सैव बीजगणिताह्वयतामुपेता ।। १५० ।। }

{Our predecessors, intelligent mathematicians have spread the thought
that different letters of alphabet are useful and this they have done
in\\
}

\end{document}
