\documentclass[]{article}
\usepackage{lmodern}
\usepackage{amssymb,amsmath}
\usepackage{ifxetex,ifluatex}
\usepackage{fixltx2e} % provides \textsubscript
\ifnum 0\ifxetex 1\fi\ifluatex 1\fi=0 % if pdftex
  \usepackage[T1]{fontenc}
  \usepackage[utf8]{inputenc}
\else % if luatex or xelatex
  \ifxetex
    \usepackage{mathspec}
  \else
    \usepackage{fontspec}
  \fi
  \defaultfontfeatures{Ligatures=TeX,Scale=MatchLowercase}
\fi
% use upquote if available, for straight quotes in verbatim environments
\IfFileExists{upquote.sty}{\usepackage{upquote}}{}
% use microtype if available
\IfFileExists{microtype.sty}{%
\usepackage[]{microtype}
\UseMicrotypeSet[protrusion]{basicmath} % disable protrusion for tt fonts
}{}
\PassOptionsToPackage{hyphens}{url} % url is loaded by hyperref
\usepackage[unicode=true]{hyperref}
\hypersetup{
            pdfborder={0 0 0},
            breaklinks=true}
\urlstyle{same}  % don't use monospace font for urls
\IfFileExists{parskip.sty}{%
\usepackage{parskip}
}{% else
\setlength{\parindent}{0pt}
\setlength{\parskip}{6pt plus 2pt minus 1pt}
}
\setlength{\emergencystretch}{3em}  % prevent overfull lines
\providecommand{\tightlist}{%
  \setlength{\itemsep}{0pt}\setlength{\parskip}{0pt}}
\setcounter{secnumdepth}{0}
% Redefines (sub)paragraphs to behave more like sections
\ifx\paragraph\undefined\else
\let\oldparagraph\paragraph
\renewcommand{\paragraph}[1]{\oldparagraph{#1}\mbox{}}
\fi
\ifx\subparagraph\undefined\else
\let\oldsubparagraph\subparagraph
\renewcommand{\subparagraph}[1]{\oldsubparagraph{#1}\mbox{}}
\fi

% set default figure placement to htbp
\makeatletter
\def\fps@figure{htbp}
\makeatother


\date{}

\begin{document}

{\ldots{}.45\ldots{}.}

{process of वर्गप्रकृति. After that two sides should be equated.}

{तौ राशी वद यत्कृत्योः सप्ताष्टगुणयोर्युतिः । }

{मूलदा स्याद्वियोगस्तु मूलदो रूपसंयुतः ।। १६० ।। }

{Find two numbers such that when they are separately multiplied by 7 and
8 and then the products are added we get a perfect square; and when 1 is
added to the difference of the products, it is a perfect square.}

{घनवर्गयुतिवर्गो ययो राश्योः प्रजायते । }

{समासोऽपि ययोर्वर्गस्तौ राशी शीघ्रमानय ।। १६१ ।। }

{Find two numbers such that the cube of one added to the square of the
other gives a perfect square; and the sum of the two also gives a square
number.}

{सभाविते वर्णकृती तु यत्र तन्मूलमादाय तु शेषकस्य । }

{इष्टोद्ध्रृतस्येष्टविवर्जितस्य दलेन तुल्यं हि तदेव कार्यम् ।। १६२ ।। }

{If in the other side of an equation there are squares of two unknowns
with their product, we should extract a square root and the remainder
may be divided by a desired number and then decreased by that number.
After the difference is halved it may be equated with the square root.}

{ययोर्वर्गयुतिर्घातयुता मूलप्रदा भवेत् । }

{तन्मूलगुणितो योगः सरूपश्चाऽऽशु तौ वद ।। १६३ ।। }

{Find two numbers such that the sum of their squares added to their
product is a square number; and the square root multiplied by the sum of
the numbers increased by 1 is perfect square.}

{यत्स्यात्साल्पवधार्धतो घनपदं यद्वर्गयोगात्पदं }

{ये योगान्तरयोर्द्विकाभ्यधिकयोर्वर्गान्तरात्साष्टकात् ।}{\\
}

\end{document}
