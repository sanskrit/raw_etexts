\documentclass[]{article}
\usepackage{lmodern}
\usepackage{amssymb,amsmath}
\usepackage{ifxetex,ifluatex}
\usepackage{fixltx2e} % provides \textsubscript
\ifnum 0\ifxetex 1\fi\ifluatex 1\fi=0 % if pdftex
  \usepackage[T1]{fontenc}
  \usepackage[utf8]{inputenc}
\else % if luatex or xelatex
  \ifxetex
    \usepackage{mathspec}
  \else
    \usepackage{fontspec}
  \fi
  \defaultfontfeatures{Ligatures=TeX,Scale=MatchLowercase}
\fi
% use upquote if available, for straight quotes in verbatim environments
\IfFileExists{upquote.sty}{\usepackage{upquote}}{}
% use microtype if available
\IfFileExists{microtype.sty}{%
\usepackage[]{microtype}
\UseMicrotypeSet[protrusion]{basicmath} % disable protrusion for tt fonts
}{}
\PassOptionsToPackage{hyphens}{url} % url is loaded by hyperref
\usepackage[unicode=true]{hyperref}
\hypersetup{
            pdfborder={0 0 0},
            breaklinks=true}
\urlstyle{same}  % don't use monospace font for urls
\IfFileExists{parskip.sty}{%
\usepackage{parskip}
}{% else
\setlength{\parindent}{0pt}
\setlength{\parskip}{6pt plus 2pt minus 1pt}
}
\setlength{\emergencystretch}{3em}  % prevent overfull lines
\providecommand{\tightlist}{%
  \setlength{\itemsep}{0pt}\setlength{\parskip}{0pt}}
\setcounter{secnumdepth}{0}
% Redefines (sub)paragraphs to behave more like sections
\ifx\paragraph\undefined\else
\let\oldparagraph\paragraph
\renewcommand{\paragraph}[1]{\oldparagraph{#1}\mbox{}}
\fi
\ifx\subparagraph\undefined\else
\let\oldsubparagraph\subparagraph
\renewcommand{\subparagraph}[1]{\oldsubparagraph{#1}\mbox{}}
\fi

% set default figure placement to htbp
\makeatletter
\def\fps@figure{htbp}
\makeatother


\date{}

\begin{document}

{\ldots{}.24\ldots{}.}

{ment and divisor. We get गुण and लब्धि by the process of kuttak. The
square of this गुण should be subtracted from the given prakriti or this
prakriti should be subtracted from the square of the guna so that the
difference may be small. This difference divided by the augment b gives
new value of b. If the square has been subtracted from prakriti, sign
must be changed for the new augment. The quotient obtained by kuttak
will be new x. From the new values of x and b we should get the new
value of y. This method of getting new values for x, y and b from the
previous ones is known as चक्रवाल or cyclic.}

{In this way for any augment 4, 2 or 1 we get integral values for the
variables. From augments 4 or 2 we can come to augment 1 with the help
of bhavana process or other methods.}

{का सप्तषष्टिगुणिता कृति रेकयुक्ता का चैक षष्टिनिहता च सखे सरूपा । }

{स्यान् मूलदा यदि कृतिप्रकृतिर्नितान्तं त्वच्चेतसि प्रवद तात }

{ततालतावत् ।। ७६ । }

{Give the rational solutions of (1) 67 x}{2}{ + 1 = y}{2}{ and 61x}{2}{
+ 1 = y}{2}

{रूपशुद्धौ खिलोद्दिष्टं वर्गयोगो गुणो न चेत् ।। ७७ ।।}{ }

{The augment being minus one, if the coefficient is not the sum of two
squares solution is not possible.}

{अखिले कृतिमूलाभ्यां द्विधा रूपं विभाजितम् । }

{द्विधा ह्रस्वपदं ज्येष्ठं ततो रूपविशोधने । }

{पूर्ववद् वा प्रसाध्येते पदे रूपविशोधने ।। ७८ ।। }

{We should take the roots of the two squares which form the coefficient.
Dividing 1 by these roots we shall get two separate values for x. From
these we can find corresponding values for y.\\
}

\end{document}
