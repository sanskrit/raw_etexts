\documentclass[]{article}
\usepackage{lmodern}
\usepackage{amssymb,amsmath}
\usepackage{ifxetex,ifluatex}
\usepackage{fixltx2e} % provides \textsubscript
\ifnum 0\ifxetex 1\fi\ifluatex 1\fi=0 % if pdftex
  \usepackage[T1]{fontenc}
  \usepackage[utf8]{inputenc}
\else % if luatex or xelatex
  \ifxetex
    \usepackage{mathspec}
  \else
    \usepackage{fontspec}
  \fi
  \defaultfontfeatures{Ligatures=TeX,Scale=MatchLowercase}
\fi
% use upquote if available, for straight quotes in verbatim environments
\IfFileExists{upquote.sty}{\usepackage{upquote}}{}
% use microtype if available
\IfFileExists{microtype.sty}{%
\usepackage[]{microtype}
\UseMicrotypeSet[protrusion]{basicmath} % disable protrusion for tt fonts
}{}
\PassOptionsToPackage{hyphens}{url} % url is loaded by hyperref
\usepackage[unicode=true]{hyperref}
\hypersetup{
            pdfborder={0 0 0},
            breaklinks=true}
\urlstyle{same}  % don't use monospace font for urls
\IfFileExists{parskip.sty}{%
\usepackage{parskip}
}{% else
\setlength{\parindent}{0pt}
\setlength{\parskip}{6pt plus 2pt minus 1pt}
}
\setlength{\emergencystretch}{3em}  % prevent overfull lines
\providecommand{\tightlist}{%
  \setlength{\itemsep}{0pt}\setlength{\parskip}{0pt}}
\setcounter{secnumdepth}{0}
% Redefines (sub)paragraphs to behave more like sections
\ifx\paragraph\undefined\else
\let\oldparagraph\paragraph
\renewcommand{\paragraph}[1]{\oldparagraph{#1}\mbox{}}
\fi
\ifx\subparagraph\undefined\else
\let\oldsubparagraph\subparagraph
\renewcommand{\subparagraph}[1]{\oldsubparagraph{#1}\mbox{}}
\fi

% set default figure placement to htbp
\makeatletter
\def\fps@figure{htbp}
\makeatother


\date{}

\begin{document}

{\ldots{}.15\ldots{}.}

{वर्गेण योगकरणी विहृता विशुध्येत् खण्डानि तत्कृतिपदस्य यथेप्सितानि । }

{कृत्वा तदीयकृतयः खलु पूर्वलब्ध्या क्षण्णा भवन्ति पृथगेवमिमाः }

{करण्यः ।। ३९ ।। }

{To analyse a surd divide the number by a square number. The root of the
square can be split into parts as one desires. After squaring the parts
we get separate surds.}

{द्विकत्रिपञ्चप्रमिताः करण्यस्तासां कृतिं द्वित्रिकसंख्ययोश्च । }

{षट्पञ्चकद्वित्रिकसंमितानां पृथक्पृथङ्मे कथयाऽऽशु विद्वन् । । }

{अष्टादशाष्टद्विकसंमितानां कृती कृतीनां च सखे पदानि ।। ४० ।। }

{Give the squares of (1) }{√}{2 + }{√}{3 +}{√}{5 (2) }{√}{2 + }{√}{3
(3)}{√}{6 + }{√}{5 + }{√}{2 + }{√}{3 and (4) }{√}{18 + }{√}{8 +}{√}{2
seaprately and find the square roots of the results.}

{वर्गे करण्या यदि वा करण्योस्तुल्यानि रूपाण्यथवा बहूनाम् }

{विशोधयेद् रूपकृतेः पदेन शेषस्य रूपाणि युतोनितानि । । }

{पृथक् तदर्धे करणीद्वयं स्यान् मूलेऽथ बह्वी करणी तयोर्या । }

{रूपाणि तान्येवमतोऽपि भूयः शेषाः करण्यो यदि सन्ति वर्गे ।। ४१ ।। }

{In a square there may be one or more than one surd. Squaring the
integral term we should subtract numbers equivalent to one or more
surds. The remainder should give a square root. This square root should
be added to and subtracted from the integral term. Putting them
separately and dividing by 2 we get two surds. If any surds are left in
the square, fixing one of the two surds, the process should be
repeated.}

{ऋणात्मिका चेत् करणी कृतौ स्याद् धनात्मिका तां परिकल्प्य साध्ये । }

{मूले करण्यावनयोरभीष्टा क्षयात्मिकैका सुधियावगम्या ।। ४२ ।। }

{If in the square there be a negative surd, regarding it to be positive
the square root should be found. Out of the surds in the answer one may
be taken as negative.\\
}

\end{document}
