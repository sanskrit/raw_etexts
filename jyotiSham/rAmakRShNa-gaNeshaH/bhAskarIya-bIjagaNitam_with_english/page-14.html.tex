\documentclass[]{article}
\usepackage{lmodern}
\usepackage{amssymb,amsmath}
\usepackage{ifxetex,ifluatex}
\usepackage{fixltx2e} % provides \textsubscript
\ifnum 0\ifxetex 1\fi\ifluatex 1\fi=0 % if pdftex
  \usepackage[T1]{fontenc}
  \usepackage[utf8]{inputenc}
\else % if luatex or xelatex
  \ifxetex
    \usepackage{mathspec}
  \else
    \usepackage{fontspec}
  \fi
  \defaultfontfeatures{Ligatures=TeX,Scale=MatchLowercase}
\fi
% use upquote if available, for straight quotes in verbatim environments
\IfFileExists{upquote.sty}{\usepackage{upquote}}{}
% use microtype if available
\IfFileExists{microtype.sty}{%
\usepackage[]{microtype}
\UseMicrotypeSet[protrusion]{basicmath} % disable protrusion for tt fonts
}{}
\PassOptionsToPackage{hyphens}{url} % url is loaded by hyperref
\usepackage[unicode=true]{hyperref}
\hypersetup{
            pdfborder={0 0 0},
            breaklinks=true}
\urlstyle{same}  % don't use monospace font for urls
\IfFileExists{parskip.sty}{%
\usepackage{parskip}
}{% else
\setlength{\parindent}{0pt}
\setlength{\parskip}{6pt plus 2pt minus 1pt}
}
\setlength{\emergencystretch}{3em}  % prevent overfull lines
\providecommand{\tightlist}{%
  \setlength{\itemsep}{0pt}\setlength{\parskip}{0pt}}
\setcounter{secnumdepth}{0}
% Redefines (sub)paragraphs to behave more like sections
\ifx\paragraph\undefined\else
\let\oldparagraph\paragraph
\renewcommand{\paragraph}[1]{\oldparagraph{#1}\mbox{}}
\fi
\ifx\subparagraph\undefined\else
\let\oldsubparagraph\subparagraph
\renewcommand{\subparagraph}[1]{\oldsubparagraph{#1}\mbox{}}
\fi

% set default figure placement to htbp
\makeatletter
\def\fps@figure{htbp}
\makeatother


\date{}

\begin{document}

{\ldots{}.12\ldots{}.}

{plicated separately. All these partial products are then added. This
method is to be applied to algebraic number, its square and surd. Here
the method is quite like the partial product method in arithmetic.}

{यावत्तावत्पञ्चकं व्येकरूपं यावत्तावद्भिस्त्रिभिः सद्विरूपैः । }

{संगुण्य द्राग् ब्रूहि गुण्यं गुणं वा व्यस्तं स्वर्णं कल्पयित्वा च
विद्वन् ।। २८ ।। }

{Multiply 5x- 1 by 3x + 2 and give the product. Changing the signs of
one of these expressions form the product.}

{भाज्याच्छेदः शुध्यति प्रच्युतः सन् स्वेषु स्वेषु स्थानकेषु क्रमेण । }

{यैयैर्वर्णैः संगुणो यैश्च रूपैर् भागाहोर लब्धयस्ताःस्युरत्र ।। २९ ।। }

{From the dividend places are removed one by one after subtracting the
products of the divisor with appropriate terms. When nothing is left the
sum of those terms is the full quotient.}

{रूपैः षडभिर्वर्जितानां चतुर्णामव्यक्तानां ब्रूहि वर्गं सखे मे ।। ३० ।।
}

{Please give me the square of 4x- 6.}

{कृतिभ्य आदाय पदानि तेषां द्वयोर्द्वयोश्चाभिहतिं द्विनिघ्नीम् । }

{शेषात् त्यजेद् रूपपदं गृहीत्वा चेत्सन्ति रूपाणि तथैव शेषम् ।। ३१ ।। }

{To find the square root of an expression, we take some square terms
from that and find their roots. Taking these roots in pairs we form
their product and remove twice the product and square terms from the
expression. This is to be done till all terms are removed. If there is
an absolute term we expect its square root in the final result.}

{यावत्तावत्कालकनीलवर्णास्त्रिपञ्चसप्तधनम् । }

{द्वित्रैकमितैः क्षयगैः सहिता रहिताः कति स्युस्तैः ।। ३२ ।। }

{To 3x + 5x +7z if -2x-3y-1z is added what is the sum? From 3x + 5y +7z
if -2x-3y-1z is removed what is the remainder?\\
}

\end{document}
