\documentclass[]{article}
\usepackage{lmodern}
\usepackage{amssymb,amsmath}
\usepackage{ifxetex,ifluatex}
\usepackage{fixltx2e} % provides \textsubscript
\ifnum 0\ifxetex 1\fi\ifluatex 1\fi=0 % if pdftex
  \usepackage[T1]{fontenc}
  \usepackage[utf8]{inputenc}
\else % if luatex or xelatex
  \ifxetex
    \usepackage{mathspec}
  \else
    \usepackage{fontspec}
  \fi
  \defaultfontfeatures{Ligatures=TeX,Scale=MatchLowercase}
\fi
% use upquote if available, for straight quotes in verbatim environments
\IfFileExists{upquote.sty}{\usepackage{upquote}}{}
% use microtype if available
\IfFileExists{microtype.sty}{%
\usepackage[]{microtype}
\UseMicrotypeSet[protrusion]{basicmath} % disable protrusion for tt fonts
}{}
\PassOptionsToPackage{hyphens}{url} % url is loaded by hyperref
\usepackage[unicode=true]{hyperref}
\hypersetup{
            pdfborder={0 0 0},
            breaklinks=true}
\urlstyle{same}  % don't use monospace font for urls
\IfFileExists{parskip.sty}{%
\usepackage{parskip}
}{% else
\setlength{\parindent}{0pt}
\setlength{\parskip}{6pt plus 2pt minus 1pt}
}
\setlength{\emergencystretch}{3em}  % prevent overfull lines
\providecommand{\tightlist}{%
  \setlength{\itemsep}{0pt}\setlength{\parskip}{0pt}}
\setcounter{secnumdepth}{0}
% Redefines (sub)paragraphs to behave more like sections
\ifx\paragraph\undefined\else
\let\oldparagraph\paragraph
\renewcommand{\paragraph}[1]{\oldparagraph{#1}\mbox{}}
\fi
\ifx\subparagraph\undefined\else
\let\oldsubparagraph\subparagraph
\renewcommand{\subparagraph}[1]{\oldsubparagraph{#1}\mbox{}}
\fi

% set default figure placement to htbp
\makeatletter
\def\fps@figure{htbp}
\makeatother


\date{}

\begin{document}

{Appendix}

{Some words denoting numbers}

{ Page}

{1 एक, इन्दु, क्षिति, मही, रूप 13}

{2 द्वि, नेत्र 12}

{3 त्रि, पावक, हुताशन 17}

{4 चतुर्, श्रुति 23}

{5 पञ्च 12}

{6 षट्, ऋतु 12}

{7 सप्त 14}

{8 अष्ट, गज, नाग 17}

{9 नव 9}

{10 दश 17}

{11 एकादश, रुद्र 17}

{12 द्वादश, अर्क, रवि, सूर्य 14}

{13 त्रयोदश, विश्व 17 }

{14 चतुर्दश 22}

{15 पञ्चदश, तिथि 17}

{16 षोडश, नृप 49}

{17 सप्तदश 17 }

{18 अष्टादश 15}

{19 एकोनविंशति}

{20 विंशति, नख 37, 50}

{21 एकविंशति 26}

{23 त्रयोविंशति 41}

{24 सिद्ध 17}

{26 षड्विंशति 41}

{27 भ 14}

{30 त्रिंशत् 41}

{32 द्वात्रिंशत्, दन्त, द्वित्रि 17}

{33 त्रयस्त्रिंशत् 26}

{35 पञ्चत्रिंशत् 35 }

{40 चत्वारिंशत् 19}

{47 =50 -3 त्र्यूनं शतार्धं 28}

{50 पञ्चाशत् 26}

{52 द्विपञ्चाशत् 26}

{53 = 50 + 3 पञ्चाशत् त्रियुत् 50}

{56 षट् पञ्चाशत् 37}

{58 = 60 - 2}{ }{द्यून}{ }{षष्टि 51}

{60 षष्टि 21}

{61 एकषष्टि 24}

{62 द्विषष्टि 51}

{63 त्रिषष्टि 22}

{65 पञ्चषष्टि 20}

{67 सप्तषष्टि 24}

{70 सप्तति 25}

{75 पञ्चसप्तति 25}

{80 अशीति 17}

{90 नवति 20}

{100 शत 20}

{195 = 200 - 5 पञ्चवर्जित् शतद्वय 20}

{200 द्विशति 35}

{221 एक विंशतियुतं शतद्वयं 20}

{300 त्रिशती 25}

{600 षट्शती 37}

{10000 अयुत 35}

{Twice द्विगुण द्विघ्न, Thrice त्रिगुण 13}

{Twenty Times नख संगुण Half दल 34}

{0 शून्य, ख, वियत् 9 }

\end{document}
