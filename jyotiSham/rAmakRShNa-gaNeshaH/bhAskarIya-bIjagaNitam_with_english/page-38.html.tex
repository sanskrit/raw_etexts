\documentclass[]{article}
\usepackage{lmodern}
\usepackage{amssymb,amsmath}
\usepackage{ifxetex,ifluatex}
\usepackage{fixltx2e} % provides \textsubscript
\ifnum 0\ifxetex 1\fi\ifluatex 1\fi=0 % if pdftex
  \usepackage[T1]{fontenc}
  \usepackage[utf8]{inputenc}
\else % if luatex or xelatex
  \ifxetex
    \usepackage{mathspec}
  \else
    \usepackage{fontspec}
  \fi
  \defaultfontfeatures{Ligatures=TeX,Scale=MatchLowercase}
\fi
% use upquote if available, for straight quotes in verbatim environments
\IfFileExists{upquote.sty}{\usepackage{upquote}}{}
% use microtype if available
\IfFileExists{microtype.sty}{%
\usepackage[]{microtype}
\UseMicrotypeSet[protrusion]{basicmath} % disable protrusion for tt fonts
}{}
\PassOptionsToPackage{hyphens}{url} % url is loaded by hyperref
\usepackage[unicode=true]{hyperref}
\hypersetup{
            pdfborder={0 0 0},
            breaklinks=true}
\urlstyle{same}  % don't use monospace font for urls
\IfFileExists{parskip.sty}{%
\usepackage{parskip}
}{% else
\setlength{\parindent}{0pt}
\setlength{\parskip}{6pt plus 2pt minus 1pt}
}
\setlength{\emergencystretch}{3em}  % prevent overfull lines
\providecommand{\tightlist}{%
  \setlength{\itemsep}{0pt}\setlength{\parskip}{0pt}}
\setcounter{secnumdepth}{0}
% Redefines (sub)paragraphs to behave more like sections
\ifx\paragraph\undefined\else
\let\oldparagraph\paragraph
\renewcommand{\paragraph}[1]{\oldparagraph{#1}\mbox{}}
\fi
\ifx\subparagraph\undefined\else
\let\oldsubparagraph\subparagraph
\renewcommand{\subparagraph}[1]{\oldsubparagraph{#1}\mbox{}}
\fi

% set default figure placement to htbp
\makeatletter
\def\fps@figure{htbp}
\makeatother


\date{}

\begin{document}

{\ldots{}.36\ldots{}.}

{number, (number/5 - 3)2 went to a cave and remaining 1 climbed a tree.
What was the number in the troupe?}

{What is the length of the shadow of gnomon of 12 fingers height, if
after subtracting from that length 1/3 of the shadow hypotenuse, 14
fingers are left?}

{चत्वारो राशयः के ते मूलदा ये द्विसंयुताः । }

{द्वयोर्द्वयोर्यथासन्नघाताश्चाष्टादशान्विताः । । }

{मूलदाः सर्वमूलैक्यादेकादशयुतात्पदम् । }

{त्रयोदश सखे जातं बीजज्ञ वद तान्मम ।।१२६।।}

{Four numbers are such that if 2 is added to each they become sqaures.
If we form their products by taking two numbers from consecutive pairs
and increase them by 18, the three become squares. If we add all the 7
square roots and add 11 we get 13 as the square root of the sum. Please
give me those four numbers.}

{राशिक्षेपाद्वधक्षेपो यद्गुणस्तत्पदोत्तरम् । }

{अव्यक्तराशयः कल्प्या वर्गिताः क्षेपवर्जिताः ।। १२७ ।। }

{{[}In the above example 18 is called augment for product and 2 is
augment for number {]}.}

{Divide the augment for product by the augment for number and get the
square root of the quotient. This is the common difference of some four
numbers. Taking these as y, y + 3, y + 6 and y + 9, from their squares
we subtract 2 and get the four unknowns.}

{क्षेत्रे तिथिनखैस्तुल्ये दोःकोटी तत्र का श्रुतिः । }

{उपपत्तिश्च रूढस्य गणितस्यास्य कथ्यताम् ।। १२८ ।। }

{To find the hypotenuse of a right angled triangle when the two sides
are 15 and 20 is a}

\end{document}
