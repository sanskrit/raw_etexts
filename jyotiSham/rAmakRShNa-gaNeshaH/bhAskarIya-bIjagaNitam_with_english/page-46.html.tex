\documentclass[]{article}
\usepackage{lmodern}
\usepackage{amssymb,amsmath}
\usepackage{ifxetex,ifluatex}
\usepackage{fixltx2e} % provides \textsubscript
\ifnum 0\ifxetex 1\fi\ifluatex 1\fi=0 % if pdftex
  \usepackage[T1]{fontenc}
  \usepackage[utf8]{inputenc}
\else % if luatex or xelatex
  \ifxetex
    \usepackage{mathspec}
  \else
    \usepackage{fontspec}
  \fi
  \defaultfontfeatures{Ligatures=TeX,Scale=MatchLowercase}
\fi
% use upquote if available, for straight quotes in verbatim environments
\IfFileExists{upquote.sty}{\usepackage{upquote}}{}
% use microtype if available
\IfFileExists{microtype.sty}{%
\usepackage[]{microtype}
\UseMicrotypeSet[protrusion]{basicmath} % disable protrusion for tt fonts
}{}
\PassOptionsToPackage{hyphens}{url} % url is loaded by hyperref
\usepackage[unicode=true]{hyperref}
\hypersetup{
            pdfborder={0 0 0},
            breaklinks=true}
\urlstyle{same}  % don't use monospace font for urls
\IfFileExists{parskip.sty}{%
\usepackage{parskip}
}{% else
\setlength{\parindent}{0pt}
\setlength{\parskip}{6pt plus 2pt minus 1pt}
}
\setlength{\emergencystretch}{3em}  % prevent overfull lines
\providecommand{\tightlist}{%
  \setlength{\itemsep}{0pt}\setlength{\parskip}{0pt}}
\setcounter{secnumdepth}{0}
% Redefines (sub)paragraphs to behave more like sections
\ifx\paragraph\undefined\else
\let\oldparagraph\paragraph
\renewcommand{\paragraph}[1]{\oldparagraph{#1}\mbox{}}
\fi
\ifx\subparagraph\undefined\else
\let\oldsubparagraph\subparagraph
\renewcommand{\subparagraph}[1]{\oldsubparagraph{#1}\mbox{}}
\fi

% set default figure placement to htbp
\makeatletter
\def\fps@figure{htbp}
\makeatother


\date{}

\begin{document}

{\ldots{}.44\ldots{}.}

{ज्येष्ठ with the square of कनिष्ठ as new ज्येष्ठ. The process should be
completed as before. }

{यस्य वर्गकृतिः पञ्चगुणा वर्गशतोनिता । }

{मूलदा जायते राशिं गणितज्ञ वदाऽऽशु तम् ।। १५५ ।। }

{Find the value of x from 5x}{4}{ - 100x}{2}{ = y}{2}

{कयोः स्यादन्तरे वर्गो वर्गयोगो ययोर्घनः । }

{तौ राशी कथयाभिन्नो बहुधा बीजवित्तम ।। १५६ ।। }

{x - y is a perfect sqaure and x}{3}{ + y}{3}{ is a cube. Please give
several intergral values of (x, y)}

{साव्यक्तरूपो यदि वर्णवर्गस्तदाऽन्यवर्णस्य कृतेः समं तम् । }

{कृत्वा पदं तस्य तदन्यपक्षे वर्गप्रकृत्योक्तवदेव मूले । । }

{कनिष्ठमाद्येन पदेन तुल्यं ज्येष्ठं द्वितीयेन समं विदध्यात् ।। १५७ । । }

{If it is possible to find the square root of one side and the second
side contains unknown, that side may be assumed to be sqaure of some
unknown. The sqaure root of the first side should be found and the
square root of the other side should be found by the process of
वर्गप्रकृति. That will be ज्येष्ठ मूल, the कनिष्ठ मूल may be equated
with the sqaure root first found. From this we can get the value of x,
after getting the value of second unknown.}

{त्रिकादिद्वयुत्तरश्रेढ्यां गच्छे बवापि च यत्फलम् । }

{तदेव त्रिगुणं कस्मिन्नन्यगच्छे भवेद्वद ।। १५८ ।। }

{The first term of an A. P. is 3 and the common difference is 2. Three
times the sum of x terms is equal to the sum of y terms. Find (x, y).}

{सरूपके वर्णकृती तु यत्र तत्रेच्छयैकां प्रकृतिं प्रकल्प्य । }

{शेषं ततः क्षेपकमुक्तवच्च मूले विदध्यादसकृत्समत्वे ।। १५९ ।। }

{Where on the other side we have squares of two unknowns and
arithmetical number, we should regard the co-efficient of one unknown as
प्रकृति and the rest as augment and get the roots by the}

\end{document}
