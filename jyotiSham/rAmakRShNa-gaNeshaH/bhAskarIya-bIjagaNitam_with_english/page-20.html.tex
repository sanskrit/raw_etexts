\documentclass[]{article}
\usepackage{lmodern}
\usepackage{amssymb,amsmath}
\usepackage{ifxetex,ifluatex}
\usepackage{fixltx2e} % provides \textsubscript
\ifnum 0\ifxetex 1\fi\ifluatex 1\fi=0 % if pdftex
  \usepackage[T1]{fontenc}
  \usepackage[utf8]{inputenc}
\else % if luatex or xelatex
  \ifxetex
    \usepackage{mathspec}
  \else
    \usepackage{fontspec}
  \fi
  \defaultfontfeatures{Ligatures=TeX,Scale=MatchLowercase}
\fi
% use upquote if available, for straight quotes in verbatim environments
\IfFileExists{upquote.sty}{\usepackage{upquote}}{}
% use microtype if available
\IfFileExists{microtype.sty}{%
\usepackage[]{microtype}
\UseMicrotypeSet[protrusion]{basicmath} % disable protrusion for tt fonts
}{}
\PassOptionsToPackage{hyphens}{url} % url is loaded by hyperref
\usepackage[unicode=true]{hyperref}
\hypersetup{
            pdfborder={0 0 0},
            breaklinks=true}
\urlstyle{same}  % don't use monospace font for urls
\IfFileExists{parskip.sty}{%
\usepackage{parskip}
}{% else
\setlength{\parindent}{0pt}
\setlength{\parskip}{6pt plus 2pt minus 1pt}
}
\setlength{\emergencystretch}{3em}  % prevent overfull lines
\providecommand{\tightlist}{%
  \setlength{\itemsep}{0pt}\setlength{\parskip}{0pt}}
\setcounter{secnumdepth}{0}
% Redefines (sub)paragraphs to behave more like sections
\ifx\paragraph\undefined\else
\let\oldparagraph\paragraph
\renewcommand{\paragraph}[1]{\oldparagraph{#1}\mbox{}}
\fi
\ifx\subparagraph\undefined\else
\let\oldsubparagraph\subparagraph
\renewcommand{\subparagraph}[1]{\oldsubparagraph{#1}\mbox{}}
\fi

% set default figure placement to htbp
\makeatletter
\def\fps@figure{htbp}
\makeatother


\date{}

\begin{document}

{\ldots{}.18\ldots{}.}

{simplify the equation. If the H. C. F. and a and b does not divide `c'
then the example is improper.}

{परस्परं भाजितयोर्ययोर्यः शेषस्तयोः स्याद् अपवर्तनं सः । }

{तेनापवर्तेन विभाजितौ यौ तौ भाज्यहारौ दृढसंज्ञकौ स्तः ।। }

{मिथौ भजेत्तौ दृढभाज्यहारौ यावद्विभाज्ये भवतीह रूपम् । }

{फलान्यधोऽधस्तदधो निवेश्यः क्षेपस्तथाऽन्ते खमुपान्तिमेन । }

{स्वोर्ध्वेहतेऽन्त्येन युते तदन्त्यं त्यजन्मुहुः स्याद् इति राशियुग्मम्
। }

{ऊर्ध्वो विभाज्येन दृढेन तष्टः फलं गुणः स्याद् अपरो हरेण ।। ५१ ।। }

{By the continued division method of finding the H. C. F. we find the
common divisor of a and b if any. Having removed the common factors of a
and b if any our a and b are now pucca for the process. Now we carry the
continued division method with दृढभाज्य and दृढहार i. e. a, b till we
arrive at remainder 1. The quotients are placed one below the other in
succession, in a vertical column and below them क्षेप i. e. c and zero
at the end. Rule for the process is: the penultimate number is to
multiply the number (quotient) over it and to this product the ultimate
number is added and the sum is put above i. e. in the row of the
multiplicand. The last number is discarded. Continuing this process we
arrive at two numbers at the top rows. Dividing the upper number by a we
get the remainder as the value of y (लब्धि). And dividing the other
number by b we get the remainder as the value of x (गुण).}

{एवं तदेवात्र यदा समास्ताः स्युर्लब्धयश्चेद् विषमास्तदानीम् । }

{यथागतौ लब्धिगुणौ विशोध्यौ स्वतक्षणाच्छेषमितो तु तौ स्तः ।। ५२ ।। }

{When the number of quotients is even the process gives गुण and लब्धि
correctly. But when the number is odd, values obtained must be
subtracted from b and a respectively to get the correct values.}

\end{document}
