\documentclass[]{article}
\usepackage{lmodern}
\usepackage{amssymb,amsmath}
\usepackage{ifxetex,ifluatex}
\usepackage{fixltx2e} % provides \textsubscript
\ifnum 0\ifxetex 1\fi\ifluatex 1\fi=0 % if pdftex
  \usepackage[T1]{fontenc}
  \usepackage[utf8]{inputenc}
\else % if luatex or xelatex
  \ifxetex
    \usepackage{mathspec}
  \else
    \usepackage{fontspec}
  \fi
  \defaultfontfeatures{Ligatures=TeX,Scale=MatchLowercase}
\fi
% use upquote if available, for straight quotes in verbatim environments
\IfFileExists{upquote.sty}{\usepackage{upquote}}{}
% use microtype if available
\IfFileExists{microtype.sty}{%
\usepackage[]{microtype}
\UseMicrotypeSet[protrusion]{basicmath} % disable protrusion for tt fonts
}{}
\PassOptionsToPackage{hyphens}{url} % url is loaded by hyperref
\usepackage[unicode=true]{hyperref}
\hypersetup{
            pdfborder={0 0 0},
            breaklinks=true}
\urlstyle{same}  % don't use monospace font for urls
\IfFileExists{parskip.sty}{%
\usepackage{parskip}
}{% else
\setlength{\parindent}{0pt}
\setlength{\parskip}{6pt plus 2pt minus 1pt}
}
\setlength{\emergencystretch}{3em}  % prevent overfull lines
\providecommand{\tightlist}{%
  \setlength{\itemsep}{0pt}\setlength{\parskip}{0pt}}
\setcounter{secnumdepth}{0}
% Redefines (sub)paragraphs to behave more like sections
\ifx\paragraph\undefined\else
\let\oldparagraph\paragraph
\renewcommand{\paragraph}[1]{\oldparagraph{#1}\mbox{}}
\fi
\ifx\subparagraph\undefined\else
\let\oldsubparagraph\subparagraph
\renewcommand{\subparagraph}[1]{\oldsubparagraph{#1}\mbox{}}
\fi

% set default figure placement to htbp
\makeatletter
\def\fps@figure{htbp}
\makeatother


\date{}

\begin{document}

{\ldots{}.22\ldots{}. }

{sum as dividend; adding the two values of c we take the sum with a
negative sign as the remainder. What we get is called mixed pulveriser.}

{कः पञ्चनिध्नो विहृतस्त्रिषष्ट्या सष्तावशेषोऽथ स एव राशिः । }

{दशाहतः स्याद् विहृतस्त्रिषष्ट्या चतुर्दशाग्रो वद राशिमेनम् ।।६९।। }

{What is that number which multiplied by 5 and divided by 63 gives 7 as
remainder and when multiplied by 10 and divided by 63 gives 14 as
remainder?}

{६ वर्गप्रकृतिः }

{Equation of the form ax}{3}{ + b = y}{3}

{इष्टं ह्रस्वं तस्य वर्गः प्रकृत्या क्षण्णौ युक्तो वर्जिता वा स येन । }

{मूलं दद्यात् क्षेपकं तं धनर्णं मूलं तच्च ज्येष्ठमूलं वदन्ति ।। ७० ।।}{
}

{What is desired is x, the first variable. By multiplying the square of
the desired by प्रकृति and adding or subtracting something we get a
square number whose root is the second variable. The augment b may be
positive or negative. }

{ह्रस्वज्येष्ठक्षेपकान् न्यस्य तेषां तानन्यान्वाऽधो निवेश्य क्रमेण । }

{साध्यान्येभ्यो भावनाभिर्बहूनि मूलान्येषां भावना प्रोच्यतेऽतः ।। }

{वज्राभ्यासौ ज्येष्ठलघ्वोस्तदैक्यं ह्रस्वं लघ्वोराहतिश्च प्रकृत्या । }

{क्षुण्णा ज्येष्ठाभ्यासयुग्ज्येष्ठमूलं तत्राभ्यासः क्षेपयोः क्षेपकः
स्यात् ।। }

{ह्रस्वं वज्राभ्यासयोरन्तरं वा लघ्वोर्घातो यः प्रकृत्या विनिघ्नः । }

{घातो यश्च ज्येष्ठयोस्तद्वियोगो ज्येष्ठं क्षेपोऽत्रापि च क्षेपघातः ।।७ १
।। }

{Put down x, y, b in this order. Below them write the same or other
ह्रस्व, ज्येष्ठ and क्षेपक satisfying a similar equation with the same
प्रकृति, a, From these by a process called भावना we can have many values
for x, y. This is why the process is called bhavana (generator). By
cross-multiplying and\\
}

\end{document}
