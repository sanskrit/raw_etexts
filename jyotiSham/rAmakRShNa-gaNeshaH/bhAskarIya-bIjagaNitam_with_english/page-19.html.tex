\documentclass[]{article}
\usepackage{lmodern}
\usepackage{amssymb,amsmath}
\usepackage{ifxetex,ifluatex}
\usepackage{fixltx2e} % provides \textsubscript
\ifnum 0\ifxetex 1\fi\ifluatex 1\fi=0 % if pdftex
  \usepackage[T1]{fontenc}
  \usepackage[utf8]{inputenc}
\else % if luatex or xelatex
  \ifxetex
    \usepackage{mathspec}
  \else
    \usepackage{fontspec}
  \fi
  \defaultfontfeatures{Ligatures=TeX,Scale=MatchLowercase}
\fi
% use upquote if available, for straight quotes in verbatim environments
\IfFileExists{upquote.sty}{\usepackage{upquote}}{}
% use microtype if available
\IfFileExists{microtype.sty}{%
\usepackage[]{microtype}
\UseMicrotypeSet[protrusion]{basicmath} % disable protrusion for tt fonts
}{}
\PassOptionsToPackage{hyphens}{url} % url is loaded by hyperref
\usepackage[unicode=true]{hyperref}
\hypersetup{
            pdfborder={0 0 0},
            breaklinks=true}
\urlstyle{same}  % don't use monospace font for urls
\IfFileExists{parskip.sty}{%
\usepackage{parskip}
}{% else
\setlength{\parindent}{0pt}
\setlength{\parskip}{6pt plus 2pt minus 1pt}
}
\setlength{\emergencystretch}{3em}  % prevent overfull lines
\providecommand{\tightlist}{%
  \setlength{\itemsep}{0pt}\setlength{\parskip}{0pt}}
\setcounter{secnumdepth}{0}
% Redefines (sub)paragraphs to behave more like sections
\ifx\paragraph\undefined\else
\let\oldparagraph\paragraph
\renewcommand{\paragraph}[1]{\oldparagraph{#1}\mbox{}}
\fi
\ifx\subparagraph\undefined\else
\let\oldsubparagraph\subparagraph
\renewcommand{\subparagraph}[1]{\oldsubparagraph{#1}\mbox{}}
\fi

% set default figure placement to htbp
\makeatletter
\def\fps@figure{htbp}
\makeatother


\date{}

\begin{document}

{\ldots{}.17\ldots{}.}

{be equivalent to the surds in the final result. Otherwise the given
example is wrong.}

{वर्गे यत्र करण्यो दन्तैः सिद्धैर्गजैर्मिता विद्वन् । }

{रूपैर्दशभिरूपेताः किं मूलं ब्रूहि तस्य स्यात् ।। ४५ ।। }

{10 + }{√}{32 + }{√}{24 + }{√}{8 is the square of an expression
involving surds. Find its square root.}

{वर्गे यत्र करण्यास्तिथिविश्वहुताशनैश्चतुर्गणितैः । }

{तुल्या दशरूपाढ्याः किं मूलं ब्रूहि तस्य स्यात् ।। ४६।। }

{10 + }{√}{60 + }{√}{52 + }{√}{12 is the sqaure of an expression. Find
its square root.}

{अष्टौ षट पञ्चाशत् षष्टिः करणीत्रयं कृतौ यत्र । }

{रूपैर्दशभिरूपेतं किं मूलं ब्रूहि तस्य स्यात् ।। ४७ ।। }

{Give the square root of 10 + }{√}{8 + }{√}{56 + }{√}{60.}

{चतुर्गुणाः सूर्यतिथीषु रूद्रनागर्तवो यत्र कृतौ करण्यः । }

{सविश्वरूपा वद तत्पदं ते यद्यस्ति बीजे पटुताभिमानः ।। ४८ ।। }

{Please give the square root of }

{13 + }{√}{48 + }{√}{60 + }{√}{20 + }{√}{44 + }{√}{32 + }{√}{24.}

{चत्वारिंशदशीतिद्विशतीतुल्याः करण्यश्चेत् । }

{सप्तदशरूपयुक्तास्तत्र कृतौ किं पदं ब्रूहि ।। ४९ । । }

{Please give the square root of }

{17 + }{√}{40 + }{√}{80 + }{√}{200}

{५ कुट्टकविवरणम । }

{Equation of the from ax + c = by.}

{भाज्यो हारः क्षेपकश्चापवर्त्यः केनाप्यादौ संभवे कुट्टकार्थम् । }

{येन च्छिन्नौ भाज्यहारौ न तेन क्षेपश्चैतद् दुष्टमुद्दिष्टमेव ।। ५० ।। }

{Here `a' is dividend, `b' is divisor, `c' is remainder. To solve a
कुट्टक (pulveriser) first of all we should ask if a, b, c have a common
divisor. In that case let us remove the common divisor and\\
}

\end{document}
