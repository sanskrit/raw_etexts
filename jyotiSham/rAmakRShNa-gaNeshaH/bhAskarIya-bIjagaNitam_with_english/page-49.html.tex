\documentclass[]{article}
\usepackage{lmodern}
\usepackage{amssymb,amsmath}
\usepackage{ifxetex,ifluatex}
\usepackage{fixltx2e} % provides \textsubscript
\ifnum 0\ifxetex 1\fi\ifluatex 1\fi=0 % if pdftex
  \usepackage[T1]{fontenc}
  \usepackage[utf8]{inputenc}
\else % if luatex or xelatex
  \ifxetex
    \usepackage{mathspec}
  \else
    \usepackage{fontspec}
  \fi
  \defaultfontfeatures{Ligatures=TeX,Scale=MatchLowercase}
\fi
% use upquote if available, for straight quotes in verbatim environments
\IfFileExists{upquote.sty}{\usepackage{upquote}}{}
% use microtype if available
\IfFileExists{microtype.sty}{%
\usepackage[]{microtype}
\UseMicrotypeSet[protrusion]{basicmath} % disable protrusion for tt fonts
}{}
\PassOptionsToPackage{hyphens}{url} % url is loaded by hyperref
\usepackage[unicode=true]{hyperref}
\hypersetup{
            pdfborder={0 0 0},
            breaklinks=true}
\urlstyle{same}  % don't use monospace font for urls
\IfFileExists{parskip.sty}{%
\usepackage{parskip}
}{% else
\setlength{\parindent}{0pt}
\setlength{\parskip}{6pt plus 2pt minus 1pt}
}
\setlength{\emergencystretch}{3em}  % prevent overfull lines
\providecommand{\tightlist}{%
  \setlength{\itemsep}{0pt}\setlength{\parskip}{0pt}}
\setcounter{secnumdepth}{0}
% Redefines (sub)paragraphs to behave more like sections
\ifx\paragraph\undefined\else
\let\oldparagraph\paragraph
\renewcommand{\paragraph}[1]{\oldparagraph{#1}\mbox{}}
\fi
\ifx\subparagraph\undefined\else
\let\oldsubparagraph\subparagraph
\renewcommand{\subparagraph}[1]{\oldsubparagraph{#1}\mbox{}}
\fi

% set default figure placement to htbp
\makeatletter
\def\fps@figure{htbp}
\makeatother


\date{}

\begin{document}

{\ldots{}.47\ldots{}.}

{subtracted and we get योग and वियोग respectively. After that by संक्रमण
we can get the unknowns.}

{राश्योर् योगवियोगकौ त्रिसहितौ वर्गौ भवेता तयोर् }

{वर्गैक्यं चतुरूनितं रवियुतं वर्गान्तरं स्यात् कृतिः । }

{साल्पं घातदलं घनः पदयुतिस्तेषां द्वियुक्ता कृतिस् }

{तौ राशी वद कोमलामलमते षट् सप्त हित्वा परौ ।। १६७ ।। }

{There are two numbers such that their sum or difference increased by 3
are perfect squares. The sum of their sqaures decreased by 4 is a
square. The difference of their squares increased by 12 is a square. If
half their product is increased by the smaller number we get a cube.
Again the sum of the 5 roots increased by 2 is a square. Tell me the two
numbers other than 6 and 7.}

{राश्योर्ययोः कृतिवियुती चैकेन संयुतौ वर्गौ । }

{रहिते वा तौ राशी गणयित्वा कथय यदि वेत्सि ।। १६८ ।।}{ }

{Please give me two numbers such that when (1) We add 1 to the sum of
their squares and to the difference of the squares the results in each
case is a square number (2) We subtract 1 from the sum of their squares
and from the difference of their squares the result is a square number.}

{यत्राव्यक्तं सरूपं हि तत्र तन्मानमानयेत् । }

{सरूपस्यान्यवर्णस्य कृत्वा कृत्यादिना समम् ।। }

{राशिं तेन समुत्थाप्य कुर्यात्भयोऽपरां क्रियाम् । }

{सरूपेणान्यवर्णेन कृत्वा पूर्वपदं समम् ।। १६९ ।। }

{If on one side there is unknown added to a number and the other side is
the square of an expression, we can assume that as the square of a third
unknown. Squaring and doing any process we should find the value of the
unknown. Substituting that value we should do another process. Giving a
numerical value to the third unknown,}

\end{document}
