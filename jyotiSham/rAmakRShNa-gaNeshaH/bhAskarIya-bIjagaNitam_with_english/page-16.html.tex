\documentclass[]{article}
\usepackage{lmodern}
\usepackage{amssymb,amsmath}
\usepackage{ifxetex,ifluatex}
\usepackage{fixltx2e} % provides \textsubscript
\ifnum 0\ifxetex 1\fi\ifluatex 1\fi=0 % if pdftex
  \usepackage[T1]{fontenc}
  \usepackage[utf8]{inputenc}
\else % if luatex or xelatex
  \ifxetex
    \usepackage{mathspec}
  \else
    \usepackage{fontspec}
  \fi
  \defaultfontfeatures{Ligatures=TeX,Scale=MatchLowercase}
\fi
% use upquote if available, for straight quotes in verbatim environments
\IfFileExists{upquote.sty}{\usepackage{upquote}}{}
% use microtype if available
\IfFileExists{microtype.sty}{%
\usepackage[]{microtype}
\UseMicrotypeSet[protrusion]{basicmath} % disable protrusion for tt fonts
}{}
\PassOptionsToPackage{hyphens}{url} % url is loaded by hyperref
\usepackage[unicode=true]{hyperref}
\hypersetup{
            pdfborder={0 0 0},
            breaklinks=true}
\urlstyle{same}  % don't use monospace font for urls
\IfFileExists{parskip.sty}{%
\usepackage{parskip}
}{% else
\setlength{\parindent}{0pt}
\setlength{\parskip}{6pt plus 2pt minus 1pt}
}
\setlength{\emergencystretch}{3em}  % prevent overfull lines
\providecommand{\tightlist}{%
  \setlength{\itemsep}{0pt}\setlength{\parskip}{0pt}}
\setcounter{secnumdepth}{0}
% Redefines (sub)paragraphs to behave more like sections
\ifx\paragraph\undefined\else
\let\oldparagraph\paragraph
\renewcommand{\paragraph}[1]{\oldparagraph{#1}\mbox{}}
\fi
\ifx\subparagraph\undefined\else
\let\oldsubparagraph\subparagraph
\renewcommand{\subparagraph}[1]{\oldsubparagraph{#1}\mbox{}}
\fi

% set default figure placement to htbp
\makeatletter
\def\fps@figure{htbp}
\makeatother


\date{}

\begin{document}

{\ldots{}.14\ldots{}.}

{of the two given surds. If the square root does not exist, the surds
should be kept separately.}

{द्विकाष्टीमत्योस्त्रिभसंख्ययोश्च योगान्तरे ब्रूहि सखे करण्योः । }

{त्रिसप्तमित्योश्च चिरं विचिन्त्य चेत् षड्विधं वेत्सि सखे }

{करण्याः ।। ३५ ।। }

{Give the sum and difference of the pair of surds (1) }{√}{2 and }{√}{8
(2)}{√}{3 and }{√}{27. After due thought give the sum and difference of
}{√}{3 and }{√}{7.}

{द्विज्यष्टसंख्यागुणकः करण्योर्गुण्यस्त्रिसंख्या च सपञ्चरूपा । }

{वधं प्रचक्ष्वाऽऽशु विपञ्चरूपे गुणेऽथवा व्यर्कमिते करण्यौ ।। ३६ ।। }

{Find the product of }{√}{2 + }{√}{3 + }{√}{8 and }{√}{3+5. Also give
the product of }{√}{3+5 and }{√}{3 + }{√}{12-5.}

{क्षयो भवेच्च क्षयरूपवर्गश्चेत्साध्यतेऽसौ करणीत्वहेतोः । }

{ऋणात्मिकायाश्च तथा करण्या मलं क्षयो रूपविधानहेतोः ।। ३७ ।। }

{If for converting a negative number to a surd the negative number is
squared, the sign of the surd must be kept negative. In like manner if a
negative surd is changed to an integer after finding the square root,
the integer must bear the negative sign,}

{धनर्णताव्यत्ययमीप्सितायाश्छेदे करण्या असकृद् विधाय । }

{तादृक्छिदा भाज्यहरौ निहन्याद् एकैव यावत् करणी हरेस्यात् । । }

{भाज्यास्तया भाज्यगताः करण्यो लब्धाः करण्यो यदि योगजाः स्युः । }

{विश्लेषसूत्रेण पृथक् च कार्या यथा तथा प्रष्टरभीप्सिताः स्युः ।। ३८ ।। }

{To simplify the denominator, we should change the sign of one surd in
that and multiply both numerator and denominator by the expression thus
obtained. This should be repeated till only one surd is left in the
denominator. Dividing the numerator by this surd if the surds so
obtained can be analysed into more surds that should be done by the rule
given in the next stanza. In this way desired surds can be had.\\
}

\end{document}
