\documentclass[]{article}
\usepackage{lmodern}
\usepackage{amssymb,amsmath}
\usepackage{ifxetex,ifluatex}
\usepackage{fixltx2e} % provides \textsubscript
\ifnum 0\ifxetex 1\fi\ifluatex 1\fi=0 % if pdftex
  \usepackage[T1]{fontenc}
  \usepackage[utf8]{inputenc}
\else % if luatex or xelatex
  \ifxetex
    \usepackage{mathspec}
  \else
    \usepackage{fontspec}
  \fi
  \defaultfontfeatures{Ligatures=TeX,Scale=MatchLowercase}
\fi
% use upquote if available, for straight quotes in verbatim environments
\IfFileExists{upquote.sty}{\usepackage{upquote}}{}
% use microtype if available
\IfFileExists{microtype.sty}{%
\usepackage[]{microtype}
\UseMicrotypeSet[protrusion]{basicmath} % disable protrusion for tt fonts
}{}
\PassOptionsToPackage{hyphens}{url} % url is loaded by hyperref
\usepackage[unicode=true]{hyperref}
\hypersetup{
            pdfborder={0 0 0},
            breaklinks=true}
\urlstyle{same}  % don't use monospace font for urls
\IfFileExists{parskip.sty}{%
\usepackage{parskip}
}{% else
\setlength{\parindent}{0pt}
\setlength{\parskip}{6pt plus 2pt minus 1pt}
}
\setlength{\emergencystretch}{3em}  % prevent overfull lines
\providecommand{\tightlist}{%
  \setlength{\itemsep}{0pt}\setlength{\parskip}{0pt}}
\setcounter{secnumdepth}{0}
% Redefines (sub)paragraphs to behave more like sections
\ifx\paragraph\undefined\else
\let\oldparagraph\paragraph
\renewcommand{\paragraph}[1]{\oldparagraph{#1}\mbox{}}
\fi
\ifx\subparagraph\undefined\else
\let\oldsubparagraph\subparagraph
\renewcommand{\subparagraph}[1]{\oldsubparagraph{#1}\mbox{}}
\fi

% set default figure placement to htbp
\makeatletter
\def\fps@figure{htbp}
\makeatother


\date{}

\begin{document}

{\ldots{}.32\ldots{}.}

{On a plane ground a bamboo of 32 hands is standing. Due to force of the
wind it cracked at one place. Its end touched the ground at a distance
of 16 hands from the bottom of the bamboo. Tell me at how many hands
from the bottom the bamboo got cracked.}

{चक्रक्रौंचाकुलितसलिले क्वापि दृष्टं तडागे }

{तोयादूर्ध्वं कमलकलिकाग्रं वितस्तिप्रमाणम् । }

{मन्दं मन्दं चलितमनिलेनाहतं हस्तयुग्मे }

{तस्मिन्मग्नं गणक कथय क्षिप्रमम्बुप्रमाणम् ।। ११२ ।। }

{In a pond whose water was visited by herons and geese, the end of a
lotus bud was seen at a height of one vitasti from the water level.
Gradually it moved because of of the wind and dipped into the water at a
distance of two hands. Please tell me the depth of water in the pond.}

{वृक्षाद्धस्तशतोच्छ्र्याच्छतयुगं वापीं कपिः कोऽप्यगाद् }

{उत्तीर्याथ परो द्रुतं श्रुतिपथात्प्रोड्डीय किंचिद्रुमात् । }

{जातैवं समता तयोर्यदि गतवुड्डीयमानं कियद् }

{विद्वंश्चेत्सुपरिश्रमोऽस्ति गणिते क्षिप्रं तदाचक्ष्व मे ।। ११३।। }

{Two monkeys were perching on a tree 100 hands in height. One of them
climbed down the tree and went to a well situated at a distance of 200
hands. The other quickly flew upwards a little and came to the same well
by a straight path. If both of them had traversed equal distances, find
to what height the second monkey flew upwards.}

{पञ्चदशदशकरोच्छ्रायवेण्वोरज्ञातमध्यभूमिकयोः । }

{इतरेतरमूलाग्रगसूत्रयुतेर्लम्बमानमाचक्ष्व ।। ११४ ।। }

{There are two bamboos of height 15 hands and 10 hands. The distance
between them is unknown. Two strings join the top of each bamboo to the
bottom of the other. Find the height of the point\\
}

\end{document}
