\documentclass[11pt, openany]{book}
\usepackage[text={4.65in,7.45in}, centering, includefoot]{geometry}

\usepackage[table, x11names]{xcolor}
%\include{alias}

\usepackage{fontspec,realscripts}
\usepackage{polyglossia}

\setdefaultlanguage{sanskrit}
\setotherlanguage{english}
\setmainfont[Scale=1]{Times New Roman}
\newfontfamily\regular[Scale=1]{Times New Roman}
\defaultfontfeatures[Scale=MatchUppercase]{Ligatures=TeX} 
\newfontfamily\sanskritfont[Script=Devanagari]{Shobhika}
\newfontfamily\englishfont[Language=English, Script=Latin]{Linux Libertine O}
\newfontfamily\ab[Script=Devanagari, Color=purple]{Shobhika-Bold}
\newfontfamily\qt[Script=Devanagari, Scale=1, Color=violet]{Shobhika-Regular}
\newfontfamily\en[Language=English, Script=Latin]{Linux Libertine O}
%\newfontfamily\bqt[Script=Devanagari, Scale=1, Color=brown]{Shobhika-Regular}
%\newfontfamily\s[Script=Devanagari, Scale=0.9]{Shobhika-Regular}
\newfontfamily\s[Script=Devanagari, Scale=0.9]{Shobhika-Regular}
\newcommand{\devanagarinumeral}[1]{%
	\devanagaridigits{\number\csname c@#1\endcsname}}
\usepackage{fancyhdr}
\pagestyle{fancy}
\renewcommand{\headrulewidth}{0pt}
%\newfontfamily\e[Scale=0.8]{Shobhika-Regular}
\XeTeXgenerateactualtext=1
\usepackage{enumerate}
%\pagestyle{plain}
%\pagestyle{empty}
\usepackage{afterpage}
\usepackage{amsmath}
\usepackage{amssymb}
\usepackage{tikz}
\usepackage{graphicx}
\usepackage{longtable}
\usepackage{multirow}
\usepackage{footnote}
%\usepackage{dblfnote} 
\usepackage{xspace}
%\newcommand\nd{\textsuperscript{nd}\xspace}
\usepackage{array}
\usepackage{emptypage}
\usepackage{hyperref}   % Package for hyperlinks
\hypersetup{
	colorlinks,
	citecolor=black,
	filecolor=black,
	linkcolor=blue,
	urlcolor=black
}
\begin{document}
\thispagestyle{empty}
\begin{center}
\vspace{2cm}{\Large{\textbf{अनन्तशयनविश्वविद्यालयः}}}

\vspace{0.6cm}
\large\textbf{अनन्तशयनसंस्कृतग्रन्थावलिः~।} \\
\vspace{0.6cm}
\textbf{ग्रन्थाङ्कः~१८५.} \\
\vspace{0.6cm}
{\Large\textbf{श्रीमदार्यभटाचार्यविरचितम् }}\\
\vspace{3mm}
\Huge\textbf{आर्यभटीयं}\\
\vspace{3mm}
\Large\textbf{गार्ग्यकेरलनीलकण्ठसोमसुत्वविरचित-} \\
\vspace{2mm}
\Large\textbf{भाष्योपेतम्~।} \\
\vspace{0.4cm}
\textbf{(तृतीयः सम्पुटः \textendash\ गोलपादः)}\\
\end{center}

\begin{figure}[h!]
   \centering
  \includegraphics[scale=0.4]{latex/Capture1.JPG}
\end{figure}

\begin{center}
\large\textbf{प्रकाशकः} \\
\vspace{4mm}
\textbf{शूरनाड् कुञ्जन् पिल्ल, एम्.~ए. }\\
\vspace{2mm}
\large{पौरस्त्यग्रन्थप्रकाशनकार्यालयाध्यक्षः~।} \\
\vspace{15mm}
\large{अनन्तशयने 'भास्करमुद्रणालये' मुद्रितम्~।} \\
\vspace{2mm}
१९५७. 
\end{center}
\thispagestyle{empty}
\newpage
\thispagestyle{empty}
%\frontmatter
%\begingroup
%\renewcommand{\thetable}{\Roman{table}} 
\begin{center}
\large\textbf{\en PREFACE}
\end{center}
%\begin{onehalfspace}
\vspace{0.2cm}
%{\englishfont
{\en This is the third and the final volume of \emph{Āryabhaṭīya}. The first and the second volumes were published as Nos.~101 and 110 of the TRIVANDRUM SANSKRIT SERIES.\\

Āryabhaṭa, the celebrated astronomer, flourished in the fifth century A.D, at Pāṭalīputra, the capital of the Gupta Emperors. This is the earliest treatise on
astronomy of known authorship and is justly considered as one of the greatest works on the subject. (See the Introduction to Vol.~I. of \emph{Āryabhaṭīya} \textendash\ No.~101 of the T.S.S.)\\

\emph{Āryabhaṭīya} is a concise work designated to give very briefly and in pithy language the principles of Indian astronomy and mathematics. It contains only 121 verses in the \emph{Āryā} metre, divided into 4 \emph{pādas}, \emph{Gītikāpāda} (13 verses), \emph{Gaṇitapāda} (33), \emph{Kālakriyāpāda} (25) and \emph{Golapāda} (50).\\

Written in terse language and in the conventional manner, the work requires elucidation for proper understanding. But we do not find many commentaries of this very valuable work. The excellent \emph{Bhāṣya} of Nīlakaṇṭha Somasutvan published from this Library and carried to completion in this volume, is therefore, bound to be of great interest and immense value to the students of this science.\\

Somasutvan whose full name as given in the \emph{Bhāṣya} was Gārgya Kerala Nīlakaṇṭha Somasutvan was a highly accomplished astronomer of Kerala who lived in the 16th century A.D. He was popularly known as Keḷallūr Comātiri. (See \emph{Kerala Sāhitya Caritram} \textendash\ Mahākavi Ullūr \textendash\ Vol.~II. pp.~170). He is known to be 
the author of}
%\thispagestyle{plain}
%\end{onehalfspace}
%\thispagestyle{empty}
\newpage
\thispagestyle{empty}
\begin{center}2\\\end{center}
\vspace{0.5cm}
%%%%%%%%%%%%%%%%%%%%%%%%%%%%%%%%%%%%%%%%%%%%%%%%%%%%%%%%%%%%%

%\vspace{1mm}

\noindent {\en five works \textendash\ \emph{Āryabhaṭīya Bhāṣya, Tantrasaṅgraha, Siddhāntadarpaṇa, Golasāra} and \emph{Candraccāyāgaṇita}. He has given some facts about himself in \emph{Āryabhaṭīya Bhāṣya} at the end of the first part of \emph{Gaṇitapāda}}\textendash

\begin{sloppypar} 
\begin{quote} 
इति श्रीकुण्डग्रामजेन
गार्ग्यगोत्रेणाश्वलायनेन भाट्टेन केरलसद्ग्रामगृहस्थेन श्रीश्वेतारण्यनाथपरमेश्वरकरुणाधिकरणभूतविग्रहेण 
जातवेदःपुत्रेण शङ्कराग्रजेन जातवेदोमातुलेन दृग्गणितनिर्मापकपरमेश्वरपुत्रश्रीदामोदरात्तज्योतिषामयनेन रवितः आत्तवेदान्तशास्त्रेण सुब्रह्मण्यसहृदयेन नीलकण्ठेन सोमसुता विरचितविविधगणितग्रन्थेन दृष्टबहूपपत्तिना स्थापितपरमार्थेन कालेन शङ्कराद्यनिर्मिते श्रीमदार्यभटाचार्यविरचितसिद्धान्तव्याख्याने महाभाष्ये युक्तिप्रतिपादनपरे त्यक्तान्यथाप्रतिपत्तौ निरस्तदुर्व्याख्याप्रपञ्चे समुद्धाटितगूढार्थे सकलजनपदजातमनुजहिते निदर्शितगीतिपादार्थे सर्वज्योतिषामयनरहस्यार्थनिदर्शके समुदाहृतमाधवादिगणितज्ञाचार्यकृतयुक्तिसमुदाये
निरस्ताखिलविप्रतिपत्तिप्रपञ्चसमुपजनितसर्वज्योतिषामयनविमलहृदयसरसिजविकासे निर्मले गम्भीरे अन्यूनानतिरिक्ते 
गणितपादगतार्यात्रयस्त्रिंशद्व्याख्यानं समाप्तम्~। 
\end{quote} 
\end{sloppypar} 
{\en From this it is clear that he was a resident of \emph{Tṛkkaṇṭiyūr} in South Malabar. It can also be gathered from other statements of his own,}

\begin{quote} 
यन्मयात्र केषाञ्चित्सूत्राणां तद्युक्तीः प्रतिपाद्य कौषीतकिनाढ्येन नारायणाख्येन व्याख्यानं कारितम् अतस्तदेवात्र 
लिख्यते~।' 'इतीदं प्रथमे वयस्येव वर्तमानेन मया द्वितीयवयसि स्थितेन कौषीतकिनाढ्येन कारितम् अत्र केषाञ्चिद्युक्तयः पुनरस्मदनुजेन शङ्कराख्येन तत्समीपेऽध्यापयता वर्तमानेन तस्मै प्रतिपादिताः, तस्याढ्यत्वात्स्वातन्त्र्याच्च तत्र व्यापारश्च निर्वृत्तः~। त्वस्मिन् स्वर्गते पुनरत एव मयाद्य प्रवयसा ज्ञाता युक्तीः प्रतिपादयितुं भास्करादिभिरन्यथा व्याख्यातानां कर्माण्यपि प्रतिपादयितुं यथाकथञ्चिदेव व्याख्यानमारब्धम्~।' 'तत्रेयं त्रिपाद्यस्माभिर्व्याचिख्यसिता, यतः तद्व्याख्येयरूपत्वाद्गीतिकापादस्यैतद्व्याख्यानेनैव अर्थः प्रकाशेत्~।'
\end{quote} 


%\end{onehalfspace}
%\thispagestyle{empty}
\newpage
\thispagestyle{empty}
\begin{center}3\\\end{center}
\vspace{0.2cm}
%%%%%%%%%%%%%%%%%%%%%%%%%%%%%%%%%%%%%%%%%%%%%%%%%%%%%%%%%%%%%%
%\begin{center}3\\\end{center}
%\vspace{1mm}
\begin{quote}
\textbf{इति कौषीतकी श्रुत्वा नेत्रनारायाणः प्रभुः~। \\
मह्यं न्यवेदयत्तस्मै तदैवं प्रत्यपादयम्~॥} etc.,
\end{quote}

\noindent {\en that the commentary of \emph{Āryabhaṭīya} was originally undertaken at the behest of his patron \renewcommand{\thefootnote}{*}\footnote{Netranārāyaṇa is the title of Alvānceri Tamprākkal, a Nampūtiri religious dignity of South Malabar, enjoying high social standing.}Netranārāyaṇa and that it was completed after the death of his patron and during the author's old age. The claim made by the author on behalf of this great \emph{Bhāṣya} cannot be considered to be exaggerated in any way.\\

In the publication of this work, I have been assisted by Sri K.~S. Mahādeva Sāstri, Curator (now retired), Sri N.~Raghavan Nair, Senior Pandit, Sri M.~P.~Parameswaram
Nampūtiri, Sri N.~Parameśwara Sāstri and Sri V.~Nārāyaṇana Nampūtiri, M.~A., B.~T., Pandits in this Library.}

\begin{table}[h!]
 \centering
 \begin{tabular}{ccc}
 {\en University Manuscripts}  & \multirow{3}{*}{$\Bigg\}$}& {\en SURANĀḌ KUNJAN PILLAI,} \\
 {\en Library, Trivandrum,}  & & \\
       21-10-1957. &  & \emph{\en Honorary Director.} \\
    \end{tabular}
\end{table}
%\endgroup  

%\thispagestyle{empty}
\newpage
\thispagestyle{empty}
%%%%%%%%%%%%%%%%%%%%%%%%%%%%%%%%%%%%%%%%%%%%%%%%%%%%%%%%%%%%%
\begin{center}
\large\textbf{उपोद्घातः} 
\end{center}

तृतीयश्चायमन्तिमः सम्पुटः सभाष्यस्यार्यभटीयस्य~। प्रथमद्वितीयौ चैतदीयसम्पुटौ अस्यामेवानन्तशयनसंस्कृतग्रन्थावल्यां १०१, ११०
तमाङ्कतया प्रकाशितचरौ~।\\

अस्य प्रणेता सुप्रथितो ज्यौतिषिक आर्यभटः क्रिस्त्वब्दीयपञ्चमशतके गुप्तसम्राजां मण्डलान्तर्गतं पाटलीपुत्रनगरमावसत्~। तदीयश्चायं निबन्धो ज्ञातकर्तृकेषु ज्योतिःशास्त्रविषयकग्रन्थेषु आदिमः सर्वोत्तमश्च वरीवर्ति~। \\

ग्रन्थेऽस्मिन् ज्योतिःशास्त्रतत्त्वानि संक्षिप्य प्रतिपादितानि~। पादचतुष्टयीमितेऽस्मिन् \textendash\ गीतिकापादे त्रयोदश, गणितपादे त्रयस्त्रिंशत्, कालक्रियापादे पञ्चविंशतिः, गोलपादे पञ्चाशच्चेति आहत्य एकविंशत्युत्तरशतमार्याः सन्ति~।\\

अतिसंक्षेपेण सङ्केतबहुलतया च निबद्धस्य अस्य अर्थावगतिः ऋते व्याख्यभ्यो न सुष्ठु सम्भवति; परन्तु विरला एवं ता इति
नीलकण्ठसोमसुत्वना विरचितं महदिदं भाष्यं ग्रन्थालयादस्मात् पूर्वं प्राकाशि, तदिदमस्मिन् सम्पुटे समाप्तमिति ज्योतिःशास्त्राध्येतृभ्यस्तोषाय भवेदिति सुदृढं विश्वसिमः~। \\

भाष्यकारोऽयं गार्ग्यकेरलनीलकण्ठसोमसुत्वाः 'केळल्लूर् चोमातिरि' इति केरलेषु प्रथां गतः केरलीयज्योतिर्विदां प्राग्रसरेष्वन्यतमः क्रिस्त्वब्दीयषोडशशतकजीवी चासीत्~। एतत्कर्तृकाश्च आर्यभटीयभाष्यं, तन्त्रसङ्ग्रहः, 
सिद्धान्तदर्पणं, गोलसारः, चन्द्रच्छायागणितमिति समुपलभ्यन्ते पञ्च ज्योतिर्विषयका निबन्धाः~। आर्यभटीयभाष्यस्थगणितपादान्ते भाष्यकारेणानेन
स्वविषयकाः केचन वृत्तान्ता उक्ताः\textendash 
%\thispagestyle{empty}
\newpage
\thispagestyle{empty}
%%%%%%%%%%%%%%%%%%%%%%%%%%%%%%%%%%%%%%%%%%%%%%%%%%%%%
\begin{center} २\\ \end{center}
%\vspace{2mm}
\begin{sloppypar} 
\begin{quote} 
	इति श्रीकुण्डग्रामजेन
	गार्ग्यगोत्रेणाश्वलायनेन भाट्टेन केरलसद्ग्रामगृहस्थेन श्रीश्वेतारण्यनाथपरमेश्वरकरुणाधिकरणभूतविग्रहेण 
	जातवेदःपुत्रेण शङ्कराग्रजेन जातवेदोमातुलेन दृग्गणितनिर्मापकपरमेश्वरपुत्रश्रीदामोदरात्तज्योतिषामयनेन रवितः आत्तवेदान्तशास्त्रेण सुब्रह्मण्यसहृदयेन नीलकण्ठेन सोमसुता विरचितविविधगणितग्रन्थेन दृष्टबहूपपत्तिना स्थापितपरमार्थेन कालेन शङ्कराद्यनिर्मिते श्रीमदार्यभटाचार्यविरचितसिद्धान्तव्याख्याने महाभाष्ये युक्तिप्रतिपादनपरे त्यक्तान्यथाप्रतिपत्तौ निरस्तदुर्व्याख्याप्रपञ्चे समुद्धाटितगूढार्थे सकलजनपदजातमनुजहिते निदर्शितगीतिपादार्थे सर्वज्योतिषामयनरहस्यार्थनिदर्शके समुदाहृतमाधवादिगणितज्ञाचार्यकृतयुक्तिसमुदाये
	निरस्ताखिलविप्रतिपत्तिप्रपञ्चसमुपजनितसर्वज्योतिषामयनविमलहृदयसरसिजविकासे निर्मले गम्भीरे अन्यूनानतिरिक्ते 
	गणितपादगतार्यात्रयस्त्रिंशद्व्याख्यानं समाप्तम्~। 
\end{quote} 
\end{sloppypar} 

\indent वाक्येनानेन भाष्यकारोऽयं नीलकण्ठसोमसुत्वा दक्षिणमलबार्जिल्लान्तर्गततृक्कण्टियूर्देशवास्तव्योऽभूदिति सुव्यक्तम्~। 

\begin{quote} 
	यन्मयात्र केषाञ्चित्सूत्राणां तद्युक्तीः प्रतिपाद्य कौषीतकिनाढ्येन नारायणाख्येन व्याख्यानं कारितम् अतस्तदेवात्र 
	लिख्यते~।' 'इतीदं प्रथमे वयस्येव वर्तमानेन मया द्वितीयवयसि स्थितेन कौषीतकिनाढ्येन कारितम् अत्र केषाञ्चिद्युक्तयः पुनरस्मदनुजेन शङ्कराख्येन तत्समीपेऽध्यापयता वर्तमानेन तस्मै प्रतिपादिताः, तस्याढ्यत्वात्स्वातन्त्र्याच्च तत्र व्यापारश्च निर्वृत्तः~। त्वस्मिन् स्वर्गते पुनरत एव मयाद्य प्रवयसा ज्ञाता युक्तीः प्रतिपादयितुं भास्करादिभिरन्यथा व्याख्यातानां कर्माण्यपि प्रतिपादयितुं यथाकथञ्चिदेव व्याख्यानमारब्धम्~।' 'तत्रेयं त्रिपाद्यस्माभिर्व्याचिख्यसिता, यतः तद्व्याख्येयरूपत्वाद्गीतिकापादस्यैतद्व्याख्यानेनैव अर्थः प्रकाशेत्~।'
\end{quote} 

\begin{quote}
\textbf{इति कौषीतकी श्रुत्वा नेत्रनारायाणः प्रभुः~। \\
मह्यं न्यवेदयत्तस्मै तदैवं प्रत्यपादयम्~॥} 
\end{quote}
%\thispagestyle{empty}
\newpage
\thispagestyle{empty}
%%%%%%%%%%%%%%%%%%%%%%%%%%%%%%%%%%%%%%%%%%%%%%%%%%%%%%%%%%
\begin{center} ३\\ \end{center}
%\vspace{4mm}
\indent इत्यादिप्रस्तावेन प्रभोर्नेत्रनारायणस्य प्रचोदनया आर्यभटीयभाष्यनिर्मितिः स्वेनारब्धा, पुरस्कर्तुर्जीवितशेषं स्वस्य
वार्द्धके भाष्यमिदं समापितं च इत्यादि स्पष्टमवगम्यते~। \\

ग्रन्थोऽयं प्रकाश्यमानो ज्योतिःशास्त्रविषयकेषु निबन्धेषु नान्यसामान्यं महिमानमावहतीति वयं विश्वसिमः~। \\

ग्रन्थस्यास्य प्रकाशने साह्यमाचरितवद्भ्यः के. एस्. महादेवशास्त्रिभ्यः, एन्. राघवन् नायर्, एम्. पि. परमेश्वरन् 
नम्पूतिरि, एन्. परमेश्वरशास्त्री, वि. नारायणन् नम्पूतिरि एम्. ए., इत्येतेभ्यश्च सुबहु धारयामि~। 
\vspace{2cm} 
\begin{table}[h!]
    \centering
    \begin{tabular}{ccc}
     अनन्तशयनम्    &\multirow{2}{*}{$\Bigg\}$}& \textbf{शूरनाड् कुञ्जन पिल्ल}, \\
      २१-१०-१९५७.&   & \textbf{संस्कृतग्रन्थप्रकाशनकार्यालयाध्यक्षः}~।\\
    \end{tabular}
\end{table}

%\thispagestyle{empty}
\newpage
\thispagestyle{empty}
%%%%%%%%%%%%%%%%%%%%%%%%%%%%%%%%%%%%%%%%%%%%%%%%%%%%%%%%
\begin{center}
\textbf{विषयानुक्रमणी} \\
\rule{0.08\linewidth}{0.3pt}\end{center} 
\renewcommand{\arraystretch}{1.25}

\begin{tabular}{lp{2cm}p{1cm}r}
  \hspace{1cm} \textbf{विषयः} &&& \textbf{पृष्ठम्} \\
&&&\\
%\hfill
अपमण्डलसंस्थानम् & ....&& २\\
अपक्रममण्डलचारिणो ग्रहाः &.... &&३\\
विक्षेपमण्डलसंस्थानम् &....& &४\\
चन्द्रादीनामुदयास्तमयपरिज्ञानम् &....& &१५\\
भूम्यादेः प्रकाशहेतुः &....&& १६\\
कक्ष्याभुवोः संस्थानम् &....& &२१\\
भूगोलस्वरूम् &....&& २२\\
भूमेः वृद्ध्यपचयौ &.... &&२३\\
भूमेः प्राग्गमनम् &....&& २४\\ 
भपञ्जरस्य भ्रमणहेतुः &.... &&,,\\
मेरुप्रमाणं तत्स्वरूपञ्च& ....&& २५\\
मेरुबडवामुखावस्थानम् &....&& २७\\
भूचतुर्भागान्तरालगताश्चतस्रो नगर्यः &....&& २८\\
लङ्कोज्जयिन्योरन्तरालप्रदेशः &.... &&,,\\
भचक्रस्य दृश्यादृश्यविभागः &....&& ३०\\
भचक्रे देवासुरदृश्यभागः &....&& ३१\\
देवादीनां दिनप्रमाणम् &.... &&,,\\
गोलकल्पना &....&& ३२\\
द्रष्टृवशादधऊर्ध्वविभागः &....&& ३५\\
दृङ्मण्डलं दृग्क्षेपमण्डलञ्च & ....&& ३६\\
गोलप्रमाणोपायः &....&& ३८\\
क्षेत्रकल्पनाप्रकारोऽक्षावलम्बकौ च &....&& ३९\\
स्वाहोरात्रार्धम् &....&& ४२\\
निरक्षदेशे राश्युदयप्रमाणम् &....&& ४५\\
दिननिशोः क्षयवृद्ध्यानयनम् &....&& ४७\\
 \end{tabular}
%\thispagestyle{empty}


\newpage
\thispagestyle{empty}
%%%%%%%%%%%%%%%%%%%%%%%%%%%%%%%%%%%%%%%%%%%%%%%%%%%%%%%%%%%%%
\begin{center}२\\\end{center}
\vspace{0.5cm}
\renewcommand{\arraystretch}{1.25}
  \begin{tabular}{llp{0.6cm}r}
\hspace{1cm} \underline{विषयः} &&& \underline{पृष्ठम्} \\
&&&\\
स्वदेशराश्युदयः &....&& ५०\\
इष्टकाले शङ्क्वानयनम् &....&& ५६\\
शङ्क्वग्रानयनम् &....&& ५८\\
अर्काग्रानयनम् &....&&,,\\
अर्कस्य समप्रवेशकाले शङ्क्वानयनम् &....&& ६६\\
मध्याह्नशङ्कुस्तच्छाया च& ....&& ६७\\
दृक्क्षेपज्यानयनम् &....&& ७२\\
दृग्गतिज्यालम्बनयोजनानयनम् &....&& ८१\\
विक्षेपेण दृक्कर्म &....&& ८३\\
आयनं दृक्कर्म &....&& ८९\\
अर्केन्दुग्रहणस्वरूपम् &....&& ९५\\
ग्रहणकालः &....&& ९६\\
भूच्छायादैर्घ्यम् &....&& ९७\\
व्यासयोजनानयनम् &.... &&,,\\
स्थित्यर्धानयनम्  &....&& ९९\\
विमर्दार्धकालानयनम् &.... &&,,\\
ग्रस्तशेषप्रमाणम् &....&& १००\\
तात्कालिकग्रासप्रमाणम् &....&& १०१\\
आक्षवलनं आयनबलनञ्च & ....&& १०२\\
गृहीतबिम्बवर्णाः &....&& १०६\\
सूर्यग्रहणेऽदृश्यभागः &....&& १२७\\
शास्त्रप्रतिपादितग्रहगत्यादेः दृग्संवादात् स्फुटत्वम् &....&& १२८ \\
शास्त्रस्य मूलम् &....&& १५९\\
उपसंहारः &....&& १६१\\
\end{tabular}
\vspace{0.1cm}
\begin{center}\rule{0.08\linewidth}{0.5pt}\end{center}
%\thispagestyle{empty}

\newpage
%%%%%%%%%%%%%%%%%%%%%%%%%%%%%%%%%%%%%%%%%%%%%%%%%%%%%%%%%%%%%
\begin{center}
\textbf{\vspace{2cm}{॥ श्रीः ॥}}

\textbf{\large श्रीमदार्यभटाचार्यविरचितम्}

\vspace{0.2cm}{\textbf{\LARGE आर्यभटीयं}}

\vspace{0.2cm}\textbf{\large गार्ग्यकेरलनीलकण्ठसोमसुत्वविरचितेन}

\vspace{0.2cm}\textbf{\large भाष्येण समेतम्~।}

\rule{3cm}{.3mm} 

\vspace{0.4cm}
(तृतीयो भागः)

\vspace{0.2cm}
\large \textbf{गोलपादः~।}\\
%\rule{5cm}{0.2mm}\\
\vspace{0.4cm}

\textbf{स्वर्क्षदेशोदितज्योतिर्गोभिर्योऽर्कः स्वरश्मिभिः~। \\
प्रकाशयति भूगोलं भगोलं च नमामि तम्~॥\\
शिष्यैस्तत्त्वं मया ज्ञेयमुच्यमानं भगोलयोः~। \\
इति ब्रवीति यस्तस्मै भटाय श्रीमते नमः~॥} \\
\end{center}

कल्पादेर्वा तत्तद्युगादेर्वा स्वग्रन्थकरणकालात् प्रभृति वा एतावति काले गते ग्रहाणां राश्यादिकं स्फुटमेतावदित्येतावदेवेह
मध्यमस्फुटकर्मभ्यां ज्ञायते~। न पुनस्तस्य स्फुटसावनेन सह सम्बन्धः~। यदा पुनरहर्गणात् त्रैराशिकेन मध्यममानीय स्फुटीक्रियते तदापि तस्य 
स्फुटस्यार्क्षमानेनैव सह सम्बन्धो ज्ञायते~। एतावत्यार्क्षमाने याते एतावच्च स्फुटमिति~। तेनापि न स्फुटसावनेन सह सम्बन्धो ज्ञातः~। स्फुटसावनमानगत एव कालः शङ्कुकपालयन्त्रादिभिर्ज्ञायते~। ततस्तेन सहास्य सम्बन्धे ज्ञात एवेदानीं राजकुमारे जाते एतावत् स्फुटमिति वक्तुं शक्यम्~। ग्रहणादावपि सावनेन सह सम्बन्धे ज्ञात एव तत्सदसद्भावस्पर्शमोक्षादिकं ज्ञातुं\\

%\thispagestyle{empty}
\afterpage{\fancyhead[CE] {आर्यभटीये सभाष्ये }}
\afterpage{\fancyhead[CO]{गोलपादः~। }}
\afterpage{\fancyhead[LE,RO]{\thepage}}
\cfoot{}

\newpage
%%%%%%%%%%%%%%%%%%%%%%%%%%%%%%%%%%%%%%%%%%%%%%%%%%%%%%%%%%%%%%
\renewcommand{\thepage}{\devanagarinumeral{page}}
\setcounter{page}{2}

%\vspace{3cm} २  \hspace{4cm}  आर्यभटीये सभाष्ये

\noindent शक्यमिति तन्न्यायप्रदर्शनपरो गोलपाद आरभ्यते~। तत्र ग्रहभ्रमणस्य प्रवहवायुभ्रमणवैलक्षण्यमेव मध्यमादन्यस्य प्रायेण कृत्स्त्रस्य ग्रहगणितकर्मणो मूलमिति तद्वैलक्षण्यं प्रथमं दर्शयति\textendash
\begin{quote}
{\ab मेषादेः कन्यान्तं सममुदगपमण्डलार्धमपयातम्~॥\\
तौल्यादेर्मीनान्तं शेषार्धं दक्षिणेनैव~॥~१~॥}
\end{quote}

इति~। ग्रहचक्रस्य प्रागर्धं सममुदगेवापयातम् ; इतरदर्धं समं दक्षिणतश्च~। कुत उदग्दक्षिणतश्चापयातम्~। समपूर्वापरत इत्येतच्च सममुदगित्यनेनैव सिद्धं, दक्षिणोत्तरप्रतियोगित्वात् पूर्वापरयोः~। यद्येकान्तरयोः विदिशोरपयातं तर्हीदृशविदिङ्मार्गोऽवधिरित्यादिकं लोके न्यायसिद्धमेव~। अस्य 
मण्डलाकारत्वात्तदवधिभूतः प्रदेशोऽपि मण्डलाकार एव, यदपेक्षयास्य तिर्यक्त्वम्~। एकस्यैवोभयपार्श्वगतत्वे तदपेक्षयास्य तिर्यक्त्वमेव युज्यते; न पुनरेकं मण्डलम\renewcommand{\thefootnote}{१}\footnote{अर्धांश \textendash\ क. घ.}र्धशः छित्त्वा 
तस्योदग्दक्षिणपार्श्वयोः पूर्वापरतया क्षिप्यते तदैवास्य ऋजुतया भाव्यम्~। अत्र तु न तथेति शेषार्धमित्यत्र शेषशब्देन द्योत्यते~। तस्मात्तदवधिभूतमण्डलापेक्षयास्य तिर्यक्त्वमेव युज्यते~। तच्चाध ऊर्ध्वमेव~। तस्य तथात्व एवास्य तदभितोऽर्धशोगतत्वं\renewcommand{\thefootnote}{२}\footnote{अर्धशोऽधोगंतत्व \textendash\ क.}
युज्यते~। तस्य तिर्यक्त्वे तु तदपेक्षया तस्यार्धयोरुपर्यधोगतत्वमेव युज्यते~। तस्मात् तन्मण्डलं पूर्वापरमेव~। कुत्र पुनस्तन्मण्डलं पूर्वापरमध ऊर्ध्वञ्च~। क्वचिदेव तस्याध ऊर्ध्वत्वं संभवति~। भूमेर्गोलाकारत्वस्य सर्वत्र प्राणिनिवाससद्भावस्य च ऊर्ध्वाधोदिशोर्द्रष्ट्रपेक्षया भेदस्य च
वक्ष्यमाणत्वात्~। {\qt अजार्कोदयाच्च लङ्कायाम्} इति लङ्कामधिकृत्य शास्त्रप्रणयनात् तत्रैव तस्याध ऊर्ध्वत्वम्~। तदेव वायुगोलमध्यम्~। घटिकमण्डलमिति तस्याख्या च लोके प्रसिद्धा~। कथमस्य लोकप्रसिद्धिः, ग्रहगणितादन्यत्रोपयोगाभावादितिचेत् नैष दोषः~। यश्च यैर्व्यहरति स तत्र लोक इति हि न्यायविदो वदन्ति~। 
%\thispagestyle{empty} 

\newpage 

\begin{quote}
{\qt लोक्येते यत्र शब्दार्थौ लोकस्तेन स उच्यते}
\end{quote} 

\indent इति च~। तस्माद्घटिकामण्डलादेवमपयातमपक्रमाख्यमिदं मण्डलम्~। तदपेक्षयापक्रान्तत्वादेवपाक्रमशब्देनोच्यत इति ; तदेवापमण्डलमिति चोच्यते~। तस्य यदर्धं मेषादेः प्रभृति कन्यान्तं तद्घटिकाख्यात् क्रमेणोदग्दिश्येवापयातम्~। ततस्तन्यध्यस्य मिथुनकर्कटसन्धेः परममपयानं च सिद्धम्~। तदेव {\qt भापक्रमो ग्रहांशा} इत्यत्र भेति पठितम्~। तत्र चतुर्विंशतिभागात्मकं तयोर्विवरमित्यर्थः~। ज्योतिश्चक्रपञ्चदशभागतुल्यं च तत्~। एवं तुलादिमीनान्तापरार्ध\renewcommand{\thefootnote}{१}\footnote{मीनान्तार्ध \textendash\ क. ग. घ.} स्य च तन्मध्यधनुरन्तरमपि\renewcommand{\thefootnote}{२}\footnote{मध्यं धनुरन्तरमपि \textendash\ ग. घ.} तावदपयातम्~। तत् ज्याकारं वा चापाकारं वा परममपयानमित्येष संशयोंऽशशब्देन व्युदस्यते~। परिधिभाग एव ह्यंशशब्देन व्यवहारस्तन्त्रकाराणां, कलाभिरेव ज्याव्यवहार इति नियमात्~। तस्मात्तत्परिच्छेदाय अध ऊर्ध्वं दक्षिणोत्तरमपि किञ्चिन्मण्डलं कल्पनीयं, यद्गतं तद्विवरं चतुर्विशतिभागात्मकम्~॥~१~॥ \\

\indent एवमयमपक्रममण्डलस्य घटिकामण्डलस्य च प्रदेशतः सम्बन्धः~। तयोरन्यतरस्य कर्णत्वे चतुर्विंशतिभागार्धज्यातुल्या भुजा~।	तदग्रयोरन्तरं केवलं चतुर्विंशतिभागसमस्तज्यातुल्यं च~। किमर्थं पुनरिदं कल्प्यत इत्याह\textendash

\begin{quote}
{\ab ताराग्रहेन्दुपाता भ्रमन्त्यजस्रमपमण्डलेऽर्कश्च~।\\
 अर्काच्च मण्डलार्धे भ्रमति हि तस्मिन् क्षितिच्छाया~॥ २ ॥}
\end{quote}

\indent इति~। अर्कोऽजस्रमपमण्डल एव भ्रमति~। न कदाचिदपि तत उत्तरतो दक्षिणतो वा भ्रश्यति, इत्येष एव तस्य पन्थाः~। न केवलं 
तस्यैवायं पन्थाः; ताराग्रहेन्दूनां पाताश्च तत्रैव भ्रमन्ति~। क्षितिबिम्बेन तिरोहितत्वात् भास्वतस्तद्रश्मयो यत्र न प्रसरन्ति, तत्परिच्छिन्नो य आकाशप्रदेशः स इह भूच्छायाशब्देनाभिधीयते~। सा च सूच्यग्रा ; क्षितिमण्डलान्महत्त्वादर्कमण्डलस्य~। सा चापक्रममण्डल एव मध्यं
कृत्वा 
\newpage 

\noindent सदा भ्रमति ; अपमण्डलमार्गमभितोऽर्धशो विभक्तत्वात् क्षितिमण्डलस्य~। भ्रमन्त्यास्तस्याः कियती गतिः, अर्कान्मण्डलार्धे हि सा भ्रमति ; तस्मादर्कस्फुटे राशिषट्कं प्रक्षिप्य तत्स्फुटं ज्ञेयम्~। अत एवाह मुञ्जालः \textendash\ {\qt छायाग्रहः
सषड्भोऽर्क} इति~॥~२~॥ \\

एवमेतेषामष्टानामयमेव मार्गः~। इतरेषां चन्द्रादीनां षण्णां\renewcommand{\thefootnote}{१}\footnote{इतरेषां षण्णाम् \textendash\ क. ग.} ग्रहाणां पुनः को मार्ग इत्यत आह\textendash  

\begin{quote}
 {\ab अपमण्डलस्य चन्द्रः पाताद्यात्युत्तरेण दक्षिणतः~।\\
कुजगुरुकोणाश्चैवं शीघ्रोच्चेनापि बुधशुक्रौ~॥~३~॥}
\end{quote}

\indent इति~। तेषामप्ययमेव मार्गः~। किन्तु ते अजस्रं न तत्रैव भ्रमन्ति~। कुतश्चित् कारणविशेषात् तत उत्तरतो दक्षिणतश्च विक्षिप्यमाणत्वात्~। तेषां पुनः कदा वा तत्प्राप्तिः, कुतः प्रभृति वोत्तरतो दक्षिणतश्च विक्षिप्यते~। 
चन्द्रस्तावत् स्वपातात् प्रभृत्यपमण्डलादुत्तरतो दक्षिणतश्च याति इति प्रतिभगणं तस्यापक्रममण्डलयोगप्रदेशो भिद्यते~। पातयोर्गतिमत्त्वात् विलोमगौ च तौ चन्द्रपातौ~। तत्र प्रथमपातात्\renewcommand{\thefootnote}{२}\footnote{पातादुत्तरतः \textendash\ ग. घ. } स्वर्भानोरुत्तरतः, द्वितीयपातात्\renewcommand{\thefootnote}{३}\footnote{पातात्प्रभृति दक्षिणतः \textendash ग. घ.} केतोः दक्षिणत इत्येतत् क्रमवशात् सिद्धम्~। ततः कस्मिंश्चित्पर्ययेऽपमण्डलस्य येन प्रदेशेन संयोगः ततो द्वितीये पर्यये ततः पृष्ठत एव संयोगः~। तत एकोनविंशत्या संवत्सरैः सर्वेषामप्यपमण्डलपरिधिभागानां पातेन संयोगात् कृत्स्नस्याप्यपमण्डलपरिधेश्चन्द्रयोगयोग्यता स्यादेव; इत्ययमेवास्यापि मार्गः~। यतस्तत एवोभयतः साम्येन विक्षिप्यते~। प्रतिपर्ययं पातद्वयसाम्ये अपमण्डलसंयोगः। यदा पुनरुभौ पातौ चन्द्राकान्तप्रदेशात् राशित्रयान्तरितौ तदापक्रममण्डलात् परमो विक्षेपः~। तत्र यदा प्रथमपातात् राशित्रयाधिक्यं चन्द्रस्फुटस्य तदा स परमविक्षेप उत्तरः, द्वितीयपाताद्राशित्रयाधिक्ये 

\newpage 

\noindent परमविक्षेपो दक्षिणः~। ए\renewcommand{\thefootnote}{१}\footnote{एवं पातद्वयात्प्रभृति क्रमेण विक्षेपवृद्धिः ; एवं राशित्रयातिकान्ते युग्मे पदे ऋमेण विक्षेपो हीयमानः ङ. छ.}वं प्रथमपातात् क्रमेणापमण्डलमार्गादुदगुदग्विप्रकृष्यते~। द्वितीयपातात्प्रभृति क्रमेण दक्षिणतश्च~। एवं राशित्रयेण विवृद्धो विक्षेपः पुनः युग्मे पदे क्रमाद्धीयमानः पातद्वये शून्यतामेतीति~। कुजगुरुकोणाश्चैवं स्वस्वप्रथमपातादुत्तरतः द्वितीयपाताद्दक्षिणतश्च विक्षिप्यन्ते, अपि च मन्दस्फुटवशाच्च~। यथा चन्द्रस्य मन्दस्फुटवशाद्विक्षेपः तथैवैषां त्रयाणामपि~। एतदुक्तं भवति~। तेषामपि मन्दकर्णस्यैव विक्षेपः, ततो मन्दकक्ष्यामण्डलमध्यस्थद्रष्ट्रपेक्षया एष एव विक्षेपः~। 
पातोनमन्दस्फुटभुजाज्यातो यस्त्रैराशिकानीतः स एव भगोलनाभिगतस्य द्रष्टुरितो भिद्यते~। यो भिन्नः स एव च भगोलकलाप्रमितो विक्षेपः~। कुतः पुनरनयोर्भेदः, उच्यते~। भगोलमध्यनाभिकस्य कार्त्स्न्येनापमण्डलमार्गगस्य शीघ्रवृत्तस्य परिधौ यः शीघ्रोच्चसमः प्रदेशः तद्धि मन्दकर्मणि कक्ष्यामण्डलकेन्द्रमिति कालक्रियापाद एवोक्तम्~। तदेव मन्दोच्चनीचवृत्तस्य कर्णमण्डलस्य च केन्द्रम्~। एवमेतानि त्रीणि मण्डलानि अपमण्डलमार्गमभितोऽर्धश उत्तरतो दक्षिणतश्च विक्षिप्तानि~। तद्विक्षेपावधिभूतयोस्तत्तत्प्रथमद्वितीयपातयोः भ्रमणाधारत्वेन 
मन्दकर्णमण्डलमविक्षिप्तमपि कल्पनीयम्~। तत्रैव तावजस्रं वामं समानं च भ्रमतः~। ननु पूर्वं कालक्रियापादे छेद्यके प्रदर्शितस्य कस्यचिदपि मण्डलस्य न विक्षेपः, तत् कथमत्र केषांचिद्विक्षिप्तत्वमुच्यते~। नैष दोषः~। द्वयी हि स्फुटयुक्तिः~। तयोरेकैव छेद्यके प्रदर्श्या~। इतरांशो गोल एव प्रदर्श्यः~। अतो गोल एव कार्त्स्न्येन प्रदर्श्या~। एवं हि कालक्रियापादानन्तर्यं गोलपादस्य स्यात्~। ताराग्रहाणां तु विक्षेपवशादपि स्फुटस्य कियांश्चिद्भेदो युज्यत एव~। स चाल्पत्वादेवोपेक्षितः शास्त्रकारैः~। अपक्रममण्डलमार्गश्च भगोलनाभितः प्रभृति नक्षत्रकक्ष्यान्तं यावत्कल्पनीयः~। अतो 

\newpage


\noindent भगोलमध्यादूर्ध्वमपि शीघ्रोच्चनीचवृत्तस्य व्यासार्धतुल्ये\renewcommand{\thefootnote}{१}\footnote{तुल्येऽप्प्रप \textendash\ क. घ. ङ.}ऽपक्रममार्ग एव मन्दस्फुटयुक्त्यर्थं कल्प्यमानस्य गोलस्य घनमध्यम्~। तत्र कक्ष्या मण्डलादिकं मन्दस्फुटे पाततुल्ये कार्त्स्न्येनापक्रममण्डलमार्गस्थमेव, न विक्षिप्तम्~। ततः प्रभृत्येव विक्षेप इति मन्दस्फुटगोल एवापक्रममण्डले पातस्य भ्रमणम्~। तेन मन्दस्फुटादेव पा\renewcommand{\thefootnote}{२}\footnote{तः}तं विशोध्या\renewcommand{\thefootnote}{३}\footnote{ध्यः\textendash\ क. ग. घ.} तद्भुजाज्यां परमविक्षेपहतां त्रिज्यया हृत्वा 
तद्गोलविक्षेपो लभ्यते~। स च मन्दकर्णकलाप्रमित एव विशेषः~। स एव पुनर्भगोलकलाप्रमितः कियानिति त्रैराशिकान्तरेण ज्ञेयम्~। कुतः पुनरुभयीनां कलानां परिमाणभेदः? भगोलनाभित एष प्रवृत्ता भगोलकला मदस्फुटगोलघनमध्यत एव प्रवृत्ता मन्दकर्णकला इति~। शीघ्रकेन्द्रे मकरादिगे भगोलकला अन्याभ्यो महत्य एव कर्क्यादावल्पीयस्यश्च~। कुतः पुनः शीघ्रकेन्द्रायनवशात् तासामन्याभ्योऽल्पत्वं महत्त्वं च ; यतः शीघ्रकर्ण एव ग्रहाधिकस्य भगोलस्य व्यासार्धं ; स च शीघ्रकेन्द्रे मकरादौ मन्दकर्णान्महान्, कर्क्यादौ मन्दकर्णादल्पश्च~। यतो मन्दकर्ण 
एव शीघ्रकोटिफलं संस्कृत्य तत् कर्णं आनीयते~। विक्षेपे सति भूताराग्रहविवरमपि पूर्वप्रदर्शितात् भिन्नम्~। कथम्? 
विक्षेपे सति मन्दकर्णवर्गात्\renewcommand{\thefootnote}{४}\footnote{वशात् \textendash\ घ.} तद्ग्रहमध्यकक्ष्याप्रमितविक्षेपवर्गं त्यक्त्वा मूलीकृत्य यल्लब्धं सा तत्र विक्षेपकोटिः~। तस्यामेव शीघ्रकोटिफलं संस्कृत्य प्राग्वच्छीघ्रकर्ण आनेयः~। तदा न मन्दकर्णे कोटिफलं संस्कार्यमिति~। तद्वशाच्छीघ्रकर्णश्च भिद्यते~। अत एव शीघ्रभुजाफलं च~। तन्निमित्तत्वात् स्फुटस्यापि भेदः~। कथं पुनर्मध्यकक्ष्याकलाप्रमितविक्षेप आनीयते~। तत्रैवं त्रैराशिकम् \textendash\ यदि मन्दकर्णव्यासार्धे कक्ष्याकला आनीतमन्दकर्णकलातुल्याः, तदा मन्दकर्णकलाप्रमिते विक्षेपे कियत्यो मध्यकक्ष्याकलाः स्युरिति~। तत्र पातोनमन्दस्फुटभुजानीतस्य विक्षेपस्यावशिष्टमन्दकलाकर्णो गुणकारः~। व्यासार्धमेव 
\newpage

\noindent भागहारः~। फलं ग्रहापक्रममण्डलान्तरं मध्यकक्ष्याकलाप्रमितम्~। सा मन्दकर्णस्य भुजा~। तत्कोट्यामेव 
तत्कर्णवर्गविश्लेषमूलतुल्यायां शीघ्रफलं संस्कार्यं, तावत्येव मन्दगोलघनमध्याद्ग्रहस्यापमण्डलमार्गेणोच्छ्रितिरिति~। तत्\renewcommand{\thefootnote}{१}\footnote{कोटिमण्डलं \textendash\ ख.} कोटिफलं मन्दगोलपतमपमण्डलस्यैकभाग एव कार्त्स्न्येन विक्षिप्तम्~। अत एव कर्णमण्डलात् तद्व्यासार्धमल्पम्~। तद्विक्षेपकोटितुल्यमेव स्यात् ; तन्मार्ग एव तदा शीघ्रोच्चनीचवृत्तमपि कल्प्यताम्~। तदा तच्च कार्त्स्न्येन मध्यकलाप्रमितविक्षेपान्तरितमेवापक्रममार्गादिति~। तदानीं स्फुटकर्मणि कल्प्यमानं छेद्यकं कृत्स्नमपक्रममण्डलस्यैकपार्श्वगतम्~। तेनापि न स्फुटस्य विशेषः~। तच्छीघ्रकर्णपरिमाणवशादेव केवलं भुजाफलभेदाद्भेदः ; न 
पुनर्विक्षेपवशात् भेदः ; यतोऽपक्रममण्डलवैपरीत्येनैव विक्षेपः~। तेनैकस्यामेव कलायां विक्षेपे सत्यसति च ग्रहो वर्तत इति 
विक्षेपवशात् स्फुटस्य भेदाभावः~। यतो भगोले तिलमात्रमपि न राशिरहितः प्रदेशः, द्वादशराशिभिर्व्याप्त एव नक्षत्रकक्ष्यावच्छिन्नो भगोलः कृत्स्नोऽपि, तस्माद्राशिविभागप्रदर्शनार्थमपक्रममण्डलवैपरीत्येनापक्रममण्डलगतद्वादशराशिसन्धिस्पृङ्मण्डलषट्कं परिकल्पनीयम्~। तदन्तरालगा 
मेषादयो द्वादशराशयः~। एवमयं द्वादशराशो भगोलः~। तस्माद्विक्षेपेण न स्फुटभेदः~। तदा मन्दकर्णविक्षेपकोट्यां शीघ्रकोटिफलं संस्कृत्यानीत एव शीघ्रकर्णः, तदा तेनैव व्यासार्धहतं शीघ्रभुजाफलमपि हार्यम्~। तद्धनुरेव\renewcommand{\thefootnote}{२}\footnote{वं \textendash\ ग} मन्दस्फुटे स्फुटग्रहसिद्ध्यर्थं संस्कार्यमिति कर्णभेदनिमित्त एव विक्षेपे सति स्फुटभेदः~। कथं पुनः प्रकृतं भगोलकलाप्रमितविक्षेपानयनं ; 
यदर्थमेतत्सर्वं प्रदर्शितम्~। उच्यते~। य इह पातोनमन्दस्फुटभुजां परमविक्षेपेण हत्वा त्रिज्याप्तं मन्दकर्णकलाप्रमितं विक्षेपमेवाविशिष्टमन्दकर्णेन हत्वा त्रिज्यया हृत्वा लब्धो मध्यकक्ष्याकलाप्रमितो विक्षेपः ; तमेव व्यासार्धेन हत्वा 

\newpage

\noindent विक्षेपकोट्यानीतशीघ्रकर्णवर्गे मध्यकलाप्रमितविक्षेपवर्गं संयोज्य पदीकृतेन भूताराग्रहविवरेण हरेत्~। तत्र लब्धः स्फुटविक्षेपः~। तत्र पूर्वं व्यासार्धं गुणकारः, स एव पुनर्भागहारश्च~। ततस्तयोस्तुल्यत्वान्नष्टयोः परमविक्षेपहतपातोनमन्दस्फुटभुजायाः त्रिज्याहतस्य 
मन्दकर्णकलाप्रमितस्य विक्षेपस्याविशिष्टो मन्दकर्णो गुणकारः~। विक्षेपकोटिद्वारानीतशीघ्रकर्णविक्षेपवर्गयोगमूलात्मकं भूताराग्रहविवरं भागहारः~। फलं ज्योतिश्चक्रकलाप्रमितस्फुटविक्षेप इति~। एवं कुजगुरुकोणानां विक्षेपानयनयुक्तिः प्रदर्शिता कुजगुरुकोणाश्चैवमित्यनेन~। अथ बुधशुक्रयोः भ्रमणप्रकारं तद्देशविशेषं च दर्शयति \textendash\ शीघ्रोच्चेनापि बुधशुक्राविति~। बुधशुक्रावपि स्वस्वपातात प्रभृत्यपमण्दलादुत्तरतो दक्षिणतश्च यातः, शीघ्रोच्चेन शीघ्रोच्चवशादपि मन्दस्फुटवशाच्च मन्दफलसंस्कृताच्छीघ्रोच्चात् पातं विशोध्य तयोर्विक्षेप आनीयत 
इत्यर्थः~। नन्वेवं बुधस्य द्वाविंशत्यैवाहोरात्रैर्वर्धमानस्य विक्षेपस्य महत्त्वनिवृत्तिः स्यात् ; पुनर्ह्रासवशात् तावद्भिरेव
दिनैः शून्यतापि स्यात्~। एवमपक्रममण्डलादेकपार्श्व एव गमनं चतुश्चत्वारिंशद्दिनान्येव ; पुनरितरपार्श्वे तावन्त्येव दिनानि 
गच्छति~। एवमष्टाशीत्यैव दिनैर्विक्षेपस्यैकः पर्यायः परिसमाप्तः स्यात्~; यतोऽष्टाशीत्यैव दिनैः शीघ्रभगणपरिपूर्तिः~। 
शीघ्रवशाच्च विक्षेप उक्तः~। कथमेतद्यु\renewcommand{\thefootnote}{१}\footnote{युज्यते स्वबिम्बस्य \textendash\ ग. घ. ङ}ज्यते~। ननु स्वबिम्बस्य विक्षेपः स्वभ्रमणवशादेव भवितुमर्हति~। न पुनरन्यभ्रमणवशादिति~। 
सत्यम्~। न पुनरन्यस्य भ्रमणवशादन्यस्य विक्षेप उपपद्यते~। तस्माद्बुधोऽष्टशीत्यैव दिनैः स्वभ्रमणवृत्तं पूरयति~। तस्मात् 
स्वभ्रमणवृत्तपरिमाणं योजनात्मकमाकाशकक्ष्यायाः सुगुशिथृनलब्धं न ख्युघृलब्धम्~। एतच्च नोपपद्यते~; यत एकेनैव संवत्सरेण 
तत्परिभ्रमणमुपलभ्यते, नैवाष्टाशीत्या दिनैः~। सत्यं~; भगोलपरिभ्रमणं तस्याप्येकेनैवाब्देन~। यस्मिन्वृत्ते 
स्वयमन्यैर्ग्रहैः समानयोजनं भ्रमति 

\newpage


\noindent तत्परिपूरण\renewcommand{\thefootnote}{१}\footnote{तत्पूरणं \textendash\ ग ङ.}मेकस्मिन्नब्दे चतुः करोति~। शुक्रोऽपि दिनानां पञ्चविंशत्यधिकशतद्वयेन स्वभ्रमणवृत्तं पूरयति~। तयोः 
पातश्च तत्कर्णमण्डले कक्ष्यामण्डले वा वर्तते~। एतदुक्तं भवति \textendash\ तयोर्भ्रमणवृत्तेन न भूः कबलीक्रियते~। ततो बहिरेव सदा भूः~। भगोलैकपार्श्व एव तद्वृत्तस्य परिसमाप्तत्वात् तद्भगणेन न द्वादशराशिषु चारः स्यात्~। तयोरपि वस्तुत आदित्यमध्यम एव 
शीघ्रोच्चम्~। शीघ्रोच्चभगणत्वेन पठिता\renewcommand{\thefootnote}{२}\footnote{ताः स्वस्वभगणाः \textendash\ ग. घ.} एव स्वभगणाः~। तथाप्यादित्यभ्रमणवशादेव द्वादशराशिषु चारः स्यात्~। शीघ्रवृत्तस्य स्वकक्ष्यायाः महत्त्वात्~। शीघ्रोच्चनीचवृत्तस्याप्येकभागगमेव स्वभ्रमणवृत्तम्~। यथा कुजादीनामपि शीघ्रोच्चं स्वमन्दकक्ष्यामण्डलादिकमाकर्षति, एवमेवैतयोरपि~। अनयोः पुनस्तदाकर्षणवशादेव द्वादशराशिषु चार इति यो विशेषः तेनोभे अपि वृत्ते व्यत्यस्य कल्प्येते\renewcommand{\thefootnote}{३}\footnote{'ते~। उच्चमध्यमे व्यत्यस्य कल्प्येते~। तयो'  च} ; तयोः शीघ्रोच्चनीचवृत्तस्य षष्टिशतत्रयांशेनैव परिमाय स्वकक्ष्यावृत्तमेव शीघ्रोच्चनीचवृत्तत्वेन कल्प्यते मन्दोच्चनीचवृत्तं च तदंशनैव परिमाय पठितम्~। तत एवं तयोर्न्याय्यं स्फ्रुटकर्म, स्वशीघ्रोच्चत्वेनाभिमतात् वस्तुतः स्वमध्यमात् स्वमन्दोच्चं विशोध्य भुजाकोटिज्ये गृहीत्वा वास्तवेनैवोच्चनीचपरिधिनोभे अपि हत्वा षष्ट्युत्तरशतत्रयेण हृत्वा लब्धयोः फलयोर्भुजाफलं चापीकृत्य तस्मिन्नेव मध्यमे संस्कुर्याद्
यस्मान्मन्दोच्चं विशोधितम्~। अस्य मान्दत्वान्मेषादावृणं तुलादौ धनं च कार्यम्~। पुनस्ताभ्यामेव कोटिभुजाभ्यां व्यासार्धेन 
चाविशेषकर्णमानीय क्वचित् संरक्ष्य रविमध्यमात् तन्मन्दस्फुटं विशोध्य दोःकोटिज्ये एवमानीय परिधिं स्फुटीकुर्यात्~। व्यसार्धं 
षष्टिशतत्रयेण हत्वा शीघ्रपरिधिना हृत्वाप्ते फले व्यासार्धेनाशीकृते एव~। शीघ्रन्यायेनागतां कोटिज्यां मन्दकर्णेन हत्वा त्रिज्याप्तां मृगकर्क्याद्योः प्राग्वत् संस्कृत्य पुनर्भुजाज्यां मन्दकर्णेन हत्वा त्रिज्यया हृत्वा तदुभयं वर्गीकृत्य सकृत् कर्णमानीय मानीय 

\newpage

\noindent मन्दकर्णसिद्धां भुजाज्यां व्यासार्धेन हत्वा शीघ्रकर्णेन हृत्वाप्तं चापीकृत्य रविमध्यम एव संस्कुर्यात्~। तदा बुधशुक्रौ स्फुटौ भवतः\renewcommand{\thefootnote}{१}\footnote{त एवएवमे \textendash\ ङ}~। एवमेकेनैव शीघ्रफलेन संस्कृतमादित्यमध्यमं तयोः स्फुटं स्यात्~। एवं मन्दस्फुटपातयोः साम्ये स्फुटीकरणम्~। भेदे तु मन्दस्फुटात् स्वपातं विशोध्य भुजाज्यां गृहीत्वा स्वकक्ष्याकलाप्रमितेन वास्तवेन परमविक्षेपेण हत्वा पुनर्मन्दकर्णेन च हत्वा त्रिज्याकृत्याप्तं\renewcommand{\thefootnote}{२}\footnote{ज्याप्तं \textendash\ ग. घ. ङ} विक्षेपं वर्गीकृत्य मन्दकर्णवर्गाद्विशोध्य मूलीकुर्यात् ; सा मन्दकर्णविक्षेपकोटिः~। पुनः शीघ्रोच्चात् मन्दस्फुटं विशोध्य भुजाकोटिज्ये नीत्वा भुजाज्यां नवभिर्हत्वा त्रिज्ययाप्तं फलम् अर्धोनचत्वारिंशदधिकशतादेकत्रिंशज्ज्यार्धाद्विशोध्य व्यासार्धं षष्ट्युत्तरशतत्रयेण हत्वानेन स्फुटपरिधिना हृत्वाप्तं फलं व्यासार्धत्वेनांशीकृत्य शीघ्रदोःकोटिज्ये विक्षेपकोटिघ्ने त्रिज्याहृते दोःकोटिफलत्वेनांशीकृत्य ताभ्यां तद्व्यासार्धेन च शीघ्रकर्णं सकृदानीय भुजाफलत्वेनांशीकृतं व्यासार्धेन हत्वा तेनैव कर्णेन हृत्वाप्तं चापीकृत्य\renewcommand{\thefootnote}{३}\footnote{कृत्य रविमध्यमे \textendash\ ग.} केवले रविमध्यमे संस्कुर्यात्~। 
तदा बुधशुक्रौ स्फुटौ भवत इति, विक्षेपे सति विशेषः~। एवमानीतशीघ्रकर्णवर्गे विक्षेपवर्गं युक्त्वा मूलीकृतं स्वभूताराग्रहविवरं तन्मध्यकक्ष्यायोजनैर्हत्वा व्यासार्धेन हरेत्~। लब्धं\renewcommand{\thefootnote}{४}\footnote{ब्ध \textendash\ घ.} स्फुटयोजनकर्णं त्रयाणामपि स्वभूविवरमध्यमयोजनकर्णेन हत्वा त्रिज्यया हरेत्~। स स्फुटकक्ष्यायोजनकर्णः~। एतद्युक्तिर्गोल एव प्रदर्श्या~। 
सूर्यस्य पूर्वप्रदर्शितछेद्यकेनैव स्फुटयुक्तिः सिध्यति, विक्षेपाभावात्~। विक्षेपवतां गोल एव प्रदर्श्या~। पातसाम्ये तेषामपि छेद्यकप्रदर्शितेनैव न्यायेन निरूप्या इति, विक्षेपे सत्येव गोले प्रदर्श्यते~। तत्र प्रथमं चन्द्रस्य प्रदर्श्यते \textendash\ अपक्रममण्डले चन्द्रोच्चसमप्रदेशे सूत्रस्यैकमग्रं 
बद्ध्वा ततश्चक्रार्धान्तरेऽन्यदग्रं बध्नीयात्~। तदुच्चनीचसूत्रम्~। तस्मिन्नुच्चभागेऽन्त्यफलतुल्ये लाञ्छयित्वा नीचसूत्रेऽपि तावत्यन्तरे लाञ्छनं कुर्यात्~। 
तदुभयस्पृष्टपरिधिकमपक्रममण्डलानु- 

\newpage

\noindent सार्येवान्त्यफलतुल्यव्यासार्धं मन्दोच्चनीचवृत्तं बध्नीयात्~। तत्रापि तद्द्वाशांशो राशिरित्यादि क्षेत्रविभागोऽपमण्डल इव 
कल्प्यः~। तस्मिंश्चन्द्रपाततुल्ये प्रदेशे चक्रार्धान्तरिते च तावत्प्रमाणमन्यन्मण्डलं तिर्यग्बध्नीयात्~। यथा ततो 
राशित्रयान्तरिते स्वांशैरर्धोनपञ्चमभाग\renewcommand{\thefootnote}{१}\footnote{र्धपञ्चमभाग}मितं तदन्तरालम्~। ततः 
पाताद्राशित्रयान्तरिते ततस्तिर्यगन्यन्मण्डलं तत्तुल्यं समदक्षिणोत्तरं बध्नीयात्~। तस्मिन् तिर्यग्बद्धे अपक्रममण्डलाद्विक्षिप्तोच्चनीचवृत्ते
चन्द्रोच्चतुल्ये प्रदेशे केन्द्रं कृत्वा त्रिज्यातुल्यव्यासार्धं प्रतिमण्डलं बध्नीयात्~। तदपि विक्षिप्तोच्चनीचवृत्तानुसार्येव बध्नीयात्, 
यथोभयोरेकमेव व्यासार्धं स्यात्~। कक्ष्यामण्डलमप्यपक्रममण्डले पातद्वये तत्तुल्यमेव तिर्यग्बध्नीयात्~। कक्ष्यामण्डलमपि 
विक्षिप्तोच्चनीचवृत्तगर्भं, तद्दिगनुसार्येव प्रतिमण्डलं च~। तत्रोच्चनीचसूत्रं प्रतिमण्डलान्तं कार्यम्~। तस्मिन् प्रतिमण्डलेऽपि समविभागा भागादयः कल्प्याः~। तेषु चन्द्रमध्यमप्रदेशे स्फटिकादिना चन्द्रबिम्बं च दर्शयितव्यम्~। तत्र कक्ष्यामण्डलात्प्रभृति चन्द्रबिम्बान्तं
स्फुटसूत्रं बध्नीयात्~। व्याससूत्राण्यपि त्रिषु पूर्वापरायतानि बध्नीयात्~। तदधोर्धं मेषादिकम्~; उत्तरार्धं तुलादिक\renewcommand{\thefootnote}{२}\footnote{गं \textendash\ ग. घ. ङ}म्~। पुनरपि कक्ष्यामण्डलाकेन्द्रात् 
तन्मध्य\renewcommand{\thefootnote}{३}\footnote{मध्य \textendash\ क. ग. ङ.}तुल्यतत्परिधिप्रदेशप्रापि तद्बद्धोच्चनीचवृत्तापरपरिध्यन्तं सूत्रं बध्नीयात्~। तद्बद्धोच्चनीचवृत्तमपि कक्ष्याप्रतिमण्डलसंश्लिष्टं कुर्यात्~।
अत्र विक्षेपयुक्तिः स्फुटयुक्तिश्च प्रदर्श्या~। तत्र यदोच्चमपि पातद्वयात् तुल्यान्तरालं तदा प्रतिमण्डलोच्चदेश एव तत्कर्णकलाभिः खत्रिघनतुल्याभिः विक्षिप्तः~। उच्चसूत्रस्पृक्कक्ष्यामण्डलपरिधिश्च ततोऽल्पीयसीभिरपि खत्रिघनसङ्ख्याभिरेव स्ववृत्तकलाभिर्विक्षिप्तः~। भगोलकेन्द्रात् प्रभृत्यामन्दकर्णाग्रं तदन्तरालं खत्रिघनसङ्ख्याभिरेव ज्योतिश्चक्रकलाभिः प्रमितम्~। तासां परिमाणस्यैव केन्द्रात् प्रभृति महत्त्वमा कर्णाग्रात्~। ततः प्रभृत्यल्पत्वं चा केन्द्रात्~। सर्वत्र साम्यमेव सङ्ख्यया तासामिति~। पातोनभुजाज्यां परमविक्षेपेण हत्वा त्रिज्ययैवाप्तं फलं विक्षेपः, 


\newpage

\noindent न पुनः कर्णेन हृत्वाप्तं फलमित्यवगन्तव्यम्~। एवं पातमन्दोच्चान्तरे भत्रये उच्चनीचरेखाग्रद्वयं परमविक्षिप्तम्~। 
यदा पुनः पातोच्चयोस्तुल्यत्वं तदोच्चनीचवृत्तस्य\renewcommand{\thefootnote}{३}\footnote{सूत्रस्य \textendash\ क.} न मनागपि विक्षेपः~। एवं पातोनोच्चभुजानुसारी उच्चनीचसूत्रविक्षेपः~। यदि कर्णेन लभ्यत्वं स्याद्विक्षेपस्य तर्हि प्रतिमण्डलेनैव विक्षिप्तेन भूयेत ; उच्चनीचवृत्तेनापक्रममण्डलानुसारिणैव भाव्यम्~। तत्परिधिस्थप्रतिमण्डलमात्रस्यैव विक्षेपे प्रतिमण्डलस्यापि मध्यकलातुल्यत्वान्मध्यकलाभिः प्रमित एव विक्षेपः स्यादिति तस्य ज्योतिश्चक्रकलाप्रमितत्वाय त्रैराशिकान्तरं कार्यम्~। तद्यथा \textendash\ यद्यविशिष्टकर्णकलाभिः मध्यकलाभिर्व्यासार्धतुल्या भगोलकला लभ्यन्ते तदा मध्यविक्षेपकलाभिः कियत्यो ज्योतिश्चक्रकला इति~। व्यासार्धं गुणकारः, मन्दकर्णो हारकः, पातोनभुजायाश्च परमविक्षेपो गुणकारः, व्यासार्धं भागहार इति~। व्यासार्धस्यैवैकत्र गुणकारत्वादन्यत्र भागहारत्वाच्च तेन गुणनं भागहरणं च न कार्यमिति~। पातोनभुजायाः परमविक्षेपो गुणकारः मन्दकर्णो भागहारः~। तर्हि मध्यमादेव पातो विशोध्यः, कक्ष्यामण्डलस्य विक्षेपाभावात्~। तत्र प्रतिमण्डलस्याविक्षिप्तस्यावधित्वेनाविक्षिप्तमपमण्डलानुसार्येवान्यत् प्रतिमण्डलं बन्धनीयम्~। तयोर्योगस्यैव तदा पातत्वम्~। तदा चन्द्रस्य प्रतिमण्डलस्थपातयोगे तन्मध्यममेव पाततुल्यं स्यादिति मध्यमे पाततुल्य एव तर्हि विक्षेपाभावः~। तत्र पातस्योच्चस्य च तुल्यत्वेऽपि ततोऽन्यप्रदेशगे चन्द्रे चन्द्रमध्यमस्फुटयोर्भेदात् पातोनमध्यमस्य पातोनस्फुटस्य च भेदः स्यात्~। यदा पातात् त्रिराश्यन्तरितमुच्चं तदा पातासन्ने चन्द्रे मध्यमस्फुटान्तरमन्त्यफलतुल्यम्~। तदा पातोनभुजाभेदश्च तावान्~। तदा तदन्तरस्य विक्षेपस्य कलाश्चतुर्विंशतिः~। एवं पक्षद्वयेऽपि विक्षेपस्य चतुर्विंशत्या लिप्ताभिरन्तरं स्यात्, तदा ग्रहणादिषु महान्भेदः~। तत्र पातचन्द्रस्फुटान्तरानीतविक्षेपस्य ग्रहणादिषु संवादान्न


\newpage

\noindent मध्यमात् पातविशोधनं कार्यमिति निर्णीयते~। अत एव विक्षेपे व्यासार्धमेव भागहारः ; न कर्णः~। तथा च सूर्यसिद्धान्तवचनं \textendash\ {\qt विक्षेपघ्नानत्यकर्णाप्ता विक्षेपस्त्रिज्यया विधोः} इति~। अस्मत्परमगुरुणापि सिद्धान्तदीपिकायामेतत्प्रतिपादितं\textendash 

\begin{quote}
{\qt दोःफलस्य यथा कर्णसाध्यत्वं नास्ति शीतगोः~।\\
 तथा क्षेपफलस्यापि नेन्दोः स्यात् कर्णसाध्यता~॥} 
\end{quote}

(इति~।) उक्तं च सूर्यसिद्धान्ते \textendash\ {\qt विक्षेपस्त्रिज्यया विधोः} इति, {\qt तस्मादयं पक्षो ग्राह्यो दृष्टिसमो ह्ययम्} इति~। अतो यदुक्तं भास्करेण {\qt कर्णेन ह्रियते लब्धो विक्षेपः सौम्यदक्षिणः} इति तन्न साध्वित्यभिप्रायः~। ताराग्रहाणां पुनः शीघ्रोच्चनीचवृत्तं न विक्षिप्तम्~। अपमण्डलानुसारिणि तत्परिधौ यत्रादित्यसूत्रं स्पृशति तत्रैव पञ्चानां स्वस्वोच्चनीचवृत्तपरिधौ मन्दोच्चनीचवृत्तकर्णमण्डलयोर्मध्यम्~। तद्द्वयं विक्षिप्तं चार्धशोऽपमण्डलमभितः ; तत्कक्ष्यामण्डलं च तथैव विक्षिप्तम्~। मन्दप्रतिमण्डलं पुनरर्धशो न विक्षिप्तं ; तत्केन्द्रस्य विक्षिप्तमन्दपरिधिगत्वात्~। तत्रापि यदा मन्दोच्चपातयोः साम्यं तदा प्रतिमण्डलमप्यर्धश एव विक्षिप्तम् ; उच्चनीचरेखाया अपक्रममण्डलगतत्वात्~। यदा पुनस्तद्द्वयं राशित्रयान्तरितं तदा प्रतिमण्डलस्योच्चभागस्य\renewcommand{\thefootnote}{१}\footnote{भागगस्य \textendash\ क. ग. घ.} मण्डलार्धाधिक्यं, नीचभागस्य विक्षिप्तस्य तदूनत्वं च~। तद्व्यासार्धमन्त्यफलाधिकमुच्चभागे विक्षिप्तं नीचभागे च तदूनमिति विशेषः~। एवं पातोच्चान्तरभुजाफलतुल्यम् आधिक्यं न्यूनत्वं च~। ततस्तद्द्विगुणतुल्यं तयोर्विक्षिप्तभागयोः परस्परमन्तरं यतो व्यासार्धत एव परस्परं न्यूनत्वमाधिक्यं च~। एवं तेषां मन्दकर्णमण्डलमेवार्धशो विक्षिप्तं ; न प्रतिमण्डलम्~। अत एव मन्दस्फुटात् पातविशोधनं शीघ्रकर्णेनैव हरणं च ; न पुनर्भूताराग्रहविवरेण तद्धरणं युक्तम्~। अत उक्तं {\qt अन्त्यकर्णाप्ते}ति~। पञ्चानां चतुर्थकर्णेनैव विक्षेपानयने हरणं कार्यमित्यर्थः~। सूर्यसिद्धान्तादिषु व्यासार्धे शीघ्रकोटिफलं संस्कृत्य तद्भुजावर्गयोगः पदीकृत एव शीघ्रकर्णत्वेनोक्तः ; स एव च विक्षेपानयने ग्राह्यः~। तत्र पञ्चानां स्फुटकर्माणि हि 

\newpage

\noindent चत्वारि सन्ति, मन्दकर्णमण्डलस्यैव विक्षिप्तत्वात्~। तस्य च शीघ्रकर्मणि प्रतिमण्डलत्वात्, तत्र स्फुटमध्यमात् 
पातविशोधनं शीघ्रकर्णेन हरणं च युज्यते~। अत्र पञ्चानां स्फुटयुक्तिरेकधैव~। बुधशुक्रयोः शीघ्रनीचोच्चवृत्तस्य स्वकक्ष्यातो महत्त्वात् 
मध्यमोच्चयोर्व्यत्यासः कक्ष्योच्चनीचवृत्तयोरपि, इत्येतावानेव भेदः ; न पुनश्छेद्यके वा गोलबन्धे वा तद्युक्तौ वा विशेषः~। परिधिपरमविक्षेपादिकं पुनः परीक्ष्यैव ज्ञातव्यम्~। न पुनरार्यभटशिष्याणां स भारः, युक्तिपरत्वादार्यभटीयस्य~। 
ग्रहाणां कक्ष्याक्रमः शीघ्रमन्दनीचोच्चवृत्ति\renewcommand{\thefootnote}{१}\footnote{ त्त \textendash\ घ. ङ}क्रमश्च 
तेषामाधिक्यन्यूनत्वे च इत्येतादृशमेव युक्तिनिरूपणदशायां ज्ञेयम्~। तेषां परिमाणं पुनः परीक्ष्यैव ज्ञेयम्~। तत्परीक्षाप्रकारश्च वक्ष्यते {\qt क्षितिरवियोगादि}त्यादिना~। परीक्षणप्रकारांस्तदपेक्षिता युक्तीश्च तत्रैव प्रदर्शयिष्यामः~। सर्वथापि तत्तत्कालभवो दृष्टिसंवादः सङ्ख्याविशेषाणां तदा तदा परीक्ष्यैव निर्णेतुं शक्य इति तत्परीक्षण एव शास्त्रकारैः अभिनिवेशः कर्तव्यः ; न परिमाणविशेषे~। स पुनः शास्त्रज्ञैः 
तत्तत्कालभवैर्निर्णेतुं शक्यः~। अत एव तन्त्रज्ञलक्षणे वराहमिहिरेणोक्तं \textendash\ {\qt सिद्धान्तभेदेऽप्ययननिवृत्तौ 
सममण्डललेखासंप्रयोगाभ्युदितांशकानां च छायायन्त्रदृग्गणितसाम्येन प्रतिपादनकुशलः, ग्रहाणां शीघ्रमन्दनीचोच्चयाम्योत्तरगतिकारणाभिज्ञः सूर्याचन्द्रमसोश्च ग्रहणे प्रग्रहविमोक्षदेशकालविद् अनागतानां च ग्रहयुद्धसमागमादीनाम् आदेष्टे}त्यादि~। {\qt तन्त्रज्ञो 
भवती}त्य(न्ते? ने)न वाक्यशेषेण सर्वेषां सम्बन्धः~। मानसव्याख्यानेऽपि स्फुटप्रदर्शनानन्तरमुक्तम्~। ननु 
पैतामहादिभेदेनेत्यादिप्रश्नस्योत्तरत्वेन पञ्चसिद्धान्तास्तावत् क्वचित्काले प्रमाणमेवेत्यवगन्तव्यम्~। अपि च यः सिद्धान्तो 
दर्शनाविसंवादी भवति सोऽन्वेषणीयः~। दर्शनसंवादश्च तदानीन्तनैः परीक्षकैः ग्रहणादौ विज्ञातव्य इत्यादि~। तस्मात् परीक्ष्य 
निर्णीतैरेव गण\renewcommand{\thefootnote}{२}\footnote{गुण \textendash\ ग. ङ.}यित्वा कालो वक्तव्य इत्यत्रैव ग्रन्थकरणे तात्पर्यम्~। परीक्षावस्थायां पुनः परिमाणस्थौल्यसौक्ष्म्यादितारतम्यपरत्त्वमिति परीक्षणापेक्षिता युक्तयश्च तन्त्रकारैर्वक्तव्याः~। 

\newpage

\noindent न पुनरस्यैतावत् परिमाणमिति वक्तव्यम्~। यदि तदप्युच्येत तर्हि शिष्याणां तेनैव कृतार्थता स्यात्~। तन्मा भूदिति तत्प्रदर्शनेऽपि नाना प्रदर्श्यते~। तेषां तत्र\renewcommand{\thefootnote}{१}\footnote{न्त्र \textendash\ ग. घ. ङ.}
संशयद्वारा जिज्ञासोत्थापनायैव तत्~। अतो युक्तय एव वक्तव्याः~। ताश्च {\qt कक्ष्यामण्डले}त्यादिभिः {\qt अपमण्डलस्ये}त्यद्यार्यया  
स्फुटकर्मापेक्षिताः कार्त्स्न्येन प्रदर्शिताः~। न्यायसाम्यादतिदेश्यत्वादेवोक्तानामिति भावः~॥~३~॥ \\

\indent चन्द्रादयो दिवाकरात् कियत्यन्तरे दृश्याः स्युर्भूच्छायांस्थैर्द्रष्टृभिरित्याह\textendash 
\begin{quote}
{\ab चन्द्रोंऽशैर्द्वादशभिरविक्षिप्तोऽर्कान्तरस्थितैर्दृश्यः~। \\
नवभिर्भृगुर्भृगोस्तैर्द्व्यधिकैर्द्व्यधिकैर्यथा श्लक्ष्णाः~॥~४~॥} 
\end{quote}

\indent इति~। अर्कान्तरस्थितैर्द्वादशभिरंशैश्चन्द्रो दृश्यः~। अंशाश्च घटिकामण्डलगाः~। अर्कार्धास्तमये घटिकामण्डलस्य योंऽशोऽस्तमेति ततः प्रभृत्यंशद्वादशकेऽस्तंगते यदि चन्द्रो नास्तंगतः तर्हि दृश्यः, प्रतिपदि वा द्वितीयायां वा~। एवमपरपक्षान्तेऽमावास्यायां वा चतुर्दश्यां वा घटिकामण्डले रव्युदय उद्यतः प्रदेशात् प्रत्यगंशद्वादशकं क्षितिजादधोगतं चेत् चन्द्रोऽपि तदानीं दृश्यार्धगतश्चेद्दृश्यः~। अन्यदा सूर्येण 
सहास्तंगतः~। प्रतिपत्पर्वणोः प्रायेण न दृश्यः~। अत एव नष्टचन्द्रः स काल उच्यते~। भृगुर्नवभिर्दृश्यः ; 
अर्कान्तरस्थितैरंशैरित्यनुवर्तते~। भृगोस्तैर्द्व्यधिकैर्द्व्यधिकैरन्येऽपि दृश्याः~। भृगुसम्बन्धिभिर्नवभिरंशैरेव द्व्यधिकैर्द्व्यधिकैर्यथा श्लक्ष्णा दृश्याः~। श्लक्ष्णक्रमश्च गीतिकापाद उक्तः~। {\qt भृगुगुरुबुधशनिभौमाः शशिङञणनमांशकाः} इति~। मण्डलाल्पत्वानुरूपं भागाधिक्यमपेक्षन्त इत्यर्थः~। तेन बुधशुक्रयोः क्रमचारे वक्रचारे च लिप्तामानभेदादुदयास्तमययोः स्वार्कान्तरभागभेदश्च सूचितः~। कथं, श्लक्ष्णत्वानुरूपं 
भागमहत्त्वे चन्द्रस्य भृगोः भा\renewcommand{\thefootnote}{२}\footnote{चन्द्रभृग्वोर्भा \textendash\ ग. घ. ङ.}गाधिक्यं, महत्त्वात्तन्मानस्य~॥~४~॥\\

\indent न केवलं बिम्बस्य लिप्तामानाधिक्यवशाद्भागानामल्पत्वं मानाल्पत्वानुरूपं भागानामाधिक्यं च, रश्मिवशादेव तन्नियमः~।
अधिकरश्मयोऽल्पैर्भागैर्दृश्याः,

\newpage

\noindent न्यूनरश्मयोऽधिकैरेव~। कथं पुनश्चन्द्रस्य तदा शुक्रान्न्यूनरश्मित्वमित्याह\textendash 
\begin{quote}
{\ab भूग्रहभानां गोलार्धानि स्वच्छायया विवर्णानि~। \\
 अर्धानि यथासारं सूर्याभिमुखानि दीप्यन्ते~॥~५~॥}
\end{quote}

\indent इति~। भुवः षण्णां ग्रहाणां च शेषाणां भानां च गोलार्धानि \textendash\ गोलाकाराणां बिम्बानामेकमेकमर्धं स्वच्छायया विवर्णम्~। यत्पुनरितरदर्धं सूर्याभिमुखं, तत् सदा दीप्यते च~। एवं सर्वेषामर्धानि सूर्यरश्मिभिर्दीप्यन्ते ; अर्धानि विवर्णानि~; 
स्वबिम्बनैव तिरोहितत्वात् ; तद्भागापेक्षया रवेः~। आदित्यरश्मिसम्पर्कादेवैषां ज्योतिष्ट्वं, न स्वतः दीप्यमानस्याप्यर्धस्यापि यथासारमेव दीप्तिः~। अतिधबले सिते संक्रान्तरश्मयोऽपि अतिधवलाः ; रक्ते संक्रान्ताश्च रक्ताः~। 
यथा स्फटिके प्रवाले च संक्रान्तानां तत्तत्सावर्ण्यं काठिन्यमार्दवानुरूपं च तत्प्रसरणं शोभा च~। चन्द्रस्य तावत् 
कठिनांशसहितत्वात् तत्प्रदेशस्य शोभाभावात् सकलङ्क इवासौ दृश्यते~। मन्त्रार्थवशाच्चैत सिद्धम्~। {\qt यददश्चन्द्रमसि 
कृष्णं तदपीह} इति~। एवं यदर्धमादित्याभिमुखं तदेव दूरस्थैर्दृश्यं, नेतरत्~। कक्ष्याक्रमश्च {\qt भानामध} इत्यादिना पूर्वमेव प्रदर्शितः~। तत्र 
सर्वेषां ज्योतिषामध एव चन्द्रमाः~। तस्मादमावास्यायां बहिरर्धमेवादित्याभिमुखम्~। अस्माकं चन्द्रकक्ष्यान्तर्गतत्वात्
सदैवान्तर्गतमेवार्धं दृश्यम्~। अतोऽस्माभिरादित्याभिमुखं तदर्धं दीप्यमानं न दृश्यमिति नष्टतया प्रतीयते~। पौर्णमास्यां तु 
सूर्याच्चक्रार्धान्तरितत्वाच्चन्द्रस्य भूम्यभिमुखमेवार्धम् आदित्याभिमुखम्~। अस्माभिरपि भूगतत्वात्तदर्धमेव दृश्यमिति तदर्धं
कार्त्स्न्येन दीप्यमानं दृश्यते~। तन्मण्डलं पुनर्गोलाकारमपि दर्पणाकारमिवाभिमन्यामहे; अतिदूरगतत्वादस्माकं द्रष्ट्ट्टणाम्~।
गोलाकारत्वमपि दृश्यमानार्धरश्मिप्रसरप्रकारान्यथानुपपक्त्या कल्प्यते~। दर्पणवृत्ताकारत्वे तस्यैकपार्श्वगतेऽर्के तत्तलं 
सकलं दृश्येत\renewcommand{\thefootnote}{१}\footnote{दृश्यते \textendash\ घ. ङ.}, मध्ये तिरोधायकाभावात्~। इतरपार्श्वगते तत्तलं तद्व्यासार्धसूत्रगते 

\newpage

\noindent नेम्यर्धमादित्याभिमुखं दृश्यम् ; इतरदर्धं न दृश्यम्~। तन्नेमिव्यासात् आदित्यव्यासस्य महत्त्वे\renewcommand{\thefootnote}{१}\footnote{महीयस्त्वे} तत्तलद्वयमपि दृश्यम् ; अल्पत्वे द्वयमपि न दृश्यम्~। नैवं चन्द्रबिम्बे रश्मिप्रसरः~। क्रमेण वर्धमानं सितं पक्षार्धेन दृश्यमानार्धस्य प्रत्यगर्धं पूर्णम्~। पक्षान्ते पुनः कृत्स्नमपि दीप्तं दृश्यते~। अपरपक्षे 
ह्रा\renewcommand{\thefootnote}{२}\footnote{ग्रा \textendash\ ख. ङ.}सश्च क्रमेणैव तस्माद्गोलाकारत्वं निर्णीयते~। तत् पुनरर्कोनचन्द्रस्य प्रथमे पदे उत्क्रमज्यानुसारेण वर्धते ; द्वितीये पदे तु क्रमज्यानुसारेण~। अपरपक्षेऽप्योजे पदे उत्क्रमज्यानुसारेणैव ह्रासः ; युग्मे च क्रमज्यानुसारेण~। गोलाकारत्वे चैवमेव युज्यते\renewcommand{\thefootnote}{३}\footnote{ते~। कलमा \textendash\ ख. ग. घ. ङ.}~। तस्मात् सितप्रतिपदन्ते तल्लिप्ताव्यासे सितमानं न कलामात्रमपि, कलापादत्रयमित\renewcommand{\thefootnote}{४}\footnote{कलात्र्यंशमित \textendash\ ख. ङ.} एव 
सितमानस्यापि\renewcommand{\thefootnote}{५}\footnote{नमघ्यस्यापि \textendash\ क. ग. घ.} विस्तारः~। अग्रयोः पुनस्तीक्ष्णविषाणत्वात् क्रमेणाल्पत्वमेव स्यादिति तदानीं शुक्रबिम्बान्न्यूनत्वात् द्वादशभिरेवार्कान्तरस्थिनैरंशर्दृश्यः~। किमर्यमविक्षिप्तग्रहणम्~। अर्धोदया\renewcommand{\thefootnote}{६}\footnote{अर्कोदया \textendash\ क. ख. ग. घ. ङ.}स्तमयाभ्यां गम्यगतकालाधिक्यवशादेव दृश्यत्वं, न पुनः पञ्चदशभिरप्यंशैरर्धास्तमय एव दृश्यत्वम्~। न केवलं देशविप्रकर्ष एव दृश्यत्वे कारणं, कालविप्रकर्षश्च कारणम्~। पुनरेभ्योऽपि विप्रकृष्टाश्चन्द्रशुक्रगुरवोऽस्तमयात् प्रागेव वा दृश्येरन्~। एवमुदयादूर्ध्वमपि~। न तथान्ये~। ते रोत्रावेव दृश्याः~। कुजबिम्बादपि सूक्ष्मास्ताराः पुनः सप्तदशभ्योऽप्यधिकै\renewcommand{\thefootnote}{७}\footnote{दशाधिकै \textendash\ ख. ङ.}रेवांशैर्दृश्याः~। यदा पुनरतिसूक्ष्ममपि नक्षत्रं दृश्यते, तावदन्ता हि सन्ध्या~। यस्मात् सन्ध्यालक्षणमेवमाह वराहमिहिरः\textendash\ 
 
\begin{quote}
{\qt अर्धास्तमयात् सन्ध्या व्यक्तीभूता न तारका यावत्~। \\ 
 तेजःपरिहाणिमुखाद्भानोरर्धोदयं यावत~॥} 
\end{quote}

\noindent इति~। तस्मात् कालवशादेव दृश्यत्वं, तत्सूचनार्थमविक्षिप्तग्रहणम्~। विक्षेपवशादप्युदयास्तमययोर्भेदस्य वक्ष्यमाणत्वात~। ननु सर्वदवाप्येकमर्धं सर्वेषां दीप्तम्~। तदनुपपन्नम्~। भूग्रहभानां कतमेनचित् कस्यचित् कदाचित्तिरोधानसम्भवात्~। तस्मात् पौर्णमास्यन्ते अविक्षिप्तस्य चन्द्रस्येकमर्धमपि न

\newpage

\noindent दीप्तम्~। यतस्तदानीं सूर्यरश्मयो भूम्या प्रतिबध्यन्ते चन्द्राभिमुखं गच्छन्तः~। तस्माद्विक्षिप्त एव तदानीं दृश्यः ; 
नैष दोषः~। तदानीमदृश्यत्वस्य वक्ष्यमाणत्वात्~। वक्ष्यति हि\textendash  
\begin{quote}
{\qt स्फुटशशिमासान्तेऽर्कं पातासन्नो यदा प्रविशतीन्दुः~। \\
 भूच्छायां पक्षान्ते तदाधिकोनं ग्रहणमध्यम्~॥}
\end{quote}
इति~। ननु शुक्लाष्टम्यर्धे चन्द्रमण्डलस्य न केवलं प्रत्यगर्धमेव दीप्तं, प्रागर्धेऽपि कियतश्चिद्भागस्य दीप्त्या भाव्यम् कुतः? यतस्तदानीं चन्द्रबिम्बपार्श्वादधोगत एवादित्यः, न समपार्श्वगतः~। तस्माच्चन्द्रबिम्बघनमध्यात् आदित्याभिमुखं प्रसरत् सूत्रं चन्द्रपार्श्ववलयादधोगतमेव~।
तत्सूत्रस्पृष्टप्रदेश एव हि चन्द्रस्य दीप्तार्धमध्यम्~। तच्च चन्द्रपार्श्ववलयादधो यावति प्रदेशे स्पृशति ततस्समन्तात् चन्द्रबिम्बपरिधिभागावधिकमर्धं दीप्तम्~। तद्दीप्तार्धं प्रागर्धेऽपि प्रसरति~। चन्द्रगोल\renewcommand{\thefootnote}{१}\footnote{लद \textendash\ क. ग. घ.}गदक्षिणोत्तरमण्डलात् प्राक् तावत्पर्यन्तं दीप्तमेव, पार्श्ववलयाद्यावति दीप्तार्धमध्यम्~। तस्मात् शुक्लाष्टम्यर्धे प्रागर्धेऽपि कियांश्चित् प्रदेशोऽस्माभिर्दृश्य इति~। सत्यमेवैतत् ; केन पुनरर्धमेव तदा दृश्यमित्युक्तम्~। नन्विदानीमेवोक्तं, प्रथमे पदे उत्क्रमज्यावशात् सितवृद्धिरिति~। भास्करादिभिरप्युक्तं सितमानानयनम्~। तत्राप्यष्टम्यर्धे लिप्ताव्यासार्धतुल्यमेव सितमानम्~। चन्द्रबिम्बपरिधिपादान्तमुत्क्रमवशादेव वर्धते ; पुनः क्रमज्यानुसारेण चेत्येवमस्माकमभिप्रायः~। आदित्यस्फुटमपि चन्द्रगोलगमेव विवक्षितम्~। कथं पुनस्तदानयनम्~। शीघ्रस्फुटन्यायेनेति ब्रूमः~। सर्वत्रापि द्रष्टृदेशवशाद्दर्शनभेदः शीघ्रस्फुटन्यायेनावगन्तव्यः~। अत एव मया सिद्धान्तदर्पणे सर्वत्र स्फुटन्यायातिदेशः प्रदर्शितः सामान्येन\textendash  
\begin{quote}
{\qt ज्ञातभोगग्रहं वृत्तं सर्वत्र प्रतिमण्डलम्~। \\
कक्ष्यावृत्तं च तत्तुल्यं ज्ञेयं भोगप्रदेशगम्~॥ \\
त एवाप्युच्चनीचाख्ये बहिश्चेदितरेतरम्~। \\
तद्व्यासार्धान्तरे चान्ये तन्मध्यान्तरनिर्मिते~॥} 
\end{quote}

\newpage

\begin{quote}
{\qt ग्रहोच्चयोविपर्यस्तौ भोगावप्यत्र कल्पयेद्~।}
\end{quote}

\noindent इत्यादिना~। स्फुटस्य भगोलगतत्वाद्भगोलमध्यगतस्य द्रष्टुस्तादृगेव ग्रहदर्शनम्~। भूपृष्ठगतस्य पुनरन्यादृशम्~। 
तत्रापि प्रतिदेशं भिद्यते~। एवम् अन्तरिक्षस्थानामपि प्रतिदेशं भिद्यते~। तत्र भगोलमध्यापेक्षया रविभोगो ज्ञातः~। तस्मादपक्रममार्गे रविस्फुटयोजनकर्णवृत्तम्~। तदिह ज्ञातभोगग्रहम्~। तत्र मेषादितः प्रभृति सूर्याक्रान्तावधिको राश्यादिको ज्ञात इति~। तत्र 
ग्रहभुक्तभागस्य ज्ञातत्वात् ज्ञातभोगग्रहं तद्वृत्तम्~। ज्ञेयभोगप्रदेश इह चन्द्रबिम्बघनमध्यमेव ; यतस्तदपेक्षया समन्ततो गतिरिह ज्ञेया~। तस्मात् तन्मध्यगकेन्द्रमिह कक्ष्यामण्डलम्~। तच्च स्फुटयोजनकर्णतुल्यम्~। तत्र चन्द्रस्य विक्षेपाभावमङ्गीकृत्यैव प्रथमं प्रदर्श्यते~। विक्षेपवशात् यो भेदः स पुनरेव प्रदर्श्यः, इतरथा दुर्विज्ञेयत्वात्~। तस्मादपक्रममण्डलानुसार्येव स्वनाभिस्थचन्द्रबिम्बं कक्ष्यामण्डलमपि~। उभयोः केन्द्रविवरमेवात्राप्युच्चनीचवृत्तव्यासार्धम्~। तच्च चन्द्रस्फुटयोजनकर्णतुल्यम्~। अपक्रममण्डलमध्यस्य चन्द्राक्रान्तापमण्डलप्रदेशात् चक्रार्धान्तरप्रापिसूत्रमध्यगतत्वात् चक्रार्धसहितं चन्द्रस्फुटमेव तत्रोच्चं कल्पनीयम्~। चन्द्रस्फुटं च नीचम्~। आदित्यस्फुटं मध्यमत्वेन च कल्प्यम्~। तत्र आदित्यस्फुटात् सषड्भं चन्द्रं विशोध्य भुजाकोटिज्ये गृहीत्वा ते उभे चन्द्रस्फुटयोजनकर्णेन हत्वा रविस्फुटयोजनकर्णेन हरेत्~। तत्र लब्धे तत्फले भवतः~। तयोः कोटिफलं मृगकर्क्यादिवशात् व्यासार्धे संस्कृत्य शीघ्रन्यायेनैव सकृत्कर्णमानीय तद्भुजा\renewcommand{\thefootnote}{१}\footnote{य भुजा \textendash\ ग. घ. ङ.}फलं त्रिज्यया हत्वा तेन कर्णेन हृत्वाप्तं फलं चापीकृत्य तत्केन्द्रे मेषादिगे रविस्फुटे धनं तुलादिगे ऋण\renewcommand{\thefootnote}{१}\footnote{लादावृणं  \textendash\ क. ग. घ.}मपि कुर्यात्~। तं रविं चन्द्रस्फुटात् विशोध्य ओजे पदे भुजाया उत्क्रमज्यामानीय चन्द्रलिप्ताव्यासार्धेन हत्वा त्रिज्ययैव हरेत्~। तत्र लब्धं सितमानम्~। युग्मे पदे तूत्क्रमज्ययैव 
तथानीतमसितमानम्~। अपरपक्षे त्वोजे पदे असितमानं, सितमानं च युग्मे~। यद्वा उभयत्रापि युग्मपदे कोटिचापस्य क्रमज्यां गृहीत्वा तां त्रिज्यया 

\newpage

\noindent योजयित्वा लिप्ताव्यासेन हत्वा ज्योतिश्चक्रव्यासेनैव हरेत्~। तत्र लब्धं सितमानम्। नन्वेवमानीतमपि सितमानं सदा न 
स्फुटत\renewcommand{\thefootnote}{१}\footnote{क \textendash\ ख. ग. घ. ङ.}रम्~। भगोलघनमघ्यग एव भूगोलमध्ये स्फुटतरं स्यात्~। न पुनर्वायुगोलापेक्षया भगोलस्योत्सर्पणेऽपसर्पणे च~। तदापि सूर्याचन्द्रमसोः स्फुटयोर्भगोलगतत्वात् द्रष्ट्राधारस्य भुवो भगोलमध्यतो विप्रकृष्टत्वाद्द्रष्टुः स्वदृग्गोलपरिध्यन्तर्भागगतस्य चन्द्रमण्डलार्धस्य दृश्यत्वाद्
द्रष्टृदृश्यादृश्यार्ध\renewcommand{\thefootnote}{२}\footnote{ष्टुर्दृश्यादृश्यार्ध \textendash\ क. ग. घ.}योर्विभागभेदात् तद्गतसितमानमपि भिद्यते~। अपक्रममण्डलान्तर्गते चन्द्रमण्डलार्धे यावत् सितमानं तदेव पूर्वमानीतम्~। इदानीं दृग्गोलान्तर्गतार्धसितमानमिति विशेषः~। सत्यम्~। किन्तु तेन सितमानेन नातीव भेदः ; चन्द्रच्छायादौ तु महान् भेदः स्यात्~। तद्भास्करादिभिरेव नोक्तम्~। उक्तं च श्रीपतिमुञ्जालादिभिः~। तच्चार्यभटाभिमतमेव, इन्दूच्चादिति सूचितत्वात्~। तदन्त्यफलं गगननृपतुल्यं श्रीपतिना प्रदर्शितम्~। तदपीन्दूच्चोनार्ककोट्यां त्रिज्यामितायामेव तावत् स्यात्~। तत्कोट्याः\renewcommand{\thefootnote}{३}\footnote{कोटि \textendash\ ग.} शून्यत्वेऽन्त्यफलमपि शून्यम्~। अतस्तां कोटिज्यां गगननृपैर्हत्वा त्रिज्यया विभक्तं तात्कालिकमुच्चनीचवृत्तव्यासार्धम्~। तत्रेन्दूच्चोनार्कस्य मकरादित्वेऽर्कस्फुटमेव तदुच्चम् ; कर्क्यादित्वे भूच्छायास्फुटम्~। शीघ्रस्फुटन्यायेनैवेतःपरं कर्म, तस्मादिन्दूच्चस्य रविभूच्छायाभ्यां
राशित्रयान्तरितत्वेऽस्यान्त्यफलस्य\renewcommand{\thefootnote}{४}\footnote{फल \textendash\ ग. घ. ङ.} शून्यत्वादिदं कर्मैव न कार्यम्~। अतोऽपक्रममण्डलस्य सूर्यापेक्षया पार्श्वस्थे चन्द्रतुङ्गे भगोलगतिर्भूगोलेऽपि सुषमैव~। चन्द्रतुङ्गस्य भूच्छायातुल्यत्वेऽर्कतुल्यत्वे च दुष्षमा~। सूर्यापेक्षयारोहणेऽवरोहणे च तदाद्यन्तसमय एव दुष्षमत्वम्~। तन्मध्ये तु सुषमत्वमेव इति~। अत्रापि {\qt मध्ये युगस्य सुषमादावन्ते दुष्षमेन्दूच्चात्} इत्येतदर्धं कृत्स्नमपि योजनीयम्~। तत्र युगशब्देन चन्द्रतुङ्गस्यार्कभूच्छाययोरन्यतरयोगोऽपि वक्ष्यते~। तस्माद्भूच्छायायोगात प्रभृति सूर्ययोगान्तमेकं युगम्~। ततो भूच्छायायोगान्तमितरद्युगम्~। तयोर्मध्य उभयपार्श्वस्थे तुङ्गे सुषमावस्था~। 

\newpage

\noindent तत्प्रभृति यदर्धं तदोत्सर्पिणी~। ततः पश्चादर्धं यत्तत्कालं\renewcommand{\thefootnote}{१}\footnote{पश्चाद्यदर्धं तत्कालं \textendash\ क. ग. घ.} कृत्स्नमपसर्पिणी~। सुषमाद्वयान्तरं वा युगमित्यत्र विशेषः~॥~५~॥ \\

एवं चन्द्रबिम्बस्य भूगोलाकारत्वमन्यथानुपपत्त्योपपाद्य भूबिम्बस्याप्याकारादिकं त्रिभिः पद्यैर्दर्शयति वृत्तेत्यादिमिः\textendash 
\begin{quote}
{\qt वृत्तभपञ्जरमध्ये कक्ष्यापरिवेष्टितः खमध्यगतः~। \\
 मृज्जलशिखिवायुमयो भूगोलः सर्वतोवृत्तः~॥~६~॥}
 \end{quote}
\begin{sloppypar}  
\indent इति~। गोलाकारस्य भपञ्जरस्य मध्ये अपक्रममण्डलमार्गे खमध्यगतो भूगोलः~। खमध्यगत इत्यनेनाधारनैरपेक्ष्यमुक्तम्~। कक्ष्यापरिवेष्टितः, चन्द्रादिशन्यन्तानां कक्ष्या\renewcommand{\thefootnote}{२}\footnote{क्ष्याभिः परि \textendash\ ङ.}भिः सप्तभिः परिवेष्टितः~। तासामपि भपञ्जरमध्यगतत्वात् तत्परिवेष्टनमुपपद्यते~। अत\renewcommand{\thefootnote}{३}\footnote{भूगोलस्य अत \textendash\ क. ख. ग.} एव भूच्छायाया अपक्रममण्डलगतत्वमप्युपपद्यते~। आदित्यस्य भूगोलस्य च विक्षेपाभावाद्भूच्छायाया अपि न विक्षेपः, भूम्यादित्यबिम्बप्रोतसूत्रतत्वाद्भूच्छायायाः~। मृज्जलशिखिवायुमयः, पाञ्चभौतिक इत्यर्थः~। खमध्यगतत्वोक्तेः अवकाशप्रदानात् सर्वस्यापि मूर्तस्य वस्तुनो विद्यत एवाकाश\renewcommand{\thefootnote}{४}\footnote{एवावकाश \textendash\ क. ख. ग. घ. ङ.}सम्बन्धः~। वराहमिहिरेणापि भूगोलस्य पाञ्चभौतिकत्वं निराधारत्वं च प्रतिपादितम्\textendash 
\end{sloppypar}

\begin{quote}
{\qt पञ्चमहाभूतमयस्तारागणपञ्जरे महीगोलः~। \\
खेऽयस्कान्तान्तःस्थो लोह इवावस्थितो वृत्तः~॥}
\end{quote}

\noindent इति~। सर्वतोवृत्तः, न पुनर्दर्पणवृत्ताकारः ; तिर्यगूर्ध्वमधश्च वृत्ताकार एव~। तिर्यक्परिणाहश्चोर्ध्वाधःपरिणाहश्च सर्वे समाना 
एव~। न च कूश्माण्डादिवद्दीर्घवृत्तः~। तत्तदवयवापेक्षया च परिणाहानां बहुत्वम्~। तद्यथा \textendash\ भूवृत्ततुल्यं वलयं द्रष्टुरूर्ध्वमधस्ताद्विन्यस्य ऊर्ध्वमधश्च कीलकद्वयेन तदवयवमवष्टभ्य भ्रामयेत्~। तथा भ्राम्यमाणं सर्वासु विदिक्षु चावान्तरदिक्षु च भूमिपरिणाहसाम्य एव भ्राम्यमाणं न प्रतिबध्यते~। तद्वलयस्य भूपृष्ठस्य च विवरं न स्यात्~। एवं तिरश्चीनमपि वलयं पार्श्वद्वयगतमेथ्योराहितं 

\newpage

\noindent भ्रा\renewcommand{\thefootnote}{१}\footnote{भ्रम \textendash\ ख. ग. घ. ङ.}मयित्वा सर्वतोवृत्तत्वं प्रदर्शनीयम्~। प्रदर्शितं च ज्योतिषां भूमेश्च~। श्रूयते हि सर्वेषां परिमण्डलत्वं, तच्चार्थवादसिद्धम्~। अन्यथानुपपत्त्यैव निर्णेयम्~॥~६~॥\\


\indent भूमेरपि समवृत्तत्वमन्यथानुपपत्त्या साधयति\textendash 
\begin{quote}

{\ab यद्वत्कदम्बपुष्पग्रन्थिः प्रचितः समन्ततः कुसुमैः~। \\
तद्वद्धि सर्वसत्त्वैर्जलजैः स्थलजैश्च भूगोलः~॥~७~॥} 
\end{quote}

\indent इति~। हिः हेतौ ; यतो भूगोलस्य सर्वसत्त्वैः समन्ततः प्रचितत्वं तत एव निराधारत्वं गोलाकारत्वं च सिद्धम्~। भूपृष्ठे न कश्चित्प्रदेशः प्राणिनिवासरहितो दृश्यते इति जलजैः स्थलजैरित्युक्तम्~। यादसामपि जलद्वारा भूगोल एवाधारः ; एवं विलवर्तिनामपि~। पतननिवारणार्थं ह्याधारोऽपेक्ष्यते~। ततः समन्ततो वस्तूनां भूमावेव पतनं, अत एव भुव आधारापेक्षा नैव स्यात्~। ब्रह्माण्डकटाहघनमध्यात् समन्ततस्तदवयवानां मण्डलार्धान्तरितानां पिपतिषया\renewcommand{\thefootnote}{२}\footnote{तानामपिपतिषया \textendash\ ङ.}न्योन्यसंघर्षात् पिण्डीक्रियमाणस्य भूगोलस्य दार्ढ्यमेव स्यात्~। अत एव समन्ततो गौरवसाम्यमपि सिद्धम्~। अत एव गोलाकारत्वं च~। तस्मादनुवाद एवान्यथानुपपत्तिसिद्धस्य गोलाकारत्वस्य श्रुतौ भूचन्द्रबिम्बयोः~। इतरेषां परिमण्डलत्वस्य श्रूयमाणस्य प्रमाणान्तरविरोधाभावात् भूतार्थवादत्वम्~। यत्पुनर्भूमण्डलस्य महापरिमाणत्वं सप्तद्वीपसमुद्रादिमत्त्वमाकारान्तराणि च, मेरोरपि महापरिमाणत्वमिति~। एतेषां स्मर्यमाणानां श्रूयमाणानां वा गुणवादत्वमेव~। अर्थवादात्मकत्वात् तेषां विधिविशेषत्वात् स्वार्थे तात्पर्याभावात्~। तेषामपि प्रमाणान्तरविरोधादेव गुणवादत्वं निर्णीयते, इत्येतदौपनिषदानामपि मतमेव~। यतस्तैरर्थवादस्य त्रैविध्यमङ्गीक्रियते\textendash  
\begin{quote}
{\qt विरोधे गुणवादः स्यादनुवादोऽवधारिते~। \\
भूतार्थवादस्तद्धानादर्थवादस्त्रिधा मतः~॥} 
\end{quote}
इति~॥~७~॥ \\

\indent ननु कथं शैलसरित्समुद्रादिमतो भूगोलस्य समवृत्तत्वमुच्यत इत्यत आह\textendash 

\newpage

\begin{quote}
\renewcommand{\thefootnote}{*}\footnote{ब्राह्मदिवसेन \textendash\ मुद्रितपुस्तकपाठः}{\ab ब्राह्मदिवसेन भूमेरुपरिष्टाद्योजनं भवति वृद्धिः~। \\
दिनतुल्ययैव रात्र्या मृदुपचितायास्तदिह हानिः~॥~८~॥} 
\end{quote}

\indent इति~। ब्राह्मदिवसेन चतुर्युगसहस्रात्मकेन कालेन भूमेरुपरिष्टाद्वृद्धिः स्यात्~; तच्च कियद्वर्धते ; योजनम्~। 
भूमेः परिमाणं ब्राह्मदिवसेनैकं योजनं कृत्स्नं वर्धते~। अत्यन्तसंयोगे द्वितीया~। कुतः, मृदुपचितायाः, 
मृद्भिरुपचितत्वात्~। तत्र तत्र मृद्भिरुपचीयते हि भूः~। क्वचित्प्रदेशे शिलादिप्राचुर्यात् कर्षकप्रवृत्त्यभावात्तत्र\renewcommand{\thefootnote}{१}\footnote{त्र तत्र प \textendash\ क.} पततां शीर्णपर्णादीनां मृगशरीराणां च निरसनाभावात्\renewcommand{\thefootnote}{२}\footnote{त् प्र \textendash\ ग. घ.} तत्प्रदेशस्योपरिष्टात् वृद्धिः स्यात्~। महता कालेन काठिन्यान्मृदंशोऽपि शिलात्वेन परिणमेत्~। ग्रामादिष्वपि पनसादिवृक्षाणां व्रीह्यादिधान्यानां चोपबृंहणाय गुल्मगोमयादिकं प्रक्षिपन्ति, एवं समन्ततो वर्धते हीदानीमपि भूः~। इतरप्रमाणवशान्निम्नतापि स्यात्~। अत एव निम्नोन्नताकारतया दृश्यते~। प्रकृत्या वृत्ताकारैव भूः~। तस्या विक्रियमाणत्वात् यो विशेषस्तेनापि न वृत्ताकारताया हानिः स्यात्~। भूपरिमाणापेक्षया निम्नोन्नततापरिमाणस्यात्यल्पत्वादिति भावः~। तर्ह्येवं क्रमेण वृद्धिसम्भवात् कालस्यानन्त्यात् आब्रह्माण्डकटाहं वर्धेतेति चेत् तत्राह \textendash\ दिनतुल्ययैव रात्र्येति~। दिनतुल्ययैव रात्र्या हानिश्च स्यात्~। तद्योजनम्~। हानेरपि तावदेव परिमाणं, वर्धनमपि नावयवाधिक्यात्~। कुतस्तर्हि अन्तर्गतावयवस्य वृक्षाद्यात्मना बहिरुद्गमनात् 
पुनस्तेषां तत्रैव मृदेकीभावेऽपि न तावन्निबिडत्वं स्यादवयवानामित्येतत् मृदुपचिताया इत्यनेन द्योतितम्~। तस्य पुनस्तावता कालेन ब्रह्मणो रात्रौ तस्मिन्नेवोपक्षीणत्वात् प्रकृतित्वेनावस्थानमपि स्याद्भुव इति कल्पान्तरेऽप्येतावत्येव भूरित्यर्थः~॥~८~॥ \\

\indent कुङिशिबुणॢष्खृ\renewcommand{\thefootnote}{+}\footnote{ख्षृ \textendash\ मुद्रितपाठः}प्रागिति या भूभ्रमणसङ्ख्योक्ता सा भग्नमणसङ्ख्यैव भानां प्रतीत्यनुसारेण भूभ्रमणावृत्तित्वेनोक्ता इति तत्प्रतीत्युपपत्त्या प्रदर्श्यते\textendash  

\newpage

\begin{quote}
{\ab अनुलोमगतिर्नौस्थः पश्यत्यचलं विलोमगं यद्वत्~। \\
 अचलानि भानि तद्वत् समपश्चिमगानि लङ्कायाम्~॥~९~॥} 
\end{quote}

\indent इति~। नन्वनेनापि भूभ्रमणमेवोच्यते~। यथा नद्यादिषु नावोह्ममानः पुरुषः स्वयं नाव्येव तिष्ठन्ननुलोमगतिः सन् तत्तीरगं पर्वतादिकमचलं विलोमगं स्वगतिवैपरीत्येन स्वाभिमुखमागच्छत् पश्यति, भूस्थोऽपि स्वयं भूतले क्वचिदेव तिष्ठन् भ्रमता भूबिम्बेन भ्राम्यमाणो भगोलस्थान्यचलानि भानि लङ्कायां सभपश्चिमगानीव पश्यति इति~। नैवं युज्यते उत्तरसूत्रविरोधात्~। कथं पुनरस्यार्थः, नौस्थ इव भान्यपि 
प्रवहवायुना भ्राम्यमाणानि लङ्कायामेव समपश्चिमगानि भूमिगतान्यचलानि वस्तूनि प्राङ्मुखं भ्रमन्तीव पश्यन्ति\renewcommand{\thefootnote}{१}\footnote{ति \textendash\ क. ग. घ.} इति~। कथं पुनरस्योत्तरसूत्रेण संवादः~। भपञ्जरः सग्रहः प्रवहेण वायुना क्षिप्तः लङ्कासमपश्चिमगो नित्यं सर्वदा समजवं भ्रमति, तत्तत्कक्ष्यासु नानाजवं यथा तत्र तत्र षष्ट्या घटिकामिरेव तदावृत्तिः पूर्येत तथैव सर्वेऽवयवा भ्रमन्तीति~॥~९~॥\\ 

\indent कुतः किमर्थं वा प्रवहस्य वायोरेवं भ्रमणमित्यत आह\textendash 

\begin{quote}
{\ab उदयास्तमयनिमित्तं नित्यं प्रवहेण वायुना क्षिप्तः~। \\
 लङ्कासमपश्चिमगो भपञ्जरः सग्रहो भ्रमति~॥~१०~॥} 
\end{quote}

\indent इति~। यथा वा सर्वेषा\renewcommand{\thefootnote}{२}\footnote{यथा सर्वेषां \textendash\ ग. घ. ङ.}मुदयास्तमयौ स्यातां दिननिशे च तथाकार्यमिति~। अयमभिप्रायः \textendash\ सर्वेषां दिनरात्र्यादिविभागार्थं प्राणिनां कर्मशक्त्या एवं भ्रमति~। तथाचोक्तम्\textendash  

\begin{quote}
{\qt नक्षत्रग्रहपञ्जरमहर्निशं लोककर्मविक्षिप्तम्~।\\
 भ्रमति शुभाशुभमखिलं प्रकाशयत् पूर्वकर्मकृतम्~॥} 
\end{quote}
 
\indent इति~। परमेश्वरप्रेरणयैव वायुर्वाति ; परमेश्वराद्भीतः सन्नादित्योऽप्युदेतीत्यादिकं श्रूयते च~। श्रीमद्भागवतेऽपि 
पञ्चमस्कन्धे\textendash\ \renewcommand{\thefootnote}{*}\footnote{वायुवशाः कर्मसारथयः परिवर्तन्ते~। एवं ज्योतिर्गणाः प्रकृतिपुरुषसंयोगानुगृहीताः कर्मनिर्मितगतयः इति मुद्रितभागवतपाठः~।}{\qt वायुसारथयः कर्मनिर्मितगतयो भुवि न पतन्ति} इति ग्रहाणां 

\newpage

\noindent कर्मनिर्मितगतित्वं प्रदर्शितम्~। तस्माज्ज्योतिश्चक्रस्यैव भ्रमणं, न भुवः~। तत्र लङ्कायां समपश्चिमगानीत्यनेन द्रष्ट्रपेक्षया भ्रमणप्रकारभेदः प्रदर्शितः~। जिष्णुनन्दनश्च भूमावपि प्रतिदेशं ग्रहनक्षत्रभ्रमणप्रकारभेदमाह\textendash 

\begin{quote}
{\qt ग्रहनक्षत्रभ्रमणं न समं सर्वत्र भवति भूम्थानाम्~।\\
तद्विज्ञानं गोलाद्यतस्ततो गोलमभिधास्ये~॥} 
\end{quote}
इति~। तद्भ्रमणप्रकारभेद एव हि दिननिशोः क्षयवृद्धिकारणम्~। तेन चरसंस्कारोऽप्यत्रैव सूचितः~॥~१०~॥\\ 

किञ्चान्यथानुपपत्त्यैव मेरोर्महापरिमाणत्वं भूमेर्दर्पणवृत्ताकारत्वं च निराकर्तव्यमित्याह\textendash 

\begin{quote}
{\ab मेरुर्योजनमात्रः प्रभाकरो हिमवता परिक्षिप्तः~। \\
 नन्दनवनस्य मध्ये रत्नमयः सर्वतोवृत्तः~॥~११~॥} 
\end{quote}

\indent इति~। मेरोश्चतुरशीतिसहस्रयोजनोच्छ्रायत्वे भूमेर्गोलाकारत्वेऽपि भारतवर्षात् प्रभृत्येव कानिचिज्ज्योतींष्यस्मदपेक्षया मेरुस्तिरोदध्यात् ; किमुत दर्पणवृत्ताकारत्वे~। अत्राप्यन्यथानुपपत्तिम् एतत् विवृण्वन्नेव भास्कर आह\textendash  
\begin{quote}
{\qt यत्र तोयनिधिमेखलातले नाम्नमेति मिथुनान्तसंस्थितः~। \\
तप्तहाटकनिभो दिवाकरस्तत्र भोऽक्षपरिमाणमुच्यताम्~॥} 
\end{quote}

\noindent इति~। तस्मात् कुलपर्वतानां महोत्सेधत्वादिकं गुणवाद एवेति न तत्राभिनिवेशः कार्योऽस्माभिः~। तथा च ब्रह्मसिद्धान्तेऽपि\textendash  
\begin{quote}
{\qt पञ्चाशत्कोटिविस्तीर्णा केवलं कल्पिता मही~। \\
स्वल्पराज्यमदान्धानां विषादाय विरक्तये~॥}
\end{quote}

\noindent इति~। अन्यत्रैव तस्य तात्पर्यम्~। किञ्च पौराणिकानां परस्परविरोधादप्यत्र तात्पर्याभावो निर्णेय इति श्रीपतिनापि प्रदर्शितम्\textendash 


\newpage

\begin{quote}
{\qt आदर्शोदरसोदरा भगवती विश्वम्भरा कीर्तिता \\
 कैश्चित् कैश्चन कूर्मपृष्ठसदृशी कैश्चित् सरोजाकृतिः~।\\
अस्माकं तु कदम्बवपुष्पनिचयग्रन्थे समा सम्मता\\
सर्वत्रासुमतां चयेन निचिता तोयस्थलस्थायिनाम्~॥}  
\end{quote}

\noindent इति~। भूविषयान्यथानुपपत्तिरपि तेनैव प्रदर्शिता\textendash  

\begin{quote}
{\qt चन्द्रादित्यग्रहणमुदयास्तौ युतिश्च ग्रहाणां \\
 शृङ्गोन्नामस्तुहिनमहसश्चित्रकर्म प्रभायाः~। \\
एतैरस्या उदितपरिधेः प्रत्ययैः पञ्चभिश्च \\
प्रत्याख्याता बहुपरिधितानन्तता चेयमुर्व्याः~॥} 
\end{quote}

इति~। भुव आधारान्तरकल्पनायामनवस्थाप्रसङ्गश्च स्यादिति च स एवाह\textendash  
\begin{center}
{\qt धर्ता धरित्र्या यदि हन्त मूर्तस्तस्यापरस्तस्य परस्ततोऽन्यः~।\\
एवं हि तेषामनवस्थितिः स्यात् ततो हि युक्ता भुव एव शक्तिः~॥}
\end{center}

इति~। अपि च आधारान्तरापेक्षायां विश्वम्भरात्वमपि हीयेत~। लोके हि विश्वम्भरैव विश्वम्भरा, नानन्तदिग्गजादिः~। न केवलं ध्रुवादधो 
नक्षत्राणां प्राचा भ्रमतां दृश्यमानत्वादेव\renewcommand{\thefootnote}{१}\footnote{अदृश्यमानत्वादेव \textendash\ घ.} मेरोर्महोच्छ्रितिर्निरस्या~। प्रत्यक्षप्रतियोग्यभावप्रमाणेन चेति प्रदर्शयितुं प्रभाकरत्वोक्तिः\renewcommand{\thefootnote}{२}\footnote{प्रभाकरोक्ति \textendash\ ख. ग- घ. ङ.}~। स्वर्णरत्नादिमयत्वादेव प्रभाणामाकर इति दूरादपि रश्मिप्रतिफलनादप्यतिशयेन भास्वरत्वात् दूरगतैरप्यस्माभिर्दृश्येतेति योग्यत्वानुपलब्ध्या च चतुरशीतिसाहस्रयोजनत्वं मेरोर्निरस्तम्~। तत्र हिमप्राचुर्यात् समन्ततः पर्वतस्य च संभवात् हिमवता परिक्षिप्त इत्युक्तम्~। तेन हेमकूटादीनां निरासः क्रियते~। यस्मात् भूपरिध्यष्टांशोऽस्माभिः तत्पृष्ठे परिभ्रमद्भिर्दृश्यते, ततोऽपि च तावत्येव च मेरुः~। तस्माद्धिमवन्मेर्वोरन्तराले हेमकूटनिषधयोस्तदन्तरालखण्डयोश्च न सम्भवः, अल्पविस्तारत्वात् हिमवन्मेर्वन्तरालस्य~। ततो नन्दनवनस्य प्राकारायमाणेन हिमवतैव परिक्षिप्तोऽयम्~। मेरुपृष्ठस्य स्वर्गत्वं 

\newpage

\noindent तत्परितो नन्दनवनं च नास्माभिः निरस्यते इत्येवमर्थमाह \textendash\ नन्दनवनस्य मध्य इति~। वृत्ताकारस्य स्मर्यमाणस्य प्रमाणान्तरविरोधाभावात्~। यद्यत् प्रमाणान्तरविरुद्धं तत्तदेवास्माभिर्गुणवादतया निरस्यते, न पुनरविरुद्धमपि~। भट्टकुमारेण\renewcommand{\thefootnote}{१}\footnote{भट्टकुमारेण च \textendash\ क. ख. घ. ड.} प्रभाकरेण 
चार्थवादानामन्यपरत्वात् यथाश्रुतार्थे प्रामाण्याभाव एवाङ्गीक्रियते, नैवमस्माभिः~। प्रमाणान्तरविरुद्धे गुणवादत्वं सर्वैरप्यङ्गीकार्यम्~। कक्ष्याक्रमश्च पुराणेषु नाना प्रदर्श्यते~। तस्मात् पुराणेषु प्रमाणान्तरविरुद्धस्य विधिनैकवाक्यत्वादत्र तात्पर्याभावात् पुराणस्य ज्योतिःशास्त्रस्य च न विरोधः~। 
विषयापहार एव हि विरोधः~। नाप्यस्मद्विषयो भूम्याकारपरिमाणादिस्तैरपह्रियते, अतद्विषयत्वात्तस्य~। तैरुच्यमान उपासनादिः अस्माभिरपि नापह्रियत इति~। अत एवोक्तमस्मदाचार्येण गोलदीपिकायाम्\textendash 

\begin{quote}
{\qt उपपत्त्या हीनोक्तिर्विदुषां सर्वाप्युपासनार्था स्यात्~। }
\end{quote} 
इति~॥~११~॥ \\

\indent अथ देशान्तरयुक्तिं प्रदर्शयति त्रिभिः पद्यैः स्वर्मेरू स्थलमध्य इत्यादिभिः\textendash  

\begin{quote}
{\ab स्वर्मेरू स्थलमध्ये नरको बडवामुखं च जलमध्ये~। \\
अमरमरा मन्यन्ते परस्परमध:स्थिता नियतम्~॥~१२~॥}
\end{quote}

\indent इति~। तत्र तावत् स्वर्नरकयोर्न देशान्तरकर्म कर्तव्यं, यतस्तत्र संवत्सरान्तमध्ययोरेवोदयास्तमयौ~। ततः सौराब्दगण एव 
तत्राहर्गणश्च, तयोः परस्परं मण्डलार्धान्तरितत्वात्\renewcommand{\thefootnote}{२}\footnote{परस्परमण्डलार्धान्तरितत्वात् \textendash\ ग.} ~। स्वःस्थाः सुरा नरकस्था	असुराश्च परस्परमध एव स्थिताः, घनभूमध्यादेवोर्ध्वदिशः प्रवृत्तेः~। यद्यपि तयोरूर्ध्वत्वं सममेव तथापि परस्परप्रतीत्यनुसारेण परस्परमधःस्थितत्वं स्वयमूर्ध्वस्थितत्वं च युज्यत एव~। वस्तुतस्तु समतिर्यग्गतत्वमेवोभयेषां वृत्ताकारत्वादेव तिर्यग्दिशश्च~। अत एव भूपृष्ठे परिभ्रमतां च वृत्ताकारत्वेऽपि पथः समतिरश्चीनतैव प्रतिभाति~॥~१२~॥ 

\newpage

\indent एवमेवान्येषामपि मण्डलार्धान्तरितानां मिथोऽधोगतत्वप्रतीतिः स्यात्~। तत्रोभयत्रापि दिननिशोर्व्यत्यासश्च स्यादित्याह\textendash  

\begin{quote}
{\ab उदयो यो लङ्कायां सोऽस्तमयः सवितुरेव सिद्धपुरे~। \\
 मध्याह्नो यवकोट्यां रोमकविषयेऽर्धरात्रः स्यात्~॥~१३~॥} 
\end{quote}

\indent इति~। यो लङ्कायामुदयकालः स एव सिद्धपुरे सवितुरस्तमयकालः, अत एव लङ्कायां मध्याह्नः सिद्धपुरेऽर्धरात्र 
इति च सिद्धम्~। एवमुभयेषां दिनस्य निशायाश्च कार्त्स्न्येन व्यत्यासः स्यात्, यथा देवानामसुराणां च~। यो लङ्कायामुदयः 
सिद्धपुरेऽस्तमयो वा स एव यवकोट्यां मध्याह्नः, रोमकविषयेऽर्धरात्रश्च स्यात्~। एवमेते भूपृष्ठगताः षट् प्रदेशाः
भूपरिधिपादान्तरिताः प्रदर्शिताः~। तत्र ये निरक्षदेशगताश्चत्वारः प्रदर्शिताः, तेषु यः 
स्वदेशसमीपग\renewcommand{\thefootnote}{१}\footnote{गस्त \textendash\ क. ख. ग. घ. ङ.}तस्तस्य स्वदेशस्य चान्तरालगतैर्योजनैस्तत्र तत्र देशान्तरं ज्ञेयम्~। तत्र लङ्कायामौदयिकः सिद्धान्तः, सिद्धपुरे आस्तमयिकः\renewcommand{\thefootnote}{२}\footnote{क इत्यादिकं \textendash\ क. घ.} सिद्धान्त इत्यादिकमप्यनेनैव सिद्धम्~॥~१३~॥\\ 

\indent अस्माभिः पुनः भारतवर्ष एव तत्र तत्र द्रष्ट्रनुरूपं देशान्तरं प्रदर्शनीयं, यतस्तत्रैव मनुष्याः सन्ति~। मनुष्याधिकारं च शास्त्रमिति लङ्कामेर्वन्तरालगतैव रेखा देशान्तरावधित्वेन प्रदर्श्यते\textendash 

\begin{quote}
{\ab स्थलजलमध्याल्लङ्का भूकक्ष्याया भवेच्चतुर्भागे~।}
\end{quote}

\indent इति~। अनेन स्थलमध्याज्जलमध्याच्च भूकक्ष्यायाश्चतुर्भागे हि लङ्का इति~।\\

\indent एतदुक्तन्यायेनैव सिद्धमिति, तदनूद्य लङ्काया अस्माभिः प्राप्तुमशक्यत्वात् तद्रेखागतो भारत एव वर्षे यः कश्चित्प्रदेश प्रदर्श्य इति, प्रसिद्धं तद्गतमुज्जयिन्याख्यं नगरमुत्तरार्धेन प्रदर्श्यते\textendash 

\begin{quote}
{\ab उज्जयिनी लङ्कायाः पञ्चदशांशे\renewcommand{\thefootnote}{३}\footnote{'तच्चतुरंशे' मुद्रितपाठः} समोत्तरतः~॥~१४~॥}
\end{quote}

\indent इति~। उज्जयिनी नाम नगरी लङ्कायाः समोत्तरत एव वर्तते~। सा च लङ्कातः कियद्दूर इत्याकाङ्क्षायामाह \textendash\ पञ्चदशांश इति~। भूकक्ष्याया 

\newpage 

\noindent इत्यत्राप्यनुवर्तनीयम्~। लङ्कामेर्वन्तरालरेरवयामेव लङ्कातः प्रभृति भूकक्ष्यापञ्चदशांशे सा नगरी~। ततस्तस्याः 
समरेखागतत्वात् तत्पूर्वापरदेशवर्तिनां ततः स्वदेशविप्रकर्षानुरूपं प्राक् पश्चाच्चोदयस्यापि विप्रकर्षात् स्वस्वोदयकालभवत्वायाहर्गणानीत मध्यमे स्फुटे वा तदन्तरालकालेन त्रैराशिकानीतं फलम्, ऋणं, धनं च कार्यमित्यभिप्रायः~। तच्चतुरंश इति केचित् पठन्ति~। तत्पक्षे 
तच्छब्देन भूकक्ष्याचतुर्भागः परामृश्यते~। चतुर्भागस्य च चतुर्भागः पुनः षोडशांश एव, इत्युभयोरर्थविशेषः स्यादेव~। न वस्तुनि विकल्पश्च युज्यते~। तस्मादेक एव साधीयान् पाठः, तत्र कतरस्य साधुत्वमित्येतदुज्जयिनीगतया विषुवच्छाययैव निर्णेयम्~। अवन्त्याख्यजनपदस्य लङ्कातो भूकक्ष्यापञ्चदशांशान्तरितत्वं शास्त्रान्तरेष्वपि प्रसिद्धम्~। यतः श्रीजैष्णव एवमाह\textendash  

\begin{quote}
{\qt लङ्कोत्तरतोऽवन्ती भूपरिधेः पञ्चदशभागे~॥} 
\end{quote}

\noindent इति~। आवन्तिको वराहमिहिरश्च तथैवाह\textendash 
 
\begin{quote}
{\qt मिथुनान्ते च कुवृत्तादंशचतुविंशतिं विहायोच्चैः~।\\
 भ्रमति हि रविरमराणां समोपरिष्टात् तदावन्त्याम्~॥} 
\end{quote}

\noindent इति~। इति तद्गताक्षस्य चतुर्विशाविभागत्वप्रदर्शनात्~। वृत्तपञ्चदशांश एव हि चतुर्विशतिभागात्मकः न षोडशांशः ; यतः 
कृत्स्नेऽपि मण्डले षष्ट्युत्तरशतत्रयसङ्ख्याभागः~। चतुर्विशतिभागश्च षष्ट्युत्तरशतत्रयस्य पञ्चदशांश एव~। तस्मात् पञ्चदशांश इत्ययमेव पाठः साधीयान्~। यद्वा जनपदस्य विस्तीर्णत्वादक्षस्य च प्रतिदेशं भेदादक्षचापस्य चतुर्विंशतिसङ्ख्यात्वमपि क्वचित् सम्भवत्येव~। तच्चोज्जयिन्यां वा 
नवेति तत्रस्थैरेव निर्णेतुं शक्यम्~। तत्र वराहमिहिरेण स्वग्रामापेक्षयैव चतुर्विंशतिभागत्वं प्रदर्शितम्~। तदनुसारेणैव जिष्णुनन्दनेनापि ततो दक्षिणत एव चोज्जयिनी~। तत्र च सार्धद्वाविंशतिरेवाक्षभागा इति च सम्भाव्यम्~। तर्हि स एव पाठः साधीयान् यतोऽत्रोज्जयिन्यामेवाक्षः प्रदर्श्यते~। समरेखापि सूक्ष्मैव इति तद्गतस्य जनपदमात्रस्यैव प्रदर्शने संशय एव स्यात्~। तत्रापि क्वगता

\newpage

\noindent समरेखेति~। तन्मा भूदिति तद्गता नगर्येव प्रदर्शिता~। तस्मादवन्तिसमयाम्योत्तररेखा\renewcommand{\thefootnote}{१}\footnote{याम्योदग्रेखा \textendash\ क. ग. घ.} 
पूर्वापराध्वनेत्यादावप्युज्जयिन्येव समरेखागतत्वेन विवक्षिता~। अत एवाहाजिताव्याख्यायां विजयाख्यायां पुराणप्रामाण्यप्रदर्शनपरवार्तिकव्यख्याने \textendash\ {\qt गणितं सर्वं यथासिद्धम् उज्जयिन्यामेव} इत्यादि~। तत्रापि यथासिद्धं लङ्कायामेवेति नोच्यते~। तच्चास्माभिः प्राप्य एव यः कश्चिद्देशविशेषो वक्तव्य इति~। भास्करश्रीपत्यादिभिः पुनर्बहूनि समरेखागतानि नगराणि प्रदर्शितानि~। तेषु समरेखानग\renewcommand{\thefootnote}{२}\footnote{खरनग \textendash\ ख. ग. घ.}रादिषु मध्ये यत् स्वदेशसमीपगं तद्देशाक्षं स्वदेशाक्षं च ज्ञात्वा तद्देशगं चन्द्रनिमीलनोन्मीलनकालं स्वदेशभवचन्द्रनिमीलनादिकालं च ज्ञात्वा देशान्तरकालोऽवगन्तव्यः~। तदवगमनप्रकारं च परीक्षासूत्रव्याख्यान एव प्रदर्शयिष्यामः~। देशान्तरस्य परीक्ष्यैवावगन्तव्यत्वात्, परीक्षायुक्त्यन्तर्भूतत्वात् तद्युक्तेश्च~॥~१४~॥ \\

\indent समरेखायामपि भूपृष्ठगतस्वाद्द्रष्टुः सूर्यादीनामुदयस्यास्तमयस्य च लम्बनवशाद्भेदात् तन्निमित्तोऽपि यः कश्चित् संस्कारः कार्य एवेति तत्प्रदर्शनायाह\textendash 

\begin{quote}
{\ab भूव्यासार्धेनोनं दृश्यं देशात् समाद्भगोलालार्धम्~। \\
 अर्धं भूमिच्छन्नं भूव्यासार्धाधिकं चैव~॥~१५~॥} 
\end{quote}

\indent इति~। समीकृतं भूपृष्ठप्रदेशमधिष्ठाय प्रेक्षमाणेनापि द्रष्ट्रा स्वोर्ध्वगं भगोलार्धं कार्त्स्न्येन न दृश्यम्~। तर्हि कियद्दृश्यम्~। भूव्यासार्धेनोनमेव दृश्यम्~। अधोर्ध्वं भूव्यासार्धेनाधिकं भूमिच्छन्नमेव~। अतस्तस्य तस्य द्रष्टुः भूपार्श्वप्रदेशादुपरि भूव्याभासार्धतुल्यं प्रदेशं गत्वैव 
तद्द्दृश्यत्वं प्राप्नोति~। भूव्यासार्धयोजनस्य\renewcommand{\thefootnote}{३}\footnote{सयोजनस्य \textendash\ ग. ङ.} सर्वकक्ष्यासु साम्यात्~। तद्भ्रमणकालो भिन्नपरिमाणः~। तदानयनाय दृग्गोलार्धोदयार्धास्तमयकालजो भगोलपार्श्वात् उन्नामात्मकः शङ्कुः प्रथममानेयः~। स च भूव्यासार्धात् त्रिज्याघ्नात् 

\newpage

\noindent स्फुटयोजनव्यासार्धाप्त एव~। ततो वक्ष्यमाणशङ्क्वानयनविपरीतकर्मणा तदसवश्च साध्याः~। तद्गतिरुदये योज्या~। अस्तमये शोध्या च~। तत्र सूर्यस्य चत्वार एवासवः स्युः~। ततः सूर्योदये गण्यमानानां ग्रहाणां प्राणचतुष्कसम्बन्धिनी स्फुटगतिर्मध्यमगतिर्वा उदये धनं
कार्यं, ऋणं  चास्तमये, तावद्भुक्त्वोदेति, तावद्भुक्त्वास्त\renewcommand{\thefootnote}{१}\footnote{तावदभुक्त्वास्तं \textendash\ ख. ग. ङ.}मेतीति~। व्यस्तं च वक्रगतौ~। यदि चन्द्रोदयकाले ग्रहा गण्यन्ते तदा चन्द्रस्य नीचगतत्वे कक्ष्याया अत्यल्पत्वात् स्वाहोरात्रासुपिण्डस्य महत्त्वाच्च दश विनाडिका अपि लभ्येरन्~। तत्कालसम्बन्धिनी स्वगतिर्लिप्ताद्वयादप्यधिका~। एवं चन्द्रोदयकालजस्य चन्द्रस्फुटस्यानयने लिप्ताद्वयमेवान्तरम् इत्यल्पत्वादेवोपेक्ष्यते~। तथापि सूक्ष्मदर्शिभिः कार्य एवायं संस्कार इत्यभिप्रायः~॥~१५~॥ \\

\indent दृश्यस्यापि भगोलार्धस्य देशवशाद्भेदमाह देवाः पश्यन्तीत्यार्याद्वयेन\textendash 
\begin{quote}
{\ab देवाः पश्यन्ति भगोलार्धमुदङ्मेरुसंस्थिताः सव्यम्~। \\
 अपसव्यगं तथार्धं दक्षिणबडवामुखे प्रेताः~॥~१६~॥} 
\end{quote}

\indent इति~। देवाः उदग्गतं भगोलार्धं सव्यं भ्रमत् पश्यन्ति ; तेषां मेरुस्थत्वाद्ध्रुवस्य\renewcommand{\thefootnote}{२}\footnote{ध्रुवरय समो \textendash\ ग. घ.} च समोपरिगतत्वाद्~। 
दक्षिणबडवामुखे प्रेताः, प्रेतास्तु दक्षिणमर्धं अपसव्यगं भ्रमच्च पश्यन्ति~। उदगर्धगतानि ज्योतींषि च देवाः सर्वदैव भ्रमन्ति पश्यन्ति~। न कदाचिदपि तत्र तेषामस्तमयः~। दक्षिणार्धगतानि कदाचिदपि न पश्यन्ति ; न कदाचिदपि तत्र तेषामुदयः~। प्रेताश्च दक्षिणार्धगतानि सदैव भ्रमन्ति पश्यन्ति~। नोदयार्धगतानि\renewcommand{\thefootnote}{३}\footnote{दग्गतानि गोलार्धानि~। त \textendash\ ख. ग. ङ.}~। तथापि तयोरपि दिनरात्रिविभागः स्य़ादेव~॥~१६~॥ \\

\indent यतोऽपक्रममण्डलस्य तिर्यक्त्वात् तत्र भ्रमन्नर्कः पर्यायेणोभयार्धं गच्छतीत्याह\textendash 
\begin{quote}
{\ab रविवर्षार्धं देवाः पश्यन्त्युदितं रविं तथा प्रेताः~। \\
शशिमासार्धं पितरः शशिगाः कुदिनार्धमिह मनुजाः~॥~१७~॥}  
\end{quote}

\newpage

\indent इति~। तत्रोभयत्रापि विषुवति रविबिम्बमध्यप्रदेश उभयैरपि न दृश्यः ; भूव्यासार्धोनस्यैवोर्ध्वार्धस्य दृश्यत्वात्~। एवं 
पितरश्च शशिमासार्धमुदितं रविं पश्यन्ति~। तेषां शक्षिमण्डलगतत्वात्~। एवमेवेहापि भूमिसावनदिनार्धं मनुजा उदितंं रविं पश्यन्ति~। इत्येकेनैवादित्येन जगति सर्वत्रापि दिनरात्रिविभागः~। तेषां सर्वेषां स्वाधारेण तिरोधानादेव भास्वतोऽप्यदृश्यत्वम्~। एवं तत्तदाधारे\renewcommand{\thefootnote}{१}\footnote{तदाधारे \textendash\ ख. ग. घ.} उदयास्तमयमध्यन्दिनार्धरात्रादयः कालविशेषाः सदा\renewcommand{\thefootnote}{२}\footnote{सदापि \textendash\ क. ग. घ. ग.} स्युस्तत्तत्प्रदेशापेक्षया~। तस्मात् सर्वत्र तत्तदाधारे सूर्याभिमुखस्यार्धस्येतरार्धस्य च सन्धौ सर्वदा सन्ध्यैव, प्रकाशमानेऽर्धे दिवा इतरदर्धे\renewcommand{\thefootnote}{३}\footnote{र्धे च \textendash\ क. घ.} नक्तं चेति~। तयोरर्धयोरपि रविभ्रमणवशात् भ्रमतोस्तत्सन्धिभागस्यापि तदनुरूपं भ्रमणवशात् प्रतिक्षणं प्रदेशान्तरेषूदयास्तमयौ करोतीति 
तद्वशाद्देशान्तरसंस्कारोऽपि प्रतिदेशं भिन्नः~॥~१७~॥ \\


\indent एवं देशान्तरसंस्कारहेतुं\renewcommand{\thefootnote}{४}\footnote{देशान्तरहेतुं \textendash\ घ.} प्रदर्श्य चरसंस्कारोपपत्तिप्रदर्शनाय गोलबन्धं प्रदर्शयति पञ्चभिरार्याभिः\textendash 
\begin{quote}
{\ab पूर्वापरमध ऊर्ध्वं मण्डलमथ दक्षिणोत्तरं चैव~। \\
 क्षितिजं समपार्श्वस्थं भानां यत्रोदयास्तमयौ~॥~१८~॥} 
\end{quote}

\indent इति~। तत्र पूर्वापरमध ऊर्ध्वं यन्मण्डलं तत्सममण्डलमित्याख्यायते~। दक्षिणोत्तरमण्डलस्य दक्षिणोत्तरमित्येवाख्या\renewcommand{\thefootnote}{५}\footnote{मेवाख्या \textendash\ ग. घ. ङ.}~। तस्याप्यध ऊर्ध्वत्वं विवक्षितम्~। तद्वशात् पूर्वापरकपालविभागः ; नतोन्नतविभागश्च~। क्षितिजं समपार्श्वास्थं क्षितेः समपार्श्वगतं मण्डलं तदपि प्रतिद्रष्टृ भिन्नम्~। भूपार्श्वस्यापि द्रष्टृवशाद्भेदात्~। यत्र भानामुदयास्तमयौ स्तः तत्रापि क्षितिजप्रागर्धे उदयः, प्रत्पनर्देऽरतभयश्च~। ननूक्तं भूव्यासार्धेनोनं दृश्यमिति, तत्कथं भानां पार्श्व एवोदयः~। क्षितिजादूर्ध्वं भूव्यासार्धान्तरित एव हि तेषामप्युदयास्तमयाभ्यां भाव्यम्~। नैष दोषः~। भूतले तिष्ठतः पुरुषस्य भकक्ष्यायां क्षितिजान्तं दृश्यत्वात्~। कथंं नक्षत्रकक्ष्यायामूर्ध्वार्धस्य 

\newpage

\noindent कृत्स्नस्यापि दृश्यत्वम्~। समीकृते भूतले तिष्ठतः पुरुषस्य चक्षुर्गोलकात् प्रभृति भूपरिधिस्पृष्टं प्रसार्यमाणं सूत्रं तत्पार्श्वगतं
व्यासार्धाग्रं स्पृशति तस्य द्रष्ट्रपेक्षया समतिरश्चीनत्वाभावात्~। क्व तर्हि तत्सूत्रं भूपरिधिं स्पृशति~। भूव्यासार्धात् पुरुषमात्राधिकेन व्यासार्धेन
भ्राम्यमाणे यद्वृत्तमुत्पद्यते तद्भूवृत्तादधिकं, द्रष्टृचक्षुर्गतपरिधिगत्वात्~। तद्गतार्धज्या हि तत्सूत्रभूपरिधिसंयोगात् द्रष्टृदृगन्तः प्रदेशः~। तच्छरश्च 
द्वित्राङ्गुलोनं पुरुषप्रमाणं दृष्ट्यूर्ध्वांशेनोनत्वात्~। तत्रापि वृत्ते {\qt शरसंवर्गोऽर्धज्यावर्ग} इत्यनेन तदर्धज्यानेया~। {\qt नृषि योजनम्} इत्युक्तत्वात्~। 
योजनानामष्टसहस्रगुणनेनांशीकृतानां पुरुषप्रमाणत्वात् शरगुणितानां तेषाम् ईषन्न्यूनत्वमेव स्यात्~। पुरुषप्रमाणादीषन्न्यूनत्वात् दृष्ट्यधोगतभागस्य~। तन्मूलतुल्या हि तदर्धज्या~। कर्णाकारे तत्सूत्रे तावतोंऽशस्य भुजापि द्विगुणपुरुषप्रमाणात् ईषन्न्यूनैव~। ज्या च पुनः भूगोले द्रष्ट्रधिष्ठितात् प्रदेशात् प्रभृति दृक्सूत्रस्पृक्परिध्यन्ता ज्या~। भूतलादपि तावदधोगता\renewcommand{\thefootnote}{१}\footnote{ता सा \textendash\ ग. ङ.} हि सा ज्या तस्मात् तच्छरद्वययोगतुल्यं हि दृग्गोलात् भूपरिधिस्पृष्टस्याधोगमनम्~। तस्मात् भूगर्ताधज्यात्मिकया कोट्या ईषन्न्यूनं द्विगुणितपुरूषप्रमाणं कर्णसूत्रस्य भुजात्मकमधोगमनं लभ्यते~। तदा भकक्ष्याव्यासार्धगतपुरुषप्रमाणैः कियदिति? लब्धं तदधोगमनं 
प्रायशो भूव्यासार्धतुल्यं~; न तु ततो न्यूनं ; ईषदधिकमेव स्यात्~। तस्मात् भानां तत्रैवोदयास्तमयौ~। एवं लग्नमपि भानुस्फुटतुल्यं भानोः क्षितिजप्राप्तिसमय एव, न पुनस्तदुदुये~। चन्द्रोदये पुनश्चन्द्रस्फुटाल्लग्नस्याधिक्यं महदेव अस्तमयेऽस्तलग्नस्य न्यूनत्वमपि तावत् स्यादिति भूम्यासक्तिवशात् दृश्यकालस्याल्पत्वम् अदृश्यकालस्य महत्त्वं च स्यादिति~। क्षितिजात्प्रभृति द्युगतासुवशादेव लग्नानयनमपि कार्यम्~। तस्मान्मेषादिराशीनां ताराणां च क्षितिज एवोदय इत्येतद्भानामित्यनेन सूचितम्~॥~१८~॥

\newpage

\begin{quote}
{\ab पूर्वापरदिग्लग्नं क्षितिजादक्षाग्रयोश्च लग्नं यत्~। \\
 उन्मण्डलं भवेत् तत् क्षयवृद्धी यत्र दिवसनिशोः~॥~१९~॥}  
\end{quote}

\indent इति~। पूर्वापरदिशोः क्षितिज एव लग्नम्~। दक्षिणोत्तरक्षितिजप्रदेशाभ्यामक्षचापान्तरिते च दक्षिणोत्तरमण्डले लग्नम्~। तत्रोत्तरक्षितिजादूर्ध्वं दक्षिणक्षितिजादधश्चेत्येतच्चाक्षाग्रयोरित्यनेनैव सिद्धं, गोलाक्षदण्डस्य उदगर्धस्य उन्नतत्वात्~। तर्हि 
क्षितिजादित्येतदनर्थकं, पूर्वापरदिग्लग्नमक्षाग्रयोश्च लग्नमित्येतावतैव सिद्धत्वादिति चेत्, न~। क्षितिजोन्मण्डलयोर्विवरस्य पूर्वापरदिग्भ्यां\renewcommand{\thefootnote}{१}\footnote{पूर्वापरदिश्यां} क्रमेण 
वर्धमानत्वात्, तदन्तराले क्षितिजोन्मण्डलविवरं त्रैराशिकेनानीय तत्र\renewcommand{\thefootnote}{२}\footnote{यते~। अत्र \textendash\ क. ग. घ. ङ.} भ्रमतां सर्वेषां ग्रहभानां क्षयवृद्ध्यर्धासवो ज्ञायन्त इति तत्सूचनार्थं क्षितिजग्रहणम्~। उन्मण्डलज्यायां लम्बकतुल्यायामक्षज्यातुल्यं तद्विवरम्~। 
तदेष्टापक्रमज्यया कियदिति~? अत्र प्रमाणफले तद्योगाद्राशित्रयान्तरिते प्रदेशे लम्बज्याक्षज्याभ्यां तुल्ये कोटिभुजे ; कर्णस्तु क्षितिजव्यासार्धमेव~। सैव तदा तत्राग्रज्या इतीह तात्पर्यम्~। क्षितिजादुद्गतत्वादेवोन्मण्डलसंज्ञापि युज्यते~। ननु एकमर्धमेव क्षितिजादुद्गतम्,~; अन्यदवनतं, 
तस्मात् अपमण्डलसंज्ञयापि भवितव्यम्~। नैतदस्ति, तदर्धस्यादृश्यत्वात्~। दृश्यार्धवशाद्धि व्यपदेशो युज्यते, इत्युन्मण्डलसंज्ञैव युज्यते यत्र दिवसनिशोः क्षयवृद्धी ज्ञायेते~। क्षितिजोन्मण्डलान्तरालभ्रमणकालेन हि दिवसनिशोः क्षयो 
वृद्धिश्च स्यातामिति~। तत्र दिनवृद्धौ क्षपायाः क्षयः, दिनस्य क्षये क्षपायाः वृद्धिः~। तयोर्योगस्य अहोरात्रस्य षष्टिघटिकाः~। 
किन्तु उत्तरायणे दिनचरगत्या न्यूनत्वं षष्टिघटिकाभ्योऽहोरात्रस्य स्यात्~। क्षितिजोत्तरभागस्य क्रमेणाधोगतत्वात्~। दक्षिणायने दिनचरभेदेनाधिक्यं च~। उन्मण्डलापेक्षया क्षितिजदक्षिणभागस्य क्रमेणोन्नतत्वात्~। एवं उदयद्वयान्तरालम्~। 

\newpage

\noindent व्यस्तमस्तद्वयान्तरालम्~। यतः सर्वेषां स्वाहोरात्राणां पश्चाद्भाग\renewcommand{\thefootnote}{१}\footnote{पार्श्वभाग \textendash\ क.ग. घ.}उन्मण्डलस्थः ततस्ताराणां भ्रमणमध ऊर्ध्वं च तुल्यकालम्~। ग्रहाणां पुनर्गतिभेदादपि दिननिशोर्भेदात् निरक्षदेशेऽपि दिननिशोर्भेदः स्यात्~। गतेः\renewcommand{\thefootnote}{२}\footnote{तौ \textendash\ ख. ग. घ. ङ.} प्रतिक्षणं भिद्यमानत्वात् निरन्तरदिननिशोरपि भेदात्~। किञ्च गतिक्षेत्रकालभेदादीप भिद्यते ; निरक्षदेशेऽपि राशीनां परस्परमतुल्यकालोदयत्वात्~। तस्मात् ताराणामेव निरक्षदेशेऽपि दिननिशोः साम्यं स्यादिति~॥~१९~॥ \\

\indent एषां मण्डलानां दिग्विभागसापेक्षत्वात् तिसृणां दिशां स्वरूपं दर्शयति\textendash 

\begin{quote}
{\ab पूर्वापरदिग्रेखाधश्चोर्ध्वा दक्षिणोत्तरस्था च~।\\
 एतासां सम्पातो द्रष्टा यस्मिन् भवेद्देशे~॥~२०~॥} 
\end{quote}

\indent इति~। यस्मिन् देशे द्रष्टा स्थित इति शेषः~। तत्र एतासां रेखाणां सम्पातो भवेत्~। का पुनः तास्तिस्रो रेखाः~। तासामेका पूर्वापरदिग्रेखा, अध ऊर्ध्वा च, दक्षिणोत्तरस्था च~। अधश्चोर्ध्वेति व्यस्योक्तत्वात् अधोरेखा च ऊर्ध्वरेखा चेति द्वे
इति कस्मान्न व्याख्यायते\renewcommand{\thefootnote}{३}\footnote{व्यायते \textendash\ ख. ग. घ.}~। एकत्वे कथं वा\renewcommand{\thefootnote}{४}\footnote{व \textendash\ ख. ङ.} तद्योजना~? पूर्वापरदिग्रेखा दक्षिणोत्तरस्था इत्यत्र यथान्योन्यं प्रतियोगिन्योर्दिशोर्ग्रहणेऽपि रेखाया विभक्त्यैव एकत्वं प्रतीयते, युक्तिश्च तथैव~। एकस्या एव रेखाया अग्रयोः परस्परापेक्षया प्रतियोगिदिग्गतत्वेन भाव्यमेवेति पौर्वापर्यमेकस्था एव युज्यते~। न केवलमग्रयोरेव पौर्वापर्यम्~। एकस्मादग्रादितराग्राभिमुखं गच्छतां पिपीलिकादीनां तद्दिगभिमुखानाम् अन्यस्मादग्रात् प्रभृति एतदग्रान्तं गच्छतामपि एकैव सा रेखा पन्थाः~। तस्मात् तन्निरन्तरावयवानां सर्वेषामपि परस्परापेक्षया पौर्वापर्यं स्यादेव~। तस्मादध ऊर्ध्वरेखाप्येकैव~। तर्हि अध ऊर्ध्वरेखेति समस्यैव 
प्रयोक्तव्यं, न व्यस्य~। नैष दोषः~। अधःशब्दस्याव्ययत्वात् पञ्चम्यन्त सः~। अधोदिशःप्रभृत्यूर्ध्वैका रेखेत्यर्थः~। एतदुक्तं


\newpage

\noindent भवति \textendash\ अधोदिश आदेर्घनभूमध्यात्प्रभृति द्रष्टृप्रापिणी नक्षत्रकक्ष्यान्ता या रेखा सा तृतीया~। द्रष्टृदिग्देश एवैतासां सम्पातः~। कुतः पुनः गोलबन्धस्य मध्ये रेखाप्रदर्शनस्य सङ्गतिः तदपेक्षितत्वे ततः प्रागेव वक्तव्यम्~। न, यदि प्रागेवोच्येत 
तर्हि {\qt क्षितिजं समपार्श्वस्थम्} इत्यत्रापि द्रष्टृसमपार्श्वस्थता क्षितिजस्यापि स्यात्~। इष्यते च क्षितिजस्य भूम्यपेक्षया समपार्श्वस्थितिः~। 
पूर्वापरदक्षिणोत्तरमण्डलयोरपि घनभूमध्यमेव मध्यम्~। न पुनर्द्रष्ट्रधिष्ठितभूपृष्ठम्~। पूर्वापरत्वं दक्षिणोत्तरत्वं च द्रष्ट्रपेक्षम्~। घनभूमध्ये 
पूर्वापरदक्षिणोत्तरादिविशेषाभावात्~। किञ्च तासु द्वे वृत्ताकारे एव~। सममण्डलानुसारि भूपृष्ठगतं वृत्तमेव पूर्वापररेखा, दक्षिणोत्तरमण्डलानुसारि दक्षिणोत्तरा~। तस्मादत्रैवास्य सूत्रस्य सङ्गतिः~॥~२०~॥ \\

\indent अथ अवसरप्राप्तं दृङ्मण्डलस्वरूपमाह\textendash 

\begin{quote}
{\ab ऊर्ध्वमधस्ताद्द्रष्टुः ज्ञेयं दृङ्मण्डलं ग्रहाभिमुखम्~। \\
 दृक्क्षेपमण्डलमपि प्राग्लग्नं स्यात् त्रिराश्यूनम्~॥~२१~॥} 
\end{quote}

\indent इति~। द्रष्टुरूर्ध्वमधस्तात् स्थितं ग्रहाभिमुखं दृङमण्डलं ज्ञेयम्~। एवं दृग्गोलगतमिदं दृङ्मण्डलं 
दृक्क्षेपमण्डलस्वरूपमपि ज्ञेयम्~। कथं~? त्रिराश्यूनं प्राग्लग्नं स्यात्~। त्रिराश्यूनं प्राग्लग्नमेव दृक्क्षेपमण्डलभुक्तं राश्यादिकमित्यर्थः~। उदयास्तलग्नयोर्मध्यं हि त्रिराश्यूनं प्राग्लग्नम्~। अस्तलग्नात् त्रिराश्यधिकं च~। उदयास्तलग्नयोश्चक्रार्धान्तरितत्वात्~। तत्प्रदेशप्रापि 
दृक्क्षेपमण्डलमपमण्डलविपरीतदिक्कं च~। इतरथा तन्मध्यमेव त्रिराश्यूनप्राग्लग्नतुल्यम्~। न दक्षिणोत्तरार्धे~। अपक्रममण्डलापेक्षया तिर्यक्त्वे तन्मण्डलं कृत्स्नमपि त्रिराश्यूनप्राग्लग्नावगाढं स्यात्~। अपक्रममण्डलवैपरीत्येनायता हि मेषादिराशयः तदंशाश्च~। अत एवोक्तं सूर्यसिद्धान्तेऽपि\textendash  

\begin{quote} 
{\qt क्षेत्राण्येवमजादीनां तिर्यक्त्वेन प्रकल्पयेत्}
\end{quote} 

\newpage

\noindent इति~। तस्मादपक्रममण्डलविपरीतं तद्दृश्यादृश्यमध्यप्रापि मण्डलं दृक्क्षेपमण्डलं, तन्मध्यमपि अपक्रममण्डलमध्यमेव~। एतदपि प्राग्लग्नं स्यात् त्रिराश्यूनमित्यनेनैव सिद्धम्~। तन्मण्डले लग्नं दृक्क्षेपलग्नमिति सदा\renewcommand{\thefootnote}{१}\footnote{सादृश्य \textendash\ ख. ग. घ. ङ.} दृश्यार्धमध्यमेव दृक्क्षेपलग्नम्~। मध्यलग्नं तु दक्षिणोत्तरमण्डललग्नत्वात् विषुवतोरूद्यतोरेव दृश्यार्धमध्यगतं स्यात्~। यत्र पुनः परमापक्रमचापोनमक्षचापं 
तत्र मध्यलग्नोदगपक्रमेऽक्ष\renewcommand{\thefootnote}{२}\footnote{क्रममक्ष \textendash\ ङ.}तुल्येऽपि दृश्यार्धमध्यं मध्यलग्नं स्यात्~। तस्माद्दृङ्मण्डलस्यैव द्रष्टृमध्यत्वं\renewcommand{\thefootnote}{३}\footnote{त्वं पुनः \textendash\ ख. घ. ङ.}, न 
पुनर्दृक्क्षेपमण्डलस्यापि~। अधऊर्ध्वत्वं पुनरर्थात् सिद्धम्~। अपक्रममण्डलविपरीतगतत्वाद्दृश्यार्धप्रापित्वाच्च~। एतस्मिन्नुभयस्मिन्नपि सति सममण्डलमध्यप्रापित्वमपि स्यादेव~। तेनोर्ध्वमधस्तादिति अत्र नानुवर्तनीयम् ; येन द्रष्टृमध्यत्वं प्रसज्येत~। कः पुनः दृक्क्षेपमण्डलस्यापि दृग्गोलगतत्वे दोषः~? दृक्क्षेपज्या तत्कोटिभ्यां हि द्रष्ट्रपेक्षयापक्रममण्डलस्य तिर्यक्त्वं परिच्छिद्यते~। तत्र ऊर्ध्वाधःसूत्रापेक्षया तिर्यक्त्त्वं दृक्क्षेपज्यायाः, क्षितिजापेक्षया तिर्यक्त्वं तत्कोटेः~। अतोऽपक्रममण्डलस्य घनभूमध्यनाभिकत्वात् तत्तिर्यक्त्वपरिच्छेदकस्यापि तन्नाभिकत्वमेव युज्यते~। दृग्गोले पुनर्दृक्क्षेपज्याया आधिक्यं तत्कोट्या न्यूनत्वं च स्यात्~। सर्वत्र त्रिज्यैव हि व्यासार्धम्~। तस्माद्दृग्गोलगतदृक्क्षेपज्याभुजकं व्यासार्धकर्णकं क्षेत्रमन्यादृशमेव भगोलगतदृक्क्षेपज्याभुजाकादिति दृग्गोलगत\renewcommand{\thefootnote}{४}\footnote{दृग्गत \textendash\ ख. ग. घ.}भुजाकोटिभ्यां तत्कर्णेन चापक्रममण्डलविषयं त्रैराशिकमपि न युज्यत एव~। इत्ययं महानेव दोषो दृग्गोलगतत्वे\renewcommand{\thefootnote}{५}\footnote{दृग्गत \textendash\ ख. ग. घ.} दृक्क्षेपमण्डलस्य~। {\qt मध्यज्योदयजीव} इत्यादिना दृक्क्षेपज्यानयनोपपत्तिरपि सेत्स्यति~। तस्माद्भगोलगतमेव दृक्क्षेपमण्डलम्~॥~२१~॥ \\

\indent कालज्ञानसाधनभूतेषु जलयन्त्रेषु बहुषु सत्स्वेकस्य गोलयन्त्रस्य भ्रमणं प्रदर्शयति सर्वोपलक्षणार्थम्~। सर्वेषां बीजयोग\renewcommand{\thefootnote}{६}\footnote{योग्य \textendash\ ख. ङ.}न्यायसाम्यात् तत्र गोलयन्त्रभ्रमणे छायादियुक्तिश्च प्रदर्शनीयेति तस्योपयोगबाहुल्यात् तदेव प्रदर्श्यते\textendash 

\newpage

\begin{quote}
{\ab काष्ठमयं समवृत्तं समन्ततः समगुरुं लघुं गोलम्~। \\
पारततैलजलैस्तं भ्रमयेत् स्वधिया च कालसमम्~॥~२२~॥} 
\end{quote}

\indent इति~। काष्ठमयं न लोहादिमयम्~। समवृत्तं, कौशलाभावाद्यदि समवृत्तता हीयेत तर्हि कालस्यान्यथात्वं स्यात्~। गौरवेणापि 
समन्ततः साम्येन भाव्यम्~। यदि क्वचित् गुरुत्वाधिक्यं स्यात् र्ह्यलाबुसूत्रनिरपेक्षं झटित्येव तत्प्रदेशो भ्रमेत्~। पुनरलाबुबद्धसूत्रेणाकृष्यमाणमपि नोर्ध्वं भ्रमेत्, गुरुभागस्याधोगतत्वात्~। लघुं, लाघवायैव हि काष्ठमयत्वमुक्तम्~। एवं बद्धं तं पारततैलजलैः स्वधिया भ्रमयेत्~। अस्य युक्तिः 
स्वयमपि ज्ञातुं शक्या, गणितयुक्तिवत् दुर्विज्ञेयत्वाभावात्~। कालसमं च भ्रमयेत्, कालसमत्वाय नलकस्योर्ध्वाधः परिणाहभेदः कर्तव्यः~। साम्ये तु जलाधिक्ये तद्गौरवादतिवेगेन जलस्रवणात् तदनुरूपमलाबुवेगस्याप्याधिक्यात् मध्याह्नात् प्रागेव गोलचतुर्भागभ्रमणं स्यात्~। पुनः पुनः क्रमेण मान्द्यं च~। तस्मात् तत्सर्वं स्वयमेव परीक्ष्य सर्वावयवं समकालं भ्रमयेत्~। तत्प्रकारश्च गुरूपदेशतः स्वयमप्यभ्यूह्य ज्ञेयः~।
 
\begin{quote}
{\qt एकाकी योजयेद्बीजं यन्त्रविस्मयकारकम्}
\end{quote}

\noindent इत्युक्तत्वात्\renewcommand{\thefootnote}{१}\footnote{इत्युक्त्वा \textendash\ ङ.}~। गोलस्याधोऽर्धं नलकालाब्धादिकं च वस्त्रादिना प्रच्छाद्यम्~। तत्र सममण्डलदक्षिणोत्तरक्षितिजादीनां भ्रामणं न कार्यं, मध्याह्नोदयाद्यवधिभूतत्वात् तेषामिति गोलद्वयं कार्यम्~। तयोः बहिष्ठः खगोल इत्याख्यायते ; अन्तःस्थो भगोलश्च~। तत्रेदानीं {\qt पूर्वापरम्} इत्यादिना प्रदर्शितः खगोल एव~। भगोलस्तु {\qt मेषादेः कन्यान्तम्} इत्यादिना पूर्वमेव प्रदर्शितः~। तत्र घटिकामण्डलस्यापरस्वस्तिके कीलकं निधाय तस्मिन् सूत्रस्यैकमग्रं बद्ध्वा अधो विषुवन्मण्डलपृष्ठेन प्राङ्मुखं नीत्वा तत उपर्याकृष्य प्रत्यङ्मुखं तेनैव नीत्वा तदग्रबद्धं पारतपूरितालाबुजलपूर्णे 
नलके निदध्यात्~। ततो 

\newpage

\noindent नलकस्याधःछिद्रं विवृतं, तेन जलं निस्रवति~। नलकस्थजलमधोधो गच्छति~। तद्वशाच्च तत्रस्थमलाबु पारतपूर्त्या गुरु\renewcommand{\thefootnote}{१}\footnote{पारतगुरु \textendash\ ख. ग. ङ.}त्वाज्जलं मुञ्चद्गोलं प्रत्यङ्मुखमाकर्षति~। एवं त्रिंशद्घटिकाभिरर्धसम्मितं यथा जलं भवति, गोलस्य चार्धं भ्रमति, तथा स्वबुद्ध्या जलनिस्रावो योज्यः~। एवमपराभिस्त्रिंशद्घटिकाभिः नलकस्थमवशिष्टं जलं यथा निश्शेषं स्रवति, अलाबु च नलकस्थले भवति, तद्वद्धसूत्रं चर्जुतया नलकान्तर्लम्बितं, गोलश्च समकालो भ्रमति, तथा च स्वधिया कालसमं गोलं भ्रमयेत्~। तत्राक्षबाहुल्यानुरूपं घटिकामण्डलस्य तिर्यक्त्वात् तत्पृष्ठगतस्य सूत्रस्य गुरुद्रव्यबद्धाग्रत्वात् तिर्यग्भूतं घटिकामण्डलं विहाय स्वयमृजू भवेत्, कालभेदं विदध्यात्~। तद्यथा घटिकामण्डलं न जह्यात्, तथा तत्पृष्ठस्य निम्नता कार्या~। तस्माद्घटिकामण्डलं नूपुराकारं कार्यमित्याद्यपि स्वधियेत्यनेनैव सूचितम्~॥~२२~॥ \\

\indent न केवलं कालज्ञानसाधनत्वमेवास्य गोलस्य, अपितु एतदेव यन्त्रं ग्रहनक्षत्रादि विवरज्ञानसाधनं, तत्प्रकारमाह\textendash 
\begin{quote}
{\ab दृग्गोलार्धकपाले ज्यार्धेन विकल्पयेद् भगोलार्धम्~। \\
 विषुवज्जीवाक्षभुजा तस्यास्त्ववलम्बकः कोटिः~॥~२३~॥} 
\end{quote}

\indent इति~। दृग्गोलार्धगतं यद् भगोलार्धं तज्ज्यार्धेन परिच्छिन्द्यात्~। एकस्मादितरस्य विप्रकर्षो ज्यार्धेनैव परिच्छेद्यः ; तदन्तरालक्षेत्रस्य चापाकारत्वेऽप्यृजुतया विप्रकर्षो ज्यार्धेनैव ज्ञेयः~। तच्च ज्यार्धं दृग्गोलार्धकपालगतमेव, द्रष्ट्रपेक्षया तद्विप्रकर्षस्य दृग्गोललिप्ताप्रमितत्वात्~। तासां भगोलकलाप्रमितत्वाय यत्नः कार्य इत्यर्थः~। एवं दृग्गोलार्धगतया विषुवद्दिनमध्याह्नछायया शीघ्रस्फुटन्यायेन साधिता या भगोलगता विषुवज्जीवा सा ह्यक्षभुजा गोलाक्षदण्डकर्णस्य भुजा~। तस्याः कोटिस्तु लम्बक 

\newpage

\noindent एव~। विषुवद्दिनमध्याह्नमहाशङ्कुतुल्य इत्यर्थः~। दृग्गोलभगोलगतयोर्यः पुनरितरेतरं परिणामः स उपरिष्टाद्वक्ष्यते\textendash 

\begin{quote} 
{\qt क्षितिजे स्वा दृक्छाया भूव्यासार्धं नभोमध्यात्} 
\end{quote}  

\indent इति~। अक्षावलम्बकप्रदर्शनाय\renewcommand{\thefootnote}{१}\footnote{कस्वरूपप्रदर्शनाय \textendash\ क.} सूत्रं बध्नीयात्~। तदर्थं पूर्वं घटिकोन्मण्डलयोः व्याससूत्रं बध्नीयात्~। तत्र घटिकामण्डलदक्षिणोत्तरमण्डलयोरुपरि स्वस्तिके सूत्रस्यैकमग्रं बद्ध्वाग्रान्तरमधः स्वस्तिके च बध्नीयात्~। एवमुन्मण्डलदक्षिणोत्तरयोः उत्तरस्वस्तिकात् प्रभृति दीक्षणस्वस्तिकान्तं सूत्रं बध्नीयात्~। एवं सममण्डलदक्षिणोत्तरमण्डलयोरुपरिस्वस्तिकात् प्रभृत्यधःस्वस्तिकान्तं सूत्रं बध्नीयात्~। एवं क्षितिजेऽपि याम्यस्वस्तिकात् प्रभृत्युदक्स्वस्तिकान्तं बध्नीयात्~। एवमेषां मण्डलानां चतुर्णामेकैकं व्याससूत्रं प्रत्यक्स्वस्तिकात्\renewcommand{\thefootnote}{२}\footnote{त्यक् सूत्रात् \textendash\ ग. घ. ङ.} प्रभृति प्राक्स्वस्तिकान्तम्~। चतुर्णामपि साधारणं व्याससूत्रम्~। तत्र घटिकामण्डलव्यासमप्युभयं खगोलावधिकं कुर्यात्~। पुनर्दक्षिणोत्तरसूत्रे उपरिघटिका व्यासस्पृष्टे सूत्रस्यैकमग्रं बद्ध्वा गोलान्तरुदङ्मुखं सूत्रं नीत्वा खमध्यात् तावत्यन्तरे तस्मिन्नेव बद्ध्वा, सूत्रशेषमधो नीत्वाधःस्वस्तिकादुत्तरतश्च तावत्यन्तरे बद्ध्वा, शेषं दक्षिणतो नीत्वा, अधःस्वस्तिकात् दक्षिणतस्तावत्यन्तरे बद्ध्वा, सूत्रशेषमूर्ध्वं नीत्वा, पूर्वबद्धाग्रेण सह बध्नीयात्~। तथा सति चत्वार्यायतचतुरश्रक्षेत्राणि अक्षभुजानि लम्बकोटिकानि\renewcommand{\thefootnote}{३}\footnote{म्बिककोटि \textendash\ क. ग. घ.} भवन्ति~। यद्वा उन्मण्डलदक्षिणोत्तरस्वस्तिके सूत्रस्यैकमग्रं बद्ध्वा क्षितिजात् तावत्यन्तरे दक्षिणोत्तरमण्डल एव बद्ध्वा उन्मण्डले अपरस्वस्तिकान्तं सूत्रं नीत्वायःशलाकाग्रे बद्ध्वा, क्षितिजात् तावत्यन्तरे दक्षिणोत्तरवृत्ते बद्ध्वा, पूर्वबद्धाग्रेण सह बध्नीयात्~। तथापि तादृशमायतचतुरश्रक्षेत्रचतुष्टयं स्यात्~। तत्र पूर्वप्रदर्शितानां दक्षिणोत्तरायताक्षज्यातुल्या भुजा, लम्बकतुल्या अधऊर्ध्वायता कोटिः~। उन्मण्डलस्पृष्टानां क्षितिजोन्मण्डलान्तरालज्या अधऊर्ध्वायताक्षज्यातुल्या भुजा लम्बकतुल्या दक्षिणोत्तरायता कोटिः~। तत्राक्ष- 

\newpage

\noindent शरोनं क्षितिजव्यासार्धमेव क्षेत्रद्वयस्य साधारणी कोटिः~। दक्षिणोत्तरमण्डलगतार्धज्यारूपे तयोरितरे कोट्यौ~। एवं 
सममण्डलव्यासार्धमक्षशरोनमेव पूर्वप्रदर्शितानां चतुर्णां साधारणी कोटिः~। दक्षिणोत्तरवृत्तगता अधऊर्ध्वायता जीवाश्चतुर्णामितराः कोटयः~। एवं स्वस्वविषयानुसारि वायुगोलतिर्यक्त्वमक्षावलम्बकाभ्यां मीयते~। भगोलतिर्यक्त्वं दृक्क्षेपतत्कोटिभ्याम्~। एवं प्रतिक्षेत्रं 
निरक्षस्वदेशविप्रकर्षानुरूपं भिन्नमिदं क्षेत्रं देशान्तराद्युपपत्तिसिद्ध्यर्थम्~। भूमावपि स्वस्वदेशभूवृत्तं भिन्नम्~। तत्प्रदर्शनार्थं भूमावपि वृत्तानि कल्पनीयानि~। स्वदेशसमदक्षिणोत्तरं मेरुबडवामुखप्रापि दक्षिणोत्तरं मण्डलं खगोलस्थयाम्योतरमण्डलाधोगतं निरक्षदेशेऽपि समपूर्वापरं 
भूवृत्ततुल्यं घटिकामण्डलाधोगतं कल्प्यम्~। पूर्वापरं पूर्वापररेखानुसारि पूर्वमेव प्रदर्शितम्~। निरक्षदेशवृत्तस्य स्वदेशपूर्वापरवृत्तस्य च यदन्तरालं दक्षिणोत्तरमण्डलगतं तदेव तयोः परमं विवरम्~। ततः परिधिपादान्तरे प्राक्पश्चाच्च तयोः सम्पातः~। स्वदेशसमदक्षिणोत्तरगतानां सर्वेषां तत्सम्पातौ समपूर्वापरदिशोरेव भूपार्श्वयोः तत्सम्पाते निरक्षवृत्तविपरीतं स्वदेशवृत्तविपरीतं च, तत्तत्समतिरश्चीनं मण्डल\renewcommand{\thefootnote}{१}\footnote{लं क \textendash\ ख. ग. घ. ङ.}द्वयं कल्प्यम्~। तत्र निरक्षसमतिरश्चीनमुन्मण्डलं, सममण्डलतिरश्चीनं क्षितिजम्~। तत्रोन्मण्डले निरक्षवृत्तादुत्तरतः प्राक्पश्चाच्च स्वदेशाक्षचापान्तरे चिह्नं कृत्वा तत्प्रदेशप्रापि स्वदेशप्रापि चान्येभ्योऽल्पं लम्बतुल्यव्यासार्धं वृत्तं पूर्वापरमालिखेत्~। तत् स्वदेशभूवृत्तम्~। तद्व्यासार्धं भूगोलकलाभिः लम्बतुल्यम्~। तत्र प्रतियोजनमुदयकालविप्रकर्षभेदस्तुल्य एव सर्वत्र इति स्वदेशभूवृत्तगतैः समरेखास्वदेशान्तरालयोजनैः त्रैराशिकेन 
देशान्तरसंस्कार आनीयते~। तद्यथा \textendash\ यदि स्वदेशभूवृत्तयोजनैः दिनगतिः लभ्यते, तदा देशान्तरयोजनैः कियतीति\renewcommand{\thefootnote}{२}\footnote{दिति \textendash\ क.}~। देशान्तरकालेऽप्येवं 

\newpage

\noindent त्रैराशिकम्~। यदि स्वदेशभूवृतयोजनैः देशान्तरनाड्यः षष्टिर्लभ्यन्ते, तदामीभिर्देशान्तरयोजनैः कियत्य इति~। तद्व्यस्तकर्मणा काले ज्ञाते योजनानयनमपि~॥~२३~॥ \\

\indent एवं स्वदेशेऽपि मध्यसावनोदये ग्रहमध्याद्यानयनं प्रदर्श्य तस्यैव स्फुटसावनौदयिकत्वापादनाय मध्यमस्फुटसावनोदयान्तरालं प्रदर्शयितुमार्याचतुष्टयमाह\textendash  

\begin{quote}
{\ab इष्टापक्रमवर्गं व्यासार्धकृतेर्विशोध्य यन्मूलम्~। \\
 विषुवदुदग्दक्षिणतस्तदहोरात्रार्धविष्कम्भः~॥~२४~॥} 
\end{quote}
 
\indent इति~। इष्टापक्रममानयनं पूर्वमेव प्रदर्शितम्~। क्व पुनरिष्टापक्रमानयनं प्रदर्शितम्~? तदेव हि गोलपादस्यादौ प्रदर्शितम्~। ननु तत्र न किञ्चित् गणितकर्म प्रदर्शितम्~। ग्रहगमनप्रकार\renewcommand{\thefootnote}{१}\footnote{रमेव\textendash\ ग. ङ.}कथनमेव केवलं कृतम्~। तत्कथनेनैव त्रैराशिकज्यार्धसूत्रन्यायसहकृतेनापक्रमानयनस्य सिद्धत्वात् तत्परत्वात् तस्य~। तत्रैवं त्रैराशिकं \textendash\ यदि विषुवतः प्रभृत्य\renewcommand{\thefootnote}{२}\footnote{र्धया त्रि \textendash\ ख. ग. घ. ङ.}र्धज्यया त्रिज्यातुल्यया घटिकामण्डलात् तज्ज्याग्रविप्रकर्षः परमापक्रमज्यातुल्यो लभ्यते, तदा विषुवत एवापक्रममण्डलाभीष्टप्रदेशाग्रज्यया तदग्रविप्रकर्षः कियानितीष्टापक्रमो लभ्यते~। कुतो जीवया त्रैराशिकेनानयनम्~। यथा \textendash\ मध्यमगतेः स्ववृत्तचापगताया अपि अन्योन्यं वा अहर्गणेन वा त्रैराशिकेनैव सिद्धिः, एवमत्रापि चापगतयैव गत्या दक्षिणोत्तरगतिरानेतुं न शक्यते~। उच्यते \textendash\ तत्प्रतिपादनायापक्रममण्डले विषुवतोश्च सर्वत्र जीवाः कल्प्याः~। 
तत्रापक्रममण्डलस्यायनद्वयान्तं यद्व्याससूत्रं तत्प्रतियोगितया जीवाः कल्प्यन्ते~। घटिकामण्डले च दक्षिणोत्तरविषुवत्सम्पातव्यासप्रतियोगिन्यः सर्वासामितरव्यासप्रतियोगिन्यश्च कोटयः~। पुनरयनस्पृग्विषुवन्मण्डले तस्मिन् मिथुनान्तयोगे सूत्रस्यैकमग्रं बद्ध्वा घटिकामण्डलाद्-

\newpage

\noindent दक्षिणतश्च तावत्यन्तरे बध्नीयात्~। एवं चापान्तात्प्रभृत्युत्तरतो नीत्वा घटिकामण्डलादुत्तरतश्च तावत्यन्तरे बध्नीयात्~। तयोरर्धं परमापक्रमज्या~। तयोरूदगग्रावगाहि याम्याग्रान्तरावगाहि च सूत्रद्वयं बध्नीयात्~। तदर्धं काष्ठान्त्यस्वाहोरात्रव्यासार्धम्~। तत्रायनावधिकमपक्रममण्डलव्यासार्धं कर्णः~। परमापक्रमशरोनं घटिकामण्डलव्यासार्धं च कोटिः, तद्व्यासार्धान्तरालानयनत्रैराशिकमेव प्रथमं 
निरूप्यताम्~। तद्गतज्यार्ध\renewcommand{\thefootnote}{१}\footnote{ज्याग्र \textendash\ क.}स्पृष्टपरिधिप्रदेशस्यापि
तावानेव विप्रकर्ष इति~। तत्रैवं त्रैराशिकम् \textendash\ यदि तद्व्यासार्धेन सकलेन कर्णभूतेन परमापक्रमज्यातुल्या भुजा लभ्यते, तदैकराशिज्यातुल्येन तदर्धेन कियतीत्यादि~। मेषान्ताद्यपक्रम एव तद्व्यासार्धार्धस्य\renewcommand{\thefootnote}{२}\footnote{सार्धस्य \textendash\ ङ.} भुजात्वेन लभ्यते~। एवं सर्वत्र जीवाभिः त्रैराशिके ययोर्वृत्तयोरन्तरालविषयं त्रैराशिकं तद्व्यासार्धयोरेव कर्णकोटिकल्पनया इच्छात्मकस्यापि तद्व्याससूत्रैकदेशस्य इतरव्याससूत्रात् विप्रकर्ष इच्छाफलत्वेनानेयः~। स एव तदिच्छाज्याग्रपरिधिसंयोगस्यापि तत्कोटिवृत्तात् प्रकर्षः~। एवं सर्वत्र द्वयोर्द्वयोः वृत्तक्षेत्रयोः तुल्यपरिमाणयोः सदेशनाभिकयोः
साधारणव्यासाग्रस्पृष्टपरिधिप्रदेशात् कर्णत्वेन कल्पितवृत्तस्याभीष्टपरिधिप्रदेशान्तज्ययेच्छाभूतया तद्वृत्तव्यासार्धेन कर्णभूतेन च प्रमाणेन
तदग्रस्येतरव्यासविप्रकर्षेण फलेन चेष्टज्याग्रस्य तत्स्पृष्टपरिधिप्रदेशस्य वा इतरवृत्तस्य तत्परिध्यन्तरालप्रदेशेषु योऽवयवः समीपगः तस्मात् 
स्वविप्रकर्षः स्यात्~। तत्र भुजाकोटिकर्णेषु 
तत्परमान्तरालव्यासार्धसंश्रितया\renewcommand{\thefootnote}{३}\footnote{सम्बन्धितया \textendash\ क. ग. ङ.} कल्प्यमानेषु कल्पनालाघवात् त्रैराशिकयुक्तेः सुगमत्वं स्यात्~। अत एव विष्कम्भार्धे यथेष्टानीति विष्कम्भार्धे ज्यार्धाना कल्पनं प्रदर्शितम्~। एवमपक्रममण्डलस्येष्टप्रदेशस्य वायुगोलमध्यप्रदेशात् घटिकामण्डलमार्गात् विप्रकर्षे ज्ञाते सति भ्रमता वायुगोलेन भ्राम्यमाणस्य तदवयवस्य 
यद्भ्रमणवृत्तं तत् तस्याहोरात्रवृत्तमित्युच्यते~। तद्घटिकामण्डलादल्पपरिमाणं वायुगोलैकपार्श्व एव परिसमाप्तत्वात्~। तद्व्यासार्धानयनमिह प्रदर्श्यते~। इष्टापक्रमवर्गं ग्रहाणां ताराणां वा योऽपक्रमस्तात्कालिको वायुगोलमध्याद्विप्रकर्षः स इष्टापक्रमः ; तस्य वर्गं 

\newpage

\noindent व्यासार्धस्य कृतेर्विशोध्य मूलीकृत्य लब्धं यन्मूलं तदेव तस्य ग्रहस्य नक्षत्रस्य वा अहोरात्रार्धविष्कम्भः~। अहोरात्रशब्देन
अहोरात्रवृत्तमुच्यते~। तस्यार्धविष्कम्भः तद्व्यासार्धं विषुवदुदग्दक्षिणतो गतः~। विषुवत उदग्गतानां ज्योतिषां स्वाहोरात्रवृत्तं कार्त्स्न्येनोदगेव गतम्~।
दक्षिणगोलगतानां च दक्षिणत एव तद्गोलपरिणाहादल्पत्वादहोरात्रवृत्तस्य~। एकगोलगतयोरपि द्वयोर्वृत्तयोः तद्गोलपरिणाहाभ्यां तुल्यव्यासपरिणाहयोरेव 
मण्डलार्धान्तरितौ सम्पातौ स्तः~। एवं ध्रुवद्वयीमध्यगतानां ज्योतिषां स्वाहोरात्राणि ध्रुवासक्तिवशादल्पप्रमाणानि घटिकामण्डलासक्तिक्रमेण 
महाप्रमाणानि च~। कुतः पुनरिह कोट्यानयनं प्रदर्शितं, {\qt यश्चैव भुजावर्ग} इत्यनेनैव सिद्धत्वात्~। न चैतत्सूत्रं गणितप्रदर्शनपरं,
तत्क्षेत्रप्रदर्शनपरमेव~। अत्रापक्रमभुजकं व्यासार्धकर्णकमपि यत्किञ्चित्क्षेत्रं कल्पनीयम्~। न पुनरपक्रमभुजकमिष्टज्याकर्णकमेव ; यद्गतेन त्रैराशिकेनेष्टापक्रम आनीतः~। क्व पुनस्तर्हीष्टापक्रमभुजायाः कर्णः तत्कोटिर्वा कल्प्या इति तत्प्रदेशतत्कोटितुल्यव्यासार्धवृत्तप्रदर्शनपरमेवेदम्~। तस्मात् इष्टापक्रमवर्गं व्यासार्धकृतेर्विशोध्य यन्मूलमित्यस्यायमेवार्थः~।  इष्टाषक्रमभुजकस्य\renewcommand{\thefootnote}{१}\footnote{जस्य \textendash\ ग. ङ.} गोलव्यासार्धकर्णकस्य या कोटिः सा तदहोरात्रार्धविष्कम्भः\renewcommand{\thefootnote}{२}\footnote{त्रविष्कम्भः \textendash\ ग. ङ.}~। यदिष्टत्वेन विवक्षितं ग्रहो वाप्युडु वा तस्याहोरात्रवृत्तस्य यद्व्यासार्धं तत् तस्य कोटिः~। यदि तज्ज्योतिर्विषुवत\renewcommand{\thefootnote}{३}\footnote{त उद \textendash\ क. ग. घ. ङ.} एव उदग्गतं तर्हि तद्वृत्तमप्युदग्गतमेव~। तद्विषुवतो यावद्विप्रकृष्टं तावद्विप्रकृष्टं च तदिति तत्कोटिमण्डलस्वरूपमेवेह प्रदर्शितम्~। कर्णश्च कोट्यग्रगः~। तस्माद्
गोलमध्यादहोरात्रवृत्तावधि यद्व्यासार्धं स कर्णः~। कर्णकोट्यन्तरालतुल्या हि भुजा~। तस्माद्घटिकामण्डलव्याससूत्रस्य स्वाहोरात्रव्याससूत्रस्य च यद्विवरं सा भुजा~। अपक्रमज्यातुल्यं च तद्विवरम्~। अतोऽपक्रमज्यैव तद्भुजा~। घटिकामण्डलमध्यादपक्रमज्याग्रान्तं सूत्रं व्यासार्धम्~।
तदग्रभ्रमणवृत्तमेव 

\newpage

\noindent स्वाहोरात्रवृत्तम्~। तर्हि तदहोरात्रवृत्तमपि घटिकामण्डलतुल्यम् ; यतः तदग्रभ्रमणवृत्तमेतत्\renewcommand{\thefootnote}{१}\footnote{णमेतत् \textendash\ ग. घ. ङ.}~। नैतदस्ति तिर्यक्त्वात् तस्य कर्णस्य~। भ्रमतस्तस्य व्यासार्धसूत्रस्य यदितरदग्रं तद्वृत्तस्य तन्नाभिकत्वाभावात्~। तद्व्यासार्धाग्रभ्रमणवृत्तस्य केन्द्रादिष्टापक्रमतुल्यान्तरे हि तद्व्यासार्धस्येतराग्रम्~। तद्विप्रकर्षसूत्रं च तद्वृत्ताद्विपरीतदिक्कम्~। अतस्तत्कर्णस्य कोटिरेव तद्वृत्तव्यासार्धम्~। यथा लातचक्रस्य व्यासार्धं भ्राम्यमाणादुन्मुकादल्पपरिमाणमेव स्यादेवं प्रवहवायुना भ्राम्यमाणस्य तिर्यग्भूतस्य गोलव्यासार्धस्य कर्णात्मकस्य भ्रमणवृत्तव्यासार्धमपि तद्वृत्तव्यासार्धादल्पमेवेति स्वाहोरात्रवृत्तं प्रदर्शितम्~॥~२४~॥ \\

\indent यानि त्रीणि विषुवन्मण्डलानि घटिकामण्डलमयनद्वयस्पृष्टं विषुवद्द्वयस्पृष्टं च तेषु घटिकामण्डलविपरीते समतिरश्चीने एवेतरे~। तत्र घटिकामण्डलेन विषुवत्स्पृष्टेन च विभक्तस्य वायुगोलस्य ये चत्वारः खण्डाः तेषु द्वावपक्रममण्डलेनापि भेदितौ~। तेष्वेक पादो मेषाद्यर्धेन धनुराकारेण विदारितः~। तौल्यादिराशिषट्केण तत्प्रतियोगिदिग्गतः पादश्च~। तत्र घटिकामण्डलापक्रममण्डलान्तरालमल्पम् अपक्रममण्डलस्य स्वमध्यस्पृष्टविषुवन्मण्डलस्य चान्तरालं महत्~। तद्विभागश्च अयनस्पृष्टविषुवन्मण्डले दृश्यः~। स्वस्यापक्रममण्डलस्य च यः संयोगः स्वस्य\renewcommand{\thefootnote}{२}\footnote{तस्य \textendash\ क.} घटिकामण्डलस्य च यः संयोगः तदन्तरालं तत्परिधौ चतुर्विशतिभागमितम्~। स्वस्य विषुवत्स्पृष्टमण्डलस्य च यः संयोगः तस्यापक्रममण्डलगतायनस्य च यदन्तरालं तत्परिधौ षष्टि\renewcommand{\thefootnote}{३}\footnote{षट्षष्टि \textendash\ ग. घ. ङ.}भागमितम्~। एवमयनगतविषुवन्मण्डलपरिधेः पादस्य नवीतभागमितस्य विभागः, तत्राल्पक्षेत्रगतं त्रैराशिकं प्रदर्शितम्~। तदितरान्तरालस्याप्यपक्रममण्डलभ्रमणकालो\renewcommand{\thefootnote}{४}\footnote{कलो \textendash\ ख. ग. ङ.}पयोगात् तद्गतं त्रैराशिमिदानीं प्रदर्श्यते\textendash 
\begin{quote}   
{\ab इष्टज्यागुणितमहोरात्रव्यासार्धमेव काष्ठान्त्यम्~। \\
स्वाहोरात्रार्धहृतफलमजाल्लङ्कोदयप्राग्ज्याः~॥~२५~॥} 
\end{quote}

\newpage

\indent इति~। अत्रापि विषुवतः प्रवृत्तापक्रममण्डलेष्टप्रदेशभुजाज्यैवेच्छा प्रमाणमपि व्यासार्धमेव~। इतरान्तरालभागस्य षट्षष्टिभागमितस्य ज्या 
परमापक्रमज्यायाः कोटिः फलम्~। इच्छाफलं तु स्वाहोरात्रवृत्तगता ज्या~। विषुवत्स्पृष्टविषुवन्मण्डलतः प्रवृत्ता इत्येकं त्रैराशिकम्~। अत्र
त्रैराशिकद्वयं युगपत् क्रियते~। इयं स्वाहोरात्रगतैव ज्येच्छा~। स्वाहोरात्रविष्कम्भार्धं प्रमाणम् ; व्यासार्धं फलम्~। इच्छाफलमपक्रममण्डलगतेष्टज्या चापस्य भ्रमणकालासूनां जीवा~। स्वाहोरात्रवृत्तस्य घटिकामण्डलादल्पत्वेऽपि तदुभयभ्रमणकालस्तुल्य एव~। यथा घटस्य भ्राम्यमाणस्य तदुदरस्य 
च बिलस्य च बुध्नस्य च परिणाहभेदेऽपि भ्रमणकालस्तुल्यः~। एवं स्वाहोरात्रस्यापि भ्रमणे खखषट्घनतुल्या एवासवः~। तस्मात् तस्य 
खखषट्घनांशस्य भ्रमणकाल एकः प्राणः~। ततस्तद्गतानां कलानामल्पत्वात् ताभिर्मीयमाना सा जीवा महती स्यात्~। तत्रैवं त्रैराशिकम् \textendash\ यदि स्वाहोरात्रव्यासार्धगताभिः घटिकामण्डलकलाभिः स्ववृत्तकला व्यासार्धतुल्यसङ्ख्या लभ्यन्ते तदेष्टापक्रमभुजकस्यापक्रमगतेष्टज्याकर्णकस्य कोट्या घटिकामण्डलकलामितया स्वाहोरात्रवृत्तगतकलामिता कियतीति~। पूर्वत्र व्यासार्धं भागहारः, अत्र गुणकार इति, इष्टज्यायाः
काष्ठान्त्यस्वाहोरात्रार्धमेव गुणकारः~। स्वाहोरात्रार्धं भागहारः, फलं स्वाहोरात्रकलामिता पूर्वापरायता इष्टज्याकर्णस्य कोटिः~। तच्चापीकृतं 
स्वाहोरात्रकलामितं तत्कोटिचापम्~। तेषां हि प्रत्येकं भ्रमणकालः प्राणः~। ततस्तावद्भिः प्राणैः स्वाहोरात्रगतं तत्कोटिचापं भ्रमति~।
तत्तुल्यल्यश्चेष्टज्यासम्बन्ध्यपक्रमचापभ्रमणकालः ; यत उभयोरपि भ्रमणारम्भ परिसमाप्तिश्च युगपत् स्यात्~।
परिसमाप्त्योरुभयोर्यौगपद्यमुभयोरग्रस्य संश्लिष्टत्वादेव सिद्धम्~। आरम्भस्य यौगपद्यं पुनरितराग्रयोर्विषुवन्मण्डलादेव

\newpage

\noindent प्रवृत्तत्वात्~। विषुवन्मण्डलस्य हि घटिकामण्डलसमतिरश्चीनत्वात् प्रागर्धं कृत्स्त्रं युगपदेवोदेति~। इतरदर्धं कृत्स्नं युगपदेवास्तमेति च~। एवं विश्लिष्टाग्रयोरप्युदयस्य यौगपद्यात् तदुभयचापखण्डयोर्भ्रमणकालस्तुल्य एव~। आरम्भस्य यौगपद्याय ह्यजादित्युक्तम्~। अजशब्देन विषुवत्प्रदेश उपलक्ष्यते~। ज्याशब्देन चापीकरणयोग्यत्वं सूचितम्~। प्राक्शब्देन जीवायाः प्रागायतत्वमुच्यते~। स्वाहोरात्रसम्बन्धिफलचापस्य यावता कालेन भ्रमणमिष्टज्याकर्णस्य तिर्यक्त्वात् तत्सम्बन्धिचापस्य महतोऽपि तावतैव कालेन भ्रमणं युक्तमिति प्राग्ग्रहणेन द्योत्यते~। प्रागादि दिग्व्यवस्था च प्रवहवायुभ्रमणवशादेव~। अत एव स्वाहोरात्रवृत्तानां पूर्वापरत्वम्~। तदपेक्षयैवापक्रममण्डलस्य तिर्यक्त्वमपि~। निरक्षदेशापेक्षया सर्वाण्यपि स्वाहोरात्रमण्डलानि पूर्वापरायतान्येव~। उत्तरतः पुनरक्षमहत्त्वानुरूपं स्वाहोरात्रोपरिभागस्य दक्षिणतोऽवनामः,
अधोभागस्योत्तरत उन्नामश्च स्यात्~। अतस्तस्योन्नतप्रदेशस्यावनतप्रदेशस्य चाध ऊर्ध्वत्वं विहन्येत~। तथापि पूर्वापरपार्श्वयोः पूर्वापरत्वं स्यादेव~। अत एवोर्ध्वप्रदेशस्याधोभागस्य च परस्परं दाक्षिणोत्तरत्वं च स्यात्~। अधोभागाद्दक्षिणत एवोर्ध्वभागः, ऊर्ध्वभागादुत्तरत एवाधोभागश्चेति~।
तद्वशादेव सर्वत्रापि दिङ्नियम इति च स्वाहोरात्रजीवायाः प्रागायतत्वोक्त्यैव द्योतितः\renewcommand{\thefootnote}{१}\footnote{तम् \textendash\ क. ग. घ. ङ.}~॥~२५~॥ \\


\indent एवमपक्रममण्डलावयवानां विषुवत्प्रत्यासन्नप्रदेशस्य भ्रमणकालस्याल्पत्वं ततः क्रमेण महत्त्वं च तत्त्वतः प्रदर्श्य साक्षे देशे पुनस्ततोऽपि
विशेषं प्रदर्शयितुं क्षयवृद्ध्यर्धासवः प्रदर्श्यन्ते\textendash 
\begin{quote}
{\ab इष्टापक्रमगुणितामक्षज्यां लम्बकेन हृत्वा या~। \\
 स्वाहोरात्रे क्षितिजा क्षयवृद्धिज्या दिननिशोः सा~॥~२६~॥} 
\end{quote}

\newpage

\indent इति~। अपक्रमयुक्तिवदेव क्षितिज्यायुक्तिरपि, उभयत्रापि गोलपृष्ठगतयोः गोलतुल्यपरिणाहयोः द्वयोर्वृत्तयोश्चक्रार्द्धान्तरितसम्पातत्वात्~। 
तत्सम्पातात् प्रभृति तदन्तरालानयनविषयत्वात् त्रैराशिकस्य~। तथापि तस्य चास्य चेच्छाप्रमाणभेदादेव भेदः~। तत्र हि कर्णात्मके इच्छाप्रमाणे,
अत्र कोट्यात्मके~। उभयत्रापि भुजैव प्रमाणफलमिच्छाफलमपि~। स्वाहोरात्रवृत्तगतत्वात्, क्षितिज्याया उन्मण्डलविपरीतदिक्कत्वात्,
क्षितिजोन्मण्डलान्तरालावगाहित्वाच्च~। क्षितिजगतानां ज्यानामेव कर्णत्वम्~। उन्मण्डलव्यासानुरूपा च कोटिः~। उन्मण्डलात् प्रभृति क्षितिजाग्रा
क्षितिज्या भुजा~। अत्र प्रमाणफलक्षेत्रप्रदर्शनाय क्षितिजोत्तरस्वस्तिके सूत्रस्यैकमग्रं बद्ध्वा गोलान्तरुपरि नीत्वा उन्मण्डलादूर्ध्वं
क्षितिजतुल्यान्तरे दक्षिणोत्तरवृत्तेऽन्यदग्रं बध्नीयात्~। तस्याधोगतमर्धं क्षितिजव्यासार्धकर्णस्य भुजा~। तच्छरोनं उन्मण्डलव्यासार्धं कोटिः~। सा च लम्बज्यातुल्या, अक्षकोटित्वात्~। सेह प्रमाणम्~। अक्षज्यातुल्या तद्भुजा फलम्~। उन्मण्डलघटिकामण्डलसम्पातात् प्रभृति स्वाहोरात्रावधिज्योन्मण्डलगता इष्टापक्रमतुल्येच्छा~। सा च उन्मण्डलव्यासानुसारिणी, उन्मण्डलपरिध्येकदेशस्य ज्यात्वात्~। सापि पूर्ववदुन्मण्डलोदग्व्यासार्धे वा कल्प्या~। एवं उत्तरगोलगतं क्षितिज्याभुजकं क्षेत्रं प्रदर्शितम्~। दक्षिणगोलेऽपि क्षितिजदक्षिणस्वस्तिके सूत्रस्यैकमग्नं बद्ध्वा अधो नीत्वा उन्मण्डलादधोऽपि क्षितिजान्तरे दक्षिणोत्तरवृत्ते बध्नीयात्~। अत्र तदुपर्यर्धं फलम्~। उन्मण्डलव्यासार्धं तच्छरोनं कोटिः~। सैव प्रमाणम्~। उन्मण्डलवृत्तस्थेष्टापक्रमज्यैवेच्छा~। स्वाहोरात्रमण्डलगतैवैच्छाफलात्मिका क्षितिज्या~। उभयत्रापि 
क्षितिजगतार्काग्रज्यैव कर्णः~। तत्र कर्णस्य अज्ञातत्वादेककोट्या इच्छात्वम्~। अत एव शरोनव्या\renewcommand{\thefootnote}{१}\footnote{शरोव्या \textendash\ ग.}सार्धस्य कोट्यात्मकस्य प्रमाणत्वं च~। इतरथेच्छाप्रमाणयोः समानजातित्वं न स्यादिति~। यदि प्रथममर्काग्रानीयते तर्हि 

\newpage

\noindent तस्या इच्छात्वं क्षितिजव्यासार्धस्य प्रमाणत्वं च~। तदाप्यक्षज्यैव फलम्~। क्षितिज्यैवेच्छाफलमपि~। सा क्षितिजा क्षितिसम्बन्धिनी ज्या~। सा स्वाहोरात्रे स्थिता~। सा हि दिननिशोः क्षयवृद्धिज्या~। तच्चापभ्रमणकालेन हि दिननिशोः क्षयवृद्धी स्याताम्~। स्वाहोरात्रे भ्रमतां ज्योतिषामुन्मण्डलादूर्ध्वभ्रमणकालस्य अधोगतस्वाहोरात्रभ्रमणकालस्य च तुल्यतया भाव्यम्~। तत्पार्श्वगतत्वादुन्मण्डलस्य~। तत उन्मण्डलाद्यावत्यन्तरे क्षितिजं स्पृशति तत्र हि उदयः~। तत्प्रदेशश्चोत्तरगोले उन्मण्डलादधोगत एव~। तत्रोदितस्य 
ज्योतिष उन्मण्डलप्राप्तिर्यावता कालेन तावता दिनार्धस्याधिक्यं स्यात्~।
उन्मण्डलदक्षिणोत्तर\renewcommand{\thefootnote}{१}\footnote{रान्तराल \textendash\ ग. घ. ङ.}मण्डलान्तरालभ्रमणा\renewcommand{\thefootnote}{२}\footnote{रभ्रमण \textendash\ ख.}कालो हि अहोरात्रपादः~। निरक्षदेशे तत्तुल्यमेव\renewcommand{\thefootnote}{३}\footnote{मेव सर्वेषां \textendash\ क.} दिनार्धं सर्वेषामपि~। साक्षदेशे पुनरुत्तरगोले क्षितिजस्य अधोगतत्वादुन्मण्डलक्षितिजान्तरालभ्रमणकालेनाधिकम्~। दक्षिणगोले तु उन्मण्डलादूर्ध्वगतत्वात् क्षितिजस्य तदन्तरालभ्रमणकालेन न्यूनं च~। कथं पुनस्तद्भ्रमणकाल आनीयते~। तदानयनमपि स्वाहोरात्र इत्यनेन सूचितम्~। स्वाहोरात्रे स्यिता हि क्षितिज्या~। स्वाहोरात्रवृत्तं च गोलपरिणाहादल्पम्~। तस्माल्लङ्कोदयन्यायेनात्रापि गोलकलाभिर्मितायाः क्षितिज्यायाः स्वकलाभिर्मीयमानाया अधिक्यं स्यात्~। मापकभूतानां स्वकलानामल्पत्वात्~। इति लङ्कोदयप्राग्ज्यावदत्रापि त्रैराशिकम्~। त्रैराशिकवाचोयुक्तिश्चैवम् \textendash\ यदि स्वाहोरात्रव्यासार्धतुल्याभिर्गोलकलाभिस्त्रिज्यातुल्याः स्वाहोरात्रवृत्तकला लभ्यन्ते, तदा क्षितिज्यातुल्याभिंर्गोलकलाभिः कियत्यः स्वाहोरात्रवृत्तकला लभ्या इति~। तत्र व्यासार्धं गुणकारः~। स्वाहोरात्रमितव्यासार्धं\renewcommand{\thefootnote}{४}\footnote{रात्रव्यास \textendash\ ख.} भागहारः~। फलं स्वाहोरात्रवृत्तगता स्वकखखषट्घनांशप्रमिता क्षितिजोन्मण्डलान्तरचापज्या~। ततस्तच्चापं तद्भ्रमणासवः, {\qt प्राणेनैति कलां भम्} इति लिप्ताप्राणयोस्तुल्यत्वोक्तेः~। उपचयापचयरूपेण क्षयवृद्धी चरत इति~। {\qt त एव चरशब्देनाभिधीयेते}

\newpage

\noindent इति गोविन्दस्वामिना प्रसिद्धायाश्चरसंज्ञायाः प्रवृत्तिनिमित्तं दर्शितम्~। कस्योपचयापचयरूपेणेत्येतन्महाभाष्य एव सिद्धम्~।
{\qt षण्मुहूर्ताश्चराचराः ते कदाचिदहर्गच्छन्ति कदाचिद्रात्रिम्} इत्यत्रेति भावः~। तत्रोत्तरायणे अह्न उपचयं निशाया अपचयं च कुर्वन्तश्चरन्ति~। दक्षिणायने तु व्यत्ययेन~। उत्तरायणादौ दिनमानं चतुर्विशतिनाडिकाः, षट्त्रिंशच्च निशामानम्~। ततः प्रभृति क्रमेण निशातो विनिर्गच्छन्तोऽवयवशो दिनं प्रविशन्तस्तदन्ते दिनं षट्त्रिंशन्नाड्यात्मकं, चतुर्विंशतिनाड्यात्मिकां च रात्रिं कुर्वन्ति~। ततः क्षणान्तरात्प्रभृति अह्नोऽवयवशो विनिर्गच्छन्नो रात्रिं प्रविशन्तश्च चरन्तस्तदन्ते रात्रिं षट्त्रिंशन्नाड्यात्मिकां दिनञ्च चतुर्विंशतिनाड्यात्मकं कुर्वन्ति~। एवं सर्वदापि चरन्तः क्षणमपि न तिष्ठन्तीत्यर्थः~॥~२६~॥ \\

\indent लङ्कोदयात् स्वदेशकालोदयस्य चरवशाद्यो भेदः तमेव प्रदर्शयति\textendash 

\begin{quote}
{\ab उदयति हि चक्रपादश्चरदलहीनेन दिवसपादेन~। \\
 प्रथमोऽन्त्यश्चाथान्यौ तत्सहितेन क्रमोत्क्रमतः~॥~२७~॥} 
\end{quote}

\indent इति~। यद्यपि चक्रपादस्य विषुवतः प्रभृत्यायनान्तं महापरिमाणत्वं तदवयवानां तुल्यानामेव भ्रमणकालस्योक्तं तथापि चक्रपादस्य कृत्स्नभ्रमणकालो दिनपादतुल्य एवेति निरक्षदेशे दिनपादेनैव चक्रपादो भ्रमति~। साक्षदेशे पुनश्चरदलहीनेन अहोरात्रपादेन चक्रपादः
प्रथमोऽन्त्यश्च भ्रमति~। कर्क्यादिपादौ पुनश्चरदलसहितेन दिनपादेन~। तत्रापि क्रमो विवक्ष्यते~। विषुवद्द्वयमभितो याश्चतस्रः कलाः तासां निरक्षदेशे भ्रमणकालस्तुल्य एव~। स द्वितीयादिभ्रमणकालादल्पश्च~। स चैकस्मात् प्राणात् व्यासार्धतुल्यच्छेदैः
व्यासार्धान्त्यस्वाहोरात्रान्तरालतुल्यैरवयवैर्हीनश्च~। स स्वकलासम्बन्धिचरसंस्कृतः तासां स्वदेशभ्रमणकलाः\renewcommand{\thefootnote}{१}\footnote{कालाः \textendash\ क. ख. ग.} स्युः~। एवं 
तत्तत्प्रदेशलङ्कोदयासवः तत्तच्चरदलासुभिः हीना युक्ताश्च मृगकर्क्याद्यो-

\newpage

\noindent र्विषुवतः\renewcommand{\thefootnote}{१}\footnote{तः तत्तद् \textendash\ ङ.} प्रभृति तत्तद्भुजाचापस्य स्वदेशोदयासवः\renewcommand{\thefootnote}{२}\footnote{शोदयाः \textendash\ ग. घ. ङ.}
स्युरित्यर्थः~। एवमेतदन्तेन ग्रन्थेन ग्रहाणां स्वस्वमध्यसावनस्य स्फुटसावनस्य च अन्तरालकालः\renewcommand{\thefootnote}{३}\footnote{लाः \textendash\ ग. घ.} कार्त्स्न्येन प्रतिपादितः~। कथं पुनः सावनस्य द्वैविध्यं मध्यसावनं स्फुटसावनं चेति~। तद्भेदकारणं सूर्यसावने तावत् प्रदर्श्यते~। तस्य तस्योदयद्वयान्तरं हि स्वकीयसावनम्~। तत्र सूर्यसावनं {\qt रविभूयोगा दिवसा} इति विशेषेण च प्रदर्शितम्~। चन्द्रादिसावनमपि {\qt भगणा द्वयोर्द्वयोः} इत्यादिना सामान्यन्यायेन सिद्ध्यति~। सूर्यसिद्धान्ते तेषामपि\textendash

\begin{quote}
{\qt भोदया भगणैः स्वैः स्वैरूनास्तस्योदया युगे}
\end{quote} 

\noindent इति सर्वेषां युगसावनदिनानि प्रदर्शितानि~। यानि पुनः सूर्यसावनदिनानि तानि युगभूदिनान्युच्यन्ते~। युगकालेन तेषां सर्वेषां तुल्यतया विभक्तो यः कालः तत्तुल्यानि सर्वाण्यपि कल्प्यमानानि मध्यमसावनदिनानि~। तैरेव मध्यमसावनानयनं युक्तं, यतस्तुल्यपरिमाणानामेव त्रैराशिकयोग्यता~। तेष्वतीतदिवसेषु यश्चरमो दिवसः तदवसानकालजमेव अहर्गणेनानीतं मध्यमादिकम्~। ततः प्राक् पश्चाच्च स्यात् सूर्योदयः~। कथम्~। अहर्गणेन त्रैराशिकानीतं रविमध्यमं यावद्राश्यादिकं घटिकामण्डले तावति प्रदेश उदिते मध्यमसावनेनानीताहर्गणर्पारेपूर्तिः~। अपक्रममण्डलस्थार्कबिम्बघनमध्यस्य क्षितिजप्राप्तौ हि स्फुटसावनदिनस्य प्रतिपत्तिश्छेदो वा~। मध्यमतुल्यघटिकामण्डलप्रदेशस्य\renewcommand{\thefootnote}{४}\footnote{स्य चोदययोः \textendash\ ङ.} स्फुटतुल्यापक्रममण्डलप्रदेशस्य चोदययोः न यौगपद्यनियमः, कादाचित्कमेव यौगपद्यम्~। प्रायेण भिन्नकालावेव हि तत्प्रदेशोदयौ~। तद्भेदकारणं त्रिविधम्~। मध्यमस्फुटभेदः, स्फुटस्यापि कालक्षेत्रभेदः\renewcommand{\thefootnote}{५}\footnote{दश्चरभेदश्च रवेः क्षितिज \textendash\ ख.},
क्षितिजोन्मण्डलोदयकालान्तरालभ्रमणभेदश्च रवेः~। तत्र सूर्यस्फुटभुजायाः काल उक्तप्रकारेण नेयः~। सूर्यस्फुटेऽयनचलनं प्रक्षिप्य, बाहुज्यां गृहीत्वा, तां पृथग्विन्यस्य अपक्रमं स्वाहोरात्रं चानीय, {\qt इष्टज्यागुणितमहोरात्रव्यासार्धम्} इत्यादिना तत्प्राण- 

\newpage

\noindent पिण्डं चानीय, पुनस्तदपक्रमज्ययैव चरदलासूनानीय, मकरादिके तत् प्राणपिण्डाद्विशोधयेत् ; कर्क्यादिके भुजाचापप्राणपिण्डे क्षिपेच्च~। तदा स प्राणपिण्डो रव्युदयस्य स्वभुजामूलविषुवदुदयस्य चान्तरालकालः स्यात्~। प्रथमपादे स एवोदग्विषुवतः प्रभृति रव्युदयगतकालः स्यात्~। द्वितीये पादे पुनर्भुजासम्बन्धिकालस्य ज्ञातत्वात् तद्भुजायाश्चावाग्विषुवतः प्रवृत्तत्वाद्रविभुजाप्राणपिण्डं षड्राशिभ्यस्त्यक्त्वा शिष्टं रव्युदयकालज आर्क्षद्युगतकालः स्यात्~। तृतीये षड्राशियुतः प्राणपिण्डः चतुर्थे मण्डलतः शोधयित्वा शिष्टमार्क्षद्युगतत्वेन ग्राह्यम्~। एवं रव्युदये आर्क्षद्युगतं स्यात्~। अहर्गणानीतमध्यमेच्छाकाले रविमध्यमतुल्यमेवार्क्षद्युगतम् इति तयोर्विश्लेषो मध्यमस्फुटसावनान्तरालकालः~। तस्य या ग्रहभुक्तिः सा मध्यमसावनात् स्फुटसावनस्याधिक्ये मध्यमसावनानीते ग्रहमध्यमे योज्या ; 
इतरथा विशोध्या~। पुनरेतत्संस्कृतं रविमध्यमं स्फुटीकृत्य प्राग्वदेव रव्युदयकालजातमार्क्षद्युगतमानयेत्~। तस्य अदृर्गणानीतरविमध्यमस्य
चान्तरालकलाश्च ग्रहभुक्त्या हत्वा चक्रकलाभिर्विभज्य लब्धमहर्गणानीते रविमध्यमे यथोक्तं संस्कुर्यात्~। पुनस्तन्मध्यमस्यापि स्फुटीकरणादिक्रमेण
अविशेषयेत्~। अविशिष्टस्य रव्युदयार्क्षद्युगतस्य अहर्गणानीतरविमध्यमस्य च अन्तरालकला एव सर्वेषां ग्रहोच्चमध्यमानामिच्छाराशिः~। तेनार्क्षदिनभवाः स्वमध्यगतीर्हत्वा चक्रकलाप्ताः कलाः तत्तन्मध्यमेऽपि रविवत् संस्कुर्यात्~। एवं सर्वेषामादित्योदयकालमध्यमं स्यात्~। कथं पुनर्मध्यमसावनाहोरात्रान्ते रविमध्यमतुल्यमार्क्षद्युगतम्~। युगकालस्य युगभूदिनांश एकं सावनदिनमिति सर्वेषां सम्प्रतिपन्नमेव~। तस्मात् युगार्क्षदिनगणे भूभगणत्वेन पठिते युगभूदिनैर्विभक्ते यत् फलं तदेकस्य सावनदिनस्यार्क्षदिनं सावयवं स्यात् इति तस्मिन् युगार्क्षदिनगणे युगभूदिनैर्ह्रियमाणे एकं फलं परिपूर्णं ग्राह्यम्~। तच्छिष्टं च रविभगणतुल्यम् ; {\qt रविभूयोगा} 

\newpage

\noindent {\qt दिवसा} इत्युक्तत्वात्~। भगणा द्वयोर्द्वयोः ये विशेषशेषा इति तयोर्विश्लेषस्य युगभूदिनत्वादेव शिष्टस्य रविभगणतुल्यत्वं सिद्धम्~। तत् शिष्टाच्चक्रकलाहतात् भूदिनैरेव विभज्य लब्धास्तदंशाः प्राणात्मका रविगतिकलातुल्याः~। तस्मात् रविमध्यमसहितमार्क्षं दिनमेव 
सावनदिनम्~। पुनः कलियातसावनदिनसम्बन्ध्यार्क्षदिनानयने यद्येकस्य सावनदिनस्य इयदार्क्षं\renewcommand{\thefootnote}{१}\footnote{यदाक्षं \textendash\ ङ.} दिनं तदा कलियातैः कियन्तीति~। तदेवाहर्गणगुणितं रविमध्यमसहितसावनदिनमेव~। तस्माद्याताहर्गणे सावनात्मके रविभगणसहिते सति अभीष्टार्कोदयात् प्रागतीतानि आर्क्षदिनानि स्युः~। पुनरादित्यभगणशेषं यन्मध्यमं वर्तमानार्क्षदिने तत्तुल्यः कालो गत इत्यादित्यमध्यमतुल्यत्वं मध्यमसावनदिनान्ते वर्तमानार्क्षदिनद्युगतस्य~।
तस्माच्चक्रकलाहते यातार्क्षदिनगणे कलीकृते रविमध्यमे च क्षिप्ते या राशिः स्यात् तावतां प्राणानां मध्यममेव याताहर्गणेनाप्यानीतम्~। स्फुटार्कोदयकालजप्राणानयने पुनरयं विशेषः \textendash\ कलीकृतरविमध्यमस्थाने देशान्तरसंस्कृताविशिष्टरविमध्यमस्फुटेनानीत आर्क्षद्युगतप्राण एव क्षेप्य इति, तदानयनपरमिदमार्यासप्तविंशत्यात्मकं वाक्यम्~। तत्रायनचलने सति रविमध्यमेऽप्ययनचलनं संस्कृत्यैव रविप्राणपिण्डेन सह विश्लेषेण कार्यम्~। यतो रविस्फुटेऽयनचलनं संस्कृत्यैव तत्प्राणा आनीताः~। याताहर्गणप्राणानयने पुनरयनचलनं न कार्यम्~। केवलमध्यममेव तत्र लिप्तोकृत्य क्षेप्यं, यत आर्क्षदिनावसानात् प्रागेव घटिकापक्रमयोगो विषुवदाख्य उदेति, अयनचलनस्य घनत्वे ऋणत्वे पश्चाच्च~। अतो वर्तमानार्क्षदिने पूर्वार्क्षदिनैकदेश एव अयनचलनतुल्यो विषुवदादित्याय प्रक्षिप्यते~। तेन न युग\renewcommand{\thefootnote}{२}\footnote{तेन युग \textendash\ घ.}यातप्राणपिण्डस्य भेदः~। स्फुटार्कोदयप्राणेषु पुनर्व्यत्ययेन च संस्कार्यम्~। आर्क्षदिनप्रक्षेपणे वर्तमानाक्षंदिनजत्वायेति बोद्धव्यम्~। एवमुदयकाले ग्रहस्फुटं प्रदर्शितम्~। किञ्च, अभीष्टकालोदयलग्नाद्यानयनमप्यत्रैव सूचितं यतोऽप- 

\newpage
\begin{sloppypar} 
\noindent क्रममण्डलावयवानामुदयप्रकारभेदनिमित्तः प्रत्यवयवमुदयकालभेदः अत्रोपपादितः~। कथम्~। अत्र हिशब्देनोपपत्तिः सूचिता~। तत्र 
प्रथमपाद आनयनक्रमेणैव उदेति~। तत्र प्रथमभागस्य यावानुन्मण्डलोदयकाले लङ्कोदयकालत्वेनानीतः~। ततः स्वदेशचरदलहीनेनैव कालेन स्वदेश 
उदेति, यतस्तदादेर्विषुवत उन्मण्डलोदयः क्षितिजोदयश्च युगपदेव~। तदग्रस्य तु उन्मण्डलोदयात् प्रागेव क्षितिजोदयः, यतो निरक्षदेशादुत्तरतः
क्षितिजोदगर्धमुन्मण्डलादधोगतम्~। क्षितिजोन्मण्डलोदयान्तरकालश्च चरदलाख्यः~। तेन साक्षनिरक्षदेशयोर्युगपत् प्रारब्धस्य साक्षे प्रागेवावसानात् 
निरक्षदेशोदयकालादल्पत्वं साक्षदेशोदयकालस्य~। एवमन्त्यपादेऽन्त्यभागस्य युगपदुदयावसानं, भिन्नकालश्च प्रारम्भः~। तत्र क्षितिजस्योर्ध्वगतत्वादुन्मण्डलोदयात् पश्चादेव क्षितिजोदय इति पश्चात् प्रारम्भात् युगपत् समाप्तैश्च निरक्षोदयादल्पकालत्वं साक्षोदयस्य~। एवमुदगयनगतानां सर्वेषामवयवानामूर्ध्वमुदयारम्भादधः परिसमाप्तिश्चाल्पकालत्वम्~। विषुवतो विप्रकृष्टानां\renewcommand{\thefootnote}{१}\footnote{ऽविप्रकृष्टानां \textendash\ ङ.} क्रमेणाधिक्यं च स्यात्~। लङ्कोदयानां तावदाधिक्यं गुण्यस्य साम्येऽपि हारकाणां तत्तत्स्वाहोरात्राणां क्रमेण ह्रासात् सकलायाः पिण्डज्याया गुणकारत्वाच्च युज्यत एव~। गणितसिद्धं चैतत् सर्वं स्वदेशोदयानामपि चरसंस्कारवशात् पूर्वखण्डापेक्षयोत्तरखण्डानां महत्त्वे शोध्यानां चरखण्डानां क्रमेणाल्पत्वादेव सिद्धम्~। दक्षिणायनावयवानां च उदयारम्भक्षितिजप्रदेशाद्दक्षिणत एवावसानात् तत्प्रदेशस्य वायुगोल ऊर्ध्वगतत्वादितरस्मात् तत्तच्चरदलखण्डाः क्षेप्याः~। तत्र कर्क्यादित्रिके क्रमेण, तुलादित्रिके चोत्क्रमेण चरदलखण्डानां महत्त्वात् तत्संस्काराल्लङ्कोदयानां निम्नत्वं प्रतिपूर्येत, धनात्मकत्वात् तत्र चरस्य~। उदग्विषुवत्यृणात्मकत्वात् कालक्षेत्रयोर्वैषम्यं महदेव स्यात्~। {\qt क्रमोत्क्रमत} इत्यनेन एतद्विवक्ष्यते~। ओजे पदे भुजायाः क्रमेणोदयः~। युग्मपदेऽपि भुजाया 
\end{sloppypar} 
\newpage

\noindent एवोत्क्रमेणोदयः~। न पुना राश्यादीनाम् अवाक्च्छिरत्वात् तत्र भुजाया इति~। एवं तत्तत्क्षेत्रावयवानामुदयकाले ज्ञाते इष्टद्युगतेन
तत्कालोदयलग्ना\renewcommand{\thefootnote}{१}\footnote{ग्नमपि \textendash\ ङ.}नयनमपि सुगमम्~। कथम्~। उदये तावदादित्यस्फुटतुल्यं हि लग्नम्~। तत्र द्युगतसम्बन्धिक्षेत्रे क्षिप्ते खलु उदयलग्नं स्यात्~। तदर्थं मख्यादिभिरानीता स्वदेशोदयप्राणा मृगकर्क्याद्योः पृथक् पृथगवधार्याः~। तेषु सायनार्कवर्तमानचापोदयप्राणैः तद्गन्तव्यकला हत्वा\renewcommand{\thefootnote}{२}\footnote{हृत्वा \textendash\ क.}, मख्या विभज्य, लब्धान् प्राणान् द्युगतप्राणेभ्यो हित्वा, चापशेषं रवौ क्षिप्त्वा, तदेष्यचापानां यावतां प्राणा द्युगतशेषाच्छोध्यास्तावतश्चापभागानपि क्षिप्त्वा शेषासून् मख्या हत्वा, तदुदयप्राणैर्हृत्वा, आप्ताः कलाश्च क्षिप्त्वा, अयनचलनं विशोधयेत्~। तदुदयलग्नम्~। षड्भाधिकमस्तलग्नं यत उद्यत उदयकालतुल्योऽस्तं गच्छतोऽस्तकालेऽपि~। यद्वा लङ्कोदयेषु मृगकर्क्याद्योर्व्यत्ययेन चरं संस्कृत्य पठितैरस्तप्राणैर्निशागतासुभिश्च एवमेवास्तलग्नमानयेत्~। द्युगतकालज्ञानाय प्राग्वदर्कलग्नयोरार्क्षद्युगतमानीय विश्लेषयेत्~। अत्रोभयथापि रविस्तात्कालिक एव ग्राह्यो, नौदयिकः, द्युगतनाड्यादेः सावनावयवत्वात्~। औदयिकार्कग्रहणे हि द्युगतासवोऽप्यार्क्षा एव ग्राह्याः, तेषामप्रसिद्धत्वात्~। सावनद्युगतेनैव लग्नमानीयते~। एवमेव लङ्कोदयैः नतासुभिश्च मध्यलग्नानयनं कार्यम्~। तच्च षड्भाधिकं पाताललग्नम्~। एतत्सर्वमत्रोक्तैर्गणितगोलन्यायैः सिद्धमेवेति भावः~॥~२७~॥ \\

\indent एवमभीष्टकाले\renewcommand{\thefootnote}{३}\footnote{लग्रहा \textendash\ ख.} ग्रहाद्यानयनं प्रदर्श्य तत्कालज्ञापनार्थं छायानयनं प्रदर्श्यते~। छाययैव हि दिने गतकाल\renewcommand{\thefootnote}{४}\footnote{दिनगतकाल \textendash\ ङ.}विज्ञानं, रात्रावपि यस्य कस्यचित् ज्योतिषः छाययैव गतकालो ज्ञातुं शक्य इति छायाह्रासवृद्धिहेतुः शङ्कुः प्रदर्श्यते\textendash 


\newpage

\begin{quote}
{\ab स्वाहोरात्रेष्टज्यां क्षितिजादवलम्बकाहतां कृत्वा~। \\
 विष्कम्भार्धविभक्ते दिनस्य गतशेषयोः शङ्कुः~॥~२८~॥} 
\end{quote}
 
\indent इति~। एतस्मिन्नहर्गण एतावति द्युगते देवदत्तस्योपनयनं कार्यमिति पूर्वोक्तेनैव गणितेन ज्ञाते स कालः पुनरिदानीं प्राप्तो वा 
नवेत्येतच्छङ्कुयन्त्रेण छायां परिच्छिद्यैव ज्ञातुं शक्यमिति, तदर्थमपि यत्नान्तरं कार्यमिति तत्प्रदर्शनपरत्वादस्य ग्रन्थस्य पूर्वेण सह
सम्बन्धः~। अभीष्टकाले या स्वाहोरात्रवृत्तगता ज्या सा स्वाहोरात्रेष्टज्या~। सा दिनगतस्य दिनगन्तव्यस्य वा प्राणात्मकस्य जीवैव~। ततः क्षितिजादेव प्रवृत्ता ग्राह्या, क्षितिजे हि उदयमस्तं च गच्छतीति~। एतदुक्तं भवति~\textendash\ स्वाहोरात्रे यत्राभीष्टकालेऽर्कबिम्बघनमध्यं वर्तते, क्षितिजसमश्च यः प्रदेशः तदन्तरालज्या न ह्यर्धात्मिका स्यात्~। यतः स्वाहोरात्रपार्श्वतः प्रवृत्तेह जीवा निरक्षदेशेऽधऊर्ध्वायता स्यात्~। साक्षदेशेऽपि तस्या दक्षिणोत्तरतयैव तिर्यक्त्वम्~। नापि पूर्वापरतया~। क्षितिजस्य स्वाहोरात्रपार्श्वत्वाभावात्~। उन्मण्डलप्रदेश एव हि स्वाहोरात्राणां पार्श्वभागः~। 
तस्मादुत्तरगोले उन्मण्डलादधोगतत्वात् क्षितिजस्य ततः प्रवृत्ता स्वाहोरात्रवृत्तार्धज्या ईषदुत्ताना स्यात्~। ततस्तस्योर्ध्वाग्रमधोऽग्रात्
प्राग्गतम्~। दक्षिणगोले पुनस्तदग्रस्यावनतत्वात् ऊर्ध्वाग्नं प्रत्यग्गतम्~। प्रत्यक्कपालं तु विपरीतम्~। तस्मादुभयथापि तस्या जीवायास्तिर्यक्त्वं स्यात्~। तन्माभूदित्युन्मण्डलादेव ज्याग्रहणं कार्यम्~। तत उत्तरगोले द्युगतासुभ्यश्चरासूंस्त्यक्त्वा शिष्टत एव जीवा ग्राह्या, दक्षिणगोले तु चरप्राणसहितेभ्यः~। सा चार्धात्मिकैव~। कर्णभूतया तया तु तत्कोटिरूपः शङ्कुखण्ड एवोदग्गोले लभ्यते~। न कृत्स्नः शङ्कुः ; भूपार्श्वादूर्ध्वगतादुन्मण्डलादेव तत्कर्णभूताया जीवायाः प्रवृत्तेः~। दक्षिणगोले चोन्मण्डलादूध्वर्गता ज्यार्धात्मिकैव~। सा क्षितिज्योनैव श्रुतिः
शङ्कुकोटिकत्र्यश्रस्य ~।

\newpage

\noindent अतः सापि नार्धज्या~। किन्तु खण्डज्यैव~। दृङ्मण्डलप्रतिमण्डलगतः शङ्कुस्तु क्षितिजदृङ्मण्डलसम्पातात्प्रभृति
ग्रहबिम्बघनमध्याग्रार्धज्या~। सा ह्यत्र कोटिः~। सैवेहानेया, तदधीनत्वाच्छायायाः~। ततस्तत्सम्बन्धी कर्ण एवेच्छात्वेन ग्राह्यः~। स च क्षितिजादेव प्रवृत्तः~। अत उक्तं क्षितिजादिति~। कथं पुनस्तस्य कर्णस्यानयनम्~। उन्मण्डलात् प्रवृत्तायां जीवायां क्षितिजोन्मण्डलान्तरालगता तच्चरज्यापि योज्या~। तदानीं कृत्स्नः कर्णः सिद्ध्यति सौम्ये~। याम्ये शोध्या च~। सा पुनर्न समस्तज्या, नाप्यर्धज्या~। एवमस्य क्षेत्रस्याकृतिः~। उत्तरगोले अर्धादधिकं तच्चापक्षेत्रम्~। दक्षिणगोलेऽर्धादूनमेव~। का तर्हि, खण्डज्यात्मिकैव~। एवमानीता जीवा स्वाहोरात्रार्धखखषड्घनांशकलाप्रमितैव ; यतो दिनगतशेषयोः प्राणानामेव जीवा गृहीता~। स्वाहोरात्रखखषड्घनांशभ्रमणकालतुल्य 
एव च प्राणः~। तस्मात् तस्या ज्योतिश्चक्रकलाप्रमिताया लाभायेदं त्रैराशिकम् \textendash\ यदि त्रिज्यातुल्याभिः स्वकलाभिरपक्रमकोटितयानीतस्वाहोरात्रव्यासार्धतुल्या ज्योतिश्चक्रकला लभ्यन्ते, तदैतावतीभिः स्वकलाभिः कियत्यो ज्योतिश्चक्रकला लभ्या इति~। स्वाहोरात्रव्यासार्धेन हत्वा त्रिज्यया हरणं कार्यं तस्याः, तदर्थं स्वाहोरात्रग्रहणम्~। तया शङ्क्वानयने पुनरिदं त्रैराशिकम् \textendash\ यदि साक्षदेशे विष्कम्भार्धतुल्यायाः स्वाहोरात्रजीवायाः कर्णभूताया लम्बकतुल्या कोटिर्लभ्यते, तदास्याः स्वाहोरात्रेष्टज्यायाः कियतीति स्वाहोरात्रेष्टज्याग्रस्योछ्रितिर्लभ्यते~। सैव रवेरुछ्रितिः, रव्यग्रत्वात् स्वाहोरात्रेष्टज्यायाः~। एवं दिनगतस्य शेषस्य वा शङ्कुः कार्य इत्यर्थः~। इतःपरं कूर्म गणितपादोक्तयुक्तया सिद्धम्~। तत्र {\qt यश्चैव भुजावर्ग} इत्यादिना शङ्कुत्रिज्यावर्गान्तरमूलं छायेति सिद्धम् स्ववृत्तकल्पनया त्रैराशिकसूत्रेण च द्वादशाङ्गुलशङ्कुच्छायाकर्णाङ्गुलानयनमपि सिद्धम् इति, न किञ्चिदिह वक्तव्यमवशिष्टमिति भावः~। छायायां प्रमाय ज्ञातायां दिनगतगन्तव्यानयनमपि विपरीतकर्मणैंव सिद्धम्~॥~२८~॥ 


\newpage

\indent अतः प्रासङ्गिकमेतत् क्षेत्रगतभुजानयनं प्रदर्श्यते\textendash  
\begin{quote}
{\ab विषुवज्जीवागुणितः स्वेष्टः शङ्कुः स्वलम्बकेन हतः~। \\
 अस्तमयोदयसूत्राद्दक्षिणतः सूर्यशङ्क्वग्रम्~॥~२९~॥} 
\end{quote}

\indent इति~। तर्हि तदपि न वक्तव्यम्~। त्रैराशिकसूत्रेणैव सिद्धत्वात्~। नैष दोषः~। एतद्युक्तिवैशद्याय छायाभुजानयने शङ्क्वग्रोपयोगप्रदर्शनाय चेदं सूत्रमारभ्यत इति~। तस्मादेवमिह क्षेत्रं प्रदर्शनीयम्~। पूर्वस्वस्तिकात् अग्रान्तरे क्षितिजे सूत्रस्यैकमग्रं बद्ध्वा परस्वस्तिकादपि तावत्यन्तर एव बध्नीयात्~। एवमुभयोर्गोलयोरस्तोदयसूत्रं प्रदर्श्यम्~। स्वाहोरात्रे पुनः यत्र भास्वद्बिम्बमवतिष्ठते, तत्र सूत्रस्यैकमग्रं बद्ध्वा अन्यदग्रमस्तोदयसूत्रे तत्सन्निकृष्टप्रदेशे बध्नीयात्~। सा स्वाहोरात्रेष्टज्या~। उन्मण्डलेऽपि स्वाहोरात्रस्पृष्टयोः पूर्वापरदिशोः सूत्र\renewcommand{\thefootnote}{१}\footnote{त्राग्र \textendash\ ग. घ. ङ.}स्याग्रद्वयं बध्नीयात्, तदुन्मण्डलास्तोदयसूत्रम्~। यः पुनरुभयोरस्तोदयसूत्रयोरन्तरालतुल्यस्वाहोरात्रेष्टज्याखण्डः, सा क्षितिज्या~। सैव स्वकलाप्रमिता चरज्या~। एवं स्वाहोरात्रेष्टज्याकर्णस्य\renewcommand{\thefootnote}{२}\footnote{खण्डस्य \textendash\ क.} द्वौ खण्डौ पृथगानीय योजितौ~। महाशङ्कुरेवास्य क्षेत्रस्य कोटिः~। तन्मूलस्य अस्तोदयसूत्रस्य चान्तरालं शङ्क्वग्रम्~। तच्च नित्यदक्षिणम् ; तदुक्तम्\textendash 
 
\begin{center}
{\qt अस्तमयोदयसूत्राद्दक्षिणतः सूर्यशङ्क्वग्रम्} 
\end{center}
इति~। सा भुजा~॥~२९~॥ \\

\indent यः पुनः क्षितिज्याभुजकस्यापक्रमकोटिकस्य कर्णः पूर्वमनुक्तः स इदानीं प्रदर्श्यते\textendash 
\begin{quote}
{\ab परमापक्रमजीवामिष्टज्यार्धाहतां ततो विभजेत्~।\\
 ज्यालम्बकेन लब्धार्काग्रा पूर्वापरे क्षितिजे~॥~३०~॥} 
\end{quote}

\newpage

\indent इति~। इष्टशब्देन आदित्यभुजा विवक्ष्यते, अर्काग्राया इहानीयमानत्वात्~। प्रदर्शनं चैतदिन्द्वग्रादेः~। तदयमर्थः \textendash\ 
परमापक्रमजीवामादित्यभुजज्यया हत्वा ज्यालम्बकेन विभजेत्~। तत्र लब्धार्काग्रा ज्या~। उद्यदादित्यप्रदेशेऽग्रं यस्याः सा जीवार्काग्रा~। पूर्वापरे क्षितिजे सा पूर्वापरक्षितिजार्धयोः पूर्वापरसूत्रादर्कोदयास्तमयविप्रकर्षप्रर्दर्शनायावतिष्ठते~। अत्रापि त्रैराशिकद्वयमेकीकृतम्~। तत्र प्रथमेन अपक्रम आनीयते~। तच्च पूर्वमेव प्रदर्शितम्~। द्वितीयत्रैराशिकक्षेत्रप्रदर्शनाय स्वाहोरात्रोन्मण्डलसम्पातादुभयतः क्षितिजोन्मण्डलविवरान्तरे स्वाहोरात्रवृत्ते
सूत्राग्रद्वयं बध्नीयात्~। तदर्धं क्षितिज्या~। इतरगोलेऽपि तावति स्वाहोरात्रमण्डले एवमेव क्षितिज्यां बध्नीयात्~। क्षितिज्यासूत्रमध्ययोः पुनरन्यत् सूत्रं बध्नीयात्~। तदर्धमपक्रमज्योन्मण्डलगता~। तद्घटिकामण्डलसम्पाते\renewcommand{\thefootnote}{१}\footnote{तद्घटिकासम्पाते \textendash\ ख. ग. घ. ङ.} सूत्रस्यैकमग्नं बद्ध्वा अन्यत् क्षितिज्याग्रे क्षितिजे बध्नीयात्~। सा अर्काग्रा ; सा कर्णभूता~। अपक्रमज्यया कोट्येच्छात्मिकया लम्बकेन प्रमाणेन चेहार्काग्रानीयते~। यदि लम्बकतुल्यया कोट्या व्यासार्धतुल्यः कर्णो लभ्यते, तदापक्रमतुल्यया कियानित्यर्काग्रा ज्या लभ्यते~। तत्रापक्रमानयने व्यासार्धं भागहारः~। तेनार्काग्रानयने व्यासार्धं गुणकारः~। 
ततस्तयोस्तुल्यत्वान्नष्टयोरिएष्टज्यार्धस्य परमापक्रमज्या गुणकारः, लम्बको भागहारः, फलमर्काग्रा इति~। विक्षेपवतां पुनः स्फुटापक्रमज्ययैव
स्वाग्रज्यानयनं, न पुनः सायनभुजज्यया पातोनभुजज्यया च~। तेषां विक्षेपवशात् परमापक्रमज्याया नानात्वात्~। ततस्तेषां
द्वित्तीयत्रैराशिकं पृथगेव कार्यम् इति ज्ञापयितुमर्काग्रेति विशिष्यते~। अत एव भास्करोऽप्येवमाह\textendash
 
\begin{quote}
{\qt क्षुण्णां परमया क्रान्त्या भुजाज्यामुष्णदीधितेः~। \\
     लम्बकेन विभज्याप्तामर्काग्रां तां प्रचक्षते~॥} 
\end{quote}

\noindent इति~। त्रैराशिके हि इच्छाराशिना फलहननमेव युक्तम्~। इति च स्थापितं 

\newpage

\noindent त्रैराशिकसूत्रे~। तदविघातार्थमिह परमापक्रमजीवामित्युक्तम्~। भास्करस्तु फलसाम्यात् {\qt क्षुण्णां परमया क्रान्त्या भुजाज्याम्} इति विपर्ययेणोक्तवान्~। छायाभुजानयनमप्यत्रैव सिद्धम्~। यतोऽग्रद्वययोगो वियोगश्च याम्योत्तरगोलयोश्छाया भुजा~। तद्योगवियोगयुक्तिश्च स्पष्टा ; यतः पूर्वापरसूत्रस्य अस्तोदयसूत्रस्य\renewcommand{\thefootnote}{१}\footnote{उदयास्तसूत्रस्य \textendash\ क. ख. ग. ङ.} चान्तरालमर्काग्रा~। याम्यगोले ह्यस्तोदयसूत्रं पूर्वापरसूत्राद्याम्यदिग्गतम्~। ततोऽपि दक्षिणतः शङ्क्वग्रान्तरे च शङ्कुः~। ततस्तदन्तरालद्वययोग एव शङ्कुपूर्वापरसूत्रान्तरालम्~। सैव छाया भुजा दक्षिणोत्तरायता~। तत्पूर्वापरसूत्रसम्पातस्य वृत्तमध्यस्य च अन्तरालं कोटिः~। तत्कर्णभूता हि छाया~। उदग्गोलेऽग्रयोर्भिन्नदिक्कत्वाद्वियोगः~। तत्रार्काग्रायाः शङ्क्वग्रस्याल्पत्वे शङ्क्वग्रविशुद्धार्काग्रा यावती पूर्वापरसूत्रादुदक् तावत्प्रदेशगतः शङ्कुः~। ततः सा सौम्यैव भुजा~। यदा पुनः शङ्क्वग्रस्याधिक्यात् ततोऽर्काग्रा विशोध्यते, तदास्तोदयसूत्रात् दक्षिणतो गच्छता शङ्कुना पूर्वापरसूत्रस्योल्लङ्घनात् तत्र शिष्टा भुजा याम्यैव~। तच्छायावर्गान्तरमूलं कोटिः~। सा प्राक्कपाले प्राची, प्रतीची च प्रत्यक्कपाले, एतद्द्वयं द्वादशघ्नं शङ्कुभक्तमङ्गुलात्मकं व्यस्तदिक्कं च~। अनवगतदिक्कस्य दिगवगमनमप्येतत्तुल्यैस्त्रिभिः सूत्रैः सुकरम्~। अनवगताक्षस्य दिगवगमनं तु अयनसन्धौ पूर्वापरकपालयोश्छाययोस्तुल्ययोः तदग्रद्वयान्तरालसूत्रेण तद्विपरीतसूत्रेण च स्यात्~। वृत्ताकारसमतलस्थलीगतनक्षत्रास्तोदयसूत्रवशाद्वा\renewcommand{\thefootnote}{२}\footnote{दावे \textendash\ ख. ग. ङ.} वेद्यम्~। इत्येतदप्यत्रोक्तैः स्वाहोरात्रभूम्याकृत्यादिभिः सिद्धम्~। किञ्च मध्याह्ने छायाह्रासवृद्ध्योर्मान्द्यात्
तत्कोट्याः शैघ्र्याच्च छायानीतकालात् कोट्यानीतकालः सूक्ष्मः इति कालस्य सूक्ष्मतया ज्ञानमपि कोट्याः प्रयोजनम्~। शैघ्र्यं च कोट्या नतकालज्यधीनत्वात्~। नतज्यायाः स्वाहोरात्रदक्षिणोत्तरमण्डलसम्पातात् प्रवृत्तेः पूर्वापरायतत्वात् कोटित्वम्~। सापि द्युज्याहता त्रिज्याहृतैव कोटिः~। 

\newpage

\noindent तदाङ्गुलानयनं प्राग्वत्~। भुजाङ्गुलानयनमप्यग्रयोरङ्गुलं पृथगानीय तत्संयोगेन वियोगेन वा कार्यम्~। तत्र शङ्क्वग्रांशः 
सर्वदा विषुवच्छायासमः~। यतो विषुवच्छायाभुजकत्र्यश्रसमानाकृतित्वं तन्निमित्तानामितरेषामपि~। गणितकर्मणाप्येतज्ज्ञेयम्~। यदि द्वादशाङ्गुलशङ्कुकोटिकस्य विषुवच्छाया भुजा तदावलम्बककोटिकस्य कियतीति~। यदाक्षज्या\renewcommand{\thefootnote}{१}\footnote{यदक्षज्या \textendash\ ख.}विषयं त्रैराशिकं तत्र विषुवच्छायाया लम्बको गुणकारः~। द्वादशसङ्ख्यो हारः~। शङ्क्वग्रानयने पुनरक्षज्याया इष्टशङ्कुः गुणकारः~। लम्बको भागहारः~। फलं शङ्क्वग्रम्~। यदीष्टशङ्कोः शङ्क्वग्रं भुजा तदा द्वादशसङ्ख्यस्य कियतीति~। शङ्क्वग्रसम्बन्धिछायाभुजाङ्गुलानयने त्रैराशिकम्~। तेषु पूर्वत्र विषुवच्छायाया गुणकारो लम्बकः~। द्वितीये भागहारः~। ततस्तयोस्तुल्यत्वान्नष्टयोर्विषुवच्छायाया इष्टशङ्कुः गुणकारः~। द्वादशसङ्ख्यो हारः~। फलं शङ्क्वग्रम्~। अङ्गुलानयने पुनः शङ्क्वग्रस्य द्वादशसङ्ख्यो गुणकारः~। इष्टशङ्कुः हारः~। ततः पूर्वत्र यौ गुणकारभागहारौ तावेवोत्तरत्र भागहारगुणकारौ स्तः~। अतः पूर्वं न्यस्ता विषुवच्छायैव त्रैराशिकत्रयेणापि लभ्यते~। अत्र द्वे एव वा त्रैराशिके स्तः, न त्रीणि सन्ति~। यथा \textendash\ यदि द्वादशसङ्ख्याकोट्या 
विषुवद्भा\renewcommand{\thefootnote}{२}\footnote{द्भात् \textendash\ ख. ग. ङ.} भुजा तदेष्टष्टशङ्कोः कियतीत्याद्यः~। तत्र शङ्क्वग्रं लभ्यते~। यदीष्टशङ्कोः शङ्क्वग्रं भुजा तदा द्वादशसङ्ख्यस्य कियतीति द्वितीयम्~। तत्रापि विषुवच्छायैव लभ्यते, इति भुजायास्तदंशः सौम्य एव सदा, नित्यदक्षिणत्वाच्छङ्क्वग्रस्य छायायाश्च तद्विपरीतदिकत्वात्~। तस्मात् पूर्वापररेखाया उत्तरतो विषुवद्भाङ्गुलविप्रकृष्टायां रेखायामेव मध्याह्नादन्यत्रापि विषुवति तात्कालिकच्छायाग्रं स्पृशति~। अतः सापि विषुवत्संज्ञरेखा~। अर्काग्रायां सत्यां पुनः सौम्येतरयोगोंलर्योर्विषुवद्रेखाया याम्योत्तरयोर्दिशोः तात्कालिकाग्राङ्गुलविप्रकृष्टे प्रदेश एव छायाग्रं स्यात्~। तदुक्तं ब्रह्मसिद्धान्ते\textendash  


\newpage

\begin{quote}
{\qt इष्टच्छायाग्रविषुवन्मध्यमग्राभिधं भवेत्~।}
\end{quote} 

इति~। सूर्यसिद्धान्तेऽप्युक्तम्\textendash 
 
\begin{quote}
{\qt इष्टच्छायाविषुवतोर्मध्यमग्राभिधीयते~।}
\end{quote} 

\noindent इति~। अर्काग्राया विशेष्यस्य ज्यात्वात् स्त्रीत्वम्~। ततः तत्सम्बन्धिनी छायाभुजाप्यग्राभिधीयते इत्याशयः~। सा प्रतिक्षणं भिन्ना~। तद्भेदकारणं द्विविधं \textendash\ छायाकर्णाङ्गुलभेदोऽग्रज्याभेदश्च~। अर्काग्राया अपि कारणद्वयं विद्यते \textendash\ अपक्रमो लम्बकश्च~। अतोऽङ्गुलात्मिकाया अग्राया भेदकारणं त्रिविधं \textendash\ रविभुजाज्या, लम्बकज्या, छायाकर्णश्च~। ननु छायाकर्ण एव क्षितिजासत्तिविंप्रकर्षवशाज्जायमानयोर्वृद्धिह्रासयोर्हेतुः, न छाया नापि कोटिः, न च नतकालः\renewcommand{\thefootnote}{१}\footnote{कालः \textendash\ ख. ग.}, इति कथं निश्चीयते~। 
उच्यते \textendash\ यदि त्रिज्यातुल्यविष्कम्भार्धे भगोले इयत्यश्छायाकर्णकोटिभुजार्काग्रादिलिप्ताः, तदा छायाकर्णाङ्गुलतुल्यविष्कम्भार्धे कियन्ति 
छायाद्यङ्गुलानीति ह्यत्र त्रैराशिकम्~। तत्र भगोलगतानां छायादीनामेव नतभेदाद्भेदो नार्काग्रायाः~। सा हि एकस्मिन् भूवृत्ते
पूर्वापरदिग्गतानां नतभेदेऽपि समैव तात्कालिकी~। तदङ्गुलानयनत्रैराशिके सर्वत्र सदापि व्यासार्धमेव भागहारः~। तत्रेच्छाभूतस्य छायाकर्णस्यैव नतवशाद्भेदः~। ततस्तद्वशादेव तदिच्छाफलस्य अर्काग्राङ्गुलसङ्ख्याया भेदः यानि पुनरग्रा\renewcommand{\thefootnote}{२}\footnote{त्रा \textendash\ ङ.}ङ्गुलविषयाणि त्रैराशिकान्तराणि (तानि) अल्पमहागोलगतछायादीच्छाप्रमाणानि~। उभयगोलगतानां छायादीच्छाप्रमाणानामेककालिको मिथः सम्बन्धः~। तत्कर्णयोरिहैव~। ततः कर्णप्रतिनिधित्वेनैवाल्पगोलगतानां छायादीनां\renewcommand{\thefootnote}{३}\footnote{छायाङ्गुलादीनां \textendash\ क.} गुणकारत्वं महाक्षेत्रगतानां भागहारत्वं चेति छायाकर्णाङ्गुलवृद्धिह्रासवदेव दिनावयवेऽष्वङ्गुलात्मिकाया अर्काग्राया अपि वृद्धिह्रासौ~। यद्येतावद्भिश्छायाङ्गुलैरेतावदग्राङ्गुलं लब्धं, तदैतावद्भिः कियदिति न त्रैराशिकं कर्तुं युक्तम्~। औपाधिकत्वात् तत्सम्बन्धस्य~। छाया- 

\newpage

\noindent कर्णाग्राङ्गुलयोरेवानौपाधिकः सम्बन्धः~। यः पुनरपक्रमवशात् भेदः, स चायनसन्धितोऽन्यत्र सवितरि दिगवगमने विशेषं प्रदर्शंयितुमुक्तः\textendash

\begin{quote}
{\qt भेदात् पूर्वापरक्रान्त्योः छायाकर्णाङ्गुलाहतात्~। \\
 लम्बकाप्तं पूर्वबिन्दोर्नीत्वा कार्योऽत्र सोऽयनात्~॥} 
\end{quote}

\noindent इति~। अपि च छायावृद्धिह्रासवत् तद्भुजाकोटिवृद्धिह्रासौ नैव युज्येते~। यदि छायावशात् तयोर्वृद्धिह्रासौ, तर्हि कालभेदेऽप्येकदिङ्मुखी स्यात् छाया~। दृश्यते च प्रतिक्षणं भिन्नदिक्कैव, यतो मध्याह्ने दक्षिणोत्तरायता अन्यदा दिगन्तराभिमुखी च~। एतत्सर्वं छायाकर्णाग्रैः त्रैराशिकभुजा कोट्यादिन्यायज्ञैर्ज्ञातुं शक्यम्~। समशङ्कुन्यायेन च सेत्स्यतीति~॥~३०~॥\\

\indent तत्सारभूतसमशङ्क्वानयनमाह\textendash

\begin{quote}
{\ab सा विषुवज्ज्योना चेत् विषुवदुदग्लम्बकेन सङ्गुणिता~। \\
 विषुवज्ज्यया विभक्ता लब्धः पूर्वापरे शङ्कुः~॥~३१~॥} 
\end{quote}

\indent इति~। सा अपक्रमजीवा विषुवदुदग् उदग्गोलगा\renewcommand{\thefootnote}{१}\footnote{विषुवदुदग्गोलगता \textendash\ ग. घ. ङ.} विषुवज्ज्योना चेदेव रवेः ग्रहाणां भानां वा सममण्डलप्रवेशः सम्भवति~। इति दक्षिणापथे उत्तरगोलेऽपि सममण्डलप्रवेशस्य कादाचित्कत्वात् तदानयनविषयः प्रदर्शितः~। लम्बकेन सङ्गुणिता सार्काग्रैव नापक्रमज्या विषुवज्ज्यया विभक्ता कार्या~। तत्र लब्धः\renewcommand{\thefootnote}{२}\footnote{लब्धा \textendash\ ख. ग.} पूर्वापरे शङ्कुः~। पूर्वापरमण्डलस्थः शङ्कुः सममण्डलशङ्कुरिति यावत्~। अत्रापि सममण्डलप्रवेशकालभवा स्वाहोरात्रेष्टज्यैव कर्णः~। सममण्डलशङ्कुः कोटिः ; तत्कालशङ्क्वग्रमेव भुजा~। तच्च अर्काग्रतुल्यं, समपूर्वापरसूत्रस्थत्वाच्छङ्कोः~। पूर्वापरास्तोदयसूत्रान्तरालतुल्या ह्यग्रज्या~। शङ्क्वग्रे च सममण्डलप्रवेशे तत्तुल्यमेवेति तयोः साम्यम्~। अत्र सममण्दलशङ्कुमूले सूत्रस्यैकमग्रं बद्ध्वा उन्मण्डलास्तोदयसूत्रस्वाहोरात्रेष्टज्यासम्पातेऽन्यदग्रं बध्नीयात्~। तदा द्विधा भिन्नमिदं त्र्यश्र\renewcommand{\thefootnote}{३}\footnote{भिन्नं त्र्यश्र \textendash\ ग.}क्षेत्रम्~। निरक्षदेशे तेषां 

\newpage

\noindent कोटिकर्णयोः साम्यात् भुजाशून्यत्वाच्च\renewcommand{\thefootnote}{१}\footnote{शून्यतामेति \textendash\ ग. ङ.}, त्र्यश्रोऽपि शून्यतामेति~। ततः क्रमेण अक्षावलम्बकभुजाकोटिवृद्धिह्रासानुरूपमेतत् त्र्यश्रत्रयं प्रतिदेशं नानाकारमपि एकस्मिन् देशे तुल्याकारमेव, यतस्तेषां बाहुसक्ता वृद्धिर्ध्रुवोच्छ्रयानुसारिणी~। त्रिष्वपि महतः क्षेत्रस्यार्काग्रा भुजा, समशङ्कुः कोटिः, स्वाहोरात्रेष्टज्यैव कर्णश्च~। अधःखण्डस्येष्टापक्रमज्या कोटिः, क्षितिज्या भुजा, अर्काग्रा कर्णः~। ऊर्ध्वखण्डस्य तु समशङ्कुः कर्णः, इष्टापक्रमज्यां भ्रुजा, 
उन्मण्डलोर्ध्वगतस्वाहोरात्रेष्टज्यार्धात्मिका कोटिः~। एवमिदं त्र्यश्रत्रयम्~। तत्र भुजाकोटिकर्णेष्वेकेन ज्ञातेन
अक्षावलम्बकव्यासार्धैस्त्रैराशिकेनैवान्यानयनं समानमेव~। तत्र अर्काग्रज्यया समशङ्क्वानयन एव त्रैराशिकम्~। अक्षज्यातुल्यया भुजया लम्बकतुल्या कोटिर्लभ्यते~। तदग्रज्यया कियतीति~। इत्यादिकं सर्वं तत्क्षेत्रप्रदर्शनेनैव सिद्धम्~। एवं बहुधा समशङ्क्वानयनं स्यात्~। अत एव भास्करोऽप्याह\textendash 

\begin{quote}
{\qt पलज्योनामुदक्क्रान्तिं विष्कम्भार्धहतां हरेत्~। \\
समपूर्वापरः शङ्कुर्लब्धार्कस्य पलज्यया~॥} 
\end{quote}
 
\noindent इति~। तत्र भुजया कर्ण आनीयते इति भुजात्मिका पलज्याप्रमाणम्~। कर्णात्मकं व्यासार्धं च फलम्~। अत एव विष्कम्भार्धगुणनं पलज्याहरणं चापक्रमस्य युज्यते~। एतत्त्रयमपि पूर्वप्रदर्शितत्र्यश्रोर्ध्वखण्डगतम्~। सममण्डलस्थेऽर्के छायाकर्णानयनमपि बहुधा कार्यम्~। यदि समशङ्कोः व्यासार्धं कर्णः तदा द्वादशाङ्गुलशङ्कोः कियानिति हि तत्र त्रैराशिकम्~। द्वादशहताया वा अक्षज्यायाः क्रान्तिज्याप्तः सममण्डलगेऽर्के छायाकर्णः~। यतो विष्कम्भार्धाक्षज्यांशाः क्रान्तिज्यातुल्याः समशङ्कुत्वं लभन्ते~। यद्वेमे व्यस्तत्रैराशिके~। यदि व्यासार्धतुल्ये शङ्कौ द्वादशसङ्ख्या छायाकर्णः तदा एतावति शङ्कौ कियान् छायाकर्ण इति~। अस्य त्रैराशिकस्य व्यस्तत्वात् इच्छाराशिः भागहारः~। प्रमाणराशिः गुणकारः~। एवमपि द्वादशघ्नात् 

\newpage

\noindent व्यासार्धाच्छङ्कुनाप्तश्छायाकर्णः~। यद्यक्षज्यातुल्य उदगपक्रमे द्वादशसङ्ख्यः सममण्डलछायाकर्णः, तदैतावत्यपक्रमे कियानिति द्वितीयम्~। इहेष्टापक्रमस्येच्छात्वात् भागहारत्वम्~। अक्षज्यायाः प्रमाणत्वात् गुणकारत्वं च~। द्वादशसङ्ख्यं प्रमाणफलं गुण्यम्~। यद्वा विषुवच्छायाहताल्लम्बकादक्षोनोदक्क्रान्तिज्याप्तः सममण्डलछायाकर्णः~। कुतः, द्वादशहताक्षज्याया विषुवद्भाघ्नलम्बकज्यायाश्च सङ्ख्यासाम्यात्~। कुतः सङ्ख्यासाम्यम्~। महाक्षेत्रगतभुजाया अल्पक्षेत्रगतकोट्याश्च संवर्गोऽक्षज्याद्वादशाभ्यासः~। महाक्षेत्रगतकोट्या अल्पक्षेत्रगतभुजायाश्च घातो विषुवज्ज्यालम्बकवधः~। अतः सङ्ख्यासाम्यम्~। यद्वा यदि द्वादशकोटिकस्य विषुवद्भा भुजा 
तदावलम्बककोटिकस्य कियती इति~। यदक्षज्यानयने त्रैराशिकं तत्र विषुवद्भालम्बकाभ्यासात् द्वादशभिरक्षज्या लभ्यते~। तस्यां पुनर्द्वादशभिरेव हतायां स एव घातः स्यात् ; यतो द्वादशभिरात्मलाभोऽभूद् इत्युभयोर्घातयोस्तुल्यत्वम्~। मध्याह्नछायाकर्णौ प्रमाय ज्ञात्वा ताभ्यामपि सममण्डलकर्ण आनेतुं शक्यः~। कथम्, यदि पूर्वापरविषुवद्रेखान्तरालगं छायाग्रं\renewcommand{\thefootnote}{१}\footnote{रालगछायाग्रं \textendash\ ख. ग. घ. ङ.} तर्ह्येव तस्मिन्नहनि भास्वतः सममण्डलप्रवेशः स्यात्, तदा विषुवच्छायाया मध्याह्नछायां त्यक्त्वा शिष्टमग्राङ्गुलं ज्ञेयम्~। यद्यनेन मध्याह्नाग्रया तात्कालिकछायाकर्णो लभ्यते तदा विषुवच्छायातुल्यया अग्रया कियानिति~। उक्तं च सूर्यसिद्धान्ते\textendash  
\begin{quote}
{\qt लम्बाक्षजीवे विषुवच्छायाद्वादशसङ्गुणे~। \\
क्रान्तिज्याप्तौ तु तौ कर्णौ सममण्डलगे रवौ~॥ \\
सौम्याक्षोना यदा क्रान्तिः स्यात् तदा द्युदलश्रवः~। \\
विषुवच्छाययाभ्यस्तः कर्णो मध्याग्रयोद्धृतः~॥  }
\end{quote}

\noindent इति~। सममण्डलशङ्कुनार्कस्फुटानयनमप्यनयैव युक्त्या सेत्स्यति~। तदप्याह भास्करः\textendash  

\newpage

\begin{quote}
{\qt छायाविधानसम्प्राप्तः शङ्कुः क्षुण्णः पलज्यया~। \\
क्रान्त्या परमया भक्तो लब्धजीवा कलाधनुः\renewcommand{\thefootnote}{१}\footnote{नः \textendash\ ख. ग. ङ.}~॥\\
तिग्मांशुर्मण्डलार्धाच्च परिशुद्धोऽभिधीयते~। \\
सममण्डलदिङ्मार्गशङ्कुछायाप्रसाधितः~॥} 
\end{quote} 

\noindent इति~। समशङ्क्वानयनविपरीतकर्मैवैतत्~। तत्र अपक्रमविषयं समशङ्कुविषयं च त्रैराशिकमेकीकृत्य समशङ्क्वानयनस्य वैपरीत्यमस्य~। तत्र पूर्वत्र व्यासार्धं भागहारः~। उत्तरत्र गुणकारः~। ततस्तयोस्तुल्यत्वान्नष्टयोरिष्टभुजज्यायाः परमापक्रमो गुणकारः~। अक्षज्या भागहारः~। फलं समशङ्कुरिति~। अत एवाक्षज्याहतः समशङ्कुर्व्यासार्धाप्तो भानुभुजाज्यापि स्यात्~। एतद्द्वारा सममण्डलप्रवेशनतानयनमपि सेत्स्यति~। तद्यथा \textendash\ उन्मण्डलात्प्रभृति सममण्डलान्ता स्वाहोरात्रेष्टज्या समशङ्कुकर्णस्य कोटिरित्युक्तम्~। तद्भुजा च अपक्रमज्या~। तस्मादपक्रमज्यां लम्बकेन हत्वा क्षितिज्यया हृत्वा लब्धा तत्कोटिः~। तस्याः स्वाहोरात्रार्धस्य च वर्गान्तरमूलं दक्षिणोत्तरसममण्डलान्तरालगतस्वाहोरात्रचापस्य ज्या~। सैव तदानीं महाछायापि~। अन्यदा छायाकर्णस्य 
कोटिरेव सवितृवियदन्तरस्वाहोरात्रचापज्या~। रविभुजया अपक्रमज्यानयने व्यासार्धं भागहारः~। परमापक्रमो गुणकारः~। पुनस्तया 
कोट्यानयने लम्बको गुणकारः~। अक्षज्या भागहारः~। कोटिज्यायाः स्वाहोरात्रवृत्तगतत्वात् तद्भ्रमणकालानयनं स्वाहोरात्रप्रमितयैव कोट्या कार्यम् इति~। तत्परिणामे व्यासार्धं गुणकारः~। स्वाहोरात्रार्धं भागहारः~। ततस्तयोस्तुल्यत्वान्नष्टयोर्भानुभुजाज्याया लम्बकपरमापक्रमघातो गुणकारः~। अक्षज्यास्वाहोरात्रार्धघातो भागहारः~। फलम् उन्मण्डलसममण्डलान्तरकालज्या~। तत्कोटिः सवितृवियदन्तरकालज्या~। तच्चापं सममण्डलनतासवः~। तच्चाह भास्करः\textendash  

\newpage

\begin{quote} 
{\qt भानोर्भुजामभिहतां परमापमेन \\
 द्युव्यासभेदभजिताप्तहताक्षकोटिः~। \\
 अक्षज्ययाप्तकृतिशुद्धकृतेस्त्रिमौर्व्या \\
मूलस्य काष्ठमसवो गगनावधेर्वा~॥} 
\end{quote}
इति~॥~३१~॥ \\

\indent परशङ्कुनेष्टशङ्क्वानयनाय तं प्रदर्शयति\textendash 
\begin{quote}
{\ab क्षितिजादुन्नतभागानां या ज्या सा परो भवेच्छङ्कुः~। \\
 मध्यान्नतभागज्या छाया शङ्कोस्तु तस्यैव~॥~३२~॥} 
\end{quote}
 
\indent इति~। क्षितिजादुन्नतभागानां क्षितिजदक्षिणोत्तरमण्डलस्वस्तिकात् प्रभृति स्वाहोरात्रोर्ध्वप्रदेशान्तं यावद्\renewcommand{\thefootnote}{१}\footnote{वद्द \textendash\ ग. घ. ङ.} ये दक्षिणोत्तरमण्डलगता
भागाः, त इह क्षितिजादुन्नतभागा विवक्षिताः~। तेषां या ज्या सा तस्मिन्नहनि परः शङ्कुः~। खमध्यात् दक्षिणोत्तरमण्डलमार्गेण स्वाहोरात्रान्तं ये नतभागाः तज्ज्यैव तस्य शङ्कोश्छाया च~। कथं पुनः तच्छङ्क्वानयनम्~। अक्षापक्रमघातं द्युज्यालम्बकघातं\renewcommand{\thefootnote}{२}\footnote{ते सं \textendash\ ग.} च संयोज्य
त्रिज्य\renewcommand{\thefootnote}{३}\footnote{ज्याप्त \textendash\ क., ज्ययाप्त \textendash\ ख. ग. घ.}याहृत्वाप्तमुत्तरगोले 
परः शङ्कुः~। वियोज्य त्रिज्यया हृत्वाप्तं दक्षिणगोले~। एतच्च खण्डज्यानयनयुक्तिसिद्धम्~। अनेन पुनस्तद्दिने शङ्क्वानयनमेवम्~। यदि 
दिनार्धकालज्यया परः शङ्कुर्लभ्यते, तदा दिनगतगन्तव्यज्यया कियानिति~। ते ज्ये चाह सूर्यसिद्धान्ते\textendash 

\begin{quote} 
{\qt त्रिज्योदक्चरजा युक्ता \renewcommand{\thefootnote}{*}\footnote{'याम्यायां तद्विवर्जिता' इति मुद्रितपाठः}याम्याया\renewcommand{\thefootnote}{४}\footnote{याम्यया \textendash\ क. ग.} तु विवर्जिता~। अन्त्या}
\end{quote}
 
\noindent इति~। इयं दिनार्धकालज्या~। सैव नतोत्क्रमज्योना अभीष्टकालज्या~। ताभ्यां तद्दिनार्धछायाकर्णेन च इष्टछायाकर्णानयनमपि व्यस्तत्रैराशिकेनानेयम्~। अन्त्यया इयान् कर्णो लभ्यते, इष्टकालज्यया कियानिति~। तत्रास्य व्यस्तत्रैराशिकत्वादिच्छैव भागहारः~। तस्मात्
तद्दिनमध्याह्नकर्णस्य

\newpage

\noindent अन्त्या गुणकारः~। इष्टकालज्या भागहारः~। कुतः पुनर्व्यस्तत्रैराशिके इच्छाराशिर्भागहारः~। इच्छावृद्धौ फलस्य ह्रासाद्, इच्छाह्रासे फलस्यवृद्धेश्च\renewcommand{\thefootnote}{१}\footnote{फलवृद्धेश्च \textendash\ ङ.}~। तथाचोक्तम्\textendash 
 
\begin{quote} 
{\qt व्यस्तत्रैराशिकफलमिच्छाभक्तः प्रमाणफलघातः~।}
\end{quote} 

\noindent इति~। क्षितिज्या\renewcommand{\thefootnote}{२}\footnote{जा \textendash\ क.}द्युदलसमासेन विश्लेषेण वा मध्याह्नकर्णं हत्वा स्वाहोरात्रेष्टज्यया हरेत्~। तत्रापि इष्टकालछायाकर्णो लभ्यते~। सदापि वा त्रिज्याविषुवत्कर्णघातात् स्वाहोरात्रेष्टज्यया इष्टकर्ण आप्यते~। तेन यदा कदाचिज्ज्ञातं छायाकर्णं वा तत्सम्बन्धिस्वाहोरात्रेष्टज्यया क्षितिजोर्ध्वगतया हत्वा तस्मिन्नेव भूवृत्ते सर्वत्रापि सदापि क्षितिजोर्ध्वगतस्वाहोरात्रज्यया इष्टकालछायाकर्णो लभ्यः ; इत्यादिकं परग्रहणस्य त्रैराशिकसूचनात् सिद्धम्~। तच्छायाविषयमपि शङ्कुनिरपेक्षं कर्म कर्तुं शक्यम् इत्युत्तरार्धेनोच्यते~। तस्य शङ्कोश्छायापि मध्यान्नतभागज्यैव~। यद्यपि ग्रहबिम्बघनमध्यान्ता खमध्यात् प्रवृत्तार्धज्यैव सर्वत्र छायादृङ्मडलानुसारिणी, क्षितिजात् प्रवृत्ता च शङ्कुः तथापि दक्षिणोत्तरमण्डलादन्यत्र नतोन्नतभागानां विविच्य ग्रहणं दुष्करमिति तज्ज्यात्वेनानेतुं शक्यते~। मध्याह्ने तु अक्षापक्रमचापसंयोगवियोगाभ्यां याम्योदग्गोलयोर्नतभागा ग्राह्याः~। लम्बकापक्रमवियोगयोगाभ्यां चोन्नताः~। नवत्यधिकानां नवतिविशोधनं च न्यायसिद्धम्~। अत्रापि अक्षापक्रमज्ययोर्द्युज्या\renewcommand{\thefootnote}{३}\footnote{मयोर्द्युज्या \textendash\ ग. घ. ङ.}लम्बकहतयोस्त्रियेयाहृतयोः संयोगे वियोगे च सूक्ष्मता स्यात्~। अत्रैव विदिक्छायाया अपि सङ्गतिः~। पूर्वापरायताया दक्षिणोत्तरायतायाश्च प्रदर्शितत्वात्~। सा स्वभुजाकोट्योः साम्यादेव सुगमा इति भावः~। तदानयनमविशेषकर्मणाह भास्करः~। तन्न चारु, प्रकारान्तराणां विद्यमानत्वात्~। गत्यन्तराभाव एव ह्यविशेषणं युक्तम्~। कोणशङ्कुछायाविषयेषु बहुषु उपायेषु विद्यमानेषु क्वचिदिदमेव सूत्रमावृत्त्या योज्यम्~। कथम्, अपरः 

\newpage

\noindent दिग्विदिग्गतेषु शङ्कुषु योऽनुक्तः कोणगतः शङ्कुः सोऽपि क्षितिजादुन्नतभागानां ज्यैव\renewcommand{\thefootnote}{१}\footnote{भागज्यैव, यतः कोणमण्डलेऽपि \textendash\ ख. घ. ङ.}~। तच्छायापि स्वमध्यान्नतभागज्यैव, यतः कोणमण्डलेऽपि स्वाहोरात्रस्पृष्टप्रदेशावधिकाः क्षितिजादुन्नतभागाः\renewcommand{\thefootnote}{२}\footnote{भागाश्च विविच्य \textendash\ ग. घ. ङ.}
खमध्यान्नतभागाश्च विविच्यावगन्तुं शक्याः~। तत्रापि तज्ज्ये एव शङ्कुछाये भवेतामिति~। तत्र 
दक्षिणोत्तरमण्डलेऽक्षापक्रमावलग्बकादिसंयोगवियोगमात्रेणैव सिद्ध्येयुः~। इह तु 
ततस्त्रैराशिकसिद्धकोणमण्डलगतकर्णभूतज्याद्वयचापसंयोगवियोगतस्तत्सिद्धिरिति विशेषः~। तत्रेमे श्लोकाः\textendash 
 
\begin{quote}
{\qt द्विघ्नो यः पलभावर्गः शङ्कुवर्गयुतश्च सः~। \\
 तन्मूलगुणहाराभ्यां विषुवे स्यात् विदिक्प्रभा~॥ \\
 व्यासार्धात् द्वादशघ्नात्तु तच्छङ्कुर्हारकोद्धृतः~। \\
 अक्षज्यया हरेच्छायां हत्वेष्टापक्रमज्यया~॥ \\
 तद्भाधनुर्भिदैक्यज्याकोणभागोलयोः क्रमात्~। \\
 लब्धस्य विषुवत्कोणशङ्कोर्योगान्तरान्नरः~॥}
\end{quote}

\noindent इति~। अत्रेदं त्रैराशिकम्~। यद्यक्षज्यातुल्यया भुजया घटिकाकोणमण्डलसम्पातात् खमध्यावधिका ज्या कर्णभूता लभ्यते, तदेष्टापक्रमज्यया भुजया कियतीति~। घटिकाकोणमण्डलसम्पातात्प्रभृति द्युवृत्तकोणमण्डलसम्पातावधिका ज्या लभ्यते~। यथापक्रमानयनमपि त्रैराशिकसिद्धत्वान्नोक्तं\renewcommand{\thefootnote}{३}\footnote{न्नोक्तं, तत्र मण्डलद्वयसम्पातात् \textendash\ ख. ग. घ. ङ.}, तथैवेदमपि नोक्तम्~। तत्र मण्डलद्वयसम्पातात्प्रभृति कर्णानुसारिमण्डलज्यया तदग्रस्येतरमण्डलाद्विप्रकर्षो भुजारूपो गुणः साध्यते~। इह तु भुजया कर्णरूपो गुणः, इत्येव केवलं विशेषः~। विषुवत्स्थे भास्वति च कोणमण्डलगतेऽग्राङ्गुलाभावात् भुजाकोट्योरुभयोरपि पलभातुल्यत्वात् पलभाङ्गुलवर्गात् द्विगुणान्मूलमेव छायाङ्गुलमपि~। ततस्तच्छङ्कुकर्णैस्त्रैराशिकेन कोणमण्डलगतनतोन्नतज्ये छायाशङ्कुरूपे सेत्स्यतः~। 
तच्छायातुल्य एव अक्षतुल्यस्यापक्रमस्य भुजात्मकस्य कर्णः~। तत 

\newpage

\noindent इष्टापक्रमस्य कियानिति कोणघटिकामण्डलसम्पातात् द्युवृत्तकोणमण्डलसम्पातावधिका ज्या लभ्यते~। तच्चापं कोणमण्डलगतं विषुवदर्कान्तरालम्~। अक्षसम्बन्धिनः कर्णस्य चापं कोणमण्डलगतं विषुवत्खमध्यान्तरालम्~। शिष्टं पूर्ववत्~। अनयैवोपपत्त्या इष्टदिगभिमुखशङ्कुच्छाये अपि ज्ञेये~। तत्रोद्देशकेन क्षितिजगतं पूर्वापरस्वस्तिकदृङ्मण्डलान्तरालचापं याम्योदक्स्वस्तिकदृङ्मण्डलविवरं वा उद्दिश्य शङ्कुछायादिकं प्रष्टव्यम्~। तयोश्च दृङ्मण्डलाग्रे स्वस्तिकाभ्यां प्रवृत्ते ये ज्ये तयोर्दक्षिणोत्तरायता हि त्रिज्यावृत्तपरिणता छाया भुजा~। अन्या च कोटिः~। तत इदं त्रैराशिकम् \textendash\ यद्यनया भुजया इयती कोटिर्लभ्यते तदा विषुवच्छायातुल्यया कियतीति~। अङ्गुलात्मिका कोटिर्लभ्यते~। तद्विषुवच्छायावर्गयोगमूलं च छायाङ्गुलम्~। तच्छङ्कुवर्गयोगमूलं स्ववृत्तविष्कम्भार्धाख्यः कर्णः~। शेषं पूर्ववत्~। तदपि श्लोकैर्निबद्ध्यते\textendash  
\begin{quote}
{\qt दृङ्मण्डल\renewcommand{\thefootnote}{१}\footnote{मण्डल \textendash\ ख. ग. घ. ङ.}याम्योदक्स्वस्तिकविवरज्यया हता पलभा~। \\
 इतरस्वस्तिकविवरज्याप्ता कोटिर्भुजा पलभा~॥\\
 कर्णो विषुवति तद्दिक्प्रभाकृतियुतैश्च तच्छङ्कोः~। \\
 मूलं छायाकर्णो युक्त्या तच्छङ्कुदृग्गुणौ कुर्यात्~॥ \\
 इष्टापक्रमगुणितं दृग्गुणमक्षज्यया हरेल्लब्धम्~। \\
 इष्टापक्रमवत् स्यादक्षज्यावत् स दृग्गुणस्ताभ्याम्~॥ \\
 छायानरौ दिनार्धवदिष्टाशायां नयेत् सदैव भुवि~। \\
 तत्कोटिभ्यां लम्बकदिनगुणकार्यं च कार्यमिह~॥}
\end{quote}

\noindent इति~। घटिकाकोणमण्डलसमतिरश्चीनमण्डलकल्पनायामपि तदिष्टदिगुन्नत्यवनत्यादिभिरप्येतद्विषयाणि कर्मान्तराणि सिद्ध्येयुः~।
द्वयोर्द्वयोः मण्डलयोः समतिरश्चीनत्वं तत्सम्पाताभ्यां चक्रपादान्तरितस्पृशो वृत्तस्य स्यात्~। तथा सति मण्डलद्वयभिन्नगोलकपालमध्यस्पृक्त्वमपि तिरश्चीन- 

\newpage

\noindent मण्डलस्य स्यादितीह ध्रुवौ विदिगन्तरक्षितिजप्रदेशौ च स्पृशति~। तत्कर्माण्यत्रोक्तयुक्तिसिद्धत्वात् ग्रन्थविस्तरभयाच्च मया नोच्यन्ते~।
यद्वा दृक्क्षेपमण्डलगत्या यावत्यपक्रममण्डलस्योन्नतिः तद्भागा वा क्षितिजादुन्नतभागाः, तेषां जीवा दृक्क्षेपकोटिः~। सा तत्कालेऽपक्रममण्डलपरिधिप्रदेशशङ्कुनां परः शङ्कुः~। तेन फलेन व्यासार्धकर्णेन चादित्यलग्नविवरज्यया च अभीष्टप्रदेशशङ्कुरानेयः~। तदुक्तं भास्करेण\textendash 

\begin{quote}
{\qt आदित्यलग्नविवरांशगुणेन हत्वा \\
तत्कालमध्यपरिनिष्ठितलम्बकाख्यम्~। \\
विष्कम्भभेदभजितस्फुटशङ्कुरुक्तः \\
तद्गोलभेदकृतिशुद्धपदं प्रभा वा~॥}
\end{quote}

\noindent इति~। तत्कालमध्यपरिनिष्ठितलम्बकं च स एवाह\textendash  
\begin{quote}
{\qt रविकक्ष्यामध्यज्यात्रिज्याकृत्योर्विशेषपदम्~। \\
तत्कालमध्यजातो गोलज्ञैर्लम्बकः कथितः~॥} 
\end{quote}
 
\noindent इति~। तत्र मध्यज्याशब्देन दृक्क्षेपज्या विवक्षिता, दृश्यार्धमध्यगत्वाद्\renewcommand{\thefootnote}{१}\footnote{गतत्वात् \textendash\ ख. ङ.} दृक्क्षेपलग्नस्य~। अत एव तस्य मध्यलग्नत्वमप्याह सूर्यसिद्धान्ते\textendash 

\begin{quote}
{\qt मध्यलग्नसमे भानौ हरिजस्य न सम्भवः~।} 
\end{quote}

\noindent इति~। यतो दृक्क्षेपलग्नसमे भानौ लम्बनस्य शून्यता तस्मात् दृक्क्षेपलग्नमिह मध्यलग्नमित्युक्तम्~। तस्मात् दृङ्क्षेपमण्डलगतोन्नतज्या
पर\renewcommand{\thefootnote}{२}\footnote{परः \textendash\ ग.}शङ्कुः~। दृक्क्षेपमण्डलमध्यात् तदवधिका या सा नतज्या, सैव तस्य परशङ्कोः छाया~। एतद्विपरीतकर्मणा शङ्कुना लग्नानयनमपि कार्यम्~। उक्तं च तन्मया तन्त्रसङ्ग्रहे\textendash  

\begin{quote}
{\qt मध्याह्नाद्वा नतप्राणा निशीथाद्वोन्नतासवः~। \\
 ये\renewcommand{\thefootnote}{३}\footnote{एतद् \textendash\ ग.} तद्बाणोनितास्त्रिज्याश्चरज्याढ्या नता यदि~॥ \\
उन्नताश्चेच्चरज्योना गोले याम्ये विपर्ययात्~। \\
द्युज्यालम्बकघातघ्ना त्रिज्याप्ता च पुनर्हृता~॥} 
\end{quote}

\newpage

\begin{quote}
{\qt कोट्या दृक्क्षेपजीवाया लब्धचापं रवौ क्षिपेत्~। \\
 तल्लग्नं प्राक्कपाले स्यान्निशि चेत्तद्विवर्जितम्~॥ \\
 प्रत्यग्गतेऽस्तलग्नं स्यात् व्यस्तमेव दिवानिशोः~॥ \\
 प्राक्पश्चाल्लग्नयोर्मध्यं लग्नं दृक्क्षेपसंज्ञितम्~॥} 
\end{quote}
\noindent इति~॥~३२~॥\\

\indent अथ तयोरानयनमाह\textendash 
\begin{quote}
{\ab मध्यज्योदयजीवासंवर्गे व्यासदलहृते यत् स्यात्~। \\
 तन्मध्यज्याकृत्योर्विशेषमूलं स्वदृक्क्षेपः~॥~३३~॥} 
\end{quote}

\indent इति~। अत्र प्रथमं तत्क्षेत्रं प्रदर्श्यते~। प्राक्पश्चाल्लग्नान्तं यत् सूत्रं यच्च क्षितिजदृक्क्षेपमण्डलसम्पातावधिकं ते च मिथः समतिरश्चीने क्रमेणापमण्डलदृक्क्षेपव्यासौ~। लग्ना(द)ध ऊर्ध्वमपि यद्वृत्तं\renewcommand{\thefootnote}{१}\footnote{वृत्तं दृक्क्षेप \textendash\ ग. घ. ङ.} तत्दृक्क्षेपसममण्डलसंज्ञम्, अपमण्डलतद्विवरं दृक्क्षेपाख्यम्~। परोऽत्र दृक्क्षेपः मध्याख्यलग्नजेन स्वज्यानीतेन फलेनेच्छया त्रिज्यया प्रमाणेन क्षितिजमध्यलग्नान्तरगुणेन च आनेयः~। तत्र मध्यज्ययेच्छया दृक्क्षेपदक्षिणोत्तरमण्डलविवरगुणेन दक्षिणोत्तरव्यासार्धकर्णेन च इच्छाफलभूतो मध्यलग्रदृक्क्षेपलग्नविवरगुणः प्रथममानीयते~। मण्डलार्धान्तरितसम्पातवृत्तद्वयविवरे त्रैराशिके सर्वत्रापक्रमन्यायोऽतिदेश्य इति
पूर्वमेवोक्तत्वात् अत्र न किञ्चिदपि वक्तव्यम्~। मध्यज्यापि रविमध्याह्नछायावत् सिद्धा~। सापि माधवोक्तज्याद्वयसंयोगवियोगत्रैराशिकाभ्यामेव सिद्धा~। तत्र याम्यापक्रमाक्षयोः योगः कार्यः~। सौम्यापक्रमाक्षयोर्विश्लेषश्च~। तत्रापि लम्बकमध्यलग्नापक्रमयोरक्षमध्यलग्नस्वाहोरात्रयोश्च घातयोर्योगो वियोगो वा त्रिज्यया हर्तव्यः~। वियोगेऽपक्रमलम्बघाते शिष्टे सौम्ये मध्यज्यादृक्क्षेपज्ये, अन्यदा याम्ये एव~। तत्र क्षितिजे दक्षिणस्वस्तिके
सूत्रस्यैकमग्रं बद्ध्वा ततः प्राक् पश्चाद्वा दृक्क्षेपदिशि दृक्क्षेपक्षितिजयोगादपर\renewcommand{\thefootnote}{२}\footnote{दुप \textendash\ क.}तोऽपि

\newpage

\noindent तावत्यन्तरे क्षितिजे बध्नीयात्~। तदर्धं दक्षिणोत्तरसूत्राग्रस्पृष्टमिह फलज्या~। सा चोदयज्यातुल्या, यतो दृक्क्षेपदक्षिणोत्तरमण्डलयोर्विवरेण तुल्यमस्तोदयलग्नसममण्डलस्य पूर्वापरसममण्डलस्य च विवरम्~। तत्रोदयाग्रज्या दक्षिणोत्तरायततया प्रसिद्धा~। तदनुसारेण याम्योत्तरदृक्क्षेपमण्डलविवरभवा क्षितिजगता पूर्वापरतया प्रतीयेत~। इह तु दृक्क्षेपमण्डलतः प्रवृत्ता दक्षिणोत्तरमण्डलस्वस्तिकाग्रा कल्प्या, यतोऽपक्रममण्डलस्य दृक्क्षेपलग्नात् सर्वेभ्यः स्वावयवेभ्यः उन्नता\renewcommand{\thefootnote}{१}\footnote{ताद्दक्षि \textendash\ ग.}
दक्षिणोत्तरापक्रममण्डलसम्पातान्तार्धज्या ज्ञेया, मध्यलग्नदृक्क्षेपकोटिकस्य मध्यज्याकर्णकस्य त्र्यश्रस्य भुजाया ज्ञेयत्वात्~।
दृक्क्षेपलग्नमध्यलग्नान्तरालज्या हि तद्भुजा~। यादृगिच्छाफलं प्रमाणफलेनापि तादृशा भवितव्यमिति क्षितिजेऽपि दक्षिणोत्तरमण्डलाग्रा कल्प्यते~। मण्डलद्वयान्तरज्योभयथापि तुल्यैवेति तत्तत्फलानुसारेण यथेष्टं कल्प्येत्यपि मण्डलान्तरालन्यायत्वेन प्रदर्शितम्~। अत्र पुनः कोटिमण्डलमपि कल्प्यं, यद्गता कोट्यात्मिका मध्यलग्नदृक्क्षेपज्या~। तच्च मध्यलग्नस्पृष्टं दृक्क्षेपमण्डलात् सर्वत्र समान्तरालम्~। तदन्तरालमिह मध्यज्योदयजीवासंवर्गात् व्यासदलहृतम्~। तदेव लग्नद्वयान्तरालज्यार्धम्~। तत्प्रदर्शनार्थं मध्यलग्ने सूत्रस्यैकमग्रं बद्ध्वा दृक्क्षेपमण्डलापरभागेऽपि अपक्रममण्डले तावत्यन्तरेऽन्यदग्रं बध्नीयात्~। मध्यलग्नस्पृष्टं तदर्धमिह मध्यज्याकर्णस्य भुजा~।
दृक्क्षेपमण्डलक्षितिजसम्पातान्मध्यलग्नदृक्क्षेपलग्नान्तरालचापसमे दक्षिणोत्तरस्वस्तिकासन्ने चिह्नं कृत्वा तन्मण्डलार्धे क्षितिजदृक्क्षेपसम्पाताच्च तावत्यन्तरे दक्षिणोत्तरस्वस्तिकापरभागेऽपि चिह्नं कृत्वा तदुभयस्पृगधऊर्ध्वं मण्डलं बध्नीयात्~। तत् खमध्यादपि लग्नसममण्डले तावत्यन्तरे स्पृशति~। ततो मध्यलग्नान्तार्धज्यामध्यलग्नदृक्क्षेपज्येह कोटि त्वेनानीयते~। इहापमण्डललग्नसममण्डलयोरन्तरालं क्षितिजात् प्रभृति

\newpage

\noindent वर्धमानं कृत्स्नमपि दृक्क्षेपशब्दवाच्यम्~। यथापक्रममण्डलेष्टप्रदेशस्य घटिकामण्डलात् विप्रकर्षोऽपक्रमशब्देनोच्यते, एवमत्रापि लग्नसममण्डलादपक्रममण्डलगतेष्टप्रदेशस्य विप्रकर्षो दृक्क्षेपशब्दवाच्य इति भावः~। अत उक्तं \textendash\ {\qt तन्मध्यज्याकृत्योविंशेषमूलं स्वदृक्क्षेप} इति~। एतद्भुजाकोट्योः मध्यज्याकर्णकत्वान्मध्यज्यामण्डलगते ज्ये एव ते इति तद्वैशद्याय मध्यज्यामण्डलमपीह वन्धनीयम्~। तद्यथा \textendash\ मध्यज्याद्विगुणव्यासं समवृत्तं निर्माय उपरि स्वस्तितकमभितः सर्वेष्वधऊर्ध्वमण्डलेषु मध्यज्याचापान्तरितेषु बध्नीयात्~। तन्मध्यज्यामण्डलम्~। यत्पुनरिह सूत्रमपक्रममण्डले दृक्क्षेपमण्डलात् तुल्यान्तरालं बद्धं तदर्ध\renewcommand{\thefootnote}{१}\footnote{यदर्ध \textendash\ क.}मिहेच्छाफलं भुजारूपं तदिदानीं 
मध्यज्यामण्डलस्पृग् भवति~। अतो मध्यज्यामण्डलजीवा च सा~। मध्यज्यामण्डलं च दृक्क्षेपापक्रममण्डलेऽपक्रमण्डलाद्वहिरेव\renewcommand{\thefootnote}{२}\footnote{दृक्क्षेपमण्डलाद्बहिरेव \textendash\ क, दृक्क्षेपापक्रममण्डलाद्बहिरेव \textendash\ ख. ग. घ. ङ.} स्पृशति, दृक्क्षेपगुणान्मध्यज्याया आधिक्यात्~। तदाधिक्यं च मध्यलग्नाद्दृक्क्षेपलग्नस्योच्छ्रितत्वात्~। दृश्यार्धमध्यगतत्वादेव दृक्क्षेपलग्नस्यापक्रममण्डला\renewcommand{\thefootnote}{३}\footnote{पमण्डला \textendash\ ङ.}वयवान्तरेभ्य उच्छ्रितिः~। अत्र स्वदृक्क्षेपशब्दोक्ता मध्यज्याकर्णकोटिरपिमध्यज्यामण्डले प्रदर्श्या~। मध्यज्यामण्डले यत्र लग्नसममण्डलं स्पृशति तत उभयतस्तुल्यान्तरालं मध्यलग्नबद्धं यत् सूत्रं तस्य मध्यलग्नस्पृष्टार्धमिह कोटिः~। एतत् सर्वं छेद्यकेऽपि प्रदर्श्यम्~। प्रथमं व्यासार्धसूत्रेण क्षितिजवृत्तमालिख्य तन्मध्यमेव मध्यं कृत्वा मध्यज्यासूत्रेणापि वृत्तमालिखेत्~। तत्र पूर्वापररेखां दक्षिणोत्तररेखां चालिख्य पूर्वरेखाग्रवेधाल्लग्नदिशि लग्नाग्राचापतुल्यान्तरात् प्रभृति वृत्तकेन्द्रगामिनीमस्तलग्नगतान्याग्रां रेखामालिख्य तन्मध्यगतमत्स्येन दृक्क्षेपमण्डलव्यासरेखामपि लिखेत्~। पुनः मध्यज्यामण्डलदक्षिणोत्तरसम्पातस्पृक्सूत्रं दृक्क्षेपमण्डलव्यासतः सर्वत्र तुल्यान्तरालं प्रसार्य रेखां कुर्यात्~। पुनरपि मध्यज्यादक्षिणोत्तररेखासम्पातस्पृष्टं लग्नसगमण्डलापक्रममण्डलयोः साधारणात् व्यासार्धात् 

\newpage

\noindent कात्स्नर्येन तुल्यान्तरालं तद्दिगनुसारि सूत्रं प्रसार्यापि क्षितिजान्तां\renewcommand{\thefootnote}{१}\footnote{न्तं \textendash\ ख.} रेखां कुर्यात्~। एवमिह क्षेत्रच्छेदः~। येह लग्नात् समान्तराला रेखा इहोक्तिक्रमाच्चरमा तदर्धमेव मध्यलग्नदृक्क्षेपकोटिः~। अतो दृग्गतिमण्डलव्यासार्धमपि न तदेव~। तस्माद्गोलेऽपि तदनुरूपं मण्डलं बन्धनीयम्~। तच्च दृक्क्षेपलग्नस्पृष्टम्~। तन्मघ्यलग्नस्पृष्टकोटिरेखाया मध्यज्यामण्डलान्तर्गतो यो भागः तदर्धमिह भुजा~। या पुनरितरापि वृत्तकेन्द्रविप्रकृष्टा रेखा तस्या मध्यज्यामण्डलान्तर्गतार्धमिह\renewcommand{\thefootnote}{२}\footnote{तमिह \textendash\ ग. घ. ङ.} कोटिः~। यद्वा चरमरेखाकेन्द्रविप्रकर्षो वा दृक्क्षेपमण्डलव्यासार्धगता कोटिः~। केन्द्रेतररेखाविप्रकर्षो भुजा~। सा च लग्नसममण्डलव्यासार्धगता~। एवमिदमायतचतुरश्रं क्षेत्रम्~। अत्र दृक्क्षेपसम्बन्धिन्यः सर्वोपपत्तयः प्रदर्श्याः~। एवं मध्यलग्नस्य स्वदृक्क्षेपेण
फलेन स्वक्षितिजान्तरालज्यया प्रमाणेन च दृक्क्षेपलग्नद्वयबद्धसूत्रस्यार्धेनेच्छाभूतेन च तात्कालिकपरमदृक्क्षेपो ज्ञेयः इति तत् त्रैराशिकसूचनाय स्वग्रहणं कृतम्~। अयं तु मध्यलग्नस्य स्वदृक्क्षेपः~। कः पुनः तदानीं परमदृक्क्षेप इत्येव पुनरिहावशिष्टा जिज्ञासा~। सा तु त्रैराशिकसूत्रेणापि शमयितव्या~। तच्च सङ्गमग्रामजेन गोलतत्त्वविदा माधवेन प्रदर्शितम्\textendash  

\begin{quote}
{\qt लग्नं त्रिभोनं दृक्क्षेपलग्नं तन्मध्यलग्नयोः~। \\
वर्गीकृत्यान्तरालज्यां मध्यज्यावर्गतस्त्यजेत्~॥ \\
त्रिज्याकृतेश्च तन्मूले क्रमशो गुणहारकौ~। \\
ताभ्यां दृक्क्षेपसंसिद्धिः त्रिज्याया\renewcommand{\thefootnote}{३}\footnote{ज्यया \textendash\ ग. घ. ङ.} जायते सदा~॥ }
\end{quote}

\noindent इति~। अत्र या दृक्क्षेपलग्नमध्यलग्नान्तरालजीवा सैव मध्यज्योदयजीवासंवर्गात् व्यासदलाप्ता~। इतःपरमुभयत्रापि समानं 
कर्म~। तथाहि \textendash\ {\qt तन्मध्यज्याकृत्योर्विशेषमूलं स्वदृक्क्षेप} इति मध्यज्याकर्णकोटिरिह स्वदृक्क्षेपतयोक्ता~। सैव माधवेनापि अन्तरालज्याया मध्यज्यायाश्च वर्गान्तरमूलत्वेन 

\newpage

\noindent प्रदर्शिता~। यत् पुनरिह स्वशब्देन सूचितं त्रैराशिकं तदेव माधवेन विस्पष्टं प्रदर्शितम्~। कथम्~। त्रिज्याकृतेश्च तां वर्गीकृतामन्तरालज्यां व्यासार्धकृतेश्च त्यजेत्~। तन्मूले शिष्टयोर्मूले क्रमशो गुणहारकौ मध्यज्यावर्गतस्त्यक्त्वा मूलीकृतं गुणकारः, त्रिज्याकृतेस्त्यक्त्वा मूलीकृतं हारकः~। ताभ्यां फलप्रमाणाभ्यां इच्छायास्त्रिज्याया दृक्क्षेपसंसिद्धिः~। दृक्क्षेपस्य सम्यक्सिद्धिस्तत्त्वज्ञानमिति यावत्, सदा जायते~। इह फलप्रमाणयोरपि प्रतिक्षणं भेदः स्यात्~। न केवलं क्षेत्रावयवभेदेनैव भेदः~। अपक्रममण्डललग्नसममण्डलयोरन्तरालक्षेत्रस्य प्रतिक्षणं भिद्यमानत्वेऽपि इदं त्रैराशिकं सदापि युक्तियुक्तम्~। इच्छाप्रमाणक्षेत्रयोरुभयोरपि तुल्याकारत्वात्~। तयोर्नानाकारत्व एव हि परस्परं त्रैराशिकं न युज्यते~। इति स्वशब्देनास्य फलत्वप्रदर्शनमुपपन्नमेवेति माधवस्याप्यभिप्रायः~। तत्र त्रिज्याकृतेरन्तरालज्यावर्गं विशोष्य यन्मूलीकृतं क्षितिजमध्यलग्नान्तरालज्यैव सा~। यतः क्षितिजलग्नाद्दृक्क्षेपलग्नं राशित्रयान्तरितम्~। दृक्क्षेपलग्नात् प्राग्लग्नभागे कदाचिन्मध्यलग्नं, कदाचिदस्तलग्नभागे~। तत्रैवं विभागः ; यदोदयज्याया मध्यज्यायाश्च दिक्साम्यं, तदा दृक्क्षेपलग्नात् प्राग्लग्नभागे मध्यलग्नम्~। यदा पुनर्मध्यज्याया अस्तज्याया दिक्साम्यं तदा अस्तलग्नभागे~। सर्वथापि दृक्क्षेपलग्नमध्यलग्नान्तरालचापोनं राशित्रयं मध्यलग्नासन्नक्षितिजलग्नयोरन्तलचापम् इति व्यासार्धवर्गात् तद्विशोधनम्~। तत्रैवं त्रैराशिकम् \textendash\ क्षितिजात् प्रभृति मध्यलग्नान्तज्यया कर्णभूतया मध्यलग्न दृक्क्षेपो दृक्क्षेपत्वेन लभ्यते, तदा लग्नात् प्रभृति दृक्क्षेपलग्नान्त\renewcommand{\thefootnote}{१}\footnote{न्त्य \textendash\ ङ.}ज्यया त्रिज्यया कियानिति परमदृक्क्षेपो लभ्यते~। स च प्रतिक्षणं भिन्नः~। तथाप्यैककालिकाभ्यां मध्पल्लग्नदृक्क्षेपतदुन्नतज्याभ्यां व्यासार्धेन च तात्कालिकः परमदृक्क्षेप आनेय इत्यर्थः~। यत् पुनर्भास्करादिभिरुक्तं \textendash\ {\qt त्रिज्याप्तफलस्य मध्यज्याकर्णस्य च वर्गविश्लेषमूलमेव दृक्क्षेप}

\newpage

\noindent इति, तदयुक्तम्~। यतो दृक्क्षेपलग्नस्य लग्नसममण्डलव्यासार्धाद्विप्रकर्षो दृक्क्षेपः~। विप्रकर्षश्च दृक्क्षेपलग्नव्यासार्धे घनभूमध्यात् प्रभृति
दृक्क्षेपलग्नान्तं क्रमेण वर्धते~। एवं वर्धमानो विप्रकर्षः सममण्डलव्यासार्धसूत्रादपमण्डलाग्रव्यासार्धाग्रस्य यावान् विप्रकर्षः\renewcommand{\thefootnote}{१}\footnote{र्षः~। स एव हि विप्रकर्षः~। स \textendash\ क.}~। स एव परमो दृक्क्षेपः~। इह यः पुनर्मध्यज्याकर्णवर्गात् भुजावर्गं विशोध्य मूलीकृतः कोटिरूपो दृक्क्षेपः, स हि दृक्क्षेपलग्नमध्यलग्नान्तरालशरोनव्यासार्धखण्डाग्रभव एव
विप्रकर्षः~। स च मध्यज्यामण्डलगत एव चरमरेखाकेन्द्रविप्रकर्षः~। यत् पुनर्दृग्गतिमण्डलं तस्य लग्नसममण्डलस्य च विप्रकर्ष एव परमो दृक्क्षेपः~। इहानीतस्वदृक्क्षेपो मध्यलग्नस्पृष्टस्वकोटिमण्डलसममण्डलयोरन्तरालसम एव~। तच्चान्तर्गतं मण्डलं\renewcommand{\thefootnote}{२}\footnote{तमण्डलं \textendash\ ग.}
दृक्क्षेपलग्नाग्रस्पृ\renewcommand{\thefootnote}{३}\footnote{स्पृष्टदृग्गति \textendash\ ङ.}ष्टम्~। दृग्गतिमण्डलात् स्वाहोरात्रवच्च सर्वेषां कोटिमण्डलानामवस्थितिः~। तत्र लग्नसममण्दलात् यावद्विप्रकृष्टं दृक्क्षेपलग्नस्पृष्टं दृग्गतिमण्डलं मध्यलग्नस्पृष्टं स्वदृक्क्षेपकोटिमण्डलं न तावद्विप्रकृष्टम्~। मध्यज्यामण्डलोपरिगतापक्रमचापस्य यः शरः तस्यापि दृक्क्षेपलग्नाग्रव्यासार्धसूत्रवत् तिर्यक्त्त्वान्मध्यलग्नस्पृष्टा दृक्क्षेपलग्नस्य विप्रकृष्टता~। स्वमध्यात् दृक्क्षेप\renewcommand{\thefootnote}{४}\footnote{मध्यदृक्क्षेप \textendash\ ग.}दिगनुसारेण स्वदृक्क्षेपकोटिमण्डलस्य च दृग्गतिमण्डलान्महत्त्वं लग्नसममण्डलासन्नत्वात् स्वदृक्क्षेपकोटिमण्डलस्य~। तस्मात् स्वदृक्क्षेपकोटिमण्डलस्य दृग्गतिमण्डलस्य च यदन्तरालं तावता दृक्क्षेपज्यातोऽल्पत्वं, स्वदृक्क्षेपस्य स्यात्~। तस्मात् सर्वदापि 
स्वदृक्क्षेपज्यातोऽधिकत्वमेव नत्यादिसाधनभूतपरमदृक्क्षेपज्यायाः~। अतिस्पष्टमेवेदम्~। लग्नात्प्रभृत्यपक्रममण्डलपरिधेर्लग्नसममण्डलाद्
विप्रकर्षः क्रमेण वर्धमानो लग्नात् राशित्रयान्तरे दृश्यार्धमध्ये दृक्क्षेपलग्नाख्य एव महान् अवयवान्तरविप्रकर्षात्~। यथायन एव परमापक्रमो महान्~। 
तस्माद्दृक्क्षेपलग्नादधोगताया मध्यज्यामण्डलापक्रममण्डलयोः सम्पातद्वयस्पृष्टाया जीवाया दृक्क्षेपलग्नात् द्रष्टुरूर्ध्वसूत्रासन्नत्वादल्प
एव विप्रकर्षः~। 

\newpage

\noindent मध्यज्यामण्डलापक्रममण्डलस्वदृक्क्षेपकोटिमण्डलसाधारणभूतायाः समस्तज्याया मध्यस्य सममण्डलव्यासार्धस्य च ऋजुतया यद्विवरं मध्यज्यामण्डलगतं तदेवेह मध्यज्यावर्गस्य तद्भुजावर्गस्य च विश्लेषमूलम्~। तच्च दृक्क्षेप\renewcommand{\thefootnote}{१}\footnote{पमण्डलस्य \textendash\ क. ङ.}लग्नस्य सममण्डलव्यासार्धस्य च विवरात् दृक्क्षेपज्याख्यादल्पमेव~। तस्मादत्रानीतस्वदृक्क्षेपो दृक्क्षेपलग्नदृक्क्षेपज्याया अल्प एव स्यादिति शतकृत्वः शपथयामः~। अत एव सूर्यसिद्धान्तेऽप्येवमुक्तम्\textendash 

\begin{quote}
{\qt लग्नं पर्वविनाडीनां कुर्यात् स्वैरुदयासुभिः~। \\
तज्ज्यान्त्यापक्रमज्याघ्ना लम्बज्याप्तोदयाभिधा~॥ \\
तदा लङ्कोदयैर्लग्नं मध्यसंज्ञं यथोदितम्~। \\
तत्क्रान्त्यक्षांशसंयोगो दिक्साम्येऽन्तरमन्यथा~॥ \\
शिष्टान्नतांशास्तन्मौर्वी मध्यज्या साभिधीयते~। \\
मध्योदयज्ययाभ्यस्ता त्रिज्याप्ता वर्गितात् फलात्~॥ \\
मध्यज्यावर्गषिश्लेषाद्दृक्क्षेपः शेषतः पदम्~। \\
तत्त्रिज्यावर्गविश्लेषान्मूलं शङ्कुः स दृग्गतिः~॥ \\
नतांशबाहुकोटिज्ये स्फुटे दृक्क्षेपदृग्गती~॥} 
\end{quote}

\noindent इति~। अत्रापि पूर्वोक्तस्य दृक्क्षेपस्या\renewcommand{\thefootnote}{२}\footnote{पूर्वोक्तदृक्क्षेपस्य \textendash\ ग.}स्फुटत्वख्यापनाय {\qt नतांशबाहुकोटिज्ये स्फुटे दृक्क्षेपदृग्गती} इत्युक्तम्~। खमध्यदृक्क्षेपलग्नस्य दृक्क्षेपमण्डलगता ये नतभागाः तज्ज्यैव स्फुटा दृक्क्षेपज्या~। अत्रोक्तस्य स्थौल्यमेव~। तस्य दृग्ज्यैव नतांशज्या~। दृग्ज्यानयनं चोक्तम्~। तस्मात् रविवद्दृक्क्षेपलग्नस्य दृग्ज्यैवानीयतां, सैव स्फुटा दृक्क्षेपज्या~। तत्कोटिज्या च स्फुटा\renewcommand{\thefootnote}{३}\footnote{स्पष्टा \textendash\ क. ग. घ.} दृग्गतिः~। किमर्थं तर्हि स्थूलो दृक्क्षेपः प्रदर्शितः~। मध्यलग्नात् दृक्क्षेपलग्नस्य भेदप्रदर्शनार्थम्~। भेदे ज्ञात एव हि
दृक्क्षेपलग्नस्य दृक्क्षेपज्यायाश्च तत्त्वतः प्रतिपत्तिरिति, तत्क्षेत्रप्रदर्शनपरमेवेदं कर्म~। मध्यज्याकर्णस्य भुजा हि मध्यलग्नदृक्क्षेपलग्नयोरन्तरालज्या इति 

\newpage

\noindent भुजाप्रदर्शनेनैवात्र कृतार्थता स्यात्~। तद्द्वारा दृक्क्षेपोऽप्यानेय इति तत्कोट्यानयनमप्युक्तम्~। अस्याः कोट्या ईषदधिक एव सदा दृक्क्षेपः~। तद्भुजाया अल्पत्वेऽस्य दृक्क्षेपस्य स्थौल्यमप्यल्पमेव~। तन्महत्त्व एर पुनर्दृग्ज्यावदानयनम्~। यदि कस्यचित् पुनरनेनापि तदानयनं स्फुरेत् तर्हि तथैवानीयताम् तेन इहैकं त्रैराशिकमेवावशिष्टम्~। तदप्यपक्रमतिर्यक्त्वानुरूपम्~। यदिदं मध्यलग्नान्ताया जीवायास्तिर्यक्त्वं तद्व्यासार्धेऽपि सुगममेवेत्यभिप्रायः~। तस्मादत्र विक्षेपसंस्कारो न कार्यः~। अत एवोक्तं परमेश्वराचार्येण\textendash 
 
\begin{quote}
{\qt मध्यलग्नस्य विक्षेपो न कार्यो लग्नजस्तथा~।} 
\end{quote}
 
\noindent इति~। अपक्रममण्डलदृक्क्षेपज्याया एव नतिसाधनत्वं, न पुनर्विक्षेपमण्डलगताया इत्येतच्च सिद्धान्तदीपिकायामुपपादितम् \textendash\ {\qt अपक्रमानुसारेण नतिः, तस्माद्विधोरपि दृश्यते, न तु विक्षेपवशादिति च निश्चितम्'} इति~। {\qt अतो न कार्यो विक्षेपसंस्कारो नतिसाधने} इति च~। तस्मात् माधवोक्तत्रैराशिकमेव स्वग्रहणेनार्यभटेन ज्ञापितं, न पुनर्विक्षेपभेदात् प्रतिग्रहं दृक्क्षेपज्याभेदः~। इयं गोलयुक्तिः लङ्कोदयानयनेऽपि समाना, यतः सर्वत्रापि गोलघनमध्यनाभिकमण्डलद्वयविवरनिमित्तानां गणितानां स्यादेव युक्तिसाम्यम्~। तत्र लङ्कोदयानयनं प्रकृतिः, दृक्क्षेपाद्यानयनं विकृतिः~। अपक्रममण्डलस्य विषुवन्मण्डलानां च विवरस्य स्थिरत्वात् तद्विवरनिमित्तस्य प्रकृतित्वम्~। तत्रापि भगोलवायुगोलमध्यगतत्वात् अपक्रमघटिकामण्डलयोः प्रसिद्धत्वात् तद्विवरनिमित्तस्य प्रकृतित्वं युक्तं, 
यतो दृक्क्षेपविषयस्य तद्विकृतित्वं युक्तमिति भावः~। अत्र दक्षिणोत्तरमण्डलमपक्रमस्थानीयं, घटिकामण्डलस्थानीयं च दृक्क्षेपमण्डलम्~। 
क्षितिजस्पृष्टदृक्क्षेपमण्डलव्यासार्धदिक्समतिरश्चीनत्वात् दक्षिणोत्तरमण्डलस्पृष्टाग्रत्वाच्च~। इहेच्छाफलभूताया भुजायाः स्वाहोरात्रस्थानीयं च

\newpage

\noindent दृक्क्षेपमण्डलैकपार्श्वगतं मध्यलग्नस्पृष्टं दक्षिणोत्तरस्वस्तिकयोः यन्मध्यलग्नसमानदिक्कं\renewcommand{\thefootnote}{१}\footnote{दिक्},
तत्स्पृष्टैकाग्रेतरस्वस्तिकादुदयज्याचापद्विगुणान्तरक्षितिजस्पृृष्टापराग्रा या रेखा सा अन्त्यद्युज्यास्थानीया~। तद्दृक्क्षेपमण्डलव्यासार्धयोरन्तरालं कृत्स्नमुदयज्यातुल्यम्~। उदयज्या च परमापक्रमस्थानीया~। त्रैराशिकानीता मध्यज्याकर्णभुजा इष्टापक्रमवत् कल्प्या~। या पुनः स्वदृक्क्षेपशब्दोक्ता तत्कोटिः, सा पुनरिष्टज्यागुणितात् काष्ठान्त्यस्वाहोरात्रात् त्रिज्याहृतवत् कल्प्या~। सा तत्र निगूढा इहाविष्कृता~।
लग्नसममण्डलं च विषुवत्स्पृष्टविषुवन्मण्डलवत्~। लग्नद्वयं च ध्रुवद्वयवत्~। तत्रेष्टज्याग्रे ध्रुवद्वये च स्पृष्टं यन्मडलं तद्वदिहापक्रममण्डलम्~।
एतद्दृक्क्षेपानयनं छेद्यके सुगमम्~। लङ्कोदयानयनयुक्तिस्तु उदग्विषुवत्युद्यति कल्प्यमाने छेद्यके प्रदर्शयितुं न शक्या~। ततो गोले सूत्राणि बद्ध्वैव प्रदर्श्या~। तत्क्षेत्रस्यादृश्यार्धगतत्वात् तत्र बुद्धिर्न स्फुरेत्~। अतो विषुवत्प्रदेशं खमध्यगं कृत्वा तद्युक्तिर्निरूप्या, छेद्यके प्रदर्श्या वा~।
तदपि प्राग्ज्येत्यनेन सूचितम्~। एतद्भुजात्रैराशिकोक्तिरिच्छाप्रमाणभूतकर्णयोर्दक्षिणोत्तरमण्डलगतत्वख्यापनार्थम्~। तत्रेष्टापक्रमवर्गमित्यादिना
स्वाहोरात्रमण्डलस्य प्रदर्शितत्वाद्, एतद्भुजाकोटिमण्डलमपि शक्यं कल्पयितुम्। तद्विष्कम्भार्धमपि भुजात्रिज्यान्तरवर्गमूलमेव~। इतीह दृक्क्षेपानयनेऽपि
तस्य हारकत्वमपि सिद्धम्~। उदयज्यात्रिज्यावर्गान्तरमूलस्य गुण्यत्वं च गुणकारत्वं च मध्यजीवायाः~। अत्र न मध्यज्याकोटिः त्रैराशिकेनानीयते ; किन्तु स्वभुजावर्गान्तरमूलीकरणेन~। ततः कोटिमण्डलगतैव सा~। सा कोटिमण्डले तावतिथांशस्य ज्या~। ततो महति मण्डले कर्णभूतत्रिज्याव्यासार्धेऽपि तावतिथांशस्य कियतीति त्रैराशिकान्तरं कार्यमिह परमदृक्क्षेपसिद्ध्यर्थम्~। यदि तत्रापीष्टापक्रमवर्गमिष्टज्यावर्गाद्
विशोध्य मूलीकृत्य कोटिरानीयते, तस्या\renewcommand{\thefootnote}{२}\footnote{स्यां \textendash\ ङ.} लङ्कोदयप्राग्ज्यायाः स्वाहोरात्रमण्डलगतत्वात् 

\newpage

\noindent मख्यादिभिश्चापीकर्तुमशक्यत्वात् तत्कलानां\renewcommand{\thefootnote}{१}\footnote{त्का \textendash\ ख. ग. घ. ड.} स्वभ्रमणकालप्राणसाम्याभावाच्च त्रिज्यावृत्ततावतिथांशज्यात्वं त्रिज्यागुणनं स्वाहोरात्रहरणं च कार्यम्~। तस्माद्गुणोपसंहारन्यायेनोभयत्राप्युभयं कर्म योज्यम्~॥~२२~॥\\
\begin{sloppypar} 	
\indent ग्रहणग्रहयोगादौ दृग्ज्यामण्डलगतस्य लम्बन\renewcommand{\thefootnote}{२}\footnote{नस्य \textendash\ ख. ग. घ. ड.}कर्णात्मकस्यापक्रममण्डलानुसारिकोट्याः तद्विपरीतदिशो नत्याख्याया भुजायाश्चानयनेऽपि\renewcommand{\thefootnote}{३}\footnote{नयने \textendash\ ङ.} दृक्क्षेपस्योपयोगात् तत्र दृक्क्षेपान्नत्यानयनं दृग्गतेश्च कोटिलम्बनानयनमितीष्टदृग्गति ज्यानयनमिहोच्यते\textendash 
\end{sloppypar}

\begin{quote}
{\ab दृग्दृक्क्षेपकृतिविशेषितस्य मूलं स्वदृग्गतिः कुवशात्~।\\
क्षितिजे स्वा दृक्छाया~भूव्यासार्धं नभोमध्यात्~॥~३४~॥}
\end{quote}
 
इति~। सूर्यसिद्धान्ते तु परमैव दृग्गतिः प्रदर्शिता~। तया पुनः चन्द्रादीनामिष्टदृग्गत्यानयनं प्रदर्श्यते परमेश्वराचार्येण\textendash  

\begin{quote} 
{\qt दृक्क्षेपस्य कृतिं विशोध्य कृतितो व्यासस्य शेषस्य य-\\
न्मूलं तेन समभ्यसेत् द्युमणिदृक्क्षेपान्तरोत्थं गुणम्~।\\
व्यासार्धेन तु संहरेत् पुनरपि प्राप्तं त्रिषड्दन्तिभिः\\
प्राप्ता लम्बननाडिकाः पुनरपि स्पष्टा स्युरेवं कृताः~॥}
\end{quote} 

\noindent इति~। यथेष्टापक्रमवर्गमपमण्डलभुजाज्यावर्गाद्विशोध्य मूलीकृत्य स्वाहोरात्रवृत्तस्था लङ्कोदयप्राग्ज्या अपक्रमेष्टज्याकर्णस्य कोटिरूपानीयते, तद्वदेव दृग्दृक्क्षेपकृतिविशेषितस्य मूलमित्येदपि~। तत्रेष्टज्यागुणितमित्यादिना यदन्तर्नीतं प्रथमत्रैराशिकं  प्रदर्शितं तद्वदेव परमेश्वरोक्तं दृग्गगतिमण्डलकल्पनं पुनः पूर्वप्रदशितात् विलक्षणमपि स्यात्~। अपक्रममण्डलं घटिकामण्डलस्थानीयं कल्पयित्वा दृक्क्षेपकोटिमण्डलं स्वाहोरात्रवत् कल्पयेत्~। पूर्वं सममण्डलस्वाहोरात्रवत् प्रदर्शितम्~। तत्कल्पनमपि कुवशादित्यनेन दर्शितम्~। अयमभिप्रायः \textendash\ अपक्रममण्डलानुसारि भूगोलेऽपि मण्डलं कल्पयेत्~। तन्निरक्षदेशभूवृत्ततुल्यम्~। तदपक्रममण्डलवच्चलति~। तच्चलन-

\newpage

\noindent वशाद्दृक्क्षेपदेशोऽपि भिद्येत~। तत् पुनरुदगवन्त्यवधि~। समुद्रेऽपि निरक्षदेशात् तावदवधि, अपक्रममण्डलवत् भ्रमद्भूमण्डलं परिरभते~। तस्मात् तदन्तरालप्रदेशे सर्वत्रापि एकस्मिन्नार्क्षदिने अदृक्क्षेपत्वं सम्भवति~। द्वादशराशिषूद्यत्सु यदा यदा यत्र यत्र यद्दिगनुसारि 
अदृक्क्षेपमण्डलं तत एकपार्श्वे तद्दिगनुसारि ततो द्रष्टृविप्रकर्षवशाद्ध्रसद्दृग्गतिमण्डलमदृक्क्षेपमण्डलसर्वावयवात् भूपरिधिपा\renewcommand{\thefootnote}{१}\footnote{प \textendash\ ख. ङ.}दावधि गत्वा 
उभयतः शून्यतामेति~। तत्रेष्टकाले यत् स्वदृग्गतिमण्डलं तदादित्यसूत्रप्रापि~। अदृक्क्षेपमण्डलविपरीतं यद्वृत्तं ततो यावद्विप्रकृष्टं
दृग्गतिमण्डलगत्या स्वस्थानं तदर्धज्येह दृग्गतिर्भूगता~। सैव लम्बनलिप्तेच्छा~। दृक्क्षेपदृग्गतिमण्डलान्तरालज्या दृक्क्षेपाख्या नतिलिप्तेच्छा~। दृग्ज्यैव मुख्यलम्बनलिप्तेच्छा~। इत्येतत् सर्वमिह कुवशादित्यनेन सूचितम्~। तत्र दृग्ज्यागतलम्बनलिप्तानयनेनैव नतिलम्बनक्षेत्रे उभे अपि सिद्धे
इति, तदेव प्रदर्शयति \textendash\ {\qt क्षितिजे स्वा दृक्छाया भूव्यासार्धं नभोमध्यात्~।} इति~। नभोमध्ये छायाभावादेव दृक्छायाया अप्यभावः~। यथोच्चप्रदेशे मध्यमस्फुटज्ययोरन्तरस्याप्यभावस्तयोरेवाभावात् एवमत्रापि~। शीघ्रस्फुटयुक्तिवदेवात्रापि युक्तिः~। तत्र भगोलगता दृग्ज्या घनभूमध्यगतद्रष्टृसम्बन्धिनी इति भगोलनाभिकं ग्रहबिम्बान्तं भूपृष्ठगतद्रष्टुरधऊर्ध्वगतैकव्यासं क्षितिजस्थतद्विपरीतव्यासं यद्वृत्तं तदिह प्रतिमण्डलस्थानीयम्~। द्रष्टृमध्यमपि तावत्परिमाणमधऊर्ध्वगं ग्रहदिङ्मार्गगं तदिह कक्ष्यामण्डलस्थानीयम्~। दृङ्गमण्डलं च कर्णमण्डलं, तद्ग्रहदृग्गोलगतम्~। तत्र स्वमध्यं नीचस्थानीयं ज्ञेयग्रहमण्डलादधोगतत्वात् प्रतिमण्डलस्य~। भूमिछन्नभगोलार्धमध्यमुच्चस्थानीयं द्रष्टुर्विप्रकृष्टत्वात् तस्य~। तत्र घनभूमध्यगतद्रष्टृग्रह सूत्रस्य यावदवनामः भूपृष्ठगतद्रष्टुर्ग्रहसूत्रस्य ततोऽप्यवनामः स्यात्, 
द्वयोरूर्ध्वाग्रयोरप्रदेशगत्वेऽपीतराग्रयोरधऊर्ध्वविप्रकर्षात्~। घनभूमध्यात् 

\newpage

\noindent विनिर्गतं द्रष्टुरूर्ध्वसूत्रं यत् खमध्यान्तं तन्मूले गणितानीतछाया कर्णमूलं स्पृशति, ज्ञेयायाश्छायायाः कर्णस्य मूलं\renewcommand{\thefootnote}{१}\footnote{कर्णमूलं \textendash\ ङ.} तस्मिन्नेव सूत्रे भूतले स्पृशति, इति तस्य तिर्यक्त्वस्याधिक्यं स्यात्~। तस्यावनामस्य यावदाधिक्यं तदेव दृङ्मण्डलगतं लम्बनं दृग्ज्याचापे संयोज्यम्~। शङ्कुचापाच्च शोध्यम्~। भूपृष्ठगतस्य द्रष्टुस्तावेव छायाशङ्कू~। तत्र यत् कक्ष्यामण्डलतया कल्पितं तदन्तं ग्रहसूत्रं नीत्वा तत्स्पृष्टपरिधौ सूत्रस्यैकमग्रं बद्ध्वा तत्कक्ष्यामण्डले दृङ्मण्डलानुसारिणि खमध्यादितरभागेऽपि तावत्यन्तरेऽन्यदग्रं बध्नीयात्~। तदर्धमिह लम्बिता छाया~। घनभूमध्यनाभिके दृङ्ममण्डलानुसारिणि प्रतिमण्डलाख्ये यत्र ग्रहो वर्तते खमध्यात् तावदन्तं यच्चापं, कक्ष्यामण्डलखमध्यादपि ग्रहभागे सूत्रस्यैकमग्रं बद्ध्वा इतरभागेऽपि तावत्यन्तरेऽन्यत् बध्नीयात्~। तदर्धं गणितानीतच्छायातुल्यम्~। भगोलमध्यप्रतिमण्डलग्रहखमध्यविप्रकर्षसाम्यादस्यापि~। प्रतिमण्डलगता हि स्वाहोरात्रेष्टज्याद्वारानीता छाया~। तत्तुल्यत्वात् कक्ष्यामण्डलप्रदर्शिताया अपि तद्गतग्रहबिम्बान्तच्छायाया अल्पत्वम्, इति छायाद्वयमप्येकस्मिन्मण्डले प्रदर्श्यमाने तच्छायाचापान्तरात्मकस्य लम्बनस्य सुगमत्वं स्यात्~। यद्वा दृग्गोलेऽपि तच्छायाद्वयं प्रदर्श्यम्~। दृग्गोलेऽपि केवला छाया भगोलकलाप्रमिता~। तस्या दृग्गोलकलाप्रमितत्वे सङ्ख्याया आधिक्यं स्यात्~। तदाधिक्यमेव वा लम्बनम्~। यद्वा दृग्गोलेऽपि स्वकलाप्रमिता तावत्सङ्ख्या कल्प्या, यावत्सङ्ख्या कक्ष्याप्रतिमण्डलयोः समाना कल्पिता, सा भगोलकला~। दृग्गोलवृत्तकलानां दृश्यार्धगतानां ततोऽल्पत्वात् तत्प्रमितेयमल्पैव छाया~। मापकभेदादेव सङ्ख्यासाम्यं विवक्षितम्~। सा दृग्गोले ग्रहादूर्ध्वमेव विश्राम्यति, तात्द्दीर्घाभावाज्ज्यारूपत्वाच्च~। तस्याः स्ववृत्तगतग्रहावधिज्यायाश्च यदन्तरं तद्वा लम्बनम्~। सर्वथापि भूमध्यगतस्य द्रष्टुस्तत्स्पृष्ठगतस्य च खमध्यात् ग्रहबिम्बान्ता 

\newpage

\noindent छाया एकेनैव मापकेन मीयमाना तुल्यसङ्ख्यापि तत्तद्द्रष्टृमध्यग्रहबिम्बपरिधिवृत्तकलाभिर्मीयमाना न 
तुल्यसङ्ख्या कर्णसूत्रस्य\renewcommand{\thefootnote}{१}\footnote{सूत्रस्य \textendash\ ग. ङ.}~।
तत्सङ्ख्याभेदवशात् भूपृष्ठगतस्य द्रष्टुर्ग्रहदृष्टिसूत्रं कर्णात्मकमन्यस्माल्लम्बते~। तत्र दृग्गोले कक्ष्यामण्डले च ये द्वे द्वे छाये प्रदर्शिते
ग्रहावधिका ततऊर्ध्वगता च, तयोरूर्ध्वगता प्रतिमण्डलछायातुल्यैव~। तयोर्यौ कर्णौ द्रष्ट्रधिष्ठितप्रदेशादेव प्रवृत्तौ,
तयोर्ग्रहबिम्बघनमध्यप्रापिणोऽन्यस्मात् कर्णात् योऽवनामः, ऊर्ध्वसूत्रादन्यस्मादपि विप्रकृष्टाग्रत्वात् अधोग्रयोः संस्पृष्टत्वाच्च, तत्र यदल्पावनामः कर्णः तस्य घनभूमध्यगतकर्णस्य च दिगेकैव तिर्यक्त्वसाम्यात्~। तस्मात् भूपृष्ठगतस्य ग्रहप्रापिकर्णस्यान्यस्मात् यदग्रस्यावनामः तस्यैकस्मिन् गोल एव प्रदर्शने कर्णयोर्विवरस्य दृश्यमानत्वात् विस्पष्टता\renewcommand{\thefootnote}{२}\footnote{त्वातिस्पष्टता \textendash\ ख. घ. ङ.} स्यादिति, दृग्गोल एवोभयमपि प्रदर्शितम्~। तस्य लम्बनस्य त्रैराशिकप्रदर्शनाय प्रमाणफले इह प्रदर्श्येते \textendash\ क्षितिजे स्वादृक्छायेति~। क्षितिजे छायाया व्यासार्धतुल्यत्वाद्, भूव्यासार्धलिप्तातुल्यमन्त्यफलं दृक्छाया, तदा ततोऽल्पाया नभोमध्यात् प्रवृत्ताया दृग्ज्यायायाः कियती दृक्छायेति~। तत्र भूव्यासार्धं गुणकारः~। व्यासार्धं भागहारः~। फलं योजनात्मकं लम्बनम्~। तल्लिप्तीकरणे पुनर्व्यासार्धं गुणकारः~। स्फुटयोजनकर्णो भागहारः~। तत्र प्रथमत्रैराशिके व्यासार्धं भागहारः~। सोऽन्यत्र गुणकारः~। ततस्तयोस्तुल्यत्वान्नष्टयोरिष्टछायाया भूव्यासार्धं गुणकारः~। स्फुटयोजनकर्णो भागहारः~। फलं लम्बनलिप्ताः~। यद्वैकमेवेदं त्रैराशिकं \textendash\ यदि स्फुटयोजनकर्णव्यासार्धे छायातुल्या लिप्ता ज्या, तदा भूव्यासार्धकर्णे भूवृत्ते कियती लिप्ता ज्या, इत्युच्चनीचवृत्तस्थानीये भूमण्डले भुजाफललब्धिः~। यद्वा शङ्कुछायाभ्यां 
कोटिभुजाफले आनीय त्रिज्यायाः कोटिफलं त्यक्त्वा शिष्टस्य भुजाफलवर्गस्य च योगमूलं दृक्कर्णः~। महाछायां त्रिज्यया हत्वा दृक्कर्णेन


\newpage

\noindent हरेत्~। तत्रापि लम्बिता छाया लभ्यते~। तच्छायावर्गं त्रिज्यावर्गाद्विशोध्य शिष्टस्य मूलं तच्छङ्कुः~। एवं सर्वत्र
स्फुटशङ्कुछाये आनेये~। विपरीतच्छायायां स्ववृत्तविष्कम्भार्धकर्णेन द्वादशाङ्गुलशङ्कुच्छायया चानीतां महाच्छायां भूव्यामार्धयोजनैर्हत्वा स्फुटयोजनकर्णेन हरेत्~। लब्धचापं छायाङ्गुलानीतदृग्ज्याचापाद्विशोधयेत्~। बिम्बलिप्ताव्यासार्धं च क्षिपेत्~। तज्ज्यावर्गं त्रिज्यावर्गतस्त्यक्त्वा मूलीकृत्य लब्धेन शङ्कुना विपरीतकर्मणा गतगन्तव्यकालानयनम्~। कालेन छायानयने शङ्कुचापे लिप्ताव्यासार्धं क्षिपेत् ; भानोरूर्ध्वार्धादपि रश्मिस्फुरणात् तदवधिकत्वाय छायायाः~। एवमेव दृग्गतिज्यां भूव्यासार्धयोजनैर्हत्वा स्फुटयोजनकर्णाप्तं 
लिप्ताद्यं लम्बनं दृक्क्षेपलग्रादधिके\renewcommand{\thefootnote}{१}\footnote{ग्नाधिके \textendash\ ग.} ग्रहे  योजयेत्~। एवं
संस्कृतलम्बनयोर्ग्रहयोः यदा साम्यं तदैव तद्बिम्बयोः परः सन्निकर्षः~। भेदे पुनस्तदन्तरं षष्ट्या हत्वा तद्गत्यन्तरेण हृत्वाप्तं नाड्यादिकः कालः~। शीघ्रगतेर्ग्रहस्याधिक्ये गतो योगः अन्यदा गम्यः~। ततो योगकालद्युगताप्तये इष्टकालद्युगतात् क्रमाच्छोध्यो योज्यश्च स योगकालः~। पुनस्तत्रापि लग्नादिकमानीय दृक्क्षेपदृग्गतिदृग्ज्यादिद्वारा पूर्ववदुभयोर्लम्बनमानीय स्वे स्वे स्फुटे संस्कृत्य तदन्तरादपि कालमानीय योगकालं 
निश्चलीकुर्यात्~। पुनर्योगकाले विक्षेपवतो विक्षेपमानीय दृक्क्षेपज्यायाश्च भूव्यासार्धस्फुटयोजनकर्णाभ्यां पूर्ववन्नतिलिप्ता\renewcommand{\thefootnote}{२}\footnote{न्नती लिप्ता} आनीय क्षेपदृक्क्षेपयोर्दिक्साम्ये योगं कुर्यात्~। भेदे च विश्लेषम्~। उभयोर्नतियोगो वियोगो वा नतित्वेन ग्राह्य:~। नत्यानयने दृक्क्षेपश्च स्वतत्कालविक्षेपसंस्कृत एव 
ग्राह्यः~। उभयोर्लग्नसममण्डलादेकदिग्गतत्वे वियोगः~। नानादिग्गतत्वे संयोगः~। पुनस्तन्नतिविक्षे\renewcommand{\thefootnote}{३}\footnote{न्नतिक्षे \textendash\ ख. ग. घ. ङ.}पयोर्योगो वियोगो वा स्फुटनतिः~। भूव्यामसस्येव कृत्स्नबिम्बव्यासस्यापि लिप्तामानं नेयम्~। तत्र छाद्यछादकयोर्बिम्बमानयोगार्धात् स्फुटनतेरल्पत्वे सूर्याचन्द्रमसोश्चेत् ग्रहणसम्भवः~। तारा-

\newpage

\noindent ग्रहयोश्चेद्भेदः~। स्फुटनतेराधिक्ये उभयमपि नैव स्यात्~। इत्येतत् सर्वं न्यायसिद्धमेव~॥~३४~॥ \\
\begin{sloppypar} 
\indent चन्द्रग्रहणस्य सर्वत्राप्येकप्रकारत्वेऽपि यत्र ग्रहणस्थितिकाले तस्य दृश्यत्वं तत्रैव ग्रहणस्यापि दृश्यत्वं तच्छायापि 
तस्य दृश्यत्व एव गण्या इति तदुदयास्तमयलग्नप्रदर्शनायाह\textendash 
\end{sloppypar} 
\begin{quote}
{\ab विक्षेपगुणाक्षज्या लम्बकभजिता\renewcommand{\thefootnote}{१}\footnote{भक्ता \textendash\ इति मुद्रितपाठः~।} भवेदृणमुदक्स्थे~। \\
उदये धनमस्तमये दक्षिणगे धनमृणं चन्द्रे~॥~३५~॥} 
\end{quote}

\indent इति~। एकस्मिन्नेव साक्षदेशे विक्षेपसदसद्भावयोर्यदुदयविवरमयनान्तस्यचन्द्रस्य तदिहानेयम्~। तच्च केवलापक्रमस्य स्फुटापक्रमस्य 
च चरभेद एव~। तस्माद्विक्षेपशब्देनेह विक्षेपसम्बन्ध्यपक्रम एव विवक्ष्यते~। स च विक्षेपसम्बन्धिज्याखण्ड एव~। ज्याखण्डश्च तच्चापमध्यकोट्या तत्समस्तज्यां निहत्य व्यासार्धेन हृत्वाप्तं फलम्~। स्वाहोरात्रज्यैव हि अपक्रमस्य कोटिः~। चापमध्यगतस्वाहोरात्रमिह इष्टापक्रमस्फुटापक्रमस्वाहोरात्रयोगार्धमेव प्रायशः~। पुनस्तत्खण्डज्यानीतायाः क्षितिज्याया\renewcommand{\thefootnote}{२}\footnote{ज्यानीतायाः \textendash\ ख. ग. ङ.}श्चरवद्व्यासार्धगुणनं स्वाहोरात्रहरणं च कार्यम्~। तस्मात् पूर्वत्रैराशिकव्यस्तत्वात् द्वितीयस्य तदुभयमपि न कर्तव्यम् इति दिक्षेपमेवाक्षज्यया हत्वा लम्बकेन हरेत्~। तत्राक्षज्यायाः फलत्वाद्विक्षेपस्य चेच्छात्वाद्विक्षेपगुणाक्षज्या लम्बकभजितेत्युक्तम्~। तत्फलमुदक्स्थे चन्द्रमसि उदये ऋणं भवेत्~। अस्तमये धनं च भवेत्~। दक्षिणगे चन्द्रमसि क्रमाद्धनमृणं च भवेत्~। चन्द्रग्रहणं\renewcommand{\thefootnote}{३}\footnote{णे \textendash\ क.} विक्षेपवदुपलक्षणम्~।
नन्वेतत्फलमपि स्थूलमेव~। केवलमपक्रमखण्डज्यानीतमेव न चरान्तरम्~। किं तर्हि ततोऽप्यधिकमेव~। स्फुटापक्रमस्याधिक्ये तावत् केवलाप\renewcommand{\thefootnote}{४}\footnote{वेलाप \textendash\ ख. घ. ङ.}क्रमचरानयने स्वक्षितिज्याया व्यासार्धहतायाः स्वाहोरात्रार्धमेव भागहारः~। 

\newpage

\noindent तच्च तदानीं स्फुटस्वाहोरात्रादधिकमेव~। अपक्रमाल्पत्वे स्वाहोरात्रस्याधिक्यात्~। स्फुटापक्रमचरानयनेऽपि स्वाहोरात्रव्यासार्धं भागहारः~। तस्मात् क्षितिज्याखण्ड एव न भेदकारणं, सकलायाः क्षितिज्याया हारभेदश्च~। तत्र स्फुटक्षितिज्यायाः केवलापक्रमक्षितिज्यातुल्यो यः खण्डः सोऽपि स्फुटस्वाहोरात्रेणैव हर्तव्यः~। तस्मात् केवलक्षितिज्यां स्वाहोरात्रान्तरेण हत्वा स्फुटस्वाहोरात्रेण हृत्वाप्तं फलमप्यस्मिन्
दर्शनसंस्कारे योजनीयम्~। एवं स्फुटापक्रमस्याल्पत्वेऽपि ज्याखण्डानीताच्चरादधिकमेव चरान्तरं, तदा स्फुटचरात् केवलचरस्याधिक्यात्~। तत्स्वाहोरात्रस्य न्यूनत्वात् तद्धृतस्य फलस्य ततोऽप्याधिक्यात् तत्रापि स्वाहोरात्रान्तरोत्यफलं क्षेप्यम्~। किञ्च तत्प्राणानां लिप्तीकृतानामेव तत्र
संस्कार्यत्वं यतोऽत्र\renewcommand{\thefootnote}{१}\footnote{त्र लग्नस्य \textendash\ ग. घ. ङ.} ग्रहोदयलग्नस्यानेयत्वम्~। अत्र तु ग्रहस्फुटतुल्यापक्रमप्रदेशोदयस्य विक्षिप्तग्रहोदयस्य चान्तरालभवाः प्राणा एवानीयन्ते~। तावता कालेन लग्नस्य यदन्तरं तदेवात्र संस्कार्यमिति~। नन्वत्र स्वाहोरात्रयोगार्धमपि न विक्षेपमध्यस्याहोरात्रं स्यात्~। विक्षेपदलयोरपि खण्डज्याभेदादिति चेत्, नैष दोषः~। तस्यान्तरस्याल्पत्वात् व्यावहारिकत्वाच्चास्य कर्मणः~। अत एव माधवोऽप्याह चापीकरणे\textendash 
 
\begin{quote}
{\qt ज्ययोरासन्नयोर्भेदभक्तस्तत्कोटियोगतः~। \\
छेदस्तेन हृता\renewcommand{\thefootnote}{२}\footnote{नाहृता \textendash\ ग. ङ.} द्विघ्ना त्रिज्या तद्धनुरन्तरम्~॥} 
\end{quote}

\noindent इति~। अत्र ज्याग्रहणे ऊनाधिकधनुषोऽर्धेन त्रिज्यां हृत्वा यश्छदे अानीतः स एवेहाप्यानीयेते~। तत्र तन्मध्पकोट्या द्विघ्नायाश्छेदाप्तं फलं ज्याखण्डतया लभ्यते~। इह तद्वैपरीत्येन ज्याखण्डहृता द्विघ्ना कोटि\renewcommand{\thefootnote}{३}\footnote{घ्नकोटिः \textendash\ ग.}रेव छेदः~। द्विघ्ना मध्यकोटिरेव ऊनाधिकधनुरग्रद्वयकोटियोगः~। तदर्धस्य मध्यकोटित्वाद्~। इति, कोटियोगार्धमेव तेनापि मध्यकोटित्वेनाङ्गीकृतम्~। तत्र ज्याचापग्रहणवाक्ययोर्मध्यगतं पथ्यावक्राह्वयार्धं 

\newpage

\noindent किमर्थम्? ज्याग्रहणे भुजाकोट्योरुभयोरपि ग्राह्यत्वे तयोरल्पामिष्टदोःकोटीत्यादिनानीय तत्त्रिज्यावर्गविश्लेषमूलेन महत्येवानेया~। इतरथा स्थौल्यं स्यादिति, स्थौल्यनिराकरणपरमिदम्\textendash 

\begin{center}
{\qt तत्राल्पीयः कृतिं त्यत्क्वा पदं त्रिज्याकृतेः परः~।}
\end{center}

\noindent इति~। तत्र भुजाकोटिगुणयोरुभयोर्मध्ये अल्पीयसो गुणस्यैव कृतिं त्रिज्याकृतेस्त्यक्त्वानीतं पदं परः परो महान् गुणः~। न तु महीयसः कृतिं त्रिज्याकृतेस्त्यक्त्वानीतं पदमल्पगुणत्वेन ग्राह्यम्~। यथा तत्पदस्यावयवोपेक्षया जायमानस्य स्थौल्यस्याधिक्यात्~। तत्रोभयोरपि कर्मणोः {\qt यश्चैव भुजावर्गः कोटिवर्गश्च कर्णवर्गः सः} इति युक्तियुक्तत्वेऽपि महतो राशेर्वर्गस्यावयवोपेक्षया जायमानस्यान्तरस्य महत्त्वं स्यात्~। 
ततस्तस्मिन् त्रिज्यावर्गात् त्यक्ते शिष्टस्याप्यन्तरस्य महत्त्वं\renewcommand{\thefootnote}{१}\footnote{शिष्टस्याप्यन्तरमहत्त्वं \textendash\ ग. घ. ङ} स्यात्~। तद्यथा \textendash\ अन्त्यचापगतगुणस्य रूपार्धा\renewcommand{\thefootnote}{२}\footnote{अन्त्यचापगतरूपार्धा \textendash\ ग. घ.}वयवोपेक्षायां वर्गे तज्ज्यातुल्यमन्तरं स्यात्~। ततस्तदितरवर्गेऽपि तावदन्तरं स्यात्~। तन्मूलं च प्रथमचापगतगुण एव~। तस्य शतसङ्ख्यत्वे द्विगुणेनं तेन शतद्वयेनान्त्यखण्डगुणे ह्रियमाणे फलं सप्तदशसङ्ख्यम्~। तस्मात् तत्र तावदन्तरं स्यात्~। यदि पुनः पञ्चसङ्ख्य एव तत्राल्पो गुणः तर्हि दशभिर्ह्रियमाणस्यान्त्यचापगुणस्य शतत्रयादप्याधिक्यात् तन्मूलं तावदन्तरं स्यादिति महानेवायं दोषः~। तस्मात् सर्वत्राप्येतत् चिन्त्यम्~। एवमत्रापि स्वाहोरात्रवर्गं त्रिज्याकृतेस्त्यक्त्वा मूलीकृतस्यापक्रमस्य स्थौल्यमेव स्यात~। अपक्रमत्रिज्यावर्गान्तरस्य स्वाहोरात्रस्य न स्थौल्यमित्यपक्रमाल्पत्वे स्वाहोरात्रभेदस्याल्पत्वात्~। यदस्य दर्शनसंस्कारस्य द्वितीयः संस्कारः प्रदर्शितः तस्याल्पत्वात् तस्मिन् अक्रियमाणेऽपि न व्यवहारे विशेष उपलभ्यः~। मानसे पुनश्चरान्तरप्राणलिप्ता एव संस्कार्यत्वेनोक्ताः\textendash  
\begin{quote}
{\qt तिथिघ्नात् चरसंस्कारात् स्वोदयेनांशकादिकम्\renewcommand{\thefootnote}{३}\footnote{दिकः \textendash\ ख. ग. घ. ङ.}~। \\
स्वर्णं क्षेपवशात् कार्यं ग्रहे षड्भयुतेऽन्यथा~॥}
\end{quote}

\newpage

\noindent इति~। तत्र {\qt नखघ्ना विषुवच्छाये}त्यादिना द्विगुणस्य चरस्योक्तत्वात् तत्संस्कारस्यापि द्विगुणात्मकस्योक्तत्वात् चरसंस्कारस्य तिथिगुणने चरदलान्तरविनाडीनां त्रिंशद्गुणनमेव कृतं स्यात्~। तत्रेदमपि त्रैराशिकं \textendash\ यदि स्वदेशराश्युदयविनाडीभिस्त्रिंशदंशा लभ्यन्ते तदा विक्षेपसम्बन्धिचरविनाडीभिः कियन्त इति~। चरान्तरविनाड्यस्त्रिंशद्धतास्तद्राश्युदयविनाडीभिः ह्रियन्ते~॥~३५~॥ \\

\indent एवं साक्षदेशसाधारणं विक्षेपहेतुकं लग्नान्तरं प्रदर्श्य साक्षनिरक्षसाधारणं लग्नान्तरं प्रदर्शयति\textendash  
\begin{quote}
{\ab विक्षेपापक्रमगुणमुत्क्रमणं विस्तरार्धकृतिभक्तम्~। \\
 उदगृणधनमुदगयने दक्षिणगे धनमृणं याम्ये~॥~३६~॥} 
\end{quote}

\noindent इति~। अस्य युक्तिर्निरक्षगोल एव प्रदर्श्या~। अपक्रममण्डले ग्रहस्फुटतुल्यो यः प्रदेशस्ततोऽपक्रममण्डलवैपरीत्येनैव हि ग्रहा
विक्षिप्यन्ते~। अतो ग्रहस्पृष्टराशिकूटमण्डलगत एव विक्षेपः~। एवं भूता एव ताराणामपि विक्षेपाः प्रदर्श्यन्ते~। ताराणां तु विक्षेपाधिक्यसम्भवात्
स्वाधिष्ठितराशेश्चतुर्थे राशावपि लगत्युदयः सम्भवति विषुवत्समीपतो विक्षिप्तानाम्~। अयनतो विक्षिप्तानां तु निरक्षदेशे न मनागप्यन्तरं स्यादुदयस्य~। तेन यदा यद्यन्नक्षत्रमयनस्थं तदा तस्य विक्षेपमहत्त्वेनायनदर्शनसंस्कारः कियांश्चिदपि स्यात्~। यदा यद्यन्नक्षत्रं विषुवतो विक्षिप्तं तस्य
विक्षेपमहत्त्वे तदाक्रान्तापक्रममण्डलकलायाः प्रायेण विक्षेपचापकलान्तरमपि तदुदयलग्नं स्यात्~। कुतः पुनरत्र यदा यन्नक्षत्रमयनस्थितमिति, नक्षत्राणां निश्चलत्वेऽपि अयनविषुवत्प्रापणं कादाचित्कमिवोच्यते~। अत्रोच्यते \textendash\ ताराणां भगोल एव निश्चलत्वम्~। यथा चित्रा सदापि कन्यातुलासन्धिगैव~। एवं रेवत्यपि मीनमेषसन्धिगता~। यथावोत्तराषाढा सर्वदापि

\newpage

\noindent पूर्वाषाढमध्यगता, यथा चागस्त्योऽपि मिथुनान्तादेव सर्वदा याम्यायामशीतितमे भागे स्थितः, एवं सर्वा अपि तारकाः सर्वदापि मेषादि राशिषु तदंशेष्वपि क्वचिदेव सर्वदापि वर्तन्ते~। न पुनः स्वस्थानात् कदाचिदपि चलन्ति~। तथापि कृत्स्नस्य भगोलस्यैव चलनात् तत्स्था अपि तारास्तदनुरूपं चलन्त्येव~। यथा {\qt अशीतिभागे याम्यायामगस्त्यो मिथुनान्तगः} इत्युक्तोऽगस्त्य इदानीं दक्षिणायनकाष्ठगत 
एव~। यत इदानीं मिथुनस्य पूर्वार्ध एवायनप्रदेशः~। तस्मादिदानीं तस्याप्य\renewcommand{\thefootnote}{१}\footnote{प्या \textendash\ क. ख. घ. ङ.}यनदर्शनसंस्कारः स्यादेव~। तस्मादयनसन्धिस्थानां ज्योतिषां निरक्षदेशोदयसमये तद्राशिकूटेमण्डलं निरक्षदेशक्षितिजेनोन्मण्डलाख्येन सहैकता गतम्~। तदानीं तस्यापक्रममण्डलस्याधऊर्ध्वत्वात् दृक्क्षेपाभावात्~। राशिकूटोन्नतिर्हि दृक्क्षेपः~। यदा पुनर्निरक्षदेश उदग्विषुवदुदेति, तदा राशिचक्रस्यातितिर्यक्त्वात् विषुवतो विक्षिप्तानामुन्मण्डलात् विप्रकृष्टत्वं स्यात्~। यद्युदग्विक्षिप्तः, तर्हि उन्मण्डलादूर्ध्वस्थः~। यदोदग्विषुवतो दक्षिणतो विक्षिप्तः तदाधस्थः~। एवं विक्षिप्तस्य लग्ने स्वस्फुटसमे शङ्कुरिहानीयते~। तस्योर्ध्वमुखत्वे ऋणत्वम्~। उदये शङ्कोरुन्मण्डलादधोगतत्वे 
धनत्वं च~। अस्तमये तद्विपरीतम्~। अत्रैवं त्रैराशिकं \textendash\ यदि त्रिज्यातुल्येन विक्षेपेण दृक्क्षेपतुल्या भुजा लभ्यते, तदेष्टविक्षेपेण कियतीति ग्रहस्फुटसमे लग्ने ग्रहशङ्कुर्लभ्यते~। दृक्क्षेपश्च राशिकूटभ्रमणवशात् प्रतिक्षणं भिन्नः~। उत्तरायणकाष्ठादावुद्यति सति, पूर्वस्वस्तिकाद्दक्षिणतः परमापक्रमतुल्ये लग्नत्वात् तस्य राशिकूटप्रदेशोऽप्युत्तरस्वस्तिकात् प्राक्परमापक्रमतुल्ये उन्मण्डलप्रेदेशे उदेति~। दक्षिणराशिकूटप्रदेशोऽपि दक्षिणस्वस्तिकात् पश्चात् तावति प्रदेशे तदास्तमेति~। ततः प्रवहवायुभ्रमणवशात् पञ्चदशघटिकातुल्ये काले मध्याह्नं प्राप्नोत्युदग्गतः, उदग्विषुवत उद्यत्त्वात् तदानीम्~। अन्यश्च निशीथं गतः~। ततोऽयनान्तात्प्रभृति 

\newpage

\noindent उत्क्रमेण यावन्तोऽसवो ग्रहाक्रान्तभुजाया ऊर्ध्वगताः स्युः, तज्जीवयेह त्रैराशिकेन तयोः क्षितिजाद्विप्रकर्ष आनेयः~। तत्क्षितिजविप्रकर्ष एव हि दृक्क्षेपः~। तस्माल्लङ्कोदयप्राग्ज्याचापं राशित्रयात् त्यक्त्वा शिष्टस्य जीवां गृहीत्वा तया राशिकूटस्वाहोरात्रेष्टज्यामानयेत्~। कथम्~। परमापक्रमतुल्यैव हि राशिकूटद्युज्या~। ततोऽयनान्तग्रहान्तरालकालज्यां परमापक्रमज्यया हत्वा त्रिज्यया हरेत्~। निरक्षदेशेऽक्षाभावात् स्वाहोरात्रेष्टज्यैव तच्छङ्कुश्च~। तत इदं त्रैराशिकं \textendash\ यदि त्रिज्यातुल्यया विक्षेपज्यया राशिकूटोन्नतितुल्या भुजा लभ्यते तदेष्टविक्षेपज्यया कियतीति ग्रहस्फुटतुल्ये लग्ने ग्रहशङ्कुर्लभ्यते~। तत्र प्रथमत्रैराशिके परमापक्रमज्यया गुणकारः~। व्यासार्धं भागहारः~। द्वितीये विक्षेपज्यया गुणकारः, भागहारश्च व्यासार्धमेव~। अत उक्तं \textendash\ {\qt विक्षेपापक्रमगुणमुत्क्रमणं विस्तरार्धकृतिभक्तम्} इति~। 
विक्षेपापक्रमाभ्यां गुणितं विक्षेपापक्रमगुणम्~। येन हि गुणकारेण गुण्यो गुणितः स पुनर्गुण्यस्तेन न्यस्तात् तावद्गुणः स्यात्~। उत्क्रमणशब्देनायनान्तात् प्रभृति ग्रहाक्रान्तावधिको गृह्यते~। {\qt दृग्गोलार्धकपाले ज्यार्धेन विकल्पयेत् भगोलार्धम्} इत्युक्तत्वात्
द्वयोर्द्वयोरन्तरालप्रदेशो ज्यार्धेनैवेह विकल्पनीयः~। तस्मान्न केवलं तच्चापमिह ग्राह्यम्~। किं तर्हि तज्ज्यैव ग्राह्या~। तस्मात् ग्रहभुजाकलाचापस्य\renewcommand{\thefootnote}{१}\footnote{भुजाचापस्य \textendash\ ख. ग. घ. ङ.} यत् कोटिचापं तज्जीवेहोत्क्रमणशब्देनोच्यते~। तस्याः प्रथमत्रैराशिके परमापक्रमो गुणकारः~। द्वितीये विक्षेपश्च~। ततस्तद्घातो गुणकारः~। उभयत्रापि
व्यासार्धस्यैव भागहारत्वात् तद्वर्ग एव भागहारः~। तत्फलं पुनरुदगयनस्थे ग्रहे उदग्दक्षिणगे विक्षेपे यथाक्रममृणधनं, याम्येऽयने 
उदग्दक्षिणगे विक्षेपे धनमृणम्~। विक्षेपायनयोर्दिक्साम्ये लग्नात् प्रत्यग्गतत्वाद्ग्रहस्य ग्रहोदयलग्नसिद्ध्यर्थं तत्फलं लग्नादृणम्~। दिशोवैपरीत्ये लग्नात् प्रत्यग्गतत्वाद्ग्रहस्य तदन्तरफलं ग्रहस्फुटे क्षेप्यं, ग्रहोदयलग्नसिद्ध्यर्थम्~। एतदेव लाघविकेन मयेन\textendash  

\newpage

\begin{quote}
{\qt सत्रिभग्रहजक्रान्तिभागघ्नाः क्षेपलिप्तिकाः~।\\
विकलाः स्वमृणं क्रान्तिक्षेपयोर्भिन्नतुल्ययोः~॥}
\end{quote}

\noindent इत्युक्तम्~। कथं पुनरिह शङ्कोरेव धनर्णत्वमुक्तम्~। तद्भ्रमणकालक्षेत्रस्य हि धनर्णता युक्ता~। सत्यम्~। तत्सम्बन्धि क्षेत्रस्यैव हि धनर्णत्वं युज्यते~। तदर्थं पुनरपि त्रैराशिकद्वयं कार्यम्~। तेन शङ्कुना कोटिभूतेन कर्णरूपा विक्षेपकोटिमण्डलगता ग्रहोन्मण्डलान्तरालज्या प्रथमत्रैराशिकेनानीयते~। तस्याः पुनर्विक्षेपकोटिमण्डलगतत्वात् तत्कलानामल्पत्वात् तन्मण्डलस्यापि स्वकलाभिर्व्यासार्धतुल्यत्वात् तस्याः कोटिमण्डलजीवाया व्यासार्धवृत्तपरिणमनं द्वितीयेन~। तत्रैवं प्रथमत्रैराशिकं \textendash\ यदि दृक्क्षेपको\renewcommand{\thefootnote}{१}\footnote{क्षेपको \textendash\ ख. ग. घ. ङ}ट्या व्यासार्धतुल्यः कर्णो लभ्यते, तदा तच्छङ्कुकोटिकस्य कियान् कर्ण इति विक्षेपकोटिमण्डलगता ग्रहक्षितिजान्तरगता ज्या लभ्यते~। तस्याः पुनर्व्यासार्धवृत्तपरिणमनेऽपि व्यासार्धं गुणकारः~। विक्षेपकोटिर्भागहारः~। फलं ग्रहस्पृष्टराशिकूटमण्डलस्य ग्रहोदयेऽपक्रममण्डलोद्यत्प्रदेशराशिकूटमण्डलस्य चान्तरालज्या~। ततस्तच्चापीकृतं ग्रहस्फुटे संस्कार्यम्~। एवमिह त्रैराशिकचतुष्टयं विद्यते~। तत्र पूर्वयोर्व्यासार्धं भागहारः, उत्तरयोर्व्यासार्धमेव गुणकारश्च~। ततश्चतुर्षु तुल्यत्वान्नष्टेषु सायनग्रहकोटिज्यायाः क्षेपपरमापक्रमघातो गुणकारः~। सायनग्रहकालकोट्यपक्रमस्वाहोरात्रस्य विक्षेपकोट्याश्च घातो भागहारः~। फलं ग्रहोदयलग्नग्रहस्फुटयोरन्तरालज्या~। ग्रहोदयकालकोटिज्येह ग्राह्येत्यत्राविशेषश्च कार्यः~। अनयैवोपपत्त्या साक्षदेशेऽपि दर्शनसंस्कारयुगलं युगपत्कर्तुं शक्यम्~। तच्चाह माधवः\textendash 

\begin{quote}
{\qt विक्षेपदृक्क्षेपवधे त्रिमौर्व्या\\
 निहत्य तत्कोटिवधेन भक्ते~।\\
धनुर्धनर्णं हरिदैक्यभेदात्\\
तयोः शशाङ्काद्युदयेऽन्यथास्ते~॥}
\end{quote}

\newpage

\noindent इति~। अत्र दृक्क्षेपानयनस्य पृथक्कृतत्वात् त्रैराशिकत्रयमेवावशिष्टम्~। तच्चैवं \textendash\ यदि त्रिज्यातुल्यया विक्षेपज्यया दृक्क्षेपज्यातुल्या भुजा लभ्यते, तदेष्टविक्षेपज्यया कियतीति प्रथमम्~। सा पुनः द्वितीयत्रैराशिके कोटिः, तयेदं त्रैराशिकं \textendash\ यदि दृक्क्षेपकोट्या त्रिज्यातुल्यः कर्णः, तदास्याः कियानिति द्वितीयम्~। तस्य व्यासार्धपरिणमने पुनरिदं \textendash\ विक्षेपकोटितुल्यायां तज्जीवायां त्रिज्यातुल्या स्ववृत्तखखषड्घनांशज्या लभ्यते, तदास्या विक्षेपकोटिमण्डलगतायाः कियतीति~। तत्र पूर्वत्रैराशिके विक्षेपस्य दृक्क्षेपो गुणकारः, व्यासार्धं भागहारः~। द्वितीये व्यासार्धं गुणकारः, दृक्क्षेपकोटिर्भागहारः~। तृतीये तु व्यासार्धमेव गुणकारः, विक्षेपकोटिर्भागहारः~। तत्र त्रिज्यया सकृद्धरणं कार्यं\renewcommand{\thefootnote}{१}\footnote{कार्यम् एकेनैव \textendash\ ख. ग. घ. ङ.}, द्विर्गुणनं च~। ततः सकृदेव गुणनं कार्यम्~। एकेनैव गुणनहरणयोः कृतयोरकृतयोश्च फलस्य तुल्यत्वात्~। तस्माद्विक्षेपस्य दृक्क्षेपत्रिज्ये गुणकारौ, विक्षेपदृक्क्षेपकोटिघातश्च भागहारः~। अत उक्तम् \textendash\ {\qt विक्षेपदृक्क्षेपहते त्रिमौर्व्या निहत्य तत्कोटिवधेन भक्ते} इति~। अत्र यच्छब्दतच्छब्दावध्याहार्यौ~। यल्लब्धं तस्य धनुरिति~। शशाङ्काद्युदये तयोर्विक्षेपदृक्क्षेपयोः हरिदैक्यभेदाद्धनर्णम् ; अन्यथा चास्ते~। हरिद्भेदे धनम्, ऋणमैक्ये~। उदय अस्त इत्येताभ्यां उदयकालजोऽस्तकालजश्च विक्षेपदृक्क्षेपाविह ग्राह्याविति सिद्धम्~। तत एवाविशेषकरणमपि सिद्धम्~। ग्रहोदयलग्ने निमित्तं दृक्क्षेपानयनं, दृक्क्षेपनिमित्तं च ग्रहोदयलग्नानयनमित्यन्योन्याश्रयत्वात् तत्परिहारायैव हि सर्वत्राविशेषः क्रियते~। मानसेऽपि दर्शनसंस्कारयुगलं युगपत् क्रियते\textendash 

\begin{quote}
{\qt विक्षेपो भिन्नतुल्यांशवलनघ्नः खखाङ्ककैः~।\\
हृतोऽंशास्तैर्युतोनः सन् ग्रहोऽर्कास्तान्तरेऽस्तगः~॥}
\end{quote}

\noindent इति~। वलनघ्नाद्विक्षेपान्नवशतहृतं यत् तदंशादि वलनविक्षेपयोर्भिन्नदिक्कयो- 

\newpage
\begin{sloppypar} 
\noindent र्ग्रहे योज्यं, तुल्यदिक्कयोः शोध्यम्~। एवं दृक्कर्मसंस्कृतः सन् ग्रहोयदार्का\renewcommand{\thefootnote}{१}\footnote{ग्रहोदयार्क \textendash\ ख. ग. घ. ङ.}स्तद्वयान्तरालगस्तदास्तङ्गतः, यदा तद्बहिर्गतस्तदा दृश्यः~। अस्तार्कद्वयमप्युक्तं\textendash 
\end{sloppypar} 
\begin{quote}
{\qt सूर्याष्टिविश्वरुद्राष्ट\renewcommand{\thefootnote}{२}\footnote{ष्टि \textendash\ ख. घ. ङ}तिथ्यंशघ्नैः खखाग्निभिः~।\\
    प्राग्भोदयाप्तैर्युक्तोनः सूर्योऽस्तार्कः शशाङ्कतः~॥} 
\end{quote}

\noindent इति~। चन्द्रादीनां सूर्यादिभिर्गुणिताः खखाग्नयः~। पृथक् पृथक् प्राग्भोदयविनाडीभिः सूर्यग्रहाक्रान्तराश्युदयविनाडीभिराप्तमंशादिकमर्काच्छोधयेत्~। तदेकोऽस्तार्कः~। पृथक्स्थात् तत्सप्तमराश्युदयविनाड्याप्तमर्के क्षिपेत्,
स इतरोऽस्तार्कः~। कुतो वलनविक्षेपघातात् खखाङ्काप्तमंशादि दृक्कर्मफलं स्यात्~। तद्धि वलनं नाम\renewcommand{\thefootnote}{३}\footnote{म~। मुख \textendash\ ख. ङ.}, ग्रहस्य मुखपुच्छयोः पार्श्वयोर्वा यत्तिर्यग्गमनम्~। अपक्रममण्डलगतघनमध्यस्य ग्रहबिम्बस्य\renewcommand{\thefootnote}{४}\footnote{स्य दृक्क्षेपदिशि क्रम \textendash\ ग.} प्राक्प्रतीच्योर्यत्रापक्रममण्डलं बिम्बमुद्भिद्य विनिःसरति, तयोर्मुखत्वं पुच्छत्वं च~। तत्र मुखवलनं सौम्यं चेत् पुच्छवलनं याम्यम्~। तदोदक्पार्श्वमुन्नतम् ; अवनतं च दक्षिणपार्श्वम्~। तत्र बिम्बस्योर्ध्वभागस्य दृक्क्षेपदिशि वलनं तद्विपरीतदिशि चाधोभागस्येति नियमः~। ग्रहस्य मुखपुच्छयोर्यावद्वलनं तावदेव पार्श्वयोरपि~। विक्षेपश्च तद्दिगनुसारी~। मानसोक्तं च वलनं रदमण्डलगतमेव~। अतो व्यासार्धत्वेन षोडशकं गृहीतम्~। वलनाङ्गुलं च दृक्क्षेपतया त्रिज्याविक्षेपकोट्योरल्पान्तरत्वादुपेक्षितं च~। तत्साधनं त्रैराशिकम्~। अतो विक्षेपस्य वलनगुणनं वलनषोडशक\renewcommand{\thefootnote}{५}\footnote{शांशक \textendash\ ख. घ. ङ.}वर्गान्तरमूलेन हरणं चेह कार्यम्~। ततः कोटिसामीप्याय पञ्चदशसङ्ख्यो हारो गृहीतः~। फलस्यांशकत्वाय षष्ट्या गुणितश्च~। ततः खखाङ्कसङ्ख्यो हारः~। फलमंशादिकम्~। अत्र वलनयोः स्थौल्यात् तद्भेदैक्यात्मकपारमार्थिकवलनस्यापि स्थौल्यादेव दृक्फलस्यापि स्थौल्यम्~। तथापि धीमतां स्फुरेदेव युक्तिः~। अत्र मानसवलनस्य प्राक्कपाले दृक्क्षेपविपरीतदिक्कत्वात् 

\newpage

\noindent प्रत्यक्कपाले समानदिक्कत्वाच्च धनर्णसाम्यमपि स्यात्~।
\begin{quote}
{\qt युतिमध्यनताभ्यस्ता पलभा भानुभाजिता~।\\
  प्रागुदग्दक्षिणं पश्चाद्वलनं रदमण्डले~॥\\
ग्रहेणायनयोरल्पमन्तरं द्विघ्नमायनम्~।\\
वलनं स्यात् तयोर्योगविश्लेषात् पारमार्थिकम्~॥}
\end{quote}

\noindent इति हि वलनानयनमुक्तम्~। व्यासार्धमण्डलगतायनवलनस्य कोट्यपक्रमसाम्यादायनस्य चाक्षसाम्यात् तद्योगस्य विश्लेषस्य वा प्रायेण दृक्क्षेपसाम्याच्च तदुपपद्यते~। एवं कृतदर्शनसंस्कारयुगले ग्रहे लग्नादधिके ग्रहो न दृश्यः, अस्तलग्नादूने च~। अस्तलग्नादूर्ध्वमुदयलग्नादधश्च ग्रहो दृश्यार्धगतत्वाद्दृश्य एव~॥~३६~॥ \\

\indent यदर्थमिह दृश्यादृश्यत्वमुक्तं तदेव ग्रहणं प्रदर्शयति~। तत्र द्वयोर्ग्रहणयोश्छाद्यछादकभावोपपत्तिप्रदर्शनपुरःसरं 
ग्रहणयोर्विशेषं प्रदर्शयति\textendash 

\begin{quote}
{\ab चन्द्रो जलमर्कोऽग्निर्मृद्भूच्छायापि या तमस्तद्धि~।\\
छादयति शशी सूर्यं, शशिनं महती च भूच्छाया~॥~३७~॥}  
\end{quote}
 
\indent इति~। चन्द्रबिम्बस्य जलमयत्वादर्कबिम्बस्य च तेजोधातुत्वात् तद्बिम्बे तत्तेजः सम्पर्कादेव तस्य प्रकाशमानत्वात् भूच्छायायाश्चन्द्रं
प्रतिग्राहकत्वमुपपद्यते~। यत्र भूम्या तिरोहितत्वात् सूर्यबिम्बस्य कश्चिदंशोऽपि न दृश्यते~। तद्भाग एवान्तरिक्षे तद्रश्मिराहित्यात् भूच्छायेत्युच्यते
इति तद्देशं प्रविष्टस्य चन्द्रस्यापि तद्रश्मिसम्पर्काभावात् दीप्त्यभाव एव ग्रहणम्~। प्रसिद्धं चैतच्छास्त्रान्तरे\textendash 

\begin{quote}
{\qt तेजसा गोलकः सूर्यो ग्रहर्क्षाण्यम्बुगोलकाः~।\\
प्रभावन्तो हि दृश्यन्ते सूर्यरश्मिविदीपिताः~॥}
\end{quote}
	
\noindent इति~। चन्द्रस्य सूर्यादधोगतत्वाज्जलांशप्राचुर्येऽपि सूर्यं प्रति छादकत्वमपि युज्यते~। तर्हि सितस्यापि कालक्रियापादसमुद्घाटितयुक्त्या
वक्रसमये

\newpage

\noindent सूर्यादधोगतत्वात् तत्संयोगे विक्षेपाभावे कदाचित् सूर्यछादकत्वं कुतो न स्यात्~। स्फटिकवत् स्वच्छत्वात् तद्बिम्बस्येति ब्रूमः~। चन्द्रस्य जलप्राचुर्येऽपि पञ्चीकरणमात्रापेक्षितान्मृदंशस्याधिक्याच्छादकत्वं युज्यते~। मृद्भूरित्यत्रापि मृत्प्राचुर्यकमेव भुवो विवक्षितं, मृज्जलशिखिवायुमय इति पाञ्चभौतिकत्वोक्तेः, एवं चन्द्रो जलमित्यादावपि जलप्राचुर्यमेव विवक्षितम्~। छायायास्तमस्त्वं लोके प्रसिद्धमेवेति हिशब्देन द्योतितम्~। अत एतद्युज्यते \textendash\ {\qt छादयति शशी सूर्यं, शशिनं महती च भूच्छाया} इति~। भूच्छायायाश्छादिकायाश्छाद्याच्छशिनो महत्त्वादेव खण्डग्रहणे दृश्यभागस्य कुण्ठविषाणत्वमपि~। सूर्यग्रहणे तु प्रायेणोभयोः साम्यादेव तीक्ष्णविषाणत्वं चेति य उभयोर्ग्रहणयोः विशेषः, स महच्छब्देन द्योत्यते~॥~३७~॥ \\

\indent तत्र रविभूच्छाययोर्विक्षेपाभावात् चन्द्रस्यापि विक्षेपाभावेऽल्पत्व एव वा ग्रहणसम्भवः, एवं पर्वपातापेक्षया ग्रहणयोः कादाचित्कत्वम्~। तत्रापि भूच्छायाया महत्त्वात् चन्द्रग्रहणस्य लम्बनेन विशेषाभावात् क्वाचित्कत्वाभावाच्च\renewcommand{\thefootnote}{१}\footnote{कल्पाभावात् \textendash\ ग. ङ.} तद्ग्रहणानि भूयांसि स्युः ; विरलानि च भास्वद्ग्रहणानि युगे इत्याह\textendash  

\begin{quote}
{\ab स्फुटशशिमासान्तेऽर्कं पातासन्नो यदा प्रविशतीन्दुः~।\\
     भूच्छायां पक्षान्ते तदाधिकोनं ग्रहणमध्यम्~॥~३८~॥}
\end{quote}

\indent इति~। इन्दुः पातासन्नः स्फुटशशिमासान्ते यदार्कं प्रविशति, यदा स एव पक्षान्ते भूच्छायां प्रविशति तदा सूर्याचन्द्रमसोर्ग्रहणमध्यम्
अधिकोनं तत ऊर्ध्वमधश्च चलति, ग्रहणमध्यम्~। सूर्यस्य लम्बनवशाच्चलनं, विक्षेपवशाच्चन्द्रस्य~। लम्बनस्य क्षितिजान्ते नाडिकाचतुष्कान्तं
वृद्धेर्दिनाद्यान्त्ययामयोः पर्वणि सूर्यग्रहणमध्यं रात्रावेवापि स्यात्~।

\newpage

\noindent छादकस्यातिदूरगत्वात्~। तदपि न सार्वत्रिकं, यथा मेघच्छादनम्~। तथा च सूर्यसिद्धान्ते\textendash 
	
\begin{quote}
{\qt छादको भास्करस्येन्दुरधस्ताद्घनवद्भवेत्~।}
\end{quote} 

\noindent इति~। कथं पुनर्विक्षेपवशात् चन्द्रग्रहणमध्यस्य पक्षान्ताच्चलनम्~। तदपि इष्टग्रासेनैव सेत्स्यति~॥~३८~॥\\
	
\indent सूर्यग्रहणासाधारणलम्बनादिकं वस्तु पूर्वमेव प्रदर्शितम्~। अथ सोमग्रहणासाधारणं भूच्छायादिकं प्रदर्शयति\textendash 

\begin{quote}
{\ab भूरविविवरं विभजेत् भूगुणितं तु रविभूविशेषेण~।\\
छायाया दीर्घत्वं\renewcommand{\thefootnote}{१}\footnote{भूच्छायादीर्घत्वम्' इति मुद्रितपाठः~।} लब्धं भूगोलविष्कम्भात्~॥~३९~॥
	
छायाग्रचन्द्रविवरं भूविष्कम्भेण तत् समभ्यस्तम्~।\\
भूच्छायया विभक्तं विद्यात् तमसः स्वविष्कम्भम्~॥~४०~॥}
\end{quote}	

\indent इति~। भूरविविवरं भूबिम्बघनमध्यस्य रविबिम्बघनमध्यस्य च यदन्तरालम्~। यस्य स्फुटयोजनकर्णाख्या~। भूगुणितं तद्रविभूविशेषेण
रविबिम्बस्य {\qt घ्रिञे}ति प्रदर्शितस्य भूमेश्च {\qt ञिले}ति प्रदर्शितस्य च बिम्बस्य योजनव्यासयोर्विशेषेण भूमियोजनव्यासं सूर्ययोजनव्यासाद्विशोध्य शिष्टेन शतत्रयाधिकसहस्रत्रयेण विभजेत्~। तत्र लब्धं भूच्छायाया दैर्घ्यं भूगोलविष्कम्भात् प्रभृति क्रमेण हीयमानपरिणाहम् एतावत्प्रदेशं गत्वा सूच्यग्रवदल्पीभूय शून्यतामियात्~। अस्योपपत्तिप्रदर्शनाय रविविष्कम्भात् प्रभृति भूविष्कम्भप्रापि सूत्रद्वयं तावत्प्रसारयेत्, यत्र तयोर्योगः स्यात्~। अथैवमेव भूमौ लिखेत्~। भूच्छायाग्रात् प्रभृति सूर्यबिम्बघनमध्यान्तमिह लम्बः~। तद्विवरस्य क्रमेण ह्रासात् तेन त्रैराशिकं \textendash\ रविभूविशेषतुल्येन ह्रासेन भुजारूपेण रविस्फुटयोजनकर्णतुल्य आयामो लभ्यते, तदा भूव्यासतुल्येन ह्रासेन कियानायाम इति~। भूछाया- 

\newpage

\noindent मध्यसूत्रस्य दैर्घ्यं लभ्यते~। अथवा भूव्यासार्धमिह शङ्कुः~। भूरविविवरं शङ्कुभुजाविवरं, रविबिम्बव्यासार्धं भुजा~। तत्र शङ्कुगुणमित्यादिना इह छायादैर्घ्यमानीयते~। किन्तु द्विगुणाभ्यामिह कर्म क्रियते, भूव्यासस्य रविभूविशेषस्य चार्धीकरणाभावे लाघवं स्यादिति~।
ननु कथं भूगोलविष्कम्भादित्युक्तम्~। रविबिम्बाद्भूबिम्बस्याल्पत्वात् दीप्यमानप्रदेशस्यार्धादाधिक्यात्\renewcommand{\thefootnote}{१}\footnote{अर्धाधिक्यात् \textendash\ ख. ग. ङ.} भूछायाभागस्य न्यूनत्वाच्च भूविष्कम्भाग्रात् भूच्छायाभागे कियच्चिच्चापं विहायैव रविभू\renewcommand{\thefootnote}{२}\footnote{विर्भू \textendash\ क.}परिधिगं सूत्रं स्पृशति~। लम्बसूत्राद्यावत् कर्णसूत्रस्य तिर्यक्त्वं विष्कम्भाग्रचापात् एतत्स्पृष्टस्य चापस्यापि तावता तिर्यक्त्वेन भाव्यम्~। सत्यम्~। भूव्यासार्धं रविभूविशेषार्धेन हत्वा रविस्फुटयोजनकर्णतद्बिम्बविशेषार्धवर्गयोगपदेन हृत्वा यदाप्तं तावत्या अर्धज्याया अग्र एव भूगोलविष्कम्भात् स्पृशति~। ततस्तच्छरं द्विगुणं भूव्यासार्धाद्विशोध्य शिष्टेन तुल्यं सूत्रयोर्भूपरिधिस्पृग्विवरम्~। तेनापि न फलभेदः~। सूर्यव्यासार्धादप्येवमानीतं यत्फलं तद्विष्कम्भादपि तावदर्धं एवैतत् सूत्रं स्पृशति~। तस्मात् इच्छाप्रमाणयोरेकरूपेणैव ह्रासान्न फलभेदः~। तस्या भूच्छायाया अवधेरेव भेदः~। न भूविष्कम्भात् प्रवृत्तमिदं छायादैर्घ्यम्, अत्रानीतार्धज्यान्तरात् प्रभृत्येव~। तस्मात् सार्धज्या भूव्यासार्धदैर्घ्ये क्षेप्या~। ततस्तस्मात् छायादैर्घ्यात्
चन्द्रस्फुटयोजनकर्णं विशोध्य शिष्टं छायाग्रचन्द्रविवरम्~। तदेव भूविष्कम्भेण समभ्यस्य भूछायया विभक्तं तमसः स्वविष्कम्भो भवेत्~।
चन्द्रकक्ष्याप्रदेशगतं तमोविष्कम्भं विद्यात्~। अत्रापि पूर्वलब्धज्याशरद्वयोनेनैव विष्कम्भेण गुणनं कार्यम्~। केवलयैव भूच्छायया हरणञ्च~।
न पुनस्तदर्धज्यायुक्तया~। युक्तिप्रथनार्थमेवायं विशेषः प्रदर्शितः~। न पुनरयं विशेषो ग्रहणे प्रत्यक्षत उपलभ्यः~। शास्त्रेषु नानोच्यमानबिम्बपरिमाणभेदा\renewcommand{\thefootnote}{२}\footnote{दात् जा \textendash\ ख. ङ.}ज्जायमानो भेद इतोऽपि सुमहान्~। तस्माद्युक्ति- 

\newpage

\noindent निरूपणदशायामेवेयं कथा, न पुनर्व्यवहारकाले~। ईदृशमल्पफलं कर्म तत्र तत्र तन्त्रकारैरप्रदर्शितं युक्तिविद्भिरुपलभ्यम्~। अस्यापि लिप्ताव्यासानयनं चन्द्रवदेव~॥~३९\textendash ४०~॥ \\

\indent अथ छाद्यच्छादकमानाभ्यां तद्विवरेण च स्थित्यर्धानयनं प्रदर्श्यते\textendash 
\begin{quote}
{\ab तच्छशिसम्पर्कार्धकृतेः शशिविक्षेपवर्गितमपोह्य~।\\
स्थित्यर्धं तन्मूलं\renewcommand{\thefootnote}{१}\footnote{सम्पर्कार्धस्य कृतेः शशिविक्षेपस्य वर्गितं शोध्यम्~। स्थित्यर्धमस्य मूलं \ldots \ldots~॥ इति मुद्रितपाठः~।} ज्ञेयं चन्द्रार्कदिनभोगाद्~॥~४१~॥}
\end{quote}
\indent इति~। तच्छशिसम्पर्कार्धस्य लिप्तात्मकस्य कृतेः शशिविक्षेपवर्गितमपोह्य यच्छिष्टं तन्मूलं स्थित्यर्धं ज्ञेयम्~। तत् स्थित्यर्धक्षेत्रम्~।
तस्य कालः पुनश्चन्द्रार्कदिनभोगाभ्यां ज्ञेयः~। अत्रेदं त्रैराशिकं \textendash\ यदि दिनगत्यन्तरेण षष्टिर्नाड्यो लभ्यन्ते तदानेन स्थित्यर्धक्षेत्रेण कियत्य इति~। स्पर्शमोक्षस्थित्यर्धयोः स्वस्वकालशशिविक्षेप एव वर्गीकृत्य सम्पर्कार्धवर्गाच्छोध्यः~। एतच्चेष्टकालग्रासेन वक्ष्यमाणेनैव सिद्धम्~॥~४१~॥ \\

\indent अथ निमीलनोन्मीलनकालज्ञानार्थमाह\textendash 

\begin{quote}
{\ab चन्द्रव्यासार्धोनस्य वर्गितं यत् तमोमयार्धस्य~।\\
	विक्षेपकृतिविहीनं तस्मान्मूलं विमर्दार्धम्~॥~४२~॥} 
\end{quote}

\indent इति~। चन्द्रव्यासार्धोनस्य तमोमयार्धस्य यद्वर्गितं विक्षेपकृतिविहीनं तस्मान्मूलं विमर्दार्धमपि ज्ञेयम्~। तच्च चन्द्रार्कदिनभोगादेव~।
तच्च षष्टिघ्नं गत्यन्तरहृतं विमर्दाधम्~। स्थित्यर्धं विमर्दार्धमपि प्रथममास्थितं\renewcommand{\thefootnote}{२}\footnote{प्रथममानीतं \textendash\ क.}  स्पर्शमोक्षयोर्निमीलनोन्मीलनयोः साधारणमेव~। पुनः स्वस्वकालविक्षेपानीतयोरेव भेदः~। एवं स्थित्यर्धद्वयं विमर्दार्धद्वयं च परमग्रासकालाच्छोध्यं योज्यं च~। तदा स्पर्शमोक्षौ निमीलनोन्मीलने च स्याताम्~। कथं पुनश्शशाङ्कस्य परमग्रासकालः पक्षान्ताच्चलति ; तत्पार्श्वभेदाद्यल्प- 

\newpage 
\noindent ग्रास एव~। तत्र सम्पर्कार्धस्य षष्टिसङ्ख्यत्वे पातभागे घटिकार्धेन विक्षेपस्य लिप्तार्धन्यूनत्वं स्यात्~। तद्गत्यन्तरस्य लिप्ताषट्कत्वात् तद्वर्गस्य सम्पर्कादल्पत्वात् तद्वर्गसंयोगमूलेन लिप्तार्धादल्पमेवाधिक्यम्~। ततस्तत्र स्थित्यर्धक्षेत्रवशाज्जायमानादाधिक्यात् विक्षेपह्रासवशाज्जायमानस्याल्पत्वस्याधिक्यम्~। तस्मात् तदानीं बिम्बान्तरस्य न्यूनत्वात् तदैव परमो ग्रासः~। एवं प्रतिपदि राकायां वा परमग्रासः सम्भवति तदपीष्टग्रासेनैव सिद्धम्~॥~४२~॥ \\

\indent समस्तग्रहणाभावे पर्वान्तचन्द्रमण्डलव्यासे कियान् प्रदेशो दृश्य इत्येतत्प्रदर्शयितु-माह\textendash  
\begin{quote}
{\ab तमसो विष्कम्भार्धं शशिविष्कम्भार्धवर्जितमपोह्य~।\\
	विक्षेपाद्यच्छेषं न गृह्यते तच्छशाङ्कस्य भूच्छाया~॥~४३~॥}
\end{quote}

\indent इति~। भूच्छायामध्यस्य चन्द्रबिम्बमध्यस्य च यत् पर्वान्तकाले विवरं स एव तदानीं विक्षेपः~। पर्वान्ते विक्षेपस्य शून्यत्वे तमोमद्ध्यं
एव शशिमध्यमपि~। तदानीं समस्तग्रासश्च स्यात्~। तमसो विष्कम्भार्धस्य आधिक्यात्~। तच्चन्द्रपरिधेः समन्तत उभयोर्व्यासार्धान्तरप्रदेश एव
भूच्छायापरिधिः~। अत एवार्करश्मिविप्रकर्षात् ग्रस्तस्य चन्द्रबिम्बस्य वर्णविशेषं वक्ष्यति {\qt स कृष्णताम्रस्तमोमध्ये} इति~। यस्मिन् पर्वणि पुनरुभयोर्व्यासार्धान्तरतुल्यो विक्षेपः तदा भूच्छायामध्यात् तावद्विप्रकर्षे चन्द्रबिम्बमध्यम्~। ततो विक्षेपसूत्रमार्गेण सूत्रे प्रसार्यमाणे
चन्द्रबिम्बमध्यात् स्वपरिधेः स्वव्यासार्धान्तरितत्वात् भूच्छायाव्यासार्धस्यापि विक्षेपतुल्यस्य खण्डस्याधोगतत्वात् तत्परिधेरपि चन्द्रबिम्बपरिधौ संयोगः स्यात्~। तेन तदानीमपि समस्तग्रासः स्यात्~। विक्षेपस्य तत आधिक्ये यावदाधिक्यं स्यात् चन्द्रव्यासार्धस्य, तावान् प्रदेशो भूच्छायाया
बहिर्गतः

\newpage

\noindent स्यात्~। अतः स प्रदेशो दृश्यः~। इत्येतत्क्षेत्रप्रदर्शनाय {\qt तमसो विष्कम्भार्धं शशिविष्कम्भार्धवर्जितमपोह्य विक्षेपाद्यच्छिष्टं न गृह्यते तच्छशाङ्कस्य} इत्युक्तम्~। एवं सम्पर्कार्धमपि तद्व्यासार्धैक्यमिति युक्तिनिरूपणे कल्प्यम्~। यत उभयोर्व्यासार्धयोगसाम्ये विक्षेपस्योभयोः परिधिसंयोगः स्यात्~। ततो विक्षेपस्याल्पत्व एव वृत्तयोः संश्लेषः स्यात्~। आधिक्ये उभयोर्मिथो बहिर्गतत्वादेव ग्रहणाभावः~। विक्षेपस्याल्पत्वेऽपि प्राक्पश्चाच्च कदा परिधिसंयोग इत्येतत् सम्पर्कार्धवृत्तगतया क्षेपकोट्यैव ज्ञायते~। यदा\renewcommand{\thefootnote}{१}\footnote{यदा न \textendash\ क.} क्षेपकोटितुल्यं पूर्वापरविवरमुभयोः स्यात्, तदा बिम्बमध्ययोर्ऋजुतया विवरं तत्कर्णतुल्यं स्यात्~। अत एव सम्पर्कार्धतुल्यस्य कर्णस्य वर्गात् क्षेपवर्गं विशोध्य शिष्टस्य मूलं स्थित्यर्धक्षेत्रं ज्ञेयमित्युक्तम्~॥~४३~॥\\
	
\indent अनयैवोपपत्त्येष्टग्रासानयनमपि प्रदर्शयति\textendash 
\begin{quote}
{\ab विक्षेपवर्गसहितात् स्थित्यर्धादिष्टवर्जितान्मूलम्~।\\
	सम्पर्कार्धाच्छोध्यं शेषस्तात्कालिको ग्रासः~॥~४४~॥} 
\end{quote}

\indent इति~। स्थित्यर्धादिष्टवर्जितात्, पुनरपि किं विशिष्टात् विक्षेपवर्गसहितात्~। वर्ग एव वर्गो योज्य इति लक्षणया तद्वर्ग एवोच्यते~।
इष्टवर्जितं स्यित्यर्धमपीष्टकालपूर्वापरविवरम्~। तच्च स्फुटान्तरतुल्यम्~। तस्मात् इष्टकाले षड्भयुक्तार्कस्य चन्द्रस्फुटस्य च यद्विवरं
विक्षेपवर्गसहितात् तद्वर्गाद्यन्मूलं तदिष्टकाले द्वयोर्बिम्बकेद्रान्तरं स्यात्~। यतो\renewcommand{\thefootnote}{२}\footnote{द्वयोः \textendash\ ग, ययोः \textendash\ घ.} बिम्बान्तरे सम्पर्कार्धतुल्ये तयोः परिधिसंयोगः तत एव बिम्बान्तरे सम्पर्कार्धात् त्यक्ते शिष्टं यत् चन्द्रबिम्बस्य, तावान् प्रदेशो भूच्छायायां प्रविष्टः स्यादिति तत्तुल्य एवेष्टकाले ग्रासः~। एवं तत्काले ग्रस्तस्य दृश्यभागस्य च विभागः~। कथं पुनः सकलेऽपि बिम्बे इत्येतत् {\qt ग्रासोन} सूत्रे प्रदर्शितमिति शशिग्रहणमध्यस्य पक्षान्ताच्चलनम्~। शशितमसोः

\newpage

\noindent परमासत्तिर्विक्षेपापक्रममण्डलयोः मध्यवृत्तभुक्तिसाम्य एव हि स्यात्~। तदैव ग्रहणमध्यं च~। स्फुटसाम्ये तु विक्षेपकोटिमण्डलापक्रममण्डलयोः भुक्तभागसाम्यमेव स्यात्~। तदुक्तं मया ग्रहणनिर्णये\textendash  

\begin{quote}
{\qt परमक्षेपकोटिघ्नः पातोनार्कभुजागुणः~। \\
 स्वेष्टविक्षेपकोट्याप्तः तत्क्षेपकृतियोगतः~॥ \\
 पदं यच्चापितं यच्च पातोनार्कभुजाधनुः~। \\
 तद्विशेषं हतं षष्ट्या गत्यन्तरहृतं क्षिपेत्~॥ \\
 पर्वान्ते युक्पदे क्षेपे शोधयेत् विषमे पदे~।\\
 एवं कृतोऽपि पर्वान्तः सूर्येन्द्वोर्ग्रहणे स्फुटम्~॥} 
\end{quote}
 
\noindent इति~। अस्य युक्तिः मण्डलान्तरालन्यायेन सिद्ध्यति~। स्थित्यर्धाद्यानयनं सूर्यस्यापि समानम्~। तत्र नत्यादिसङ्कीर्णत्वात्
स्थित्यर्धयुक्तिमात्रं पृथगुपादाय प्रदर्शयितुं लम्बनादिरहिते शशिग्रहणे तद्युक्तिः प्रदर्शिता~। अत एव भास्करेणापि लघुभास्करीये सोमग्रहणं प्रथमं प्रदर्शितम्~॥~४४~॥\\

\indent अथोभयोः समानं दिग्वलनं प्रदर्शयितुमाह~। तच्च द्विविधं दर्शनसंस्कारवत्~। तत्राक्षं प्रथमं दर्शयति\textendash 

\begin{quote}
{\ab मध्याह्नात् क्रमगुणितोऽक्षो दक्षिणतोऽर्धविस्तरहृतो दिक्~।\\
	स्थित्यर्धाच्चार्केन्द्वोस्त्रिराशिसहितायनात् स्पर्शे~॥~४५~॥} 
\end{quote}

\indent इति~। निरक्षदेशेऽयनान्तस्थस्य ग्रहस्य दिग्वलनं कदाचिदपि न स्यात्~। इह तु मध्याह्न एव तदभावः~। तद्युक्तिः घटिकामण्डल एव
प्रदर्श्या स्वाहोरात्रान्तर्भावात् तस्यापि~। निरक्षदेश एव हि स्वाहोरात्रवृत्तानि पूर्वापरायतानि~। तत\renewcommand{\thefootnote}{१}\footnote{यतः \textendash\ ग. घ.} उत्तरतस्तूर्ध्वभागो दक्षिणत एवावनतः~। अधोभागस्तूत्तरत एवोन्नतः, तेषामधऊर्ध्वदिगपेक्षया तिर्यक्त्वमक्षेण परिछिद्यते~। अधऊर्ध्वायतत्वं च लम्बकेन~। तद्व्याससूत्रे व्यासार्धतुल्यप्रदेशस्य कर्णात्मकस्य लम्बकतुल्यैव कोटिः~। अक्षतुल्यैव भुजा~।

\newpage

\noindent तेषामुन्मण्डलस्पृष्टावयवा एव विषुवत्स्थानीयाः, तत्र तिर्यक्त्वस्याधिक्यात् दक्षिणोत्तरमण्डलस्पृष्टा एवायनस्थानीयाः तत्परिधिभागस्य समपूर्वापरत्वात् दिग्वलनाभावात्~। तत्र तावद्घटिकोन्मण्डलसम्पाते रवावुद्यति निरक्षदेशस्थानां तद्घनमध्यात् तत्पृष्ठे यत्रोन्मण्डलमुद्भिद्य विनिर्गच्छति, स एव पार्श्वदेशः~। यत्र तत्पृष्ठे घटिकामण्डलं स्पृशति तत्प्रदेशादूर्ध्वोधोगतौ~। ततो यत् साक्षदेशे दिग्वलनं दिक्सूत्राणां तिर्यग्गमनं
तदिहाक्षदिग्वलनम्~। तच्चोदये व्यासार्धमण्डलेऽक्षचापतुल्यम्~। यतः साक्षदेशे उन्मण्डलदक्षिणोत्तरव्यासाग्रयोः क्षितिजादक्षतुल्यो विप्रकर्षः, ततो
बिम्बस्य दक्षिणोत्तरयोरक्षतुल्यमेव दिग्वलनम्~। अधऊर्ध्वदिशोश्च सममण्डलाद्घटिकामण्डलस्य विप्रकर्षोऽक्षतुल्य एव~। एवमुदयेऽक्षचापतुल्यमेव दिग्वलनम्~। तस्मात् व्यासार्धमण्डलवलनात् ग्रहबिम्बदिग्वलनमपि त्रैराशिकसिद्धम्~। मध्याह्ने पुनः ग्रहस्य दिग्वलनं न स्यात् ; स्वाहोरात्रमण्डलानां दक्षिणोत्तरस्पृष्टभागस्य समपूर्वापरत्वात्~। यदा पुनः पञ्चघटिकातुल्यो नतकालः, तदा घटिकामण्डलदक्षिणोत्तरस्वस्तिकात् प्राग्भागे घटिकामण्डलद्वादशांशे रविबिम्बघनमध्यं स्यात्~। तदानीं तन्मध्यं व्यासार्धमण्डलं, ततः प्राक् पश्चाच्च घटिकामण्डलपरिधिपादांशे स्पृशति~। तदा घटिकामण्डलापरस्वस्तिकादूर्ध्वं तद्द्वादशांशे तद्व्यासार्धमण्डलं स्पृशति~। पूर्वस्वस्तिकादधश्च द्वादशांशे घटिकामण्डलतत्प्रदेशयोः सममण्डलात् विप्रकर्षोऽक्षार्धतुल्यः~। यतः क्षितिजप्रदेशात् प्रभृति व्यासार्धतुल्ये खमध्येऽक्षज्यातुल्यो विप्रकर्षः, घटिकामण्डलोन्नतज्यया एकराशिज्यया सममण्डलाद्विप्रकर्षः कियन् लभ्यते इति तत्र त्रैराशिकम्~। तच्च अपक्रमवदेव~। तस्मात् एकराशिज्यायास्त्रिज्यार्धतुल्यत्वात् अक्षज्यार्धमेव तत्र वलनम्~। तस्मात् पञ्चघटिके नतकाले तत्क्रमज्ययैव एकराशिज्याया वलनमानेयम्~। न पुनरेकराश्युत्क्रमज्यया~। तस्मात् अत्राप्युत्क्रमशब्द ऊर्ध्वतः प्रवृत्तत्वादेव वर्तते~।

\newpage

\noindent न पुनस्तेनोत्क्रमज्या विवक्ष्यते~। साक्षात्कृतज्याचापसम्बन्धानां विशदमेवैतत् ज्याखण्डन्यायसाम्यादपि\textendash  
\begin{quote}
{\qt समपूर्वापरं वृत्तं घटिकास्थार्कमध्यगम्~।\\
सममण्डलतस्तस्य विवरं सर्वगं समम्~॥

अक्षजं वलनं भानोः परिधौ विवरं तयोः~।\\
ज्याखण्डे क्षितिजाद्भानोः मानार्धे श्रवणेऽक्षदोः~॥

सम्पर्कार्धेऽक्षदोःखण्डो नाना तल्लेखनानुगम्~।\\
धनुर्ज्या मध्यकोटिघ्ना त्रिज्याप्ता गुणखण्डकः~॥

अक्षघ्नस्त्रिज्यया भक्तो धनुर्ज्यावृत्तजं हि तत्~।\\
तस्मान्नतासु मख्यादि ज्याक्षघ्ना त्रिज्यया हृता~॥

त्रिज्यामण्डलजं तद्धि ज्ञेयमिष्टेऽनुपाततः~।\\
नतज्याक्षज्ययोर्घातः त्रिज्याप्तस्तस्य कार्मुकम्~॥

इत्युक्तं सूर्यसिद्धान्ते त्रिज्यावृत्ते हि तद्धनुः~।\\
एवमेवायनं चिन्त्यं द्युज्यापक्रमवृत्तयोः~॥

विवरं भास्वतो वृत्ते मानैक्यार्धे शशिन्यपि~।\\
दोःखण्डापक्रमस्तत्र तच्चापवलने कलाः~॥

तस्मादुत्क्रमशब्दार्थं भास्करो वलनद्वये~।\\
तथैवायनदृक्क0र्मण्यन्यथा प्रत्यपद्यत\renewcommand{\thefootnote}{१}\footnote{प्रतिपद्यते \textendash\ ग. घ.}~॥}
\end{quote}

\noindent इति~। अक्षवलनस्य बिम्बोर्ध्वभागे दक्षिणतो दिक्~। नित्यदक्षिणत्वादक्षस्य~। आयनवलनदिक्स्पर्शेऽयनात् चन्द्रस्य प्रकृतत्वात् तत्स्पर्शे उत्तरायणस्थे ग्रहे उत्तरमायनं वलनम्~। दक्षिणायने याम्यम् ; एवं पूर्वार्धे~। चन्द्रस्य हि पूर्वार्धे स्पर्शोपलब्धिः, यतः स्वयं शीघ्रगतिः
सन् भूच्छायायाः प्रत्यग्भागे स्वबिम्बप्राग्भागेन तां प्रविशति~। तस्मादुभयोरपि पूर्वार्धेऽयनवत् दिक्~। तत एवापरार्धे व्यत्ययेन चेति सिद्धम्~। 

\newpage

\noindent एवमायनवलनदिक्~। आक्षस्यापि पूर्वार्धे सौम्यैव दिक्~। किं पुनरायनं वलनम्~। अर्केन्द्वोस्त्रिराशिसहितायनात् सिद्धं त्रिराशिसहितस्यार्कस्येन्दोर्वा यदपक्रमचापं तदेवायनं वलनम्~। एवं मध्यग्रहणवलनमुक्त्वा स्पर्शमोक्षयोरप्याह \textendash\ {\qt स्थित्यर्धाच्च} इति~। स्थित्यर्धशब्देन स्पर्शमोक्षौ विवक्ष्येते~। स्पर्शकालान्मोक्षकालाच्चैवं नतासुज्यया आक्षवलनं नयेत्~। आक्षवलनस्य हि स्पर्शमोक्षयोर्महान् भेद इति तत्रैव स्थित्यर्धाच्चेत्युक्तम्~। आयनमपि तात्कालिकमानेयम्~। {\qt अर्केन्द्वोस्त्रिराशिसहितायनात् स्पर्श} इत्यत्रापि स्पर्शग्रहणात् मध्यग्रहणे दक्षिणोत्तरदिग्भ्यामेव बिम्बपार्श्वभागस्य वलनम्~। उत्तरायणे तद्बिम्बपरिध्यभिमुखस्य दिक्सूत्रात् इडाभागे वलनचापात् त्रैराशिकानीतं रव्यादिपरिधौ वा सम्पर्कार्धपरिधौ वा आनेयम्~। दक्षिणायने तु पिङ्गलाभागे आक्षं तत्~। तद्दिश इडाभाग एव मध्यग्रहणे वलितसूत्रेण विक्षेपं स्फुटनतिं वा नीत्वा तदग्रे चन्द्रबिम्बमालेखनीयम्~। रविबिम्बं भूच्छायाबिम्बं वा वलनवृत्तमध्ये~। सूर्यसिद्धान्ते पुनस्त्रिज्यावृत्तात् सप्तत्यंशवृत्ते परिणमितं वलनद्वयम्~। सम्पर्कार्धे भास्करेण~। मानसे पुनः रदमण्डले परिणमितम्~। यथेष्टपक्षाङ्गीकाराय सूत्रे न विशेषः कृतः, विश्वतोमुखत्वात्सूत्रस्य~। इष्टकालेऽपि दिग्वलनसूत्रेण पूर्वतोऽपरतो वा कोटिसूत्रं नयेत्~। ततो मत्स्यमुखपुच्छानुसारि विक्षेपसूत्रं च नीत्वा तदग्रे चन्द्रबिम्बमालिखेत्~। स्पर्शकाले प्रत्यग्दिशि कोटिसूत्रं नयेत्~। तदा रविभूच्छायापेक्षया तद्दिग्गतत्वात् तस्य~। एवं मोक्षेऽपि प्राच्याम्~। इष्टकाले बिम्बान्तरवृत्ते वा वलनं नेयम्~। द्वयोः वलनचापयोर्योगो वियोगो वा सर्वत्र वलनं विवक्षितम्~। ग्रहयोगेऽपि दिग्वलनसूत्रे वलनवृत्तमध्यात् यथादिशं द्वयोर्विक्षेपं नीत्वा तदग्रे ग्रहयोर्बिम्बे आलिखेत्~। भेदोल्लेखापसव्यांशुविमर्दादिकं सर्वं ज्ञेयम्~। सूर्यग्रहणस्य विमर्दस्थाने सूर्यबिम्बात् चन्द्रबिम्बस्याल्पत्वे मध्यतमस्कार्धमेव विमर्दार्धम्~। दिग्वलनवशात् क्षेपवशाच्च ग्रहणे हनुभेदपार्श्वभेदपायुभेदादिकं

\newpage

\noindent सर्वं ज्ञेयम्~। यतः संहितायां तद्वशात् फलविशेषाः प्रदर्शिताः~॥~४~॥\\

\indent ग्रहणेऽपि शशिनोऽस्मद्दृग्गोचरगतत्वात् ग्रस्तभागस्यादित्यरश्मिसन्निधानविप्रकर्षानुरूपं प्रतीयमानो वर्णविशेषः प्रदर्श्यते\textendash 
\begin{quote}
{\ab प्रग्रहणान्ते धूम्रः खण्डग्रहणे शशी भवति कृष्णः~।\\
	सर्वग्रासे कपिलः स कृष्णताम्रस्तमोमध्ये~॥~४६~॥} 
\end{quote}
 
 \indent इति~। प्रग्रहणं ग्रहणप्रारम्भः~। तस्मिन् ग्रहणान्ते च ग्रस्तभागो धूम्रः~। खण्डग्रहणे स्पर्शकालात् ग्रासमहत्त्वेन चन्द्रबिम्बस्य
द्वेधा खण्डितत्वे स्पष्टे ग्रस्तभागे कृष्णो भवति शशी~। सर्वग्रासे कपिलः चन्द्रबिम्बे कार्त्स्न्येन भूच्छायां प्रविष्टे कपिलवर्णो दृश्यते शशी~।
तत्रापि आदित्यरश्मिसन्निधानात् भूच्छायापरिधिसमीपगतस्येषच्छौक्ल्यं स्यात्~। इतरभागस्य कपिलत्वम्~। तमोमध्ये स कृष्णताम्रः, यदा भूच्छायानाभिगतः तदा ताम्रवर्णत्वेऽपीषत्कार्ष्ण्यं स्यात्~। कथं पुनर्मध्यकाले विक्षेपाभावे सति सोमग्रहणे सम्पर्कार्धं कृत्स्नं स्थित्यर्धक्षेत्रं न स्यात्~। यतो भूच्छायाबिम्बव्यासमार्गेणैव तदा चन्द्रो गच्छति~। (तदा ?) तद्व्यासार्धयोगतुल्यं भूच्छायाचन्द्रस्फुटविवरं यदा तदैव स्पर्शो मोक्षश्च स्याताम्~। तदा सम्पर्कार्धेन सकलेन स्थित्यर्धक्षेत्रत्वेन भूयेत~। तस्मान्मध्यविक्षेपेणैव स्थित्यर्धमानेयम्~। न पुनः स्पर्शमोक्षभवाभ्यां विक्षेपाभ्याम्~। अत एव केचित् विक्षेपसम्पर्कार्धवर्गविश्लेषे मध्यविक्षेपस्येष्टवर्गविक्षेपस्य च दिशोः साम्ये मध्येष्टविक्षेपान्तरवर्गं, भेदे च तद्विक्षेपयोगवर्गं च प्रक्षिप्य मूलीकृत्य स्थित्यर्धमानयन्ति~। युज्यते चैतत्~। विक्षेपमण्डले यत्र स्पर्शकाले चन्द्रस्तिष्ठति, यत्र च मध्यकाले, तदन्तरालं हि स्थित्यर्धक्षेत्रम्~। एवमानीयमानमपि तत् तुल्यमिति~। नैवं विक्षेपमण्डलकल्पनं युक्तम्~। केन पुनरिदं विक्षेपमण्डलं कल्पितम्~। ननु सर्वैरपि व्याख्यातृभिरपक्रममण्डलगतपातद्वयस्पृक्, ततः परिधिभाग\renewcommand{\thefootnote}{१}\footnote{पाद \textendash\ ग. घ.}योः परमविक्षेपान्तरालं 

\newpage

\noindent च विक्षेपमण्डलबन्धः प्रदर्शितः~। उत्सूत्रमेवेदं कल्पनम्~। आर्यभटेन त्वन्यथैव विक्षेपः प्रदर्शितः~। {\qt अपमण्डलस्य चन्द्रः
पाताद्यात्युत्तरणे दक्षिणतः} इति~। एतदुक्तं भवति \textendash\ इहानीता चन्द्रस्फुटसङ्ख्या अपक्रममण्डलमेषादितः प्रभृति यत् कलावधिका ततोऽपक्रममण्डलविपरीतमण्डलेनैव राशिकूटाभिमुखमेव विक्षिप्यमाणा ग्रहा यान्ति~। तस्मात् पातोनचन्द्रस्फुटकलाश्च पातात् प्रभृति चन्द्रविक्षेपमूलावधिका एव~। न पुनः पातात् प्रभृति कर्णरूपेण चन्द्रबिम्बघनमध्यावधिकाः कलाः तत्कल्पितविक्षेपमण्डलगताः~। तस्मात् अपक्रममण्डलगतिरेव स्फुटगतिः, न विक्षेपमण्डलगता~। तस्मात् तत्कल्पितविक्षेपमण्डलस्य सम्पर्कार्धान्तर्गतो यः खण्डः न तत्तुल्या ग्रहणस्थितिकालगत्यन्तरकलाः\renewcommand{\thefootnote}{१}\footnote{कालाः \textendash\ ग.}~। तस्मात् तन्मध्येष्टविक्षेपान्तरवर्गो वा तद्योगवर्गो वा सम्पर्कार्धेष्टविक्षेपवर्गविश्लेषे न क्षेप्यः~। एतच्च इष्टग्रासप्रदर्शने विस्पष्टम्~। {\qt विक्षेपवर्गसहितात्
स्थित्यर्धादिष्टवर्जितान्मूलम्~। सम्पर्कार्धाच्छोध्यम्} इति~। अत्रेष्टशब्देन स्पर्शस्थित्यर्धे गतं विवक्षितम्~। मोक्षस्थित्यर्धे च गन्तव्यम्~। तद्वर्जितं स्थित्यर्धं मध्येष्टकालान्तरगत्यन्तरम्~। तद्वर्गात् विक्षेपवर्गसहितान्मूलं ग्राह्यग्राहकबिम्बमध्यान्तरम्~। तस्मिन् सम्पर्कार्धाच्छुद्धे शिष्टस्तात्कालिको ग्रास इति~। तस्मात् अपक्रमण्डलानुसार्येव गत्यन्तरम्~। तस्मात् चन्द्रगतिरप्यपक्रममण्डलानुसारिण्येव ; न पुनर्विक्षेपमण्डलानुसारिणी~। यदि विक्षेपमण्डलानुसारिणी स्फुटगतिः, तर्ह्यपक्रममण्डलगतिसिद्धये लङ्कोदयानयनन्यायेनानीतोऽपक्रममण्डलभोग एव स्फुटत्वेन ग्राह्य इति स्फुटेऽपि प्राणकलान्तर वद्विशेषः स्यात्~। विक्षेपस्यापक्रमादल्पत्वान्न तावान् भेद इत्येव विशेषः इति चन्द्रस्फुटसिद्धये स्फुटीफरणात्परमपि यत्नान्तरं कार्यम्~। तच्च पूर्वाचार्येषु केनचिदपि न प्रदर्शितम्~। तस्मात् यदि विक्षेपमण्डलं कल्प्येत स्फुटयुक्तिवैशद्याय, तदापि ग्रहस्य तात्कालिकविक्षेपतुल्यमेवा-
 
\newpage

\noindent पक्रमविक्षेपमण्डलयोः परममन्तरालम्~। ग्रहाक्रान्तप्रदेश एव चापक्रमात् तावदन्तरितम्~। ततः चक्रपादान्तरितयोरेवापक्रमविक्षेपमण्डलयोः सम्पातश्च, न पातस्थितप्रदेशे~। ग्रहे पातस्थे तु मण्डलयोः कार्त्स्न्येनैव सम्पात इत्येकतां गतमेव मण्डलद्वयं तदा ततः पुनर्ग्रहैर्विक्षेपवशात् तदपक्रममण्डलवशान्निष्कृष्यमाणं तत्कालविक्षेपतुल्यमेव विव्रियते~। पाताच्चक्रपादान्तरिते ग्रहे तु मण्डलयोर्विवरं परमविक्षेपतुल्यमपि स्यात्~। अन्यदा सदैवेष्टविक्षेपतुल्यमेव तयोः परमं विवरम्~। अत्र पक्षद्वयेऽपि अपक्रमचापविक्षेपचापयोर्योगो वियोगो वा न स्फुटापक्रमचापं स्यात्~। अत एव गोलविदा माधवेन विक्षेपवतां स्फुटापक्रमानयने गणितविशेषः प्रदर्शितः\textendash  

\begin{quote}
{\qt परमापक्रमकोट्या विक्षेपज्यां निहत्य तत्कोट्या~।\\
 इष्टक्रान्तिं चोभे त्रिज्याप्ते योगविरहयोग्ये~स्तः~॥\\
सदिशोः संयुतिरनयोर्वियुतिर्विदिशोरपक्रमः~स्पष्टः~।\\
स्पष्टापक्रमकोटिर्द्युज्या विक्षेपमण्डले वसताम्~॥}
\end{quote}

\noindent इति~। इष्टविक्षेपज्यां परमापक्रमकोट्या {\qt शशिकृतशशिराम} सङ्ख्यया सावयवया निरवयवया वा निहत्य तत्कोट्येष्टविक्षेपकोट्या व्यासार्धकर्णभवया विक्षिप्तग्रहसम्बन्धिनीं इष्टक्रान्तिं जीवारूपां निहत्य ते उभे त्रिज्याप्ते ये ते योगविरहयोग्ये स्तः~। अत्रापि योगो वियोगो वा त्रिज्यया हार्यः~। कदा पुनस्तत्संयोगः, कदा वा वियोगः~। सदिशोरनयोः संयुतिः~। समाने दिशौ ययोः ते सदिशौ, तयोः संयुतिः संयोग एव कार्यः~। विदिशोस्तु वियुतिः~। विविधे नानारूपे दिशौ ययोः~। यदैकस्या याम्या दिगन्यस्या सौम्या तदा तयोर्वियुतिः कार्या~। एवं युक्तो वियुक्तो वा अपक्रमः स्पष्टः~। न पुनः चापद्वयसंयोगस्य वियोगस्य वा गुणः स्पष्टः~। एतावानेव विक्षेपवतां विशेषः~। तेषामपि स्वाहोरात्रव्यासार्धं स्पष्टापक्रमकोटिरेव विक्षेपमण्डले वसतां, न पुनर्गच्छताम्~। अपक्रममण्डल एव विक्षिप्ता अपि स्वगत्या गच्छन्ति~।

\newpage

\noindent अपक्रममण्डले पुनर्यत्र ग्रहो वसति तद्वसतेरेव विक्षेपः~। तस्मात् अपक्रममण्डलस्यापि न विक्षेपः स्यात्~। स्वस्वकलामार्गेणैव~ विक्षेपः~। भगोलस्य मध्यभागगतमेव ह्यपक्रममण्डलम्~। मध्यभागश्च सूक्ष्म एव, न पुनस्तत्रैव राशयस्सन्ति~। तत्पार्श्वयोरपि चक्रपादान्तं सन्त्येव हि राशयः~। अपक्रममण्डलपरिध्यवयवेभ्यश्चक्रपादान्तरितौ प्रदेशावेव राशिकूटाख्यौ~। तस्मात् स्वेनैव मार्गेण विक्षिप्तत्वात् ग्रहाणां स्फुटकलासु न भेदः स्यात्~। विक्षिप्तेऽपि तावदेव स्फुटम्~। न पुनः स्फुटस्याल्पत्वमधिकत्वं वा विक्षेपवशाज्जायत इति विक्षिप्तोऽपि तत्रैव वसति~। तत्र वसन्नेव विक्षिप्यते~। न पुनर्विक्षेपवशात् ततश्चलनमिति {\qt वसता}मित्यनेन प्रदर्शितम्~। यदि पुनर्व्याख्यातृकल्पितविक्षेपमण्डलगतिरेव स्फुटा, तदा तु अपक्रममण्डलविक्षेपमण्डलयोः सम्पातात् कलाशतकान्तरिते ग्रहे मण्डलसम्पातं केन्द्रं कृत्वा पातग्रहान्तरव्यासार्धं वृत्तं लिखेत्~। तत्र तत्सम्पातकेन्द्रात् कलाशतकशलाकायां भ्राम्यमाणायां पूर्वमपक्रममण्डलज्यारूपेण स्थिता शलाका~। ततो वृत्ताकारेण विकृष्यमाणा अपक्रममण्डलस्य स्फुटप्रदेशासन्नसमतिरश्चीना तज्ज्या~। तच्छलाकाव्यासार्धवृत्तगतत्वात्~। स्फुटात् पातासन्नप्रदेश एव तद्ग्रहस्यातिसन्निकृष्टोऽपक्रममण्डलप्रदेश इति स्फुटस्य अल्पत्वं स्यात्~। तदा न च स्फुटकलामार्गेण विक्षिप्यते~। अपक्रममण्डलकलायाः समतिरश्चीन एव स मार्गः~। तन्मार्गगमेव राशिकूटमण्डलम्~। दृक्क्षेपमण्डलमपि अपक्रममण्डलसमतिरश्चीनमेव~। तस्मात् देशविशेषापेक्षया अपक्रममण्डलस्य कृत्स्नस्यापि भागस्य एकदैवोदयलग्नत्वं अस्तलग्नत्वं दृक्क्षेपलग्नत्वं, मध्यलग्नत्वमपि सम्भवति~। एकस्मिन् देशे तु क्रमेणैव ते भवन्ति~। देशभेदेष्वेव हि सर्वदेशानां सर्वलग्नत्वं सम्भवति~। इति यत्राभीष्टकाले ग्रहस्फुटतुल्यं दृक्क्षेपलग्नं तत्स्थदृष्ट्यात्र
युक्तिः प्रदर्श्यते~। तत्र च निरक्षगोल एव स्फुटापक्रमस्य विक्षेपस्य च देशवशाद्विशेषाभावाद्यं कञ्चिद्द्रष्टारमपेक्ष्य ज्ञातोऽपक्रमादिकः सर्वसाधारण
एवेति प्रातिस्विके निरूपणेऽपि न भेदः~। तस्माद्दृक्क्षेपलग्नात् यदि

\newpage

\noindent घटिकामण्डलानुसारी विक्षेपः तदा तयोर्वियोगः स्फुटापक्रमः~। अपक्रममण्डलाद्बहिर्विक्षिप्ते सति संयोग एव स्फुटापक्रम इति तत्संयोगः प्रथमम् उदाह्रियते~। तत्र दृक्क्षेपलग्नात् दृक्क्षेपमण्डलानुसारेणापक्रममण्डलादधोगते ग्रहे दृक्क्षेपविक्षेपयोरेव योगयोग्यत्वं, न पुनरपक्रमविक्षेपयोः~। कः पुनः दृक्क्षेपलग्नापक्रमस्य दृक्क्षेपजीवायाश्च विशेषः दृक्क्षेपज्या अपक्रममण्डलविपरीतदिक्का, दृक्क्षेपापक्रममण्डलयोः परस्परं विपरीतदिक्कत्वात्~। दृक्क्षेपलग्नापक्रमस्तु घटिकामण्डलविपरीतदिक्क एव ; तस्मात् दृक्क्षेपज्याकर्णस्य कोटिरूपैव अपक्रमज्या~। तया
स्वकर्णरूपां दृक्क्षेपज्यामानीय तस्या विक्षेपज्यायाश्च दृक्क्षेपमण्डलगतत्वात् स्वमार्गगतया दृक्क्षेपज्ययैव, विक्षेपस्य योगो वियोगो वा कार्यः~। स च माधवोक्तवसन्ततिलकेनैव क्रियताम्~। एवं घटिकामण्डलादपक्रममण्डलविपरीतदिशा ज्ञातस्य ग्रहविप्रकर्षस्यापक्रमत्वाभावात्, घटिकामण्डलविपरीतदिक्कस्यैव अपक्रमत्वात् दृक्क्षेपविक्षेपयोगस्य वियोगस्य वा कर्णात्मकस्य कोटिरेव विक्षिप्तस्य ग्रहस्य स्फुटापक्रमज्या~। एवं यदानयनमिह प्रदर्शितं, तदेव निरूप्यमाणमन्याकारतां भजते~। तदेव कर्म {\qt परमापक्रमकोट्ये}त्युक्तम्~। तद्यथा \textendash\ इह ग्रहस्फुटादानीतेनापक्रमेण ग्रहस्फुटस्य दृक्क्षेपलग्नत्वे या दृक्क्षेपज्या निरक्षदेशजा, सा प्रथममानेया~। तच्च एवं \textendash\ 
सत्रिभग्रहद्युज्यया कोटिभूतया व्यासार्धतुल्यः कर्णो लभ्यते, तदेष्टापक्रमज्यया कोट्या कियानिति दृक्क्षेपो लभ्यते~। अत्रापि दृक्क्षेपप्रदर्शित एव क्षेत्रच्छेदः~। तत्र ग्रहस्य दृक्क्षेपलग्नत्वकल्पनायां सत्रिभग्रहस्फुटमेव प्राग्लग्नम्~। तदपक्रमज्यैवोदयज्यापि, निरक्षदेशविषयत्वान्निरूपणस्य~। तत्र {\qt मध्यज्योदयजीवासंवर्ग} इत्यस्य कर्णेन भुजानयनं क्रियते~। इहापि तस्मिन्नेव क्षेत्रे कोट्या कर्णानयनं क्रियत इति जिज्ञास्यभेदात् क्रियामात्रस्यैव भेदः~। न तु क्षेत्रस्यापि एकमेव हि उभयत्रापि क्षेत्रम्~। किन्तु, तत्र क्षितिजे दृक्क्षेपमण्डलान्तरालज्या
दक्षिणस्वस्तिकाग्रा कल्प्यते, तत्र मध्यज्याकर्णस्य दक्षिणोत्तरमण्डलगतत्वात्~। यतः

\newpage

\noindent तत्कोटिर्मध्यलग्नदृक्क्षेपः~। इह तु दृक्क्षेपमण्डलव्यासस्यैव कर्णत्वं, न दक्षिणोत्तरमण्डलस्य~। अतोऽत्र दृक्क्षेपमण्डलक्षितिजमण्डलयोर्योगान्ता दक्षिणस्वस्तिकादारब्धा\renewcommand{\thefootnote}{१}\footnote{दक्षिणस्वस्तिकाग्रा कल्प्यते~। तत्र मध्यज्या कर्णस्यादारब्धा \textendash\ ग. घ.} इहोदयज्यातुल्या तदन्तरालज्या~। ततः तत्कोटिमण्डलमपि समदक्षिणोत्तरमेव~। दक्षिणोत्तरमण्डलत उदयज्यातुल्यमेव सर्वत्र तद्विवरम्~। तच्च मकरादित्रिकस्थे ग्रहे दक्षिणोत्तरमण्डलात् प्राग्गतमेव, तदा सत्रिभग्रहस्य प्राग्लग्नस्य मेषादित्वात्~। उदयज्यायाः पदसन्धितः सव्याग्रत्वात् इयमपि दक्षिणस्वस्तिकात् सव्याग्रैव भुजा~। सव्यभागश्च दक्षिणाभिमुखस्य प्राग्भाग एवेति मेषादावुदयलग्ने दृक्क्षेपमण्डलकोटिः दक्षिणोत्तरमण्डलात् प्रागेव भुजान्तरितप्रदेशे वर्तते~। कर्क्यादिस्थेऽपि ग्रहे लग्नस्य जूकादित्वात् प्राग्गतमेव दृक्क्षेपकोटिमण्डलम्~। ओजपदस्थे ग्रहे प्रत्यगेव केवलम्~। सत्रिभग्रहयोर्दिक्साम्यात्~। तत्रोदयज्याया ग्रहस्य कोट्यपक्रमतुल्यत्वात् तत्कोटिर्ग्रहकोटिज्यातुल्यैव~। इह विक्षेपापक्रमयोगतः प्राक् पश्चाच्च त्रैराशिकं परस्परं वैपरीत्येनैव~। तयोः प्रथमत्रैराशिके कोट्या कर्ण आनीयते~। तत्प्रत्यापत्त्यर्थं पुनः कर्णेन कोटिश्च~। तत्र द्वितीययुक्तिः सुगमा अपक्रमवदेवेति~। तद्विपरीतत्वात् प्रथमस्यापि न दुरवबोधा~। तत्र घटिकामण्डलस्य ग्रहदृक्क्षेपमण्डलस्य च योगात् प्रभृति दृक्क्षेपलग्नान्तस्य कर्णस्य दृक्क्षेपज्यातुल्यस्य कोटिरेव~। घटिकामण्डलादपमण्डलाभीष्टप्रदेशविप्रकर्षोऽप्यपक्रमाख्यः~। तया त्रैराशिकम् \textendash\ ऊर्ध्वस्वस्तिकात प्रभृति क्षितिजान्तं यद्दृक्क्षेपमण्डलव्यासार्धं तद्यदि कोटिस्वाहोरात्रार्धस्य कोटिरूपस्य कर्णः, तदा ग्रहापक्रमतुल्यायाः कोट्याः कियान् कर्ण इति दृक्क्षेपो लभ्यते~। पुनः तस्य विक्षेपस्य च योग एवात्रोदाह्रियते~। तेनैव वियोगकल्पोऽपि सिद्ध्यतीति~। तत्र विक्षेपकोट्या दृक्क्षेपो हन्तव्यः~। दृक्क्षेपकोट्या च विक्षेपः~। पुनस्तद्योगः त्रिज्यया हर्तव्यः~। तेन कर्णेन तत्कोट्यानयने त्रैराशिकमेवं \textendash\ यदि खमध्यात् प्रभृति दृक्क्षेपमण्डलगतया व्यासार्धतुल्यया जीवया ग्रहकोटिस्वाहोरात्रतुल्या कोटिर्लभ्यते, तदा दृक्क्षेप-

\newpage


\noindent चापयोगज्यया कर्णभूतया कियतीति~। सैव स्फुटापक्रमज्या~। तत्र त्रयाणां कर्मणामेकीकरणं, प्रथमचरमयोर्मध्यमकर्मण्येव निलापनेन~। तत्र प्रथमत्रैराशिके त्रिज्या गुणकारः, कोटिस्वाहोरात्रार्धं भागहारः~। द्वितीये पुनः स्फुटापक्रमकर्णस्य कोटिस्वाहोरात्रार्धं गुणकारः, व्यासार्धं भागहारः~। तत्र प्रथमत्रैराशिकसिद्धस्य दृक्क्षेपस्य विक्षेपकोटिः गुणकारः, व्यासार्धं भागहारः, फलं विक्षेपेण संयोज्यम्~। पुनरपि तत्कर्मोभयविषयं क्रियते~। तदपि कृत्वा वा संयोगः~। तत्र वर्गसम्बन्धाभावात् फलविशेषाभावात्~। तत्र यदपक्रमविषयं त्रैराशिकत्रयं तत्राद्ये व्यासार्धं गुणकारः, कोटिस्वाहोरात्रार्धं भागहारः~। अन्त्ये व्यत्ययेन च~। ततः तदुभयं न कर्तव्यम्~। ततो मध्यमत्रैराशिकमेव तत्र कार्यम्~। तत्र विक्षेपकोटिः गुणकारः, व्यासार्धं च भागहारः~। यत् पुनर्विक्षेपविषयं त्रैराशिकद्वयं तत्राद्ये दृक्क्षेपकोटिः गुणकारः, व्यासार्धं भागहारः~।
तत्र त्रैराशिकक्रमभेदात् फलभेदाभावात् द्वितीयं वा प्रथमं क्रियताम्~। तत्र दृक्क्षेपकोटिः गुणकारः, व्यासार्धं भागहारः, फलम् अपक्रमकर्णेन
दृक्क्षेपकोटिः योगार्हम्~। अस्य केवलापक्रमेण योगार्हत्वाय पुनः कोटिस्वाहोरात्रगुणनं व्यासार्धहरणं च कार्यमिति प्रथमपक्षे क्रमः~। तत्र
व्यासार्धे भागहारे दृक्क्षेपकोटिः गुणकारः, कोटि\renewcommand{\thefootnote}{१}\footnote{टिः \textendash\ क.}स्वाहोरात्रे भागहारे को गुणकारः, इति दृक्क्षेपकोट्या ग्रहकोट्यपक्रम\renewcommand{\thefootnote}{२}\footnote{मः \textendash\ क.}स्वाहोरात्रस्य च घातात् द्युव्यासार्धेन हृतं फलं गुणकारः~। तच्च परमापक्रमकोटितुल्यम्~। कुतः ? उच्यते \textendash\ सर्वथापि कोटिः गुणकारः, कर्णो भागहारः, तत्र व्यासार्धस्य भागहारत्वे दृक्क्षेपो भुजा, तयोः कोटिर्गुणकारः, कोटि\renewcommand{\thefootnote}{३}\footnote{टिः \textendash\ क.}स्वाहोरात्रस्य भागहारत्वे अपक्रम एव भुजा। तद्भुजानयने एवं त्रैराशिकम्। व्यासार्धकर्णस्य दृक्क्षेपो भुजा; तदा कोटि\renewcommand{\thefootnote}{४}\footnote{टिः \textendash\ क.}स्वाहोरात्रकर्णस्य का भुजेति~। पूर्वमपक्रमादेव दृक्क्षेप आनीतः ; एतद्विपरीतमेव तत्कर्म~। तत्र त्रैराशिकत्रये प्रथमत्रैराशिकेन दृक्क्षेप आनीयते~। तत्रापक्रमस्य व्यासार्धं गुणकारः~।

\newpage

\noindent कोटि\renewcommand{\thefootnote}{१}\footnote{टिः \textendash\ इत्येव प्रायेण सर्वत्र क पाठे}स्वाहोरात्रार्धं भागहारः~। तद्व्यत्ययेन दृक्क्षेपादपक्रमानयने~। तस्मात् कोटिस्वाहोरात्रस्य भागहारत्वे तद्व्यासार्धवृत्तगता भुजा दृक्क्षेपस्थानीया अपक्रम एव~। तस्मात् अपक्रमवर्गं कोटिस्वाहोरात्रवर्गात् विशोध्यापि मूलीकृतं यत् सैवेह कोटिः~। तस्या एव गुणकारत्वमपि~। कोटिस्वाहोरात्रस्य भागहारत्वे, कोटिस्वाहोरात्रं च ग्रहकोट्यपक्रमवर्गं व्यासार्धवर्गात् विशोध्य मूलीकृतमेव~। तस्मात् कोटिस्वाहोरात्रार्धस्य वर्गो, व्यासार्धवर्गात् ग्रहकोट्यपक्रमवर्गरहितः~। तस्मात् भुजापक्रमवर्गश्च शोध्यः~। कोट्यपक्रमवर्गस्य भुजापक्रमवर्गस्य च योगः परमापक्रमवर्ग एव, परमापक्रमज्यावृत्तगतत्वात् तयोः~। तस्मात् व्यासार्धपरमापक्रमयोर्वर्गविश्लेषमूलमेव गुणकारः~। तस्यैव कोटिस्वाहोरात्रकर्णस्य ग्रहापक्रमभुजायाश्च कोटित्वात्~। तस्मात् विक्षेपज्यां परमापक्रमकोट्या निहत्य, कोटिस्वाहोरात्रेण हृत्वा, पुनर्द्वितीयत्रैराशिके कोटिस्वाहोरात्रेण गुणनं, त्रिज्यया हरणं च कार्यम्~। तस्मात् कोटिस्वाहोरात्रेण गुणनं, हरणं चोभयमपि न कर्तव्यम्~। तस्मात् परमापक्रमकोटिः गुणकारः, व्यासार्धं भागहारः~। अत उक्तम् \textendash\ {\qt परमापक्रमकोट्या विक्षेपज्यां निहत्य तत्कोट्या इष्टक्रान्तिं चोभे त्रिज्याप्ते} इति~। त्रैराशिकक्रमव्यत्ययेऽपि प्रथमं विक्षेपं
कोटिस्वाहोरात्रेण हत्वा त्रिज्यया हृत्वा पुनः परमापक्रमकोट्या हत्वा स्वाहोरात्रेण हरणं च कार्यम्~। तत्रापि कोटिस्वाहोरात्रस्यैव गुणकारत्वं भागहारत्वं चेति विक्षेपस्य परमापक्रमकोटिः गुणकारः, व्यासार्धं भागहार इति~। अपक्रमे पुनरेकमेव त्रैराशिकं कार्यम्~। तत्र विक्षेपकोटिस्त्रिज्या च
गुणकारभागहारौ~। तस्मात् एवमेव दृक्क्षेपमण्डलेन राशिकूटद्वयबद्धेन मार्गेणापक्रममण्डलाद्विक्षिप्तस्य ग्रहस्य तद्विक्षेपस्य तिर्यक्त्वात्
घटिकामण्डलापेक्षया ऋजुत्वमानेयम्~। ऋजुत्वं च कोटिस्वाहोरात्रगुणनत्रिज्या-

\newpage

\noindent हरणाभ्यां स्यात्~। पुनस्तस्यापक्रमस्य कोटिस्वाहोरात्रवृत्तगतत्वात् तद्योगवियोगयोः {\qt जीवे परस्परे}त्याद्युक्तकर्मणा क्रियमाणयोः स्ववृत्तकोट्यानयने स्वस्ववर्गं कोटिस्वाहोरात्रवर्गाद्विशोध्य शेषस्य मूलीकरणं कार्यम्~। तयोरितरस्य कोट्या द्वौ भुजौ गुणयित्वा कोटिस्वाहोरात्रेणैव हृत्वा योगो वियोगो वा कार्यः~। एतदेव प्रथमतः प्राप्तं कर्म~। तत्र विक्षेपस्य ऋजूकरणायापक्रमयोगवियोगयोग्यत्वाय च ये त्रैराशिके तयोरेकीकरणमेवेह प्रदर्शितम्~। एवमस्य युक्तियुक्तत्वात् केवलयोश्चापयोर्योगात् वियोगात् वा गृहीताया ज्यायाः स्थूलत्वमेव, भिन्नदिक्कत्वात् उभयोः~। घटिकामण्डलविपरीतदिगेवापक्रमज्या~। अपक्रममण्डलविपरीतदिगेव विक्षेपज्या~। घटिकामण्डलविप्रकर्ष एवापक्रमश्च~।
तस्माद्विक्षेपस्यापि घटिकामण्डलविपरीतदिक्कत्वायेह प्रथमं त्रैराशिकं क्रियते~। एतत् सर्वमपमण्डलस्य {\qt चन्द्रः पातात् यात्युत्तरेण दक्षिणत} इति~। विक्षेपस्वरूपदर्शनादेव सूचितम्~। आयनदर्शनसंस्कारोऽपि तत एव सिद्धः~। तद्विवरणमेव {\qt विक्षेपापक्रमगुण}मित्येतत्~। तदेव विक्षेपवृत्तगतं वलनम्~। तत्र कोट्यपक्रमस्य व्यासार्धमण्डलवलनस्य विक्षेपगुणनं व्यासार्धहरणं च विक्षेपमण्डलगतत्वायैव~। अत एव वलनस्य सिद्धत्वात् त्रिराशिसहितायनादित्येतावदेवोक्तम्~। यदुक्तमिह ग्रहणे तत् सर्वं ग्रहयोगे ग्रहनक्षत्रयोगे च समानम्~। तत्रापि द्वितीयस्फुटचन्द्र एव ग्राह्यः~। ग्रहणेऽपि तद्गतिरेव चन्द्रस्य ग्राह्या~। योजनकर्णानयने कलाकर्णश्चाविशिष्टमन्दकर्णे द्वितीयकोटिफलं संस्कृत्य तद्भुजावर्गयोगात् सिद्ध एव
ग्राह्यः~। तत्सिद्धेन स्फुटयोजनकर्णेनैव लम्बनबिम्बमानाद्यानयनमपीष्यते~। व्यतीपातानयने द्वितीयस्फुटमकिञ्चित्करम्~। विक्षेपश्च
व्यासार्धहृत एव ग्राह्यः~। बिम्बमानमपि मन्दकर्णसिद्धस्फुटयोजनकर्णेनैव नेयम्~। सूर्येन्द्वोर्ह्रसद्वर्धमानापक्रमयोः साम्य एव व्यतीपातमध्यम्~।
यतस्तयोरेव साम्ययोः चन्द्रार्कयोगजत्वम्~। इतरयोस्तिथिजत्वमेव, पक्षान्तसामीप्यात्
 
\newpage


\noindent तयोः~। तत्स्पर्शमोक्षौ च स्थित्यर्धन्यायेनैव सिद्धौ~। चन्द्रस्य प्रतिक्षणं परमापक्रमभेदेऽपि सायनपातात् तात्कालिकं परमापक्रमं प्रथमं नीत्वा पुनरेकेनैव त्रैराशिकेन स्फुटापक्रमज्या ज्ञातुं शक्या~। तद्युक्तिप्रदर्शनाय अपक्रममण्डलस्य पातद्वयस्पृष्टं ततश्चक्रपादे परमविक्षेपान्तरितं
पूर्वैरुक्तं विक्षेपमण्डलमप्यङ्गीक्रियताम्~। अपक्रममण्डलभुक्तभाग एव स्फुटः~। न त्वेवं कल्पितविक्षेपमण्डलभुक्तभागः~। उभयोर्भेदात्~। यत् पुनरन्यादृशविक्षेपमण्डलकल्पनमस्माभिरुक्तं, तत् उभयोर्भुक्तभागसाम्यायैव~। यथेहापक्रमविक्षेपमण्डलयोः सम्पातः सर्वत्र स्यात् न तया घटिकाविक्षेपमण्डलयोः~। तत्सम्पातस्तु विषुवतः कतिपयकलाधिकभागदशकान्तमेव चलति~। परमापक्रमस्य च पातकोटिवशात् प्रायेण सार्धशतद्वयकलाभिर्मृगकर्क्याद्योर्वृद्धिर्ह्रासश्च स्यात्~। एतच्च निरक्षगोले प्रदर्श्यम्~। तत्र तावत् प्रथमद्वितीयपातयोः प्राक्पश्चिमविषुवत्स्थयोः परमविक्षेपापक्रमचापयोगज्यैव परमापक्रमज्या~। अपक्रममण्डलविषुवत्येव विक्षेपमण्डलविषुवच्च~। तयोरेव पातयोर्याम्योदग्विषुवत्स्थयोस्तच्चापवियोगज्या च, इत्येतत् सुगममेव~। विक्षेपवतामपि तात्कालिकपरमापक्रमो ध्रुवद्वयस्य गोलस्थस्य विक्षेपमण्डलपार्श्वद्वयस्य च विवरमेव सर्वदापि~। तद्युक्तिप्रदर्शनार्थमपक्रममण्डले विषुवतः प्रभृति विक्षेपमण्डलं पातवद्वामं भ्रामयेत्~। तदापक्रममण्डलपार्श्वद्वयं राशिकूटाख्यं मेधीकृत्य विक्षेपमण्डलपार्श्वद्वयमपि भ्रमति~। यथा प्रवहेण भ्राम्यमाणेऽपक्रममण्डले ध्रुवद्वयं मेधीकृत्य
राशिकूटद्वयमपि भ्रमति, तथा चन्द्रस्य तात्कालिकस्फुटकक्ष्यागोलस्थं राशिकूटं मेधीकृत्य विक्षेपमण्डलपार्श्वद्वयं, पातेन भ्राम्यमाणं विक्षेपमण्डलमनु भ्रमति~। तद्भ्रमणवृत्तमपि परमविक्षेपव्यासार्धं राशिकूटनाभिकम्~। यथा राशिकूटद्युवृत्तव्यासार्धं परमापक्रमतुल्यम्~। एवं भ्रमतस्तस्य ध्रुवस्य चान्तरालं यावत् तावदेव हि घटिकामण्डलस्य विक्षेपमण्डलस्य च तात्कालिकं परमं विवरम्~। घटिकामण्डलस्य ग्रहभ्रमणवृत्तस्य च परममन्तरालमेव 

\newpage

\noindent हि परमापक्रमः~। स च चन्द्रस्य विक्षेपमण्डलभ्रमणवशात् प्रतिक्षणं भिन्नः~। तदानयनं चैवम्\textendash 
\begin{quote}
{\qt परमक्षेपकोटिघ्नं जिनभागगुणं हरेत्~।\\
त्रिज्यया क्षेपवृत्तस्य नाभ्युच्छ्रय इहाप्यते~॥

पातस्य सायनस्याथ दोःकोटिज्ये उभे हते~।\\
क्षिप्त्या परमया त्रिज्याभक्ते स्यातां च तत्फले~॥

अन्त्यद्युज्याहतं तत्र कोटिजं त्रिज्यया हरेत्~।\\
नाभ्युत्सेधे च तत्स्वर्णं मृगकर्क्यादिपातजम्~॥

तद्बाहुफलवर्गैक्यमूलं क्रान्तिः परा विधोः~।\\
त्रिज्याघ्नं दोःफलं भक्तं तया चलनमायनम्~॥

जूकक्रियादिगे पाते स्वर्णं तत्सायने विधौ~।\\
तद्बाहुज्या हता क्रान्त्या तदा परमया स्वया~॥

त्रिज्याप्तापक्रमज्येन्दोः स्फुटा तात्कालिकी भवेत्~॥}
\end{quote}

\begin{sloppypar} 
\noindent इति~। इह विक्षेपमण्डलपार्श्वभ्रमणवृत्तनाभ्युच्छ्रयानयनयुक्तिज्याद्वयसंयोगवियोगयुक्त्यन्तर्भूतैव~। इह विषुवत्स्थे पाते कर्म च तदेव~। तदा योगवियोगयोग्यायोर्गुणयोरेकः परमापक्रमः, इतरश्च परमविक्षेप एव~। तत्र तयोः परस्परकोटिगुणनं, त्रिज्यया हरणं, फलयोः संयोजनं वियोजनं, वा कर्म~। इहोक्तमपि तदेव, एवं ह्यत्र प्रथमत्रैराशिकम्~। यदि भगोलमध्यात् राशिकूटान्तस्य व्यासार्धकर्णस्य परमापक्रमज्यातुल्या राशिकूटोन्नतिर्भुजा तदा तस्मिन्नेव व्यासार्धे परमविक्षेपकोटितुल्यस्यांशस्य कियतीति~। यद्ययनाग्रस्यापक्रममण्डलव्यासार्धकर्णस्यान्त्याद्युज्यातुल्या कोटिः, तदा विक्षेपमण्डलपार्श्वभ्रमणवृत्तस्य व्यासार्धस्य परमविक्षेपतुल्यस्य कियतीति द्वितीयम्~। तेनात्रापि परयोः क्रान्तिविक्षेपयोः परस्परकोटिगुणनं, त्रिज्याहरणं च क्रियते~। उदग्विषुवत्स्थे पाते
फलयोः संयोजनमितरस्थे वियोजनं चेत्युभयत्राभिन्नमेव कर्म~। अनयोः
\end{sloppypar} 
\newpage

\begin{sloppypar} 
\noindent प्रथमत्रैराशिकं तात्कालिकेन्दुपरमापक्रमानयने सर्वदापि समानम्~। पातभुजायां सत्यां त्रैराशिकद्वयफलसंयोगवियोगतोऽतिरिक्तं कर्म तद्भुजावर्गयोग\renewcommand{\thefootnote}{१}\footnote{जाफलवर्गयोग \textendash\ ख. ग.}मूलीकरणमपि स्यात् द्वितीयत्रैराशिके चायं विशेषः \textendash\ यद्यपमण्डलानयनान्तस्य व्यासार्धस्यान्त्यद्युज्यातुल्या कोटिः, तदा तत्समानदिशः परमविक्षेपवृत्तव्यासार्धस्य\renewcommand{\thefootnote}{२}\footnote{वृत्तस्यार्धस्य \textendash\ ग.} भुजाकोटिफलतुल्यांशस्य कियतीति~। प्रथमत्रैराशिके परमापक्रमस्य परमविक्षेपकोटिरेव गुणकारः~। न पुनः योगयोग्ययोरितरस्य कोटिफलस्य कोटिः~। अतः प्रथमत्रैराशिकेच्छाफलमपि तुल्यमेव सदा इति सकृदानीयावधार्यमेव~। तच्छ्रोतव्यं वा गुरोः~। कथं भूते पुनरिह दोः कोटिफले? उच्यते \textendash\ यदिह प्रागेव प्रदर्शितं परमविक्षेपतुल्यव्यासार्धवृत्तं राशिकूटद्वयबद्धसूत्रस्थनाभिकं, तत्र भ्रमतो विक्षेपमण्डलपार्श्वस्य भुजाकोटिज्ये एव ते~। तस्य राशिकूटायनस्पृष्टमण्डलपरमविक्षेपवृत्तसम्पाताद्विप्रकर्षो भुजा, राशिकूटविषुवत्स्पृष्टमण्डलविप्रकर्षः कोटिः~। तत एव तद्वृत्तनिष्पादनाय सायनपातदोःकोटिज्ययोः परमविक्षेपगुणनं त्रिज्यया हरणं\renewcommand{\thefootnote}{३}\footnote{ज्याहरणं \textendash\ ख. ग. घ.} च क्रियते~। तथैवेदम्~। यथा स्फुटकर्मण्युच्चनीचवृत्तव्यासार्धत्रिज्याभ्यां गुणहाराभ्यां स्वपरिधिगतदोःकोटिफलनिष्पादनं कर्णानयनमप्युभयत्र समानम्~। किन्तु अत्र कोटिफलस्य तिर्यक्त्वात् द्वितीयत्रैराशिकेन तदृजूकरणे तस्याल्पत्वं स्यात्~। तत्र कक्ष्यामण्डलोच्चनीचवृत्तान्तरे व्यासार्धतुल्य एव सदा कोटिफलं संस्क्रियते~। अत्र गोलाक्षदण्डविक्षेपवृत्तकेन्द्रान्तरे तदिह प्रथमत्रैराशिकानीतम्~। तत् सदा तुल्यमेव~। कोटिफलसंस्कृतस्य तस्य दोःफलस्य च वर्गयोगमूलं कर्णः~। तत्तुल्या हि घटिकामण्डलपार्श्वस्थध्रुवात् प्रवृत्ता विक्षेपमण्डलपार्श्वाग्रज्या~। सा च
घटिकामण्डलविक्षेपमण्डलपरमान्तरालतुल्या~। समानगोलगयोः गोलघनमध्यनाभिकयोर्मण्डलयोः परमान्तरालतुल्यत्वात् पार्श्वान्तरस्य तत्प्रदर्शनाय दारुमयं 
\end{sloppypar}
\newpage

\noindent गोलं पूर्वापरदक्षिणोत्तरसमतिरश्चीनरेखात्रयविभक्ताष्टखण्डं विन्यस्यैकस्यां रेखायां वंशशलाकादिकृतं तत्तुल्यं मण्डलं विन्यस्य उभयोः प्रतिस्वस्तिकयोरवष्टभ्यान्ययोः प्रतिस्वस्तिकयोर्गृहीत्वा भ्रामयेत्~। भ्राम्यमाणस्य तस्य गृहीतभागयोः तत्स्वस्तिकाभ्यां विप्रकर्षो यथा यथा वर्धते तथा तथैव तद्रेखाविपरीतरेखयोः सम्पाताभ्यां मण्डलपार्श्वयोरपि विप्रकर्षो वर्धते इत्येतत् युक्त्या परिमायापि निर्णेतुं शक्यम्~। मण्डलात्
तत्पार्श्वाभ्यां सदापि गोलपादान्तरिताभ्यां भाव्यम्~। यतः अपक्रममण्डलायनाभ्यां विषुवद्भ्यां च विक्षेपमण्डलानयनयोः विषुवतोरपि
पातवशाच्चलनं स्यात्~। तदुक्तम्\textendash 
\begin{quote}
{\qt त्रिज्याघ्नं दोःफलं भक्तं तया चलनमायनम्}
\end{quote}

\noindent इति~। अत्रोक्तं भुजाफलं त्रिज्यया हत्वा अत्र कर्णतयानीतया तात्कालिकपरमक्रान्त्या हरेत्~। तत्र लब्धं विक्षेपमण्डलस्य अयनचलनम्~। तच्च सायने पाते जूकादिगे सायने विधौ धनं कुर्यात्~। मेषादिगे पाते सायनचन्द्रात् विशोधयेत्~। अत्रेदं त्रैराशिकं \textendash\ यदि ध्रुवसंश्लिष्टयोः विक्षेपापक्रममण्डलपार्श्वप्रापिणोर्गोलपृष्ठगतयोर्मण्डलयोर्विवरं तात्कालिकपरमक्रान्तिव्यासार्धमण्डले भुजाफलतुल्यं, तदा त्रिज्यातुल्यव्यासार्धे घटिकामण्डले कियतीति~। अयनचलनमेव विषुवच्चलनमपि~। स्वस्वायनात् राशित्रयान्तरितत्वात् स्वस्वविषुवतः~। अयनचलनाभावे तु एतत्संस्कृतचन्द्रबाहुज्यायाः तात्कालिकपरमक्रान्त्या त्रैराशिकेन स्फुटापक्रमज्यानेया~। गोलायनपदविभागश्च विहितोभयायनचलनवशादेव~। एतत् सर्वं कालक्रियापादोक्तस्फुटयुक्तिन्यायेनैव सिद्धम्~। यद्दृक्क्षेपानयनमुक्तमस्माभिश्चन्द्रच्छायाविषये प्रबन्धे\textendash 

\begin{quote}
{\qt अन्त्यद्युज्याहताक्षाद्यं\renewcommand{\thefootnote}{१}\footnote{द्यत्रिज्याप्तं \textendash\ क. घ.} त्रिज्याप्तं यश्च लम्बकः~।\\
काललग्नोत्थकोटिघ्नः करार्थाब्ध्युरगैर्हृतः~॥}
\end{quote}

\newpage

\begin{quote}
{\qt दृक्क्षेपस्तद्भिदैक्यं च काले कर्किमृगादिगे~।\\
विश्लेषे लम्बजाधिक्ये सौम्यो याम्योऽन्यदा सदा~॥}
\end{quote}
\begin{sloppypar} 
\noindent इति तत्राप्ययमेव न्यायो योज्यः~। अत्र राशिकूटोन्नतिरेवानीयते, तत्तुल्यत्वात् दृक्क्षेपस्य~। तत्रापि राशिकूटद्युज्यावृत्तनाभ्युच्छ्रय
एव {\qt अन्त्यद्युज्याहताक्षाद्यं त्रिज्याप्तम्}\renewcommand{\thefootnote}{१}\footnote{क्षाद्यत्रिज्याप्तम् \textendash\ घ.}, इत्यनेनानीयते~। स एकः खण्डः~। परमापक्रमस्वदेशलम्बककाल\renewcommand{\thefootnote}{२}\footnote{लम्बकाल \textendash\ ग.}लग्नकोटिघातात् त्रिज्यावर्गाप्तः इतरः खण्डः~। मृगादिगे काललग्ने तयोर्योगो राशिकूटशङ्कुः~। कर्क्यादिगे तु वियोगः~। वियोगे द्वितीयखण्डाधिक्ये शिष्टो याम्यराशिकूटशङ्कुः~। तद्वैपरीत्येन दृक्क्षेपलग्नदिक्~। एकस्य क्षितिजादूर्ध्वगतः शङ्कुरितरस्य क्षितिजादधोगतश्च सदापि तुल्यावेव~। कुतोऽत्र काललग्नभुजापक्रमवर्गीकरणादिकं न क्रियते~। तद्विनैव दृक्क्षेपसिद्धेः~। यतो राशिकूटशङ्कुतुल्य एव दृक्क्षेपः, न पुनर्याम्योदक्स्वस्तिकराशिकूटान्तरज्यातुल्यः~। दृक्क्षेपमण्डलस्य द्रष्ट्रपेक्षया अधऊर्ध्वत्वात् तस्यापक्रममण्डलसम्पातस्य दृक्क्षेपलग्नत्वात् राशिकूटशङ्कुरेव तच्छायातुल्यः, यतो दृक्क्षेपमण्डलगतं खमध्यक्षितिजान्तरालं, दृक्क्षेपलग्नराशिकूटान्तरालं च मण्डलचतुर्भागतुल्यं,
कर्णस्य च तिर्यक्त्वान्न शङ्कुत्वम्, अतोऽस्य शङ्कोः काललग्नभुजापक्रमस्य च वर्गयोगादिकं न कार्यम्~। यद्वा
स्वाहोरात्रेष्टज्यामित्याद्युक्तशङ्क्वानयनमेवेदम्~। अत्र क्षितिज्यासम्बन्धी शङ्कुखण्डः प्रथममानीयते~।
राशिकूटापक्रमस्यान्त्यस्वाहोरात्रतुल्यत्वात्~। तेन गुणितामक्षज्यां लम्बकेन हत्वा हि स्वाहोरात्रगता क्षितिज्यावाप्यते~। तस्या अप्यवलम्बकहताया
विष्कम्भार्धेन लब्धः क्षितिजोन्मण्डलान्तरालस्वाहोरात्रेष्टज्यासम्बन्धी शङ्कुः~। तत्र लम्बकगुणनहरणयोः अकरणेऽन्त्यद्युज्यैवाक्षज्याया गुणकारः, त्रिज्यैव भागहारः~। फलं राशिकूटद्युज्यावृत्तनाभ्युच्छ्रयः~। पूर्वोक्तयुक्त्यैकमेवेदं त्रैराशिकम्~। खण्डान्तरानयने तु इष्टकालराशिकूटोन्मण्डल- 
\end{sloppypar} 
\newpage

\noindent संयोगकालयोरन्तरालासव एव काललग्नकोटिचापकलाः~। ततस्तज्ज्यां परमापक्रमतुल्येन स्वाहोरात्रेण हत्वा त्रिज्यया हृत्वा उन्मण्डलात् प्रवृत्ता स्वाहोरात्रेष्टज्या लभ्यते~। साप्यवलम्बकाहता विष्कम्भार्धेन भाज्या~। तत्र लम्बकापक्रमौ गुणकारौ, त्रिज्यावर्गश्च भागहारः~। तत्र लम्बकस्य अनित्यत्वात् तद्धननं कार्यमेव~। परमापक्रमस्य गुणकारस्य त्रिज्यावर्गस्य भागहारस्य च लघुतन्त्रन्यायेनाल्पीकरणं कार्यम्~। तत्रेदं त्रैराशिकं \textendash\ यदि परमापक्रमगुणकारे त्रिज्यावर्गो भागहारः तदा रूपे गुणकारे कियानिति~। लब्धो हारः करार्थाब्ध्युरगसङ्ख्यः, अत उक्तं \textendash\ {\qt यश्च लम्बकः काललग्नोत्थकोटिघ्नः करार्थाब्ध्युरगैर्हृत} इति~। फलयोः संयोगो वियोगो वा राशिकूटशङ्कुरिति, चन्द्रस्य शृङ्गोन्नतिरपि वलनानुसारिणी~। विक्षेपे सति तद्विपरीतदिक्कं तृतीयमपि वलनं स्यात्~। तदुक्तं मानसे \textendash  

\begin{quote} 
{\qt विक्षेपव्योमधृत्यंशसंस्कृतं वलनं स्फुटम्~।}
\end{quote}  
इति~। तत्स्थूलं लाघविकत्वात् आचार्यस्य~। किञ्च भास्करमुञ्जालाद्युक्ता शृङ्गोन्नतिः सममण्डलस्थ एव शशिनि संवदते, नान्यत्र~। यतश्चन्द्रबिम्बमध्यदक्षिणोत्तरसूत्रात् तच्छृङ्गयोः प्राक् पश्चाच्च विप्रकर्ष एव तत्र ज्ञायते~। दृङ्मण्डलसमतिरश्चीनसूत्राच्छृङ्गयोर्विप्रकर्ष एव हि
शृङ्गोन्नतिः~। सा च रवीन्द्वोः छाया भुजासाम्येऽपि स्यात्~। तदुक्तपरिलेखने तदा समे एव शृङ्गाग्रे~। समदिक्कयोर्भुजयोर्विश्लेष एव परिलेखने बाहुत्वेनेष्टत्वात् तस्य शून्यत्वात्~। तस्मात् दृङ्मण्डलानुसारिणी शृङ्गोन्नतिरन्यादृशी~। तदानयनं च तन्त्रसङ्ग्रहे प्रदर्शितम्\textendash  

\begin{quote}
{\qt कृतलम्बनचन्द्रार्कविवराज्ज्याशरौ नयेत्~।\\
 तज्ज्यामिन्दुनतीषुघ्नां त्रिज्याप्तां गुणतस्त्यजेत्~॥
 
नतीषुफलकृत्योश्च भेदान्मूलमिषौ क्षिपेत्~।\\
गुणबाणौ तथाभूतावुच्येते विवरोद्भवौ~॥}
\end{quote}

\newpage

\begin{quote}
{\qt विवरोत्थशरस्यार्कनतिबाणस्य चान्तरम्~।\\
अन्तरज्या च या यच्च नतिज्याविवरं तयोः~॥

योग एव दिशोर्भेदे नत्योस्तत्र शशीनयोः~।\\
त्रयाणां वर्गयोगस्य मूलं बिम्बान्तरं स्फुटम्~॥

कृतलम्बनचन्द्रार्कविवरे भत्रयाधिके~।\\
तदन्तरदलस्यैव ज्या ग्राह्या न शरस्तदा~॥

रवीन्दुनतबाणघ्ना द्विष्ठा त्रिज्याहृता पृथक्~।\\
योज्ये स्वस्वफलोने ते जीवा सैवान्तरोद्भवा~॥

नतीषुफलकृत्योर्ये भेदमूले तदन्तरम्~।\\
द्वितीयं चरमः प्राग्वन्नतियोगो भिदापि वा~॥

त्रयाणामपि वर्गैक्यमूलं बिम्बान्तरं तदा~।\\
सदा बिम्बान्तरार्धस्य चापं द्विगुणमन्तरम्~॥

रवीन्द्वोर्वलये तत्र द्रष्टृमध्योभयस्पृशि~।\\
तद्बाहुज्या च कोटिज्या चन्द्रदृक्कर्णताडिता~॥

भुजाकोटिफले भक्ते रविदृक्कर्णयोजनैः~।\\
ताभ्यां त्रिभज्यया कर्णाच्छीघ्रन्यायेन दोःफलम्~॥

चापितं धनमेवात्र बिम्बान्तरधनुष्यदः~।\\
कर्क्येणादौ त्रिभज्यायां स्वर्णं कोटिफलन्त्विह~॥

सितमानार्थमेवैवं रवीन्द्वन्तरमिष्यते~।\\
उत्क्रमज्या ततो ग्राह्या क्रमज्या च समे पदे~॥

बिम्बमानाहताद्बाणात् त्रिज्याढ्याच्च पदाधिके~।\\
तद्गुणात् कृत्स्नविष्कम्भभक्तमेव सितं तदा~॥

प्राग्वच्छायाभुजां भानोः स्वाग्राभ्यां च विधोर्नयेत्~।\\
शङ्क्वग्रं सौम्यदिक्कं स्याद् अदृश्यार्धगते ग्रहे~॥}
\end{quote}

\newpage

\begin{quote}
{\qt दृक्कर्णानयने शङ्कोः फलमप्यत्र योजयेत्~।\\
व्यासार्धात् तद्धताच्छायाभक्ते क्षितिजगे उभे~॥

योगस्तद्धनुषोः कार्यो यथा युक्त्यन्तर तथा~।\\
छायाबाह्वोर्दिशोर्भेदे योगः साम्येऽन्तरं तथा~॥

विश्लेषे चन्द्रबाहुश्चेत् शिष्टः स्याद् व्यत्ययेन दिक्~।\\
सूर्यस्यैव तयोऽन्यत्र ग्राह्या दिक् योगभेदयोः~॥

स्वभूम्यन्तरनिघ्ना स्वा दृग्ज्या दृक्कर्णभाजिता~।\\
अर्केन्द्वोः सुस्फुटा दृग्ज्या द्रष्टुर्भूपृष्ठगस्य हि~॥

बाहुचापान्तरज्याघ्ना दृग्ज्या त्रिज्याहृता रवेः~।\\
चन्द्रबिम्बार्धनिघ्नाथ बिम्बान्तरभुजा हृता~॥

उन्नतिश्चन्द्रशृङ्गस्य नतिर्वार्धगुणात्मिका~।\\
वर्गत्रयैक्यमूलस्य दलस्य द्विगुणं धनुः~॥

यत्तस्य बाहुजीवात्र बिम्बान्तरभुजोदिता~। \\
चन्द्रबिम्बार्धमानेन लिखेद्वृत्तं तु तद्गते~॥

रेखे द्वे दिग्विभागार्थं प्रत्यग्रेखाग्रतः पुनः~।\\
नीत्वा शृङ्गोन्नतेर्मानं प्राग्वदर्धगुणात्मकम्~॥

चन्द्रादर्कदिशीन्दोस्तु परिधौ प्राग्विपर्ययात्~।\\
बिन्दुं कृत्वा लिखेद्रेखां तन्मार्गेण सितं नयेत्~॥

प्रत्यगग्रात् सिते पक्षे प्रागग्रादसितेऽपि च~।\\
सितान्ते बिन्दुमाधाय तिर्यग्रेखाग्रयोस्ततः~॥

बिन्दू कृत्वा लिखेद्वृत्तं बिन्दवो नेमिगा यथा~।\\
वृत्तान्तराकृतिश्चन्द्रः शृङ्गोन्नत्या प्रदर्श्यताम्~॥

व्यस्तदिक्कोऽर्कबाहुः स्यात् तयोर्नानाकपालयोः~।\\
प्रत्यासन्नरवेर्भागात् इहाप्यन्तर्नयेत्सितम्~॥}
\end{quote}
 
\newpage

\begin{quote} 
{\qt अन्यस्मादसितं वापि सर्वमन्यद्यथोदितम्~॥}
 \end{quote}

\noindent इति~। अत्र प्रथमं सार्धेनानुष्टुबष्टकेन बिम्बान्तरानयनं प्रदर्श्यते, तन्मूलत्वात् सितमानस्य, दृङ्मण्डलात् द्विस्पृग्वृत्तवलनस्य च~। तत्र पद्यचतुष्के सामान्यविधिः ; शेषे विशेषविधिः~। द्रष्टरि घनभूमध्यस्थे चन्द्रे चाविक्षिप्ते द्वयोः स्फुटान्तरमेव बिम्बान्तरधनुः~। तत्समस्तज्या च
बिम्बान्तरम्~। विक्षेपे सत्यपि दृक्क्षेपवन्मण्डलद्वयान्तरालयुक्त्यैव सिद्धम्~। अत्रापक्रममण्डलगतार्केन्दुस्फुटसूत्रगम् अपक्रममण्डलसमतिरश्चीनं
घनभूमध्यनाभिकं राशिकूटगतसम्पातद्वयं मण्डलद्वयं विवक्षितम्~। यथा लग्नसममण्डलमपक्रममण्डलं च~। तत्रादित्यसूत्रगं राशिकूटमण्डलं लग्नसममण्डलवत्~। चन्द्रराशिकूटमण्डलमपक्रममण्डलवत्~। अपक्रममण्डलं च दृक्क्षेपमण्डलवत्~। खमध्यस्थमिव सूर्यं प्रकल्प्य तत्र मण्डलान्तराणि तद्व्यासार्धानि भुजाकोटिसूत्राणि रेखा वा प्राग्वत् प्रकल्प्य दृक्क्षेपादियुक्तयोऽखिला अत्रातिदेश्याः~। तद्यथा \textendash\ अत्र प्रथमं बिम्बान्तरालार्धज्या दृक्क्षेपानयनविपरीतकर्मणानीयते~। तत्समस्तज्या च पुनर्ज्याच्छेदविधानन्यायेन~। अत्र चन्द्रराशिकूटमण्डले अपक्रममण्डलवत् कल्पिते चन्द्रबिम्बघनमध्यं मध्यलग्नस्थानीयम्~। द्विस्पृग्वृत्तं दक्षिणोत्तरवत्~। मध्यज्यास्थाने बिम्बान्तरालार्धज्या~। दृक्क्षेपस्थाने स्फुटान्तरालार्धज्या च~। सेह ज्ञाता~। तत्र मध्यलग्नदृक्क्षेपलग्नान्तरालगुणवदत्र दृक्क्षेपः~। स च ज्ञातः~। बिम्बान्तरालार्धज्या च तत्कर्णः~। स इह ज्ञेयः~। तदर्थं तत्कोटिः प्रथममानीयते~। सा च मध्यलग्नदृक्क्षेपवत् स्थिता~। ततो मध्यलग्नदृक्क्षेपात् परमविक्षेपानयनविपरीतकर्मणैव तत्कोटिरानेया~। तच्चैवं \textendash\ विक्षेपवर्गं त्रिज्याकृतेर्विशोध्य, पदीकृतेन विक्षेपकोटिमण्डलव्यासार्धेन स्फुटान्तरालज्यां हत्वा त्रिज्यया हरेत्~। तत्रेयं
कोटिराप्यते~। दृक्क्षेपकोटिमण्डलगता स्फुटान्तरालज्यैव सा~। यथा तत्र क्षितिजमध्यलग्नान्तरालज्या वि(क्षेप)मण्लडगता मध्यलग्नदृक्क्षेपात्मिका कोटिः~।

\newpage

\noindent सा तत्र त्रिज्याहता कोटिव्यासार्धेन ह्रियते~। इह तु तद्वैपरीत्येन कोटिव्यासार्धहता त्रिज्यया ह्रियते~। अत्र पुनर्गणितलाघवाय हारकात् गुणकारं विशोध्य शिष्टेन प्रतिराशि तं गुण्यं सङ्गुणय्य हारकेण हृत्वाप्तं पृथक्स्थात्~। गुण्यात् विशोध्यते~। अत्रेन्दुनतीषु तुल्यं हि गुणहारान्तरम्~। अस्याः कोट्या बाहोर्विक्षेपज्ययाश्च वर्गयोगमूलं बिम्बान्तरालार्धज्या~। तत्स्वशरवर्गयोगमूलं बिम्बान्तरसमस्तज्या~। अत्र
स्फुटान्तरशरे ततो बिम्बान्तरशरस्याधिकं भागं प्रक्षिप्य बिम्बान्तरशर आनीयते~। न पुनर्ज्याच्छेदविधानवत्~। यथा दृक्क्षेपज्यामण्डलादधोगतं
मध्यज्यामण्डलं तथात्रापि स्फुटान्तरालार्धज्यामण्डलात् बिम्बान्तरालार्धज्यामण्डलं भगोलमध्यासन्नम्~। शरान्तरं च तयोरधऊर्ध्वविवरम्~। तच्च
विक्षेपशरस्य कोटिरेव~। ततस्तद्युक्तस्फुटान्तरशरो बिम्बान्तरशरः स्यात्~। अत्र विक्षेपमण्डलगता कोटिरपि न चन्द्रस्पृष्टकोटिमण्डले कल्प्यते~।
किन्तु अपक्रममण्डलगतचन्द्रस्फुटसूत्रस्य विक्षेपसूत्रस्य च सम्पातात् रविस्फुटान्तरालतया, यतस्तत्र कल्प्यमानायाः कोट्यास्तद्बाहुभूतविक्षेपस्य
च वर्गयोगपदं बिम्बान्तरालार्धज्याकर्णतया प्रतीयेत~। तस्या विक्षेपशरबाहुत्यागेनानयनमपि सुगमम्~। स्फुटान्तरालज्याया ईषन्न्यूनत्वात्
तस्याः~। विक्षेपशरतद्बाहुवर्गान्तरमूलं तयोरूर्ध्वाधोविवरम्~। ततस्तदेव स्फुटान्तरालशरे क्षेप्यं, बिम्बान्तरशरसिद्ध्यर्थम्~। यदि चन्द्रस्फुटसूत्रस्य स्फुटान्तरालज्या भुजा, तदा विक्षेपशरस्य कियतीति इह त्रैराशिकम्~। बिम्बान्तरसमस्तज्यायाः साध्यत्वात् त्रयाणां वर्गयोगः पदीक्रियते~। चन्द्रस्फुटसूत्रस्याधऊर्ध्वत्वकल्पनायां स्फुटविवरज्यायाः तद्बाणविक्षेपशरान्तरस्य च वर्गयोगमूलम् आदित्यस्फुटसूत्रचन्द्रकक्ष्यापक्रममण्डलसम्पातस्य चन्द्रस्फुटसूत्रविक्षेपसूत्रयोः सम्पातस्य चान्तरालं स्यात्~। तद्विक्षेपवर्गयोगमूलं च बिम्बान्तरसमस्तज्या स्यात्, विक्षेपाग्रगतत्वाच्चन्द्रबिम्बघनमध्यस्य~। सा हि भगोलमध्यात् प्रवृत्तयोः चन्द्रार्क-

\newpage

\noindent बिम्बघनमध्यप्रोतयोः सूत्रयोरन्तरालकलानां समस्तज्या~। अर्केन्द्वन्तरे राशित्रयाधिके तु तत्स्फुटसूत्रस्पृष्टयोः अपक्रममण्डलप्रदेशयोर्मध्यं खमध्ये कल्पयित्वा बिम्बान्तरानयनयुक्तिः प्रदर्श्या~। तत्रापि विक्षेपाभावेऽन्तरालार्धज्याद्विगुणैव बिम्बान्तरम्~। सति विक्षेपेऽर्धज्यां विक्षेपशरेण हत्वा त्रिज्ययाप्तं तच्छरभुजा~। तेन शरकोटिं चानीय तद्भुजां द्विगुणाया अर्धज्याया विशोध्य शिष्टस्य शरकोट्याश्च विक्षेपस्य च वर्गयोगमूलं बिम्बान्तरम्~। स्फुटान्तरे राशित्रयोनेऽपि समानमेवैतत् कर्म~। यदिदं घनभूमध्यगत\renewcommand{\thefootnote}{१}\footnote{ ते \textendash\ क. ग.}द्रष्टृबिम्बान्तरं, ततो लम्बननतिभ्यामेव भूपृष्ठवर्तिनां स्वस्वदृग्गोलगतबिम्बान्तरस्य विशेषः~। तदेवाह तन्त्रसङ्ग्रहे \textendash\ {\qt भूमध्यसम्बन्धिनो व्यर्थत्वात् युक्तिप्रतिपादनलाघवायैव हीह प्रथमं
तदुक्तिः}~। अत्र तु चन्द्रस्य दृक्क्षेपलग्नस्थकल्पनायां खमध्यस्थकल्पनायां च यो विशेष उक्तः तत् कर्मोभयमपि कार्यम्~। उभयोरपि नतिसम्भवात्\renewcommand{\thefootnote}{२}\footnote{सम्भवत्वात् \textendash\ ख.}~। किन्तु उभयोर्नतियोगो वियोगो वेह नतित्वेन ग्राह्यः~। अत्र सितमानानयनयुक्तिः पूर्वमेव प्रदर्शिता~। शृङ्गोन्नतियुक्तिरेव पुनरवशिष्टा~। तत्र यदा रवीन्द्वोर्दृङ्मण्डलमेकमेव स्यात् तदा न मनागपि शृङ्गोन्नतिः~। चन्द्रबिम्बदृश्यादृश्यार्धसन्धिगतवलयपार्श्वगत्वादेव उभयोः शृङ्गयोः~। अन्यदेन्दुदृङ्मण्डलात् यद्दिशि रविदृङ्मण्डलं तद्दिक्स्थं शृङ्गमुन्नतम्
अन्यद् अवनतं च~। तत्रापि रवीन्दुदृक्कर्णस्पृष्टं द्रष्टृमध्यं यद्वृत्तं, तद्गतं बिम्बान्तरालसूत्रं चन्द्रपृष्ठे यं प्रदेशमुद्भिद्य चन्द्रकक्ष्यावधिकं रविसूत्रं
गच्छति, तत्र हि चन्द्रकलामानार्धमध्यम्~। तस्य दृङ्मण्डलात् यावत् तिर्यग्गमनं तावत्येव हि शृङ्गस्योन्नतिरवनतिर्वा~। अतस्तदेवेह वलनत्वेन
ग्राह्यम्~। तच्च दृङ्मण्डलान्तरालाधीनमिति क्षितिजगतदृङ्मण्डलान्तरालमिह प्रथममानीयते~। तत्रापि सममण्डलात् दृङ्मण्डलान्तरालं पृथक् पृथगानीय
तत्संयोगं वियोगं वा कृत्वा दृङ्मण्डलद्वयान्तरालद्वयमिह साध्यते~।

\newpage

\noindent तत्रेदं त्रैराशिकं \textendash\ यद्येतावत्या दृग्ज्यया एतावान् बाहुर्लभ्यते, तदा त्रिज्यातुल्यया कियानिति~। तद्योगो वियोगो वा यावान् तावान् हि चन्द्रदृङ्मण्डलात् रविदृङ्मण्डलस्य परमो विप्रकर्षः~। तेन त्रैराशिकं \textendash\ यदि त्रिज्यया इयान् विप्रकर्षः, तदा रविदृग्ज्यया कियानिति बिम्बान्तरालसम्बन्धिवलनं लभ्यते~। ततश्चेदं त्रैराशिकं \textendash\ यदि बिम्बान्तरालज्याया वलनमेतावत् तदा चन्द्रबिम्बव्यासार्धस्य कियदिति~। तत्परिलेखनं तु विस्पष्टम्~। भगोलविषये सर्वत्र प्रथमस्फुटमेव ग्राह्यम्~। दृग्विषये सर्वत्र द्वितीयस्फुटमेव~। अत्र एव श्रीपतिनापि {\qt स्फुटगणितदृगैक्यं कर्तुमिच्छद्भिरत्र} इत्युक्तम्~। तस्माच्छायाद्यानयने द्वितीयस्फुटचन्द्रादेव सायनादपक्रमो ग्राह्यः~। व्यतीपात एव प्रथमस्फुटात्~। छायादौ विक्षेपश्चान्त्यकर्णहृत एव ग्राह्यः~। पुनः {\qt परमापक्रमकोट्या} इत्यादिना स्फुटापक्रमश्च कार्यः~। ताराणां छायाद्यानयने उदयास्तमयकालानयने च सायनांशेभ्यः स्वयोगभागेभ्य एवापक्रमानयनमायनदर्शनसंस्कारश्चेष्यते~। एवमानीतापक्रमेण नियतेन स्वस्वपठितविक्षेपेण च {\qt परमापक्रमकोट्ये}त्यादिना {\qt परमक्षेपकोटिघ्न}मित्यादिना वा स्फुटापक्रममानीय शङ्कुच्छायानयनमपि कार्यम्~। भगोलस्य अयनचलनवशात् वायुगोलापेक्षया कृत्स्नस्यापि तिर्यक्त्वात् चलनान्नक्षत्राणां कालवशादपक्रमभेदश्च स्यात्~। विक्षेपः सर्वदापि नियत
एव~। तेषां सर्वेषामपि योगभागाः स्वस्वविक्षेपाश्च {\qt क्षितिरवियोगात्} इति वक्ष्यमाणप्रकारेणैवानेयाः~। तेषामादित्यासन्नानामुदयास्तमयभागाश्च
परीक्ष्यैव ज्ञेयाः~। तत्राश्विन्यादीनां प्रसिद्धानामगस्त्यादीनां च योगभागविक्षेपोदयास्तमयभागाश्च पठिताः~। तथा च सूर्यसिद्धान्ते~॥~४६~॥\\


\indent एवमशेषं ग्रहगणितन्यायं प्रदर्श्य ज्योतिर्ज्ञानस्यादेशफलत्वादनादेशग्रहणग्रहयुद्धाद्यादेशे प्रणिहितेन भाव्यमादेष्ट्रेति बोधयितुं सूर्यग्रहणे
स्फुटगणितविदामपीषद्ग्रासे यथागतग्रासमानादेशे लोके हास्यता स्यादिति तस्यानादेश्यावधिप्रदर्शनायाह\textendash 

\newpage

\begin{quote}
{\ab सूर्येन्दुपरिधियोगेऽर्काष्टमभागो भवत्यनादेश्यः~।\\
भानोर्भासुरभावात् स्वच्छतमत्वा\renewcommand{\thefootnote}{१}\footnote{तनुत्वा \textendash\ क. ख.}च्च शशिपरिधेः~॥~४७~॥} 
\end{quote}

\indent इति~। अर्कलिप्ताव्यासाष्टमभागो ग्रस्तोऽप्यनादेश्यो भवति~। कुतः? अदृश्यत्वात्~। अदृश्यत्वे कारणं भानोर्भासुरभावात् शशिपरिधेः स्वच्छतमत्वाच्च~। इत्यदृश्यत्वे कारणद्वयमुक्तम्~। ततोऽष्टमांशे ग्रस्त एव स्पर्शो वाच्यः~। अष्टमांशेऽवशिष्टे मोक्षश्च~। यदा पुनः परमग्रासोऽप्यष्टमांशादल्पः, तदा ग्रहणमेवानादेश्यम्~। यतः श्रौतकर्मणि ग्रहणदर्शन एव प्रायश्चित्तविधिर्विधीयते, मन्त्रलिङ्गाच्च दृष्ट एव
ग्रहणे प्रायश्चित्तस्य कर्तव्यतावगमात्~। तस्माददृश्यत्वे ग्रहसद्भाव एव न वाच्यः~। {\qt सूर्येन्दुपरिधियोग} इत्यनेन शशिपरिधेः स्वच्छतमत्वस्य नादृश्यत्वोपपत्तिः प्रदर्शिता~। चन्द्रमण्डलस्य कियतिचिदंशे सूर्यमण्डलं प्रविष्टेऽपि चन्द्रपार्श्वसंलग्नैः सूर्यरश्मिभिः ग्रस्तांशस्य परिपूरणादपि
अदृश्यत्वं स्यात्~। ग्रस्तभागस्य सूर्यस्यातिभासुरत्वात् सूर्यमण्डलान्निर्गच्छतां रश्मीनां निबिडत्वात्तरङ्गवच्चलत्वप्रतीतेश्च परिधेर्निम्नभागं पूरयन्त्येव
ते~। परिधिसंश्लेष एव पुनरियान् विशेषः~। यदि खण्डग्रहणेऽपि ग्रस्तभागस्याल्पत्वं दृश्यभागस्याधिक्यं च प्रतीयेत, भानोर्भासुरत्वादेव,
तथापि यन्त्रेण निरीक्ष्यमाणे ग्रस्ताग्रस्तविभागस्तत्त्वतः परिच्छेत्तुं शक्यत एव~। परिधिसंयोगे पुनरुपायान्तरेणापि नोपलब्धुं शक्यो ग्रस्तभाग
इति ज्ञापनार्थं सूर्येन्दुपरिधियोग इत्युक्तम्~। तस्मात् ग्रस्तस्याष्टमभागाधिक्य एव ग्रहणमादेश्यम्~। अष्टमभागश्च प्रायेण कलाचतुष्टयमितः~।
सूर्यसिद्धान्ते तु लिप्तात्रयस्यैवानादेश्यत्वमुक्तम्\textendash 
\begin{quote}
{\qt लिप्तात्रयमपि ग्रस्तं तीक्ष्ण\renewcommand{\thefootnote}{२}\footnote{ग्रस्ततीक्ष्ण \textendash\ घ.}त्वान्न विवस्वतः~।} 
\end{quote}

\noindent इति~। ततो मन्यामहे मध्याह्ने लिप्ताचतुष्टयमपि ग्रस्तमनादेश्यं स्यात्~। 

\newpage

\noindent क्षितिजासन्नत्वे तु तैक्ष्ण्यस्य न्यूनत्वाल्लिप्तात्रयादधिकमपि ग्रहणं द्रष्टुं शक्यमिति विभाग इति~॥~४७~॥\\
	
\indent एवमिह ब्रह्मसिद्धान्तादिगणितस्कन्धगतयुक्तिकलापो गणितकालक्रियागोलभेदेन त्रेधा विभज्य पादत्रयेण कार्त्स्येन प्रदर्शितः~। अयं न
विप्रतिपत्तियोग्यः, स्वसंवेद्यत्वात् युक्तीनाम्~। यत् पुनर्भगणादीनां क्वचित् केषाञ्चिदाचार्यैर्नानाप्रतिपादनं, तत् परिमाणानां सावयवत्वात्, अवयवानां चानन्त्यात्, कालदैर्घ्यात्तदा वृत्तौ, तदवयवानां तावद्गुणितत्वात्~। यद्वर्धमानं स्थौल्यं तदपि तन्मूलप्रमाणैः प्रत्यक्षादिभिरेव निराकार्यम्~। ततस्तेषां प्रत्यक्षादिभिः परीक्षणं स्वपरीक्षाप्रदर्शनच्छलेन प्रदर्शयति\textendash 
\begin{quote}
{\qt क्षितिरवियोगाद् दिनकृद्रवीन्दुयोगात् प्रसाधितश्चेन्दुः~।\\
	शशिताराग्रहयोगात् तथैव ताराग्रहाः सर्वे~॥~४८~॥} 
\end{quote}

\indent	इति~। क्षितिरवियोगात् दिनकृत् प्रसाधितः~। भूतलस्य रविबिम्बमध्यस्य च एकसूत्रमध्यगतत्वं तद्योगेन विवक्ष्यते~। एवं दिनकृतः
प्रसाधनम्~। एवं रविगतौ निर्णीतायां रवीन्दुयोगादिन्दुश्च प्रसाद्व्य ; ग्रहणदर्शनान्निर्णेय इत्यर्थः~। एवं शशिनि ज्ञाते शशिताराग्रहयोगात् सर्व
एव ताराग्रहास्तथैव निर्णेयाः~। एवमिदानीं कल्यब्दे षष्टिषष्टिमिते भवता क्षितिरवियोगाद्दिनकृत् प्रसाधित इत्येतद्युज्यत एव, अयनस्य प्रकृतिस्थत्वादिदानीम्~। इतः परन्तु क्षेप्येऽयनचलने वर्धमाने भवच्छिष्यप्रशिष्यादिभिः भवद्व्यंश्यैरियदन्तरमयनचलने इयद् रवि\renewcommand{\thefootnote}{१}\footnote{इयान् रवि \textendash\ ग. घ.}स्फुटे इति विभज्य ज्ञातुमशक्यत्वात् कथं क्षितिरवियोगाद्दिनकृन्निर्णेयः~। अयनचलनानयनस्यापि स्थौल्यात् परीक्ष्यैव तन्निर्णयस्याप्युक्तवादिति चेत् ; तदाप्यस्माभिः पूर्वाचार्यैश्च प्रदर्शिताभिः नक्षत्रयोगकलाभिः तद्विक्षेपलिप्ताभिश्च ग्रहनक्षत्रयोगेन ग्रहा निर्णेतुं शक्याः~। एवं शशिनि निर्णीते ग्रहणदर्शनात् रविश्च निर्णेयः\textendash 

\newpage

\begin{quote}
{\qt श्रुतौ स्मृतौ पुराणेषु कल्पमीमांसयोरपि~।\\
	ज्योतिर्गतिः प्रसिद्धापि साद्ध्यतेऽत्रानुमेयता~॥} 
\end{quote}
 
\noindent तत्र तावच्छ्रुतौ\textendash  
\begin{quote}
{\qt स्मृतिः प्रत्यक्षमैतिह्यमनुमानश्चतुष्टयम्~।\\
	एतैरादित्यमण्डलं सर्वैरेव विधास्यते~॥} 
\end{quote}
 
\noindent इत्याद्यनुवाके कालप्रमाणं प्रदर्श्यते~। अत्रादित्यमण्डलस्य कालनिर्वाहकत्वात् तच्छब्देन काल एव विवक्ष्यत इत्युत्तरमन्त्रैर्निर्णीयते~। तदयमर्थः \textendash\ एतैः स्मृतिप्रत्यक्षैतिह्यानुमानैश्चतुर्भिः सर्वैरेव कालो विधास्यत इति~। तत्र केषाञ्चित् कर्मणां कालः प्रत्यक्षेणैवावगम्यते~। {\qt तस्मात् ब्राह्मणोऽहोरात्रस्य संयोगे सन्ध्यामुपास्ते~। सज्योतिष्या ज्योतिषो दर्शनाद्} इति श्रुतेः~।

\begin{quote}
{\qt अहोरात्रस्य यः सन्धिः सूर्यनक्षत्रवर्जितः~।\\
सा तु सन्ध्या समाख्याता मुनिभिस्तत्त्वदर्शिभिः~॥

पूर्वां सन्ध्यां जपस्तिष्ठेत् सावित्रीमार्कदर्शनात्~।\\
पश्चिमां तु समासीत सम्यगार्क्षविभावनात्~॥}
\end{quote}

\noindent इत्यादिस्मृतिभ्यश्च~। यानि पुनर्नैमित्तिकान्यभ्युदितेष्ठ्यादीनि ग्रहणादिदर्शननिमित्तं वा तेषां च कालावगतिः प्रत्यक्षेणापि श्रूयते, स्मर्यते च~। सन्ध्योपासनादिकालस्यापि क्वचित् कदाचिद् मेघाच्छादनादिना अनुमेयतैव स्यात्~। अनुमानेऽपि स्मृतिप्रत्यक्षैतिह्यानां सहकारिता
स्यादेव~। अत एव भगवती श्रुतिः सर्वैरित्याह~। तत्र अश्विन्यादीनां ज्योतिश्चक्रे स्थिरत्वं ग्रहाणां गतिमत्वं च सुगममेव~। तत्र घटादिवदेव कृत्तिकादीनामपि वृद्धव्यवहारेण संज्ञासंज्ञिसम्बन्धोऽवगम्यते~। ननु `{\qt षट्कृतिकानक्षत्रम्}' `{\qt अमी ये सुभगे दिवि विवृतौ नाम तारके}' इत्यादिभिः

\newpage

\noindent नानाशाखास्वाम्नातैः वेदवाक्यैरेव नक्षत्राणां संज्ञासंज्ञिसम्बन्धोऽवगम्यते~। मैवम्~। लोकत एव शब्दार्थसम्बन्धावगतिरिति शास्त्रेषु स्थापितत्वात्~।
\begin{quote} 
{\qt लोक्येते यत्र शब्दार्थौ लोकस्तेन स उच्यते~।}
\end{quote}

\noindent इति वृद्धव्यवहार एव लोकशब्देनोच्यते~। अमी विचृतौ नाम तारके इत्यङ्गुल्यादिभिर्निर्देश्य बोधयितुमशक्यत्वाच्छ्रुतेः~। वृद्धव्यवहारे
तु गामानयेत्यादिवाक्यप्रयोगानन्तरं शिष्यादिभिः क्रियमाणं गवानयनादिकं दृष्ट्वैव बालैः शब्दार्थसम्बन्धोऽवगन्तुं शक्यः~। तस्मात् प्रसिद्धा एव
अश्विन्यादयस्तारा अगस्त्यादयः सप्तर्ष्यादयश्च~। अस्तु वा वाजिभादीनां वाजिमुखाद्याकृतिस्मरणात् ज्योतिःशास्त्रोक्तलक्षणेनैव तन्निर्णयः~।
स्मर्यन्ते च ब्रह्मगर्गमयादिभिः स्वप्रणीतेषु ज्योतिःशास्त्रेषु सप्तविंशतिनक्षत्रारूढ ज्योतिश्चक्रप्रदेशविशेषाः, अगस्त्याद्यारूढाश्च~। तेन गोलयुक्त्यभिज्ञानां
तत्तद्विक्षेपयोगकलाद्वारेण शङ्कुयष्टिधनुश्चक्रादिभिर्यन्त्रैः प्रत्यक्षपरीक्षणेनैव तन्निर्णयः स्यात्~। तत्र धनुर्यन्त्रपरीक्षणेन प्राग्वत् बिम्बान्तरमानेयम्~।
शङ्कुयन्त्रेणोभयोः छायाभुजाकोटिसंवादात् निर्णयः~। एवमवशिष्टानामपि ज्योतिषां यावदपेक्षं प्रदेशा ज्ञेयाः~। अतिप्रसिद्धयोः
सूर्याचन्द्रमसोरपि श्रुतिस्मृतिभ्यां संज्ञासंज्ञिसम्बन्धनिर्णयः स्यात्~। तत्र श्रुतिः \textendash\ `{\qt सूर्य एकाकी चरति}' '{\qt चन्द्रमा जायते पुनः}' '{\qt नवो नवो भवति
जायमानः~।}' '{\qt आदित्यो वा एष एतन्मण्डलं तपति~।}' एवमादिका~।

\begin{quote}
{\qt कलानामपि नैवेन्दोर्मृतिर्ह्यस्य कुहूरिव~।} 
\end{quote} 

\noindent इत्याद्या स्मृतिरपि~। भौमादीनां बिम्बा अपि स्थूलया भागवताद्युक्तगत्या\renewcommand{\thefootnote}{१}\footnote{युक्त्या \textendash\ क. ग. घ.}, ततः सूक्ष्मया ज्योतिःशास्त्रोक्तगत्यापि\renewcommand{\thefootnote}{२}\footnote{युक्त्या \textendash\ क. ग. घ.} निर्णेतुं
शक्याः~। कक्ष्याक्रमश्च ग्रहणादौ छाद्यछादकभावेन वक्रारम्भनिवृत्योर्ग्रहनक्षत्रयोगे प्राक्प्रतीच्योः कपालयोर्नक्षत्रात् प्राक्प्रतीच्योर्लम्बनदर्शनेन च
ज्ञेयः~। पञ्चसिद्धान्तेषु गर्गादिप्रणीतेषु तन्त्रेषु च नानापठितैः भूदिनभगणैस्तदुक्त- 

\newpage

\noindent प्रकारेण कक्ष्यायोजनमानीयापि क्रमो निर्णेतुं शक्यः~। तेष्वविसंवादात् कक्ष्याक्रमस्य तद्योजनसङ्ख्यास्वेव विसंवादात्~। यतस्तदनन्तरं क्रमोल्लङ्घनाय नालम्~। ऊर्ध्वाधोगतिसद्भावश्च यन्त्रैः परिच्छिद्यमानानां चन्द्रादिबिम्बकलानां वृद्धिह्रासानुरूपं कल्प्यः~। गतिकलानां मानकलानां च वृद्धिह्रासयोरैकरूप्यात्~। सार्वकालीनं योजनगतिसाम्यमपि शक्यं कल्पयितुं सूर्येन्द्वोः~। तत्र चन्द्रस्य प्रायेण नाक्षत्रमासद्वयोनेन
सौराब्दनवकेन द्वादशराशिषु क्रमेणाल्पतममध्यमशीघ्रतमानां गतीनां सञ्चारावगमात् तद्भ्रमणवृत्तस्यापि भूम्यपेक्षया परिभ्रमणं शक्यं
कल्पयितुम्~। तद्भ्रमणमेव तस्य मन्दोच्चगतिः~। तदानयनमेवमाह मानसे\textendash 

\begin{quote}
{\qt द्युगणाद् द्विगुणाब्दोनाच्चन्द्रोच्चांशा नवोद्धृताः~।\\
स्ववेदघ्नाब्दसंयुक्ताः साष्टांशाब्दकलोनिताः~॥}
\end{quote}

\noindent इति~। मेषादिमासषट्कात् तुलादिमासषट्कस्य ईषदधिकेन दिनानां सार्धाष्टकेनोनत्वात् इदानीं रवेरुत्तरगोले गतेर्मान्द्यं, दक्षिणगोले
शैघ्र्यं च ज्ञेयम्~। ग्रहस्फुटज्ञानोपायश्च भास्करेणैवमुक्तः\textendash 
\begin{quote}
{\qt ग्रीवासमां भगणभागविभक्तवृत्तां\\
 कुर्यात् स्थलीं समतलां कृतदिग्विभागाम्~।\\
तस्यां जलेशदिशि मण्डलमध्यदृष्टिः\\
विध्याद्रविं परिधिलग्नमनाकुलात्मा~॥

पूर्वरेखाग्रवेधस्य रविवेधस्य चान्तरम्~।\\
अर्काग्रा चापनिर्माणं परिधौ भागलक्षिते~॥
 
अर्काग्रा ज्या भवेत् तस्य तन्नतिज्याविशेषजा~।\\
लिप्ता शङ्क्वग्रजीवाया दक्षिणे चोत्तरेऽन्यथा~॥

दक्षिणाभिमुखी छाया यदा भवति भास्वतः~।\\
नतिज्यारहितार्काग्रा शङ्क्वग्रं कथ्यते तदा~॥}
\end{quote}

\newpage
\begin{quote}
{\qt विद्धि तेन विषुवत्प्रभां सतीं\\
 पूर्ववच्चपललम्बकौ पुनः~।\\
 वेदितव्यविदितग्रहान्तरं\\
 नाडिकाभिरवगम्य तत्त्वतः~॥\\
 षड्गुणाश्च घटिका लवास्तु\renewcommand{\thefootnote}{१}\footnote{श्च \textendash\ क. ग.} तैः\\
 पूर्वपश्चिमदिशि स्थिते क्षयः~।\\
 उच्यते धनमथ क्रमेण तु\\
 ज्ञातचारनिचयैः सदा बुधैः~॥\\
 एवं नक्षत्रताराणां ग्रहैस्ताराभिरेव च~।\\
 साधितं क्षेत्रनिर्माणं युक्त्या सर्वत्र सर्वदा~॥}
\end{quote}

\noindent इति~। अनेन मध्यमादीनां ज्ञानोपायः प्रदर्श्यते\textendash 

\begin{quote}
{\qt गर्गादिस्मृतनक्षत्रक्षेपयोगकलागतम्~।\\
 मध्यलग्नं प्रकल्प्यार्कं ग्रहमध्याह्नकालजम्~॥
 
मध्यलग्नं नतप्राणैर्भस्य लङ्कोदयासुभिः~।\\
नीत्वा तन्मध्यजीवां च मध्यच्छायां ग्रहस्य च~॥

ग्रहस्फुटं तदेव स्यात् तयोः साम्येऽन्यथान्तरम्~।\\
क्षेपः स्यात् चापयोर्व्यस्तं कुर्यात् दृक्कर्म तद्ग्रहः~॥

मध्यच्छाया नृपाढ्येन्दोः स्वैकाश्चांशोनिता स्कुटा~।\\
भमध्याह्नात् दिनार्धान्तनाड्यश्चेत् त्रिंशतोऽधिकाः~॥

खखाष्ट्यश्विविशुद्धास्तत्प्राणाः स्युः प्राक्कपालगाः~।\\
आर्क्षनाडीखषड्वर्गभागा अत्रासवः स्मृताः~॥

ऋक्षलग्नात् खमध्यस्थग्रहाक्रान्तकलाप्तये~।\\
कन्यातुलाभसन्धिस्था चित्रा यस्मात् तदन्तरात्~॥} 
\end{quote}

\newpage

\begin{quote}
{\qt ज्ञेयो ग्रहोऽप्यविक्षिप्ततिष्यभात् कर्किमध्यगात्~।\\
यत्कलावधिचक्रार्धे मान्द्यमन्यदले जवः~॥

भानोरुच्चं च नीचं च कल्प्यं तन्मध्यगं क्रमात्~।\\
उच्चनीचविभक्तार्धे तुल्यकालेन गच्छति~॥

स्याच्च लाघवमाचार्यस्मृतोच्चस्य परीक्षणे~।\\
त्रिभोनोच्चं प्रकल्प्यार्कं सायनं तं पृथङ् न्यसेत्~॥

संस्कृतायनदृक्कर्म नक्षत्रमपि सायनम्~।\\
उभयोः स्वस्वलिप्तासु भेदं कृत्वा यथाविधि~॥

तदन्तरकलाः प्राणा मध्याह्नविवरे तयोः~।\\
यत् त्यक्तं तस्य मध्याह्नाद् अन्याहर्दलकालजाः~॥

सूर्यास्तमितभस्यापि सूर्यस्याप्यन्तरं यदा~।\\
जिज्ञास्यते तदा सूर्यात् दूरस्थ यापि कस्यचित्~॥

भस्य चास्तमितस्यापि ज्ञानैः प्राग्विवरासुभिः~।\\
ज्ञेयास्तदन्तरप्राणाः सूर्यस्यास्तमितस्य च~॥

परीक्ष्याप्यन्तरप्राणान् अर्कदूरस्थतारयोः~।\\
दृश्यभादेकभागस्थं विवरासुद्वयं यदि~॥

तद्भेदो विवरप्राणः सूर्यस्यास्तङ्गतस्य च~।\\
दृश्यादूर्ध्वगतत्वे तु द्वयोरल्पमधोगतम्~॥

अधस्थत्वेऽधिकासु स्यात् अतोऽप्यस्तमितोऽन्यथा~।\\
दृश्यपार्श्वद्वयप्राणयोगचक्रकलान्तरम्~॥

त्रिभोनोच्चसमार्कस्य ततः षड्भाधिकस्य च~।\\
एकस्यैव तु भस्यापि मध्याह्नविवरासुभिः~॥

यस्मिन्नह्नि रवेर्विद्यात् उच्चाद् भत्रयमन्तरम्~।\\
तदासन्नदिनार्धस्य भमध्याह्नस्य चान्तरम्~॥} 
\end{quote}

\newpage

\begin{quote}
{\qt परीक्ष्य तद्गतासूनामानीतानां यदन्तरम्~।\\
स्फुटकेन्द्रायनान्तस्तन्मध्याह्नात् तावदन्तरे~॥

तदन्तरालकालात्तु वत्सरार्धं विशोधयेत्~।\\
शिष्टसावनमध्यार्ककलार्धं परमं फलम्~॥

एवमेवोच्चसाम्यस्य कालो ज्ञेयः परीक्षकैः~।\\
ज्ञातेऽर्कमध्यमे ज्ञेयं बुधोच्चं तत्स्फुटेऽपि च~॥

अर्कतद्बिम्बयोर्भेदं यन्त्रैर्ज्ञात्वापि तत्स्फुटम्~।\\
प्राच्यां दृष्टस्य नृच्छाये ज्ञात्वा नाडीः कपालतः~॥

भास्वच्छायावगत्यन्ताः प्रतीच्यां तु तदादिकाः~।\\
बुधशङ्क्वगत्यन्ता रवेः कालं विशोध्य तु~॥

युक्त्वा बुधस्य कालं च ज्ञेयास्तद्विवरासवः~।\\
शोध्याश्चौदयिकात् काललग्नादास्तमये धनम्~॥

चरं चासुकलाभेदं व्यस्तं कृत्वा विशेषयेत्~।\\
प्राग्लग्नं चास्तलग्नं स्यात् क्रमाद् ज्ञस्योदयास्तयोः~॥

छायाबाहोर्महाबाहुं शङ्कुं शङ्क्वग्रमानयेत्~।\\
तदैक्यभेदतोऽग्रज्यां युक्त्या क्रान्तिं समानयेत्~॥

तल्लग्नक्रान्तिचापैक्यं दिशोर्भेदेऽन्यथान्तरम्~।\\
क्षेपो युक्त्या च दिक् ज्ञेया दृक्कर्मयुगलं ततः~॥

व्यस्तं संस्कृत्य तल्लग्ने सकृत् ज्ञेयस्फुटो ग्रहः~।\\
सप्तविंशतिभागान्तं गच्छेदुपरि भास्वतः~॥

स्त्रियां ज्ञोऽधो नवांशानां पञ्चकं याति नाधिकम्~।\\
मीने तूपर्यधो व्यस्तमर्कज्ञविवरांशकाः~॥

तत्त्वधृत्यंशका मध्यात् स्फुटान्तरमुपर्यधः~।\\
सिंहे कुम्भे ततो व्यस्तं स्वर्णं मान्दं तयोरतः~॥}
\end{quote}
 
\newpage

\begin{quote}
{\qt तदन्तरदलं मान्दं फलं तु परमं स्मृतम्~।\\
तद्योगदलमेवापि शैघ्रं स्यात् परमं फलम्~॥

मैत्रकृत्तिकयोर्मध्ये कदाचिदपि नाधिकम्~।\\
शैघ्रादन्त्यफलान्मध्यस्फुटयोर्लक्ष्यऽतेन्तरम्~॥

दृश्यते चान्तरं तावदधश्चोर्ध्वं कदाचन~।\\
आद्यत्र्यंशे ततः क्वापि मन्दोच्चं वृश्चिके विदः~॥

इत्यादौ विदिते स्मृत्या लाघवं स्यात् परीक्षणे~।\\
व्याख्यातः शुक्र एतेन स्त्रियां शकनवांशकान्~॥

ऊर्ध्वं गच्छेदधोऽध्यर्धराशिमेव झषेऽन्यथा~।\\
अर्कान्मन्दफलस्यास्य स्वर्णसाम्यात् कनीयसः~॥

आर्द्रान्तार्कं समं गच्छेत् अभितोऽत्रोच्चमस्य तु~।\\
कन्यायां मध्यमादूर्ध्वं किन्नरा भस्य नाडिकाः~॥

अधो लोकाधिका गच्छेन्मीनगात् व्यत्ययादपि~।\\
स्तुहीति पठितं मान्दं तद्दलं परमं फलम्~॥

तद्योगार्धं फलं शैघ्रमन्त्यं सूतसुरोदितम्~।\\
यदैवाद्यपदान्तः स्याद् युगपन्मन्दशीघ्रयोः~॥

मन्दाद्यान्ते तृतीयान्तः शीघ्रस्यापि यदा तदा~।\\
तृतीयान्तेऽपि मन्दस्य शैघ्रस्यौजपदान्तयोः~॥

उच्चयोः पञ्चषैर्भागैर्भेदेऽपि न फले भिदा~।\\
पदान्ते फलयोर्मान्द्यात् मान्दे भार्धं भिदास्तु वा~॥

ततः प्राग्वत् पृथक्कार्ये फले द्वे मन्दशीघ्रयोः~।\\
उच्चयोर्भेदतो भेदो निराकार्यः फलेऽपि सन्~॥

मिथः परीक्षया भूयः केन्द्राङ्घ्रिष्वविशेषतः~।\\
मन्दौजान्ते द्वितीयान्तः शैघ्रस्य स्याद् यदा तदा~॥}
\end{quote}

\newpage
\begin{quote}
{\qt शीघ्रोच्चं तु परीक्ष्यं स्याद्धीमता सूक्ष्मताप्तये~।\\
वक्रारम्भात् प्रभृत्यस्ते शैघ्राद्वा परमात् फलात्~॥

परीक्ष्यानुदिनं लेख्यं स्फुटार्कादन्तरं भृगोः~।\\
उदयादुदये तद्वद् अह्नामासप्ततेरपि~॥

अर्कशुक्रान्तरे तत्तु दिनमान्दफलान्तरम्~।\\
मौढ्यात् प्राग्धनमच्छार्कमान्दान्तरमृणं पुनः~॥

मध्यस्फुटाच्छयोर्भेदाः पदयोः स्युर्दिने दिने~।\\
मीने मान्दान्तरं व्यस्तं साम्यकालोऽनुपाततः~।।

मध्यस्फुटान्तरे तुल्ये पदयोर्द्वे यदा यदा~।\\
तन्मध्यकालजं शुक्रस्फुटमध्यं तु षड्भयुग्~॥

शीघ्रोच्चमत्र तन्मध्यसाम्ये वापि परीक्ष्यताम्~।\\
अन्त्ये पदे फलादन्त्यादास्वदृष्टिदिनावधेः~॥

आद्येऽप्युदयतः पश्चात् स्वदृष्टिद्युगणावधेः~।\\
वृश्चिकान्ते रवौ वोच्चे वृद्धे बाले च वक्रिणि~॥

तुल्यत्वात् सितविक्षिप्तेर्लाघवं स्यात् परीक्षणे~।\\
इहानुलोमयोगेऽपि पातशीघ्रसमत्वतः~॥

वृषकीटान्तगे सूर्ये समः क्षेपः पदद्वये~।\\
ज्येष्ठाद्यांशे रवौ वोच्चे वक्रिणोर्वृद्धवालयोः~॥

भृगोर्दोःफलयोर्मान्दे प्रायोऽर्कफलसाम्यतः~।\\
मृगकर्कटयोर्वक्रयोगे विंशांशके तथा~॥

चक्रे शुक्रार्कयोर्योगो विंशत्या गुरुपर्ययैः~।\\
कृत्स्ने चरति केन्द्राद्यपदान्तेऽब्देऽत्र सद्गुरौ~॥

शबरैर्वत्सरैरेकदिनाढ्यैः केन्द्रपर्ययाः~।\\
सेवकाः स्युः शरादानैर्देवकीघटिकोनितैः~॥}
\end{quote}

\newpage

\begin{quote}
{\qt योगस्थानानि पञ्च स्युरब्दानामष्टकेऽष्टके~।\\
मीनान्ते मृगधृत्यंशे कीटे षट्के हरौ जिने~॥

यमे भानौ युगादौ स्युः क्रमात् स्थानानि पञ्च वै~।\\
क्रमाच्चलन्त्यधस्तानि त्रीनंशान् दशवत्सरैः~॥

सकृत्सकृदुभौ योगौ पञ्चस्वब्दाष्टके ततः~।\\
दशकृत्वोऽस्तमेत्यच्छो दशकृत्व उदेति च~॥

एको यत्र यदा योगस्तत्रान्योऽब्दचतुष्टये~।\\
गाथालिप्तोनिते भानौ तत्पञ्चांशे निरन्तरे~॥

उक्तेषु तु परीक्षायाः स्थाने यत्र युतिस्तदा~।\\
अब्दाष्टके समीपे वा परीक्षेतोक्तवत् पुरा~॥

प्रत्यक्षैतिह्ययोगाभ्यां ज्ञेया खेटोच्चयोर्गतिः~।\\
मध्यमानां विरोधात्तु शास्त्रोक्तानामनिर्णयात्~॥

मिथः प्रत्यक्षतो वापि विरोधात् तन्निराकृतेः~।\\
संहितोक्तगुणैर्युक्तदक्षशिष्यगुणान्वितः(?)~॥

दैवज्ञः सुभृतो राज्ञा धार्मिकेण जयैषिणा~।\\
ग्रहणग्रहयोगास्तवक्रोडुग्रहयोगजान्~॥

कालान् ब्रूयात् स शिष्येभ्यो गुप्तये शिष्यसन्ततौ~।\\
ऐतिह्याख्यं प्रमाणं तत् प्राहेहोपनिषच्छ्रुतिः~॥

लेख्यं तत् पुस्तके शिष्यैः पद्यं वा गद्यमेव वा~।\\
संस्कृतैः प्राकृतैर्वापि स्वीयया देशभाषया~॥

भेदोल्लेखाविशेषेण ग्रहयोर्ग्रहतारयोः~।\\
अप्रमत्तेन विज्ञेयाः सर्वे लेख्याः प्रयत्नतः~॥

आसन्नक्षेपयोर्वापि बिम्बान्तरकला अपि~।\\
युगपत् पर्ययान्तेऽस्मात् परीक्ष्या रविखेटयोः~॥}
\end{quote}

\newpage

\begin{quote}
{\qt स्ववंशजैर्युतिर्वाह्नि भगणान्तो यदा तदा~।\\
बिम्बान्तरं परीक्ष्यं स्यादभितो निशयोर्द्वयोः~॥

लेख्यं तैरपि तत् सर्वं ग्रहयोगस्य पुस्तके~।\\
आसन्नभगणान्तोऽब्दगणः पूर्वमुदीरितः~॥

रव्यच्छशीघ्रयोः सूक्ष्मो दैवाधीनोऽपि वा स च~।\\
रविज्ञशीघ्रयोर्योगः प्रायो गोचरवत्सरैः~॥

तत्त्वाब्दादूर्ध्वमासन्ने सौराच्चान्द्रादधस्तदा~।\\
षट्त्रिंशल्लिप्तिकोनांशत्रिकोणोर्ध्वं त्रयोदशात्~॥

कविः शिवाय धात्रीजो ज्ञकेन्द्रभगणाः क्रमात्~।\\
कृष्णोत्सवदिनैस्तस्य कवि\renewcommand{\thefootnote}{१}\footnote{कविः \textendash\ ख.}केन्द्रस्य पर्ययाः~॥

मनीषाब्दे गुरोः पूर्णा भगणाः प्रायशो गजैः~।\\
यतीन्द्रे विह्वले वापि योगश्चक्रे समीपगः~॥

स राशिभगणो भानोरंशत्रयसमन्वितः~।\\
अंशस्वरांशसंयुक्तो मन्त्रिणः केन्द्रपर्यये~॥

योगाद् वक्रगतेर्मध्ये तदर्धं रविमध्यमम्~।\\
केन्द्रस्यैकादशे पूर्णे द्वितीये भगणे गुरोः~॥

देवलिप्ताधिकाब्ध्यंशैरधिके रविमध्यमे~।\\
मान्दकेन्द्रपदाद्यन्ते पदसन्धिश्चले यदा~॥

प्राग्वदोजान्तयोर्वृत्ते फलयुग्भेदसङ्क्रमात्~।\\
स्फुटस्य स्फुटमध्यस्याप्यर्कमध्यात् भषट्कगे~॥

साम्यात् षड्भाधिकं भानुमध्यं स्यात् स्फुटमध्यमम्~।\\
प्रतिराशि परीक्ष्यं स्याद् द्वादशाब्दैर्गुरोश्च तत्~॥

प्रतिसप्तांशकं ज्ञेयं सप्तभिर्भगणैरपि~।\\
चक्रार्धयोर्यतः साम्यं स्फुटमध्यस्य भोगयोः~॥} 
\end{quote}

\newpage

\begin{quote}
{\qt कालस्य तत्प्रदेशस्थे ह्युच्चनीचे ग्रहस्य तु~।\\
आधिक्यं मकरादेः स्यात् कर्क्यादेरल्पकालता~॥

कालद्वयान्तरे भुक्तिर्मध्यमा यावती ततः~।\\
चतुरंशफलं मान्दं परमं स्याद् द्युचारिणाम्~॥

मान्दस्याद्यपदान्तेऽर्कमध्यमं गुरुमध्यमात्~।\\
अधऊर्ध्वं तृतीयान्ते तन्मन्दान्त्यफलान्तरे~॥

यदा तदा तयोर्योगो मध्यमस्फुटमध्ययोः~।\\
प्रागस्तमयतः पश्चादुदयात् प्राग्वदन्तरम्~॥

ग्रहार्कस्फुटयोर्ज्ञेयं स्फुटमध्यार्कमध्ययोः~।\\
विवरं केन्द्रदोश्चापं फलमोजान्तवृत्तजम्~॥

आनीय दोर्गुणाल्लब्धं स्फुटमध्यस्फुटान्तरम्~।\\
हारो गुणश्च तज्जीवे ह्योजान्तपरिधेस्ततः~॥

तत्कालपरिधेर्लाभस्तस्य चौजान्तजस्य च~।\\
अन्तरं त्रिज्यया हत्वा कोटिबाणहृतं ततः~॥

तत्कालवृत्तसामीप्यं यथा स्वर्णं तथायने~।\\
वृत्ते तद्गोलसन्ध्योः स्याद् मान्दे वृत्ते उभे तथा~॥

त्र्यंशोनं भचतुष्कं प्रा(हा?गा)र्धरात्रिकमीरितम्~।\\
मन्दोच्चमसृजः पश्चादष्टभागाधिकं ततः~॥

गीतिकायां भटेनोक्तं दहेति स्वपरिभाषया~।\\
आर्धरात्रिकमेवोक्तं प्रथमं खण्डखाद्यके~॥

तस्यैवोत्तरभागे तदत्यष्ट्यंशसमन्वितम्~।\\
तच्च श्रीपतिमुञ्जालसूर्यदेवादिसम्मतम्~॥

तदेव परमाचार्यो ममाह परमेश्वरः~।\\
पञ्च पञ्चाशदब्दान् यः परीक्ष्य करणं स्फुटम्~॥}
\end{quote}

\newpage
\begin{quote}
{\qt आह दृग्गणिताख्यं यत् तत्र मुञ्जालकोदितम्~।\\
ताराग्रहाणां मन्दोच्चमार्के श्रीपतिनोदितम्~॥

मन्दोच्चं तत्सुतः प्राह तन्मतं मह्यमेव च~।\\
सूर्यान्मन्दोच्चांशा वसुतुरगाः पर्वताश्च सत्र्यंशाः~॥

स्वररवयः खाकृतयो द्विनगभुवोऽशीतिरभ्रजिनाः~।\\
ध्रुवार्यापञ्चके त्वेका मानसे कथिता त्वियम्~॥

सत्र्यंशं भचतुष्कं स्यात् सूर्यसिद्धान्तपर्ययैः~।\\
धन्यो ज्ञोऽपि स सौम्यस्य ज्ञानिन पुत्रेशमन्त्रिणः~॥

धिगन्धकारमच्छस्य स्थूलोच्चं खस्थमार्किजम्~।\\
कल्यब्दात् भगणघ्नात् स्वं लक्षद्वय(हृताः?तात्) कलाः~॥

कल्पादिष्टघ्नभगणान्नानानिभहृतं फलम्~।\\
मन्दोच्चं भगणादि स्यान्मन्दोच्चभगणा अमी~॥

सज्जलं वानरो दत्तिलं ज्ञानधीः\\
मार्गणो धीगनुद्भास्करात् पर्ययाः~।\\
भास्करो देहवान् वासुकिर्गानधी\\
रक्षितो भूमिजात् पातजाः पर्ययाः~॥

आर्धरात्रिकसिद्धान्ते भटोक्तान्याह भास्करः~।\\
अष्टिरष्टौ जिना रुद्रा विंशतिर्द्व्यधिकाः क्रमात्~॥

दशघ्ना गुरुशुक्रार्किभौमज्ञांशाः स्वमन्दजाः~।\\
गुरुभूमिजयोरेव स्थूलता महतीह तु~॥

स्थूलान्यौदयिके सर्वाण्यार्केरेवाल्पमन्तरम्~।\\
परीक्षकाणां संवादात् बहूनामेवमिष्यते~॥

परीक्षा चोदनार्थं हि द्वेधाहार्यभटः स्वयम्~।\\
नहसैर्दिवसैरेकः सावनैः केन्द्रपर्ययः~॥}
\end{quote}

\newpage
\begin{quote}
{\qt दिनद्वयाधिकैर्धीस्थैरसृजोऽर्कस्य पर्ययैः~।\\
केन्द्रजा भगणा स्थूला असङ्ख्या गोतलैस्तथा~॥

अंशपादाधिकश्चार्कैर्गोवधैः फलपर्ययाः~।\\
एकोनषष्टिवर्षैर्द्वौ सुम्भलिप्ताधिकाः शनेः~॥

यत्किञ्चिद्भं यदा छन्नं ग्रहणोल्लेख एव वा~।\\
अप्रमत्तेन तद्वाच्यं दैवज्ञेन विपश्चिता~॥

कालश्च लक्षणं भस्य समीपस्थैस्तु तारकैः~।\\
सकलं लिखितं पद्यैर्गद्यैर्वा पुस्तकेषु तत्~॥

दृष्ट्वाप्यधीत्य वा केचिद् भविष्यन्तः स्ववंशजाः~।\\
गोतलाब्दे परीक्ष्यैव निर्णयेयुर्मुहुर्मुहुः~॥

तेनैव भेन संयोगे सुसूक्ष्मां मध्यमां गतिम्~।\\
ऐतिह्योक्तार्कमध्येन पुनरप्यर्कमध्यमे~॥

तुल्ये योगे युगं सूर्यग्रहयोः समयोऽन्तरे~।\\
धीस्थसर्वफलमान्यवत्सरेष्वाद्ययोगरविमध्यतः पुनः~।\\
तद्भखेटसमतार्कमध्यमं भिन्नमेव खलु चाधऊर्ध्वयोः~॥

तारकायाः स्थिरत्वात् तद्ग्रहस्फुटसमत्वतः~।\\
ऐतिह्यवाक्यसिद्धं तत्प्रतिराश्य ग्रहस्फुटम्~॥

उभाभ्यां रविमध्याभ्यां तन्मध्ये चाविशेषयेत्~।\\
कर्मणा विपरीतेन निश्चलग्रहमध्ययोः~॥

अन्तरेणाधिका वोना भगणाः कालभेदजाः~।} 
\end{quote} 

\indent अथ विव्रियते पादो {\qt रवीन्दुयोगात् प्रसाधितश्चेन्दुः} इति~। तत्र शशिग्रहणैः प्रथमं तत्पातस्य निर्णयः कार्यः~।
\begin{quote}
{\qt सूर्योपरागतश्चापि बिम्बलम्बननिर्णयः~।\\
सकृष्णताम्रतामध्ये ग्रहणस्य यदा विधोः~॥}
\end{quote}

\newpage
\begin{quote}
{\qt तदा सषड्भमर्कस्य स्फुटमेव विधोः स्फुटम्~।\\
तदा पातश्च तत्तुल्यबिम्बयोगोऽपि तत्र तु~॥

ग्रहणस्थितिकालोत्थगत्यन्तरसमं भवेत्~॥

संशये धूम्रतामात्रे ग्रहणे ब्रह्मदैवते~।\\
सम्पर्कार्धसमः क्षेपः प्रायोऽपि यमदैवते~॥

ब्राह्मादिभ्यस्तु सप्तभ्यो हिमगौ पर्वणि ग्रहः~।\\
मुहुर्मुहुः स एव स्यात् गते तत्पर्ययाष्टके~॥

द्युगणः प्रथमं ज्ञेयो नाडिकात्रितयोनितः~।\\
एषु क्षेपं परीक्ष्यैव विज्ञेयाः पातपर्ययाः~॥

गिरीन्द्रे पर्वणि प्रायस्तदेव ग्रहणं मुहुः~।\\
चन्द्रकेन्द्रस्य चासत्तिर्गिरीन्द्रे मासि चैन्दवे~॥

गानघ्नहिमगौ तद्वद् द्युगणे तोयसङ्कुले~।\\
गिरीन्द्रैः पञ्चभिर्युक्तैः षड्घ्नैर्हिमगुसङ्ख्यकैः~॥

मासैस्तैर्गीतरागज्ञैः पूर्यन्ते केन्द्रपर्ययाः~।\\
धन्या हि मूलतालज्ञा नाड्यादिर्द्युगणस्तथा~॥

मासैः कुमारसङ्ख्यैः स्युर्धर्तारः केन्द्रपर्ययाः~।\\
षट्चयेषु मुखैर्मासैः सर्वैर्न्न समता यतः~॥

ततः कुमारमासाः स्युर्ग्रहणेऽनुपयोगिनः~।\\
गोपघ्ना गीतरागास्ते पातपर्वत्वमागताः~॥

साष्टांशांशद्वयाधिक्यात् पातस्येह पुनः पुनः~।\\
आरभेत यदि ब्राह्मादुत्सृजेत् पातपर्वताः~॥

परीक्षामारभेतातो ग्रहणाद् यमदैवतात्~।\\
आग्नेयवारुणादीनि भविष्यन्त्युत्क्रमात्ततः~॥} 
\end{quote}

\newpage

\begin{quote}
{\qt त्रयोदशसु षट् च द्विः सकृदन्यतमो ग्रहः~।\\
चतुर्दशे युगे न स्याद् ग्रहः पञ्चदशेऽपि वा~॥

युगं हिमगुमासं चेद् ब्राह्मादारभ्य तु क्रमात्~।\\
एकैकं शतकृत्वः स्यात् प्रायो ब्राह्मादिसप्तसु~॥

षण्मासोत्तरवृद्ध्या विज्ञेयाः सप्त देवताः क्रमशः~।\\
ब्रह्मशशीन्द्रकुबेरा वरुणाग्नियमाश्च विज्ञेयाः~॥

संहितायामिति प्रोक्ता फलभेदा अपि क्रमात्~।\\
अब्दाष्टादशकेऽप्यन्त्यादुत्क्रमेण मुहुर्मुहुः~॥

एकैकमष्टकृत्वः स्याद् ग्रहणं तद्युगेष्विह~।\\
दृश्ये चन्द्रे ग्रहो दृश्यो ह्यदृश्यार्धगते न च~॥

मासषट्के परीक्षेत ब्राह्मादारभ्य तु क्रमात्~।\\
दिननाड्यादिकाः काला विवरे तु द्वयोर्द्वयोः~॥

भिन्नाः स्युग्रहयोस्तस्मात् भुजाफलविनिर्णयः~।\\
आद्यग्रहे रवेः केन्द्रं भत्रयं शशिनः पुनः~॥

बिम्बश्रियेद्ध इत्येवं यदि तद्विवरं परम्~।\\
यदा भनवकं भानोरिन्दोर्बाला प्रियाङ्गना~॥

पार्थनाड्यधिकः कालो मध्यात् पूर्वोदितेऽन्तरे~।\\
अन्ययोर्विवरे कालः तावतोनस्तु मध्यमात्~॥

मासषट्के विनाड्यादीन्यहान्यर्केन्दुमध्यमात्~।\\
ध्यानध्येयस्थसङ्ख्यानि भेदो दोःफलतस्ततः~॥

स्वर्णदोःफलतो भानोर्नाड्यः स्युः सार्धमूर्छनाः~।\\
पञ्चाशन्नाडिका व्यर्धा स्वर्णदोःफलतो विधोः~॥

हंसाढ्येऽह्न्यधिके रात्रौ स्यातामिन्दूदयास्तयोः~।\\
स्पर्शमोक्षौ कदाचित्तदैन्द्रकौवेरयोर्यदि~॥}
\end{quote}

\newpage
\begin{quote}
{\qt 	
निमीलनद्वये कालं ज्ञात्वा वोन्मीलनद्वये~।\\
परमग्रासकालौ च यथायुक्त्यवगम्य च~॥

अर्कदोःफलविज्ञानाज्ज्ञेये चेन्दोः फले द्वयोः~।\\
स्वर्णाभ्यां वा समाभ्यां वा फलाभ्यामुच्चनिर्णयः~॥

तच्छायादिवसाः पूर्णा ह्यत्यासन्नग्रहान्तरे~।\\
मध्यमे मासकालेऽर्कफलात् भुक्त्यन्तरागतः~॥

कालो दोःफलवत्स्वर्णं व्यस्तमाद्यग्रहे द्वयोः~।\\
षण्मासान्तरयोराद्ये स्वर्णं दोःफलवद्रवेः~॥

प्रथमे व्यस्तमर्केन्द्वोः कालो गत्यन्तरोद्भवः~।\\
दृष्टकालस्य तस्यापि साम्ये तुल्ये फले विधोः~॥

आद्ये बालप्रियं केन्द्रं द्वितीये सुखसर्पणम्~।\\
षड्भाढ्यं वोभयोः केन्द्रं षण्मासे वैष्णवं नुमः~॥

सार्धैकादशनाडीभ्यः कालभेदेऽधिके द्वयोः~।\\
केन्द्रं स्याद्भिन्नगोलस्थं स्वल्पे स्यादेकगोलगम्~॥

दृष्टकालाधिके तूर्ध्वफलाधिक्यमृणे द्वये~।\\
आद्यग्रहफलाधिक्यं धने स्वल्पेऽन्यथा द्वयम्~॥

गोलभेदेऽधिके दृष्टे धनं पूर्वमृणं परम्~।\\
व्यस्तमूने गतेर्भेदात् कालभेदः फले समे~॥

ऋणे फलद्वये बिम्बमाद्येऽल्पं महदन्तिमे~।\\
धने व्यस्तं द्वयोरल्पं पादोनं महतस्ततः~॥

यन्त्रेण केन्द्रभेदस्तु विज्ञेयोऽत्र फले समे~।\\
सर्वव्यासे लताव्यासे चार्धं बिम्बं महल्लघु~॥

कौबेरे ग्रहणे बिम्बे छाद्यच्छादकयोः पृथक्~।\\
ज्ञेये स्थितिविमर्दार्धे कालं ज्ञात्वेह धीमता~॥}

\end{quote}
\newpage

\begin{quote}
{\qt विमर्दस्थितिकालौ द्वौ गत्यन्तहतौ हरेत्~।\\
 षष्ट्या बिम्बवियोगश्च योगश्च क्रमशो भवेत्~॥ 
 
ततः सङ्क्रमणेनैव ज्ञेयं बिम्बं पृथक् पृथक्~।\\
यद्वा विमर्दकालेन हत्वा बिम्बयुतिं हरेत्~॥
 
स्थितिकालेन तत्राप्तं वियोगो बिम्बयोर्द्वयोः~।\\
ह्यस्तनादिन्दुमध्याह्नादा पूर्वेन्दुदिनार्धतः~॥
 
आ च श्वस्तनमध्याह्नात् तत्षष्ठिघटिकान्तरम्~।\\
यत् तत्प्राणान् भलिप्ताघ्नात् निरक्षासुभिराहरेत्~॥
 
गत्यन्तरकला लब्धास्तद्योगार्धमिहेष्यते~।\\
षण्मासोत्तरवृद्ध्या यानि ग्रहणानि सप्तषण्मासैः~॥
 
आद्येऽन्त्येऽपि च तेषु ग्रहणं स्वल्पं त्रिके तृतीयादौ~।\\
सकलग्रहणं प्रायो द्वयोः सकलमेव मध्यमे त्रितये~॥
 
अभितः समे विमर्दे क्षेपाभावस्तु मध्यमे ज्ञेयः~॥
 
स्थितिकालं विमर्दं वा गत्यन्तरहतं हरेत्~।\\
षष्ट्याप्तकृतिहीनाभ्यां वर्गाभ्यां योगभेदयोः~॥
 
मूलार्धं मध्यविक्षेपो भेदः स्यात् स्पर्शमोक्षयोः~।\\
स्थित्यर्धेन्दुगतेर्लिप्ता लोकाप्तास्तूत्तरत्रिके~॥
 
स्वर्णं क्रमात् त्रिषु व्यस्तमैन्द्रान्तेष्वसकृत् तथा~।\\
छाद्यच्छादकयोर्द्रष्टर्येकसूत्रगतेषु च~॥
 
सूर्यबिम्बं च विज्ञेयं सूक्ष्मं स्थित्यर्धतोऽपि च~।\\
लम्बनेनापि विज्ञेया कक्ष्या च हिमदीधितेः~॥
 
स्वमध्ये क्षितिजे चापि ग्रहणे कालभेदतः~।\\
ज्ञेया लम्बननाङ्यः स्युस्ताभिस्तल्लिप्तिका अपि~॥
 
पक्षान्तनिर्णये जाते ग्रहणस्तु विधोस्तथा~।\\
मासान्तोऽपि विनिर्णेयः क्रान्त्यक्षांशैतश्च युक्तितः~॥}
\end{quote}

\newpage

\begin{quote}
{\qt याम्यतः प्रतिनिवृत्तिकालतः सौम्यतश्च विदितं यदन्तरम्~।\\
भास्करस्य दलितं तदेव हि क्रान्तिमाहुरधिकां पुरातनाः~॥
	
वराहमिहिरणोक्तमेवं क्रान्तिपरीक्षणम्~।\\
याम्योत्तरदिशोर्यत्र मध्याह्नं कुरुते रविः~॥

नतांशयोर्युतेरर्धं क्रान्तिरक्षोन्तरस्य च~।\\
याम्यदिश्येव यत्रार्को मध्याह्नं याति तत्र तु~॥

भेदार्धं क्रान्तिरक्षस्तु नतयोगार्धमेव च~।\\
अयनान्तार्कयुक्कोटिरुणं चलनमायनम्~॥
	
धनमोजपदस्थार्ककोटिचापं ग्रहे च भे~।\\
पाते विषुवतीन्दोश्च स्फुटक्रान्तिः परा तथा~॥
	
पातेऽन्यविषुवत्स्थेऽल्पा परमक्रान्तिरिष्यते~।\\
अन्तरार्धं तयोः क्षेपः परमश्च विधोर्भवेत्~॥
	
मध्याह्ननतचापाभ्यां शोध्यं तल्लम्बनं विधोः~।\\
तारानतान्तराद्भूभेः परिधिर्देशयोर्द्वयोः~॥
	
यत्र यद्भं खमध्यस्थं यत्र चैकांशको नतः~।\\
चतुर्दशान्तरालेऽर्कसिद्धान्ते योजनानि तु~॥
	
देशयोर्भटसिद्धान्ते स्वस्त्र्यंशोनानि तानि हि~।\\
चक्रांशघ्नानि तानि स्युः कृत्स्नस्य परिधेर्भुवः~॥
	
याम्योदक्सूत्रगं वंशैरन्तरालं प्रमाय वा~।\\
चक्रभागहते वंशा भुवः स्युः परिधेस्तदा~॥
	
ज्ञातेष्वेतेषु शास्त्रोक्तगोलयुक्तिविदाखिलम्~।\\
ज्ञेयं लग्नादिकं सोऽथ गणयेल्लम्बनं विना~॥
	
स्पर्शमोक्षौ च यौ दृष्टौ याः स्युर्नाड्योऽन्तरे तयोः~।\\
त्रिज्याघ्ना दृग्गतिज्याप्तास्ताः स्युर्लम्बनजाः पराः~॥}
\end{quote}
\newpage

\begin{quote}
{\qt अर्केन्दुमध्यभुक्तिघ्ना षष्ट्याप्ता लिप्तिकान्तयोः\renewcommand{\thefootnote}{१}\footnote{लिप्तिकास्तयोः \textendash\ ग}~।\\
त्रिज्याघ्नं तद्धृतं वृत्तं भुवः कक्ष्या द्वयोरपि~॥

दशघ्नाश्चक्रलिप्ता वा कक्ष्या कल्प्या विधोरिह~।\\
तिथिघ्ना सूर्यसिद्धान्ते तद्घ्नलम्बनलिप्तिकाः~॥
	
परास्त्रिज्याहृता भूमेर्वृत्तं तद्योजनैर्मितम्~।\\
इन्दोः स्वभगणाभ्यस्ता कक्ष्यार्कभगणैर्हृता~॥
	
अर्ककक्ष्या यतो भुक्तिभेदाद् दृक्क्षेपतो नतिः~।\\
स्वमध्ये विषुवे यस्मात् दिनं प्राणाष्टकोनितम्~॥
	
नाडीभ्यस्त्रिंशतो रात्रिर्यद्वा प्राणाष्टकाधिका~।\\
स्फुटयोजनकर्णघ्नास्त्रिज्याप्ता लम्बलिप्तिकाः~॥
	
रवीन्दुबिम्बविष्कम्भयोजनानि सदैकधा~।\\
परीक्षेन्दोर्द्वितीयस्याप्युच्यते स्फुटकर्मणः~॥
	
येन केनचिदेवापि नक्षत्रेण मुहुर्मुहुः~।\\
मिथो दूरगताभ्यां वा द्वाभ्यां कतिपयैरपि~॥
	
अल्पविक्षेपकैर्योगेष्विन्दोः कार्याप्यतन्द्रितैः~।\\
छन्नानि यानि वा स्पृष्टानीन्दुना तानि तान्यपि~॥
	
लेख्यानि लक्षणैर्भानि तत्तदिन्दुस्फुटान्यपि~।\\
भत्रयं विवरं भानोरिन्दूच्चस्य यदा तदा~॥

तत्तत्स्फुटसमा\renewcommand{\thefootnote}{२}\footnote{सम \textendash\ ख. ग.}स्तेषां योगभानां यदा पुनः~।\\
समौ षड्भान्तरौ तद्वत् तन्मध्यस्थभसंयुतिः~॥
	
तदेन्दुस्फुटयोर्भेददलं तु परमं फलम्~।\\
तदा मुक्तेत्रऽहनीन्दूच्चयोगो निशि घटीद्वये~॥
	
विंश्यामर्केण पातेन द्वाविंश्यां ह्योभसंयुतिः~।\\
अर्केन्दूच्चयुतिश्चाह्नां देहमोक्षगणेऽपि च~॥} 
\end{quote}

\newpage

\begin{quote}
{\qt अब्दाष्टादशके भूयः प्राग्वदेव क्रमाद् युतिः~।\\
 आदौ दिवानिशोः पर्व यद्याद्येऽन्यत्तदन्तगम्~॥
 
ऊर्ध्वाष्टादशके तस्मान्महीमातृयुतेऽहनि~।\\
प्रत्यग्दृश्यादृश्येऽन्यत् पर्व पर्याययोर्द्वयोः~॥
 
प्राग्वदेव गता नाङ्यः पूर्णेऽष्टादशकत्रये~।\\
गोलसन्धिसमीपेऽर्कचन्द्रकेन्द्रेऽन्तरं महत्~॥
 
लम्बनेनापि सूर्यस्य मनुनाड्यन्तरे भवेत्~।\\
प्राक् प्राङ्नक्षत्रयोगश्च वर्धते तिथिदोःफलम्~॥
 
पूर्णबाहुर्यदा कश्चित् अर्केन्द्वोर्न्निर्भुजः परः~।\\
पूर्णबाहोस्तदा वृत्तं परीक्षेतायनद्वये~॥
 
निमीलनादिभेदेन खमध्ये क्षितिजद्वये~।\\
प्राङ्मध्यापरदेशेषु ज्ञातदेशान्तरेषु वा~॥
 
गोलज्ञैस्तद्गतैर्ज्ञेयं वृत्तं तात्कालिकं त्रिषु~।\\
मध्याह्ने ग्रासमध्ये वाप्येकस्मिन् ग्रहणे दिने~॥
 
स्पर्शमोक्षकपालैक्ये स्पर्शमोक्षनिमीलनैः~।\\
द्वयोर्वृत्तत्रयं ज्ञेयं ग्रहे कस्यापि तद्यथा~॥
 
भटवृत्तेन वौजान्तमयवृत्तेन वा स्फुटम्~।\\
आनीय पूर्णबाहोस्तु गणयेद् ग्रहणं स्फुटम्~॥
 
दृष्टानीतनिमीलनोत्थगतिभेदाभ्यस्तमन्त्यं गुणं\\
भङ्क्त्वा तन्नतजीवया दिनदलप्रोक्तस्ववृत्ताहतम्~।\\
मध्याह्नान्त्यफलोद्धृतं दिनदले वृत्ते धनर्णं तथा\\
मध्योक्तादितरद् यथा क्षितिजगं वृत्तं परं व्यत्ययात्~॥ 

ऋणस्वपरिधी रवेर्दिनदले मनूनां द्वयं\\
नभोऽश्विकलितोनितं तिथिघटी नतस्य क्रमात्~। \\
सुरेशवरुणाशयोः नखकलाभिराढ्योनितं\\
विवर्जितसमन्वितं द्युदलवत् त्रियामादले~॥}
\end{quote}

\newpage

\begin{quote}
{\qt रदद्वयं सिद्धकलोनमिन्दोर्मध्यं दिने\renewcommand{\thefootnote}{१}\footnote{नेष्य \textendash\ ग. घ.} द्व्यक्षकलोनयुक् प्राक्~।\\
पश्चाद्युतोनं रविवन्नतस्य ऋणस्वसंज्ञं परिधिद्वयं स्यात्~॥
 
तद्दिनार्धपरिधिद्वयान्तरेणाहता स्वनतशिञ्जिनीहृता~।\\
त्रिज्ययाथ परिधिर्दिनार्धजे हीनके स्वमधिके त्वृणं स्फुटम्~॥
 
इत्याह श्रीपतिर्भानुचन्द्रयोः परमे फले~।\\
प्राक् पश्चाच्च महान् भेदस्ताराचन्द्रसमागमे~॥
 
इति संक्षेपतः प्रोक्ता परीक्षा ज्योतिषामिह~।\\
कालमानचतुष्कस्य श्रुतस्य विवृतिस्त्वियम्~॥}

\end{quote}

\indent इह तु कदाचिदपि कर्मणां विच्छेदासम्भवात् तदाधारभूतकालबोधकानां प्रबन्धविशेषाणामप्यविच्छेदात् तदुक्तगोलगणितयुक्तिविद्भिः
परीक्षकैः तत्तत्करणेषु विशेषाधानमेव कार्यमित्यतोऽपि लाघवं स्यात्~। सुन्दरराजप्रश्नोत्तरे मयोक्तप्यत्रानुसन्धेयम्~। तत्रेदानीमार्यभटजिष्णु
नन्दनभट्टोत्पलमुञ्जालकपरमेश्वराद्याचार्यैः परीक्षितग्रहगत्या ग्रहणादिकमानीय परीक्षा क्रियतेऽस्माभिः~। तद्यथा \textendash\ कल्यब्दे तावद्
षष्टिवर्गपरिमिते आर्यभटोक्तभगणाद्यानीतमध्यमादिकं प्रायशो दृक्सममभूदिति तत्प्रमाणादवसीयते~। तदुक्तभगणादीनां नानाप्रतिपादनात्
तत्प्रभृति ग्रहगतेः क्रमेण स्थूलता स्यात्~। ततस्तत्परिहारार्थमन्यैरपि संस्काराः प्रदर्शिताः\textendash 
\begin{quote}
{\qt चन्द्रे बाणकरा बीजाश्चन्द्रोच्चे मनुभूमयः~।\\
 कुजे शून्यशरा ज्ञेयाः खाग्निवेदा बुधस्य तु~॥
 
 गुरोः खपञ्च विज्ञेयाः शुक्रे खाष्टनिशाकराः~।\\
 शनेः शशिकराः प्रोक्ता राहोः षण्णवति स्मृताः~॥
 
 भवभानूनिते शाके बीजघ्ने शबरोद्धृते~।\\
 फलं लिप्ता विलिप्ताश्च ज्ञारार्कीणां धनं भवेत्~॥}
\end{quote}

\newpage

\begin{quote}
{\qt राहुचन्द्रोच्चजीवानामृणं कार्यं भृगोरपि~॥}
\end{quote}

\noindent अन्यस्तमेव सविशेषमाह\textendash 

\begin{quote} 
{\qt वाग्भावोनाच्छकाब्दाद्धनशतलयहान्मन्दवैलक्ष्यरागैः\\
 प्राप्ताभिर्लिप्तिकाभिर्विरहिततनवश्चन्द्रतत्तुङ्गपाताः~।\\
शोभानीरूढसंविद्गणकनरहतामागराप्ताः कुजाद्याः\\
संयुक्ता ज्ञारसौराः सुरगुरुभृगुजौ वर्जितौ भानुवर्जम्~॥}
\end{quote}

\noindent इति~। लल्लस्त्वाह\textendash 
\begin{quote}
{\qt शाके नखाब्धिरहिते शशिनोऽर्क्षदस्रैः\\
 तत्तुङ्गतः कृतशिवेस्तमसः षडङ्कैः~।\\
शैलाब्धिभिः सुरगुरोर्गुणिते सितोच्चात्\\
शोध्यं त्रिपञ्चकहतेऽभ्रशराक्षिभक्ते~॥

स्तम्बेरमाम्बुधिहते क्षितिनन्दनस्य\\
सूर्यात्मजस्य गुणितेऽम्बरलोचनैश्च~।\\
व्योमाग्निवेदनिहते विदधीत लब्धं\\
शीतांशुसूनुचलतुङ्गकलासु वृद्धिम्~॥}
\end{quote}

\noindent इति~। अस्यायमभिप्रायः \textendash\ भवभानूनिते शाक इति वदता {\qt षष्ट्यब्दानां षष्टि}रित्यस्यार्थोऽप्यन्यथा गृहीतः~। तस्मादेतद्वाक्याद् भवभानुशबरयोगमिते शाके बाणकरादिलिप्ता आर्यभटोक्तचन्द्रादिमध्यमेषु संस्कार्या इत्येतावदेव ग्राह्यम्~। न पुनस्तदुक्तेच्छाप्रमाणे~। किन्तु ते अप्येतद्वाक्यादवगम्येते~। कथं भटाब्द इच्छाराशिः, भवभानुशबरयोगमिते शाके यो भटाब्दः स प्रमाणराशिरिति~। अत उक्तम् {\qt अष्टशराक्षिभक्त} इति~। मूर्च्छनाब्धिरहित इति वक्तव्ये {\qt नखाब्धिरहित} इति वदतो लल्लस्येच्छाया एकाधिक्यात् तत्फलस्य यदन्तरं तत् सर्वदा सममेव~। तच्च स्वल्पमिति न दोषाय भवति, प्रत्युत गुण एवासौ, यस्मादार्यभटानीत-


\newpage

\noindent चन्द्रात् तत्करणकालेऽपि दृक्संवादाय किञ्चिच्छोद्ध्यम्~। इत्युत्तरकालग्रहणादिदर्शनादेव निर्णीयत इत्यभिप्रायः~। प्रदर्शितश्चार्यभटेनान्येन वा तदानीं संस्कारः\textendash 
\begin{quote}
{\qt वस्वेकेषु युगघ्नं मनुयुगमर्कादिमध्यमचतुर्णाम्~।\\
धनमृणमृणमथ देयम्~॥}
\end{quote}

\noindent इति~। बुधादीनामप्यार्यभटसिद्धान्तयोरेव भगणभेदाद् ग्रन्थकरणकालेऽपि तन्मध्यगानां स्थूलता युज्यते~। इति खाग्निवेदादीनां बाणकरादिकलानामपि स्थूलता~। अनेनैव न्यायेन लल्लाद्युक्तानां फलानामपि स्थूलता~। तस्मात् सूक्ष्मतमः संस्कारोऽन्वेष्यः~। तथा कृतशरवसुमितशाके मुञ्जालकोक्तमपि ग्रहमध्यमादिकं प्रायशो दृक्समम्~। स चैवमाह\textendash 

\begin{quote}
{\qt कृतशरवसुमितशाके चैत्रादौ सौरिवारमध्याह्ने~।\\
 राश्यादिरजनृपार्का रविरिन्दुर्भवधृतिद्वियमाः~॥
	
द्व्युत्कृतिखाग्नियुगोत्कृतिकराब्धयः खाष्टनववसुद्वियमाः~।\\
गोष्टाविंशतितानाः कुजादयः सूर्यभगणान्ते~॥
	
सूर्यान्मन्दोच्चांशा वसुतुरगाः पर्वतास्तु सत्र्यंशाः~।\\
स्वररवयः खाकृतयो द्विनगभुवोऽशीतिरद्रिजिनाः~॥
	
सङ्क्रान्तितिथिध्रुवकाः शक्रा वसुनवरसेषवो राहोः~।\\
कृतयमवसुरसदशका दशाहताः शेषपातांशाः~॥
	
अयनचलनाः षडंशाः पञ्चाशल्लिप्तिकास्तथैकैका~।\\
प्रत्यब्दं तत्सहितो रविरुत्तरविषुवदादिः स्यात्~॥}
\end{quote}

\noindent इति~। तथा त्रीषुविश्वभिते शाके परमेश्वराचार्यप्रणीतदृग्गणितानीतमध्यमादिकं प्रायशो दृक्समम्~। तथा च तद्वाक्यम्\textendash 

\begin{quote}
{\qt एवं दृग्गणितं शाके त्रीषुविश्वमिते कृतम्~।\\
 परमादीश्वरेणैतत् प्रायो भवति दृक्समम्~॥}
\end{quote}

\newpage

\noindent इति~। तत्रार्यभटोक्तप्रकारेण भवभानुशबरयोगमिते शाके मध्यममानीय भवभानूनिते शाक इत्यादिना संस्कृत्य, त्रीषुविश्वमिते पुनः परमेश्वरोक्तप्रकारेणाप्यानीय, तयोर्विश्लेषं कृत्वा, तत्फलत्वेनाङ्गीकृत्य, तदन्तरकालं प्रमाणतया परिगृह्य, आर्यभटपरमेश्वरग्रन्थकरणकालान्तरमिच्छाराशिं कृत्वा, त्रैराशिकेनानीतां ग्रहगतिं त्रीषुविश्वमितशाकानीताद् विशोध्य,
आनीतमार्यभटग्रन्थकरणकालमध्यममार्यभटानीततत्कालमध्यमात् भिन्नम्~। तस्मात् एकतमस्यचित् स्वकाले दृग्वैषम्यं निर्णेयं, तत्सर्वेषां वा
कल्प्यम्~। तच्च आर्यभटीयस्यैव सिद्धान्तत्वात् युगभगणनिःशेषतापारतन्त्र्यादल्पवयः कर्तृकत्वाच्च युज्यते\textendash 

\begin{quote}
{\qt ज्ञात्वेवैतत् स्वसंवेद्यमार्यभटोऽब्रवीत्~।\\
 त्र्यधिका विंशतिश्चाब्दा जन्मने मे गता इति~॥}
\end{quote}

\noindent तत्र ग्रहणहेतुरविचन्द्रतुङ्गपातानां {\qt वस्वेकेषु युगघ्नं} इत्याद्युक्तसंस्कारकरणमुपपन्नम्~। यद्यपि भूदिनेष्टकं क्षिप्त्वा तिथौ समीकरणं कार्यं, तथापि गर्गाभिमतभूदिनपरिग्रहाय तावानेव भेदः सोढव्य इति भावः~। तत्र यद् ग्रन्थकरणकाले कलेर्दिव्याब्ददशके गतेऽपि तिथिसंवादाय लिप्ताचतुष्कं त्याज्यं, तदौदयिकाच्चन्द्रादेव वा त्याज्यम्~। उभयथापि तिथिसाम्यात्~। ततो बाणकरेभ्यो लिप्ताचतुष्के शुद्धे यच्छिष्यते, तदेवाष्टशराक्षिसम्मिताब्दस्य प्रमाणस्य फलत्वेन ग्राह्यम्~। तत्कलाश्व प्रायो मूर्च्छनासमाः~। ततश्चन्द्रे शशिकरा बीजा इति पाठः कार्यः, तथाचेदानीमपि प्रायेण ग्रहणादिकं संवदते~। नन्वार्यभटीयानीतबुधशीघ्रोच्चमध्यमात् अत्रोक्तत्रैराशिकोन्नीतम् आर्यभटकाले भागषट्काधिकम्~।
तच्च अयुक्तम्~। यस्मादाचार्येणैव तन्त्रान्तरे विंशत्योनाः तद्भगणाः पठिताः~। तैरानीतं च तदानीमन्यस्माद् भागषट्कोनम्~। तस्मात् 
आर्यभटोक्तद्वितीयसिद्धान्तानीतान्नाधिकं बुधशीघ्रोच्चमिति निर्णीतम्~। न्यूनतायामेव पुनः सन्देहः~। तस्मात् तत्संस्कारकर्तुर्वा परमेश्वरस्य वा

\newpage

\noindent स्खलनं युक्तं, नत्वार्यभटस्येति~। मन्द मैवम्\textendash 
\begin{quote}
{\qt त्रिशती भूदिने क्षेप्या ह्यवमेभ्यो विशोध्यते~।\\
ज्ञगुर्वोर्भगणेभ्योऽपि विंशतिश्च ततोऽब्धयः~॥}
\end{quote}

\noindent इत्यत्रैवं वा योजना~। ज्ञगुर्वोर्भगणेभ्यो विंशतिरब्धयश्च, तथा क्षेप्याः शोध्याश्च क्रमेण~। ज्ञभगणेषु विंशतिः क्षेप्याः~। भृगुभगणेषु चाब्धयः शोध्यन्त इति~। तस्मात् आर्यभटोक्तपूर्वतन्त्रानीतबुधशीघ्रोच्चं सूक्ष्मतरमन्याभ्यामपि सम्यक् परीक्षितमिति सारभूतम्~। शुक्रस्य च पूर्वतन्त्रानीतमन्यस्माद् द्वासप्ततिकलान्यूनम्~। उक्तत्रैराशिकानीतं च तयोर्मध्येऽवतिष्ठते~। अपि च भौममन्दोच्चमपि पूर्वतन्त्रे विंशतिभागाधिकं
राशित्रयं पठितम्~। द्वितीये पुनर्भागद्वयोनं राशिचतुष्टयम्~। श्रीजैष्णवमुञ्जालकपरमेश्वरादिभिः पुनर्भागसप्ताधिकं राशिचतुष्टयं निर्णीतम्~।
तस्माद् आर्यभटीयानीतमध्यवर्तिनैव भाव्यं तात्त्विकेनेति न नियमः~। मन्दमन्दोच्चं तु पूर्वतन्त्रोक्तं, राश्यष्टकमिदानीमपि सूक्ष्मम्~। तस्माद्
आचार्येण दिङ्मात्रप्रदर्शनार्थमेव भगणादिकमभिहितम्~। अत एव हि न्यायमार्गप्रतिपादकात् प्रबन्धाद्बहिरेव सङ्ख्याभिधानं कृतं,
ग्रहकक्ष्याबिम्बयोजनदिनगत्यादिद्वारा लम्बनादिनिरूपणार्थमिति सर्वमनवद्यम्~। एवं कृतशरवसुमितशाकध्रुवकाश्च मुञ्जालकपठिता अनेनैव त्रैराशिकेनानीतैः प्रायशः समाः~। तेन मानसकारेण पूर्वोक्तेष्वेव परीक्षापूर्वकं स्थूलपरित्यागः सूक्ष्मग्रहणं चैव कृतम्~। न पुनरावापोद्वापौ~। ततः
तद्ध्रुवका अपि तदानीं प्रायिकाः~। मध्यमाद्यानयनमपि लाघविकेन प्रायिकमेव प्रदर्शितम्। अत एवोक्तम्\textendash 

\begin{quote}
{\qt चैत्रादौ वारसङ्क्रान्तितिथ्यर्केन्दूच्च सध्रुवान्~।\\
ज्ञात्वान्यांश्चार्कवर्षादावाजन्म गणयेत्ततः~॥}
	\end{quote}
	
\noindent इति~। तत् परीक्षा दिङ्मात्रं च\textendash 

\begin{quote}
{\qt षडक्षाङ्गुलयष्ट्यग्रे दृङ्मध्यादंशकोऽङ्गुलम्~।}
\end{quote}

\newpage

\noindent इति प्रदर्शितम्~। परमेश्वराचार्येण पुनर्ग्रहणग्रहयोगादिकं यन्त्रैः पञ्चपञ्चाशद्वर्षकालं सम्यक् परीक्षितम्~। आह चैवम् \textendash\ {\qt ग्रहेन्द्रा
पञ्चपञ्चाशद्वर्षकालषड्त(?)न्तर} इत्यादि~। तस्य गोलवित्तमत्वं च गोलदीपिकादिभिः तत्कृतैः प्रबन्धविशेषैर्ज्ञायते~। अतोऽन्येषां करणानां स्वस्वकाले यावत् सूक्ष्मत्वं ततोऽप्येतत्कृतस्य दृग्गणिताख्यस्य सूक्ष्मतमत्वमुपपद्यते~। अपि च परमेश्वरोक्तमन्दोच्चपरिध्यादीनामार्यभटोक्तेभ्यः सूक्ष्मत्वमिदानीमपि ग्रहयोगादिष्वस्माभिर्ज्ञायते~। मध्यमानां हि गुणहारयोः स्थूलत्वात् कालदैर्घ्यात् स्थूलत्वं वर्धते, न पुनर्वृत्तादीनाम्~। मन्दोच्चानां तु प्राग्गतित्वात् आर्यभटाद्युक्तेभ्यो न्यूनत्वमिदानीं नोपपद्यते~। न्यूनत्वं च गुरुशुक्रमन्दोच्चयोरन्यैर्बहुभिरप्यङ्गीकृतम्~। तस्मात् आर्यभटीयानीतग्रहस्फुटस्य स्थूलतैव युक्ता~। स्फुटस्य स्थूलत्वे मध्यमोच्चपरिध्यादीनां स्थूलता स्यात्~। भास्करादीनां पुनर्युक्तिगम्येऽपि वलनविक्षेपदर्शनसंस्कारादौ भ्रान्तिरभूत्~। तस्मात् तैरपि न सम्यक् परीक्षितम्~। परमेश्वरस्तु रुद्रपरमेश्वरात्मजनारायणमाधवादिभ्यो गोलविद्भ्यो गणितगोलयुक्तीरपि बाल्य एव सम्यग्गृहीत्वा तेभ्य एव क्रियमानप्रयोगस्य दृग्विसंवादं तत्कारणं चावधार्य शास्त्राण्यपि बहून्यालोच्य पञ्चपञ्चाशद्वर्षकालं निरीक्ष्य, ग्रहणग्रहयोगादिषु परीक्ष्य, समदृग्गणितं करणं चकार~। तस्यादावेय स्वयमेवोक्तम\textendash 

\begin{quote}
{\qt सर्वं दृश्यन्ते विहगा दृष्ट्या भिन्नाः परहितोदिताः~।\\
प्रत्यक्षदृष्टाः स्पष्टाः स्युर्ग्रहा शास्त्रेष्वितीरितम्~॥

सत्कर्मोदितकालस्य ग्रहा हि ज्ञानसाधनम्~।\\
अस्पष्टविहगैः सिद्धः शुद्धः कालो न कर्मणि~॥

ये तु शास्त्रविदस्तद्वद् गोलयुक्तिविदश्च तैः~।\\
स्फुटखेचरविज्ञाने यत्नः कार्यो द्विजैरतः~॥

सञ्चिन्त्येति समालोच्य पूर्वतन्त्राणि यत्नतः~।\\
स्फुटयुक्तिं खेचराणां गोलदृष्ट्या समीक्ष्य च~॥}
\end{quote}

\newpage

\begin{quote}
{\qt स्फुटखेचरविज्ञानं शिष्यैर्यैः प्रार्थितं द्विजैः~।\\
तेभ्यो दृग्गणितं नाम गणितं क्रियते मया~॥}
\end{quote}

\noindent इति~। अतःपरमपि शिष्यप्रशिष्यसन्तानपरम्परया परीक्ष्यमाणा मध्यमगतिरपि सुसूक्ष्मा ज्ञातुं शक्या~। तदर्थं करणान्ते तत्कालश्च
प्रदर्शितः, {\qt शाके त्रीषुविश्वमिते कृतम्} इति~। त्रैराशिकसिद्ध्यर्थं वाग्भावेत्याद्युक्तसंस्कारसहितार्यभटीयानीतमध्यमानामपि सूक्ष्मं
संस्कारान्तरमप्यनेनैव न्यायेन कर्तव्यम्~। तत्र {\qt त्रीषुविश्वमिते शाके} वाग्भावादि संस्कृतस्यार्यभटीयमध्यमस्य दृग्गणितसिद्धस्य च यावदन्तरं
तत्फलत्वेन ग्राह्यम्~। षड्विंशत्यूना सप्तशती प्रमाणम्~। वाग्भावमागरयोगहीनश्शकाब्द इच्छाराशिः~। एभिस्त्रैराशिकानीतं फलं पुनरपि तत्रैव
संस्कार्यम्~। एवमानीतं चिरमप्यविसंवादि~। किन्तु तत्रापि विशेषाधानेन सुसूक्ष्मत्वं सम्पादनीयम्~। अतः परीक्ष्यैव सर्वैरपि शिष्येभ्य
उपदेष्टव्यम्~। तथा सति न विसंवादः कदाचिदपि भविष्यति~। जायमानस्यापि परिहार्यत्वात्~। अश्वत्थग्रामजो भार्गवः परमेश्वरः
सिद्धान्तदीपिकायां स्वदृष्टेषु कानिचिद् ग्रहणान्युदाहृत्य तैः कल्पितानि अर्केन्दुतुङ्गपातमध्यमान्यपि कस्मिंश्चिद्द्युगणे ध्रुवकत्वेन प्राह\textendash 

\begin{quote}
{\qt द्युगणे सप्तनागाग्निगुणेषुरसभूमिते~।\\
गोकर्णे ग्रहणं भानोर्दृष्टं नात्र निलातटे~॥

शून्याग्निभूशरेष्वङ्गभूतुल्ये द्युगणे रवेः~।\\
गोकर्णे ग्रहणं दृष्टं निलाब्ध्योः सङ्गमेऽत्र न~॥

प्रोक्ते दिनेऽपि बिम्बस्य पार्श्वे वर्णस्य भेदनम्~।\\
कैश्चित्कुमारैरत्रापि कल्पितं वा निलातटे~॥

द्युगणे शशिशून्याक्षिबाणाङ्गशशिसम्मिते~।\\
सुव्यक्तं ग्रहणं दृष्टं नावाक्षेत्रेऽत्र तीक्ष्णगोः~॥}
\end{quote}

\newpage

\begin{quote}
{\qt स्पर्शोपलब्धौ पदभा चत्वारिंशन्मितात्र तु~।\\
पञ्चत्रिंशन्मितेत्येके वदन्ति व्यक्तिभेदतः~॥

द्विषड्रसेषुपञ्चाङ्गविधुभिर्द्युगणे मिते~।\\
ईषद्ग्रस्तो रविर्दृष्टो निलायां तु सुदृष्टिभिः~॥

स्पर्शकाले तु पदभा तत्र पञ्चदशोन्मिता~।\\
प्रायशो मोक्षकाले तु दशभिर्वा दलोनितैः~॥

अस्मिन् दिने ग्रहेर्कस्य मण्डलं पश्यतां नृणाम्~।\\
नातितप्ते दृशौ मान्द्यं तैक्ष्ण्यस्यातोऽत्र कल्पितम्~॥

कृतद्विबाणरामाब्धिषट्चन्द्रैर्द्युगणे समे~।\\
पदमैकादशारम्भे मोक्षो दृष्टोऽपराह्नजः~॥

स्थितिकालोऽधिकोऽह्यत्र नाडिकानवकादतः~।\\
कार्यमेवाविशेषादि धीमता द्युगणैरपि~॥

द्युगणे रसतिथ्यद्रिवेदषड्भूमिसम्मिते~।\\
मुक्तेऽर्केऽस्तमयो दृष्टो नाडीपादोऽस्तिवान्तरे~॥

द्व्यक्ष्यद्यष्टाब्धिषट्चन्द्रसम्मिते द्युगणे पुनः~।\\
स्पर्शे तु पदभार्कस्य चतुर्विंशतिसम्मिता~॥

अह्नां गणेऽब्धिनागाब्धिपञ्चेष्वङ्कैकसम्मिते~।\\
मोक्षकालो रवेः सार्धैः पञ्चभिः पदभा मिता~॥

दिनौघेऽद्यब्धिषट्पञ्चबाणाङ्गैकमिते विधोः~।\\
संस्पर्शे तिथिभिर्द्वाभ्यां मोक्षे च पदभा मिता~॥

रामरन्ध्रयमप्राणबाणषट्च्छशिसम्मिते~।\\
द्युगणे नैव दृष्टं खे ग्रहणं शीतदीधितेः~॥

वेदरन्ध्ररसाक्षीषुरसशीतांशुभिर्मिते~।\\
द्युगणे शीतगुर्दृष्टः ईषद्ग्रस्तोऽम्बरे नृभिः~॥}
\end{quote}

\newpage

\begin{quote}
{\qt मन्वङ्गाब्धीषुषट्चन्द्रैः सम्मिते द्युगणे विधोः~।\\
	दृष्टो विमर्दस्तद्वत् त्रिखाब्धित्रीषुरसेन्दुभिः~॥
	
उक्तेभ्योऽन्ये चोपरागा मया दृष्टा विवस्वतः~।\\
इन्दोश्च बहवो दृष्टाः ते तु नोदाहृता इह~॥
	
एतानतीतोपरागान् सञ्चिन्त्य परिकल्पिताः~।\\
विलिख्यन्ते मया भानुचन्द्रचन्द्रोच्चराहवः~॥
	
द्युगणे व्योमशून्याद्रिचन्द्रेषुरसभूमिते~।\\
सूर्यस्य मेषसंस्थस्य तिथिभिः सम्मिताः कलाः~॥
	
घटस्थस्य विधोर्भागा वेदा लिप्ता रसैर्मिताः~।\\
तुङ्गस्य कर्कटे भागा नव लिप्ताः स्वरेषवः~॥
	
पातस्य सिंहे त्रियमा भागाः प्राणशराः कलाः~।\\
अर्धाधिकं गृहीतं वै प्रोक्तेष्वर्केन्दुराहुषु~॥
	
अस्मिन् काले रवीन्दूच्चपातानां स्थितिरीदृशी~।\\
गुरुषु स्वर्गतेष्वत्र सर्वग्रासो मया रवेः~॥
	
नावाक्षेत्रगतेभ्यः प्रागुक्त्वा कालश्च दर्शितः~।\\
हंसो विहततापोनो द्युगणोऽब्दो हितः शिवः~॥
	
दृष्टो याम्यांशको देशे त्रिपुराल्पपलाङ्गुले~।\\
अष्टादशविनाड्यल्पे काले देशान्तरे धने~॥
	
गोकर्णस्थैरुदग्भागः सम्यग्दृष्टो जनैः किल~।\\
मोक्षो मध्याह्न एवात्र स्पर्शे शङ्कुसमा प्रभा~॥
	
ग्रहेन्द्राः पञ्चपञ्चाशद्वर्षकालं निरीक्षिताः~।\\
मया तदा दृशा भिन्ना दृष्टाः परहितोदिताः~
	
उदाहरणमप्यत्र किञ्चित्तेषां विलिख्यते~।\\
अजो गणेशतुल्येऽह्नि बुधवारनिशोषसि~॥}
\end{quote}

\newpage

\begin{quote}
{\qt आरे चन्द्रसमे भागा गणिते षट् तदन्तरे~।\\
पार्थोऽसौ सोमतुल्येऽह्नि सूर्यवारनिशोषसि~॥
	
मन्दं चन्द्रसमे भागाः पञ्च तद्विवरोद्भवाः~।\\
गर्भे स्याद्विष्णुतुल्येऽह्नि सूर्यपुत्रदिनोदये~॥

समौ सितज्ञौ विवरे पञ्चांशा गणिते तयोः~।\\
वने राजगताढ्येऽह्नि प्राच्यां दृष्टो बुधस्तथा~॥

अर्कज्ञयोरन्तरांशास्तदाष्टौ दलवर्जिताः~।\\
गजोऽसौ सोमतुल्येऽह्नि निशान्ते बुधवारगे~॥

समकालावस्तमयौ शशिनश्च गिरां पतेः~।\\
तयोस्तदानीं विवरे गणितेऽर्धयुतोऽंशकः~॥

दाता योगशताढ्येऽह्नि गिरां प्रतिदिनोदयात्~।\\
प्राङ्नाड्यां चन्द्रबिम्बेन छन्नो दृष्टः सुतो रवेः~॥

विवरे तु तयोर्भागास्तदानीं वेदसम्मिताः~।\\
गोत्रज्ञः समतुल्येऽह्नि रात्रौ वारे तु शीतगोः~॥

शन्यारबिम्बसंस्पर्शः सम्यग्रदृष्टो महाजनैः~।\\
गणिते तु तयोर्भागः विवरे तु दशाधिकाः~॥} 
\end{quote}

\noindent तावेणे सत्सु तुल्यौ, शन्यारसंयोगः~। राज्ञा नुन्नो हन्त ज्योक्, चन्द्रेण छन्नो गुरुः~। द्वौ खगौ सक्तौ चेड्यौ~। गुरुशुक्रयोगः~। द्यौः मन्दहसिताढ्या, शनिशुक्रयोगः~॥~४८~॥ \\

\indent इह ग्रहगणितसाधनेषु सङ्ख्याविशेषेषु केषाञ्चिदतात्त्विकत्वात् सङ्ख्यापादं पृथक्कृत्य गीतिकापादे निक्षिप्य, तस्यापि गौरवमापादयितुं तत्रापि गणितकालक्रियागोलयुक्तिकलापं संक्षेपेण प्रदर्श्य, सङ्ख्यामूलप्रमाणप्रदर्शनपरेणानेन सूत्रेण तद्गतं स्थौल्यदोषं परिहरन् शास्त्रार्थं निरवद्यं
परिपूर्योपसंहरन्नुपरितनेन सूत्रेण स्वशास्त्रं स्तुवन् पूर्वशास्त्रेभ्योऽस्य विशेषं दर्शयति\textendash 

\newpage
\begin{quote}
{\ab सदसज्ज्ञानसमुद्रात् समुद्धृतं देवताप्रसादेन~।\\
सज्ज्ञानोत्तमरत्नं मया निमग्नं स्वमतिनावा~॥~४९~॥}
\end{quote}

\indent इति~। यथा समुद्रे रत्नाकरेऽपि नौकादिभिर्यानैर्यान्तो दाशादयो वणिजश्च तदुपरिगतैः पद्मकुमुदाकरादिभिराछन्नत्वात् तत्तलनिमग्नं रत्नजातमनुपलभमाना मत्स्यशङ्खनक्रतिमिङ्गिलादिभिर्दुरवगाहत्वात् च तत्र निमग्नं रत्नमजानाना, लब्धैर्मत्स्यशङ्खादिभिरेव कृतार्था निवर्तन्ते~।
एवं गणितस्कन्धेऽपि सिद्धान्तमहातन्त्रादिभिर्नानाविधैः प्रबन्धैर्विस्तृतत्वात् गाम्भीर्याच्च समुद्रत्वेन रूप्ये अध्ययनश्रवणादिभेरवगाहमाना अपि तत्र
निमग्नं सज्ज्ञानोत्तमरत्नमसज्ज्ञानैरावृतत्वादनुपलभमाना एवासज्ज्ञानैरेव लब्धैः कृतार्थान् निवर्तमानानेव कांश्चिन्मन्दमतीन् दृष्ट्वा मया
स्वयंभुवः प्रसादेन स्वमत्याख्यया नावा दूरं गत्वा तत्रावगाह्य तत्रासारान् यादस्थानीयान् पदार्थानुत्सृज्य तत्तलं प्राप्य युक्त्यात्मकं
सज्ज्ञानोत्तमरत्नमाचिन्वता समुच्चित्योद्धृत्य प्रदर्शितम्~। रत्नगृघ्नुभिः गृह्यन्तां रत्नानीति~। एतदुक्तं भवति \textendash\ मध्यमस्फुटानयनादिकं दश भेदं ग्रहगणितं तत्साधनसङ्ख्यास्थौल्याद् ज्याग्रहणवापीकरणादिकर्मणां च युक्तिभिस्साद्ध्यत्वादल्पफलानां केषाञ्चित् कर्मणामुपेक्षयानुक्तत्वाच्च तद्युक्तीनां तैस्तिरोहितत्वात् कस्यचिदपि स्वयमुत्प्रेक्ष्य ज्ञातुमशक्यत्वाच्च तत्रानवगाहमानाः क्रियामात्रेणैव कृतार्था दैवज्ञत्वं प्रपद्यमाना एव दृश्यन्ते इति~। तान्प्रत्येव श्रीपतिरपि सदण्डमाह\textendash  
\begin{quote}
{\qt अविदितपदलक्ष्मा वावदूकोऽभियात-\\
श्चतुरबुधसभायां यद्वदायाति हास्यम्~।\\
ग्रहगणितविदेवं गोलतत्त्वानभिज्ञो\\
गणकसदसि विद्वच्चक्रवक्रोक्तिहीनः~॥}
\end{quote}

\newpage

\begin{quote}
{\qt यद्वदिन्दुरहिता न शर्वरीभाति शीलरहिता न चाङ्गना~।\\
	ब्राह्मणश्च न चरित्रवर्जितस्तद्वदेव गणकोऽप्यगोलविद्~॥}
\end{quote}

\noindent इति~। तत्र सारभूतैर्युक्तिविशेषैरेवाल्पफलानामुपेक्षितानां कर्मणामुपलब्धिः~। प्रमाणफलादीनामतात्त्विकत्वेन इच्छाकालमहत्त्वानुरूपं
वर्धमानमपि तदन्तरं युक्तिविद्भिरेव यन्त्रसहकृतया दृष्ट्या परिछेत्तुं शक्यम्~। तथाचाह गर्गश्च\textendash 

\begin{quote}
{\qt तयैव ज्ञानभूयस्त्वाद् दैवज्ञस्यावधारणा~।}
\end{quote}

\noindent इति~। तस्मादयुक्तिविदां कर्ममात्रपराणां ज्ञानभूयस्त्वाभावात् तिथ्यादिप्रतिपत्तिछेदादयोऽपि न निर्णेतुं शक्या इति, सर्वदापि
कैश्चिद्युक्तिविद्भिः भाव्यम्~। इतरथा श्रौतस्मार्तकर्माण्यकाल एव क्रियेरन्~। कस्यचिदपि प्रतिपत्तिछेदनिर्णयाभावात्~। तदप्याह गर्गः\textendash 

\begin{quote}
{\qt मुहूर्ततिथिनक्षत्रमृतवश्चायने तथा~।\\
सर्वाण्येवाकुलानि स्युः न स्यात् सांवत्सरो यदि~॥

नगरद्वारि लोष्ठस्य यद्वत्स्यादुपयाचितम्~।\\
आदेशस्तद्वदज्ञानां यत्सत्यं न विभाव्यते~॥
	
नक्षत्रसूचकोद्दिष्टमुपवासं करोति यः~।\\
स याति चान्धतामिस्रं सार्धमृक्षविडम्बकैः~॥
	
न सांवत्सरिके देशे वस्तव्यं भूतिमिच्छता~।\\
चक्षुर्भूतो हि यत्रैष पापं तत्र न विद्यते~॥}
\end{quote}	
	
\noindent इत्यादि~। नक्षत्रसूचकलक्षणं च स एवमाह\textendash 
\begin{quote}
{\qt अविदित्वैवं यः शास्त्रं दैवज्ञत्वं प्रपद्यते~।\\
	स पङ्क्तिदूषकः पापो ज्ञेयो नक्षत्रसूचकः~॥}
	\end{quote}

\noindent इति एवं युक्तिज्ञानाभावे कर्मणां वैकल्यान्महान् दोषः स्यात्~। इति सर्वेष्वपि जनपदेषु सर्वदापि कतिपयैर्दैवविद्भिर्गोलयुक्तिविद्भिर्भाव्यम्~। तत्र विशेषेण कलावश्रद्दधानत्वात् मन्दमतित्वाच्च दस्युराजाद्युपद्रुतत्वाच्च,

\newpage

\noindent सिद्धान्तादिप्रबन्धनिरूढं गणितसामान्यन्यायं विविच्य ज्ञातुमशक्यत्वात् मया विवच्यार्पितैरेवाशेषगणितव्यापिभिर्न्यायैः यावदपेक्षं विषयविशेषानुरूपं कर्मविशेषं निरूप्य ज्ञात्वा तेन मार्गेण गणयित्वा तिथ्यादिप्रतिपत्तिछेदौ कथं तत्वतो ज्ञेयौ स्यातामिति युगानुरूपेणाल्पेनैव
प्रबन्धेन सूत्ररूपेण गणितन्यायकलापः कार्त्स्न्येन निबद्ध इत्यर्थः~॥~४९~॥\\

\indent अथान्त्येन सूत्रेण {\qt मङ्गलान्तानि हि शास्त्राणि प्रथन्ते वीरपुरुषाणि च भवन्त्यायुष्मन्तश्च श्रोतार} इत्युक्तेरेवमस्य शास्त्रस्य
माहात्म्यातिशयोऽपि, तस्य प्रतिकञ्चुकचिकीर्षूनदृष्टशङ्काजनकेन भेदेन वशमानीय ततो व्यावर्तयिष्यान् शिष्यानस्याध्येतृञ्छ्रोतॄन् वक्तृश्च प्रत्याशिषमाशास्ते\textendash 

\begin{quote}
{\ab आर्यभटीयं नाम्ना पूर्वं स्वायम्भुवं सदा सत्यम्~।\\
सुकृतायुषोः प्रणाशः\renewcommand{\thefootnote}{१}\footnote{णाशं \textendash\ इति मुद्रितपाठः} कुरुते प्रतिकञ्चुकं योऽस्य~॥~५०~॥}
\end{quote}

\indent इति~। यदेतच्छास्त्रं पूर्वं जगत्सृष्टौ ब्रह्ममुखविनिस्सृतत्वात् नाम्ना स्वायम्भुवमासीत्~। तदेवेदमिदानीं नाम्नार्यभटीयमभूद्, अस्मन्मुखविनिःसृतत्वात् केवलं शब्दमात्रस्यैवोभयोर्भेदः~। अर्थजातं तदेव कृत्स्नं, न मनागपि भेदः~। अतोऽर्थात्मना सदा सत्यमेवेदं न्यायानां
कदाचिदप्यबाध्यत्वात्~। ज्ञानात्मकत्वादेवाबाध्यत्वमपि~। जडस्यैव हि नित्यारोपितत्वात् तत्तद्भेदज्ञानेन बाध्यत्वमिति भावः, एवमेव
सर्वविद्यास्थानानामप्यर्थात्मना नित्यमेव~। तत् प्रतिपादकशब्दानामेव शिक्षाद्यङ्गानामुपाङ्गानां च पौरुषेयत्वादनित्यत्वम्~। अत एवाहुः वार्तिकारः\textendash 

\begin{quote}
 
{\qt सत्यं नित्यैव मीमांसा न्यायपिण्डात्मिकेष्यते~।\\
संक्षेपविस्तरग्रन्थैः कृत्रिमैस्तु निबद्ध्यते~॥}
\end{quote}

\noindent इति~। निरुक्तवार्तिके सामान्येनाप्युक्त पद्मपादाचार्येण\textendash 

\begin{quote} 
{\qt विद्यास्थानानि नित्यानि तेषां ग्रन्थाः सकर्तृकाः~।}
\end{quote}

\newpage

\noindent इति~। विद्यास्थानशब्दनिरुक्त्या च बृहट्टीकायां विद्यानां नित्यत्वं प्रदर्शितम्\textendash 

\begin{quote}
{\qt स्थानान्येव हि विद्यानां नित्यानीत्यवधार्यताम्~।\\
  ग्रन्थोऽन्योन्यश्च तेनैव विद्यास्थानाभिर्धेयता\renewcommand{\thefootnote}{१}\footnote{यतः \textendash\ क.}~॥}
\end{quote}

\noindent इति~। तस्मात् इदमिदानीम् अस्मन्मुखविनिस्सृतमपि सदा सत्यमेव वेदाङ्गम्~। वेदपुरुषस्य चक्षुष्ट्वेन स्तुतमङ्गम्\textendash 

\begin{quote}
{\qt अत ऊर्ध्वं तु छन्दांसि शुक्लेषु नियतः पठेत्~।\\
वेदाङ्गानि रहस्यं च कृष्णपक्षेषु सम्पठेत्~॥}
\end{quote}

\noindent इति~। स्मर्यमाणे स्वकालेऽध्येयं श्रोतव्यं च द्विजातिभिः तदकरणाज्जायमानं प्रत्यवायं परिहर्तुं स्वाध्यायविधिनैव विहितत्वात् साङ्गवेदाध्ययनस्य~। अत एवाह गर्गोऽपि\textendash 

\begin{quote}
{\qt श्रूयतां स्वर्ग्यमायुष्यं धन्यं पुण्यं यशस्करम्~।\\
ज्ञानविज्ञानसम्पन्नं द्विजानां पावनं परम्~॥
	
कालज्ञानमिदं पुण्यं आद्यं हि ज्ञानमुत्तमम्~।\\
सिसृक्षुणा पुरा वेदानेतद् दृष्टं स्वयम्भुवा~॥

वेदाङ्गमाद्यं वेदानां क्रियाणां च प्रसाधकम्~।\\
ज्योतिर्ज्ञानं द्विजेन्द्राणामतो वेद्यं विदुर्बुधाः~॥
	
ज्योतिश्चक्रात्तु लोकस्य सर्वस्योक्तं शुभाशुभम्~।\\
ज्योतिर्ज्ञानं तु यो वेत्ति स वेत्ति परमां गतिम्~॥
	
तद्भावभाविनं युक्तं तं देवा ब्राह्मणं विदुः~।\\
तस्मात् पूर्वमधीयीत ज्योतिर्ज्ञानं द्विजोत्तमाः~॥
	
धर्मसूत्रं ततः पश्चात् यज्ञधर्मविधिक्रियाम्~।\\
तस्मात् पुण्यसमं वेदैर्यज्ञचक्षुः सनातनम्~॥
	
स्वर्ग्यमध्येयमव्यग्रैः ब्राह्मणैः संशितव्रतैः~॥}
\end{quote}

\newpage


\noindent इति~। तस्मात् एवम्भूतस्यास्य शास्त्रस्य यः प्रतिकञ्चुकं कुरुते तस्य सुकृतायुषोः प्रणाशः स्यादिति सम्प्रति दण्डः प्रयोगः क्रियते~। कथमस्य दण्डात्मकत्वम्\textendash 

\begin{quote}
{\qt पूर्वं धिग्दण्ड एवासीद् वाग्दण्डस्तदनन्तरम्~।\\
आसीदादानदण्डोऽपि वधदण्डोऽद्य वर्तते~॥}
\end{quote}

\noindent इति~। दण्डानां चतुर्विधत्वाद्वाग्दण्डाख्योऽयम्~। तत्रादानवधदण्डौ राजभिरेव कार्यौ~। धिग्वाग्दण्डस्त्वेव ब्राह्मणैः~। अत एवाह मनुः\textendash  
\begin{quote}
{\qt वाचि दण्डो ब्राह्मणानां क्षत्रियाणां भुजार्पितः~।}
\end{quote} 

\noindent इति~। ब्राह्मणमात्रस्यापि \textendash\ भयङ्कर एव वाग्दण्डः~। किमुत सर्वज्ञस्य भगवत आर्यभटस्य~। अतः क एव तस्मान्न बिभेति~। सर्वेऽपि बिभ्युरेवेति स्वप्रतिभटेषु दण्डप्रयोग एवायं स्वानुकूलेषु स्वशिष्येष्वाशिष्ट्वेन पर्यवत्स्यति~। अतोऽस्य मङ्गलात्मकत्वम्~। अतोऽस्य तात्पर्यमेव महाभास्करीये {\qt तपोभिराप्तम्} इत्यादिनाविष्क्रियते~। अत एव गोविन्दस्वामिनाप्येतत्पद्यमेवमवतारितम् \textendash\ {\qt अनेन पुनरार्यभटीयस्य
माहात्म्यं, तस्य च तदध्येतृश्रोतॄणां चाशिषो वक्ता दर्शयति} इति~। आचार्येणैवात्र कृता आशिषो दर्शयत्येव परं भास्करः, न पुनः स्वयमाशिष आशास्त इति दृशिग्रहणात् सूचितम्~। तेनास्याशीःपरत्वेन योजनायैव तद्दर्शनमपीति च द्योत्यते~। इतरथा मङ्गलपरत्वेनाव्याख्येयत्वप्रसङ्गात्~। अत
एव वक्तृग्रहणं कृतम्~। वक्ता आर्यभटीयस्य प्रवक्तेत्यर्थः~। व्याख्यातृभिः मन्दमतीनां मूलग्रन्थे जायमानान्यथानुपपत्तिः निवारणीयैव~। शिष्यशब्देनाध्येतारः, श्रोतारश्च गृहीताः~। यतः कैश्चिद् अध्ययनपरैरेव भाव्यम्~। तत् पाठशुद्ध्यर्थं, व्याख्यानकुशलानां व्याख्यानमात्रपरत्वाद्
इति~। शब्दावधारणकुशलैर्ग्रन्थोऽवधार्यः, व्याख्यानकुशलैर्व्याख्यातव्यश्च इति~। द्विविधैरेव पुरुषैः शास्त्रं कार्त्स्न्येन रक्ष्यते इति~।
\newpage

\noindent तदर्थमेवाह व्यासोऽपि\textendash 
\begin{quote}
{\qt द्विविधं संहिताज्ञानं दीपयन्ति मनीषिणः~।\\
व्याख्याने कुशलाः केचित् केचिद्ग्रन्थस्य धारणे~॥}
\end{quote}

\noindent इति~। स्मर्यते चाध्ययनस्यापि फलम् \textendash\ {\qt इदं शास्त्रमधीयान इत्यादौ}~। तस्मात् ये पुनरिदमेव शास्त्रमधीयानाः व्याचक्षाणाश्च रक्षन्ति, तेषां प्रतिकञ्चुकमकुर्वतां सुकृतायुषी वर्धेताम्~। ते सुकृतिनो दीर्घायुषश्च भूयासुरित्यस्मिन्नर्थेऽस्य पर्यवसानमित्यर्थः~। एदमिदमस्माभिर्यथामति व्याख्यातम्~॥~५०~॥
	
\begin{center}
{\qt नमः स्वयंभुवे तस्मै यत्प्रसादादिदं कृतम्~।\\
	नमो भगवते तस्मै श्रीमदार्यभटाय च~॥
	
शिष्यं तत्त्वेन विचार्याचार्यभटसूत्रभाष्यमिदम्~।\\
यदि स न्यायाल्लिप्सेदस्मै दातव्यमेव शङ्कर ते~॥}
\end{center} 
	
\begin{center}	
 ॥~इति गोलपादव्याख्यानं समाप्तम्~॥ 
\end{center}

\begin{center}
\textbf{इति श्रीमदार्यभटीयभाष्यं\\
	गार्ग्यकेरलनीलकण्ठसुत्वविरचितं समाप्तम्~॥} 
\end{center}
 

 \begin{center} 
{\qt प्राचां खेचरतां स्वतः प्रवहतश्चैव प्रतीचां भ्रमो\\
वृत्ते यत्र यतस्तदुन्नतिनती स्तो देशतो ज्योतिषाम्~।\\
विक्षेपोऽस्य समन्तता गतियशान्मध्याद्भगोलस्प स\\
ज्योतिश्चक्रकुमध्यमर्कशशिनोरूर्ध्वं परेषां ततः~॥
 
 देशे महति काले वा स्फुटार्थं यस्य दर्शनम्~।\\
सूत्रकाराय तस्मै स्तान्नमो भाष्यकृतेऽपि च~॥} 
\end{center} 

	\begin{center}
	  
	\textbf{॥~आर्यभटीयं समाप्तम्~॥}
	
\vspace{2mm}
\textbf{॥~शुभं भूयात्~॥}
	\end{center}

\afterpage{\fancyhead[CE] {}}
\afterpage{\fancyhead[CO]{}}
\afterpage{\fancyhead[LE,RO]{\thepage}}
\cfoot{}

\newpage

\pagestyle{empty}
\begin{center}
\textbf{\underline {स्मृतग्रन्थानुक्रमणी~।}}
\end{center}
%\begin{singlespace}
\begin{longtable}{clcc}
\textbf{पृष्ठम्} & \textbf{वाक्यानि} & \textbf{ग्रन्थनाम}  & \textbf{कर्तृनाम}\\
 
 \vspace{-3mm}
&&&\\
 \endhead
३ & लोक्यते यत्र शब्दार्थौ\textendash & \ldots \ldots &  \ldots \ldots \\
४ & छायाग्रहः सषट्भोऽर्क\textendash & \ldots \ldots & मुञ्जालः\\
१३ & दोःफलस्य यथा कर्णः & सिद्धान्तदीपिका &  \ldots \ldots \\
,, & विक्षेपघ्नान्त्यकर्णाप्ता\textendash & सूर्यसिद्धान्तः &  \ldots \ldots \\
,, & विक्षेपस्तृज्यया &,,&  \ldots \ldots\\
,, & कर्णेन ह्रियते लब्धो\textendash &  \ldots \ldots & भास्करः\\
१४ & सिद्धान्तभेदेऽप्ययन\textendash &  \ldots \ldots &वराहमिहिरः\\
१७ & अर्धास्तमयात् सन्ध्या\textendash & \ldots \ldots & ,,\\
१८ & ज्ञातभोगग्रहं वृत्तं\textendash & सिद्धान्तदर्पणम्& नीलकण्ठः\\
२१ & पञ्चमहाभूतमयः\textendash &  \ldots \ldots & वराहमिहिरः\\
२२ & विरोधेगुणव\textendash & \ldots \ldots & औपनिषदिकाः\\
२४ & नक्षत्रग्रहपञ्जरः\textendash &  \ldots \ldots \\
२५ & ग्रहनक्षत्रभ्र\textendash &  \ldots \ldots & विष्णुनन्दनः\\
,, & यत्र तोयनिधि\textendash &  \ldots \ldots & भास्करः\\
,, & पञ्चाशत् कोटि\textendash & ब्रह्मसिद्धान्तः &  \ldots \ldots \\
२६ & आदर्शोदरसोदर\textendash &  \ldots \ldots & श्रीपतिः\\
,, & चन्द्रादित्यग्रहण\textendash &  \ldots \ldots & ,,\\
,, & धर्ता धरित्र्या यदि\textendash &  \ldots \ldots & ,,\\
२७ & उपपत्त्या हीनोक्तिः\textendash & गोलदीपिका &  \ldots \ldots \\
२९ & लङ्कोत्तरतोऽवन्ती\textendash &  \ldots \ldots & जैष्णवः\\
,, & मिथुनान्ते च कुवृत्ताद्\textendash &  \ldots \ldots & वराहमिहिरः\\
३० & गणितं सर्वं यथासिद्धं\textendash & अजितव्याख्या &  \ldots \ldots \\
३६ & क्षेत्राण्येवमजादीनां\textendash & सूर्यसिद्धान्तः &  \ldots \ldots \\
३८ & एकाकी योजयेत्\textendash &   \ldots \ldots &  \ldots \ldots \\
५० & षण्मुहूर्ताश्चराचराः\textendash & महाभाष्यम् &  \ldots \ldots \\
\newpage
%%%%%%%%%%%%%%%%%%%%%%%%%%%%%%%%%%%%%%%%%%%%%%%%%%%%%%%%%%%%%%%

५१ &भोदया भगणैः\textendash & सूर्यसिद्धान्तः &  \ldots \ldots \\
५९& क्षुण्णां परमया क्रान्त्या\textendash &   \ldots \ldots & भास्करः\\
६२ & इष्टच्छायाग्र\textendash & ब्रह्मसिद्धान्तः &  \ldots \ldots \\
,,& इष्टच्छायाविषुवतो\textendash & सूर्यसिद्धान्तः &  \ldots \ldots \\
६३& भेदात् पूर्वापरक्रान्त्या\textendash &  \ldots \ldots &  \ldots \ldots \\
६४ & फलज्योनामुदक्\textendash &  \ldots \ldots & भास्करः\\
६५ & लम्बाक्षजीवे\textendash & सूर्यसिद्धान्तः &  \ldots \ldots \\
६६ & छायाविधानसम्प्राप्तः & ,, &  \ldots \ldots\\
६७ & भानोर्भुजामभिहतां & \ldots \ldots & भास्करः\\
,, & त्रिज्योदक् चरजा\textendash & सूर्यसिद्धान्तः& \ldots \ldots \\
६८ & व्यस्तत्रैराशिक\textendash & \ldots \ldots & \ldots \ldots \\
६९ & द्विघ्नो यः पलभावर्गः\textendash & \ldots \ldots & \ldots \ldots \\
७१ & आदित्यलग्नविवर\textendash & \ldots \ldots & भास्करः\\
,, & रविकक्ष्या मध्यज्या\textendash & \ldots \ldots & ,,\\
,,&  मध्यलग्नसमे\textendash &सूर्यसिद्धान्तः & \ldots \ldots \\
,, & मध्याह्नाद्वा नतप्राणा\textendash & तन्त्रसङ्ग्रहः & नीलकण्ठसुत्वः\\
७५ & लग्नं त्रिभोनं\textendash &\ldots \ldots &माधवः\\
७८ & लग्नं पर्वविनाडीनां\textendash & सूर्यसिद्धान्तः & \ldots \ldots \\
७९ & मध्यलग्नस्य\textendash & \ldots \ldots& परमेश्वराचार्यः\\
८१ & दृक्क्षेपस्य कृतिं\textendash & \ldots \ldots & ,,\\
८७ & ज्ययोरासन्नयो\textendash & \ldots \ldots & माधवः\\
८८ & तिथिघ्नान् चरसंस्कारान्\textendash & मानसगणितम् & \ldots \ldots\\
९२ & सत्रिभग्रहज\textendash & \ldots \ldots &मयः\\
,, & विक्षेपदृक्क्षेप\textendash & \ldots \ldots & माधवः\\
९३ & विक्षेपो भिन्नतुल्यांश\textendash & मानसगणितम् & \ldots \ldots\\
९५ & युतिमध्यनताभ्यस्ता\textendash &,,& \ldots \ldots \\
,,& तेजसां गोलकः\textendash & \ldots \ldots & \ldots \ldots \\
\pagebreak
%%%%%%%%%%%%%%%%%%%%%%%%%%%%%%%%%%%%%%%%%%%%%%%%%%%%%%%%%%

९७ & छादको भास्करस्येन्दु\textendash & सूर्यसिद्धान्तः & \ldots \ldots\\
१०२ & परमक्षेपकोटिघ्नः\textendash &\ldots \ldots & नीलकण्ठः\\
१०४ & समपूर्वापरं वृत्तं\textendash & \ldots \ldots & \ldots \ldots\\
१०८ & परमापक्रमकोट्या\textendash &  \ldots \ldots & माधवः\\
११६ & परमक्षेपकोटिघ्नं\textendash & \ldots \ldots & \ldots \ldots \\
११८ & अन्त्यद्युज्याहता\textendash & \ldots \ldots & नीलकण्ठः\\
१२० & विक्षेपव्योमधृत्यंश\textendash & मानसगणितम् & \ldots \ldots \\
,, & कृतलम्बनचन्द्रार्क\textendash & तन्त्रसङ्ग्रहः & \ldots \ldots \\
१२७ & लिप्तात्रयमपि\textendash & सूर्यसिद्धान्तः  & \ldots \ldots\\
१२९ & स्मृतिः प्रत्यक्षमैतिह्य\textendash & श्रुतिः & \ldots \ldots \\
,, & तस्माद् ब्राह्मणो\textendash &,,& \ldots \ldots \\
,, & अहोरात्रस्य यः\textendash & स्मृतिः & \ldots \ldots \\
१३० & लोक्येते यत्र\textendash & \ldots \ldots & \ldots \ldots \\
,, & सूर्य एकाकी चरति\textendash & श्रुतिः & \ldots \ldots \\
,, & चन्द्रमा जायते\textendash & ,, & \ldots \ldots \\
,, & नवो नवो भवति\textendash & ,, & \ldots \ldots \\
,, & आदित्यो वा एषः\textendash & ,, & \ldots \ldots \\
,, & कलानामपि नैव\textendash & स्मृतिः & \ldots \ldots \\
१३१ & द्युगणाद् दिगुणा\textendash & मानसगणितम् & \ldots \ldots \\
,,& ग्रीवासमां भगण\textendash & \ldots \ldots & भास्करः\\
१३२ & गर्गादिस्मृतनक्षत्र\textendash & \ldots \ldots & ,,\\
१४९ & चन्द्र बाणकरा\textendash & \ldots \ldots & \ldots \ldots \\
१५० & वाग्भावो\textendash & \ldots \ldots & \ldots \ldots \\
,, & शाके नखाब्धिरहिते\textendash & \ldots \ldots & लल्लः\\
१५१ & कृतशरवसुमित\textendash & \ldots \ldots & मुञ्जालः\\
,, & एवं दृग्गणितं\textendash & \ldots \ldots & परमेश्वरः\\
\pagebreak
%%%%%%%%%%%%%%%%%%%%%%%%%%%%%%%%%%%%%%%%%%%%%%%%%%%%%%%%%%%%
१५४ & सर्वं दृश्यन्ते विहगाः\textendash & \ldots \ldots & परमेश्वरः\\
१५५ & द्युगणो सप्तनागाग्नि\textendash & सिद्धान्तदीपिका & परमेश्वरः (भार्गवः)\\
१५९ & अविदितदलक्ष्मा\textendash  & \ldots \ldots & श्रीपतिः\\
१६० & तथैव ज्ञानभूयस्त्वाद्\textendash & \ldots \ldots & गर्गः\\
,, & मुहूर्ततिथिनक्षत्र\textendash & \ldots \ldots & ,,\\
,, & अविदित्वैव यच्छास्त्रं\textendash & \ldots \ldots &,,\\
१६१ & सत्यं नित्यैव मीमांसा\textendash & \ldots \ldots & वार्तिककाराः\\
,, & विद्यास्थानानि नित्यानि\textendash & \ldots \ldots & पद्मपादाचार्यः\\
१६२ & स्थानान्येव हि\textendash & बृहट्टीका &  \ldots \ldots \\
,, & श्रूयतां स्वर्ग्यमायुष्यं\textendash & \ldots \ldots & गर्गः\\
१६३ & पूर्वं धिग्दण्ड एवासीत्\textendash & \ldots \ldots & \ldots \ldots\\
,,& वाचि दण्डो ब्राह्मणानां\textendash &\ldots \ldots & मनुः\\
,, & अनेन पुनरार्यभटीयस्य\textendash &\ldots \ldots & गोविन्दस्वामी\\
१६४ & द्विविधं संहिताज्ञानं\textendash & \ldots \ldots & व्यासः
\end{longtable}
\begin{center}
 \hspace{1cm}   \rule{0.1\linewidth}{0.5pt}
\end{center}
%\end{singlespace}


 \end{document}