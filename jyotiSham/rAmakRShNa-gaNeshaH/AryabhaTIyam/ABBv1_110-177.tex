\documentclass[11pt, openany]{book}
\usepackage[text={4.65in,7.45in}, centering, includefoot]{geometry}
\usepackage[table, x11names]{xcolor}
\usepackage{fontspec,realscripts}
\usepackage{polyglossia}
\setdefaultlanguage{sanskrit}
\setotherlanguage{english}
\setmainfont[Scale=1]{Shobhika}
\newfontfamily\s[Script=Devanagari, Scale=0.9]{Shobhika}
\newfontfamily\regular{Linux Libertine O}
%\newfontfamily\regular[Scale=1]{Times New Roman}
%\newfontfamily\sanskritfont[Script=Devanagari]{Shobhika}
%\newfontfamily\englishfont[Language=English, Script=Latin]{Linux Libertine O}
\newfontfamily\en[Language=English, Script=Latin]{Linux Libertine O}
\newfontfamily\ab[Script=Devanagari, Color=purple]{Shobhika-Bold}
\newfontfamily\qt[Script=Devanagari, Scale=1, Color=violet]{Shobhika-Regular}
\newcommand{\devanagarinumeral}[1]{%
	\devanagaridigits{\number \csname c@#1\endcsname}} % for devanagari page numbers
%\usepackage[Devanagari, Latin]{ucharclasses}
%\setTransitionTo{Devanagari}{\s}
%\setTransitionFrom{Devanagari}{\regular}
\usepackage{enumerate}
%\pagestyle{plain}
\usepackage{fancyhdr}
\pagestyle{fancy}
\renewcommand{\headrulewidth}{0pt}
\usepackage{afterpage}
\usepackage{multirow}
\usepackage{amsmath}
\usepackage{amssymb}
\usepackage{graphicx}
\usepackage{longtable}
\usepackage{footnote}
\usepackage{vwcol}
\usepackage{perpage}
\MakePerPage{footnote}
\usepackage{xspace}
\usepackage{array}
\usepackage{emptypage}
\usepackage{hyperref}% Package for hyperlinks
\hypersetup{colorlinks,
	citecolor=black,
	filecolor=black,
	linkcolor=blue,
	urlcolor=black}
\XeTeXgenerateactualtext=1 % for searchable pdf
\begin{document}
\newpage

\noindent पदादितः प्रभृति तदर्धचापज्यातुल्यः~। तन्निमितं भुजाकर्णयोर्दलद्विगुणभावस्य सम्बन्धस्य तुल्यत्वम्~। अत्र पुनश्चतुरंशचतुर्गुणभावः कर्णभुजयोः सम्बन्धः~। अन्यत्र पुनर्मिथः पादत्रयत्र्यंशचतुष्कभावः~। तत्र कर्णापेक्षया पादत्रयमितो बाहुः तत्र बाहोस्त्र्यंशचतुष्कमितश्च कर्ण इति कर्णपादत्रयभावो बाहोः तत्त्र्यंशचतुष्कभावः कर्णस्येत्येवं रूप उभयोः परिमाणतः सम्बन्धः~। यथा शब्देन सहार्थस्य वाच्यभावः सम्बन्धः अर्थस्य शब्दवाच्यत्वात्, शन्दस्यार्थेन च स्वापेक्षया वाचकभाव इति परस्परं सम्बन्धो वाच्यवाचकभावः, एवमत्रापि पादत्रयभावस्त्र्यंशचतुष्कभावश्चेति द्वौ सम्बन्धौ~। सम्बन्धस्य च द्विनिष्ठत्वादुभयोरपि सम्बन्धयोस्तावेव सम्बन्धिनाविति परस्परसम्बन्धोऽयमेकीकृततया वाच्यवाचकभावः सबन्ध इतिवद् पादत्रयत्र्यंशचतुष्कभावः सम्बन्ध इति च वक्तव्यमेव एवमितरेतरावधिकयोः सम्बन्धयोरन्यतरावधिकयोश्च भेदादेव विपरीतकर्मणि,

\begin{quote}
{\qt अथ स्वांशाधिकोने तु लवाढ्योनो हरो हरः~।\\
अंशस्त्वविकृतस्तत्र विलोमे शेषमुक्तवत्~॥}
\end{quote}

\noindent इति {\qt यः क्षेपः सोऽपचयोऽपचयः क्षेपश्चे}त्यमुमंशं विवृण्वता आनुलोम्ये त्र्यंशसंयोजने प्रातिलोम्ये पुनश्चतुरंशवियोजनं कार्यम्,
आनुलोम्ये चतुरंशवियोजने पुनः प्रातिलोम्ये त्र्यंशसंयोजनं चेत्ययं विशेषः प्रदर्शितः~। एवं वृत्तभवयोः क्षेत्रयोर्व्यासार्धसमस्तज्याकर्णकयोस्त्रयाणामैककालिकः सम्बन्ध उभयोस्तुल्य एव~। यदैकदा वृत्तगतभुजाकोटिव्यासार्धकर्णानां त्रयाणां परस्परं सम्बन्धो यादृशः तदैव व्यासार्धशलाकास्पृष्टसमस्तज्याकर्णस्य तद्भुजाकोट्योः शरखण्डस्य\renewcommand{\thefootnote}{१}\footnote{शरखण्डस्य भुजाखण्डस्य \textendash\ ख. पाठः.} च त्रयाणां तादृश एव सम्बन्धः~। यदा पुनरन्यादृश एकक्षेत्रगतानां सम्बन्धः तदेतरक्षेत्रगतानामपि त्रयाणां मिथः सम्बन्धश्च तादृश एव~। अतस्तयोः क्षेत्रयोरेकस्मिंस्त्रिषु परिमितेष्वन्यस्मिन् ज्ञातपरिमाणेनैकेनान्ययोः परिमाणं त्रैराशिकेन ज्ञातुं शक्यमिति सर्वत्राप्ययं न्यायः समान एव क्षत्रत्रैराशिके\renewcommand{\thefootnote}{२}\footnote{केन}~। तथाप्यत्र
क्षेत्रपृथक्त्वान्मन्दमतीनां व्यामोहो जायेत~। तन्मा भूदिति विस्तरेणैतत् प्रतिपादितम्~। क्षेत्रैक्ये\renewcommand{\thefootnote}{३}\footnote{क्षेत्रे \textendash\ क. पाठः.} पुनः सुगमैव त्रैराशिकयुक्तिः~। तत्रावयवावयविभावेनैव केवलं क्षेत्रभेदः कल्प्यते~। यथा \textendash\ प्रतिमण्डलग(तो ? त)ग्रहोच्चनीचरेखाविप्रकर्षभुजाकोटिभ्यां व्यासार्धेन च प्रतिमण्डलमध्यगतोच्चनीचवृत्तेऽपि उच्चनीचरेखाग्रह-

\newpage

\noindent सूत्रविप्रकर्षभुजातत्कोट्योरानयनमिच्छात्मकेनोच्चनीचव्यासार्धेन क्रियते~। तद्वृत्तस्येच्छात्वे प्रतिमण्डलपरिणाहस्यैव प्रमाणत्वम्~। तदापि
प्रतिमण्डलगतभुजाकोटिज्ये एव फले~। इच्छाफले चोच्चनीचवृत्तगतभुजाकोटिज्ये, तत्रोच्चनीचवृत्तगत\renewcommand{\thefootnote}{१}\footnote{र्भ \textendash\ क. पाठः.}त्वात्~। प्रतिमण्डलस्यावयवित्वात् तद्वृत्तस्य उच्चनीचवृत्तान्तर्गतक्षेत्रस्य तु तदवयवत्वं चेत्यवयवावयविभावात् कर्णसूत्रस्यैकत्वाच्चोभयत्राप्युच्चनीचरेखायाश्चैकत्वात् तद्द्वयविवरात्मकभुजयोरप्यवयवावयविनोरिव मिथः सम्बन्धः सुगम एव~। य(था ? दा) वात्रैव व्यासार्धकर्णस्याष्टमी ज्या भुजा तदा तत्कर्णैकदेशस्य शरोनितस्य भुजा कियतीति त्रैराशिकमित्यत्रापि सुगमः सम्बन्धः~। एवं समस्तज्याकर्णस्य क्षेत्रस्यावयवभूतस्यावयविक्षेत्रमन्यदुत्पाद्यम्~। कथम्~। समस्तज्याव्यासार्धकर्णयोर्योग एव केन्द्रं कृत्वैतद्वृत्तसममन्यद्वृत्तमालिखेत्~। तथा सत्युभयोरपि वृत्तयोः साम्यात् कर्णश्च तुल्य एव~। समस्तज्यानुसार्येव तत्कर्णश्च~। तस्य पूर्वकर्णापेक्षया समतिरश्चीनत्वात् पूर्ववृत्तभुजाकोट्यपेक्षया तद्भुजाकोट्योरपि समतिरश्चीनत्वं स्यात्~। ततः पर्ववृत्ते भुजा दक्षिणोत्तरायता यदि तत्स्थनीया द्वितीयवृत्तभुजा पूर्वापरायता~। कोटिश्च दक्षिणोत्तरायता~। ते एवात्र फले कल्प्ये~। तद्गतं व्यासार्धं च प्रमाणम्~। समस्तज्या चेच्छा~। तद्गतकोटिभुजात्मकौ ज्याशर\renewcommand{\thefootnote}{२}\footnote{शर}खण्डावेवेच्छाफले~। तत्प्रदर्शनाय तत्रापि पूर्वापररेखां
दक्षिणोत्तररेखां च कुर्यात्~। ते च प्रतिक्षणं कार्ये~। यदा तद्वृत्तं कर्णशलाकाबद्धं ग्रहानुरूपं भ्रमति तद्ब(न्धा ? द्धा) समस्तज्यानुरूपिणी
व्यासार्धतुल्या शलाका च~। सा च तत्र तदा कर्णतया कल्प्या~। प्रथमवृत्तगतकर्णो यदा पूर्वपरायतः तदेतरकर्णो दक्षिणोत्तरायतः~। एवं सर्वदापि
पूर्वृत्तकर्णसमतिरश्चीनो द्वितीयवृत्तकर्णः~। ततः पूर्वा\renewcommand{\thefootnote}{३}\footnote{कर्णगतपूर्वा \textendash\ क. पाठः.}परसूत्रात् प्रथमवृत्तकर्णाग्रं भ्रमदुत्तरतो याव(द्)विप्रकृष्टम् अन्यकर्णदक्षिणाग्रमपि तद्गतदक्षिणोत्तररेखायाः प्राक् तावद्विप्रकृष्टं स्यादिति पूर्ववृत्त\renewcommand{\thefootnote}{४}\footnote{स्यादिति वृत्त \textendash\ ख. पाठः.}भुजज्यैव तत्रापि भुजाज्या~। एकस्या दक्षिणोत्तरायतत्वमितरस्याः पूर्वापरायतत्वमित्येव केवलं विशेषः~। तत्र च कर्णयोर्मिथो व्यत्यस्तदिक्कतैव हेतुः~। तस्मादुभयत्रापि तुल्याकारं क्षेत्रम्~। तत्रैवं त्रैराशिकवाचोयुक्तिः~। द्वितीयवृत्ते व्यासार्धकर्णस्येयती भुजा तदा तदंशभूतस्य समस्तज्याकर्णस्य कियती भुजेति दक्षिणोत्तरायता भुजा लभ्यते~। एवमपि न कश्चिद्विशेषः, यतः पूर्ववृत्ततुल्यैव तत्रापि भुजेति~। यदा पूर्ववृत्तेऽष्टमी

\newpage

\noindent ज्या दक्षिणोत्तरायता भुजा ईशपादस्था तदा द्वितीयवृत्तेऽपि भुजा अष्टमी ज्यैव~। सा तस्मिन्नग्निकोणप(दा ? द)स्था पूर्वापरायता च~। एवं प्रदर्शिते मन्दबुद्धेरपि प्रत्ययः स्यात्~। तस्मात् समस्तज्यया खण्डज्यानयनमनवद्यम्~। तत्र प्रथमया चापज्ययैव द्वितीयादिज्यानयनमप्येतेनैव त्रैराशिकेन प्रदर्शितम्~। चापमध्यभुजाकोट्यानयनद्वारं तत्र भुजाखण्डस्य भुजानुरूपत्वं च प्रदर्शितम्~। प्रदर्शितं च खण्डज्यान्तराणां अतएव भुजानुरूपत्वम्~। तत्परमेव चेदं सूत्रम्~। तच्च विस्पष्टं, तत्र प्राकृतवदेव निरुप्यते, न पुनरस्य गणितस्य त्रैराशिकमङ्गीकृत्य~। यथा लोके\renewcommand{\thefootnote}{१}\footnote{कैः \textendash\ क. पाठः.}
प्रस्थादिमात्रव्रीहीणां कुडुबादिमापकेनैतावन्तस्तण्डुलाः स्युरिति प्रतिव्रीहिप्रस्थं तण्डुलमाने ज्ञाते तत्क्षेत्रभवानां तथा नियतानां व्रीहीणां तण्डुलज्ञानं प्राकृतानामपि स्यात्~। एकैकस्य व्रीहिप्रस्थस्यैतावन्तस्तण्डुला इति ज्ञात्वा द्वितीयप्रस्थस्य च तावन्त इत्येवं खार्यादिषु यावन्तः प्रस्थाः सन्ति तत्फलस्यापि तावद्गुणनं कृत्वा तत्फलं ज्ञेयम्~। पुनः प्रस्थैकदेशस्यापि प्रस्थाद्यावानंशो व्रीहिभागः फलस्यापि
तावानंशस्तत्फलमिति च ज्ञायते~। ते पुनरेतत्त्रैराशिकमिति न जानन्तीत्येव विशेषः~। अतस्त्रैराशिकमप्रदर्श्यैव प्राकृतबुद्ध्यनुसारेणैव तदानयनं
प्रदर्श्यते~। तच्च विस्पष्टत्वायैव~। तथाहि \textendash\ {\qt प्रथमाच्चापज्यार्धाद् यैरूनं खण्डितं द्वितीयार्धम्} इत्युच्यमाने श्रोतॄणां मखिभख्याद्यन्तरेषु बुद्धिः प्रसरेत्~। तदा तेषां मध्ये कतिपयानामन्तराणामेकाद्येकचयत्वमप्येतच्छ्रवणात् प्रागपि कस्यचित् प्रतिभाति~। तत्प्रतिभानं तेषां कथं स्यादिति खण्डजीवाः पाठिताः~। खण्डजीवानां पुनर्महत्त्वा(त्ता ? त्तदे)न्तरमेव दुर्ज्ञानम्~। कुतः पुनस्तदन्तरान्तरे बुद्धिः प्रसरतीत्यन्तरान्तरेष्वपि बुद्धिप्रसरणा(त्वं ? र्थं) खण्डज्याः पठिताः~। तत् कस्य हेतोरिति जिज्ञासा चैकाद्येकचयत्वज्ञानानन्तरमेव स्यादिति तज्जिज्ञासूनामोष्ठपरिस्पन्देऽप्येतद्विषयोऽयं व्यापारः~। किं पुनस्तस्मिन् प्रदर्श्यमाने~। तद्विषयज्ञानं कुतो न स्यात्~। तथै\renewcommand{\thefootnote}{२}\footnote{तत्रैव \textendash\ ख. पाठः.}तावत्युच्यमान आकाङ्क्षा स्यात् प्रथमद्वितीयान्तरं किमर्थमनूदितमिति~। यच्छब्दश्रवणाद्विधेयाकाङ्क्षा जायेत, तस्यां जातायां तत्प्रथमज्यार्धांशैरित्येतावति श्रुते प्रथमज्यार्धस्य हारकत्वमपि प्रतीयेत यतस्तदंशत्वं\renewcommand{\thefootnote}{३}\footnote{तदंशं \textendash\ क. पाठः.} तद्धृतस्यैव स्यात्~। यतःकुतश्चित् प्रथमज्यार्धेन हृतं फलं यत्तत् प्रथमज्यार्धांशशब्देनोच्यत इति हार्यबुद्धिरपि सामान्येन स्यात्~। ततस्त-

\newpage

\noindent द्विशेषा\renewcommand{\thefootnote}{१}\footnote{द्विशेषाणां} अाकाङ्क्षा च स्यात्~। प्रथमज्यार्धांशैरिति बहुवचनेनेच्छानां बहुत्वमपि प्रतीयेत~। इतरथा तत्फलस्य समुदायतयैकत्वादेकवचनमेव प्रयुज्येतेति~। तस्मात् यस्याः कस्याश्चिदिच्छायाः प्रथमज्यार्धांशा ये तैः किमित्याकाङ्क्षा ऊनानीत्यनेन परिपूर्यते~। तत्तत्प्रथमज्यार्धांशैरूनानि इति तयोरन्वयप्रतीतिः, तैस्तै\renewcommand{\thefootnote}{२}\footnote{प्रतीतिः~। ततस्तै}रिति तच्छब्दश्रवणादनुदितस्यैवायं परामर्श इति च प्रतीतिः स्यात्~। तस्मात् प्रथमज्यार्धात् खण्डितं द्वितीयार्धं यैरूनं तावद्भिः प्रथमज्यार्धांशैः तावद्गुणितैः प्रथमज्यार्धांशैः~। तस्मादिच्छाराशौ यावन्ति
प्रथमज्यार्धानि भान्ति तेषां सर्वेषां फलं प्रत्येकं प्रथमद्वितीयखण्डज्यान्तरतुल्यं तावतां सर्वेषां स्यादिति चतुर्गुणनं कार्यम्~। तैरूनानि पुनः कानि
स्युरित्याकाङ्क्षायां शेषाणीत्युच्यते~। अत्र किमुपयुक्तं यस्येमानि शेषाणि~। प्रथमं च द्वितीयं च खण्डज्यार्धमुपयुक्तम्~। तस्मात् तृतीयादीनि शेषाणि
जायन्त इति चाध्याहर्तव्यम्~। कुत ऊनानीति तदवध्यपेक्षायामनन्तरप्रकृतमेव द्वितीयं तदवधित्वेन गृह्णाति~। तत्र का पुनरिच्छा इतीयमाकाङ्क्षा इच्छाप्रमाणयोः समानजातित्वेनैव पूर्येत~। प्रथमज्यायामिच्छायां प्रथमद्वितीययोः खण्डयोरन्तरं हि फलमिति प्रथमज्यैव प्रमाणत्वेनाङ्गीकृतेति तस्या हारकत्वोक्तेरेव ज्ञेयम्~। तत्र खण्डज्यान्तरात् प्रथमात् प्रभृति द्व्यधिकत्वात्\renewcommand{\thefootnote}{३}\footnote{त्वं \textendash\ क. पाठः.} तज्जिज्ञासा चिरोषिता बुद्धौ~। अतः प्रायेण पूर्वेच्छा(या)
द्वितीयेच्छया द्विगुणीकृतया भाव्यम्~। तस्माद्द्वितीयज्यैवात्रेच्छा स्यादिति युज्येत, द्वयोः पिण्डरूपत्वात्~। इच्छाप्रमाणयोः समानजातीयत्वमपि तथा सति स्यात्~। तस्माद्द्वितीयाद्याः पिण्डज्या एव द्वितीयाद्यन्तरानयने इच्छाराशयः स्युः~। द्वितीयतृतीयखण्डज्यान्तरं हि द्वितीयान्तरम्~। ततस्तदानयने सा प्रथमज्यया हर्तव्या~। तत् प्रायेण द्विकं स्यात्~। तदूना च भखिः फखिः सम्पद्येत~। एवं तत्तत्पिण्डज्यायाः प्रथमज्ययैव हृतं फलं प्रथमद्वितीयान्तरेणैकेन हतमनन्तरानीताच्छोध्यमित्यर्थः प्रतीयेत~। तस्मादिदं त्रैराशिकमेतस्मात् चापभागादल्पेषु महत्स्वपि चापभागेषु साधारणमेवेति ज्ञात्वा यत्र प्रथमाद्वितीययोः खण्डज्ययोरन्तरमितो महद्वाल्पं वा स्यात् तत्र फलगुणनमवश्यं कार्यमेवेति प्रथमद्वितीयान्तरेणात्र गुणनमुक्तम् उपपत्ति(र्ज्ञा ? ज्ञा)पनायेति बुद्ध्वापि तत्र परीक्षेच्छा स्यात् तत्संवादार्थात्~। फलसंवादे ह्यर्थनिश्चयः स्यात्~। तेनेतो

\newpage

\noindent द्विगुणेषु चापखण्डेषु चतुर्गुणेषु वा परीक्षापि क्रियेत~। चतुर्गुणत्वे पञ्चदशभागात्मकत्वाच्चापस्य प्रथमाद्वितीययोः खण्डज्ययोरन्तरं महदेव स्यात्~। ततस्तदन्तेरण राशिज्यां हत्वार्धशाशिज्यया ह्रियेत~। तत्र लब्धं फलं च द्वितीयखण्डज्यातः शोधयेत्~। तत्र शिष्टं तृतीयज्याखण्डः चतुर्विंशतिपक्षपठितद्वा(द ? विं)शा(ष्टम ? ष्टादश)पि(ख ?)ण्डगुणयोरन्तरतुल्यो जायेत~। तनैवमेवार्थ इति च निर्णयः स्यात्~। अतोऽल्पाक्षरमसन्दिग्धमित्युक्तमसन्दिग्धार्थत्वमपि स्यादेवास्य सूत्रस्य~। तर्कापेक्षा पुनर्दोषाय न भवति~। प्रमाणानुग्राहकस्तर्क इति हि न्यायाविदो वदन्ति~। अतो गुणायैव तर्कापेक्षायुक्तिपराणां वाक्यानाम्~। विस्पष्टमुच्यमाने हि युक्तिप्रतीतिर्न
स्यादनिरूप्यैवार्थज्ञानोत्पत्तेरित्य(य)मत्र गुणः~। तस्मादनेनैव त्रैराशिकेनेतरकर्मनिरपेक्षेणैव मख्यादयः सिद्ध्यन्ति~। निरपेक्षत्वायैव हि प्रथमाच्चापज्यार्धादिति चापतुल्याया एव प्रथमज्यायाः प्रमाणत्वेनोक्तिः~। इतरथा कतिपय(स्याः ?) जीवायाः प्रमाणत्वे तत्परिमाणानुक्तौ च तज्ज्ञानाय यत्नान्तरं कर्तव्यमिति चापज्यैव प्रमाणतया गृहीता~। इतरासां सर्वासामपि पर्यायेण प्रमाणत्वसम्भवेऽपि क्रियतश्चापस्य पुनर्ज्याचापतुल्येत्येतदपि कलादिचापस्य स्वज्यातुल्यत्वनिश्चयात् ततः प्रभृत्येवारभ्यताम्~। यद्वा मख्यादिषु दृष्टेन खण्डज्यान्तराणाभादितः
प्रभृत्येकाद्येकोत्तरगुणितत्वेन पिण्डज्याधर्मेणेतोऽधस्तात् तद्धर्मस्य प्रायिकत्वं क्रमेण हीयेतेति पदादौ सूक्ष्मतरं स्यादिति तदपि शरसंवर्गन्यायेनैव
सिद्धम्~। तदर्थं राशेस्त्रिंशांशादिषु यं(कि ? क)ञ्चिच्चापं परिगृह्याविशेषकर्मणा तच्छरो ज्ञेयः~। कथं पुनरत्राविशेषकर्म~। तस्याः स्वमतिकल्पितायाश्चापज्याया वर्गं व्यासेनैव हृत्वाप्तं तच्छरत्वेन गृहीत्वा तदूनेन् व्यासेन पुनरपि तमेव वर्गं हृत्वा लब्धेन शरेण मुहुरेतत् कुर्याद्यावदविशेषः~। तदा शरः सूक्ष्म एव~। तच्छरवर्गं ज्यावर्गे युक्त्वा मूलीकृतं तत् चापार्ध\renewcommand{\thefootnote}{१}\footnote{र्धं \textendash\ क. पाठः.} समस्तज्या स्यात्~। तस्याः\renewcommand{\thefootnote}{२}\footnote{तस्य \textendash\ ख. पाठः.} पूर्वकल्पितचापज्यार्धादाधिक्यमधिकं चेत् पुनस्तदर्धपरम्परास्वेवमेव ज्या आनेयाः~। जायमानस्य पूर्वदलसाम्यं यदा स्यात् तावदन्तं\renewcommand{\thefootnote}{३}\footnote{न्त} पुनरप्येवमानेयम्~। एवं प्रथम\renewcommand{\thefootnote}{४}\footnote{नेयम्~। प्रथम \textendash\ क. पाठः.}चापज्या लभ्या~। यदि ततः प्रागेव परितोषः स्यात् तर्हि तच्चापमेव चापज्याङ्गीकार्येति तदपि ज्ञेयम्~। अत एतत्सूत्रविवरणमेव माधवेनाप्यनेन वसन्ततिलकेन कृतम्~। तत्र द्वितीयवाक्यस्य कोऽर्थः

\newpage

\noindent कीदृशं वा कर्म कीद्दशी वा\renewcommand{\thefootnote}{१}\footnote{च \textendash\ ख. पाठः.} युक्तिरिति चेदयमर्थः \textendash\ यद्वेत्यनेन पूर्वोक्तस्य वक्ष्यमाणस्य चैकविषयत्वमुच्यते~। अतएव विकल्पः~। तस्मात् तत्रैव पक्षान्तरमेतत्~। स्वलम्बकृतिभेदपदीकृते द्वे ते एव द्वे जीवे~। ये स्वलम्बकृतिभेदपदीकृते ते अप्य\renewcommand{\thefootnote}{२}\footnote{कृते अप्य \textendash\ क. पाठः.}न्योन्ययोगविरहानुगुणे स्याताम्~। स्वस्याः स्वलम्बकस्य च वर्गान्तरमूलं योगवियोगार्हम्~। तस्मादेकमेव लम्बवर्गं\renewcommand{\thefootnote}{३}\footnote{वर्ग \textendash\ ख. पाठः.} द्वयोरपि वर्गाभ्यां विशोध्य मूलीकृत्य योजयेद्वियोजयेद्वा~। तन्मूलद्वयमपि पूर्वानीतद्वयमेव स्यादिति कथं तत्तुल्यत्वं निश्चीयते~। एकस्याधि(क्य ? क)त्वेऽन्यस्याल्पत्वेऽपि योगस्य तुल्यत्वमेव स्यात्~। सत्यम्~। योगस्तुल्य\renewcommand{\thefootnote}{४}\footnote{योगतुल्य} एव स्यात्~। अन्तरे निमीलितदृष्टिः खलु भवान्, यतोऽन्तरे महानेव भेदः~। महतोऽल्पत्वेनैव शिष्टस्य फलस्याल्पत्वं स्यात्~। त्याज्यस्य महत्त्वे च शिष्टस्याल्पत्वमेव स्यात्~। महतो महत्वेऽल्पस्य\renewcommand{\thefootnote}{५}\footnote{महतोऽल्पस्य} न्यूनत्वे चोभयमप्याधिक्यहेतुः स्यात्~। तस्मात् तत्तुल्यमेव फलद्वयं मूलद्वयञ्च~। तस्मात् तत्र यदानीयते तदेव प्रकारान्तरेणानीयते इति न केवलं योगस्यैव साम्यं वियोगस्य वा~। तस्माद्योगवियोगयोग्यत्वापादकं कर्मापि द्विविधं विद्यत इत्यायातम्~। कथं पुनरत्र गणितयोर्द्वयोः फलसाम्यमिति सङ्ख्या युक्त्वापि ज्ञातुं शक्यं क्षेत्रान्तर कल्पनयापि च~। यत्र प्रथ(म ? मं) सङ्ख्याद्वारा निरूपय्ते~। तत्र या पदसन्धितः प्रवृत्ता महती ज्या पारकल्प्यते या\renewcommand{\thefootnote}{६}\footnote{न \textendash\ क. पाठः.} पुनर्द्वितीया च तत्कर्णस्पर्शिन्यर्धज्या च तस्याः शरोनेन भागेन महतीं ज्यां हत्वा व्यासार्धेन हृत्वाप्तं फलं कथं पुनर्वर्गमूलाभ्यामानेतुं शक्यमिति ह्यत्र निरूपणीयम्~। अत्र ज्ययोरुभयोरप्यल्पत्वमेव प्रथमे कर्मण्यपि स्यात् व्यासार्धाद्धारकादल्पेनैव
गुण्यत इति~। अत्रापि द्वयोर्वर्गाभ्यां यत्किञ्चिद्विशोध्य मूलीकृतावुभयोरप्यल्पत्वमेव स्यात्~। कथं पुनः साम्यमपि~। किं पुनस्तयोरल्पत्वे निमित्तं गुणकारस्य हारकादूनेन भागेन गुणनस्याकृतत्वात्~। तस्मादितरेतरगुणितत्रिज्याप्तभागो वा स्वस्याः स्वस्यास्त्याज्यः~। शिष्टस्यापि योगविरहयोग्यत्वं
स्यादेवेति चेदानीं निर्णीतम्~। तत्र हारकान्न्यूनो गुणकारोऽपि यदि हारकवर्गाद्यस्यकस्यचिद्वर्ग विशोध्य मूलीकृते स्यात् तर्ह्येवेदं युज्येत~। तस्मात्
कस्य वर्गविशोधनेन तदानेतुं शक्यमिति प्रथमं निरूप्यम्~। किं पुनस्तज्ज्ञाने फलम्~। तत्स्थानीयं किमत्रापीति ज्ञानम्~। तस्मिन् ज्ञाते तद्युक्तिरेव
ज्ञाता

\newpage 

\noindent स्यात्~। हारकान्न्यूनो गुणकारः, इतरशरोनं व्यासार्धं\renewcommand{\thefootnote}{१}\footnote{व्यासार्धं शरोनं व्यासार्धं \textendash\ क. पाठः.} कोटिरेवेति च पूर्वमेवोक्तम्\renewcommand{\thefootnote}{२}\footnote{पूर्वोक्तम्\textendash\ ख. पाठः.}~। तस्मात् तस्य द्वितीयज्याकोटित्वाद्व्यसार्धावर्गात् द्वितीयज्यावर्गं विशोध्य मूलीकृत्यापि स्वा कोटिर्ज्ञेया, न शरत्यागेनैव~। तस्मादत्रापि तत्स्थानीयं किमित्यत्राप्येवं त्रैराशिकम्~। यदि व्यासार्धकर्णस्य भुजा द्वितीया ज्या तदा प्रथमज्याकर्णस्य कियती ज्येत्यानीतं यत् तस्येह वर्गः प्रथमज्याया विशोध्यः~। यथा व्यासार्धस्य द्वितीयज्याकोट्या सह सम्बन्धः तथा प्रथमज्याया एवमानीतफलेनापि सम्बन्धः~। तस्मात् तस्य वर्गः प्रशमज्यावर्गाद्विशोध्यः~। तत्र द्वितीयज्या फलम्~। व्यासार्धमेव प्रमाणम्~। प्रथमज्या चेच्छा~। तस्मात् प्रथमज्यां द्वितीयज्यया निहत्यात्र व्यासार्धेन ह्रियते~। तस्माद्भुजज्ययोरेव संवर्गो व्यासा(र्ध)हृत इह लम्बत्वेन विवक्षित इति प्रथमज्याह्रासनिरूपणन्यायेन सिद्धम्~। द्वितीयज्याह्रासकारणगवेषणयापि ज्ययोः संवर्गो व्यासार्धहृत एव लम्ब इति सेत्स्यति~। तत्र द्वितीयज्या प्रथमज्यायाः कोट्या प्रथमकर्मणि हन्यते~। तद्ह्रासवशाद्धि तद्धानिः~। द्वितीयज्याकोट्याः कथं पुनर्हारकाद्धानिः स्यादिति निरूपणे तद्वर्गात् प्रथमज्यावर्गविशोधनमूलीकरणवशात् व्यासार्धात् तद्धानिः~। तस्मादत्र प्रथमभुजास्थानीयं किमित्यत्राप्येवं त्रैराशिकं \textendash\ व्यासार्धस्य
प्रथमज्या भुजा अत्राल्पीयस्या द्वितीयज्यायाः कियतीति~। अत्र द्वितीयज्यैवेच्छा प्रथमज्या फलमितीच्छाफलयोर्व्यत्यास एव केवलम्~। तत्रापि प्रमाणं व्यासार्धमेवेतीच्छाफलयोर्धातस्तुल्य एवोभयत्र~। हारकश्च तुल्यः~। अतएव फलस्यापि तुल्यत्वमुक्तम्~। तस्मात् तत्रापि पूर्वोक्तमेव लम्बतया ग्राह्यमिति लम्बानयनगणितवासना~। कथं पुनरत्र क्षेत्रकल्पना~। सापि योगविषया तावत् प्रदर्श्यते~। तत्र तयोरुभयोश्चापयोरेकीकृता ज्येह साध्या~। सैवात्र भूमिः~। या पुनर्द्वितीयज्या तत्साधनभूता सैव सव्यभुजा~। या च पुनः साधनभूतयोर्महती सैव दक्षिणभुजा~। कथं सा न दक्षिणभुजा स्यात्~। सा हि तत्कर्णभूतव्यासार्धाग्रात् प्रभृति प्रवृत्ता समदक्षिणतो गता व्यासार्धपर्यवसिता भूम्या उपर्येव स्यात्~। तस्याः सर्वेऽवयवा अपि भुवस्तुल्यविप्रकर्षाः, न कश्चिद्भागो भुवं स्पृशति~। सा कथं भुजेति युज्यते~। उच्यते~। तत्तुल्यां शलाकां भूमिदक्षिणाग्रस्पृष्टमूलां कर्णद्वितीयज्यायोगस्पृष्टाग्रां विन्यसेत्~। सा दक्षिणभुजा~।

\newpage

\noindent तस्मात् सा तत्तुल्येति सैवेत्युक्तम्~। तामेव ज्यारूपेण बद्धामाकृष्य भूमिसव्यभुजाग्रयोर्बध्यमाना त्र्यश्रबाहुत्वं प्राप्नुयात्~। सा पूर्वं
चतुरश्रभुजाभूत्~। कथं पुनर्भूमिदक्षिणाग्रस्य सव्यभुजोर्ध्वाग्रस्य चान्तरालं तज्ज्यातुल्यमित्यस्योत्तरं पूर्वमेव दत्तं, तदाबाधायाः
प्रथमवाक्योक्तत्रैराशिकेनानीतायास्तद्भुजायाश्च तुल्यत्वप्रतिपादनात्~। दक्षिणाबाधावर्गस्य लम्बवर्गस्य च योगमूलं प्रथमज्यातुल्यमिति च तत एव सिद्वम्~। इति चतुर्थपादोपपत्तिः~॥~१२~॥\\

खण्डज्यानयनत्रैराशिके इच्छाप्रमाणयोस्तत्फलयोरपि क्षेत्रद्वयगतत्वादस्पष्टत्वादेकस्मिन्नेव क्षेत्र\renewcommand{\thefootnote}{१}\footnote{क्षेत्रभुजयोः \textendash\ क. पाठः.} उभ(ज\renewcommand{\thefootnote}{२}\footnote{जा}?)योरंशयोः परस्परं
त्रैराशिकेनानयनं प्रदर्शयितुं प्रथमं तावत् क्षेत्रसिद्धिं दर्शयति\textendash 

\begin{quote}
{\ab वृत्तं भ्रमेण साध्यं त्रिभुजं च चतुर्भुजं च कर्णाभ्याम्~।\\
साध्या जलेन समभूरध ऊर्ध्वं लम्बकेनैव~॥~१३~॥}
\end{quote}

इति~। वक्ष्यमाणस्यापि शङ्कुच्छायाकर्णक्षेत्रविशेषस्य वृत्तगतायतचतुरश्रान्तर्भूतत्र्यश्रात्मकत्वाद्वृत्तादीनां कर्तृदोषजनितासमीचीनतापरिहारोपायोऽनेन प्रदर्श्यते~। तत्र तावद्वृत्तं भ्रमेण साध्यमित्येतत् पूर्वमेव व्याख्यातम्~। त्रिभुजं च चतुर्भुजं च कर्णाभ्यां साध्यम्~। त्र्यश्रस्य भुजाकोट्योरेवोद्देशकेनोद्दिष्टयोः सत्योस्तत् क्षेत्रं शक्यं कर्तुम्~। तत्रापि कर्णं स्वयमानीय ज्ञात्वा तदग्रान्तरालस्य तत्तुल्यत्वेनैव भुजाकोट्योर्मिथः
समतिर्यक्त्वनिर्णयः~। कर्णेनान्तरालमप्रमा\renewcommand{\thefootnote}{३}\footnote{दा \textendash\ ख. पाठः.}य यथाकथञ्चिल्लिखितयोर्भुजाकोट्योरुभयाग्रप्रापिण्या रेखाया लेखने भुजाकोटिकर्णत्र्यश्रत्वमेव हीयेतेति कर्णेनैव तत्साधुत्वनिर्णयः कार्यः~। तथैवायतचतुरश्रस्य समचतुरश्रस्यापि कर्णाभ्यामेव विस्तारायामयोर्भुजाकोटिरूपत्वनिर्णयः~। तत्र कर्णयोस्तुल्यत्वादेव
तन्निर्णयः कार्यः, भुजप्रतिभुजयोर्मिथस्तुल्यत्वात्~। तयोश्चातुल्यत्वे विषमचतुरश्रत्वापत्तेः~। समभूः पुनर्जलेनैव साध्या~। परिलेखनादौ भुवः समीकरणमवश्यं कार्यम्~। विशेषतस्तु छायाकर्मणि जीवादिपरिलेखने पुनः कुतश्चिद्भागात् प्रभृति क्रमेण प्रावण्येन दोषः स्यात्~। निम्नप्रतिपूरणं च
तत्राप्यवश्यं कार्यम्~। छायाकर्मणि प्रवणभागाभिमुख्यां छायायां तस्या आधिक्यमुन्नताभिमुख्यां च ह्रस्वत्वमिति महान् दोष इति तत्र जलैनैव समीकरणनिर्णयः~। तन्निर्णयप्रकारो गोविन्दस्वामिना भाष्ये प्रदर्शितः~। यथा \textendash\ चक्षुस्सूत्रसमी-

\newpage

\noindent कृते धरातले निर्वाते (ति ? त्रि) काष्ठोप(र्य) (? र्यु)त्पूत\renewcommand{\thefootnote}{१}\footnote{र्यद्भुत \textendash\ क. पाठः.}जलपूर्णं घटं निधायाधः छिद्रं कुर्यात् यथा तदुदकमेकधारं प्रस्रवति~। तत्प्रस्रुतोदकवृत्तभावेन धरातलसमत्वावगतिरिति~। नन्वत्र न समीकरणमुक्तम्~। समीकृतस्य धरातलस्य समत्वनिर्णय एवोपायोऽयं प्रदर्शितः~। {\qt साध्या जलेन समभूरि}ति ह्याचार्येण तत्साधनमप्युक्तम्~। तत् कथम्~। यावतो भूतलस्य समीकरणं कार्यं ततो बहिः समन्तात् खात्वा तत्र जलमासिच्य तन्मध्यगतां स्थलीं तज्जलसमतया क्रकचादिरूपेण द्राघीयसा जलादुपरिगतं भागं विदार्यापनीय पुनः ऋजूकृतसूत्रेणापि कृत्स्नस्य तत्तलस्य साम्यमुत्याद्य\renewcommand{\thefootnote}{२}\footnote{साम्यमापाद्य \textendash\ ख. पाठः.} पुनरुक्तप्रकारेण परीक्षणमपि कार्यमिति भावः~। अध ऊर्ध्वं लम्बकेनैव अध ऊर्ध्वं यत् क्षेत्रं तल्लम्बकेनैव साध्यम्~। तिस्र एव हि जगति सर्वत्रापि दिशः~। श्रूयन्ते च तास्तिस्र एव श्रुतौ~। तासां तिर्यगधऊर्ध्वसंज्ञानां विभागोऽपि माधवेन प्रदर्शितः~।

\begin{quote}
{\qt तिस्रो दिशो जगति तिर्यगुपर्यधस्ता\\
	स्वाद्या ह्युपाधिभिरनेकविधेह भाति~।\\
तत्राध इत्यनुगते\renewcommand{\thefootnote}{३}\footnote{ज}र्जगदण्डमध्ये\\
यत्र स्थितिं क्षितिरुपैति निराश्रयैव~॥}
\end{quote}

\noindent इति~।

\begin{quote}
{\qt अस्त्यन्तोऽधोदिशः कश्चिदादिरूर्ध्वादेशस्तथा~।\\
पूर्वापरदिशोर्नोभावुभावुदगवाग्दिशोः~॥}
\end{quote}

\noindent इति~। तत्र तिर्यग्दिशः प्राच्यादयोऽष्टौ भेदा औपाधिका एव~। तिरश्चीनात्मक(त)यैकविधत्वमेव स्वतः~। ज्योतिर्भ्रमणनिमित्त एव तद्भेदः न स्वा\renewcommand{\thefootnote}{४}\footnote{तद्भेदः स्वा \textendash\ क. पाठः.}भाविकः~। तत्र तिसृषु मध्ये या दिगध इत्याख्यायते जगदण्डस्य ब्रह्याण्डकटाहस्य सम (स्ता ? न्ता)न्मध्ये~। आपे(क्ष ? क्षि)की हि दिक्~। सा कथं क्वचिदेव प्रदेशेऽवस्थाप्या इत्येतत्परिहारोऽस्त्यन्तोऽधोदिशः कश्चिदित्यादावुत्तरपद्ये दृश्यः~। तस्यान्त एवात्र जगदण्डमध्येऽवस्थाप्यते~। आदिरूर्ध्वदिशस्तथा ऊर्ध्वदिशः कश्चिदादिरस्ति~। स च तथा जगदण्डमध्य एव~। यद्यपि शनैश्चरकक्ष्यापेक्षया गुर्वादिकक्ष्याणामधोगतत्वमपि स्याद्भूम्यपेक्षया चोर्ध्वदिग्गतत्वं, तथापि न जगदण्डमध्यस्य कञ्चित् प्रदेशमपेक्ष्याप्यूर्ध्वत्वम्~। जगदण्डमध्यात् समन्ततो ये प्रदेशास्तेभ्यः सर्वेभ्योऽप्यध

\newpage

\noindent एव जगदण्डमध्यमिति जगदण्डमध्यमेवाधोदिशः पर्यवसानम्~। ततः समन्तात् प्रदेशाः सर्वेऽपि ऊर्ध्वदिग्गता एव~। इत्यूर्ध्वदिशोऽपि आदिरपि तत्रैव~। पूर्वापरदिशोः पुनरेवं न कश्चिदादिरन्तश्च विद्येते~। उदगवाग्दिशोरुभौ स्तः आद्यन्तौ भवत एव~। तत्र मेरावुदग्दिशोऽन्तः, तत्रैवादिश्च
दक्षिणदि(शा ? शः)~। यः पुनर(सु ? न्य)भागे विनिर्गतो मेरुभागः तत्रैव दक्षिणदिशोऽन्तः उत्तरदिश आदिश्चेति~। कुतः पुनर्जगदण्डमध्यं तत्\renewcommand{\thefootnote}{१}\footnote{तस्मात्} समन्ततः प्रदेशानां सर्वेषामध एव स्यादित्यत आह \textendash\ {\qt अनुगते}रिति~। गुरुद्रव्याणां तत्रैवानुगतेः~। गुरुद्रव्यपतनं ह्यधोदिशो लक्षणम्~। यत्र च क्षितिरपि निराश्रयैव स्थितिमुपैति तत्र हि भूमिर्निराधारैव तिष्ठति~। तस्या निराधारत्वेऽपि सर्वाधोगतत्वमेव हेतुः~। यतो जग\renewcommand{\thefootnote}{२}\footnote{यतो न जग}दण्डमध्य एव सर्वेषां
द्रव्याणां पिपतिषा ततो भूम्यवयवाश्च सर्वेऽप्यहमहमिकया तत्र पतन्तः परस्परं प्रतिबद्धत्वात् तामेव गोलाकारां पिण्डीकुर्वन्ति~।
मण्डलार्धान्तरितप्रदेशयोर्द्बयोः द्वयोर्मल्लयोरिव परस्परं विजिगीषतो(?)र्गुरुत्वसाम्यादेव परस्परं प्रतिबन्धः स्यादिति\renewcommand{\thefootnote}{३}\footnote{स्यात् \textendash\ क. पाठः.} समन्ततो गौरवसाम्ये गोलाकारत्वमेव स्यात्~। श्रूयते च गोलाकारत्वं दिवः पृथिव्याश्च ज्योतिषां च~। तस्मात् परिमण्डलैव पृथिवी निराधारा च~। सूर्य\renewcommand{\thefootnote}{४}\footnote{तथा च सूर्य}सिद्धान्तेऽप्युक्तं\textendash

\begin{quote}
{\qt मध्ये समन्तादण्डस्य भूगोलो व्योम्नि तिष्ठति~।\\
बिभ्राणः परमां शक्तिं ब्रह्यणो धारणात्मिकाम्~॥}
\end{quote}

\noindent इति~। एवं सर्वेषामध एव भूः~। अत एव च कालक्रियापादे वक्ष्यति\textendash

\begin{quote}
{\qt भानामधः शनैश्चरसुरगुरुभौमार्कशुक्रबुधचन्द्राः~।\\
एषामधश्च भूमिर्मेधीभूता खमध्यस्था~॥}
\end{quote}

\noindent इति~। तस्मात् सर्वेषां भूरेवाधः खगानां भूतानां च स्थावरजङ्गमानाम्~। अत एव भूपृष्ठे समन्ततः प्राणिनिवासश्च सम्भवति~। तत्सम्भ\renewcommand{\thefootnote}{५}\footnote{भा \textendash\ ख. पाठः.}वश्च गोलपादेन वक्ष्यते\textendash

\begin{quote}
{\qt यद्वत्कदम्बपुष्पग्रन्थिः प्रचितः समन्ततः कुसुमैः~।\\
तद्वद्धि सर्वसत्त्वैर्जलजैः स्थलजैश्च भूगोलः~॥}
\end{quote}

\noindent इति~। भूमावङ्गुलस्याष्टभागोऽपि न (भूमौ ?) प्राणिरहित इति व्यासेनाप्युक्तम्~। अदृश्या अणीयांसः प्राणिनस्तर्कगम्या बहवः सन्तीति भीमसेन-

\newpage

\noindent वाक्येनापि व्यासेन हरवादे प्रदर्शितः~। तस्मात् प्रतिद्रष्टृभिन्ने एवाधऊर्ध्वदिशौ~। तथापि भूप्रदेशेषु सर्वत्रापि गुरुद्रव्य\renewcommand{\thefootnote}{१}\footnote{या \textendash\ ख. पाठः.}ब(न्धे ? द्धे)न
सूत्रेणावलम्बितेनाधऊर्ध्वत्वं निर्णेयम्~॥~१३~॥\\
\begin{sloppypar} 
\indent एवं समीकृतभूतले स्थापितस्य शङ्कोः ऋजोः ज्योतिःसन्निधानाज्जायमानायाश्च्छायायाः\renewcommand{\thefootnote}{२}\footnote{जायमानायाः} परिमाणेन शङ्कुपरिमाणेन चरतस्य रश्मिमतो वस्तुनश्चर्जुतया विप्रकर्षे ज्ञाते तयोस्तिरश्चीनविवरमधऊर्ध्वदिगनुसारिविवरं च ज्ञातुं यः क्षेत्रविशेषः कल्पनीयस्तं दर्श(यत ? यितु)माह\textendash
\end{sloppypar} 

\begin{quote}
{\ab शङ्कोः प्रमाणवर्गं छायावर्गेण संयुतं कृत्वा~।\\
यत्तस्य वर्गमूलं विष्कम्भार्धं स्ववृत्तस्य~॥~१४~॥}
\end{quote}

इति~। शङ्कोः प्रमाणवर्गं न पुनः शङ्कुवर्गम्~। येनाङ्गुलादिना मापकेन शङ्कुर्मीयते शङ्कोस्तदङ्गुलादिसङ्ख्यावर्गं छायावर्गेण छायाप्रमाणवर्गेण संयुतं कृत्वा शङ्कुतुल्यचतुर्बाहुकं यच्चतुरश्रं क्षेत्रं छायातुल्यचतुर्बाहुकं च यत् ते उभे सम्यग्युते चतुरश्रात्मनैकीकृते कृत्वा तस्य वर्गमूलं यल्लभ्यते
तत् स्ववृत्तस्य विष्कम्भार्धम्~। यद्वृत्तगते भुजाकोटिज्ये शङ्कुच्छाये तद्धि तयोः स्ववृत्तं तस्य विष्कम्भार्धं तत्~। अत्र स्वशब्देन यद्वृत्तं
व्यावर्त्यते तच्चैवं रूपम्~। तत्रेदं वृत्तं छायाग्रनाभिकं शङ्कुमस्तकस्पृष्टपरिधिकमधऊर्ध्वायितम्\renewcommand{\thefootnote}{३}\footnote{ऊर्ध्वायितम्~। अन्य} एवमन्यदप्येतच्छायाग्रनाभिकम् अधऊर्ध्वायितं तज्जयोतिरूर्ध्वाग्रस्पृष्टपरिधिकं कल्पनीयम्~। तद्गतेनैकेन ज्ञातेनैतद्गतैस्त्रिभिरपि ज्ञातैः तद्गतेतरज्ञानोपायभूतगणितकर्मोक्तिप्रदर्शनायेदं क्षेत्रद्वयं कल्प्यते~॥~१४~॥\\

एवं क्षेत्रद्वये प्रदर्शिते खण्डज्या\renewcommand{\thefootnote}{४}\footnote{ज्यान \textendash\ क. पाठः.}न्तरानयनसूत्रदर्शितत्रैराशिकन्यायोऽत्रापि स्वयमेवातिदेश्यः शिष्यैरिति तत्कर्म न प्रदर्श्यते~। तथाप्यादित्यच्छायानयने यो विशेषः तत्स्फुरणं स्यादिति तदर्थं प्रदीपच्छायानयनच्छलेन तदानयनमुत्तरमूत्रेण प्रदर्शयति\textendash

\begin{quote}
{\ab शङ्कुगुणं शङ्कुभुजाविवरं शङ्कुभुजयोर्विशेषहृतम्~।\\
यल्लब्धं सा छाया ज्ञेया शङ्कोः स्वमूलाद्धि~॥~१५~॥}
\end{quote}

इति~। भुजाशब्देन दीपस्तम्भ उक्तः~। तस्य च शङ्कोश्च यदन्तरालक्षेत्रं तिर्यगायतं तच्छङ्कुगुणं शङ्कुभुजयोर्विशेषेण शङ्कुमस्तकस्य
ज्योतिषश्चोत्सेधान्तरेण हृतं कृत्वा यल्लब्धं सा हि शङ्कोः स्वमूलात् प्रवृत्ता छाया प्रसिद्धा~। आदित्यस्य पुनर्महाशङ्कुभूतलसमतिरश्चीनात् प्रदेशात् न प्रवृत्तेति दीपस्तम्भाद्विशेषः~। कुतस्तर्हि त(स्य ? स्याः) प्रवृत्तिः~। भूपार्श्वसमतिरश्चीनात्

\newpage

\noindent क्षितिजप्रदेशादेव प्रवृत्ता~। ततस्तत्र शङ्कुभुजयोर्विशेषः कलात्मकस्य भूव्यासार्धस्य महाशङ्कोश्च विशेष एव~। न पुनर्द्वादशाङ्गुलशङ्कोः
महाशङ्कोश्च विशेषः~। तदप्येतस्मिन्नुच्यमाने स्फुरतीति भावः~॥~१५~॥\\

तन्न्यायेन दीपस्तम्भच्छायाग्रविवरस्तम्भोत्सेधतत्कर्णानामन्यतमेन ज्ञातेनेतरानयनमेवं सिध्यतीति तेषां त्रयाणामप्यानयनायोत्तरकत्रमारभ्यते\textendash

\begin{quote}
{\ab छायागुणितं छायाग्रविवरमूनेन भाजिता कोटी~।\\
शङ्कुगुणा कोटी सा छायाभक्ता भुजा भवति~॥~१६~॥

यश्चैव भुजावर्गः कोटीवर्गश्च कर्णवर्गः सः~।}
\end{quote}

इति~। आर्यार्धैस्त्रिभिः क्रमात् कोटि\renewcommand{\thefootnote}{१}\footnote{त्रिभिः कोटि}भुजाकर्णानामानयनं प्रदर्श्यते~। प्रदीपादे रश्मिगोचरे समीकृते भूतले एकं शङ्कुं स्थापयित्वा
तच्छायामार्गेण सूत्रं नीत्वा तच्छायाग्रात् दूरत एव तन्मार्गेणेतरमपि तत्समानं शङ्कुं स्थापयेत्~। तत्र पूर्वस्य छायाल्पा द्विर्तायस्य च महती~। तच्छायाद्वयं शङ्कुमापके(णे ? नै)व प्रमाय क्वचिद्विन्यस्य द्वितीयं शङ्कुमुद्धृत्य पूर्वशङ्कोः छायाग्रात् प्रभृति द्वितीयशङ्कुच्छायाग्रान्तं भूभागं समीकृतमपि तेनैव मापकेन प्रमाय तच्छायाग्रविवरं प्रथमं प्रथमशङ्कुच्छायया गुणितं कृत्वा तस्य यो भागो द्वितीयच्छायाया ऊनः तेन~। प्रथमशङ्कुच्छायां
द्वितीयशङ्कुच्छायायास्त्यक्त्वा शिष्टेनेत्यर्थः~। तत्र यल्लब्धं सा कोटिः प्रथमच्छायाग्रदीपस्तम्भयोर्विवरावगाहिनी~। एवमेव द्वितीयशङ्कुच्छायाग्रात् प्रभृत्यपि स्तम्भान्तानेया~। एवं कोट्यानयनमुक्तम्~। सा कोटी शङ्कुगुणा स्वच्छायया भक्ता भुजा दीपोत्सेधश्च\renewcommand{\thefootnote}{२}\footnote{जीवोत्सेधश्च} भवति~। कः पुनरस्य कर्ण इत्यत आह \textendash\ {\qt यश्चैव भुजावर्गः कोटीवर्गश्च कर्णवर्गः स} इति~। नन्विदं पुनरुक्तं शङ्कोः प्रमाणवर्गमित्यादिना कर्णवर्गस्योक्तत्वादिति चेत्~। नैष दोषः~। तत्सूत्रं स्ववृत्तविष्कम्भार्धपरम्, एतत्तु भुजाकोटिकर्णक्षेत्रेषु या सा\renewcommand{\thefootnote}{३}\footnote{क्षेत्रेषु सा}मान्यात्मिका युक्तिस्तत्प्रदर्शनपरमिति पौनरुक्त्याभावः~। का पुनरत्र युक्तिः~। तत्प्रदर्शनाय पूर्वमुद्धृतं शङ्कुं तत्रैव स्थापयित्वा तच्छायाग्राद्बिन्दोः प्रभृति तन्मस्तकप्रापि दीपज्वालाग्रान्तं कर्णसूत्रं कल्पयेत्~। एवं प्रथमच्छायाग्रात् प्रभृति च तत्कर्णसूत्रं, दीपस्तम्भ\renewcommand{\thefootnote}{४}\footnote{सूत्रं, स्तम्भ \textendash\ क. पाठः.}मूलात् प्रभृति च तत्तच्छायाग्रान्तं, दीपस्तम्भश्चैक इत्येते चत्वारो राशयः~। छायाग्रविवर सङ्ख्यश्च पञ्चमो राशिः~। एवमस्मिन्नन्तर्गर्भक्षेत्रे पञ्च राशयः कल्प्याः~। एवं

\newpage 

\noindent पुनरेतादृशं क्षेत्रं स्थापितशङ्कुतुल्यभुजकं भूतले लिखेत्~। कथम्~। स्थापितशङ्कुतुल्यामूर्ध्वायतां रेखां लिखित्वा द्वितीयशङ्कुच्छायातुल्यां रेखां शङ्कुमूलात् प्रभृति समतिरश्चीनां लिखेत्~। तस्यां शङ्कुमूलात् प्रथमच्छायातुल्ये प्रदेशे बिन्दुं कृत्वा तस्माद्द्वितीयच्छायाग्राच्च शङ्कुमस्तकप्रापिण्यौ रेखे
कुर्यात्~। तदप्यन्तर्गर्भं पूर्वलिखितक्षेत्रतुल्याकारम्~। इदं पुनरल्पम् अन्यन्महदित्येव केवलं विशेषः~। अत्रापि शङ्कुश्च छाये च त\renewcommand{\thefootnote}{१}\footnote{शङ्कुच्छायैव त \textendash\ क. पाठः.}त्कर्णौ च छायाग्रविवरं चेति षड्राशयः सन्ति~। अत्र पुनस्ते सर्वेऽपि विदिताः~। शङ्कुश्च छाये च तत्तुल्यतया विदिताः~। छाययोर्विश्ले\renewcommand{\thefootnote}{२}\footnote{शे \textendash\ क. पाठः.}षं कृत्वा तद्विवरं च छाययोर्वर्गयोः शङ्कुवर्गं क्षिप्त्वा मूलीकृत्य तत्कर्णौ च ज्ञेयौ~। एवं विदितैरेतैर्महाक्षेत्रगताश्च वेद्याः~। तत्र छायाग्रविवरं प्रमाय विदितम्~। अन्ये पुनरतिदूरत्वाद्दीपस्य प्रमाय ज्ञातुं न शक्याः~। तस्माद्गणितेनैव ते पञ्चापि वेदितव्याः~। एकेनैव छायाग्रविवरेण विदितेना(ल्पा ?)ल्पक्षेत्रे सर्वेषां विदितत्वात्~। तद्वदेवान्यत्रापि मिथः परिमाणसम्बन्धः~। यथा महाक्षेत्रे छायाग्रविवरस्याल्पक्षेत्रे छायाविवरस्योनशब्दोक्तस्य(च) मिथः सम्बन्धः,
एवमेवान्येषामप्यल्पक्षेत्रगतानां महाक्षेत्रतानां च पञ्चानां मिथः सम्बन्धः~। तस्मात् तयोः क्षेत्रयोः कृत्स्नयोरपि तथैव परिमाण(त)स्सम्बन्धः~। यथा पुनर्महाक्षेत्रगतस्य छायाग्रविवरस्य तद्गतैरन्यैः पञ्चभिः सम्बन्धः एवमल्पक्षेत्रेऽप्यूनस्येतरैः पञ्चभिः सम्बन्धः~। यथा पुनरल्पक्षेत्रगतानां षण्णां द्वयोर्द्वयोः, एवं तत्स्थानगतयोर्महाक्षेत्रगतयोरपि द्वयोर्द्वयोः~। एवमल्पक्षेत्रगतैः षड्भिर्विदितैरन्येष्वेकेनैव च विदितेन तद्गता इतरे पञ्चापि त्रैराशिकेनैव ज्ञेयाः~। तत्र न कस्यचिदपि वर्गमूलापेक्षा~। तत्रैवं त्रैराशिकपञ्चकं \textendash\ यद्यल्पक्षेत्रगतेनोनेनाल्पा कोटिरल्पच्छाया लभ्यते तदा महाक्षेत्रगतेन छायाग्रविवरेणाल्पा कोटी कियतीति, ए(कं ? वं) यद्यल्पक्षेत्रगतेनोनशब्दोक्तेनैव महती कोटिर्द्वितीय\renewcommand{\thefootnote}{३}\footnote{कोटी द्वितीय \textendash\ ख. पाठः.}शङ्कुच्छायातुल्या लभ्यते तदान्यस्मिन् क्षेत्रे छायाग्रविवरेण कियती लभ्या, तथाल्पक्षेत्रे ऊनेन शङ्कतुल्यो भुजो लब्धः अन्यक्षेत्रे तत्स्थानीयेन छायाग्रविवरेण कियान् भुजो लभ्यः, एवमल्पक्षेत्रगतेनोनेनाल्पच्छायाशङ्कुवर्गयोगतुल्योऽल्पः कर्णो लभ्यः तदा छायाग्रविवरेणाल्पः कर्णः कियान् लभ्य इति, तथैवोनेन द्वितीयच्छायाशङ्कुवर्गयोगमूलतुल्यः कर्णो महाँल्लभ्यः तदा छायाविवरेण महान् कर्णः कियान् लभ्य इत्येते सर्वेऽपि त्रैराशिकेनैव ज्ञातुं शक्याः~।

\newpage

\noindent कर्णौ पुनस्तत्तद्भुजाकोटिवर्गयोगमूलेन च वेद्यावित्यवसरप्राप्ता भुजाकोटिकर्णेषु ज्ञातयोर्द्वयोर्द्वयोरितरज्ञानार्थं क्रियमाणस्य
वर्गयोगमूलीकरणस्य युक्तिरत्रैव प्रदर्श्येति तत्प्रदर्शनपरतैवास्य सूत्रस्य~। कर्णयोस्त्रै\renewcommand{\thefootnote}{१}\footnote{योगस्त्रै \textendash\ क. पाठः.}राशिकेनैव सिद्धत्वात् न तत्परत्वमिति सिद्धम्~। कथं पुनरनयोर्महाक्षेत्राल्पक्षेत्रयोः परस्परतुल्याकारत्वं निर्णीतं, येन निर्णीतेन तयोर्लिङ्गलिङ्गिभावो निर्णीयते~। (अ)तुल्याकारत्वं हि सुगममेवानयोः~। यन्महाक्षेत्रे तदन्तर्गर्भितमल्पं त्र्यश्रं तस्य च महतश्च तावन्नानाकारत्वमेव, यतस्तत्र विस्तृतिभेदाद्भेदः~। तुल्या ह्युभयोर्भुजा, कोटी तु भिद्येते~। अत एव कर्णौ च~। एकस्य तुल्यत्वेऽन्ययोर्भेदाद्धि आकारभेदः स्यात्~। अत एव च द्वयोः क्षेत्रगतानां परस्परं त्रैराशिकयोग्यत्वाभावश्च~। तस्मादत्रोभयोः क्षेत्रयोरप्यन्तस्त्र्यश्रस्य बहिस्त्र्यश्रस्य च परस्परमाकारभेदः सिद्धः~। एतयोः क्षेत्रयोरुभयोरप्यन्तस्त्र्यश्रयोः परस्परं तुल्याकारत्वमस्त्येव, यतो महाक्षेत्रगतस्याल्पत्र्यश्रस्यैकदेश एवाल्पक्षेत्रगतान्तस्त्र्यश्रम्~। महाक्षेत्रस्याल्पकर्णे यत्र स्वशङ्क्वग्रं स्पृशति तदधोगतं तच्छङ्कुच्छायाभुजाकोटिकं तत्कर्णैकदेशकर्णं तस्य कृत्स्नस्य त्र्यश्रस्यांश एव~। तदूर्ध्वगतं क्षेत्रमन्योंऽशः~। तयोरुभयोरप्याकारसाम्यं स्यात्~। नन्वेताभ्यामेव खण्डाभ्याम् अन्तस्त्र्यश्रं कृत्स्नं न व्याप्तम् अवशिष्टोऽपि कश्चिदंश आयतचतुरश्रशङ्कुभुजको\renewcommand{\thefootnote}{२}\footnote{भुजा \textendash\ क. पाठः.}ऽवशिष्यते~। तत्सहितस्यैव शङ्कुभुजकत्र्यश्रस्योर्ध्वगतखण्डापेक्षयेतरखण्डत्वम्~। शङ्कुमस्तकात् प्रभृति तिर्यक् छिद्यते~। तेन द्विधैव विभज्यते क्षेत्रम्~। तत उभयोर्न तुल्याकारत्वं, यत ऊर्ध्वगतं त्र्यशम् अन्यद्विषमचतुरश्रम्~। तयोराकारभे(द ? दोऽ)स्तु तेन किं त्र्यश्रयोरायातम्~। तथापि
शङ्कुभुजकस्येतरत्र्यश्रस्याधोगतस्योर्ध्वगतस्य च शङ्कुभुजाविवरकोटिकस्य किं तुल्याकारत्वं हीयते~। यः पुनस्तृतीयः खण्डः शङ्कुतुल्यबाहुः कोट्यन्तरतुल्यकोटिः (न ? स) पुनरत्र नोपयोगी~। अत्र त्र्यश्रस्य कृत्स्नस्य फलज्ञाने स खण्डो नोपेक्षणीयः~। इदानीं
तूपयोगाभावादुपेक्षणीय एव~। त्रैराशिकोपपत्तौ पुनः कर्णान्तर्भागे एकैककोटिकं तदुचितभुजमनेकं क्षेत्रं कल्पनीयम्~। अत एव क्रकचधारादृष्टान्तेन पूर्वं
तत् त्रैराशिकं प्रदर्शितम्~। तत्सर्व(म)त्रानुसन्धेयम्~। तस्मादत्रान्तस्त्र्यश्रगतयोः शङ्कुमस्तकपरिच्छिन्नयोस्तुल्याकारत्वमस्त्येव~। अत एव शङ्कुगुणं
शङ्कुभुजाविवरमित्येतन्त्रैराशिकं प्रवर्तते~। तत्रोर्ध्वगतक्षेत्रस्य शङ्कोरतिरिक्तो यो दीप-

\newpage

\noindent स्तम्भस्य खण्डः तस्य यावच्छेदस्य यावन्तोंऽशाः शङ्कुभुजाविवरं शङ्कोरपि तावच्छेदस्य तावन्तोंऽशाः छाया~। सर्वत्रैवं रूप एव सम्बन्धः~। एवम्भूत एव सम्बन्ध इहेच्छाप्रमाणयोस्त्रैराशिकोपयोगी~। ज्ञातनियमेन नियतपरिमाणयोः परिमाणविशिष्टतया ज्ञातेनान्यतेरणान्यतरस्यापि परिमाणविशिष्टस्यानुमानमेव हि त्रैराशिकम्~। एतदेव यो यथा नियत इत्यादिना व्याप्तिनिर्णयगतेन सिद्धान्तसंक्षेपपद्येन सूचितम्~। एवमेव महाक्षेत्रस्यान्तर्गर्भतया कल्पितस्य कृत्स्नस्य विप्रकृष्टच्छायान्तस्य क्षेत्रस्य विप्रकृष्टशङ्कुच्छायाकर्णक्षेत्रस्य तदंश\renewcommand{\thefootnote}{१}\footnote{ग \textendash\ क. पाठः.} भूतस्यापि मिथः सम्बन्धः~। एवम्भूते सम्बन्धेऽल्पयोर्महतोश्च निर्णीते एतत्त्रैराशिकं युज्यत इति सिद्धम्~। अतः सूत्रकारेण विवक्षिता त्रैराशिकवाचोयुक्तिरीदृशी \textendash\ ऊनेन छायाग्रविवरतुल्यो विप्रकर्षो लब्धः, कियान् पुनर्महत्या छायया लभ्य इति विप्रकृष्टच्छायाग्रान्तायाः कोट्या आनयने~। प्रथमच्छायाग्रान्ताया अप्येवमेव~। इयानेव विशेषः~। ऊनेन छायाग्रविवरं लब्धं सन्निकृष्टशङ्कुच्छायाह्रासेन कियदिति~। ऊनेनेति वदतोऽयमभिप्रायः \textendash\ दीपस्य योऽधोगतो भागस्तत्सन्निकर्षविप्रकर्षवशात् शङ्कोर्दीपोत्सेधादल्पोत्सेधस्य छायाया ह्रासवृद्धी~। दीपस्तम्भादधिकोत्सेधस्य पुनः शङ्कोर्भूमौ न पर्यवस्यति प्रासादादितल एव तत्पर्यवसानम्~। तत्रापि समीकृते तत्तले त्रैराशिकं प्रवर्तत एव~। रश्मीनां प्रसरतामृजुत्वादेव सर्वत्र छायाया ज्ञेयत्वम्~।
इह न केवलं छायाग्रविवरेणैव दीपसम्बन्धिभुजाकोटी ज्ञेये, शङ्कुविवरेणापि~। दीपाद्विप्रकृष्टस्य छायापरिमाणात् सन्निकृष्टशङ्कुच्छायापरिमाणे एतावता
न्यूने, अल्पच्छायाशङ्कुरितरस्मात् एतावता विप्रकृष्टः, महत्या छायया कृत्स्नयापि शून्यभूतायामन्यस्यां छायायां तच्छङ्कोरितरस्मात् कियान् विप्रकर्ष इति शङ्कुभुजाविवरे अपि द्वे त्रैराशिकेनैवानेये~। अतोऽत्र महाक्षेत्रविषयाणि सप्त त्रैराशिकानि स्युः~। छायाग्रविवरं शङ्कुविवरं च शङ्क्वङ्गुलमितेन मापकेनैव प्रमाय ज्ञेये~। ते द्वे विवरे इच्छाराशी~। एवं तत्र नव राशयः सन्ति~। तत्र छायाग्रविवरेणेच्छाभूतेन शङ्कोः परस्परं विवरेणेच्छात्मकेन
द्वे अपि शङ्कुभुजाविवरे छायाग्रभुजाविवरे (शङ्कुभुजाविवरे ?) च~। त्रैराशिकयोरिच्छाप्रमाणयोस्तुल्यत्वेऽपि फलभेदादेवेच्छाफलभेदः~। छायाग्रविवराद्यावता शङ्कुविवरस्य ह्रासः तावतैवांशेन छायाग्रभुजाविवरात् शङ्कुभुजा-

\newpage

\noindent विवरस्यापि ह्रास इत्येतदपि {\qt शङ्कुगुणं शङ्कुभुजाविवरमि}त्यनेनैव सिद्धम्~। तेन शङ्कुभुजाविवरविषयत्रैराशिकद्वयमनवद्यमिति~। अथ {\qt यश्चैव भुजावर्ग} इत्यस्योपपत्तिः प्रदर्श्यते~। एतद्विवरणभूतं हि,

\begin{quote}
{\qt सम्पर्कस्य हि वर्गाद्विशोधयेदेव वर्गसम्पर्कम्~।\\
यत्तस्य भवत्यर्धं विद्याद्गुणकारसंवर्गम्~॥

द्विकृतिगुणात् संवर्गाद्द्व्यन्तरवर्गेण संयुतान्मूलम्~।\\
अन्तरयुक्तं हीनं तद्गुणकारद्वयं दलितम्~॥}
\end{quote}

\noindent इति सूत्रद्वयम्~। कथम्~। सम्पर्कस्य भुजाकोट्योर्योगस्य यो वर्गः, यस्य क्षेत्रस्य भुजाकोटियोगतुल्याश्चत्वारो बाहवः तत्फलं हि तद्योगवर्गः, तस्मात् वर्गस\renewcommand{\thefootnote}{१}\footnote{वर्गद्वय स}म्पर्कं वर्गसम्पर्कफलात्मकं चतुरश्रं क्षेत्रम्~। कर्णतुल्यबाहुकं समचतुरश्रं क्षेत्रमित्येतद्यश्चैवेत्यादि सूत्रेणोक्तम्~। तत्क्षेत्रं पुनरत्र क्व वा कल्प्यते यत्सम्पर्कवर्गात् त्याज्यम्~। तत्प्रदर्शनाय भुजाकोटिसंयोगतुल्यबाहुक्षेत्रस्याग्नेयकोणादुत्तरतः प्राग्बाहौ जिज्ञासितकर्णभुजाकोटितुल्यान्तरे बिन्दुं कुर्यात्~। एवमीशकोणात् प्रभृत्युत्तरबाहौ तावदन्तरे, वायुकोणात् प्रभृति पश्चिमबाहावपि भुजान्तरे, नैर्ऋतादपि या(म्या ? म्य)बाहौ तावदन्तरे~। एवं चतुर्णां बाहूनामितरखण्डा जिज्ञासितकर्णकोटितुल्याः~। तत्राग्निकोणात् प्रत्यग्याम्यबाहौ कोटितुल्यान्तरे यो बिन्दुः तत आरभ्य प्राक्सूत्रगतबिन्द्वन्तां रेखां लिखेत्~। ततःप्रभृत्युत्तरबाहुबिन्द्वन्तां च~। ततः प्रत्यग्बिन्द्वग्रां पुनर्याम्यबिन्द्वन्तां च~। एवं कृते तत्क्षेत्रमध्ये जिज्ञासितकर्णतुल्यचतुर्बाहुकं क्षेत्रं यदुत्पन्नं तद्बहिर्भागोऽन्तर्भागश्च विभज्य प्र\renewcommand{\thefootnote}{२}\footnote{भागश्च प्र}दर्श्यत
एतद्विवरणसूत्राभ्याम्~। तत्र प्रथमेनान्तर्गतं चतुरश्रं क्षेत्रमितरेण चत्वारि त्र्यश्राणि परितः स्थितानि~। तान्येव भुजाकोटिकर्णक्षेत्राणि, यत्र कर्णो\renewcommand{\thefootnote}{३}\footnote{करणा} जिज्ञास्यते\renewcommand{\thefootnote}{४}\footnote{जिज्ञास्यन्ते \textendash\ क. पाठः.}~। अन्तर्गतस्य चतुरश्रस्य पुनरपि विभागः कार्यः~। कथम्~। यानि परितस्त्र्यश्राणि तान्यर्धायतानि~। तेषामायतचतुरश्रत्वाय यानीतरार्धान्यपेक्षितानि तानि यथा दृश्यानि स्युस्तथैव विभागः कार्यः~। तद्यथा \textendash\ प्राग्बाहुगतबिन्दोः प्रत्यगायता तत्कोटितुल्या रेखा लेख्या~। एवमुत्तरबिन्दोर्या(मा ? म्या)यता प्रत्यग्बिन्दोः

\newpage

\noindent पूर्वायता याम्यबिन्दोरूदगायता च~। तथा सति तेषामग्राणि संहितानि स्युः~। तेनैव तदन्तर्भागे भुजाकोट्यन्तरतुल्यबाहुकमपि समचतुरश्रं क्षेत्रं स्यात्~। इति क्षेत्रच्छेदः~।\\

एवं स्थिते सूत्रद्वयमारभ्यते सम्पर्कस्य हि वर्गादित्यादि~। तस्मात् सम्पर्कवर्गक्षेत्रान्मध्यगते जिज्ञासित\renewcommand{\thefootnote}{१}\footnote{ते}कर्णचतुर्बाहुके उद्धृत्यान्यत्र
लिखिते यच्छिष्टमुद्धृतं च तत्र शिष्टं यत् तत्र ह्यर्धायतचतुरश्राणि चत्वारि सन्ति, उद्धृते तानि च भुजाकोट्यन्तरचतुरश्रबाहुकं चेति पञ्च खण्डास्तदवयवभूताः स्युः~। यत्तस्य भवत्यर्धं, विशोधन उक्ते शिष्टस्य हि सर्वत्र\renewcommand{\thefootnote}{२}\footnote{सर्वस्य \textendash\ क. पाठः.} ग्रहणमिति शिष्टस्य यदर्धं, ते द्वे आयतचतुरश्रक्षेत्रे~। तस्मिन् संश्लिष्टकर्णे व्यत्यस्ताग्रे संयोजयेत्~। तदैकमायतचतुरश्रं पूर्णं स्यात्~। तस्य विस्तारो भुजातुल्यः~। आयामश्च कोटितुल्यः~। तत्तयोर्भुजाकोट्योः संवर्गं विद्यात्~। उभयोरपि गुणकारत्वोक्तिरभ्यास उभयोरपि गुणकारत्वसम्भवात्~। भुजाकोटी इह राशी विवक्षितौ~। ययोः सम्पर्कश्च सम्पर्क उक्तः~। एवमन्यस्मिन्नर्धेऽपि~। एवं त्र्यश्रक्षेत्रयोरेकीकृतयोरप्यायतचतुरश्रं यदुत्पद्यते तत्फलमपि भुजाकोट्यभ्यासतुल्यम्~। तद्विस्तारायामयोरपि तत्तुल्यत्वात्~। एवं योगवर्गाद्वर्गयोगे विशोधिते द्विघ्नघाततुल्यं शिष्टम्~। अन्येन कर्णबाहुकचतुरश्रेणाप्येवं भूतमायतचतुरश्रद्वयमुत्पादयितुं शक्यं, यतस्तत्रापि ताव(त्ये ? न्त्येव) चत्वारि त्र्यश्राणि सन्ति~। न पुनस्तान्येव तत्र सन्ति ततोऽतिरिक्तं भुजाकोट्यन्तरबाहुचतुर्भुजं तन्मध्यगतमपि विद्यते~। एवं तस्य पञ्चखण्डत्वमुक्तम्~। एतत्सर्वं द्वाभ्यां सूत्राभ्यां विस्पष्टीक्रियते~। तत्र पूर्वसूत्रं चतुःखण्डविषयम् उत्तरसूत्रं पञ्चखण्डविषयं कृत्स्नविषयं वा~। तथाहि \textendash\ द्विकृतिगुणात् संवर्गात् संवर्ग एकैकमायतचतुरश्रम्~। वर्गः समचतुरश्र इत्यनेनैव संवर्ग आयतचतुरश्र इत्यपि सिद्धम्~। अत उक्तं {\qt सर्वेषां क्षेत्राणां प्रसध्य पार्श्वे फलं तदभ्यास} इति~। तस्माद्द्विकृतिगुणात् द्वयोः कृत्या चतुःसङ्ख्यया गुणितात्~। पुनरपि किंविशिष्टात्~। द्व्यन्तरवर्गेण संयुताद्द्वयोर्भुजाकोट्योरन्तरस्य वर्गेण संयुताद्यन्मूलं लभ्यते तत् प्रतिराश्यैकस्मिन् भुजाकोट्यन्त(रः ? रं) योजयेत्
इतरस्माद्विशोधयेत्~। तद्द्वयं दलितं गुणकारद्वयं स्याद्भुजाकोटी स्याताम्~।

\newpage

\noindent संयोज्यार्धीकृतं तयोर्महती वियोज्य चार्धीकृतमल्पेति विभागः~। एतदुक्तं भवति~। ययो राश्योः संवर्गश्चतुर्भिर्गुणितस्तयोरेवान्तरवर्गेण
संयुक्तश्च तन्मूलं तयोरेव राश्योः सम्पर्कः स्यात्~। तस्माद्योगवर्गे घाताश्चत्वारः सन्ति अन्तरवर्गश्चैक इति पञ्चखण्डाः~। तत्र घातद्वयं विनान्यत् त्रयं
वर्गयोगतुल्यम्~। अत एवाह भास्करः\textendash

\begin{quote}
{\qt राश्योरन्तरवर्गेण द्विघ्ने घाते युते तयोः~।\\
वर्गयोगो भवेदेवम्}
\end{quote}

\noindent इति~। अनेनैव खण्डयित्वा वर्गीकरणमपि सिद्धम्~। क्व पुनरिदमुक्तम्~। न तावद्वर्गपरिकर्मणि~। इदं हि क्षेत्रव्यवहारादौ भुजाकोटिकर्णविषयगणितप्रदर्शनानन्तरम्~। एव ह्यत्र पाठक्रमः\textendash

\begin{quote}
{\qt इष्टाद्बाहोर्यत्}\renewcommand{\thefootnote}{१}\footnote{र्यस्या \textendash\ ख. पाठः.} {\qt स्यात्~। तत्समतिर्यग्दिशीतरो बाहुः~।}\renewcommand{\thefootnote}{*}\footnote{"इष्टो बाहुर्यः स्यात् तत्स्पर्धिन्यां दिशीतरो बाहुः" इति मुद्रितलीलावतीपाठः}
\end{quote}

\noindent इत्यस्यानन्तरमेतत् पठ्यते~। अत\renewcommand{\thefootnote}{२}\footnote{पठ्यते स}एव भुजाकोटिकर्णयुक्तिप्रदर्शनपरत्वमप्यस्य सूचितम्~। तत्र घातक्षेत्रं कर्णानुसारेण छित्त्वान्तरवर्गक्षेत्रं
चतुरश्रात्मकं क्वचित् कृत्वा तस्मिंश्चत्वार्यर्धायतचतुरश्रक्षेत्राणि संयोजयेत्~। कथम्~। तस्यैकतमबाहुना द्राघीयसो बाहोरल्पबाहुना योगात् प्रभृति तावत्प्रदेशं सश्लिष्टं कृत्वा पुनरन्यस्य त्र्यश्रस्य भुजाकोटियोगप्रदेशं तत्कोणे कृत्वा संश्लिष्टं कुर्यात्~। तथा सत्यन्तरचतुरश्रातिरिक्तेन खण्डेनास्य द्वितीयबाहुः कृत्स्नः संश्लिष्टः स्यात्~। तस्य चतुरश्रस्य द्वितीयं पार्श्वमस्य\renewcommand{\thefootnote}{३}\footnote{मध्य \textendash\ क. पाठः.} भुजाकोटियोगात् प्रभृति तत्तुल्येन कोट्यवयवेन संश्लिष्टं स्यात्~। तस्य द्वितीयस्य कोट्या अन्तरचतुरश्रातिरिक्तो यो भागस्तेन सह तृतीयस्य भागं कार्त्स्न्येनाभि\renewcommand{\thefootnote}{४}\footnote{नाति \textendash\ ख. पाठः.}सन्धाय तृतीयस्याप्यतिरिक्तप्रदेशेन चतुर्थस्य बाहुं सन्धाय जिज्ञासितकर्णबाहुं चतुरश्रं कुर्यात्~। तथा सति वर्गयोगेन कृत्स्नेन ज्ञेयकर्णबाहुं\renewcommand{\thefootnote}{५}\footnote{बाहु \textendash\ क. पाठः.} चतुरश्रमेवं कुर्यादिति भावः~। कथं पुनर्भुजातुल्यचतुरश्रेण कोटितुल्यचतुरश्रेण च घातद्वयमन्तरवर्गश्चोत्पाद्येते~। तदपि कोटिचतुर्बाहुकस्य भुजाचतुर्बाहुकस्य च कोणद्वयं संयोज्य(शि ?)श्लि)ष्टं कृत्वा तत्र महतश्चतुरश्रस्याल्पबाहुसंश्लिष्टभागादितरः खण्डो यस्त-

\newpage

\noindent दग्रकोणात् प्रभृत्यपरपार्श्वेऽपि भुजाकोट्यन्तरभागे बिन्दुं कृत्वा बिन्दुमध्यमत्स्येन क्षेत्रान्तः सूत्रं प्रसारयेत्~। तन्मार्गेण महच्चतुरश्रं
छित्वा तद्बहिर्भागं पृथक् कुर्यात्~। तद्भुजाकोट्यायामविस्तारमायतचतुरश्रं स्यात्~। तदितरभागो यः तस्य पुनर्नायतचतुरश्रत्वं नापि समचतुरश्रत्वम्~। शिखरसद्भावात् पञ्चबाहुकमेव तत्~। तच्छिखरे छिन्ने आयतचतुरश्रमितरतुल्यं स्यात्~। छिन्नः शिखरभागश्च भुजाकोट्यन्तरचतुर्बाहुः~। एवमायतचतुरश्रे द्वे घाततुल्ये~। तृतीयमन्तरबाहुचतुरश्रमिति भावः~। तस्मान्न केवलमत्रानीतभुजाकोट्योः कर्णप्रदर्शनमात्रपरमिदं सूत्रं, सामान्यविषय एव~। ततः सर्वेषां
भुजाकोटिकर्मक्षेत्राणां सामान्येनैकस्याज्ञातस्येतराभ्यां ज्ञाताभ्यामानयने यो न्यायः तत्प्रदर्शनपरमेवेदम्~। अतएव शङ्कोः प्रमाणवर्गमित्यादिवन्न विशेष(प्रदर्शनपर)मेतत्~। तद्धि स्ववृत्तविष्कम्भार्धानयनपरमेव~। इदं तु सामान्यनिष्ठम्~। अत एव सिद्धवस्तुप्रदर्शनपरवदुक्तं {\qt यश्चैव भुजावर्गः कोटीवर्गश्च कर्णवर्गः स} इति~। न पुनर्भुजाकोटिकर्णानाम् अन्यतमानयनमात्रं\renewcommand{\thefootnote}{१}\footnote{मत्र \textendash\ ख. पाठः.} प्रदर्श्यते~। यथा पुनर्भागं हरेदित्यादिषु लिङादयः प्रयुज्यन्ते नैवमत्र लिङ्लोट्तव्यादिकं प्रयुज्यते~। तस्मान्निधिमानेष भूभाग इत्यादिवाक्येन सदृशमेतत्~। तत्रापि कार्यपरत्वं विद्यत इति चेदस्तु तदत्रा\renewcommand{\thefootnote}{२}\footnote{तत्रा}पीष्यताम्~। यदि कर्णो ज्ञेयस्तर्हि तयोर्वर्गयोगो मूलीकार्यः~। यदा पुनर्बाहोर्ज्ञेयत्वं तदा कर्णवर्गाद्कोटिवर्गमपास्य शिष्टं मूलीकार्यम्~। यदा पुनः कोटिर्ज्ञेया तदा कर्णवर्गाद्बाहुवर्गमपास्य~। तस्मात् यस्य यदा यदिष्टं तदा तत् कुर्यादिति सामान्यपरमेवैतत्~। तच्च भुजाकोटिकर्णक्षेत्रेषु सर्वत्रापि समानमेव~। यदा पुनस्त्रैराशिके इच्छाप्रमाणक्षेत्रयोरेवमाकारविशेषविवक्षापि न विद्यते, भुजाकोट्याकारत्वमेव केवलमत्रेष्टं यत्र\renewcommand{\thefootnote}{३}\footnote{यदा} भुजाकोट्याकारता हीयेत तत्रैवैतन्न व्याप्नोति~। तस्माद्भुजाकोटिकर्णक्षेत्रे सर्वत्र व्याप्तत्वान्नाव्याप्तिः, नाप्यतिव्याप्तिः, भुजा कोटिशब्दोच्चारणात्~। कः पु\renewcommand{\thefootnote}{४}\footnote{णात्~। पु \textendash\ क. पाठः.}नर्भुजाकोटिकर्णक्षेत्राणां सामान्याकारः को वा विक्रियमाण आकारः~। भुजाकोट्योरितरेतरसमतिर्यगायतत्वं सामान्याकारः~। अत एव विषमत्र्यश्रेऽपि यत्र भुजाकोटिकर्णाकारत्वं न स्यात् तत्रापि तदन्तर्भूतभुजाकोटिकर्णकल्पनयैव तत्परिच्छेदः~। तद्विभागार्थमेव च लम्बः कल्प्यते~।

\newpage

\noindent लम्बभूम्योः पुनः परस्परं व्यस्तदिक्कत्वं स्यादेव~। यतो भूः समीकृता समतिरश्चीनैव~। न पुनस्तादृशभूतलस्य मनागप्यूर्ध्वायतत्वम्
उन्नत्यवनत्यभावादवयवानाम्~। लम्बश्चाधऊर्ध्वायत एव~। अतएवोक्तं {\qt अधऊर्ध्वं लम्बकेनैव} इति~। एवमेतयोः परस्परं व्यस्तदिक्कत्वाद्भुजाकोटित्वं युज्यत एवेत्येतज्ज्ञापनार्थमेव भूमिलम्बशब्दाभ्यां तयोरुक्तिः~। भुजाकोटिकर्णन्यायेन त्रैराशिकन्यायेन चोभाभ्यां सकलं ग्रहगणितं व्याप्तम्~। अत एवादितो भुजाकोट्याकारप्रदर्शनायैव हि त्रिभुजस्य फलशरीरमित्यारब्धम्~। यतस्तत्र क्षेत्रफलन्यायेनैव तदाकारः सिद्ध्यति, भुजार्धस्य समदलकोट्याश्च संवर्गः फलमिति ह्यत्रोक्तं, ततस्तस्यायतचतुरश्रत्वापादनं क्रियते~। आयतचतुरश्रत्वे हि घाततुल्यफलत्वम्~। तच्च बहुधा प्रदर्श्य फलप्रदर्शनमुपसंहरता दृढीकृतं च {\qt सर्वेषां क्षेत्राणां प्रसाध्य पार्श्वे फलं तदभ्यास} इति~। यथा वृत्तादिक्षेत्राणां मया आयतचतुरश्रत्वमापाद्यैव फलं प्रदर्शितम्, एवमन्यत्राप्यायतचतुरश्रतामापाद्य फलं चिन्त्यमित्यनेन सर्वेषु क्षेत्रेषु फलन्यायोऽतिदिश्यते~। तदत्राप्यायतचतुरश्रत्वं स्फुरेत्~। तद्विस्तारश्चतुर्भुजार्धसम इति चतुर्भुजार्धशब्देन प्रदर्शितम्~। तस्मात् तत्तुल्यत्र्यश्रे एकस्मात् कोणात् तदुभयस्पृष्टेतरबाहुमध्ये च तल्लम्बो यथा तद्भुजापेक्षया व्यस्तदिक्कः, एवमन्यत्रापि भुजाकोट्योरिति~। तत आरभ्यैतावत्पर्यन्तं सर्वत्र भुजाकोटिक्षेत्रपरिशीलनमेव कार्यते~। एतद्विस्पष्टमुक्तं भास्करेण\textendash

\begin{quote}
{\qt इष्टाद्बाहोर्यत्\renewcommand{\thefootnote}{१}\footnote{र्यस्या \textendash\ ख. पाठः.} स्यात् तत्समतिर्यग्दिशीतरो बाहुः~।\\
त्र्यश्रे चतुरश्रे वा सा कोटिः कीर्तिता तज्ज्ञैः~॥

तत्कृत्योर्योगपदं कर्णो दोःकर्णवर्गयोर्विवरात्~।\\
मूलं कोटिः कोटिश्रुतिकृत्योरन्तरात् पदं बाहुः~॥}
\end{quote}

\noindent इति~। भुजाकोटिकर्णात्मकत्र्यश्रक्षेत्रस्य सर्वत्र तुल्यमाकारं प्रदर्श्य तत्कर्म चोक्त्वा तदुपपत्तिश्च\textendash

\begin{quote}
{\qt राश्योरन्तरवर्गेण द्विघ्ने घाते युते तयोः~।\\
वर्गयोगो भवेदेवं तयोर्योगान्तराहतिः~॥\\
वर्गान्तरं भवेदेवं ज्ञेयं सर्वत्र धीमता~॥}
\end{quote}

\noindent इति विस्पष्टं प्रदर्शिता~॥~${\hbox{१६}} \dfrac{\hbox{१}}{\hbox{८}}$~॥

\newpage

यदेतदार्यार्धेन प्रतिपादितं तज्ज्याछेदविधानोपयोगि च~। तत्र परिधिपादस्य त्रिभुजेन छेदने भुजाकोटिवर्गयोगः कार्यः~। तत् पूर्वमेव मया प्रदर्शितं
द्वादशज्यानयने~। तत्र हि कर्णः साध्यः~। यत्र पुनः कर्णेन भुजाकोट्योरन्यतरेण चेतरानयनं तत्र चतुरश्रं कल्प्यते~। तत्र चतुर्भुजाछेदने वर्गविश्लेषमूलं कार्यम्~। तच्चैकराश्यर्धज्यया षोडश्या आनयने प्रदर्शितम्~। अनन्तरमनन्तरोक्तखण्डज्यानयनोपयोगिन्यायः प्रदर्श्यते\textendash

\begin{quote}
{\ab वृत्ते शरसंवर्गोऽर्धज्यावर्गः स खलु धनुषोः~॥~१७~॥}
\end{quote}

इति~। अत्र द्वयोर्धनुषोर्यदल्पं यत् तदर्धमर्धचापाकारं तत्सम्बन्धिज्यावर्गोऽत्रानीयते, तस्य या समस्तज्या तद्वर्गो व्यासहृत एवाल्पः शरः~। यत्पुनरल्पशरस्य समस्तचापं तस्य समस्तज्या अत्रानीतवर्गमूलज्यया द्विगुणया तुल्या~। अतस्तस्याः समस्तज्याया वर्गो व्यासहृत एतद्द्विगुण\renewcommand{\thefootnote}{१}\footnote{एव द्विगुण}चापशरः स्यात्~। ज्याशरयोरुभयोर्ज्ञातयोरप्यज्ञातव्यासानयनमनेनैव सिद्धम्~। तदुक्तं भास्करेण\textendash

\begin{quote}
{\qt जीवार्धवर्गे शरभक्तयुक्ते व्यासप्रमाणं प्रवदन्ति वृत्ते~॥}
\end{quote}

\noindent इति~। जीवार्धवर्गस्य शरसंवर्गत्वात् जीवार्धवर्गे धनुषोरन्यतरस्य शरे च ज्ञाते तद्वर्गाच्छरहृतोऽन्यः शरः, यत उभयोः संवर्गोऽन्यतरेण हृतोऽन्यतरः स्यादिति~। एवमुभयोः शरयोर्ज्ञातयोः तद्योगं कृत्वा व्यासो ज्ञेयः, यतो व्यासखण्डावेवोभौ शरौ~। किञ्च शरसंवर्गेऽर्धज्यावर्गतुल्ये शरवर्गे च युक्ते समस्तज्यावर्गः स्यात्, शरार्धज्ययोर्भुजाकोटित्वात् कर्णत्वाच्च समस्तज्यायाः~। तत्र यौ द्वौ भागौ तत्रार्धज्यावर्गे महाशरगुणितोऽल्पशरः, शरवर्ग\renewcommand{\thefootnote}{२}\footnote{शरसंवर्ग \textendash\ क. पाठः.}श्चाल्पशरगुणितोऽल्पशरः~। तद्योगः पुनर्व्यासगुणितोऽल्पशर एव, यतोऽल्पशर एव व्यासखण्डाभ्यामुभाभ्यां पृथग्घातौ योज्येते समस्तज्यावर्गसिद्ध्यर्थम्~। व्यासखण्डाभ्यां पृथक् पृथग्गुणितस्य योगः, समस्तेन व्यासेन गुणितश्च तुल्यावेव स्याताम्~। तस्मात् समस्तज्यावर्गेऽल्पशरहृते कृत्स्नो व्यासः स्यात्~। एवं समचापस्य प्रमाय विदितस्य समस्तज्यावर्गो व्यासहृतस्तच्छरः स्यादिति सिद्धम्~। तदेव लघुभास्करीयव्याख्यायां सुन्दर्यामुक्तं\textendash

\newpage

\begin{quote}
{\qt राशिजीवासमभ्यस्तसमबाणसमुत्थितम्~।\\
कृतिमूलं भवेज्ज्यार्धं तद्बाणधनुषोऽर्धके~॥

तत्त्रिज्यावर्गविश्लेषान्मूलं\renewcommand{\thefootnote}{१}\footnote{ल \textendash\ क. पाठः.}  ज्या तत्पदाह\renewcommand{\thefootnote}{२}\footnote{ग}ते(:)~।\\
तयोनं विस्तरस्यार्धं बाणः पदगतो भवेत्~॥}
\end{quote}

\noindent इति~। ननु पूर्वं शरस्य व्यासगुणनमुक्तम्, इह तु राशिजीवागुणनम्, अतो विषम उपन्यास इति चेत्~। नैष दोषः~। तत्र तद्बाणधनुषोऽर्धस्य समस्तज्यानयने व्यासगुणनमुक्तम्~। इह तु तदर्धस्य~। शरसंवर्गात् बाणवर्गयुताच्चतुरंश एव पुनस्तदर्धज्यावर्गः~।
शरसंवर्गमूलज्यासम्बन्धिश्चापार्ध\renewcommand{\thefootnote}{३}\footnote{चापादर्ध}स्यार्धज्यानयनमिहोच्यते~। तद्वर्गश्च तत्समस्तज्यावर्गात् चतुरंश एवेति चतुरंशत्वाय व्यासचतुरंशतुल्यया राशिज्यया गुण्यते~। अत उक्तं \textendash\ {\qt कृतिमूलं भवेज्ज्यार्धं तद्बाणधनुषोऽर्धके} इति~। अतएव समशब्दोपादानम्~। यावतां चापानामर्धज्याया आनिनीषा ततो द्विगुण\renewcommand{\thefootnote}{४}\footnote{द्विगुणस्य \textendash\ ख. पाठः.}चापस्य शरादेवोक्तप्रकारेण तदानयनं कार्यम्~। बाणसम्बन्धिनोऽर्धचापस्यानेनार्धज्यानयनमुच्यत इति द्विगुणत्वाद्युग्मत्वमेवास्य युज्यते~। यस्यकस्यचिदपि निरवयवस्य राशेर्द्विगुणीकृतस्य युग्मत्वमेव स्यात्~। द्विगुणीक्रियमाणस्य युग्मत्वे तावद्द्विगुणितस्यापि युग्मत्वमेव युक्तम्~। ओजत्वेऽपि द्वन्द्वेभ्योऽतिरिक्तस्य चरमस्य द्विगुणीकरणाद्युग्मत्वमेव स्यादिति युग्मचापानां बाणेनैव सर्वा ज्या इहानेया इति~। अतः सिद्धं {\qt समस्तज्यावर्गाद्व्यासहृतः शर} इति~। अर्धज्यावर्गस्य समस्तज्यावर्गस्य चान्तरं शरवर्ग एव~। एवमर्धज्याशरवर्गयोगश्चतुर्भिर्हृतस्ततोऽर्धस्यार्धज्यावर्गः~। यतः समस्तज्यार्धात्मकचापसम्बन्धिनी ततस्तदर्धमेव तदर्धस्यार्धज्या~। कृत्स्नस्य चापस्य समस्तज्यार्धं ह्यर्धचापस्यार्धज्या~। धनुषश्चार्धं जीवायाश्चार्धमेवेति द्वे अप्यर्धचापमर्धजीवेति चोच्यते, न पुनस्तदर्धं कृत्स्नचापज्यार्धमित्युच्यते, चापार्धसम्बन्धितयैवाङ्गीकारः, यतो मध्यमोच्चविवरगतेन चापेनोच्चनीचरेखाया ग्रहस्य च विप्रकर्षो ज्ञेयः, न पुनरुच्चस्य ग्रहस्य च~। एवं दृङ्मण्डलगतशङ्कुरपि न क्षितिजार्धान्तरम्~। किन्तर्हि~। क्षितिजव्याससूत्रादुत्सेध एव~। तद्व्यासाग्रस्य रवेश्च विवरगता

\newpage

\noindent समस्तज्या पुनस्तस्योत्सेधः~। खमध्यमासादयतः सूर्यस्य तत्समस्तज्या क्रमेणानमन्ती क्रमादूर्ध्वायतत्वं जहाति, उदयक्षण एव तस्या ऊर्ध्वायतत्वं, पुनरपराह्णेऽपि क्रमेणानमन्ती अस्तमये समतिरश्चीना स्यादिति तदानीं क्षि\renewcommand{\thefootnote}{१}\footnote{दिति क्षि \textendash\ क. पाठः.}तिजव्यासतामापन्ना~। अतस्तद्वशान्न द्वादशाङ्गुलच्छायावृद्धिह्रासौ~। स्ववृत्तविष्कम्भार्धानुसारेण तद्बिम्बान्तं प्रसारितस्य सूत्रस्य भुजाकोटिवशादेव तद्वृद्धिह्रासौ~। तद्भुजाकोट्यौ च दृङ्मण्डलगते ज्ये~। तत्र दृग्ज्या\renewcommand{\thefootnote}{२}\footnote{दृश्या} खमध्यात् प्रभृति सूर्यबिम्बान्तार्धज्या~। सा समोर्ध्वाधःसूत्रमेव स्पृश(न्ती ? ति)न खमध्यगतदृग्ज्यामण्डलपरिधिप्रदेशम्~। एवं हि तद्वन्धनं दृङ्मण्डले खमध्यात् सूर्यापरभागे च तावदन्तरे बिन्दुं कृत्वा ततो गोलान्तः\renewcommand{\thefootnote}{३}\footnote{गोलान्त \textendash\ ख. पाठः.} सूर्यबिम्बप्रापि यत्सूत्रं नीयते तदर्धमिह खमध्यात् प्रभृति सूर्यान्तस्य चापस्यैव ज्यार्धं, न पुनस्तदुभयाग्रावधिकस्य~। यद्यपि कृत्स्नचापस्य ज्याया अर्धमेव तत्, तथापि तस्य ज्यार्धमिति न व्यपदेशः~। कथं तर्हि~। चापार्धस्य ज्यार्धमिति हि व्यपदिश्यते~। एवं चापीकरणेऽपि~। तेन तदर्धचापमेवानीयते~। अत उक्तं \textendash\ {\qt धनुषश्चार्धं जीवायाश्चार्ध}मिति~। अर्धत्वसाम्याच्चापार्धस्यार्धज्येत्युच्यते~। तस्यार्धस्य या पुनः समस्तज्या तच्चापार्धस्यार्धज्यापि तदर्धतुल्या~। अतस्तद्वर्गादर्धज्यावर्गश्चतुरंश एव~। एवं तदर्धज्यानयनम्~। पुनरप्येवमेव तत्तच्चापार्धस्य ज्यार्धानयनम्~। एवं चापार्धपरम्पराणामर्धज्यावर्गानयने तत्तच्छरवर्गः पुनः पुनः क्षेप्यः~। शरवर्गं क्षिप्त्वा पुनश्चतुर्भिर्हत्वा तत्रापि तच्छरवर्गं क्षिप्त्वा पुनरपि चतुर्भिर्हार्यम्~। एवं प्रतिज्यावर्गानयनं शरवर्गक्षेपश्चतुर्हरणं चेति द्वे एव कर्मणी स्तः~। तत्तच्छरमानीयैव तच्छरवर्गः क्षेप्तुं युक्त इति शरश्च तत्र तत्रानेयः~। तथापि तत्तत्समस्तज्यावर्गस्य द्वावेवांशौ चतुरंशौ स्वद्विगुणचापार्धज्यावर्गचतुरंशस्तच्छरवर्गचतुरंशश्चेति~। तयोर्ह्रासप्रकारोऽत्र निरूप्यः~। तत्र प्रथमस्य समस्तज्यावर्गस्य चतुरंशपरम्परामार्ग एकः~। तत्र क्षिप्यमाणस्य तत्तच्छरवर्गस्य(गिती? गति)र्हि\renewcommand{\thefootnote}{४}\footnote{तर्हि \textendash\ क. पाठः.} पृथङ् निरूप्या~। तत्र प्रथमभूताया अर्धज्याया द्विगुणायास्तत्समस्तज्याभूताया वर्गाद्व्यासहृतस्वशरः पुनः स्वशरवर्गयुक्तात् प्रथमभूतार्धज्या-

\newpage

\noindent वर्गात् व्यासहृतस्तस्यै\renewcommand{\thefootnote}{१}\footnote{तदस्यै \textendash\ ख. पाठः.}वार्धज्यासम्बन्धिनः शरः~। स पुनः प्रायेण चतुरंश एव~। एवं तत्तदर्धस्य शरस्तत्तच्छरा\renewcommand{\thefootnote}{२}\footnote{शरस्तच्छरा}च्चतुरंश एव, यतस्तदर्धस्य समस्तज्या पूर्वपूर्वसमस्तज्याया अर्धतुल्या~। तस्माच्छराणां चतुरंशत्वात् पूर्वशरवर्गात् षोडशांश एवोत्तरोत्तरशरवर्ग\renewcommand{\thefootnote}{३}\footnote{शरपद \textendash\ क. पाठः.} इति शरवर्गषोडशांशपरम्परैवैको भागः~। एवं प्रथमार्धज्यावर्गचतुरंशपरम्परापूर्वशरवर्गषोडशांशपरम्परा तत्तदर्धशरवर्गषोडशांशपरम्परायुक्ता अन्यमार्गगता~। तस्याः परम्परायां बहु\renewcommand{\thefootnote}{४}\footnote{बाहु \textendash\ ख. पाठः.}शाखत्वात् तन्निरू(पेण ? पणे) प्रणिहितमनसा भाव्यम्~। कथं पुनस्तस्या बहु\renewcommand{\thefootnote}{५}\footnote{बाहु \textendash\ ख. पाठः.}शाखत्वं येन दुर्निरूपणत्वं तस्याः~। तत्राद्येऽर्धज्यावर्गे तच्छावर्ग क्षिप्त्वा चतुर्भिर्ह्रियते~। तत्र योंऽशस्तच्छरवर्गचतुरंशभूतः स पुनरितरपरम्परायै देयः, तज्जातीयत्वात् तस्य~। तत्र शिष्टः केवल एवार्धज्याचतुरंश इति सा शुद्धैव, अन्यसम्पर्काभावात्~। एवमुभयाशात्मके सम्पन्नार्धज्यावर्गे पुनर्द्वितीयशरवर्गं क्षिप्त्वा चतुर्भिर्ह्रियते~। स पुनः पूर्वशरवर्गात् प्रायशः षोडशांशः~। तच्छरस्य तदुभयात्मकार्धज्यावर्गादुभयचतुरंशमिश्रिताद्व्यासार्धाप्तत्वात् तस्य पूर्वशरचतुरंशादीषदाधिक्यम्~।
पूर्वशरवर्गचतुरंशभूताद्व्यासहृतेनाधिकत्वात्~। एवं पुनः पुनरपि शरवर्गस्य षोडशांशात् ईषदाधिक्यं स्यात्~। तस्याधिक्यस्य चापाल्पत्वानुरूपं
क्रमेणाल्पत्वं स्यात्~। चापस्याल्पत्वं चैकस्य शरस्य व्यासासन्नतया अल्पस्य शून्यासन्नतया च~। अतएव महतः शरस्य परार्धदशकत्वमितरस्यैकत्वं च यत्र स्यात्, तत आरभ्य निरूप्यमाणे शरवर्गपरम्पराया अपि षोडशांशत्वमेवांशीकृत्य निरूप्यमाणेऽपि सूक्ष्मता स्यादित्यूर्ध्वमुत्प्लुत्य निरूपणं
प्रदर्शितम्~। तत्प्रदर्शनायैव हीदं सूत्रमप्यारब्धं {\qt वृत्ते शरसंवर्ग} इति~। एवं चापार्ध परम्परागतार्धज्योशरयोर्वर्गाश्चतुरंशतया षोडशांशतया च क्रमान्न्यूनाः~। अन्त्यादुत्क्रमेण चेद्गुणोत्तरतया अन्त्यशरवगार्त् प्रभृत्यधोधः षोडशगुणाः~। अन्त्याच्छरवर्गात् षोडशगुण उपान्त्यशरवर्गः~। एवं तत्तदधोधोगताः ततस्ततः षोडशगुणिताः~। एवं द्विगुणोत्तराणां चापानां शरवर्गाः पदादावेव षोडशगुणाः~। ततः प्रभृति क्रमेण न्यूनगुणाः~। तत्र ज्याचापयोरन्तरस्य पुनः पुनर्न्यूनत्वं चापपरिमाणाल्पत्वक्रमेणेति तत्तदर्धचापानामर्धज्यापरम्परा

\newpage

\noindent शरपरम्परा चानीयमाना न क्वचिदपि पर्यवस्यति आनन्त्याद्विभागस्य~। ततः कियन्तञ्चित् प्रदेशं गत्वा चापस्य जीवायाश्चाल्पीयस्त्वमापाद्य चापज्यान्तरं च शून्यप्रायं लब्ध्वा पुनरपि कल्प्यमानमन्तरमत्यल्पमपि कौशलाज्\renewcommand{\thefootnote}{१}\footnote{कौशलं} ज्ञेयम्~। पुनरपि तत्तदर्धभवानां क्रियाविशेषाणामनन्तानामपि भवन्तं फलविशेषं समुच्चित्य युगपदेकेनैव कर्मणानेयम्~। कथं ततःप्रभृत्यपि कर्तव्यानामर्धज्यावर्गाणां परम्परयोरपि द्विधैवैवं गतिः~। स्वस्वचतुरंशमार्ग एका~। स्वस्वशरकृतिषोडशांशपरम्परा चान्या~। तेषां पुनरुत्क्रमेण गुणनं कृत्वैव एतच्चापगतज्याचापान्तरं ज्ञेयम्~। यथा \textendash\ तत्र कृत्स्नतया कल्पितस्य चापस्य कृत्स्नज्यावर्गाच्चतुरंशस्तदर्धस्यार्धज्या~। तत्र तच्छरवर्गात् षोडशांशश्च योज्यस्तद्गतसमस्तज्यावर्गत्वाय पुनस्तन्मूलीकृत्य द्विगुणं कृत्वैव तत्कृत्स्नचापो लभ्येत~। तदा कृत्स्नचापस्य जीवाया ईषदधिकत्वं च स्यात्~। कुतः~। शरवर्गयुक्तस्य मूलीकृत्य द्विगुणीकृतत्वात्~। तत्र कृत्स्नचापस्यापि वर्ग एव ज्ञेयः तर्हि मूलीकरणं न कार्यम्~।
कृत्स्नस्यार्धज्यावर्गतश्चतुरंशे शरवर्गांशं प्रक्षिप्य चतुर्गुणनमेव कार्यम्~। एवं पुनः पुनः~। त\renewcommand{\thefootnote}{२}\footnote{पुनः~। त}स्माज्जीवावर्गस्य न चतुर्भिर्हरणं कार्यं, चतुर्गुणनस्य कार्यत्वात्~। उभयस्याप्यकरणेऽपि फलविशेषाभावात्~। तत्र शरवर्गस्य पुनः पुनः षो\renewcommand{\thefootnote}{३}\footnote{पुनः षो}डशांशस्यैव योज्यत्वात् ज्यावर्गचतुरंशे युक्त्वा तेन सह चतुर्गुणनस्यापि कार्यत्वाज्ज्यावर्गस्य हरणाभावे पूर्वशरवर्गाच्चतुरंश एव तत्र क्षेप्यः, चतुर्हृतज्यावर्गे क्षिप्तस्य षोडशांशस्य पुनस्तेन सह चतुर्गुणनस्य कार्यत्वात्~। षोडशांशस्य चतुर्गुणितस्य पूर्वापेक्षया चतुरंशत्वात्~। एवं शरवर्गस्य चतुरंशपरम्परैव योज्या प्रथम एव ज्यावर्गे~। ततः पुनर्द्विगुणनपरम्परा न कार्या~। कथं पुनरेवमानीतस्यान्तरान्तरा\renewcommand{\thefootnote}{४}\footnote{तरान्तरात् \textendash\ क. पाठः.} आद्ये ज्यावर्गे स्वशरवर्गं कृत्स्नं क्षिप्त्वा पुनरर्धीकृत्य पूर्वशरवर्गषोडशांशं क्षिप्त्वा
पुनरप्यर्धीकृत्यार्धीकृते पूर्वपूर्वक्षिप्तशरवर्गषोडशांशं तत्र तत्र क्षिप्त्वानीतस्य वर्गराशेर्यावत्कृत्वोऽर्धीकृतस्तावत्कृत्वो द्विगुणनं कृत्वानीतस्यापि फलसाम्यं स्यात्~। उच्यते~। तत्र ज्यावर्गपरम्परायां यश्चरमः स एकोंऽशः~। तावत्कृत्वो दलितस्य चापवर्गस्य तस्य तन्मूलीकरणमकृत्वा तद्वर्गस्यैव प्रत्यापादने क्रियमाणे ताव-

\newpage

\begin{sloppypar}
\noindent त्कृत्वश्चतुर्गुणनमेव कार्यम्~। चापस्य च द्विगुणनमेव यतस्ततो\renewcommand{\thefootnote}{१}\footnote{यतस्ततस्तो \textendash\ क. पाठः.}ऽर्धचापस्य कृत्स्नचापवर्गाच्चतुरंशत्वमेव स्यात्~। अतश्चरमस्य चतुरंशस्य मूलीकृतस्य तच्चापत्वमापन्नस्य पूर्वपूर्वचापत्वप्राप्तये द्विगुणीकरणमेव कार्यम्~। द्विगुणोत्तराणां मूलानां वर्गस्य चतुर्गुणोत्तरत्वाद्वर्गपरम्पराया उत्क्रमेण चतुर्गुणनमेव कार्यम्~। तस्माद्धरणस्य गुणनस्य चाकरणेऽपि फलसाम्यात् किमर्थं बहुकृत्वो हरणं गुणनं च क्रियत इत्यर्धज्यावर्गोऽविकृत एव स्थाप्यः~। यत्पुनस्ततोऽधिकृतं\renewcommand{\thefootnote}{२}\footnote{कृत्वं} चापवर्गस्य तत्र तत्र शरवर्गक्षेपेण स्यात् तमंशं पृथगानीयाविकृते तज्ज्यावर्गे क्षिप्त्वा कृत्स्नश्चापवर्ग आनेयः~। तत्र तत्र शरवर्गपरम्पराया एकीकरणं कथम्~। प्रथमं तावदर्धज्यावर्गे शरवर्ग एव कृत्स्नः क्षेप्यः तयोः कर्णभूतायाः समस्तज्याया वर्गसिद्धये~। पुनस्तच्चतुरंशे चेदर्धचाप\renewcommand{\thefootnote}{३}\footnote{चापस्य} शरस्य चतुरंशत्वात् तद्वर्गः पूर्वशरवर्गात् षोडशांश एव~। त\renewcommand{\thefootnote}{४}\footnote{एव~। तत्र त \textendash\ क. पाठः.}स्मिन् ज्याशरवर्गयोगे चतुर्भिर्हृते चेत् पूर्वं क्षिप्तस्य शरवर्गस्य षोडशांश एव क्षेप्यः~। चतुर्भिर्हरणमकृत्वा ज्याशरवर्गयोगेनानीते समस्तज्यावर्गे लाघवाय पूर्वशरवर्गचतुरंश एव क्षेप्यः~। तथा तस्यापि चतुर्हृते क्षेप्यस्य चतुर्गुणनेन कृत्स्नचापसंबन्धित्वं स्यात्~। एवं पुनः पुनः षोडशांशचतुरंश एव\renewcommand{\thefootnote}{५}\footnote{पूर्वं} क्षेप्यः~। पूर्वपूर्वचतुरंश एव सः~। तस्मात् केवले ज्यावर्गे तच्छरवर्गं कृत्स्नं च पुनस्तत्तदर्ध\renewcommand{\thefootnote}{६}\footnote{तदर्ध \textendash\ क. पाठः.}भवषोडशांशपरम्पराणां प्रत्यापत्त्या कृत्स्नचापभवत्वाय तावत्कृत्वश्चतुर्गुणनस्यापि कार्यत्वात् चतुरंश एवाविकृते क्षेप्यः स्यात्~। ज्यावर्गचतुरंशपरम्परायामेव शरवर्गषोडशांशपरम्पराणां क्षेप्यत्वमुक्तं, न पुनरविकृते~। तस्मात् प्रथमशरवर्गे
तच्चतुरंशपरम्परां कार्त्स्न्येन क्षिप्त्वा पुनरविकृते ज्यावर्गे संयोज्य तत्कृत्स्नचापस्य वर्गो लभ्यः~। चतुरंशपरम्परायाः पुनः कार्त्स्न्येन लाभाय त्रिभिरेव शरवर्गो हर्तव्यः, यतश्चतुरंशपरम्परासमुदायः कृत्स्नोऽपि त्र्यंशत्वमेवापद्यते~। कथं पुनस्तावदेव वर्धते तावद्वर्धते च~। उच्यते~। एवं यस्तुल्यच्छेदपरभागपरम्पराया अनन्ताया अपि संयोगः तस्यानन्तानामपि कल्प्यमानस्य योगस्याद्यावयविनः परम्परांशच्छेदादेकोनच्छेदांशसाम्यं सर्वत्रापि समानमेव~। तद्यथा\textendash
\end{sloppypar} 

\newpage

\noindent चतुरंशपरम्परायामेव तावत् प्रथमं प्रतिपाद्यते~। ये राशेर्द्वादशांशास्तेषां त्रिकं हि चतुरंशः~। चतुष्कं च त्र्यंशः~। तच्चतुष्टये त्र्यंशात्मके
भागत्रयं चतुरंशेनापूर्णम्~। यः पुनस्तस्य चतुर्थोंऽशः तस्यापि पादत्रयं चतुरंशस्य चतुरंशेनापूर्णम्~। द्वादशांशानां त्रयाणां ये पादास्तत्तुल्या एव तस्यापि
चतुर्थस्य, तस्याप्यवयविनो द्वादशांशत्वात्~। तस्मात् त्रयाणां तुर्यांशः तत्तुर्यांशत्रयतुल्य इति तस्य चरमः पाद एवावशिष्टः~। तत्राप्यवशिष्टस्यान्येषां च परस्परं साम्यात् तुरीयोंऽशोऽपि प्रत्येकं तुल्यः~। इति त्रयाणांं तुर्यांशश्चतुर्थपादत्रयतुल्य एवेति तत्रापि तत्तुरीयांश एवावशिष्टः~। ततस्तेनैवावयविनस्तृतीयांशस्यापरिपूर्तिः~। एवमवशिष्टस्य तृतीयांशनिम्नभागस्य पूर्वपूर्वपादत्रयतुर्यांशसमुदायेन शिष्टांशस्य सर्वदापि
पादत्रयपूरणं स्यादिति द्वादशांशस्य चतुरंशभागप्राग्भागेषु चरमांशेनैकेनैव सर्वदा त्र्यंशस्य निम्नता~। तत्र न चतुर्थाङ्घ्रिपरम्परा ग्राह्या~। तत्र
यावत्कृत्वस्तत्च्चत\renewcommand{\thefootnote}{१}\footnote{स्तच्च \textendash\ क. पाठः.}तुरंशाः क्षिप्ताः तावत्कृत्वो न्यस्तानां चतुष्काणां राशीनां तेषां घातच्छेदस्तस्य~। अंशः पुनरेक एव~। तस्माद्द्वादशांशस्य चतुष्कपरम्पराघातांशक एक एव शरवर्गत्र्यंशान्निम्नोंऽशः~। तस्य पुनः पुनरतिसूक्ष्मत्वादेव न केवलं त्र्यंशत्वेनाङ्गीकारः, निरूप्यमाणस्य वा क्रियमाणस्य वानन्त्यात्~। आनन्त्यादेव शिष्टत्वादेव कर्मणस्तस्यापरिपूर्तिर्भाति~। एवं सर्वदापि सावशेषाणां कर्मणां परम्परायां कार्त्स्न्येनाकृष्यात्र सन्निहितायां परिपूर्तिः स्यादेवेति निश्चीयते चतुर्गुणोत्तरे गुणोत्तराख्ये गणितेऽपि~। तत्र पृथ्वग्रत्वं रूपात् प्रभृत्यारभ्यमाणत्वात्~। अत्र महतः प्रभृति चतुरंशानां न
चतुर्गुणोत्तरत्वं चतुरंशोत्तरत्वमेव स्यात्~। तेनात्र चरमांशस्य कृत्स्नधने यावत्य आवृत्तयः स्युस्तत्र रूपस्य तावदावृत्तिः~। तस्याप्युत्क्रमेण निरूप्यमाणस्य चतुरंशोत्तरत्वमेव न चतुर्गुणोत्तरत्वम्~। अत एव सूच्यग्रतुल्यत्वं च~। तत्रेयानेवोभयोर्विशेषः~। तस्य पर्यवसानं स्याद्रूपे~। अतो नानन्त्यम्~। अत एकोनत्वं तस्य~। अत्र पुनरानन्त्यात् अनन्तानां कार्त्स्न्येन परिगृहीतत्वायैकोनच्छेदांशक्षेपणमेव कार्यमिति~। अत एव तत्र गुणोत्तरे व्येकमिति व्येकत्वोक्तिः~। तत्र तावद्द्विगुणोत्तरपरम्परायां विस्पष्टतास्तीति तद्युक्तिरपि प्रदर्श्यते~। तत्र

\newpage

\noindent यदाष्टसङ्ख्यो गच्छस्तत्र गच्छतुल्यानां द्विकानां घात एव गुणवर्गजं फलं न तच्चरमधनम्~। सप्तानां द्विकानां घात एव तत्र चरमधनम्, यत एकादिद्विगुणोत्तरे द्वितीयपद एव द्विकं स्यात्~। ततस्तद्द्विगुणं तृतीये~। अतो द्विकयोर्घात एव तृतीये, न त्रयाणां द्विकानां घातः~। तस्मादेकोनगच्छतुल्यनां द्विकानां घातोऽन्त्यधनम्, आदे रूपतुल्यत्वात्~। यद्वा आदौ रूपं विन्यस्य पुनरेकोनगच्छतुल्येषु स्थानेषु द्विकानि च विन्यस्य गच्छतुल्यानां तेषां द्विकानां संवर्गोऽन्त्यधनम् उभयोर्विशेषाभावात्~। एकगुणितस्य केवलस्य च विशेषाभावात्~। वृत्तसङ्ख्यायां पुनरेकाक्षरस्यापि गुरुलघुभेदेन भेदद्वय सम्भवात् अक्षरसङ्ख्यातुल्यस्थानेषु द्विकानि विन्यस्य तेषां परस्परगुणनेन वृत्तसङ्ख्यानेया~। अत्र पुनः पादतुल्येन गच्छेन तदनन्तरपदधनमेव गुणवर्गजं स्यादित्यष्टमपदगतादष्टाविंशत्युत्तरशताद्द्विगुणमेवाष्टसङ्ख्येन गच्छेनानीतं गुणवर्गफलम्~। तत् पुनर्व्येकमष्टपद\renewcommand{\thefootnote}{१}\footnote{पाद}स्थगुणोत्तरसमुदायः~। अष्टमपदे उपान्त्यादिपदगतधनसंयोगे अष्टमात् पदात् तत्तत्पदधनतुल्येन न्यूनत्वात्~। तथाहि \textendash\ उपान्त्यपदगतं चतुष्षष्टिसङ्ख्यम्~। तस्यान्त्येन साम्याभावस्तावतैव न्यूनत्वात्~। एवं तदुपर्युपर्यर्धं तावता तावता न्यूनमित्येकसङ्ख्ये आदिमे क्षिप्तेऽप्येकेन न्यूनता स्यादिति पूर्वसमुदायस्यान्त्याद्व्येकत्वम्~। अत एवान्त्यधनं द्विघ्नं व्येकं सर्वधनं स्यात्~। तत्र व्येकगुणोत्तरस्थाने पुनरत्र व्येकच्छेदहरणमेव युक्तं, महतो राशेरारभ्यमाणत्वात्~। न पुनरुत्तरोत्तरं गुणितत्वम्~। अतो भागप्रभागपरम्परैवेयम्~। अत्र यथा चतुरंशपरम्परा एवं गुणोत्तरेऽपि प्रातिलोम्येन निरूप्यमाणस्य चतुर्गुणोत्तरत्वे चतुरंशपरत्वमेव स्यात्~। अतो व्येकेन गुणेन चतुर्गुणोत्तरादौ त्र्यादिभिरेव सर्वेषां युगपदानयने हरणं कार्यम्~। एवमत्रापि व्येकच्छेदेनैव हर्तव्यम्~। अस्यानन्त्याद्व्येकमित्युक्तम्~। एकशोधनमेव कार्य\renewcommand{\thefootnote}{२}\footnote{मेव न कार्य}मित्येव विशेषः~।
यदि तत्राप्यूर्ध्वत एवारभ्य\renewcommand{\thefootnote}{३}\footnote{आरभ्य \textendash\ ख. पाठः.} भागप्रभागादिपरम्पराया रूपमतीत्यापि तदधोऽप्यानन्त्यं कल्प्यते तर्हि कार्त्स्न्येनाकर्षणाय व्येकमित्येतत् कर्म न
कर्तव्यम्~। त्रिगुणोत्तरे चतुर्गच्छे तावदुदाहरिष्यामः~। द्विगुणोत्तरे व्येकगुणस्यैकत्वात् अस्पष्टत्वात् त्रिगुणोत्तर एव\renewcommand{\thefootnote}{४}\footnote{वैवयप \textendash\ क. पाठः.} त्र्यंशपरम्परायां वा व्येकगुणहरणे
विशेषः

\newpage

\noindent स्यादिति~। तत्र चतुस्सङ्ख्यं गच्छं विन्यस्य द्विरर्धीकरणं कार्यम्~। ततो वर्गचिह्नद्वयं च स्थाप्यम्~। पुनरर्धीकर्तुमशक्यत्वात् तदेकत्याग एव कार्यः~। अत एव सर्वत्र चिह्नेष्वन्ते गुणचिह्नस्थापनं क्रियते तत्र गुणनचिह्ननिमित्तं प्रथमं त्रिभिर्गुणनं कार्यम्~। रूपस्य तस्य त्रिकस्य
पुनर्द्विर्वर्गीकरणे कृते एकाशीतिः सम्पद्यन्ते~। तस्या व्येकत्वेऽशीतिः शिष्यते~। तत्र व्येकगुणो द्विकः, तेन द्वाभ्यां हृताशीतिंश्चत्वारिंशत्~। तत् सङ्कलितेनापि निर्णेतुं शक्यम्~। एकस्य त्रिकस्य च सङ्कलितं चतुष्कं तस्य नवकस्य च त्रयोदशकं तस्य सप्तविंशतेश्च चत्वारिंशत्~। तत्र यदि सप्तविंशतेः प्रभृत्युत्क्रमेण पूर्वपूर्वत्र्यंशात्मकानां रूपमतिक्रम्यापि त\renewcommand{\thefootnote}{१}\footnote{यापि पुनरपि त \textendash\ ख. पाठः.}त्त्र्यंशत्वम् एवं पुनः पुनरपि तत्त्र्यंशः~। एवं त्र्यंशपरम्परायां तत्र चत्वारिंशत्सङ्ख्ये यच्चरममेकरूपं तद्व्येकेन छेदेन द्विसङ्ख्येन हरेत्~। तत्फलं रूपस्यार्धम्~। तस्मिंश्चत्वारिंशति प्रक्षिप्ते सार्धचत्वारिंशत् तेषामनन्तानां सकलानामपि राशीनां संयोगः स्यात्~। एवं पुच्छभागस्य कार्त्स्न्येनाकर्षः~। एवमत्रापि चतुरंशपरम्परायास्तस्य शरवर्गे स्वत्र्यंशे क्षिप्ते चतुरंशपरम्परा कार्त्स्न्येन संयुक्ता स्यात्~। दशगुणोत्तरेऽप्युदाहरिष्यामः, तत्रातिलाघवं स्यादिति~। तद्यथा \textendash\ नवकमादिं प्रकल्प्य दशगुणोत्तरराशयोऽष्टादशकगच्छस्य प्रातिलोम्येन गण्यमानाः परार्धनवकात् प्रभृत्यधोधःस्थानेऽप्यङ्कानां नवकमेकः स्यात्~। स्थानान्तरगतत्वादेव दशांशत्वं दशगुणो\renewcommand{\thefootnote}{२}\footnote{गुणो \textendash\ क. पाठः.}त्तरत्वञ्च~। अतोऽष्टादशसु स्थानेषु समा एवाङ्काः~। एवमष्टादशस्थाननवकात् प्रभृत्येतत्स्थानगतनव\renewcommand{\thefootnote}{३}\footnote{स्थाननव}कान्तं या दशांशपरम्परा तत्रापि सूच्यग्रतया तत्तदधोधोंशानामप्यानन्त्यं कल्प्येत~। गुणकस्य वा छेदस्य वाप्येकस्य एकस्थानगतनवांश एव प्रक्षेप्यः, तत्परम्परायाः सकलस्यापि रूपस्यैकस्मिन् रूप एव परिसमाप्तत्वात्~। आदिस्थानगतनवके तस्मिन् रूपे क्षिप्ते सति तत्र शिष्टा\renewcommand{\thefootnote}{४}\footnote{शिष्ट \textendash\ ख. पाठः.}या दशक\renewcommand{\thefootnote}{५}\footnote{त \textendash\ क. पाठः.} आरोपणेन विषमस्थानं शून्यमेव स्यात्~। द्वितीयनवकेऽप्यारोपितेन सह दशकत्वात् तत्स्थानमपि शून्यमेव~। एवं शतादिस्थानगतानां नवकानां पूर्वपूवारोपितेन सह दशकत्वापत्तेस्ततः कार्त्स्न्येनारोपणात् शून्यत्वमेव स्थानाष्टादशकानाम्~। एवमष्टादशात् स्थानादप्यारोपितमेकमेकोनविंशेऽपि स्थाने स्यात्~। एवं परार्धनवकात् प्रभृति या दशांशपरम्परानन्ता तत्सकलयोगे परार्धदशकमेव नाप्य-

\newpage

\noindent धिकं नाप्यूनम्~। एवमणुमात्रेणापि न्यूनत्वमतिरिक्तत्वं वा न स्यादिति निर्णीतम्~। पुच्छच्छेदेन यस्यकस्यचिदादित्व एव व्येकत्वं कार्यम्~। न पुना राशीनामसङ्ख्येयत्वे~। तत्र व्येकेच्छेदोद्धरणमेव केवलं युक्तम्~। एवमुक्तन्यायेन जीवार्धवर्गे सत्र्यंशं शरवर्गं क्षिप्त्वा ज्यावर्गांशपरम्परा शरवर्गांशपरम्परा च कार्त्स्न्येन गृहीते स्याताम्~। अत एव तन्मूलं\renewcommand{\thefootnote}{१}\footnote{तत्स्थूलं} तदर्धज्याया अर्धात्मकं धनुः स्यात्~। तद्वृत्ते पुनस्तावन्ति धनूंषि यावन्ति स्युः, तावद्गुणिते तस्मिंस्तत्परिधिसङ्ख्यापि स्यात्~। एवं परिध्यानयनमपि तत्र विस्पष्टं स्यात्~। {\qt वृत्ते शरसंवर्ग} इति ज्याछेदविधानेनैव परिध्यानयने सिद्धे लाघवाय विस्पष्टत्वाय चैवमप्युक्तम्~। अत एव तत्कर्म मया गोलसारेऽप्युक्तं\textendash 

\begin{quote}
{\qt सत्र्यंशादिषुवर्गाज्ज्यावर्गाढ्यात् पदं धनुः प्रायः~।}
\end{quote}

\noindent इति~। कथं पुनरंशपरम्परायां कार्त्स्न्येन गृहीतायामप्यस्य प्रायिकत्वम्~। सत्यम्~। चतुरंशपरम्परा कार्त्स्न्येन गृहीता चतुरंशत्वस्यैव~।
तत्परम्परायाश्चतुरंशत्वस्य प्रायिकत्वादेव प्रायिकत्वम्~। प्रायिकत्वमपीषदधिकत्वेनांशानां चतुरंशत आधिक्यादेव तत एवमानीताद्धनुषोऽप्यधिकमेव वास्तवं धनुः~। चापस्याधिक्ये पुनरन्तरं\renewcommand{\thefootnote}{२}\footnote{पुरनन्तरं \textendash\ क. पाठः.} महदेव स्यात्~। अतोऽल्पत्व एवैतत् कर्मयुक्तम्~। अत एवैकस्यात्यल्पत्वमितरस्य शरस्य व्यासात्यासन्नत्वं च यत्र स्यात् तत्रैवेदमुच्यते स्त्र्यंशादिषुवर्गादिति~। तच्च प्राक्तनेन ग्रन्थेनोक्तं,

\begin{quote}
{\qt अर्धज्यादिकमेवं युक्त्यानेयं मुहुर्मुहुर्वृत्ते~।}
\end{quote}

\noindent इति~। तत्र ज्याशराणां सर्वेषां करणीगतत्वात् अवयवोपेक्षया जायमानस्य स्थौल्यस्य न परिहारः~। महावर्गगुणनन्यायेनापि न कृत्स्नपरिग्रहः स्यात्, सौक्ष्म्यतारतम्यमेव स्यात्~। तथाप्यङ्कबाहुल्यात् क्रियावृत्तिबाहुल्याच्च यत्नगौरवात् प्रमादश्च स्यात्~। तन्मा भूदिति प्रथमत
एवातिविप्रकृष्टशरद्वयव्यासकल्पनया तेषामकरणीगतत्वेन चातिसूक्ष्मत्वं क्रियालाघवं च स्यादितीदं सूत्रमारब्धम्~। अपि च ज्याग्रहणे चापगतगन्तव्यांशज्याखण्डानयनं वा तत्समस्तज्यानयनं वानेनैव सिद्ध्यति~। कथम्~। {\qt सत्र्यंशादिषुवर्गादि}त्येतद्द्वारा तत्सिद्धिः~। तत्रार्धज्यावर्गे सत्र्यंशशरवर्गे युक्ते सति चापवर्गः स्यात्~। तेनार्धज्याचापयोर्वर्गान्तरमेवैतत्~। तदेव द्विघ्नया ज्यया द्विघ्नेन चापेन वा

\newpage

\noindent हृत्वाप्तं ज्याचापयोरन्तरं स्यादित्येतत् पूर्वप्रतिपादितन्यायेन सिद्ध्यति~। तत्र द्विघ्नया ज्यया ह्रियमाणे तत्फलवर्गः शेषाच्छोध्यः~। तत्सहितेन हारेण हृत्वा पुनरपि तत्फलं क्षेप्यं वा~। द्विघ्नचापेन ह्रियमाणे तत्फलवर्गः शेषे क्षेप्यः~। यद्वा हारात् फलं विशोध्य तेन हृत्वा पुनरपि तत्फलं शोध्यमित्येवं द्वयी गतिः पूर्वमेव प्रदर्शिता~। ज्याचाप\renewcommand{\thefootnote}{१}\footnote{चापे}योगेन वा ह्रियताम्~। तदा न शो\renewcommand{\thefootnote}{२}\footnote{तदानयनशो \textendash\ ख. पाठः.}धनक्षेपणे कार्ये~। यतो योगान्तराहतिरेव वर्गान्तरम्~। तच्च प्रतिपादितम्~। कथं पुनः शरवर्गो जात इति चात्र निरूपणीयम्~। यस्य सत्र्यंशवर्गस्य ज्याचापयोगेन हरणं कार्यं तन्मूलभूतः शरोऽपि कथमानीत इति~। तस्यैव चापस्य समस्तज्यावर्गाद्व्यासेन हृतः खलु शरः~। स एव स्वेनैव हतः पुनस्तद्वर्गश्च~। तस्मात् समस्तज्यावर्गात् व्यासहृतयोर्द्वयोर्घातः शरवर्गः~। तत्र फलयोः संवर्गे कार्ये हार्ययोर्घातात् हारघातेनापि हरणं कार्यम्, क्रमविशेषात् फलभेदाभावस्य पूर्वमेव प्रदर्शितत्वात्~। तस्मात् समस्तज्यावर्गवर्गो व्यासवर्गेण हृत एव शरवर्गः~। तस्य पुनः स्वत्र्यंशाढ्यत्वाय चतुभिर्गुणनं त्रिभिर्हरणं च कार्यम्~। स्वत्र्यंशचतुष्कं हि सत्र्यंशम्~। तस्मात् त्रिकस्यापि हारकत्वात् व्यासवर्गस्त्रिभिर्गुणितो हार्यः~। यत्पुनश्चतुर्भिर्गुणनं कार्यमिति तेन समस्तज्यावर्गवर्गश्चतुर्गुणितो हार्यः~। यद्वा चतुर्भिर्हृतेन हारकेण समस्तज्यावर्गवर्गो हर्तव्यः~। व्यासवर्गश्चतुर्भिर्हृतो व्यासार्धवर्ग एव~।
तदमाद्व्यासार्धवर्गेणैव हर्तव्यः न व्यासवर्गेण~। तेन हृतो यः समस्तज्यावर्गवर्गः स एव\renewcommand{\thefootnote}{३}\footnote{णैव हृत एव \textendash\ क. पाठः.} त्र्यंशः शरवर्गः~। स पुनर्ज्याचापयोगेनापि हर्तव्यः~। तत्रापि प्रथमं समस्तज्यावर्गवर्गो ज्याचापवर्गेणैव हृत एव वा व्यासार्धवर्गेण हर्तव्यः क्रमभेदेऽपि फलभेदाभावात्~। अर्धज्याचापयोगो हारः प्रायेण द्विघ्नसमस्तज्यातुल्य एव~। तेन समस्तज्यावर्गवर्गो द्विघ्नया समस्तज्यया\renewcommand{\thefootnote}{४}\footnote{ज्या} हृतो व्यासार्धवर्गेण हर्तव्य इत्यायातम्~। एवं हर्तव्ये यदि केवलयैव समस्तज्यया हरणं क्रियते तर्हि हारको द्वाभ्यां हन्तव्यः, यतः समस्तज्या द्विघ्ना\renewcommand{\thefootnote}{५}\footnote{द्विघ्नापि \textendash\ ख. पाठः.} हारक एवेति तस्मिन् कार्यं द्विगुणीकरणमन्यस्मिन् हारके कृ\renewcommand{\thefootnote}{६}\footnote{हृ}तेऽपि फलस्य विशेषाभावात्, त्रिघ्नो व्यासार्ध\renewcommand{\thefootnote}{७}\footnote{व्यास \textendash\ क. पाठः.}वर्गो हार इत्युक्तः~। स पुनर्द्विगुणितः षड्गुणो व्यासार्धवर्गः स्यात्~। स एवास्य हारः~। कस्य~। समस्तज्ययैव केवलया

\newpage

\noindent हृतस्य~। समस्तज्यावर्गवर्गस्यैव वर्गवर्गः पुनर्मूलेन\renewcommand{\thefootnote}{१}\footnote{मूलेनैव \textendash\ ख. पाठः.} हृतस्तस्या एव घनः स्यात्, यतः सदृशचतुष्कसंवर्गो वर्गवर्गः~। तस्मिन् पुनर्मूलेन हृते चतुर्थेन सदृशेन गुणनात् प्राग्यावान् तावानेव पुनस्तेन हत्वा तेनैव हृतेऽपि~। तस्मात् सदृशत्रयसंवर्ग एव वर्गवर्गः स्वमूलेन हृतः~। तस्मात् समसाज्याघनात् षड्गुणितेन व्यासार्धवर्गेण हृतं धनुर्ज्यान्तरमेव~। अत एवोक्तं मया तन्त्रसङ्गहे\textendash 

\begin{quote}
{\qt शिष्टचापघनषष्ठभागतो विस्तरार्धकृतिभक्तवर्जितम्~।\\
शिष्टचापमिह शिञ्जिनी भवेत् स्पष्टता भवति चाल्पतावशात्~॥}
\end{quote}

\noindent इति~। तत्र घनषष्ठांशस्य हार्यत्वे व्यासार्धवर्गस्य षड्गुणनं न कार्यं षड्भिर्हतत्वाद्धार्यस्य~। समस्तज्याघन इत्येव पूर्वोक्तन्यायेनायातं तत् किं शिष्टचापघन इत्युक्तम्~। अर्धज्याचापयोगार्धात् समस्तज्यायाः सदैवाधिक्यमेव स्यात्~। ततस्तया हृतफलस्याल्पत्वं स्यात्~। तस्माद्घनादीषदधिकस्यैव\renewcommand{\thefootnote}{१}\footnote{स्यैव तस्यैव \textendash\ क. पाठः.} हार्यत्वं युक्तम्~। खण्डज्यान्तरदृष्टेन न्यायेनापि हार्यस्य घनादीषदधिकत्वं सिद्धम्~। कः पुनः खण्डज्या(न्तर)दृष्टो न्यायः~। निरन्तरचापद्वयखण्डान्तरमेकाद्येकोत्तरतया दृष्टम्~। ततः प्रथमचापज्यार्धाद्द्वितीयादिज्याखण्डस्य न्यूनांशा एकाद्येकोत्तरसङ्कलिततुल्या एव~। ततो
द्वित्र्यादिचापानां तत्पिण्डज्यायाश्चान्तरमेकाद्येकोत्तरसङ्कलितैक्यतुल्यम्~। तच्च घनादीषदधिकम्~। यस्मात् तदानयनमेवमुक्तम्,

\begin{quote}
{\qt एकोत्तराद्युपचितेर्गच्छाद्येकोत्तरत्रिसंवर्गः~।\\
षड्भक्तः स चितिघनः सैक(घनपदो ? पदघनो) विमूलो वा~॥}
\end{quote}

\noindent इति~। कः पुनः खण्डज्यान्तराणामेकाद्येकोत्तरत्वे हेतुः~। भुजानुसारेण वर्धमानत्वमेव तद्धेतुः~। ननु भुजाचापसमानवृद्धित्व एव एकाद्येकोत्तरत्वं युक्तम्~। नैव चापानुसारिणी खण्डज्यान्तरवृद्धिः भुजाज्यानुसारिण्येव, यतो भुजाज्याया एवात्रेच्छत्वं न तु चापस्य~। तस्मात् प्रतिचापखण्डमस्य फलस्य तुल्यत्वाभावाच्चयस्य क्रमेण न्यूनत्वमेव स्यात्, न पुनः श्रेढीक्षेत्रवदाद्यन्तमेकरूप एव~। चापज्याखण्डानां पुनःपुनर्न्यूनत्वात् तदनुसारेणैवास्य चयोऽपि~। तच्चापखण्डेष्वेव द्रष्टव्यम्~। ञ्ख्यन्तमेव ह्येकोत्तरत्वम्~। णखि-

\newpage

\noindent ञ\renewcommand{\thefootnote}{१}\footnote{म}ख्योर्यावदन्तरं तावदेव पुनर्ञ\renewcommand{\thefootnote}{२}\footnote{र्न}खिङ\renewcommand{\thefootnote}{३}\footnote{म \textendash\ क. पाठः.}ख्योरपि~। एवं फछयोरन्तरमपि पञ्चदशैव~। एकोत्तरत्वे त्रयोविंशचतुर्विंशयोरन्तरेण त्रयोविंशत्या भाव्यम्~। यस्मात् प्रथमद्वितीयान्तरमेकं ततः पुनः क्रमेणैकोत्तरत्वेऽन्तरेषु त्रयोविंशं हि तत्, तस्मात् पुनःपुनश्च यस्य न्यूनत्वात् तत्तद्धनेच्छायास्त्रेराशि(का)नीतं कृत्स्नं न त्याज्यं स्यात्~। तस्मात् पुनःपुनः घनान्न्यूनयैवेच्छया भाव्यम्~। अत एवाह माधवः\textendash 

\begin{quote}
{\qt विद्वांस्तुन्नबलः कपीशनिचयः सर्वार्थशीलस्थिरो\renewcommand{\thefootnote}{४}\footnote{स्थितो \textendash\ ख. पाठः.}\\
निर्विद्धाङ्गनरेन्द्ररुङ्निग(ति ? दि)तेष्वेषु क्रमात् पञ्चसु~।\\
आधस्त्याद्गुणितादभीष्टधनुषः कृत्या विहृ\renewcommand{\thefootnote}{५}\footnote{कृ}त्यान्तिम-\\
स्याप्तं शोध्यमुपर्युपर्यथ घने नैवं धनुष्यन्ततः~॥}
\end{quote}

\noindent इति~। तस्मात् पदादावेव घनात् त्रैराशिकानीतादाधिक्यं ज्याचापान्तरस्य, विमूलसैकपदघनानुसारित्वं च~। उपर्युपरि घनायातादल्पत्वं च दृष्टम्~। तस्मात् क्वचित् साम्येनापि भाव्यम्~। तच्च पदादिसमीप एव~। तस्माद्राश्यष्टमांशचापादूनस्य चापस्य स्वघनात् त्रैराशिकानीतं\renewcommand{\thefootnote}{६}\footnote{त} ज्याचापान्तरं\renewcommand{\thefootnote}{७}\footnote{र} सूक्ष्ममेव~। तत उपर्युपर्येव स्थौल्यम्~। तदुक्तं \textendash\ {\qt स्पष्टता भवति चाल्पतावशादि}ति~। तस्मादूनाधिकधनुषोर्ज्याचापान्तरानयने चापघनाद्व्यासार्धवर्गेण षड्गुणितन हृत्वाप्तम् ऊनाधिकधनुष एव त्यक्त्या शिष्टं तस्यार्धज्यैव~। ततस्तस्याश्च पठितज्या(या)श्च मिथः कोटिहतयोर्योगवियोगतोऽभी(ष्ट)ज्यानयनमप्यनेनैव सिद्धम्~। का पुनरेकोत्तरादिसूत्रस्य युक्तिः~। तत्सर्वं तत्र द्रष्टव्यम्~। यन्मयात्र केषाञ्चित् सूत्राणां तद्युक्तीः प्रतिपाद्य कौषीतकिनाढ्येन नारायणाख्येन व्याख्यानं कारितम् अतस्तदेवात्र लिख्यते~। अथ समस्तगतप्रदीपस्य दीपस्तम्भस्य समीपवर्तिनः शङ्कोश्छायानयनोपायमाह\textendash 

\begin{quote}
{\qt शङ्कुगुणं शङ्कुभुजाविवरं शङ्कुभुजयोर्विशेषहृतम्~।\\
यल्लब्धं सा छाया ज्ञेया शङ्कोः स्वमूलाद्धि~॥}
\end{quote}

\noindent इति~। ततः दीपस्तम्भस्य शङ्कोस्तयोरन्तरालस्य चेयत्ता(या)मवगतायामज्ञाताया\renewcommand{\thefootnote}{८}\footnote{यां \textendash\ क. पाठः.}श्छायाया आनयनं क्रियत इति द्रष्टव्यम्~। शङ्कुभुजाविवरमिति

\newpage

\noindent भुजाशब्देन तद्रूपेण स्थितो दीपस्तम्भ उच्यते~। विवरशब्दोऽन्तरालवचनः~। शङ्कुमूलदीपस्तम्भमूलयोरन्तरालमित्यर्थः~। विशेषो विश्लेषः~। इयमत्रोपपत्तिः \textendash\ तुल्यस्वभावे क्षेत्रान्तरे ज्ञाताभ्यां भुजाकोटिभ्यां क्षेत्रान्तरगतायाः को\renewcommand{\thefootnote}{१}\footnote{गतायाः पुनस्तत्क्षेत्रं को\textendash\ ख. पाठः.}टेरानयनमेवात्र क्रियते, यथा महाशङ्कुमहाछायाभ्यां द्वादशाङ्गुलशङ्कोः छायायाः~। किं पुनस्तत्क्षेत्रं, यच्छङ्कुभुजकेन छायाकोटिकेन क्षे\renewcommand{\thefootnote}{२}\footnote{भुजकेन क्षे \textendash\ क. पाठः.}त्रेण तुल्यस्वभावम्~। उच्यते~। शङ्कुपरिमितात् प्रदेशादूर्ध्वगतो यो भागो दीपस्तम्भस्य सोऽत्र भुजा~। स पुनः शङ्कुदीपस्तम्भयोर्विश्लेष एव~। शङ्क्वग्रभुजामूलान्तरालं कोटिः~। तच्च शङ्कुमूलदीपस्तम्भमूलयोरन्तरालमेव, तयोरन्तरालस्य सर्वत्र तुल्यत्वात्~। ततस्ताभ्यां त्रैराशिकं \textendash\ यदि क्षेत्रान्तरगताया
भुजाया इयती कोटिः, तदेष्टभुजायाः कियतीति~। कथं पुनरनयोः क्षेत्रयोस्तुल्यस्वभावत्वम्~। उच्यते~। एतदेव हि सर्वत्रापि
क्षेत्रयोस्तुल्यस्वभावत्वं, यदुभयत्रापि भुजाकोट्योर्वृद्धिह्रासयोरैकरूप्यम्~। एकस्मिन् क्षेत्रे कोटेर्यावत्याधिक्ये न्यूनत्वे वा जाते भुजाया यावदाधिक्यं न्यूनत्वं वा जातम्, अन्यस्मिन्नपि कोटेस्तावत्याधिक्ये न्यूनत्वे वा जाते भुजाया अपि तावदाधिक्यं न्यूनत्वंं वा यदि स्यात् तर्हि तयोः क्षेत्रयोस्तुल्यस्वभावत्वं स्यात्~। इतरथा न~। तच्च कर्णादिना~। यदि कर्णस्य (तर्यञ्च ? तिर्यक्त्व)मुभयत्रापि समानं स्यात् तर्हि भुजाकोट्योर्वृद्धिह्रासयोरेकरूपत्वमपि स्यात्~। अन्यथा न~। तद्यथा \textendash\ यस्मिन् क्षेत्रद्वये भुजाकोट्योर्वृद्धिह्रासयोरैकरूप्यं तत्र भुजाकोट्योर्विभागोऽपि तुल्य एव स्यात्~।~। यद्येकस्मिन् कोट्यर्धतुल्या भुजा तर्ह्यन्यस्मिन्नपि कोट्यर्धतुल्यैव भुजा~। यदि वैकस्मिन् कोटेस्त्र्यंशतुल्या भुजा तर्ह्यन्यस्मिन्नपि कोटेस्त्र्यंशतुल्यैव भुजा~। एवं विभागान्तेरष्वपि द्रष्टव्यम्~। कथं पुनरिदमवसीयत इति चेत्~। उच्यते~। एकस्मिन् क्षेत्रे यावत्यां कोट्यां यावती भुजा, अन्यस्मिन्नपि तावत्यां कोट्यां तावत्येव भुजा भवति, उभयत्रापि भुजाकोट्योरेकेनैव प्रकारेण प्रवृद्धत्वात्~। अतस्तयोर्विभागोऽपि तुल्य एव स्यात्~। अतो यस्मिन् क्षेत्रद्वये भुजाकोट्योर्विभागस्य तुल्यत्वं तत्र तयोर्वृद्धिह्रासयोरैकरूप्यमस्त्येवेति निश्चीयते~। भुजाकोटिविभागतुल्यत्वं च कर्णतिर्य(क्त्वो ? क्त्वं चो)भयत्रापि तुल्यत्व एव भवति~। तदिदं छेद्यके

\newpage

\noindent प्रदर्श्यते~। समे भूपृष्ठे विशंत्यङ्गु(लं ? ल)परिमितां समपूर्वापरदिशा स्थितां रेखां कृत्वा तस्याः पश्चिमाग्रात् प्रवृत्तां समदक्षिणोत्तरदिशा स्थितां दशाङ्गुलपरिमितां रेखां कुर्यात्~। तत्र प्रथमा कोटिः, द्वितीया भुजा~। पुनः कोट्यग्रात् प्रवृत्तां भुजाग्रप्रापिणीं रेखां कुर्यात्, सा\renewcommand{\thefootnote}{१}\footnote{स} कर्णः~। अत्र कोट्यर्धतुल्या भुजा~। पुनः कर्णपरिमितामृज्वीं काञ्चिच्छलाकां कर्णरेखायां विन्यस्य तस्याः कोट्या संस्पृ\renewcommand{\thefootnote}{२}\footnote{सृ \textendash\ ख. पाठः.}ष्टमग्रं वा भु(जा ? ज)या संस्पृ\renewcommand{\thefootnote}{३}\footnote{सृ \textendash\ ख. पाठः.}ष्टमग्रं वाङ्गुल्यावष्टभ्यान्यदग्रं भुजाकोटिसम्पाताभिमुखं किञ्चित् सारयेत्~। तत्र यदि भु(जायाः ? जया) संस्पृ\renewcommand{\thefootnote}{४}\footnote{सृ \textendash\ ख. पाठः.}ष्टमग्रं सार्येत तदा कर्णस्य तिर्यक्त्वं किञ्चिन्न्यूनं भवति~। यदि पुनः कोट्यासंस्पृ\renewcommand{\thefootnote}{५}\footnote{सृ \textendash\ ख. पाठः.}ष्टमग्रं सार्येत तदा कर्णतिर्यक्त्वं किञ्चिदधिकं स्यात्~। तथा भुजया संस्पृ\renewcommand{\thefootnote}{६}\footnote{सृ \textendash\ ख. पाठः.}ष्टस्याग्रस्य सारेण कृते भुजायाः किञ्चिन्न्यूनत्वं स्यात्, कोट्या संस्पृ\renewcommand{\thefootnote}{७}\footnote{सृ \textendash\ ख. पाठः.}ष्टस्याग्रस्य सारेण कोटेः~। तेन कर्णतिर्यक्त्वस्य न्यूनत्वेऽधिकत्वे च तस्मिन् क्षेत्रे भुजाकोट्योर्विभागः पूर्वक्षेत्रवन्न स्यात्~। अत एव तयोर्वृद्धिह्रासावपि पूर्वक्षेत्रवन्न स्याताम्~। तस्मात् कर्णतिर्यक्त्वस्य वैलक्षण्यात् पूर्वक्षेत्रविलक्षणमेवैतत् क्षेत्रम्~। यदि पुनः कर्णरेखायां विन्यस्तां शलाकां तिर्यक्त्वमबाधित्वा भुजाकोटिसम्पाताभिमुखं किञ्चिदुत्सार्य क्षेत्रान्तरं
सम्पाद्येत, तदा भुजाकोट्योरग्राभ्यां किञ्चिदध एव शलाकायाः संयोगः स्यात्~। तेन भुजाकोट्योरुभयोरप्यत्र किञ्चिन्न्यूनत्वं जातम्~। तत्र
कोटेर्यावन्न्यूनत्वं जातं तदर्धतुल्येमव भुजाया न्यूनत्वम्~। कथं पुनरिदमवसीयते~। उच्यते~। यदि शलाकातिर्यक्त्वमबाधित्वैव भुजाकोटिसम्पातावधि सार्येत तर्हि भुजाकोट्योरुभयोरपि साकल्येनैव ह्रासः स्यात्~। तत्र कोटिह्रासार्धतुल्यो भुजाह्रासः, कोट्यर्धतुल्यत्वाद्भुजायाः~। अतो भुजाकोटिसंपातप्राप्तेः प्रागपि कोटिह्रासार्धतुल्यो भुजाह्रास इति निश्चीयते~। अतो द्वितीये क्षेत्रेऽपि प्रथमक्षेत्रतुल्य एव भुजाकोट्योर्विभागः,
विभागानुसारेणैव ह्रासस्य जातत्वात्~। एष एव न्यायो विभागान्तरेष्वपि द्रष्टव्यः~। तदेवं सर्वत्रापि कर्णतिर्यक्त्वस्य तुल्यत्वे भुजाकोटिविभागस्यापि
तुल्यत्वमेव स्यात्~। अत एव भुजाकोट्योर्वृद्धिह्रासयोरपि तुल्यत्वं स्यात्~। अत्र च क्षेत्रद्वयगतयोः कर्णयोस्तिर्यक्त्वं तुल्यमेव~।
दीपस्तम्भाग्रच्छायाग्रान्तराल-

\newpage

\noindent वर्ति यत् सूत्रं दीपस्तम्भभुजकस्य दीपस्तम्भमूलच्छायाग्रान्तरालकोटिकस्य क्षेत्रस्य कर्णात्मकं तस्य शङ्क्वग्रादधोगतो यो भागः स एवात्रैकस्य क्षेत्रस्य कर्णः~। शङ्क्वग्रादूर्ध्वगतो यो भागः तस्य स एवान्यस्य कर्णः~। अत एकस्यैव कर्णस्यांशावेव तौ कर्णौ~। तस्मात् तयोस्तिर्यक्त्वं तुल्यमेव भवति~। अत एव तयोः क्षेत्रयोस्त्रैराशिकयोग्यत्वमप्यस्त्येव~। तस्मात् सर्वमवदातम्~। अथ दीपस्तम्भसमीपवर्तिनोस्तुल्यपरिमाणयोर्द्वयोः शङ्क्वोः छायाभ्यां छायाग्रयोरन्तरालेन शङ्कुभ्यां चाज्ञातयोर्भुजाकोट्योरानयनोपायमाह\textendash 

\begin{quote}
{\qt छायागुणितं छायाग्रविवरमूनेन भाजिता कोटी~।\\
शङ्कुगुणा कोटी सा छायाभक्ता भुजा भवति~॥}
\end{quote}

\noindent इति~। छायाग्रयोरन्तरालमभीष्टया छायया गुणितं कृत्वा तस्मादूनेन छाययोर्विश्लेषेण विभज्य लब्धा कोटिर्भवति\renewcommand{\thefootnote}{१}\footnote{कोटी भवति \textendash\ ख. पाठः.}, यया छायया छायाग्रविवरं गुणितं तदग्रस्य दीपस्तम्भमूलस्य चान्तरालं भवतीत्यर्थः~। सा कोटी पुनः शङ्कुना गुणिता तच्छायया विभक्ता भुजा भवति, दीपस्तम्भपरिमाणं भवति~। अत्रैकः शङ्कुः दीपस्तम्भसमीपवर्ती, अन्यस्तु ततः किञ्चित् विप्रकृष्टदेशस्थः~। एते त्रयोऽप्येकसूत्रगताश्चेति द्रष्टव्यम्~।
अत्र वासना छेद्यके प्रदर्श्यते \textendash\ समायामवनौ समदक्षिणोत्तरदिशा स्थितामशीत्यङ्गुलपरिमितां रेखामालिख्य तस्या उत्तराग्रात् प्रवृत्तां प्रागग्रां षष्ट्यङ्गुलपरिमितां रेखां कुर्यात्~। तत्र प्रथमाधिका कोटिः~। द्वितीया दीपस्तम्भः~। पुनर्दीपस्तम्भमूलाद्दक्षिणतश्चत्वारिंशदङ्गुलपरिमितात्
चतुष्षष्ट्यङ्गुलपरिमिताच्च प्रदेशात्\renewcommand{\thefootnote}{२}\footnote{च प्रदेशात्} प्रवृत्ते प्रागग्रे द्वादशाङ्गुलपरिमिते रेखे~। तयोः शङ्कुत्वं च द्रष्टव्यम्~। पुनरासन्नस्य शङ्कोर्मूलाद्दक्षिणतो दशाङ्गुलपरिमितात् प्रदेशात् प्रवृत्तां तच्छङ्क्वग्रसंस्पर्शिनीं दीपस्तम्भाग्रप्रापिणीं रेखां कृत्वा कोटेर्दक्षिणाग्रात्\renewcommand{\thefootnote}{३}\footnote{दक्षिणात् \textendash\ क. पाठः.} प्रवृत्तां विप्रकृष्टशङ्क्वग्रसंस्पर्शिनीं दीपस्तम्भाग्रप्रापिणीमेव रेखां कृत्वा कुर्यात्~। तयोः कर्णत्वं च द्रष्टव्यम्~। उभयत्रापि कर्णशङ्कुमूलान्तरालं छाया~। पुनर्विप्रकृष्टस्य शङ्कोर्मूलाद्दक्षिणतो दशाङ्गुलपरिमितात् प्रदेशात् प्रवृत्तां तच्छङ्क्वग्रप्रापिणीं रेखामालिखेत्~। तस्या दक्षिणतः स्थितो यो दक्षिणोत्तररेखाभागः स छाययोर्विश्लेषः~। अत्र विप्रकृष्टशङ्कुभुजकं तच्छायाकोटिकं यदवान्तरं\renewcommand{\thefootnote}{४}\footnote{र \textendash\ ख. पाठः.} क्षेत्रं तन्महाक्षेत्रेण तुल्यस्वभावम्, अस्य कर्णस्य

\newpage

\noindent महाक्षेत्रैकदेशत्वेनोभयत्रापि कर्णतिर्यक्त्वस्य तुल्यत्वात्~। यः पुनरस्य शङ्कुमूलाद्दशाङ्गुलतः प्रवृत्तेन शङ्क्वग्रप्रापिणावान्तरकर्णेन कृतो
विभागः स च सन्निकृष्टशङ्कुमूलाद्दशाङ्गुलतः प्रवृत्तेन दीपस्तम्भाग्रप्रापिणा महाक्षेत्रावान्तरकर्णेन कृतेन महाक्षेत्रविभागेन सरूप एव स्यात्, अवान्तरकर्णयोरपि तिर्यक्त्वस्य तुल्यत्वात्~। तथाहि \textendash\ सन्निकृष्टस्य शङ्कोरग्रादधोगतो यो भागो महाक्षेत्रावान्तरकर्णस्य तस्य
तावदवान्तर\renewcommand{\thefootnote}{१}\footnote{रक \textendash\ ख. पाठः.}क्षेत्रावान्तरकर्णतुल्यमेव तिर्यक्त्वम्~। कथम्~। स खलु सन्निकृष्टशङ्कुभुजकस्य तच्छायाकोटिकस्य क्षेत्रस्य कर्ण एव~। अवान्तरक्षेत्रावान्तरकर्णोऽपि तथाविधस्यैव क्षेत्रस्य कर्णः~। विप्रकृष्टशङ्कोः सन्निकृष्टशङ्कुतुल्यत्वात्  तच्छायातुल्यत्वाच्च तत्कोटेः\renewcommand{\thefootnote}{२}\footnote{तुल्यत्वात् तत्कोटेः \textendash\ क. पाठः.}~। अतो भुजाकोट्योस्तुल्यत्वात् कर्णयोस्तिर्यक्त्वमपि तुल्यमेव स्यात्~। अत एव शङ्क्वग्रादूर्ध्वगतस्यापि भागस्यावान्तरक्षेत्रावान्तरकर्णतुल्यमेव तिर्यक्त्वम्~। नह्येकस्मिन्नेव कर्णतिर्यक्त्वं भिद्यते~। अतो महाक्षेत्रावान्तरकर्णस्यावान्तरक्षेत्रावान्तरकर्णस्य च तिर्यक्त्वं तुल्यमेवेति स्थितम्~। तेन तत्कृतः कोटेर्विभागश्च क्षेत्रद्वयेऽपि सरूप एव स्याद्, अवान्तरकर्णतिर्यक्त्वाधीनत्वात् तस्य~। यदि महाक्षेत्रावन्तरकर्णोभयपार्श्वगतौ कोटिभागौ
तुल्यौ स्यातां तर्ह्यवान्तरक्षेत्रावान्तरकर्णोभयपार्श्वगतौ कोटिभागावपि तुल्यावेव स्याताम्~। यदि पुनर्महाक्षेत्रावान्तरकर्णोत्तरतः स्थितः कोटिभागो
दक्षिणतः स्थिताद्द्विगुणः तर्ह्यवान्तरक्षेत्रावान्तरकर्णोत्तरतः स्थितः कोटिभागोऽपि दक्षिणतः स्थिताद्द्विगुण एव भवति~। एवं सर्वेष्वपि विभागेषु
सारूप्यं द्रष्टव्यम्~। तस्मात् क्षेत्रयोस्तद्विभागस्य च सरूपत्वात् त्रैराशिकयोग्यत्वमनयोरस्त्येवेति निश्चीयते~। तत्रावान्तरक्षेत्रे कर्णोभयपार्श्वगतौ
कोटिभागौ ज्ञातौ, दक्षिणतः स्थितः छाययोर्विश्लेषः उत्तरतः स्थितश्च सन्निकृष्टशङ्कुच्छायातुल्यत्वात् तज्ज्ञानेनैवागतः~। महाक्षेत्रे तु कर्णाद्दक्षिणतः स्थितः कोटेर्भागो ज्ञातः, तस्य छायाग्रविवरात्मकत्वात्~। तत इदं त्रैराशिकं \textendash\ यद्यवान्तरक्षेत्रे कर्णाद्दक्षिणतः स्थितेन छायाविश्लेषात्मकेन कोटिभागेन विप्रकृष्टशङ्कुच्छायातुल्या सकला कोटिर्लभ्यते, तदा महाक्षेत्रे कर्णाद्दक्षिणतः स्थितेन छायाग्रविवरात्मकेन कोटिभागेन कियती सकला कोटिरिति~। तत्र महाक्षेत्र-

\newpage

\noindent गता सकला कोटिर्लभ्यते~। अत्र छायाविश्लेषः प्रमाणराशिः~। विप्रकृष्टशङ्कुच्छाया फलराशिः~। छायाग्रविवरमिच्छाराशिः~। अथापरं
त्रैराशिकं \textendash\ यद्यवान्तरक्षेत्रे कर्णाद्दक्षिणतः स्थितेन कोटिभागेनोत्तरतः स्थितः सन्निकृष्टशङ्कुच्छायातु(ल्या ? त्यः) कोटिभागो लभ्यते, तदा महाक्षेत्रे कर्णाद्दक्षिणतः स्थितेन कोटिभागेनोत्तरतः स्थितः कियानिति~। अत्र सन्निकृष्टशङ्कुच्छायावधिकः कोटिभागो लभ्यते~। अत्रापीच्छाप्रमाणराशी पूर्वोक्तावेव\renewcommand{\thefootnote}{*}\footnote{इह क-ख मातृकयोः कियांश्चिदंशो लुप्तः~। ख-मातृकायां लुप्तस्य स्थाने प्रदेशान्तरस्थं किञ्चिद्वाक्यजातं प्रक्षिप्तं च~।} ते सम्बद्धाग्रे रेखे कुर्यात्~। तथाकृते तत् पञ्चाङ्गुलकोटिकं क्षेत्रं भवति~। पुनस्तदीशकोणसम्बद्धनिर्ऋतिकोणं विंशत्यङ्गुलभुजाकोटिकं क्षेत्रमालिखेत्~। पुनस्तदीशकोणात् प्रवृत्ते समपूर्वापरदिशा समदक्षिणोत्तरदिशा च स्थिते शताङ्गुलायते क्षेत्रे पूर्वोत्तररेखाप्रापिण्यौ रेखे कुर्यात्~। तच्छताङ्गुलभुजाकोटिकं क्षेत्रं भवति~। एवमेतानि त्रीण्यपि पूर्वलिखितमहाक्षेत्रान्तर्वर्तीन्यवान्तरक्षेत्राणि भवन्ति~। तत्र प्रथमं क्षेत्रं पञ्चभुजाकोटिकत्वात् पञ्चविंशत्युत्तरशतसङ्ख्यस्य राशेः प्रथमपदवर्गक्षेत्रं भवति~। द्वितीयं विंशतिभुजाकोटिकत्वात् द्वितीयपदवर्गक्षेत्रम्~। तृतीयं शतभुजाकोटिकत्वात् तृतीयपद वर्गक्षेत्रमिति द्रष्टव्यम्~। एवमेतानि त्रीण्यपि वर्गक्षेत्राणि~। अथावर्गक्षेत्राणि प्रदर्श्यन्ते \textendash\ द्वितीयवर्गक्षेत्रवायव्यकोणात् प्रवृत्ते शताङ्गुलपञ्चाङ्गुलायते महाक्षेत्रोत्तरपश्चिमरेखाप्रापिण्यौ रेखे कुर्यात्~। पुनस्तदाग्नेयकोणात् प्रवृत्ते महाक्षेत्रपूर्वदक्षिणरेखाप्रापिण्यौ पूर्वरेखातुल्ये रेखे कुर्यात्~। तथा सति षड\renewcommand{\thefootnote}{१}\footnote{षड्}वर्गक्षेत्राणि भवन्ति~। प्रथमवर्गक्षेत्रादुत्तरतः पञ्चभुजकं
विंशतिकोटिकं तदुत्तरतः पञ्चभुजकं शतकोटिकम्~। प्रथम\renewcommand{\thefootnote}{२}\footnote{भुजकं प्रथम}वर्गक्षेत्रात् पूर्वतोऽप्येवम्~। द्वितीयवर्गक्षेत्रादुत्तरतः पूर्वतश्च विंशतिभुजके शतकोटिके\renewcommand{\thefootnote}{३}\footnote{भुजके कोटिके \textendash\ क. पाठः.}~। एवं
वर्गक्षेत्रैः सह नव क्षेत्राणि भवन्ति~। अथवा पूर्ववन्महाक्षेत्रमालिख्य तन्निर्ऋतिकोणादुत्तरतः पञ्चाङ्गुलपरिमितात् पञ्चविंशत्यङ्गुलपरिमिताच्च प्रदेशात्
प्रवृत्ते समपूर्वापरायते क्षेत्रपूर्वरेखाप्रापिण्यौ रेखे कुर्यात्~। पुनस्तस्मादेव कोणात् पूर्वतः पञ्चाङ्गुलपरिमितात् पञ्चविंशत्यङ्गुलपरिमिताच्च प्रदेशात्
प्रवृत्ते सम-

\newpage

\noindent दक्षिणोत्तरायते क्षेत्रोत्तररेखाप्रापिण्यौ रेखे कुर्यात्~। एवं कृते पूर्वप्रदर्शितानि क्षेत्राणि भवन्ति~। तत्र कोणदिशा स्थितानि समचतुरश्राणि त्रीणि वर्गक्षेत्राणि, शिष्टान्यवर्गक्षेत्राणि~। एतेषामवान्तरक्षेत्राणां क्षेत्रफलानि {\qt समद्विघात} इत्युक्तन्यायेन सम्पाद्यन्ते~। तेषु सम्पादितेषु महाक्षेत्रक्षेत्रफलान्यपि सम्पादितान्येव, अवान्तरक्षेत्रपरिपूर्णत्वान्महाक्षेत्रस्य~। तत्र {\qt स्थाप्योऽन्त्यवर्ग} इत्यन्त्यवर्गे स्थापिते
शतभुजाकोटिकस्यान्त्यवर्गक्षेत्रस्यायुतसङ्ख्यानि क्षेत्रफलानि भवन्ति~। यद्यपि शतस्थाने स्थापितत्वेनास्य शतसङ्ख्यत्वं, तथा{\qt प्युत्सार्य पुनश्च राशिमि}तीतरयोः पदयोर्वर्गीकरणे द्विरुत्सारणे कृतेऽयुतसङ्ख्यत्वं भवत्येव~। ये पुनरन्त्यवर्गक्षेत्रस्य समपश्चिमतः स्थिते अवर्गक्षेत्रे तयोः प्रथममन्त्यवर्गक्षेत्रतुल्यायामत्वात् शतकोटिकं, द्विती(य ? यं) वर्गक्षेत्रविस्तारत्वाद्विंशतिभुजकं च भवति~। तेन शतगुणिता विंशतिः तत्क्षेत्रक्षेत्रफलं द्विसहस्रसङ्ख्यं भवति, भुजाकोटिसंवर्गः क्षेत्रफलमिति प्राक् प्रदर्शितत्वात्~। अतोऽन्त्यपदेन शतसङ्ख्येन विंशतिसङ्ख्ये द्वितीयपदे गुणिते तत्क्षेत्र\renewcommand{\thefootnote}{१}\footnote{क्षेत्रे}क्षेत्रफलं भवति~। द्वितीयमप्यन्त्यवर्गक्षेत्रतुल्यायामत्वात् शतकोटिकं प्रथमवर्गक्षेत्रतुल्यविस्तारत्वात् पञ्चभुजकं च भवति~। तेन शतगुणिता पञ्चसङ्ख्या तत्क्षेत्रफलं पञ्चशतसङ्ख्यं भवति~। अतस्तत्रान्त्यपदगुणितं प्रथमपदं क्षेत्रफलम्~। एवमन्त्यपदेनेतरपदद्वये गुणिते क्षेत्रद्वयसम्बन्धि क्षेत्रफलं स्यात्~। एवमेवान्त्यवर्गक्षेत्रस्य समदक्षिणतः स्थितयोरपि क्षेत्रयोः क्षेत्रफलानि सम्पादनीयानि, पूर्वोक्तक्षेत्रद्वयतुल्यभुजाकोटितुल्यत्वात् तयोः~। तेन गुण्यो राशिः द्विगुणः कर्तव्यः~। तत्र गुण्ये द्विगुणिते गुणकारे द्विगुणिते वा फलवैषम्याभावाद्गुणकारस्य द्विगुणनं कृतम्~। तस्माद्द्विगुणितेनान्त्यपदेनेतरपदद्वयं हत्वा स्वस्वोपरि स्थापयेत्~। तथा सति क्षेत्रचतुष्टयसम्बन्धि क्षेत्रफलं स्यात्~। तदिदमुक्तं {\qt द्विगुणान्त्यनिघ्नाः स्वस्वोपरिष्टाच्च तथापरेऽङ्का} इति~। अत्राप्युत्सारणेन\renewcommand{\thefootnote}{२}\footnote{सारितेन \textendash\ क. पाठः.} शतगुणितत्वं द्रष्टव्यम्~। पुनरुत्सार्य द्वितीयपदवर्गे स्थापिते द्वितीयवर्गक्षेत्रक्षेत्रफलं स्यात~। द्वितीय\renewcommand{\thefootnote}{३}\footnote{फलं द्वितीय \textendash\ ख. पाठः.}वर्गक्षेत्रात् पश्चिमतो दक्षिणतश्च स्थितस्य द्वितीयवर्गक्षेत्रकोटिकस्य प्रथमवर्गक्षेत्रभुजकस्य क्षेत्रद्वयस्य क्षेत्रफलानयनं पूर्ववद्द्रष्टव्यम्~। पुनरप्युत्सार्य प्रथमवर्गक्षेत्रसम्बन्धि क्षेत्रफलं

\newpage

\noindent स्यात्~। एवं कृते महाक्षेत्रमपि सम्पूर्णं स्यात्~। एवमन्यत्रापि द्रष्टव्यम्~। यद्वा समद्विघात इत्याद्यया प्रक्रियया तुल्यसङ्ख्ययो राश्योः संवर्ग एव क्रियते~। तथाहि \textendash\ पञ्चविंशत्यधिकशतसङ्ख्यस्य राशेः स्वेनैव गुणने क्रियमाणे प्रथममन्त्यपदमन्त्यपदेन गुणनीयम्~। तत्रान्त्यपदयोस्तुल्यसङ्ख्यत्वाद् अन्त्यपदवर्गस्तयोराहतिः स्यात्~। तदेतदुक्तं \textendash\ {\qt स्थाप्योऽन्त्यवर्ग} इति~। पुनस्तदनन्तरे स्थानेऽन्त्यपदेन गुणितं द्वितीयपदं स्थाप्यं, तदनन्तरे चान्त्यपदेन गुणितं प्रथमपदम्~। पुनर्गुणकारराशिमुत्सार्य द्वितीयान्त्ययोः संवर्गः स्थाप्यः~। स च गुणकारस्योत्सारितत्वाद् अन्त्ययोः पदयोः संवर्गो यत्र स्थापितस्तदनन्तर एव स्थाने स्थाप्यः~। एवमन्त्यपदसंवर्गादनन्तरे स्थाने द्वितीयन्त्ययोः संवर्गो द्विः स्थाप्यो जातः~। अनेन न्यायेन ततोऽप्यनन्तरे स्थाने प्रथमान्त्ययोः संवर्गोऽपि द्विः स्थाप्यः स्यात्~। तदिदमुक्तं \textendash\ {\qt द्विगुणान्त्यनिघ्नाः स्वस्वोपरिष्टाच्च तथापरेऽङ्का} इति~। एवं कृते गुण्यस्यान्त्यं पदं सकलेन गुणकारेण द्विगुणितं\renewcommand{\thefootnote}{१}\footnote{गुणितं \textendash\ ख. पाठः.} भवति~। इतरपदद्वयमपि
गुणकारस्यान्त्येन पदेन गुणितं स्यात्~। पुनर्द्वितीयस्य द्वितीयेन प्रथमेन च गुणनं प्रथमस्य द्वितीयेन प्रथमेन च गुणनं परिशिष्टं जातम्~। तत्र द्वितीयवर्गस्थापनेन द्वितीयस्य द्वितीयेन गुणनं क्रियते तुल्यसङ्ख्यत्वात् तयोः~। स च प्रथमस्थापितादन्त्यवर्गादेकान्तरिते स्थाने स्थाप्यो भवति, द्वितीयेन द्वितीये गुण्यमाने गुण्यस्य गुणकारस्य चान्त्यात् पदात् अनन्तरत्वेनान्त्यसंवर्गाद् अनन्तरे स्थाने स्थापितस्य द्वितीयान्त्यसंवर्गस्याप्यनन्तरे स्थाने स्थाप्यत्वात् तत्संवर्गस्य~। तदेतदाह \textendash\ {\qt उत्सार्य पुनश्च राशि}मिति~। पुनर्द्विगुणितेन द्वितीयपदेन निहते प्रथमपदे स्वोपीर स्थापिते पूर्वोक्तन्यायेन द्वितीयप्रथमयोः प्रथमद्वितीययोश्च पदयोराहतिः स्थापिता स्यात्~। प्रथमयोः संवर्गस्य द्वितीययोः संवर्गादेकान्तरिते स्थाने स्थाप्यत्वात् प्रथमपदं पुनरप्युत्सार्य तद्वर्गे स्वोपरि स्थापिते प्रथमपदयोराहतिः स्थापिता स्यात्~। एवं सर्वत्र द्रष्टव्यम्~। तदेवं सर्वत्र समचतुरश्रक्षेत्रक्षेत्रफलसङ्ख्यैव वर्गः~। तस्माद्यश्चैवेत्यादिना सूत्रार्धेन भुजाक्षेत्रस्य कोटिक्षेत्रस्य च क्षेत्रफलयोगः कर्णक्षेत्रक्षेत्रफलसङ्ख्या भवतीत्युक्तं स्यात्~। तत्र यदि भुजाक्षेत्रजानि कोटिक्षेत्रजानि च क्षेत्रफलानि

\newpage

\noindent निश्शेषाणि कर्णक्षेत्रे निरन्तरमन्तर्भवेयुः तर्हि तयोः क्षेत्रफलयोगस्य कर्णक्षेत्रक्षेत्रफलत्वमुपपन्नं स्यात्~। क्षेत्रफलान्तर्भावश्च क्षेत्रयोरन्तर्भावे
भवत्येव, क्षेत्रावयवत्वात् क्षेत्रफलानाम्~। अवयविनोऽन्तर्भावेऽवयवानामप्यन्तर्भावादिति कर्णक्षे(त्र ? त्रे) भुजाकोटिक्षेत्रयोरन्तर्भावः~। तत्प्रकारः
प्रदर्श्यते~। तद्यथा \textendash\ समायामवनौ पूर्वापरायतं द्वादशभुजकं षोडशकोटिकं क्षेत्रमालिख्य तदाग्नेयकोणात् प्रवृत्तां वायव्यकोणप्रापिणीं रेखां कुर्यात्~। सा हि तस्य कर्णः~। पुनस्तदग्राभ्यां प्रवृत्ते तद्विपरीतदिशा स्थिते तत्तुल्यपरिमाणे रेखे कुर्यात्~। पुनस्तयोरन्यतरस्याग्रात् प्रवृत्तामितराग्रप्रापिणीं रेखां कुर्यात्~। तथा कृते तत् पूर्वक्षेत्रकर्णतुल्यभुजाकोटिकं क्षेत्रं स्यात्~। पुनः प्रथमक्षेत्रभुजा(तुल्य)भुजाकोटिकं कोटितुल्यभुजाकोटिकं च मृद्दार्वादिना क्षेत्रद्वयं निर्माय तयोः कोटिक्षेत्रं प्रथमक्षेत्रभुजातुल्ये प्रदेशे छिन्द्यात्~। तथा कृते तत्क्षेत्रं भवति~। तत्र प्रथमं प्रथमक्षेत्रभुजातुल्ये प्रदेशे छिन्नत्वात् तत्तुल्यभुजकं प्राक् प्रथमक्षेत्रकोटितुल्यभुजाकोटिकत्वात् तत्तुल्यकोटिकम्~। अतस्तत् प्रथमक्षेत्रतुल्यम्~। द्वितीयं तु प्रथमक्षेत्रभुजाकोट्यन्तरतु(ल्यं ? ल्यभुजकं) तत्कोटितुल्यकोटिकं च भवति, कोटितुल्यस्य भुजातुल्यप्रदेशे छिन्नत्वात्~। पुनस्तद्द्वितीयं क्षेत्रमपि प्रथमक्षेत्रभुजातुल्यप्रदेशे छिन्द्यात्~। तथा कृते ये क्षेत्रे भवतः तयोरुभयोरपि भुजाकोट्यन्तरतुल्या भुजा, प्रागेव तत्तुल्यभुजकत्वात्~। कोटिस्तु प्रथमक्षेत्रभुजातुल्ये प्रदेशे छिन्नत्वेन तत्तुल्यैकस्य~। इतरस्य तु तच्छेषत्वाद्भुजाकोट्यन्तरतुल्यैव कोटिरपि~। अतस्तद्भुजाकोट्यन्तरतुल्यभुजाकोटिकम्~। प्रथमं तु प्रथमक्षेत्रभुजातुल्यकोटिकं भुजाकोट्यन्तरतुल्यभुजकम्~। तयोः प्रथममादाय भुजाक्षेत्रस्य केनचिद्भागेन सम्बद्धकोटिकं कृत्वा निधाय लाक्षादिना भुजाक्षेत्रेण सुश्लिष्टं\renewcommand{\thefootnote}{१}\footnote{प्रथममादाय भुजाकोट्यन्तरभुजकत्वाद्भुजाकोट्यन्तरयुक्ताया भुजायाः कोटितुल्यत्वाच्च प्रथमक्षेत्रकोटितुल्यस्यापि कोटिः सुश्लिष्टं} कुर्यात्~। एवं कृते तदपि प्रथमक्षेत्रतुल्यं क्षेत्रं भवति, भुजाक्षेत्रेण संश्लिष्टस्य क्षेत्रस्य प्रथमक्षेत्रभुजाकोट्यन्तरभुजकत्वाद्भुजाकोट्यन्तरयुक्ताया भुजायाः कोटितुल्यत्वाच्च~। प्रथमक्षेत्रकोटितु(ल्य ? ल्या)स्यापि कोटिः, संश्लिष्टस्य प्रथमक्षेत्रभुजातुल्यकोटिकत्वात् कोट्या च संश्लेषात्~। प्रथमक्षेत्रभुजातुल्यैवास्यापि भुजा~। अतः प्रथमक्षेत्रतुल्यमेवै\renewcommand{\thefootnote}{२}\footnote{व \textendash\ क. पाठः.}तदपि क्षेत्रम्~। तदेवं भुजाकोटिक्षेत्राभ्यां

\newpage

\noindent सम्पादितानि त्रीणि क्षेत्राणि भवन्ति~। तत्र क्षेत्रद्वयं प्रथमक्षेत्रतुल्यम्~। अन्यत् प्रथमक्षेत्रभुजाकोट्यन्तरतुल्यभुजाकोटिकम्~। तेषु
प्रथमक्षेत्रतुल्यं क्षेत्रद्वयं कर्णाकारेण खण्डयेत्~। ततस्तान्यर्धायतचतुरश्राणि चत्वारि क्षेत्राणि भवन्ति~। तेषां च भुजाकोटिकर्णाः प्रथमक्षेत्रभुजाकोटिकर्णतुल्याः~। प्रथमक्षेत्रतुल्य \ldots \ldots सम्बद्धकर्णाकारेण खण्डितत्वात्~। पूर्वलिखि(ते ? त)कर्णक्षेत्रभुजाकोटितुल्याश्चैतेषां कर्णाः, तस्य भुजाकोट्योरेतेषां कर्णानां च प्रथमक्षेत्रकर्णतुल्यत्वात्~। तेष्वेकमा(द ? दाय) बाहुना
सम्बद्धकर्णं\renewcommand{\thefootnote}{१}\footnote{तेष्वेक सम्बद्धकर्णं} निदध्यात्~। पुनर्द्वितीयमादाय प्रथमनिहितस्य कोट्या सम्बद्धभुजकं कर्णक्षेत्रस्य पूर्वस्मादनन्तेरण बाहुना सम्बद्धकर्णं च निदध्यात्~। तृतीयं पुनर्द्वितीयस्य कोट्या सम्बद्धभुजकं कर्णक्षेत्रतृतीयबाहुसम्बद्धकर्णं निदध्यात्~। चतुर्थमपि तृतीयस्य कोट्या सम्बद्धभुजकं प्रथमस्य भुजया सम्बद्धकोटिकं कर्णक्षेत्रचतुर्थबाहुसम्बद्धकर्णं च निदध्यात्~। तथा सति कर्णक्षेत्रं प्रायेण तैः परिपूर्णं स्यात्~। मध्ये भुजाकोट्यन्तरतुल्यभुजाकोटिकं क्षेत्रमपरिपूर्णं स्यात्~। तथाहि \textendash\ प्रथमनिहितस्य कोट्या सम्बद्धा खलु द्वितीयस्य भुजा\renewcommand{\thefootnote}{२}\footnote{भुजायान}~। तत्र द्वितीयस्य भुजयानवगाढो यो भागः प्रथमस्य कोटेः स एवापरिपूर्णस्य क्षेत्रस्यैको बाहुः, इतरस्य द्वितीयभुजाव्याप्तत्वेन पूर्णत्वात्~। स च भुजाकोट्यन्तरतुल्यः, भुजयावगाढस्य भागस्य भुजातुल्यत्वेन शिष्टस्य कोटिभागस्य भुजाकोट्यन्तरतुल्यत्वात्~। तृतीयस्य भुजयानवगाढो द्वितीयकोटिभागो यः स द्वितीयो बाहुः~। चतुर्थस्य भुजयानवगाढस्तृतीयस्य कोटेर्यो भागः स तृतीयबाहुः~। प्रथमभुजयानवगाढश्चतुर्थकोटिभागश्चतुर्थो बाहुः~। सर्वत्र भुजाकोट्यन्तरत्वं पूर्ववद्द्रष्टव्यम्~। एवं यद्भुजाकोट्यन्तरतुल्यभुजाकोटिकमपरिपूर्णं क्षेत्रं स्यात् तत् पूर्वं ख(ण्ड ? ण्ड)तेन कोटिक्षेत्रशेषेण भुजाकोट्यन्तरतुल्यभुजाकोटिकेन तस्मिन्निहितेन परिपूर्णं स्यात्~। तथा सति कर्णक्षेत्रं भुजाकोटिक्षेत्राभ्यां परिपूर्णं भवति~। उत्तरार्धेन वृत्तक्षेत्रस्थजीवाया उभयभागस्थिताभ्यां शराभ्यां तज्ज्यार्धानयनोपायमाह\textendash 

\begin{quote}
{\qt वृत्ते शरसंवर्गोऽर्धज्यावर्गः स खलु धनुषोः~।}
\end{quote}

\noindent इति~। वृत्ते स्थितयोः शरयोर्यः संवर्गः स खलु तद्धनुषोः\renewcommand{\thefootnote}{३}\footnote{धनुषोः \textendash\ क. पाठः.} साधारणभूताया जीवाया अर्धस्य वर्गो भवतीत्यर्थः~। अत्र वासनां वक्तुं प्रथमं

\newpage

\noindent क्षेत्रं प्रदर्श्यते \textendash\ समायां भूमौ कर्कटकेन वृत्तमालिख्य तास्मिन् पूर्वापरदक्षिणोत्तररेखे कुर्यात्~। ततः पूर्वापररेखापरिध्योः प्राक्सम्पातादुत्तरतः परिधावभीष्टप्रदेशे बिन्दुं कृत्वा दक्षिणतोऽपि ता\renewcommand{\thefootnote}{१}\footnote{दक्षिणतो ता}वत्येवान्तरे द्वितीयं बिन्दुं कुर्यात्~। पुनस्तद्बिन्दुद्वयसंस्पर्शिसूत्रं प्रसार्य तदनुसारिणीं रेखां कुर्यात्~। सा खलु धनुषोः साधारणभूता ज्या~। तदवच्छिन्नौ\renewcommand{\thefootnote}{२}\footnote{न्न} परिधेः प्राक् पश्चाद्भागौ धनु(षि ? षी)~। जीवायाः पूर्वतः पश्चिमतश्च स्थितौ पूर्वापररेखाभागौ शरौ~। पुनर्जीवाया अन्यतराग्रात् प्रवृत्तां वृत्तकेन्द्रप्रापिणीं रेखां कुर्यात्~। एवं स्थिते वासना प्रदर्श्यते \textendash\ अत्र हि ज्यार्धं भुजा~। वृत्तकेन्द्रजीवान्तरालवर्तिपूर्वापररेखाखण्डं कोटिः~। जीवाग्रात् प्रवृत्ता वृत्तकेन्द्रप्रापिणी रेखा कर्णः~। एवमिदमर्धायतचतुरश्रं क्षेत्रम्~। तत्र कर्णवर्गात् कोटिवर्गेऽपनीते शिष्टं भुजात्मकस्य ज्यार्धस्य वर्गो भवति~। कर्णवर्गो नाम भुजाकोटिवर्गयोग एवेति\renewcommand{\thefootnote}{३}\footnote{वर्ग एवेति} पूर्वार्धे प्रदर्शितत्वात्~। अतः कर्णकोट्योर्वर्गान्तरमेव भुजावर्गः~। वृत्तकेन्द्रपरिध्यन्तरालस्य सर्वत्र व्यासार्धतुल्यत्वाद्व्यासार्धतुल्य एवात्र कर्णः~। ततो व्यासार्धकोट्योर्वर्गान्तरं भुजावर्गो भवति~। तयोर्वर्गान्तरं च योगेऽन्तरेण गुणिते स्यात्~। तद्यथा \textendash\ द्वयो राश्योर्वर्गान्तरे सम्पादयितव्ये तयोरधिकसङ्ख्यस्य राशेर्वर्गान्न्यूनसङ्ख्यस्य वर्गः शोधनीयः~। त\renewcommand{\thefootnote}{४}\footnote{तयोरधिकसङ्ख्यस्यावशिष्टो \ldots \ldots त \textendash\ क. पाठः.}द(र्थः ? र्थं) तयोर्वर्गौ सम्पादनीयौ~। तत्सम्पादने चाधिकसङ्ख्यो राशिरधिकसङ्ख्येन राशिना गुणनीयः, न्यूनसङ्ख्यो न्यूनसङ्ख्येन गुणनीयः~। तत्र गुण्यं गुणकारं च खण्ड(गुणनन्यायेन खण्डयेत्~। तद्यथा \textendash\ ) अधिकस्य राशेर्गुण्यगुणकारौ द्वेधा खण्डयेत्, उभयत्राप्येकः खण्डो न्यूनराशितुल्यः कार्यः~। तेनान्यो राश्यन्तरतुल्यो जातः~। एवं स्थिते गुण्यस्य न्यूनराशितुल्यः खण्डो राश्यन्तरतुल्य(: खण्डश्च) न्यूनराशितुल्येन खण्डेन निहतौ कार्यौ~। पुनश्च गुण्यखण्डद्वयं गुणकारस्य राश्यन्तरतुल्येन खण्डेन निहतं कार्यम्~। एवं चत्वारः संवर्गाः कार्याः~। तत्र प्रथमो न्यूनराशितुल्ययोः संवर्गः~। एतेषां
योगोऽधिकस्य राशेर्वर्गो भवति~। तत्राधिकराशिवर्गान्न्यूनराशिवर्गस्य शोधनीयत्वात् न्यूनराशितुल्ययोः संवर्गोऽत्र न कर्तव्यः~। तेन त्रय एव संवर्गाः कार्याः~। तत्र द्वौ न्यूनराश्यन्तरयोः संवर्गौ~। एको राश्यन्तरयोः संवर्गः~। त्रिष्वपि

\newpage

\noindent संवर्गेषु राश्यन्तरस्य विद्यमानत्वात् सर्वत्र राश्यन्तरं गुणकारत्वेन परिकल्पयेत्, न्यूनराशिद्वयं राश्यन्तरं च गुण्यत्वेन~। तथा सति त्रयाणामपि गुण्यानां राश्यन्तरमेव गुणकारः~। तेन गुण्यानां योगो राश्यन्तरेण गुणनीयो जातः~। तेषां योगश्च राश्योर्योग एव~। तथाहि \textendash\ राश्यन्तरस्य द्वयोर्न्यूनराश्योश्चात्र योगः कर्तव्यः~। तत्रैकस्य न्यूनराशेः राश्यन्तरस्य च योगे कृते सोऽधिकराशिर्भवति, अन्तरसहितस्य न्यूनराशेरधिकराशित्वात्~। तस्मिन् पुनरितरो न्यूनराशिर्योजनीयः ततोऽधिकराशिर्न्यूनराशिसहितस्तेषां योग एव~। अस्य राश्यन्तरं गुणकारः~। तस्माद्राश्योर्योगेऽन्तरेण गुणिते तयोर्वर्गान्तरं स्यादिति युक्तमेतत्~। अत्र पुनर्व्यासार्धकोट्योर्योग एवाधिकः शरः, अन्तरं न्यूनः शरः~। तद्यथा \textendash\ जीवायाः पश्चिमतः स्थितो यः पूर्वापररेखाभागः स खल्वत्राधिकः शरः~। तस्य वृत्तकेन्द्रजीवान्तरालवर्तिखण्डं कोटिरिति प्रागेव प्रदर्शितम्~। वृत्तकेन्द्रपश्चिमसम्पातान्तरालखण्डं व्यासार्धमेव हि~। अतो व्यासार्धकोट्योर्योग एवाधिकः शरः (वृत्तकेन्द्र)प्राक्सम्पातान्तरालं व्यासार्धम्~। केन्द्रजीवान्तरालं कोटिः~। तत्र यावतांशेन कोटेरधिकं व्यासार्धं स एव तयोरन्तरम्~। स च कोटेर्जीवावधिकत्वाद् जीवाप्राक्सम्पातान्तरालात्मकः~। स एव च न्यूनः शरः~। अतोऽधिकशरे न्यूनशरेण गुणिते व्यासार्धकोट्योर्योगस्तयोरन्तरेण गुणितः स्यात्~। तस्माद्युक्तमिदं गणितकर्म~॥~१७~॥\\

अथान्योन्यस्मिन्नन्तर्भूतैकदेशयोर्विषमयोर्वृत्तयोर्व्यासाभ्यां ग्रासेन च सम्पात-शरयोरानयनमाह\textendash 

\begin{quote}
{\ab ग्रासोने द्वे वृत्ते ग्रासगुणे भाजयेत् पृथक्त्वेन~।\\
ग्रासोनयोगलब्धौ\renewcommand{\thefootnote}{*}\footnote{'योगभक्ते' इति मुद्रितपुस्तकपाठः.} सम्पातशरौ परस्परतः~॥~१८~॥}
\end{quote}

इति~। ग्रास इत्यन्योन्यस्मिन्नन्तर्भूतो व्यासैकदेश उच्यते~। वृत्तशब्दो व्यासवचनः~। ग्रासोनयोगलब्धौ ग्रासोनयोर्व्यासयोर्योगेन लब्धौ~। अनेनानुवादरूपेण भाजको दर्शितः~। परस्पर(त) इति महतो वृत्तस्य व्यासाल्लब्धोऽल्पस्य वृत्तस्य शरः, अल्पस्य व्यासाल्लब्धो महतः शर इत्यर्थः~। अत्र वासनाछेद्यकं प्रदर्श्यते \textendash\ समायां भूमौ कर्कटकेन\renewcommand{\thefootnote}{१}\footnote{कर्कटेन \textendash\ ख. पाठः.} वृत्तमालिख्य तत्पूर्वतः पूर्वस्मात् न्यूनपरिमाणेन कर्कटकेन प्रथमवृत्तान्तर्भूतैकदेशं द्वितीयं वृत्तमा-

\newpage

\noindent लिखेत्~। पुनस्तद्वृत्तद्वयव्यापिनीं पूर्वापररेखां कृत्वा वृत्तसम्पातद्वयसंस्पर्शि सूत्रं प्रसार्य तदनुसारिणी रेखां कुर्यात्~। तत्रान्योन्यस्मिन्नन्तर्भूतौ परिधिभागौ सम्पातद्वयावच्छिन्नत्वात् सम्पातधनुषी~। सम्पातद्वयसंस्पर्शिनी दक्षिणोत्तरायता रेखा सम्पातधनुषोर्जीवारूपेण स्थिता सम्पातजीवा~। तस्याः पूर्वापरभागगतौ ग्रासभागौ धनुर्मध्यजीवान्तरालवर्तित्वात् सम्पातशरौ~। तत्रापरगतो भागोऽल्पस्य वृत्तस्य शरः~। तेन सोऽधिकशरः~। पूर्वगतो भागो महतः शरः~। तेन स न्यूनशरः~। ननु विरुद्धमिदम् अल्पस्य वृत्तस्य शरोऽधिकः, महतो वृत्तस्य शरो न्यून इति~। (वृत्तस्याल्प)त्वे (तं ? तत्)परिधेर्वक्रताधिका भवति~। महत्त्वे न्यूना स्यात्~। अतोऽल्पस्य वृत्तस्य धनुषो वक्रताधिका स्यात् महतो न्यूना, परिध्येकत्वाद्धनुषः~। धनुर्वक्रताया आधिक्यं न्यूनत्वं च शरस्याधिक्ये न्यूनत्वे च भवति~। अतोऽल्पस्य वृत्तस्य शरोऽधिकः महतो न्यूनः, तावेतौ शरावत्रानीयेते~। तत्राधिकशरस्य यावानंशो न्यूनशरः, तावानेवांशः स्वशरोनस्याधिकव्यासस्य स्वशरोनो न्यूनव्यासः~। यद्यधिकशरस्यार्धं न्यूनशरः तर्ह्यधिकस्य शरोनव्यासस्य अर्धं न्यूनशरोनव्यासः~। यदा पुनरधिकशरस्य त्र्यंशो न्यूनशरः तदाधिकशरोनव्यासस्यापि त्र्यंशो न्यूनशरोनव्यासः~। यदाधिकशरञ्चांशतुल्यो न्यूनशरः तदाधिकशरोनव्यासपञ्चांशतुल्यो न्यूनशरोनव्यासः~। एवं सर्वत्र शरसमानयोगक्षेमावेव शरोनव्यासौ~। तथाहि \textendash\ अत्राधिकशरेण न्यूने शरोनव्यासे गुणिते न्यूनशरेणाधिके शरोनव्यासे गुणिते च फलसाम्यं स्यात्~। अधिकशरन्यूनशरोनव्यासौ नाम सम्पातजीवापेक्षयाल्पस्य वृत्तस्य शरावेव~। तथा न्यूनशराधिकशरोनव्यासावपि महतो वृत्तस्य शरौ~। शरयोः संवर्गोऽर्धज्यावर्ग एवेति पूर्वं प्रदर्शितम्~। अत्र पुनरुभयोर्वृत्तयोः सम्पातजीवैव जीवा~। अतोऽधिकशरन्यूनशरोनव्यासयोः संवर्गो न्यूनशराधिकशरोनव्यासयोः संवर्गश्च तस्या एवार्धस्य वर्गो भवति~। तस्मादस्त्येव फलसाम्यम्~। तदेतत्फलसाम्यं शरोनव्यासयोः शरसमानयोगक्षेमत्वेन विना न सम्भवति~। तथाहि \textendash\ यदाधिकशरार्धतुल्यो न्यूनशरस्तदा तेनाधिकशरोनव्यासे गुणितेऽधिकशरेण न्यूनशरोनव्यासे गुणिते च यत्फलसाम्यं तन्न्यूनशरोनव्यासस्याधिकशरोनव्यासार्धतुल्यत्व एव भवति~। तद्यथा \textendash\ यदि गुण्यौ शरोनव्यासौ समौ स्यातां तर्हि फलसाम्यं न स्यात्, गुणकारयोः शरयोरसमत्वात्~। किन्तु अधिकशरगुणितफलार्धतुल्यमेव न्यूनशरगुणितं फलम्, अधि-

\newpage

\noindent कशरार्धतुल्यत्वात् न्यूनशरस्य~। अतो न्यूनशरगुण्यस्येतरस्मादाधिक्यमेव युक्तम्~। तत्रापि कियदाधिक्यमिति वीक्षायां\renewcommand{\thefootnote}{१}\footnote{विवक्षा \textendash\ क. पाठः.} द्विगुणितत्वमेवेति स्यात् प्रत्य(या ? यः)~। तथाहि \textendash\ शरोनव्यासयोः साम्ये सत्यधिकशरगुणितफलार्धतुल्यमेव न्यूनशरगुणितं फलमित्युक्तम्~। अतस्तस्मिन् द्विगुणिते फलसाम्यं स्यात्~। अत्र (सु ? अधिकशरो)न(व्यासे) न्यूनशरेणैव गुणिते फलसाम्यात् प्रागेव तस्य द्विगुणितत्वं कल्प्यम्~। अतोऽधिकशरोनव्यासार्धतुल्यो न्यूनशरोनव्यासः~। यत्र पुनरधिकशरत्र्यंशतुल्यो न्यूनशरस्तत्रापि शरोनव्यासयोः सा(म्यं ? म्ये) सति फलसाम्यं न स्यात्~। किन्तु अधिकशरगुणितफलत्र्यंशतुल्यं तत्र न्यूनशरगुणितं फलम्~। अतस्तस्मिंस्त्रि(भि)र्गुर्णिते फलसाम्यं स्यात्~। अत्रापि न्यूनशरेणैव गुणिते फलसाम्यदर्शनात् प्रागेव तस्य त्रिगुणितत्वं कल्प्यम्~। अतस्तत्राधिकशरोनव्यासत्र्यंशतुल्यो न्यूनशरोनव्यासः~। एवं सर्वत्र द्रष्टव्यम्~। तदेवं फलसाम्यानुपपत्तिः शरोनव्यासयोः शरसमानयोगक्षेमत्वं गमयति~। शरोनव्यासवद्ग्रासोनव्यासावपि शरसमानयोगक्षेमौ~। तथाहि \textendash\ ग्रासो नाम शरयोर्योग एव~। अन्योन्यस्मिन्नन्तर्भूतो व्यासैकदेशो ग्रासः~। सम्पातजीवाया उभयपार्श्वस्थितौ ग्रासभागौ शराविति प्राक् प्रदर्शितत्वात्~। अतः शरद्वयरहितो व्यास एव ग्रासोनव्यासः~। तथा सत्यधिकशरहीनोऽधिकशरोनव्यासोऽधिको ग्रासोनव्यासः स्यात्~। महतो वृत्तस्य शरो न्यून इत्युक्तम्~। अतस्तेन हीनोऽधिको व्यासोऽधिकशरोनव्यासः~। स पुनरधिकशरेण च हीनः शरद्वयहीन एव स्यात्~। अनेनैव न्यायेन न्यूनशरहीनो न्यूनशरोनव्यासो न्यूनग्रासोनव्यास इत्यपि सिद्धम्~। तत एव ग्रासोनव्यासयोः शरसमानयोगक्षेमत्वमपि सिद्ध्यति~। तथाहि \textendash\ यत्र न्यूनशरोऽधिकशरार्धतुल्यस्तत्र न्यूनशरोनव्यासोऽप्यधिकशरोनव्यासार्धतुल्य इति प्रागेव प्रदर्शितम्~। तत्र ग्रासोनव्याससम्पादनार्थं न्यूनशरोनव्यासान्न्यूनशरेऽधिकशरोनव्यासादधिकशरे च शोधिते द्विगुणाद्द्विगुणमर्धादर्धं च शोधितं स्यात्, शरोनव्यासयोः शरयोश्च समानयोगक्षेमत्वात्~। तेन पुनरपि तयोः पूर्ववदेव विभागः स्यात्~। अधिको द्विगुणः अन्यस्तदर्धमिति द्विगुणाद्द्विगुणेऽर्धादर्धे च शोधिते शोधनात् प्राग् यः शरोनव्यासयोर्द्विगुणितत्वार्धत्वलक्षणो विभागस्तदनुसारेणैव शोधनं

\newpage

\noindent कृतं स्यात्~। अतो विभागः पुनरपि तदवस्थ एव स्यात्~। अनेनैव न्यायेन ग्रासोनव्यासयोः शरसमानयोगक्षेमत्वं सर्वत्र द्रष्टव्यम्~। तदिदमुदाहरणेन स्पष्टीक्रियते~। तत्र वृत्तव्यासावेकविंशतिषट्त्रिंशत्सङ्ख्यौ~। ग्रासः षट्सङ्ख्यः~। गणितानीतौ सम्पातशरौ द्विचतुस्सङ्ख्यौ~। अत्राधिकशरार्धतुल्यो न्यूनशरः~। महतो वृत्तस्य व्यासात् षट्त्रिंशत्सङ्ख्यात् तच्छरे द्विसङ्ख्ये शोधिते शेषोऽधिकशरोनव्यासश्चतुस्त्रिंशत्सङ्ख्यः~। अल्पस्य वृत्तस्य व्यासादेकविंशतिसङ्ख्यात् तच्छरे चतुस्सङ्ख्ये शोधिते शेषो न्यूनशरोनव्यासः सप्तदशसङ्ख्यः~। अतोऽधिकशरोनव्यासार्धतुल्यो न्यूनशरोनव्यासः~। अधिकव्यासात् षट्त्रिंशत्सङ्ख्यात् षट्सङ्ख्ये ग्रासेऽधिकशरोनव्यासात् चतुस्त्रिंशत्सङ्ख्याच्चतुस्सङ्ख्येऽधिकशरे वा शोधिते शेषोऽधिको ग्रासोनव्यासस्त्रिंशत्सङ्ख्यः~। न्यूनव्यासादेकविंशतिसङ्ख्यात् (षट्)सङ्ख्ये ग्रासे न्यूनशरोनव्यासात् सप्तदशसङ्ख्याद्द्विसङ्ख्ये न्यूनशरे वा शोधिते शेषो न्यूनग्रासोनव्यासः पञ्चदशसङ्ख्यः~। अतोऽधिकग्रासोनव्यासार्धतुल्यो न्यूनो ग्रासोनव्यासः~। एवं सर्वत्रोदाहरणीयं स्यात्~। सर्वत्र शरसमानयोगक्षेमावेव ग्रासोनव्यासाविति सुव्यक्तम्~। अतो न्यूनशरेणाधिके ग्रासोनव्यासे गुणिते अधिकशेरण न्यूने ग्रासोनव्यासे गुणिते च फलयोः साम्यं स्यात्~। तद्यथा \textendash\ यत्राधिकशरार्धतुल्यो न्यूनशरः तत्राधिकग्रासोनव्यासार्धतुल्यो न्यूनग्रासोनव्यास इत्युक्तम्~। अतस्तस्य गुण्ययोर्ग्रासोनव्यासयोर्वैषम्यात् फलवैषम्यं स्यात्~। तेनाधिकस्य ग्रासोनव्यासस्यार्धीकरणे वा न्यूनस्य द्विगुणने वा कृत एव फलसाम्यं स्यात्~। तत्र यद्यधिकस्यार्धीकरणं कर्तव्यं तद्गुणकारस्य न्यूनशरस्येतरगुणकारार्धतुल्यत्वात् कृतमेव, गुण्येऽर्धीकृते गुणकारेऽर्धीकृते वा फलवैषम्याभावात्~। यदि वा न्यूनस्य द्विगुणनं कर्तव्यं तदपि\renewcommand{\thefootnote}{१}\footnote{तदपि तन्न्यूने \textendash\ क. पाठः.} तद्गुणकारस्याधिकशरस्येतरगुणकाराद्द्विगुणितत्वात् कृतमेव, गुण्ये द्विगुणिते गुणकारे द्विगुणिते वा फलवैषम्याभावात्~। अतोऽत्र न्यूनशरेणाधिके ग्रासोनव्यासे अधिकशरेण न्यूने ग्रासोनव्यासे च गुणिते फलसाम्यं भवत्येव~। एष एव न्यायः शरयोर्विभागान्तरेष्वपि द्रष्टव्यः~। तदेवं सर्वत्र न्यूनशराधिकग्रासोनव्यासयोः (संवर्गः अधिकशरन्यूनग्रासोनव्यासयोः) संवर्ग एव~। तयोश्च संवर्गोऽन्ययोः संवर्ग\renewcommand{\thefootnote}{२}\footnote{संवर्ग एव} इति स्थितम्~। एवं स्थिते पूर्वोक्त\renewcommand{\thefootnote}{३}\footnote{पूर्वत्रोक्त \textendash\ ख. पाठः.}गणितवासना प्रद-

\newpage

\noindent र्श्यते \textendash\ तत्र प्रथमं वृत्तयोर्व्यासाभ्यां ग्रासे शोधिते ग्रासोनव्यासौ भवतः~। पुनस्तयोर्ग्रासेन गुणने कुते शरौ ग्रासोनव्यासयोगेन गुणितौ स्याताम्~। तत्राधिके शरोन\renewcommand{\thefootnote}{१}\footnote{ग्रासोन \textendash\ क. पाठः.}व्यासे ग्रासेन गुणितेऽधिकशरोन\renewcommand{\thefootnote}{२}\footnote{शसोग्रा \textendash\ ख. पाठः.}(व्यासो) ग्रासोनव्यासयोगेन गुणितः स्यात्~। तथाहि \textendash\ ग्रासो नाम शरयोर्योग एवेत्युक्तम्~। तथा सति ग्रासेन गुणनं शरयोगेन गुणनमेव~। अतो ग्रासेन तयोर्गुणने कृते शरयोगेन गुणनमेव कृतं स्यात्~। तत्राधिके ग्रासोनव्यासे ग्रासेन गुणिते गुणगुण्ययोः परस्परगुणकारत्वादधिके\renewcommand{\thefootnote}{३}\footnote{को}न ग्रासोनव्यासेन शरयोग एव गुणितः स्यात्~। तथा सति तस्य संवर्गस्यैकोंऽशोऽधिकग्रासोनव्यासाधिकशरसंवर्गात्मकः~। अन्योंऽशोऽधिकग्रासोनव्यासन्यूनशरसंवर्गात्मकः~। तत्र द्वितीयो
न्यूनग्रासोनव्यासाधिकशरसंवर्गतुल्ये एव स्यात्~। तत्तुल्यत्वं च प्रागेव प्रदर्शितम्~। अतः संवर्गद्वयेऽपि गुण्योऽधिकशर एव~। गुणकार एकत्राधिको ग्रासोनव्यासः, अन्यत्र न्यूनो ग्रासोनव्यासः~। अतस्तयोर्योगेनाधिकशरे गुणिते तत्संवर्गद्वयं स्यात्~। तस्मात् ग्रासेनाधिके ग्रासोनव्यासे गुणिते ग्रासोनव्यासयोगेनाधिकशर एव गुणितः स्यात्~। तस्मिन् पुनर्ग्रासोनव्यासयोगेन हृतेऽधिकः शरः स्यात्, गुणकारेणैव हृतत्वात्~। अनेनैव न्यायेन न्यूने ग्रासोन\renewcommand{\thefootnote}{४}\footnote{न्यायेन ग्रासोन}व्यासे ग्रासेन गुणिते ग्रासोनव्यासयोगेन हृते च न्यूनशरः स्यात्~। अथवा त्रैराशिकमेवेदम्~। यदि ग्रासोनव्यासयोगस्य न्यूनाधिकावंशौ न्यूनाधिकग्रासोनव्यासतुल्यौ तदा शरयोगात्मकस्य ग्रासस्य न्यूनाधिकावंशौ कियन्ताविति~। अत्र ग्रास इच्छाराशिः~। न्यूनोऽधिको ग्रासोनव्यासः फलराशिः~। ग्रासोनव्यासयोगः प्रमाणराशिः~। अतो न्यूनेऽधिके वा ग्रासोनव्यासे\renewcommand{\thefootnote}{५}\footnote{व्यासेन} ग्रासे(न) गुणिते ग्रासोनव्यासयोगेन विभक्ते क्रमान्न्यूनोऽधिको वा शरः स्यात्, ग्रासोनव्यासयोः शरसमानयोगक्षेमत्वस्य प्रदर्शितत्वात्~। त्रैराशिकयोग्यत्वमप्यत्रास्त्येव~। तस्मादुभयथाप्युपपन्नमेवेदं गणितम्~। अत्र\renewcommand{\thefootnote}{६}\footnote{अथ} केचिदाहुः~। लौकिकगणितोपयोग्येवैतद्गणितं, न ग्रहगणितोपयोगि, अहर्गणानयनादौ ग्रहगणिते क्वचिदप्येवंविधस्य गणितस्यादृष्टत्वादिति~। अत्र ब्रूमः~। यद्यप्यह\renewcommand{\thefootnote}{७}\footnote{यद्यह \textendash\ क. पाठः.}र्गणानयनादौ ग्रहगणित एवंविधस्य गणितस्यादृष्टत्वं तथापि ग्रहणे विद्यत एवास्योपयोगः~। कथम्~। उच्यते~। सूर्यग्रहणे चन्द्रग्रहणे वा समस्तग्रहणाभावे बिम्बस्य

\newpage

\noindent कियान् भागो ग्रस्तो भविष्यतीति जिज्ञासायां तदवगतिसाधनत्वेनास्य ग्रहण उपयोगः~। ननु ग्रस्तस्य भागस्येयत्तानेन विनापि सुगमैव~। सम्पर्कार्धाद्विक्षेपेऽपनीते यः शेषः स ग्राह्यबिम्बस्य यावानंशस्तावानेवांशस्तस्य ग्रस्तो भविष्यतीति~। नैतत् सारम्~। गणितानीतं बिम्बं तद्व्यास एव~। ग्राहकबिम्बव्यासे संसृष्टो यो भागो ग्राह्यबिम्बव्यासस्य स एव विक्षेपहीनं सम्पर्कार्धं, न ग्राहकबिम्बेन संसृष्टो ग्राह्यबिम्बभागः~। स एव तस्य ग्रस्तो भागः~। तेन ग्राह्यबिम्रव्यासस्यैतावान् भागो ग्रस्तो भविष्यतीत्येव तत्र लभ्यते, न पुनर्ग्राह्यबिम्बव्यासस्यैतावान् भागो ग्रस्तो
भविष्यतीति~। ननु व्यासस्य तावत्यंशे ग्रस्ते बिम्बस्यापि तावान् स्वांशो ग्रस्तो भवति~। अतो गणितानीतस्य व्यासत्वेऽपि न कश्चिद्दोषः~। मैवम्~। ग्राहकबिम्बेन संसृष्टो यो भागो ग्राह्यबिम्बस्य स एव हि ग्रस्तो भवति, नान्यः~। स च ग्राहकबिम्बपरिध्यवच्छिन्नो ग्राह्यबिम्बैकदेशः~। तत्र यदि विक्षेपहीनं सम्पर्कार्धं ग्राह्यबिम्बव्यासार्धतुल्यं स्यात् तदा व्यासार्धमेव ग्रस्तं स्यात् न बिम्बस्य, व्यासात् प्रभृत्युभयतोऽपि ग्राहकबिम्बपरिधेर्वक्रतया
ग्राह्यबिम्बमध्यरेखातः क्रमेण विप्रकृष्टत्वात्~। अतो नैतावन्मात्रेण ग्रस्तभागेयत्ता ज्ञातुं शक्यते~। कस्तर्हि तज्ज्ञानोपायः~। उच्यते~। विक्षेपहीनं सम्पर्कार्धं संसृष्टयो\renewcommand{\thefootnote}{१}\footnote{संस्पृष्टयो \textendash\ क. पाठः.}र्बिम्बयोर्ग्रास एव~। अतस्तेन बिम्बव्यासाभ्यां चोक्तन्यायेन सम्पातशरावानीय तयोरन्यतरेण {\qt वृत्ते शरसंवर्ग} इत्युक्तन्यायेन (सम्पातजीवानेया~।) सम्पातजीवायास्तथाविधत्वस्य दर्शितत्वात्~। पुनस्तेन निहतां त्रिज्यां प्रतिराश्यैकां ग्राह्यबिम्बार्धेनान्यां ग्राहकबिम्बार्धेन च हरेत्~। ननु किमनेन त्रैराशिकेन कृतं स्यात्~। पूर्वानीतस्य ज्यार्धस्य त्रिज्यावृत्ते परिणमनम्~। कथम्~। यदि बिम्बार्धतुल्ये व्यासार्ध(तुल्य ?) इय(ती ? त्) ज्यार्धं तदा त्रिज्यातुल्ये कियदिति~। किं पुनरस्य त्रिज्यावृत्ते परिणमनेन प्रयोजनम्\renewcommand{\thefootnote}{२}\footnote{किमनेन \ldots\ प्रयोजनम्}~। चापीकरणसौकर्यम्~। त्रिज्यावृत्त(ज्या ? जा)नामेव जीवानां लोके प्रसिद्धत्वात् ताभिरेव चापीकरणं सुकरं स्यात्~। अन्यथा जीवानामप्यानयनं कर्तव्यं स्यात्~। अतो गणितलाघवं न स्यात्~। पुनस्तज्ज्यार्धद्वयं चापीकृत्य प्राग्ग्राह्यबिम्बार्धेन हृतं ग्राह्यबिम्बार्धेन ग्राहकबिम्बार्धेन च हृतं ग्राहकबिम्बार्धेन च निहत्य त्रिज्यया विभज्य बिम्बार्धवृत्तयोः परिणमयेत्~। तथा सति ते बिम्बार्ध\renewcommand{\thefootnote}{३}\footnote{र्धनिह}वृत्तगते

\newpage

\noindent सम्पातजीवार्धसम्बन्धिनी चापार्धे भवतः~। अत्रेदं त्रैराशिकं \textendash\ यदि त्रिज्यावृत्त इयच्चापार्धं तदा बिम्बार्धवृत्ते कियदिति~। पुनस्ते चापार्धे स्वस्वबिम्बार्धनिहते कार्ये~। तथा कृते तत्संवर्गद्वयं द्वयोरपि सम्पातधनुषोऽग्राभ्यां प्रवृत्तयोः स्वस्वबिम्बकेन्द्रप्रापिण्यो रेखयोरन्तरालवर्ती यः प्रदेशो बिम्बयोस्तत्सम्बन्धि क्षेत्रफलं स्यात्~। ग्राहकबिम्बार्धतच्चापार्धयोः संवर्गस्तद्बिम्बैकदेशस्य क्षेत्रफलम्~। ग्राह्यबिम्बार्धतच्चापार्धयोः संवर्गस्तद्बिम्बैकदेशस्य क्षेत्रफलमिति ज्ञेयम्~। का पुनरत्रोपपत्तिरिति चेत्~। उच्यते~। सर्वत्रापि धनुषोऽग्राभ्यां प्रवृत्तयोः केन्द्रप्रापिण्यो रेखयोरन्तरालेन कस्मिंश्चिच्चतुरश्रक्षेत्रे कृते तस्य भुजाकोटी व्यासार्धचापार्धतुल्ये भवतः~। तद्यथा \textendash\ रेखयोरन्तरालं
धनुर्मध्यादारभ्य केन्द्राव(पि ? धि) खण्डयेत्~। तत्र यत् खण्डद्वयं जातं तदपि पुनस्तथैव खण्डयेत्~। पुनस्तत्र जातान् खण्डानपि तथैव खण्डयेत्~। भूयोऽप्येवं खण्डयेद्यावच्चापखण्डानां वक्रता निवर्तते~। तथा कृते तानि त्र्यश्राणि क्षेत्राणि भवन्ति~। तेषां सर्वेषां पार्श्वानि च
व्यासार्धतुल्यानि, केन्द्रपरिध्यन्तरालस्य सर्वत्र व्यासार्धतुल्यत्वात्~। (ते ? तैः) पुनश्चतुरश्रं क्षेत्रं कुर्यात्~। कथम्~। तेष्वन्योन्यसंस्पृष्ट\renewcommand{\thefootnote}{१}\footnote{संसृष्ट \textendash\ ख. पाठः.}पार्श्वेषु व्यत्यस्ताग्रेषु च विन्यस्तेषु चतुरश्रं क्षेत्रं स्यात्~। तस्य भुजा व्यासार्धं, तेषां व्यासार्धतुल्यायामत्वात्~। कोटिश्चापार्धं, परस्परसम्बद्धास्तेषां भूमय एव हि कोटिः~। ताश्च चापखण्डान्येव~। तदर्धमेका कोटिः अन्यदर्धमितरा कोटिः, व्यत्यस्ताग्रतया विन्यस्तत्वात्~। तेन चापार्धव्यासार्धयोः संवर्गस्तत्क्षेत्रभुजाकोट्योः संवर्ग एव~। भुजाकोटिसंवर्गः क्षेत्रफलमित्युक्तम्~। एतच्च पुनः क्षेत्रफलं चापाग्राभ्यां प्रवृत्तयो रेखयोरन्तरालात्मकस्य पूर्वोक्तस्य क्षेत्रस्य क्षेत्रफलमेव, तुल्यत्वात् क्षेत्रयोः~। पूर्वप्रदर्शितेषु त्र्यश्रेषु खण्डेषु व्यत्यस्ताग्रतया विन्यस्तेषु चतुरश्रं क्षेत्रं स्यात् इत्येतावानेव विशेषः~। {\qt समपरिणाहस्यार्धं विष्कम्भार्धहतमेव वृत्तफलमि}ति वृत्तक्षेत्रफलसम्पादनोपायं प्रदर्शयता सूत्रकारेणाप्येतत्सर्वं सूचितमेव~। तत्र सकलवृत्तसम्बन्धिनां क्षेत्रफलानां सम्पादनीयत्वात् सकलस्य परिधेरर्धं व्यासार्धेन गुणनीयम्~। अत्र तु वृत्तैकदेशसम्बन्धीन्येव क्षेत्रफलानि सम्पादनीयानि~। अतस्तत्सम्बान्धिपरिध्येकदेशात्मकस्य चापस्यार्धमेव व्यासार्धेन गुणनीयम् इत्येष एव विशेषः~। उभ-

\newpage

\noindent यत्राप्येकैव वासना~। अतस्तत्तद्बिम्बार्धचापार्धयोः संवर्गस्तत्तद्बिम्बगतस्य पूर्वोक्तक्षेत्रस्य क्षेत्रफलमेवेति सिद्धम्~। ननु वृत्तक्षेत्रैकदेशसम्बन्धीन्येवैतानि क्षेत्रफलानि नतु घनवृत्तक्षेत्रैकदेशसम्बन्धीनि~। तानि चात्र सम्पादयितव्यानि, अर्केन्दुबिम्बयोः समघनवृत्तत्वात्~। सत्यम्~। ग्रस्तभागेयत्तावगतिः खल्वत्र सा(ध्यः ? ध्या)~। तेन तत्साधन\renewcommand{\thefootnote}{१}\footnote{साधन}भूतानामेव सम्पादनं कर्तव्यं नेतरेषां, प्रयोजनाभावात्~। एतेषु पुनः क्षेत्रफलेषु चापजीवान्तरालसम्बन्धीन्येवात्र ग्राह्याणि~। अतोऽन्यानि जीवाकेन्द्रान्तरालसम्बन्धीन्येतेभ्यः शोधनीयानि, अनुपयोगित्वात्~। तानि च तत्तच्छरोनेन तत्तद्बिम्बार्धेन ज्यार्धे निहते स्युः~। कथम्~। उच्यते~। द्विसमत्र्यश्रं क्षेत्रमेवेदम्~। यत् पूर्वोक्तस्य क्षेत्रस्य जीवाकेन्द्रान्तरालं तस्य पार्श्वद्वयं\renewcommand{\thefootnote}{*}\footnote{'र्श्वद्वयमि'त्यारभ्य 'तुल्यसङ्ख्यत्वादेवोक्तमि'त्यन्तं वाक्यजातं ख. पाठः ऽत्रादृष्टं ११८ तमपृष्ठे 'अत्रापि इच्छाप्रमाणराशी पूर्वोक्तावेव' इत्यनन्तरं दृष्यते~।} व्यासार्धतुल्यं, भूमिसम्पातजीवैव केन्द्रात् प्रवृत्ता ज्यामध्यप्रापिणी रेखावलम्बः~। सैव ज्याचापमध्यान्तरालरूपेण शरेण युता व्यासार्धम्~। अतो व्यासार्धाच्छरेऽपनीते शेषोऽवलम्बः~। अतस्तेन भूम्यर्धे निहते तत्क्षेत्रक्षेत्रफलानि\renewcommand{\thefootnote}{२}\footnote{तत्क्षेत्रफलानि \textendash\ क. पाठः.} स्युः~। तद्यथा \textendash\ त्रिभुजं क्षेत्रं\renewcommand{\thefootnote}{३}\footnote{त्रिभुजक्षेत्रं \textendash\ ख. पाठः.} भूमध्यादारभ्यावलम्बानुसारेण खण्डयेत्~। तथा खण्डितेऽर्धायतचतुरश्रे द्वे क्षेत्रे भवतः~। तयोर्भुजे भूम्यर्धतुल्ये, भूमध्यादारभ्य खण्डितत्वात्~।
कोटी लम्बतुल्ये, तदनुसारेण खण्डितत्वात्~। पुनस्ताभ्यां चतुरश्रं क्षेत्रं कुर्यात्~। कथम्~। तयोर्भुजे विरुद्धदिग्गते कार्ये, कर्णौ च परस्परसम्बद्धौ कार्यौ~। तथा सति तच्चतुरश्रं क्षेत्रं भवति~। तस्य भुजाकेटी भूम्यर्धतु(ल्यावे ? ल्ये) एव, चतुरश्रीकृतेऽपि भुजाकोट्योर्विशेषाभावात्~। अतो लम्बेन भूम्यर्धे गुणिते तत्क्षेत्रक्षेत्रफलनि स्युः~। तान्येव त्रिभुजस्य क्षेत्रस्यापि क्षेत्रफलानि, तुल्यत्वात् क्षेत्रयोः~। यदा भुजे विरुद्धदिग्गते कर्णौ च परस्परसम्बद्धौ तदा तच्चतुरश्रं क्षेत्रं स्यात्~। यदा पुनरेकदिग्गते भुजाकोटी च परस्परसम्बद्धे तदा तत् त्रिभुजं क्षेत्रं स्यादित्येव विशेषः~। तस्माद्भूम्यर्धलम्बयोः संवर्ग एव त्रिभुजक्षेत्रक्षेत्रफलम्~। तथाचोक्तं \textendash\ {\qt त्रिभुजस्य फलशरीरं समदलकोटीभुजार्धसंवर्गः} इति~। अत एतेषु क्षेत्रफलेषु पूर्वानीतेभ्यः क्षेत्रफलेभ्यः शोधितेषु

\newpage

\noindent जीवाकेन्द्रान्तरालसम्बन्धीनि क्षेत्रफलानि शोधितानि भवन्ति~। तस्माच्छिष्टानि जीवाचापान्तरालसम्बन्धीनि क्षेत्रफलानि~। एवं बिम्बद्वयगतान्यपि सम्पादनीयानि~। पुनस्तेषां योगे कृते ग्रस्तभागसम्बन्धीनि क्षेत्रफलानि भवन्ति~। सम्पातधनुषोऽन्तरालवर्ती यो भागो ग्राह्यबिम्बस्य स एव हि ग्रस्तो भवति, नान्यः~। तस्य सम्पातजीवावच्छिन्नौ यावंशौ तयोः क्षेत्रफलान्येवात्रानीतानि~। अतस्तेषां योगे कृते ग्रस्तभागसम्बन्धीन्येव क्षेत्रफलानि भवन्ति~। पुनः {\qt चतुरधिकं शतम्} इत्यादि सूत्रेण प्रदर्शिताभ्यां फलप्रमाणाभ्यां ग्राह्यबिम्बपरिधिमानीय तदर्धं बिम्बार्धेन निहत्य ग्राह्यबिम्बगतानि क्षेत्रफलान्यानयेत्~। तेषु ग्रस्तभागक्षेत्रफलैर्हृतेषु यत् फलं लभ्यते तावानंशो ग्राह्यबिम्बस्य ग्रस्तो भवति~। ननु चन्द्रग्रहणे भवत्वेवं, सूर्यग्रहणे पुनर्नैतद्युक्तम्~। कथम्~। चन्द्रकक्ष्या खल्वर्ककक्ष्यातो न्यूनपरिमाणा, अर्कस्याधःस्थितत्वात् चन्द्र(स्य)~। तेन चन्द्रकक्ष्यागताः कला अप्यर्कबिम्बगताभ्यः कलाभ्यो न्यूनपरिमाणाः~। तेन ताभिरानीतानि क्षेत्रफलान्यप्यर्कबिम्बकलाभिरानीतेभ्यः क्षेत्रफलेभ्यो न्यूनपरिमाणानि~। अतो ग्रस्तभागसम्बन्धिनां क्षेत्रफलानां भिन्नपरिमाणत्वात् अयुक्तमेवेदं गणितं सूर्यग्रहण इति~। नैष दोषः~। अर्ककक्ष्यागताभिः कलाभिरेव सर्वाण्यपि क्षेत्रफलान्यत्रानीयन्ते न चन्द्रकक्ष्यागताभिः~। ननु कथं चन्द्रबिम्बगताभिः कलाभिरानीतानाम् अर्ककक्ष्यागताभिः कलाभिरानी(य ? त)त्वम्~। उच्यते~। अर्ककक्ष्यागतो यः प्रदेशः चन्द्रबिम्बेनाच्छाद्यते तद्गताभिः कलाभिरेव क्षेत्रफलान्यानीतानि न चन्द्रबिम्बगताभिः~। स च वृत्तक्षेत्रविशेषः, आच्छादकस्य चन्द्रबिम्बस्य वृत्ताकारत्वात्~। तेन सम्पातशरादिकं सर्वं तद्गतमेव, न चन्द्रबिम्बगतम्~। गणितानीतानां चन्द्रबिम्बगतानां कलानां तत्साधनत्वं चन्द्रबिम्बे तदाच्छाद्ये वृत्ते च कलानां तुल्यसङ्ख्यत्वादेवोक्तम्\renewcommand{\thefootnote}{*}\footnote{इत ऊर्ध्वं किञ्चित्लुप्तं प्रतिभाति}~॥~१८~॥

\begin{quote}
{\ab (इष्टं व्येकं दलितं सपूर्वमुत्तरगुणं समुखमध्यम्~।\\
इष्टगुणितमिष्टधनं त्वथवाद्यन्तं पदार्धहतम्~॥~१९~॥)}
\end{quote}

\dotfill\ इष्टधनं भवतीत्यर्थः~। यदा पुनरादित आरभ्य केषाञ्चिद्वा सर्वेषां वा पदानां धनं ज्ञातुमिष्टं तदा पूर्वस्याभावात् सपूर्वमित्यनेन विना कर्म

\newpage

\noindent कर्तव्यम्~। यदि पुनः प्रथमपदव्यतिरिक्तस्यैकस्यैव पदस्य धनं ज्ञातुमिष्टं स्यात्, तदा दलितं\renewcommand{\thefootnote}{१}\footnote{त} मध्यमित्याभ्यां पदाभ्यां हीनेन पूर्वार्धेन कर्म कर्तव्यम्~। इष्टमित्यत्रैकत्वं च विवक्षितम्~। अत्र वासना द्विमुखे पञ्चोत्तरे पञ्चदशगच्छे श्रेढीक्षेत्रविशे(ष ? षे) प्रदर्श्यते~। तत्र प्रथमं
तत्क्षेत्रसम्पादनप्रकारः कथ्यते \textendash\ समायां भूमौ दक्षिणोत्तरायतं प(ञ्चां ? ञ्चद)शाङ्गुलभुजकं द्विसप्तत्यङ्गुलकोटिकं चतुरश्रं क्षेत्रमालिख्य तत्क्षेत्रमेकैकाङ्गुलान्त(र ? रि)ताभिश्चतुर्दशमी रेखाभिर्दक्षिणोत्तरदिशा पञ्चदशधा खण्डयेत्~। ततस्तस्मिन् पञ्चदश पदानि स्युः~। पुनः पूर्वापरदिशापि द्विसप्ततिधा खण्डयेत्~। तथा सति तानि पञ्चदशापि पदानि प्रत्येकं द्विसप्ततिसङ्ख्यक्षेत्रयुतानि भवन्ति~। पुनस्तेषु पदेषु पश्चिमं प्रथमं पदं परिकल्प्य तस्मिन् क्षेत्रद्वयं द्वितीयादिषु च पूर्वस्मात् पूर्वस्मात् पञ्च\renewcommand{\thefootnote}{२}\footnote{पूर्वस्मात् पञ्च \textendash\ ख. पाठः.}पञ्चोत्तराणि क्षेत्राणि दक्षिणतोऽवशेषयेत्, अन्यानि परिमार्जयेत्~। ततस्तदुक्तलक्षणं श्रेढीक्षेत्रं स्यात्~। अस्मिन् क्षेत्रे पञ्चमादीनि नवपदानीष्टानीति च कल्पयेत्~। तत्र प्रथमं तन्मध्यगतस्य पदस्य धनमानीयते पूर्वार्धेनेत्युक्तम्~। तदर्थं द्वितीयादीनां तदवधिकानां पदानां सङ्ख्या सम्पाद्यते {\qt इष्टं व्येकं दलितं सपूर्वमि}त्यनेन~। तच्चैवम्~। द्वितीयादीनि
इष्टमध्यपर्यन्तानि पदानि खल्वत्र सम्पादनीयानि~। अतो मध्यमादूर्ध्वगतानि पदानीष्टेभ्यः शोधनीयानि~। द्वितीयादीनीष्टाधःस्थितानि क्षेप्याणि च भवन्ति~। तदेवानेन क्रियते~। तत्र व्येकं दलितमित्यनेन मध्यमपदादूर्ध्वगतानां पदानां शोधनं क्रियते~। यदि यथास्थितानामिष्टानां दलनं क्रियते तदा मध्यमात् तदप्यर्धमपनीतं स्यात्~। तच्च न कर्तव्यं मध्यमस्य पदस्य ग्राह्यकोटिनिक्षिप्तत्वात्~। तेन तदुद्धृत्य शिष्टानां दलनमुक्तम्~। तथा सति मध्यमादूर्ध्वगतानि पदानि शोधितानि भवन्ति, मध्यमपदादूर्ध्वमधश्च पदानां तुल्यसङ्ख्यत्वात्~। सपूर्वमित्यनेन द्वितीयादीनामिष्टाधःस्थितानां पदानां क्षेपः क्रियते~। ननु नात्र द्वितीयादीनामेव क्षेपोऽभिहितः सपूर्वमित्यविशेषेणाभिधानात् प्रथमपदस्यापि क्षेप्यत्वोक्तेः~। नैष दोषः~। अर्धीकरणे मध्यमपदस्योद्धृतत्वात् तस्यापि क्षेप्यत्वं ज्ञातम्~। तेन तस्यैवात्र प्रथमपदव्याजेन क्षेप्यत्वमुक्तं न प्रथमपदस्य~। एवं कृते द्वितीयादीनाम् इष्टमध्यमावधिकानां पदानां सङ्ख्या स्यात्~। तया पुनरिष्टमध्यम-

\newpage

\noindent पदधनं सम्पाद्यते उत्तरगुणं समुखमित्यनेन~। तत्रोत्तरगुणमित्यनेन तद्गताश्चयाः सम्पाद्यन्ते~। समुखमित्यनेन मुखमिति विभागः~। सर्वेष्वपि पदेषु प्रथमपदतुल्यो भागो मुखम् अन्यश्चयात्मक इति द्रष्टव्यम्~। इयमत्रोपपत्तिः \textendash\ द्वितीये पदे तावदेकश्चयो भवति~। द्वितीया(द्) द्वितीये द्वौ~। तृतीये त्रयः~। एवं चतुर्थादिष्वपि द्रष्टव्यम्~। द्वितीयपदात् प्रभृत्येकैकाधिकत्वाच्चयानाम्~। अतो द्वितीयात् पदात्  यावतिथगभीष्टं पदं तावन्त एव तस्मिंश्चया भवन्ति~। अतो द्वितीयादितत्तदभीष्टपदपर्यन्तानां पदानां सङ्ख्याया चये गुणिते तत्ततदगतं चयात्मकं धनं भवति~। पुनस्तात् मुखेन च संयुक्तं तत्पदधनं स्यात्~। अत्र पुनरिष्टमध्यमपद\renewcommand{\thefootnote}{१}\footnote{मध्यपद \textendash\ ख. पाठः.}स्याभीष्टत्वात् द्वितीयादीनां तत्पर्यन्तानां पदानां सङ्ख्यायाश्चयेन गुणनमुक्तम्~। एवमानीतम् इष्टमध्यमपदधनमिष्टेन गुणितमिष्टधनं स्यादित्युक्तम्~। तत्रैषा वासना \textendash\ इष्टानां सर्वेषामपि पदानां मध्यमपदतुल्यधनत्वे यद्धनं (य ? त)दिहेष्टधनमित्यनीयते, इष्टधनेन तेषां मध्यमपदतुल्यधनत्वस्य सम्पादयितुं शक्यत्वात्~। तथाहि \textendash\ मध्यमपदात् ऊर्ध्वगतमनन्तरं यत् तत् पदं मध्यमपदादेकेन चयेनाधिकं भवति~। अधोगतमनन्तरं पदमेकेन चयेन न्यूनम्~। अत ऊर्ध्वगतस्य पदस्याधिकं चयमुद्धृत्याधोगतस्य पदस्यान्तेन योजयेत्~। तथा सति ते द्वे अपि मध्यमतुल्ये भवतः~। ऊर्ध्वगतस्याधिकेन हीनत्वादधोगतस्य न्यूनेन सहितत्वाच्च~। मध्यमपदात् पुनरेकान्तरितमूर्ध्वगतं पदं मध्यमपदाच्चयद्वयेनाधिकम्, अधोगतमेकान्तरितं पदं चयद्वयेन न्यूनम्~। अतस्तदप्यधिकं पदद्वयं न्यूनेन योजयेत्~। तथा सति ते अपि मध्यमपदतुल्ये भवतः~। पुनस्त्येवमेवोपर्युपरि स्थितानधिकांश्चयानधोऽधःस्थितेषु पदेषु क्षिपेत्~। तथा सति सर्वाण्यपीष्टपदानि मध्यमपदतुल्यानि भवन्ति~। तत्रेष्टेन मध्यमपदधने गुणित एवं सम्पादितानां पदानां धनं\renewcommand{\thefootnote}{२}\footnote{तानां धनं \textendash\ क. पाठः.} भवति~। तच्चेष्टधनेन तुल्यमिष्टधनेन सम्पादित्वादेतेषां पदानाम्~। तस्मात् साधूक्तमिष्टगणितमिष्टधनमिति~। ननु विषमेष्विष्टेषु भवदेवं, समेषु पुनर्मध्यमस्य पदस्याभावात् कथं तद्धनपूर्वकमिष्टधनं सम्पद्येत~। उच्यते~। यद्यपि समेष्विष्टेषु पारमार्थिकस्य मध्यमपदस्याभावस्तथापि तन्मध्य एकं परिकल्प्य तद्धनमानीयेष्टधनं सम्पादयितुं शक्यम्~। तत्र यदा
\newpage

\noindent द्विमुखे चतुरुत्तरे पञ्चगच्छे क्षेत्रे द्वितीयादीनि चत्वारि पदानीष्टानि स्युस्तदा तृतीयपदमध्यादारभ्य चतुर्थमध्यावधिकं प्रदेशमेकं पदं परिकल्पयेत्~। ततस्तृतीयपदाच्चयार्धेनाधिकं भवति, चतुर्थपदमध्यावधिकत्वात्~। अतः सार्धयोर्द्वितीयतृतीययोः पदयोश्चयेन गुणितयोस्तद्धनं स्यात्~। एवं सर्वेषु समेष्विष्टेषु द्रष्टव्यम्~। द्वितीयस्य प्रकारस्येयं वासना \textendash\ आद्यान्त्यधनयोगे पदार्धेन गुणिते आद्यान्त्यधनयोगतुल्यधनानि पदार्धतुल्यसङ्ख्यानि पदानि भवन्ति~। तानि चेष्टध\renewcommand{\thefootnote}{१}\footnote{पदेन \textendash\ क. पाठः.}नेन सम्पादयितुं शक्यानि~। तद्यथा \textendash\ इष्टाद्यं पदमुद्धृत्यान्त्यपदान्तेन संयोजयेत्~। तथा सति तदाद्यान्त्यधनयोगतुल्यधनं पदं भवति~। पुनर्द्वितीयं पदमुपान्त्येन पदेन संयोजयेत्~। ततस्त(दा ? द)प्याद्यान्त्यधनयोगतुल्यधनं पदं भवति~। कथम्~। द्वितीयस्य पदस्य हि द्वावंशौ~। प्रथमपदतुल्य एकोंऽशोऽन्यश्चयात्मकः~। तत्र चयात्मकेनांशेन युतमुपान्त्यं पदमन्त्यपदतुल्यं भवति, अन्त्यादुपान्त्यस्यैकेन चयेन हीनत्वात्~। तत् पुनः प्रथमपदतुल्येनांशेन च युक्तमाद्या(न्त्य)धनयोगतुल्यधनं भवति~। पुनस्तृतीयादीनि पदान्युपान्त्यादधोऽधोगतैः पदैः क्रमेण संयोजयेत्~। एवमिष्टाधोऽर्धगतानि पदान्यूर्ध्वार्धगतैः पदैः संयोजयेत्~। तथा सति तानि पूर्वोक्तन्यायेनाद्यान्त्यधनयोगतुल्यधनानि स्युः~। तैदेवमिष्टधनेनाद्यान्त्यधनयोगतुल्यधनानीष्टार्धतुल्यसङ्ख्यानि पदानि सम्पादयितुं शक्यानि~। तस्मात् तद्धनमिष्टधनमेव स्यात्~। यदि पुनः सर्वधनं सम्पादनीयं भवेत् तदा सर्वाण्यपि पदानीष्टानि परिकल्न्य तैरेवमेव सर्वधनं सम्पादयेत्~॥~१९~॥\\

अथ सर्वधने यथाकथञ्चिज्ज्ञाते तेनाज्ञातस्य गच्छस्यानयनमाह\textendash 

\begin{quote}
{\ab गच्छोऽष्टोत्तरगुणिताद्द्विगुणाद्युत्तरविशेषवर्गयुतात्~।\\
मूलं द्विगुणाद्यूनं स्वोत्तरभजितं सरूपार्धम्~॥~२०~॥}
\end{quote}

इति~। पूर्वसूत्रे सर्वधनस्यापि प्रकृतत्वात् सर्वधनमत्र विशेष्यत्वेन विवक्षितम्~। अष्टाभिरूत्तरेण च गुणिताद्द्विगुणितस्यादेरूत्तरस्य च यो\renewcommand{\thefootnote}{२}\footnote{उत्तरस्य यो \textendash\ ख. पाठः.} विशेषः, तद्वर्गेण संयुक्तात् सर्वधनात् यन्मूलं तद्द्विगुणादिना रहितं स्वीयेनोत्तरेण भक्तं रूपेण सहितमर्धीकृतं गच्छो भवतीत्यर्थः~। इयमत्र वासना \textendash\ सर्वधनं नाम श्रेढीक्षेत्रमेव~। अतस्तस्मिन्नष्टाभिर्गुणितेऽष्टौ श्रेढीक्षेत्राणि स्युः~। तैः पुनश्चत्वारि दीर्घचतुरश्राणि क्षेत्राणि सम्पादयेत्~। तत्सम्पादनप्रकार-

\newpage

\noindent श्चैवं \textendash\ प्रथम(स्ते ? न्ते)ष्वेकं पुरतो विन्यस्य तस्य मुखेनान्यस्यान्त्यं पदं, द्वितीयादिभिः पदैरुपान्त्यादीनि पदानि च\renewcommand{\thefootnote}{१}\footnote{दीनि च \textendash\ ख. पाठः.} संश्लेषयेत्~। एवं कृते तद्दीर्घचतुरश्रं क्षेत्रं भवति~। एवमेवान्यैः षड्भिरपि क्षेत्रैस्त्रीणि क्षेत्राणि सम्पादयेत्~। तान्येतानि चत्वार्यपि क्षेत्राणि सर्वधने अष्टाभिर्गुणिते भवन्ति~। तेषामायामोऽन्त्यपदमुखयोगतुल्यः, परस्परमन्त्यपदाभ्यां मुखयोः संश्लिष्टत्वात्~। विस्तारो गच्छतुल्यः~। अष्टगुणिते सर्वधने पुनरुत्तरेण गुणिते तान्येवोत्तरेण गुणितानि~। तदा तेषां चतुर्णामप्युत्तरेण गुणितत्वात् तथाविधानि चतुर्गुणितोत्तरसङ्ख्यानि क्षेत्राणि भवन्ति~। तैः पुनः क्षेत्रचतुष्टयं सम्पादयेत्~। प्रथममेकमादाय तस्यायतेन प्रदेशेनान्यस्यायतं प्रदेशं संश्लेषयेत्~। पुनस्तस्यायतेन प्रदेशेनान्यस्यायतं प्रदेशम्~। एवमेवोत्तरसङ्ख्यैः क्षेत्रैरेकं क्षेत्रं सम्पाद्यान्यानि त्रीण्यपि क्षेत्राण्येवमेव सम्पादयेत्~। तेषां च द्वे द्वे पार्श्वे गच्छोत्तरघाततुल्ये, गच्छतुल्यविस्तारैरुत्तरतुल्यसङ्ख्यैः क्षेत्रैः सम्पादितत्वात्~। अन्ये त्वन्त्यधनमुखयोगतुल्ये, तत्तुल्यायामैः क्षेत्रैः सम्पादितत्वात्~। अतोऽन्त्यधनमुखयोगादधिके गच्छोत्तरघाते\renewcommand{\thefootnote}{२}\footnote{घाते तु \textendash\ क. पाठः.} गच्छोत्तरघाततुल्यस्तेषामायामः~। अन्त्यधनमुखयोगतुल्यो विस्तारः~। यदा पुनरन्त्यधनमुखयोगात् न्यूनो गच्छोत्तरघातस्तदान्त्यधनमुखयोगतुल्य आयामः~। गच्छोत्तरघाततुल्यो विस्तारः~। यदा त्वन्त्यधनमुखयोगगच्छोत्तरघातौ तुल्यौ स्यातां तदा समचतुरश्राणि तानि क्षेत्राणीति द्रष्टव्यम्~। दीर्घचतुरश्रेष्वायामः कोटिः विस्तारो भुजेति च कल्पनीयम्~। तैः पुनः समचतुरश्रं क्षेत्रं सम्पादयेत्~। तच्चैवं \textendash\ प्रथमं तेष्वेकं पूर्वस्यांं दिशि दक्षिणोत्तरदिशा निदध्यात्~। पुनस्तस्य दक्षिणभुजया संश्लिष्टः कोटेर्भुजातुल्यः प्राग्भागो यथा स्यात् तथा द्वितीयं दक्षिणस्यां दिशि पूर्वापरदिशा निदध्यात्~। अतस्तस्य भुजाकोट्यन्तरतुल्यः पश्चाद्गतो भागः परिशिष्टो भवति~। तृतीयं पुनर्द्वितीयस्य
पश्चिमभुजासंश्लिष्टभुजाकोटितुल्यकोटिभागं पश्चिमायां दिशि दक्षिणोत्तरदिशा निदध्यात्~। अतस्तस्य भुजाकोट्यन्तरतुल्यो भाग उत्तरतः परिशिष्टः स्यात्~। चतुर्थं च तृतीयस्योत्तरभुजया संश्लिष्टभुजातुल्यकोटिभागं प्रथमस्य भुजातुल्यकोटिभागेन संश्लिष्टपूर्वभुजकमुत्तरस्यां दिशि पूर्वापरदिशा
निदध्यात्~। अतस्तयोरपि भुजाकोट्यन्तरतुल्यौ भागौ प्राग्दक्षिणतश्च परिशिष्टौ भवतः~।

\newpage
\begin{sloppypar} 
\noindent एवं कृते तत् समचतुरश्रं क्षेत्रं स्यात्~। तस्य च मध्ये भुजाकोट्यन्तरतुल्यभुजाकोटिकं क्षेत्र\renewcommand{\thefootnote}{१}\footnote{त्रं}(म)परिपूर्णं भवति, सर्वत्रापि भुजाकोट्यन्तरतुल्यस्य भागस्य परिशिष्टत्वात्~। अतस्तत्परिपूर्त्यर्थं भुजाकोट्यन्तरवर्गस्तस्मिन् क्षेप्यो जातः~। स च द्विगुणाद्युत्तरविशेषवर्गे क्षिप्ते क्षिप्तो भवति~। भुजाकोट्यन्तर(स्य) द्विगुणाद्युत्तरविशेषस्य च तुल्यत्वात्~। तथाहि \textendash\ यैश्चतुर्भिः क्षेत्रैरेतत्क्षेत्रं सम्पादितं तानि गच्छोत्तरघातेऽन्त्यधनमुखयोग\renewcommand{\thefootnote}{२}\footnote{योगान्त \textendash\ क. पाठः.}(तुल्ये) तुल्यभुजकानि, न्यूनेऽन्त्य(धन)मुखयोगतुल्यकोटिकानि गच्छोत्तरघाततुल्यभुजकानीति प्रागेव प्रदर्शितम्~। तेनोभयथापि गच्छोत्तरघातान्त्यधनमुखयोगान्तरमेव भुजाकोट्यन्तरम्~। तत्र द्विगुणमुखसहितो व्येकस्य गच्छस्योत्तरस्य च घात एवान्त्यधनमुखयोगः~। कथम्~। व्येकगच्छोत्तरघाते मुखसंयुतेऽन्त्यधनं स्यादिति पूर्वसूत्रवासनायामेव प्रदर्शितम्~। अत्र पुनरन्त्यधनमुखयोगस्य सम्पाद्यत्वाद्द्विगुणितं\renewcommand{\thefootnote}{३}\footnote{त \textendash\ ख. पाठः.} मुखं व्येकगच्छोत्तरघाते क्षेप्यं जातम्~। अतो गच्छोत्तरघातस्य द्विगुणमुखसंयुतव्येकगच्छोत्तरघातस्य चान्तरमेवास्य भुजाकोट्यन्तरम्~। तयोरन्तरं च द्विगुणितस्य मुखस्योत्तरस्य चान्तरमेव~। यदा गच्छोत्तरघाता(द्) द्विगुणमुखसहितो व्येकगच्छोत्तरघातो न्यूनः, तदा सोऽन्यस्मादुत्तरेण न्यूनः, उत्तरगुणितेनैकेन रहितत्वात्~। द्विगुणितेन मुखेनाधिकः, तस्यान्यस्मिन्नविद्यमानत्वात्~। अतो द्विगुणितेन मुखेन तुल्यो योंऽश उत्तरस्य सोऽस्मिन्नप्यस्त्येवेति न सर्वे\renewcommand{\thefootnote}{४}\footnote{वेति सर्वे}णोत्तरेणासौ न्यूनः, किन्तु द्विगुणितान्मुखादधिको योंऽश उत्तरस्य तेनैव~। अतो द्विगुणाद्युत्तरविशेष एवात्र भुजाकोट्यन्तरम्~। यदा पुनर्गच्छोत्तरघातादधिको द्विगुणमुखसहितो\renewcommand{\thefootnote}{५}\footnote{सव्ये \textendash\ क. पाठः.} व्येकगच्छोत्तरघातः तदापि सोऽन्यस्मादुत्तेरण न्यूनः, द्विगुणमुखेनाधिकः~। तत्र न सर्वेणासौ द्विगुणमुखेनाधिकः, किन्तु एकस्योत्तरस्यास्मिन्नविद्यमानत्वात् ततोऽधिकेनांशेनैव~। अतस्तत्रापि द्विगुणाद्युत्तरविशेष एव भुजाकोट्यन्तरम्~। यदा पुनर्भुजाकोट्योस्तुल्यत्वं, तदा द्विगुणाद्युत्तरयोरपि तुल्यत्वमेव स्यात्~। भुजाकोट्योस्तुल्यत्वस्यान्यथानुपपत्तेः~। एकेनोत्तेरण रहितस्य गच्छोत्तरघातस्य द्विगुणादिना रहितस्यान्यस्य च तुल्यत्वमेव स्यात्, तदा द्वयोरपि व्येकगच्छोत्तरघातत्वात्~। अतस्ताभ्यां सहितयोर्वैषम्यं तयो-
\end{sloppypar} 
\newpage

\noindent र्वैषम्येण विना न स्यात्~। अत एव द्विगुणितमुखा\renewcommand{\thefootnote}{१}\footnote{द्विगुणितान्मुखा \textendash\ ख. पाठः.}दुत्तरेऽधिके गच्छोत्तरघातस्याधिक्यम्, उत्तराद्द्विगुणमुखेऽधिकेऽन्यस्याधिक्यमित्यपि द्रष्टव्यम्~। तदेवं सर्वथापि द्विगुणाद्युत्तरविशेषतुल्यमेव भुजाकोट्यन्तरम्~। अतस्तद्वर्गे प्रक्षिप्ते तत्क्षेत्रं परिपूर्णं स्यात्~। तस्य च भुजाकोटी गच्छोत्तरघातस्य द्विगुणमुखसहितस्य व्येकगच्छोत्तरघातस्य च योगेन तुल्ये, तत्तुल्यभुजाकोटिकैः क्षेत्रैः सम्पादितत्वात्~। तस्माद्गच्छोत्तरघातस्य द्विगुणमुखसहितस्य व्येकगच्छोत्तरघातस्य च योगेन तुल्यं तन्मूलम्~। तत् पुनर्द्विगुणादिना रहितं गच्छोत्तरघातस्य व्येकगच्छो(त्तरघा)तस्य च योगः स्यात्~। तस्मिन्नुत्तरेण भक्ते गच्छस्य व्येकगच्छस्य च योगो भवेत्~। स पुनरेकेन युक्तो द्विगुणितो गच्छः स्यात्~। अतस्तस्मिन्नर्धीकृते गच्छो भवति~॥~२०~॥\\

अथ चितिघनानयनोपायमाह\textendash 

\begin{quote}
{\ab एकोत्तराद्युपचितेर्गच्छाद्येकोत्तरत्रिसंवर्गः~।\\
षड्भक्तः स चितिघनः सैकपदघनो विमूलो वा~॥~२१~॥}
\end{quote}

इति~। गच्छाद्येकोत्तरत्रिसंवर्गो यः स षड्भक्तः एकोत्तराद्युपचितेश्चितिघनो भवतीति योजना~। गच्छादीनामेकैकोत्तराणां त्रयाणां राशीनां संवर्गो गच्छाद्येकोत्तरत्रिसंवर्गः~। गच्छ एवैको राशिः, सैको गच्छो द्वितीयः, द्वियुतो गच्छस्तृतीयः~। उपचितिरित्युपचितधनसमूहात्मकं श्रेढीक्षेत्रमुच्यते~। एकसङ्ख्यावुत्तरा(दि ? दी) यस्या उपचितेः सा तथोक्ता~। अनेन यत्रोत्तरस्यादेश्चैकसङ्ख्यत्वं तत्रै\renewcommand{\thefootnote}{२}\footnote{तथै \textendash\ क. पाठः.}वैतद्गणितं युक्तं नान्यत्रेति द्योत्यते~। सर्वधनेऽन्त्यपदविहीनरय क्षेत्रस्य धनं प्रक्षिपेत्~। तद्योगे पुनरन्त्योपान्त्याभ्यां विहीनस्य क्षेत्रस्य धनं प्रक्षिपेत्~। एवमन्त्यपदादारभ्यैकैकेन पदेन विहीनस्य क्षेत्रस्य धनानि पूर्वस्मिन् पूर्वस्मिन् योगे प्रक्षिपेत्~। एवं जातो यो राशिः सोऽत्र चितिघन इत्युच्यते~। पुनः प्रकारान्तरेण चितिघनानयनमाह \textendash\ {\qt सैकपदघनो विमूलो वे}ति~। सैकस्य पदस्य गच्छस्य घनः स्वमूलेन सैकपदेन विहीनः षड्भक्तो वा चितिघन इत्यर्थः~। अथ वासना \textendash\ अत्र हि गच्छाद्येकोत्तरत्रिसंवर्गे षड्भक्ते चितिघनो भवतीत्युक्तम्~। तच्च षड्गुणितस्य चितिघनस्य गच्छाद्येकोत्तरत्रिसंवर्गस्य च तुल्यत्व एव युक्तमिति तयोस्तुल्यत्वं

\newpage

\noindent प्रदर्श्यते~। {\qt सदृशत्रयसंवर्ग} इत्युक्तन्यायेन गच्छतुल्योत्सेधं सैकगच्छतुल्यविस्तारं द्वियुतगच्छतुल्यायामं क्षेत्रमेव हि गच्छाद्येकोत्तरत्रिसंवर्गः~। समत्रिघात उत्सेधादयस्तुल्याः, विषमत्रिघाते विषमा इत्येतावानेव विशेषः~। एतत् पुनः क्षेत्रं षड्गुणितेन चितिघनेनापि सम्पादयितुं शक्यम्~। तथाहि \textendash\ सर्वधने षड्गुणिते षट् श्रेढीक्षेत्राणि भवन्ति~। तैः पुनः पूर्वसूत्रोक्तप्रकारेण त्रीणि दीर्घचतुरश्राणि क्षेत्राणि सम्पादयेत्~। अतस्तेषां कोटिरन्त्यधनमुखयोगतुल्या, भुजा गच्छतुल्या इति पूर्वसूत्रे प्रदर्शितम्~। अत्र श्रेढीक्षेत्रफलानामुत्सेधायामविस्तारा एकाङ्गुलपरिमिताः कल्पिताः~। तेनैतानि क्षेत्राण्येकाङ्गुलोत्सेधानि द्रष्टव्यानि~। पुनरन्त्यधनहीने सर्वधने षड्गुणितेऽन्त्यपदहीनानि षट् श्रेढीक्षेत्राणि भवन्ति~। तैरपि पूर्ववदेव\renewcommand{\thefootnote}{१}\footnote{वं \textendash\ क. पाठः.} त्रीणि क्षेत्राणि कुर्यात्~। तेषामायामविस्तारौ पूर्वेभ्य एकाङ्गुलोनौ~। उपान्त्यधनमुखयोगतुल्यो ह्यत्रायामः~। उपान्त्यधनं चान्त्यधनादेकोनम्~। अत एकाङ्गुलोनत्वमाया(मा)ङ्गुल(स्य)~। विस्तारस्यैकाङ्गुलोनत्वं स्पष्टम्~। सर्वेषामप्येकाङ्गुलपरिमित एवोत्सेधः, सर्वेषां क्षेत्रफलानां तथात्वात्~। पुनरन्त्योपान्त्यधनहीने सर्वधने षड्गुणितेऽन्त्योपान्त्यपदहीनानि षट् श्रेढीक्षेत्राणि स्तुः~। तैरपि पूर्ववत् त्रीणि
क्षेत्राणि कुर्यात्~। तानि चोक्तन्यायेन स्वपूर्वेभ्य एकाङ्गुलोनायामविस्ताराणि~। पुनरप्येवमेव पूर्वस्मात् पूर्वस्मादेकैकपदहीनैर्मुखावधिकैः षड्गुणितैः
श्रेढीक्षेत्रावयवैः स्वस्वपूर्वेभ्य एकैकाङ्गुलोनायामविस्ताराणि त्रीणि त्रीणि क्षेत्राणि कुर्यात्~। अतएव तेषां त्रिकाणि गच्छतुल्यसङ्ख्यानि भवन्ति~। एवं चितिघने षड्गुणित एतानि क्षेत्राणि भवन्ति~। तैः पुनर्गच्छाद्येकोत्तरत्रिसंवर्गात्मकं क्षेत्रं सम्पादयेत्~। तद्यथा \textendash\ प्रथमं सर्वधनेन कृतानि क्षेत्राण्यादाय तेष्वेकं भूमौ दक्षिणोत्तरायतं विन्यस्य द्वितीयं तस्य पश्चिमतस्तत्पार्श्वेन संस्पृष्टं कृत्वा भूमौ विन्यसेत्~। तत्रायं विशेषः \textendash\ यथा तस्य विस्तार ऊर्ध्वधोदिशा स्थितो भवेत्, आयामश्च दक्षिणोत्तरदिशा, तथा भि(त्रि ? त्ति)रूपेण विन्यस्याधः कर्तव्यः~। तृतीयं पुनः प्रथमस्योत्तरतः तत्पूर्वपार्श्वेन समपूर्वपार्श्वमुत्तरपार्श्वेन संश्लिष्टं पूर्वापरायतं पश्चिमस्योत्तरपार्श्वेन सकलेन संश्लिष्टं कृत्वा भित्त्याकारेण निदध्यात्~। तथा सति गच्छाद्येकोत्तरत्रिसंवर्गात्मकस्य क्षेत्रस्योत्तरपश्चिमगते पार्श्वे भवेताम्~। तत्रोत्तरपार्श्वं सैकगच्छतुल्यायामम्, उत्तर-

\newpage

\noindent पार्श्वतया विन्यस्तस्य क्षेत्रस्य तथात्वात्~। कथम्~। सर्वधनेन कृतानां क्षेत्राणामायामोऽन्त्यधनमुखयोगतुल्य इति प्राक् प्रदर्शितम्~।
अन्त्यधनमुखयोगश्चैकोत्तराद्युपचितौ सैकगच्छतुल्य एव स्यात्, तत्रादेरुत्तरस्य चैकसङ्ख्यत्वेनान्त्यधनस्य गच्छेन तुल्यत्वात्, एकेन मुखेन युक्तस्य तस्य\renewcommand{\thefootnote}{१}\footnote{मुखेन तस्य} सैकगच्छतुल्यत्वाच्च~। पश्चिमपार्श्वं द्वियुतगच्छतुल्यायामम्~। पश्चिमक्षेत्रायामस्योत्तरक्षेत्रपश्चिमपार्श्वविस्तारस्य च योग एव हि तत्पार्श्वम्~। तत्र पश्चिमक्षेत्रायामः सैकगच्छतुल्य इत्युक्तम्~। अन्यस्त्वेकाङ्गुलपरिमितः, उत्तरक्षेत्रस्य प्राक्तन उत्सेधो हि सः~। स चैकाङ्गुलपरिमित इत्युक्तम्~। अतो द्वियुतगच्छतुल्यस्तत्पार्श्वायामः~। द्वयोरपि पार्श्वयोरुत्सेधो गच्छतुल्यः, क्षेत्रयोर्विस्तारस्य तथात्वात्~। एवमेतैस्त्रिभिः क्षेत्रैर्द्वादशाश्रस्य क्षेत्रस्योत्सेधविस्तारायाभेष्वेकैकाङ्गुलपरिमितः\renewcommand{\thefootnote}{२}\footnote{क्षेत्रस्योत्सेधायामविस्तारेष्वेकैकाङ्गुलपरिमितः \textendash\ ख. पाठः.} प्रदेशः परिपूर्णो जातः, तेषामेकाङ्गुलोत्सेधस्य प्राक् प्रदर्शितत्वात्~। तत्प्रकारश्चैवम् \textendash\ अन्त्यधनहीनेन सर्वधनेन कृतानि क्षेत्राण्यादाय तेष्वेकमुत्तरतो विन्यस्तस्य दक्षिणपार्श्वेन, पश्चिमतो विन्यस्तस्य पूर्वपार्श्वेन च संश्लिष्टं पूर्वापरायतं प्रथमतो विन्यस्तस्योपरि भित्त्याकारेण विन्यस्येत्~। तदा त(दु)त्सेध उत्तरपश्चिमयोरुत्सेधेन समो भवति~। तद्विस्तारस्यैकाङ्गुलोनत्वेन
सम्भाव्यमानस्योत्सेधवैषम्यस्यैकाङ्गुलोत्सेधस्य प्रथमतो विन्यस्तस्योपरि विन्यस्त(स्य ?)त्वेन परिहृतत्वात्~। प्राग्भागोऽपि प्रथमोत्तरयोः प्राग्भागाभ्यां समः~। उत्तरतो विन्यस्तादेकाङ्गुलोनो ह्यस्यायामः~। तत्पश्चिमपार्श्वात् पूर्वत एकाङ्गुलपरिमिते प्रदेश एवा(न्य ? स्य) पश्चिमपार्श्वं भवति, तस्य च पश्चिमपार्श्वावधिकत्वादस्य च तत्पूर्वपार्श्वावधिकत्वात्~। अतस्तयोः पूर्वपार्श्वे समे एव भव(ति ? तः)~। अतएव प्रथमं विन्यस्तस्यास्य च पूर्वपार्श्वे अपि समे~। द्वितीयं पुनः पश्चिमस्य पूर्वपार्श्वेन द्वितीयोत्तरस्य दक्षिणपार्श्वेन च संश्लिष्टं दक्षिणोत्तरायतं प्रथमस्योपरि भित्याकारेणैव विन्यस्येत्~। ततस्तस्याप्युत्सेधोऽन्यैः समो भवति~। दक्षिणपार्श्वं च प्रथमपश्चिमयोर्दक्षिणपार्श्वाभ्यां समं भवति~। पश्चिमादेकाङ्गुलोनो ह्यस्यायामः~। पश्चिमस्योत्तरपार्श्वाद्दक्षिणत एकाङ्गुलपरिमिते प्रदेशे चास्योत्तरं पार्श्वम्~। तत्र द्वितीयो(त्तरो ?)त्तरपार्श्वावधिकत्वात् अस्य च तद्दक्षिणपार्श्वाव-

\newpage
\begin{sloppypar} 
\noindent धिकत्वात्~। अतस्तयोर्दक्षिणपार्श्वयोः साम्यमेव स्यात्~। एवं कृते क्षेत्रस्यायामे विस्तारे च द्व्यङ्गुलपरिमितः प्रदेशः परिपूर्णः स्यात्~।
अतोऽपरिपूर्णस्य प्रदेशस्यायामो गच्छतुल्यः, विस्तारो व्येकगच्छतुल्यः~। तत्र\renewcommand{\thefootnote}{१}\footnote{गच्छतुल्यः~। तत्र \textendash\ ख. पाठः.} पुनस्तृतीयं (प्रथम)द्वितीयाभ्यां पश्चिमोत्तराभ्यां क्रमेण संश्लिष्टपश्चिमोत्तरपार्श्वं दक्षिणोत्तरायतं प्रथमस्योपरि तदनुसारेण विन्यस्येत्~। तदा तस्य पूर्वदक्षिणे पार्श्वे प्रथमस्य पूर्वदक्षिणपार्श्वाभ्यां समे भवतः, अपरिपूर्णस्य प्रदेशस्य च तुल्यायामविस्तारत्वात्~। एवं द्वितीये त्रिके विन्यस्तेऽपि क्षेत्र\renewcommand{\thefootnote}{२}\footnote{विन्यस्ते क्षेत्र}स्योत्सेधायामविस्ताराणामेकैकाङ्गुलपरिपूर्तिर्जाता~। एवमन्येष्वपि त्रिकेषु क्रमेणैवमेव विन्यस्तेष्वेकैकाङ्गुलपरिपूर्तिर्द्रष्टव्या~। तत्र व्येकगच्छसङ्ख्येषु त्रिकेषु विन्यस्तेषु क्षेत्रस्योत्सेधायामविस्तारेषु व्येकगच्छेतुल्यो भागः परिपूर्णो भवति~। तदानीमपरिपूर्णः प्रदेशस्त्र्यङ्गुलायामः, द्वियुतगच्छतुल्यायामत्वात् क्षेत्रस्य~। तस्य विस्तारो द्व्यङ्गुलपरिमितः, क्षेत्रस्य सैकगच्छतुल्यविस्तारत्वात्~। उत्सेध एकङ्गुलपरिमितः, गच्छतुल्योत्सेधत्वात् क्षेत्रस्य~। पुनरुत्तरपश्चिमयोर्विन्यस्तयोरपरिपूर्णस्य प्रदेशस्यायामो द्व्यङ्गुलपरिमितः~। विस्तारोत्सेधावेकाङ्गुलपरिमितौ~। अन्त्यत्रिकमपि द्व्यङ्गुलायामम् एकाङ्गुलविस्तारोत्सेधम्~। अतस्त(त्तृती ? त्त्रित)ये तत्र विन्यस्ते क्षेत्रं परिपूर्णं स्यात्~। तस्माद्गच्छाद्येकोत्तरत्रिसंवर्गः षड्गुणितेन चितिघनेन तुल्यः~। अतस्तस्मिन् षड्भिर्भक्ते चितिघनो भवतीत्युक्तमुपपन्नम्~। द्वितीयस्य प्रकारस्येयं वासना \textendash\ सैकपदतुल्योत्सेधायामविस्तारं घनक्षेत्रमेव हि सैकपदघनः~। अतस्तस्मात् सैकपदे शोधिते तत्क्षेत्रं सैकपदतुल्यायामेनैकाङ्गुलोत्सेधविस्तारेण खण्डेन हीनं स्यात्~। तच्च क्षेत्रं पूर्वप्रदर्शितेन क्षेत्रेण सम्पादयितुं शक्यम्~। तद्यथा \textendash\ पूर्वप्रदर्शितं क्षेत्रं दक्षिणपार्श्वादुत्तरत एकाङ्गुलपरिमिते प्रदेशे पूर्वापरदिशा खण्डयेत्~। तथा सत्युत्तरं खण्डं सैकपदतुल्यायामविस्तारं भवति~। दक्षिणे खण्डे पुनः शायिते तस्योत्सेध एकाङ्गुलपरिमितः, विस्तारः पदतुल्यः, आयामः सैकपदतुल्यः~। एतत् पुनरुत्तरस्योपरि पूर्वापरायतं तद्दक्षिण\renewcommand{\thefootnote}{३}\footnote{दक्षिण}पार्श्चेन\renewcommand{\thefootnote}{४}\footnote{पार्श्वोनं \textendash\ क. पाठः.} समदक्षिणपार्श्वं कृत्वा विन्यसेत्~। ततोऽस्य पूर्वापरे पार्श्वे अपि तस्य पूर्वापरपार्श्वाभ्यां समे स्यातां, तस्य विस्तारस्यायामस्य च सैकपदतुल्यत्वात्~।
\end{sloppypar} 
\newpage

\noindent उत्तरं पार्श्वं तु तस्योत्तरपार्श्वात् दक्षिणत एकाङ्गुलपरिमिते प्रदेशे भवति, अस्य विस्तारस्य पदतुल्यत्वात् तस्यायामस्य च सैकपदतुल्यत्वात्~। अतस्तत्रैकाङ्गुलविस्तारोत्सेधं सैकपदतुल्यायामं खण्डमपरिपूर्णं भवति~। ततो दक्षिणत उत्सेधोऽपि सैकपदतुल्यः, क्षेत्रस्य गच्छतुल्योत्सेधस्योपर्येकाङ्गुलोत्सेधस्य विन्यस्तत्वात्~। एवं पूर्वप्रदर्शितेन क्षेत्रेण सम्पादितं सैकपदायामेनैकाङ्गुलोत्सेधविस्तारेण खण्डेन हीनं सैकपदतुल्योत्सेधविस्तारायामं\renewcommand{\thefootnote}{१}\footnote{सैकपदतुल्योत्सेधायामविस्तारं \textendash\ ख. पाठः.} क्षेत्रमिदं जातम्~। अथवा अत्र यत् सैकपदस्य घनीकरणं तस्मात् स्व\renewcommand{\thefootnote}{२}\footnote{स्वे}मूलविशोधनं च क्रियते~। तेन गच्छाद्येकोत्तराणां त्रयाणां राशीना संवर्ग एव प्रकारान्तरेण क्रियते~। तथाहि \textendash\ गच्छाद्येकोत्तरत्रिसंवर्गे कर्तव्ये हि प्रथमं पदस्य सैकपदेन गुणनं कर्तव्यम्~। अत्र पुनः प्रथमं सैकपदस्य सैकपदेन गुणनं क्रियते~। (ते)न पदसैकयोः संवर्गात् सैकपदतुल्या या सङ्ख्या (त)याधिकोऽयं\renewcommand{\thefootnote}{३}\footnote{यो} राशिर्भवति, सैकपदेन गुणितेनाधिकत्वात्~। अतोऽस्माद्राशेः सैकपदं विशोध्य शिष्टेद्वियुतेन पदेन निहते गच्छाद्येकोत्तरत्रिसंवर्गः स्यात्~। अत्र पुनस्तद्विशोधनं\renewcommand{\thefootnote}{४}\footnote{पुनर्विशोधनं} क्रियते सैकेन पदेनैव च गुण्यते~। अतस्तद्गुणिताद्राशेः सैकपदेन गुणितं सैकपदं शोध्यं जातं, शोध्यस्य सैकपदस्य सैकपदेन गुणितत्वात्~। पदसैकपदघातः क्षेपोऽपि जातः, गुणकारस्यैकहीनत्वेनास्य संवर्गस्यैकगुणितेन पदसैकपदघातात्मकेन गुण्यराशिना हीनत्वात्~। अतः पदसैकपदघातस्य सैकपदवर्गस्य च यो विश्लेषः स एवात्र शोधनीयः~। स च सैकपदतुल्यः~। अतः सैकपदघनात् सैकपदे शोधिते गच्छाद्येकोत्तरत्रिसंवर्गो भवतीति युक्तम्~॥~२१~॥\\

अथ वर्गचितिघनघनचितिघनयोरानयनमाह\textendash 

\begin{quote}
{\ab सैकसगच्छपदानां क्रमात् त्रिसंवर्गितस्य षष्ठोंऽशः~।\\
वर्गचितिघनः स भवेच्चितिवर्गो घनचितिघनश्च~॥~२२~॥}
\end{quote}

इति~। सैक (स) गच्छपदानामित्यनेन सैकपदसगच्छ\renewcommand{\thefootnote}{५}\footnote{सैकपदगच्छ \textendash\ क. पाठः.}सैकपदपदानि विवक्षितानि~। त्रिसंवर्गितस्य, संवर्ग एव संवर्गितं, त्रिसंवर्गस्येत्यर्थः~। चितिवर्गः सङ्कलितवर्गः~। एतदुक्तं भवति \textendash\ सैकसगच्छपदानां संवर्गस्य षष्ठोंऽशः संवर्गचितिघनो भवति, चितिवर्गो घनचितिघनश्च भवतीति~। गच्छवर्गेऽन्त्य-

\newpage

\noindent पदहीनस्य गच्छस्य वर्गं प्रक्षिपेत्~। तस्मिन् पुनरन्त्योपान्त्यपदहीनस्य गच्छस्य वर्गं प्रक्षिपेत्~। तस्मिन् पुनः पदत्रयहीनस्य गच्छस्य वर्गं प्रक्षिपेत्~। पुनरप्येकैकपदहीनस्य गच्छस्य वर्गं पूर्वस्मिन्नेव राशौ प्रक्षिपेत्~। एवं जातो यो राशिः सोऽत्र वर्गचितिघन इत्युच्यते~। घनचितिघनेऽप्येष एव न्यायः~। तत्र गच्छघनेऽन्येषां घनानां प्रक्षेपः कर्तव्य इत्येतावानेव विशेषः~। अथ वासना~। तत्र प्रथमं\renewcommand{\thefootnote}{१}\footnote{वासना~। प्रथमं \textendash\ क. पाठः.} वर्गचितिघनवासना प्रदर्श्यते \textendash\ अत्र हि सैकपदादि राशित्रयसंवर्गस्य षष्ठोंऽशो वर्गचितिघनो भवतीत्युक्तम्~। तच्च षड्गुणितस्य वर्गचितिघनस्य राशित्रयसंवर्गस्य च तुल्यत्व एव युक्तमिति तयोस्तुल्यत्वं प्रदर्श्यते~। पदतुल्योत्सेधं सैकपदतुल्यविस्तारं सैकसगच्छ\renewcommand{\thefootnote}{२}\footnote{विस्तारं सगच्छ \textendash\ ख. पाठः.}पदतुल्यायाम क्षेत्रं राशित्रयसंवर्गः~। एतत् पुनः क्षेत्रं षड्गुणितेन वर्गचितिघनेनापि सम्पादयितुं शक्यम्~। तथाहि \textendash\ गच्छवर्गे षड्गुणिते षड्गच्छवर्गात्मकानि क्षेत्राणि भवन्ति~। तेषु द्वे द्वे क्षेत्रे संयोज्य त्रीणि क्षेत्राणि सम्पादयेत्~। अतएव तेषामायामो द्विगुणितेन गच्छेन तुल्यः~। विस्तारो गच्छतुल्यः~। पुनरेकैकहीनस्य गच्छस्य वर्गैः षड्गुणितैरेवमेव त्रीणि त्रीणि क्षेत्राणि सम्पादयेत्~। अतस्तेषां विस्तारः पूर्वेभ्यः पूर्वेभ्य एकैकहीनः, आयामो द्वाभ्यां द्वाभ्यां हीनः~। सर्वेषामप्येतेषामुत्सेध एकसङ्ख्यो द्रष्टव्यः~। एतैः पुनः क्षेत्रैस्तद्राशित्रयसंवर्गात्यकं क्षेत्रं सम्पादयेत्~। तद्यथा \textendash\ प्रथमं गच्छवर्गेण कृतानि क्षेत्राण्यादाय तेष्वेकं दक्षिणोत्तरायतं विन्यस्य द्वितीयं तत्पश्चिमतः पूर्ववद्भित्त्याकारेण विन्यस्येत्~। तृतीयं पुनः सैकगच्छतुल्ये भागे खण्डयेत्~। तथा सति तयोः खण्डयोरेकं सैकगच्छतुल्यायामं गच्छतुल्यविस्तारम्~। अन्यस्य पुनः प्राक्तन आयामो व्येकगच्छतुल्यः~। स पुनरिदानीं विस्तारः परिकल्पनीयः~। प्राक्तनो विस्तार इदानीमायामश्च परिकल्पनीयः, न्यूनस्य भागस्य विस्तारत्वेनाधिकस्यायामत्वेन च प्रसिद्धत्वात्~। अतस्तद्गच्छतुल्यायामं व्येकगच्छतुल्यविस्तारम्~। तयोः सैकगच्छतुल्योयामं खण्डमुत्तरतो द्वयोरप्युत्तरपार्श्वाभ्यां संश्लिष्टं पूर्वापरायतं विन्यस्येत्~। तथा साते राशित्रयसंवर्गात्मकस्य क्षेत्रस्य पश्चिमोत्तरपार्श्वे स्याताम्~। तत्रोत्तरंं पार्श्वं सैकगच्छतुल्यायामं, तत्र विन्यस्तस्य क्षेत्रस्य तथात्वात्~। पश्चिमं पार्श्वं सगच्छसैकपदतुल्यायामम्~। पश्चिमक्षेत्रायामस्योत्तरक्षेत्रपश्चिमपार्श्वविस्तारस्य च योग

\newpage

\noindent एव हि तत्पार्श्वम्~। तत्र पश्चिमक्षेत्रायामो द्विगुणितेन गच्छेन तुल्य इत्युक्तम् अन्यस्त्वेकाङ्गुलपरिमितः~। अतः सगच्छसैकपदतुल्यस्तत्पार्श्वायामः~। द्वयोरपि पार्श्वयोरुत्सेधो गच्छतुल्यः, क्षेत्रयोर्विस्तारस्य तथात्वात्~। पुनरन्यत् क्षेत्रं प्रथमविन्यस्तस्योपरि दक्षिणतः पूर्वापरायतं विन्यस्येत्~। तत्र तस्य दक्षिणं पार्श्वं प्रथमविन्यस्तस्य दक्षिणपार्श्वस्य समोपरिष्टाद्यथा भवेत् तथा विन्यासः कर्तव्यः~। पश्चिमं पार्श्वं पश्चिमस्य पूर्वपार्श्वेन संश्लिष्टं कर्तव्यम्~। तथा सति क्षेत्रस्य दक्षिणं पार्श्वं रयात्~। तच्च सैकगच्छतुल्यायामम्~। कथम्~। दक्षिणतो विन्यस्तं क्षेत्रं गच्छतुल्यायाममित्युक्तम्~। तत् पुनः पश्चिमस्यैकाङ्गुलपरिमितेन दक्षिणपार्श्वेन\renewcommand{\thefootnote}{१}\footnote{पार्श्वे \textendash\ क. पाठः.} युक्तं सैकगच्छतुल्यं स्यात्~। दक्षिणपार्श्वस्याप्युत्सेधो गच्छतुल्यः, तत्र विन्यस्तस्य व्येकगच्छतुल्यविस्तारस्य क्षेत्रस्यैकाङ्गुलोत्सेधस्य प्रथमविन्यस्तस्योपरि विन्यस्तत्वात्~। पुन(र्वे ? र्व्ये)कगच्छवर्गेण सम्पादितेषु क्षेत्रेष्वेकं गच्छतुल्ये भागे खण्डयेत्~। ततस्तयोः खण्डयोरेकं गच्छतुल्यायामं व्येकगच्छतुल्यविस्तारम्~। अन्यस्य पूर्ववदायामविस्तारयोर्व्यत्यासे कल्पिते व्येकगच्छतुल्य आयामः, द्वाभ्यां हीनेन गच्छेन तुल्यो विस्तारः~। कथम्~। प्रथमात् त्रिकात् द्वाभ्यां हीनो द्वितीयस्य त्रिकस्यायाम इत्युक्तम्~। प्रथमत्रिक(श्च ? ञ्च) द्विगुणितगच्छतुल्यायामम्~। अतो द्विहीनस्य गच्छस्य च योगेन तुल्यो द्वितीयत्रिकायामः~। अतस्तस्मिन् गच्छतुल्ये प्रदेशे खण्डिते शिष्टस्य द्विहीनगच्छतुल्य आयामः~। स एवात्र विस्तारत्वेन कल्पितः~। अतो द्विहीनगच्छतुल्योऽस्य विस्तारः~। एवमेतानि चत्वारि क्षेत्राणि भवन्ति~। तेषु (तेषु ?) प्रथ(मः ? मं) गच्छतुल्यायामं क्षेत्रमुत्तरस्य दक्षिणतस्तेन पश्चिमेन च संश्लिष्टं पूर्वापरायतं भित्त्याकारेण विन्यस्येत्~। तथा सति तस्योत्सेधः प्रथमस्योत्तरस्योत्सेधेन तुल्यः स्यात्, व्येकगच्छतुल्यविस्तारत्वात्~। पूर्वपार्श्वं च तेन समं गच्छतुल्यायामत्वात्~। पुनरखण्डितयोरकं पश्चिमस्य पूर्वतस्तेन द्वितीयोत्तरेण च संश्लिष्टं दक्षिणोत्तरायतं विन्यस्येत्~। ततस्तस्यापि पूर्ववदुत्सेधोऽन्येन तुल्यः~। दक्षिणपार्श्वं पुनः प्रथमस्य पश्चिमस्य दक्षिणपार्श्वादुत्तरत एकाङ्गुलपरिमिते प्रदेशे स्यात्~। कथम्~। प्रथमात् पश्चिमा(द्य ? द् द्व्य)ङ्गुलहीनोऽस्यायाम इत्युक्तम्~। तत्र
यद्युभयोरप्युत्तरपार्श्वे समे स्यातां तर्हि दक्षिणपार्श्वयोर्द्व्यङ्गुलमन्तरं स्यात्~। अत्र पुनः प्रथमस्योत्तरपार्श्वाद्दक्षिणत एकाङ्गुलपरिमिते प्रदेशे
द्वितीयस्योत्तरं

\newpage

\noindent पार्श्वम्~। अतो दक्षिणपार्श्वयोरेकाङ्गुलपरिमितमन्तरं स्यात्~। दक्षिणस्योत्तरपार्श्वं च प्रथमस्य पश्चिमस्य, प्रथमं भूमौ विन्यस्तस्य च
दक्षिणपार्श्वाभ्यामुत्तरत एकाङ्गुलपरिमिते प्रदेशे भवति, प्रथमं विन्यस्तस्योपरि विन्यस्तत्वात्~। अतो द्वितीयस्य पश्चिमस्य दक्षिणपार्श्वं दक्षिणेन संश्लिष्टं भवति~। पुनरन्यदखण्डितं क्षेत्रं दक्षिणोत्तरायतं प्रथमं विन्यस्तस्योपरि तदनुसारेण विन्यस्येत्~। तथा सति तस्योत्तरं पार्श्वं द्वितीयेनोत्तरेण, दक्षिणं पार्श्वं दक्षिणेन च संश्लिष्टं स्यात्, प्रथमं विन्यस्तात् द्व्यङ्गुलहीनायामत्वात्~। पश्चिमं पार्श्वं द्वितीयेन पश्चिमेन संश्लिष्टं भवति~। अतएव पूर्वपार्श्वं प्रथमं
विन्यस्तस्य पूर्वपार्श्वेन समं स्यात्, व्येकगच्छतुल्यविस्तारत्वात्~। पुनरन्यत् क्षेत्रं दक्षिणस्योत्तरपार्श्वेन पश्चिम(सा ? स्य) पूर्वपार्श्वेन च सं\renewcommand{\thefootnote}{१}\footnote{पार्श्वेन सं}श्लिष्टं पूर्वापरायतं विन्यस्येत्~। तथा सति तस्य पूर्वपार्श्वं दक्षिणस्य चाधोगतयोश्च पूर्वपार्श्वैः समं स्यात्, व्येकगच्छतुल्यायामत्वात्~। उत्सेधोऽप्यन्यैस्तुल्यः,
द्विहीनगच्छतुल्यविस्तारत्वात् द्वयोरुपरि विन्यस्तत्वाच्च~। पुनरन्यानि त्रिकाण्यप्यनेनैव न्यायेन विन्यस्येत्~। ततस्तत् क्षेत्रं परिपूर्णं स्यात्~। तत्रैकैकस्मिंस्त्रिके विन्यस्ते क्षेत्रस्य विस्तारोत्सेधयोरेकैकाङ्गुलपरिपूर्तिर्भवति~। आयामस्य च द्व्यङ्गुलपरिपूर्तिः~। उभयत्रापि क्षेत्रविन्यासात्~। अतो
व्येकगच्छतुल्यसङ्ख्येषु त्रिकेषु विन्यस्तेषु क्षेत्रस्योत्सेधे विस्तारे च तावान् प्रदेशः परिपूर्णो भवति~। आयामे तु द्विहीनेन द्विगुणितगच्छेन तुल्यः प्रदेशः परिपू(र्णं ? र्णः) स्यात्, उत्तरतो दक्षिणतश्च व्येकगच्छतुल्यस्य प्रदेशस्य परिपूर्णत्वात्~। अतस्तत्रापरिपूर्णस्य प्रदेशस्योत्सेध एकाङ्गुलपरिमितः~। विस्तारो द्व्यङ्गुलपरिमितः~। आयामस्त्र्यङ्गुलपरिमितः~। सोऽपि शिष्टे त्रिके विन्यस्ते परिपूर्णो भवति~। अत्रोत्तरतो विन्यस्तव्यस्य
खण्डनं न कार्यम्~। अतएव दक्षिणतो विन्यासश्च न कर्तव्य इति विशेषः~। तत्रोत्तरपश्चिमयोर्विन्यस्तयोर्द्व्यङ्गुलायाम एकाङ्गुलोत्सेधविस्तारः
प्रदेशोऽपरिपूर्णो भवति~। स च द्व्यङ्गुलायामेनैकाङ्गुलोत्सेधविस्तोरण क्षेत्रेण परिपूर्णो भवतीति~। अथ घनचिति\renewcommand{\thefootnote}{२}\footnote{अथ चिति \textendash\ क. पाठः.}घनवासना \textendash\ अत्र चितिवर्गो घनचितिघनो भवतीत्युक्तम्~। तच्च चितिवर्गघनचितिघनयोस्तुल्यत्व एव युक्तम्~। तयोस्तुल्यत्वं च चितिवर्गेण घनचितिघनस्य सम्पादयितुं शक्य(त्वं ? त्वे)

\newpage

\noindent युक्तम्~। तत्सम्पादनप्रकारश्चैवम् \textendash\ तत्र प्रथमं चितितुल्यभुजाकोटिकमेकाङ्गुलोत्सेधं चितिवर्गात्मकं क्षेत्रं सम्पादयेत्~।
तत्पुनराग्नेयकोणादुत्तरतः पश्चिमतश्च गच्छतुल्यात् प्रदेशादारभ्य पूर्वापरदिशा दक्षिणोत्तरदिशा च खण्डयेत्~। तथा सति तानि चत्वारि क्षेत्राणि भवन्ति~। तत्राग्नेयकोणगतं गच्छतुल्यभुजाकोटिकं, कोणादुभयतोऽपि तत्तुल्ये प्रदेशे खण्डितत्वात्~। तस्योत्तरपश्चिमगते गच्छतुल्यविस्तारे~। आयामस्तु तयोरन्त्यधनहीनचितितुल्यः, गच्छतुल्येन भागेन हीनत्वात्, एकोत्तराद्युपचितौ गच्छान्त्यघनयोस्तुल्यत्वाच्च~। अन्यदन्त्यधनहीनचितितुल्यभुजाकोटिकं, पूर्वतो दक्षिणतश्च गच्छतुल्येन भागेन हीनत्वात्~। पुनः कोणगतस्योभयपार्श्वगते वक्ष्यमाणप्रकारेण खण्डयेत्~। तत्र तत्समीपवर्तिनी व्येकगच्छतुल्यविस्तारे कार्ये~। तदनन्तरे द्विहीनगच्छतुल्यविस्तारे~। पुनस्त्रिहीनगच्छतुल्यविस्तारे~। पूर्वस्मात् पूर्वस्मादेकैकाङ्गुलहीनविस्ताराणि तानि यथा स्युस्तथा खण्डनं कार्यम्~। एवं खण्डित उभयत्रापि व्येकगच्छसङ्ख्यानि गच्छतुल्यायामानि क्षेत्राणि भवन्ति~। कथं पुनस्तेषां व्येकगच्छतुल्यसङ्ख्यत्वम्~। उच्यते~। अन्त्यधनहीनचितितुल्यं तयोरायाम इति प्राक् प्रदर्शितम्~। अत उपान्त्यादीनि पदानि तयोः क्रमेण स्थितानि~। तानि च व्येकगच्छतुल्यसङ्ख्यानि~। एकैकस्य पदस्यान्ते च खण्डनं कृतम्, उपान्त्यादिपदधनानां व्येकगच्छादीनां च तुल्यत्वात्~। अतो व्येकगच्छतुल्या तेषां सङ्ख्या~। तेषु पुनरधोर्धगतान्यूर्ध्वार्धे गतैः संयोजयेत्~। तत्रैकाङ्गुलविस्तारं व्येकगच्छतुल्यविस्तारेण संयोजयेत्~। एवं द्व्यङ्गुलविस्तारादीनि च द्विहीनगच्छतुल्यविस्तारादिभिः क्रमेण संयोजयेत्~। एवमुभयत्रापि कुर्यात्~। तथा सत्यूर्ध्वार्धगतानि सर्वाण्यपि गच्छतुल्यभुजाकोटिकानि क्षेत्राणि भवन्ति~। ऊर्ध्वार्धगतानामूनस्य विस्तारस्य तत्तुल्यविस्तारैरधो(र्ध्व ? र्ध)गतैः पूरितत्वात्~। यदा पुनरोजसङ्ख्यानि तानि क्षेत्राणि त(तो ? दो)भयत्रापि मध्य एकं क्षेत्रं परिशिष्टं स्यात्~। तस्य विस्तारो गच्छार्धतुल्य एव~। तथाहि \textendash\ अत्र ह्येकाङ्गुलविस्तारादीनि गच्छतुल्यविस्तारान्तानि पूर्वस्मात्पूर्वस्मादेकैकाङ्गुलाधिकविस्ताराणि गच्छतुल्यसङ्ख्यानि क्षेत्राणि क्रमेण स्थितानि~। तेषु पूर्वार्धगतानामन्त्यं यत्क्षेत्रं तदेव गच्छतुल्यविस्तारेण हीनानां तेषां मध्यमं भवति, तदा तस्योपर्यधश्च क्षेत्राणां तुल्यसङ्ख्यत्वात्~। उपरि ताव-

\newpage

\noindent देकहीनेन गच्छार्धेन तुल्यसङ्ख्यानि क्षेत्राणि, गच्छतुल्यविस्तारेण हीनत्वात्~। अधोऽपि ताव(त्ये ? न्त्ये)व, पूर्वार्धान्त्यस्य मध्यमत्वेन परिकल्पितत्वात्~। अतः पूर्वार्धान्त्यमेव तदा मध्यमं भवति~। पूर्वार्धान्त्यस्य च गच्छार्धतुल्य एव विस्तारो युक्तः~। कथम्~। अत्र प्रथमं
क्षेत्रमेकाङ्गुलविस्तारं, द्वितीयं द्व्यङ्गुलविस्तारं, तृतीयं त्र्यङ्गुलविस्तारम्~। एवं प्रथमादारभ्य गणिते यावतिथं क्षेत्रं तावांस्तस्य विस्तारोऽपीति हि स्थितिः\renewcommand{\thefootnote}{१}\footnote{हि इति स्थितिः \textendash\ क. पाठः.}~। अतः प्रथमादारभ्य गण्यमानेषु तेषु यद्गच्छार्धसङ्ख्याविशिष्टं स्यात्, तद्गच्छार्धतुल्यविस्तारं भवति~। पूर्वार्धान्त्यं च तथा गण्यमाने
गच्छार्धसङ्ख्याविशिष्टं, पूर्वार्धान्त्यत्वादेव~। अतस्तस्य गच्छार्धतुल्यो विस्तारः, एवं विधस्य क्षेत्रस्य पुनरुभयत्रापि विद्यमानत्वात्~। तयोः संयोगे कृते तदपि गच्छतुल्यविस्तारं क्षेत्रं भवति~। एवमो(ज ? जे)युग्मे च द्वाभ्यां द्वाभ्यां सम्पादितानि व्येकगच्छतुल्यसङ्ख्यानि क्षेत्राणि भवन्ति~। उभयत्रापि व्येकगच्छतुल्यसङ्ख्यानि क्षेत्राणीति प्राक् प्रदर्शितम्~। द्वाभ्यां द्वाभ्यां चैकैकं क्षेत्रं सम्पादितम्~। अत उभयत्रापि व्येकगच्छार्धतुल्यसङ्ख्यानि क्षेत्राणि स्युः~। तद्योगश्च व्येकगच्छतुल्यः~। अतो व्येकगच्छतुल्या तेषां सङ्ख्या~। तानि पुनः कोणगतस्योपर्युपरि विन्यस्येत्~। तदा तद्गच्छतुल्यायामविस्तारोत्सेधं घनक्षेत्रं स्यात्, सर्वेषामेव गच्छतुल्यायामविस्तारत्वात्~। आयामविस्तारौ तावद्गच्छतुल्यौ~। एकाङ्गुलोत्सेधस्य कोणगतस्योपर्येकाङ्गुलोत्सेधानां व्येकगच्छतुल्यसङ्ख्यानां विन्यस्तत्वात्~। उत्सेधोऽपि गच्छतुल्य एव~। एवमिदं गच्छ(ध ? घ)नात्मकं क्षेत्रं जातम्~। पुनरन्त्यधनहीनचितितुल्यभुजाकोटिकेन चतुर्थेन क्षेत्रेण व्येकगच्छादीनां घनक्षेत्राणि सम्पादयेत्~। तत्र प्रथमं कोणादुभयतो व्येकगच्छतुल्ये प्रदेशे खण्डनं कार्यम्~। पुनर्द्विहीनगच्छतुल्ये प्रदेशे~। एवं पूर्वस्मात्पूर्वस्मात् एकैकाङ्गुलहीने प्रदेशे खण्डयेदिति विशेषः~। पूर्ववदेवं कृते तानि व्येकगच्छादीनां घनक्षेत्राणि भवन्ति~॥~२२~॥\\

गुणगुण्ययो रश्योः संवर्गे कर्तव्य उपायान्तरमाह\textendash 
\begin{quote}
{\ab सम्पर्कस्य हि वर्गाद्विशोधयेदेव वर्गसम्पर्कम्~।\\
यत् तस्य भवत्यर्धं विद्याद्गुणकारसंवर्गम्~॥~२३~॥}
\end{quote}

इति~। सम्पर्को योगः~। राश्योर्योगस्य वर्गा(न्तर ? त्त)योरेव वर्ग- 

\newpage

\noindent योगं विशोधयेत्~। तत्र शिष्टस्य यदर्धं तद्गुणगुण्ययोः परस्परापेक्षया गुणकारत्वसम्भवाद्गुणकारसंवर्गमित्युक्तम्~। इयमत्र वासना \textendash\ राश्योर्योगस्य वर्गः तयोरेव वर्गयोगस्य द्विगुणितस्य संवर्गस्य च योग एव~। तथाहि \textendash\ योगस्य वर्गे कर्तव्ये योगो योगेन गुणनीयः~। तत्र गुण्यगुणकारौ खण्डयित्वा गु(णेन ? णने) कृतेऽपि विशेषाभावात् यथैकं खण्डमधिकराशितुल्यं स्यादन्यच्च न्यूनराशितुल्यं तथा द्वावपि खण्डयेत्~। एवं खण्डिते न्यूनाधिकराश्यात्मकौ गुण्यौ प्रत्येकं द्वाभ्यामपि न्यूनाधिकराश्यात्मकाभ्यां गुणकाराभ्यां गुणनीयौ~। तत्र गुण्ये\renewcommand{\thefootnote}{१}\footnote{गुणनीयौ~। गुण्ये}ऽधिकराशौ न्यूनेन गुणकारराशिना गुणिते गु(णो ? णके) न्यूनराशावधिकेन गुणकारराशिना गुणिते च गुण्यगुणकारयोः संवर्गद्वयं स्यात्~। तयोर्योगो द्विगुण\renewcommand{\thefootnote}{१}\footnote{णः \textendash\ क. पाठः.}संवर्गो भवति~। पुनर्गुण्येऽधिकराशावधिकेन गुणकारराशिना न्यूनराशौ न्यूनराशिना च गुणिते तयोर्वर्गौ स्याताम्~। तयोर्योगो वर्गयोगः~। अतो वर्गयोगस्य द्विगुणितस्य संवर्गस्य च योगो योगवर्गो भवति~। अतएव योगवर्गाद्वर्गयोगे विशोधिते शि(ष्टे ? ष्टो) द्विगुणितः संवर्गः स्यात्~। अतस्तमिन्नर्धीकृते संवर्गो भवतीति युक्तम्~। अत्र वासना क्षेत्रेऽपि प्रदर्शयितुं शक्या~। तथाहि \textendash\ योगतुल्यभुजाकोटिकं क्षेत्रमेव हि योगवर्गः~। वर्गयोगस्य द्विगुणितसंवर्गस्य(च) योगोऽपि तदेव~। तद्यथा \textendash\ अधिकराशेर्वर्गोऽधिकराशितुल्यभुजाकोटिकं क्षेत्रम्~। न्यूनराशेर्वर्गो न्यूनराशितुल्यभुजाकोटिकं क्षेत्रम्~। राश्योः संवर्गो न्यूनराशितुल्यभुजकम् अधिकराशितुल्यकोटिकं क्षेत्रम्~। तस्मिन् पुनर्द्विगुणिते तथाविधं क्षेत्रद्वयं स्यात्~। एवमेतैश्चतुर्भिः क्षेत्रैर्योगतुल्यभुजाकोटिकं क्षेत्रं सम्पादयितुं शक्यम्~। तच्चैवं \textendash\ प्रथमं तेष्वधिकराशिवर्गात्मकं क्षेत्रं भूमौ विन्यस्य तस्याग्नेयकोणेन सम्बद्धवायव्यको(ण ? णं) न्यूनराशिवर्गात्मकं क्षेत्रं विन्यस्येत्~। ततो राशिवर्गात्मकयोः क्षेत्रयोरेकमादायाधिकराशिवर्गक्षेत्रस्य दक्षिणपार्श्वेन संश्लिष्टोत्तरपा(र्श्व ? र्श्वं) न्यूनराशिवर्गक्षेत्रपश्चिमपार्श्वेन संश्लिष्टपूर्वपार्श्वं पूर्वापरायतं निदध्यात्~। तदा तस्य (तस्य ?) पश्चिमपार्श्वमधिकक्षेत्रपश्चिमपार्श्वेन समं भवति, तत्तुल्यायामत्वात्~। अतोऽनयोः क्षेत्रयोः पश्चिमपार्श्वयोर्योग एव हि सम्पद्यमानस्य क्षेत्रस्य पश्चिमभुजा~। सा च न्यूनाधिकराशियोगतुल्या, अधिकराशिक्षेत्रपश्चिमपार्श्वस्याधिकराशितुल्य-

\newpage

\noindent त्वात् राशिसंवर्गक्षेत्रपश्चिमपार्श्वस्य न्यूनराशितुल्यत्वाच्च~। तस्य दक्षिणपार्श्वं पुनर्न्यूनराशिक्षेत्रदक्षिणपार्श्वेन (सं ? समं) भवति, तत्तु(ल्य)विस्तारत्वात्~। अतस्तयोर्दक्षिणपार्श्वयोर्योगः सम्पद्यमानस्य क्षेत्रस्य दक्षिणभुजा~। सापि\renewcommand{\thefootnote}{१}\footnote{सा परिधियो} राशियोगतुल्या, राशिसंवर्गक्षेत्रदक्षिणपार्श्वस्याधिकराशितुल्यत्वात् न्यूनराशिक्षेत्रदक्षिणपार्श्वस्य न्यूनराशितुल्यत्वाच्च~। पुनर्द्वितीयं राशिसंवर्गक्षेत्रं न्यूनराशिक्षेत्रोत्तरपार्श्वेन
संश्लिष्टदक्षिणपार्श्वमधिकराशिक्षेत्रपूर्वपार्श्वेन संश्लिष्टपश्चिमपार्श्वं दक्षिणोत्तरायतं विन्यस्येत्~। तदा तस्य पूर्वोत्तरे पार्श्वे क्रमेण न्यूनराशिक्षेत्रपूर्वपार्श्वेनाधिकराशिक्षेत्रोत्तरपार्श्वेन (च) समे भवतः~। न्यूनराशिक्षेत्रतुल्यविस्तारत्वात् अधिकराशिक्षेत्रतुल्यायामत्वाच्च तस्य~। अतः सम्पाद्यमानस्य क्षेत्रस्य पूर्वोत्तरभुजे अपि पूर्ववद्राशियोगतुल्ये जाते~। एवमेतैश्चतुर्भिः क्षेत्रैः सम्पादितं राशियोगवर्गात्मकं क्षेत्रमेतज्जातम्~।
तस्मात्पुनर्वर्गयोगे(न ?) विशोधिते न्यूनराशिक्षेत्रमधिकराशिक्षेत्रं चे विशोधितं स्यात्~। अतस्तत्र राशिसंवर्गात्मके क्षेत्रे परिशिष्टे~। पुनरर्धीकृते तयोरेकमपनीतं स्यात्~। अतः शिष्टं राशिसंवर्गात्मकं क्षेत्रं भवतीति युक्तम्~॥~२३~॥\\

अथ राश्योः संवर्गेऽन्तरे च\renewcommand{\thefootnote}{२}\footnote{च अज्ञात} ज्ञातेऽज्ञातयोस्तयोरानयनो\renewcommand{\thefootnote}{३}\footnote{ऽज्ञातयोरानयनो \textendash\ क. पाठः.}पायमाह\textendash 

\begin{quote}
{\ab द्विकृतिगुणात् संवर्गात् द्व्यन्तरवर्गेण संयुतान्मूलम्~।\\
अन्तरयुक्तं हीनं तद्गुणकारद्वयं दलितम्~॥~२४~॥}
\end{quote}

इति~। द्वयोः कृत्या चतुर्भिर्गुणिताद्राश्योः संवर्गात् द्वयो राश्योरन्तरस्य वर्गेण सं\renewcommand{\thefootnote}{४}\footnote{वर्गसं \textendash\ क. पाठः.}युताद्यन्मूलं तद्द्विष्ठम् अन्तेरण युक्तं हीनमर्धीकृतं च राशिद्वयं भवति~। युक्तस्यार्धमधिको राशिः~। हीनस्यार्धं न्यूनो राशिरित्यर्थः~। अथ वासना \textendash\ द्विकृतिगुणे संवर्गे द्व्यन्तरवर्गेण संयुते राशियोगस्य वर्गो भवति~। तथाहि \textendash\ वर्गयोगसहितो द्विगुणितः संवर्गो\renewcommand{\thefootnote}{५}\footnote{द्विगुणितसंवर्गो \textendash\ ख. पाठः.} योगवर्गो भवतीति पूर्वसूत्र एव प्रदर्शितम्~। वर्गयोगः पुनर्द्विगुणितस्य संवर्गस्य राश्यन्तरवर्गस्य च योग एव~। तद्यथा \textendash\ वर्गयोगे हि सम्पादनीये राश्योर्वर्गौ पृथक् सम्पादनीयौ~। अतोऽधिकराशिरधिकराशिना गुणनीयः~। न्यूनो राशिर्न्यूनराशिना च गुणनीयः~। तत्राधिकराशौ गुण्यमाने गुणकारभूतस्याधिकराशेः खण्डनं कृत्वैकं खण्डं न्यूनराशितुल्यमन्यद्राश्यन्तरतुल्यं च कुर्यात्~। अतोऽधिकरा(शिना ? शिः) न्यूनराशिना राश्यन्तरेण च गुणनीयो जातः~। तत्र

\newpage

\noindent न्यूनराशिना गुणिते राश्योः संवर्ग एव स्यात्~। अतो द्विगुणिते संवर्गे तृतीयोऽपि संवर्गः क्षेप्यो जातः~। पुनरन्तरेणाधिकराशिर्गुणनीयः~। तत्र गुण्यमधिकराशिमपि पूर्वदेव खण्डयेत्~। तथा सत्यन्तरेण न्यूनराशेरन्तरस्य च गुणनं कर्तव्यम्~। तत्रान्तरेणान्तरे गुणितेऽन्तवर्ग एव स्यात~। अतोऽन्तरवर्गोऽप्यत्र क्षेप्यो जातः~। पुनरन्तरेण न्यूनराशिर्गुणनीयः~। न्यूनराशिवर्गसम्पादनार्थं न्यूनराशिना न्यूनराशिश्च गुणनीयः~। अत उभयत्रापि न्यूनराशिरेव गुण्यः~। गुणकारः पुनरेकत्रान्तरम्, अन्यत्र न्यूनराशिः~। अत उभयत्रापि न्यूनराशेरेव गुण्यत्वाद्गुणकारयोर्योगेन न्यूनराशी गुणयेत्~। तद्योगश्चाधिकराशितुल्य एव~। अन्तरे(ण स)हितस्य न्यूनराशेस्तत्तुल्यत्वात्~। अतोऽधिकराशिना न्यूनराशिर्गुणनीयो जातः~। एवं चतुर्थोऽपि संवर्गः क्षेप्यो जातः~। अतः संवर्गचतुष्टयेऽन्तरसहिते योगवर्गो भवतीति युक्तम्~। तस्मिन् पुनर्मूलीकृते योगो भवति~। स पुनरन्तरेण युक्तो द्विगुणितोऽधिकराशिः स्यात्~। अतस्तस्मिन् दलिते(ऽधिकराशिर्भवति~। अन्तरेण हीनो द्विगुणितो न्यूनराशिः स्यात्~। अतस्तस्मिन् दलिते) न्यूनराशिर्भवति~। अत्र वासना क्षेत्रेऽपि प्रदर्शयितुं शक्या~। तद्यथा \textendash\ द्विकृतिगुणिते संवर्गे न्यूनराशिभुजका(न् ? नि) अधिकराशिकोटिकानि चत्वारि क्षेत्राणि भवन्ति~। अन्तरवर्गश्चान्तरतुल्यभुजाकोटिकं क्षेत्रम्~। एवमेतैः पञ्चभिः क्षेत्रैः {\qt गच्छोऽष्टोत्तरगुणितादि}त्यादिसूत्रवासनायां प्रदर्शितेन न्यायेन चतुरश्रं क्षेत्रं सम्पादयेत्~। अतस्तद्राशियोगतुल्यभुजाकोटिक क्षेत्रं स्यात्~। तस्य मूलं राशियोगः~। शेषं पूर्ववत्~॥~२४~॥\\

अथ मूलफलानयनोपायमाह\textendash

\begin{quote}
{\ab मूलफलं सफलं कालमूलगुणमर्धमूलकृतियुक्तम्~।\\
मूलं मूलार्धोनं कालहृतं स्यात् स्वमूलफलम्~॥~२५~॥}
\end{quote}

इति~। उत्तमर्णेनाधमर्णाय यत् पणादिकं द्र(ष्ट ?)व्यं दीयते तदिह मूल्यमित्युच्यते~। तस्य सम्वत्सरे सम्वत्सरे मासि मासि या या वृद्धिरधमर्णेनोत्तमर्णाय देया सा मूलफलम्~। एकस्मिन् सम्वत्सरे मासे वा लब्धं मूलफलं कस्मैचित्तयैव वृद्ध्या दत्त्वा केषुचित्सम्वत्सरेषु मासेषु वा व्य\renewcommand{\thefootnote}{१}\footnote{प्य \textendash\ ख. पाठः.}ती-

\newpage

\noindent तेषु तस्य यत्फलं लभ्यते तत्सहितं प्रथमफलं सफलमित्युक्तम्~। व्यतीतानां सम्वत्सराणां मासानां वा सङ्ख्या कालः~। अ(र्धं ? र्धमूलं) मूलार्धम्~। एतदुक्तं भवति \textendash\ स्वफलेन सहितं मूलफलं कालेन च मूलेन च गुणितं मूलार्धस्य कृत्या युक्तं मूलीकुर्यात्~। तत्र लब्धं मूलं\renewcommand{\thefootnote}{१}\footnote{मूला} मूलार्धेनोनं कालेन हृतं स्वमूलफलं स्यात्~। द्वितीयफलेन रहितं मूलफलं स्वयमेव स्यादित्यर्थः~। अत्र मूले, द्वितीयफलसहिते मूलफले काले च ज्ञातेऽज्ञातस्य केवलस्य मूलफलस्यानयनं क्रियत इति द्रष्टव्यम्~। अत्रेयं वासना \textendash\ सफलं\renewcommand{\thefootnote}{२}\footnote{फलं} मूलफलं कालेन मूलेन च निहत्य तस्मिन् मूलार्धवर्गे क्षिप्ते कालगुणितस्य मूलफलस्य मूलार्धस्य च यो योगस्तस्य वर्गो भवति~। तथाहि \textendash\ कालगुणितमूलफलात्मकस्य राशे\renewcommand{\thefootnote}{३}\footnote{राशेश्च \textendash\ क. पाठः.}र्मूलार्धात्मकस्य राशेश्च योगस्य वर्गे सम्पाद्ये राशियोगो राशियोगेन गुणनीयः~। तत्र गुणकारराशिं गुण्यराशिं च खण्डयित्वोभयत्राप्येकं खण्डं कालगुणितमूलफलमपरं\renewcommand{\thefootnote}{४}\footnote{फलतुल्यमपरं} मूलार्धं च कुर्यात्~। तथा सति कालगुणितमूलफलं कालगुणितमूलफलेन मूलार्धेन च गुणनीयम्~। मूलार्धं च ताभ्यामेव गुणनीयं जातम्~। एवं गुणितानां तेषां (योगः) कालगुणित\renewcommand{\thefootnote}{५}\footnote{तेषां गुणित}मूलफलस्य मूलार्धस्य च योगस्य वर्गो भवेत्~। एतदेवात्रापि क्रियते~। कथम्~। चतुर्णां (सं)वर्गाणां योगो राशियोगवर्ग इति प्राक् प्रदर्शितम्~। तत्र कालगुणितमूलफलवर्ग एकः संवर्गः~। कालगुणितमूलफलमूलार्धसंवर्गौ द्वौ~। मूलार्धसंवर्गोऽन्यः~। एवमेतेषु चतुर्षु संवर्गेष्वाद्यानां त्रयाणां संवर्गाणां योगः सफले मूलफले कालेन मूलेन च गुणिते स्यात्~। तत्र मूलफले कालेन मूलेन च गुणिते कालगुणितमूलफलमूलार्धसंवर्गद्वयं स्यात्~। कथम्~। कालेन गुणिते (मूलफले) काल\renewcommand{\thefootnote}{६}\footnote{गुणिते तत् काल \textendash\ ख. पाठः.}गुणितमूलफलं स्यात्~। तत्पुनर्यदि मूलार्धेन द्वाभ्यां च गुण्येत तदा कालगुणितमूलफलमूलार्धसंवर्गद्वययोगो भवेदिति स्पष्टम्~। अतः कालगुणितमूलफले द्वाभ्यां गुणितेन मूलार्धेन गुणितेऽपि तत्संवर्ग\renewcommand{\thefootnote}{७}\footnote{संवर्गस्य \textendash\ क. पाठः.}द्वययोगः स्यात्।~। द्वाभ्यां गुणितं मूलार्धं मूलमेव~। तस्मात् कालगुणिते मूलफले मूलेन गुणिते कालगुणितमूलफलमूलार्धसंवर्गद्वययोगो भवत्येव~। मूलफलस्य फले(न ?) पुनः कालेन मूलेन च गुणिते कालगुणितमूलफलवर्गो भवति~। तथाहि \textendash\ कालगुणितमूलफलवर्गे हि सम्पादनीये कालगुणितमूलफलं काल-

\newpage

\noindent गुणितमूलफलेन गुणनीयम्~। अत्र पुनर्मूलस्य यावतांशेन तुल्यं मूलफलं तया सङ्ख्यया गुणितः कालेन हृतश्च गुणकारराशिर्गुणकारत्वेन परिकल्पितः~। तस्य मूलस्य\renewcommand{\thefootnote}{१}\footnote{कालं मूलस्य} यावतांशेन तुल्यं मूलफलं तया सङ्ख्यया कालगुणिते मूलफले गुणिते कालगुणितमूलं स्यात्~।  तच्चैवं \textendash\ तया सङ्ख्यया तावन्मूले हृते मूलफलं स्यात्, मूलतदंशत्वान्मूलफलस्य~। अतस्तया सङ्ख्यया मूलफले गुणिते मूलमपि स्यात्~। अतएव कालगुणिते मूलफले तया सङ्ख्यया गुणिते कालगुणितमूलं स्यादिति युक्तमेव~। तस्मिन् पुनः कालेन हृते मूलं भवेत्~। मूलं चात्र गुणकारत्वेनोक्तं\renewcommand{\thefootnote}{२}\footnote{क्ते} मूलगुणमिति~। तस्मात् पूर्वोक्तया सङ्ख्यया गुणितः कालेन हृतश्च गुणकारराशिरत्र गुणकारत्वेन परिकल्पितः~। एवं गुणकारस्य तया सङ्ख्यया गुणितत्वात् कालेन हृतत्वाच्च गुण्यो राशिस्तया सङ्ख्यया हर्तव्यः कालेन गुणनीयश्च संवृत्तः~। तत्र कालगुणितमूलफलात्मके गुण्यराशौ पूर्वोक्तया सङ्ख्यया हृते मूलफलस्य फलं स्यात्~। तथाहि \textendash\ मूलफले तावत तया सङ्ख्यया हृते मूलफलस्यैककालसम्बन्धि फलं स्यात~। मूलस्य यावतांशेन तुल्यं मूलफलं मूलफलस्यापि तावतांशेन तुल्यत्वात् तत्फलस्य~। अतएवोभयत्रापि वृद्धेरैकरूप्यमुक्तम्~। अतः कालगुणिते मूलफले तया सङ्खयया हृते कालगुणितं तत्फलं स्यात्~। (काल)गुणितं चात्र मूलफलस्य फलं, तस्य बहुकालसम्बन्धित्वात्~। अतः पूर्वोक्तया सङ्ख्यया हृतं कालगुणितं मूलफलमेव मूलफलस्य फलम्~। तस्मिन् पुनः कालेन गुणिते गुण्यो राशिः स्यात्~। तस्मिन् पुनर्मूलेन गुणकारेण गुणिते कालगुणितमूलफलस्य वर्गो भवतीति युक्तमेव~। तदेवं मूलफले कालेन मूलेन च गुणिते कालगुणितमूलफलमूलार्धसंवर्गद्वययोगो भवति~। मूलफलस्य फले कालेन मूलेन च गुणिते कालगुणितमूलफलवर्गो भवति~। अतो मूलफलतत्फलयोगे कालेन मूलेन च गुणिते कालगुणितमूलफलमूलार्धसंवर्गद्वयस्य कालगुणितमूलफलवर्गस्य च यो योगो भवति~। तस्मिन् पुनर्मूलार्धवर्गे क्षिप्ते पूर्वोक्तानां चतुर्णां संवर्गाणां योगः स्यात्~। स च कालगुणितमूलफलमूलार्धयोगवर्ग\renewcommand{\thefootnote}{३}\footnote{वर्गयोग \textendash\ ख. पाठः.} एवेति प्राक् प्रदर्शितम्~। अतस्तस्य मूलं कालगुणितमूलफलमूलार्धयोगः~। तस्मात् मूला(ले ? र्धे)ऽपनीते\renewcommand{\thefootnote}{४}\footnote{मूलफलेऽपनीते \textendash\ क. पाठः.} शिष्टं कालगुणितमूलफलम्~। तस्मिन् पुनः कालेन हृते मूलफलं स्यात्~।

\newpage

\noindent इति युक्तमेवेदं गणितम्~। अत्र हि मूलफलतत्फलयोर्योगे मूले काले च ज्ञाते मूलफलानयनोपायः कथितः~। यदा पुनर्मूलमूलफल\renewcommand{\thefootnote}{१}\footnote{पुनर्मूलफल}योर्योगो मूलफलस्य (का ? फ)लं कालश्च ज्ञायते (य ? त)दापि मूलफलमानेतुं शक्यत एव~। तत्रेत्थं प्रक्रिया \textendash\ मूलमूलफलयोर्योगं कालेन मूलफलस्य\renewcommand{\thefootnote}{२}\footnote{मूलस्य} फलेन (च) निहत्य तस्मिन् मूलफल(फल)स्यार्धस्य वर्गं प्रक्षिप्य मूलीकुर्यात्~। तस्मात् मूल(फल)फलस्यार्धं विशोध्य शिष्टं कालेन हरेत् तन्मूलफलं स्यात्~। अत्र मूलमूलफलयोगं\renewcommand{\thefootnote}{३}\footnote{मूलफलयोगं} कालेन द्वितीयफलेन च निहत्य तस्मिन् द्वितीयफलार्धस्य वर्गे क्षिप्ते कालगुणितमूलफलद्वितीयफलार्धयोगवर्गो भवति~। तत्र मूल(मूल)फलयोगे कालेन द्वितीयफलेन च निहते कालगुणितमूलफलद्वितीयफलार्धसंवर्गद्वयस्य कालगुणितमूलफलवर्गस्य च योगः स्यात्~। तस्मिन् द्वितीयफलार्धस्य वर्गे क्षिप्ते चतुर्णां संवर्गाणां योगः स्यात्~। अत्रोपपत्तिः पूर्वोक्तन्यायेन द्रष्टव्या~। अतस्तन्मूलाद्द्वितीयफलार्धं विशोध्य शिष्टे कालेन हृते मूलफलं स्यात्~। अत्रायं श्लोकः\textendash

\begin{quote}
{\qt सफलपदं कालगुणं प्रफलघ्नं प्रफलवर्गपादयुतम्~।\\
मूलं प्रफलार्धोनं कालहृतं भवति मूलफलम्~॥}
\end{quote}

\noindent इति~॥२५~॥\\

अथ त्रैराशिकेनेच्छाफलानयनोपायमाह\textendash

\begin{quote}
{\ab त्रैराशिकफलराशिं तमथेच्छाराशिना हृतं कृत्वा~।\\
लब्धं प्रमाणभजितं तस्मादिच्छाफलमिदं स्यात्~॥~२६~॥}
\end{quote}

इति~। इच्छाफलप्रमाणैस्त्रिभी राशिभिः साध्यं गणितं त्रैराशिकम्~। तत्र यः फलराशिस्तमिच्छाराशिना हत्वा प्रमाणराशिना विभज्य तस्मात् यल्लब्धमिदमिच्छाफलं\renewcommand{\thefootnote}{४}\footnote{फलसम्बन्धिन} स्यात्~। एतत्सम्बन्धिन एतावन्त इति ज्ञाते एतत्सम्बन्धिनः कियन्त इत्यस्यां जिज्ञासायाम् अस्य गणितस्योपयोगः~। यथा \textendash\ धीजगन्नूपुराहर्गणसम्बन्धिनो रविभगणास्तत्समङ्ख्या इति ज्ञातेऽभीष्टाहर्गणसम्बन्धिनः कियन्त इत्यादिका जिज्ञासा~। तत्र धीजगन्नूपुरस्थानीयः प्रमाणराशिः, तत्समस्थानीयः फलराशिः, अभीष्टा\renewcommand{\thefootnote}{५}\footnote{स्थानीयः अभीष्टा \textendash\ क. पाठः.}हर्गणस्थानीय इच्छाराशिः~। इयमत्रोपपत्तिः \textendash\ धीजगन्नूपुरसङ्ख्येऽहर्गणे तत्समसङ्ख्यो रविभगण इति

\newpage

\noindent प्रागेवावगतम्~। अतोऽभीष्टाहर्गणेऽपि धीजगन्नूपुरतुल्यस्यांशस्य तत्समसङ्ख्यो रविभगण इत्यनुमातुं शक्यते, समानजातीयत्वादुभयोः~। एवं सर्वत्रापीच्छाराशौ\renewcommand{\thefootnote}{१}\footnote{राशेः} प्रमाणराशितुल्ये फलराशितुल्यमिच्छाफलम्~। तत्र यदि द्विगुणितेन प्रमाणराशिना तुल्य(मि ? इ)च्छाराशिः तदा द्विगुणितेन फलराशिना तुल्यमिच्छाफलमित्यप्यनुमेयम्~। अनेनैव न्यायेन यया सङ्ख्यया गुणितेन प्रमाणराशिना तुल्य इच्छाराशिस्तया सङ्ख्यया गुणितेन फलराशिना तुल्यमिच्छाफलमिति स्थितम्~। सा च सङ्ख्या प्रमाणराशिनेच्छाराशौ विभक्ते स्यात्~। तया पुनः फलराशौ गुणिते इच्छाफलं स्यात्~। अतः सर्वत्रापि प्रमाणेनेच्छां विभज्य तेन फले गुणिते इच्छाफलं स्यात्~। एतदेवात्रापि क्रियते~। तत्र यत् प्रथमं प्रमाणेनेच्छाया हरणं कर्तव्यं तदत्र पश्चात्क्रियते, तथा कृतेऽपि फलवैषम्याभावात्, इत्येतावानेव विशेषः~। अत एव केवलेनैवेच्छाराशिना गुणनं पश्चात् प्रमाणराशिना हरणं चोक्तम्~। यत्र पुनः प्रमाणराशेर्न्यून इच्छाराशिः तत्रेच्छाराशिना प्रमाणराशिं विभज्य तेन फलराशौ विभक्ते इच्छाफलं स्यात्~। तथाहि \textendash\ अत्रापि प्रमाणतुल्यायामिच्छायां फल\renewcommand{\thefootnote}{२}\footnote{मिच्छाफल}तुल्यमिच्छाफलम्~। यदा पुनः प्रमाणार्धतुल्येच्छा तदा फलार्धतुल्यमिच्छाफलम्~। प्रमाणत्र्यंशतुल्यायामिच्छायां फलत्र्यंशतुल्यमिच्छाफलम्~। एवं सर्वत्रापि प्रमाणस्य यावतांशेन तुल्येच्छा फलस्यापि तावतांशेन तुल्यमिच्छाफलम्~। सा च सङ्ख्येच्छया\renewcommand{\thefootnote}{३}\footnote{सङ्ख्येच्छायाः} प्रमाणे विभक्ते स्यात्~। तया पुनः फले विभक्ते\renewcommand{\thefootnote}{४}\footnote{भक्ते \textendash\ क. पाठः.} इच्छाफलं स्यात्~। अतः सर्वत्रापीच्छया प्रमाणं विभज्य तेन फले विभक्ते इच्छाफलं स्यादिति युक्तम्~। तत्र यदिच्छया प्रमाणस्य हरणं कर्तव्यं तदकृत्वा केवलैनैव प्रमाणेन फले विभक्ते तत्फलमिच्छया पुनर्गुणनीयं भवति~। कथम्~। हारकस्य महत्त्वे फलस्य न्यूनत्वं भवति~। इच्छया हरणाभावेनेच्छया गुणितत्वमत्र हारकस्य महत्त्वम्~। अतस्तेन हृतं फलमिच्छया हृतेन फलेन तुल्यमेव भवति~। अतस्तदिच्छया गुणनीयं जातम्~। अतः प्रमाणेन फलं विभज्येच्छया गुणिते इच्छाफलं स्यात्~। तत्र प्रमाणेन हरणं पश्चादपि कर्तुं शक्यम्~। अत एवेच्छया फलं निहत्य प्रमाणेन हरणमत्रोक्तम्~। एवमुभयथापीच्छया फले हते प्रमाणेन विभक्ते इच्छाफलं भवतीति युक्तम्~। अत्र प्रमाणादधिकायामिच्छायां प्रथमोक्तः प्रकारः,

\newpage

\noindent न्यूनायामपर इति (नि)यमो नास्ति~। उभयत्राप्युभयथा वासना द्रष्टव्या~। अथवा प्रमाणेन पुनः फलं विभज्य तेनेच्छायां गुणितायां फलेन प्रमाणं विभज्य तेनेच्छायां विभक्तायां चेच्छाफलं स्यात्~। अत्रापि वासना पूर्वोक्तन्यायेन द्रष्टव्या~। सर्वथापीच्छाफलयोर्घातः प्रमाणेन हर्तव्यो भवति~। यत्र पुनरिच्छाया वृद्धौ फलस्य ह्रास इच्छाया ह्रासे वा फलस्य वृद्धिस्तत्र व्यस्तत्रैराशिकं कर्तव्यम्~। तथा चोक्तम\textendash 

\begin{quote}
{\qt इच्छावृद्धौ फलह्रास इच्छाह्रासेऽधिकं फलम्~।\\		
यत्र तत्र हि कर्तव्यं व्यस्तत्रैराशिकं बुधैः~॥}
\end{quote}

\noindent इति~। तत्प्रकारश्चोक्तः\textendash 

\begin{quote}
{\qt प्रमाणेन फलं हत्वा विभजेदिच्छया बुधः~।\\
व्यस्तत्रैराशिकं (ह्येतत्) ज्ञेयं सर्वत्र धीमता~॥}
\end{quote}

\noindent इति~। अत्रेयं वासना \textendash\ अत्रापि प्रमाणतुल्यायामिच्छायां फलतुल्यमेवेच्छाफलम्~। यदि पुनः प्रमाणार्धतुल्येच्छा तदा द्विगुणितेन फलेन तुल्यमिच्छाफलम् इच्छाह्रासानुसारेण तत्फलस्य वृद्धेः~। अनेनैव न्यायेन प्रमाणस्य यावतांशेन तुल्येच्छा तद्गुणितेन फलेन तुल्यमिच्छाफलमिति गम्यते~। सा च सङ्ख्येच्छया प्रमाणे विभक्ते स्यात्~। तस्मादिच्छया प्रमाणं विभज्य तेन फले गुणिते इच्छाफलं स्यात्~। अत्राप्येतदेव क्रियते~। इच्छया हरणं पश्चात्क्रियत इत्येव विशेषः~। यदा पुनः प्रमाणादधिकेच्छा\renewcommand{\thefootnote}{१}\footnote{प्रमाणाधिकेच्छा} तदा प्रमाणेनेच्छां विभज्य तत्फलेन फलं विभजेत्~। तदेच्छाफलं स्यात~। तथाहि \textendash\ अ\renewcommand{\thefootnote}{२}\footnote{त \textendash\ ख. पाठः.}त्रापि प्रमाणतुल्यायामिच्छायां फलतुल्यमिच्छाफलम्~। यदा पुनर्द्विगुणितेन प्रमाणेन तुल्येच्छा तदा फलार्धतुल्यमिच्छाफलम्, इच्छावृद्धौ फलस्य ह्रासात्~। एवं यद्गुणितेन प्रमाणेन तुल्येच्छा फलस्य तावतां(ये ? शे)न तुल्यमेवेच्छाफलं भवति~। सा च सङ्ख्या प्रमाणेनेच्छायां विभक्तायां स्यात्~। अतः प्रमाणेनेच्छां विभज्य तत्फलेन फले विभक्ते इच्छाफलं स्यादिति युक्तम्~। अत्राप्येतदेव क्रियते~। तत्र यत् प्रमाणेनेच्छाया हरणं कर्तव्यं, तदकृत्वेच्छया फले विभक्ते पुनः प्रमाणेन गुणनं कार्यम्~। तत्र प्रथमं प्रमाणेन गुणनं क्रियते इच्छया हरणं पश्चात्क्रियत इति विशेषः~। अत्रापि प्रथमद्वितीययोः प्रकारयोर्नियमो न~। उभयत्राप्युभयथा वासना योजयितुं

\newpage

\noindent शक्येति द्रष्टव्यम्~। इतीदं प्रथमे वयस्येव वर्तमानेन मया द्वितीयवयसि स्थितेन कौषीतकिनाढ्येन कारितम्~। अत्र केषाञ्चिद्युक्तयः पुनरस्मदनुजेन शङ्कराख्येन तत्समीपेऽध्यापयता वर्तमानेन तस्मै प्रतिपादिताः~। तस्याढ्यत्वात्स्वातन्त्र्याच्च तत्र व्यापारश्च निर्वृत्तः~। तस्मिन्
स्वर्गते\renewcommand{\thefootnote}{१}\footnote{तत्र स्वर्गते \textendash\ क. पाठः.} पुनरत एव मयाद्य प्रवयसा ज्ञाता युक्तीः प्रतिपादयितुं भास्करादिभिरन्यथाव्याख्यातानां कर्माण्यपि प्रतिपादयितुं यथाकथञ्चिदेव व्याख्यानमारब्धम्~॥~२६~॥\\

एवं त्रैराशिकं वर्गकर्म च तत्तद्विषयनिष्ठतया प्रतिपाद्य तत्परम्परयावगम्यं विषयविशेषं च प्रदर्श्य तयोः सामान्य(स्या ? न्या)यं च प्रकाश्येच्छादीनां
सांशत्वे यो विशेषस्तं प्रदर्शयितुमुत्तरसूत्रमारभते\textendash

\begin{quote}
{\ab छेदाः परस्परहता भवन्ति गुणकारभागहाराणाम्~।\\
छेदगुणं सच्छेदं परस्परं तत्सवर्णत्वम्~॥~२७~॥}
\end{quote}

इति~। गुणकारभागहाराणां त्रयाणां छेदाः परस्परहतास्तत्रैव विलीयन्ते~। गुणकारशब्देनैव फलराशिरिच्छाराशिश्च गृह्येते {\qt गुणकारद्वयमि}त्यादिवत्~। अ(त्र ? त) एव बहुत्वं च~। गुणकारयोर्भागहारस्य चेत्यर्थः~। परस्परहताः, गुणकारच्छेदौ भागहारहतौ भागहारच्छेदश्च गुणकारहतः~। एवं कृते ते छेदास्त्याज्या एव, तैः पुनः प्रयोजनाभावात्~। पुनस्तैर्गुणकारभागहारैरेव गुणनं हरणं वा कार्यम्~। नन्वेवं त्रिभ्योऽतिरिक्ता राशयः स्युः, गुणकाराभ्यां चाभ्यां हतौ हारकच्छेदौ हारकेण हतौ गुणकारच्छेदौ च~। मन्द ! मैवम्~। हननशब्देन संवर्गस्यैव विवक्षितत्वात्, गुणकारच्छेदयोर्हारकस्य चेति त्रयाणामाहतिर्हारकः गुणकारयोर्हारकच्छेदस्य च संवर्गो हार्य इति~। एतदुक्तं भवति \textendash\ गुणगुण्ययोर्घातो हारकच्छेदेन हत एव (का ? हा)र्यः~। हारकश्च गुणगुण्यच्छेदघातहतो हारक इति हारकस्य गुणकारच्छेदयोश्चैकारिमित्रतयैकत्रैवैदस्पर्यात् हार्यस्य हारकच्छेदस्य च एककोटित्वान्मिथः सम्बन्ध इति भावः~। यत्र पुनर्बहूनां द्वयोर्वा सवर्णना कार्या तद्वस्तु मिथश्छेदगुणं कार्यम्~। किन्तु न स्वयमेव छेदगुणं कार्यम्~। अपितु स्वच्छेदश्च छेदगुणः~। एवमन्येऽपि राशयः सच्छेदाः परस्परहताः कार्याः~। तदेतत्सवर्णत्वम्~। एवं कृते सति सर्वेषां सवर्णत्वं स्यात्~। अत्र सच्छेदमिति छेदस्यांशे शेषत्वं
द्योत्यते~।

\newpage

\noindent अंशराशीनां शेषि(त?)त्वं च~। अत एव सिद्धं पूर्वत्रापि गुणकारहारकाणाम् अंशा एव स्वशब्देनोक्ता इति~। तत्रापि छेदस्याप्राधान्यात्~। हारकांशराशिर्गुण्ययोरन्यतरच्छेदाहतः सन्नितरच्छेदेन च हतो हारकः~। गुण्यांशराशिर्वा गुणकारांशराशिर्वा हारकच्छेदेन हतो गुण्यो गुणकारश्च स्यातामित्येतावानेवांशेषु तत्र छेदैर्विशेष आधेयः~। किं पुनरत्र वर्णशब्देनोच्यते~। न तावत् गोत्वादिजातिः, सङ्ख्याविशेषाणां तत्सम्बन्धाभावात्~। नाप्यक्षराणि च~। जातिरेवात्र वर्णशब्देनोच्यते~। सा च परिमाणेषु वर्तमाना~। परिमाणेषु कः पुन\renewcommand{\thefootnote}{१}\footnote{परिमाणेषु पुन}र्बहुष्वनुवृत्त इतरेभ्यो व्यावृत्तश्च गुणः, येन तत्सम्बन्धानाम् अपरसामान्यत्वं प्रतीयेत~। उच्येते~। अंशेषु तावदंशत्वं सर्वत्र साधारणम्~। तथापि परिमेयाल्पत्व\renewcommand{\thefootnote}{२}\footnote{परिमेयत्वाल्प}महत्त्वानुरूपं तत्रापि महत्त्वमल्पत्वं चारोप्यते~। तद्वशाज्जायमाना भेदा अनन्ता एव~। तद्यथा \textendash\ परिमेयानां व्यक्तीनां ये द्व्यंशा अर्धशब्दवाच्याः ते सर्वेऽप्येकजातिकाः~। ये पुनस्त्र्यंशास्ते तदपेक्षयान्यजातिकाः~। त्र्यंशत्वेनैभ्यो व्यावृत्ताः चतुरंशादिभ्यश्च~। त्र्यंशत्वसामान्यं च स्वेषु सर्वेष्वनुवर्तते~। एवं क्रमेण पञ्चांशादयोऽप्येकोत्तरच्छेदाः यथापेक्षं कल्प्याः~। सङ्ख्येयानां सङ्ख्यानां चानन्त्यात् तेऽप्यनन्ता एव~। त एव केरलेषु लुप्तरेफेण पर्णशब्देन प्रयुज्यन्ते~। तत्र यदि कश्चित् पृच्छति पणत्र्यंशद्वितयं मयास्मै देयम् अनेन च मह्यं पणपञ्चांशत्रितयं देयं, तत आवयोः केन कस्मै कियदवशिष्टं देयमिति~। तत्र तौ राशी सवर्णयित्वैव तदुत्तरं\renewcommand{\thefootnote}{३}\footnote{सवर्णयित्वैतदुत्तरं \textendash\ क. पाठः.} देयमिति~। त्र्यंशद्वये पञ्चभिर्गुणिते दशांशा स्युः~। पञ्चांशत्रितये त्रिभिर्गुणिते नवांशाश्च~। त्र्यंश\renewcommand{\thefootnote}{४}\footnote{नवांशाश्च तत् त्र्यंश \textendash\ क. पाठः.}द्वयमधिकं दशसङ्ख्यत्वात्~। तेन दशभ्यो नवके विशोधिते येन त्र्यंशद्वितयं देयं तेन देयोंशोऽवशिष्यते~। स रूपस्य कतिथोंऽशः, कथं वाप्युभयेषामंशानां सावर्ण्यम्~। तत्र छेदयोः पश्चत्रिकयोर्घात उभयोश्छेदः~। तत उभये पश्चदशांशा इति तेषां जातिज्ञाना(र्थः ? र्थं) सच्छेदमित्युक्तम्~।
कथं पुनरुभयोः परस्परं\renewcommand{\thefootnote}{५}\footnote{परस्पर \textendash\ ख. पाठः.} छेदगुणितयोः सवर्णत्वम्~। तत्र यौ त्र्यंशौ तयोः प्रत्येकं पञ्चधा विभक्तयोः पञ्चांशः स्युः~। तत्तुल्य एवावशिष्टस्तृतीयो भागः येन तस्य राशे रूपान्न्यूनत्वम्~। तस्मिन्नपि पञ्चांशाः स्युः~। ततस्तादृशा अंशा एकस्मिन् रूपे पञ्चदश स्युः~। अतस्ते रूपस्य पञ्चदशांशा इति निर्णीयते~।

\newpage

\noindent ये च पुनरन्यस्मिन् राशौ त्रयः पञ्चांशास्तेषु च प्रत्येकं त्रिधा विभक्तेषु प्रत्येकं त्रयस्त्रयोंऽशाः स्युः~। एवं शिष्टयोरपि~। एवं पञ्चांशेषु
पञ्चस्वपि प्रभागाः सम्भूय पञ्चदश स्युः~। एवं तेऽपि रूपस्य पञ्चदशांशाः~। उभयोः छेदेऽपि पूर्वच्छेदद्वयघातः पञ्चदशांशः~। तेन हि तज्जातिरवगम्यते एते रूपस्य\renewcommand{\thefootnote}{१}\footnote{गम्यते रूपस्य \textendash\ क. पाठः.} पञ्चदशांशा इति~। अत एवांशानां प्राधान्यं छेदस्याप्राधान्यं च युज्यत एव~। अत उक्तं छेदगुण सच्छेदं परस्परमिति~॥~२७~॥\\

ग्रहगणिते पुनर्मध्यमादीनां विपरीतकर्माप्यभिधीयते गर्गादिभिः~। अतस्तल्लाघवाय विपरीतकर्मणीतरस्माद्भेदं प्रदर्शयति तेनोक्तिलाघवं स्यादिति तत्र तत्र
विपरीतकर्मणोऽनेन न्यायेन सिद्धत्वात् तन्न पृथग् वक्तव्यमिति\textendash

\begin{quote}
{\qt गुणकारा भागहरा भागहरा ये भवन्ति गुणकाराः~।\\
यः क्षेपः सोऽपचयोऽपचयः क्षेपश्च विपरीते~॥~२८~॥}
\end{quote}

इति~। विपरीतकर्माणां प्रयोजनमपि मीमांसायां सिद्धम्~। ज्योतिःशास्त्रे युगपरिवृत्तिपरिमाणद्वारेण चन्द्रादित्यादिगतिविभागेन तिथिनक्षत्रज्ञानमविच्छिन्नसम्प्रदायगाणितानुमानमूलमिति प्रमाणाध्याये गणितस्कन्धप्रामाण्यप्रतिपादकस्य वार्त्तिकस्य व्याख्यायामजितायाम्
अविच्छिन्नसम्प्रदायपदं विवृण्वतैतदुक्तम्~। गणितोन्नीतस्य चन्द्रादेर्देशविशेषान्वयस्य प्रत्यक्षेणैव संवादः~। ततो निश्चितान्वयस्य परस्मै गणितलिङ्गोपदेशः,
ततस्तस्याप्तोपदेशावगतान्वयस्यानुमानं संवादः परस्मै चोपदेश इति सं(पा ? प्र)दायाविच्छेदात् प्रामाण्यम्~। तन्निर्णये विपरीतकर्मापेक्षान्यत्र तेनैव दर्शिता तस्मिन्नेवाधिकरणे~। मूलं चैतयोरेतन्मूलगणितफलेन मानान्तरसंवादिना परावृत्य गणितरूपम्~। विजयाख्ये तदभिप्रायश्चैवं दर्शितः~। अयमभिप्रायः \textendash\ प्रथममुपदेशत एतयोर्देशकालपरिमाणयोर्ज्ञानं, पुनस्तन्मूलं गणनं, ततस्तत्फलस्य ग्रहणादेः प्रत्यक्षादिना संवादः, ततस्तेन प्रत्यक्षादिना संवादिना फलेन पुनरुपदेशावगतदेशकालपरिमाणयोः प्रातिलोम्येन गणनां कृत्वा तत्त्वनिर्णयः~। तदुक्तं परावृत्ये(ती ?)ति~। तस्मात् ग्रहणादीनामपि विपरीतकर्म ज्ञेयमिति तत्रापि प्रसिद्धम्~। आदिशब्देन ग्रहाणां मिथो योगो ग्रहनक्षत्रयोगश्चोदयास्तमयादिकं च विवक्षितम्~। तथा च जातकेऽप्युक्तम्\textendash

\newpage


\begin{quote}
{\qt योगे ग्रहाणां ग्रहणेऽर्कसोमयोः\\
मौढ्ये तथा वक्रगतौ च पञ्चसु~।\\
दृष्टानुरूपं करणं यदन्वहं\\
तेन ग्रहेन्द्रान् गणयेत् त्रिवारकम्~॥}
\end{quote}

\noindent इति~। जातकरणेऽपि तथैवोक्तं\textendash 

\begin{quote}
{\qt ग्रहणग्रहयोगादौ बहुशो यत् परीक्षितम्~।\\
करणं तेन सङ्गण्य ज्ञेयाः सूर्यादयो नृभिः~॥}
\end{quote}

\noindent इति~। पराशरहोरायां सामान्येनाप्युक्तं\textendash

\begin{quote}
{\qt यदा यश्चैव सिद्धान्तो गणिते दृक्समो भवेत्}
\end{quote}

\noindent इति~। तस्मात् प्रत्यक्षोपलब्धग्रहयोगादिना विपरीतगणितेन ग्रहाणां स्फुटनिर्णयः कर्तव्य इत्यस्य ग्रहगणिते महानुपयोगः~॥~२८~॥\\

यदोद्देशकेन बहूनां राशीनां समुदाये प्रष्टव्ये तेष्वेकैकं विना तत्तदितरसमुदायं पृथक् पृथगुद्दिश्य एतेषां समुदायः कियान् पृथग्भूता वा राशयः कियन्त इति पृच्छति (तदा) तदानयनायाह\textendash

\begin{quote}
{\ab राश्यूनं राश्यूनं गच्छधनं पिण्डितं पृथक्त्वेन~।\\
व्येकेन पदेन हृतं सर्वधनं तद्भवत्येव~॥~२९~॥}
\end{quote}
\begin{sloppypar} 
इति~। सर्वधनस्य व्येकपद(ह ? हृ) तस्यात्रोद्दिष्टत्वात्\renewcommand{\thefootnote}{१}\footnote{तस्यात्रोद्दिष्टात् \textendash\ ख. पाठः.} उद्देशकालापवैपरीत्यमेवास्यापीति विपरीतकर्मानन्तरमस्य सङ्गतिः~। यावत्कृत्वः समुदाय उद्दि(ष्टाः ? ष्टः) तत्सङ्ख्येह पदशब्देनोच्यते, तावन्तः पृथग्भूता राशयस्तावत्सु पदेषु स्थिता इति~। तेषु प्रतिप्रश्नमेकैकस्य राशेः परित्यागात्
पर्यायेण सर्वे राशयः सकृत्सकृत् परित्यक्ताः स्युः~। तेन तदुद्दिष्टसमुदाययोगात् सकलराशिसमुदाययोगात् तावत्कृत्वः कृतादेकगुणितेन सकलसमुदायेन न्यून(म्) एव स्यात्~। तस्मात् व्येकपदगुणित एव सकलसमुदायः, तदुद्दिष्टसमुदायस्य तुल्यत्वात्~। तदुद्दिष्टसमुदाययोगे (व्येकेन पदेन हृते) सकलसमुदायः स्यात्~। तस्मात् पृथक्स्थितादेकैकोनराशिसमुदाये पृथक् पृथक् त्यक्ते शिष्टं तत्तद्राशिसङ्ख्या च स्यादिति तत्समुदाययोगस्य व्येकेन पदेन हरणमेव युक्तं नतु सकलेन गच्छेनेति सिद्धम्~। एवकारेण न्यूनातिरेकव्यावर्तकेन युक्तिरेव सूचिता~॥~२९~॥\\
\end{sloppypar} 
\newpage

यस्य निष्कपणादिषु द्वौ राशी धनम् इतरस्य च तावेव द्वौ राशी~। यस्य महतां रूपाणाम् आधिक्यं तस्याल्पानां पणादीनां राशेर्न्यूनसङ्ख्यत्वम्~। इतरस्य महतां समुदायस्याल्पत्वम् अल्पसमुदायसङ्ख्याधिका~। उभयोरपि तुल्ये एव धने~। तयोर्द्वयोरपि भिन्नजाती यौ राशी उद्दिश्य तत्र महत्स्वैकैका व्यक्तिरल्पात् कियतोऽर्हतीति पृष्टे महदर्घज्ञानायाह\textendash

\begin{quote}
{\ab गुलिकान्तरेण विभजेत् द्वयोः पुरुषयोस्तु रूपकविशेषम्~।\\
लब्धं गुलिकामूल्यं यद्यर्थकृतं भवति तुल्यम्~॥~३०~॥}
\end{quote}

इति~। उभयोरपि तत्तद्राश्योस्तुल्यांशस्य धनं सममेव~। यत् पुनर्द्वयोर्विवरं तयोर्विवरयोरपि द्वयोर्धनसाम्येन भाव्यम्~। अन्यथा विषमधनत्वापत्तेः~। तत्र तयोरुभयोर्विवरयोरल्पस्यैव बहुसङ्ख्यत्वं युक्तम्~। महतामल्पसङ्ख्यत्वमेव युक्तम्~। तस्मान्महत्सङ्ख्ययाल्पयाल्पानां सङ्ख्या महती हर्तव्या~। तत्र यल्लब्धं तावतोऽल्पानर्हति महत्स्वेकैकमित्येतदुपपत्तिः सुगमैव~॥~३०~॥\\

ग्रहयोरेकमार्गेणैव गच्छतोर्भिन्नदिग्गातकयोर्वा सर्वदैव जवस्तुल्यो लिप्ताभिर्मिथो भिन्नश्च~। तयोरन्तरालगतं प्रदेशं ज्ञात्वा गतिपरिमाणं च निर्णीय
तयोर्योगः कदाभूत् भविष्यति वेत्येतज्ज्ञानोपायमाह\textendash

\begin{quote}
{\ab भक्ते विलोमविवरे गतियोगेनानुलोमविवरौ द्वौ~।\\
गत्यन्तरेण भक्तौ द्वियोगकालावतीतैष्यौ~॥~३१~॥}
\end{quote}

इति~। तत्र यद्येकदिक्कयोरल्पगतिः पुरस्सरः तदा\renewcommand{\thefootnote}{१}\footnote{कदा} भविष्यत्येव\renewcommand{\thefootnote}{२}\footnote{भवष्यतीत्येव} योगः~। यदा पुनः शीघ्रगतिः पुरस्तरः तदा गत एव~। परस्परं विलोमगतिकयोस्त्वाभिमुख्ये भविष्यत्येव योगः~। परस्परं पृष्ठगतत्वेऽतीत एव~। तदन्तरं भिन्नदिक्कयोर्गतियोगेन हार्यम्~। समानदिक्कगतिकयोस्तु गत्यन्तरेण च~। तत्र (लब्धम्?) इष्टकालस्य योगकालस्य चान्तरालवर्ती दिनादिकः कालो लभ्यते, दिनकालभवयोर्गत्योर्विवक्षितत्वादित्येतदापे सुगममेव~। तत्र ग्रहाणां पुनश्चिरं गतिसाम्यं न स्यादित्यासन्नयोगयोरेवान्तरेणानीतं दिनादिकं वास्तवं, स्यात्~। अत एवाह मानसे\textendash

\begin{quote}
{\qt ग्रहयोरन्तरे स्वल्पेऽनल्पभुक्तेः पुरस्सरः~।\\
यदाल्पगतिरेष्यः\renewcommand{\thefootnote}{३}\footnote{गतिरेषः} स्यात् तदा योगो\renewcommand{\thefootnote}{४}\footnote{योगे}ऽन्यथा गतः\renewcommand{\thefootnote}{१}\footnote{गतिः \textendash\ क. पाठः.}~॥}
\end{quote}

\newpage

\begin{quote}
{\qt युक्त्या भिन्नदिशोर्गत्योरन्तरेणैकदिक्कयोः~।\\
ग्रहान्तराद्दिनानि स्युस्तैः समावनुपाततः~॥}
\end{quote}

\noindent इति~। कस्मिन् देशे तयोः समागमोऽभूत् वा भविष्यति वेत्येतच्चानुपाततो ज्ञेयम्~। कथम्~। तं कालं दिनादिकं स्वस्वदिनगत्या हत्वा तावदन्तरे प्रदेशे योगः~। कुतः~। यस्मिन्नेतन्निरूपणकाले\renewcommand{\thefootnote}{१}\footnote{कालो \textendash\ क. पाठः.} वर्तते~। एवमुभयोरपि तत्तदाधारभूतदेशात् तस्य तस्य स्वस्थानात् तावति देशे योगः~। यद्वा तदन्तरालदेशपरिमाणं स्वस्वभुक्त्या हत्वा गत्यन्तरेण गतियोगेन वा हरेत्~। तावद्योजनान्तरे कलाद्यन्तरिते वा देशे योग इति साम्यकालप्रदेशान्वयश्चोभयोर्ज्ञातव्यः~। तत्र पूर्वोक्तः पक्षः {\qt तैः समावनुपातत} इति प्रदर्शितः~। तैर्दिनैरनुपाततः समौ कार्यौ~। तयोर्मार्गस्य यं कञ्चित् प्रदेशमवधित्वेनाङ्गीकृत्य ततःप्रभृति कियति दूरे तयोर्योग इति तदन्तरालगतयोजनासाम्यादेव तयोः समत्वं विवक्षितम्~। तत् पुनर्ग्रहयोरेव कार्यं, तत्र देशादेरवधित्वेन प्रसिद्धत्वात्~। न तथा नदीसमुद्रभूतलादिषु यः कश्चित् प्रदेशोऽवधित्वेन प्रसिद्ध इति सामान्यन्यायपरत्वादत्र तदनुक्तिः~॥~३१~॥\\

एवं लोकशास्त्रयोः प्रसिद्धं गणितजातं सकलं प्रदर्श्य ग्रहगत्यनुमानोपयोगि कुट्टाकाराख्यं गणितविशेषं करिकाद्वयेनाह\textendash

\begin{quote}
{\ab अधिकाग्रभागहारं छिन्द्यादूनाग्रभागहारेण~।\\
शेषपरस्परभक्तं मतिशुणमग्रान्तरे क्षिप्तम्~॥~३२~॥
			
अधउपरिगुणितमन्त्ययुगूनाग्रच्छेदभाजिते शेषम्~।\\
अधिकाग्रच्छेदगुणं द्विच्छेदाग्रमधिकाग्रयुतम्~॥३३~॥}
\end{quote}

इति~। तदेतल्लौकिकोदाहरणद्वारास्माभिः प्रदर्श्यते~। यत् पुनः {\qt क्षितिरवियोगाद्दिनकृदि}त्यादिसूत्रं शास्त्रान्तर्गतं तद्व्याख्याने पुनरस्य ग्रहगणितातिदेशो विस्तरेण करिष्यत इत्यत्रैतद्युक्तिमात्रमेव प्रदर्श्यते~। कस्मिंश्चिद्राशावनेन हृतेऽ(न ? ने)न चापहृतेऽयं चायं च शेषो योऽपहृतः स क इति केनचित् पृष्टे तदानयनोपायप्रदर्शनपरमिदं सूत्रम्~।

\begin{quote}
{\qt द्वौ वंशौ तुल्यमानौ यौ तौ प्रमायावशेषितौ~।\\
ईशहस्तेन मानेन ततो वस्वधिकेन च~।}
\end{quote}

\newpage

\begin{quote}
{\qt पञ्चसङ्ख्यस्त्रिसङ्ख्यश्च कियद्धस्तौ च तौ वद~।\\
पञ्चकं ह्यधिकाग्रं स्यादिहोनाग्रं त्रिकं तथा~॥}
\end{quote}

\noindent इति~। तत्रैकादशसङ्ख्योऽधिकाग्रभागहारः~। एकोनविंशतिसङ्ख्यश्चोनाग्रभागहारः~। ऊनमग्रं शेषो यस्य स ऊनाग्रो भागहारः~। येन हृतशिष्टमधिकं सोऽधिकाग्रभागहारः~। तत्र यद्यधिकाग्रभागहार ऊनाग्रभागहारादधिकः तदा तमधिकाग्रं भागहारमूनाग्रेण भागहारेण हृत्वा शिष्टमेव परस्परहरणे ऊनाग्रभागहारप्रतियोगि~। तयोः पुनः परस्परहरणं कार्यम्~। तेनाल्पीकृतेनाधिकाग्रभागहारेण ततोऽधिकमूनाग्रभागहारं प्रथमं हृत्वा तत्फलं च क्वचिद्विन्यस्य ऊनाग्रभागहारशेषेणान्यमपि हरेत्~। तस्याल्पीकृतस्य पुनरिदानीमूनाग्रशेषादधिकत्वाद्धरणयोग्यता~। तत्फलमपि तदधो विन्यस्येत्\renewcommand{\thefootnote}{१}\footnote{विन्यसेत्}~। एवं सकृत्~। परस्परहरणम्~। एवं मुहुर्वा ह्रियताम्~। एवं द्विर्द्विर्हृत्वाधिकाग्रभागहारशेषेऽल्पे सत्येव मतिकल्पना कार्या~। तत्र यदा मनसि मतिः प्रतिभाति तदा मतिः कल्प्या~। यदि कस्यचित् कदाचित् परस्परहरणात् प्रागप्यधिकाग्रभागहारस्याल्पत्वे मतिः स्फुरेत्, तर्हि तत्रैव मतिः कल्या~। किं पुनः कल्पितया तया क्रियते~। अधिकाग्रभागहारशिष्टस्य गुणनम्~। यत् पुनर्मतिगुणमधिकाग्रभागहारशिष्टं तदग्रान्तरे अग्रयोरुभयोरन्तरे
क्षिप्त्वाप्यूनाग्रभागहारेण हृते यथा शेषस्य शून्यता स्यात् तथा हरणं यया सङ्ख्ययाधिकाग्रभागहारशेषे गुणिते भवति सैव सङ्ख्या मतिशब्देनोच्यते~। तां मतिं फलद्वन्द्वादधो निधाय मत्या हतादधिकाग्रभागहारशेषादग्रान्तरयुतादूनाग्रभागहारशेषेण हृत्वाप्तं फलमपि मतेरधो विन्यस्य न्यस्तेषु फलद्वन्द्वेषु मतेरूर्ध्वस्थं फलं मत्या हत्वा तदधोगतमन्त्य\renewcommand{\thefootnote}{२}\footnote{न्त्यं \textendash\ ख. पाठः.}फलं च तस्मिन् क्षिपेत्~। एवं पुनः पुनरप्यधउपरि गुणनम्~। अधस्थेनोपरिस्थस्य गुणनं कृत्वा तदपि तदा यदन्त्यं तेन युतं कार्यम् एवं मुहुरधउपरिगुणनम्~। अन्त्ययोगश्च कार्यः~। एवं वल्ल्युपसंहारेण यदा पुनर्द्वावेव राशी भवतः तदान्त्याभावादधउपरिगुणिते गुणकारादन्यस्या(न्त ? न्त्य)स्य क्षेप्यस्याशक्यत्वात् तत्र परिसमाप्तमेवैतत् कर्म~। तदनन्तरमपि किं कर्तव्यमित्याह \textendash\ तयोरुपरिस्थराशावूनाग्रच्छेदभाजिते शेषम\renewcommand{\thefootnote}{३}\footnote{छेदम \textendash\ क. पाठः.}धिकाग्रच्छेदगुणं कृत्वा पुनरप्यधिकमेवाग्रं तत्रैव योजयित्वा यल्लभ्यते तद्द्विच्छेदाग्रम्~। द्वौ छेदावग्रे च यस्य तद्द्विच्छेदाग्रम्~।

\newpage

\noindent तथाभूतो राशिरेव सः~। यो राशिः पूर्वं द्वाभ्यां हा\renewcommand{\thefootnote}{१}\footnote{हो \textendash\ क. पाठः.}राभ्यां हृत उद्देशकेनोद्दिष्टः कस्मिश्चिद्राशावनेन हृतेऽ(न ? नेन) चापहृतेऽयं चायं च शेषो योऽपहृतः स क इति स एवैष इत्यर्थः~। अस्मिन्नुदाहरणेऽधिकाग्रभागहारस्यैवाल्पत्वात् तेनैकादशसङ्ख्येनैकोनविंशतिसङ्ख्यमूनाग्रभागहारमेव प्रथमं हरेत्~। तत्र लब्धं फलमेकं क्वचिद्विन्यस्य शिष्टेनाष्टकेन एकादशसङ्ख्यं हृत्वाप्तं चैकं तदधो विन्यस्य शिष्टेन त्रिकेणाष्टकं हृत्वा लब्धं फलं द्विसङ्ख्यं तदधो विन्यस्य शिष्टेन द्विकेन त्रिकं हृत्वा लब्धं फलमेकमेव स्यात्~। तत्र लब्धं फलमप्येकं तदधो विन्यसेत्~। तत्राधोगतः शेष एकसङ्ख्यः~। उपरिगतो द्विसङ्ख्यः~। तत्राधोगतस्याधिकाग्रभागहारशेषत्वात् तस्यैव मतिकल्पना कार्या~। तस्मिन् द्विसङ्ख्यया मत्या हत्वा अग्रयोः पञ्चकत्रिकयोरन्तरे द्विके क्षिप्ते सति द्विकेनोपरिगतेन शेषेण हृते निश्शेषता स्यात्~। तत्फलं च द्विसङ्ख्यम्~। तत्र द्विसङ्ख्यया मत्या तदुपरिस्थमेकसङ्ख्यं फलं हत्वा तस्मिन्नन्त्ययुते चतुस्सङ्ख्यं स्यात्~। तेन च तदुपरिस्थे द्विके हते द्विसङ्ख्येनान्त्येन च युते तत्र दशकं स्यात्~। तेन तदूर्ध्वगत एकसङ्ख्ये हते चतुष्केणान्त्येन च युते तन्मनुसङ्ख्यम्~। तेन तदुपरिस्थमेकं हत्वा अन्त्येन दशकेन युते चतुर्विंशतिसङ्ख्यो राशिरुपरिस्थः~। तस्मिन्नूनाग्रभागहारेणैकोनविंशत्या हृते शेषं पञ्चसङ्ख्यम्~। तदधिकाग्रच्छेदेनैकादशसङ्ख्येन गुणितं पञ्चपञ्चाशत्सङ्ख्यम्~। अधिकमग्रं पञ्चसङ्ख्यं योजयित्वा लब्धो द्विच्छेदाग्रराशिः षष्टिसङ्ख्यः~। तस्मिन्नेवैकादशसङ्ख्येन एकोनविंशत्या च हृते पञ्चकं त्रिकं शिष्टमित्येतत् साग्रकुट्टाकाराख्यं कर्म~। निरग्रकुट्टाकारे पुनरुपरिस्थे चतुर्विंशतिसङ्ख्ये एकोनविंशत्या हृते यच्छिष्टं पञ्चकं स एव पृष्टो गुणकारः~। अधस्थ\renewcommand{\thefootnote}{२}\footnote{अथ स्व \textendash\ ख. पाठः.}राशेरेकादशहृतशिष्टं तत्फलं चेत्यधिकाग्रच्छेदगुणमधिकाग्रयुतमित्युक्तं कर्म तत्र न कार्यम्~। अतस्तदपेक्षया सशेषोऽयं कुट्टाकार इति (वि)शेषः~। तत्र गुणकारविषय एव प्रश्नः~। तत्फलमप्यनेनैव कर्मणा सिध्यतीत्यधःस्थो राशिरितरेण राशिना ह्रियते~। तत्र शिष्टं तत्फलम्~। तत् पुनरेतद्गुणकारे ज्ञाते तेन भाज्यं हत्वा शेषं क्षिप्त्वा विशोध्य वा हारेण हृत्वा विज्ञेयमिति गुणकारविषय एव कुट्टाकारश्च~। तत्र नान्तरीयकतया फलं च सिद्ध्येदिति तच्च स्वीक्रियते~। तत्र कथम्भूतः प्रश्नः~। येन हतेऽस्मि-

\newpage

\noindent न्ननेन संयुक्ते वियुक्ते वानेनापहृते निश्शेषो भवति भागः स कस्तत्फलं वा किमिति केनचित् पृष्टे तदानयनोपायः~। {\qt सङ्ख्यान्तराश्रयत्वाभावादन्यनिरपेक्षत्वाद्वा निरग्र} इति हि भाष्यकारवचनम्~। अस्मिन्नुदाहरणे एवं निरग्रविषयः प्रश्नः \textendash\ एकादशसङ्ख्ये (भ ? भा)ज्ये येन गुणकारेण हते अग्रान्तरं द्विकं क्षिप्त्वैकोनविंशतिसङ्ख्येन हारकेण हृत्वा शेष एव न स्यात्\renewcommand{\thefootnote}{२}\footnote{एव स्यात्}~। अतएव फलं च निरवयवं लभ्यते~। स गुणकारः कियान् तत्र लब्धं फलं च कियदिति निरग्रविषयः प्रश्नः~। उभयत्रापि राश्योरावृत्तिस्तदन्तरेऽग्रान्तरतुल्ये ज्ञातव्या~। उद्देशकेनोद्दिष्टयोस्तयोर्गुण्यः पुनर्भाज्यशब्देनोच्यते~। (ता ? त)मुद्दिश्य गुणकारः पृच्छ्यते~। तत्र न केवलं तावेवोद्देश्यौ~। हरणे यो वाञ्छितः शेषः स चोद्देश्यः, इतरथा व्यवस्थाभावात्~। गुणकारस्य न प्रष्टव्यत्वम्~। तत्र शेषो द्विविधः~। गन्तव्यशेषो गतशेषश्च~। तत्र यदि गन्तव्यशेष इष्टस्तर्हि गुणकारगुणिते भाज्ये तं संयोज्योद्दिष्टेन हारकेण ह्रियमाणे निश्शेषतया भाव्यम्~। इतरथा घाताच्छेषं विशोध्य ह्रियमाणे
निःशेषत्वमिष्यते~। तत्र शेष\renewcommand{\thefootnote}{२}\footnote{तत्र तत्र शेष \textendash\ क. पाठः.} एव तयोर्घातयोर्विशेषः~। ननु कथमत्र घातयोरित्युक्तम्~। अत्र ह्येक एव गुणः ततस्तेन गुणिते भाज्य एक एवात्र
घातः, (न) ततोऽन्यो द्वितीयः कश्चिदुपलभ्यते~। सत्यम्~। एक एव स घातः~। तत्र शेषे क्षिप्ते शुद्धे वा यत् स्यात् सोऽन्यो घातः~। स कयो
राश्योर्घातः~। हारकस्य तत्फलस्य च~। यथात्रैकादशके भाज्ये गुणकारेण पञ्चकेन हते द्विकेन क्षेप्यशेषेण च युक्ते सप्तपञ्चाशत्सङ्ख्यो राशिः स्यात्~। स एव ह्येकोनविंशत्या ह्रियमाणो निश्शेषः स्यात्~। तत्र फलं च त्रिसङ्ख्यम्~। तस्य फलस्य हारकस्य च घात एव सप्तपञ्चाशत्सङ्ख्यो राशिः~। इतरो घातः पञ्चपञ्चाशत्सङ्ख्यः~। तयोरन्तरं च द्विकम्~। तत एतदेव प्रश्नवाक्यस्य तात्पर्यम् \textendash\ एकादशसङ्ख्ये राशौ कियत्कृत्वः कृते एकोनविंशतिसङ्खये च कियकृत्वः कृते एकोनविंशतेराहतिर्द्वयेनाधिका स्यात्~। एवं क्षेप्यशेषे हारकस्य (शोध्यशेषे) फलस्य चाधिक्यम्~। क्षेप्यशेषे पुनर्गुणकारभाज्ययोर्घातस्याधिक्यमित्येव विशेषः~। तत्र यावकृत्वः कृतस्य भाज्यस्य भागहारस्य चान्तरं द्विकं स्यात् आवृत्तस्य हारकस्य द्विकेनाधिक्यं च तौ कावित्येवमेव प्रष्टव्यम्~। शोध्यशेषेऽपि यावत्कृत्वः कृताद्धारकात् यावकृत्वः कृतो भाज्य

\newpage

\noindent एतावता न्यूनः तौ कियन्ताविति तयोरुभयोरावृत्तिसङ्ख्यैव गुणकारत्वेन फलत्वेन च पृच्छ्यते~। तदन्तरेणैवोद्दिष्टेन तन्नियमश्च सेत्स्यति~। साग्रेऽपि तावेव राशी उपरिस्थ ऊनाग्रभागहारेण हृते अधस्थे चाधिकाग्रभागहारेण हृते शे(षा\renewcommand{\thefootnote}{१}\footnote{ष} ? षौ) स्या(न्तां ? ताम्)~। तत्रोपरिशेषो गुणकारः अधस्थो राशिः फलम् इ(तर ? ति त)योरावृत्तिसङ्ख्ये एव साग्रकुट्टाकारे शिष्टे स्याताम् इत्येतावत्पर्यन्तं न कश्चिद्विशेषः~। तस्यैवान्याकारत्वभ्रमायैव तत्रैवेषद्विशेषमा\renewcommand{\thefootnote}{१}\footnote{तत्रैवैतद्विशेषमा \textendash\ क. पाठः.}(धवा ? धा)य पृच्छ्यते~। अतो हारभाज्ययोर्विपर्यस्तयोरपि न क्रियाभेदः~। एकस्मिन् पक्षे शेषः क्षेप्यश्चेदन्यस्मिन् पक्षे शोध्य एव~। अन्यत् सर्वं तुल्यमेव~। यत्र पुनर्भाज्यहारयोरपवर्तनं कर्तुं शक्यं तत्रापवर्तिताभ्यामपि कुट्टाकारः कर्तुं शक्यः, तत्र मतिकल्पनादौ लाघवं च स्यादित्यपवर्तनयुक्तिश्चात्र प्रदर्श्या~। उस्तं च भास्करेण कुट्टाकाराङ्गतया भाज्यहारयोरपवर्तनेन प्रथमं द्रढीकरणं\textendash

\begin{quote}
{\qt भूदिनेष्टगणान्योन्यभक्तशेषेण भाजितौ~।\\  
हारभाज्यौ दृढौ स्यातां कुट्टाकारं तयोर्विदुः~॥}
\end{quote}

\noindent इति~। तयोः परस्परभक्तशेषेण हृतयोर्दृढत्वं\renewcommand{\thefootnote}{३}\footnote{शेषदृढत्वं \textendash\ ख. पाठः.} च पुनरेकेनैव राशिना हरणे उभयोरपि निश्शेषत्वाभावात्~। कथं पुनस्तदभावः~। तत्रातुल्याभ्यामुभयोर्हरणे तयोर्निश्शेषत्वं क्वचित् सम्भवति~। तयोर्येन हारेणैकस्य निश्शेषत्वं तेनैव ह्रियमाणेऽन्यस्य न निःशेषत्वम्~। येन\renewcommand{\thefootnote}{४}\footnote{तेन \textendash\ क. पाठः.} हृते पुनस्तस्य निश्शेषत्वं तेनान्यस्यापि निश्शेषत्वं न स्यात्~। एकेन राशिनोभयोर्बहूनां वा निश्शेषहरणमपवर्तनशब्देनोच्यते~। एवं निश्शेषं हृतयोर्दार्ढ्यं स्यादेव~। तत्र तावद्भाज्यस्य रविभगणस्य निश्शेषहरणसमर्था बहवो राशयो विद्यन्ते~। भूदिनस्यापि बहवः स्युः~। उभयोरपि निश्शेषहरणसमर्था अपि कतिचित् सन्ति~। तेषु यो महान् स एवात्र परस्परहरणे शेषत्वेनार्विष्क्रियते~। तत्र, 

\begin{quote}
{\qt शतमष्टोत्तरं भानोश्चतुर्भिरयुतैर्हतम्~।}
\end{quote}

\noindent इति हि रविभगणा उक्ताः~। तस्मिन्नुक्तमात्रेऽपि चतुर्भिरयुतैर्हरणे निश्शेषता स्यात् रविभगणस्येत्येतत् प्रतीयत एव~। पुनरपि निरूप्यमाणे द्वाभ्यां हतैश्चतुर्भिरयुतैर्हृते निःशेषता स्यादिति पुनस्त्रिभिरयुतैः पुनश्चतुर्भिः षड्भिरष्टाभिर्नवभिस्तत्फलभूतैश्च हरणे निःशेषता स्यात्।~। तत्र यानि भूदिनानि

\newpage

\begin{quote}
{\qt व्योमशून्यशराद्रीन्दुरन्ध्राद्यद्रिशरेन्दवः~।}
\end{quote}

\begin{sloppypar} 
\noindent इत्युक्तनि तानि न सहस्रदिभिर्निश्शेषं हर्तुं शक्यानि, यतस्तत्राद्यं स्थानद्वितयमेव शून्यं स्यादिति~। तानि पुन\renewcommand{\thefootnote}{१}\footnote{स्यादिति न पुन}रेकषष्ट्या नवाब्धिवेदाग्नि(भि)र्नवाष्टाग्निखरूपयमलैश्चैकषष्ट्या गुणितैस्तैश्च हर्तुं शक्यानि~। तेष्वन्यतमेनापि\renewcommand{\thefootnote}{२}\footnote{विभज्य \textendash\ क. पाठः.} भाज्यराशिर्न निश्शेषं हर्तुं शक्यः~। उभयोर्निश्शेषहरणसमर्थाः पुनर्बहवः सन्ति, द्वित्रिचतुष्पश्चषट्सङ्ख्या दशद्वादशपञ्चदशविंशतिपञ्चविंशतित्रिंशदाद्याश्च बहवः शतत्रयपञ्चकपञ्चदशकपञ्चविंशतिसङ्ख्याश्च~। तेभ्यः सर्वेभ्यो महान् पुनः खव्योमेषुशैलसङ्ख्यः~। तयोः परस्परहरणे एष एवान्तेऽवशिष्यते~। कथं परस्परहरणेऽपवर्तनं शिष्यते~। भाज्यभाजकयोरुभयोरप्युभयसाधारणेषु हारकेषु महान् राशिः शेते~। बह्वावृत्त्या यद्यपि तयोस्तस्यावृत्तिसङ्ख्या न समा तथापि तेनैवारब्धौ तौ~। अतः परस्परहरणे तदारब्धाभ्यां हारकाभ्यां तदावृत्तय एव काश्चनापास्यन्ते~। ततस्तत्र तत्र जायमानाः शेषा अपि तदावृत्तयः स्युः~। य\renewcommand{\thefootnote}{३}\footnote{त \textendash\ ख. पाठः.}तो राश्योरुभयोरपि तदावृत्तत्वं त(दा ? तः) शेषाणामपि तदावृत्तत्वमेव सम्भवति, यतस्तत्समुदायात् तैरेव कैश्चिद्राशिभिः स राशिर्ह्रियते~। हरणे हारकं येन केनचिद्गुणयित्वा कृत्स्नश\renewcommand{\thefootnote}{४}\footnote{कृत्स्नत \textendash\ क. पाठः.} एव हारका\renewcommand{\thefootnote}{५}\footnote{हारा}स्ततस्त्यज्य(ते ? न्ते), न पुनरर्धशः पादशो वा~। तस्माच्छेषेऽपि कृत्स्नश एव ते वर्तन्ते~। एवं पुनश्शेषेणापि तत्समुदायेन पूर्वं हारकभूतादपि तत्समुदायाद्धरणेन\renewcommand{\thefootnote}{६}\footnote{णे \textendash\ ख. पाठः.} त्यज्यन्ते इति तच्छेषेऽपि समानैवेयं युक्तिः~। एवमेकस्य शून्यत्वेऽपि अन्यत्र यः शिष्यते तत्र स एक एव शिष्यते न पुनर्द्व्यादिभिर्हतः, यतस्तत्र शिष्ट एव महान् विवक्षितः~। तेन तयोर्हरणे उभयोरपि निश्शेषता स्यादिति~। अस्य युक्तिर्व्युत्क्रमेण सिद्धान्तदीपिकायां प्रदर्शिता\textendash
\end{sloppypar}

\begin{quote}
{\qt राश्योरन्योन्यहरणे महत्यल्पेन संहृते~।\\
यः शेषः स्वल्पराशिश्च तौ हार्यौ सङ्ख्यया यया~॥

तयैव सङ्ख्यया हार्यो भवेद्राशिर्महानपि~।\\
त्यक्ता ये महतस्तेऽङ्काः स्वल्पराशिहता यतः~॥

एवं भूयोऽपि सञ्चिन्त्या हा(र्या ? र्यता)धिकहीनयोः~।\\
अन्त्यशेषेण शेषो हि हार्योऽन्यो हृत एव हि~॥}
\end{quote}

\newpage

\begin{quote}
{\qt अन्योन्यभक्तशेषेण तस्माद्भाजकभाज्ययोः~।\\
विभक्तौ तौ तु निश्शेषौ भवतो युक्तिरीदृशी~॥}
\end{quote}

\noindent इति~। तत्र यः शिष्टो राशिस्तेनासौ स्वयं निश्शेष हर्तुं शक्यः~। अन्यत्रस्थः स्वहार्यो\renewcommand{\thefootnote}{१}\footnote{हार्यो}ऽनेनैव निश्शेषतया हृत एव~। तस्मात् स राशिरेतत्समुदाय एव हरणात् प्राक् तत्र स्थितः~। स एव कियत्कृत्वश्चिदावृत्त\renewcommand{\thefootnote}{२}\footnote{कियत्कृत्वः कश्चिदावृत्त} एवान्यस्मादपि ततः पूर्वं त्यक्तः~। तत्र शेषश्च तस्यावृत्तिरेव~। तस्मात् तत्र चरमशेषस्थाने प्राक् स्थितोऽपि तत्समुदाय एव~। चरमशेषेण हृतश्च तत्समुदाय इति चोक्तम्~। तत्र ततः\renewcommand{\thefootnote}{३}\footnote{तत्र} प्राक् स्थितो राशिर्यः तत्सिद्ध्यर्थं तस्य हारकोऽन्यस्तत्फलगुणितः क्षेप्यः~। तस्मात् तत्राप्युभयोरंशयोस्तदारब्धत्वमेव स्यात्~। पूर्वं तद्धारकतया स्थितस्य तदावृत्तत्वं साधितमेव~। तत्र तस्मिन्नप्येतस्मिन् स्वफलहते क्षिप्ते तस्यापि तदारब्धत्वमेव युज्यते~। एवं तत्तद्द्वन्द्वात् पूर्वद्वन्द्वस्यापि तदारब्धत्वमेव युक्तमिति पूर्वन्यस्तयोरपि तदारब्धत्वनिर्णयात् तेन हृतयोस्तयोर्निश्शेषता स्यात्~। ततो महता\renewcommand{\thefootnote}{४}\footnote{स्यात्~। महता \textendash\ क. पाठः.} केनचिदपि न निःशेषत्वं द्रष्टुं शक्यं तयोः~। ततोऽल्पैर्हरणे निःशेषहरणं शक्यं स्यात् कर्तुम्~। तत्र तयोर्दृढतापि न स्यात्, पुनरन्येन हृत्वाप्यल्पीकार्यत्वात्~। महता हृतयोः पुनस्ततोऽल्पत्वापादनं\renewcommand{\thefootnote}{५}\footnote{पादनत्वं} न शक्यमिति तत्र स्थिरत्वं स्यादिति दृढत्वम्~। यथा(त्र वि\renewcommand{\thefootnote}{६}\footnote{यथा रवि} ? त्रापि) भाज्यभाजकयोर्द्व्यादिभिश्च हरणे निःशेषता स्यात्~। तथापि तयोर्न दृढत्वं तेनैव स्यात्, पुनः शताद्यैरपि हर्तुं शक्यत्वात्~। ततोऽप्यल्पीयस्त्वाय ततश्चलनसम्भवात् दृढत्वा\renewcommand{\thefootnote}{७}\footnote{दृढता \textendash\ ख. पाठः.}भावः~। एतयोः पुनस्ततो गन्तुमवकाशाभावाद्दृढतैवेत्यपवर्तनयुक्तिः\renewcommand{\thefootnote}{८}\footnote{दृढतेत्येवेत्यपवर्तनयुक्तिः}~। तत्र यथेष्टं शेषोऽप्युद्देष्टव्यः~। अपवर्तिताभ्यां कुट्टाकारे पुनरुद्देशकेन निरूप्यैव शेषोद्देशः कार्यः, अन्यथैवोद्दिशन् जाड्यमेव लभत इति~। अत एवोक्तं\textendash

\begin{quote}
{\qt परस्परं भाजितयोर्ययोर्यच्छेषं तयोः\renewcommand{\thefootnote}{*}\footnote{'भाजितयोर्यः शेषस्तयोः' इति} स्यादपवर्तनं तत्\renewcommand{\thefootnote}{$\dagger$}\footnote{'सः' इति च मुद्रितलीलावतीपाठः.}~।\\
तेनापवर्तेन विभाजितौ यौ तौ भाज्यहारौ दृढसंज्ञितौ स्तः~॥
		
भाज्यो हारः क्षेपकश्चापवर्त्यः केनाप्यादौ सम्भवे कुट्टकार्थम्~।\\
येनच्छिन्नौ भाज्यहारौ न तेन क्षेपश्चैतद् दुष्टमुद्दिष्टमेव~॥}
\end{quote}

\newpage

\noindent इत्युक्तत्वात् तदुद्देशस्य दुष्टत्वात्~। तस्मादेकादयोऽपवर्तनगुणिता एव तत्रानपवर्तिते शेषतयोद्देश्याः~। अनपवर्तितयोः परस्परहरणे
तयोस्तत्तत्स्थानगतशेषाणां च सर्वदैवापवर्तनेन हरणे निःशेषता प्रतिपादिता~। ततस्तत्रा\renewcommand{\thefootnote}{१}\footnote{तत्र तत्रा}पवर्तनात् प्राग्गताः सङ्ख्याविशेषा न शेषतयोद्देष्टुं शक्याः,
किन्त्वपवर्तनतुल्या एव~। ततोऽप्यूर्ध्वमपि द्विघ्नापवर्तनतुल्य एव शेष उद्देश्यः~। न तयोरन्तरालेषु कश्चन सङ्ख्याविशेषः~। एवमपवर्तनसमुदाया एव तत्र शिष्टाः स्युः~। अपवर्तितयोः\renewcommand{\thefootnote}{२}\footnote{एव~। अपवर्तितयोः} परस्परहरणे त्वपवर्तितभागहारादधोगताः सङ्ख्याविशेषाः सर्व एव पर्यायेण शेषाः स्युरिति यथेष्टमुद्देश्याः~। हारकादधिक एव नोद्देश्यः~। भगणभूदिनयोर्भाज्यहारकत्वे भगणशेषा एव हि शेषाः~। प्रष्टव्यश्चेच्छाभूतः फलस्य भगणस्य गुणकारभूतोऽहर्गण एव~। यातभगणाश्चेच्छाफलभूताः~। तत्रापवर्तितो भागहारो नवाष्टाग्निखरूपयमलसङ्ख्यो रवेः~। भाज्यश्च षट्सप्तपञ्चसङ्ख्यः~। तावत्सु वर्षेषु हारकतुल्या दिवसा निरवयवाः स्युरिति स कालो युगशब्देनोच्यते~। तत्र रवेः कुट्टाकारे षट्सप्तपञ्चमितवर्षमेकं युगम्~। दिवसा नवाष्टाग्निखरूपयमलसङ्ख्याः~। अत उक्तम्\textendash

\begin{quote}
{\qt अहरात्मकमत्र स्यात् धीजगन्नूपुरं युगम्~।}
\end{quote}

\noindent इति~। तत्रैकस्मिन् युगे युगादितः प्रभृति प्रतिदिनं ये ये शेषा आयुगान्तं परस्परं भिन्नास्त एव युगान्तरेष्वपि क्रमेणैव शेषाः स्युः~। इति वर्तमानयुगगताहर्गण एव तत्र कुट्टाकारेण ज्ञेयः~। न पुनरती(ति ? त)युगसङ्ख्या ज्ञेया\renewcommand{\thefootnote}{३}\footnote{ज्ञेयाः \textendash\ क. पाठः.}~। तेन ग्रहसामान्ययुगाहर्गणो न ज्ञेयः~। अतः पृच्छकं प्रत्यहर्गणे प्रदर्शिते स यदि न संवदेत तर्हि पुनः पुनरपि भागहारं क्षिपत्वा क्षिपत्वा पृच्छेत् एष वा किं त्वया पृच्छ्यत इति~। एवं यावत्संवादं भागहारो मुहुर्मुहुः क्षेप्यः~। यावत्कृत्वः क्षिप्ते तस्य संवादो जायते भगणे भाज्योऽपि तावत्कृत्वः क्षेप्यः~। तस्माद्भागहारेऽसकृत् क्षिप्ते भाज्योऽप्यसकृत्\renewcommand{\thefootnote}{४}\footnote{भाज्योप्यधः सकृत् \textendash\ ख. पाठः.} क्षेप्यः~। एवं मुहुर्मुहुरुभयं क्षेप्यं यावत्संवादम्~। तदुक्तं\textendash

\begin{quote}
{\qt प्रक्षिप्य भागहारं कुट्टाकारे पुनः पुनः प्राज्ञः~।\\
योज्यं च भागलब्धं भाज्ये प्रस्तारयुक्त्यैव~॥}
\end{quote}

\newpage

\noindent इति~। तत्रैकस्मिन् युगेऽपि प्रतिदिनं भिन्नाः शेषाः केन क्रमेणोत्पद्यन्ते~। न पुनरेकद्व्यादिक्रमेण~। यद्येकद्व्यादिक्रमेणोत्पद्येरन् तर्हि शेषतुल्य एवाहर्गणोऽपि स्यादिति शेषमुद्दिश्याहर्गणप्रश्नो न युज्यते~। एवं हारकादधोगतैः सङ्ख्याविशेषैः सर्वैरेव शेषतया भाव्यञ्च~। हारकादतिरेकायोगात् प्रतिदिनं भिन्नत्वाच्च युगदिनतुल्याः शेषाः स्युरिति चेत्~। उच्यते~। तत्तद्युगे प्रथमदिने भाज्यतुल्य एव शेषः~। यस्मादेकेन दिनेन फलरूपो भगणो हन्यते तस्मात् प्रथमदिने भाज्य एव शेषः~। तस्य युगाहैर्हर्तुमशक्यत्वाच्छेषत्वम्~। पुनरपि द्व्यादिगुणिता दृढभगणा एवाद्येऽब्दे क्रमेण शेषाः स्युः~। तस्माद्भाज्यादिभाज्यचयाः पञ्चषष्ट्युत्तरशतत्रयान्तं शेषाः~। पुनर्द्वितीयादिषु वर्षेषु मुख\renewcommand{\thefootnote}{१}\footnote{मुख्य}स्यैव भेदः~। सर्वेष्वपि वर्षेषु भाज्य एव चयः~। तत्र कुट्टाकारे\renewcommand{\thefootnote}{२}\footnote{र} प्रथमहरणे भाज्यमुपरि न्यस्य तदधःस्थे भागहारे सकृद्धृते यः शेषः स नवमनुसङ्ख्यः~। तेन तस्मिन्नहर्गणेऽपि नैको भगणः पूर्णो लभ्यते~। कथं शेषे सत्यपूर्णता~। न पुनर्भगणगुणितस्याहर्गणस्य भागहारहरणे यः शेष स एव भगणशेषः~। अयं पुनर्गन्तव्यशेष एव\renewcommand{\thefootnote}{३}\footnote{एवं} यतः पञ्चषड\renewcommand{\thefootnote}{४}\footnote{द्घ \textendash\ क. पाठः.}(नि ? ग्नि)तुल्येन गुणकारेण गुणितो भगणो भागहारादिह प्रथमहरणे त्यज्यते~। य(तः ? त्र) प्रथमं केवलो भागहार एव भाज्येन ह्रियते, तत्र शिष्टस्य कृत्स्नाद्भागहारात् द्यु\renewcommand{\thefootnote}{५}\footnote{हारद्यु \textendash\ ख. पाठः.}गणगुणितस्य भाज्यस्य यावद्भिर्न्यूनत्वं त एव शिष्यन्ते~। अहर्गणगुणितस्य भाज्यस्य भागहारतुल्यत्व एव हि भगणपरिपूर्त्तिः~। तस्मात् तच्छेषतुल्ये राशौ तत्र क्षिप्त एव भागहारे(णा ? ण) हृते निःशेषता स्यादित्ये(कः ? कं) फलं पूर्येत~। तत्रापि सोंऽशो न वास्तवः~। ततः पुनश्चोद्भूते भागहारादाधिक्यमपि स्याच्छेषस्य~। स च नवमनुतुल्यशेषहीनभाज्यसङ्ख्यः~। स च भवेदसङ्ख्यः~। तत्र यः पूर्वस्माद्दिनाच्चरमस्य दिनस्योपचयो जातः तदंशेन भगणपरिपूरणं कृत्वा शिष्टमेव शेषतया ग्राह्यम् इति द्वितीयवर्षादौ भवेदतुल्यमुखम्~। पुनरपि षट्सप्तपञ्चतुल्य एव चयः~। एवं त्रिंशत्स्वरतुल्येऽहर्गणे पूर्वस्माद्द्विगुणो गन्तव्यो द्विगुणनवमनुतुल्यः~। तस्मात् ततः परेऽहनि योऽत्र भाज्यः तत्समात् प्रतिदिनोपचयात् पूर्वदिनगन्तव्यशेषं\renewcommand{\thefootnote}{६}\footnote{ष एव \textendash\ क. पाठः.} नवगनुयुगतुल्यं त्यक्त्वा यः शेषो वसुभसङ्ख्यः

\newpage

\noindent स एव रूपाग्निस्वरतुल्येऽहर्गणे तृतीयवर्षाद्युदये शेष इति तत्तुल्यं तस्मिन् वर्षे मुखम्~। एवं प्रतिवर्षं मुखभेदाच्चयसाम्याच्च नानाभूतः शेषः~। तत्र सर्वेषामेकद्व्यादीनां शेषत्वसम्भवादनिरूप्यैव यदृच्छया वाच्यायात एव शेषो वाच्यः~। किन्तु हारकान्न्यूनत्वमेव निरूपणीयं, ततोऽधिकत्वं माभूदिति~। यदा पुनर्नवमनुसङ्ख्यो गन्तव्यशेषो वर्षगुणितो भाज्यात् त्यक्तुं न शक्यः, तदा तद्विपरीतशेष एकाधिकदिनेऽपि गन्तव्य एव~। ततः पूर्वस्मिन् वर्षे तृतीया(स्ते ? न्ते) नवमनुतुल्येन शेषेण भाज्ये हृते यः शेषः स न गन्तव्यशेषः, अपितु गतशेष एव~। यतो गन्तव्यशेषोपचयादहर्गणे प्रक्षिप्यमाणेनैकेन दिनेन जात उपचयोऽधिकः स्यात्~। तस्मात् कुट्टाकारे द्वितीयहरणे शेषोऽपगतशेष एव~। एवं पुनः पुनः पद\renewcommand{\thefootnote}{१}\footnote{पुनः पद \textendash\ ख. पाठः.}स्यौजत्वे उपरिगतः शेषो गन्तव्यशेष एव, युग्मत्वे\renewcommand{\thefootnote}{२}\footnote{युग्मास्या}ऽधोगतस्याल्पत्वात् स शेषो गतशेष एवेति नियमो द्रष्टव्यः~। अतः सर्वदा शेषस्यैकविधत्वायैव ह्यधिकमपि भाज्यमल्पीकृत्य परस्परहरणमुक्तम् \textendash\ {\qt अधिकाग्रभागहारं छिन्द्यादूनाग्रभागहारेण शेषपरस्परभक्तमि}ति~। ननु लौकिककुट्टाकारविषयमेव तदल्पीकरणम्~। ग्रहगणिते पुनः कस्यचिदपि भगणस्य भूदिनादधिकत्वाभावान्न ग्रहगतिविषयम्~। नैवम्~। न केवलं भगणस्यैवात्र भाज्यत्वं भागकलादीनामपि प्रश्नवाक्यानुसोरण भाज्यत्वसम्भवात् ग्रहगणितेऽप्येतत् समानम्~। यत्र भागशेषं कलाशेषं वोद्दिश्य तत्र भागीकृतं कलीकृतं वा भगणं भाज्यत्वेन गृहीत्वा तेन प्रश्नः कृत इति ज्ञेयम्~। यतस्तत्राशीकृतं भगणमहर्गणेन हत्वा दृढहारकेणैव हृते यः शेषः स उक्त इति न तत्र\renewcommand{\thefootnote}{३}\footnote{इति तत्र \textendash\ क. पाठः.} केवलभगणो भाज्यः~। तत्र शशिबुधशुक्राणां भागीकृतानामेव हारकादाधिक्यं स्याद्भागाधिकत्वात् दिनगतेः~। कलीकृताः पुनः सर्वेषामपि भाज्या दृढभागहारादधिका एव, यतः कस्यचिदपि (न) विकलामात्रगतित्वम्~। सर्वेषां मन्दगतेः शनैश्चरस्यापि दिनगतिः कलाद्वयादधिकैव~। मन्दोच्चभगणैः पातभगणैश्चात्र न कुट्टाकारः कर्तुं शक्यः, कुजादीनां तद्भगणस्यानुपदिष्टत्वात्~। चन्द्रस्य तु तयोरपि पातमन्दोच्चयोर्गत्योः कलात्रिकषट्काभ्यामाधिक्यादेव कलीकृतानां दृढहारादाधिक्यम्~। अत उक्तं\textendash

\newpage

\begin{quote}
{\qt भाज्योऽधिको यदि भवेत् खलु हारराशेः\\
तत्राधिकं समपनीय तथैव कर्म~।\\
तेनाधिकेन गुणितो गुणकारराशिः\\
युक्तोऽधरेण स भवेत् पृथगत्र लब्धम्~॥}
\end{quote}

\noindent इति~। वेलाकुट्टाकारे पुनः कलीकृतानामपि हारकादल्पत्वं सम्भवत्येव, तत्राभीष्टवेलाविषयत्वात् प्रश्नस्य~। यामकालहोराना(स्या? ड्या)दिवशादुद्देशकवाक्यानुसारेण तत्तच्छेदगुणितस्यैव भागहारत्वात् तच्छेषाणां पुनरंशीकृताद्धारकादाधिक्यमेव न युज्यते~। तदधोगताः शेषा यथेष्टमुद्देष्टव्या एव~। एवमेतत्सर्वं साधारणमेव स्थिर\renewcommand{\thefootnote}{१}\footnote{साधारणवत् स्थिर \textendash\ क. पाठः.}कुट्टाकारेऽपि~। भाज्यभाजकयोरन्यतरस्य द्वयस्य वांशीकरणे पृथक्पृथक् गुणकारशेषौ ग्राह्यौ~। कः पुनः स्थिरकुट्टाकारः~। यत्र रूपमात्रे शेषे गुणकारमानीय पुनः कुट्टाकारं विनाप्युद्देशकेन यथेष्टमुद्दिष्टानां गुणकारा गुणनमात्रेणैव सिद्ध्यन्ति तत्र सकृत्कृतेनैव कुट्टाकारेण स्वयं कृतेन वा गुर्वादिभिः कृतेन वा तदुपदेशेनैव तज्जातीयेषु विषयेषु सर्वत्रोत्तरं वक्तुं शक्यमिति (तस्य ?) स्थिरीकृत्य ध्रुववत् पठितत्वात् स्थिरत्वं तस्य~। तदर्थमेव तद्द्वयं फलं च गोविन्दस्वामिनोक्तं\textendash

\begin{quote}
{\qt रूपापचयसिद्धोऽयं गुणकारो रवेर्भवेत्~।\\
दस्रखाङ्गाहिरन्ध्राख्यो लब्धं रन्ध्रेषुनेत्रकम्~॥}
\end{quote}

\noindent इत्यादिभिः~। तत्र प्रतिवत्सरं नानाजायमानानां मुखानां मध्ये तत्तत्फलमितेष्वादिभूतानां नवमनुनवार्कादीनां परम्परायां यदन्त्यमुखं रूपसमं स्यात् तत्र यौ वल्ल्युपसंहारेणानीतौ तदहर्गणभगणौ तावेवेहोपदिष्टौ, तस्येतरशेषेभ्यो न्यूनत्वात्, सर्वशेषसाधारणभूतत्वात्~। ततो द्वि(ति ? त्रि)गुणादिशेषाणां द्वित्र्यादिगुणिते तदहर्गणे सम्भवस्य निर्णयादिष्टशेषगुणनेनानीतो योऽहर्गणः स उद्देशकोद्दिष्टशेष एव स्यात्~। यथा रूपशेषस्याहर्गणस्याभीष्टशेषगुणनया तच्छेषाहर्गणसिद्धिः, एवं तद्भगणस्याप्यभीष्टशेषगुणनेनैव तत्सम्बन्धित्वं स्यात्~। तत्र रूपशेषस्याहर्गणस्येष्टशेषगुणनेन लब्धो यो गुणकाररूपोऽहर्गणः तस्मिन्नुद्देशकोद्दिष्ट एव शेष इत्येव निर्णीयते~। न पुनस्तस्मादहर्गणात् प्राक् तावच्छेषः क्वचिदपि न स्यादिति न निर्णीतम्~। सम्भवति च प्रागपि बहुकृत्वस्तावान् शेषः~। कथम्~। एत\renewcommand{\thefootnote}{३}\footnote{एक \textendash\ ख. पाठः.}-

\newpage

\noindent स्मिन् रन्ध्रेषुनेत्रे शतेन गुणिते योऽब्दगणो जायते तत्र षट्सप्तपञ्चवर्षाणि बहूनि युगानि सन्ति~। तदन्तर्भूतेषु सर्वेषु, युगेषु एकस्मिन्नेकस्मिन् दिने एतावाञ्छेषोऽभूत्~। तस्मात् यावन्ति युगान्यतीतानि एतावाञ्छेषोऽपि तावत्कृत्वोऽभूदिति नैतावुद्देशकाय देयावहर्गणभगणौ~। ततस्तस्मादतीतयुगेष्वपास्तेषु वर्तमानयुगगतोऽहर्गणः सिद्ध्यति~। तावत्यहर्गण एव प्रथममेतावाञ्छेषः स्यात्~। प्रथमभूताहर्गण एव चोत्तरतया देयः, प्रथमातिक्रमे कारणाभावादिति~। तत्रापरितुष्यत एव पृच्छकाय तस्मिन्नेवापरितोषं हारकं मुहुर्मुहुः प्रक्षिप्य सोऽहर्गणो वक्तव्यः~। तस्मादत्रातीतयुगानां
तक्षणेन त्यागः कार्यः~। अतएवोक्तं\textendash

\begin{quote}
{\qt रूपमेकमपास्यापि कुट्टाकारः प्रसाध्यते~।\\
गुणकारोऽथ लब्धं च राशी स्यातामुपर्यधः~॥

इष्टेन शेषमभिहत्य भजेद्दृढाभ्यां\\
शेषं दिनादि भगणादि च कीर्त्यतेऽत्र~।}
\end{quote}

\noindent इति~। वेलाकुट्टाकारमपेक्ष्य दिनादीत्युक्तम्~। राशिशेषाद्यपेक्षयैवेतरत्राप्यादिग्रहणम्~। एवमेकेनैव वर्धितेनाहर्गणभगणावानीय स्थिरकुट्टाकारे तक्ष्यते~। अन्यत्र द्वाभ्यां शेषाभ्यां वर्धिताभ्याम्~। तत्र यावद्गुणिते नवार्कमिते शेषे उद्दिष्टशेषे क्षिप्ते त्यक्ते वा यावद्गुणितेन नवमनुमितेन शेषेण यावद्गुणितेन साम्यं स्यात्, तत्राल्पसङ्ख्यो हार्यो यावता गुण्यते सा मतिः, यद्गुणितस्य हारकस्य शेषयुतहार्यतुल्यत्वं तत् फलं, {\qt भाज्याद्धरः शुध्यति यद्गुणः स्यादन्त्यात् फलं तदि}त्युक्तत्वात्~। तत्र यावता च तत्तच्छेषौ गुणितौ तदहर्गणौ च तावता तावता गुणयित्वा योज्यौ~। तदेव चात्र वल्ल्युपसंहारेण क्रियते~। यदि सकृद्धृतभाजकशेष एव मत्या हन्यते तदा मत्या तत्फलं हन्तव्यम्~। प्रथमफलतुल्ये हि गुणकारे तावत् प्रथमहृतशेष एव गन्तव्यशेषः~। स एव शेषो यदा केनचिद्राशिना हतो यः, तत्तुल्ये गन्तव्यशेषे यो गुणकारः सोऽपि पूर्वगुणकारात् तावद्गुण एवेति तात्सिद्ध्यर्थं स च फलं च मत्यैव हन्यते~। मत्या हतः स शेषो यदा भाज्येन हृतः, तदा तावद्भिर्दिनैस्तद्गुणकारस्याधिक्यं स्यात्~। तत्र य शिष्यते स एव तदानीं गन्तव्यशेषः~। तत्र यद्युद्दिष्टशेषं विशोध्य भाज्येन

\newpage

\noindent ह्रियते तदापि निश्शेषतायां\renewcommand{\thefootnote}{१}\footnote{तया \textendash\ ख. पाठः.} त्यक्तः शेष एव गन्तव्यशेषः~। भाज्येन ह्रियमाणे यत् फलं, तद्दिवसहतो\renewcommand{\thefootnote}{२}\footnote{गतो \textendash\ क. पाठः.} यो भाज्यो हरणे ततस्त्यक्तः तस्य गतशेषत्वात् तत्त्यक्तशेष एव गन्तव्यशेषः~। तत्र यदि शेषो न शोध्यते तदा मतिगुणितस्यान्यस्माद्गतशेषाच्छेषेणाधिक्यात् गन्तव्यशेषस्य तावतातिरिक्तता~। तत्र यद्युद्दिष्टशेषः शोध्येत, तदा तच्छुद्धशेष एव निःशेषं ह्रियेत~। तदा यद्गुणितो भाज्यः ततस्त्यक्तः तद्धि मतिफलम्~। तावता दिनेनाधिक्ये तावता शेषस्याधिक्यं स्यात्, गतशेषस्य चाधिक्यम्~। तेन गन्तव्यशेषस्य चेन्न्यूनतैव युक्ता~। तत्र यद्युद्दिष्टशेषोऽपि न शोध्यते, नापि योज्यते, तत्र केवलो मतिहत एव प्रथमशेषो भाज्येन ह्रियते च, तदा गन्तव्यशेषस्याधिक्यात् मतिहतपूर्वफलस्य मतिफलस्य च योगतुल्येऽहर्गणे तावता शेषेण न्यूनेन भाव्यम्~। यदि निःशेषता स्यात् तदा भागहारतुल्य एव स गुणकारः, अन्यत्र निश्शेषत्वासम्भवात्~। यदा पुनस्तस्य प्रथमशेषस्य मत्यान्येन वाभीष्टेन केनचिद्धनने कृते भाज्येन ह्रियमाणस्य तस्य गन्तव्यशेषो यः तावता तस्येतरस्मान्न्यूनत्वात् तावता गतशेषस्य चाधिक्यात् मतिहत(स ? स्य) भाज्यस्य च गतशेषत्वाद्गतशेष एव तावांस्तद्दिने~। तत्रापि हतपूर्वफलस्य मतिफलस्य च योग एव गुणकारत्वेन विवक्ष्यते~। ततस्तावति गुणकारे गतशेष एवावशिष्यते~। तस्मात् यं कञ्चिच्छेषं क्षिप्त्वा भाजकशेषे हृते वल्ल्युपसंहारेणानीतेऽहर्गणे क्षेपसमो गतशेषः~। यं कञ्चित् त्यक्त्वा भाज्येन निश्शेषं ह्रियते चेत् गन्तव्यशेषस्य तावताधिक्याद्गन्तव्यशेष एव स इति तत्र क्षेपणशोधनयोर्वैपरीत्यं स्यात्, यतो गतशेषस्य शोधनं गन्तव्यशेषस्य क्षेपणं चोक्तम्~।

\begin{quote}
{\qt केनाहतोऽयमपनीय यथास्य शेषं\\
भागं ददाति परिशुद्धमिति प्रचिन्त्य~।\\
आप्तां मतिं तां विनिधाय वल्ल्या\\
नित्यं ह्यधोधः क्रमशश्च लब्धम्~॥}
\end{quote}

\noindent इति पूर्वमुक्त्वा

\begin{quote}
{\qt गन्तव्यमिष्टं यदि कस्यचित् स्यात्\\
गन्तव्ययोगादिदमेव कर्म~।}
\end{quote}

\newpage

\noindent इति पुनर्गन्तव्यशेषे विशेषस्योक्तत्वाद्गतशेषस्यैव पूर्वमपनयनमुक्तमिति गम्यते~। तस्माद्गतशेषस्य शोधनं गन्तव्यशेषस्य च क्षेपणमाचार्याणां मतम्~। यदि सकृद्धृतेऽप्युपरिस्थो भाज्य एव चेत् येनकेनचिद्धन्येत भाज्यशेषेण च ह्रियेत तत्र\renewcommand{\thefootnote}{१}\footnote{तत तत्र \textendash\ ख. पाठः.} शिष्टे गतशेषे सति योऽहर्गणः
तत्सिद्ध्यर्थं पूर्व\renewcommand{\thefootnote}{२}\footnote{पूर्वं}द्वितीयफलयोर्घाते येन मतिस्थानीयेन भाज्यो हतः स च प्रक्षेप्यः, तदा तदहर्गणः स्यादिति~। तत्र तन्मतिफलमेव पूर्वफलस्याधः स्थाप्यम्~। मतिस्थनीयो राशिश्च तदधः, फलयोर्घाते तस्य क्षेप्यत्वात्~। अधउपरिगुणितमन्त्ययुगित्यस्यापि घातात् यदि तत्फलं सर्वाधो न्यस्येत, तदुपरि च भाज्यस्य गुणकारः, तदोपान्त्येन तदूर्ध्वे हतेऽभीष्टो राशिर्न लभ्येत~। अत्रापि फलयोर्घात एव स्वभाज्यगुणकारः प्रक्षेप्यः इतीदानीं प्रतिपादितया युक्त्यैव सिद्धम्~। अतो भास्करोक्तव्यत्यासेन तौ मतितत्फलराशी स्थाप्यौ~। यदि पुनः प्रथमशेषेण भाज्यं च हृत्वा भाज्यशेषं च येनकेनचिद्धत्वा प्रथमेन भाजकशेषेणैव हरेत्, तदा यः शिष्टः तावति गत\renewcommand{\thefootnote}{३}\footnote{शिष्टः तं मतिगत}शेषे योऽहर्गणः तदानयनं कथमित्यत्र निरूपणीयम्~। तत्र प्रथमशेषे सति गन्तव्यशेषे प्रथमफलमेव गुणकार\renewcommand{\thefootnote}{४}\footnote{कारत्व} इति तावत् सुगममेव~। तत्फलगुणिताद्भाज्यात् तद्भाजकस्य तत्र शिष्टेनाधिक्यं स्यात्~। तावताधिक्यं च भाजकस्य, भाज्यहृतस्य ह्येतच्छिष्टं शिष्यते च~। भाजकस्याधिक्यात् तस्येदानीं भज्यता\renewcommand{\thefootnote}{५}\footnote{भाज्यत्वात् \textendash\ क. पाठः.} भाज्यस्य च भाजक(त्वात् ? त्वम्~।) येन यद्ध्रियते यच्च तत्र शिष्टं तत्फलगुणितस्य तद्धारकस्य चान्तरमेव तच्छिष्टम्~। अन्तरं च तद्गुणिताद्भाज्यात् भाजकस्यातिरिक्तभाग एव~। तस्मात् तत्फलगुणितो भाज्यराशिस्तत्र शिष्टेन न्यून इति गन्तव्यक्षेपणेन पूरयित्वा ह्रियमाणे पूर्वफलादेकाधिकं फलं च लभ्यम्~। अत्र पुनस्तत्क्षेपाभावे फलमेकमपि न पूर्णम्~। ततः शेषं प्रक्षिप्याप्तमेकं परिपूर्णं स्यात् इति प्रथमफलेन भाज्ये गुणिते रूपफलस्यैव हृतशेषो गन्तव्य इति~। प्रथमशेषो येन हन्यते तेन रूपहनने तत्फलं च स्यात्~। तत्र भाज्यस्य प्रथमशेषेण हरणेऽपि तत्फलेन प्रथमशेषो हन्यत एव, यतो येन हत्वा हारको विशोध्यते तस्यैव फलत्वम्~। तस्माद्द्वितीयफलतुल्या प्रथमफलस्यावृत्तिरिति तत्फलद्वयघाततुल्ये गुणकारे द्वितीयफलहतप्रथमशेष एव गन्तव्य-

\newpage

\noindent शेषः~। किन्तु भाज्यो येन गुणितो ह्रियते तावद्भिर्दिवसैः फलघाताद्गुणकारस्याधिक्यं स्यात्~। तदा शेषस्यापि तेन गुणितेन भाज्येनाधिक्याद्द्वितीयफलगुणितेन प्रथमशेषेण न्यूनत्वाच्च तदन्तरतुल्य एव तत्र शेषः~। तच्चान्तरमपि मतिगुणितस्य भाज्यस्यैकदेश एव, यतो मतिगुणितं भाज्यं प्रथमशेषेण हृत्वा तत्फलं च शेषश्च लभ्येते, तत्र शेषो ह्रियमाणस्यैकदेश एवेति~। द्वितीयफलगुणितात् प्रथमशेषान्मतिगुणितस्य भाज्यस्यैवाधिक्यात् तच्छेषो गतशेष एव~। एवं द्वितीयहरण एव मतिकल्पना कार्या~। न पुनर्हृत्वा शिष्टस्यैव मतिगुणना कार्येति नियमः~। इति द्वितीयमतिकल्पनास्थानं भाज्य एव, तृतीयस्थानमेव भाज्यशेष इति मतिकल्पनायाः स्थानोत्कर्षवशाद्वल्ल्युपसंहारादिविशेषश्चिन्त्य इति प्रथमशेषमतिहननन्यायनिरूपणानन्तरं केवलभाज्यस्य मतिहननप्रथमशेषहननाभ्यां\renewcommand{\thefootnote}{१}\footnote{भ्यां यो \textendash\ क. पाठः.} पूर्वस्मात् कर्मणो यो विशेषः तदुपपत्तिं निरूप्यैव प्रथमशेषहृतभाज्यशेषमतिकल्पनायुक्तिः पुनरेव निरूप्येति ततः प्राक्तनकर्मणो युक्तिरिदानीं निरूप्यते~। तत्र क्षेपस्य शोधनं क्षेपं
वा कृत्वैव यथाप्राप्तस्य शेषस्याहर्गण एव निरूप्यताम्~। तेन तद्गुणकारस्य मत्याख्या नैव स्यात् विमृश्यकार्यत्वाभावात्~। तस्माद्भाज्यमिष्टेन
हत्वा प्रथमेन भाजकशेषेणैव हृते यः शेषः तावति मण्डलादिशेषे सति कियानहर्गण इत्येव प्रथमं निरूप्यताम्~। तथापि वल्ल्युपसंहारयुक्तिः प्रकाशेत
यतो नानाविषयज्ञानं युगपन्न स्यात्~। युगपज्ज्ञानानुत्पत्तिः खलु मनसो लिङ्गम्~। तेन शेषक्षेपशोधनयुक्तिजिज्ञासा तिष्ठतु~। तत्रोदाहृते रविमण्डलशेषे
गन्तव्ये नवमनुमिते भाजकाद्भाज्यहृतं फलं पञ्चषष्ट्युत्तरं शतत्रयमेव नवमनुमिते गन्तव्यशेषे उपचितः स्यात् , यावानिह केनचिद्गुणितात् भाज्यात् त्यक्तो राशिः~। यतः प्रत्यब्दं भाजकशेषतुल्य उपचयः, स एव द्वितीयफलगुणितो यः तस्य तावदावृत्तत्वात् तावत्सु वर्षेषु गन्तव्यशेषः फलगुणितप्रथमशेषतुल्यः~। तस्य च गन्तव्यशेषत्वाच्छेषस्यापचय एवायम्~। तस्मादितोऽधिक उपचयः केनचिद्गुणितो भाज्यः, यतस्तस्मादेष विशोध्यते~। तत्र शेषो गत एव~। ततो येन राशिना भाज्यो गुणितः स तावद्भिर्दिनैर्जायमान उपचय एव~। तस्मात् फलद्वयघाते भाज्ये गुणकारं क्षिप्त्वा यो गुणकारो लभ्यते तत्र गतशेष एव हृतशेषः~। तस्मात् तत्र यथाप्राप्तस्य शेषस्य गतशेषत्वम्, उपचयापचययो-

\newpage

\noindent र्विश्लेष उपचयस्याधिक्यात्~। ततस्तावति गतशेषे गुणकारभूताहर्गणसिद्ध्यर्थं फलद्वयघातो भाज्यगुणकारश्च प्रक्षेप्यः~। भाज्यगुणकारतुल्यैर्दिनैरेव तद्गुणितभाज्यतुल्य उपचयः स्यात्, यतो दिन\renewcommand{\thefootnote}{१}\footnote{यतो तयोर्दिन \textendash\ ख. पाठः.}गणो भाज्येन हन्यते~। तस्मात् प्रतिदिनं भाज्यतुल्य एवोपचयः~। ततस्तत्समुदायात्
प्रत्यब्दं योऽपचयः पूर्वहृतशेषः स प्रथमफलतुल्यस्याहर्गणस्यैव गन्तव्यशेषः~। स यावद्गुणितोऽन्यस्माच्छोध्यत इति तत्फलमपि तावता गुणनीयम्~। तद्घाततुल्यस्याहर्गणस्य स एव गन्तव्यशेषः, यो द्वितीयफलगुणितः प्रथमशेषः~। तस्य गन्वव्यशेषत्वादपचयात्मकत्वात् इष्टगुणिताद्भाज्यात् तद्विशोध्यत इति तावन्ति दिनानि केवलान्येव घाते योज्यानि, न पुनः केनचिद्गुणितानि तानि~। तस्माद्भाज्यस्य गुणकारः स इष्टराशिः मतिस्थानीयोऽपि फलघाते क्षेप्य एव~। यतस्तत्संयुक्तफलघाताहर्गणस्य स द्वितीयशेष एव गतशेषः~। तस्मात् तत्र भाज्यस्य गुणकारः सर्वाध एव स्थाप्यः, यतस्तस्य फलघाते क्षेपणमेव कार्यं न पुनस्तेनान्यो राशिर्गुणनीय इति~। अन्त्यस्यैव च योज्यत्वोक्तेः अधउपरिगुणितमन्त्ययुगिति प्रथमफलस्य द्वितीयफलस्य च अधउपरिस्थितयोः स गुणकारो योज्य\renewcommand{\thefootnote}{२}\footnote{योज्यत} एव, इति मतिस्थानीयस्तत्फलादध एव तत्र स्थाप्यः~। तस्माच्छेषयोर्द्वयोर्महति मत्या हतेऽल्पेन च हृतेऽत्र मति\renewcommand{\thefootnote}{३}\footnote{हृते मति \textendash\ क. पाठः.}फलं पूर्वफलादनन्तरस्थाधः स्थाप्यम्~। ततोऽप्यध एव मतिः स्थाप्या~। तत्र यथोक्तं वल्ल्युपसंहारे कृते यो गुणकारो लभ्यते तावत्यहर्गणे द्वितीयहृतशेष एव गन्तव्यशेषः~। तत्रापि योजनीयं सूत्रं {\qt शेषपरस्परभक्तं मतिगुणमि}त्येवोक्तत्वात्~। न पुनः परस्परहरणानन्तरं मतिगुणनमित्यत्र किञ्चिल्लिङ्गमस्ति~। यद्धरणं क्रियते तत् परस्परमेव कार्यम् इत्येवात्रोच्यते~। मतिगुणनस्यापि तत्रान्तर्भावो न निवार्यते~। द्वितीयहरणं मत्या गुणयित्वा न कार्यमिति~। अपिच प्रथमं हृत्वाल्पीयसः शेषस्य मतिगुणने पुनरपि तस्यैव हरणमिति हरणवैषम्यमपि स्यात्~। भाजकस्थानात् सकृदेव हृतं भाज्यस्थानात् द्विर्हृतम् इति परस्परता हीयेतेति भाज्यस्थानगतस्य पराजय एव~। इति द्वितीयहरणात् प्रागेव मतिकल्पनायां शब्दत आर्जस्य (?)~। तस्मिन् भाज्ये भाजकशेषेणापि हृते मतिकल्पनायां लाघवं स्यात्~। तया मत्या प्रागेव गुणितेऽपि न कश्चिद्विशेषः, यतस्तच्छेषात् मतिगुणितादुद्दिष्टशेषे विशोधिते निःशेषहरणं स्यादिति निःशेषहरणमेव मतिकल्पनायाः प्रयोजनं, तयैव मत्या कृत्स्ने भाज्ये

\newpage

\noindent गुणितेऽपि~। यतः शेषादितरभागस्य निःशेषहरणं कृतं ततस्तावतो भागस्य येनकेनचिद्गुणितस्यापि तेनैव भाज्यशेषेण ह्रियमाणस्य निःशेषत्वमेव स्यात्~। तत्फलस्यैव पूर्वफलान्मतिगुणत्वाद्विशेषः~। भाजकशेषस्यावृत्तिरेव भाज्यशेषस्य हृतांशः~। शेषांशस्यैव भाजकशेषादल्पत्वात् निःशेषहरणसंवादाभावः~। हृतो भागः संवदत एव~। ततस्तावतो भागस्य येनकेनचिद्गुणितस्यापि तावद्भागस्य समुदाय एव गुणितोऽपीति समुदायिनः प्रत्येकं संवादे समुदायस्यापि संवादः स्यादेव~। ततस्तच्छेषे कल्पितया मत्या (पि ?) गुणितेऽपि कृ(त्स्नो ? त्स्ने) भाज्ये गुणितेऽपि यो विशेषः स मतिगुणनात् प्राक् हृतभागस्य समुदाय एवेति स सर्वथा\renewcommand{\thefootnote}{१}\footnote{एवेति सर्वथा \textendash\ ख. पाठः.} निःशेषं हर्तुं शक्यः~। शेषादप्य\renewcommand{\thefootnote}{२}\footnote{शेषमप्य \textendash\ क. पाठः.}भीष्टशेषं शोधयित्वा क्षिप्त्वा वा यदि निःशेषहरणं शक्यं स्यात् तर्हि कृत्स्नस्यापि भाज्यस्य तया मत्या गुणितस्येष्टक्षेपसंस्कृतस्य निःशेषहरणं शक्यं स्यात्~। द्वितीय\renewcommand{\thefootnote}{३}\footnote{शक्यम्~। द्वितीय \textendash\ क. पाठः.}फलगुणितभाजकशेषतुल्यस्य भाज्यांशस्य निःशेषं हार्यत्वाच्च~। शेषतुल्यस्य भाज्यांशस्यापि मतिगुणितस्य शेषसंस्कृतस्यापि निःशेषं हर्तुं शक्यत्वात् तदुभयांशव्यतिरिक्तस्य तत्रासम्भवान्न्यूनत्वासम्भवाच्च कृत्स्नभाज्यस्यापि मतिगुणितस्येष्टशेषसंस्कृतस्य शिष्टस्योभयांशात्मकस्यापि निःशेषहरणं स्यादेवेत्युभयथापीष्टगुणकारः सिद्ध्यत्येव~। तत्र फलस्थाने फलद्वयादधो मतिरेव स्थाप्या, तदधो मतिफलं च~। यतोऽत्र मत्या फलद्वयघातो गुण्यः तत्फलेन च पूर्वफलमेव~। तत्र मत्या गुणितेन द्वितीयफलेन प्रथमफले गुणिते यत् फलं स्यात् तदेव फलद्वयघाते मत्या गुणितेऽपि~। इति मत्या स्वो\renewcommand{\thefootnote}{४}\footnote{स्यो \textendash\ क. पाठः.}परिस्थं द्वितीयफलं हत्वा तद्घातेन पुनः सर्वोपरिस्थं प्रथमफलमपि वा गुण्यतां, मतिफलेन पूर्वफलस्यैव गुण्यत्वात्~। तत्प्रथमफलगुणकारे मतिद्वितीयफलघाते मतिफलं क्षिप्त्वा तेन प्रथमफले गुणिते मतिगुणितः पूर्वफलघातः मतिहतपूर्वफलं च संयुक्तं स्यात्~। तत्फलसंयोग एव तत्र जिज्ञास्यो गुणकारः~। कथं तत्र तस्य गुणकारत्वम्~। उच्यते~। यदा भाज्यहृतभाजकशेषतुल्यः क्षेप्यशेषः तदा तत्फलमेव गुणकार इति पूर्वमेव प्रतिपादितः~। तच्छेषेण च भाज्ये हृते स भाज्यशेषो यावान् तत्तुल्ये गतशेषे फलद्वयघातः सरूप एव गुणकार इत्येतच्च प्रदर्शितम्~। तत्र केवलभाज्यस्यैव हार्यत्वा-

\newpage

\noindent न्म(ति? ते)रेकसङ्ख्यत्वादधउपरिगुणिते तद्रूपमेव मतिस्थानीयं क्षेप्यम्~। तत्र फलस्योर्ध्वस्थापनं मतेरधस्थापनं चोक्तम्~। अतस्तस्य
रूपस्यान्त्यत्वादुपर्यधस्थयोः फलयोर्घाते तस्य योज्यत्वं सिद्धम्~। ततो द्वितीयशेषतुल्ये शोध्यशेषे फलघातः सरूप एव गुणकारः~। अभीष्टशेषः पुनरन्य एव स्यात्, भाज्यभाजकयोः परस्परं ह्रियमाणयोर्ये शेषाः तेष्वन्यतम एवोद्देशकेनोद्देश्य इति नियमाभावात्~। इति तच्छेषसम्बन्ध्यहर्गणसिद्ध्यर्थं तच्छेषेष्वन्यतमं मत्या हत्वोद्दिष्टशेषं संस्कृत्य निःशेषं ह्रियते~। ततः शेषस्य सङ्ख्याभेदमात्रेण वल्ल्युपसंहारस्य तद्युक्तेश्च भेदो न स्यात्~। ततो यंकञ्चिच्छेषं कल्पयित्वापि वल्ल्युपसंहारादियुक्तिर्निरूप्यैव~। त(त्ररवि? त्रापि) भाज्यभाजकयोः परस्परहृतयोर्यो भाज्यशेषो नवार्कसङ्ख्यः, तत्र नवाधिकं शतं शोध्यशेषं कल्पयित्वात्र युक्तिः प्रथमं प्रदर्श्यते~। तत्र द्विसङ्ख्या मतिः, लब्धं चैकम्~। तत्र द्वितीयशेष एव नवार्कसङ्ख्ये शोध्यशेषे\renewcommand{\thefootnote}{१}\footnote{शेषं \textendash\ क. पाठः.} फलद्वयघातो रूपयुक्तो गुणकार इत्येतत् प्रतिपादितम्~। तद्द्विगुणे शेषे तु गुणकारश्च द्विगुणः स्यादिति पूर्वगुणकारे मतिहते द्विघ्ननवार्कतुल्ये शेषे यो गुणकारः स स्यात्~। ततो नवमनुतुल्योनशेषस्याभीष्टत्वात्\renewcommand{\thefootnote}{२}\footnote{स्या \textendash\ ख. पाठः.} अस्माच्छेषात् तावदूनो गन्तव्यशेषो यावतोऽहर्गणस्य स्यात् तमपि मतिगुणे प्रकृत एव योजयित्वा स च लभ्यते~। तस्य गन्तव्यत्वात् तत्फलपूरणायास्य मतिगुणितशेषस्य तावानंशो देय इत्यस्मात् तावति शुद्धे अभीष्टशेषः स्यात्~। अस्मात् तावदूनस्याभीष्टत्वात् तस्मिन् सशेषे शोधिते, हरणे निःशेषत्वमपि स्यात्~। एवं सति मतिगुणि(ता)त् द्वितीयशेषादभीष्टशेषं त्यक्त्वा प्रथमेन भाजकस्थानस्थेन शेषेण तुल्यत्वात् तेन ह्रियमाणे एकं फलं लभ्यते~। तत्र शेषे त्यक्ते नवार्कतुल्यं शिष्टम्~। तावतैव प्रथमफलस्य भाज्यहतस्य भाजकान्न्यूनत्वम्~। ततस्तत्र प्रथमतुल्ये गुणकारे क्षिप्ते तद्धतस्य भाज्यस्य भाजकतुल्यत्वाय शिष्टं नवार्कसङ्ख्यं कृत्स्नं देयम्~। तत्र यः शेषो नवाशा\renewcommand{\thefootnote}{३}\footnote{नवांशः \textendash\ क. पाठः.}तुल्यः शोध्यशेषः शोधितः तावानेव शेषः स्यादित्यशेषस्य तस्याहर्गणः सः~। एवमन्यत्रापि फलद्वयघातस्य सरूपस्य मतिहतस्य, मतिफलहतस्य प्रथमफलस्य च योग एव गुणकारः~। तत्र पूर्वफलद्वयघाते रूपसहिते मत्या हन्तव्ये रूपं पृथक्कृत्य घात एव मत्या हन्येत तर्हि मतिरेव पुनस्तत्र क्षेप्या~। पृथक्कृतस्य रूपस्यापि

\newpage

\noindent मतिहतस्य मतितुल्यत्वात् तावतैवेतरांशस्य मतिहतस्य न्यूनत्वाच्च मतियुते सोंऽशः पूर्येत~। यः पुनरन्योंऽशः मतिफलहतपूर्वफलात्मकः तदर्थं पूर्वफलं मतिफलेन हन्तव्यम्~। तच्च मतिद्वितीयफलघातेन मतिफलसहितेन सर्वोपरिस्थे पूर्वफले हते उभयांशस्यापि लाभात् तदन्तर्भूतमेवेति तस्य न पृथग्घननं कर्तव्यमिति मत्या हतेन तदुपरिफलेन मतिफलसहितेन तदुपरिफले हते तत्राभीष्टगुणकारः स्यादिति तत्र मतेरध एव तत्फलं स्थाप्यं तस्यान्त्यत्वसिद्ध्यर्थम्~। अन्त्यमेव ह्युपान्त्यहते स्वोपरिस्थे योज्यम्~। अत्र मतिहते द्वितीयफले मतिफलस्य च क्षेप्यत्वम्~। तस्मात् फलपदक्रम एवमेव सर्वत्राल्पशेषस्य मतिहनने~। अधिकस्य मतिहनने च व्यत्यस्तः क्रमः~। एवं परस्परहरणे यो न्यायः\renewcommand{\thefootnote}{१}\footnote{अन्यः} प्रदर्शितः, स एव मुहुरपि परस्परहरणे योज्यः~। विषमपदे क्षेपणशोधनयोर्व्यत्यासश्चोक्तन्यायेनैव सिद्धः~। शेषयोरल्पमहतोर्मतिगुणितयोर्विशेषश्च~। तस्मात् प्रथमपरस्परहरणे यो न्यायः स एव मुहुर्मुहुरपि परस्परहरणे योज्य इत्येतन्न्यायकलापस्य कृत्स्नस्यापि परिग्रहाय परस्परहरणमुक्तम्~। अत्र सर्वत्रापि गुणकारस्य भाजकतक्षणं फलस्य भाज्यतक्षणं च ग्राह्यं लाघवाय परिगृहीतानां युगान्तराणां परित्यागाय~। तथा च परस्परहतशेषयोः गुणकाररूपया मत्या वल्ल्युपसंहारेण तक्षणेन च परस्परहरणात् प्राङ्न्यस्तयोर्भाज्यभाजकयोः हरणात् प्रागपि कल्पनीया मतिस्तत्फलं चैवानीयेते~। यदि तत्रैव सा मतिः स्फुरेत् तर्हि हरणादिकमपि न कार्यम्~। तदस्फुरणादेव परस्परहरणादिकं क्रियते~। एवं परस्परहरणेऽपि यौ शेषौ तत्रापि मत्यस्फुरणादेव पुनरपि परस्परं ह्रियते~। तत्रा\renewcommand{\thefootnote}{२}\footnote{तथा \textendash\ क. पाठः.}पि वल्ल्युपसंहारेण सकृत्परस्परहृतशेषस्थाने या मतिः स्फुरिता\renewcommand{\thefootnote}{३}\footnote{मतिरस्फुरिता \textendash\ ख. पाठः.} सापि तावत्पर्यन्तवल्ल्युपसंहारेणैव~। तद्गतभाज्यभाजकशेषाभ्यां तक्षणेन च तद्गतं फलं मतिश्च सिद्ध्यतः~। तत्र तक्षाकरणादेव सर्ववल्ल्युपसंहारे तक्षणे फलयोर्महत्त्वं स्यादिति सर्वत्र न्यायसाम्यात् {\qt शेषपरस्परभक्तं मतिगुणमग्रान्तरे क्षिप्तम्~। अधउपरिगुणितमन्त्ययुगूनाग्रच्छेदभाजिते शेषम्} इत्येतदन्तन्यायेनैव तत्र निरग्रकुट्टाकारे गुणकारः सिद्ध्यति~। एतावदेव कुट्टाकारशरीरम्~।

\begin{quote} 
{\qt कुट्टाकारे स्पष्टे भट\renewcommand{\thefootnote}{४}\footnote{भट्ट \textendash\ क. पाठः.}भास्करगोभिरुज्ज्वलेऽत्र मतिः~।\\
फलवत्यखिलेऽपि पदे कार्या बहुधा मुहुर्मुहुस्तष्टा~॥}
\end{quote}

\newpage

\begin{quote}
{\qt ऋक्षादेर्महतोऽल्पानां तदंशादिषु कस्यचित्~।\\
दर्शने द्युगु(णं? णात्) वा तन्मध्यमं भगणा(न्? द्) गता(न्? त्)~॥

शेषेणोर्ध्वांशतच्छेषौ तच्छिदो भाज्यता तदा~।\\
भाजकास्ते सहैवांशच्छेदैर्वाप्यपवर्तिताः~॥}
\end{quote}
\noindent इति श्रीकुण्डग्रामजेन\renewcommand{\thefootnote}{१}\footnote{श्रीकण्ठग्रामजेन} गार्ग्यगोत्रेणाश्वलायनेन भाट्टेन केरलसद्ग्रामगृहस्थेन श्रीश्वेतारण्यनाथपरमेश्वरकरुणाधिकरणभूतविग्रहेण जातवेदःपुत्रेण शङ्कराग्रजेन जातवेदोमातुलेन दृग्गणित\renewcommand{\thefootnote}{२}\footnote{दृग्गणितगणित \textendash\ ख. पाठः.}  निर्मापकपरमेश्वरपुत्रश्रीदामोदरात्तज्योतिषामयनेन र(चि? वि)त आत्तवेदान्तशास्त्रेण सुब्रह्मण्यसहृदयेन नीलकण्ठेन सोमसुता
विरचितविविधगणितग्रन्थेन दृष्टबहूपपत्तिना स्थापितपरमार्थेन कालेन शङ्कराद्यनिर्मिते\renewcommand{\thefootnote}{३}\footnote{शङ्करार्यभनिर्मिते \textendash\ क. पाठः.} श्रीमदार्यभटाचार्यविरचितसिद्धान्तव्याख्याने महाभाष्ये उत्तरभागे युक्तिप्रतिपादनपरे त्यक्तान्यथाप्रतिपत्तौ निरस्तदुर्व्याख्याप्रपञ्चे (स)मुद्घाटितगूढार्थे सकलजनपदजातमनुजहिते निदर्शितगीतिपादार्थे सर्वज्योतिषामयनरहस्यार्थनिदर्शके समुदाहृतमाधवादिगणितज्ञाचार्यकृतयुक्तिसमुदाये निरस्ताखिलविप्रतिपत्तिप्रपञ्चसमुपजनितसर्वज्यो\renewcommand{\thefootnote}{४}\footnote{कलज्यो \textendash\ ख. पाठः.}तिषामयनविदमलहृदयसरसिजविकासे निर्मले गम्भीरे अन्यूनानतिरिक्ते गणितपादगतार्यात्रयस्त्रिंशद्वाख्यानं समाप्तम्~॥
		
\vspace{2cm}		
\begin{center} 
\textbf{गणितपादः समाप्तः~।}\\
\vspace{.5cm}
\textbf{शुभं भूयात्~।}
\end{center}


\newpage 
\begin{center}
\begin{longtable}{|c|p{4.2cm}|p{3cm}|p{2.7cm}|}
	%\begin{longtable}{|c|l|l|l|}
		\caption{\textbf{स्मृतग्रन्थाद्यनुक्रमणी}}\\
		\hline
		{\textbf{पृष्ठम्}} &~~~~~~{\textbf{वाक्यानि}} & ~~~~{\textbf{ग्रन्थनाम}} &~ {\textbf{कर्तृनाम}} \\
		\hline
		१ & इष्टं हि विदुषां लोके\textemdash  & \ldots  & \ldots \\
		\hline
		२ & एक एव हि भूतात्मा\textemdash  & \ldots  & \dots \\
		\hline
		,, & प्रथमं सर्वशास्त्राणां\textemdash  & \ldots  & \ldots \\
		\hline
		,, & बिभेत्यल्पश्रुताद्वेदो\textemdash  & \ldots  & \ldots \\
		\hline
		,, & स्वयं स्वयम्भुवा सृष्टं\textemdash  & \ldots  & वृद्धगर्गः \\
		\hline
		,, & सिसृक्षुणा पुरा सृष्टं\textemdash  & \ldots  & \ldots \\
		\hline
		,, & आब्रह्मादिविनिस्सृत\textemdash  & \ldots  & वराहमिहिरः \\
		\hline
		३ & गणितज्ञो गोलज्ञो\textemdash  & \ldots  & \ldots \\
		\hline
		४ & गन्ते\textemdash  & \ldots\  & पिङ्गलः\\
		\hline
		,, & वर्गस्तावत्कृतिश्चेति\textemdash  & वैजयन्ती  & \ldots \\
		\hline
		५, ११९ & समद्विघातः कृतिरुच्यते\textemdash  & \ldots  & \ldots\\
		\hline
		९, ११, २३ & गुण्यस्त्वधोऽधो गुण\textemdash  & \ldots & \ldots\\
		\hline
		९ & खण्डद्वयस्याभिहतिः\textemdash  & \ldots  & \ldots \\
		\hline
		,, & एवं मुहुर्वर्गघन\textemdash  & \ldots  & \ldots \\
		\hline
		११ & पृथग्दोःकोटिवर्गाभ्यां\textemdash  & गर्गसंहिता  & \ldots \\
		\hline
		१२ & वर्गेण महतेष्टेन हतात्\textemdash  & \ldots  & भास्करः \\
		\hline
		,, & भक्तो गुणः शुद्ध्यति येन\textemdash & \ldots  & \ldots \\
		\hline
		१४ & गुणद्वयस्य संवर्गो\textemdash & \ldots  & गोविन्दस्वामी \\
		\hline
		,, & छेदघ्नरूपेषु लवा\textemdash & \ldots  & \ldots\\
		\hline
		१५, १८  & व्येकपदघ्नचयो मुख\textemdash & \ldots  & भास्करः \\
		\hline
		१५ & तयोर्योगान्तराहतिः\textemdash  & ,,  & \ldots \\
		\hline
		१७ & वर्गयोगपदे साध्ये\textemdash & \ldots  & \ldots \\
		\hline
		१९ & योगे खं क्षेपसमम्\textemdash & \ldots  & \ldots \\
		\hline
		,, & अल्पाक्षरमसन्दिग्धं\textemdash  & \ldots  & \ldots \\
		\hline
		२० & आद्यं घनस्थानमथाघने\textemdash  & \ldots  & भास्करः \\
		\hline
		२० & विपरीते विपरीतं न्याय्यम्\textemdash  & \ldots  & \ldots \\
		\hline
		२१, १७२ & भाज्याद्धरः शुद्ध्यति यद्गुणः\textemdash  & \ldots  & \ldots \\
		\hline
		२१ & समत्रिघातश्च घनः\textemdash & \ldots & भास्करः  \\
		\hline
		२५  & समद्वादशबाहौ तु\textemdash & \ldots  & \ldots \\
		\hline
		२९  & त्रिभुजे भुजयोर्योग\textemdash & \ldots & \ldots \\
		\hline
		२९, ७४, ९८, १०० & राश्योरन्तरवर्गेण\textemdash  & \ldots  & भास्करः \\
		\hline
		३२ & \ldots & महाभास्करीयभाष्यम्  & गोविन्दस्वामी  \\
		\hline
		३४  & इच्छां फलेन संहत्य\textemdash  & \ldots  & \ldots \\
		\hline
		३६  & द्विघ्ना कर्णकृतिर्भक्ता\textemdash & \ldots  & सूर्यदेवः \\
		\hline
		३८ & तैलिकचक्रस्य यथा\textemdash & \ldots  & \ldots  \\
		\hline
		४२ & व्यासे भनन्दाग्निहते\textemdash  & \ldots  & भास्करः \\
		\hline
		,, & विबुधनेत्रगजाहि\textemdash & \ldots  & सङ्ग्रमग्रामजो माधवः  \\
		\hline
		,, & कृतकानित्यवत् व्यास\textemdash & \ldots  & \ldots \\
		\hline
		४७, ७५, ७६ & राशिलिप्ताष्टमो भागः\textemdash & सूर्यसिद्धान्तः  & मयः \\
		\hline
		४८ & एकचापसमस्तज्या\textemdash  & \ldots  & \ldots \\
		\hline
		५३ & द्विघ्नान्त्यखण्डनिघ्नात्\textemdash & गोलसारः  & नीलकण्ठः \\
		\hline
		,, & भाजकाद्गुणकारेण\textemdash & \ldots \ & \ldots \\
		\hline
		५४ & यो यथा नियतो येन\textemdash & व्याप्तिनिर्णयः  & पार्थसारथिमिश्रः  \\
		\hline
		,, & शङ्कुच्छायां वा रवि  & \ldots  & ,, \\
		\hline
		५५ & इष्टदोःकोटिधनुषोः\textemdash  & \ldots  & माधवः  \\
		\hline
		,, & श्रेष्ठं नाम वरिष्ठानां\textemdash  & \ldots  & ,, \\
		\hline
		५८ & जीवे परस्परनिजेतर\textemdash & \ldots  & सङ्गमग्रामजो माधवः \\
		\hline
		६३, ११० & सत्र्यंशादिषुवर्गाज्ज्या\textemdash & गोलसारः  & नीलकण्ठः \\
		\hline
		६४ & इष्टयोराहतिर्द्विघ्नी\textemdash & \ldots  & भास्करः \\
		\hline
		६६, ७१ & \ldots  & \ldots & माधवभास्करौ  \\
		\hline
		६६ & कृतियोगस्तयोरेव\textemdash & \ldots  & \ldots \\
		\hline
		६९, ७०, ८५ & \ldots  & \ldots  & माधवः \\
		\hline
		८१ & अथ स्वांशाधिकोने तु\textemdash & \ldots  & \ldots \\
		\hline
		८९ & तिस्रो दिशो जगति\textemdash  & \ldots  & माधवः \\
		\hline
		,, & अस्त्यन्तोऽधोदिशः\textemdash  & \ldots & \ldots \\
		\hline
		९० & मध्ये समन्तादण्डस्य\textemdash & सूर्यसिद्धान्तः  & \ldots \\
		\hline
		९८, १०० & इष्टाद्बाहोर्यत् स्यात्\textemdash  & \ldots  & भास्करः \\
		\hline
		१०१ & जीवार्धवर्गे शरभक्त\textemdash  & \ldots  & ,, \\
		\hline
		१०२ & राशिजीवासमभ्यस्त\textemdash & सुन्दरी (लघुभास्करीयव्याख्या)  & \ldots \\
		\hline
		११० & अर्धज्यादिकमेवं\textemdash & \ldots  & \ldots \\
		\hline
		११२ & शिष्टचापघनषष्ठभाग\textemdash & तन्त्रसङ्ग्रहः  & नीलकण्ठः \\
		\hline
		११३ & विद्वांस्तुन्नबलः कपी\textemdash & \ldots  & माधवः \\
		\hline
		,, & स्पष्टता भवति चाल्प\textemdash & \ldots  & \ldots \\
		\hline
		११३, १५६ & \ldots   & \ldots  & कौषीतकिनारायणः \\
		\hline
		१५३  & सफलपदं कालगुणं\textemdash & \ldots  & \ldots \\
		\hline
		१५५ & इच्छावृद्धौ फलह्रासः\textemdash & \ldots  & \ldots \\
		\hline
		,, & प्रमाणेन फलं हत्वा\textemdash  & \ldots  & \ldots \\
		\hline
		१५६, १८०  & \ldots  & \ldots  & शङ्करः \\
		\hline
		१५६ & \ldots  & \ldots  & भास्करादयः \\
		\hline
		१५८ & गणितोन्नीतस्य चन्द्रादेः\textemdash  & अजिता~(वार्त्तिकव्याख्या) & \ldots \\
		\hline
		,, & \ldots  & विजयाख्यः & \ldots \\
		\hline
		१५९ & योगे ग्रहाणां ग्रहणे\textemdash & जातकम्  & \ldots  \\
		\hline
		,, & ग्रहणग्रहयोगादौ\textemdash  & जातकरणम्  & \ldots \\
		\hline
		,, & यदा यश्चैव सिद्धान्तो\textemdash & पराशरहोरा  & \ldots \\
		\hline
		१६०  &  ग्रहयोरन्तरे स्वल्पे\textemdash & मानसम्  & \ldots \\
		\hline
		१६१ & द्वौ वंशौ तुल्यमानौ यौ\textemdash & \ldots  & \ldots \\
		\hline
		१६४ & सङ्ख्यान्तराश्रयत्वा\textemdash & \ldots  & भाष्यकारः \\
		\hline
		१६५ & भूदिनेष्टगणान्योन्य\textemdash & \ldots  & भास्करः \\
		\hline
		,, & शतमष्टोत्तरं भानोः\textemdash & \ldots  & \ldots \\
		\hline
		१६६  & व्योमशून्यशराद्रीन्दु\textemdash & \ldots & \ldots \\
		\hline
		,, & राश्योरन्योन्यहरणे\textemdash & सिद्धान्तदीपिका   & \ldots \\
		\hline
		१६७  & परस्परं भाजितयो\textemdash & \ldots  & \ldots \\
		\hline
		१६८ & अहरात्मकमत्र स्यात्\textemdash  & \ldots  & \ldots  \\
		\hline
		,,  &  प्रक्षिप्य भागहारं\textemdash  & \ldots  & \ldots \\
		\hline
		१७१  & भाज्योऽधिको यदि भवेत्\textemdash & \ldots   & \ldots \\
		\hline
		,, & रूपापचयसिद्धोऽयं\textemdash & \ldots  &  गोविन्दस्वामी \\
		\hline
		१७२ & रूपमेकमपास्यापि\textemdash  & \ldots  & \ldots \\
		\hline
		१७३ & केनाहतोऽयमपनीय\textemdash  & \ldots  &\ldots \\
		\hline
		,, & गन्तव्यमिष्टं यदि\textemdash  & \ldots  &   \ldots \\
		\hline
		१७९ & \ldots  & \ldots  & भट्टभास्करः \\
		\hline
		१८० & \ldots  & \ldots  & जातवेदाः  \\
		\hline
		,, & \ldots  & दृग्गणितम्  & परमेश्वरः \\
		\hline
		,, & \ldots  & \ldots  & श्रीदामोदरः \\
		\hline
		,, & \ldots  & \ldots  & सुब्रह्मण्यः \\
		\hline
		,, & \ldots  & \ldots  &  माधवादयः \\
		\hline
		
	\end{longtable}
\end{center}
\end{document}