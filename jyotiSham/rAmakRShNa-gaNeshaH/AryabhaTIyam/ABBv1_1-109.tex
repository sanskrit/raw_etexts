\documentclass[11pt, openany]{book}
\usepackage[text={4.65in,7.45in}, centering, includefoot]{geometry}
\usepackage[table, x11names]{xcolor}
\usepackage{fontspec,realscripts}
\usepackage{polyglossia}
\setdefaultlanguage{sanskrit}
\setotherlanguage{english}
\setmainfont[Scale=1]{Shobhika}
\newfontfamily\s[Script=Devanagari, Scale=0.9]{Shobhika}
\newfontfamily\regular{Linux Libertine O}
%\newfontfamily\regular[Scale=1]{Times New Roman}
%\newfontfamily\sanskritfont[Script=Devanagari]{Shobhika}
%\newfontfamily\englishfont[Language=English, Script=Latin]{Linux Libertine O}
\newfontfamily\en[Language=English, Script=Latin]{Linux Libertine O}
\newfontfamily\ab[Script=Devanagari, Color=purple]{Shobhika-Bold}
\newfontfamily\qt[Script=Devanagari, Scale=1, Color=violet]{Shobhika-Regular}
\newcommand{\devanagarinumeral}[1]{%
	\devanagaridigits{\number \csname c@#1\endcsname}} % for devanagari page numbers
%\usepackage[Devanagari, Latin]{ucharclasses}
%\setTransitionTo{Devanagari}{\s}
%\setTransitionFrom{Devanagari}{\regular}
\usepackage{enumerate}
%\pagestyle{plain}
\usepackage{fancyhdr}
\pagestyle{fancy}
\renewcommand{\headrulewidth}{0pt}
\usepackage{afterpage}
\usepackage{multirow}
\usepackage{amsmath}
\usepackage{amssymb}
\usepackage{graphicx}
\usepackage{longtable}
\usepackage{footnote}
\usepackage{vwcol}
\usepackage{perpage}
\MakePerPage{footnote}
\usepackage{xspace}
\usepackage{array}
\usepackage{emptypage}
\usepackage{hyperref}% Package for hyperlinks
\hypersetup{colorlinks,
	citecolor=black,
	filecolor=black,
	linkcolor=blue,
	urlcolor=black}
\XeTeXgenerateactualtext=1 % for searchable pdf
\begin{document}
\addtolength{\oddsidemargin}{0.1in}
	\addtolength{\evensidemargin}{0.1in}
	\addtolength{\textwidth}{1.75in}

	\addtolength{\topmargin}{-.875in}
	\addtolength{\textheight}{1.75in}
	
\thispagestyle{empty}
\begin{center}
	\large{\en TRIVANDRUM SANSKRIT SERIES}
	\vspace{0.3cm}
	
	{\en No.~CL.}\\
	\vspace{0.3cm} {\en Śrī Setu Lakṣmī Prasādamālā.}
	\vspace{0.3cm}
	
	{\en No.~XXIII.}
	
	\vspace{1.5cm} 
	{\en THE}
	
	\vspace{0.3cm}
	{\LARGE{\en ĀRYABHAṬĪYA}}
	
	\vspace{0.3cm} 
	
	{\en OF} 
	
	\vspace{0.3cm}
	\textbf{\en ĀRYABHAṬĀCĀRYA}
	
	\vspace{0.3cm} 
	
	{\footnotesize{\en WITH THE \emph{BHĀṢYA} OF}
		
		\vspace{0.2cm} \en NĪLAKAṆṬHA SOMASUTVAN}
	
	\vspace{1.5cm}{\footnotesize\textbf{\en EDITED BY}}
	
	\vspace{0.3cm}
	\textbf{\en K.~SĀMBAŚIVA ŚĀSTRĪ,}\\
	\vspace{0.2cm}
	\emph{\en Curator of the Department for the Publication \\
			of Oriental Manuscripts, Trivandrum.}\\
	\rule{2cm}{0.3mm}\\
	\vspace{0.4cm}
	\rule{6cm}{0.3mm}\\
	{\textbf{\en Part~I \textendash\ \emph{\en Gaṇitapāda.}}}\\
	\rule{6cm}{0.4mm}
	\vspace{0.5cm}
	
	\small{\en PUBLISHED UNDER THE AUTHORITY OF THE GOVERNMENT OF\\
		HER HIGHNESS THE MAHARANI REGENT OF TRAVANCORE.}\\
	\rule{2cm}{0.3mm}
	\vspace{0.3cm}
	
	\small{\en TRIVANDRUM:\\
		PRINTED BY THE SUPERINTENDENT, GOVERNMENT PRESS,}
	
	1931.
\end{center}

\emph{\en All Rights Reserved.}]\


%\hspace{4.8cm}{}\textbf{सरस्वती - विहारः}


\begin{center}
	\vspace{2cm}{\Large{\underline{\textbf{अनन्तशयनसंस्कृतग्रन्थावलिः~।}}}}
	
	\vspace{0.4cm}\large\textbf{ग्रन्थाङ्कः~१०१.}
	
	\vspace{0.2cm}{\Large\textbf{श्रीसेतुलक्ष्मीप्रसादमाला }}
	
	\vspace{0.2cm}\textbf{ग्रन्थाङ्कः~१३.}
	
	\vspace{0.2cm}
	
	{\Large\textbf{श्रीमदार्यभटाचार्यविरचितम् }}
	
	\vspace{0.2cm}{\textbf{\huge\textbf{ आर्यभटीयं }}}
	
	\vspace{0.2cm}\textbf{गार्ग्यकेरलनीलकण्ठसोमसुत्वविरचित- }
	
	\textbf{ भाष्योपेतम्~।}
	
	\vspace{0.2cm}
	
	\textbf{ पौरस्त्यग्रन्थप्रकाशनकार्याध्यक्षेण \\ के. साम्बशिवशास्त्रिणा\\ संशोधितम्~।}
	
	\rule{2cm}{0.3mm}
	
	\vspace{0.1cm}
	
	\begin{center}
		\rule{6cm}{0.3mm}\\
		\vspace{0.2cm}
		\textbf{ प्रथमः सम्पुटः \textendash\ गणितपादः~। }\\
		\rule{6cm}{0.3mm}
	\end{center}
	
	\vspace{0.1cm}
	
	\textbf{ तच्च \\ अनन्तशयने \\ महामहिमश्रीसेतुलक्ष्मीमहाराज्ञीशासनेन \\ राजकीयमुद्रणयन्त्रालये तदध्यक्षेण \\ मुद्रयित्वा प्रकाशितम्~।}
	
	\rule{2cm}{0.3mm}
	
	\vspace{3cm}{\textbf{कोलम्बाब्दाः ११०५. क्रैस्ताब्दाः १९३०. }}
	\thispagestyle{empty} 
\end{center}
\newpage
\hfill\break
\thispagestyle{empty}
\begin{center}
	\Large\textbf{॥~श्री~॥}\\
	\textbf{श्रीपद्मनाभसेवि-\\न्यखिलश्रीवर्धनी महाराज्ञी~। \\ श्रीसेतुलक्ष्म्यभिख्या \\ प्रत्यक्षा जयति वञ्चिभूलक्ष्मीः~॥}
	
	\vspace{.5cm}
	
	\textbf{ ग्रन्थावलिरियमिन्धे \\ प्रसाधिता तत्प्रसादगुणगुम्फा~।\\ श्रीसहितसेतुलक्ष्मी-\\ प्रसादमाला सुवर्णमणिचित्रा~॥}
\end{center}

\newpage
\thispagestyle{empty}

\begin{center}
\textbf{PREFACE}
\end{center}

{\en This edition of the work is based on three palm leaf manuscripts respectively marked} \textbf{ क, ख} {\en and} \textbf{ग.} {\en The first two were obtained from the Palace Library, Trivandrum and the third from the library of the Raja of Kilimanur. The manuscript} \textbf{क} {\en went out of hand when its paper transcript was taken up for examination for the Press and it has not since been available for use, and hence no description of it is given here. The manuscript} \textbf{ख} {\en runs up to a portion of the 25th \emph{Sūtra} in the \emph{Golapāda}, while the manuscript} \textbf{ग} {\en contains some fragments of the \emph{Gaṇitapāda} and the whole of the \emph{Golapāda}. These two manuscripts which alone were available for collation are legibly written and appear to be about two centuries old. There
is a noticeable gap in both the manuscripts} \textbf{क} {\en and} \textbf{ख} {\en after the passage} \textbf{अत्रापीच्छाप्रमाणराशीपूर्वोक्तावेव} {\en on page 118 in the \emph{Bhāṣya} of the 17th \emph{Sūtra} of the \emph{Gaṇitapāda} and the same in} \textbf{ख} {\en was found filled up by} \textbf{र्श्वद्वयं व्यासार्धतुल्यं} {\en which is evidently a piece of the \emph{Bhāṣya} text having noconnection with the context. As this piece however, was found to fit in another gap found in the \emph{Bhāṣya} of the 18th
\emph{Sūtra} in the \emph{Gaṇitapāda}, it was accordingly placed there. There is also a third gap in both the manuscripts} \textbf{क} {\en and} \textbf{ख}
{\en after the passage} \textbf{तुल्यसङ्ख्यत्वादेवोक्तम्} {\en in the \emph{Bhāṣya} on the
18th \emph{Sūtra} in the \emph{Gaṇitapāda}. All my attempts to fill up this gap have hitherto been in vain. I secured a loan of the manuscript of Nīlakaṇtha \emph{Bhāṣya} from the Central Manuscript Library, Baroda, but to my surprise and disappointment, the same gap was found in that copy also as in our manuscript} \textbf{क.} {\en This circumstance kindled my curiosity to know how two manuscripts coming from such two distant countries contained one and the same gap, but I postpone my enquiry in the matter to a future occasion, when I shall have secured other manuscripts of the work for the Department. complete manuscript of the work has not yet been procured; however, in view of the rare merit of the \emph{Bhāṣya}, I have placed the first part before the learned public.\\

I take this opportunity of expressing my sincere appreciation of the commendable patience and enthusiasm shown by the Pandits of the department in preparing this difficult and erudite work for the Press, notwithstanding the fact that the manuscript materials at their disposal were far from satisfactory.}

\vspace{2cm}
\hspace{6.5cm} \textbf{\en K.~SĀMBAŚIVA ŚĀSTRĪ}


\newpage
\thispagestyle{empty}
\begin{center}
\textbf{INTRODUCTION}
\end{center}

{\en With the publication of the first part containing the \emph{Gaṇitapāda} of \emph{\en Āryabhaṭīya}, the Trivandrum Sanskrit Series is numerically entering on the second centesimal cycle. The \emph{Jyotiśśāstra} is composed of three branches, viz., \emph{Gaṇita, Saṃhitā} and \emph{Horā}, and the \emph{Āryabhaṭīya} deals with the \emph{Gaṇita} branch. The work consists of four parts or \emph{Pādas}, namely \emph{Gītikāpāda, Gaṇitapāda, Kālakriyāpāda} and \emph{Golapāda}. There is another division of the work, the first consisting of \emph{Gītikāpāda} of 13 \emph{Ārya} verses and the second, comprised of the three other \emph{Pādas} containing 108 \emph{Ārya} verses; and thus the work contains on the whole 121 \emph{Ārya} verses.

The work had been a mine of knowledge for Lalla, Muñjāla and Bhāskarācārya who respectively wrote \emph{Śiṣyadhīvrddhida}, \emph{Mānasakaraṇa} and \emph{Siddhāntaśiromaṇi} after observations of the planetary movements. After studying the \emph{Brahmasiddhānta} and other work of his predecessors and observing the planetary movements of his time, Āryabhaṭa wrote his work for the benefit of the succeeding generations of students.}

\begin{minipage}[t]{0.15\textwidth}
\vspace{3cm}
\textbf{{\en Āryabhaṭa}}
\end{minipage} 
%\hfill
\begin{minipage}[t]{0.55\textwidth} 
\vspace{-3mm}
{\en In explaining the line} \textbf{कुसुमपुरेऽभ्यर्चितं ज्ञानम्} {\en in the introductory verse of the \emph{Gaṇitapāda}, the commentator Bhāskara observes} \textbf{अयं किल स्वायंभुवसिद्धान्तः कुसुमपुरनिवासिभिः कृतिभिः पूजितः सत्स्वपि पौलिशरोमकवासिष्ठसौर्येषु, तेनाह कुसुमपुरेऽभ्यर्चितं ज्ञानमिति} {\en that is Āryabhaṭa followed the Svayambhuva Siddāntha as it had been highly respected by the learned people of Kusumapura. At the outset of his \emph{Bhāṣya} Nīlakaṇṭha says} \textbf{अश्मकजनपदजात आर्यभटाचार्यः} {\en that is Āryabhaṭa was a native of the country called Aśmaka. It is said that this country was a part of Southern India, while some take it to be the same as the ancient Travancore; see for example Apte's Sanskrit Dictionary. The work of Āryabhaṭa has long been popular in Kerala more than in any other country and the commentators of the work are all of them known to be Keraliyas. It is therefore possible that Āryabhaṭa was a native of Kerala and migrated to Kusumapura, the imperial capital of the Guptas, for patronage; and it is a matter for congratulation for he was a native of Travancore, a part of the Kerala country.}
\end{minipage} 

\newpage
\thispagestyle{empty}

\begin{center}{2}\end{center}

{\en From the \emph{Ārya} verse,}

\begin{quote}
\textbf{षष्टयब्दानां षष्टिर्यदा व्यतीतास्त्रयश्च युगपादाः~।\\
त्र्यधिका विंशतिरब्दास्तदेह मम जन्मनोऽतीताः~॥}
\end{quote}
\vspace{-3mm}
\hspace{7.5cm}(\emph{\en Kālakriyāpāda Śloka}. 10)\\

\noindent {\en we learn that Āryabhaṭa flourished in the latter half of the 5th century A.D., and that this work which was written by him when he was only 23 years old, was received by the public very favourably though he was so young.}\\

\begin{minipage}[t]{0.15\textwidth}
	\vspace{2cm}
\textbf{{\en Keraliyas and their development of~\emph{Jyotiṣa}}}
\end{minipage} 
%\hfill
\begin{minipage}[t]{0.55\textwidth} 
\vspace{-4mm}
{\en Since the time of Bhāskarācārya, Northern India is not known to have produced as many original writers on \emph{Jyotiṣśāstra} as the Kerala country in Southern India. Verily, the stars that shone on the sky of Kerala might have felt highly propitiated in as much as the intellect of the Keraliyas found in astronomy a fitting field to work upon and produced \emph{Bṛhadbhāskarīya, Dṛggaṇita, Tantrasaṅgraha, Siddhāntadarpaṇa} and many other valuable works. It is a pity that these precious works are still slumbering in the archives of this country, though their publications are highly to be wished for.}
\end{minipage} \\

%\vspace{2cm}
\begin{minipage}[t]{0.15\textwidth}
\vspace{2cm}
\textbf{\emph{\en Āryabhaṭīya Bhāṣya}}
\end{minipage} 
%\hfill
\begin{minipage}[t]{0.55\textwidth} 
%\vspace{-3mm}
{\en Among the valuable works on Astronomy, one is the \emph{Bhāṣya} on \emph{Āryabhaṭīya} by Nīlakaṇṭha. It is worthy of note that no one who is not a Keralīya has hitherto ventured to write either a \emph{Bhāṣya} or a \emph{Vyākhyā} on the \emph{Āryabhaṭīya}. The \emph{Vyākhyā} named \emph{Bhaṭadīpikā} on the work, printed and published in Holland was written by no other than Parameśvarācārya of Kerala, who, for the first time, propounded the \emph{Dṛggaṇitatantra} after 55 years of study. This fact is evident from the following verses found in the introduction of his commentaries on the \emph{Āryabhaṭīya} and \emph{Līlāvatī}},
\end{minipage}
\begin{quote}
\textbf{ लीलावती भास्करीयं लघु चान्यच्च मानसम्~।\\
व्याख्यातं शिष्यबोधार्थं येन प्राक् तेन चाधुना~॥}

\textbf{ तन्त्रस्यार्यभटीयस्य व्याख्याल्पा क्रियते मया~।\\
परमादीश्वराख्येन नाम्नात्र भटदीपिका~॥}

\textbf{नीलायाः सागरस्यापि तीरस्थः परमेश्वरः~।\\
व्याख्यानमस्मै बालाय लीलावत्याः करोम्यहम्~॥}
\end{quote}

\noindent {\en which state clearly that Parameśvarācārya lived on the bank of the River \emph{Nīla} near the sea shore in North Kerala.}

\newpage

\begin{center} 3 \end{center}
%\vspace{4mm}
\thispagestyle{empty} 
{\en The \emph{Bhāṣya} of Nīlakaṇṭha is called \emph{Mahābhāṣya} as is seen from the words of the author himself,}

\begin{quote} 
\textbf{श्रीमदार्यभटाचार्यविरचितसिद्धान्तव्याख्याने महाभाष्ये उत्तरभागे
युक्तिप्रतिपादनपरे त्यक्तान्यथाप्रतिपत्तौ निरस्तदुर्व्याख्याप्रपञ्चे
समुद्घाटितगूढार्थे सकलजनपदजातमनुजहिते निदर्शितगीतिपादार्थे सर्वज्योतिषामयनरहस्यार्थनिदर्शके
समुदाहृतमाधवादिगणितज्ञाचार्यकृतयुक्तिसमुदाये
निरस्ताखिलविप्रतिपत्तिप्रपञ्चसमुपजनितसर्वज्योतिषामयनविदमलहृदयसरसिजविकासे निर्मले गम्भीरे अन्यूनानतिरिक्ते गणितपादगतार्यात्रयस्त्रिंशदव्याख्यानं समाप्तम्~।}~ (p.~180). 
\end{quote}

\noindent {\en As it closely follows the methods adopted by Pataṅjali in his \emph{Vyākaraṇamahābhāṣya}, I think this \emph{Bhāṣya} fully deserves the name \emph{Mahābhāṣya} while the epithets} \textbf{युक्तिप्रतिपादनपरे, त्यक्तान्यथाप्रतिपत्तौ} {\en \& c. used as referring to \emph{Mahābhāṣya} in the above quotation also go to justify the title. It is a matter for satisfaction that we have been enabled to publish a work on astronomy which goes a long way to remove the charge levelled against \emph{Jyotiṣa} that there is no expository \emph{Bhāṣya} in it as in other \emph{Śāstras}. From the following observations in the \emph{Bhāṣya}},

\begin{quote} 
\textbf{यन्मयात्र केषाञ्चित् सूत्राणां तद्युक्तीः प्रतिपाद्य कौषीतकिनाढ्येन नारायणाख्येन व्याख्यानं कारितं अतस्तदेवात्र लिख्यते~॥} 
~(p.~113) \\

\textbf{इतीदं प्रथमे वयस्येव वर्तमानेन मया द्वितीयवयसि स्थितेन कौषीतकिनाढ्येन कारितम्~। अत्र केषाञ्चिद् युक्तयः पुनरस्मदनुजेन शङ्कराख्येन तत्समीपेऽध्यापयता वर्तमानेन तस्मै प्रतिपादिताः~। तस्याढ्यत्वात् स्वातन्त्र्याच्च तत्र व्यापारश्च निर्वृत्तः~। तस्मिन् स्वर्गते पुनरत एव मयाद्य प्रवयसा ज्ञाता युक्तीः प्रतिपादयितुं भास्करादिभिरन्यथाव्याख्यातानां कर्माण्यपि प्रतिपादयितुं यथा कथञ्चिदेव
व्याख्यानमारब्धम्~।}\hspace{5 cm} (p.~156)
\end{quote} 
{\en it appears that the author is adding to his text the portion of another \emph{Bhāṣya} that was caused to be written by Kauṣītaki Nārāyaṇa, but this additional portion of the text runs only on the \emph{Sūtras} from 15th to 17th but not on those from 18th to 26th, though the author tells us at the close of the 20th \emph{Sūtra} that he has transcribed the \emph{Bhāṣya} of Kauṣītaki up to that extent. The author had not commented on the \emph{Gītikāpāda} evidently with the idea that the meaning of it would be clear to anyone who studied his \emph{Bhāṣya} on the three other \emph{Pādas} as he himself explicitly says,}

\begin{quote} 
\textbf{तत्रेयं त्रिपाद्यस्माभिर्व्याचिख्यासिता, यतस्तद्व्याख्येयरूपत्वात् गीतिकापादस्यैतद्व्याख्यानेनैवार्थः प्रकाशेत}
\end{quote} 

\newpage

\begin{center} 4 \end{center}

\thispagestyle{empty} 
\indent {\en This \emph{Bhāṣya} which from its rare merits surpasses those of Bhāskarācārya, Sūryadeva, Ghaṭīgopa and others will, I think, be found useful not only by the students of \emph{Jyotiśśāstra}, but also by others interested in Sanskrit literature.

The closing colophon of the \emph{Bhāṣya}},
\begin{quote} 
\textbf{ इति श्रीकुण्डग्रामजेन गार्ग्यगोत्रेणाश्वलायनेन \ldots \ldots \ldots \ldots \ldots \ldots \\
\ldots \ldots गणितपादगतार्यात्रयस्रिंशद्व्याख्यानं समाप्तम्~।}\hspace{1 cm} (p.~180)
\end{quote}

\begin{minipage}[t]{0.15\textwidth}
\vspace{.8cm}
\textbf{{\en Nīlakaṇṭha Somasutvan}}
\end{minipage} 
%\hfill
\begingroup
\renewcommand{\thefootnote}{\fnsymbol{footnote}}
\begin{minipage}[t]{0.55\textwidth} 
{\en reveals that Nīlakaṇṭha was a disciple of Dāmodara, son of Parameśvarācārya, that he was a native of Kuṇḍa Grāma\makebox[0pt][l]{\footnotemark} and belonged to \emph{Garga Gotra} and \emph{Āśvalāyana Śākhā} and that both his father and maternal uncles were named Jātaveda and his brother Śaṅkara.}
\end{minipage} 	
\footnotetext{Kuṇḍagrāma is Trikanḍiyūr in British Malabar.}\addtocounter{footnote}{-1}
\endgroup

\vspace{.5cm} \noindent {\en As the commentary on \emph{Tantrasaṅgraha} explains the first line in the introductory verse,}

\begin{quote} 
\textbf{हे विष्णो! निहितं कृत्स्नं जगत् त्वय्येव कारणे~।\\
ज्योतिषां ज्योतिषे तस्मै नमो नारायणाय ते~॥}
\end{quote}

\noindent {\en and the third line of the concluding verse},

\begin{quote} 
\textbf{गोलः कालक्रिया चापि द्योत्यतेऽत्र मया स्फुटम्~।\\
लक्ष्मीशनिहितध्यानैरिष्टं सर्वं हि लभ्यते~॥}
\end{quote} 

\noindent {\en as also representing two chronograms of the \emph{Kali} days on which respectively the work was begun and finished, the date of Nīlakaṇṭha may be fixed between 1450 A.~D. 1550 A.~D. From the line}

\begin{quote} 
\textbf{एवं दृग्गणितं शाके त्रीषुविश्वमिते कृतम्~।}
\end{quote}

\noindent {\en in the \emph{Dṛggaṇita} of Parameśvarācārya, we learn that it was written in \emph{Śaka} 1353 corresponding to 1430 A.~D. It is quite possible therefore that Nīlakaṇṭha, a disciple of Paramesvarācārya wrote his \emph{Tantrasaṅgraha} some 70 years after the latter wrote his \emph{Dṛggaṇita}. Nīlakaṇṭha has also written \emph{Golasāra}, \emph{Tantrasaṅgraha, Siddhāntadarpaṇa} and many other original works; and, among these \emph{Golasāra} and \emph{Tantrasaṅgraha} must have been written earlier than the present \emph{Bhāṣya} which refers to them by name thus,}

\begin{quote} 
\textbf{एतत्सर्वमस्माभिर्गोलसारे प्रदर्शितम्~।\\
द्विघ्नान्त्यखण्डनिघ्नात् तत्तज्ज्यार्धात् त्रिभज्याप्तम्~।\\
अन्त्यादिखण्डयुक्तं त्याज्यं स्यात् पूर्वपूर्वगुणसिद्ध्ये~॥} \hspace{1.5cm}(p.~53)
\end{quote}

\newpage

\begin{center}{ 5}\end{center}
\thispagestyle{empty} 
{\s अत एवोक्तं मया तन्त्रसङ्ग्रहे}\textendash 
\begin{quote}
\textbf{शिष्टचापघनषष्ठभागतो विस्तरार्धकृतिभक्तवर्जितम्~।\\
शिष्टचापमिह शिञ्जिनी भवेत् स्पष्टता भवति चाल्पतावशात्~॥} 
\end{quote}
\vspace{-2mm}
\hspace{9cm} (p.~112)\\

\noindent {\en That Nīlakaṇṭha had a very respectable place among the great astronomers of Kerala can be learned from the fact that the author of \emph{Sphuṭanirṇaya} mentions him along with the venerable persons whom he pays homage to in the introductory verse of the work,}

\begin{quote}
\textbf{ब्रह्माणं मिहिरं वसिष्ठपुलिशौ गर्गं मयं लोमशं\\
श्रीपत्यार्यभटौ वराहमिहिरं लल्लं च मुञ्जालकम्~।\\
गोविन्दं परमेश्वरं सतनयं श्रीनीलकण्ठं गुरून्\\
वन्दे गोलविदश्च माधवमुखान् वाल्मीकिमुख्यान् कवीन्~॥}
\end{quote}

{\en The following passage in the \emph{Bhāṣya} shows that Nīlakaṇṭha had a brother named Śaṅkara who was equally well versed in the \emph{Jyotiśsāstra}}\textendash \\

\begin{minipage}[t]{0.20\textwidth}
\vspace{.1cm}
\textbf{{\en Śaṅkara brother of Nīlakaṇṭha}}
\end{minipage} 	
\begin{minipage}[t]{0.45\textwidth}
%\vspace{1cm}
 \textbf{अत्र केषाञ्चिद्युक्तयः पुनरस्मदनुजेन
शङ्कराख्येन तत्समीपेऽध्यापयता वर्तमानेन तस्मै प्रतिपादिताः}~~~\hspace{5cm} (P.~156) \\
\end{minipage}  \\

\noindent {\en Mention may be made here of the sidelight thrown by the works of Nīlakaṇṭha on certain obscure points in regard to Thunchattu Ezhuttacchan, justly regarded as the father of modern Malayalam literature. In my consultations with Mr. Justice P.~K.~Narayana Pillai B.A., B.L., of the Travancore High Court, who is a member of my Advisory Board, he brought to my notice, the reference to \textbf{Śrī Nīlakaṇṭha Guru}, in the \emph{Harināma Kīrtanam}, one of Ezhuttacchan's works. This leads to the information that our Nīlakaṇṭha was the Guru or teacher of Ezhuttacchan. This is strengthened by the allusion to Netranārāyaṇa in Malayalam \emph{Brahmāṇḍapurāṇam} which is another work by Ezhuttacchan or by a disciple of his, as some would have it. Netranārāyaṇa, as is well known, is the titular appellation of the Azhuvāñceri Tamprākkal. The Kauṣītaki referred to by Nīlakaṇṭha in the passages quoted above is none other than the Azhuvāñceri Tamprākkal, one of the foremost, if not the foremost dignitary in the West Coast Hierarchy. This is evident from the following reference,}


\begin{quote}
 \textbf{इति कौषीतकी श्रुत्वा नेत्रनारायणः प्रभुः~।\\
मह्यं न्यवेदयत् तस्मै तदैवं प्रत्यपादयम्~॥}
\end{quote}
\noindent {\en made by Nīlakaṇṭha himself. In the light of this information, the necessary landmarks to fix Ezhuttacchan's date, a point involving some controversy among Malayalam scholars}
 	
\newpage

\begin{center}
    6
\end{center}
\thispagestyle{empty} 
\noindent {\en become easily available, since Nīlakaṇṭha's date is easily gatherable from the chronograms quoted from the \emph{Tantrasaṅgraha}. This aspect of the subject has been discussed by Mr. Justice Narayana Pillai, in his lectures on Ezhuttacchan delivered recently at the instance of the University of Madras. As he observed during the course of those lectures, Nīlakaṇṭha reveals himself as a solid mass of erudition, submerged some how or other under the current of time.\\

Dāmodara, Guru of Nīlakaṇṭha and son of Parameśvarācārya has written a work named \emph{Muhūrtābharaṇa} as is clear from the following introductory verse of the work,}

\begin{quote}
\textbf{आचायार्यभटीयसूत्रितमहागूढोक्तिमुक्तावली-\\
मालालङ्कृतयो जयन्ति विमला वाचो यदीयाः शुभाः~।\\
सूक्ष्मा यत्प्रतिभा च गूढगणितं निश्शेषकालक्रियां\\
भूगोलं ग्रहवास्तवञ्च तदिदं विश्वं स्फुटं पश्यति~॥\\
तस्यात्मजः शिष्यवरः प्रसादमाश्रित्य दामोदरनामधेयः~।\\
मुहूर्तशास्त्राभरणं गुणाढ्यं विचित्रवृत्तं रुचिरं चकार~॥}
\end{quote}

{\en Nīlakaṇṭha was well-versed not only in \emph{Jyotiśśāstra}, but also in other branches of knowledge such as \emph{Mīmāṃsā, Nyāya, Vyākaraṇa} and \emph{Vedānta} and in support of this statement may be cited the following passages, among others from his \emph{Bhāṣya}},\\

{\s अत एवोक्तं पार्थसारथिमिश्रेण व्याप्तिनिर्णये}\textendash 
\begin{quote}
\textbf{यो यथा नियतो येन यादृशेन यथाविधः~।\\
स तथा तादृशस्यैव तादृशोऽन्यत्र बोधकः~॥}
\end{quote}

{\en In the \emph{Gaṇitapāda Bhāṣya}, the author cites as authority Vṛddhagarga, Varāhamihira, Piṅgala and other ancient teachers; Bhāskara, Govindasvāmi, Sūryadeva, Mādhava and other later authors; \emph{Vaijayantī Gargasaṃhitā, Sūryasiddhānta} and other works; thus suggesting that his \emph{Bhāṣya}
was written on the line of ancient authoritative commentaries.\\

His ripe scholarship on \emph{Jyotiśśāstra} seasoned as it is with his close acquaintance with the works of various ancient teachers by adding to the importance of the \emph{Āryabhaṭīya}, has brought lasting credit to the Kerala people. We are gratified that our desire to enrich the series by publishing more \emph{Jyotiṣa} works as expressed in the introduction of the \emph{Horāśāstra} (No.~91 of the Trivandrum Sanskrit Series), has now been realised to this extent.}\\

\noindent{{\en Trivandrum,}
	
	\noindent$\emph{11-11-1105.}$}
\hspace{5cm} \textbf{\en K.~SĀMBAŚIVA ŚĀSTRĪ.}

\newpage

\begin{center}
\textbf{उपोद्घातः~।}
\end{center}
\thispagestyle{empty} 
\indent अस्यार्यभटीयभाष्यग्रन्थस्य प्रकाशने परिशोधनोपयुक्ताः \textbf{क}-\textbf{ख}-\textbf{ग}-संज्ञितास्त्रय आदर्शाः~। तेषु आद्यौ द्वौ महाराजग्रन्थशालसम्बन्धिनौ~। तृतीयः किलिमानूर्राजस्वामिकश्च~। तत्र \textbf{क}-संज्ञितस्य प्रतिरूपणाधारस्य परिशोधनावसरेऽनुपलब्धेस्तदीयं याथार्थ्यमशक्यवचनं जातम्~। द्वितीयश्च \textbf{ख}-संज्ञितो गणित-कालक्रिया-पादाभ्यां गोलपादे पञ्चविंशसूत्रैकदेशभाष्यभागेन च सम्पुटितः~। तृतीयस्त्वादिमे कियताचन गणितपादभागेनानन्तरं समग्रेण गोलपादेन च सङ्घटितः~। परिशोधनोपयुक्तावुभावपि \textbf{ख}-\textbf{ग}-संज्ञौ द्विशतवर्षज्येष्ठौ
सुष्ठु लिखितौ~। अस्मिन् मुद्रितपुस्तके ११८ तमपुटे नक्षत्रचिह्नोत्तरं (\textbf{अत्रापीच्छा प्रमाणराशी पूर्वोक्तावेव} इत्येतदुत्तरं) \textbf{क}-\textbf{ख}-मातृकयोः कियांश्चिदंशो लुप्तः प्रतिभातः~। \textbf{ख}-मातृकायां लुप्तस्थाने मुद्रितैतत्पुस्तकीय १३१ तमपुटे दृष्टानि नक्षत्रचिह्नोत्तराणि \textbf{र्श्वद्वयं व्यासार्धतुल्यम्} इत्यादीनि कतिचन वाक्यानि प्रक्षिप्तानि लक्ष्यन्ते~। परन्तु सोऽयं भागोऽस्मन्मुद्रितपुस्तकरीत्या \textbf{क}-मातृका गत्या च १३१ तमपुट एव स्थानमर्हतीति तथैव कृतः~। उभयोरपि \textbf{क}-\textbf{ख}-मातृकयोः सममेव १३२ तमपुटे नक्षत्रचिह्नोत्तरं (\textbf{तुल्यसङ्ख्यत्वादेवोक्तमि}त्यस्यानन्तरम्) अष्टादशसूत्रस्यान्तिम एकोनविंशसूत्रस्यादिमश्च भाष्यांशो लुप्तः~। लुप्तस्यास्य भागस्य परिपूरणाय बहुव्यवस्यतापि मया न फलमुपलब्धम्~। बरोडादेशीयप्राचीनग्रन्थप्रकाशनकार्यालयादपि मातृकामेकामेतदर्थे समपादयम्~। सापि दैवात् तत्रैव खिला दृष्टा, यत्रास्माकं \textbf{क}-मातृका लुप्तलिप्ती विकला~। कथमतिदविष्ठविदेशस्थितयोरनयोरेतादृशी समावस्था समगतेति प्राप्तावसरेऽपि कौतुककारिणि विचारे तादृशानां मातृकान्तराणां सम्पादनसमनन्तराय सन्दर्भाय सद्यो विरतोऽस्मि~। कदानु कुत्रवेमं परिपूर्णं भागं सम्पाद्यासमित्यधुनाप्यविरतप्रोत्साहनो व्याप्रिये~। अनुपलब्धेऽपि समग्रे मातृकान्तरे श्लाघनीयमदसीयमर्थनिरूपणप्रपञ्चनजातमभिज्ञजनसमक्षमचिरादेवाविर्भावयितुमहमकृतविलम्ब एवामुं भागमधुना प्राकाशयम्~।

\newpage

\begin{center} 
	२ 
\end{center}
\thispagestyle{empty} 
\indent सुपरिशुद्धमातृकान्तरवैकल्येऽपि क्षमया श्रमसहा अस्मत्पण्डिताः प्रकृत्या गहनमिदमन्यदुष्प्रवेशं भाष्यं परिशोध्य मुद्रणानुगुणं कृतवन्त इति निकाममभिनन्दनमर्हन्ति~॥\\

\noindent \emph{अनन्तशयनम्,}\\
\noindent \emph{११-११-११०५.}\hspace{5.9cm}  \textbf{के.~साम्बशिवशास्त्री.}

\newpage

\begin{center}
\textbf{अवतारिका~।}
\end{center}
\thispagestyle{empty} 
%\vspace{1cm}
\begin{sloppypar} 
प्रथमोऽयं सम्पुटो गणितः सभाष्यस्या\textbf{र्यभटीयस्यास्मदनन्तशयनसंस्कृतग्रन्थावलि}द्वितीयशतकस्य च~। गणित-संहिता-होराख्यैस्त्रिभिः
स्कन्धैरुपचिताकृतेः किल ज्योतिस्तन्त्रमहातरोरादिमं गणितस्कन्धमधिरुह्य लब्धसञ्चारं सर्वतः सुमनोभिराममार्यभटीयम्~। सन्ति च तत्रास्मिन् गीतिका-गणित-कालक्रिया-गोलाख्याश्चत्वारः पादाः~। एषु गीतिकया पूर्वस्त्रिभिरपरैरुत्तर इति च द्वौ पूर्वोत्तरौ प्रबन्धौ तत्र स्तः~। अनयोः पूर्वे
त्रयोदशोत्तरेऽष्टोत्तरं शतमिति सन्त्याहत्यैकविंशत्युत्तरं शतमार्यारब्धानि सूत्राणि,
\end{sloppypar} 
\begin{minipage}[t]{0.15\textwidth}
\vspace{.01cm}
\textbf{आर्यभटीयम्}
\end{minipage} 
%\hfill
\begin{minipage}[t]{0.55\textwidth} 
यैः परिपूर्णं प्रस्तुत\textbf{मार्यभटीयम्}~। श्रीमतां \textbf{लल्ल-मुञ्जालक-भास्कराचार्या}दीनां~~~~तदा~~~तदा व्यतियतीर्ग्रहगतीरनुसृत्य 
\end{minipage} 
\begin{sloppypar} 
\noindent \textbf{शिष्यधीवृद्धिद-मानस-सिद्धान्तशिरोमण्या}दीनां ग्रन्थरत्नानां विरचने तदिद\textbf{मार्यभटीयं} कापि सारखनिरेवाभवत्~। परिशील्य \textbf{ब्रह्मसिद्धान्ता}दीनि पूर्वशास्त्राणि, परिशोध्य ग्रहगतीः, सङ्गृह्य सारं, समनन्तरेभ्यः शिष्येभ्यः सूत्रात्मना सम्पिण्ड्य समर्पितमिदं तन्त्रमन्यादृशमेव कमप्यसाधारणं महिमानमात्मनः पुष्णातीति निश्चप्रचोऽयमर्थो मन्ये न पुनः पल्लवनमर्हति\\
\end{sloppypar} 
\noindent श्रीमान् \textbf{आर्यभटाचार्य}श्च\textendash 

\begin{quote}
\textbf{ब्रह्मकुशशिबुधभृगुरविकुजगुरुकोणभगणान् नमस्कृत्य~।\\
आर्यभटस्त्विह निगदति कुसुमपुरेऽभ्यर्चितं ज्ञानम्~॥} 
\end{quote}
\vspace{-2mm}
\hspace{8cm} (गणितपादः सू. १)

\begin{quote}
\textbf{षष्ट्यब्दानां षष्टिर्यदा व्यतीतास्त्रयश्च युगपादाः~।\\
त्र्यधिका विंशतिरब्दास्तदेह मम जन्मनोऽतीताः~॥} 
\end{quote}
\vspace{-2mm}
\hspace{7.5cm} (कालक्रियापादः सू. १०) \\

\begin{minipage}[t]{0.10\textwidth}
\vspace{.3cm}
\textbf{आर्यभटः}
\end{minipage} 
%\hfill
\begin{minipage}[t]{0.55\textwidth} 
\noindent इत्याभ्यामार्याभ्यां \textbf{पाटलीपुत्रान्तर्गतकुसुमपुराभिजनः क्रिस्त्वब्दीयपञ्चमशतकोत्तरार्धजीवी}ति स्पष्टमवगम्यते~। तत्रैव दृश्यमानौ\textendash 
\end{minipage}
\newpage

\begin{center}
	२ 
\end{center}
\thispagestyle{empty} 
\begin{quote}
\textbf{त्र्यधिका विंशतिरब्दास्तदेह मम जन्मनोऽतीताः~।\\
कुसुमपुरेऽभ्यर्चितं ज्ञानम्~।}
\end{quote}

\noindent इति भागावाचार्यस्य त्रयोविंशे वयसि ग्रन्थनिर्मितिं तत्समकालमेव ग्रन्थबहुमतिं च स्पष्टमाचक्षाते~। अस्मद्भाष्ये\textendash 

\begin{quote} 
\textbf{अश्मकजनपदजात आर्यभटाचार्यः}~~~ (पुटं.~१)
\end{quote} 
\noindent इति दर्शनात् कोऽ\textbf{प्यश्मकाभिधानो} देश आचार्यजन्मभूमिरिति सिद्ध्यति~। स चायं दक्षिणभारतान्तर्गतः कश्चिदन्य एव वा स्यात्, आहोस्वित् जनपदपदस्वारस्यात् तस्यैव वा \textbf{कुसुमपुरस्य} व्यापकः सामान्यदेशो वा भवेत्~। \\

\begin{minipage}[t]{0.15\textwidth}
\vspace{2cm}
\textbf{ केरला ज्योतिस्तन्त्रप्रचारश्च}
\end{minipage} 
%\hfill
\begin{minipage}[t]{0.55\textwidth} 
सिद्धान्तशिरोमणिकर्तुः श्रीमतो \textbf{भास्कराचार्यात्} परतो न तावन्त औत्तराहा ज्योतिस्तन्त्रेऽस्मिन् स्वतन्त्रान् ग्रन्थान् प्रणयन्त
उपलब्धाः, यावन्तो दाक्षिणात्याः केरलीयाः~। केरलान् खलु प्रकृतिसुभगान् देशान् प्रभयोपरि परिस्तृणन्ति ज्योतिर्मण्डलानि
मन्ये तावत् प्रसन्नानि यावता केरलीयानामहमहमिकया लब्धप्रकाशा बुद्धिरस्मिंस्तन्त्रे क्रमशः क्रममाणा नैकानि प्रौढिमन्ति
\textbf{बृहद्भास्करीयदृग्गणित-तन्त्रसङ्ग्रह-सिद्धान्तदर्पणादीनि} ग्रन्थरत्नानि प्रकाशयितुमुद्यमवती जाता~। अद्ययावदिमानि च रत्नान्यलब्धसूर्यालोकानि तेषु तेषु जरत्तमेषु ग्रन्थशालेष्वेव कुहचन शेरत इति हन्त भोः! शान्तं पापमेषामुपरिष्टादचिरेण प्रकाशाय कल्पताम्~। 
\end{minipage} \vspace{2mm}

जाग्रत्स्वेतादृशेषु ग्रन्थरत्नेष्वन्यतमं किमप्यनर्घं रत्नमिदं \textbf{नीलकण्ठीयभाष्यं} नाम~। \textbf{आर्यभटीयस्य} भाष्यं व्याख्या वा किमपीयता कालेनाकेरलीयप्रणीतं नोपलब्धम्~। \textbf{मुम्बापुर्यां हाळन्ड्देशे} च मुद्रापितप्रसिद्धीकृता च सा \textbf{भटदीपिका} व्याख्या पञ्चपञ्चाशतः परिवत्सरान् कृतव्यवसितेरिदमुपक्रमं \textbf{दृग्गणितं} प्रचारितवतः केरलीयस्य \textbf{श्रीपरमेश्वराचार्य}स्यैव~। इदं चार्यभटीयव्याख्याप्रारम्भे, 

\begin{quote}
\textbf{लीलावती भास्करीयं लघु चान्यच्च मानसम्~।\\
व्याख्यातं शिष्यबोधार्थं येन प्राक् तेन चाधुना~॥\\
तन्त्रस्यार्यभटीयस्य व्याख्याल्पा क्रियते मया~।\\
परमादीश्वराख्येन नाम्नात्र भटदीपिका~॥}
\end{quote}

\noindent इति दृष्टस्यानुगुणं लीलावतीव्याख्यानोपक्रमे\textendash 

\newpage

\begin{center} 
	३ 
\end{center}

\thispagestyle{empty} 
\begin{quote}
\textbf{नीलायाः सागरस्यापि तीरस्थः परमेश्वरः~।\\
व्याख्यानमस्मै बालाय लीलावत्याः करोम्यहम्~॥}
\end{quote}

\noindent इति लीलावतीव्याख्याकर्तुः \textbf{परमेश्वराचार्यस्य} नीलासागरतीराभिजनत्वप्रतिपादनात् स्पष्टमवगम्यते~। यतो हि तीरं नीलासागरयोरुत्तरकेरलान्तर्गतम्~।\\ 

अस्य च \textbf{नीलकण्ठीय}भाष्यस्य महाभाष्यमिति संज्ञा~। सा च
\begin{quote} 
\textbf{श्रीमदार्यभटाचार्यविरचितसिद्धान्तव्याख्याने महाभाष्ये उत्तरभागे युक्तिप्रतिपादनपरे त्यक्तान्यथाप्रतिपत्तौ निरस्तदुर्व्याख्याप्रपञ्चे
समुद्घाटितगूढार्थे सकलजनपदजातमनुजहिते निदर्शितगीतिपादार्थे सर्वज्योतिषामयनरहस्यार्थनिदर्शके समुदाहृतमाधवादिगणितज्ञाचार्यकृतयुक्तिसमुदाये निरस्ताखिलविप्रतिपत्तिप्रपञ्चसमुपजनितसर्वज्योतिषामयनविदमलहृदयसरसिजविकासे निर्मले गम्भीरे अन्यूनानतिरिक्ते गणितपादगतार्यात्रयस्त्रिंशद्व्याख्यानं समाप्तम्}~~~(पुटं.~१८०)
\end{quote} 
\noindent इति स्वयमेव कथनात्, \textbf{पातञ्जलमहाभाष्ये} समुट्टङ्कितां पद्धतिमनुसृत्यार्थप्रपञ्चनयुक्तिनिरूपणचर्चावितानवाकोवाक्यसमुपबृंहणादिभ्यश्च
सुतरां सङ्गच्छते~। विशिष्य च समुद्धृते भागे स्वीकृतानि महाभाष्यविशेषणानि \textbf{युक्तिप्रतिपादनपरे} इत्यादीनि \textbf{अन्यूनानतिरिक्ते} इत्यन्तानि निकाममस्य महाभाष्यतां समर्थयितुं जाग्रतीति विपश्चितामपरोक्षोऽयमर्थः~। अपूर्वोऽयं
पूर्वपक्षसिद्धान्तात्मा भाष्यग्रन्थो ज्योतिस्तन्त्रस्य तन्त्रान्तरवदुपपादनप्रपञ्चनवैरल्यापख्यातिं बाढं परिहरतीति सद्योऽस्माकमभिमानः स्थाने वल्गति~। \\

\indent इह कश्चिदयं विशेषः, यः ११३ तमपुटे १५६ तमपुटे च दृश्यमानाभ्यां

\begin{quote} 
\textbf{यन्मयात्र केषाञ्चित् सूत्राणां तद्युक्तीः प्रतिपाद्य
कौषीतकिनाढ्येन नारायणाख्येन व्याख्यानं कारितम् अतस्तदेवात्र लिख्यते}~~(पुटं.~११३)
\end{quote} 
इति,

\begin{quote} 
\textbf{इतीदं प्रथमे वयस्येव वर्तमानेन मया द्वितीयवयसि स्थितेन कौषीतकिनाढ्येन कारितम्~। अत्र केषाञ्चिद्युक्तयः पुनरस्मदनुजेन शङ्करा-}
\end{quote}
\newpage

\begin{center} ४ \end{center}
\thispagestyle{empty}
\begin{quote}
\textbf{ख्येन तत्समीपेऽध्यापयता वर्तमानेन तस्मै प्रतिपादिताः~। तस्याढ्यत्वात् स्वातन्त्र्याच्च तत्र व्यापारश्च निर्वृत्तः~। तस्मिन्
स्वर्गते पुनरत एव मयाद्य प्रवयसा ज्ञाता युक्तीः प्रतिपादयितुं भास्करादिभिरन्यथाव्याख्यातानां कर्माण्यपि प्रतिपादयितुं यथाकथञ्चिदेव व्याख्यानमारब्धम्~।}
\end{quote}

\noindent इति च भागाभ्यां कौषीतकिना \textbf{नारायणेन} कारितं व्याख्यानं स्वयमनुवदति परन्तु सोऽयमनुवादः पञ्चदश-षोडश-सप्तदशानां सूत्राणां पङ्कत्यावृत्तेरिव अाष्टादशात् आषड्विंशमावृत्तिविरहात् न ज्ञायत इति~। \\

अपरं च चतुष्पाद्यात्मकस्य सम्पूर्णस्य तन्त्रस्य प्रथमो गीतिकापादो न पृथगिह व्याख्यातो दृश्यते~। उत्तरत्रिपादीव्याख्यानेनैव गतार्थत्वात्~।
इदं च भाष्यकार एव स्पष्टमाह~। यथा\textendash 

\begin{quote} 
\textbf{तत्रेयं त्रिपाद्यस्माभिर्व्याचिख्यासिता, यतस्तव्याख्येयरूपत्वाद्गीतिकापादस्यैतद्व्याख्यानेनैवार्थः प्रकाशेत}
\end{quote} 

\noindent इति~। \\

\textbf{श्रीमद्भास्कराचार्य-सूर्यदेवयज्व-घटीगोपादीनाम}नेकेषां प्रामाणिकानां मिषत्सु भाष्यव्याख्यानादिषु विवरणमहिम्ना बहूपपत्तिचतुरिम्णा
युक्तिभूम्ना चाग्रिममिदं भाष्यं न केवलं ज्योतिर्विदामुपकारकम् अपितु तन्त्रान्तरीयाणामपि कौतुकावहं विजयते~। 

\begin{minipage}[t]{0.15\textwidth}
\vspace{.8cm}
\textbf{नीलकण्ठसोम-सुत्वा}
\end{minipage} 
%\hfill
\begin{minipage}[t]{0.55\textwidth} 
अस्माकं भाष्यकर्ता \textbf{श्रीनीलकण्ठसोमसुत्वा} दृग्गणितकर्तुः \textbf{श्रीपरमेश्वराचार्यस्य सूनोर्दामोदर}पण्डितस्य शिष्यः \textbf{श्रीकुण्डग्रामाभिजनः} गार्ग्यगोत्रजातः आश्वलायनशाखास्थितः \textbf{जातवेदः} पुत्रः \textbf{शङ्कराग्रजः} कस्यचन \textbf{जातवेदसो} मातुल इत्यादि प्रकृतग्रन्थान्तिमभागात् ज्ञायते~। तथाहि\textendash 
\end{minipage}

\begin{quote}
\textbf{इति श्रीकुण्डग्रामजेन गार्ग्यगोत्रेण आश्वलायनेन भाट्टेन केरलसद्ग्रामगृहस्थेन श्रीश्वेतारण्यनाथपरमेश्वरकरुणाधिकरणभूतविग्रहेण जातवेदः पुत्रेण शङ्कराग्रजेन जातवेदो मातुलेन दृग्गणितनिर्मापकपरमेश्वरपुत्रश्रीदामोदरात्तज्योतिषामयनेन रवित आत्तवेदान्तशास्त्रेण सुब्रह्मण्यसहृदयेन नीलकण्ठेन सोमसुता विरचितविविध-}
\end{quote}

\newpage

\begin{center}
५
\end{center}
\thispagestyle{empty}
\begin{quote} 
\textbf{गणितग्रन्थेन दृष्टबहूपपत्तिना स्थापितपरमार्थेन कालेन शङ्कराद्यनिर्मिते श्रीमदार्यभटाचार्यविरचितसिद्धान्तव्याख्याने महाभाष्ये उत्तरभागे} \ldots \ldots \ldots \ldots  \ldots \ldots  \ldots \ldots  \ldots \ldots  \ldots \ldots  \ldots \ldots \ldots \ldots  \ldots \ldots  \ldots \ldots  \ldots \ldots  \ldots \ldots \\
 \ldots \ldots \ldots \ldots  \ldots \ldots  \ldots \ldots  \ldots \ldots  \ldots \ldots  \ldots \ldots \ldots \ldots  \ldots \ldots  \ldots \ldots  \ldots \ldots  \ldots \ldots  \ldots  \ldots \ldots \\ 
\textbf{गणितपादगतार्यात्रयस्त्रिंशद्व्याख्यानं समाप्तम्~।}~इति~।~~(पुटं.~१८०)
\end{quote} 

तन्त्रसङ्ग्रहस्योपक्रमे\textendash 

\begin{quote}
\textbf{हे विष्णो! निहितं कृत्स्नं जगत् त्वय्येव कारणे~।\\
ज्योतिषां ज्योतिषे तस्मै नमो नारायणाय ते~॥}
\end{quote}

\noindent इति, \\

उपसंहारे\textendash 

\begin{quote}
\textbf{गोलः कालक्रिया चापि द्योत्यतेऽत्र मया स्फुटम्~।\\
लक्ष्मीशनिहितध्यानैरिष्टं सर्वं हि लभ्यते~॥}
\end{quote}

\begin{sloppypar}
\noindent इति च दृश्यमानयोः श्लोकयोः \textbf{`हे विष्णो! निहितं कृत्स्नं'} \textbf{`लक्ष्मीशनिहितध्यानैः'} इति घटितौ द्वौ पादौ श्रीनारायणस्मरणेन मङ्गलार्थमुपयुज्यमानावपि तदानीन्तनकलिदिनसूचनार्थमप्युपकुरुत इति तद्व्याख्यातोऽवगमात्\renewcommand{\thefootnote}{*}\footnote{अत्र च \textbf{मङ्गलाचारयुक्तानां विनिपातो न विद्यते} इत्युक्तेनीत्या माङ्गलिकेनाचार्येणेमं श्लोकमादितो ब्रुवता प्रथमपादेन प्रबन्धारम्भदिनकल्यहर्ग्रणश्चाक्षरसंख्ययोपदिष्टः~। समाप्तिसमयाहर्गणश्च \textbf{लक्ष्मीशनिहितध्यानै}रित्यन्ते भविष्यति~। \\

तथाहि \textendash\ \textbf{हे विष्णो निहितं कृस्त्नं}  $=$ १६८०५४८ (कलिदिनं) $=$ (४६०१
मीनमासः २६ दि. कलिव.) $=$ (६७६ मनिमासः २६ दि. कोलम्बाब्दः) \hspace{1cm} तन्त्रसङ्ग्रहोपक्रमदिवसः~।\\
	
\textbf{लक्ष्मीशनिहितध्यानैः} $=$ १६८०५५३ (कलिदिनं) $=$ (४६०२ मेषमासः १ दि. कलिव) $=$ (६७६ मेषमासः १. दि कोलम्बाब्दः) \hspace{1cm} तन्त्रसङ्ग्रहोपक्रमदिवसः~।}
क्रिस्त्वब्दीयपञ्चदशशतकोत्तरार्धषोडशशतकपूर्वार्धयो(१४५०--१५५०)रन्तरालपरिमितः कालोऽस्य जीवितसमय इति लभ्यते~। 
\end{sloppypar} 
\newpage

\begin{center}
	६
\end{center}
\thispagestyle{empty}
इदं च,

\begin{quote} 
\textbf{एवं दृग्गणितं शाके त्रीषुविश्वमिते कृतम्~।}
\end{quote}

\noindent इति \textbf{दृग्गणित}वचनोन्नीत १३५३ शकवर्षेभ्यः प्रति सम्पादित ४५३२ कलिसम्वत्सरे निर्मितदृग्गणितकर्तृश्रीपरमेश्वराचार्यपुत्रशिष्यस्य
\textbf{नीलकण्ठस्य} सप्ततेः संवत्सरेभ्यः समनन्तरं \textbf{तन्त्रसङ्ग्रह}निर्मित्या निकाममनुरुद्ध्यते~।\\

अनेन च \textbf{गोलसार-तन्त्रसङ्ग्रह-सिद्धान्तदर्पणादयो} बहवः स्वतन्त्रा ग्रन्था निर्मिताः~। एषु \textbf{गोलसारस्तन्त्रसङ्ग्रहश्च} प्रकृतभाष्यात् प्रागेव जातौ ज्ञायेते~। यतो भाष्येऽस्मिन्\textendash \\

एतत्सर्वमस्माभिर्गोलसारे प्रदर्शितम्~।
\begin{quote}
\textbf{द्विघ्नान्त्यखण्डनिघ्नात् तत्तज्ज्यार्धात् त्रिभज्याप्तम्~।\\
अन्त्यादिखण्डयुक्तं त्याज्यं स्यात् पूर्वपूर्वगुणसिद्ध्यै~॥}~~~(पुटं.~५३)
\end{quote}
 
इति \textbf{गोलसारः},\\

अत एवोक्तं मया \textbf{तन्त्रसङ्ग्रहे}\textendash 

\begin{quote}
\textbf{शिष्टचापघनषष्ठभागतो विस्तरार्धकृतिभक्तवर्जितम्~।\\
शिष्टचापमिह शिञ्जिनी भवेत् स्पष्टता भवति चाल्पतावशात्~॥} 
\end{quote}
\vspace{-2mm}
\hspace{9.5cm}(पुटं.~११२)\\

इति \textbf{तन्त्रसङ्ग्रहश्च} नामग्राहं गृह्येते~। \\

अस्मद्भाष्यकर्तुः \textbf{श्रीनीलकण्ठसोमयाजिनः} सुप्रसिद्धानां वन्दनीयमहिम्नां केरलीयज्योतिर्विदां कोटौ कापि गणनीयता आसीदिति स्फुटनिर्णयकारस्य वन्दनश्लोक एतन्नामधेयघटनादवगन्तुं शक्यम्~। स हि श्लोकः\textendash 

\begin{quote}
\textbf{ब्रह्माणं मिहिरं वसिष्ठपुलिशौ गर्गं मयं लोमशं\\
श्रीपत्यार्यभटौ वराहमिहिरं लल्लं च मुञ्जालकम्~।\\
गोविन्दं परमेश्वरं सतनयं श्रीनीलकण्ठं गुरून् \\
वन्दे गोलविदश्च माधवमुखान् वाल्मीकिमुख्यान् कवीन्~।}
\end{quote}

\newpage

\begin{center}
	७ 
\end{center}
\thispagestyle{empty}
\begin{minipage}[t]{0.15\textwidth}
%	\vspace{1mm}
\textbf{नीलकण्ठसोम-सुत्वानुजः शङ्करः}
\end{minipage} 
%\hfill
\begin{minipage}[t]{0.55\textwidth} 
एतदवरजोऽपि शङ्कराख्यो ज्योतिस्तन्त्रे सुनिपुणः कोऽपि पण्डिताग्रणीरासीदित्यधोलिख्यमाना भाष्यपङ्क्तिरेव
स्पष्टीकरोति~। यथा\textendash 
\end{minipage}
 
\begin{quote} 
\textbf{इतीदं प्रथमे वयस्येव वर्तमानेन मया द्वितीयवयसि स्थितेन कौषीतकिनाढ्येन कारितम्~। अत्र केषाञ्चिद्युक्तयः पुनरस्मदनुजेन शङ्कराख्येन तत्समीपेऽध्यापयता वर्तमानेन तस्मै प्रतिपादिताः~। तस्याढ्यत्वात् स्वातन्त्र्याच्च तत्र व्यापारश्च निर्वृत्तः~। तस्मिन् स्वर्गते पुनरत एव मयाद्य प्रवयसा ज्ञाता युक्तीः प्रतिपादयितुं भास्करादिभिरन्यथाव्याख्यातानां कर्माण्यपि प्रतिपादयितुं यथाकथञ्चिदेव व्याख्यानमारब्धम्~।}
(पुटं.~१५६) 
\end{quote}

एतद्गुरुनाथश्च \textbf{श्रीपरमेश्वराचार्यात्मजो दामोदर}पण्डितो \textbf{मुहूर्ताभरणग्रन्थ}कर्तेति,
\begin{quote}
\textbf{आचार्यार्यभटीयसूत्रितमहागूढोक्तिमुक्तावली\\
मालालङ्कृतयो जयन्ति विमला वाचो यदीयाः शुभाः~।\\
सूक्ष्मा यत्प्रतिभा च गूढगणितं निश्शेषकालक्रियां\\
भूगोलं ग्रहवास्तवञ्च तदिदं विश्वं स्फुटं पश्यति~॥\\
तस्यात्मजः शिष्यवरः प्रसादमाश्रित्य दामोदरनामधेयः~।\\
मुहूर्तशास्त्राभरणं गुणाढ्यं विचित्रवृत्तं रुचिरं चकार~॥}
\end{quote}

\noindent इति पद्याभ्यां व्यक्तीभवति~। \\

अयं सोमयाजी न केवलं ज्योतिस्तन्त्रे किन्तु शास्त्रान्तरेष्वपि मीमांसा न्याय-व्याकरण-वेदान्तेषु समं परिचिती पण्डित आसीदिति तदिदमेव भाष्यं ततस्ततो लक्ष्यमुपलक्ष्यते~। तदर्था \textbf{चेयमुदाहरणदिक्} मीमांसायाम्\textendash \\

अत एवोक्तं पार्थसारथिमिश्रेण व्याप्तिनिर्णये-
\begin{quote}
\textbf{यो यथा नियतो येन यादृशेन यथाविधः~।\\
स तथा तादृशस्यैव तादृशोऽन्यत्र बोधकः~॥}
\end{quote}

\noindent इति~। \\

अनुमाने लिङ्गलिङ्गिनोर्व्याप्तिनियम एवमेवेत्यभिप्रायः~। त्रैराशिकं चानुमानम्~। अत एवैतद्विवरणे तेनैव गणितविषयोदहृतिः कृता~। \textbf{शङ्कुच्छायां}

\newpage

\begin{center}
८
\end{center}
\thispagestyle{empty} 
\noindent \textbf{वा रविर्दिविष्ठो भूमिष्ठा}मित्यादिना तस्यैव नभोमध्ये स्थितिस्तामेवाध्यर्धपञ्चदशघटिकातिभ्रान्तामित्यन्तेन ग्रन्थेन (पुटं.~५४) \\

\noindent इत्यादि~।\\

\indent अन्यच्च\textendash \\
\begin{sloppypar} 
प्रस्तुतेऽस्मिन् गणितपादभाष्ये भाष्यकारोऽयं \textbf{वृद्धगर्ग-वराहमिहिरपिङ्गलादीन् प्राचः, भास्कर-गोविन्दस्वामि-सूर्यदेव-माधवादीन्} अर्वाचश्च महितमहिम्न आचार्यान् प्रामाणिकतया स्मरन् \textbf{वैजयन्ती-गर्गसंहिता-सूर्य-सिद्धान्तादीन्} ग्रन्थान् प्रमाणयंश्च स्वभाष्यस्य सप्रमाणतामभिव्यनक्ति~। सर्वथा प्रमाणभूतनैकाचार्यग्रन्थपरिचयपचेलिममस्य ज्योतिर्ज्ञानं मन्ये केरलीयानां बहुमाननाम् \textbf{आर्यभटीय}तन्त्रस्य सर्वतो विजयप्रतिष्ठापनां च रूढमूलामावोढुमुदितं चरितार्थम्~।\\
\end{sloppypar} 

इतः पूर्वमत्रैव ग्रन्थावलौ प्रकाशितस्य \textbf{होराविवरण}स्यावतारिकायामन्ते कृतामाशंसामस्य महाभाष्यस्य प्रकाशनेन सफलयन्नयमहं च चरितार्थः~॥\\

\noindent \emph{अनन्तशयनम्,}\\
\noindent \emph{११-११-११०५  }\hspace{5.9cm}  \textbf{के.~साम्बशिवशास्त्री.}


\newpage

\thispagestyle{empty}
%%%%%%%%%%%%%%%%%%%%%%%%%%%%%%%%%%%%%%%%%%%%%%%%%%%%%%%%
\begin{center}
	\textbf{विषयानुक्रमणी} \\
	\rule{0.08\linewidth}{0.3pt}\end{center} 
\renewcommand{\arraystretch}{1.25}
\vspace{1cm}
\begin{tabular}{lp{2cm}p{.2cm}r}
\hspace{1cm} \textbf{विषयः} &&& \textbf{पृष्ठम्} \\
&&&\\
मङ्गलाचरणपुरस्सरं प्रतिपाद्यवस्तुनिर्देशः & .... && ~ १\\
ब्रह्मणः सर्वशास्त्राणामादिकर्तृत्वे प्रमाणानि  & ....&& ~  २\\
अस्य ग्रन्थस्य ब्रह्मसिद्धान्तमूलकत्वम् \\ एतदन्तर्गतन्यायकलापस्य तत्प्रसादसिद्धत्वं च  & ....&& ~  ३\\
विषय प्रयोजने  & ....&&  ,,\\
गोलज्ञप्रशंसा  & ....&&  ~ ,,\\
कृत्स्नस्यापि गणितस्य सङ्ख्यामूलकत्वात् \\ प्रथमं तत्स्वरूपप्रतिपादनम्  & ....&& ~ ३\\
ज्ञातसङ्ख्यास्वरूपस्य सङ्कलितादिकं परिकर्मचतुष्टयं विस्पष्ट- 
\\ युक्तित्वात् स्वयं स्फुरेदिति सङ्ख्यास्वरूपप्रदर्शनानन्तरं 
 \\ वर्गस्वरूपस्य तत्फलस्य च प्रदर्शनम्   & ....&&  ~  ४\\
घनस्वरूपं तत्फलं  & ....&&  ~ ,,\\
वर्गीकरणस्य स्वमूलवैपरीत्येन सिद्धेर्वर्गमूलानयनम्  & ....&&  ~ ५\\
प्रसक्तानुप्रसक्त्या गीतिकापादोक्तपरिभाषासूत्रार्थवर्णनम्  & ....&&  ~  ६\\
खण्डगुणनन्यायेन गुणनफलानयनम्  & ....&& ~ ९\\
खण्डवर्गानयनद्वारा कृत्स्नवर्गानयनम्   & ....&&  ~ ,,\\
भुजाकोटिकर्णेषु त्रिषु द्वयोर्ज्ञातयोरितरज्ञाने \\ वर्गमूलेयोरुपयोगप्रदर्शनम्  & ....&&  ~  ११\\
 आसन्नमूलज्ञानोपायः  & ....&&  ~ १२\\
एकस्मिन् विषयेऽनेकत्रैराशिकसन्निपाते \\ गोविन्दस्वाम्युक्तलघुक्रियाप्रदर्शनम्  
 & ....&&  ~ १४\\
छेद्यकद्वारा वर्गमूलोपपत्तिः  & ....&& ~ १६\\
वर्गयोगपदानयनम्  & ....&&  ~ १७\\
घनीकरणस्य स्वमूलवैपरीत्येन \\ सिद्धेर्घनमूलीकरणप्रदर्शनम् & ....&&  ~ १९\\
भास्करोक्तरीत्या घनमूलानयनम्   & ....&&  ~  २०\\
भास्करोक्तरीत्या घनानयनम्   & ....&&  ~ २१ \\

\end{tabular}

\newpage

\begin{center}
	२
\end{center}
\vspace{3mm}
\thispagestyle{empty}
%%%%%%%%%%%%%%%%%%%%%%%%%%%%%%%%%%%%%%%%%%%%%%%%%%%%%%%%
%\begin{center}
%\textbf{विषयानुक्रमणी} \\
%\rule{0.08\linewidth}{0.3pt}\end{center} 
%\renewcommand{\arraystretch}{1.25}
%\vspace{1cm}
\begin{tabular}{lp{1cm}p{.2cm}r}
\hspace{1cm} \textbf{विषयः} &&& \textbf{पृष्ठम्} \\
&&&\\
खण्डघनद्वारा कृत्स्नस्य राशेर्घनानयनम्  & ....&&  ~ २३\\
क्षेत्रकल्पनया घनमूलोपपत्तिः & ....&&  ~ २५\\
अशेषक्षेत्रयुक्तीः प्रदर्शयिष्यतः तदुपयोगिषडश्रक्षेत्रन्यायस्य \\प्रदर्शनम् & ....&& ~ २७\\
वर्गान्तरस्य योगान्तरघाततुल्यत्वम् & ....&&  ~ २९\\
प्रसक्तानुप्रसक्त्या वक्ष्यमाणपातरेखानयनम्  & ....&&  ~ ३०\\
पातरेखानयनक्रियोपपत्तिप्रदर्शनार्थं त्रैराशिकक्रियाप्रदर्शनं,\\
महाभास्करीयभाष्ये गोविन्दस्वाम्युक्तस्य `त्रैराशिके'-  \\
त्यादिश्लोकार्थस्य विवरणं च & ....&& ~ ३२\\
षडश्रक्षेत्रफलानयनम् & ....&&  ~ ३५\\
वृत्तक्षेत्रफलानयनं तद्घनफलानयनं च  & ....&&  ~ ३७\\
विषमचतुरश्रगत न्यायकलापं प्रदर्शयितुं तत्सारभूत-\\पातरेखादेः
स्वरूपप्रदर्शनम् & ....&&  ~ ३९\\
सर्वेषामपि क्षेत्राणां विस्तारायामौ प्रसाध्य फलं नेयमित्यस्य  \\
 न्यायस्य सर्वत्रातिदेशकथनम् & ....&&  ~ ४०\\
फलप्रकरणोपसंहरणानन्तरं ज्याप्रकरणारम्भः, परिधेः षड्भाग-  \\
ज्यायाः व्यासार्धतुल्यताप्रदर्शनं च & ....&&  ,, \\
परिधिव्यासयोर्मिथः परिमाणतः सम्बन्धः प्रतिपादयितुं तयोः \\
प्रथमतः प्रायिकस्य सङ्ख्यासम्बन्धस्य प्रतिपादनम्  & ....&&  ~ ४१\\
भास्करोक्तरीत्या परिध्यानयनम्  & ....&&  ~ ४२\\
सङ्गमग्रामजमाधवोक्तात्यासन्नपरिधिसङ्ख्याप्रदर्शनं, तस्याति\\
सूक्ष्मताप्रतिपादनं च  & ....&&  ~ ,,\\
ज्याबाणयोरानयनार्थं क्षेत्रच्छेदप्रदर्शनम् & ....&&  ~ ,,\\
प्रथमाद्वितययोरर्धज्ययोर्ज्ञातयोस्त्रैराशिकेनेतरज्यानयनम्  & ....&&  ~ ४५\\
गीतिकापादोक्तखण्डज्यानयनम्  & ....&&  ~ ४६\\
खण्डज्यानयनवासना  & ....&&  ~ ४८\\
खण्डज्यानां क्रमेण ह्रासे तदन्तराणां वृद्धौ च युक्तिप्रदर्शनम्  & ....&&  ~ ५०\\
त्रैराशिकानुमानयोरैक्यप्रदर्शनम्  & ....&&  ~ ५४\\
माधवोदिताः तत्परादिकलान्ता महाज्याः  & ....&&  ~ ५५\\
\end{tabular}	


\newpage
\begin{center}
	३
\end{center}
\vspace{3mm}
\thispagestyle{empty}
%%%%%%%%%%%%%%%%%%%%%%%%%%%%%%%%%%%%%%%%%%%%%%%%%%%%%%%%
%\begin{center}
%\textbf{विषयानुक्रमणी} \\
%\rule{0.08\linewidth}{0.3pt}\end{center} 
%\renewcommand{\arraystretch}{1.25}
%\vspace{1cm}
\begin{tabular}{lp{2cm}p{.2cm}r}
\hspace{1cm} \textbf{विषयः} &&& \textbf{पृष्ठम्} \\
&&&\\
इष्टदोःकोटिधनुषोः समीपसमीरिताभ्यां ज्याभ्याम् \\अभीष्टप्रदेशजयोर्दोःकोटिजीवयोरानयनम्  & ....&& ~ ५५\\
एकवृत्तगतयोर्निरन्तरयोः परिधिखण्डयोस्तुल्ययोरतुल्ययोर्वा \\
पृथक् पृथगर्धज्ययोर्विदितयोरेकीकृतस्य तच्चापद्वय-\\
स्यार्धज्यायास्त्रैराशिकेनानयनम् & ....&& ~ ५८\\
तद्विषयस्य सङ्गमग्रामजमाधवोक्तस्य `जीवे परस्परनिजेतर-\\
मौर्विकाभ्याम्' इत्यस्य पद्यस्यार्थविवरणम्, उदाहरणतया \\
ज्यानामानयनं च & ....&& ~ ,,\\
`यद्वा स्वलम्बकृतिभेदपदीकृते द्वे' इति तदीयचतुर्थपादप्रति-\\
पादितस्य प्रकारान्तरस्योपपादनम्  & ....&& ~ ८६\\
वृत्तादिक्षेत्रसिद्धिप्रदर्शनम् & ....&& ~ ८८\\
तिर्यगधऊर्ध्वसंज्ञानां तिसृणां दिशां व्यवस्था & ....&& ~ ८९\\
भूमेर्निराधारत्वं, गोलाकारत्वं च & ....&& ~ ९०\\
भूपृष्ठे समन्ततः प्राणिनिवाससद्भावः & ....&&~ ,,\\
इष्टवृत्तव्यासार्धानयनम् & ....&& ~ ९१\\
आदित्यच्छायानयने यो विशेषस्तत्स्फुरणं स्यादिति दर्शयितुं  
\\प्रदीपच्छायानयनच्छलेन तदानयनम्  & ....&& ~ ,, \\
दीपस्तम्भच्छायाग्रविवरस्तम्भोत्सेधतत्कर्णानामन्यतमेन \\ ज्ञातेनेतरानयनम् 
 & ....&&~ ९२\\
भुजाकोटिकर्णेषु त्रिषु द्वयोर्ज्ञातयोरितरानयनम् & ....&&~ ,,\\
`यश्चैव भुजावर्गः कोटीवर्गश्च कर्णवर्गः सः' इत्यस्योपपत्तिः  & ....&&~ \\
अर्धज्यानयनम् & ....&&~ १०१\\
`शङ्कुगुणं शङ्कुभुजाविवरम्' इत्यादीनां त्रयाणां सूत्राणां  \\
कौषीतकिनाढ्येन नारायणेन कृतं व्याख्यान्तरम् & ....&&~  ११३\\
अन्योन्यस्मिन्नन्तर्भूतैकदेशयोर्विषमयोर्वृत्तयोर्व्यासाभ्यां ग्रासेन \\
च सम्पातशरयोरानयनम् & ....&&~ १२४\\
श्रेढीफलानयनम् & .... && १३२\\
इष्टधनानयनवासना & .... && १३४\\
\end{tabular}	


\newpage



\begin{center}
४
\end{center}
\vspace{3mm}
\thispagestyle{empty}
%%%%%%%%%%%%%%%%%%%%%%%%%%%%%%%%%%%%%%%%%%%%%%%%%%%%%%%%
%\begin{center}
%\textbf{विषयानुक्रमणी} \\
%\rule{0.08\linewidth}{0.3pt}\end{center} 
\renewcommand{\arraystretch}{1.25}
%\vspace{1cm}
\begin{tabular}{lp{2cm}p{.2cm}r}
\hspace{1cm} \textbf{विषयः} &&& \textbf{पृष्ठम्} \\
&&&\\ 
`अथवाद्यन्तं पदार्धहतम्' इत्युक्तस्य द्वितीयस्य श्रेढीफलानयन-\\
प्रकारस्य वासना & ....&&~ १३५\\
सर्वधने ज्ञाते तेनाज्ञातस्य गच्छस्यानयनम्  & ....&&~ ,,\\
श्रेढीक्षेत्रद्वारा गच्छानयनवासना & ....&&~ ,,\\
चितिघनानयनम्  & ....&& ~ १३८\\
चितिघनानयनवासना & ....&& ~ ,,\\
`सैकपदघनो विमूलो वा' इत्युक्तस्य द्वितीयस्य चितिघना-\\
नयनप्रकारस्य वासना & ....&&~ १४१\\
वर्गचितिघन-घनचितिघनयोरानयनम्  & ....&&~ १४२\\
वर्गचितिघनवासना  & ....&&~ १४३\\
घनचितिघनवासना & ....&&~ १४५\\
गुणगुण्ययो राश्योः संवर्गे कर्तव्य उपायान्तरम्  & ....&& ~ १४७\\
राश्योः संवर्गेऽन्तरे च ज्ञातेऽज्ञातयो राश्योरानयनम् & ....&&~ १४९\\
मूलफलानयनम् & ....&&~ १५०\\
तद्वासना & ....&&~  १५१\\
त्रैराशिकेनेच्छाफलानयनम्  & ....&&~ १५३\\
व्यस्तत्रैराशिक इच्छाफलानयनम्  & ....&&~ १५५ \\
भिन्नानां सवर्णीकरणम्  & ....&&~ १५६\\
व्यस्तविधावितरस्मात् भेदप्रदर्शनम्  & ....&&~ १५८ \\
सङ्घधनानयनं सर्वधनानयनं च & ....&&~ १५९ \\
अव्यक्तमूल्यानां मूल्यज्ञानोपायः & ....&&~ १६० \\
ग्रहान्तरात् ग्रहयोगकालानयनम्  & ....&&~ ,,\\
ग्रहगत्यनुमानोपयोगि कुट्टाकारगणितम् & ....&&~ १६१\\
साग्रनिरग्रयोः कुट्टाकारयोः क्रियाभेदः  & ....&&~ १६३\\
कुट्टाकाराङ्गतया भाज्यहारयोरपवर्तनेन दृढीकरणम्  & ....&&~ १६५\\
सिद्धान्तदीपिकायां व्युत्क्रमेण प्रदर्शिता अपवर्तनयुक्तिः  & ....&&~ १६६\\
कुट्टाकारभेदः  & ....&& ~ १७१\\
वल्ल्युपसंहारयुक्तिः & ....&& ~ १७८\\
भाष्यकर्तुर्देशगोत्रनामधेयादयः  & ....&&~ १८०\\
\end{tabular}	

\newpage

\begin{center}
\large{॥~श्रीः~॥}

\vspace{0.3cm}
श्रीमदार्यभटाचार्यविरचितम्\\
\vspace{0.3cm}
{\Huge\textbf{आर्यभटीयं}}

\vspace{0.3cm}
गार्ग्यकेरलनीलकण्ठसोमसुत्वविरचितेन\\
भाष्येण समेतम्~।

\begin{center}
\rule{2cm}{.5mm} 
\end{center}

\vspace{0.3cm}\textbf{गणितपादः}
\end{center}

\begin{quote}
{\ab ब्रह्मकुशशिबुधभृगुरविकुजगुरुकोणभगणान् नमस्कृत्य~।\\
आर्यभटस्त्विह निगदति कुसुमपुरेऽभ्यर्चितं ज्ञानम्~॥~१~॥}
\end{quote}

\begin{quote}
\textbf{वागजमहीक्षपाकृज्ज्ञ\renewcommand{\thefootnote}{\s १}\footnote{\s कज्ञ \textendash\ क. ख. पाठः}शुक्रसूर्यार\renewcommand{\thefootnote}{\s २}\footnote{\s सूरिश}जीवशनिभानि~।\\
भगवन्तं चार्यभटं नत्वा व्याख्यायतेऽथ तत्तन्त्रम्~॥}
\end{quote}

इह खलु वर्तमानस्य ब्रह्मण आयुष ऊर्ध्वार्धे प्रथमकल्पे वैवस्वताख्यसप्तममन्वन्तरेऽष्टाविंशे कृष्णद्वैपायनव्यासे च चतु\renewcommand{\thefootnote}{३}\footnote{व्यासे चतु}र्युगे कल्यादितो\renewcommand{\thefootnote}{४}\footnote{के \textendash\ ग. पाठः}
दिव्याब्ददशके गते अश्मकजनपदजात आर्यभटाचार्यो ब्रह्यादिमुखविनिस्सृतानि पुरातनान्यखिलानि ज्योतिःशास्त्राण्यालक्ष्य ततः सारभूतं ग्रहगणितन्यायकलापं पृथगुपादाय कार्त्स्न्येन प्रतिपादयितुमार्यभटीयं नाम सिद्धान्तम् {\qt इष्टं हि विदुषां लोके समासव्यासधारणमि}ति न्यायमनुसरन् संक्षेपविस्तराभ्यां प्रबन्धद्वयात्मकं चकार~। तत्र त्रयोदशार्यारब्धः प्रथमः प्रबन्धः~। उत्तरोऽष्टोत्तरशतार्यारब्धः~। स च गणितकालक्रियागोलाख्यपादत्रयात्मकः~। तत्र गणितपादस्त्रयस्त्रिंशदार्यारब्धः~। कालक्रियापादस्तु पञ्चविंशत्यार्याभिरारब्धः~। गोलपादस्तु पञ्चाशता~। तत्रेयं त्रिपाद्यस्माभिर्व्याचिख्यासिता, यतस्तद्व्याख्येयरूपत्वाद्गीतिकापादस्यैतद्व्याख्यानेनैवार्थः प्रकाशेत~। 

%G.P.T. 2106. 500. 4-4-1103 B

\afterpage{\fancyhead[CE] {\s आर्यभटीये सभाष्ये }}
\afterpage{\fancyhead[CO]{\s गणितपादः~। }}
\afterpage{\fancyhead[LE,RO]{\thepage}}
\cfoot{}
\newpage
%%%%%%%%%%%%%%%%%%%%%%%%%%%%%%%%%%%%%%%%%%%%%%%%%%%%%%%%%%%%%%
\renewcommand{\thepage}{\devanagarinumeral{page}}
\setcounter{page}{2}


%\noindent{२}\hspace{4cm} {आर्यभटीये सभाष्ये }

\vspace{0.3cm}\noindent{तस्येयमाद्यार्या \textendash\ {\qt ब्रह्मे}ति~। अनेन सूत्रेण मङ्गलाचरणपुरस्सरं विषयादिकं प्रदर्श्यते~। समानार्थं चैतत्

\begin{quote}
{\qt प्रणिपत्यैकमनेकं कं सत्यां देवतां परं ब्रह्म~।\\
आर्यभटस्त्रीणि गदति गणितं कालक्रियां गोलम्~॥}
\end{quote}

\noindent इत्यनेन\renewcommand{\thefootnote}{\s १}\footnote{\s इत्यादिना \textendash\ ग. पाठः.}~। तेनात्रापि जगत्कारणभूतं ब्रह्म कार्यजातं च नमस्क्रियते~। तत्र हीतरथैकत्वानेकत्वयोर्विरोधात्~। एवं सति 

\begin{quote}
{\qt एक एव हि भूतात्मा भूते भूते व्यवस्थितः~।\\
एकधा बहुधा चैव दृश्यते जलचन्द्रवत्~॥}
\end{quote}

\noindent इत्यविरोधाच्च कार्यकारणभेदेनोभयात्मकत्वमङ्गीकृत्य तन्नमस्कारः कृतः~। तेनात्रापि ब्रह्मशब्देन परं ब्रह्म चतुर्मुखश्च विवक्ष्यते, यतस्तत्र कं ब्रह्माणं प्रणिपत्येति तन्नमस्कारश्च कृतः~। इष्यते च चतुर्मुखमुखाम्भोजविनिस्सृतत्वाच्छास्त्रस्य तदारम्भे तन्नमस्कारः~। इतरथा तत्प्रसादमन्तरेण तदर्थाप्रतीतेः~। स्मर्यते च तत्रतत्र सर्वशास्त्राणामादिकर्तृत्वं ब्रह्मणः\textendash 

\begin{quote}
{\qt प्रथमं सर्वशास्त्राणां पुराणं ब्रह्मणा स्मृतम्~।\\
अनन्तरं तु वक्त्रेभ्यो वेदास्तस्य विनिस्मृताः~॥\\
बिभेत्यल्पश्रुताद् वेदो मामयं प्रतरेदिति~।}
\end{quote}

\noindent इत्यादि~। वृद्धगर्गश्चाह\textendash  

\begin{quote}
{\qt स्वयं स्वयम्भुवा सृष्टं चक्षुर्भूतं द्विजन्मनाम्~।\\
वेदाङ्गं ज्योतिषं ब्रह्म समं वेदैर्विनिस्मृतम्~॥

मया स्वयम्भुवः प्राप्तं क्रियाकालप्रसाधकम्~।\\
मत्तश्चान्यानृषीन् प्राप्तं पारम्पर्येण पुष्कलम्~॥

तैस्तथादृष्टिभिर्भूयो ग्रन्थैः स्वैः स्वैरुदाहृतम्~।}
\end{quote}

\noindent इति~। अन्यत्रापि स्मर्यते\textendash 

\begin{quote} 
{\qt सिसृक्षुणा पुरा सृष्टं वेदानेतत् स्वयम्भुवा~।}
\end{quote} 

\noindent इति~। तथाच वराहमिहिरः\textendash  

\begin{quote} 
{\qt आब्रह्मादिविनिःसृतमालक्ष्य ग्रन्थविस्तरं बहुशः\\
क्रियमाणकमेवेदम्}
\end{quote}

\newpage

%\hspace{4cm} गणितपादः~। \hspace{4cm} ३

\noindent इति~। वक्ष्यते चास्य ब्रह्मसिद्धान्तमूलत्वं तत्प्रसादसिद्धत्वं चास्य न्यायकलापस्य\textendash  
\begin{quote}
{\qt सदसज्ज्ञानसमुद्रात् समुद्धृतं देवताप्रसादेन~।\\
सज्ज्ञानोत्तमरत्नं मया निमग्नं स्वमतिनावा~॥\\
आर्यभटीयं नाम्ना पूर्वं स्वायम्भुवं सदा सत्यम्~।\\
सुकृतायुषोः प्रणाशः कुरुते प्रतिकञ्चुकं योऽस्य~॥}
\end{quote}
\noindent इति~। एवं सर्वातिशायिवस्तुनमस्कारान्महन्मङ्गलं सम्पादितम्~। अत्र पुनः कार्यविशेषाणां केषाञ्चित् पृथगुपादानात् तद्विषयत्वं चास्य प्रदर्शितम्~। तेनात्र भूग्रहभानां चरितं विषयः~। गीतिकापादोपसंहारे च विषयप्रयोजने विस्पष्टं प्रदर्शिते\textendash  

\begin{quote}
{\qt दशगीतिसूत्रमेतद् भूग्रहचरितं भपञ्जरे ज्ञात्वा~।\\
ग्रहभगणपरिभ्रमणं स याति भित्त्वा परं ब्रह्म~॥}
\end{quote}

\noindent इति~। तेन तत्रापि क्वादीनां स्वीकारार्थमनेकशब्दोपादानम्~। तत्रापि गणितकालक्रियागोलभेदेन प्रतिपाद्यं वस्तु सकलं संक्षिप्योक्तम्~। विस्पष्टं चात्र त्रैविध्यमुपरिष्टात्~। ज्ञायतेऽनेनेति ज्ञानं ग्रहगतिज्ञानसाधनं गणितच्छेद्यकगोलबन्धादि~। नहि गोलज्ञानमन्तरेण\renewcommand{\thefootnote}{१}\footnote{मनन्त} ग्रहगतिर्ज्ञातुं शक्या\renewcommand{\thefootnote}{२}\footnote{गत्या~। गो}~। गोलश्च क्षेत्रात्मकत्वाद्गणितगभ्यः~। अतएवोक्तं\textendash  

\begin{quote}
{\qt गणितज्ञो गोलज्ञो गोलज्ञो ग्रहगतिं विजानाति~।\\
यो गणितगोलबाह्यो जानाति ग्रहगतिं स कथम्~॥}
\end{quote}

\noindent इति~। भावे वा ल्युट्~। यतः श्रोतृ\renewcommand{\thefootnote}{३}\footnote{त्र \textendash\ क. ख. पाठः.}बुद्धौ ग्रहगत्यनुमानमुद्भाव्यते परोपदेशात्मकेन वाक्येन~। उपायोपेयभावलक्षणः सम्बन्धः, प्रतिपाद्यप्रतिपादकभावलक्षणो वा~॥~१~॥ \\

अत्र गणितपादे सामान्यगणितमेव प्रतिपाद्यते~। तच्च युक्तिमात्रपरम्~। कालक्रियागोलपादयोः पुनर्ग्रहगतौ तदतिदेश क्रियते~। तत्र कृत्स्नस्यापि गणितस्य सङ्ख्यामूलत्वात् प्रथमं तत्स्वरूपं प्रतिपाद्यते\textendash  

\begin{quote}
{\ab एकं दश च शतं च सहस्रमयुतनियुते तथा प्रयुतम्~।\\
कोट्यर्बुदं च वृन्दं स्थानात् स्थानं दशगुणं स्यात्~॥~२~॥}
\end{quote}

\newpage 

इति~। शतं चेत्यत्र चकारस्य पादान्तत्वात् गुरुत्वं {\qt गन्ते} (अ. १. सू. १०) इति पिङ्गलस्मरणात्~। नेदमपि संख्याविशेषाणां संज्ञाप्रदर्शनपरं सूत्रम्~। किन्तु दशगुणोत्तरत्वप्र\renewcommand{\thefootnote}{१}\footnote{रप्र \textendash\ ग. पाठः.}तिपादनपरम्~। अतः {\qt स्थानात् स्थानं दशगुणं स्याद्} इति\renewcommand{\thefootnote}{२}\footnote{शतं चे(त्य ? त्या)दि वा \textendash\ क. ख. पाठः.} वाक्यार्थः तस्य पा\renewcommand{\thefootnote}{३}\footnote{क्यं पा \textendash\ ख. पाठः.}रिभाषिकतानिरसनार्थम्~। लोकवेदमूलत्वेन नित्यत्वमेकं दश चेत्यादिभिः पदैः प्रदर्श्यते~॥~२~॥ 

\begin{quote}
{\qt ज्ञातसङ्ख्यास्वरूपस्य यतः सङ्कलितादिकम्~।\\
स्फुरेत् विस्पष्टयुक्तित्वात् परिकर्मचतुष्टयम्~॥\\
वर्गाद्येव ततोऽत्रोक्तं पद्यैस्त्रिभिरतःपरम्~।\\
वर्गोऽर्धेन घनश्चापि प्रत्येकं मूलमार्यया~॥}
\end{quote}

\begin{quote} 
{\ab वर्गः समचतुरश्रः फलं च सदृशद्वयस्य संवर्गः~।}
\end{quote} 

इति~। वर्ग इत्युक्ते समचतुरश्रं क्षेत्रं बोद्धव्यम्~। यतस्तत्क्षेत्रफलं वर्गीकरणेन सम्पाद्यते~। तद्यथा \textendash\ तत्प्रदर्शनाय समचतुरश्रं फलकं मृन्मयं वा निर्मायाङ्गुलहस्तयोजनकलादिषु येन मानेन तत्क्षेत्रं मीयते तेनैक विस्तारं विदार्य पुनः प्रत्येकमेकदीर्घं च छिन्नेषु यावन्तः\renewcommand{\thefootnote}{४}\footnote{त \ldots \ldots क्षे \textendash\ क. पाठः.} खण्डाः
स्युस्तावत्फलं तत् क्षेत्रम्~। एवं तद्गतफलान्यपि चतुरश्राणि~। न केवलं समचतुरश्रक्षेत्र एव फलानां समचतुरश्रत्वम्~। अपितु
वृत्तत्र्यश्रचापाकारादिष्वखिलेष्वपि~। तेष्वपि समचतुरश्रकोष्ठसङ्ख्या हि फलाख्या~। कथं पुनस्तदानयनमित्यत आह \textendash\ {\qt सदृशद्वयस्य संवर्ग} इति~। तुल्येषु चतुर्षु बाहुष्वेककोणस्पृष्टयोर्द्वयोः संवर्ग इति यावत्~। एतदुक्तं भवति \textendash\ यावद्बाहुकं
समचतुरश्रं क्षेत्रं तावती सङ्ख्यापि तावत्कृत्वः कृता वर्गाख्या~। उक्तं च वैजयन्त्यां\textendash 

\begin{quote}
{\qt वर्गस्तावत्कृतिश्चेति तावत्कृत्वः कृते द्वयम्~।\\
तन्मूले च पदं हेतुः\renewcommand{\thefootnote}{*}\footnote{'तन्मूले तु पुमान् हे' इति मुद्रितपाठः}} 
\end{quote}

\noindent इति~॥~$\hbox{२}\dfrac{\hbox{१}}{\hbox{२}}$~॥ 

\begin{quote} 
{\ab सदृशत्रयसंवर्गो घनस्तथा द्वादशाश्रिः स्थात्~॥~३~॥}
\end{quote} 

\newpage

इति~। द्वादशाश्रिर्घनस्तथा घनफलं\renewcommand{\thefootnote}{१}\footnote{लवद् द्वादशाश्रिः \textendash\ क. ख. पाठः.}  च द्वादशाश्रि~। वक्ष्यमाणेषु षडश्रिघनगोलादिष्वपि घनफलं द्वादशाश्र्येव~। तदानयनमपि सदृशत्रयसंवर्ग इति प्रदर्शितम्~। तुल्यानां विस्तृतिदीर्घपिण्डानां घातो घनः~। तद्यु\renewcommand{\thefootnote}{२}\footnote{(द्वि ? द्य) \textendash\ ख. पाठः.}क्तिरपि मृदादिना प्रदर्श्या~॥~३~॥\\

नन्वेतद्गुणनाख्यमेव परिकर्म न वर्गाख्यं परिकर्मान्तरमत्रोक्तम्~। एवं हि वर्गपरिकर्माहु\textendash 

\begin{quote}
{\qt समद्विघात कृतिरुच्येऽथ स्थाप्योऽन्त्यवर्गो द्विगुणान्त्यनिघ्नाः~।\\
स्वस्वोपरिष्टाच्च तथापरे\renewcommand{\thefootnote}{३}\footnote{रा \textendash\ ग. पाठः.}ऽङ्कास्त्यक्त्वन्त्यमुत्सार्य पुनश्च राशिम्~॥}
\end{quote}

\noindent इति~। सत्यं न परिकर्मान्तरमुक्तम्~। गुणनेनैव चतुरश्रफलस्य सिद्धत्वात् तदर्थे न परिकर्मान्तरमेष्टव्यम्~। तन्मूलज्ञाने पुनरेष्टव्यमेवावश्यं परिकर्मान्तरम~। तेन वर्गीकरणस्यापि तद्वैपरीत्यायैव वर्गीकरणाख्यं परिकर्मान्तरमङ्गीकृतमिति द्योतयितुं मूलमेवाहार्यभटः परिकर्मान्तरं\textendash 

\begin{quote}
{\ab भागं हरेदवर्गान्नित्यं द्विगुणेन वर्गमूलेन~।\\
वर्गाद् वर्गे शुद्धे लब्धं स्थानान्तरे मूलम्~॥~४~॥}
\end{quote}

\noindent इति~। वर्गात् विषमस्थानादन्त्यात् यावतो वर्गः शोध्यः तावतो वर्गे शुद्धे तस्य शुद्धस्य वर्गस्य मूलमेकादिनवान्तेषु यावत्सङ्ख्यं तेन
द्विगुणेनावर्गात् समस्थानात् भागं हरेत्~। नित्यं सर्वदा~। विषमस्थानादेव वर्गः शोध्यः~। अवर्गादेव भागो हर्तव्यः~। यावतिथात् स्थानात् वर्गः
शुद्धः तदधोगतं यत् समस्थानं तत एव तन्मूलेन द्विगुणेन भागं हरेत्~। तदधोगतवर्गस्थानादेव तत्फलवर्गः शोध्य इतीह नियमः स्यात्~। यदा पुनस्तत्र भागो हर्तुं न शक्यस्तदा नतु तदधोगतात् विषमस्थानाद्धर्तुं युक्तम्~। अपितु तस्याप्यधोगतात् समस्थानादेव~। यदा तत्रापि हर्तुं न 
शक्यस्तदापि तत एकान्तरितात् समस्थानादेव भागो हर्तव्यः~। तत्र यल्लब्धं तद्वर्गस्तदधोगतविषमस्थानादेव शोध्यः~। न पुनरेकान्तरितेभ्यः
स्थानेभ्य इतीह नियमोऽस्ति~। तेन भागहरणे एतन्निरूप्यं \textendash\ यावति फले गृहीते तदधस्तनात् तत्फलवर्गशोधनं कर्तुं शक्यं तावदेव फलं ग्राह्यम्~। एवं पुनरपि वर्गस्थानात् फलवर्गे शुद्धे समस्थानात् भागहरणेन यल्लब्धं तत् स्थानान्तरे मूलम्~। तत्र वर्गशोधनस्थानान्निरन्तरस्थानहरणे
शोधितवर्गमूलस्थानान्निरन्तरस्थानगतमूलं हृतफलम्~। इतरथा यावद्भ्यः समस्थानेभ्यो

\newpage

\noindent भागो हर्तुं न शक्यः पूर्वमूलात् तावत्स्थानान्तरितस्थानगतमूलं तत्फलमित्यर्थः~। एतदेव कर्म तद्वर्गराशिक्षयान्तमावर्तनीयम्~। एतदुक्तं भवति~। एवं यत् स्थानद्वयगतं मूलद्वयं लब्धं तेनापि द्विगुणेन द्वितीयवर्गशोधनस्थानात् यदधोऽनन्तरमवर्गस्थानं ततो भागं हरेत्~। तत्रापि यदा हार्यस्य हारकादल्पत्वाद्वा तत्फलवर्गस्य तदधोगतवर्गस्थानाच्छोधयितुमशक्यत्वाद्वा भागो न हर्तव्यः, तदा प्राग्वत् यस्मादवर्गाद्धरणं तन्निरन्तराधोवर्गस्थानात् तत्फलवर्गशोधनं च कर्तुं युक्तं तत्र तद्द्वयं कृत्वा पूर्वस्थापितमूलात् तावत्स्थानान्तरिते\renewcommand{\thefootnote}{१}\footnote{रेत \textendash\ ग. पाठः.} तत्फलमपि मूलत्वेन स्थापयेत् यावत्सु हरणं न कृतं तत्फलवर्गशोधनं वा~। एवं यावदाद्यविषमस्थानात् वर्गः शोध्यते तावदेवमेव कार्यम्~। तत्र
यदि निःशेषता स्यात् तदा निरवयवमूलम्~। शेषे सति सावयवम्~। यदा पुनरादितः प्रभृति कतिपयथादेव वर्गस्थानाद्वर्गे शुद्धे निःशेषता स्यात् तदा तदधो यावन्ति शून्यस्थानानि वर्गराशेः सन्ति तान्यर्धीकृत्य मूलराशेर्दक्षिणतः स्थाप्यानीत्यादिकं सुगममेवेति भावः~। सूचितं ह्येतत् परिभाषासूत्रेऽपि\textendash 

\begin{quote}
{\qt वर्गाक्षराणि वगेऽवर्गेऽवर्गाक्षराणि काद् ङ्मौ यः~।\\
खद्विनवके स्वरा नव वर्गेऽवर्गे नवान्त्यवर्गे वा~॥}
\end{quote}

\noindent इत्यत्र~। ग्रहगणितसाधनभगणादिसङ्ख्यामल्पेन ग्रन्थेन प्रतिपादयितुं परिभाषात्र क्रियते~। वर्गाक्षराणि कादीनि मान्तानि वर्गे विषमस्थाने ततोऽन्यानि यादीन्यवर्गे अवर्गस्थाने च गतां सङ्ख्यां प्रतिपादयन्ति~। काद् वर्गाक्षराणि कात् प्रभृत्येव~। ककारात् प्रभृत्येव
एकद्व्यादिसङ्ख्यां प्रत्याययन्तीत्यनेन प्रसिद्धं कटपयादित्वं वर्णानां व्युदस्यते~। तेन नञयोः शून्यत्वमपि न स्यात्~। एवं पञ्चविंशत्यन्ता सङ्ख्या वर्गाक्षरैरेव प्रतिपाद्या~। अपि च संयोगाक्षरेषु व्यञ्जनानां सर्वेषां सङ्ख्या ग्राह्या न पुनः स्वरात् पूर्वस्यैव~। तेन पञ्चविंशतेरूर्ध्वमपि काश्चित् सङ्ख्या वर्गाक्षरैरेव प्रतिपादयितुं शक्याः, ख्युघृ\renewcommand{\thefootnote}{२}\footnote{क्नख्मेत्या \textendash\ क. ख. पाठः.} इत्यादिभिः~। तेनावर्गस्थानगतैर्यादिभिस्त्रिंशदादय एव सङ्ख्या प्रत्याय्यन्त इत्याह \textendash\ ङ्मौ य इति~। किञ्चात्र स्थाननियमोऽपि न वर्णक्रमवशात्~। कथं तर्हीत्याह \textendash\ खद्विनवके स्थानद्विनवके वर्गस्थाननवकेऽवर्गस्थाननवके च स्वाङ्गभूतव्यञ्जनानि नियमयन्ति~। के~। स्वरा नव~।


\newpage

\noindent स्वराणां नवत्वं ह्रस्वदीर्घयोरभेदेन प्लुतानां चाप्रयोगात्~। संयोगे तु स्वाङ्गि\renewcommand{\thefootnote}{१}\footnote{ङ्गी \textendash\ क. ख. पाठः.}भूतस्वर एक एव संयुक्तानां स्थानं नियमयति~। यादीनां त्रिंशदादिसङ्ख्यत्वं वर्गस्थानापेक्षयैव~। इतरथा वर्गस्थान एव यादयोऽपि स्युरिति~। तेन हकारसङ्ख्यैव वर्गस्थानं प्रविशति, अवर्गस्थानापेक्षया दशसङ्ख्यत्वात्~। एवं नञयोरवर्गस्थानसम्बन्धश्च स्यात्~। टा\renewcommand{\thefootnote}{२}\footnote{भा \textendash\ क. पाठः.}दीनां स्थानद्वयसम्बन्धश्च~। नन्वेवमष्टादशादूर्ध्वस्थानगता सङ्ख्या प्रतिपादयितुं न शक्या~। क्रमस्य तत्स्थान\renewcommand{\thefootnote}{३}\footnote{तु स्थान}नियामकत्वे पुनर्यावदपेक्षं वक्तुं शक्या, इत्यस्याः परिभाषाया विषयसङ्कोचनान्न\renewcommand{\thefootnote}{४}\footnote{चान्न \textendash\ ग. पाठः.} चातुर्यमित्यत आह \textendash\ {\qt नवान्त्यवर्गे वेति}~। नवानां वर्गस्थानानामन्त्ये वर्गे वा स्वराणां यं कञ्चिद् विशेषं विधाय\renewcommand{\thefootnote}{५}\footnote{षमाधा \textendash\ ख. ग. पाठः.} प्रतिपादयितुं शक्या~। शास्त्रव्यवहारस्तु परार्धादूर्ध्वं न प्रसरति~। लोकवेदयोरपि परार्धावधय एव सङ्ख्याः प्रसिद्धाः~। एवमत्राष्टादश सङ्ख्यास्थानानि परिगृहीतानि~। अतस्तदंश एवैकं दश चेत्यादिना विव्रियते~। ओजयुग्मयोः स्थानयोर्वर्गावर्गसंज्ञाया
वर्गतन्मूलपरिकर्मापेक्षत्वात् तदप्यत्र सूच्यते~। ततस्तद्विवरणमेव भागं हरेदित्यादि च~। कथं पुनर्वर्गमूलकर्मण्योजयुग्मयोः
स्थानयोर्वर्गावर्गसंज्ञाप्रसिद्धिमात्रं दर्शयता तत्परिकर्म सूचितम्~। उच्यते~। तत्प्रदर्शने श्रोतॄणां तद्धेतुजिज्ञासा स्यात्~। ततश्चिन्तयतः प्रतिभाजुषस्तत्परिकर्मयुक्तिः कृत्स्नापि स्फुरेदिति भावः~। तथाहि \textendash\ एकस्थानगतानाम् अङ्कानां वर्ग एकस्थान एव स्थाप्य इत्येतत् सुगमम्~। एकस्यैकेन स्वतुल्येन गुणने ह्येकमेव स्यात्~। एवं द्वयादीनामपि स्वतुल्येन गुणकारेण गुणेन चतुरादय एकाशीत्यन्ता 
अङ्काः क्रमेण स्युः~। तत्र षोडशादीनां स्थानद्वयगतत्वेऽपि प्रथमस्थानापेक्षयैव षोडशत्वादिसङ्ख्यावगम्यत इत्येकस्थान एवैकाशीत्यन्ता नव वर्गाङ्काःस्थाप्याः~। एवमेव दशादि\renewcommand{\thefootnote}{६}\footnote{शमादि \textendash\ क. ख. पाठः.}नवत्यन्तानां वर्गा अपि तृतीये शतस्थाने क्रमेणैव स्थाप्याः स्युः~। नन्वेकद्व्यादीनामिव पङ्क्तिविंशत्यादीनां वर्गोऽपि स्वस्थाने द्वितीय एव स्था\renewcommand{\thefootnote}{७}\footnote{यः स्था \textendash\ ख. पाठः.}पयितुं युक्तः~। सत्यम्~। पङ्क्त्यादयोऽपि यद्येकादिभिरेव गुण्येरन् तर्हि स्वस्थान एव स्थाप्याः स्युः~। वर्गे पुनः स्वसङ्ख्ययैव सर्वे राशयो गुण्यन्ते~। तेन गुण्यस्य गुणकारस्य चैकस्थानस्य शून्यत्वेन गुणितस्यादितः स्थानद्वयस्य शून्यत्वापत्तेः प्रथमाङ्कस्थानाद्द्वितीयस्थानगताङ्कवर्गस्य स्थानद्वयोत्कर्षः स्यात्~। एवमुपरिष्टादपि वर्गीकार्याणां राशीनाम् एकैक-


\newpage

\noindent स्थानोत्कर्षे उत्तरोत्तरवर्गस्थानस्य पूर्वपूर्वस्थानगतवर्गापक्षेया निरन्तरोपरिस्थानामङ्क\renewcommand{\thefootnote}{१}\footnote{नाङ्क}वर्गाणां स्थानद्वयोत्कर्षः स्यात्~। एवं
प्रथमद्वितीयादिस्थानगताङ्कानां वर्गाः प्रथमद्वितीयादिविषमस्थानगताः स्युः~। एवं प्रथमाद्येकैकस्थानगतानामङ्कानां वर्गा विषमस्थान एव स्थाप्याः स्युः~। कथं पुनरनेकस्थानावस्थितानाम् अङ्कानां वर्ग इति चेत्, तत्राप्याद्याङ्कस्य स्थानं यावतिथं भवति तावतिथे वर्गस्थान एव तद्वर्गः स्थाप्यः~। तत्र पुनः सर्वेषामङ्कानां सर्वैर्हनने कर्तव्ये ये ये स्वस्थानाङ्कघातास्ते सर्वे स्वस्थानसम्बन्धिविषमस्थानेष्वेव
स्थाप्याः~। इतरे पुनर्यथायथं समेषु विषमेषु च स्युः~। ते च गुण्यगुणकाराङ्क\renewcommand{\thefootnote}{२}\footnote{नाङ्क }योः स्थानैक्यमेकोनं यावत् तावतिथे स्थाने स्थाप्याः स्युः~। यतः कपाटद्वयसन्धिन्यायेन गुण्यते~। तत्र बहुस्थानगताङ्कानां स्वस्वस्था\renewcommand{\thefootnote}{३}\footnote{स्वस्था \textendash\ क., सस्था \textendash\ ग. पाठः.}नयोरङ्कयोस्तुल्यत्वमेव स्यात्~। गुण्यगुणकयोः साम्यात्~। तेन तयोर्घातो वर्ग इत्युच्यते~। इतरेषां पुनर्घात एव~। तत्रान्त्यस्थानगतस्याद्यस्थानगतस्य वा वर्गे क्वचित् स्थापिते तत्सम्बन्धिषु घातेषु सर्वेषु स्थापितेष्वेव तत्समीपगस्य वर्गः स्थाप्यते~। कथमन्त्यस्थानादिके तावद्वर्गेऽन्त्यस्थानाङ्कवर्गस्थापनानन्तरमन्त्याङ्केन
द्विगुणेन हता इतरेऽङ्का स्थाप्याः~। तथाच\renewcommand{\thefootnote}{४}\footnote{पि \textendash\ क. पाठः.} तद्गुण्यानां तद्गुणकानां च घाताः स्थापिताः स्युः~। तस्य गुण्यत्वं पुनरन्त्यस्थानगुणन एव~। उपान्त्यादीनां गुण्यत्वे पुनरन्त्यस्य गुणकारत्वमेव~। एवमन्त्याङ्कस्य गुणने उपान्त्यादिभिर्गुणितोऽन्त्यो येषु स्थानेषु यावानेवमुपान्त्यादीनां गुण्यत्वेऽप्यन्त्याङ्केन गुणकेन गुणितस्तेषु स्थानेषु तावानेव~। तद्यथा \textendash\ अन्त्ययोः स्थानयोर्घातस्य स्थानं यावतिथं तत एकापकृष्टमन्त्योपान्त्ययोर्घातस्येत्येतदुभयत्रापि समानम्~। अन्त्यस्य गुण्यत्वेऽन्त्याद्गुणकारादुपान्त्यस्य गुणकारस्य
निरन्तराधोगतत्वात् तद्धतोऽन्त्योऽङ्कोऽन्त्यवर्गान्निरन्तराधस्थान एव स्यात्~। एवमुपान्त्यस्य गुण्यत्वेऽप्यन्त्यस्यान्त्यगुणनान्निरन्तराधोगतमेव घातस्थानम्~। गुण्यस्थानस्यैवैकापकृष्टत्वाद्गुणकारस्थानस्यापकर्षाभावाच्च~। एवमुभयथापि ये घाता उभयेऽपि ते यथा परिगृहीताः स्युरिति द्विगुणेनान्त्येनेतरे सर्वे गुण्यन्ते~। एवं गुणिते सति पुनरुपान्त्यान्तं यः खण्डो वर्गस्य राशेस्तद्वर्ग एव
पुनरवशिष्यत इत्येतत्खण्डगुणनन्यायेन सिद्धम्~। एवं हि खण्डगुणनमुक्तम्\textendash 

\newpage

\begin{quote}
{\qt गुण्यस्त्वधोऽधो गुणखण्डतुल्यस्तैः खण्डकैः सङ्गुणितो युतो वा~।\\
भक्तो गुणः शुध्यति येन तेन लब्ध्या च गुण्यो गुणितः फलं वा~॥\\
द्विधा भवेद् रूपविभाग एवं स्थानैः पृथग्वा गुणितः समेतः~।}
\end{quote}

\noindent इति~। तत्र स्थानविभागखण्डगुणनमाश्रित्यैतन्निरूपणीयम्~। तत्रान्त्यस्थानगत एको महान् खण्डः~। इतरोऽल्पः खण्डः~। तत्रान्त्याङ्केन महता खण्डेन गुण्यो राशिः कृत्स्न एव गुणनीयः~। इतराङ्कराशिनेतरखण्डेन च~। तत्र गुण्य\renewcommand{\thefootnote}{१}\footnote{णय \textendash\ क. पाठः.}स्यापि तथा खण्डने\renewcommand{\thefootnote}{२}\footnote{ण्डे \textendash\ ख. पाठः.} कृते गुण्या\renewcommand{\thefootnote}{३}\footnote{णा \textendash\ क. पाठः.}न्त्याङ्कः\renewcommand{\thefootnote}{४}\footnote{ङ्कस्य तु \textendash\ क. ख. पाठः.} स्वतुल्येन महता खण्डेनोपान्त्यान्तेने\renewcommand{\thefootnote}{५}\footnote{न्त्येने \textendash\ ग. पाठः.}तरखण्डेन च गुणनीयः~। गुण्य\renewcommand{\thefootnote}{६}\footnote{णय \textendash\ क. पाठः.}स्येपान्त्यान्तखण्डोऽपि तथा~। तथा सति\renewcommand{\thefootnote}{७}\footnote{तत्र \textendash\ क. ख. पाठः.} गुणिताश्चत्वारः खण्डाः स्युः~। तेषामैक्यं च कृत्स्नस्य राशेर्वर्गः~। तत्रैकः खण्डोऽन्त्यस्थानाङ्कस्य वर्गः~। द्वितीयोऽन्त्याङ्कगुणित इतरः
खण्डः~। एवं महता खण्डेन कृत्स्नोऽपि गुण्यो गुणितः स्यात्~। इतरखण्डेन गुण्यगुणने पुनरितरखण्डेन गुणितोऽन्त्याङ्क एकः~। इतरखण्डवर्गोऽन्यः~। स एवात्रावशिष्यते, अन्येषां त्रयाणां परिगृहीतत्वात्~। तत्र प्रथमखण्डोऽन्त्यवर्गस्थापनेन परिगृहीतः~। अन्त्याङ्केन गुणित इतरः खण्डः इतराङ्कैर्गुणितोऽन्त्याङ्कखण्डश्च द्विगुणेनान्त्याङ्कखण्डेनेतरेषामङ्कानां गुणने परिगृहीतौ~। एवं खण्डवर्गोऽप्युक्तः\textendash 

\begin{quote}
{\qt खण्डद्वयस्याभिहतिर्द्विनिघ्नी तत्खण्डवर्गैक्ययुता कृतिर्वा~।}
\end{quote}

\noindent इति~। उपान्त्यान्तस्य खण्डस्यापि खण्डवर्गन्यायमाश्रित्यैव वर्गः क्रियते~। तेष्वप्यन्त्योऽङ्क एकः खण्डः~। इतरेऽन्यः~। एवमाद्यस्थानाङ्कवर्गस्थापने कृत्स्नस्य राशेर्वर्गः कृतः स्यात्~। 

\begin{quote}
{\qt एवं मुहुर्वर्गघनप्रसिद्ध्यै आद्याङ्कतो वा\renewcommand{\thefootnote}{*}\footnote{'प्रसिद्धावाद्याङ्कतो वा' मुद्रितपाठः.} विधिरेष कार्यः~।}
\end{quote}

\noindent इति~। आद्याङ्कमारभ्यापि वर्गघनौ कार्यौ~। तत्राद्याङ्कात प्रभृति वर्गीकरणं {\qt भागं हरेदवर्गादि}त्यादेर्वै\renewcommand{\thefootnote}{८}\footnote{त्यद्यैर्वै \textendash\ ग. पाठः.}परीत्येन सिद्धम्~। विपरीतकर्मापि वक्ष्यति\textendash 

\begin{quote}
{\qt गुणकारा भागहरा भागहरा ये भवन्ति गुणकाराः~।\\
यः क्षेपः सोऽपचयोऽपचयः क्षेपश्च विपरीते~॥}
\end{quote}


\newpage

\noindent इति~। विपरीते परावृत्य गणिते~। आनुलोम्येन गुणने ये गुणकारास्ते प्रातिलोम्ये भागहाराः स्युः~। आनुलोम्ये ये भागहारास्ते प्रातिलोम्ये गुणकाराः~। क्षिप्यत इति क्षेपः~। आनुलोम्ये यः क्षेपः स इतरत्रापचयः~। अपचीयत इत्यपचयः~। आनुलोम्ये योऽपचयः सोऽन्यत्र क्षेप इति~। तथात्रापि मूली\renewcommand{\thefootnote}{१}\footnote{ल \textendash\ क. ख. पाठः.}करणे प्रथमस्थानस्य वर्गे शोधिते तन्मूलस्थापनं चरमं कर्म~। ततः प्रथमस्थानाङ्कं पृथग् विन्यस्य तद्वर्गीकरणं प्रथमं कार्यम्~। मूले पुनस्ततः प्राक्तनं कर्म पूर्वलब्धमूलेन द्विगुणेन द्वितीयस्थानाद्धरणं तत्फलं च प्रथमस्थानाङ्कसङ्ख्यम्~। तेन प्रथमस्थानाङ्को द्वितीयाद्यङ्कराशिना द्विगुणेन हतो द्वितीयेऽवर्गाख्ये स्थाने स्थाप्यः~। ततः प्राक् द्वितीयस्थानाङ्कवर्गशोधनं तृतीयस्थानात् कृतमिति द्वितीयस्थानवर्गस्तृतीये स्थाने द्वितीयवर्गाख्ये स्थाप्यः~। एवमुपरिष्टादप्या स्थानपरिसमाप्तेः~। गुणनमप्यानुलोम्येन प्रातिलोम्येन वा कार्यम्~। यदि गुण्यपङ्क्तावेव गुणन\renewcommand{\thefootnote}{२}\footnote{णित\textendash\ ख. ग. पाठः.}फलमपि स्थाप्यते तर्हि गुण्यस्यान्त्याङ्कात् प्रभृत्येव गुणनं कार्यम्~। यदि कश्चिद् गुण्याद्याङ्कात् प्रभृति गुणनमिच्छति तर्हि तेन बहिरेव गुणितं\renewcommand{\thefootnote}{३}\footnote{त \textendash\ ग. पाठः.} फलं स्थाप्यं न गुण्यपङ्क्तौ~। तस्यामेव स्थाप्यमाने तदूर्ध्वगतदशस्थानाद्यङ्कानां गुणितफलसंव\renewcommand{\thefootnote}{४}\footnote{क \textendash\ क. ख. पाठः.}लनेन ते न पृथग् ज्ञातुं शक्याः~। ततस्तेषामेव गुणकारेण गुणनं कर्तुं न शक्यम्~। तत्र तत्र दृष्टानामङ्कानां गुणने पुनर्गुणितानामपि मुहुर्मुहुर्गुणनात् फलाधिक्यं स्यात्~। अन्त्यात् प्रभृति गुणने तु गुणन\renewcommand{\thefootnote}{५}\footnote{णित \textendash\ ख. ग. पाठः.}फलस्याधोगमनाभावादगुणितानां च तदध एव स्थितत्वात् तेषामविकारात् त एव गुणकारेण गुणयितुं शक्या इति~।\\

ङ्मौ य इत्यनेन सङ्कलितमपि सूचितमिति व्याख्येयम्~। कथम्~। इदं तावदिहोक्तं पञ्च(तो ? कयो)र्योगे दश\renewcommand{\thefootnote}{८}\footnote{शत्वं \textendash\ ग. पाठः.}सङ्ख्यत्वं स्यादिति~। तेन षट्कचतुष्कयोर्योगेऽपि दश सम्पद्यन्ते~। योगिनोरेकस्यैकाधिकत्व इतरस्य व्येकत्वे च योगसाम्यात्~। एवं सप्तकत्रिकाद्ययोरपीत्याद्यवगन्तुं शक्यम्~। सङ्ख्यास्वरूपमात्रेणैव यथैकादिगणनं सेत्स्यति तथा सङ्कलितमपि सेत्स्यति~। यतो गणनमेव हि सङ्कलनमपि~। द्वित्वादिसङ्ख्याविशेषेषु यावत्सङ्ख्यो महान् यावांश्चाल्पः तत्र महत ऊर्ध्वं निरन्तरो यः सङ्ख्याविशेषः ततः प्रभृत्यल्प\renewcommand{\thefootnote}{७}\footnote{त्युत्पन्नसं \textendash\ क. ख. पाठः.}-

\newpage

\noindent सङ्ख्यापर्यन्तं गणिते यावती सङ्ख्या सम्पद्यते तावत्येव हि तयोर्योगसङ्ख्येति~। महत उत्क्रमेणाल्पपर्यन्तं गणिते तदधोगणितः सङ्ख्याविशेषो व्यपकलितस्यापि स्यात्~। गुणनमपि सङ्कलनेनैव सेत्स्यति~। गुणनविधावपि नवान्तानामङ्कानामेव\renewcommand{\thefootnote}{१}\footnote{नवानामङ्कानामेव} घातोऽवधारणीयः~। क्वचिदप्यङ्कानां नवाधिक्याभावात~। ते च {\qt गुण्यस्त्वधोऽधो गुणखण्डतुल्य} इत्याद्युक्तरूपविभागगुणनेनैव
सेत्स्यन्ति~। तत्र गुणकारस्य रूपविभागे यावद्रूपं विभागः कर्तव्यः~। तथा सति गुणतुल्येषु स्थानेषु गुण्ये स्थापिते तद्योग एव गुणितफलं स्यादिति सङ्कलनेनैवैकादीनां नवान्तानामङ्कानां परस्परघाताः समसङ्ख्यघाताश्च सिध्येयुः~। तत्र {\qt सदृशद्वयसंवर्गा एव वर्गा} इति वर्गमूलयोरपि त एव स्थाप्या हेया वा~। इतीदानीं परिकर्मषट्कमुक्तम्~। क्व पुनर्वर्गमूलयोर्विनियोगः~।
भुजाकोटिकर्णेषु त्रिषु द्वयोर्ज्ञातयोरितरज्ञाने तद्विनियोगः~। वक्ष्यति च\textendash 

\begin{quote}
{\qt यश्चैव भुजावर्गः कोटीवर्गश्च कर्णवर्गः सः~।}
\end{quote}

इति~। नन्वत्र वर्गकर्मैव श्रुतं न मूलम्~। कथमत्र वर्गकर्मोक्तम्~। भुजाकोट्योर्वर्गयोगः कर्णवर्ग इत्येतावदेवेहोक्तम्~। इतरदर्थसिद्धम्~। तेन तद्योगमूलं कर्ण इत्यपि सिद्धं स्यात्~। अत्र समुच्चितस्यैव हि कर्णवर्गत्वोक्तेः~। समुच्चयार्थौ हि चकारौ~। गर्गसंहितायां विस्पष्टमेतत्

\begin{quote}
{\qt पृथग्दोःकोटिवर्गाभ्यां कर्णवर्गोऽनुषज्यते~।}
\end{quote}

इति~। अयमर्थः \textendash\ पृथग्भूते\renewcommand{\thefootnote}{२}\footnote{त \textendash\ क. पाठः.} दोःकोटिचतुरश्रक्षेत्रे ये तयोः संश्लेषेण सम्पादितं समचतुरश्रं तत्कर्णतुल्यचतुर्भुजमेवेति~। नन्वेवं वर्गद्वययोगस्य वियोगस्य वा वर्गात्म\renewcommand{\thefootnote}{३}\footnote{त्मि \textendash\ ग. पाठः.}कत्वस्य कादाचित्कत्वात् तस्य वर्गत्वाभावे कथं तन्मूलात्मकः कर्णो भुजाकोट्योरन्यतरो वा स्यात्~। अत्रोच्यते~। तत्रापि तत्कर्णबाहुकसमकर्णचतुर्भुजक्षेत्रफलमेव तथाविधकोटिबाहुवर्गयोगः क्षेत्रफलमेव वर्गवियोगश्च~। तन्मूलं तु न निरवयवम्~। यत एकत्वादिसङ्ख्याविशेषाः सर्वे न वर्गराशयः~। एकादिषु निरन्तरराश्योर्वर्गान्तरमपि~। शून्यात् प्रभृत्येकादिद्विचयं श्रेढीफलमेव~। तथाहि \textendash\ शून्यस्यैकस्य च वर्गान्तरमेकम्~। द्व्येकयोर्वर्गान्तरं पुनस्त्रिसङ्ख्यम्~। द्विकत्रिकयोः पञ्चसङ्ख्यम्~। 

\newpage

\noindent एवमुत्तरोत्तरं सप्तनवैकादशत्रयोदशादिविषमसङ्ख्यं निरन्तरैकादि\renewcommand{\thefootnote}{१}\footnote{द}वर्गान्तरम्~। एवञ्च विरला एव वर्गराशयः~। तद्योगा वियोगा वा ततोऽपि भूयांसः स्युः~। यथैकभुजयोः कर्णवर्गो द्विकः, एकद्विकदोःकोटिवर्गान्तरं त्रिकम्, एकद्विकदोःकोटिवर्गयोरेकचतुष्क\renewcommand{\thefootnote}{२}\footnote{ष्ट\textendash\ क. ख. पाठः.}योर्योगः पञ्च इति दिक्~। तस्मात् ये वर्गराशयस्तन्मूलमेव निरवयवम्~। सावयवत्वेऽप्यवयवानां नेयत्ता ज्ञातुं शक्या~। तत आसन्नमूलमेव तत्र ज्ञातुं शक्यम्~। तदर्थमाह भास्करः\textendash 

\begin{quote}
{\qt वर्गेण महतेष्टेन हताच्छेदांशयोर्वधात्~।\\
पदं गुणपदक्षुण्णच्छिद्भक्तं निकटं भवेत्~॥}
\end{quote}

\noindent इति~। अत्र महता येनकेनचिद्राशिना ह(तं\renewcommand{\thefootnote}{३}\footnote{त\textendash\ ख. पाठः.} मू ? तान्मू)लमानीयते~। अतस्तद्गुणेन हृतमूलं तन्मूलं ज्ञेयम~। तत्र हस्तादे रूपभेदस्य\renewcommand{\thefootnote}{४}\footnote{रूपभूतस्य} यावतिथांश\renewcommand{\thefootnote}{५}\footnote{ग \textendash\ क. ख. पाठः.}ज्ञानेनालंभावः स्यात् तावतो वर्गेण करणी हन्तव्येति महतेत्यनेन सूचितम्~। करणीमूलं च गुणमूलेन हार्यम्~। तत्र फलं रूपात्मकं मूलं शेषोऽंशः~। तद्धारो गुणवर्गमूलतुल्यच्छेदः स्यात्~। एतत् यद्गुणमूलं ज्ञेयं ननु तेनैव वर्गो हन्तव्यः कुतः पुनस्तद्वर्गेण हन्यते वर्गः~। उच्यते~। द्व्यादिगुणोत्तरराशीनां वर्गा\renewcommand{\thefootnote}{६}\footnote{र्गा द्व्यादिवर्गा द्व्या \textendash\ क. पाठः.} द्व्यादिवर्गगुणोत्तरा एव स्युः~। नतु मूलवद् द्व्यादिगुणाः~। तद्युक्तिश्छेद्यके प्रदर्श्या~। एकहस्तमितसमचतुरश्रे तावदेकमेव फलं तद्द्विगुणे हस्तद्वयसमचतुरश्रे तु फलं हस्तचतुष्कम्~। ततः पूर्वबाहोर्द्विगुणे\renewcommand{\thefootnote}{७}\footnote{हौ द्विगु \textendash\ क. ख. पाठः.} बाहौ पूर्वफलाच्चतुर्गुणं फलमिति निर्णीयते~। एवं हस्तचतुष्काष्टकषोडशादिबाहूनां फलान्युत्तरोत्तरं चतुर्गुणानि~। एवं त्रिगुणेत्तर\renewcommand{\thefootnote}{८}\footnote{कत्वा \textendash\ क. पाठः.}बाहूनां क्षेत्राणां फलानि नवगुणोत्तराणि स्युः~। गणितकर्मणाप्येतत् सेत्स्यति~। अभीष्टराशेर्वर्गात् केनचिदिष्टेन गुणितस्य तस्य वर्गः कियद्गुणः स्यादिति ह्यत्र निरूपणीयम्~। तत्राल्पराशिः स्वगुणितः खलु तद्वर्गः~। स एव यावताभीष्टेन गुणितो महान् स च पुनः स्वतुल्येन हन्यते~। तदाल्पवर्गो गुणवर्गहतः स्यात्~। अल्पराशिना चेद्धन्येत तर्हि तद्घातोऽल्पराशिवर्गादिष्टगुण एव स्यात~। न पुनरिष्टवर्गगुणः~। स च न कस्यचिदपि राशेर्वर्गः स्यात्~।
यद्यभीष्टो गुणो वर्गराशिर्न स्यात् तथापि तस्मिन् मूलीकृते पुनर्गुणमूलहतमेव न्यस्तमूलं स्यात्~। एतच्च {\qt भक्तो गुणः शुध्यति येन तेन लब्ध्या च गुण्यो गुणितः फलं वे}त्युक्तखण्डगुणनेनैव सिद्धम्~। युक्तिसाम्यादेवोभयोः~। कथं


\newpage 

\noindent पुनर्युक्तिसाम्यमनयोः~। एवं हि खण्डगुणनयुक्तिः \textendash\ कस्मिंश्चिद्राशौ द्वादशादिभिर्ययोः कयोश्चिद्धातात्मकैर्हन्तव्ये ययोर्घातः स गुणकारस्ताभ्यामेकेन प्रथमं हत्वा हत एव पुनर्द्वितीयेन च हन्यते तदा तयोर्घातगुणितः स्यादित्येतत् सुगमम्~। यदा द्वादशभिर्हन्तव्यो राशिस्तदा तस्य द्वादशकस्य गुणकारराशेस्त्रिकचतुष्काभ्यासरूपत्वात् त्रिकहतो गुण्यराशिः पुनश्चतुष्केण च हतस्त्रिकगुणितादेव चतुर्गुणः स्यादिति पूर्वं त्रिरावृत्तः संश्चतुरावृत्तः क्रियत इति पूर्वगुण्यो द्वादशकृत्वः कृतः स्यात्~। एवमष्टादशादिभि\renewcommand{\thefootnote}{१}\footnote{शभि \textendash\ क. ख. पाठः.}र्हन्तव्येऽपि त्रिकषट्कादिहतो गुण्यराशिरष्टादशादिहतः स्यादिति~। एवमत्रापि कस्यचिद्वर्गेऽन्येन केनचिद्वर्गराशिना हन्तव्ये तन्मूलेन द्विर्हतस्तद्वर्गहतः स्यात्~। तथा कृते सति तस्य गुण्यस्य यन्मूलं गुणकारवर्गस्य च यत् तयोर्घातस्य वर्गः स्यात्~। तद्यथा \textendash\ वर्गीकृतयोः संवर्गे संवर्गे
वर्गीकृते च तुल्यमेव फलं स्यात्~। यत उभयत्रापि गुण्यगुणकाराणां तुल्यत्वमेव स्यात्~। गुणनक्रमभेद एव केवलम~। क्रमभेदाच्च न फलभेदः~। कथम्~। वर्गीकृतयोः\renewcommand{\thefootnote}{२}\footnote{योवर्गेऽपि तैरेव हन्यते \textendash\ क. पाठः.}   संवर्गे प्रथमं वर्गीकार्ययोरेकः प्रथमस्थेनैव\renewcommand{\thefootnote}{३}\footnote{प्रथमं स्वेनैव \textendash\ ख. पाठः.} हन्यते~। पुनरितरेण च पुनरपीती\renewcommand{\thefootnote}{४}\footnote{पीत}तरतुल्याभ्यां द्वाभ्याम्~। तद्वर्गगुणने खण्डगुणनाश्रयात्~। एवं स्वतुल्येनेतरतुल्याभ्यां च द्वाभ्याम्~। एवमेतैस्त्रिभिर्गुणैर्हन्यते~। संवर्गितयोर्वर्गेऽपि तैरेव हन्यते~। एकस्यान्यराशिना हननं हि संवर्गः, इति संवर्गे कृतेऽन्येन हतः स्यात्~। तस्य वर्गीकरणेऽपि पुनः खण्डगुणनन्यायाश्रयेण गुणकारस्यापि स्वतुल्यत्वेन घातात्मक एन सोऽपि स्यात्~। ययोर्घातः स्वयं तयोर्घात एव गु\renewcommand{\thefootnote}{५}\footnote{घातो गु \textendash\ क. ख. पाठः.}णकारोऽपि~। तयोरेकः स्वतुल्यः~। ताभ्यां च हन्यमाने पुनः स्वतुल्येनेतरतुल्येन च हन्यते~। तस्मात् तत्रापि तद्घातगुण्यगुणकारयोरेको गुण्यत्वेन कल्पितः स्वतुल्येन सकृद्धन्यते इतरेण च द्विः~। यथा वर्गीकृतयोः संवर्गे क्रमभेद एव केवलमुभयत्र~। घातस्य वर्गीकरणे प्रथममितरेण हत्वा पुनरपि स्वेनेतरेण च हन्यते~। वर्गयोर्घाते पुनः प्रथमं स्वेन हत्वा पुनरितरेण द्विर्हन्यत इति~। कथं पुनर्गुणने क्रमभेदे फलभेदाभावः~। यथा त्रयाणां राशीनां संवर्गे प्रथमस्य द्वितीयस्य च संवर्गस्तयोरभ्यासः स्यात्~। तत्र द्वितीयो यावान् तावदावृत्तः प्रथमो यः प्रथमोऽपि यावान् तावदावृत्तो द्वितीयोऽपि स एव~। यथा त्रिकचतुष्कयोर्घाते द्वादशको घातश्चतुरावृत्ता त्रित्वसङ्ख्या त्रिरावृत्ता चतु-

\newpage

\noindent ष्ट्वसङ्ख्या च~। एवं घातस्य गुण्यगुणकयोरितरेतरावृत्तत्वात् स एवोभयोरभ्यासश्चोच्यते~। तस्मिन्नभ्यासे पुनस्तृतीयेन केनचिद्राशिना हते यावांस्तृतीयो राशिस्तावदावृत्तः पूर्वोऽभ्यासः स्यात्~। तृतीयश्च पूर्वाभ्यासावृत्तस्तावानेव~। तत्र पुनः प्रथमः प्रथमं तृतीयेन हन्येत~।
पुनर्द्वितीयेन च~। तथापि त्रयाणां घातः स तावानेव स्यात् यावांस्त्रिष्वप्येकैको राशिरितराभ्यासावृत्तः~। यतो द्वयोरभ्यासे कृते परस्परमितरेतरावृत्तौ सन्तौ तौ राशी पुनरन्येन च हतौ पुनरपि तृतीयराशिसङ्ख्ययावृत्तौ स्याताम्~। तेन प्रत्येकं स्वेतरद्वयाभ्यासावृत्तः स्यात्~। इतरावृत्तस्याप्यन्यावृत्तेः~। अतएव {\qt तेन लब्ध्या च गुण्यो गुणित} इत्यत्र क्रमो न विवक्षितः~। तस्माद्गुणनहरणयोः क्रमभेदान्न फलभेदः~। अतएवैकस्मिन् विषयेऽनेकत्रैराशिकसन्निपाते लाघवायाह गोविन्दस्वामी\textendash 

\begin{quote}
{\qt गुणद्वयस्य संवर्गो भागहारद्वयस्य च~।\\
गुणको भागहारश्च स्यातां त्रैराशिकद्वये~॥}
\end{quote}

\noindent इति~। यदि पुनरंशीभूता करणी तदा तच्छेदेनांशं हत्वा पुनर्महता वर्गेण च हन्तव्या~। कुतः~। अंशीभूतो राशिः खलु छेदहतोंऽशिरूपराशिरेव~। यतोंऽशीभूतात् स्वच्छेदेन हृ\renewcommand{\thefootnote}{१}\footnote{ह \textendash\ ख. पाठः.}त्वाप्तं रूपात्मकं फलं स्यात्~।~। 

\begin{quote}
{\qt छेदघ्नरूपेषु लवा धनर्णमेकस्य भागा अधिकोनकाश्चेत्~।}
\end{quote}

\noindent इत्युक्तभागानुबन्धभागापवाहयोरप्येतत् सिद्धम्~। अतः छेदमात्रेण हतायाः करण्याः पुनरपि छेदहनने छेदवर्गहननं कृतं स्यात्~। तस्यां पुनर्महता वर्गेण च हतायां गुणपदच्छेदघातवर्गहता स्यात्~। अत उक्तं{\qt गुणपदक्षुण्णच्छिद्भक्त}मिति~। अत्रापि मूलीकरणे हारकार्धोनशेषस्य परित्यागादर्धाधिके शेषरूपस्य परिपूरणेन परिग्रहाच्चवयवे स्थूलता स्यात~। अत उक्तं निकटमिति~। एवं कृतेऽप्यासन्नमेव मूलं स्यात्~। न पुनः करणीमूलस्य तत्त्वतः परिच्छेदः\renewcommand{\thefootnote}{२}\footnote{दं \textendash\ ग. पाठः.} कर्तुं शक्य इत्यभिप्रायः~। ततो यावदपेक्षमंशानां सूक्ष्मत्वाय महता वर्गेण हननमुक्तम्~। तत्र यावता महता गुणने बुद्धावलंभावः स्यात् तावता हन्यात्~। महत्त्वस्यापेक्षिकत्वात् क्वचिदपि न परिसमाप्तिरिति भावः~। वक्ष्यति च \textendash\ {\qt अयुतद्वयविष्कम्भस्यासन्नो वृत्तपरिणाह} इति~। तत्र व्यासेन परिधिज्ञाने अनुमानपरम्परा स्यात्~। तत्कर्मण्यपि मूलीकरणस्यान्तर्भावादेव तस्यासन्नत्वम्~। तत्सर्वं तदवसर एव प्रतिपादयिष्यामः~। नन्वेवं सति

\newpage

\noindent सर्वत्रापि समचतुरश्रे कर्णस्य करणीगतत्वं स्यात्~। तत् कथं बौधायनेन समकर्णानयनं वर्गमूलीकरणं\renewcommand{\thefootnote}{१}\footnote{मूलीकरणं \textendash\ क. ख. पाठः.} विनाप्युक्तम्~। तेन हि समचतुरश्रबाहौ स्वत्र्यंशं त्र्यंशचतुरंशं च युक्त्वा त्र्यंशतुरीयचतुस्त्रिंशांशे ततस्त्यक्ते कर्णो भवतीत्युक्तम्~। नैष दोषः~। व्यावहारिकत्वात् तस्य~। न पुनर्बौधायनो निरंश\renewcommand{\thefootnote}{२}\footnote{रङ्कुश \textendash\ क. ख. पाठः.}त्वेन तत्कर्णं वक्तुं प्रवृत्तः~। किन्तु क्रतौ शालादिकर्मणि कर्णापेक्षत्वात् यावता तन्निर्वाहः स्यात् तावतोऽपि सूक्ष्मत्वं
स्यादेवास्यापीति न दोषः~। कथं पुनरस्य स्थूलता~। अत्र द्वादशबाहुकं समचतुरश्रं मनसि कृत्वेदं कर्माह भगवान् बौधायनः~। तत्र द्वादशकवर्गे
द्विगुणीकृतेऽष्टाशीत्यधिकं शतद्वयं स्यात्~। तच्च सप्तदशकवर्गादेकोनमेव~। तस्माद्द्विगुणसप्तदशकां(श ?)च्छेदेनैकेनांशेन सप्तदशकाद्धीयते तन्मूलम्~। कथं पुनर्द्विगुणसप्तदशकच्छेदत्वमंशस्य ज्ञायते~। द्विगुणेन वर्गमूलेन हार्यत्वाच्छिष्टस्य गन्तव्यशेषेऽपि न्यायसाम्याच्च~। अतोऽत्र न्यूनस्य रूपस्य चतुस्त्रिंशच्छेदत्वाच्चतुस्त्रिंशांशेन हीनं\renewcommand{\thefootnote}{३}\footnote{नः} सप्तदशकं\renewcommand{\thefootnote}{४}\footnote{कः} कर्ण इति द्वादशके त्र्यंशतत्तुरीयांशौ क्षिप्त्वा तत्तुरीयांशचतुस्त्रिंशांशत्याग उक्तः~। द्वादशसु तत् त्र्यंशभूतेषु चतुर्षु क्षिप्तेषु षोडश सम्पद्यन्ते~। तेष्वपि द्वादशकत्र्यंशस्य चतुष्कस्य तुरीयांशस्यैकत्वात् तस्मिन क्षिप्ते सप्तदश च~। तत एकस्य तत्तुरीयांशस्य चतुस्त्रिंशांशे त्यक्ते आसन्नः कर्णो भवति~। आसन्नत्वं चास्य चतुस्त्रिंशांशवर्गस्य योज्यत्वात्~। पुनस्तस्मात्\renewcommand{\thefootnote}{५}\footnote{स्माच्च \textendash\ ग. पाठः.} चतुस्रिंशांशोनेन सप्तदशकेन मूलेन च द्विगुणेन भागो हर्तव्यः~। तत्रापि तत्फलवर्गं क्षिप्त्वांशीभूतं च तत्फलं पूर्वमूलात् त्याज्यमिति ततोऽपि न्यूनत्वं स्यात्~। पूर्वोक्तवर्गान्तरन्यायेनाप्येतत् सिद्धम्~। यच्चोक्तमेकादिद्विचयत्वं वर्गान्तराणां तेन त्रयोदशकद्वादशकयोर्वर्गान्तरे कार्ये त्रयोदशसङ्ख्यो गच्छः~। तत्र यदन्त्यधनं तद्द्वादशत्रयोदशराश्योर्वर्गान्तरम्~। अन्त्यधनानयनमप्याह भास्करः \textendash\ {\qt व्येकपदघ्नचयो मुखयुक् स्यादन्त्यधनम्} इति~। अत्र व्येकपदं द्वादश~। तद्घ्नो द्विसङ्ख्यश्चयश्चतुर्विंशतिः~। तत्रैकं मुखं च योज्यम्~। तथा सति पञ्चविंशतिसङ्ख्यमन्त्यधनम्~। {\qt तयोर्योगान्तराहतिर्वर्गान्तरं भवेदि}त्यनेनाप्येतत् सिद्धम्~। तत्रापि द्वादशत्रयोदशयोगः पञ्चविंशतिः तदन्तरं चैकं ततस्तयोर्घातोऽपि पञ्चविंशतिरेव~।

\newpage


\noindent रूपेण हतस्य हृतस्य च विशेषाभावात्~। तस्मात् सर्वत्र द्विगुणेन मूलेन समा र्वगगतिरित्येतद्रहस्यम्~। तेन सार्धद्वादशके गच्छे वर्गगतिः पञ्चविंशतिसङ्ख्या~। यथा ज्यान्तरं चापमध्यस्य ज्यागतिरेव नतु चापाद्यन्तयोः, एवमत्रापि~। एवं दिनद्वयस्फुटान्तरमपि तन्मध्यकालस्फुटगतिरेव~। एवं सति सप्तदशके परिपूर्यमाण एवैकस्य रूपस्य वर्गगतिश्चतुस्त्रिंशत्सङ्ख्या, तत ईषदूने ईषदूना चतुस्त्रिंशत्सङ्ख्यैव वर्गगतिरपीति हारकस्य न्यूनत्वाच्च\renewcommand{\thefootnote}{१}\footnote{च्युताच्च \textendash\ क., च्यूत्वाच्च \textendash\ ख. पाठः.}तुस्त्रिंशांशादाधिक्यमिह हेयभागस्य~। एवं सर्वत्राप्यवयवग्रहणं कार्यम्~। कलात्मके राशौ तावद्विकलादिग्रहणाय मूलशेषं षष्ट्या हत्वा द्विगुणेन च मूलेन हृत्वा तत्फलवर्गश्च तत्पराभ्यस्त्याज्यः~। तत्फलं च तत्पूर्वमूलात् केवलादधो विकलास्थाने स्थाप्यम्~। पुनरप्यवयवग्रहणे कार्ये द्विगुणे रूपमूले द्विगुणं फलं क्षेप्यम्~। तत्र प्रथमं विकलीकृतस्य दशमस्था\renewcommand{\thefootnote}{२}\footnote{दशस्था \textendash\ क. ख. पाठः.}नाद्वरणे तत्फलवर्गस्तत्पराशतस्थानात् त्याज्यः~। द्वितीयस्थानवर्गस्य तृतीयस्थानगतत्वस्य पूर्वमेवोक्तत्वात्~। एवं कलादिस्थानात् प्रभृति तृतीयपञ्चमादिविषमस्थानेभ्यस्तत्परादिस्थानेभ्य एव तत्तदवयववर्गः शोध्यः~। अवर्गात् विकलादिस्थानादेव च हरणम्\renewcommand{\thefootnote}{३}\footnote{स्थानादेव हरणम् \textendash\ क. ख. पाठः.}~। तत्र सकृत् दशम\renewcommand{\thefootnote}{४}\footnote{दश \textendash\ ग. पाठः.}स्थानाद्घृते पुनस्तत्फलेऽपि द्विगुणे हारकादधः क्षिप्ते उत्सार्य तेन ह्रियमाणे पुनस्तत्परादिष्वपि विषमस्थानेष्ववर्गात् दशम\renewcommand{\thefootnote}{५}\footnote{दश \textendash\ ग. पाठः.}स्थानात् प्रभृति तद्धरणं कार्यं स्यादित्येव विशेषः~। अस्य युक्तिश्छेद्यके प्रदर्श्या~। तद्यथा \textendash\ वर्गमूलीकरणं नाम पारिकर्म कस्मिंश्चित् समचतुरश्रे
तदवान्तरखण्डेषु तन्मापकहस्तादिबाहुषु समचतुरश्रेषु ज्ञातेषु तत्कोष्ठाश्रयमहाचतुरश्रबाहुज्ञानाय खल्वपेक्ष्यते~। तत्र महाचतुरश्रगतकोष्ठसङ्ख्या मूलीकार्या~। ततस्तत्सङ्ख्याया वर्गात्मिकाया यस्मिन् कस्मिंश्चिद वर्गराशौ विशोधिते विशोधनेनापनीतैः कोष्ठैस्तन्मूलतुल्यबाहुकं समचतुरश्रं समकर्णं सम्पाद्यम्~। पुनरवशिष्टनामपि तत्संश्लेषेण तद्वर्धनं कार्यम्~। तत्र शोधितवर्गमूलतुल्यबाहुषु चतुर्ष्वेककोणस्पृग्बाहुभ्यां बहिः साम्येन वर्धनं कार्यम्~। इतरथा आयतचतुरश्रत्वापत्तेः~। अतो द्विगुणेन तन्मूलेन शिष्टाद्भागे हृते यत् फलं निरन्तरयोरुभयपार्श्वयोरपि तत्तुल्याभिः पङ्क्तिभिर्वर्धनं कृतं स्यात्~। ताश्च पङ्क्तयो दैर्घ्ये शोधितवर्गमूलतुल्या एव~। यतस्तेन द्विगुणेन ह्रियते ततस्तदुभयस्पृक्कोणे तत्फलतुल्यबाहुकं समचतुरश्र\renewcommand{\thefootnote}{६}\footnote{श्रं \textendash\ ख. पाठः.}कोष्ठं शून्यं स्यात्~। तत- 

\newpage

\noindent स्तस्य द्वे एव पार्श्वे शोधितवर्गमूलहृतफलयोगतुल्ये स्याताम्~। इतरे मूलमात्रतुल्ये एव~। तत्कोणे पुनस्तत्फलवर्गतुल्येषु कोष्ठेषु क्षिप्तेषु चत्वारोऽपि बाहवो मूलफलयोगतुल्याः स्युरिति तत्पूरणाय हृतफलवर्गश्च हृतशेषाच्छोध्यते~। एवं मुहुः क्षेत्रं वर्धनीयम्~। यदा पुनर्द्विगुणमूलेन ह्रियमाणे शेषस्याल्पत्वात् रूपफलं न पूर्यते तदा नैकैका पङ्क्तिरुभयोः पार्श्वयोः संश्लेष्या~। अपितु तदावशिष्टानां चतुरश्रफलानां विदारणेनाङ्गुलाद्यंशीकरणं कार्यम्~। तदा तेषां प्रत्येकं हस्तादिमितं दैर्घ्यम् अङ्गुलादिमितो विस्तारः~। एवं चतुर्विंशत्यादिगुणनेनांशीकृताच्छेषात् सम्पन्नचतुरश्रबाहुद्वययोगेन हृते यत् फलं लभ्यते, तत्तुल्याः पङ्क्तयः सम्पन्नबाहुविस्तृति\renewcommand{\thefootnote}{१}\footnote{त \textendash\ ख. ग. पाठः.}दैर्घ्या अङ्गुला\renewcommand{\thefootnote}{२}\footnote{ल्या}दिमितविस्तारा उभयपार्श्वयोः क्षिप्ताः कार्याः स्युः~। तत्राप्यङ्गुलाद्यंशमितसमचतुरश्रकोष्ठैः फलवर्गतुल्यैः समचतुरश्रस्यापूर्णत्वात् तत्फलवर्गश्च शोध्यः~। स च पुनर्नतेभ्य एवांशेभ्यः शोध्यः~। चतुर्विंशांशादिवर्गत्वात् तस्य~। वर्गो हि समचतुरश्रः तेन दैर्घ्येऽप्यङ्गुलमितत्वं कार्यमिति~। पूर्वं विदारितानां पुनश्छेदेनाप्यंशीकरणं कार्यमिति तेषामप्यंशानां चतुर्विंशत्यादिगु\renewcommand{\thefootnote}{३}\footnote{शगु \textendash\ क. ख. पाठः.}णनेनांशीकृतानां तदधोऽवरोपणं कृत्वा तेभ्य एव वर्गः शोध्यः~। एवं कलानां षष्ट्या गुणितानां कलात्मकेन द्विगुणमूलेन हरणं कृत्वा शेषेषु कतिपयानां षष्ट्या गुणनेनावरोपणं कृत्वा तत्पराभ्य एव विकलाफलवर्गः शोध्यः~। तत्परात्मकः खलु विकलावर्ग इति~। एवं यावदपेक्षमवयवं गृहीत्वा अन्ते हारकं दलीकुर्यात्~। तदेवाभीष्टमूलं स्यादिति~। एवं भुजाकोट्योरन्यतरस्यैव वर्गादितरमव(र्गि ? र्गयि)त्वैव द्विगुणीकृत्य तेन भागं हृत्वावाप्तवर्गमपि यथास्थानं विशोध्य फलमपि द्विगुणीकृत्य हारके क्षिप्त्वा तेनापि स्थानान्तरेभ्यो भागहरणादिकमेवं मुहुः कार्यम्~। न पुनरितरस्य वर्गकर्म~। यतस्तयोः संयोगादप्यन्यतरवर्गं विशोध्य कर्म कर्तुं शक्यम्~। अत्र पुनस्तन्न कर्तव्यम्~। ज्ञातत्वादेवेतरमूलस्य~। तत्रेमे श्लोकाः\textendash 

\begin{quote}
{\qt वर्गयोगपदे साध्ये राश्योरल्पस्य वर्गतः~।\\
द्विगुणेनेतरेणैव लब्ध\renewcommand{\thefootnote}{४}\footnote{लभ्य \textendash\ ख. ग. पाठः.}युक्तेन चान्त्यतः~॥\\
हृत्वेह महताप्येनद् यथास्थानं क्षिपेत् फलम्~॥\\
हारके तेन हारेण लभ्यं लब्धं च योजयेत्~॥}
\end{quote}

\newpage

\begin{quote}
{\qt अभितो हरणं भूयः षष्ट्या हत्वा हरन् फलम्~।\\
हारकादध\renewcommand{\thefootnote}{१}\footnote{य \textendash\ क. पाठः.} एव द्विर्द्वितीये प्रथमेऽपि वा~॥\\
विकलासु क्षिपन् हृत्वाप्यन्ते हारो दलीकृतः~।\\
कर्णः स्यात् कर्णवर्गाद्वा दोःकोट्योः कतरेण\renewcommand{\thefootnote}{२}\footnote{य \textendash\ क. पाठः.}चित्~॥\\
द्विघ्नेन लभ्यहीनेन हृत्वा लब्धं च शोधयेत्~।\\
यथास्थानं मुहुश्चैवं तदर्धं च पदं भवेत्~॥\\
बाहुर्वाप्यथ कोटिर्वाप्येवमादीह सूचितम्~।\\
युक्तिसाम्यादतः सारं\renewcommand{\thefootnote}{३}\footnote{र} मूलकर्मेह दर्शितम्~॥}
\end{quote}

\noindent इति~। स्यादेतत्~। एकद्व्यादिवर्गाणां एकादिद्विचयश्रेढीफलतुल्यत्वात् श्रेढीक्षेत्रत्वेनापि कल्पना युक्ता~। न केवलं समचतुरश्रत्वेनैव~। एतच्छ्रेढीविशेषफलानयनवर्गपरिकर्मणोः फलसाम्यं चैवम्~। {\qt आद्यन्तं पदार्धहतमि}ति हि श्रेढीफलानयनमुक्तम्~। अन्त्यधनमिह न्यायसिद्धम्~। कथम्~। मुखस्य चयतुल्यत्वाभावाद्गच्छादेकं विशोध्य शिष्टे द्विगुणीकृतेऽन्त्यधने यश्चयांशः स स्यादितरांशो मुखमेवेति पुनः रूपं च योज्यम्~। अतएवाह भास्करः \textendash\ {\qt व्येकपदघ्नचयो मुखयुक् स्यादन्त्यधनम्}~। इति~। अत्र मुखस्यैकत्वनियमात् सदा रूपमेव मुखत्वेन क्षेप्यम्~। तत्र पुनराद्याख्यं\renewcommand{\thefootnote}{४}\footnote{ख्य \textendash\ क. ख. पाठः.} मुखमपि क्षेप्यम्~। तच्चात्र रूपमेव~। अतो गच्छादेकं विशोध्य द्विगुणीकृत्य रूपे द्विः क्षिप्ते गच्छः कृत्स्न एव द्विगुणीकृतः स्यात्~। एवमाद्यन्तैक्यस्य द्विगुण\renewcommand{\thefootnote}{५}\footnote{द्विगुणित \textendash\ ख. ग. पाठः.}गच्छतुल्यत्वात् तस्मिन् गच्छार्धेन हते यत् स्यात् तदेव गच्छेन गच्छे हतेऽपि स्यात्~। केवलयोर्गच्छयोः परस्परं गुणने कार्येऽपि गुण्यगुणकयोरेकं द्विगुणीकृत्यान्यदर्धीकृत्य च गुणने कृते फलस्य विशेषाभावात् श्रेढीक्षेत्रेऽप्यस्मिंश्चतुरश्रक्षेत्रेऽपि\renewcommand{\thefootnote}{६}\footnote{क्षेत्रे} फलसाम्यमेवम्~। श्रेढीक्षेत्रं ह्येतदेवमाकारम्~। आद्यपङ्क्तावेकमेव खलु मापकबाहुकं समचतुरश्रम्~। ततः प्रभृति द्विचयत्वात् द्वितीयादिषु त्रि\renewcommand{\thefootnote}{७}\footnote{द्वि}त्वादिविषमसङ्ख्यातुल्यानि\renewcommand{\thefootnote}{८}\footnote{न्येव या}~। एवं यावद्गच्छपर्यन्तम्~। तत्रान्त्यपङ्क्तिगतं फलं द्विगुणगच्छादेकोनम्~। तत्र यो गच्छादति\renewcommand{\thefootnote}{९}\footnote{गच्छाति \textendash\ क. पाठः.}रिक्तोंऽश एकोनगच्छतुल्यस्तस्मिन्नाद्यपङ्क्तौ क्षिप्ते सा चान्त्या च गच्छ-

\newpage
\noindent तुल्या~। एवमुपान्त्येऽपि गच्छादतिरिक्तो योंऽशः स त्र्यूनगच्छतुल्यः~। पूर्वतो द्व्यूनत्वात् तस्य~। तस्मिन द्वितीयपङ्क्त्यां क्षिप्ते सा च गच्छतुल्या स्यात्~। एवमितरेषामपि सर्वेषां गच्छतुल्यदैर्घ्यत्वात् विस्तारस्य च मिथो योगाद्गच्छतुल्यत्वात् समचतुरश्रत्वं सम्पादनीयं श्रेढीक्षेत्रफलैरेव~। अत्रान्त्यधनानयनमार्यापूर्वांशोक्तेष्टधनानयनेनैव वा सिद्धम्~। तस्यायमर्थः \textendash\ अभीष्टे गच्छे यावतीनां पङ्क्तीनां निरन्तराणां फलं जिज्ञास्यते तत्सङ्ख्येहेष्टशब्देनोक्ता~। तस्मादत्रान्त्यस्यैकस्यैवेष्टत्वम्~। तस्मिन् व्येके शून्यतामापद्यते~। तेन दलितेऽपि न विशेषः~। ततः प्राक् याः पङ्क्तयः ता एव पूर्वशब्देनोक्ताः~। अन्त्ये जिज्ञासिते एकोनगच्छस्यैव पूर्वत्वं तत्सहिते शून्ये {\qt योगे खं क्षेपसममि}त्येकोनगच्छ एव स्यात्~। तस्मिन्नुत्तरेण द्वयेन गुणिते मुखेन रूपेण च सहिते इष्टमध्यगतफलं स्यात्~। तस्मिन्निष्टगुणिते पुनरिष्टधनमपि स्यात्~। तेन गुणिते न विशेष इति द्विघ्नपदं व्येकमेवात्रान्त्य\renewcommand{\thefootnote}{१}\footnote{वान्त्य \textendash\ ख. पाठः.}धनमिति~। इष्टेष्वाद्या\renewcommand{\thefootnote}{२}\footnote{द्य \textendash\ ग. पाठः.}न्त्ययोरैक्यमेवाद्यन्तमपि~। तदिष्टार्धेन हतं वेष्टधनम्~। एवमन्याकारतयापि वर्गस्य कल्प्यत्वात् वर्गः समचतुरश्र इत्युक्तेर्विषयसङ्कोचः स्यात्~। तेन विश्वतोमुखत्वमपि हीयेत\renewcommand{\thefootnote}{३}\footnote{हीयते}~। इष्यते हि\renewcommand{\thefootnote}{४}\footnote{च हि} विश्वतोमुखत्वमपि सूत्राणाम्~।} {\qt अल्पाक्षरमसन्दिग्धं सारवत् विश्वतोमुखमि}ति हि सूत्रलक्षणं वदन्ति सन्तः इति~। नैष दोषः~। अस्य भुजाकोटिकर्णक्षेत्रविषयत्वात्~। भुजाकोटिकर्णेषु ज्ञातयोर्द्वयोरितरज्ञानार्थमेव ह्यत्र वर्गमूलपरिकर्मणी\renewcommand{\thefootnote}{५}\footnote{णी \textendash\ क. ख. पाठः.} प्रोक्ते~। एतन्न्यायेन त्रैराशिकन्यायेन च व्याप्तमेव हि सकलं ग्रहगणितम् इति तदर्थमेवेह तत्प्रदर्शनम्~। तत्र च चतुरश्राकाराणि भुजाकोटिकर्णक्षेत्राणि परिकल्पनीयानि~। तद्युक्तिमुत्तरत्र स्वावसरे प्रदर्शयिष्यामः~॥~४~॥\\

एवं परिकर्मषट्कं प्रदर्श्य सप्तमस्य घनीकरणस्य स्वमूलवैपरीत्येन सिद्धिं मन्वानोऽष्टमं घनमूलीकरणं प्रदर्शयितुमाह\textendash 

\begin{quote}
{\ab अघनाद्भजेद्द्वितीयात् त्रिगुणेन घनस्य मूलवर्गेण~।\\
वर्गस्त्रिपूर्वगुणितः शोध्यः प्रथमाद् घनश्च घनात्~॥~५~॥}
\end{quote}

इति~। अत्र घनस्थानमेकं ततो द्वे अघने~। एवं स्थानत्रिकेषु सर्वेषु प्रथमं घनाख्यम् इतरद्\renewcommand{\thefootnote}{६}\footnote{रद्व \textendash\ ख. ग. पाठः.} द्वयमघनाख्यम् इत्येतदत्रैव सिद्धं द्वितीयादघ-

\newpage

\noindent नात् प्रथमादघनात् घनादिति~। एषु किं कार्यमित्याकाङ्क्षायामाह \textendash\ द्वितीयादघनाद् भजेत्~। केन~। घनस्य मूलवर्गेण त्रिगुणेन~। प्रथमादघनात् पुनर्वर्गस्त्रिपूर्वगुणितः शोध्यः~। कस्य वर्गः~। भजनानन्तर्याद् भङ्क्त्वा लब्धस्य~। घनश्च घनात् घनस्थानात् घनश्च शोध्यः~। इतीह नियम्~। कस्य घनः~। हरणानन्तरं हृतफलस्य घनः~। प्रथममन्त्यादेव घनात्~। कुतः~। घनस्य मूलवर्गेणेत्युक्तत्वात्~। घने शुद्ध एव शुद्धस्य घनस्य मूलवर्गेण त्रिगुणेन द्वितीयादघनात् भजनं विवक्षितम्~। एतदुक्तं भवति \textendash\ मूलीकार्यस्य घनराशेरन्त्यत्रिकस्याद्यस्थानात् यावतां घनः शोध्यः तस्मिन्नेकाष्टसप्तविंशति चतुष्षष्ट्यादिष्वन्यतमे शुद्धे तस्य मूलस्यैकादिनवान्तेष्वन्यतमस्य वर्गेणैकाद्येकाशीत्यन्तेष्वन्यतमेन त्रिगुणेन तदधःस्थानात् द्वितीयादघनात् भागं विभजेत्~। तत्र लब्धस्य वर्गः त्रिगु\renewcommand{\thefootnote}{१}\footnote{त्रिभिर्गु \textendash\ ख. ग. पाठः.}णितः सन् पूर्वेण घनमूलेन च गुणितः प्रथमादघनाच्छोध्यः~। द्वितीयादघनाद्धरणेन यल्लब्धं तस्य घनः शोधनस्थानादधो यद्घनस्थानं ततः शोध्यः~। एवं मुहुरिति~। भास्करश्चाह\textendash 

\begin{quote}
{\qt आद्यं घनस्थानमथाघने द्वे पुनस्तथान्त्याद् घनतो विशोध्य~।\\
घनं पृथक्स्थं पदमस्य कृत्या त्रिघ्न्या तदाद्यं विभजेत् फलं तु~॥\\
पङ्क्त्यां न्यसेत् तत्कृतिमन्त्यनिघ्रीं त्रिघ्नीं त्यजेत् तत्प्रथमात् फलस्य~।\\
घनं तदाद्याद् घनमूलमेवं पङ्क्तिर्भवेदेवमतः पुनश्च~॥}
\end{quote}

\noindent इति~। अत्रापि शोधितघनानां मूलं निरन्तरमधोऽधः स्थाप्यम्~। यदा पुनर्घनशोधनाधःस्थानात् तस्य मूलवर्गेण त्रिगुणेन हर्तुं न शक्यते तदा पुनस्तदधस्तनत्रिकेऽप्यन्त्यादेव द्वितीयादघनान्मूलवर्गेण त्रिगुणेन भागं हरेत्~। त्रिपूर्वगुणितफलवर्गशोधनमपि तदधःफलस्य घनशोधनमपि घनस्थानाद्यथा कार्यं तावदेव हरणेऽपि सर्वत्र फलं ग्राह्यम्~। येषु स्थानेषु हरणं न कृतं तत्र तत्र शून्यचिह्नमेव स्थापयेदिति~। अतएव विपरीतन्यायेन घनीकरणमपि सिद्धं {\qt विपरीते विपरीतं न्याय्यमि}\renewcommand{\thefootnote}{२}\footnote{विपरीतन्यायमि \textendash\ क. ग. पाठः.}ति~। इहान्ते प्रथमस्यानात् घनं विशोध्य तन्मूलं स्थाप्यते~। तेन घनीकरणे प्रथमं प्रथमस्थानघन आद्ये स्थाने स्थाप्यः, मूले आद्यस्थानात् घन\renewcommand{\thefootnote}{३}\footnote{नं}विशोधनात्\renewcommand{\thefootnote}{४}\footnote{धयेत् \textendash\ क. पाठः.}~। प्रागाद्यस्थानवर्गस्य त्रिपूर्वगुणितस्य शोधनमुक्तमिति प्रथमस्थानघनस्थापनानन्तरं तद्वर्गोऽन्त्यत्रिगुणितः 

\newpage

\noindent स्थाप्यः~। घनकर्मणि\renewcommand{\thefootnote}{१}\footnote{नपूर्वणि} पूर्वस्येहान्त्यत्वात्~। ततः प्राक् पूर्वलब्धघनमूलपङ्क्तिवर्गेण त्रिगुणेन हरणं कृतम्, इह तत्फलभूतस्याद्यस्थानाङ्कस्य ज्ञातत्वात् तद्धतो द्वितीयाद्यङ्कवर्गस्त्रिगुणितः\renewcommand{\thefootnote}{२}\footnote{ण \textendash\ क. पाठः.} क्षेप्यः~। तच्छोधनमेव हरणमपीति~।
\begin{quote}
{\qt भाज्याद्धरः शुध्यति यद्गुणः स्यादन्त्यात् फलं तत् खलु भागहारे~।}
\end{quote}
\noindent इति हि भागहरणमुक्तम्~। ततो विपरीतकर्मणि हारकः फलेन हन्तव्यः~। पुनस्तदुपरि चतुर्थस्थाने द्वितीयस्थानघनश्च स्थाप्य इति~। अन्त्याङ्कात् प्रभृति वा घनीकरणं कार्यम्~। तदा तत्क्रमस्य\renewcommand{\thefootnote}{३}\footnote{कर्मक्रमस्य \textendash\ ग. पाठः.} वैपरीत्यं न स्यात्~। गुणनहरणयोः क्षेपशोधनयोरेव वैपरीत्यम्~। तस्मान्मूले प्रथमं घनविशोधनमुक्तम्~। ततो घनीकरणेऽपि प्रथममन्त्यस्थानस्य घनः क्वचिद्देयः~। ततस्तद्वर्ग(मूलः?) त्रिगुणस्तदधःस्थानाङ्कगुणितो न्यस्तघनाधस्थानात् प्रभृति न्यस्तव्यः~। तत्र शोधितघनमूलवर्गेण त्रिगुणेन हरणस्योक्तत्वादत्र त्रिगुणितः पूर्वमूलवर्गः तत्फलेन तदधःस्थानगताङ्केन हतो द्वितीयेऽघने क्षिप्यते~। स एव मूले ततस्त्यज्यत इति~। ततोऽधःस्थानाङ्कवर्गः त्रिगुणितस्तदूर्ध्वगतघनमूलहतः प्रथमेऽघने स्थाप्यः, वर्गस्त्रिपूर्वगुणितः शोध्यः प्रथमादित्यस्य वैपरीत्याय~। अथोपान्त्यघनस्तदधःस्थाने स्थाप्यत इति~। तथाच भास्करः\textendash 

\begin{quote}
{\qt समत्रिघातश्च घनः प्रदिष्टः स्थाप्यो घनोऽन्त्यस्य ततोऽन्त्यवर्गः~।\\
आदित्रि\renewcommand{\thefootnote}{३}\footnote{स्त्रि \textendash\ क. पाठः.}निघ्नस्तत आदिवर्गस्त्र्यन्त्याहतोऽथादिघनश्च सर्वे~॥\\
स्थानान्तरत्वेन युता घनः स्यात् प्रकल्प्य तत्खण्डयुगं ततोऽन्त्यम्~।\\
एवं मुहुर्वर्गघनप्रसिद्ध्यै\renewcommand{\thefootnote}{*}\footnote{'द्वावाद्या' इति मुद्रितलीलावतीपाठः~।} आद्याङ्कतो वा विधिरेष कार्यः~॥}
\end{quote}

\noindent इति~। का पुनरत्र स्थानान्तरत्वेन योजने युक्तिः~। उच्यते~। अन्त्याङ्कस्य घनो यावतिथे स्थाने स्थाप्यः ततोऽधःस्थाने अन्त्याङ्कवर्ग उपान्त्यहतः स्थाप्यः~। यतो घनोऽप्यन्त्यस्थानवर्गोऽन्त्याङ्कहतः, ततोऽन्त्यवर्ग एवोपान्त्यहत एकेनैव स्थानेनापकृष्टः स्यात्~। उपान्त्यस्यैव त्रिष्वप्येकेन स्थानेन न्यूनत्वं नेतरयोः~। सदृशावन्त्याङ्कावेव हीतरौ~। ततस्तयोर्घातादन्त्याङ्कहतादेकेनैव स्थानेनोपान्त्यहतस्य न्यूनत्वं स्यादिति निर्णीयते~। स्थाप्यो घनोऽन्त्यस्य ततोऽन्त्यवर्ग आदित्रिनिघ्न इत्युक्तस्य\renewcommand{\thefootnote}{५}\footnote{क्तः स्वस्था \textendash\ क. ख. पाठः.} स्थानात् तत आदिवर्गस्त्र्यन्त्याहत 

\newpage

\noindent इत्युक्तस्य स्थानमप्येकेनैव स्थानेनापकृष्टं स्यात्~। अन्त्यवर्गादुपान्त्यवर्गस्य स्थानद्वयापकृष्टत्वात्~। अन्त्याङ्कस्य चापकर्षाभावात्~। तस्मात् घनस्थानादेकान्तरिते अघने प्रथम एव तत्स्थापनं युक्तम् इत्येषु तृतीयस्यापि द्वितीयादेकान्त\renewcommand{\thefootnote}{१}\footnote{याङ्कान्त}रितत्वं युक्तम्~। तत उपान्त्यघनस्याप्येकान्तरितत्वमेव~। अन्त्याङ्कघनात् स्थानद्वयान्तरितत्वात् उपान्त्यघनस्य~। कथं पुनर्निरन्तराङ्कयोर्घनयोरन्तरं स्थानद्वयं स्यात्~। उच्यते~। अन्त्याङ्कस्य त्रिषु स्थानेषु स्थापितस्येतरेतरं हनने तत्स्थान\renewcommand{\thefootnote}{२}\footnote{तव स्थान \textendash\ क. ख. पाठः.}सङ्ख्यायास्त्रिगुणाया द्व्यूनैव घनस्थानसङ्ख्या स्यात्~। 
यतः स्था\renewcommand{\thefootnote}{३}\footnote{सस्था \textendash\ ग. पाठः.}नाङ्कयोर्घाते तयोरङ्कयोर्यावन्ति शून्यानि तान्येव द्विगुणानि स्युः~। न पुनश्चरमस्य स्थानस्य द्विगुणत्वम्~। यतस्तच्छून्यानामध इतरस्यापि
शून्यानि स्थापयित्वा शून्योपरि स्थिताङ्कस्थान एव तद्धतिः स्थाप्यत इत्यन्त्यस्थानस्य द्विगुणत्वाभावात् द्विगुणस्थानादेकोनत्वं घातस्थानस्य~।
पुनस्तृतीयहननेऽपि तच्छून्यानामेवाधःस्थापनं न पुनः पूर्वं गुणितोऽङ्क उत्कृष्यत इति तत्रैव तध्दातः स्थाप्यत इति~। संवर्गयोर्द्वयोरप्येकैकोनत्वात्~। सदृशत्रयसंवर्गे संवर्ग्य\renewcommand{\thefootnote}{४}\footnote{र्ग \textendash\ क. ख. पाठः.}स्थानात् त्रिगुणाद् द्व्यूनत्वमन्त्यस्थानस्य~। ततोऽन्त्योपान्त्यघनयोरन्तराले द्वे स्थाने स्तः~। ते उभे अप्यघनाख्ये~। घनस्थापनायोगात्~। एवं ततोऽप्येकैकोनानाम् अङ्कानां घनस्थानानि द्व्यन्तराणि स्युः~। ततः स्थानत्रिकेषु प्रथममेव घनस्यानम् अन्ये चाघने इति नियमोऽस्त्येव~। एवमन्त्यघनस्य तद्वर्गोपान्त्यघातस्य उपान्त्यवर्गान्त्याङ्कघातस्योपान्त्यघनस्य च क्रमादेकैकोनस्थानत्वं युक्तमिति~। एवमाद्याङ्कघनतद्वर्गद्वितीयाङ्कघातादी नामप्याद्यात् प्रभृत्येकैकस्थानोत्कर्षो विज्ञेयः~। कथं पुनरत्र\renewcommand{\thefootnote}{५}\footnote{त्र च स \textendash\ ग. पाठः.} सदृशानामितरेतरं गुणने च सङ्ख्यासाम्यं स्यादिति~। तद्युक्तिः खण्डवर्गद्वारा निरूप्या~। खण्डवर्गे हि वर्गस्य चत्वारः खण्डाः स्युः~। गुण्यगुणकयोरुभयोरपि द्वेधा खण्डनात्~। तत्र खण्डयोर्वर्गौ द्वौ~। घातावपि द्वौ~। वक्ष्यति च\textendash 

\begin{quote}
{\qt सम्पर्कस्य हि वर्गाद् विशोधयेदेव वर्गसम्पर्कम्~।\\
यत्तस्य भवत्यर्धं विद्याद् गुणकारसंवर्गम्~॥}
\end{quote}

\noindent इति~। तत्र सम्पर्क एको राशिः~। तत्खण्डावितरौ~। तत्र सम्पर्कस्य कृत्स्नस्य राशेर्वर्गात् वर्गसम्पर्के विशोधिते खण्डवर्गौ द्वौ\renewcommand{\thefootnote}{६}\footnote{द्वौ वर्गाव \textendash\ क. ख. पाठः.}भागावपास्तौ स्तः~। शिष्ट-

\newpage

\noindent स्यार्धे च द्वौ भागौ~। एवं चत्वारः खण्डाः~। तत्र शेषस्यार्धं तयोः खण्डयोर्घात एव~। तदुक्तं \textendash\ यत्तस्य भवत्यर्धं विद्याद्गुणकारसंवर्गम् इति~। संवर्ग द्वयोरपीतरेतरापेक्षया गुणकारत्वात् तौ गुणकारावुक्तौ~। अर्धं तयोः खण्डयोः संवर्गं विद्यादित्यर्थः~। इत्युक्तखण्डवर्गन्यायेन वर्गं चतुर्धा विभज्य चतुर्णां मूलराशिना गुणनेऽपि तदैक्यं घनतुल्यं स्यादित्येतत् सुगमम्~। यतो वर्ग एव तन्मूलहतो घनः~। चतुरः खण्डान् पृथक् पृथङ्मूलेन निहत्य योजनेऽपि कृत्स्नस्य वर्गस्य कृत्स्नेन मूलेन हननं स्यादिति,

\begin{quote}
{\qt गुण्यस्त्वधोऽधो गुणखण्डतुल्यस्तैः खण्डकैः सङ्गुणितो युतो वा~।}
\end{quote}

\noindent इत्यनेनाप्युक्तम्~। यथा वर्गे गुण्यगुणकयोस्तुल्यता एवं घनेऽपि त्रयाणां तुल्यतया खण्डनं कृत्वा निरूप्यम्~। तत्र वर्गस्य चतुर्षु खण्डेषु यौ द्वौ वर्गात्मकौ यौ च संवर्गात्मकौ तौ द्विकावपि द्वाभ्यां\renewcommand{\thefootnote}{१}\footnote{भ्यां गु \textendash\ क. पाठः.} खण्डाभ्यां गुणनीयौ~। तत्राष्टौ खण्डाः स्युः~। तत्राल्पखण्डवर्गस्य तत्सदृशेन गुणने अल्पखण्डघनतुल्यत्वं स्यात्~। एवं महतो वर्गस्य स्वसदृशखण्डगुणनेऽपि~। एवं द्वौ खण्डौ तत्तद्धनेनैव परिगृहीतौ स्याताम्~। ये पुनरितरे षट् खण्डास्तेषु द्वौ खण्डवर्गावितरखण्डहतौ तयोर्घनेनापरिगृहीतत्वात्~। स्वसदृशखण्डहतावेव हि घनेन परिगृहीतौ~। नेतरखण्डहतौ~। तत्राल्पखण्डवर्ग इतरखण्डहतः खण्डयोर्घातोऽल्पखण्डहत एव~। गुणने क्रमभेदेन फलभेदाभावात्~। प्रथममल्पखण्डं महता खण्डेन हत्वा पुनरल्पेन च हते, प्रथममल्पखण्डं स्वसदृशेन हत्वा महता च हतेऽपि फलसाम्यं स्यात्~। अत्रापि संवर्गस्य चाल्पखण्डहननेऽल्पखण्डवर्ग इतरखण्डहत एव स्यात्~। घाते पुनरल्पेन हन्यमानेऽपि क्रमभेद एव स्यात् अल्पं महता हत्वा स्वसदृशेन हन्यत इति~। तस्मात् घात एवाल्पखण्डहत एकः खण्डः~। महतो वर्गोऽल्पेन हतश्च द्वयोर्घातो महता हत एव~। तस्मात् खण्डयोर्घातोऽल्पेन हतो महता हतश्च द्वौ खण्डौ~। ये पुनश्चत्वारोऽवशिष्टास्तेष्वप्यल्पखण्डहतघाततुल्यौ द्वौ\renewcommand{\thefootnote}{*}\footnote{इत ऊर्ध्वं महाखण्डहतघाततुल्यौ च द्वौ~। इत्यपि योज्यं प्रतिभाति~।}~। यतो वर्गस्य खण्डेषु घाततुल्यौ द्वावेव खण्डाववशिष्टौ~। इतरयोः कृत्स्नराशिना हतयोः 

\newpage

\noindent खण्डघनाभ्यां खण्डहतघाताभ्यां च परिगृहीतत्वात्~। तत्र शिष्टौ घातौ कृत्स्नेन राशिना हन्तव्यौ~। तत्र खण्डगुणनन्यायेन खण्डाभ्यां पृथक् पृथक् सङ्गुणय्य संयोजने क्रियमाणे शेषश्च परिगृहीत एव स्यात्~। तत्र घातस्य द्वाभ्यां खण्डाभ्यां पृथक् पृथग्गुणनं कार्यम्~। तत्राल्पखण्डेन हतौ घातौ द्वौ~। महता खण्डेन च हतौ द्वौ~। तत्तुल्यावेव च पूर्वमपि परिगृहीतौ~। एवमल्पेन खण्डेन हतो घातस्त्रिगुणीकार्यः~। महता खण्डेन हतश्च~। यद्वा अल्पखण्डवर्गो महता खण्डेन हत एव वा त्रि\renewcommand{\thefootnote}{१}\footnote{एव त्रि}गुणीकार्यः~। म(हता ? हा)खण्डवर्गोऽल्पखण्डहतश्च~। खण्डवर्गस्येतरखण्डहतस्य तद्वर्गपदात्मकखण्डस्य घातहतस्य च तुल्यत्वस्योपपादितत्वात्~। गु\renewcommand{\thefootnote}{२}\footnote{त्~। त्रिगु}णनमपि वर्ग एव वा कार्यम्~। गुणने क्रमभेदेन फलभेदाभावस्योक्तत्वात्~। तस्मादल्पवर्गे त्रिभिर्हते महता च हतेऽष्टसु त्रयः खण्डाः परिगृहीताः स्युः~। महतो वर्गेऽपि त्रिभिरल्पेन च हते त्रयः~। खण्डघनाभ्यामपि द्वौ~। एवमष्टानां खण्डानां परिग्रहेण\renewcommand{\thefootnote}{३}\footnote{हणे \textendash\ क. पाठः.} घनः कृत्स्न एव सम्पद्यते~। तस्मात् {\qt स्थाप्यो घनोऽन्त्यस्य ततोऽन्त्यवर्गः आदित्रिनिघ्नस्तत आदिवर्गस्त्र्यन्त्याहतः आदिघनश्चे}ति चतुर्षु स्थानेषु स्थाप्येषु अघनयोः स्थानयोस्त्रयस्त्रयः खण्डाः स्थाप्यन्ते~। घनस्थानयोश्च खण्डघनतुल्यौ इत्यन्योन्यहनने घनीकरणेऽपि फलसाम्यं सिद्धम्~। अत्रान्त्योपान्त्याङ्कद्वयमेकः खण्डः शेषोऽन्य इति द्वेधा विभज्य खण्डयोरुभयोर्घनीकरणं त्रिघ्नखण्डवर्गयोः खण्डान्तरहननं च कार्यम्~। तत्राप्यन्त्योपान्त्याङ्कद्वयात्मकः खण्डः स्थानविभागेन द्वौ खण्डौ क्रियेते~। तयोः खण्डयोर्घननेऽपि तयोर्घनयोस्तन्मध्ये च त्रिगुणवर्गयोर्मिथो हतयोः स्थानद्वये स्थापनादन्त्योपान्त्याङ्कघनः स्थानचतुष्कग\renewcommand{\thefootnote}{४}\footnote{ह \textendash\ क. पाठः.}तः~। एवं प्रथमं कल्पितयोः खण्डयोरन्त्योपान्त्यात्मकस्य खण्डस्य घनः परिगृह्यते~। इतरखण्डादप्यनयोरधोगतं तेषु चोर्ध्वगतमङ्कं पृथग्गृहीत्वा तस्य चान्त्योपान्त्याङ्कद्वयखण्डस्य च घनौ, घातहतौ त्रिगुणखण्डवर्गौ च विभज्य स्थानत्रयघनः सम्पा\renewcommand{\thefootnote}{५}\footnote{म्प \textendash\ क. पाठः.}द्यते~। तत्र स्थानद्वयात्मकस्य खण्डस्य घनः पूर्वमेव परिगृहीतः~। ततस्तद्वर्गे त्रिगुणितेऽन्येन च हतेऽन्यवर्गे च त्रिगुणीकृतेऽन्त्येन स्थानद्वयेन च हते इतरखण्डस्य घने च स्थापिते स्थानत्रयं घनीकृतं स्यात्~। एवं पुनःपुनरपि तत्तदधोगतं स्थानं पृथ-

\newpage

\noindent गादाय तत्खण्डस्य घनीकृतखण्डस्य च घनाभ्यां त्रिघ्नवर्गेतराभ्यासाभ्यां च कृत्स्नस्य घनः परिगृह्यते~। इति घनस्य युक्तिरतिविशदं प्रदर्शिता~। एषेव क्षेत्रकल्पनयापि प्रदर्श्या~। तद्यथा \textendash\ समद्वादशाश्रस्य कस्यचिद्घनक्षेत्रस्याश्राणां तुल्यतया त्रेधा खण्डनं कृत्वा अष्टौ खण्डाः पृथक्कृत्य प्रदर्श्याः~। तच्चोदाहरणपुरःसरं प्रदर्शयिष्यामः~। तत्र नवविस्तृतिदीर्घपिण्डे द्वादशाश्रे तावत् प्रदर्श्यते~। तत्र नवसङ्ख्यस्य बाहोश्चतुस्सङ्ख्य एकः खण्डः~। इतरः पञ्चसङ्ख्यः~। तत्र भूस्पृष्टादेककोणात् प्रभृति
त्रिष्वप्यश्रेषु हस्तचतुष्कमितेऽङ्कं कृत्वा विभक्ते सत्यष्टौ खण्डाः स्युः~। तत्र द्वौ समद्वादशाश्रौ~। तयोरेकश्चतुर्हस्तविस्तृतिदीर्घपिण्डः~। इतरः
पञ्चहस्तविस्तृतिदीर्घपिण्डः~। स चान्यकोणप्रतियोग्यूर्ध्वकोणगतः~। शिष्टेषु षट्सु खण्डेषु चतुस्सङ्ख्यस्योर्ध्वगत एकः इतरौ भूस्पृष्टौ तन्निरन्तरपार्श्वगतौ च इत्येतत्त्रयं समपरिमाणं समाकारं च~। इतर त्रयमपीतरेतरं समाकारं तेष्ववशिष्टभूगत एकः, पार्श्वगतयोरुपरिभागावितरौ~। अवशिष्टः पञ्चहस्तो द्वादशाश्रा न भू\renewcommand{\thefootnote}{१}\footnote{श्रौ भू}स्पृष्ठः~। तेषु शयानः खण्डस्तस्याधारः~। 

\begin{quote}
{\qt समद्वादशबाहौ तु विभक्ते च घने त्रिधा~।\\
युक्तिर्बोध्या विभागाय पृष्ठे रेखाद्वयं लिखेत्~॥\\
पूर्वापरायतं ह्येकमन्यद् याम्योत्तरायतम्~।\\
अल्पखण्डान्तरे सौम्याद् याम्याच्च महदन्तरे~॥\\
तथैव प्रत्यगश्राच्च प्रागश्राच्च यथाक्रमम्~।\\
अल्पखण्डोच्छ्रिते रेखाः कुर्यात् पार्श्व चतुष्टये~॥\\
विदारिते च तैर्मार्गैरष्टौ खण्डा भवन्ति हि~।\\
अल्पखडघनो\renewcommand{\thefootnote}{२}\footnote{नौ \textendash\ क. पाठः.} वायौ भूगतो द्वादशाश्रकः~॥\\
ततः प्राग्याम्ययोः खण्डावूर्ध्वगश्च समास्त्रयः~।\\
अल्पखण्डोच्छ्रिती द्वौ तु महाखण्डोच्छ्रितिः परः~॥\\
ऊर्ध्वभागेऽग्निकोणे यः खण्डः स महतो घनः~।\\
तदधोगत एकः स्यादुदक्पार्श्वगतः परः~॥\\
प्रत्यक्पार्श्वगतोऽन्यश्च त्रय एते मिथः समाः~।\\
षडेतेनैव खण्डाः स्युः समद्वादशबाहवः~॥}
\end{quote}

\newpage

\begin{quote}
{\qt खण्डयोः समताभावात् तत्समत्वे समा भुजाः~।\\
विषमे द्वादशाश्रेऽपि पार्श्वयोस्तु मिथः समम्~॥\

फलमूर्ध्वमधश्चापि षट्सु पृष्ठफलेषु तु\renewcommand{\thefootnote}{१}\footnote{ता \textendash\ ख. पाठः.}~। \\
मिथः प्रतिदिशोस्तुल्यं त्रिविधं स्यात् फलं ततः~॥

विस्तारायामपिण्डेषु वध एव द्वयोर्द्वयोः~।\\
विस्तारायामयोर्घात उपरिष्टात्तलेऽपि च~॥

विस्तारोच्छ्रितिघात स्याद्ध्रस्वयो पार्श्वयोर्द्वयोः~। \\
आयामोच्छ्रितिघातः स्याद् दीर्घयोः पार्श्वयोर्द्वयोः~॥

त्रिष्वेकमितरेणापि हतं घनफलं भवेत्~।\\
तदत्र त्रिषु तुल्येषु पूर्वोक्तेषु घनाप्तये~॥ 

महता हन्यतेऽल्पस्य वर्गः खण्डस्य च त्रिषु~। \\
अल्पखण्डघनेनैषां सह सन्धीयते तु यः~॥
 
भागस्तत्फलमल्पस्य वर्गतुल्यं यतस्ततः~।\\
महता हन्यते तत्तद्घनात्मकफलाप्तये~॥

महतश्च घनेनैभिः सन्धीयन्ते त्रयोऽपि ये~।\\
पिण्डेऽल्पखण्डतुल्यास्ते विस्तारायामयोः पुनः~॥

खण्डेन महता तुल्यास्तद्वर्गेऽल्पहते ततः~।\\
प्रत्येकं स्यात् फलं तेषां त्रिघ्नं समुदितं भवेत्~॥

एवं द्वेधा विभागोऽत्र षट्सु चैकीकृते त्रिके~।\\
वर्गौ त्र्यन्यहतौ खण्डघनौ यौ\renewcommand{\thefootnote}{२}\footnote{द्वौ \textendash\ क. ख. पाठः.} तद्युतिर्घनः~॥ 

घनयुक्त्युपयोगी स्यादेष खण्डघनस्त्विह~। \\
खण्डाभ्यां वा हतो राशिस्त्रिघ्नः खण्डघनैक्ययुक्~॥ 

इत्येतद्युक्तयेऽप्यत्र तुल्ययोस्त्रिकयोर्द्वयोः~।\\
एकैकं पृथगादाय संश्लिष्टे यत् त्रिकद्वयम्~॥

अल्पखण्डसमं पिण्डे विस्तारे महता समम्~।\\
कृत्स्नेन राशिना तुल्यमायामे तत्त्रयं त्विह~॥}
\end{quote}

\newpage

\begin{quote}
{\qt अल्पखण्डहतो राशिर्भूयोऽपि महता हतः~।\\
त्रिघ्नश्च स्याद् घनैक्यं च भवेदष्टासु\renewcommand{\thefootnote}{१}\footnote{देषासु \textendash\ ग. पाठः.} च द्वयम्~॥

इष्टोनयुग्राशिवधो वेष्टवर्गघ्नराशियुक्~।\\
इति द्वेधा विभक्तेऽत्र क्षेत्रे युक्तिः स्फुरेद् घने~॥

इष्टभागे विदार्यैतं खण्डमादाय योजयेत्~।\\
शिष्टेनेष्टो\renewcommand{\thefootnote}{२}\footnote{ष्टे \textendash\ क. पाठः.}नतुल्येऽस्य पार्श्वयोः क्वचिदेव च~॥

राशिनेष्टयुतेन स्यादायामोऽस्यैकपार्श्वगः~।\\
विस्तारोऽपीष्टहीनेन राशिनैव समः क्वचित्~॥

यत्रैष निहितः खण्डस्तत्र स्यान्महता समः~।\\
विस्तारः शिखरे तस्मिन् खण्डयित्वा पृथक्कृते~॥

इष्टोन\renewcommand{\thefootnote}{३}\footnote{ष्टे \textendash\ ग. पाठः.}राशिना तुल्यो विस्तारस्तद्युतेन च~।\\ 
आयामे राशिना पिण्डे कृत्स्नेनैव समो ह्ययम्~॥

खण्डः पृथक्कृतोऽन्यो यः स च राशिसमोच्छ्रितिः~।\\
विस्तारायामयोरिष्टतुल्यं घनफलं द्वयोः~॥

इष्टोनयुक्तविस्तारदैर्घ्यो राशिसमोच्छ्रितिः~।\\
यस्तत्र तद्वधोऽन्यत्र राशिनेष्टकृतिर्हता~॥

एवं क्षेत्रविभागेन घनयुक्तिरिहोदिता~।}
\end{quote}

\noindent इति~। क्व पुनरस्योपयोगः~। ज्यार्धोपदेशसूत्रे हि प्रायेण गणितपादोक्तानामुपयोगः~। तदुपयोगं तद्व्याख्याने ज्याप्रकरणे दर्शयिष्यामः~॥~५~॥ \\

अथ वृत्तमवगाह्याशेषक्षेत्रयुक्तीः प्रदर्शयिष्यंस्तदुपयोगिषडश्रक्षेत्रन्यायं प्रथमं दर्शयति\textendash 

\begin{quote}
{\ab त्रिभुजस्य फलशरीरं समदलकोटीभुजार्धसंवर्गः~।\\
ऊर्ध्वभुजातत्संवर्गार्धं स घनः षडश्रिरिति~॥~६~॥}
\end{quote}

इति~। अत्र त्रिभुजमिति समत्रिभुजं विवक्षितम्~। तस्यैवोर्ध्वभुजाद्वारा वृत्तस्य षोढा विभाग उपयोगात्~। समचतुरश्रघनक्षेत्रयोः 

\newpage

\noindent फलप्रदर्शनानन्तरं त्र्यश्रषडश्र\renewcommand{\thefootnote}{१}\footnote{श्रि \textendash\ ग. पाठः.}घनयोः फलप्रदर्शनं प्राप्तावसरमित्ययं बाह्योऽपि सम्बन्धो विवक्ष्यते~। यः समदलकोटीभुजार्धसंवर्गस्तद्धि त्रिभुजस्य फलशरीरमित्यर्थः~। तद्यथा \textendash\ समत्र्यश्रं त्वेतत् समदलकोटिमार्गेण विभज्य एकं भागमादाय व्यत्ययेनान्येन सन्दध्यात्~। यथोपरिबाहुरपि भूम्यर्धतुल्यः स्यात्~। इतरौ च समदलकोटीतुल्यौ तदायामविस्तारौ समदलकोटीभुजार्धतुल्यौ~। ततस्तयोः संवर्गस्तत्फलतुल्य स्यादिति~। कथं पुनरिह समदलकोट्यानयनम्~। {\qt यश्चैव भुजावर्गः कोटीवर्गश्च कर्णवर्गः स} इति वक्ष्यमाणन्यायेनेति ब्रूमः~। तच्चेह कोटीभुजाशब्दाभ्यामेव सूचितम्~। त्र्यश्रे बाहुद्वययोगादितरभुजासन्नप्रदेशावधिका या\renewcommand{\thefootnote}{२}\footnote{या सा} रेखा सा कोटिः~। तच्छिन्नाया भुजायाः खण्डौ च तद्बाहू~। समत्र्यश्रे त्वत्र\renewcommand{\thefootnote}{३}\footnote{त्वश्र \textendash\ क. पाठः.} भुजार्धे एव
दलयोस्तुल्ये भुजे~। कोटिः पुनः सर्वत्रापि त्र्यश्रे तद्गतावान्तरजात्यत्र्यश्रयोस्तुल्यैव~। अतोऽत्रोभयोर्दलयोः सा\renewcommand{\thefootnote}{४}\footnote{सम \textendash\ क. ख. पाठः.}धारणी कोटिः समदलशब्देनोच्यते~। एवमिदमर्धायतचतुरश्रं क्षेत्रद्वयं तुल्याकारं समपरिमाणं च~। भुजाकोट्यग्रान्तरावगाढावितरौ बाहू च तत्कर्णौ~। तत्र कर्णश्च बाहुश्च ज्ञातौ ताभ्यामिहज्ञाता कोटिरानेया~। तद्विषयं चेदं सूत्रं {\qt यश्चैव भुजावर्गः कोटीवर्गश्च कर्णवर्गः स} इति~। यश्च भुजावर्गः यश्चैव कोटीवर्गः तौ समुच्चितौ कर्णवर्गः स्यात्~। कर्णवर्गात् भुजाकोट्योरन्यतरस्य वर्गेऽपनीते इतरवर्गः शिष्यत इत्येतच्चेह सिद्धम्~। तेनात्र कर्णवर्गाद्भुजावर्गेऽपनीते यः शेषः स कोटिवर्गः~। ततस्तन्मूलं कोटिरिति~। एतत्सर्वं विषमत्र्यश्रेऽपि समानम्~। किन्तु तत्र भुजार्धमेव भुजेति न नियमः~। उभयोरवान्तरखण्डयोरतुल्ये एव 
हि तत्र भुजे~। तथापि तद्बाहू भुजाकोटिकर्णन्यायेनैव सेत्स्यतः~। तद्यथा \textendash\ यतः कोटिरुभयत्रापि समाना~। ततस्तद्वर्ग एव स्वस्वभुजावर्गे क्षिप्ते स्वस्वकर्णवर्गः\renewcommand{\thefootnote}{५}\footnote{स्वस्वकर्णः \textendash\ ग. पाठः.} स्यादिति~। भुजयोः कर्णयोरपि तुल्यमेव वर्गान्तरमपि~। कर्णवर्गभेदस्य भुजावर्गभेद एव कारणम्~। न पुनरन्यश्च~। कोटिवर्गस्येतरभागस्योभयत्रापि साम्यात्~। तस्मात् कर्णात्मकयोस्त्र्यश्रबाह्वोः आबाधात्मकयोर्भुजाख्ययोर्भूमिखण्डयोश्च वर्गान्तरं तुल्यमित्येतावज्ज्ञातम्~। न पुनस्तत् कियदिति~। अवान्तरखण्डभुजायोगश्च ज्ञातः~। भूम्याख्याया भुजायाश्चो-

\newpage

\noindent द्देशकेनैवोक्तत्वात्~। भूमितुल्यो हि बाहुयोगः~। तस्या एव खण्डयोर्बाहुत्वात्~। तत्र भुजावर्गान्तरमेवाज्ञातं कर्णयोरुद्देशकेनैव
बाहुत्वेनोद्दिष्टत्वात्~। तद्वर्गान्तरमपि ज्ञेयम्~। तत्तुल्यत्वं चावगतं भूमिखण्डवर्गयोरिति भुजावर्गान्तरमपि ज्ञेयम्~। तदानयनमेवात्रोक्तं \textendash\ {\qt त्रिभुजे भुजयोर्योगस्तदन्तरगुण} इति~। योगान्तरघाततुल्यं हि वर्गान्तरम्~। तच्चाह भास्करः\textendash 

\begin{quote}
{\qt राश्योरन्तरवर्गेण द्विघ्ने घाते युते तयोः~।\\
वर्गयोगो भवेदेवं तयोर्योगान्तराहतिः~॥\\
वर्गान्तरं भवेदेवं ज्ञेयं सर्वत्र धीमता~।}
\end{quote}

\noindent इति~। कथं पुनरिदमवसीयते \textendash\ द्वयो राश्योर्योगस्तदन्तरेण गुणितस्तयोर्वर्गान्तरं स्यादिति~। अस्य युक्तिश्चोभयथा प्रदर्श्या गणनन्यायमात्रेण क्षेत्रकल्पनया च~। तत्र छेद्यके वैशद्यं स्यात्~। तद्यथा \textendash\ महतो वर्गात्
तत्तुल्यचतुर्बाहोः क्षेत्रादल्पतुल्यबाहुके समचतुरश्रेऽपनीते यच्छिष्टं क्षेत्रं तन्न समचतुरश्रं नचाप्यायतचतुरश्रम्~। कथं भूतं तर्हि तत् क्षेत्रम्~। महतः
क्षेत्रस्यैककोणादल्पराशितुलितान्तरे तत्कोणस्पृष्टयोरुभयोरपि बाह्वोरङ्कं कृत्वा ताभ्यां सूत्रे क्षेत्रान्तर्नीत्वा तद्युतौ च बिन्दुं कुर्यात्~। कथं
तद्युतेर्नियतदेशत्वम्~। सूत्रयोः ऋजुत्वे नियतदेशैव युतिः~। ऋजुत्वं च महाक्षेत्रगताश्रस्य\renewcommand{\thefootnote}{१}\footnote{स्यास्य सू \textendash\ ग. पाठः.} सूत्रस्य चान्तरालस्यापादतलमस्तकं तुल्यत्वादेव सिद्धम्~। तत्र ये तत्सूत्रयोगान्ते\renewcommand{\thefootnote}{२}\footnote{म्त \textendash\ क. ख. पाठः.} रेखे ते एवाल्पस्य क्षेत्रस्य द्वे पार्श्वे~। इतरे च तत्तुल्ये~। कोणाङ्कद्वयान्तरमिते तस्मिन्नवान्तरक्षेत्रे विनाशिते या तद्विवरस्याकृतिः तस्या उभयविधचतुरश्रत्वं\renewcommand{\thefootnote}{३}\footnote{श्रं \textendash\ ग. पाठः.} न स्यात्~। ततस्तेनैवायतचतुरश्रं सम्पाद्य तत्फलं निरूप्यम्~। तत्फलं हि वर्गान्तरमिति~। तदेकीकरणं च महाक्षेत्रस्यान्तर्गतरेखामार्गेण तद्युतेर्बहिरपि महाक्षेत्रेतरबाहुपर्यन्तं रेखां कृत्वान्यत् क्षेत्रं निष्कृष्य तदग्रे सन्धायायाममेव वर्धयेत्~। तथा सति राशियोगतुल्य आयामः, तदन्तरतुल्यो विस्तारः, इति राशियोगान्तरयोर्घात एव तत्फलात्मकवर्गान्तरमपि तयोरिति तद्वैपरीत्येन वर्गान्तरं राशियोगेन हृत्वा लब्धमेव राश्योरन्तरमिति च ज्ञेयम्~। अत उक्तं भुवा हृत इति~। अत्र लब्धमाबाधान्तरम् आबा(ध?धा)योगश्च भूरेव~। तस्मात् भुवि तल्लब्धं क्षिप्त्वार्धिते महत्याबाधा स्यात्~। 

\newpage

\noindent भुवो लब्धं त्यक्त्वार्धितेऽल्पा च~। तत्राबाधयोरल्पस्यापि महत्या तुल्यत्वमल्पे स्वमहदन्तेरयोगेन क्रियते~। ततस्तद्युक्ताया भुवो
द्विगुणमहाबाधातुल्यत्वात् तदर्धं केवला महत्याबाधा स्यात्~। अन्तरत्यागेन महत्या अप्यल्पतुल्यत्वे द्विगुणाल्पतुल्यत्वं स्यादिति तदर्धीकरणेनाल्पाबाधा स्यादिति विषमे भुजाज्ञनोपायभूतं कर्म~। समे पुनर्भुजाया बाह्वर्धतुल्यत्वात् न तदर्थं यत्नः कार्यः~। इति बाहुतदर्धवर्गविवरमूलमेव\renewcommand{\thefootnote}{१}\footnote{वर्गमूलविवरमेव \textendash\ क. पाठः.} समदलकोटिरिति\renewcommand{\thefootnote}{२}\footnote{कोटीति \textendash\ ग. पाठः.} सिद्धम्~। अथ तद्गतषडश्रक्षेत्रफलानयनायोत्तरार्धमाह \textendash\ ऊर्ध्वभुजातत्संवर्गार्धं स घनः षडश्रिरिति~। इति~। ऊर्ध्वभुजायास्त्र्यश्रफलस्य च संवर्गस्य यदर्धं स षडश्रि\renewcommand{\thefootnote}{३}\footnote{श्र \textendash\ क. ख. पाठः.}घनः~। षडश्रिक्षेत्रगतं घनफलमिति यावत्~। कथं पुनरिहोर्ध्वभुजानयनम्~। सा पुनः पातरेखान्यायेन सेत्स्यति~। वक्ष्यति च पातरेखानयनम्\textendash 

\begin{quote}
{\qt आयामगुणे पार्श्वे तद्योगहृते स्वपातरेखे ते~।}
\end{quote}

\noindent इति~। केयं पातेरखा नाम~। उच्यते~। यथेहैककोणात् सूत्रप्रसारणमुक्तम्, एवमन्याभ्यामपि कोणाभ्यां सूत्रद्वये प्रसारिते यस्त्रयाणां योगः स्यात् स पात इत्युच्यते~। तेषां सम्पातरूपत्वात्~। ततः प्रवृत्ते ये रेखे ते पातरेखे~। कियत्पर्यन्ते पुनस्ते इत्यादि प्रदर्शनाय समत्र्यश्रान्तर्भूतं विषमचतुरश्रं कल्पनीयम्~। कथम्~। समेषु त्रिष्वपि बाहुष्वेको याम्योत्तरायतः तदग्राभ्यां प्रवृत्तयोरितरयोर्योगस्तन्मध्यात् प्राग्दिशि कल्प्यः~। तत्र दक्षिणोत्तरायतास्य भूम्याख्या~। इतरबाहुयोगात् अवलम्बितं सूत्रं लम्बः~। स च भूम्यर्धभुजकयोः साधारणी कोटिः~। भूमेरुदगग्रात् प्रभृति दक्षिणबाहुमध्यान्तं या रेखा सापि लम्बतुल्या~। बाहुद्वययोगप्रतिभुजमध्यान्तरालावगाढत्वेनाविशेषात्~। एवमेव दक्षिणाग्रात् प्रभृति सव्यभुजामध्यान्तरालावगाढस्यापि तत्तुल्यत्वम्~। एवं समदलकोटीतुल्यास्तिस्रो रेखाः स्युः~। आभिर्यद्विषमचतुरश्रमुत्पद्यते,
त्र्यश्रभूमिरेव तस्यापि भूमिः~। मुखं तु भूम्यर्धतुल्यं, यतो बाहुद्वयमध्यान्तरालं मुखम्~। मुखादूर्ध्वगतो यस्त्र्यश्रभागः तद्भूमितुल्यं हि तदधोगतचतुरश्रमुखम् एकत्वादेवोभयोः~। कथं पुनस्तस्य भूम्यर्धसाम्यम्~। ऊर्ध्वगतसव्येतर-

\newpage
\noindent भुजयोः कृत्स्नत्र्यश्रगतस\renewcommand{\thefootnote}{१}\footnote{गस}व्येतरभुजार्धतुल्यत्वात् समत्र्यश्रत्वाच्च भुवोऽपि भुजासाम्यात्~। क्षेत्रं चैतत् वृत्तान्तर्भूततया\renewcommand{\thefootnote}{२}\footnote{र्भूतया \textendash\ ग. पाठः.} कल्प्यम्~। वृत्तं प्रस्तुत्य हि पातरेखादिकं प्रदर्श्यते~। तेन कोणावलम्बितसू\renewcommand{\thefootnote}{३}\footnote{लम्बिसू \textendash\ क. पाठः.}त्रत्रयं वृत्तापरपरिध्यन्तं नेयम्~। सम्पातात् प्रभृति वृत्तपर्यन्तं यानि तेषामर्धानि तान्यरस्थानीयानि च~। एवं षडश्र\renewcommand{\thefootnote}{४}\footnote{र \textendash\ ख. ग. पाठः.}त्वं वृत्तस्य सम्पा\renewcommand{\thefootnote}{५}\footnote{म्प \textendash\ क. ग. पाठः.}द्यते~। तथाभूतेऽस्मिन् वृत्ते ज्याछेदविधानं प्रदर्शनीयमित्यभिप्रायः~। तत्र तत्संपात एव वृत्तकेन्द्रम्~। स एव त्र्यश्रस्यापि समन्तान्मध्यम्~। यतस्ततः कोणत्रयमितरेतरं तुल्यं स्वबाहुमध्यान्ततरालत्रयं च~। कथं पुनस्तेषु सूत्रेषु द्वयोर्द्वयोः सम्पाता नानादेशगता न स्युः, येन त्रयाणां सन्निपातो न स्यात्~। उच्यते~। ऊर्ध्वकोणावलम्बितसूत्रस्यावयभूतात् तत्तत्प्रदेशात् भूम्यग्रद्वयं तुल्यान्तरं स्यात्~। तथाहि \textendash\ भूम्यर्धात्\renewcommand{\thefootnote}{६}\footnote{भूमध्यात् \textendash\ ग. पाठः.} तावत् तदग्रद्वयं तुल्यान्तरालम्~। लम्बश्च तत्रैव पतति समत्र्यश्रत्वादस्य~। इतरथा सव्येतरबाह्बोः समत्वमेव हीयेत\renewcommand{\thefootnote}{७}\footnote{हीयते \textendash\ क. ख. पाठः.}~। तथा सत्याबाधे अपि न तुल्ये~। तत्प्रदर्शनाय भूमितुल्ये द्वे शलाके भूम्यग्रस्पृष्टैकाग्रे विन्यस्य तयोः शिरसोः सन्धानं कार्यम्~। एवं सन्धीयमानयोस्तयोर्नमनमुभयोरपि तुल्यमेव स्यात्~। इतरथैकस्याग्रादध एवान्यस्याग्रसंस्पर्शः~। तथा तद्बाह्वोरपि तुल्यत्वमेव न स्यात्~। किञ्च दलयोरुभयोः कोटिसाम्यादेव भुजासाम्यमपि सिद्धं तत्कर्णयोश्च मिथस्तुल्यत्वात्~। तच्च सूचितं समदलकोटीशब्देन~। तस्मात् भूमिलम्बसम्पातात् भूम्यग्रे उभे अपि तुल्ये समत्र्यश्रे~। तथा लम्बसूत्रोपरिभागेभ्यश्च भूमेः सव्याग्रं दक्षिणाग्रं च तुल्यमेव स्यात् समोर्ध्वगतप्वाल्लम्बस्य भूम्यपेक्षया समतिर्यक्त्वात्~। अग्रयोरन्यतरप्रावण्ये सत्येव तदूर्ध्वावयवेष्वेकाग्रस्य सन्निकर्ष इतराग्रस्य विप्रकर्षश्च स्यात्\renewcommand{\thefootnote}{८}\footnote{स्याताम् \textendash\ ख. पाठः.}~। उदासीनत्वे पुनस्तुल्यत्वमेव स्यात्~। एवं दक्षिणबाहुमध्यस्पृष्टे सूत्रेऽपि सकलावयवेभ्यो दक्षिणसूत्राग्रद्वयविप्रकर्षौ मिथस्तुल्यावेव स्याताम्~। एतयोः संपातश्च क्वचिदवश्यंभावी~। यतो भूमिसव्याग्रात् प्रवृत्तं भूमध्यादूर्ध्वतः प्रवृत्तं सूत्रमप्राप्य कथं दक्षिणबाहुमध्यं प्राप्नोति~। सव्यभुजातः प्रावण्यादेवास्य सूत्रस्य प्रावण्यमेव हि युक्तमग्रावधिकान्मध्यावधिकस्य~। तस्मात् तयोः सम्पातात् कोणत्रयमपि तुल्यम्~। यत ऊर्ध्वसूत्रे सर्वत्र भूम्यग्रद्वयस्य तुल्यतया भाव्यम्~।
दक्षिणभुजामध्यस्पृष्टेऽपि सर्वत्र 

\newpage

\noindent दक्षिणभुजाग्रयोर्विप्रकर्षस्यापि तुल्यतया भाव्यम्~। तस्मात् सूत्रद्वयसंयोगे भूम्यग्रयोर्दक्षिणभुजाग्रयोश्च तुल्यत्वमेव स्यात्~। दक्षिणभुजाधोग्रस्य भूमिदक्षिणाग्रस्य चैकत्रैवावस्थानम्~। अतः संपातात् तुल्यान्तरे एव तयोरितराग्रे~। एवं कोणत्रयस्यापि तद्द्वयसंपातात् तुल्यत्वम्~। अनेनैव न्यायेन सव्यभुजामध्यगतमपि सूत्रमितरयोः संयोग\renewcommand{\thefootnote}{१}\footnote{योग} एव सम्पतति~। तत्प्रदेशेष्वपि सर्वत्र तदूर्ध्वाधोग्रयोः साम्यात्~। तस्मात् सूत्रत्रयसम्पातः क्षेत्रमध्यगः~। तत् लम्बाह्वयसूत्रस्य सूत्रत्रय\renewcommand{\thefootnote}{२}\footnote{लम्बाह्वय सूत्रत्रय}सम्पातादधोगतो यो भागो भूमध्यान्तः यश्चोर्ध्वगतो मुखमध्यान्तः तावेवेहानीयेते~। तदुभयमपि पातरेखाख्यम्~। {\qt आयामगुणे पार्श्वे तद्योगहृते स्वपातरेखे ते} इति या भूमिस्पृष्टा पातरेखा सा हि भूसम्बन्धिनी मुखमध्यान्ता च मुखसम्बन्धिनीति तयोः स्वभूते ते~। तद्योगशब्देन च पार्श्वद्वयैक्यमुच्यते~। न पुनस्त्रयाणां योगः~। नापि
पार्श्वयोरन्यतरस्यायामस्य च योगः~। योगस्य द्वाभ्यामेव कृतार्थत्वात् त्रयाणां योगो निरस्यते~। कल्पनागौरवाच्च~। आयामैकतरभुजयोः ये द्वे पार्श्वे आयामगुणे तद्योगहृते ते स्वस्वपातरेखे स्याताम्~। आयामगुणं भूम्याख्यं पार्श्व भूमुखयोगहृतं भूस्पर्शिनी\renewcommand{\thefootnote}{३}\footnote{स्पर्शिनी \textendash\ क. पाठः.} पातरेखा, आयामगुणं च मुखं भूमुखयोगहृतं मुखसम्बन्धिनी चेत्यर्थः~। कीदृशीहोपपत्तिः~। त्रैराशिकं हीदं गणितकर्म~। वक्ष्यति च त्रैराशिकं\textendash 

\begin{quote}\
{\qt त्रैराशिकफलराशिं तमथेच्छाराशिना हतं कृत्वा~।\\
लब्धं प्रमाणभजितं तस्मादिच्छाफलमिदं स्यात्~॥}
\end{quote}
\noindent इति~। अस्यार्थो गोविन्दस्वामिना महाभास्करीयभाष्ये प्रदर्शितः~। कथमिदं त्रैराशिकं नाम~। इदमिह त्रैराशिकं \textendash\ त्रयो राशयः समाहृताः कारणं यस्य स राशिः कार्ये कारणोपचारात् त्रिराशिर्भवति, स प्रयोजनं यस्य तद्गणितं त्रैराशिकम्~। तत्र प्रमाणं फलमिच्छा चेति त्रयो राशयः~। तेषु तत्\renewcommand{\thefootnote}{४}\footnote{तव \textendash\ क. ख. पाठः.} प्रमाणं नाम यत इदं लब्धमिति व्यपदिशति~। लब्धं तु फलम्~। यत् पुनरनेन कियल्लभ्यत इतीदमभिधीयते तदिच्छा~। यच्च पुनर्जिज्ञास्यं तदिच्छाफलं नाम~। तत्रेच्छाहतं फलं प्रमाणेन विभजेत्~। तदेच्छाफलावाप्तिरिति~। इहापि पार्श्वद्वययोगः प्रमाणम्~। आयामश्च फलम्~। भूमिरेकत्रेच्छा~। इतरत्र च मुखम्~। यदि भूमुखयोगतुल्येनैतत्सूत्रद्वयविप्रकर्षेणायामतुल्यस्तदव-

\newpage

\noindent लम्बो लभ्यते, तदा भूमितुल्येन तद्विप्रकर्षेण कियानिति भूमिपातरेखा लभ्यते~। मुखतुल्येन\renewcommand{\thefootnote}{१}\footnote{तुल्येनैव} विप्रकर्षेण कियानिति चान्या~। क्व पुनरिहानयोः सूत्रयोर्विप्रकर्षो भूमुखयोगतुल्योऽनुभूतः, तन्मध्यगता लम्बाह्वया कोटिश्च चतुरश्रगतायामतुल्या~। येन तयोः प्रमाणफलयोः सिद्धिरिति चेत्~। ते चास्मिन्नेव चतुरश्रेऽनुभूते~। कथम्~। सूत्रत्र\renewcommand{\thefootnote}{२}\footnote{द्व \textendash\ क. पाठः.} यसंपातात् प्रभृति भूतुल्यस्तद्भागगतो विप्रकर्षः मुखतुल्यश्चेतरभागगतः~। एवं खण्डद्वयगतौ यौ विप्रकर्षौ तदैक्यतुल्यः खलु विषमचतुरश्रावच्छिन्नयोरेतयोः सूत्रयोः कृत्स्नयोर्विप्रकर्षः~। तयोः कृत्स्नयोः कर्णयोः कोटिश्चायामतुल्या~। नन्वत्रेच्छाक्षेत्रस्य प्रमाणक्षेत्रस्य च तुल्याकारत्वं न स्यात्~। प्रमाणक्षेत्रं हि चतुरश्रम्~।
त्र्यश्रं चान्यत्~। तुल्याकारयोर्हीच्छाप्रमाणक्षेत्रयोस्त्रैराशिकं युज्यते~। नैष दोषः~। तुल्याकारत्वादेवोभयोः~। कथं
पुनश्चतुरश्रत्र्यश्रयोस्तुल्याकारत्वमुपपद्यते~। सूत्रयोर्विप्रकर्षस्य तुल्यत्वादेव ह्येकाकारत्वम्~। न पुनर्वृत्तचतुरश्रादिक्षेत्र गतत्वेन~। अत्र पुनः खण्डयोरुभयोरपि तुल्याकारमेव हि सूत्रविवरम्~। यतः सूत्रखण्डयोरितरेतरमेकदिग्गतयोर्ऋजु\renewcommand{\thefootnote}{३}\footnote{गततया ऋजु \textendash\ क. ख. पाठः.}त्वादेव विप्रकर्षसाम्यम्~। तेन सूत्राधःखण्डयोः कर्णभूतयोर्वर्धमानयोर्यावानंशः सर्वत्र विप्रकर्षः, ऊर्ध्वखण्डयोरपि तावानेव सर्वत्रेति त्रैराशिकोपपत्तिः~। अनया युक्त्याप्यापरितुष्यतः शिष्यस्यैवं वा युक्तिः प्रदर्श्या \textendash\ भूम्यग्राभ्यामधः सूत्रद्वयमाकृष्य कृत्स्नमपि सूत्रद्वयं संपातादधोगतं कृत्वा त्र्यश्रतयैव वर्धिते क्षेत्रे
प्रमाणफले प्रदर्श्ये~। तथा सति संपातात् प्रभृति भूम्यग्रद्वयस्पृक्सूत्रद्वयं कृत्स्नमप्येकमार्गगतमिति कृत्स्नस्याप्येकाकारता~। तत्र पातरेखाद्वयतुल्यैव कोटिः~। अत एवास्याश्चतुरश्रगतायामतुल्यत्वमपि स्यात्~। कथम्~। यथा कर्णाकारसूत्रयोगादेकं सूत्रं तत्कोट्याकारमवलम्बितम्, एवं कर्णयोरुभयोः पृथक् पृथगकैकहस्तादितुलितेभ्यः प्रदेशेभ्यो गुरुद्रव्यबद्धानि सूत्राण्यवलम्ब्य(ताम् ?
न्ताम्)~। तथा तेषां द्वयोर्द्वयोर\renewcommand{\thefootnote}{४}\footnote{चयेर्ध्ययोर}न्तरालानां तुल्यत्वमेव स्यात्~। कर्णेऽपि तत्स्पृष्टान्तरालानां तुल्यत्वात्~। तच्चान्तरालमेकसङ्ख्यस्य कर्णस्य कोट्या\renewcommand{\thefootnote}{५}\footnote{कोट्यां \textendash\ ख. पाठः.} तुल्यमिति कर्णगतावयवानां प्रत्येकं तुल्यकोटित्वादेव त्रैराशिकोपपत्तिः~। यत्रेच्छावृद्ध्यनुसारेणैव फलस्यापि वृद्धिः स्यात्, ह्रासानुसारेणैव ह्रा(सा~? स)श्च~। एतदप्यस्मिन्नेव सूत्रे सूचितं {\qt त्रैराशिकफलराशिं तमथेच्छाराशिना हतं}

\newpage

\noindent {\qt कृत्वे}ति~। अत्रैवं वच(नं? न)व्यक्तिः \textendash\ तं पूर्वानुभूतं फलराशिमिच्छाराशिना हतं कृत्वा अथ पश्चात् तस्मात् प्रमाणभजितं तदेवेच्छाफलतया गृह्यताम् अवयवोपेक्षादोषपरिहाराय लाघवाय वा क्रियायाः~। कः पुनरत्र प्रयोगक्रमः क्रियालाघवाय वावयवोपेक्षादोषपरिहाराय वा व्यावर्त्यते~। अञ्जसा प्राप्त एवेति ब्रूमः~। कथं पुनराञ्जस्येन सिद्धं कर्म~। एतावतः प्रमाणस्यैतावत् फलमिति यद्भूयोदर्शनेन\renewcommand{\thefootnote}{१}\footnote{दर्शने \textendash\ ख. पाठः.} वा प्रमाणान्तरेण वावगतं\renewcommand{\thefootnote}{२}\footnote{तः} तत्र प्रमाणजातीयव्यक्तीनां प्रत्येकं कियत् फलं स्यादिति हि प्रथमं निरूपयितुं युक्तम्~। तत्र प्रमाणगता व्यक्तयो यावत्यः स्युः फलस्य तावानेवांशः प्रत्येकं व्यक्तिफलं स्यादित्येतत् सर्वैरपि ज्ञातुं शक्यं, यदि प्रमाणफलजात्योः परिमाणत(त्स ? स्स)म्बन्धनियमः\renewcommand{\thefootnote}{३}\footnote{परिमाणस्सम्बन्धनियमः} स्यात्~। तदभावे पुनर्नैवैकेनेतरानुमानं शक्यं, सम्बन्धाभावात्~। तस्मात् व्यक्तीनां सर्वासां फलसाम्यात् ज्ञातमेवैकव्यक्तिफलं सर्वत्रेच्छाराशिना गुणनीयम्~। तदेच्छागतव्यक्तीनां सर्वासां फलानामैक्यं स्यादिति कृत्स्रस्येच्छाराशेः फलसिद्धिरित्ययं क्रम एव सर्वेषां स्वतः स्फुरति~। अत एवोक्तम् इच्छाराशिना फलराशिं हत्वैव प्रमाणराशिना ह्रियताम्~। न पुनः प्रमाणेन फलराशिं हृत्वा पश्चादिच्छागुणनं क्रियताम्~। इच्छाराशिना फलगुणनोक्तेश्चैषैव युक्तिः प्रदर्श्या\renewcommand{\thefootnote}{४}\footnote{प्रदर्शिता \textendash\ क. पाठः.}~। इतरथा
विषयव्याप्त्यर्थमिच्छाफलयोर्घातः प्रमाणेन ह्रियताम् इत्येव वक्तव्यम्~। इच्छायाः फलेन हननस्य व्यावृत्त्यर्थं च फलराशिमिच्छाराशिना हतं कृत्वेत्युक्तम्~। तस्मात् त्रैराशिकयुक्तिप्रदर्शनपरमिदं सूत्रं, नतु कर्मक्रममात्रप्रदर्शनपरम्~। तस्माद्युक्त्यनुसारिणः कर्मण एवात्र प्रदर्शनं कृतम्~। फलसाम्यं पुनरितरथापि स्यात्~। यथाह कश्चित्\textendash 

\begin{quote}
{\qt इच्छां फलेन संहत्य प्रमाणेन विभाजयेत्~।\\
इच्छाफलं भवेल्लब्धमेवं त्रैराशिकं मतम्~॥}
\end{quote}

\noindent इति~। एवं विषयसंको(च ? चे)नापि युक्तिरेवात्र प्रदर्श्येति भावः~। अत एव ह्यस्य सारवत्त्वम्~। तस्मादत्र पातरेखयोर्युक्तमेव त्रैराशिकं, तुल्यवृद्धिह्रासवत्त्वादिच्छाप्रमाणफलयोरिति सिद्धम्~। इह पुनरूर्ध्वगता पातरेखा आयाम-

\newpage

\noindent त्र्यंशतुल्या~। ततो द्विगुणा चान्या, मुखात् द्विगुणत्वाद्भुवः~। समत्र्यश्रगतस्यायामस्यार्धमेवात्र चतुरश्रगतायामः~। अर्धमूर्ध्वत्र्यश्रगतम्, ऊर्ध्वत्र्यश्रगतकर्णात् द्विगुणत्वात् कृत्स्नस्य त्र्यश्रकर्णस्य~। तस्मात् कृत्स्नत्र्यश्रगतकोट्यर्धस्या(र्धो? धो)गतस्य चतुरश्रान्तर्भूतस्य त्रेधा विभक्तस्य भागद्वयतुल्या भूसंपातरेखा~। अतः समदलकोट्याश्च त्र्यंशतुल्या~। यस्मादेकस्यार्धस्य त्रेधा विभागे इतरस्यापि त्रेधा विभागः कार्यः~। तस्मात् षोढा विभक्तस्य लम्बस्य चत्वारः खण्डाः सूत्रत्रयसंपातकोणानन्तरगताः, तद्बाहुमध्यान्तरगतौ
च द्वाविति विभागः~। तस्माल्लम्बस्य त्र्यंशतुल्यः पातादधःखण्डः~। ततस्तद्वर्गे लम्बवर्गात् विशोधिते शिष्टं ऊर्ध्वभुजावर्गः~। ऊर्ध्वभुजायाः
कोट्या भुजात्वात् भूपातरेखायाः~। तयोः कर्णश्च कृत्स्नलम्बतुल्यः~। किमाकारं पुनरेतत् षडश्रि\renewcommand{\thefootnote}{१}\footnote{श्र \textendash\ ख. पाठः.}क्षेत्रं यद्गतमूर्ध्वभुजासमलम्बकोटिकर्णकं
क्षेत्रमुच्यते~। तत्प्रदर्शनाय त्र्यश्रबाहुतुल्या ऋज्वीस्तिस्रः शलाकास्तत्कोणेषूर्ध्वायताः कृत्वा तस्यामग्रत्रयं योजयेत्~। यत्र तासामग्रत्रयं युक्तं यदुत्सेध ऊर्ध्वभुजा, तच्च सूत्रत्रयसंपातात् समोर्ध्वमेव स्यात्, ऊर्ध्वभुजाप्रदेशेषु सर्वत्रापि कोणविप्रकर्षसाम्यात्~। एवं शलाकान्तर्भागं मृदादिनापूर्य तत्पृष्ठभागत्रयं वास्यादिना समीकुर्यात्~। एवमिदं षडश्रि\renewcommand{\thefootnote}{२}\footnote{श्र \textendash\ ख. पाठः.}क्षेत्रमूर्ध्वायतमवतिष्ठते~। तस्याश्राणां षण्णां तुल्यत्वमेव स्यात्~। यतस्त्र्यश्रबाहुतुल्यान्येव त्रीण्यश्राणि~। त(तो ? त ऊ)र्ध्वभुजा तु तेभ्यो न्यूनपरिमाणैव, बाहुतुल्यानामश्राणां तत्कर्णत्वात्~। कर्णान्न्यूने एवहि भुजाकोट्यौ~। यतस्तद्वर्गयोगमूलं कर्णः~। केवलमश्राणामेवोर्ध्वभुजाकर्णत्वम्~। समतलकोटयश्च तिस्रस्तत्कर्णा एव~। तेनोभयथापि तदानयनं कार्यम्~। तत्र तावत् समदलकोट्याः कर्णत्वे भूपातरेखैव कोटिः, ऊर्ध्वभुजा च भुजेत्येतत् प्राक्प्रदर्शितयुक्त्यैव सिद्धम्~। समदलकोट्याः कर्णत्वं च भुजामध्यात् प्रवृत्तायास्त्वस्या ऊर्ध्वभुजाग्राभिमुख्येन समोर्ध्वत्वाभावे न युज्यते~। अतस्ततोऽपि न्यूनैवोर्ध्वभुजा~। तत्र प्रथमं
समदलकोटीवर्ग आनेयः~। तत ऊर्ध्वभुजावर्गश्च~। समदलकोटीवर्गानयनं तावत्पूर्वमेवोक्तम्~। यतस्त्र्यश्रबाहुवर्गात् तदर्धवर्गेऽपनीते
बाहुवर्गपादत्रयमेवावशिष्यते~। कृत्स्नवर्गादर्धवर्गस्य चतुरंशत्वात्~। प्रदर्शितं च गुणोत्तर-

\newpage

\noindent राशिवर्गाणां मूलगुणवर्गगुणोत्तरत्वम्~। तत्तलगतायाः समदलकोट्यास्त्र्य(श्र ? श)तुल्या भूपातरेखेति च प्रतिपादितम्~। ततस्तद्वर्गो दलकोटिवर्ग\renewcommand{\thefootnote}{१}\footnote{वर्गे \textendash\ ख. पाठः.}नवांश एव~। तस्मिन् दलकोटिवर्गादपनीते दलकोटिवर्गनवांशाष्टकमेवावशिष्यते~। तच्चाश्रवर्गद्वादशांशाष्टकम्~। यतोऽश्रव(र्गा ? र्ग)द्वादशांशस्य दलकोटीवर्गनवांशस्य च तुल्यत्वम्~। कु\renewcommand{\thefootnote}{२}\footnote{यु \textendash\ क. पाठः.}तस्तयोस्तुल्यत्वम्~। अश्रवर्ग\renewcommand{\thefootnote}{३}\footnote{वर्गे \textendash\ ख. पाठः.}पादत्रयात्मकत्वात् दलकोटिवर्गस्याश्रवर्ग\renewcommand{\thefootnote}{४}\footnote{श्रकोटिवर्ग}(त्वा? द्वा)दशांशनवकत्वमपि\renewcommand{\thefootnote}{५}\footnote{मपि न \textendash\ क. पाठः.} स्यात्~। यतो द्वादशांशनवकं च पादत्रयं च तुल्यमेव~। तस्मात् तन्नवांशेऽपनीते अश्रवर्गस्य द्वादशांश एवापनीतः स्यात्~। एवं नवभ्य एकस्मिन्नपनीतेऽष्टावेवा\renewcommand{\thefootnote}{६}\footnote{वेव \textendash\ ख. पाठः.}वशिष्यन्त इत्यश्रवर्गद्वादशांशाष्टकत्वमूर्ध्वभुजावर्गस्य सिद्धम्~। ततस्त्रैराशिकेन त्र्यश्रबाहुवर्गेच्छाया इच्छाफलभूतोर्ध्वभुजावर्ग आनेतुं शक्यः~। तत्र द्वादशकं प्रमाणं फलमष्टसङ्ख्यम्~। ततस्तयोश्चतुर्भिरपवर्तितयोः त्रिसङ्ख्यकमेव प्रमाणं फलं च द्विसङ्ख्यकम्~। तदेतत् त्रैराशिकं सूर्यदेवेनाप्युक्तं\textendash 

\begin{quote}
{\qt द्विघ्ना कर्णकृतिर्भक्ता त्रिभिरूर्ध्वभुजाकृतिः~।}
\end{quote}

\noindent इति~। त्र्यश्रबाहुतुल्यकणवर्गादूर्ध्वभुजावर्गानयनमपि निरूप्यमाण एवमेव पर्यवस्यति~। तद्यथा \textendash\ तत्र
त्र्यश्रकोणादूर्ध्वभुजामस्तकान्तस्त्रयश्रबाहुतुल्यः कर्णः~। भूपातरेखोना दलकोटिरेव सूत्रत्रयसंपातकोणान्तरालमितिः~। (स? सा) कोटिः~। तस्मात् दलकोट्यास्त्र्यंशोनत्वादस्या भुजाया वर्गो दलकोटिवर्गनवांशकचतुष्कतुल्य एव~। यथा त्रयाणां वर्गान्नवसङ्ख्यात्
स्वत्र्यंशोनत्रिकस्य द्विकत्वमापन्नस्य वर्गश्चतुस्सङ्ख्यो लभ्यते, एवं सर्वत्रापि कृत्स्नस्य त्र्यंशोनवर्गस्यापि परिमाण(क? तः) सम्बन्धो नियतः~। ते(ते? ने)हापि त्रैराशिकं युज्यत एव~। कर्णवर्गद्वादशांशनवकतुल्यो दलकोटिवर्ग इति चोक्तम्~। तत ऊर्ध्वभुजानयने त्र्यंशोनः कर्णवर्ग इच्छाराशिः~। नवसङ्ख्यः प्रमाणराशिः~। कीदृशीह त्रैराशिकवाचोयुक्तिः~। ई\renewcommand{\thefootnote}{७}\footnote{की \textendash\ क. पाठः.}दृशीह त्रैराशिकवाचोयुक्तिः~। यदि कृत्स्नराशिवर्गेण नवसङ्ख्येन स्वत्र्यंशोनवर्गश्चतुस्सङ्ख्यो लभ्यते, तदा कृत्स्नाया दल कोट्या वर्गेणानेन त्र्यंशोनदलकोट्या वर्गः कियानिति कर्णवर्गादूर्ध्वभुजावर्गो लभ्यते~। तस्माद्दलकोट्या वर्गस्य कर्णवर्गद्वादशांशनवकत्वात् तस्मिंश्चतुर्भि-

\newpage

\noindent र्हत्वा नवभिर्हृते कर्णवर्गद्वादशांशचतुष्कमेवावशिष्यते, सोऽत्र कोटिवर्गः~। तस्मात् कर्णवर्गस्य त्र्यंश एव सः~। तस्मिन् कर्णवर्गादपनीते कर्णवर्गस्य त्र्यंशद्वितयं शिष्यते~। स चोर्ध्वभुजावर्गः~। तस्मात् तत्राप्येवमेव~। द्विघ्ना कर्णकृतिरित्यादिनोक्तमेव त्रैराशिकम्~। तन्मूलमूर्ध्वभुजा~। तस्याः पूर्वार्धोक्तत्र्यश्रफलस्य च संवर्गस्यार्धमेतत् षडश्रक्षेत्रघनफलमित्यर्थः~। कुतो\renewcommand{\thefootnote}{१}\footnote{त} 
नवाश्रक्षेत्रमनुक्त्वा षडश्रक्षेत्रफलमेवोक्तम्~। घनन्यायसिद्धत्वात् तस्य~। समनवाश्रस्यायतनवाश्रस्य वा तत्सम्बन्धित्र्यश्रफलस्य तदुच्छ्रितिहननेनैव घनफलं स्यादित्येतन्न्यायसिद्धम्~। तस्मादूर्ध्वभुजातत्संवर्ग एव नवाश्रफलम् इत्येतच्चाप्यनेनैव सूचितम्~। ऊर्ध्वभुजातत्संवर्गशब्दोच्चारणात् तत्र बुद्धिः प्रथमं प्रसरेत्~। तस्यार्धस्यापनीनत्वादवशिष्टं षडश्रक्षेत्रमप्यर्धतुल्यमेवेति भावः~॥~६~॥ \\

एवं त्र्यक्षचतुरश्रयोः फलं तत्सम्बन्धि घनफलं चोक्त्वा वृत्तक्षेत्रफलं तद्घनफलं चार्यार्धाभ्यामाह\textendash 

\begin{quote}
{\ab समपरिणाहस्यार्धं विष्कम्भार्धहतमेव वृत्तफलम्~।\\
तन्निजमूलेन हतं घनगोलफलं निरवशेषम्~॥~७~॥}
\end{quote}

इति~। समपरिणाहं च वक्ष्यति \textendash\ {\qt वृत्तं भ्रमेण साध्यम्} इति~। वृत्तं भ्रमेण साध्यम्~। भ्रमतीति भ्रमः कर्कटादिः~। तेन वृत्तं साध्यम्~। यद्वा भ्रामणेन कर्कटादेर्भ्रामणेन वृत्तं साध्यमित्यर्थः~। किमर्थं पुनरेवं साध्यते~। कज्जलादिरूषितेन वर्तिकाङ्कुरेणालेखनमात्रेणैव क्रियतां चित्रकर्मकुशलैः~। एवं हि लोके वृत्तमालिख्यत इति चेत्~। तत्र न समपरिणाहत्वं स्यात्, तन्नियामकाभावात्~। दृढेन कर्कटादिनालिख्यमाने तस्यैकमग्रं वृत्तकेन्द्रभाग एव दृढमवष्टभ्येतराग्रभ्रामणे क्रियमाणे परिध्यवयवानां
सर्वेषां वृत्तकेन्द्रात् समविप्रकर्षत्वं निर्णीतमेव~। कर्कटस्य दृढत्वेन कादाचित्कस्य सन्निकर्षस्य विप्रकर्षस्याप्यभावात् भ्रमतस्तस्याग्रान्तरालस्य सर्वदापि तुल्यत्वमेव स्यात्~। एवं समपरिणाहत्वनिर्णयः~। एवं भूतस्य परिणाहस्यार्धं विष्कम्भार्धेन हन्यात्~। तत्र यत् फलं तदेव वुत्तफलं, न न्यूनं नाप्यतिरिक्तमिति निर्णीयते~। कथं निर्णीयते~। उच्यते~। वृत्तक्षेत्रावयवाः सर्वे हि सूच्याकाराः, यत(स्त)न्नाभितः प्रवृत्तास्तन्नेमिपर्यवसानाः पृ(ख ? थ्व)ग्राः\renewcommand{\thefootnote}{२}\footnote{प्रघग्रा \textendash\ क. पाठः.}~।


\newpage

\noindent तेषामनन्तत्वेऽपि यावदपेक्षं सूक्ष्मत्वमापाद्य छिद्यमानानां सर्वेषां केन्द्रस्पृष्टाग्रत्वेन हि सूच्याकारता~। तेषु द्वन्द्वशो व्यस्तमूलाग्रतया संश्लिष्टेषु द्वन्द्वानां सर्वेषामायतचतुरश्रत्वं स्यात्~। ततस्तैरारब्धं क्षेत्रमप्यायतचतुरश्रं स्यात्~। यथा द्व्यणुकादिक्रमेण कार्यमारभ्यते, एवमत्राप्यणीयसां तदवयवानां द्वन्द्वशः संहितानामेव वृत्तक्षेत्रारम्भकत्वं कल्प्यते~। तथा सति सम\renewcommand{\thefootnote}{१}\footnote{तथा सम}परिणाहार्धतुल्यस्तस्यायामः~। विष्कम्भार्धतुल्यश्च विस्तारः~। ननु द्वन्द्वशः श्लिष्ठानामायामो विष्कम्भार्धतुल्यः~। ततः\renewcommand{\thefootnote}{२}\footnote{श्लिष्ठानां विष्कम्भार्धतुल्य आयामः~। तदा \textendash\ क. पाठः.} तैरारब्धस्य क्षेत्रस्याप्यायामेन विष्कम्भार्धतुल्येन भाव्यम्~। नैतद(स्त्रि ? स्ति)~। योगे हि द्वे पार्श्वे दीर्घे~। द्वे च ह्रस्वे~। तत्र ये पार्श्वे दीर्घे ये च ह्रस्वे तत्र दीर्घतुल्ये आयामः, ह्रस्वतुल्यो विस्तारः~। तस्मा(न ? न्न) द्वन्द्वशः श्लिष्टानां विप्कम्भार्धतुल्य आयामः~। तदारब्धस्य क्षेत्रस्य पुनः परिणाहार्धतुल्य एवायामः~। परिणाहार्धात् विष्कम्भार्धस्य न्यूनत्वात्~। न्यूनत्वञ्च\textendash 

\begin{quote}
{\qt चतुरधिकं शतमष्टगुणं द्वाषष्टिस्तथा सहस्राणाम्~।\\
 अयुतद्वयविष्कम्भस्यासन्नो वृत्तपरिणाहः~॥}
\end{quote}

\noindent इति वक्ष्यमाणन्यायेन सिद्धम्~। कथं पुनः परिणाहार्धतुल्यत्वमायामस्य~। व्यस्ताग्रतया द्वन्द्वशः श्लिष्टत्वात् परिणाहस्यैकमर्धमेकस्मिन् पार्श्वे इतरदन्यत्र चेति परिणाहार्धदीर्घत्वम्~। वृत्तक्षेत्रं हि व्यासमार्गेण द्वेधा विभज्य तदर्धगतस्य नेमिभागस्य धनुराकारत्वात् तदग्रे हस्ताभ्यां गृहीत्वा धनुराकारस्यर्जूक्रियमाणस्यावयवा विश्लिष्टाः स्युः~। सर्वाग्राणांवृत्तनाभावुपसंहृतानाम् ऋजूक्रियमाणास्यानां स्यादेवेतरेतर\renewcommand{\thefootnote}{३}\footnote{तरं \textendash\ ख. पाठः.} विवरं वृत्तावयवशून्यम्~। एवमुभयार्धाभ्यां संश्लिष्टाभ्यां तद्विवरपूरणं कार्यम्~। एवं सत्यायतचतुरश्रक्षेत्रतामापद्येतेति भावः~। एवं भूताश्चावयवा एवहि राश्यादयः~। तथाचोक्तं\textendash 

\begin{quote}
{\qt तैलिकचक्रस्य यथा विवरमराणां भवति नाभ्याः~।\\
 ने(म्या ? म्यां) महत् तथैव स्थितानि राश्यन्तराण्यूर्ध्वम्~॥}
\end{quote}
\noindent इति~। अस्यायमर्थः \textendash\ यथा तैलिकचक्रस्याराणां तन्नाभिप्रवृत्तानां नाभितः प्रभृति क्रमेण वर्धमानं तद्विवरं नेम्यां हि सर्वतो महत्, राश्यन्तराण्यप्यूर्ध्वं

\newpage

\noindent तथैव स्थितानि व्यवस्थितानि~। राश्यन्तराणि राश्यात्मकान्यराभ्यन्तराणि क्षेत्राणि~। यद्वा राशीनामन्तराणि राशीनामादेरन्तस्य चान्तराणि~। तान्यपि राशिसम्बन्धीनि अरान्तराणीति यावत्~। अयममिप्रायः \textendash\ भगोलनाभेः प्रभृति प्रवृत्तानां द्वादशषष्टिशतत्रयादिसङ्ख्यानां राशिभागकलादिप्रविभागार्थं कल्प्यमानानामराणां चन्द्रादिकक्ष्याप्रदेशेषु विवरं नानापरिमाणं
भगोलपृष्ठे इतरप्रदेशविवरेभ्यो महत् स्यादिति~। एवकारेण परिधिगतवक्रत्वमस्मिन्नायतचतुरश्रेऽप्याशङ्क्यमानं व्युदस्यते~। तत् पुन\renewcommand{\thefootnote}{१}\footnote{स्यते~। पुनः}र्निजमूलेन हतं घनगोलफलम्~। तच्च निरवशेषं स्यात्~। अत्र मूलशब्देन वर्गमूलमिष्यते~। कुतः~। एवमिदमायतचतुरश्रं क्षेत्रम्~। तस्य समचतुरश्रत्वमप्यापादनीयम्~। तस्य चतुरश्रत्वमापन्नस्य तुल्याश्चत्वारो बाहवः कियन्त इत्यत्र वर्गमूलीकरणस्येष्टत्वात्~। अस्मिन् फले मूलिते पुनस्तन्निर्मितचतुरश्रबाहुः स्यात्~। एवं वृत्त\renewcommand{\thefootnote}{२}\footnote{स्यात्~। वृत्त \textendash\ क. पाठः.}क्षेत्रेण समचतुरश्रं सम्पादनीयम्~। एवं घनगोलस्य समद्वादशाश्रत्वमापन्नस्यापि तच्चतुरश्रबाहुतुल्य एव द्वादश बाहवः~। तस्मात् तद्बाहुघन एव गोलघनफलमितीदं वृत्तक्षेत्रफलं चतुरश्रे कल्प्यमानं स्वमूलेन हतं स्वमूलस्य घन एव सम्प\renewcommand{\thefootnote}{३}\footnote{म्पा}द्यते~। यतः सदृशत्रयसंवर्गो घन इत्येतत् फलं स्वमूलहतमेव घनगोलफलमिति~॥~७~॥\\ 

अथ विषमचतुरश्रगतं न्यायकलापं प्रदर्शयितुकामस्तत्सारभूतं\renewcommand{\thefootnote}{४}\footnote{त \textendash\ ख. पाठः.} पातरेखादिस्वरूपं दर्शयते\textendash

\begin{quote}
{\ab आयामगुणे पार्श्वे तद्योगहृते स्वपातरेखे ते~।\\
विस्तरयोगार्धगुणे ज्ञेयं क्षेत्रफलमायामे~॥~८~॥}
\end{quote}
इति~। त्रिविधं हि चतुरश्रं समचतुरश्रमायतचतुरश्रं विषमचतुरश्रं चेति~। तत् पुनः प्रत्येकं द्विविधं नियतकर्णमनियतकर्णं चेति~। तत्र नियतकर्णस्य समचतुरश्रस्य च, तुल्यावेव कर्णौ, अनियतकर्णस्यैवातुल्यौ~। विषमचतुरश्रे पुनर्नियतकर्णे तयो\renewcommand{\thefootnote}{५}\footnote{योरपीतरम \textendash\ क. पाठः.}र्नियतयोरपीतरेतरमतुल्यत्वं स्यादिति ताभ्यामस्य वैलक्षण्यम्~। अनियतकर्णं च द्विविधं समलम्बं विषमलम्बं चेति~। विषमत्वमपि बाहूनां वैषम्यात्~। तत्र चत्वारो बाहवोऽपि परस्परं भिन्नाः 

\newpage

\noindent स्युः द्वौ वा त्रयो वा~। अवश्यमेकस्य भेदेन भवितव्यम्~। इतरत्रयस्य तुल्यत्मपि सम्भवति~। एवम्भूतमिदं विषमचतुरश्रं पूर्वं
प्रदर्शितत्र्यश्रपरतन्त्रमेवेति तस्य न स्वातन्त्र्यम्~। तत्र यत् समलम्बं विषमचतुरश्रं तत्रैव पातरेखानयनमुच्यते~। तद्युक्तिः पूर्वमेव प्रदर्शिता~। ततः क्रि (यां द्वे\renewcommand{\thefootnote}{१}\footnote{क्रयां देव \textendash\ ख. पाठः.} ? याभेद) एव विशेषः~। तथापि त्रैराशिकत्वं न हीयेतेति ततो विरम्यते~। समलम्ब एव फलानयनमप्याह \textendash\ आयामे विस्तरयोगार्धगुणे क्षेत्रफलमपि ज्ञेयम्~। तस्मिन् समलम्बे भूमुखयोगार्धगुणे यत् स्यात् तत् समलम्बविषमचतुरश्रक्षेत्रगतं फलं स्यादित्यत्रापि लम्बद्वयान्तर्गतो भाग आयतचतुरश्र एव~। तत्र च लम्ब एवायामः~। मुखतुल्यौ पुनरितरौ बाहू~। तेन तत्तुल्य एव च विस्तारः~। यौ पुनस्ततः सव्येतरभागगतौ\renewcommand{\thefootnote}{२}\footnote{गतौ तौ} भागौ तौ चार्धायतचतुरश्रात्मकौ~। तयोः स्वस्वबाहुरेव कर्णः~। लम्ब एव कोटिश्च~। उभौ भूम्यग्रगतखण्डौ च तद्बाहू~। तच्च प्रत्येकं भुजामध्ये छित्त्वा कर्णार्धे उभे संश्लेष्यायतचतुरश्रीकृते\renewcommand{\thefootnote}{३}\footnote{श्रकृतेः भाग \textendash\ क. पाठः.} (त)द्भागस्यापि लम्ब एवायामः~। स्वभुजार्धं विस्तारः~। एवमुभे अपि स्वभुजार्धविस्तारे क्षेत्रे~। एवं कृते कृत्स्नस्याप्यायतचतुरश्रस्य लम्ब एवायामः~। मुखभूम्यर्धतुल्यो विस्तारः~। अतस्तद्घातः फलं स्यादिति~॥~८~॥\\

एवं यथा त्र्यश्रवृत्तविषमचतुरश्रादिक्षेत्राणामायतचतुरश्रतामापाद्य तदायामविस्तारघातः फलत्वेनोक्तः, एवमपशिष्टानामपि विस्तारायामौ प्रमाध्य फलं
नेयमित्ययमेव न्यायः सर्वत्रातिदिश्यते\textendash 

\begin{quote}
{\ab सर्वेषां क्षेत्राणां प्रसाध्य पार्श्वे फलं तदभ्यासः~।}
\end{quote}
इति~॥ ${\hbox{८}}\dfrac{\hbox{१}}{\hbox{२}}$~॥ \\

एवं फलप्रकरणमुपसंहृत्य ज्याप्रकरणमारभ्यते\textendash 

\begin{quote} 
{\ab परिधेः षड्भागज्या विष्कम्भार्धेन सा तुल्या~॥~९~॥}
\end{quote} 

इति~। षोढा विभक्तस्य वृत्तपरिधेर्य एको भागस्तद्गतसमस्तज्या या सा विष्कम्भार्धेन तुल्या~। तद्युक्तिप्रदर्शनाय पूर्वप्रदर्शितवृत्तक्षेत्रे त्र्यश्रादिकर्मार्जयित्वा अरषट्कावशेषे अरविवरेषु षट्सु षड्ज्याः परिधिस्पृष्टोभयाग्राः कार्याः~। तत्रास्य वासना प्रदर्श्या~। एवं कृते समत्र्यश्रे क्षेत्राणि विष्कम्भार्धतुल्यबाहुकानि च षड् भवन्ति~। तेषां बाहुद्वन्द्वानां व्यासार्ध-

\newpage

\noindent तुल्यत्वं सिद्धमेव~। भूम्यात्मकानामेव बाहूनां ज्यारूपाणां व्यासार्धतुल्यत्वमिह बोध्यते~। तत्रापि व्यासार्धतुल्यायाः परिधिषड्भागावगाढत्वमिह निर्णीयते~। विष्कम्भार्धतुल्यां शलाकां जीवां वा व्यासाग्रस्पृष्टैकाग्रां परिधिस्पृष्टोभयाग्रामेकां भुजां प्रकल्प्य तदग्रस्पृष्टव्यासार्धं भूमित्वेन च कल्पयित्वा जीवाग्रान्तरात् प्रभृति केन्द्रान्तमन्यबाहुं च कल्पयेत्~। तदा व्यासार्धमध्य एव तल्लम्बः पतति, समत्र्यश्रत्वात् तस्य~। एवमुभयोः पार्श्वयोः कल्प्यमानयोरन्तरालमपि व्यासार्धतुल्यं, व्यासार्धतुल्ययो(र्भ ? र्भु)वोरर्धयो\renewcommand{\thefootnote}{१}\footnote{रन्तरयो \textendash\ क. पाठः.}रेकीकृतत्वात्~। सव्यत्र्यश्रस्य भुवो दक्षिणार्धं, दक्षिणत्र्यश्रस्य भुव उत्तरार्धं चैकीकृतं लम्बविवरावगाढमिति तदेव लम्बाग्रान्तर्नीयमानं वृत्तस्योर्ध्वार्धे पार्श्वद्वयगतत्र्यश्रावशिष्टपरिधिभागज्यात्वमाप्नोति~। तच्च व्यासार्धतुल्यम्~। त्र्यश्रबाह्वात्मिके उभयपार्श्वगते उभे जीवे अपि व्यासार्धतुल्ये एव~। इतरथा लम्बोऽत्र न व्यासार्धसूत्रमध्ये पतति~। तस्मात् वृत्तस्यैकार्धे व्यासार्धतुल्याभिस्तिसृभिर्जीवाभिः कृत्स्नं परिध्यर्धं चापत्वेन स्वीकृतं स्यात्~। एवमितरार्धेऽपि तिस्रः समाना व्यासार्धतुल्या जीवाः स्युः~। तस्मात् परिधेः षड्भागज्या विष्कम्भार्धतुल्यैवेति निर्णीयत इति~॥~९~॥ \\

परिधिव्यसयोर्मिथः परिमाणतः सम्बन्धं प्रतिपिपादयिषुः प्रथमं प्रायिकं तयोः सङ्ख्यासम्बन्धं प्रतिजानीते\textendash

\begin{quote}
{\ab चतुरधिकं शतमष्टगुणं द्वाषष्टिस्तथा सहस्राणाम्~।\\
अयुतद्वयविष्कम्भस्यासन्नो वृत्तपरिणाहः~॥~१०~॥}
\end{quote}
इति~। व्यासस्या(र्य ? यु)तद्वयांशैरष्टगुणितचतुरधिकशतोत्तरद्वाषष्टिसहस्रैर्मितः परिधिरिति परिधिव्यासयोः सङ्ख्यासम्बन्धः प्रदर्शितः~। तेनैव सिद्धं
परिधेरष्टगुणितचतुरधिकशतद्वाषष्टिसहस्रांशैरयुतद्वयसङ्ख्यैर्मितो व्यास इति च~। एवमनयोः सम्बन्धोऽल्पेषु महत्स्वपि वृत्तेषु सर्वत्र समान एव~। आसन्नः, आसन्नतयैवायुतद्वयसङ्ख्यविष्कम्भस्येयं परिधिसङ्ख्योक्ता\renewcommand{\thefootnote}{२}\footnote{सङ्ख्या स्वयमुख्योक्ता? \textendash\ ख. पाठः.}~। कुतः पुनर्वास्तवीं सङ्ख्यामुत्सृज्यासन्नैवेहोक्ता~। उच्यते~। तस्या
वक्तुमशक्यत्वात्~। कुतः~। येन मानेन मीयमानो व्यासो निरवयवः स्यात्, तेनैव मीयमानः परिधिः पुनः सावयव एव स्यात्~। येन च मीयमानः परिधि-

\newpage

\noindent र्निरवयवस्तेनैव मीयमानो व्यासोऽपि सावयव एव, इत्येकेनैव मानेन मीयमानकेरुभयोः क्वापि न निरवयवत्वं स्यात्~। महान्तमध्वानं गत्वाप्यल्पावयवत्वमेव लभ्यम्~। निरवयवत्वं तु क्वापि न लभ्यमिति भावः\renewcommand{\thefootnote}{१}\footnote{लभ्यमित्यर्थः~। \textendash\ क. पाठः.}~। कुतः पुनरनयोः शक्यापवर्तनत्वेऽप्यनपवर्त्यैव महान्तौ राशी प्रतिपादितौ~। भास्करस्तु षोडशभिरपवर्त्यैवोक्तवान्\textendash 

\begin{quote}
{\qt व्यासे भनन्दाग्निहते विभक्ते खबाणसूर्यैः परिधिः सुसूक्ष्मः~।}
\end{quote}

\noindent इति~। तथा च लाघवं स्यात् गणितस्य~। अतः सम्भवल्लघूपायत्वाख्यो दोषोऽपि स्यादिति चेत्~। तत्र हि परिधेरर्धीकरणादौ सावयवत्वं स्यात्~। अत्र तु परिध्यर्धपादादीनां निरवयवत्वमेव स्यात्~। किञ्च व्यासार्ध\renewcommand{\thefootnote}{२}\footnote{स्यात्~। व्यासार्ध}षड्भागज्यादीनामपि निरवयवत्वं स्यात्~। गुणनाप्यत्रैव लघ्वी, अङ्कबाहुल्याभावादिति न कश्चिद्दोषः~। प्रत्युत उक्तगुणयोगादियमेवोक्तिः साधीयसी~। यत् पुनरस्यासन्नतया जायमानं स्थौल्यं, तदपि न ग्रहगणिते फलति~। यतस्तत्परास्वेव वैषम्यं स्यात्~। व्यासार्धस्यापि तद्धेतुकं भुजाफलादिषु जायमानं स्थौल्यं पुनस्ततोऽप्यल्पमेव~। सङ्गमग्रामजो माधवः पुनरत्यासन्नां परिधिसङ्ख्यामुक्तवान्\textendash 

\begin{quote}
{\qt विबुधनेत्रगजांहिहुताशनत्रिगुणवेदभवारणबाहवः~।\\
नवनिखर्वमिते वृ\renewcommand{\thefootnote}{३}\footnote{हृ \textendash\ ख. पाठः.}तिविस्तरे परिधिमानमिदं जगदुर्बुधाः~॥}
\end{quote}

\noindent इति~। अत्र व्यासनिखर्वनवकांशतुल्येन मापकेन मीयमानः परिधिर्विबुधेत्यादिनोक्तः~। अतोऽयमतिसूक्ष्मः~। मापकस्य न्यूनत्वादल्पावयवत्वाच्च~। ताभ्यां फलप्रमाणाभ्यामेव व्यासपरिध्यो\renewcommand{\thefootnote}{४}\footnote{परिध्यो \textendash\ ख. पाठः.}र्ज्ञातेनेतरोऽनुमेयः~। समव्याप्तिकत्वादुभयोर्यथाविवक्षं व्याप्यव्यापकभावात्~। अत एवाहुः\textendash 

\begin{quote}
{\qt कृतकानित्यवद् व्यासपरिधी नियतौ मिथः~।}
\end{quote}

\noindent इति~। अतस्तदर्थं तयोर्नियम उक्तः~॥~१०~॥\\

ग्रहगणिते पुनर्ज्योतिश्चक्रस्य लिप्तादिना मीयमानस्य परिधेः खखषड्घनादिसङ्ख्यत्वेन व्यवस्थितत्वात् तद्व्यास\renewcommand{\thefootnote}{५}\footnote{व्यास \textendash\ क. पाठः.} एवानेयः, तेन ज्याबाणौ चेति तदर्थं क्षेत्रच्छेदः प्रदर्श्यते\textendash 

\begin{quote}
{\ab समवृत्तपरिधिपादं छिन्द्यात् त्रिभुजाच्चतुर्भुजाच्चैव~।\\
समचापज्यार्धानि तु विष्कम्भार्धे यथेष्टानि~॥~११~॥}
\end{quote}

\newpage
इति~। ग्रहगणिते त्विहार्धात्मकैरेव गुणैरुपयोगः~। वक्ष्यति च\textendash 

\begin{quote}
{\qt दृग्गोलार्धकपाले ज्यार्धेन विकल्पयेद् भगोलार्धम्~।}
\end{quote}

\noindent इति~। ततस्तैरेव स्फुट\renewcommand{\thefootnote}{१}\footnote{षट्}क्रियादिषूपयोगः~। अतएव गीतिसूत्रेऽपि कलार्धज्याः पठिताः~। तासामिहानयनमारभ्यते~। तत्र चतुर्विंशतिरेव खण्डज्याः पठिताः~। अत्र पुनस्ततोऽपि भूयसां ज्यार्धानां यावदपेक्षमानयनं प्रदर्श्यते~। न्यायस्य सर्वत्र तुल्यत्वात्, न्यायप्रदर्शनपरत्वाच्चास्य सूत्रस्य~। ज्यार्धानामानन्त्येऽपि न्यायः कृत्स्नेऽपि प्रसरतीति न्याय\renewcommand{\thefootnote}{२}\footnote{ज्ञान \textendash\ ख. पाठः.}विदा यावत्परितोषं भूयांस्यप्यानेयानीति यथेष्टानीत्युक्तम्~। समचापज्यार्धानि इतरेतरं समानां चापखण्डानां ज्यार्धान्यानीय पठनीयानीत्यर्थः~। अथवा समं स्वचापं यस्य ज्यार्धस्य तत् समचापम्~। समचापं च तज्ज्यार्धं चेति समचापज्यार्धं, तानि समचापज्यार्धानि यथेष्टान्यानेयानीति~। तेन परिध्यानयनं सूचितम्~। अर्धज्याश्च भुजाकोटिरूपतया चतुर्धा विभक्तस्य वृत्तस्यैकस्मिन् पाद एव प्रदर्श्याः~। तत्रैव तासां परिसमाप्तत्वादितरेष्वपि तत्साम्याच्च~। अत उक्तं {\qt समवृत्तपरिधिपादं छिन्द्यात्} इति~। कथं पुनस्तच्छेदः~। त्रिभुजाच्चतुर्भुजाच्च~। तदन्तस्त्रिभुजक्षेत्रकल्पनया चतुर्भुजक्षेत्रकल्पनया च~।
कथं पुनस्तत्कल्पना~। तत्र तावत् वृत्तक्षेत्रस्येशानको\renewcommand{\thefootnote}{३}\footnote{शको \textendash\ ख. पाठः.}णगते पादे ज्याच्छेदविधानं प्रदर्श्यते~। व्यासेन हि वृत्तं व्यस्यते~। मिथो व्यस्तदिक्काभ्यां द्वाभ्यां व्यासाभ्यां हि वृत्तपादाः परिच्छिद्यन्ते~। अतः केन्द्रात् प्रागायतेन व्यसार्धेनोदगायतेन चायं पादः परिच्छिद्यते~। ततस्तत्कर्णात्मिका ज्या परिधिपादस्य समस्तज्या~। तत्पादान्तर्गतेन त्र्यश्रक्षेत्रेण वृत्तपादस्य छेदः क्रियते~। तस्यास्तत्कर्णभूताया जीवाया अर्धमध्यर्धराशेरर्धज्या द्वादशी~। एकराशिज्या तु परिधेः षड्भागस्य समस्तज्याया विष्कम्भार्धतुल्यत्वोक्तेः सिद्धा व्यासार्धस्यार्धतुल्येति~। आभ्यां व्यासार्धेन चान्या एकविंशतिरर्धज्या आनीयन्ते~। तद्यथा \textendash\ पूर्वापरसूत्रपूर्वाग्रात् प्रभृत्युत्तरतः परिधौ राशिलिप्तान्तरे प्रदेशे बिन्दुं कृत्वा वृत्तकेन्द्रात् प्रभृति तत्पर्यन्तं सूत्रं नीत्वा कर्णरेखा कार्या~। वृत्तकेन्द्रात् प्रभृति पुनरुत्तरसूत्रे राश्यर्धज्यातुल्ये प्रदेशे 

\newpage

\noindent तत्कर्णाग्रान्तरालावगाहिनी रेखा कोटिरूपा~। केन्द्रात् प्रभृत्येव पुर्वसूत्रेऽपि कोटितुल्यान्तरे बिन्दुं कृत्वा तत्कर्णाग्रावगाहिनी भुजारेखा च
कार्या~। एवं सत्यायतचतुरश्रं क्षेत्रमुत्पद्यते~। एवं चतुरश्रद्वारा भुजायां ज्ञातायां कोट्यानयनं कार्यम्~। उभयोर्ज्ञातयोः पुनस्तद्बाणानयनं तु तदर्धचापात्मकगतत्र्यश्रकल्पनया~। तत्रैकराशिज्यायाः कोटिरेव राशिद्वयकाष्ठार्धज्या षोडशी~। एवं चतस्रो जीवाः सिद्धाः~। व्यासार्धादितरज्यायां शुद्धायां यः शेषः स एव हि स्वबाणः~। तस्मादेकराशिज्याबाणार्थं षोडशी ज्या व्यासार्धाच्छोध्या~। तदप्यस्मिन्नायतचतुरश्रे द्रष्टव्यम्~। तत्र दक्षिणबाहोः कोटिज्यातुल्यत्वात् तदति\renewcommand{\thefootnote}{१}\footnote{अति}रिक्तव्यासार्धखण्डस्यैकराशिबाणत्वात्~। एवं बाणे सिद्धे सति तत्परिधिसंयोगात् स्वज्यापरिधि\renewcommand{\thefootnote}{२}\footnote{परिधि}संयोगप्रापिणीं रेखां कुर्यात्~। सैव राशेः समस्तज्या~। तदर्धं च राश्यर्धस्य ज्यार्धम्, अतः सा चतुर्थी~। तत् समस्तज्यातुल्यां शलाकां
पूर्वसूत्रस्पृष्टमध्यां परिधिस्पृष्टोभयाग्रां कुर्यात्~। तदा तस्योदगर्धं राश्यर्धचापस्य ज्यार्धम्~। तत्र च रेखां कृत्वा तत्केद्रात् तदग्रप्रापिणीं रेखां कुर्यात्~। सा च व्यासार्धतुल्या~। सैव श्रुतिः~। केन्द्रादुत्तरसूत्रेऽपि चतुर्थज्यातुल्येऽन्तरे बिन्दुं कृत्वा प्राग्वदेव कर्णभुजाग्रप्रापिणीं रेखां कुर्यात्~। पूर्वसूत्रेऽपि केन्द्रात् प्रभृति तावती कोटिः~। तदूर्ध्वखण्डश्च चतुर्थो बाणः~। एवं चतुर्थ्या भुजात्मिकया तत्कोटिरूपा विंशी ज्या साध्या~। तत्रापि बाणज्यापरिधिसंयोगान्तया समस्तज्यया चापक्षेत्रं छि\renewcommand{\thefootnote}{३}\footnote{जि}त्त्वा तत्कर्णार्धतुल्या द्वितीया ज्याप्यानेया~। एवमुभाभ्यां त्रिभुजचतुर्भुजाभ्यां वृत्तपरिधिपादं मुहुर्मुहुः छित्त्वा चतुर्विंशतिरर्धज्याः साध्याः~। अतो वा भूयस्यः अष्टाचत्वारिंशदादितद्द्विगुणोत्तरसङ्ख्याः~। एवं पूर्वापरायता दक्षिणोत्तरायताश्चोभय्य\renewcommand{\thefootnote}{४}\footnote{य \textendash\ क. पाठः.}श्चतुर्विंशतिसङ्ख्याः स्युः~। एवं ज्याच्छेदविधाने कृते खण्डज्याश्च मख्याद्यक्षरपठिता विष्कम्भार्धे द्रष्टव्याः~। तदुक्तं विष्कम्भार्ध इति~। कथं पुनर्मख्यादयः कलार्धज्या विष्कम्भार्धे दृश्याः~। चक्षुषोरुन्मीलनेनैव~। तत्र प्रथमज्यायाः कोटिस्त्रयोविंशी ज्या
पूर्वापरायता उदग्व्यासार्धसूत्रे यत्र स्पृशति, तत्केन्द्रान्तरालं मखिपरिमाणम्~। यत्र च द्वाविंशी स्पृशति तस्यामेवोदगायतायां रेखायां वृत्तकेन्द्रात् प्रभृति
तदन्ता

\newpage

\noindent द्वितीया ज्या~। द्वितीयातृतीययोः कोट्योरन्तरालतुल्यो विष्कम्भार्धगतो यः खण्डः स भ\renewcommand{\thefootnote}{१}\footnote{म \textendash\ क. पाठः.}खितुल्यः~। एवं तस्मिन्नेव विष्कम्भार्धे
निरन्तरकोटिज्यापरिच्छिन्ना ये खण्डास्त एव म\renewcommand{\thefootnote}{२}\footnote{भ \textendash\ ख. पाठः.}ख्यादयो ज्याच्छेदविधानेनैव तत्र द्रष्टव्याः~। न पुनस्तदर्थमन्यो यत्नः कर्तव्यः~। एवं प्राक्सूत्रेऽपि भुजाज्यापरिच्छिन्नाः खण्डाः केन्द्रात् प्रभृति मख्यादितुल्याः~। जीवाश्च परिधिस्पृष्टा लिखिता रेखा एव~। तत्खण्डाः पुनर्व्यासयोरुभयोरेव प्रत्येकं दृश्याः~। जीवाः पुनर्नानादेशस्थाः व्यासाग्रात् प्रभृति स्वस्वबाणानुसारेण क्रमेण वृत्तकेन्द्रासन्नाः~। एवं ज्या एवोत्क्रमेण गण्यमाना बाणखण्डाः~। तस्माच्छादय एवोत्क्रमेण खण्डा गणिता उत्क्रमज्याख्या बाणा इति मख्यादिभिरेवोत्क्रज्याकार्यमपि सेत्स्यतीति भावः~। परिधिमानमप्यनयैव दिशा ज्ञेयम्~। कथं तत्र न तृतीयादिज्या(वे ? स्वे)का\renewcommand{\thefootnote}{३}\footnote{ज्याप्येकोऽप्य \textendash\ क. पाठः.}प्यस्ति\renewcommand{\thefootnote}{४}\footnote{कास्ति \textendash\ क. पाठः.}~। चापज्ययोरल्पत्वापादनमेव हि तत्र कार्यम्~। एकराशेः प्रभृति चापार्धपरम्परागतभुजाकोटिबाणा एव तदर्थमानेया इति तत्र क्रियालाघवं स्यात्~। चतुर्विंशतिज्यानयन एव ततो गौरवमिति~। तत्र चतुर्विंशतिज्यास्वष्टावेव तदर्थमानेयाः~। पुनरपि मख्यर्धचापगतभुजाकोटिबाणा आनेयाः~। पुनस्तदर्धगताः~। तत्रैव सपादषट्पञ्चाशत्कलामिते ज्याचापयोर्भेदो विलिप्ताष्टांशादीषदधिक एव~। तस्मात् तत्र यत् प्रथमं चापज्यार्धं, तस्मिन्नेव चतुरशीत्युत्तरशतत्रयगुणिते परिधिमानं स्यात्~। तत्रापि
का(भि)श्चिद् विकलाभिरेव स्थौल्यं स्यात्~। किं पुनस्ततोऽप्यर्धीकरणे~। किञ्च तत्परिहारेऽप्युपायो लघुर्विद्यते~। स उपरिष्टाद्वक्ष्यते~। एवं व्यासपरिध्योः परिमाणसम्बन्धोऽप्यत्रैवोपपादितः~। उत्तरत्रापि तत्प्रपञ्चो द्रष्टव्यः~॥~११~॥\\

एवं परिधिषण्णवत्याद्यंशेषु चापेषु प्रथमाद्वितीययोरर्धज्ययोर्ज्ञातयोरितरानयनं पुनस्त्रैराशिकेनैव कार्यमिति पूर्वप्रदर्शितात् कर्मणोऽस्यैव लाघवम्~। तत्र तु
भुजाकोटिकर्णकल्पनया प्रत्येकं वर्गमूलपरिकर्मणी कार्ये~। अत्र पुनर्गुणनहरणे एव कार्ये इति तत्त्रैराशिकप्रदर्शनायाह\textendash 

\begin{quote}
{\ab प्रथमाच्चापज्यार्धाद् यैरूनं खण्डितं द्वितीयार्धम्~।\\
तत्प्रथमज्यार्धांशैस्तैस्तैरूनानि शेषाणि~॥~१२~॥}
\end{quote}

इति~। चापमेव ज्यार्धं चापज्यार्धम्~। ज्यार्धेषु प्रथममेव हि चापतुल्यं स्यात्~। चाप\renewcommand{\thefootnote}{५}\footnote{यावत् चाप \textendash\ ख. पाठः.}साम्यमेव हि ज्याछेदविधानन्योयेनो(क्तमानत?क्तम्~। अत

\newpage

\noindent स्त)त्तदर्धचापज्यानयनं कार्यमित्याद्यस्यैव ज्यार्धस्य चापसाम्यं, (न) पुनर्द्वितीयादीनामिति~। प्रथमाच्चापज्यार्धात् खण्डितं द्वितीयार्धं
द्वितीयज्यार्धम्~। प्रथमज्यार्धोनं हि द्वितीयं खण्डज्यार्धम्~। तत् प्रथमज्यार्धाद् यैरूनं यावद्भिरूनं, तत्प्रथमज्यार्धांशैस्तैस्तैस्तावद्भिस्तावद्भिः
तत्तत्प्रथम\renewcommand{\thefootnote}{१}\footnote{तत्प्रथम \textendash\ क. पाठः.}ज्यार्धांशैरूनानि शेषाणि खण्डितानि तृतीयादिखण्डज्यार्धानि~। कुतः पुनस्तावद्भिरूनत्वम्~। स्वस्वपूर्वज्यार्धखण्डात्~। अत्र
निरन्तरयोरुभयोर्ज्यार्धयो\renewcommand{\thefootnote}{२}\footnote{र्धख(ण्ड)यो \textendash\ ख. पाठः.}रन्तराण्येवानीयन्ते~। अतः पूर्वपूर्वखण्डज्यातः फल\renewcommand{\thefootnote}{३}\footnote{प्रथम}शोधनेनोत्तरोत्तरखण्डज्याः सिद्ध्यन्ति~। तत्तत्प्रथमज्यार्धांशैरित्यत्रापि वीप्सा कार्या, यतस्तैस्तैरिति वीप्सा कार्या~। तत्तत्पिण्डज्यायाः प्रथमज्यार्धेन हृतं फलम्~। तानि तावन्ति कार्याणि, यैः\renewcommand{\thefootnote}{४}\footnote{ये} प्रथमखण्डा(द्) द्वितीयखण्ड\renewcommand{\thefootnote}{५}\footnote{प्रथमखण्ड}ज्यार्धमूनमिति~। अयमर्थः \textendash\ प्रथमद्वितीययोरन्तरेण गुणितपिण्डज्यार्धात् प्रथमज्या(ने?हृते) यल्लब्धं, तेनोनं पूर्वखण्डज्यार्धं तदुत्तरखण्डज्यार्धं स्यात्~। एतदुक्तं भवति \textendash\ द्वितीयाज्ज्यापिण्डात् प्रथमज्यार्धेन भागं हृत्वा यदाप्तं तत् पुनः प्रथमद्वितीयान्तरेण च ह\renewcommand{\thefootnote}{६}\footnote{हृ}त्वा द्वितीयात् खण्डज्यार्धाच्छोध्यम्~। तत्र शिष्टं तृतीयखण्डज्यार्धं स्यादिति~। तत्र गुणनहरणयोः क्रमभेदेन फलभेदाभावात् प्रथमद्वितीयज्याखण्डान्तरेण तत्तत्पिण्डजीवां ह\renewcommand{\thefootnote}{७}\footnote{हृ}त्वा प्रथमज्यार्धेनैव विभजेत्~। तत्फलं चानीतेषु खण्डज्यार्धेषु चरमखण्डज्यातो विशोधयेत्~। ततस्तदनन्तरखण्डज्यार्धं स्यात्~। एवमेव शेषाणि सर्वाणि खण्डज्यार्धानि स्युः~। तस्मान्मखिभख्यारेन्तरेणेषन्न्यूनेन रूपेण मखिभखियोगं द्वितीयज्यापिण्डं हत्वा मख्यैव विभजेत्~। फलमपि भखितः शोधयेत्~। तत्र शिष्टं फ\renewcommand{\thefootnote}{८}\footnote{भ \textendash\ क. पाठः.}खिसङ्ख्यं भवति~। एवं\renewcommand{\thefootnote}{९}\footnote{सङ्ख्या~। एवं \textendash\ ख. पाठः.} मख्यादित्रययोगं तृतीयज्यापिण्डं तेनैव मखिभख्यन्तरेण हत्वा तेनैव मखिसङ्ख्येन विभजेत्~। तत्र लब्धफलहीनफ\renewcommand{\thefootnote}{१०}\footnote{भ}खितुल्यं चतुर्थं खण्डज्यार्धं ध\renewcommand{\thefootnote}{११}\footnote{म \textendash\ क. पाठः.}खिसङ्ख्यं स्यात्~। एवं
तत्तज्ज्यार्धपिण्डमाद्य\renewcommand{\thefootnote}{१२}\footnote{द्यं \textendash\ ख. पाठः.}द्वितीयान्तरेणैव हत्वा प्रथमज्यार्धेनैव विभज्य लब्धं लब्धेष्वन्त्याद् विशोध्य शिष्टं तदनन्तरज्याखण्डतया ग्राह्यम्~। त\renewcommand{\thefootnote}{१३}\footnote{य \textendash\ क. पाठः.}दुक्तं शेषाणीति~। द्वितीयात् प्रभृति शेषाणि स्वपूर्वयोरन्तरस्य पूर्वशोधनेन साध्यानीति~। एतदेव त्रैराशिकं सूर्यसिद्धान्तेऽपि\textendash 

\newpage


\begin{quote}
{\qt राशिलिप्ताष्टमो भागः प्रथमज्यार्धमुच्यते~।\\
तत्तद्विभक्तुलब्धोनमिश्रितं तद्वितीयकम्~॥

आद्येनैवं\renewcommand{\thefootnote}{१}\footnote{व} क्रमात् पिण्डाद् भङ्क्त्वा लब्धोनितैर्युतैः~।\\
खण्डकैः स्युश्चतुर्विंशज्यार्धपिण्डाः क्रमादमी~॥}
\end{quote}

\noindent इत्यनेन ग्रन्थेन प्रदर्शितम्~। तत्र फलस्य प्रायेण रूपसङ्ख्यत्वात् फलगुणनं\renewcommand{\thefootnote}{२}\footnote{ने \textendash\ क. पाठः.} न प्रदर्शितम्~। अत्रापि मखिभख्योरन्तरमेकमेव~। ज्याच्छेदविधानन्यायसिद्धयोः सावयवयोः प्रथमद्वितीयखण्डयोरेवान्तरं प्रायेण सप्तविंशांशोनैका कला~। अतः फलस्य रूपसङ्ख्यत्वाश्रयणं न दुष्यति~। किन्तु तत्र फलाप्रदर्शनात् त्रैराशिकस्य निगूढत्वात् तद्युक्तिज्ञापनायेह भगवतार्यभटेन तत् त्रैराशिकं विस्पष्टं प्रदर्श्यते~। कथं पुनरत्र त्रैराशिकवाचोयुक्तिः~। उच्यते~। यद्येतावत्या पिण्डज्यया तामभितःस्थितयोस्तुल्यपरिमाणयोश्चापयोः खण्डज्यान्तरमेतावल्लभ्यते, तदैतावत्या पिण्डज्ययेमां पिण्डज्यामभितस्तत्तुल्ययोश्चापयोः खण्डज्यान्तरं कियत् स्यात्~। तस्मान्न केवलमाद्यज्याया एव प्रमाणत्वं, नाप्याद्ययोः खण्डज्ययोरन्तरस्य फलत्वञ्च, पर्यायेण सर्वासामपि जीवानां प्रमाणत्वं स्यात्~। प्रमाणभूतज्याग्रमभितः स्थितयोः वृत्तपरिधिखण्डयोस्तुल्ययोरेव खण्डज्यान्तरस्य फलत्वञ्च स्यात्~। अतोऽत्र बहूनि प्रमाणफलानि सम्भवन्ति~। इच्छा पुनरभीष्टज्या(या?)मभितश्चापखण्डयोः खण्डज्यान्तरं ज्ञेयम्~। सा हि तत्राभीष्टज्या~। कथं पुनर्ज्यासु त्रैराशिकं घटते, ज्याचापयोर्वृद्धिह्रासयोर्वै(श्व?)रूप्यात्~। ययोर्वृद्धिह्रासयोस्तुल्यरूपत्वं तयोरेवेतरेतरं लिङ्गलिङ्गिभाव उपपद्यते~। अत्र तु न चापवृद्ध्यनुरूपैव तज्ज्यावृद्धिरिति त्रैराशिकस्याप्रवृत्तेस्त्रिभुजादिक्षेत्रपरिकल्पनयैव तदानयनं प्रदर्शितमिति
चेत्~। नैष दोषः~। चापेन ज्यानयन एव त्रैराशिकस्याप्रवृत्तिः, ज्यया\renewcommand{\thefootnote}{३}\footnote{ज्या \textendash\ ख. पाठः.} चापानयनेऽपि~। तयोरेव वृद्धिह्राससाम्याभावात्~। जीवासु पुनः परस्परं लिङ्गलिङ्गिभावसम्भवात् त्रैराशिकं युक्तमेव~। यथैकवृत्तगताभिर्ज्याभिरितरवृत्तगतानां ज्यानां तुल्यमानेन मीयमानानामानयनं तत्रतत्रोच्यमानं घटते, एवमत्रापि ज्यावृद्धिह्रासवशादेव तत्खण्डान्तरवृद्धिह्रासावपीति जीवानां खण्डान्तराणां च मिथो नियमाज्ज्ञातेनान्यानयनं युक्तमेव त्रैराशिकेनेति~। अत्र 

\newpage

\noindent वासना खण्डज्यानयनद्वारा बोध्या~। कथं पुनः खण्डज्यानयनं तद्वासना वा~। उच्यते~।

\begin{quote}
{\qt एकचापसमस्तज्यां श्रुतिरूपाखिलेष्वपि~।\\
चापभागेष्विहेच्छा स्यान्मानं व्यासदलं तथा~॥\\
तत्तत्कार्मुकमध्याग्रे कोटिदोर्ज्ये फले उभे~।\\
इच्छाफले तु दोःकोट्योः खण्डज्ये ज्ञेयता ययोः~॥\\
त्रैराशिकद्वयं कार्यं चापे चापे तयोश्च तैः~।}
\end{quote}

\noindent तत्प्रदर्शनाय समवृत्तमालिख्य मातृपितृरेखे कृत्वा तत्परिधिं चापभागाङ्कितं कृत्वा व्यासार्धतुल्यां शलाकां निर्माय तदग्रादधः अर्धचापबाणान्तरे तत्समतिर्यक्शलाकामेकचापसमस्तज्यातुल्यं दृढीकृत्य तन्मूलं वृत्तकेन्द्रगं कृत्वा भ्रामयेत्~। तस्यां भ्राम्यमाणायां तदवधिकभुजाकोटिज्यास्तत्खण्डज्याश्च सर्वा एव प्रदर्श्याः~। तदग्रं यदा प्रथमचापखण्डमध्यं स्पृशति तदा तच्छलाकाग्रे अपि प्रथमचापाग्रद्वयं स्पृशतः~। तदा प्रथमचापगते ये दोःकोटिखण्डज्ये तयोरानयनयुक्तिः प्रतिपा(द्य~?~द्या)~। तत्र\renewcommand{\thefootnote}{१}\footnote{यक्तिः~। तत्र \textendash\ ख. पाठः.} शलाकाद्वयानुसारिण्यौ द्वे एव रेखे कृत्वा वा तद्युक्तिर्निरूप्या~। तत्र चापमध्याग्रा व्यासार्धतुल्या या रेखा, सा ह्येका श्रुतिः~। तद्विपरीता समस्तज्यातुल्या या तच्चापावगाहिनी रेखा, साप्यन्या~। तयोर्व्यासार्धतुल्यायाः प्रमाणत्वमिच्छात्वम् अन्यस्याश्च~। ये पुनर्व्यासार्धाग्रस्पृष्टे कोटिभुजज्ये ते एव फले~। या पुनस्तस्य चापस्यार्धज्या दक्षिणोत्तरायता, सा चैकस्मिंस्त्रैराशिक इच्छाफलम्~। यश्चा\renewcommand{\thefootnote}{२}\footnote{यद्वा \textendash\ क. पाठः.}स्य चापस्य बाणः स चान्यस्मिन्निच्छाफलम्~। एवमन्येष्वपि चापखण्डेषु तत्तच्चापमध्याग्रा रेखा व्यासार्धतुल्यैकैव सर्वत्र प्रमाणम्~। तद्व्यस्तदिक्का तत्तच्चापाग्रान्तरालतुल्या समस्तज्यापि सर्वत्र समानैव~। तत्रतत्रेच्छाफलयोरेव पुनर्विशेषः~। तत्र भुजाज्याखण्डे ज्ञेये~। तत्तच्चपमध्याग्रा कोटिज्या प्रमाणफलम्~। इच्छाफलं च भुजाखण्डज्या~। तस्या ज्ञेयत्वात्~। कोटिखण्डानयने पुनर्दोर्ज्यैव प्रमाणस्य फलम्~। कोटिखण्डज्या चेच्छाफलम्~। कथं पुनरत्रैभिस्त्रैराशिकं युज्यते~। कर्णरूपाया इच्छायाः समस्तज्यारेखायाः कर्णरूपाया व्यासार्धरेखायाश्चेतरेतरं व्यस्तदि-

\newpage

\noindent क्कत्वे सति प्रमाणफलस्येच्छाफलस्य चेतरेतरं व्यस्तदिक्कत्वात्~। व्यासार्धात्मकस्य कर्णस्य यथा यथा पूर्वपरत्वं हीयते दक्षिणोत्तरत्वं वर्धते च इच्छात्मकसमस्तज्याया अपि तथा तथा दक्षिणोत्तरत्वं हीयते वर्धते च पूर्वापरत्वम्~। किं पुनरनयोः पूर्वापरत्वं दक्षिणोत्तरत्वं वा~। कथं वा
तयोर्वृद्धिह्रासौ~। उच्यते~। यदा तावत् पूर्वापरायतत्वं तदा न दक्षिणोत्तरत्वं मनागपि, यदा पुनर्दक्षिणोत्तरायतत्वं तदा पूर्वापरत्वमपि नैव
स्यादित्येतत् सर्वेषां सम्प्रतिपन्नमेव~। यदा पुनस्तत्कर्णस्य भ्राम्यमाणस्य पूर्वापरदिगपेक्षयेषत्तिर्यक्त्वं तदा दक्षिणोत्तरत्वमपि तावत् स्यात्~।
यतस्तयोर्वृत्तमध्यगताग्रात् परिधिगताग्रस्य भ्राम्यमाणस्योत्तरत एव स्थितिः, ततो दक्षिणोत्तरत्वमपि स्यात्~। पुनरपि भ्राम्यमाणं कियन्तञ्चित् प्रदेशं गत्वा पूर्वापरदिगपेक्षया ततोऽपि तिर्यक्त्वं प्राप्नोति यतस्तदग्रगयोः\renewcommand{\thefootnote}{१}\footnote{ग्रयोः} पूर्वापरसूत्रयोः वि\renewcommand{\thefootnote}{२}\footnote{पूर्ववि \textendash\ ख. पाठः.}प्रक(र्षात्त ? र्षस्त)दानीं महान् स्यात्~। एवं पुनःपुनरप्यग्रगतपूर्वापरसूत्रविप्रकर्षानुरूपमग्रद्वयगतयोर्दक्षिणोत्तरसूत्रयोर्विप्रकर्षस्य क्रमेण ह्रासाच्च पूर्वापरत्वं क्रमेण हीयते~। तावेव विप्रकर्षौ तत्कर्णस्य भुजाकोटी स्तः~। तत्र यदि दक्षिणोत्तरत्वं भुजायास्तर्हि पूर्वापरता कोट्याः स्यात्~। प्रमाणकर्णादिच्छाकर्णस्य समतिर्यग्गतत्वात्~। प्रमाणक्षेत्रभुजाकोटिभ्यां व्यस्तदिक्के एव इच्छाक्षेत्रगते ते इति पूर्वापरायताया ज्याया व्यासार्धकर्णस्य केटित्वाद् दक्षिणोत्तरायतैव समस्तज्याकर्णस्य कोटिः~। सैव भुजाखण्डज्या~। यतो दक्षिणोत्तरायतानां जीवानां भुजात्वमिह विवक्ष्यते, ततस्तत्खण्डानामपि दक्षिणोत्तरायतत्वमेव युज्यते~। तथाहि \textendash\ प्रथमार्धज्या तावत् प्रथमचापाग्रस्पृष्टाग्रा समदक्षिणोत्तरायतैव~। एवं चापद्वयार्धज्यापि द्वितीया~। सा च द्वितीयचापाग्रस्पृष्टा समदक्षिणोत्तरायतैव~। ये च पुनस्तयोरुभयोः कोट्यौ ते अपि तत्तद्भुजाग्रात् प्रभृति प्रत्यगायते एव दक्षिणोत्तरायतसूत्रावधिके स्तः~। तत्र प्रथमज्यायाः कोटिर्द्वितीयज्यां यत्र स्पृशति तत उत्तरतो यो द्वितीयज्यायाः खण्डः स एव द्वितीयो ज्याखण्डः~। दक्षिणखण्डश्च प्रथमज्यातुल्यः~। एवं प्रथमद्वितीयज्ययोरन्तरात्मकस्य ज्याखण्डस्य दक्षिणोत्तरायतत्वम्~। (यः? यत्) पुनः कोटिखण्डो द्वितीयज्याया ऊर्ध्वगत एव तच्चापभागे कोटिज्याखण्डः~। ततः कोटिज्याखण्डस्य सर्वत्र पूर्वा-

\newpage

\noindent परायतत्वम्~। तयोरेव समस्तज्याकर्णापेक्षया कोटिभुजात्मकत्वमपि~। यतः समस्तज्या च तस्मिंश्चापखण्डे तदग्रान्तरावगाहिनी सती भुजाकोट्यग्रान्तरालतुल्या~। भुजाकोट्यग्रविवरमेव हि कर्णश्च~। तस्मादेकचापसमस्तज्यायास्तत्र कर्णत्वम्~। तत इदं त्रैराशिकम्~। यदि व्यासार्धकर्णस्य वृत्तगता पूर्वापरायता ज्या स्वाग्रस्पृष्टा कोटिः तदास्याः समस्तज्यायाः कर्णरूपायास्तत्कर्णव्यस्तदिक्काया व्यासार्धकर्णकोटिव्यस्तदिक्का दक्षिणोत्तरायता कोटिः कियतीति भुजाखाण्डज्यानयने त्रैराशिकम्~। समस्तज्याकर्णस्य कोटित्वाद्भुजाखण्डज्यायाः~। एवं पुनस्तच्चापगतकोटिखण्डानयने त्रैराशिकम्~। एतच्चापमध्यावयवविपरीतदिक्कस्य व्यासार्धतुल्यस्य कर्णस्य भुजा यदि वृत्तगता भुजाज्या दक्षिणोत्तरायता तदा तच्चापमध्यभागसमदिक्कायास्तत्कर्णव्यस्तदिक्कायाः समस्तज्यायाः कर्णरूपा या भुजा तद्भुजावि\renewcommand{\thefootnote}{१}\footnote{भुजा वि}परीतदिक्का पूर्वापरायता कियतीति समस्तज्यायास्तत्तच्चापमध्यदिगनुसारिण्या भुजारूपा कोटिखण्डज्याप्यानीयते~। एवं भुजाकोटिखण्डज्ययोः कोटिभुजाज्याह्रासवृद्ध्यनुरूपे ह्रासवृद्धी इति कर्मेदमिह चापानामिष्यते~। स्वेषु गूढयोः खण्डज्ययोर्भुजाकोट्योर्भागानां वा कलात्मनाम् एव पदादेः प्रभृति समपरिमाणानां चापभागानां तत्समस्तज्याकर्णानां च तन्मध्यदिगनुसारेण वर्धमानानां कोटिरूपा दक्षिणोत्तरायता भुजाखण्डज्याप्यानेया~। तद्भुजारूपाणां कोटिखण्डानां बाणखण्डानामप्यानयनमेवमेव~। यदेतदुक्तं तन्न केवलं परिधिपादचतुर्विंशांशचापानामेव~। नेष्यते कतिथानां\renewcommand{\thefootnote}{२}\footnote{कथितानां \textendash\ क. पाठः.} पुनः परिध्यंशानामिति चेद् (न)~। यावतिथानां कतिथानांचित्\renewcommand{\thefootnote}{३}\footnote{कथितानाञ्चित् \textendash\ क. पाठः.} सर्वेषामेवाभीष्टानाम्~। तेन भागकलामात्रमितानामपि चापानां मध्येऽभीष्टचापखण्डगतयोर्भुजाकोटिज्ययोः खण्डज्ययोरप्यानयनमेवमेवेष्यते~। इत्युक्तं खण्डज्यानयनम्~।\\

कथं पुनराद्यात् प्रभृति ह्रसतां तत्तज्ज्याखण्डानामन्तराणि क्रमेण वर्धमानान्यानीयन्ते इति तद्युक्तिरप्यत्रैव प्रदर्श्यते~। स्वस्वचापमध्याग्रकोट्यनुसारेण तत्तच्चापभुजाखण्डानां ह्रासः, स्वचापमध्याग्रभुजानुसारेण कोटिखण्डतुल्यानां भुजाचापबाणखण्डानां क्रमेण वृद्धिश्च इत्येतदिह सिद्धम्~। यस्मात् खण्डज्यानां वृद्धिह्रासावितरेतरज्यावृद्धिह्रासानुसारेण तस्मात् तदन्त-

\newpage

\noindent रानुसारेण च तदन्तराणां स्यात्~। तद्यथा \textendash\ यथा प्रथमखण्डज्यानयनसाधनं चापार्धभुजायाः कोटिज्या तत्खण्डज्यासम्बन्धिचापमध्याग्रा तथा द्वितीयखण्डज्यायाश्चाध्यर्ध\renewcommand{\thefootnote}{१}\footnote{श्चार्ध \textendash\ क. पाठः.}भुजाचापकोटिज्या सार्धद्वाविंशतिज्यासाधनम्~। अनुरूपत्वात् तयोः~। तस्मादाद्यद्वितीयखण्डज्ययोन्तरं तन्मध्यगतकोटिज्ययोरन्तरवशात् ज्ञायते~। चापमध्यगतयोः कोट्योरन्तरं च स्वसम्बन्धिचापमध्यगतभुजाज्यावशात्~। प्रथमचापस्योत्तरार्धं द्वितीयचापस्याद्यार्धं चैकीकृत्य यश्चापभागः कल्प्यते स च कृत्स्नचापतुल्यः~। तदर्धद्वयैक्यात्~। तस्मात् तन्मध्यं प्रथमज्याग्रस्पृष्टः परिधिभागः~। तस्मात् प्रथमज्यया तयोः कोट्योरन्तरं प्राग्वदेवानेयम्~। समस्तज्या पुनस्तुल्येषु चापेषु सर्वत्रैव समाना इत्येतदसकृदावेदितम्~। तस्मात् तां समस्तज्यां प्रथमभुजज्यया निहत्य व्यासार्धेनैव हृत्वा प्रथमद्वितीयचापमध्याग्रयोः कोट्योरन्तरं लभ्यते~। एवं पूर्वोक्तेनैव कर्मणा चापभागसन्धिगताभिः पठिताभिरेव भुजाज्याभिरत्र कोटिखण्डानयनं क्रियते चापमध्यगतयोः कोट्योरन्तरं ह्यत्रानीयत इति~। पूर्वत्र चापमध्यगयोः कोट्योरन्तरस्यानीयमानत्वात् तन्मध्यगतभुजज्यया समस्तज्या हन्यत इत्येव केवलं विशेषः~। तेन त्रैराशिकस्य तद्युक्तेर्वा न 
विशेषः~। एवं भुजाखण्डानयनसाधनानां कोटीनामन्तराणि कोटिखण्डानयनोक्तत्रैराशिकेनैव सिद्धानि~। तैः पुनर्भुजाखण्डान्तरानयनमेवम्~।
प्रथमचापमध्यगताया (द्वितीयचापमध्यगताया)श्च कोट्या यदन्तरं तच्च समस्तज्यया निहत्य त्रिज्ययैव हरेत्~। तत्र यल्लब्धं तदेव प्रथमद्वितीयखण्डज्ययोरन्तरम्~। तद्युक्तिश्चैवम्~। तयोः कोट्योर्या महती प्रथमचापमध्यस्पृष्टाग्रा तया समस्तज्यां निहत्य त्रिज्ययैव हृत्वाप्तं प्रथमज्याखण्डः~। आद्यस्य खण्डस्य पिण्डस्य चैकत्वात् प्रथमज्यैव सा~। द्वितीयभुजाखण्डज्यानयनमप्येवम्~। या च पुनर्द्वितीयचापमध्यगता कोटिस्तया च समस्तज्यामेव निहत्य त्रिज्ययैव हृत्वाप्तं द्वितीयो ज्याखण्डः~। इत्येतयोरुभयोरप्यानयनेऽपि तुल्ये एवेच्छाप्रमाणे~। यतः सर्वचापेषु समानैकचापसमस्तज्यैवेच्छा~। त्रिज्यैवं च प्रमाणम्~। प्रथमज्यामध्यगता कोटिः प्रथमज्याखण्डानयने फलं द्वितीयज्यामध्यगता च द्वितीयखण्डानयने इति गुण्ययोरेव केवलमुभयत्र भेदः, न पुनर्गुणकारहारकयोः~। फलस्य हि गुण्यत्वमुक्तं {\qt त्रैराशिकफलराशिं तमथेच्छा-}

\newpage

\noindent {\qt राशिना हतं कृत्वे}ति~। अतो गुण्यान्तरमात्रं पृथगुद्धृत्य तत् समस्तज्यया निहत्य त्रिज्ययैव हृत्वाप्तमिच्छाफलभूतयोः खण्डज्ययोरन्तरमित्येतत् पूर्वोक्तेन खण्डगुणनन्यायेनैव सिद्धम्~। गुण्यान्तरं चेह कोट्यन्तरम्~। यस्मात् भुजाखण्डानयने तन्मध्यगतायाः फलात्मिकायाः कोटिज्याया एव गुण्यत्वम्~। तत्कोट्यन्तरं चापसन्धिगतभुजज्यानुरूपम्~। तस्याश्च तत्र गुण्य\renewcommand{\thefootnote}{१}\footnote{ण}त्वात्~। पदादितः प्रभृति चापसन्धिश्च चापदलद्वयात्मकस्य चापभागस्य मध्यम्~। तस्मात् भुजाज्याखण्डानयनगुण्यानां चापमध्यगतकोटीनामन्तरानयने प्रथमादिभुजज्यैव गुणकारः~। तस्मात् समस्तज्यां यया भुजज्यया निहत्य त्रिज्यया हरति तत्फलात्मकं कोट्यन्तरं भुजाज्याग्रमभितश्चापभागयोरुभयोर्ज्याखण्डयोरन्तरानयने गुण्यम्~। तच्च पुनः समस्तज्यया निहत्य त्रिज्यया हृतं तामेव भुजाज्यामभितो भुजज्याखण्डयोरन्तरम्~। एवमिदं द्वाभ्यां त्रैराशिकाभ्यामानीयते~। चापमध्यकोट्यन्तरानयनविषयमेकं त्रैराशिकम्~। इतरद्भुजाखण्डान्तरानय(ने ? न)विषयम्~। तत्र पूर्वत्र पदसन्धिगता भुजाज्या समस्तज्याया गुणकारः~। हारस्त्रिज्या~। तत्फलं कोट्यन्तरम्~। तस्यैव समस्तज्यैव गुणकारः~। त्रिज्यैव हारकः~। तस्माच्चापसन्धिगतभुजाज्यायाः समस्तज्यावर्गो गुणकारः त्रिज्यावर्गो भागहारः~। फलं खण्ड\renewcommand{\thefootnote}{२}\footnote{हारः~। खण्ड}ज्यान्तरमिति\renewcommand{\thefootnote}{३}\footnote{ज्यान्तरं कृति} समानावेव सर्वत्र गुणकारभागहारौ~। सन्धिगतभुजाज्याया गुण्यत्वात् तस्याश्च तत्तत्सन्धिषु नानात्वाद्गुण्यस्यैव केवलं भेदः~। तस्मात् गुण्य\renewcommand{\thefootnote}{४}\footnote{ण \textendash\ क. पाठः.}वृद्धिह्रासानुरूपावेव फलस्यापि वृद्धिह्रासाविति~। भुजाज्यानुसारिण्येव
ज्याखण्डानां वृद्धिरिति तयोर्नियमात् ज्ञातेनान्यस्यानुमानं युक्तमेव~। अत एव तत्र त्रैराशिकं युज्यते~। यद्वा समस्तज्यावर्गः सर्वत्र गुणकारः,
त्रिज्याव(र्ग ? गो)भागहार इति~। ततस्त्रैराशिकसिद्धौ गुणकारभागहारौ प्रथमद्वितीयखण्डज्यान्तरं प्रथमज्या च स्याताम्~। कथम्~। तत्रैवं त्रैराशिकम्~। यदि त्रिज्यावर्गे हारके समस्तज्यावर्गो गुणकारः तदा प्रथमज्यामात्रे हारके कियान् गुणकार इति~। तत्र प्रथमज्यायाः समस्तज्यावर्गो गुणकारः~। त्रिज्यावर्गो भागहारः~। फलं प्रथमज्याया हारकत्वेन जायमानो गुणकारः, स एव 


\newpage

\noindent च गुणकारः प्रथमद्वितीयज्ययोरन्तरमेव~। तदानयनमप्येवमेव यतः खण्डज्यान्तरानयने समस्तज्यावर्गो गुणकारः त्रिज्यावर्गो भागहारस्तत्तत्पिण्डज्यायाः फलं पिण्डज्यामभितः खण्डज्यान्तरम्~। एवमत्रापि प्रथमज्यामेव समस्तज्यावर्गेण हत्वा त्रिज्यावर्गेणैव हृत्वाप्तं फलमपि
प्रथमद्वितीयखण्डज्ययोरन्तरमेव~। एवमेव द्वितीयादिज्यानामपि हारकत्वं यदीष्येत, तदापि त्रैराशिकेनानीतो गुणकारस्तत्र तत्र जायमानं खण्डज्यान्तरमेव~। तस्मात् ययोर्निरन्तरयोः खण्डज्ययोरन्तरं ज्ञातं, तच्चापखण्डद्वयसन्धिगतपिण्डज्या च ज्ञाता तदा ताभ्यामपि फलप्रमाणाभ्यां त्रैराशिकं कार्यम् \textendash\ एतावत्या ज्यया तदग्रस्पृष्टचापभागयोरुभयोः खण्डज्ययोरन्तरमेत्तावल्लब्धं तदानया पिण्डज्यया तदग्रस्पृष्टचापद्वयज्याखण्डान्तरं कियदिति~। एतत्सर्वमस्माभिर्गोलसारे प्रदर्शितं, 

\begin{quote}
{\qt द्विघ्नान्त्यखण्डनिघ्नात् तत्तज्ज्यार्धात् त्रिभज्याप्तम्~।\\
अन्त्यादिखण्डयुक्तं त्याज्यं स्यात् पूर्वपूर्वगुणसिद्ध्यै~॥}
\end{quote}

\noindent इत्यादिना~। अस्यायमर्थः~। अत्रोत्क्रमखण्डज्यानयनमुच्यते~। तत्रान्त्योपान्त्यज्ययोर्ज्ञातयोस्ताभ्यामितरज्यानयनार्थमिदं कर्म~। तत्र
द्विघ्ना\renewcommand{\thefootnote}{१}\footnote{घ्नो \textendash\ ख. पाठः.}न्त्यज्याखण्ड आद्यो वा बाणखण्डो द्विघ्नो गुणकारः, उभयोरेकत्वात्~। हारकः पुनस्त्रिज्यैव~। एतौ च गुणहारौ पूर्वोक्ताभ्यां लघुतन्त्रसिद्धौ~। कथं पुनर्लघुतन्त्रम्~। 

\begin{quote}
{\qt भाजकाद् गुणकारेण निहताद् येनकेनचित्~।\\
भाजको गुणकाराद् वा भाजकेनाप्यते गुणः~॥

मतिर्भवति सा सङ्ख्या ह\renewcommand{\thefootnote}{२}\footnote{क \textendash\ क. पाठः.}र्तव्यो हन्यते यया~।\\
मतिरन्यत्वमाप्नोति फलतः खण्डनं प्रति~॥}
\end{quote}

\noindent इति~। अत्र त्रिज्या मतिः, यदि त्रिज्यावर्गे हारके समस्तज्यावर्गो गुणकारः तदा त्रिज्यातुल्ये हारके कियानिति~। तत्राप्तं द्विघ्नप्रथमबाणतुल्यम्~। अत उक्तं {\qt द्विघ्नान्त्यखण्डनिघ्नादि}ति~। तत्तज्ज्यार्धं हि सर्वत्रेच्छात्वेन भगवतोक्तम्~। तस्मादान्त्यो\renewcommand{\thefootnote}{३}\footnote{त्रो \textendash\ ख. पाठः.}पान्त्यज्यैवान्त्योपान्त्यखण्डयोरन्तरानयन इच्छाराशिः~। तस्मादुपान्त्यज्याया द्विघ्नान्त्यखण्डहतायास्त्रिभज्याप्तम(ते ?
न्त्योपा)न्त्यखण्डयोरन्तरम्~। तस्मिन्नन्त्यखण्डयुक्ते उपान्त्यखण्डश्च स्यात्~। उत्क्रम-

\newpage

\noindent खण्डानां क्रमेणाधिक्यात्~। तद्धीनोपान्त्यज्या तदधोगता पिण्डज्या स्यात्~। एवं पुनः पुनरपि तत्तदन्तरयुक्तः पूर्वखण्डः स्वेच्छायाः
पिण्डज्यायास्त्याज्यः~। एवं पूर्वपूर्वगुणसिद्धिः द्विघ्नोऽन्त्यखण्डोऽपि खण्डान्त(र)स्थानीयः~। ऋणधनात्मकयोर्योगो हि खण्डान्तरस्थानीय इति त्रिज्यामभितः खण्डज्ययोरेकस्या ऋणत्वमन्यस्या धनत्वं च स्यात्~। यस्मादुपान्त्यज्यायामन्त्यखण्डं प्रक्षिप्य परमज्यानीयते पुनस्तस्याः परमज्यायास्तमे\renewcommand{\thefootnote}{१}\footnote{दे}वान्त्यज्याखण्डं त्यक्त्वा इतरपदगता तदनन्तरज्यानीयत इति तस्यर्णत्वं धनत्वं च प्रथमस्य~। एवं त्रिज्यामभितः खण्डयोरुभयोर्योग एवान्तरस्थानीय इति चाशयः~। तस्मादत्रेयं त्रैराशिकवाचोयुक्तिः \textendash\ यदि त्रिज्यातुल्यया एतावत् खण्डज्यान्तरं लब्धं तदेष्टज्यया तामभितः खण्डयोरन्तरं कियदिति~। एवं तत्र तत्र नियमानुसारिणी त्रैराशिकवाचोयुक्तिः प्रदर्श्या~। नियमश्च बहुविधः~। अतएवोक्तं पार्थसारथिमिश्रेण व्याप्तिनिर्णये\textendash

\begin{quote}
{\qt यो यथा नियतो येन यादृशेन यथाविधः\renewcommand{\thefootnote}{२}\footnote{विधिः \textendash\ ख. पाठः.}~।\\
स तथा तादृशस्यैव तादृशोऽन्यत्र बोधकः~॥}
\end{quote}

\noindent इति~। अनुमाने लिङ्गलिङ्गिनोर्व्याप्तिनियम एवमेवेत्यभिप्रायः~। त्रैराशिकं चानुमानम्~। अत एवैतद्विवरणे तेनैव गणितविषयोदाहृतिः कृता {\qt शङ्कुच्छायां वा रविर्दिविष्ठो भूमिष्ठामि}त्यादिना तस्यैव नभोमध्ये स्थितिस्तामेवाध्यर्धपञ्चदशघटिकातिभ्रान्तामित्य(ने? )न्तेन ग्रन्थेन~। इत्यलमतिविस्तरेण~। प्रकृतमनुसरामः~। एवमप्यभीष्टचापभागसन्धिगता एव ज्याः सिध्येयुः~। न पुनः सर्वावयवेषु मध्येऽभीष्टस्य प्रदेशस्य ज्या~। कथं पुनस्तदानयनम्~। उच्यते~। एवमेव चापसन्ध्यभीष्टप्रदेशयोरन्तरालात्मकस्य चापखण्डस्य खण्डज्यामानीय चापसन्धिगतज्यायां धनमृणं वा कृत्वाभीष्टजीवापि नेया~। तत्र प्रथमं चापसन्ध्यभीष्टप्रदेशयोर्मध्यगतेतरज्या ज्ञेया~। तया ह्यस्यास्तच्चापखण्डगतखण्डज्यानीयत इति~। तदर्थं च तन्मध्यगतैतद्दिगनुसारिणी ज्या ज्ञेया~। एतदनुसारिणी हि तत्खण्डज्येति~। यद्यप्येवमनवस्था प्रसज्येत तथापि यावदपेक्षमेव कर्माणि गृह्यन्ताम्~। तत्रोत्तरोत्तरं फलस्याल्पत्वादादितः प्रभृति द्वित्राण्येव कर्माणि कार्याणि~। तत्राह माधवः\textendash

\newpage

\begin{quote}
{\qt इष्टदोःकोटिधनुषोः स्वसमीपसमीरिते~।\\
ज्ये द्वे सावयवे न्यस्य कुर्यादूनाधिकं धनुः~॥

द्विघ्नतल्लिप्तिकाप्तैकशरशैलशिखीन्दवः~।\\
न्यस्याच्छेदाय च मिथस्तत्संस्कारविधित्सया~॥

छित्त्वैकां प्र\renewcommand{\thefootnote}{१}\footnote{प्रा \textendash\ ख. पाठः.}क्षिपेज्जह्यात् तद्धनुष्यधिकोनके~।\\
अन्यस्यामथ तां द्विघ्नां तथास्यामिति संस्कृतिः~॥

इति ते कृतसंस्कारे स्वगुणौ धनुषोस्तयोः~।}
\end{quote}
\noindent इति~। तेनैव विबुधनेत्रादिना प्रोक्तपरिधिव्यासाभ्यां चक्रकलात्मकपरिधिपादचतुर्विंशांशे चापखण्डे समानीय पठिता यास्तत्परान्ता ज्यार्धपिण्डाः,
 
\begin{quote}
{\qt श्रेष्ठं नाम वरिष्ठानां हिमाद्रिर्वेदभावनः~।\\
तपनो भानुसूक्तज्ञो मध्यमं विद्धि\renewcommand{\thefootnote}{२}\footnote{धि} दोहनम्~॥

धिगाज्योनाशनं कष्टं छन्नभोगाशयाम्बिका~।\\
मृगाहारो नरेशोयं वीरो रणजयोत्सुकः~॥

मूलं विशुद्धं नाळस्य गानेषु विरळा नराः~।\\
अशुद्धिगुप्ता चोरश्रीः शङ्कुकर्णो नगेश्वरः~॥

तनूजो गर्भजो मित्रं श्रीमानत्र सुखी सखे~।\\
शशी रात्रौ हिमाहारो वेगज्ञः पथि सिन्धुरः~॥

छायालयो गजो नीलो निर्मलो नास्ति सत्कुले~।\\
रात्रौ दर्पणमभ्राङ्गं नागस्तुङ्गनखो बली~॥

धीरो युवा कथालोलः पूज्यो नारीजनैर्भगः~।\\
कन्यागारे नागवल्ली देवो विश्वस्थली भृगुः~॥

तत्परादिकलान्तास्ता महाज्या माधवोदिताः~।}
\end{quote}

\noindent ता एवेह सावयवा ज्या विवक्षिताः~। ताभिरभीष्टप्रदेशजयोः दोःकोटिजीवयोरानयनमिह प्रदर्श्यते~। तत्रेष्टदोर्धनुषः कोटिधनुषश्च
स्वस्वसमीप\renewcommand{\thefootnote}{३}\footnote{स्वसमीप \textendash\ क. पाठः.}चापसन्धिपठितां भुजाज्यां कोटिज्यां च सावयवे क्वचिद्विन्यस्य तयोरुभयोः साधारणमूनाधिकधनुः कुर्यात्~। कथं पुनरुभयोः साधारण्यं तस्य~। 

\newpage

\noindent सङ्ख्यासाम्यात्~। दोःकोटिधनुषोरिष्टयोर्यस्य समीप(श्चो ? श्चा)पसन्धिरधोगतः स्यात् स एव तदितरस्य तदैवोर्ध्वगतः स्यात्~। यतो विषुवतो भुजाप्रवृत्तिः कोटिप्रवृत्तिश्चायनात्, तत ओजे पदे पदादितःप्रभृति भुजाज्या प्रवर्तते~। पदान्तात् प्रभृति च कोटिज्या~। तत्र यदा
भुजाधनुर्यंकञ्चिच्चापसन्धिमतिक्रम्य कियन्तञ्चित् प्रदेशं गत्वा तिष्ठति, तदा तत्प्रदेशस्याधोगतः सन्धिः~। कोटिधनुः पुनस्तमेव सन्धिमप्राप्य तिष्ठति~। यतस्तद्धनुषः पदान्त एवादिस्ततस्तस्याधोमुखत्वादेव तद्विवक्षा~। ततस्तत्सन्ध्यवधिकचापभागेभ्यो न्यूनमेव तद्धनुः~। तत्र पठिता हि तत्समीपज्या न्यस्ता~। भुजाधनुषः पुनरतीतचापखण्डेभ्यः कृत्स्नेभ्योऽधिकं स्वधनुः~। अतस्तदन्तरालन्यस्तभुजाज्याधनुषोऽतिरिक्तमिति तस्या अधिकधनुस्तत्~। तावतैव न्यस्तकोटिज्याधनुष इष्टकोटिधनुषोऽल्पत्वमित्येकमेव तदन्तरालमूनाधिकधनुरुच्यते~। तदेव च तच्छब्देन परामृश्यते~। {\qt एकशरशैलशिखीन्दव} इति च चतुर्गुणं व्यासार्धमुच्यते~। तस्य लिप्तात्मकत्वादूनाधिकधनुरपि लिप्तीकार्यम्~। द्विगुणिताभिरूनाधिकधनुर्लिप्ताभिराप्ताश्चतुर्गुणव्यासार्धलिप्ताः क्वचिन्न्यस्याः~। किमर्थम्~। छेदाय~। कथं पुनस्तेन हरणं कस्य वा~। न्यस्तयोर्जीवयोरेकां केवलां केनाप्यहताम्~। यद्वा रूपेण हताम्~। रूपस्यैवात्र लघुकर्मणीच्छात्वात्~। तत्रापि विशेषाभावादविकृ(ताम ? तां) छित्त्वा~। केन~। यो राशिस्तदर्थं न्यस्तः तेन~। तत्फलमन्यस्यां जीवायां या हृता ततोऽन्यस्यां क्षिपेज्जह्याद्वा~। कदा क्षिपेत् कदा वा जह्यात्~। तद्धनुषि न्यस्तज्याधनुषोऽधिके क्षिपेत्, तत ऊने जह्यात्~। पुनरपि तामेव द्विघ्नामेवं कृत्वा तेनैव छेदेन छित्त्वास्यां क्षिपेज्जह्याद्वा~। एतद्धनुषि न्यस्तज्याधनुषोऽधिके क्षिपेत् ऊने च जह्यात्~। अस्याः संस्कृतिरिति~। एवमन्यस्या अपि संस्कृतिः कार्या~। एवं कृतसंस्कारे ते उभे ज्ये तयोरभीष्टभुजाकोटिधनुषोः स्वगुणौ स्याताम्~। तस्य धनुषः स्वोगुणः स्यात्~। अभीष्टधनुःसम्बन्धी गुण एवं नेय इत्यर्थः~। का पुनरत्र युक्तिः~। इयमिहोपपत्तिः \textendash\ तत्र प्रथमेन त्रैराशिकेनोनाधिकधनुर्मध्यगतेतरज्यानीयते~। यदि भुजाज्यानेया तदा तन्मध्यगता कोटिः यदा वा कोटिज्यानेया तदा तन्मध्यगता भुजाज्यानीयते~। तत्रोनाधिकधनुर्मध्य-

\newpage

\noindent चापस(न्धि)(ज्यो ? ज्ययो)रन्तरमूनाधिकधनुरर्धतुल्यम्~। तस्य तत्समस्तज्यायाश्चापान्तरत्वात्~। तदेव तत्समस्तज्यां कल्पयित्वा संस्कार्यज्यया हत्वा व्यासार्धेन विभज्य लब्धमूनाधिकधनुरर्धसम्बन्धिनी संस्कार्येतरखण्डज्या~। तत्रोनाधिकधनुरर्धसंस्कार्यपठितज्ययोर्घातात् त्रिज्याप्तं फलं हीतरज्याखण्डः~। तत्र गुणहारौ लघूकृत्येह कर्म प्रदर्शितम्~। तद्यथा \textendash\ ऊनाधिकधनुरर्धं संस्कार्याया जीवाया गुणकारः~। व्यासार्धं भागहारः~। तत्र भाजकादित्यादिनैकसङ्ख्यां मतिं परिकल्प्य हारो लघूकृतः~। यदि तद्धनुरर्धतुल्येन गुणकारेण व्यासार्धतुल्यो भागहारो लभ्यते तदा रूपेण गुणकारेण कियानिति~। तत्रोनाधिकधनुरर्धस्य व्यासार्धस्य च चतुर्गुणनं कृत्वात्र हरणं क्रियते~। ऊनाधिकधनुरर्धं च चतुर्गुणितं हि(वि ? द्वि)घ्नमूनाधिकधनुः~। व्यासार्धं च चतुर्गुणमेकशरशैलशिखीन्दुसङ्ख्यम्~। अतस्तत्तेन ह्रियते~। चतुर्गुणनं कलापरिपूर्त्त्यर्थम्~। पादोनं हि जलेवलं व्यासार्धं प्रायशः {\qt देवो विश्वस्थलीभृगुरि}ति पठितत्वात्~। अतश्छेदेन हरणमेवात्र संस्कार्यज्यायाः कार्यम्~। न पुनः फलगुणनम्~। एकसङ्ख्यत्वादेव तस्य~। तत्र लब्धे संस्कार्येतरज्यायां संस्कृते सा ह्यूनाधिकधनुर्मध्योत्था स्यात्~। त\renewcommand{\thefootnote}{१}\footnote{क \textendash\ ख. पाठः.}या पुनरूनाधिकधनुषः कृत्स्नस्य सम्बन्धीष्टज्या\renewcommand{\thefootnote}{२}\footnote{ज्याखण्ड}संस्कार्य ज्याखण्डमानीयते~। तत्र तूनाधिकधनुः कृत्स्नमेव गुणकारः~। न पुनस्तदर्धम्~। कृत्स्नसम्बन्धिनो ज्याखण्डस्य संस्कार्यत्वात्~। तत्समस्त(स्या ? ज्या)गुणकार इति तत्र तेन व्यासार्धं हर्तव्यम्~। तदर्धेन हृतं च व्यासार्धं छेदत्वेन न्यस्तम् इत्यत्राभीष्टहाराद्द्विगुणोऽयं छेद इतीतरज्यापि द्विगुणीक्रियते~। अत उक्तम् {\qt अथ तां द्विघ्ना}मिति~। एवमेवोभयोरपि संस्कार इति~। ननु तत्रोनाधिकधनुर्मध्यचापसन्धिज्ययोरन्तरानयने तन्मध्यगतज्यैव साधनम्~। न पुनश्चापसन्धिगताः~। तयैवात्र तु ज्या\renewcommand{\thefootnote}{३}\footnote{त्र ज्या}खण्ड आनीयत इतीहापि स्थौल्यमेव~। नैष दोषः~। साप्यानीयतां, का नो हानिः~। तदानयनानुक्तिरेव दोष इति चेत्~। न~। तदानयनमप्येवमेव स्यादिति तदर्थं प्रागपि किञ्चित् त्रैराशिकं कार्यम्~। यदा भुजाज्या संस्क्रियते तदा प्रथमं कोटिज्यामनेनैव हारेण हृ\renewcommand{\thefootnote}{१}\footnote{ह \textendash\ क. पाठः.}त्वा तत्फलार्धं भुजाज्यायां
संस्कार्यम्~। तत्रोनाधिकधनुश्चतुर्भागस्य गुणकारत्वात् तेन हृतं व्यासार्धमेतच्छेदाद्द्विगुणं स्यादिति तत्फलमेतत्फलार्धतुल्यमित्यनेनैव हृतमर्धीकार्यम्~। न 
पुनस्तदर्थं हारकान्तरमानेयम्~। तत्रापि कोटिर्न केवलं पठिता ग्राह्या,

\newpage

\noindent किन्तु चापसन्धित ऊनाधिकधनुरष्टांशान्तरितज्यैव इत्येतद्दोषपरिहारार्थमपि ततः प्रागेकं त्रैराशिकं कार्यम्~। तत्र प्रथमं संस्कार्यामेव ज्यामनेनैव छेदेन हृत्वाप्तस्य फलस्य चतुरंश एव तत्कोट्यां संस्कार्यः पठिताया ऊनाधिकधनुरष्टांशान्तरितज्यासिद्ध्यै~। ऊनाधिकधनुरर्धेन हि पूर्वं छेद आनीतः~। अर्धेनानीतादष्टांशानीतस्य चतुर्गुणत्वाद्गुण्यस्यापि चतुर्हरणं कार्यं तत्फलस्य वा~। उभयथापि फलस्य तुल्यत्वादिति त्रैराशिकचतुष्टयं वा कार्यमिति भावः~। इति खण्डज्यानयनयुक्तिरेवात्रापि युक्तिरिति तज्ज्याभिरेव त्रैराशिकेन खण्डज्यानयनमपि कार्यम्~। तस्माज्जीवायामपि त्रैराशिकं प्रवर्तते~। तस्माद्भुजाखण्डज्याः कोटिज्यानुसारिवृद्धिह्रासा इति तदानयने कोटिज्याया एवेच्छात्वं प्रमाणत्वमपि~। तत्र ज्ञातभुजान्तरायाः
प्रमाणत्वम्~। ज्ञेयभुजाखण्डज्यायाः पुनरिच्छात्वम्~। खण्डज्यान्तरानयने पुनस्तेषां कोटिखण्डानुसारित्वात् कोटिखण्डानां च भुजानुसारित्वात् तदनुसार्येव खण्डज्यान्तरमिति भुजाज्यानां खण्डज्यान्तराणां च वृद्धिह्राससाम्यलक्षणः सम्बन्धः स्यादिति तदानयने भुजाज्यानाम् इच्छात्वं तासु कतमस्याश्चित्प्रमाणत्वं च युज्यत एवेत्येतदनेन सूत्रेण दर्शितम्~। अनयैव दिशा खण्डज्यान्तराणां कोट्यनुरूपत्वात् ताभिरिच्छात्मिकाभिस्तेषामानयनं युक्तं, तथा तदन्तराणां भुजानुसारित्वाद्भुजाभिश्च~। इत्यन्तरपरम्परायामप्योजानां युग्मानां च कोटिभिर्भुजाभिश्चानयनं युक्तमिति सिद्धम्~। अनयैवोपपत्त्यैकवृत्तगतयोर्निरन्तरयोः परिधिखण्डयोस्तुल्ययोरतुल्ययोर्वा पृथक् पृथगर्धज्ययोर्विदितयोरेकी\renewcommand{\thefootnote}{१}\footnote{पृथगर्धज्ययोरेकी \textendash\ क. पाठः.}कृतस्य तच्चापद्वयस्यार्धज्यापि त्रैराशिकेनैवानेतुं शक्या~। सोऽयमुपायोऽस्माभिरश्रुतपूर्वो दृष्टः~। तदनन्तरं पुनस्तद्विषयं वसन्ततिलकं सङ्गमग्रामजमाधवनिर्मितं पद्यं च श्रुतम्~। यथा\textendash

\begin{quote}
{\qt जीवे परस्परनिजेतरमौर्विकाभ्या-\\
मभ्यस्य विस्तृतिगुणेन विभज्यमाने~।\\
अन्योन्ययोगविरहानुगुणे भवेतां\\
यद्वा स्वलम्बकृतिभेदपदीकृते द्वे~॥}
\end{quote}

\newpage

\noindent इति~। एतद्वाक्यद्वयात्मकम्~। तद्विषययोर्द्वयोः कर्मणोः प्रदर्शनात्~। तत्राद्यपादत्रयात्मकमेकं वाक्यम्~। चरमः पादो वाक्यान्तरमिति विभागः~। तत्राद्ये वाक्ये त्रैराशिकेन तदानयनं प्रदर्श्यते~। अन्यस्मिन् भुजाकोटिकर्णद्वारा वर्गमूलपरिकल्पनया~। तत्र त्रैराशिकोपपत्त्यर्थं पदसन्धितः प्रभृति तुल्यान्तरालविभक्तमङ्कद्वयं कृत्वा तदग्रद्वयान्तामृज्वीं रेखां लिखेत्~। तदा सा पूर्वापरायतव्यासच्छिन्ना द्वेधा विभक्ता स्यात्~। तयोरेको
भागस्तच्चपार्धसम्बन्धिन्यर्धज्या~। केन्द्रादग्रान्तां च रेखां लिखेत्~। सा तत्र श्रुतिरूपा~। तज्ज्याविभक्तस्य व्यासार्धस्य योऽधरः खण्डः सा कोटिः~। केन्द्रात् पुनरुदग्व्यासार्धेऽपि भुजाज्यातुल्यान्तरे बिन्दुं कृत्वा ततः प्रागायतां रेखां ज्याकर्णाग्रयुगस्पृष्टां लिखेत्~। सा च कोटिः~। उदग्व्यासार्धस्य
केन्द्रात् तदन्तो यः खण्डः सा च भुजा~। एवमिदमायतचतुरश्रं क्षेत्रम्~। तद्वायुकोणात् प्रभृत्यग्निकोणान्तं च कर्णः~। स पुनर्न लेखनीयः, तेनात्र
प्रयोजनाभावात्~। तत्कर्णात् प्रभृत्युत्तरतश्च कियन्तञ्चचिद्भागं विहाय बिन्दुं कुर्यात्~। स परिधिखण्ड एक\renewcommand{\thefootnote}{१}\footnote{व \textendash\ ख. पाठः.} चापभागः~। कर्णात् तावदन्तरे पुनर्दक्षिणतश्च बिन्दुं कृत्वा बिन्दुद्वयान्तरालावगाहिनीं रेखां कुर्यात्~। तदर्धं च तदर्धज्या~। सा च यत्र तत्कर्णं स्पृशति तत्प्रदेशाच्च दक्षिणतोऽधश्च व्याससूत्रावधिकां दक्षिणोत्तरायतां पूर्वपरायतां च रेखे कुर्यात्~। द्वितीयज्याग्राच्च दक्षिणतः पूर्वापरव्यासावधिकां दक्षिणोत्तरायतामेव रेखां कुर्यात्~। सात्र जिज्ञासिता~। तस्याश्च यौ खण्डौ द्वितीयज्याकर्णयोगात् प्रत्यगायतया रेखया खण्डितौ ताविह\renewcommand{\thefootnote}{२}\footnote{खण्डिताविह \textendash\ क. पाठः.} पृथक् पृथङ्नीयेते, तदैक्यं चापद्वयस्य ज्येति~। कथं पुनस्तत्खण्डयोरानयनम्~। उच्यते~। तस्य दक्षिणखण्डानयन एवं त्रैराशिकं यदि व्यासार्धतुल्यस्य प्रथमज्या भुजा, तदा तस्यैव कर्णस्य द्वितीयज्याशरोनस्य केन्द्रावधिकस्य खण्डस्य कियतीति~। तत्तुल्य एव हि चापद्वयज्यायाः प्रदर्शितयोः खण्डयोर्दक्षिणः खण्डः~। यतस्तयोरुभयोः पूर्वविष्कम्भार्धस्य द्वितीयज्याशरोनव्यासार्धकर्णकोट्याश्चान्तरालावगाहित्वेन तुल्यत्वम्~। यः पुनरुत्तरः खण्डः तदानयनमेवम्~। यदि प्रथमज्याकर्णस्य व्यासार्धतुल्यस्य पूर्वापरा कोटिरियती तदा तत्कर्णव्यस्तदिक्कायाः श्रुतिरूपाया दक्षिणोत्तरायता कोटिः कियतीति~।

\newpage

\noindent तत्र प्रथमज्याया\renewcommand{\thefootnote}{१}\footnote{ज्ययोः} इतरशरोनव्यासार्धस्य च घातो द्वितीयज्यायाः प्रथमज्याकोट्याश्च घातोऽपि त्रिज्यया हृतौ योज्यौ~। तदा चापद्वयज्या स्यात्~। तत्र घातयोरुभयोरपि हारकस्य व्यासार्धत्वाद्घातयोगो वा व्यासार्धेन ह्रियताम्~। तत्फलं चापयोगज्यैवेति शानचा सूचितम्~। चापयोर्द्वयोरर्धज्ये ये ते परस्परनिजेतरमौर्विकाभ्यामभ्यस्य विस्तृतिगुणेन विभज्यमाने एवान्योन्ययोगविरहानुगुणे भवेताम्~। न पुनर्विभक्ते एव~। तत्र युक्त्वा हरणे द्विर्हरणं न कार्यमिति क्रियालाघवम्~। उभयत्रापि कलाविकलादिष्वर्धोने तदवयवे चोपेक्ष्यमाणे स्थौल्यमप्यल्पमेव स्यात्~। हृत्वा युक्ते
चरमेष्ववयवेष्वर्धाधि\renewcommand{\thefootnote}{२}\footnote{ष्वाधि}क्येनैकेन भेदः स्यात्~। द्विर्हरणात् क्रियागौरवं चेत्ययं विशेष इत्याशयः~। निजेतरशब्देन स्वयं भुजे चेत् कोट्यौ विवक्षिते~। {\qt परस्परनिजेतरे}त्यत्र परस्परशब्देनेतरकोट्या हननं कार्यम्~। एवमन्यस्या अपि स्वयं कोट्यौ चेन्मि(थौ ? थो)भुजाभ्यां हननं कार्यमित्युक्तं स्यात्~। विस्तृतिगुणश्चार्धात्मको विवक्षितः~। स च व्यासार्धम्~। विस्तृतिदलेनेति वा पाठः~। सर्वत्र शरोनव्यास एव कोटिरिति तयोः संयोगोपपत्तिः~। वियोगे पुनः क्षेत्रकल्पनाभेदः स्यात्~। लेख्यद्रष्टृणां शिष्याणां व्यामोहो मा भूदिति तल्लेख्यं पूर्वं न प्रदर्शितम्~। द्वितीयक्षेत्रकर्णः कोटिश्च न प्रदर्शितौ~। रेखाबाहुल्याद्धि तद्विभागमजानतां व्यामोहः स्यादिति~। कोट्यप्रदर्शनाच्च सा शरोनव्यासशब्दोनोक्ता~। वियोगोपपत्तौ पुनर्द्वितीयज्याया इतरदर्धं महाचापप्रविष्टं कर्णः~। तदग्राच्चाधःसूत्रमवलम्ब्य दक्षिणोत्तरव्यासपर्यन्तां रेखां कुर्यात्~। तत्र प्रथमत्रैराशिकसिद्धा भुजा या द्वितीयज्याशरोनव्यासार्धकर्णस्य भुजारूपा सा पूर्वमेव लिखिता~। सा चेदानीं द्विधा कृता इदानीमेव लिखितया पूर्वापरायतया रेखया~। तस्याः पुनरुदग्गतो यः खण्डः स एवात्र द्वितीयज्याकर्णस्य कोटिः~। सा च द्वितीयत्रैराशिकेन पूर्वमानीतया कोट्या तुल्या~। यत\renewcommand{\thefootnote}{३}\footnote{कोट्या~। यत}स्तत्र द्वितीयज्याया उदगर्धं कर्णः~। अत्र च दक्षिणार्धम्~। उभयत्र कर्णयोः साम्यात् कोट्या अपि\renewcommand{\thefootnote}{४}\footnote{कोट्यामपि \textendash\ क. पाठः.} साम्येन भाव्यम्~। लेखनप्रदेशभेद एव केवलं द्वयोः~। ततस्तस्यां कोट्यां शरोनव्यासभुजायास्त्यक्ता(याः? यां)तद्दक्षिणखण्डश्च स्यात्~। योगे तत्तुल्या\renewcommand{\thefootnote}{५}\footnote{स्यात्~। तत्तुल्या \textendash\ ख. पाठः.} हि चापविवरज्या~। सा परिधि-

\newpage

\noindent स्पृष्टाग्रा~। तयोरप्येतद्रेखाद्वयान्तरावगाहित्वेन तुल्यत्वादिति वियोगयुक्तिः प्रतिपाद्या~। तत्रापि वियोगं कृत्वा वा हरेत्~। हृतयोर्वा वियोगं कुर्यात्~। उभयथापि फलसाम्यात्~। योगे वियोगे च न क्रियाभेदः~। योगवियोगभेदादेव केवलं भेद इति समाने एव त्रैराशिकफले उभयत्रापि~। अत उक्तम् \textendash\ इतरेतरकोट्याभ्यस्य विभज्यमाने ते अन्योन्ययोगविरहानुगुणे भवेतामिति~। ये योगानुगुणे ते एव वियोगानुगुणे च~। तस्मात्\renewcommand{\thefootnote}{१}\footnote{तस्मिन्} फलयोगे चापद्वययोगज्या स्यात्~। वियोगे तु तद्विवरज्या च स्यात्~। ननु ज्याखण्डस्य परिधौ प्रत्यवयवं भेदेन भाव्यम्~। तत् कथमत्र द्वितीयज्यार्धसम्बन्धिनोश्चापयोरुभयोरपि खण्डज्यासाम्यं स्यात्~। तत्साम्ये\renewcommand{\thefootnote}{२}\footnote{साम्ये (सा ? ऽन्या)दृ} सत्येव हि तद्योगवियोगयोश्चापयोगवियोगज्ये स्याताम्~। मन्द! अत्र द्वितीयत्रैराशिकेन न केवलं खण्डज्यानीयते~। द्वितीयज्याया अर्धचापार्धखण्डज्ये यदि विभज्य प्रदर्शनीये तर्ह्यन्यादृ(शि ? शी)परिलेखना~। फलं च न तुल्यम्~। अत्र द्वितीयज्यार्धयोः कर्णभूतयोः कोटी एवानीयेते~। ते च तुल्ये~। तच्चापार्धयोः खण्डज्ययोः पुनरेका अस्याः कोट्या महती~। इतरा चा(ल्पम् ? ल्पा)~। तत्प्रदर्शनाय प्रथमज्याग्रादेव प्रत्यग्रेखा कार्या~। द्वितीयज्याया उत्तराग्राच्च~। तयोरन्तरालं द्वितीयज्याचापार्धयोर्बहिरर्धस्य खण्डज्या~। द्वितीयज्याया इतर क्षेत्रावगाढाग्रात् या प्रत्यग्रेखा कृता, तस्याश्च प्रथमज्याग्रस्पृष्टायाश्च महाचापप्रविष्टाल्पचापार्धस्य खण्डज्या या, सा तु द्वितीयत्रैराशिकानीतफलादधिका~। अल्पा चेतरखण्डज्या~। अतएव प्रथमज्यायां न तत्फलं योज्यते वियोज्यते च~। प्रथमज्या हि व्यासार्धकर्णस्य भुजा~। या तु द्वितीयज्याशरोनव्यासार्धस्य तस्यां हि प्रथमत्रैराशिकानीतायामत्र द्वितीयत्रैराशिकसिद्धफलं योज्यते वियोज्यते वा इत्येकस्यैव फलस्य संयोजनवियोजने न दोषाय~। इष्टदोःकोट्यादिना तु ऊनाधिकधनुषोर्महाचापबहिरन्तःप्रविष्टयोः पृथक् पृथक् खण्डज्यामानीय महाचापज्यायां योगो वियोगोऽपि क्रियत इति तत्र योगे वियोगे च फलभेदः स्यात्~। अतएव प्रथमं भुजाज्यां विभज्य लब्धं तत्कोट्यास्त्यक्त्वा शिष्टं द्विगुणीकृत्य हरणे
बहिर्गतशिष्टचापखण्डज्या लभ्यते, भुजा(य ? या)श्छे\renewcommand{\thefootnote}{३}\footnote{च्छे \textendash\ क. पाठः.}देन लब्धं फलं तस्यामेव कोट्यां

\newpage 

\noindent संयोज्य महाचापान्तर्गतगन्तव्यधनुःखण्डज्या~। सा च महती पूर्वफलसंयोजनेन कोट्या आधिक्यात्~। भुजाधनुष्यधिके पुनः कोटिधनुषश्चोनत्वात् कोट्यास्तत्फलत्यागात् बहिर्गता शिष्टचापखण्डज्याल्पा इति तत्रैव योज्यस्य वियोज्यस्य च भेदः~। तत्र कृत्स्नज्यायां हि संस्क्रियते~। अत्र तु चापद्वययोगस्य वियोगस्य च ये ज्ये तयोर्योगार्धे हि संस्क्रियते~। तद्योगार्धतुल्यं च प्रथमत्रैराशिकानीतं फलम्~। अतो द्वितीयज्यायाः कृत्स्नाया यत् कृत्स्नं चापं तस्य या खण्डज्या तदर्धमेवात्र द्वितीयत्रैराशिकानीतं फलम्~। न पुनर्महाचापान्तर्बहिर्गतखण्डयोः~। द्वितीयज्यार्धचापयोः खण्डज्ये पृथक् पृथगानीय महत्यां ज्यायां संस्क्रियते~। इष्टदोःकोट्यादिना पुनस्ते एव संस्क्रियेते इति तत्रैव धनर्णयोर्भेदः न पुनरत्र इति त्रैराशिकयुक्तिसाम्यमेवोभयत्रापि~। खण्डज्यानयने चापखण्डसमस्ता ज्या हि कर्णः~। भुजाकोटिखण्डज्ये च तत्कोटिबाहू~। अत्र पुनस्तस्याः समस्तज्याया अर्धयोः कोट्योस्तुल्यत्वाद्युक्तिसाम्यम्~। इच्छाया अर्धत्वात् फलमपि अर्धात्मकमित्येव केवलं विशेष इति प्रथमाच्चापज्यार्धादित्यादिसूत्रेण दर्शितैवात्र सर्वत्र युक्तिः न पुनर्मनागपि भेद इतीदं सर्वमप्यनेनैव सूत्रेण सिद्धम्~। अपि च व्यासात् परिध्यानयनमप्यनेनैव सिद्धम्~। कथम्~। वृत्तक्षेत्रे ये  केचिद्बाहुकोटिकर्णा निरवयवाः, तेषां (बा ? ब)हुत्वात् तत्क्षेत्राण्यपि नानाकाराणि बहूनि स्युः~। तेष्वन्यतमे क्षेत्रे तद्गतभुजाकोटिकर्णैर्निरवयवैरिदं कर्मारम्भणीयम्~। यत्र भुजाकोट्योः साम्यं स्यात् तस्यैव ह्याकारेण चतुरश्रत्वं स्यात्~। ततः प्रभृति भुजाकोटिविप्रकर्षानुरूपं विस्तारः क्रमेण ह्रसति~। विस्तारादायामस्य बाहुल्यक्रमेण द्राघीयस्त्वमपि क्षेत्रस्य प्रतीयेत~। एवं यान्यनन्तानि क्षेत्राणि तेषु कानिचिदेव निरयवभुजाकोटिकर्णकानि~। तत्रापि भुजाकोटिचापयोर्निरवयवत्वमवश्यं न स्यात् इत्येतत् सर्वमवगन्तव्यम्~। एवं भुजाकोट्योर्विप्रकर्षवशादतीव विप्रकृष्टयोरल्पस्य चापं सुगमम्~। (कथं) पुनस्तस्य सुगमत्वम्~। ज्याछेदविधानन्यायेनानीयमाने लधूपायत्वसम्भवात्~। कथं तस्य लघूपायत्वसम्भवः~। त(स्या ? न्न्या) यश्च

\begin{quote}
{\qt वृत्ते शरसंवर्गोऽर्धज्यावर्गः स खलु धनुषोः~।}
\end{quote}

\newpage

\noindent इति वक्ष्यमाणसूत्रन्यायेनैव सेत्स्यति~। तद्यथा अस्यायमर्थः~। समवृत्तं क्षेत्रं यत्र क्वापि भिन्द्यात्~। तत्र विदारणमार्गस्य ऋजुतया\renewcommand{\thefootnote}{१}\footnote{ज्ञात ऋजुतया \textendash\ ख. पाठः.} भाव्यमित्येव केवलं नियमः~। तत्राल्पखण्डश्चापाकारः~। अन्यो मृदङ्गाकारोऽपि स्यात्~। तत्राल्पस्य मध्यगतो यः शरः तदनुसारेण कृत्स्नेऽपि वुत्ते व्यासरेखां कुर्यात्~। तस्या विदारणरेखया द्वेधा खण्डिताया अल्पः खण्डोऽल्पचापस्य शरः~। इतरः खण्डो महतश्चापस्य शरः~। तयोः संवर्गो धनुषोरुभयोः साधारणभूताया अर्धज्याया वर्गः~। तद्विदारणरेखार्धस्य वर्ग इत्यर्थः~। खल्विति एतत् सम्प्रतिपन्नमेव विदुषां सर्वेषामिति तदुपपत्तिः सूचिता~। सा च योगान्तरघातस्य वर्गान्तरत्वादेव सिद्धा~। योगान्तरघातस्य वर्गान्तरसाम्यं पूर्वमेव प्रदर्शितम्~। शरसंवर्गः पुनः कयोर्वर्गान्तरं, कयोर्वा योगो महाशरः, कयोश्चान्तरमल्पशरः~। तदर्धज्याकर्णकोट्योरिति ब्रूमः~। तथाहि \textendash\ महाशरस्य व्यासार्धतुल्यो यो भागः स कर्णतुल्यः~। ज्यारूपयोर्बाहुकोट्योः कर्णस्य व्यासार्धात्मकत्वं वृत्ते सर्वत्रापि स्यात्~। यः पुनरितरो भागः स कोटितुल्यः~। शरोनव्यासार्धस्य कोटित्वादिति तद्योगो महाञ्छरः~। तत्साम्यं क्षेत्रे वा प्रदर्शनीयम्~। महाशरस्य केन्द्रे छिन्नस्य महतः खण्डस्याग्रं भ्राम्यमाणं
यावच्चापद्वयसन्धिं स्पृशति तदा व्यासार्धतुल्यः स खण्डोऽस्य क्षेत्रस्य कर्णतामापद्यते~। तत्र विस्पष्टं कर्णकोटियोगत्वं तस्य महाशरस्य~। कोटिकर्णान्तरं हि शरः~। अतः शरसंवर्गः कोटिकर्णयोगान्तघात एव~। अत एव कोटिकर्णयोर्वर्गान्तरत्वं च स्यादिति सम्प्रतिपन्नतास्य~। एवं
वृत्तपरिधे\renewcommand{\thefootnote}{२}\footnote{धि \textendash\ क. पाठः.}रल्पीयसोंऽशस्यार्धज्यावर्गे ज्ञाते तत्र तच्छरवर्गं सत्र्यंशं क्षित्वा मूलीकृते तद्धनुरर्धं स्यात्~। उक्तं चैतदस्माभिर्गोलसारे\textendash

\begin{quote}
{\qt सत्र्यंशादिषुवर्गा(ज्या ? ज्ज्या)वर्गाढ्यात् पदं धनुः प्रायः~।}
\end{quote}

\noindent इति~। अस्य युक्तिः पुनरेव प्रदर्शयिष्यते~। एवं तद्धनुषि ज्ञाते तावान्ति धनूं(प्र ? षि)वृत्ते यावन्ति सन्ति, तस्मिंस्तावद्भिर्गुणिते
परिधेस्तद्धनुःसमुदायसम्बन्धिनः परिमाणं स्यात्~। यच्च पुनस्ततोऽप्यल्पमवशिष्टं धनुः तस्य तेष्वेकस्मादपि न्यूनत्वात् तज्जीवायां च ज्ञातायां तच्छरस्यापि ज्ञेयत्वात् तद्धनुरप्येवमेवानेयम्~। तत्परिमाणमपि समुदायपरिमाणे क्षिप्ते सति कृस्त्नस्य

\newpage

\noindent परिधेः प्रमाणं स्यात्~। तत्र वृत्तगतानामल्पधनुषां सङ्ख्या चापशिष्टज्या च {\qt जीवे परस्परे}त्युक्तन्यायेनैव ज्ञेया~। तत्र पुनर्वृत्तस्य मापकनियमपारतन्त्र्याभावाच्च सूक्ष्मत्वापादने लाघवम्~। एवं चातिसूक्ष्मतापि स्यात्~। कथम्~। तत्र यतः कुतश्चिदपि
निरवयवभुजाकोटिकर्णक्षेत्रादारभ्याप्येतत्कर्मपरम्परयाल्पत्वमप्यापादनीयम्~। तत्र त्र्याद्येकोत्तरभुजाकोटिकर्णक्षेत्रात् प्रभृति कर्मपरम्परा प्रदर्श्यते~। तत्र पञ्चसङ्ख्यव्यासार्धवृ(तेति ? त्ते त्रि)सङ्ख्यार्धज्या बाहुः~। कोटिश्च चतुस्सङ्ख्या~। तत्र तयोर्योगो वियोगो वा क्रियताम्~। तत्र {\qt जीवे परस्परे}त्युक्तं कर्मोभयत्रापि कार्यम्~। तत्र व्यासार्धहरणमकृत्वापि पुनरप्येतत्कर्मावृत्त्या द्विगुणोत्तराणां चापानां वा योगवियोगचापयोरन्तरालद्वारा वा वृत्तस्यान्तं गन्तव्यम्~। कथम्~। तत्रेतरेतरकोट्या हि द्वयं हन्तव्यम्~। तस्मात् त्रिकतुल्या ज्या चतुष्ककोट्या त्रिसङ्ख्ययैव हन्तव्या~।
चतुस्सङ्ख्या चेतरकोट्या चतुस्सङ्ख्ययैव~। तद्योगो वियोगो वा व्यासार्धेन पञ्चकेन हर्तव्यः~। तत्र योगे हृते व्यासार्धमेव स्यात्~।
यतश्चतुष्कयोर्घातः षोडशसङ्ख्यः~। त्रिकयोर्घातो नवसङ्ख्यः~। ततस्तद्योगः पञ्चविंशतिसङ्ख्यः~। ततः पञ्चभिर्हृते पञ्चैव फलम्~। पञ्चसङ्ख्यश्चात्र कर्णः~। वियोगः पुनः सप्तसङ्ख्यः~। तस्मिन् व्यासार्धेन पञ्चकेन ह्रियमाणे फलं सावयवं स्यात्~। तर्हि हरणं न कार्यम् इति पूर्वव्यासार्धात् व्यासार्धमेव पञ्चगुणं कल्प्यताम्~। तथा सति पञ्चविंशतिसङ्ख्यं व्यासार्धम्~। एवं योगोऽपि न हार्यः~। एवं\renewcommand{\thefootnote}{२}\footnote{योगेऽपि~। एवं \textendash\ ख. पाठः.} सप्तसङ्ख्या या ज्येहानीता सा पूर्वचापयोरन्तरस्यैव ज्या~। मापकं च पूर्वमापकात् पञ्चांशतुल्यम्~। अतएवोक्तं भास्करेण\textendash

\begin{quote}
{\qt इष्टयोराहतिर्द्विघ्नी कोटिर्वर्गान्तरं भुजः~।\\
कृतियोगस्तयोरेवं}\renewcommand{\thefootnote}{*}\footnote{'व' इति मुद्रितलीलावतीपाठः.} {\qt कर्णश्चाकरणीगतः~॥}
\end{quote}

\noindent इति~। न केवलमकरणीगतकर्णाभ्यामेव भुजाकोटिभ्यामेतत् कर्म कार्यम्~। स्वेच्छया कल्पिताभ्यां याभ्यां काभ्याञ्चिदपि, किन्तु तत्र कर्णेन हर्तव्ये सति तदकरणेनान्यस्मिन् वृत्ते परिणम्यमाने ते अकरणीगते एव स्याताम्~। अत- 

\newpage

\noindent स्तत्कर्णोऽप्यकरणीगत एव स्यात्~। कस्मिन् पुनः परिणम्यते~। स्ववृत्तव्यासार्धवर्गव्यासार्धवृत्ते~। कुतः~। ये स्वकल्पिते इष्ट\renewcommand{\thefootnote}{१}\footnote{इष्टे \textendash\ ख. पाठः.}कोटिभुजे तयोरेव योगवियोगयोः कर्तव्ययोस्तदानुगुण्याय यत्कर्मोक्तं, तत्र तुल्ययोर्योगे तयोः परस्परनिजेतरगुणने तद्घाततुल्य एव स्वपरेतरयोर्घातः~। अतो (च्चा ? घा)त एव द्विगुणीक्रियते~। कथम्~। योगार्हयोस्तुल्यत्वे सति तत्कोट्योरपि तुल्यत्वेन भाव्यम्~। भुजाकोटी चेष्टतया कल्पिते~। तत एकस्यां जीवायां समानभुजाया इतरस्याः कोट्या अपि स्वकोट्या साम्यादिष्टभूतया कोट्यैवेष्टान्तरं हन्तव्यम्~। तस्मादिष्टयोर्घात एव तत्र परस्परनिजेतरघातः~। द्वयोरपि परस्परकोट्या हन्तव्यत्वात्~। अन्यस्यापि भुजायाः कल्पितयोरिष्टयोरन्यतरत्वात् तदेवान्यतरेणेष्टेन\renewcommand{\thefootnote}{२}\footnote{तरेष्टेन \textendash\ क. पाठः.} हन्तव्यम्~। तत उभयत्रापि कल्पितयोरिष्टयोरेव घातः परस्परकोट्या योगार्हज्यायाश्च घातः~। अत उक्तमिष्टयोराहतिर्द्विघ्नीति~। अत एव सिद्धं भुजाकोटी एवेष्टतया क(ल्प्य ? ल्पि)ते इति~। ए(व ? क)मिष्टं भुजा इतरदिष्टं कोटिः~। योगयोग्ययोर्द्वयोरपि तुल्यत्वात्~। प्रथमं कल्पितमिष्टं तयोर्द्वयोरपि (दोः)सङ्ख्या~। अन्यदिष्टं द्वयोरपि कोटिसङ्ख्या इत्यन्यकोट्या हते उभे अपीष्टाहती एवेति तयोर्हरणात् प्रागेव योगे तयोरेव योगः कार्यः इति द्विगुणीक्रियते~। एवमिष्टयोराहतिर्द्विघ्नी सती कल्पितेष्टयोर्भुजाकोट्यात्मनोः कर्णेन हर्तव्या~। तत्कर्णस्य करणीगतत्वमपि प्रायेण सम्भवति~। अनिरूप्यैव कल्प्यमानत्वादिष्टयोः~। निरूप्य कल्प्यमानत्वे हीष्टत्वमेव हीयेत नियमसद्भावादिति~। यथेष्टं कल्पितयोर्भुजाकोट्योः क(र्णो ? र्णः) करणीगत एव स्यात्~। तस्य ज्ञातुमशक्यत्वात् तेन हरणं कर्तुं\renewcommand{\thefootnote}{३}\footnote{हरणं न कर्तुं \textendash\ ख. पाठः.} शक्यम्~। तद्धरणे यत्फलं स्यात् तदेव तेनैव हरणे तद्गुणं स्यादित्यपेक्षितात् फलात् तत्कर्णगुणत्वाद्द्विघ्न्या इष्टयोराहतेरिष्टसम्बन्धिवृत्तात् कर्णगुणिते वृत्ते परिणामः स्यात्~। ततः करणीगतो यः कर्णः कल्पितेष्टवृत्तव्यासार्धात्मकः सोऽपि स्वेन हन्तव्यः फलस्य तावद्गुणितत्वात् तत्सम्बन्धिव्यासार्धात्वाय~। तस्मात् पूर्वव्यासार्धात् स्वगुणमिदं व्यासार्धम्~। तस्मात् कर्णेनाहरणे गुण एव स्यात् कर्णो व्यासार्धस्य, कर्णगुणस्य च व्यासार्धस्य ज्ञेयत्वात्~। कथं पुनरस्य ज्ञेयता~। अज्ञाते स्वमूलभूतेऽल्पे वृत्ते कथं\renewcommand{\thefootnote}{४}\footnote{स्वमूलभूतेऽल्पे कथं \textendash\ क. पाठः.} तद्वर्गात्मकं\renewcommand{\thefootnote}{५}\footnote{क \textendash\ ख. पाठः.} महावृत्तं ज्ञायते~। भुजाकोटिचापयो-

\newpage

\noindent र्योगे हि परिधिपादः कृत्स्नोऽपि स्यात्~। ततस्तद्योगज्यैव व्यासार्धम्~। तदानयनेऽपि व्यासार्धभागहा\renewcommand{\thefootnote}{१}\footnote{व्यासार्धहा \textendash\ ख. पाठः.}राकरणेन तदपि महति वृत्ते परिणतत्वात् निरवयवमेव स्यात्~। भुजाकोट्योर्योगार्हत्वाय तयोरपि परस्परनिजेतराभ्यां घाते कार्ये कोट्यास्तत्तुल्यया परेतरया हननं\renewcommand{\thefootnote}{२}\footnote{परेतयोर्हननं \textendash\ क. पाठः.} कार्यम्~। यतो\renewcommand{\thefootnote}{३}\footnote{यत \textendash\ ख. पाठः.} योगयोग्यायाः कोट्याः परा\renewcommand{\thefootnote}{४}\footnote{योग्याय परा \textendash\ क. पाठः.} भुजैव~। तदितरा च स्वा~। तया च कोट्या घातो वर्गः~। तस्माद्योगयोग्यायाः कोट्या वर्ग एवैको घातः~। एवं योगयोग्ययोः कोट्यात्मिकाया योगयोग्यभुजेतरायाश्च घातः~। योगयोग्यभुजायाः पुनर्योग्यापरनिजेतरायाश्च घातो भुजावर्ग एव~। ततस्तयोर्योगः कर्णेन हर्तव्योऽपि न ह्रियते~। स्ववृत्तव्यासार्धवर्गव्यासार्धवृत्तपरिणते च ते~। एवं वर्गयोग एव तद्वृत्तव्यासार्धम्~। यतः पूर्ववृत्तादस्य व्यासार्धं पूर्वव्यासार्धवर्गतुल्यम्~। तद्व्यासार्धस्यैवमकरणीगतत्वाज्ज्ञेयत्वाच्च तद्वियोगोऽपि व्यासार्धेन हरणाकरणेनैवास्मिन् महति वृत्ते परिणम्यत इति~। एवं स्वे\renewcommand{\thefootnote}{५}\footnote{इति~। स्वे \textendash\ ख. पाठः.}ष्टाभ्यामप्यकरणीगताभ्यां भुजाकोटिभ्यां तद्भुजाकोटिचापयोगस्य माधवभास्कराभ्यामुक्तन्यायेनानीता ज्या व्यासार्धतुल्या अकरणीगतैव स्यात्~। तद्भुजाकोटी च कल्पितेष्टयोर्भुजात्मकस्येष्टस्य यच्चापं तद्द्विगुणस्य ज्यापीष्टयोर्योगार्हत्वमापाद्य योगे कृते स्यात्(?)~। तत्कोटिश्च पुनरिष्टयोर्भुजाकोटिज्ययोर्वियोगस्यैव ज्या~। तत इष्टयोरेव भुजाकोट्यात्मकयोर्विरहार्हत्वमापाद्य वियुक्तयोस्तत्कोटिश्चाकरणीगता स्यात्~। तस्मादकरणीगताभ्यामपीष्टाभ्यां भुजाकोट्यात्मकाभ्यामितरेतरयोगयोग्यतामापाद्य व्यासार्धहरणाकरणेन सिद्धं व्यासार्धं तत्कर्णात्मकमकरणीगतमेव स्यात्~। त(दा ? था) कल्पितेष्टयोर्यदल्पं तद्भुजाचापतुल्ययोर्निरन्तरयोर्द्वयोश्चापयोर्योगस्य जीवापि~। एवमानीयमाना प्रथमेष्टज्या द्विगुणचापज्या
च व्यासार्धाहरणेन महति वृत्ते परिणम्यमाना अकरणीगतैव स्यात्~। कल्पितद्वितीयेष्टतुल्यायाः कोट्याश्चापतुल्ययोरपि
निरन्तरयोर्द्वयोश्चापयोर्योगस्य जीवापि द्वितीयेष्टतुल्ययोः पृथग्भूतचापज्ययोः परस्पर\renewcommand{\thefootnote}{६}\footnote{तुल्ययोः परस्पर}निजेतरहननमात्रेण हरणमकृत्वा योगयोग्यतामापन्नयोर्योगतुल्या ज्या महति वृत्ते परिणम्यमानापि तत्कोटिरकरणीगतैव स्यात्~। सापि द्विघ्नीष्टाहतिरेव~। तत्र परिणतवृत्तव्यासार्धं {\qt कृतियोगस्तयोरेव}\renewcommand{\thefootnote}{७}\footnote{वं \textendash\ क. पाठः.} {\qt कर्णश्चाकरणीगत} इत्युक्तम्~। इष्टयोराहतिर्द्विघ्नीति

\newpage 

\noindent तद्वृत्तगतैका ज्या, इष्टयोर्वर्गान्तरं तदितरा, इति त्रयाणामकरणीगतत्वं युक्तम्~। एवमकरणीगतैस्तैरपि स्वस्वव्यासार्धगुणितवृत्ते परिणम्यमानानां सङ्ख्यामहत्त्वेऽपि निरवयवत्वमेव स्यात्, न पुनरवयवहानादिना स्थौल्यं कदाचिदपि स्यात्~। एवं परिणतज्याभ्यामप्युभाभ्यां विसदृशाभ्यां भुजाकोटिभ्यामानीता या भुजाकोटिस्वचापयोगज्या, सा स्वोर्ध्वमहावृत्तस्य व्यासार्धं स्यात्~। ततः कर्णहरणाभावात् सदृशयोर्योगोऽपि सदृशयोर्वियोगश्चैवं कृतस्तद्भुजा कोटि च स्त इति सिद्धम्~। एतदेव मुहुर्मुहुः कर्म कार्यम्~। तत्रादितः प्रभृति प्रदर्श्यते, यस्मिन् वृत्ते एकसङ्ख्या भुजा द्विसङ्ख्या च कोटिः तयोर्वर्गयोगः पञ्च~। पञ्चानां मूलाभावात् करणीगत एवास्य कर्णः~। ये पुनरिमे एकद्विसङ्ख्ये भुजाकोटिरूपे ज्ये, ततस्ते
पञ्चपदव्यासार्धवृत्तभवे~। तयोरेकसङ्ख्याया यावच्चापं तद्द्विगुणचापस्य ज्यानयने क्षेत्रयोरुभयोस्तुल्याकारत्वात् पूर्वप्रदर्शितयोः क्षेत्रयोर्भिन्नानां
नानाकारत्वाच्च~। ततोऽत्र विशेषः पुनरियानेव~। अत्र तु पदादित एव प्रवृत्तैकसङ्ख्या या ज्या या च पुनस्तदग्रतोऽपि पदादित इव प्रवृत्ता एकसङ्ख्या ज्या तयोरुभयोरपि कोटी द्विसङ्ख्ये~। तत्र (या) प्रथमज्याद्वितीयज्यासम्बन्धिशरोनव्यासार्धेन द्विकेन हता द्विसङ्ख्या सा च तत्कर्णेन हर्तव्या~। या पुनर्द्वितीयज्याप्येकसङ्ख्या सापि प्रथमज्याकोट्या द्विसङ्ख्यया हता सैव~। सा च तत्कर्णेन हर्तव्या~। तयोर्योगश्चतुस्सङ्ख्यो वा कर्णेन हर्तव्यः~। तत्र लब्धमेकसङ्ख्यायाश्चापाद्द्विगुणस्य चापस्य ज्या~। तद्योगस्य कर्णेन हर्तुमशक्यत्वात् तद्योग एव पञ्चसङ्ख्यव्यासार्धस्य परितस्तावतोंऽशस्य ज्या~। व्यासार्धेन हृतं फलं पञ्चमूलव्यासार्धवृत्ते ता\renewcommand{\thefootnote}{१}\footnote{या}वतोंऽशस्य~। तत्राहरणेन कस्यचिदपरितोषः स्यात्~। स एवं प्रतिबोद्धव्यः पञ्चमूलेन ह्रियते यदभावात् तव विषादोऽभूत्~। तत्फलं पुनस्तेनैव पञ्चमूलेन गुण्यत एव~। किमर्थम्~। तत्कर्णस्याकरणीगतत्वाय~। कुतः पुनस्तेन कर्णस्याकरणीगतत्वं स्यात्~। ज्यारूपस्य तत्फलस्य पञ्चमूलेन हतत्वात् तद्व्यासार्धमपि पञ्चमूलसङ्ख्यं स्वेन हन्तव्यम्~। तस्मात् पञ्चमूलं तावत्\renewcommand{\thefootnote}{२}\footnote{पञ्चमूलस्तावत् \textendash\ ख. पाठः.} कृत्वा कृतं तत्र व्यासार्धम्~। ततः पञ्चसङ्ख्यम्~। तस्माद्व्यासार्धपञ्चांशतुल्येन मापकेन मिता तच्चापद्वयज्यातद्घातयोगतुल्यैव, न पुनस्तस्माद्व्यासार्धाप्तफलतुल्या~। सा च पुनरिहेष्टयो-

\newpage

\noindent राहतिर्द्विघ्नीत्यादिनानीता चतुस्सङ्ख्या तत्कोटिज्या~। या तु पञ्चमूलवृत्तभवा एकसङ्ख्या आद्येष्टतुल्या या च पुनर्द्वितीयेष्टतुल्या द्विसङ्ख्या कोटि तयोश्च ये चापे यच्च तयोश्चापयोरन्तरं तज्ज्या पञ्चसङ्ख्यव्यासार्धपरिणतैव~। तत्राप्युभयोक्तकर्मणी योज्येते~। तत्र पदसन्धितः प्रभृति द्विसङ्ख्या ज्या कल्प्या~। इतरा च तदग्रतः~। तत्र विश्लेष एव कार्य इति तदुक्तं छेद्यकमनुसन्धेयम्~। तत्र तां पदसन्धितः प्रवृत्तां द्विसङ्ख्यां
ज्यामेकसङ्ख्यायाकोट्या द्विसङ्ख्यया हत्वा पञ्चमूलेन करणीगतेन तत्कर्णेन हरेत्~। पुनर्द्वितीयज्यामेकसङ्ख्यां च प्रथमज्याया द्विसङ्ख्याया(ः) कोट्या एकसङ्ख्ययैव हत्वा पञ्चमूलेन हरेत्~। लब्धद्वये वियोजिते द्विसङ्ख्यज्याचापादेकज्याचापे विशोधिते यच्छिष्टं तज्ज्या स्यात्~। तत्रापि पुनस्तदेव फलं स्वकर्णेन पञ्चमूलेन हतं तद्धार्यतुल्यम्~। हार्यश्चात्र घातयोर्वियोगः~। घातः पुनरत्र ज्ययोर्वर्ग एव~। प्रथमत्रैराशिके द्विसङ्ख्याया इच्छायाः प्रथमाया एकसङ्ख्यायाः कोट्या द्विसङ्ख्ययैव गुणनीयत्वात्~। तत्र वस्तुत आकारतो विभेदेऽपि तयोः सङ्ख्यासाम्याद्घातोऽपि वर्ग एव~। अतो द्विसङ्ख्याया वर्गश्चतुस्सङ्ख्य एको घातः~। इतरघातः पुनरेक एव~। तयोरुभयोरप्येकसङ्ख्यत्वात्~। यत एकसङ्ख्या ज्या द्विसङ्ख्यायाः कोट्यैकसङ्ख्ययैव हन्यते ततः सोऽपीष्टयोरन्यतरस्य वर्गः~। एवं द्विसङ्ख्यस्यैकसङ्ख्यस्य च इष्टयोर्वर्गान्तरं त्रिसङ्ख्यं स्यात्~। तदेव पुनः पञ्चमूलेन हृत्वा तेनैव गुण्यते~। अतः पुनरपि त्रिसङ्ख्यत्वं न हीयते~। अतस्त्रिसङ्ख्या पञ्चसङ्ख्यव्यासार्धस्यैका ज्या~। इतरा च चतुस्सङ्ख्या~। एवं पुनस्तद्वर्गवृत्ते उभे अपि परिणाम्येते पूर्वमेव प्रदर्शिते~। तत्कर्णश्च पञ्चविंशतिसङ्ख्यः~। तस्मिन् परिणते चतुर्विशतिसप्तसङ्ख्ये स्तः~। कथं तयोश्चतुर्विंशतिसङ्ख्यत्वं सप्तसङ्ख्यत्वं चोपपद्यते~। ननु तत्र परिणतयोर्विंशतिसङ्ख्यत्वं पञ्चदशसङ्ख्यत्वं चोपपद्यते~। यतस्तद्व्यासार्धात् पञ्चविंशतिसङ्ख्यं व्यासार्धं पञ्चगुणम्~। ततस्तयोरपि पञ्चभिरेव गुण्यत्वाच्चतुस्सङ्ख्या पञ्चगुणिता विंशतिसङ्ख्या, त्रिसङ्ख्या च पञ्चगुणिता पञ्चदशसङ्ख्या च स्यात्~। सत्यम्~। तर्हि पञ्चगुणनैव युक्ता यदि द्वयोर्वृत्तयोः कलात्मकयोश्चापयोस्तुल्यत्वं स्यात्~। इह तु न तयोस्तुल्यत्वम्~। चतुस्सङ्ख्यज्याचापाच्चापस्य

\newpage

\noindent द्विगुणितस्य ज्यामानीय सैव पुनस्तत्र परिणम्यते~। न केवला चतुस्सङ्ख्या~। अतोऽस्याश्चतुर्विंशतिसङ्ख्यत्वम्~। चतुःसङ्ख्यज्यायास्त्रिसङ्ख्यज्यायाश्च चापान्तरस्य ज्यामानीय सैव तत्र परिणता सप्तसङ्ख्या~। न पुनस्त्रिसङ्ख्यैव तत्र परिणम्यते~। त्रिसङ्ख्यैव तत्र परिणम्यमाना पञ्चदशसङ्ख्या स्यात्, चतुस्सङ्ख्या च विंशतिसङ्ख्या~। मूलवृत्तचापद्विगुणचापज्ये द्वे
इष्टद्वयं कृत्वा मुहुर्मुहुस्तद्वर्गव्यासवृत्तज्ये परिणम्येते~। अतोऽत्र त्रिकचतुष्कयोरिष्टयोराहतिर्द्वादशसङ्ख्या~। सा पुनर्द्विघ्नी चतुर्विंशतिसङ्ख्या~।
त्रिकचतुष्कयोरिष्टयोर्वर्गान्तरं सप्तसङ्ख्यम्~। सा पुनस्तच्छिष्टपदज्या~। एते उभे च द्वितीयपदस्थे~। कथं तदपि माधवोक्तन्यायेनानीयमाने सिद्ध्यति~। तत्र पञ्चव्यासार्धे पञ्चविंशतिव्यासार्धे च एतत्पदान्तरप्रदर्शनं शक्यम्~। तत्र पञ्चविंशतिव्यासार्धे चेत् कर्णहृते चतुर्विंशतिसप्तसङ्ख्ये स्तः~।
पञ्चसङ्ख्यव्यासार्धे अकृतहरणे अपि चतुर्विंशतिसप्तसङ्ख्ये स्तः~। परिणामोऽनयोरेव कार्य इति~। तत्रैवं परिलेखनं \textendash\ पञ्चविंशतिव्यासार्धे पदादितः प्रभृत्युत्तरतो विंशतिसङ्ख्यार्धज्या लेख्या~। चत्वारिंशन्मितां शलाकां मध्येऽङ्कितां कृत्वा व्यासरेखास्पृष्टाङ्कां परिधिस्पृष्टोभयाग्रां कृत्वा तदुत्तरार्धे वा रेखां लिखेत्~। पञ्चव्यासार्धे चेदेतत्पञ्चांशतुल्या चतुस्सङ्ख्यैवैतत्स्थाने लेख्या~। सैव तत्पञ्चगुणे परिणतेयं विंशतिसङ्ख्या~। तदग्रात् प्रभृति केन्द्रान्तं पञ्चविंशतिसङ्ख्यां कर्णरेखां च कुर्यात्~। पुनरपि चत्वारिंशत्सङ्ख्यां शलाकां मध्येऽङ्कितां कृत्वा एतत्कर्णस्पृष्टमध्यां परिधिस्पृष्टोभयाग्रां विन्यस्य तत्सङ्ख्यार्धानुसारिणी च रेखा कार्या~। सा च विंशतिसङ्ख्या चतुस्सङ्ख्यास्थानीया~। सा प्रथमपदमुल्लङ्घ्य द्वितीयपदेऽपि प्रसृता~। ततः प्रत्यक्पदसन्धितश्चाभितस्तावदन्तरिते परिधिभागे बिन्दू कृत्वा तदुभयाग्रप्रापि सूत्रं प्रसार्य रेखां
कुर्यात्~। तदुत्तरार्धमिहानीता चतुर्विंशतिसङ्ख्या ज्या~। ततोऽस्याः पदान्तरगतत्वम्~। तदग्रात् प्रभृति केन्द्रान्तं सूत्रं प्रसार्य रेखां कुर्यात्~। सा
पञ्चविंशतिसङ्ख्या~। तदर्धज्याकर्णो व्यासार्धतुल्यः~। स कृत्स्नो द्वितीयपदगतः~। तत्कोटिः पुनरितरकर्ण एव दृश्या, यत एतच्छरोनव्यासार्धतुल्या सा~। द्वितीयविंशतिज्येतरकर्णयोगात् प्रभृति प्रत्यगायता रेखा व्यासमुल्लङ्घ्यापि द्वितीयपदस्थज्याप्रापिणी कार्या~। तद्योगादेव दक्षिणतस्तु व्यासपर्यन्ता रेखा कार्या~। सा

\newpage

\noindent प्रथमत्रैराशिक इच्छाफलम्~। द्वितीयपदस्थज्याया उत्तरखण्डो द्वितीयत्रैराशिक इच्छाफलम्~। तत्र दक्षिणोत्तरायतां पदसन्धितः प्रवृत्तां विंशतिसङ्ख्यां विंशतिसङ्ख्यद्वितीयज्याकोट्या पञ्चदशसङ्ख्यया निहत्य क्वचिद्विन्यस्य द्वितीयज्यामपि विंशतिसङ्ख्यामुभयपदगतां प्रागायताया विंशतिसङ्ख्यायाः कोट्या पञ्चदशसङ्ख्ययैव हत्वा विन्यसेत्~। ते उभे फले प्रत्येकं शतत्रयसङ्ख्ये योजयेत्~। तत् षट्छतसङ्ख्यं पञ्चविंशतिसङ्ख्येन व्यासार्धेन हरेत्~। फलं चतुर्विशतिसङ्ख्यम्~। तत्तुल्या वायुकोणपदगता ज्या~। एवं योगयोग्यतामापाद्य तयोर्योगः कृतः~। वियोगयोग्ययोः पुनरे(वा~?~का) पञ्चदशसङ्ख्या~। इतरा विंशतिसङ्ख्यैव महती पूर्वं लिखिता~। याम्योत्तरायता द्वितीया तत्कर्णस्पृष्टमध्यत्रिंशन्मितशलाकार्धमिता~। तस्या दक्षिणार्धमिह ग्राह्यम्~। यतस्तच्चापहीनस्य विंशतिसङ्ख्यज्याचापस्य ज्या एतयोर्वियोगयोग्यतामापाद्य विश्लेषणेन कार्या~। अतस्तत्र विंशतिसङ्ख्यां ज्यामितरस्याः पञ्चदशसङ्ख्यायाः कोट्या विंशतिसङ्ख्यया हत्वा स्थापयेत्~। तां पञ्चदशसङ्ख्यां पुनर्विंशतिसङ्ख्यायाः कोट्या पञ्चदशसङ्ख्ययैव हत्वा स्थापयेत्~। एवं तयोर्विंशतिपञ्चदशवर्गयोर्विश्लेषं कृत्वा शिष्टं
पञ्चविंशत्यूनशतद्वयसङ्ख्यम्~। पञ्चविंशतिसङ्ख्येन कर्णेन हृत्वाप्तं फलं सप्तसङ्ख्यं स्यात्~। तथा वायुपदगतज्यायाः पूर्वमानीतायाः कोटिरुदक्पदसन्धितः प्रवृ(त्तः ? त्ता) प्रत्यगायता~। एते एव त्रिकचतुष्कयोरिष्टयोर्लीलावत्युक्तप्रकारेणानीते कोटिभुजे~। तत्रेष्टयोस्त्रिकचतुष्कयोराहतिर्द्विघ्नी चतुर्विंशतिसङ्ख्या वायुपदगता दक्षिणोत्तरायता ज्या~। त्रिकचतुष्कयोर्वर्गान्तरतुल्या तत्पदस्थैव पूर्वापरायता
सप्तसङ्ख्या~। माधवोक्तप्रकारेणाप्येते एव सिद्ध्यतः~। तत्राप्येता एव रेखा व्यासार्धपञ्चांशेन मेयाः~। तथा सति महति वृत्ते विंशतिसङ्ख्ये एवाल्पवृत्ते
गते चतुस्सङ्ख्ये स्तः, पञ्चदशसङ्ख्ये त्रिसङ्ख्ये च~। व्यासार्धं च पञ्चसङ्ख्यमित्येव विशेषः~। तत्र चतुस्सङ्ख्ययोर्योगे कार्ये चतुस्सङ्ख्यामेकामितरकोट्या त्रिसङ्ख्ययैव हत्वा स्थापयेत्~। अन्यामपि चतुस्सङ्ख्यां प्रथमायाश्चतुस्सङ्ख्यायाः कोट्या त्रिसङ्ख्ययैव हत्वा स्थापयेत्~। ते उभे अपि द्वादशसङ्ख्ये~। तयोर्योगश्चतुर्विंशतिसङ्ख्यः पञ्चसङ्ख्येन व्यासार्धेन हर्तव्यः~। तद\renewcommand{\thefootnote}{१}\footnote{ततः \textendash\ क. पाठः.}करणे
व्यासार्धस्य पञ्चविंशत्यंशमापकेन मिता स्यात् सा चतुर्विंशतिसङ्ख्या इति योगजातेयम्~। 

\newpage

\noindent त्रिकचतुष्कसङ्ख्ययोः पुनर्वियोगः कार्यः~। तत्र चतुस्सङ्ख्या त्रिसङ्ख्याया, इतरस्याः कोट्या चतुस्सङ्ख्ययैव ह(त्वा ? ता) सती चतुर्वर्गः स्यात्~। त्रिसङ्ख्या\renewcommand{\thefootnote}{१}\footnote{स्यात्~। तत्र सङ्ख्या}मपीतरस्याश्चतुस्सङ्ख्यायाः कोट्या\renewcommand{\thefootnote}{२}\footnote{सङ्ख्याः भुजाकोट्या} त्रिसङ्ख्ययैव हत्वा त्रिवर्गत्वमापाद्य तयोर्विश्लेष इह कार्य इति चतुर्वर्गात् षोडशकात् त्रिवर्गे नवके विशोधिते शिष्टं सप्तसङ्ख्यम्~। तत्तुल्या त्रिचतुस्सङ्ख्ययोर्जीवयोः कर्णहरणाकरणात् पूर्वव्यासार्धगता द्वितीयपदस्था पूर्वापरायता ज्या स्यादिति भास्करमाधवोक्तयोः फलसाम्यादेकविषयत्वम्~। तत्र भास्करोक्तं कर्म तुल्याकारयोर्भुजाकोटिकर्णक्षेत्रयोः संयोजनेन स्वधनु\renewcommand{\thefootnote}{३}\footnote{संयोजनेन धनु}र्द्विगुणाचापजत्वम् एकाकारक्षेत्रयोरेव भुजाकोट्योश्चापवियोगजत्वमापाद्यते~। अत्र पुनस्तृतीयमपि कर्मान्त(रं) विद्यते~। तेन व्यासार्धमेवानीयते~। किमर्थं पुनर्व्यासार्धमानीयते~। चापयोगजवियोगजज्ययोरकृतहरणयोः पूर्वव्यासार्धवर्गतुल्यव्यासार्धजत्वात् तद्व्यासार्धस्य पूर्वव्यासार्धतो भेदात् तत्सिद्ध्यर्थं यत्नः कार्य इति~। कथं पुनर्भुजाकोटिज्यावर्गयोगस्य पूर्वव्यासार्धवर्गजत्वं स्यात्~। अत्र कोटिजीवाया भुजाजीवायाश्च चापयोर्योगस्य जीवैव व्यासार्धम्~। दोःकोटिचापाभ्यां पदपरि\renewcommand{\thefootnote}{४}\footnote{चापाभ्यां परि}पूर्तेः~। तत्र दोःकोटिज्ययोर्योगयोग्यतापादने परस्परनिजेतरमौर्विका स्वस्वसमैव~। कः पुनर्निजेतरशब्दस्यार्थः~। निजशब्द आत्मीयवाची~। निजा च सेतरा च निजेतरा~। सर्वासामपि जीवानां प्रत्येकमेकैकयेतरजीवया भाव्यम्~। सा च स्वचापावशिष्टपदचापज्या\renewcommand{\thefootnote}{५}\footnote{स्वचापविशिष्टचापज्या}~। सा सर्वासामितरा~। तस्मात् योगयोग्ययो(रपि?)र्द्वयोरपीतरजीवया भाव्यम्~। तत्रैका निजेतरा अन्या परेतरा~। तत्र स्वनिजेतरया न स्वा हन्यते~। कया पुनः~। परनिजेत(रा ? रया)~। परशब्देन योगयोग्ययोरित\renewcommand{\thefootnote}{६}\footnote{रिति}(र? रा) चोच्यते~। एवं जीवयोर्द्वयोर्योगयोग्ययोर्या परस्परमितरा तया परस्परनिजेतरया~। न केवलं परया तस्या, निजाया येतरा तया, स्वामभ्यस्य स्वनिजा च या या च पुनरितरा तया चान्यामभ्यस्येत्यर्थः~। यद्वा निजशब्देन योगयोग्ययोरितरोच्यते~। तयोर्ह्येकक्रियाकारकत्वेन सम्बन्धो विद्यत इति परस्परं स्वीय(त्वा ? त्वम्~।)तस्याः पुनर्या इतरा तस्याश्च कयाचिदितरया भाव्यमिति पूर्वमेवोक्तम्~। सा च नेयम्\renewcommand{\thefootnote}{७}\footnote{सापनेयम् \textendash\ क. पाठः.}~। का पुनः~। तत्पदशिष्टज्या हि सा~। सा हि तस्याः प्रतियोगिनी~। अस्याश्च पदशिष्टज्यैव प्रतियोगिनी, 

\newpage

\noindent नेतरा योगयोग्या~। योगयोग्ययो(स्सं ? स्स)ख्यमेवैकक्रियाकारकत्वात्, न पुनर्मिथः प्रतियोगित्वम्~। अतस्तयोर्निजत्वमेवेतरेतरम्~। तयोः प्रतियोगिन्यौ पुनरितरे~। निजाया इतरा निजेतरा~। अन्य(तरं ?)स्या इतरा अन्येतरा~। एवं परस्परनिजेतरमौर्विकाभ्याम्~। अयमभिप्रायः \textendash\ ययोर्योगः कार्यः ते उभे अपि यदि भुजात्वेन विवक्ष्येते तदैव परस्परकोट्याभ्यस्येति वक्तुं युक्तम्~। यदा पुनरुभयोः कोटित्वमेव तदा परस्परभुजा\renewcommand{\thefootnote}{१}\footnote{भुजाज्या \textendash\ क. पाठः.}भ्यामपि च इति वक्तुं युक्तम्~। यदा पुनरेका कोटित्वेन विवक्ष्यते इतरा पुनर्भुजात्वेन तदा पुनः परस्परकोटिभ्यां परस्परभुजाभ्यामित्युभयमपि वक्तुं न युक्तम्~। तदा कोटिं पर\renewcommand{\thefootnote}{२}\footnote{परस्पर \textendash\ क. पाठः.}कोट्या हत्वा भुजां परभुजया चेति विविच्य वक्तव्यम्~। द्वयोः कोटित्वेन वा भुजात्वेन वा साम्याभावात्~। एवं त्रिष्वपि प्रकारेषु व्याप्त्यर्थं साधारण्येनोच्यते परस्परनिजेतरमौर्विकाभ्यामिति\renewcommand{\thefootnote}{३}\footnote{भ्यामपि \textendash\ क. पाठः.}~। एवमत्र भुजाकोट्योरेव योगयोग्यत्वात् तत्र या भुजा सा कोटीतरया हन्तव्या~। अत्र च कोटीतरा स्वतुल्या~। उभयोः परस्परं भुजाकोटित्वात्~। तदभावे हि गुणगुण्ययोर्भेदः~। भुजाकोटित्वसम्बन्धे सत्येवं साम्यमेव गुणगुण्ययोः स्यात्~। तदा भुजयोः संवर्ग एवैकः~। तयोः कोटिश्चेतरस्या भुजाया इतरया तस्याः कोट्या स्वतुल्ययैव हन्तव्या~। तस्मात् स कोटिवर्गतुल्यः~। तयोर्योगः कर्णवर्ग एव~। जीवे परस्परनिजेतरमौर्विकाभ्यामभ्यस्य हरणात् पूर्वमेव कृतो यो योगः स एवायं कर्णवर्गः~। स पुनर्विस्तृतिगुणेन विभाज्यः~। विभज्यमानयोरेव योगयोग्यत्वात्~। किं योगस्य विलम्बनेनेति प्रागेव योगः कृतः~। स पुनर्यदा कर्णेन हृतस्तदा स्ववृत्तव्यासार्धमेव~। हरणात् प्राक् कस्यचिद्वृत्तस्य व्यासार्धमेव~। तच्च वृत्तमेतस्माद्वृत्ताद्धारकसङ्ख्यया आवृत्तम्~। अतोऽस्मात् तावद्गुणत्वं तस्य~। अतोऽस्य व्यासार्धस्य वर्ग एव तस्य व्यासार्धम्~। यद्वा, हरणमपि क्रियते यद्व्यासार्धगते इमे भुजाकोट्यौ तद्व्यासार्धमानेयमिति~। स पुनर्येन व्यासार्धेन हृतस्तेनैव यदि गुण्येत तर्हि पूर्वव्यासार्धात् तद्गुणवृत्तव्यासार्धं स्यात्~। किमर्थं तदानीयते~। उच्यते~। यदा तुल्ययोरेव द्वयोर्जीवयोर्योगः क्रियते तदा निजेतरमौर्विकाभ्यां गुणिते प्रत्येकमिष्टयोर्भुजाकोट्योराहतिः स्यात्~। सा द्विघ्नी तद्योगश्च स्यात्~। तस्य विस्तृतिगुणेन हरणे सावयवत्वं स्यादिति हरणं न कृतम्~। यतः सा ज्या

\newpage 

\noindent स्ववृत्तात् स्वव्यासार्धगुणे वृत्ते परिणता स्यात्~। एवं योगार्हयोर्योगः कृतः~। तत्रैव भुजाकोट्योर्वियोगः कार्यः, भुजाचापद्विगुणज्याया अानीतत्वात् तत्कोटिरप्यानेयेति~। भुजाकोट्योर्योगे कार्ये च तयोः परस्परनिजेतरमौर्विका(भ्या ? भ्य)स्तयो\renewcommand{\thefootnote}{१}\footnote{भ्यां तयो \textendash\ क. पाठः.}र्भुजाकोटिवर्गान्तरतुल्यत्वात् तद्वर्गान्तरमपि व्यासार्धेन हार्यम्~। हरणात् सावयवत्वापत्तेर्भीतः सन् ह\renewcommand{\thefootnote}{२}\footnote{सन्नह \textendash\ ख. पाठः.}रणमकुर्वन्नेव\renewcommand{\thefootnote}{३}\footnote{ह} महावृत्तसम्भवात्वमापादयति~। ततस्तयोः सम्बन्धिव्यासार्धसिद्ध्यर्थं तत्कृतियोगमपि न हरेत्~। अतः पूर्वपूर्ववृत्तात् पूवपूर्वव्यासार्धगुणे वृत्ते परिणम्यमाने ते उभे ज्ये वीचीमालावत् चक्रार्धान्तरपरिधिप्रदेशावप्राप्यैव निवर्तमाने तद्व्यासोभयाग्रात् प्रभृत्यन्तः प्रविशन्त्यावितरव्यासमासाद्यापि सन्निकृष्टे पुनरितरार्धेऽपि स्वस्वव्यासाग्रासन्नप्रदेशं प्राप्यान्योन्यं विप्रकृष्यमाणे इतरव्यासाग्रासन्नप्रदेशं प्राप्यापि\renewcommand{\thefootnote}{४}\footnote{प्राप्नुयापि} निवर्तमान कदाचिद्यदृच्छया वा कर्तृकौशलाद्वा व्यासाग्रात्यासन्नप्रदेशं प्राप्नुतः~। तदा तच्चापमल्पमेवेति तज्ज्यार्धवर्गे तच्छरवर्गं सत्र्यंशं क्षिप्त्वानीयमानं धनुः सुसूक्ष्मं स्यात्~। पुनः कृत्स्नेऽपि वुत्ते तदावृत्तिं ज्ञात्वा एतद्धनुस्तावद्गुणं कृत्वा तच्छिष्टचापमप्येवं नीत्वा संयोज्य तत्परिधिपादः कृत्स्नः परिधिर्वा ज्ञेयः~। तत्र पुनः किं कर्तृकौशलमिति तदप्युदाहरणेन प्रदर्श्यते~। अत्र तत्वव्यासार्धे ये सप्तचतुर्विशतिसङ्ख्ये ज्ये उत्पादिते तन्मार्गेणेष्टयोराहतिर्द्विघ्नीत्यादिनैव वर्गगुणोत्तरव्यासार्धवृत्ते परिणमनं कार्यम्~। यावदल्पत्वेन प\renewcommand{\thefootnote}{५}\footnote{त्वेन न प \textendash\ क. पाठः.}रितोष इत्येको मार्गः~। मार्गान्तरं चात्र मृग्यम्~। अत्र या पञ्चविंशतिसङ्ख्या व्यासार्धज्या या च पुनश्चतुर्विशतिसङ्ख्या कोटिः, तद्वदिष्टयोगद्वारोऽपि मार्गो मृग्यः~। तत्र तावत् प्रथमे मार्गे सप्तकचतुर्विंशत्योरिष्टयोर्घातोऽष्टनृपसङ्ख्यः~। स द्विगुणः षड्देवसङ्ख्यः~। तयोर्वर्गान्तरं भेषुसङ्ख्यम~। चतुर्विंशतिवर्गश्चतुर्विंशत्यूना षट्छती~। सप्तवर्ग एकोनपञ्चाशत्~। तद्योगश्च पञ्चविंशत्युत्तरा षट्छती~। तद्व्यासार्धे वृत्ते षड्देवसङ्ख्या भेषुसङ्ख्या च भुजाकोटिरूपेणावस्थिते ज्ये~। पुनरपि ताभ्यामिष्टाभ्यां पञ्चविंशत्युत्तरषट्छतीव्यासार्धवृत्ते परिणामः कार्यः~।

\newpage

\noindent एवमुत्तरोत्तरं वर्गगुणे व्यासार्धे परिणम्यमानयोरेकस्यां यदा व्यासार्धासन्नत्वम् इतरस्या अत्यल्पीयस्त्वं च यावदपेक्षं स्यातां तावदेवं कुर्यादित्यादिरेको मार्गः~। अन्यस्तु पञ्चविंशतिव्यासा(र्धं ? र्ध)तत्कोटिचतुर्विंशतिसङ्ख्ये इष्टे आश्रित्य प्र(वृ ? व)र्तमानः~। तत्रेष्टयोराहतिः षट्छती~। सा द्विघ्नी पुनर्द्वादशशतसङ्ख्या~। तद्वर्गान्तरमेकोनपञ्चाशत्, यतस्तदन्तरस्यैकत्वात् तद्योग एव वर्गान्तरम्~। सा भुजा~। तद्वर्गयोग एकाधिकं शतद्वादशकम्~। यतः पञ्चविंशतिवर्गः पञ्चविंशत्युत्तरा षट्छती~। इतरश्चतुर्विंशत्यूना सैव षट्छती~। तन्निम्नपूरणायैकोना पञ्चविंशतिरेवालमिति षट्छतीद्वयमेकाधिकं स्यादिति तत्र कोट्या व्यासार्धासन्नत्वं स्यात्, यत एकमेवान्तरम्~। ततोऽपि द्वौ मार्गौ विद्येते भजाकोटीष्टद्वारश्च कर्णकोटीष्टद्वारश्च~। तत्रापि द्वितीये कोट्याः कर्णगुणने शतद्वादशकमेकाधिकेन तेनैव गुणनीयम्~। तत्र स्थानविभागे शतद्वादशकं शतद्वादशकेन गुणनीयम् एकेन चेतरखण्डेन~। तत्रैकेन गुणितं शतद्वादशकं शतद्वादशकमेव~। शतद्वादशकं शतद्वादशकेन गुणितमयुतस्थाने चतुश्चत्वारिशंदधिकं शतम्~। (पा ? द्वा)दश(त?)कवर्गतुल्यत्वात् तत्र~। ततः खद्वय(द्वादश)काब्धिमनुसङ्ख्यो घातः~। स द्विगुणः खद्वयजिनाहिमनुयुक्सङ्ख्यः~। सा कोटिः~। तयोर्योगतुल्यं वर्गान्तरं भूव्योमजिनसङ्ख्यम्~। कृतियोगे पुनः कोटिकृतिरब्धिमनुगुणितमयुतम्~। तद्द्विघ्नं सैकं वर्गयोगः~। {\qt राश्योरन्तरवर्गेण द्विघ्ने घाते युते तयोः वर्गयोगो भेवदेवमि}त्युक्तत्वात्~। रूपाकाशजिनाष्टाष्टयमसङ्ख्यः स कर्णः~। तत्राप्येकमेव कर्णकोट्यन्तरम्~। एवमुत्तरोत्तरमपि कर्णकोटिपरम्परामार्गे एकान्तरावेव कोटिकर्णाविति तन्मार्गः साधीयान्~। कथं पुनस्तत्र सर्वेषामेकान्तरत्वं निर्णीतम्~। उच्यते~। यदेतदेकान्तरितं कोटिकर्णद्वयम् एतयोः कोटिकर्णयोरेव घातो द्विघ्नस्तदूर्ध्वगा कोटिः~। तद्वर्गयोगश्च कर्णः~। वर्गयोगश्चान्तरवर्गयुतो द्विघ्न एव घातः~। स च कर्णः~। तस्माद्द्विघ्नघाततुल्यायाः कोट्या रूपवर्गेणैकेनाधिक एव कर्णः~। तयोरप्येकान्तरितत्वात् तदूर्ध्वगावप्येकान्तरितौ~। तत एव ततश्चोर्ध्वमपि~। इत्येकान्तरमेव कर्णकोटियुगं सर्वत्रापि~। इत्यस्मिन्मार्गे एकान्तरितत्वं कर्णकोटियुगलानां सर्वेषामति निर्णीतम्~। अनेन मार्गेणापरितुष्यतामुत्प्लुत्य कियन्तञ्चित् प्रदेशं गत्वा एष एव मार्ग आश्रयणीयः~। कथमुत्प्लवनम्~। एवं ह्यत्रोत्प्लवनम्~। वर्गस्थानेषु यावदपेक्षमूर्ध्वं यत्र क्वाप्ये-

\newpage

\noindent कमिष्टं कल्पयित्वा इतरदन्यस्थानेऽप्येकं कल्पयेत्~। तद्यथा \textendash\ परार्धदशकमेकमिष्टम्~। अन्यद्रूपाधिकमेतेदव~। तत्रेष्टयोराहतिस्तृतीये स्थानाष्टादशके द्वितीये) स्थानाष्टादशकेऽप्याद्यस्थाने आदितः सप्तत्रिंशे एका सङ्ख्या एकोनविंशे स्थानेऽपि~। सा द्विघ्नी तत्रोभयत्र द्विसङ्ख्या कोटिः~। तयोरिष्टयोः पुनर्वर्गान्तरं सैका परार्धविंशतिः~। तस्मादेकोनविंशे स्थाने द्वयमाद्यस्थाने चैकम्~। तद्वर्गयोगश्च (द्विघ्न)घातादेकाधिकः~। ततस्तस्याङ्का आद्यस्थान(के? एक)सङ्ख्यः एकोनविंशे सप्तत्रिंशे च द्विसङ्ख्यः~। तस्मादष्टादशकत्रिकाद्यस्थानत्रिके क्रमोदकद्विसङ्ख्याः~। तदेव व्यासार्धम्~। तत्रापीष्टयोराहतिर्द्विघ्नीत्याद्येव कर्म कार्यम्~। तत्रापि कोटिकर्णावेवेष्टराशी कल्पयित्वा एतत्कर्मावर्तनीयम्~। एतत्सर्वं ज्यार्धसूत्रेणैव सूचितं खण्डज्यान्तरविषयत्रैराशिकप्रदर्शनेन विवृतं च~। नन्वेतत्सूत्रं नं नि\renewcommand{\thefootnote}{१}\footnote{सूत्रं नि}रपेक्षं ज्यानयने ज्याछेदविधानसूत्रसोपक्षत्वाद्, यतस्तन्न्यायानीते प्रथमाद्वितीयज्ये इह साधनतयोक्ते~। {\qt प्रथमाच्चापज्यार्धाद् यैरूनं खण्डितं द्वितीयार्धमि}ति ते एवानूद्य शेषानयन एवेहोपायप्रदर्शनादिति चेन्न~। प्रथमाद्वितीययोरानयनमप्यनेनैवन्यायेन सिद्ध्यति~। अनेनैव ज्यार्धोपदेशसहकृतेन\renewcommand{\thefootnote}{२}\footnote{देशविकृते \textendash\ क. पाठः.} सिद्ध्यति~। हन्त ज्यार्धोपदेशेनैव सर्वा जीवाः पठिताः~। किमर्थं पुनस्तेषामानयनायेदं सूत्रमारभ्यत इति चेत्, तेषां सङ्ख्यामात्रसिद्ध्यर्थमेव नैतत्सूत्रमारभ्यते~। किमर्थं तर्हि~। तद्युक्तिप्रदर्शनायैव हि केवलम्~। ज्ञातयुक्तीनां पुनरवयवाः सुग्रहाः~। तन्न्यायातिदेशेन कृत्स्नमपि ग्रहगणितं स्फुरेदिति ज्यार्धसूत्रयुक्तिपरत्वादस्य तदपेक्षत्वं न दोषः~। तच्च युक्तिपरमेव~। अन्यथा पठितस्य पुनःपाठादानर्थक्यमेव~। पठितानि हि ज्यार्धानि सूर्यसिद्धान्तादिषु~। तैरेव खण्डज्या अपि सिद्ध्येयुः~। अपिच तदानयनञ्च तेष्वेवोक्तं\textendash

\begin{quote}
{\qt राशिलिप्ताष्टमो भागः प्रथमज्यार्धमुच्यते~।\\
तत्तद्विभक्तलब्धोनमिश्रितं तद्द्वितीयकम्~॥

आद्येनैवं क्रमात् पिण्डाद् भङ्क्त्वा लब्धोनितैर्युतैः~।\\
खण्डकैः स्युश्चतुर्विंशज्यार्धपिण्डाः क्रमादमी~॥}
\end{quote}

\noindent इति~। तत्र त्रैराशिकं निगूढमिति तदाविष्करणमनेन क्रियते~। कथं तर्ह्यनेनेवाद्यद्वितीयज्ये सिद्ध्यतः~। तदपि चापशब्देन सूचितं, चापमेव ज्यार्धं-

\newpage

\noindent चापज्यार्धमिति~। तेनापि सूर्यसिद्धान्ताभिप्रायः प्रदर्शितः~। {\qt राशिलिप्ताष्टमो भागः प्रथमज्यार्धमुच्यते} इति वदतो मयस्य ज्योतिश्चक्रस्य षण्णवत्यंशे प्रायेण चापज्ययोः साम्यं स्यात्~। ते(न) राशिलिप्ताष्टमो भाग एव प्रथमज्यार्धतया ग्राह्य इत्यभिप्राय इत्येतच्चात्र चापशब्देन सूच्यते~। अस्माकं पुनस्तदेव चापज्यार्धमित्यत्र न तात्पर्यम्~। तस्यापि विलिप्तानवकान्तरितत्वेन स्थौल्यं मन्यमानानां ततोऽप्यल्पस्य चापस्य प्रथमज्यार्धतया ग्रहणमस्त्विति तत्रापि त्रैराशिकप्रसरणाय तन्निगू\renewcommand{\thefootnote}{२}\footnote{रू}ढत्रैराशिकाविष्करण एव तात्पर्यम्~। अत एव सङ्ख्याविशिष्टतया नोक्तम्~। तत्र हि राशिलिप्ताष्टमांशत्वेनैव सङ्ख्या प्रदर्शिता~। तेनैव चापभागस्य परिधिषण्णवत्यंशत्वमपि सिद्धं, द्वादशराश्यात्मकत्वाच्चक्रस्य~। अस्माभिः पुनश्चापभागनियमो नेष्यते यावत्परितोषमल्पीकरणानुग्रहाय~। अल्पीकृत्यापि कतिथस्य चिदंशस्य चापस्य ज्यासाम्यमतात्त्विकमप्येष्टव्यं व्यवहारार्थमित्येव सूर्यसिद्धान्तकारस्याभिप्राय इत्यविशेषेणोक्त्यापि सूचितम्~। तस्माद्यस्य यत् प्रथमज्यार्धतयेष्टं तत् तस्य चापतुल्यमेवाभिमतम्~। तेनैकेनैवेतरेषामानयनन्यायोऽत्र प्रदर्श्यते, न पुनर्द्वितीयज्यापेक्षास्ति~। तत्त्रैराशिकयुक्तौ सिद्धायां तदानयनमपि तयैव सिद्ध्यति~। तत्सिद्धिश्चैवं खण्डज्यानयने त्रैराशिकमेतद्व्याचक्षणैरस्माभिः प्रदर्शिता, तस्यैवात्रापि योज्यत्वात्~। प्रथमचापस्य तावत् तदेव ज्यार्धमपीति येनाङ्गीकृतं तेन समस्तज्यापि तत्तुल्यैवेत्येतदवश्यमङ्गीकार्यं, यतो ज्यार्धात् कृत्स्नाया एव स्थौल्याभावः~। यतस्तदर्धज्या द्विगुणीकृतास्य समस्तज्या स्यात्~। कृत्स्नचापज्यार्धन्तरात् अष्टां(श? श)तुल्यमेव हि तदर्धचापज्यान्तरम्~। तस्मिन् द्विगुणीकृते पुनरेतदन्तरचतुरंश एव~। ते(न) तत्समस्तज्यायाश्च सिद्धत्वाद्द्वितीयचापमध्योत्था कोटिर्ज्ञेया, प्रथमचापमध्योत्था च~। कथम्~। तां समस्तज्यां प्रथमज्याकर्णस्पृष्टमध्यां परिधिस्पृष्टोभयाग्रां विलिख्य तद्युक्तिः प्रा\renewcommand{\thefootnote}{२}\footnote{विलिख्य प्रा \textendash\ क. पाठः.}ग्वदेव प्रदर्श्या~। तत्रापीदं त्रैराशिकं \textendash\ व्यासार्धकर्णस्य प्रथमज्याकोटिरेव कोटिस्तदा समस्तज्यार्धस्य कियतीत्येकम्~।
व्यासार्धकर्णस्य चापज्यैव भुजा तदा समस्तज्याशरोनव्यासार्धस्य कियतीत्यपरम्~। तद्योगो द्वितीयचापमध्यगता भुजाज्या~। तद्वियोगः
प्रथमचापमध्यगतार्धज्या~। सैव द्विगुणीकृता समस्तज्येत्युच्यते~। तथा द्विगुणितया प्रथम-

\newpage

\noindent ज्यार्धमप्यानेयम्~। प्रथमचापमध्याग्रस्य व्यासार्धकर्णस्य स्वाग्रस्पृष्टा कोटिरियती, तदा समस्तज्याकर्णस्य कियतीति प्रथमज्याखण्डोऽपि लभ्यः~। यद्वा प्रथमं चापज्यार्धमेव चापभागार्धस्य समस्तज्यां कल्पयित्वा पूर्वव्यासाग्रात् प्रभृति परिधिस्पृष्टोभयाग्रां तां स्वार्धतुल्यया परिधिभागचतुर्थांशतुल्यभुजज्यया हत्वा त्रिज्ययैव विभज्य लब्धं चापार्धस्योत्क्रमज्या~। तां व्यासार्धात् त्यक्त्वा यच्छिष्टं लभ्यते सैव प्रथमचापमध्योत्था काटिः~। कथं पुनस्तत्फलस्य चापार्धोत्क्रमखण्डत्वम्~। चापार्धसम्बन्धिसमस्तज्याबाहुत्वात्~। यतस्तन्मध्यभुजज्यया हत्वा त्रिज्यया ह्रियत इति समस्तज्याकर्णस्य भुजैव सा समस्तज्याखण्डस्य कोटिज्याखण्डः~। भुजा चोत्क्रमखण्ड इत्यसकृदावेदितम्~। तस्मात् प्रथमज्यासूत्रगर्भेणैव न्यायेन सिद्धा को\renewcommand{\thefootnote}{१}\footnote{सिद्धान्तको \textendash\ क. पाठः.}टिरियमिति न कोटिरूपा वा भुजज्यारूपा वान्या काचिदिह जीवापेक्ष्यते, चापज्ययैव सिद्धत्वात्~। पुनस्तत्समस्तज्यया कृत्स्नचापभवया कृत्स्नचापार्धज्यया चानया प्रथमचापमध्याग्रगतयानीतया कोट्या च व्यासार्धेन च द्वितीयचापमध्योत्था कोटिरप्यानेया~। कथम्~। अत्रापि पूर्वोक्तमेव त्रैराशिकम्~। पूर्वं राश्यष्टमांशतुल्यतया वा राशित्रिंशांशमितेषु वा यथेष्टांशेषु तुल्यतया कल्पितेषु पदादितः प्रभृति यावतिथे चापखण्डे ज्याशरखण्डौ जिज्ञास्येते, तन्मध्यगतकोटिभुजाभ्यां भुजाकोट्योर्ज्याखण्डस्य च परस्परं नियमाच्छरखण्डस्य भुजायाश्च, इदानीं चापखण्डमध्यगतयोर्ज्ययोः शरयोश्चेत्येतावानेव विशेषः, न पुनस्त्रैराशिकस्य तद्युक्तेश्च~। कः पुनस्तेन जायमानो विशेषः क्रियायां फलति~। चापसन्धिगतभुजाकोटिभ्यामत्र गुणनं क्रियते~। अत्र सैव समस्तज्या गुण्या इति गुण्यराशेर्न विशेषः~। अत्र निरन्तरचापद्वयमध्यस्पृष्टोभयाग्रा सा कलप्यते वा लिख्यते वा, इदानीं चापमध्यगतयोर्भुजयोः कोट्योश्च खण्डयोरत्र जिज्ञास्यत्वात्~। न पुनश्चापसन्धिगतयोर्जीवयोः खण्डयोरत्र जिज्ञास्यत्वम्~। अतः पूर्वचापोत्तरार्धं चोत्तरचापपूर्वार्धं चैकीकृत्यान्यैः कल्पितैः समेऽप्यस्मिन् समस्तज्याया विशेषाभावात् ताभ्याम\renewcommand{\thefootnote}{२}\footnote{ताम \textendash\ क. पाठः.}र्धाभ्यां निष्पादितस्यास्य मध्यगताभ्यां प्रसिद्धचापसन्धिगगताभ्यां ताभ्यां पृथक् पृथगाहत्य त्रिज्ययैव हृत्वा तौ खण्डौ लभ्येते~। कल्पितचापसम्बन्धी (ये ? यो)

\newpage

\noindent भुजाखण्डः स पठितकोटिहताया लभ्यते शरखण्डश्च भुजाहतायाः~। एवं प्रथमज्याहतायाः समस्तज्यायास्त्रिज्याप्तः शरखण्डः
प्रथमचापमध्यो\renewcommand{\thefootnote}{१}\footnote{ख्यो \textendash\ क. पाठः.}त्थायाः कोट्याः पूर्वमानीतायाः शो(ध्या ? ध्यः)~। तत्र शिष्टं यत् सा द्वितीयचापमध्याग्रा कोटिः, शरखण्डस्यैव कोटिखण्डत्वात्~। तच्चैवं क्षेत्रं \textendash\ प्रथमचापाग्रात् केन्द्रान्तां व्यासार्धसम\renewcommand{\thefootnote}{२}\footnote{मं \textendash\ ख. पाठः.} रेखां लिखित्वा समस्तज्यां तत्स्पृष्टमध्यां तत्कर्णचापमध्यस्पृष्टोभयाग्रां लिखेत्, चापद्वयमध्यगते भुजाकोटिज्ये च~। तत्र तयोर्मध्यगतयोर्भुजाकोट्योः परस्परयोगात् खण्डितयोर्यावग्रगतखण्डौ तौ तस्य समस्तज्याकर्णस्य भुजाकोट्यात्मकौ~। तत्र दक्षिणोत्तरायतायाः द्वितीयचापमध्यगताया(भुजायाः) प्रथमचापमध्यगतायाः कोट्याः पूर्वापरायतायाश्च संयोगाद्बहिरुदगायतः खण्डो द्वितीयभुजाग्रगः~। तद्योगादेव पूर्वायतः खण्डः कोटिज्याग्रगः~। तत्र भुजाखण्डः समस्तज्याकर्णस्य कोट्यात्मकः, कोटिखण्डश्च भुजात्मकः~। एवं भुजाकोटिज्याग्राभ्यां समस्तज्यया च कर्णभूतयोत्पन्नमिदमर्धायतचतुरश्रं क्षेत्रं प्रतिचापभागमन्यादृशं, भुजाकोट्योः प्रतिचापं भेदात् कर्णस्य तुल्यत्वाच्च~। एकाकारेषु त्र्यश्रेषु महत्स्वणुषु च भुजाकोटिकर्णास्त्रय एव समानवृद्धिह्रासाः~। तस्माद्बाहुष्वेकस्य महत्त्वे इतरयोरपि तदनुरूपं महत्त्वमेव स्यात्~। अणुत्वमपि त्रयाणां तुल्यमेव~। अत्र पुनः कर्णस्य सदा साम्यमेव~। भुजाकोट्योरन्यतरस्या महत्त्वे इतरस्या अल्प(त्व)मेव स्यात्~। अन्यत्र सहैव वर्धेते ह्रसतश्च~। अत्रतु व्यस्तमेव, एकस्यां क्रमेण
वर्धमानायामन्यस्याः क्रमेण ह्रासादेकह्रासे चेतरवृद्धेः~। यथैकस्मिन् वृत्ते सदैव तुल्य एव व्यासार्धकर्णः~। कोटिभुजज्ये पुनर्नानापरिमाणे~। तथापि ते व्यासार्ध\renewcommand{\thefootnote}{३}\footnote{तथापि व्यासार्ध \textendash\ क. पाठः.}कर्णतां न ज(ह? ही )तः~। सदापि भुजाकोटिज्ययोर्व्यासार्धमेव कर्णः~। एवमत्रापि ज्याशरखण्डयोर्भुजाकोट्योः स्वचापोभयाग्रावगाहिनी समस्तज्या सर्वेषु चापेषु तुल्यरूपा सती खण्डकर्णत्वं न जहाति, नमनोन्नमनाभ्यां सर्वदा तावेवानुसरति~। भुजाकोट्यग्रस्पृष्टोभयाग्नत्वमेव कर्णत्वम्~। भुजाकोट्योरपीतरेतरसंश्लिष्टेतराग्रे कर्णाग्रे न त्यजतः~। व्यासार्धकर्णक्षेत्रस्य तच्चापसमस्तज्याकर्णस्य च प्रतिचापं नानाकारतयैव केवलं साम्यम्~। इतरेतरतुल्या-

\newpage


\noindent कारत्वमपि सदैव स्यात्~। तच्च पुनः साम्यं समस्तज्यासम्बन्धिचापमध्याग्रव्यासार्धकर्णक्षेत्रस्य तद्गतज्याशरखण्डकोटिभुजाकर्णस्य
समस्तज्याकर्णक्षेत्रस्य च सर्वदा मिथः समानाकारत्वमेव~। तत्प्रदर्शनार्थमेव व्यासार्धतुल्यां शलाकां समस्तज्यातुल्यशलाकां च कृत्वा समस्तज्यातुल्यशलाकामध्यं व्यासार्धतुल्यशलाकाग्रतः समस्तज्याशरतुल्येऽन्तरेऽधस्ताद्दृढीकृत्य भ्राम्यते~। एवं भ्राम्यमाणे शलाकाद्वये तदग्रद्वयं सदा परिधिस्पृष्टमेव~। व्यासार्धतुल्यशलाकाया मूलं सदा केन्द्रगमेवेति तस्यां भ्राम्यमाणायां यथा य(था? थं)दिक्चतुष्टयाभिमुखत्वं विज्ञाय दिगेपक्षया तिर्यक्त्वं स्यात्~। तद्वशाद्धि भुजाकोट्योर्वृद्धिह्रासौ सम्भवतः~। शलाके सदापि समपरिमाणे एव~। व्यासार्धकर्णस्य\renewcommand{\thefootnote}{१}\footnote{कर्णस्य च} समस्तज्याकर्णस्य च स्वस्वभुजाकोटिवशाज्जायमानो विकारः सर्वदा क्षेत्रयोरुभयोः समान एव~। तयोर्विकारः पुनस्तत्तत्परिध्यवयवस्पर्शवशात् प्रत्यवयवं प्रतिक्षणं वा नानाभूतः~। एवमुभयोरपि सदा विक्रियमाणत्वेऽपि तत्तत्क्षणे समान एव विकारः~। यथैकस्य क्षेत्रस्येदानीं विकारः अन्यस्यापीदानीं तथाभूत एव~। इदानीं पुनरन्या(वा ? व)न्याकारौ~। तथापि परस्परं समानाकाराविति प्रतिक्षणं क्षणान्तरेभ्यो भेदे विद्यमानेऽपीतरेतरं
साम्यमेव स्या(दि? द)तस्तयो(रे ? र)वस्था शलाकायां भ्रमन्त्यां तुल्यतया प्रदर्शनीया~। तत्स्पृष्टपरिधिप्रदेशभेदवशात् क्षणभेदेषु
देशान्तरस्थिताभ्यामुभयोर्भेदश्च दर्शनीय इति तयोरेकस्मिन् क्षेत्रे भुजाकोटिकर्णेषु त्रिष्वेकस्मिन् ज्ञातेऽन्यत्र त्रिष्वपि ज्ञातेषु
ज्ञोतैकैकक्षत्रेऽपीतरयोरानयनमन्यक्षेत्रभवैर्ज्ञातैरेव कार्यम्, उभयोरपि त्रयाणां तुल्यकालं मिथः परिमाणसम्बन्धः एकप्र(म ? मा)कार इति~। यदा व्यासार्धकर्णस्य स्वभुजा स्वार्धतुल्या, तदा समस्तज्याकर्णस्यापि स्वभुजा स्वार्धतुल्या~। तेन तदानीं कर्णभुजयोः परस्परं परिमाणतः
सम्बन्धोऽर्धद्विगुणलक्षण उभयोः क्षेत्रयोः समान एव~। भुजायाः कर्णार्धत्वं कर्णस्य भुजाद्विगुणत्वं च तदानीमुभयोः स्यादेव~। अत एव प्रथमराश्यन्तमभितः\renewcommand{\thefootnote}{२}\footnote{धः \textendash\ क. पाठः.} स्थितानां चापयुगलानां शरखण्डाः प्रथमादिज्यातुल्याः~। तत्तुल्यत्वं च पठितानां विश्लेषणेन सङ्ख्यासाम्यादपि निर्णेयम्~। न पुनर्युक्त्यैव~। तद्यथा \textendash\ अष्टमनवमचापभागौ हि चतुर्विंशत्यर्धज्यापक्षे प्रथमराश्यन्तमभितः स्थितौ~। तद्युगलशरखण्डोऽष्टमनवमयोः शरखण्डयोर्योगः~। एवमष्टमः शर-

\newpage

\noindent खण्डः~। स पुनः सप्तदशो ज्याखण्ड एव सप्तसङ्ख्यः, तस्यैव छात् प्रभृत्युत्क्रमेणाष्टमत्वात्~। षोडशो ज्याखण्डः पुनर्नवमः शरखण्डः~। स चधाहसङ्ख्यः, तस्य छात् प्रभृति नवमत्वात्~। तद्योगोऽष्टमनवमचापखण्डयुगलस्य ज्याखण्डः~। स च षोडशसप्तदशज्याखण्डयोर्योग एव~। स च सप्तदश्या जीवायाः पञ्चदश्याश्च विश्लेषमात्रेणैवोत्पाद्यः~। स पुनः प्रथमज्यार्धतुल्यः~। एवं प्रथमराश्यन्तमभितश्चापयुगलयोः शरखण्डयोगो मखिभखियोगतुल्यः~। तस्मादष्टादश्याश्चतुर्दश्यां विशोधितायां शिष्टं द्वितीयपिण्डज्यातुल्यम्~। एवं पुनः पुनरधश्चोर्ध्वं चैकैकान्तरितंयोर्ज्ययोर्भेदा राश्यन्तमभितः स्थितानां शरयुगलानां खण्डास्तृतीयादिपिण्डज्यातुल्याः स्युः~। न केवलं चतुर्विंशतिज्यापक्ष एवैवं स्यात्~। क्व पुनस्तर्हि~। अन्यत्रापि नवत्यादिज्यापक्षेष्वखिलेष्वपि~। तेन राशिद्वयजीवा एवानीय निबध्याः, तृतीयराशिभवानां पुनः संकलितपरिकर्मणैव साध्यत्वात्~। तेन नवतिपक्षे षष्टिरेवावधार्या~। तृतीयराशिभवास्त्रिंशत संकलनेनैव सिद्धाः~। एकोनषष्टितमप्रथमयोरेकषष्टितमा स्यात्~। एवं द्वितीय\renewcommand{\thefootnote}{१}\footnote{द्वितीयादि}राशिभवानामधोधोगतया प्रथमराशावुपरिगतया च (तरा ? तास्ता)स्त(त्त)त्संयोजनमात्रेणैव साध्याः~। का पुनरत्र युक्तिः~। यदा व्यासार्धतुल्या शलाका प्रथमराश्यन्तगता स्यात् तदेतरशलाकाग्रे अष्टमनवमचापमध्यस्पृष्टे स्याताम्~। तदा अष्टमचापमध्यात् प्रभृति प्रत्यगायता या कोटिज्या अध्यर्धषोडशचापज्या या च पुनर्नवमचापमध्यात् प्रभृति दक्षिणायता भुजा अर्धनवमचापज्या तद्योगात् यौ तदग्रगौ खण्डौ तद्भुजाकोटिगतमेतत् क्षेत्रम्~। तत्र महा\renewcommand{\thefootnote}{२}\footnote{तत्र तत्र महा}क्षेत्रे अष्टमी ज्या व्यासार्धर्ध\renewcommand{\thefootnote}{३}\footnote{व्यासार्ध \textendash\ क. पाठः.}समा भुजा~। तथा समस्तज्याया अपि कर्णभूतायाः\renewcommand{\thefootnote}{४}\footnote{यां \textendash\ ख. पाठः.} तदर्धसमा सैव भुजा~। शरखण्डोऽर्धशो राश्यन्तमभितः स्थितस्यैकस्यैव चापभागस्यास्य शरखण्डः~। तदानयन एवं त्रैराशिकं \textendash\ यदि व्यासार्धकर्णस्य राशिज्या भुजा तदा समस्तज्या\renewcommand{\thefootnote}{५}\footnote{समस्तज्याया अपि कर्णभूतायास्तत्र \textendash\ क. पाठः.} कर्णस्य कियतीति~। तत्र गुणकाराद्द्गिगुणत्वाद्धारकस्येच्छार्धमेव फलम्~। सा च समस्तज्यार्धतुल्या चापभागदलस्यार्धज्यैव, यतो दलस्यार्धज्या द्विगुणा कृत्स्नस्य समस्तज्या~। यस्माद्राशिषोडशांशचापेषु प्रथमज्यातुल्यः राश्यन्तमभितः स्थितस्य राश्यष्टमचापस्य शरखण्डः~। एवं राश्यन्तमभितः स्थितस्य चापस्य शरखण्डः


\end{document}