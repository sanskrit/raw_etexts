\documentclass[11pt, openany]{book}
\usepackage[text={4.65in,7.45in}, centering, includefoot]{geometry}

\usepackage[table, x11names]{xcolor}
%\include{alias}

\usepackage{fontspec,realscripts}
\usepackage{polyglossia}

\setdefaultlanguage{sanskrit}
\setotherlanguage{english}
%\setmainfont[Scale=1]{Times New Roman}
%\newfontfamily\regular[Scale=1]{Times New Roman}
\defaultfontfeatures[Scale=MatchUppercase]{Ligatures=TeX} 
\newfontfamily\sanskritfont[Script=Devanagari]{Shobhika}
\newfontfamily\englishfont[Language=English, Script=Latin]{Linux Libertine O}
\newfontfamily\ssi[Script=Devanagari, Color=purple]{Shobhika-Bold}
\newfontfamily\qt[Script=Devanagari, Scale=1, Color=violet]{Shobhika-Regular}
%\newfontfamily\bqt[Script=Devanagari, Scale=1, Color=brown]{Shobhika-Regular}
%\newfontfamily\s[Script=Devanagari, Scale=0.9]{Shobhika-Regular}
\newfontfamily\s[Script=Devanagari, Scale=0.9]{Shobhika-Regular}
\newcommand{\devanagarinumeral}[1]{%
	\devanagaridigits{\number\csname c@#1\endcsname}}
\usepackage{fancyhdr}
\pagestyle{fancy}
\renewcommand{\headrulewidth}{0pt}
%\newfontfamily\e[Scale=0.8]{Shobhika-Regular}
\XeTeXgenerateactualtext=1
\usepackage{enumerate}
%\pagestyle{plain}
%\pagestyle{empty}
\usepackage{afterpage}
\usepackage{mathtools}
\usepackage{amsmath}
\usepackage{amssymb}
\usepackage{tikz}
\usepackage{graphicx}
\usepackage{longtable}
\usepackage{multirow}
\usepackage{footnote}
%\usepackage{dblfnote} 
\usepackage{xspace}
%\newcommand\nd{\textsuperscript{nd}\xspace}
\usepackage{array}
\usepackage{emptypage}
\usepackage{hyperref}   % Package for hyperlinks
\hypersetup{
	colorlinks,
	citecolor=black,
	filecolor=black,
	linkcolor=blue,
	urlcolor=black
}
\begin{document}

\cfoot{}

\noindent १९६ \hspace{4cm} सूर्यसिद्धान्तः
\vspace{1cm}


\noindent दिक्-साधनेन दिशः सममण्डलीयाङ्किताः । एतच्चिह्नाद्वलनान्तरेण क्रान्तिवृत्तदिशां सत्त्वात् । तत्र स्पर्शमोक्षदिङ्नियमार्थं क्रान्तिवृत्तप्राच्यपरानुसारेण चन्द्रसूर्ययोः स्पर्शमोक्षौ निर्णेयौ । ग्रहभोगस्य तद्वृत्तानुसारित्वात् । शीघ्रगचन्द्रः सूर्यषड्भान्तरितभूच्छायां सूर्यगत्यनुरुद्धगमनां प्रति पश्चादागत्यमेलनारम्भं करोत्यतश्चन्द्रबिम्बस्य पूर्वभागे स्पर्शः । भूभामतिक्रम्याग्रे चन्द्रो यदा गच्छति तदा चन्द्रस्य पश्वाद्भागे भूभावियोगोऽतः पश्चान्मोक्षः । सूर्यं चन्द्रः पश्चादात्याच्छादयत्यतः सूर्यस्य पश्चिमभागे स्पर्शः पूर्वभागे मोक्ष इति~॥~४~॥\\
\noindent अथ वलनवृत्ते वलनदानमाह \textendash

%\vspace{2mm}

\begin{quote}
{\ssi यथादिशं प्राग्ग्रहणं वलनं हिमदीधितेः~।\\
 मौक्षिकं तु विपर्यस्तं विपरीतमिदं रवेः~॥~५~॥ }
 \end{quote}
%\vspace{2mm}


 चन्द्रस्य ग्राह्यस्य स्पार्शिकं वलनं पूर्वचिह्नाद्यथादिशं दक्षिणं चेद्दक्षिणाभिमुखमुत्तरं चेदुत्तराभिमुखं पूर्वापरसूचादर्धज्यावद्वलनाश्रितवृत्ते देयम् । अत एव तद्वृत्तं वलनाश्रितसञ्ज्ञम् । मौक्षिकं मोक्षकालिकं तुकाराच्चन्द्रस्य वलनम् । विपर्यस्तं विपरीतं पश्चिमचिह्नात् पूर्वापरसूत्रादर्धज्यावद्दक्षिणं चेदुत्तरदिगभिमुखमुत्तरं चेद्दक्षिणदिगभिमुखं देयमित्यर्थः । सूर्यग्रहणे विशेषमाह\textendash विपरीतमिति~। सूर्यस्य ग्राह्यस्येदं स्पार्शिकं मौक्षिकं वलनं विपरीतं व्यस्तम् । मौक्षिकं वलनं पूर्वचिह्नात्पूर्वापरसूत्रादर्धज्यावद्दक्षिणं चेद्दक्षिणदिगभिमुखमुत्तरं चेदु\textendash


\newpage

\hspace{3.5cm} गूढार्थप्रकाशकेन सहितः~। \hfill १९७
\vspace{1cm}


\noindent त्तरदिगभिमुखं स्पार्शिकं वलनं पश्चिमचिह्नात् पूर्वापरसूत्रादर्धज्यावद्दक्षिणं चेदुत्तरदिगभिमुखमुत्तरं चेद्दक्षिणदिगभिमुखं देयमित्यर्थः । अत्रोपपत्तिः। चन्द्रस्य पूर्वभागे स्पर्श इतिसममण्डलपूर्वचिह्नाद्वलनान्तरेण स्पर्श इति तद्वृत्ते यथाशं स्पार्शिकं वलनं देयम् । पश्चिमोत्तराभिमुखस्य दक्षिणत्वाद्दक्षिणाभिमुखस्योत्तरत्वान्मौक्षिकं वलनं पश्चिमचिह्नाद्विपरीतं देयम् । सूर्यस्य तु पश्चिमभागे स्पर्शात् पश्चिमचिह्नात् स्पार्शिकं वलनं व्यस्तं देयम् । पूर्वभागे मोक्ष इति मौक्षिकं वलनं पूर्वचिह्नाद्यथाशं देवमिति~॥~५~॥\\
\noindent अथ द्वितीयवृत्ते स्पार्शिकमौक्षिकविक्षेपयोर्दानमाह\textendash

%\vspace{2mm}

\begin{quote}
{\ssi वलनाग्रान्नयेन्मध्यं सूत्रं यद्यत्र संस्प्रृशेत्~।\\
 तत् समासे ततो देयौ विक्षेपौ ग्रासमौक्षिकौ~॥~६ ~॥}
\end{quote}
%\vspace{2mm}


 प्रथमवृत्ते यत्र स्पार्शिकवलनाग्रं यत्र च मौक्षिकवलनाग्रंज्ञातं तस्माद्यत् प्रत्येकं सूत्रं रेखामित्यर्थः । मध्यं वृत्तमध्यबिन्दुं केन्द्ररूपं प्रति नयेत् । तद्रेखात्मकं सूत्रं समासे समासाख्यद्वितीयवृत्तपरिधौ यत्र यस्मिन् प्रदेशे संस्पृशेत् । स्पर्शंकुर्यात् ततस्तत्सूत्रादवधिरूपात् समासवृत्तेऽर्धज्यावद्यथादिशौस्पार्शिकमौक्षिकौ विक्षेपौ यथायोग्यौ देयौ । अत्रोपपत्तिः । वलनाग्रसूत्रं मानैक्यखण्डवृत्ते यत्र लग्नं तत्र क्रान्तिवृत्तप्राच्यपरा वा ततः सूर्याच्चन्द्रस्य विक्षेपान्तरेण सत्त्वात् समासवृत्तेवलनाग्रसूचाद्विक्षेपो देयो ग्रहबिम्बकेन्द्रज्ञानार्थम् । परं सूर्यग्र\textendash


\newpage

\noindent १९८ \hspace{3.5cm} सूर्यसिद्धान्तः
\vspace{1cm}


\noindent हणे । चन्द्रग्रहणे तु चन्द्रस्य विक्षेपवृत्तस्थत्वात् तदानीतवलनदानादवगतवलनाग्ररेखा मानैक्यखण्डवृत्तै यत्र लग्ना तत्र क्रान्तिवृत्तानुसृतप्राच्यपरा विक्षपमण्डले तत्स्थाने छाद्याच्चन्द्राच्छादकः सूर्यो विक्षेपान्तरेण विक्षेपदिग्विपरीतदिशि भवतीति वलनाग्रसूत्रात् समासवृत्तेऽर्धज्यावच्छरो व्यस्तो देय इति सिद्धम् । अत एव विपरीता शशाङ्कस्तेत्यग्र उक्तम्~॥~६~॥\\
\noindent अथ ग्राह्यवृत्ते स्पर्शमोक्षस्थानज्ञानमाह\textendash

%\vspace{2mm}


\begin{quote}
{\ssi विक्षेपाग्रात् पुनः सूत्रं मध्यबिन्दुं प्रवेशयेत्~।\\
 तद्ग्राह्यबिन्दुसंस्पर्शाद्ग्रासमोक्षौ विनिर्दिशेत्~॥~७~॥}
\end{quote}
%\vspace{2mm}


 विक्षेपाग्रं समासवृत्ते यत्र लग्नं तस्मात् सूत्रं रेखामित्यर्थः । अत्र रेखा सरला नायातीति शङ्कया प्रथमतोऽवधिद्वयान्तं सूत्रं धृत्वा तदनुसारेण रेखा कार्येति सूचनार्थं सूत्रोक्तिः सर्वत्रेति ध्येयम् । पुनर्द्वितीयवारं पूर्ववलनाग्राद्रेखाया मध्यकेन्द्रावधिकायाः कृतत्वात् तथैव विक्षेपाग्राद्रेखामित्यर्थः । वृत्तमध्यरूपकेन्द्रबिन्दुं प्रति गणकः प्रवेशयेत् प्रविष्टं कुर्यादित्यर्थः। तद्रेखाग्राह्यबिम्बवृत्तपरिध्योः संयोगाद्ग्रासमोक्षौ स्पर्शमोक्षौ गणको विनिर्दिशेत् कथयेत् । स्पार्शिकशराग्रसूत्रं ग्राह्यवृत्ते यत्र लग्नं तत्र स्पर्शः । मौक्षिकशराग्रसूत्रं ग्राह्यवृत्ते यत्र लग्नं तत्र मोक्ष इत्यर्थः । अत्रोपपत्तिः । मानैक्यखण्डवृत्ते यत्र ग्राह्यकबिम्बकेन्द्रं तस्माद्ग्राहकार्धेन वृत्तं ग्राहकवृत्तं ग्राह्यवृत्ते यत्र लग्नं तत्र स्पर्शमोक्षौ भवतः । तत्र वृत्ताकरणलाघवाद्ग्राहक\textendash


\newpage

\hspace{3.5cm} गूढार्थप्रकाशकेन सहितः~। \hfill १९९ 
\vspace{1cm}


\noindent केन्द्राद्ग्राह्यकेन्द्रं यावत् सूत्रं मानैक्यखण्डमितं ग्राह्यवृत्ते यत्र लग्नं तत्र परिध्योः स्पर्शमोक्षौ स्वस्वव्यासार्धयोगात्~॥~७~॥\\
\noindent अथ ग्रहणे विक्षेपस्य दिग्व्यवस्थां मध्यग्रहणज्ञानार्थं मध्यकालिकवलनदानं च लोकाभ्यामाह\textendash

%\vspace{2mm}

\begin{quote}
{\ssi नित्यशोऽर्कस्य विक्षेपाः परिलेखे यथादिशम्~।\\
 विपरीताः शशाङ्कस्य तद्वशादथ मध्यमम्~॥~८~॥

वलनं प्राङ्मुखं देयं तद्विक्षेपैकता यदि~।\\
भेदे पश्चान्मुखं देयमिन्दोर्भानोर्विपर्ययात्~॥~९~॥} 
\end{quote}
%\vspace{2mm}


 अर्कस्य ग्रहणे चन्द्रविक्षेपाः परिलेखे ग्रहणभेददर्शनप्रकारे यथादिशं यथास्थितदिशं नित्यशो नित्यं ज्ञेयाः । चन्द्रस्य ग्रहणे चन्द्रविक्षेपा विपरीता दक्षिणाश्चेदुत्तरा उत्तराश्चेद्दक्षिणाः । एतदनुरोधेनैव स्पार्शिकमौक्षिकविक्षेपौ देयौ । न यथागतदिशाविति ज्ञेयम् । अथानन्तरं तद्वशान्मध्यग्रहणकालिकविक्षेपदिशः सकाशात् सूर्यग्रहणे मध्यग्रहणकालिकस्पष्टविक्षेपदिक्चिह्नाच्चन्द्रग्रहणे मध्यकालिकविक्षेपदिग्विपरीतदिक्चिह्नादित्यर्थः । यदि यर्हीत्यर्थः । तद्विक्षेपैकता तद्वलनं विक्षेपोमध्यग्रहणकालिकविक्षेपः । अनयोरेकतैक्यं दिक्मम्बन्धेनेति शेषः। एकदिशीत्यर्थः । अत्र चन्द्रविक्षेपदिग्यथास्थितैव य विपरीतदिगिति ध्येयम् । प्राङ्मुखं पूर्वचिह्नसम्मुखम् । वलनाश्रितवृत्तेऽर्धज्यावच्चन्द्रस्य मध्यमं वलनं मध्यग्रहणकालिकं स्फुटं वलनं देयम् । भेदे वलनविक्षेपे दिशोर्भिन्नत्वे पश्चान्मुखम् । वलना\textendash


\newpage

\noindent २०० \hspace{3.5cm} सूर्यसिद्धान्तः
\vspace{1cm}


\noindent श्रितवृत्तेऽर्धज्यावन्मध्यग्रहणकालिकं चन्द्रस्य वलनं पश्चिमचिह्नसम्मुखं देयम् । सूर्यग्रहणे विशेषमाह\textendash भानोरिति~। सूर्यग्रहणे सूर्यस्य वलनं विपर्ययादुक्तवैपरीत्यात् । एकदिशि पश्चिमचिह्नसम्मुखं भिन्नदिशि पूर्वचिह्नसम्मुखं देयमित्यर्थः । फलितार्थस्तु चन्द्रग्रहणे मध्यकालवलनदिक्तत्कालविक्षेपयथागतदिशोर्दक्षिणत्व उत्तरचिह्नाद्वलनाश्रितवृत्तेऽर्धज्यावन्मध्यवलनं पूचिह्नाभिमुखं देयम् । तयोरुत्तरत्वे दक्षिणचिह्नात् पूर्वाभिमुखं वलनं देयम् । यदि दक्षिणवलनमुत्तरविक्षेपस्तदा दक्षिणदिक्चिह्नादर्धज्यावत् पश्चिमचिह्नाभिमुखं वलनं देयम् । यद्युत्तरं वलनं दक्षिणविक्षेपस्तदा वलनाश्रितवृत्त उत्तरचिह्नात्पश्चिमचिह्नाभिमुखं वलनमर्धज्यावद्देयम् । सूर्यग्रहणे तु द्वयोर्दक्षिणत्वे वलनाश्रितवृत्ते दक्षिणचिह्नात् पश्चिमचिह्नाभिमुखं वलनं देयम् । उत्तरत्व उत्तरचिह्नात् पश्चिमाभिमुखं देयम् । यदि दक्षिणं वलनमुत्तरविक्षेपस्तदोत्तरचिह्नात् पूर्वाभिमुखम् । यद्युत्तरं वलनं दक्षिणविक्षेपस्तदा दक्षिणचिह्नात् पन्मूर्वाभिमुखं देयमिति । भास्करार्चायैस्त्वेतदुक्तफलितं लाघवेन दक्षिणोत्तरवलनं क्रमेण सव्यायसव्यं देयमित्युक्तम् । यात्रोपपत्तिः । प्रथमश्लोकोपपत्तिः स्पार्शिकमौक्षिकशरदानोपपत्तापवुक्ता । ग्राह्यबिम्बकेन्द्राद्विक्षेपान्तरेण ग्राहकबिम्बकेन्द्रं भवति । शरस्य कदम्बाभिमुखत्वेन केन्द्रात् कदम्बाभिमुखशरदानार्थं कदम्बज्ञानंवलनाश्रितवृत्त आवश्यकमतो वलनान्तरेण खदिग्भ्यः क्रान्तिवृत्तदिशां सत्त्वादुत्तरदक्षिणदिग्भ्यां मध्यवलनान्तरेण क्रान्ति\textendash


\newpage

\hspace{3.5cm} गूढार्थप्रकाशकेन सहितः~। \hfill २०१
\vspace{1cm}


\noindent वृत्तयाम्योत्तररूथकदम्बौ दक्षिणोत्तरत इति पूर्वपश्चिमानुरोधेनैतद्दानं युक्ततरम् । यद्यपि चन्द्रग्रहणे शरस्य विपरीतरदिक्त्वात् तच्छरदिग्ग्रहणेन सूर्यचन्द्रयोर्मध्यवलनदानमेकदिक्त्वेपश्चिमचिह्नाभिमुखं भिन्नदिक्त्वे पूर्वाभिमुखमित्येकोक्तिलाथवं तथापि सूर्यचन्द्रयोर्ग्रहणभेदादेकोक्तौ मन्दबुद्धीनां भ्रमसम्भवस्तद्वारणार्थं पृथगिवोक्तिः कृता । स्वतन्त्रेच्छस्य नियोगानर्हत्वाच्च~॥~९~॥\\
\noindent अथ मध्यग्रहणं श्लोकाभ्यां परिलेखे दर्शयति\textendash

%\vspace{2mm}
%\setlength{\parindent}{5em}
\begin{quote}
{\ssi वलनाग्रात् पुनः सूत्रं मध्यबिन्दुं प्रवेशयेत्~।\\
 मध्यसूत्रेण विक्षेपं वलनाभिमुखं नयेत्~॥~१०~॥ 

 विक्षेपाग्राल्लिखेद्वृत्तं ग्राहकौर्धेन तेन यत्~।\\
 ग्राह्यवृत्तं समाक्रान्तं तद्ग्रस्तं तमसा भवेत्~॥~११~॥ }
%\vspace{2mm}
\end{quote}

 वलनाद्यान्मध्यकालिकवलनाग्रात् पूर्वश्लोकोक्तात् सूत्रं रेखांमध्यबिन्दुं नृत्तमध्यचिह्नं प्रति पुनर्वारान्तरं पूर्वं स्यार्शिकमौक्षिकवलनाग्राभ्यां सूत्ररचना तथैवेत्यर्थः । प्रवेशयेत् । गणकः प्रविष्टं कुर्यात् । मध्यसूत्रेणानेन मध्यकालिकविक्षेपं मध्यवलनाग्राभिमुखं नयेत् । वृत्तमध्यबिन्दोरित्यर्थसिद्धम् । तथा च वृत्तमध्यान्मध्यवलनाग्रसूत्रे विक्षेपाङ्गुलानि गणयित्वा तदग्रे विक्षेपायचिह्नं कुर्यादिव्यर्थः । अस्माद्विक्षेपाग्राद्ग्राहकबिम्बमानार्धेन वृत्तं गणको लिखेत् । तेन वृत्तेन यद्यन्मितं ग्राह्यवृत्तं समाक्रान्तं व्याप्तम् । यद्ग्राह्यवृत्तविभागरूपं तमसान्धकाररूपेण छादकेन ग्रस्तमाच्छादितं स्यात् तन्मितं विभागं मध्यादिना


{\tiny{2 C}}

\newpage

\noindent २०१ \hspace{4cm} सूर्यसिद्धान्तः
\vspace{1cm}


\noindent लिप्तं कुर्यादित्यर्थः । अत्रोपपत्तिः । वृत्ते मध्यसूत्रं कदम्बाभिमुखं तत्र ग्राह्यकेन्द्राच्छरान्तरेण ग्राहककेन्द्रं तस्माह्वाहकार्धेन वृत्तं ग्राहकबिम्बवृत्तं तेन ग्राह्यवृत्तं यावदाक्रान्तं तावन्मध्यकाले ग्रस्तमिति तद्भागस्य कृत्स्नत्वेनाकाशे दर्शनात् तमसा ग्रस्तमित्युक्तम्~॥~९९~॥\\
\noindent ननु पूर्वकपाले ग्रहणयोः सम्भवे सर्वमुक्तमुपपन्नम् । पश्चिमकपाले ग्रहणसम्भवे परिलेखोक्तं वैपरीत्येन भवति । तथाहि । यस्यां दिशि परिलेखे स्पर्शो मोक्षो वापरकपाले तस्य पश्चिमाभिमुखत्वेव दर्शने दिग्वैपरोत्यं प्रत्यक्षमित्यत आह\textendash 

%\vspace{2mm}

\begin{quote}
{\ssi छेद्यकं लिखता भूमौ फलके वा विपश्चिता~।\\
 विपर्ययो दिशां कार्यः पूर्वापरकपालयोः~॥~१२~॥ }
 \end{quote}
%\vspace{2mm}


 भूमौ फलके काष्ठपट्टिकायामित्यर्थः । वा विकल्पे भूमौ लिखितस्येतस्ततो नयनासम्भवात् फलक इत्युक्तिः । छेद्यकं प्रागुक्तं लिखता गणकेन विपश्चिता तत्त्वज्ञेन दिशां पूर्वादिदिशां पूर्वापरकपालयोर्विपर्ययो व्यत्यासः कार्यः । यथा पूर्वकपाले सव्यक्रमेण पूर्वादिलेखनं तथापरकपाले सव्यक्रमेण पूर्वादि लेखनं न कार्यम् । किन्तु पश्चिमस्थाने पूर्वा पूर्वस्थाने पश्चिमा । उत्तरदक्षिणदिग्भागे क्रमेणोत्तरदक्षिणे लेख्ये इत्यर्थः । तेन पश्चिमकपाले ग्रहणसम्भवेऽपि परिलेखोक्तं सम्भवत्येवेति भावः । अत्रोपपत्तिः । दिग्वैपरीत्यं भवतीति पूर्वमेव वैपरीत्येन दिशां लेखने परिलेखो यथास्थितो भवतीत्युक्तम् । भास्कराचार्यैस्तु\textendash


\newpage

\hspace{3.5cm} गूढार्थप्रकाशकेन सहितः~। \hfill २०३ 
\vspace{1cm}


\noindent नैतदुक्तम् । परिलेखेनामक्यां दिश्यमुकं भवतीति ज्ञानस्यावश्यकत्वेन तस्य तत्राभावात् । न हि यथाकाशे तथा दर्शनमपेक्षितम् । भूमौ फलके वाकाशादीनां वास्तवानामभावात् । अत एव किञ्चिन्न्यूनसादृश्येन वृष्टान्तत्वमिति ध्येयम्~॥~१२~॥\\
\noindent अथामादश्यग्रहणमाह\textendash

%\vspace{2mm}

\begin{quote}
{\ssi स्वच्छत्वाद्द्वादशांशोऽपि ग्रस्तश्चन्द्रस्य दृश्यते~।\\
 लिप्तात्रयमपि ग्रस्तं तीक्ष्णत्वान्न विवस्वतः~॥~१३~॥ }
 \end{quote}
%\vspace{2mm}


 चन्द्रबिम्बस्य द्वादशांशो ग्रस्त आच्छादितः । अपिशब्दादाच्छादनेन तेजोहीनतया दृश्यतासम्भावनायामित्यर्थः । न दृश्यते । हेतुमाह\textendash स्वच्छत्वादिति~। तदतिरिक्तसम्पकर्णदृश्यभागस्य स्वच्छत्वाज्ज्योत्स्नावत्त्वात् । तथा च तज्ज्योत्स्नाधिक्येन ग्रस्तोऽप्यल्पोंशः स्वाकारेण न दृश्यते ज्योत्स्नावत्त्वेन दूरतया भासते । सूर्यस्य लिप्तात्रयं ग्रस्तमपि न दृश्यते । अत्र हेतुमाह\textendash तीक्ष्णत्वादिति~। सूर्यस्य तेजस्तैक्ष्ण्याल्लोकनयनप्रतिघातार्हत्वाच्चेत्यर्थः । वृद्धवसिष्ठेन तु \textendash

%\vspace{2mm}


\begin{quote}
{\qt ग्रस्तं शशाङ्कस्य कलाद्वयं चेत्  कंलात्ररां भानमतो' न लक्ष्यम्~। \\
 तत् किञ्चिदूनं ह्युदयास्तकाले  लक्ष्यं यतस्तौ करगम्फहीनौ~॥ }
 \end{quote}
%\vspace{2mm}


इत्युक्तम् । अत उदयास्तकाले उक्तमदृश्यं दृश्यमिति ध्येयम् ॥१३॥\\
\noindent अथेष्टग्रासपरिलेखार्थं ग्राहकमार्गज्ञानं श्लोकत्रयेणाह\textendash

 {\tiny{2 C 2}}
 
 \newpage
 

\noindent २०४ \hspace{4cm} सूर्यसिद्धान्तः
\vspace{1cm}


\begin{quote}
{\ssi स्वसञ्चितास्त्रयः कार्या विक्षेपाग्रेषु बिन्दवः~।\\
 तत्र प्राङ्मध्ययोर्मध्ये तथा मौक्षिकमध्ययोः~॥~१४~॥

लिखेन्मत्स्यौ तयोर्मध्यान्मुखपुच्छविनिःसृतम् ~।\\
प्रसार्य सूत्रद्वितयं तयोर्यत्र युतिर्भवेत् ॥~१५ ॥

तत्र* सूत्रेण विलिखेच्चापं विन्दुत्रयस्पृशा~।\\
स पन्था ग्राहकस्योक्तो येनासौ सम्प्रयास्यति~॥~१६~॥ }
\end{quote}
%\vspace{2mm}


 विक्षेपाग्रेषु स्पार्शिकमौक्षिकमाध्यविक्षेपाणां पूर्वं स्वस्वस्थाने स्पर्शमोक्षमध्यग्रहणज्ञानार्थं दत्तानामग्रिमभागेषु स्वसञ्ज्ञया सङ्केतिता बिन्दवस्त्रयः कार्याः स्पर्शशराग्रे स्पर्शचिह्नाङ्कितो बिन्दुर्मोक्षशराग्रे मोक्षचिह्नाङ्कितो बिन्दुर्मध्यशराग्रे मध्यचिह्नाङ्कितो बिन्दुरिति त्रयो बिन्दवो गणकेन स्थाप्याः । तत्रोपस्थितबिन्दुत्रयमध्ये प्राङ्मध्ययोः स्पर्शमध्यबिन्द्वौर्मध्येऽन्तराले मौक्षिकमध्ययोस्तत्सञ्ज्ञयोर्बिन्द्वोस्तथान्तराले प्रत्येकं मत्स्यं लिखेदित्यन्यतरद्वये गणको मत्स्यौ लिखेत् । तयोर्मत्स्ययोर्मध्याद्गर्भान्मुखपुच्छाभ्यां विनिःसृतं निष्काशितं प्रत्येकं सूत्रमितिसूत्रद्वितयम् । प्रसार्याग्रेऽपि स्वमार्गेण निःसार्य तयोः स्वस्वमार्गप्रसारितसूत्रयोर्यत्र प्रदेशे युतिर्योगः स्यात् तत्र प्रदेशे केन्द्रं प्रकल्प्य सूत्रेण बिन्दुत्रयस्य स्पृशा प्रकल्पितकेन्द्रबिन्दुत्रयान्यतमबिन्द्वन्तरसूत्रेण व्यासार्धरूपेणेत्यर्थः । चापं वृत्तैकदेशरूपं धनुर्बिन्दुत्रयस्पृष्टं लिखेत् । गणकः कुर्यादित्यर्थः । स चापा\textendash


\noindent \rule{\linewidth}{.5pt}

\begin{center}
 * तेन इति पाठान्तरम् ।
\end{center}

\newpage

 \hspace{3cm} गूढार्थप्रकाशकेन सहितः~। \hfill २०५
\vspace{1cm}


\noindent त्मकौ वृत्तैकदेशो ग्राहकस्य पन्था मार्गः कथितः । येन मार्गेणासौ ग्राहकः सम्प्रयास्यति ग्राह्यबिम्बाच्छादनार्थं गमिष्यति । परिलेखस्य ग्रहणकालपूर्वकालावश्यम्भावित्वात् । अत्रोपपत्तिः । इष्टेऽह्नि मध्ये प्राक्पश्चादिति त्रिप्रश्नाधिकारान्तर्गतश्लोकोपपत्तिः प्राक् प्रतिपादिता ~॥~९६~॥\\
\noindent अथेष्टग्रासपरिलेखं श्लोकत्रयेणाह\textendash

%\vspace{2mm}


\begin{quote}
{\ssi ग्राह्यग्राहकयोगार्धात् प्रोज्झ्येष्टग्रासमागतम्~।\\
 अवशिष्ष्टाङ्गुलसमां शलाकां मध्यबिन्दुतः~॥~१७~॥

तयोर्मार्गोन्मुखीं दद्याद्ग्रासतः प्राग्ग्रहाश्रिताम्~।\\
विमुञ्चतो मोक्षदिशि ग्राहकाध्वानमेव सा ॥~१८~॥

स्पृशेद्यत्र ततो वृत्तं ग्राहकार्धेन संलिखेत्~।\\
तेन ग्राह्याद्यदाक्रान्तं तत्* तमोग्रस्तमादिशेत् ॥~१९~॥}
\end{quote}
%\vspace{2mm}


 मानैक्यखण्डादिष्टकालिकाभीष्टग्रासमागतं चन्द्रग्रहणाधिकारोक्तप्रकारावगतं त्यक्त्वावशिष्टे यान्यङ्गुलानि तत्प्रमाणां शलाकां यष्टिं मध्यबिन्दुतो वृत्तत्रयमध्यकेन्द्रबिन्दोः सकाशात् तयोः स्पर्शमोक्षविक्षेपाग्रयोर्मार्गोन्मुखीं सम्बद्धमार्गचापरेखाभिमुखीं मार्गरेखासक्तां दद्यात् । कथमित्यत आह\textendash ग्रासत इति~। मध्यग्रासतः प्राक् पूर्वकाले ग्रहाश्रितां ग्रहस्पर्शस्तच्छराग्रसम्बन्धिमार्गचापरेखासक्तां शलाकाम् । विमुञ्चतो मुच्यमानान्तर्गताभीष्टग्रासस्य शलाकाम् । मोक्षदिशि । मोक्षविक्षेपाग्रसम्बन्धि\textendash


\noindent\rule{\linewidth}{.5pt}

\begin{center}
 *तदा इति पाठान्तरम् ।
\end{center}

\newpage

\noindent २०६ \hspace{4cm} सूर्यसिद्धान्तः 
\vspace{1cm}



\noindent मार्गचापरेखायां सक्तां दद्यात् । सा शलाका ग्रहाध्वानं गोहकमार्गचापरेखां यत्र यस्मिन् भागे स्यृशेत् संलग्ना स्यात् ।ततः स्थानात् । एवकारस्तदतिरिक्तव्यवच्छेदार्थः । ग्राहकमानार्धेन व्यासार्धेन वृत्तं संलिखेत् । सम्यक्प्रकारेण कुर्यात् । तेनवृत्तेन ग्राह्याद्ग्राह्यवृत्ताद्यद्यन्मितमेकदेशरूपं वृत्तमाक्रान्तं व्याप्तम् । तत् तन्मितग्राह्यवृत्तांशं तमोग्रस्तं छादकाच्छादितमभीष्टकाल आदिशेत् कथयेत् । अत्रोपपत्तिः । इष्टग्रासोनं मानैक्यखण्डं कर्णः । स तु ग्राह्यग्राहककेन्द्रान्तररूपः । अतोऽयं ग्राह्यकेन्द्रात् पूर्वज्ञातग्राहकमार्गरेखायां यत्र लग्नस्तवाभीष्टसमये ग्राहककेन्द्रम् । तस्माद्ग्राहकवृत्तेन ग्राह्यवृत्तं यदाक्रान्तं तत्काले ग्रास इति सुगमा~॥~९९~॥\\
\noindent अथ श्लोकाभ्यां निमीलनपरिलेखमाह\textendash 

%\vspace{2mm}

\begin{quote}
{\ssi मानान्तरार्धेन मितां शलाकां ग्रासदिङ्मुखीम्~।\\
 निमीलनाख्यां दद्यात् सा तन्मार्गे यत्र संस्पृशेत्~॥~२०~॥

ततो ग्राहकखण्डेन प्राग्वन्मण्डलमालिखेत्~।\\
तद्ग्राह्यमण्डलयुतिर्यत्र तत्र निमीलनम्~॥~२१~॥ }
\end{quote}
%\vspace{2mm}


 ग्राह्यग्राहकबिम्बमानयोरन्तरस्यार्धं तेन परिमिता शलाकां निमीलनसञ्ज्ञां ग्रासदिङ्मुखीं स्पार्शिकशराग्रविभागाभिमुखीं मध्यबिन्दोः सकाशाद्दद्यात् । सा निमीलनसञ्ज्ञा शलाका तन्मार्गं स्पार्शिकग्राहकभार्गं चापरेखाकारं यस्मिन् प्रदेशे संलग्ना स्यात् तत्स्थानाद्ग्राहकमानार्धेन प्राग्वत् मध्याभीष्टग्रासज्ञानार्थं


\newpage

\hspace{3cm} गूढार्थप्रकाशकेन सहितः~। \hfill २०७
\vspace{1cm}


\noindent यथा मद्वृत्तं कृतं तथेत्यर्थः । वृत्तं कुर्यात् । तद्ग्राह्यमण्डलयुतिर्लिखितवृत्तग्राह्यवृत्तयोः संयोगो यत्र यस्यां दिशि तत्र तस्यां दिशि निमीलनं ग्राह्यबिम्बस्य निमज्जनं स्थात् । अथोपपत्तिः । सम्मीलनकाले ग्राह्यग्राहककेच्छान्तरं मानार्धान्तरमितं कर्णः । अन्यथा तदनुपपत्तेः । स ग्राह्यकेन्द्रात् स्पर्शमार्गे यत्र लग्नस्तत्र ग्राहककेन्द्रम् । तस्माद्ग्राहकवृत्तं ग्राह्यमण्डलं यत्र स्पृशति तत्र निमीलनं स्पष्टम्~॥~२१~॥\\
\noindent अथोन्मीलनपरिलेखमाह\textendash

%\vspace{2mm}

\begin{quote}
{\ssi एवमुन्मीलने मोक्षदिङ्मुखो सम्प्रसारयेत्~।\\
 विलिखेन्मण्डलं प्राग्वदुन्मलिनमथोक्तवत्~॥~२२~॥ }
 \end{quote}
%\vspace{2mm}


 उन्मीलने उन्मीलनज्ञानार्थमित्यर्थः । एवं बिम्बमानान्तरार्धनितां शलाकां मोक्षदिभुखीं भौक्षिकशराग्रविभागाभिमुखीं मध्यबिन्दोः सकाशात् समासारथेद्दद्यादित्यर्थः । प्राग्वत् सम्मीलनार्थं दत्तशलाकास्पार्शिकमार्गयोगस्थानाद्ग्राहकार्धेन वृत्तं कृतं तथेत्यर्थः । मौक्षिकमार्गदत्तशलाकायोगस्थानाद्ग्राहकवृत्तं कुर्यात् । अथानन्तरमुक्तवद्वाहकग्राह्यवृत्तयोगो यस्यां तस्यां दिशीत्यर्थः । उन्मीलनं ग्राह्यबिम्बस्योन्मज्जनं स्यात् । अत्रोपपत्तिः । उन्मीलनेऽपि ग्राह्यग्राहककेन्द्रान्तरं मानार्धन्तरमितं कर्णः । परमपरमोक्षदिशीति युक्तिस्तुल्या~॥~२२~॥\\
 \noindent अथ ग्रहणेचन्द्रस्य वर्णानाह\textendash

%\vspace{2mm}

\begin{quote}
{\ssi अर्धादूने सधूम्रं स्यात् कृष्णमर्धाधिकं भवेत्~।\\
 विमुञ्चतः कृष्णताम्रं कपिलं सकलग्रहे ~॥~२३~॥}
\end{quote}
\newpage

\noindent २०८ \hspace{4cm} सूर्यसिद्धान्तः 
\vspace{1cm}


 अर्धादर्धबिम्बादूने न्यूने ग्रस्ते सति सधूम्रं ग्रासीयबिम्बं धूम्रवर्णं स्यात् । अर्धाधिकं ग्रस्तविम्बं कृष्णं स्यात् । विमुञ्चत एतदनन्तरं ग्रस्तमधिकमपि मुक्त्युन्मुखमिति मोक्षारम्भोन्मुखस्य पादोनबिम्बाधिकग्रस्तस्यासम्पूर्णस्येत्यर्थः । कृष्णताम्रं श्यामरक्तमिश्रवर्णः । सम्पूर्णग्रहणे कपिलं पिशङ्गवर्णं बिम्बं स्यात् । अत्र भूभायास्तेजोऽभावतया चन्द्राच्छादकत्वादेते वर्णाः सम्भवन्ति । सूर्यस्य तु चन्द्रो जलगोलरूप आच्छादकः स दर्शान्तदिवसेऽस्मदृश्यार्धे सदा कृष्ण एवेति कृष्ङ एव सूर्यस्य ग्रस्तोंऽशः सर्वदा ।अत एवाविकृतत्वाद्भगवता वार्णो नोक्तः~॥~९३~॥\\
 \noindent अथोक्तच्छेद्यकस्य गाप्यत्वमाह\textendash

%\vspace{2mm}

\begin{quote}
{\ssi रहस्यमेतद्देवानां न देयं यस्य कस्यचित्~।\\
सुपरीक्षितशिष्याय *देयं वत्सरवासिने~॥~२४~॥ }
\end{quote}
%\vspace{2mm}


 एतद्ग्रहणच्छेद्यकं देवानां गोप्यं वस्तु । त्रस्य कस्यचिद्यस्मै कस्मैचिदपरीक्षिताय न देयम् । कस्मैचिद्देयन्नित्यर्थागतं विवृणोति\textendash सुपरीक्षितशिष्यायेति~। सुपरीक्षितमित्यत्र हेतुगर्भं विशैषणमाह\textendash वत्सयवासिन इति~। वर्षपर्यन्तं तत्साङ्गत्या तस्य तत्त्वतया ज्ञानं भवत्येवेति भावः~॥~२४~॥\\
 \noindent अथाग्रिमग्रन्यस्यासङ्गतित्वनिरासार्धमधिकारसमाप्तिं फक्किक्याह \textendash 


\begin{center}
 इति छेद्यकाध्यायः~॥
\end{center}

\noindent \rule{\linewidth}{.5pt}

\begin{center}
 * दातव्यं ज्ञानमुत्तमम् इति पाठान्तरम्~।
\end{center}


\newpage


\hspace{3cm} गूढार्थप्रकाशकेन सहितः~। \hfill २०९
\vspace{1cm}


 ग्रहणभेदज्ञापकपरिलेखप्रतिपादनं परिपूर्तिमाप्तमित्यर्थः । इदं दशभेदग्रहगणितमित्युक्त्या गणितक्रियाभावाद्ग्रहणाधिकारान्तर्गतं नाधिकारान्तरम् । अत एवाधिकार इत्युपेक्ष्याध्याय इत्युक्तम् । 

%\vspace{2mm}

 \begin{quote}
 {\ssi रङ्गनाथेन रचिते सूर्यसिद्धान्तटिप्पणे~।\\
 छेद्यकं ग्रहणान्तं तु पूर्णं गूढप्रकाशके~॥ }
 \end{quote}
%\vspace{2mm}
\begin{center}
\noindent%\rule{7em}{.5pt}
इति श्रीसकलगणकसार्वभौमबल्लालदैवज्ञात्मजरङ्गनाथगणकविरचिते गूढार्थप्रकाशके छेद्यकाध्यायः सम्पूर्णः ॥
\end{center}


\noindent अथ युत्याभासग्रहणनिरूपणेन संस्मृततयारन्धो यन्नयुयधिकारो व्याख्यायते~।  तत्र युतिभेदानाह\textendash
%\vspace{2mm}

\begin{quote}
{\ssi ताराग्रहाणामन्योन्यं स्यातां युद्धसमागमौ~।\\
 समागमः शशाङ्केन सूर्येणास्तमनं सह~॥~१~॥ }
 \end{quote}
%\vspace{2mm}


 ताराग्रहाणां भौमादिपञ्चग्रहाणां परस्परं थोगे चुद्धसमागमौ वक्ष्यमाणलक्षणभिन्नौ स्तः । चन्द्रेण सह पञ्चतारान्यतमस्य योगः क्षमागमसञ्ज्ञः । सूर्येण सह पञ्चताराणामन्यतमस्य चन्द्रस्य वा योगस्तदस्तमनं पूर्णास्तङ्गतत्वम् । न त्वस्तमाम्नम् । युत्यभावे प्रागपरकाले तस्य सत्त्वात्~ ॥~९~॥\\
\noindent अथ युतेर्गतैय्यत्वं सार्धश्लोकेनाह \textendash

%\vspace{2mm}

 \begin{quote}
 {\ssi शीघ्रे मन्दाधिकेऽतीतः संयोगो भवितान्यथा~।\\
 द्वयोः प्राग्याआयिनोरेवं वक्रिणोस्तु विपर्ययात्~॥~२~॥

प्राग्यायित्यधिकेऽतीतो वक्रिण्येष्यः समागमः~।}
\end{quote}
{\tiny{2 D}}

\newpage

\noindent २१० \hspace{4cm} सूर्यसिद्धान्तः 
\vspace{1cm}


 ययोग्रहयोर्योगोऽभिमतस्तयोर्ग्रहयोर्मध्ये यः शीघ्रगतिर्ग्रहस्तस्मिन् मन्दाधिके मन्दगतिग्रहादधिके सति तयोः मंयोगोयुतिस्ञ्ज्ञो गतः पूर्वं जात इत्यर्थः । अन्यथा मन्दगतिग्रहे शीघ्रगतिग्रहादधिके सतीत्यर्थः । तयोर्योगो भविता एष्यः । एवमुक्तं गतैष्यत्वम् । द्वयोर्ग्रहयोः प्राग्यायिनोः पूर्वगतिकयोर्भवति । वक्रिणोर्वक्रगतिग्रहयोर्विपर्ययादुक्तवैपरोत्यात् । तुकाराङ्गतैष्यो योगो भवति । शीघ्रगतिग्रहे मन्दगतिग्रहादधिक एष्यः संयोगो मन्दगतिग्रहे शीघ्रगतिग्रहादधिके गतः संयोग इत्यर्थः । अथैकस्य वक्रत्व आह\textendash प्राग्यायिनीति~। द्वयोर्मध्य एकतरस्मिन् वक्रिणि सति तदो वक्रगतिग्रहात् पर्वगतिग्रद्देऽधिके सति गतो योगः । यदा तु पूर्वगतिग्रहाद्वक्रगतिग्रहेऽधिके सति समागमो योग एष्यः स्यात् । अत्रोपपत्तिः । पूर्वगत्योर्ग्रहयोर्मध्ये शीघ्रगस्याधिकत्वेऽग्रे योगासम्भवात् पूर्वयोगो जातः । मन्दगस्याधिकत्वे शीघ्रगस्य न्यूनत्वादग्रे योगो भविष्यति । वक्रिणोस्तु शीघ्रगस्याधिकत्वेऽग्रे तन्न्यूनत्वेन योगसम्भवादेष्यो योगो मन्दगस्याधिकत्वे शीघ्रगस्योत्तरोत्तरं न्यूनत्वसम्भवेनाग्रे योगासग्भवाङ्गतो योगः । अथ वक्रगतिग्रहात् पूर्वगतिग्रहेऽधिक उत्तरोत्तरंयोगासम्भवाद्गतो योगः । पूर्वगतिग्रहाद्वक्रगतिग्रहेऽधिके वक्रगतिग्रहस्य न्यूनत्वेनाग्रे योगसम्भवादेष्यः संयोग इति~॥~९~॥\\
\noindent अथ युतिकाले तुल्यग्रहयोरानयनं युतिकालस्य गतैष्यदिनाद्यानयनं च सार्धश्लोकत्रयेणाह\textendash


\newpage

\hspace{3cm} गूढार्थप्रकाशकेन सहितः~। \hfill २११
\vspace{1cm}


 \begin{quote}
{\ssi ग्रहान्तरकलाः स्वस्वभुक्तिलिप्तासमाहताः ~॥~३~॥
 
भुक्त्यन्तरेण विभजेदनुलोमविलोमयोः~।\\
द्वयोर्वक्रिण्यथैकस्मिन् भुक्तियोगेन भाजयेत्~॥~४~॥

लब्धं लिप्तादिकं शोध्यं गते देयं भविष्यति~।\\
विपर्ययाद्वक्रगत्योरेकस्मिंस्तु धनव्ययौ~॥~५~॥

समलिप्तौ भवेतां तौ ग्रहौ भगणसंस्थितौ~।\\
विवरं तद्वदुद्धृत्य दिनादि फलमिष्यते~॥~६~॥ }
\end{quote}
%\vspace{2mm}


 युतिसम्बन्धिनोर्ग्रहयोरभीष्टैककालिकयोरन्तरस्य कलाः पृथक् स्वस्वगतिकलाभिर्गणिताः कर्म द्वयोर्ग्रहयोरनुलोमविलोमयोर्मार्गगयोर्वक्रगयोर्वेत्यर्थः । स्फुटगत्यन्तरेण गणको भजेत्।विशेषमाह । वक्रिणीति । अथानन्तरं द्वयोर्मध्य एकतरे वक्रिणि सति तयोर्गतियोगेन भजेत् । फलं कलादि स्वं स्वं गते योगे सति ग्रहयोर्मार्गगयो शोध्यं भविष्यति । एष्ये योगे सति तयोर्देयं योज्यम् । द्वयोर्वक्रगत्योः स्वं स्वं फलं विपर्ययादुक्तवैपरीत्यात् कार्यम् । गते योगे योज्यम् । एष्ययोगे हीनमित्यर्थः । द्वयोर्मध्य एकतरे तुकाराद्वक्रिणि सति तयोर्ग्रहयोर्वक्रमार्गयोः स्वस्वकलात्मकफलाङ्कौ धनव्ययौ युतहीनौ कार्यौ। यथा हि । गतयोगे मार्गगग्रहे स्वफलं हीनं वक्रिणि ग्रहे योज्यम् । एष्ययोगे वक्रग्रहे शोध्यम् । मार्गग्रहे योज्यमिति । एवं कृते तौ युतिसम्बन्धिनौ ग्रहौ भगणसंस्थौ भगणे राश्यधिष्ठितचक्रे संस्थितिर्ययोस्तौ राश्याद्यात्मकौ समलिप्तौ समकलौ स्तः।


{\tiny{2 D 2}}

\newpage


\noindent २१२ \hspace{4cm} सूर्यसिद्धान्तः 
\vspace{1cm}


 
\noindent लिप्तापदस्य भगणावयवोपलक्षणत्वेन समौ स्त इत्यर्थः । अथ युतिकालज्ञानमाह\textendash विवरमिति~। अभीष्टकालिकयोर्युतिसम्बन्धिनोर्ग्रहयोरन्तरं कलात्मकं तद्वत् समकलोपयुक्तफस्तज्ञानार्थं यथा गतिगणितमन्तरं गतियोगेन गत्यन्तरेण भक्तं तथेत्यर्थः । तेन हरेण भक्त्वा फलं दिनादिकं गतैष्ययुतिवशादभीष्टकालाद्गतैष्यमुच्यते । तत्समये तद्यशिकाले तौ ग्रहौ समौस्त इत्यर्थः । अत्रोपपत्तिः । गत्यन्तरेण गतिकलास्तदा ग्रहान्तरकलाभिः का इति फले गतयुतौ ग्रहयोः शोध्ये । एष्ययुतौ योज्ये । द्वयोर्वक्रत्वे गत्यन्तरभक्तफले गतद्युतौ ग्रहयौर्योज्ये । एष्ययुतौ शोध्ये । वक्रग्रहस्योत्तरोत्तरं न्यूननत्वात् । अथैको वक्रो तदा तयोरन्तरं प्रत्यहं गतियोगेनोपचितम् । अतो गतियोगहरेणागतं फलं गतयोभे मार्गगग्रहे हीनं पूर्वं तस्य न्ययत्वात् ।वक्रग्रहे योज्यम् । पूर्वं तस्याधिकत्वात् । एष्ययोगमार्गगग्रह योज्यम् । उत्तरोत्तरमधिकत्वात् । वक्रग्रहे शोध्यम् । तस्याग्रे न्यूनत्वात् । गतियोगेन गत्यन्तरेण वा दिनमेकं लभ्यते तदान्तरकलाभिः किमित्यनुपातेन गतैष्यदिनाद्यम्~॥~६~॥\\
\noindent अथ दृक्कर्मार्थमुपकरणानि साध्यानीत्याह \textendash 

%\vspace{2mm}

\begin{quote}
{\ssi कृत्वा दिनक्षपामानं तथा विक्षेपलिप्तिकाः~।\\
 नतोन्नतं साधयित्वा स्वकाल्लग्नवशात् तयोः~॥~७~॥ }
 \end{quote}
%\vspace{2mm}

 तयोः समयोर्ग्रहयोर्दिनक्षपामानं प्रत्येकं दिनमानं रात्रिमानं प्रसाध्य विक्षेपकलाः । तथा प्रसाध्येत्यर्थः । अत्र भगवता
\textendash

\newpage

\hspace{3cm} गूढार्थप्रकाशकेन सहितः~। \hfill २१३
\vspace{1cm}


\noindent विक्षेपकलाः प्रसाध्येत्यस्य दिनरात्रिमानं प्रसाध्येत्येतदनन्तरमुक्तेर्दिनरात्रिमानं स्पष्टक्रान्तिजचरेण न साध्यम् । किन्तु समग्रहीयशरासंस्कृतकेवलक्रान्तिजचरेण साध्यमिति सूचितम् । समशरग्रहयोः प्रत्येकं नतकालमुन्नतकास्तं प्रसाध्य । अत्र समुच्चयार्थकं तथेत्यन्वेति । एतदर्थमेव दिनरात्रिमानं प्रसाध्येति पूर्वमुक्तम् । समनन्तरोक्तं दृक्कर्म कार्यमिति वाक्यशेषः । ननु नतोन्नतं कथं साध्यं ग्रहोदयाज्ञानात् तदवधिकालमामज्ञानाभावात् । न हि ग्रहस्य दिनरात्रिगतकालज्ञानं विनापि केवलदिनरात्रिमानाभ्यां तत्सिद्धिरत आह\textendash स्वकाल्लग्नवशादिति~। यस्मिन् काले समौ ग्रहौ जातौ तात्कालिकलग्नं पूर्वोक्तप्रकारावग तदवशात् तद्ग्रहणादित्थर्थः । स्वकात् समग्रहात्प्रत्येकमुन्नतनत कालो साध्यावित्यर्थः । एतदुक्तं भवति । युतिकालिकलग्नमधिकसञ्ज्ञं प्रकल्प्य समग्रहं न्यूनसञ्ज्ञं प्रकल्प्य। 

%\vspace{2mm}

\begin{quote}
{\qt भोग्यासूनूनकस्याथ भुक्तासूनधिकस्य च~।\\
 सम्पीड्यान्तरलग्नासूनेवं स्यात् कालसाधनम्~॥ }
 \end{quote}
%\vspace{2mm}


इति त्रिप्रश्नाधिकारोक्त्या ग्रहस्य दिनगतं रात्रिगतं प्रसाध्य दिने दिनगतशेषयो रात्रौ रात्रिगतशेषयोर्यदल्पं तदुन्नतम् । तेनोनं दिनार्धं रात्र्यर्धं वा ग्रहस्य नतम् । दिनक्षपामानं नतोन्नतमित्येकवचनेन समग्रहयोरभिन्नं दिनमानं रात्रिमानं नतमुन्नतं चेति सूचनादपि नोदय लग्नलग्नाभ्यामन्तरकालः प्रत्येकभिन्नः साध्यः । न वा स्पष्टक्रान्तिजचरेण दिनरात्रिमाने प्रत्येकं पूर्वमुदयलग्नस्यैवासिद्धेरिति स्फुटीकृतम् । अत्रोपपत्तिः । ता \textendash


\newpage

\noindent २१४ \hspace{4cm} सूर्यसिद्धान्तः
\vspace{1cm}


\noindent त्कालिकार्कलग्नाभ्यां थथा सूर्यस्योदयगतकालस्तथा तात्कालिकग्रहलग्नाभ्यां ग्रहोदयशतकालः सिद्ध्यति । यद्यपि सूर्यस्य क्रान्तिवृत्तस्थत्वात् सूर्यस्य युक्तः कालः । ग्रहस्य तु क्रान्तिवृत्त स्थत्वानियमादुक्तरीत्या गतकालस्य क्रान्तिवृत्तस्थग्रहचिह्नीयत्वेऽपि ग्रहबिम्बीयत्वाभावादयुक्तत्वम् । अत एव वक्ष्यमाणदृक्कर्म संस्कृतग्रहादानीतकालो ग्रहविम्बीयस्तथापि वक्ष्यमाणदृक्कर्मार्थं ग्रहचिह्नीयस्यैवापेक्षितत्वान्न क्षतिः~॥~ ७~॥ \\
\noindent अथाक्षदृक्कर्मतत्संस्कारं च ग्रहस्य श्लोकाभ्यामाह \textendash

%\vspace{2mm}

\begin{quote}
{\ssi विषुवच्छाययाभ्यस्ताद्विक्षेपाद्द्वादशोद्धृतात्~।\\
 फलं स्वुनतनाडीघ्नं स्वदिनार्धविभाजितम् ॥~८~॥

लब्धं प्राच्यामृणं सैम्याद्विक्षेपात् पश्चिमे धनम्~।\\
दक्षिणे प्राक्कपाले स्वं पश्चिम तु तथा क्षयः ॥~९~॥ }
\end{quote}
%\vspace{2mm}


 अक्षभया गुणिताद्ग्रहविक्षेपादानीताद्द्वादशभक्ताद्यल्लब्धं तत् स्वनतनाडीघ्नं विक्षेपसम्बन्धिग्रहस्य नतघटीभिर्गुणितं तस्यैव दिनार्धेन भक्तं रात्रौ रात्र्यर्धेनेत्यर्थसिद्धम् । अत्र समग्रहयोः पूर्वोक्तप्रकारेण दिनमाननतयोरभिन्नत्वात् स्वशब्द उभयत्रानावश्यकोऽपि युतिव्यतिरिनादृग्ग्रहाणां प्रयोजनतया साधनवैयधिकरण्यव्यावृत्त्यर्थं स्वपदं भगवता दत्तम् । वस्तुतस्तु दृग्ग्रहयोस्तुल्यत्वे भगवताग्रे युतेरुक्तत्वात् तात्कालिकयोः स्यष्टयोरतुल्यत्वेन दृक्कर्मसाधनार्थं नतदिममानयोस्तयोर्भिन्नत्वेन स्वपदं युक्तं प्रयुक्तम् । न तु स्पष्टक्रान्तिजचरोस्थन्नदिममानयो \textendash


\newpage

\hspace{3cm} गूढार्थप्रकाशकेन सहितः~। \hfill २१५
\vspace{1cm}


\noindent र्भेदान्नमभेदाञ्च स्वमित्युक्तम् । तत्साधनस्य वैयधिकरण्येनाग्नसक्तिरिति ध्येयम् । उक्तरीत्योन्तराद्विक्षेपाल्लब्धतत्कलात्मकं प्राच्यां प्राक्कपाले ग्रहस्य हीनम् । पश्चिमकपाले योज्यम् । दक्षिणे तथा विक्षेपे । तुकारात् तदुत्पन्नं फलं प्राक्कपाले योज्यं पश्चिमकपाले हीनं कार्यम्~॥~९~॥\\
\noindent अथायनदृक्कर्माह \textendash

%\vspace{2mm}

\begin{quote}
{\ssi सत्रिभग्नहजक्रान्तिभागघ्नाः क्षेपलिप्तिकाः~।\\
 विकलाः स्वमृणं क्रानतिक्षेपयोर्भिन्नतुल्य योः~॥~१०~॥ }
 \end{quote}
%\vspace{2mm}
 
 
 विक्षेपकलाः पूर्वसाधिता राशित्रययुतग्रहोत्पन्नक्रान्त्यंशैर्गुणिता विकला भवन्ति । ता अक्षदृक्कर्मसंलतग्रहे विकलास्थानेक्रान्तिक्षेपयोः सत्रिभग्रहस्य क्रान्तिर्ग्रहस्य विक्षेपः । अनयोर्भिन्नतुल्ययोर्भिन्नैकदिक्कयोः सतोः क्रमेण स्वमृणं कार्याः । अत्रोपपत्तिः । विक्षेपवृत्तस्य ग्रहथिम्बोपरिधरुवप्रोतश्लथवृत्तं स्पृष्ट्वा क्रान्तिवृत्ते ग्रहासन्ने यत्र लगति तस्य ग्रहचिमस्यान्तरे याः क्रान्तिवृत्ते कलास्ता आयनकलास्तदानयनार्थं क्षेत्रं ग्रहशरः कदम्बाभिमुखः कर्णः । तत्सम्बद्धद्युरात्रवृत्तप्रदेशध्रुवप्रोतश्लथवृत्तसम्पातयोरन्तरे द्युरात्रवृत्ते भुजः । ध्रुवप्रोतवृत्ते स्पष्टशरो ग्रहबिम्बतत्सम्पातान्तरे कोटिः । अतस्त्रिज्याकर्णेऽयनवलनज्या भुजस्तदा शरकर्णे क इत्यनुपातेन द्युरात्रवृत्ते द्युज्याप्रमाणेन भुजकलाः । न तु ग्रहचिह्नततद्वृत्तसम्पातात्तरे क्रान्तिवृत्ते भुजकलाः क्रान्तिवृत्तस्य तिर्यक्त्वेन तादृशक्रान्तिवृत्तप्रदेशस्य तिर्यक्त्वाद्भुजत्वासम्भवात् । आयनवलनज्या भुजस्त्रिज्या कर्णो यष्टिः


\newpage

\noindent २१६ \hspace{4cm} गूढार्थप्रकाशकेन सहितः~।
\vspace{1cm}


\noindent कोटिस्तद्वर्गान्तरपदरूपेति क्षेत्रं गोले प्रत्यक्षम् । अतोऽनुपाते न क्षतिः । तत्र भगवता लोकानुकम्पया गणितसुखार्थं द्युरात्रवृत्तस्थभुजकलाः क्रान्तिवृत्तस्था अङ्गीकृताः स्वल्पान्तरत्वात्।अतोऽयनवलनज्या शरकलाभिर्गुण्या त्रिज्यया भाज्येति प्राप्ते भगवतायनवलनस्य सत्रिभग्रहक्रान्तिभागत्वेनाङ्गीकारात् तद्भागा अष्टपञ्चाशता गुणनीया ज्या भवति । यतः परमाश्चतुर्विंशत्यंशा अष्टपञ्चाशता गुणिताः पञ्चोनाः परमक्रान्तिज्या जाता । इयं शरगुणा त्रिज्याभक्तायनकलास्तत्र रिकलात्मकफलार्थं षष्टिर्गुण इति सत्रिभग्रहक्रान्तिभागगुणितो ग्रहविक्षेपोऽष्टपञ्चाशत्षष्टिघातेन विंशत्यूनेन पञ्चत्रिंशच्छतेन गुण्यस्त्रिज्यया भक्त इति सिद्धम् । अत्रापि लाघवाद्गुणस्य त्रिज्यामितत्वेन स्वल्पान्तरत्वादङ्गीकाराद्गुणहरयोर्नाश इत्युपपन्नं सत्रिभेत्यादिविकला इत्यन्तम् । भास्कराचार्यैस्तु । 

%\vspace{2mm}

 \begin{quote}
 {\qt आयनं वलनमस्फटेषुणा सङ्गुणं द्युगुणभाजितं हतम्~।\\
 पूर्णपूर्णधृतिभिर्ग्रहाश्रितव्यक्षभोदयहृदायनाः कलाः~॥}
 \end{quote}
 %\vspace{2mm}


इति सूक्ष्ममस्मादुक्तम् । धनर्णोपपत्तिस्तु मकराद्युत्तरायणे दक्षिणध्रुवाद्दक्षिणकदम्बोऽधः । उत्तरध्रुवादुत्तरकदम्ब ऊर्ध्वम् ।तत्र शरो यदा तूत्तरस्तदा ग्रहबिम्बस्योत्तरकदम्बोन्मुखत्वेनोत्तरध्रुवादुन्नतत्वात् क्रान्तिवृत्तस्थग्रहचिह्नात् क्रान्तिवृत्तध्रुवप्रोतश्लथवृत्तसम्पात आयनग्रहचिह्नरूपः क्रान्तिवृन्ते पश्चाद्भवत्यत आयनविकलाः स्यष्टग्रह ऋणं कृताश्चेदायनग्रहभोगो ज्ञातः स्यात् । एवं दक्षिणशरे ग्रहबिम्बस्य दक्षिणकदम्बोन्मुखत्वेन


\newpage

\hspace{3cm} गूढार्थप्रकाशकेन सहितः~। \hfill २१७
\vspace{1cm}


\noindent ध्रुवान्नतत्वात् कान्तिवृत्ते ग्रहचि ह्नादायनग्रहचिह्नमग्र एव भवतीति धनमायनविकलाः । कर्कादिदक्षिणायने तु दक्षिणध्रुवाद्दक्षिणकदम्ब ऊर्ध्वमुत्तरध्रुवादु त्तरकदम्बौऽधः । तत्र यदि ग्रहशरो दक्षिणस्तदा ग्रहबिम्बस्य दक्षिणध्रुवादुन्नतत्वात् क्रान्तिवृत्ते ग्रहचिह्नादायनग्रहचिह्नं पश्चादत ऋणमायनम् ।यद्युत्तरशरस्तदा ग्रहबिम्बस्योत्तरध्रुवान्नतत्वाद्ग्रहचि ह्नादायनग्रहचिह्नमग्रे क्रान्तिवृत्ते भवतीत्यायनं धनमिति गोलस्थित्यायनशरदिगैक्य ऋणमयनशरदिग्भेदे धनमिति सिद्धम् । तव ग्रहायनदिशः सत्रिभग्रहगोलदिक्तुल्यत्वात् सत्रि भग्रहक्रान्तिग्रहशरयोरेकदिक्त्व ऋणं भिन्नदिक्त्वे धनमित्युपपन्नम् । अथाक्षदृक्कर्मोपपत्तिः । भूगर्भक्षितिज याम्योत्तरवृत्तसम्पातरूपसमप्रोतचलवृत्ते ग्रहबिम्बसक्ते क्रान्तिमण्डलस्य ग्रहासन्नो यत्र सम्पातस्तत्राक्षदृक्कलासंस्कृतो ग्रहस्तस्यायनग्रहस्य चान्तरे क्रान्तिवृत्तप्रदेश आक्षदृक्कलास्ताः क्षितिजस्थग्रहबिम्बे परमान्तरत्वात्परमा याम्योत्तरवृत्तस्थे ग्रहेऽयनग्रहमेवाक्षदूक्कलासंस्कृतग्रहचिह्नं भवतीति तदभावः । अतः क्षितिजस्थे ग्रहबिम्बे चलवृत्तं याम्योत्तरक्षितिजसम्पातप्रोतं क्षितिजवृत्ताभिन्नं तत्र ग्रहबिम्बसक्तं ध्रुवप्रोतचलवृत्तक्रान्ति वृत्तच्चम्पातोऽयनग्रहचिह्नरूपः क्षितिजस्थक्रान्तिवृत्तप्रदेशादूर्ध्वमधो वा याभिः कलाभिरन्तरितस्ता आक्षदृक्कलाः । आसां ज्ञानार्थं तदन्तरप्रदेशीयद्युरात्रवृत्तखण्डप्रदेशस्थासवोऽक्षजाः साधिताः । तथाहि । घृवद्वयप्रोतग्रहबिम्बगतचलवृत्ते विषुवद्वृत्तग्रहबिम्बान्तरे स्फुटा क्रान्तिः ।


{\tiny{2 E}}

\newpage

\noindent २१८ \hspace{4cm} सूर्यसिद्धान्तः
\vspace{1cm}


\noindent विषुवद्वृत्तस्थायनग्रहचिह्नान्तरे मध्यमा क्रान्तिरयनग्रहस्यायनग्रहचिह्नग्रहबिम्बान्तरे स्मुटशरः । द्वयोः क्रान्त्योरेकदिक्त्वे स्फुटक्रान्तिरधिका । तत्रोत्तरगोलेऽयनग्रहचिह्नं क्षितिजादधः स्वद्युरात्रवृत्ते क्रान्त्योश्चरान्तरासुभिर्भवति । यतोऽयनग्रहचिह्नद्युरात्रवृत्तस्थोन्मण्डलक्षितिजान्तररूपचराद्ग्रहबिम्बीयचरस्याधिकत्वेन भध्यमचरसम्बद्धक्षितिजवृत्तप्रदेशाद्ध्रुवाभिमुखसू्रं ग्रहबिम्बीयचरसम्बद्धद्युरात्रवृत्तप्रदेशे यत्र लग्नं तत्क्षितिजान्तराले चरान्तरस्य सत्त्वेन स्पष्टशरचरान्तराभ्यां कोटिभुजाभ्यामोयतचरस्रक्षेत्रस्य तद्युरात्रवृत्तद्वयमध्ये स्फुटदर्शनम्। एवं दक्षिणगोलेऽयनग्रहचिह्नं स्वद्युरात्रवृत्ते क्षितिजादूर्ध्वं क्रान्त्योश्चरन्तरासुभिरिति । क्रान्त्योर्भिन्नदिक्त्वे तु क्षितिजादयनग्रहचिह्नं स्वद्युरात्रवृत्ते क्रान्त्योश्चरयोगतुल्यासुभिरध ऊर्ध्वम् । मध्यक्रान्तिद्युरात्रवृत्त उन्मण्डलात् स्पष्टक्रान्तिचरतुल्यान्तरेण दक्षिणोत्तरगोस्त्रयोरध ऊर्ध्वमयनग्रहचिह्नस्य सत्त्वात्। क्षितिजाच्चरान्तरेणोद्वृत्तस्य सत्त्वाच्चेति । भास्कराचार्यैः।

%\vspace{2mm}

\begin{quote}
{\qt स्फुटात्फुटाक्रान्तिजयोश्चरार्धयोः \\
 ममान्यदिक्त्वेऽन्तरयोगजासवः~।\\ 
 पलोद्भवाख्या भनभःसदाम्~। }
 \end{quote}
%\vspace{2mm}


इति सूक्ष्ममाक्षदृगसुज्ञानमुक्तम् । भगवता तु पूर्वोक्तरीत्या स्फुटास्फुटक्रान्तिसंस्कारोत्पन्नस्फुटशररूपक्रान्तिखण्डस्य स्वल्पान्तरेण यथागतशरतुल्यस्य चरमाक्षदृगसव इत्यङ्गीकृत्य द्वादशकोटौ पलभा भुजस्तदा व्रिक्षेपरूपक्रान्तिकोटौ क इत्यनुपाता \textendash


\newpage

\hspace{3cm} गूढार्थप्रकाशकेन सहितः~। \hfill २१९
\vspace{1cm}


\noindent द्विक्षेपज्या फलधनुषोस्त्यागात् स्वल्पान्तरेण कुज्याचरज्ययोरभिन्नत्वेनाङ्गीकाराच्चरासव आक्षासव एता एव कला धृताः स्वल्पान्तरत्वात् । क्षितिजातिरिक्तस्थग्रहबिम्बे त्वेताः कला अभीष्टनतकालपरिणता भवन्तीति यिषुवच्छायद्येत्यादिस्वदिजाद्रविभाजितमित्यन्तम् । अत्र ग्रहे आयनं दृक्कर्म संस्कार्य तस्माद्दिनरात्रिमानादिनतं साधयित्वाक्षदृक्कर्म क्रियते तदा किञ्चित् सूक्ष्ममिति सत्रिभग्रहजेत्यादिश्लोकः सप्तमो यत्पुस्तके तत्र तूक्तं स्वतः सिद्धम् । नतानुपाते स्वपदव्यर्थप्रयोगशङ्कानवकाशश्च समग्रहयोरायनदृक्कर्मसंस्कारेण भिन्नत्वसम्भवात् तयोर्दिनमाननतयोरपि भिन्नत्वसिद्धेरित्यवधेयम् । धनर्णोपपत्तिकस्तु समप्रोतचलवृत्तं ग्रहबिम्बोपरिगं यत्र क्रान्तिवृत्ते लगति स राश्यादिभोग आक्षदृक्कर्मसंस्कृत इति प्रागुक्तम् । तत्र पूर्वकपाले तस्माद्ग्रहादायनग्रहचिह्नं क्रान्तिवृत्त उत्तरशरेऽग्रिमभागे भवति दक्षिणशरे पश्चाद्भवतीति क्रमेणर्णधनमुक्तम् । पश्चिमकपाले तूत्तरशरे पवाद्दक्षिणशरेऽग्रिमभाग इति क्रमेणायनग्रहे धनर्णं दृक्कर्मद्वयसंस्कृतो ग्रहः सिद्धो भवतोत्युपपन्नं सर्वम्~॥~९०~॥\\
\noindent अथ प्रसङ्गगदृक्कर्मसंस्कारस्थलान्याह \textendash

%\vspace{2mm}

\begin{quote}
{\ssi नक्षत्रग्रहयोगेषु ग्रहास्तोदयसाधने~।\\
 शृङ्गोन्नतौ तु चन्द्रस्य दृक्कर्मादाविद स्मृतम्~॥~११~॥ }
 \end{quote}
%\vspace{2mm}


 अत्र निमित्तसप्तमी । ग्रहनक्षत्राणां बहुत्वाद्बहुवचनम् । नक्षत्रग्रहयोर्युत्यर्थं नक्षत्रग्रहयोरिदं द्वयं दृक्कर्म स्मृतं प्रागुक्तम् ।


{\tiny{2E2}}

\newpage

\noindent २२० \hspace{4cm} सूर्यसिद्धान्तः
\vspace{1cm}


\noindent आदौ प्रथमं कार्यम् । ताभ्यामनन्तरं क्रिया कार्येत्यर्थः । अत्र नक्षत्रध्रुवकाणामायनदृक्कर्मसंस्कृतानामेवोक्तत्वादायनं दृक्कर्म न कार्यमिति ध्येयम् । ग्रहाणामस्तोदयौ नित्यास्तोदयौ सूर्यसान्निध्यजनितास्तोदयौ च । ग्रहाणामुपलक्षणत्वान्नक्षत्राणामपि । तयोः साधननिमित्तं ग्रहस्य नक्षत्रस्य वा देयम् । अत्राक्षदृक्कर्मार्थं केवलः शरः साध्यः । न तु दिनमानरात्रिमाननतोन्नते साध्ये । क्षितिजसम्बन्धेन दृग्ग्रहरूपोदयास्तलग्नस्यावश्यकत्वेन क्षितिजातिरिक्तनतपरिणामस्य व्यर्थत्वात् । युतौ तु समप्रोतचलवृत्ते युगपदर्शनार्थं तत्परिणामस्यावश्यकत्वात् । शृङ्गोन्नतिनिमित्तं चन्द्रस्य । तुकारः समुच्चयार्थकचकारपरः । अत्रापि। श्लोके पृर्वार्धोक्तमाक्षदृक्कर्म संस्कार्यमिति ध्येयम्~॥~९९~॥\\
\noindent अथ दृक्कर्मसंस्कृतग्रहयोर्युतिकालं तात्कालिकतद्विक्षेपाभां ग्रहयोर्याम्योत्तरान्तरं चाह\textendash

%\vspace{2mm}

\begin{quote}
{\ssi तात्कालिकौ पुनः कार्यौ विक्षेपौ च तयोस्ततः~।\\
 दिक्त्वल्पे त्वन्तरं भेदे योगः शिष्टं ग्रहान्तरम्~॥~१२~॥ }
\end{quote}
%\vspace{2mm}


 पुनर्द्वितीयवारं तादृशग्रहाभ्यां शोघ्रे मन्दाधिकेऽतीत इत्यादिना युतेर्गतैष्यत्वं ज्ञात्वा ग्रहान्तरकला इत्यादिना दृक्कर्मसंकृतौ समौ स्वयुतिसमये भवतः । विवरं तद्वदुद्धृत्येत्यादिना समस्पष्टग्रहकला दृक्कर्मसंस्कृतसमग्रहकालो युत्याख्यो ज्ञेयः । तस्मिन् काले साधितौ तौ ग्रहौ स्फुटावसमौ तात्कालिकौ मध्यस्पष्टादिक्रियया कार्यौ । तयोः साधितग्रत्रयोर्विक्षेपौ । चः \textendash


\newpage

 \hspace{3cm} गूढार्थप्रकाशकेन सहितः~। \hfill २२१
\vspace{1cm}


\noindent समुच्चये । कार्यौ । एतौ ग्रहौ दृक्कर्मसंस्कृतौ समौ भवत इति प्रतीतिः । नो चेत् तस्मादप्युक्तरीत्या मुहुः कालं स्थिरं कृत्वा प्रतीतिर्द्रष्टव्या । ततः सूक्ष्मयुतिसमये ग्रहयोर्विक्षेपसाधनानन्तरम् । दिक्तुल्य एकदिक्त्वे तुकाराद्विक्षेपयोरन्तरं कार्यम् । भेदे भिन्नदिक्त्वे विक्षेपयोर्योगः । शिष्टं संस्कारोत्पन्नं ग्रहान्तरम् । युतिसम्बधिनोर्ग्रहबिम्बकेन्द्रयोरन्तरालं याम्योत्तरं भवति । अत्रोपपत्तिः । दृक्कर्मसंस्कृतग्रहयोः पूर्वापरान्तराभावः समग्रोतचलवृत्त इति तयोः समत्वम् । विक्षेपाग्रे ग्रहबिम्बकेन्द्रत्वादेकदिशि विक्षेपयोरन्तरं ग्रहबिम्बकेन्द्रयोर्याम्योत्तरमन्तरं समप्रोतचलवृत्ते भिन्नदिशि शरयोर्योग एव ग्रहबिम्बकेन्द्रयोर्याम्योत्तरमन्तरं तद्वृत्ते । भास्कराचार्यैस्तु\textendash 

%\vspace{2mm}

\begin{quote}
{\qt एवं लब्धैर्ग्रहयुतिदिनैश्चालितौ तौ समौ स्त \\
स्ताभ्यां सूर्यग्रहणवदिषू संस्कृतौ स्वस्वनत्या~।\\
तौ च स्पृष्टौ तदनु विशिखौ पूर्ववत् संविधेयौ \\
दिक्साम्ये या वियुतिरनयोः संयुतिर्भिन्नदिक्त्वे~॥ \\ }
\end{quote}
%\vspace{2mm}


 इत्यनेन सूक्ष्ममुक्तम् । भगवता कृपालुना तदुपेक्षितम् । स्वल्पान्तरत्वात्~॥~९२~॥\\
 \noindent अथ पञ्चताराणां बिम्बमानकलानयनं श्लोकाभ्यामाह \textendash

%\vspace{2mm}

\begin{quote}
{\ssi कुजार्किज्ञामरेज्यानां त्रिंशदधाधवधिताः~।\\
 विष्कम्भाश्चन्द्रकक्षायां मृगोः षष्टिरुदाहृता~॥~१३~॥

त्रिएचतुः कर्णयुक्त्याप्तास्ते द्विघ्नास्त्रिज्यया हताः~।\\
स्फुटाः स्वकर्णास्तिथ्याप्ता भवेयुर्मानलिप्तिकाः~॥~१४~॥}
\end{quote}
\newpage

\noindent २२२ \hspace{4cm} सूर्यसिद्धान्तः 
\vspace{1cm}


 त्रिंशदर्धार्धवर्धितास्त्रिंशतोऽर्धं पञ्चदश तदर्धं सार्धसप्त तैरुत्तरोत्तरं युक्तास्त्रिंशत् क्रमेण भौमशनिबुधबृहस्पतीनां चन्द्र कक्षायां चन्द्राकाशगोले चन्द्रकक्षाप्रमाणेन न खकक्षाप्रमाणेनेत्यर्थः । विष्कम्भा बिम्बव्यासा योजजात्मका उक्ताः । भौमस्यत्रिंशत् । शनेः सार्धसप्तत्रिंधत् । बुधस्य पञ्चचत्वारिंशत् । गुरोः सार्धद्विपञ्चाशत् । अनेनैव क्रमेण शुक्रस्य षष्टिः । भृगोः षष्टिरित्यनेनार्धार्धेत्यस्य प्रत्येकमर्धयुक्ता इत्यर्थो निरस्तः स्वाभिमतार्थो व्यक्तीकृतश्च । ते उक्ता विष्कम्भा द्विगणास्त्रिज्ययागुणितास्त्रिचतुःकर्णयुक्त्याप्ताः । तृतीयकर्मणि चतुर्थकर्मणि च यौ कर्णौ मन्दकर्णशीघ्रकर्णौ तयोर्योगेन भक्ता इति साम्प्रदायिकव्याख्यानम् । नव्यास्तु तृतीयकर्मणि कर्णानुपातानुक्तेस्तृतीयकर्णस्य मन्दकर्धस्याप्रसिद्धेरुपपत्तिविरोधाच्च पूर्वपव्याख्यामुपेक्ष्य त्रिशब्देन चिज्या चतुःकर्णश्चतुर्थकर्मणि शीघ्रकर्णस्तयोर्योगेन भक्ता इत्यर्थं कुर्वन्ति । स्पष्टाः स्वकर्णाः स्वबिम्बव्यासा भवन्ति । पञ्चदशभक्ता बिम्बमानकला भवेयुः । अत्रोपपत्तिः । स्वस्वकक्षायां स्थिताः पञ्चताराग्रहा दूरत्वाल्लोकैश्चन्द्राकाशस्थिता इव दृश्यन्ते । अतस्तेषां वास्तवबिम्बव्यासयोजनानि खयं ज्ञातानि । यथा सूर्यबिम्बव्यासयोजनान्युक्तानि चन्द्रग्रहणाधिकारे रवेः स्वभगणाभ्यस्त इत्यादिना चन्द्रकक्षायां साधितानि तथा खभगणानुसारेणोक्तप्रकारेण चन्द्रकक्षायां साधितानि । तथा च शाकल्यसंहितायाम् । 

%\vspace{2mm}

\begin{quote}
{\qt अन्तरुन्नतवृक्षाश्च वनप्रान्तै स्थिता इव~।\\}
\end{quote}
\newpage

\hspace{3cm} गूढार्थप्रकाशकेन सहितः~। \hfill २२३
%\vspace{1cm}

\begin{quote}
{\qt दूरत्वाच्चन्द्रकक्षायां दृश्यन्ते सकला ग्रहाः~॥
 
व्यर्धाष्टवर्धितास्त्रिंशद्विष्कम्भाः शास्त्रदृष्टतः~।\\}
\end{quote}
%\vspace{2mm}


इत्येतानि त्रिज्यातुल्यशीघ्रकर्ण उक्तानि । अतः शीघ्रकर्णेऽधिके न्यूनं बिम्बग्रहस्योच्चासन्नत्वादल्पे तु नीचासन्नत्वादधिकं बिम्बमिति चिज्ययोक्तानि बिम्बानि तदेष्टशीघ्रकर्णेन कानीति व्यस्तानुपातेन युक्तमपि भगवतोपलब्ध्या त्रिज्यातोऽधिकन्यूनकर्णयोः क्रमेण व्यस्तानुपातागतादधिकं न्यूनं च बिम्बं दृष्टमतः कर्ण एव त्रिज्याशीघ्रकर्णयोगार्धमितः क्रमेण न्यूनाधिको गृहीतः । अत्र छेदं लवं च परिवर्त्य हरस्येत्यादिना द्विघ्नास्त्रिज्यागुणिता विष्कम्भास्त्रिज्याशीघ्रकर्णयोगभक्ता इत्युपपन्नम् ।

%\vspace{2mm}

\begin{quote}
{\qt त्रिचतुःकर्णयोगार्धं स्फुटकर्णोऽस्य मस्तके~।\\
त्रिज्याघ्नाः स्फुटकर्णाप्ता र्विष्कम्भास्ते स्रुटाः स्मृताः~॥}
\end{quote}
%\vspace{2mm}


इति शाकल्योक्तेश्च । अत एव बिम्बस्य द्राङ्गीचोच्चमण्डलस्थत्वेन शीघ्रकर्णस्यैव भूगर्भाद्बिम्बे सम्बन्धान्मन्दकर्णसम्बन्धस्त्वयुक्तः । नहि छेद्यके मन्दकर्णार्धाच्छीघ्रकर्णार्धे ग्रहबिम्बमस्तीति प्रतिपादितम् । येन मन्दशीघ्रकर्णयोर्योगार्धं कर्णः सूपपन्नः । शीघ्रफलानयने तथाङ्गीकारापत्तेः । भास्कराचार्यैस्तु\textendash

%\vspace{2mm}

\begin{quote}
{\qt व्यङ्घ्रीषवः सचरणा स्मृतवस्त्रिभाग \\
युक्ताद्रयो नव च सत्रिलवेषवश्च~।\\
स्युर्मध्यमास्तनुकलाः क्षितिजादिकानां \\
त्रिज्याशुकर्णविवरेण पृथग्विनिघ्नाः~॥ \\

त्रिघ्न्या निजान्त्यफलमौर्विकया विभक्ताः}
\end{quote}
\newpage

\noindent २२४ \hspace{4cm} सूर्यसिद्धान्तः
%\vspace{1cm}

\begin{quote}
{\qt लब्धेन युक्तरहिताः क्रमशः पृथक्स्थाः~।\\
 ऊनाधिके त्रिभगुणाच्छ्रवणे स्फुटा स्युः~। }
 \end{quote}
%\vspace{2mm}


इत्युपलब्ध्योक्तम् । भास्करानुवर्तिनस्तु त्रिचतुःकर्णयुक्त्याप्ता इत्यस्य चिज्याशीम्रकर्णयोर्योगार्धेन भक्ता इत्यर्थं वदन्ति~॥~१४~॥\\
\noindent अथ युतिसम्बधिनौ ग्रहौ युतिसमये दर्शनीयावित्याह \textendash

%\vspace{2mm}

\begin{quote}
{\ssi छायाभूमौ विपर्यस्ते स्वच्छायाग्रे तु दर्शयेत्~।\\
 ग्रहः स्वदर्पणान्तस्थः शङ्क्वग्रे सम्प्रदृश्यते~॥~१५~॥ }
\end{quote}
%\vspace{2mm}


 छायाभूमौ छायादानार्थं योग्यायां जलवत् समीकृतायां पृथिव्याम् । विपर्यस्ते वैपरीत्येन दत्ते स्वच्छायाग्रे ग्रहच्छायाग्रस्थाने । तुकारोऽन्ययोगव्यवच्छेदार्थैवकारपरः । स्वदर्पणान्तस्थः स्वस्य यो दर्पण आदर्शस्तत्र स्थापितस्तन्मध्यस्थितो ग्रहोग्रहप्रतिबिम्बः स्यात् । तद्गणकः शिष्याय दर्शयेत् । एतदुक्तंभवति । समभूमौ दिक्साधनं कृत्वा दिक्सम्पातस्थानाद्युतिकालिकच्छायाङ्गुलानि पूर्वापरसूत्राद्भुजविपरीतदिशि भुजान्तरेण ग्रहाधिष्ठितपूर्वापरकपालदिशि दत्वा तत्रादर्शः स्थाप्यस्तत्र प्रतिबिम्बं ग्रहस्य दिक्यम्पातस्थो गणकः शिष्याय दर्शयेदिति । अत्रोपपत्तिः । ग्रहबिम्बादवलम्बसूत्रं महाशङ्कुरूपं यत्र भूमौ पतति तत्र ग्रहबिम्बप्रतिबिम्बो भवति । तज्ज्ञानं तु खमध्याद्ग्रहबिम्बपर्यन्तं नतांशा आकाशे तथा भूमौ दिक्सम्पातस्थानान्महाशङ्कुकोटौ दृग्ज्या भुजस्तदा द्वादशाङ्गुलशङ्कुकोटौको भुज इत्यनुपातानीतच्छायामितान्तरेण ग्रहाधिष्ठितकपाले


\newpage

\hspace{3cm} गूढार्थप्रकाशकेन सहितः~। \hfill २२५
\vspace{1cm}


\noindent भवति । यथा दिक्सम्पातस्थद्वादशाङ्गुलशङ्कोश्छाया ग्रहाधिष्ठितकपालान्यकपाले भवति । तथा ग्रहप्रतिबिम्बस्थानस्थद्वादशाङ्गुलशङ्कोश्छाया दिक्सम्पाते भवति । अतो दिक्सम्पातस्थानाच्छाया ग्रहाधिष्ठितकपाले दत्ता तदग्रे ग्रहप्रतिबिम्बस्थानं ज्ञातं भवतीत्युपपन्नं छायाभूमावित्यादि खदर्पणान्तस्य इत्यन्तम् । अथ ग्रहाधिष्ठितकपालान्यकपाले छायासद्भावनियमाद्ग्रहाधिष्ठितकपाले कथं छायादानं युक्तं व्याघातादितिमन्दाशङ्कास्यरसादाह\textendash शङ्क्वग्र इति~। दिक्सम्पातस्थापितशङ्कोरग्रे मस्तक आकाशे ग्रहो दृश्यते गणकेनेति शेषः~॥~९५~॥\\
\noindent ननु कथं दृश्यत इत्यतः प्रकृतग्रहयोर्युतिसम्बन्धिनोर्दर्शनप्रकारंसार्धहोश्लोकाभ्यामाह \textendash

%\vspace{2mm}

\begin{quote}
{\ssi पञ्चहस्तोच्छितौ शङ्कू यथादिग्भ्रमसंस्थितौ~।\\
 ग्रहान्तरेण विक्षिप्तावधो हस्तनिखातगौ~॥~१६~॥

छायाकणा ततो दद्याच्छायाग्राच्छङ्कुमूर्धगौ~।\\
छायाकर्णाग्रसंयोगे संस्थितस्य प्रदर्शयेत्~॥~१७~॥

स्वशङ्कुमूर्धगो व्योम्नि ग्रहौ दिक्तुल्यतामितौ~।\\}
\end{quote}
\vspace{2mm}


 ग्रहयुनिसम्बन्धिनोर्ग्रहयोरायनदृक्कलां श्लोकपूर्वार्धोक्ताक्षदृक्कलाभ्यां संस्कृतयोस्तुल्येऽल्पान्तरेणासन्ने वोदयलग्ने स्तः । षड्भयुतयोर्ग्रहयोरायनाक्षदृक्कलासंस्कृतयोस्तुल्ये स्वल्पान्तरेणासन्ने वास्तलग्ने भवतः । यस्मिन् काले ग्रहौ द्रष्टुमभिमतौ तात्कालिकलग्नाद्रात्रौ यदुदयास्तलग्ने क्रमेण न्यूनाधिके यदि भवतस्तौ \textendash


{\tiny{2F}}

\newpage


\hspace{3.5cm} सूर्यसिद्धान्तः \hfill २२६
\vspace{1cm}


\noindent सूर्यसान्निध्यजनेतास्ताभावे दर्शनयोग्यौ । तदा पञ्चहस्तोच्छ्रितौ । चतुर्विंशशत्यङ्गुलो हस्तः । एवं पञ्चहस्तप्रमाणदीर्घौ शङ्कू काष्ठघटितसरलदण्डौ यथादिग्भ्रमसंस्थितौ युतिकाले ग्रहयोर्यादृशं दिग्भ्रमणम् । ग्रहौ प्रवहभ्रमेण पूर्वकपाले पश्चिमकपाले वा तत्र संस्थितौ स्वाधिष्ठितस्थानाद्ग्रहाधिष्ठितकपालदिशि स्थाप्यौ न ग्रहानधिष्ठितकपालदिशि । ग्रहान्तरेण दिक्तुल्ये त्वन्तरं भेदे योग इत्यादिना ज्ञातयाम्योत्तरग्रहान्तरेण कलात्मकेन विक्षिप्तौ याम्योत्तरान्तरितौ स्थाप्यौ । अत्र सोन्नतमित्यादिनाग्रहविक्षेपावङ्गुलात्मकौ कृत्वा दिक्तुल्ये त्वन्तरमित्यादिना ग्रहान्तरं ज्ञेयम् । अधो भूमेरन्तः । हस्तनिखातगौ हस्तवेधप्रमाणा या गर्ता तत्र स्थितौ भूम्यां शङ्कोर्हस्तमात्रं रोपयित्वाभूमेरूर्ध्वं शङ्कू चतुर्हस्तप्रमाणदीर्घौ स्यातामित्यर्थः । ततः शङ्क्वुमूलाभ्यां प्रत्येकं यच्छायाग्रं ग्रहानधिष्ठितकपालदिशि तस्मात् प्रत्येकमित्यर्थः । छायाकर्णौ स्वकीयौ शङ्कुमूर्धगौ निजशङ्क्वग्ररूपमस्तकप्रापिणौ गणको दद्यात् । एतदुक्तं भवति । युतिसमये लग्नं कृत्वा तात्कालिकोदयलग्नेष्टलग्नाभ्यां' पूर्ववदन्तरकालो ग्रहोदयाद्गतकालः सावनः । एवं ग्रहयोर्युतिसमयेस्वदिनगतात् त्रिप्रश्नाधिकारोक्त्रविधिना स्यफक्रान्त्या छायासाध्या । ततो यो ग्रहो दक्षिणोत्तरयोर्मध्ये यद्दिशि तच्छायातद्दिक्स्था शङ्कोर्मूलाद्ग्रहानधिष्ठितकपालदिशि पष्यर्वापरसूत्राद्भुजान्तरेण भुजदिशि देया । परमानीतच्छाया द्वादशाङ्गुलशङ्कोरिति चतुर्हस्तशङ्कुप्रमाणेन प्रसाध्य रेखा तन्मिता सम\textendash


\newpage

 \hspace{3cm} गूढार्थप्रकाशकेन सहितः~। \hfill २२७
\vspace{1cm}


\noindent भूमौ शङ्कुमूलात् कार्या । रेखाग्रे छायाग्रे ज्ञापकं चिह्ने कार्यम् । तत्र कीलादिना सूत्रं बध्वा शङ्क्वग्रसोक्तं\textendash प्रसार्यमिति~। छायाकर्णाग्रसंयोगे छायाग्रं कर्णस्य मूलरूपमग्रं तयोः सम्पाते संस्थितस्य छायाग्रस्थानकृतगर्तापविष्टशिष्यस्य गणको ग्रहावाकाशे स्वशङ्कुमूर्धगौ निजशङ्क्वग्ररूपमस्तसमसूत्रस्थितौ दृक्तुल्यतां दृष्टिगोचरतामितौ प्राप्तौ प्रदर्शयेत् सन्दर्शयेत् । अत्रोपपत्तिः । उच्चतया दर्शनार्थं पञ्चहस्तप्रमाणौ शङ्कू कृतौ । तत्रैकहस्तस्य भूमिगुप्तत्वं शङ्कुदृढत्वार्थं कृतम् । बहिः पुरुर्षप्रमाणौ चतुर्भितहस्तावशिष्टौ शङ्कोः पुरुषपर्यायेणाभिधानाच्च । शङ्कुसूत्रस्य ग्रहबिम्बसक्तत्वाद्यथा दिग्भ्रमसंस्थितावित्युक्तम् । शङ्क्वग्रसमसूत्रेण ग्रहबिम्बावस्थाननियमाद्ग्रहान्तरेण याम्योत्तरान्तरितौ स्मापितौ । अत्र यद्यपि स्वस्वस्पष्टक्रान्त्यग्रां प्रसाध्य ततः कर्णाग्रां प्रसाध्योक्तदिशा पलभासंस्कारेण स्वस्वभुजं प्रसाध्यताभ्याम् । 

%\vspace{2mm}

\begin{quote}
{\ssi दिक्तुल्ये त्वन्तरं भेदे योगः शिष्टं ग्रहान्तरम् ~। }
\end{quote}
% \vspace{2mm}
 

इत्युक्तरीत्या ग्रहान्तरं शङ्कोरन्तरं युक्तं तथापि भगवता स्वल्पान्तरेण गणितश्रमापनोदार्थमाकाशस्थितदृष्टान्तरमेव धृतम् । शङ्कोच्छायाग्राच्छायाकर्णसूत्रं ग्रहबिम्बदर्शनसूत्रमतः कर्णमूलदृशा पुरुषेण ग्रहबिम्बं द्रष्टव्यमेवेति दिक्~॥~९७~॥\\
\noindent अथ श्लोकाभ्यां पञ्चताराणां प्राक् प्रतिज्ञातौ युद्धसमागमावाह \textendash

%\vspace{2mm}

 \begin{quote}
{\ssi उल्लेखं तारकास्पर्शाद्भेदे भेदः प्रकीर्त्यते~॥~१८~॥
 
 युद्धमंशुविमर्दाख्यमंशुयोगे परस्परम् ।}
\end{quote}

{\tiny{2 F 2}}


\newpage

\noindent २२८ \hspace{4cm} सूर्यसिद्धान्तः 
\vspace{1cm}

\begin{quote}
{\ssi अंशादूनेऽपसव्याख्यं युद्धमेकोऽत्र चेदणुः~॥~१९~॥
 
समागमोंऽशादधिके भवतश्चेद्बलान्वितौ~।}
\end{quote}
%\vspace{2mm}


भौमादिपञ्चताराणां मध्ये द्वयोर्युतौ तारकास्पर्शाद्बिम्बनेम्योः स्पर्शमात्रादुल्लेखसज्ञं युद्धं वदन्ति युतिभेदज्ञाः । इदंतु द्वयोर्मानैक्यखण्डतुल्ययाम्योत्तरान्तरे भेदे मण्डलभेदे भेदो भेदसञ्ज्ञो युद्धावान्तरभेदो युद्धभैदतत्त्वज्ञैः कथ्यते । अयं भेदो मानीक्यखण्डादूने द्वयोर्याम्योत्तरान्तरे । अत्र भास्कराचार्यैस्तु\textendash

%\vspace{2mm}

\begin{quote}
{\qt मानैक्यार्धाद्द्युचरविवरेऽल्पे भवेद्भेदयोगः \\
कार्यं सूर्यग्रहवदखिलं लम्बनाद्यं स्फुटार्थं~॥\\

कल्प्योऽधस्थः सुधांशुस्तदुपरिग इमो लम्बनादिप्रसिद्ध्यै \\
किन्त्वर्कादेव लग्नं ग्रहयुतिसमये कल्पितार्कान्न साध्यम्~।\\
प्राग्वत् तल्लम्बनेन ग्रहयुतिसमयः संस्कृतः प्रस्फुटः स्यात् \\
खेटौ तौ दृष्टियोग्यौ ग्रहयुतिसमये कार्यमेवं तदैव~॥\\

याम्योदक्स्थद्युचरविवरं भेदयोगे स बाणो \\
ज्ञेयः सूर्याद्भवति च यतः शीतगुः सा शराशा~। \\
मन्दाक्रान्तोऽनृजुरपि यदाधःस्थितः स्यात् तदैन्द्र्यां \\
स्पर्शो मोक्षोऽपरोदिशि तदा पारिलेख्येऽवगम्यः~॥\\}
\end{quote}
%\vspace{2mm}


इति विशेषोऽभिहतः । भगवता तु सूक्ष्मबिम्बयोराकाशे दूरतो विविक्तदर्शनासम्भवाद्व्यर्थपरयासादुपेक्षितमिति ध्येयम् । युतावन्योन्यं किरणयोगे सत्यंशुमर्दाख्यं किरणसङ्घट्टनसञ्ज्ञं युद्धं स्यात् । द्वयोर्याम्योत्तरान्तरेंऽशात् षष्टिकलात्मकैकभागादूने \textendash


\newpage

\hspace{4cm} गूढार्थप्रकाशकेन सहितः~। \hfill २२९
\vspace{1cm}


\noindent ऽनधिके सत्यपसव्यसञ्ज्ञं युद्धं भवति । अत्र विशेषमाह\textendash एक इति~। अत्रापसव्ययुद्ध एको द्वयोरन्यतरोऽणुरणुबिम्बश्चेत्स्यात् तदापसव्यं युद्धं व्यक्तं स्यादन्यथा त्वव्यक्तं युद्धं स्यात् ।एषां चतुर्णां फलम् ।

%\vspace{2mm}

\begin{quote}
{\qt अपसव्ये विग्रहं ब्रूयात् सङ्ग्रामं रश्मिसङ्कुले~। \\
लेखनेऽमात्यपीडा स्याद्भेदने तु धनक्षयः~॥ \\ }
\end{quote}
%\vspace{2mm}


\noindent इति भार्गवीयोक्तं ज्ञेयम् । युद्धभेदानुक्त्वा समागममाह\textendash समागम इति~। द्वयोर्याम्योत्तरान्तरे षष्टिकलात्मकैकभागादभ्यधिके सति समागमो योगो भवति । अत्रापि विशेषमाह\textendash भवत इति~। युतिविषयकौ ग्रहौ बलान्वितौ बलेन ।

%\vspace{2mm}

\begin{quote}
{\qt स्थानादिबलचिन्तात्र व्यर्था केनापि न स्मृता~।\\
प्रश्नत्रयेऽथवाप्यस्मिन् स्थौल्यसौक्ष्म्यबलं स्मृतम् ~॥ }
\end{quote}
%\vspace{2mm}


\noindent इति ब्रह्मसिद्धान्तवचनात् । स्थूलमण्डलतयान्वितौ युक्तौ स्थूलबिम्बौ समावित्यर्थः । चेत् स्तस्तदा समागमस्तयोर्व्यक्तः स्यात् । अन्यथा त्वव्यक्तः समागमः।

%\vspace{2mm}

\begin{quote}
{\qt द्वावपि मयूखयक्तौ विपुलौ स्निग्धौ समागमे भवतः~।\\
अत्रान्योऽन्यं प्रीतिर्विपरीतावात्मपक्षघ्नौ~॥

युद्धं समागमो वा यद्यव्यक्तौ तु लक्षणैर्भवतः~।\\
भुवि भूभृतामपि तथा फलमव्यक्तं विनिर्दिष्टम्~॥ }
\end{quote}
\noindent इत्युक्तेः । 

\begin{quote}
{\qt भेदोल्लेखांशुसम्मर्दा अपसव्यस्तथापरः~। \\
 ततो योगो भवेदेषामेकांशकसमापनात्~॥}
\end{quote}
\newpage

\noindent २३० \hspace{4cm} सूर्यसिद्धान्तः 
\vspace{1cm}


\noindent इति काश्यपोक्तेश्च सर्वं निरवद्यम्~॥~१९~॥ \\
\noindent अथ युद्धे पराजितस्यग्रहस्य लक्षणमाह \textendash 
%\vspace{2mm}

\begin{quote}
{\ssi अपसव्ये जितो युद्धे पिहितोऽणुरदीप्तिमान्~॥~२०~॥ \\
 
 रुक्षो विवर्णो विध्वस्तो विजितो दक्षिणाश्रितः~।} 
\end{quote}
%\vspace{2mm}

 द्वयोर्मध्ये यस्तदितरेण विध्वस्तो हतः स विजितः पराजितो ज्ञेयः । हतस्य लक्षणमाह\textendash अपसव्य इति~। अपसव्ये युद्धे योजितो जयलक्षणैर्विवर्जितः । एतेनोल्लेखादित्रये सञ्ज्ञाफलं न पराजितस्य फलमिति सूचितम् । पिहित आच्छादितोऽव्यक्त इति यावत् । अणुरितरग्रहबिम्बादल्पबिम्बः । अदीप्तिमान् प्रभारहितः । रुक्षोऽस्निग्घः । विवर्णः वर्णेन स्ववर्णेन स्वाभाविकेन रहित इत्यर्थः । दक्षिणाश्रित इतरग्रहाग्रेक्षय दक्षिणदिशि स्थितः । 

%\vspace{2mm}

\begin{quote}
{\qt श्यामो वा व्यपगतरश्मिवान् कृशो वा~। \\
 आक्रान्तो विनिपतितः कृतापसव्यो\\
विज्ञेयो हत इति समग्रहो ग्रहेण~॥ }
\end{quote}
%\vspace{2mm}

इति भार्गवीयोक्तेः~॥~२०~॥\\
\noindent अथ श्लोकार्धेन जयिनो ग्रहस्य लक्षणमाह \textendash
%\vspace{2mm}

\begin{quote}
{\ssi उदकस्थो दीप्तिमान् स्थूलो जयी याम्येऽपि यो बली~॥~२१~॥ }
\end{quote}
%\vspace{2mm}

 इतरग्रहापेक्षयोत्तरदिक्स्थः । दीप्तिमान् प्रभायुक्तः । स्थूल इतरग्रहबिम्बापेक्षया पृथुबिम्बः । जयो जययुक्तः स्यात् ।

\newpage

\hspace{3cm}गूढार्थप्रकाशकेन सहितः~। \hfill २३१
\vspace{1cm}


\noindent अथोत्तरदक्षिणदिक्स्थत्वक्रमेण जयपराजयौ न स्त इत्याह\textendash  याम्य इति~। दक्षिणदिशि यो ग्रहो बली दीप्तिमान् पृथुबिम्बो भवति स जयी । अपिशब्द उत्तरदिशासमुच्चयार्थकः । तथा च जयपराजयलक्षणयोर्दिग्दानमनुपयुक्तमिति भावः~॥~२१~॥\\
\noindent अथ युद्धे विशेषमाह\textendash 

%\vspace{2mm}

\begin{quote}
{\ssi आसन्नावप्युभौ दीप्तौ भवतश्चेत् समागमः~।\\
 स्वल्पौ द्वावपि विध्वस्तौ भवेतां कूटविग्रहौ~॥~२२~॥ }
%\vspace{2mm}
\end{quote}

 उभौ द्वौ । आसन्नावेकभागान्तरगतान्तरितौ । अपिशब्दाद्युद्धलक्षणाक्रान्तौ । दीप्तौ प्रभायुक्तौ चेत् स्यातां तदा बलान्विताविति समागमलक्षणैकदेशसद्भावात् समागमाख्यं युद्धम्| द्वावपि ग्रहौ स्वल्पौ सूक्ष्मबिम्बौ विध्वस्तौ । द्वावपि पराजयलक्षणाक्रान्तौ स्यातां तदा क्रमेण कूटविग्रहसञ्ज्ञकौ युद्ध भेदौ स्याताम्~॥~२२~॥\\
 अर्थोत्सर्गतः शुक्रस्य जयलक्षणाक्रान्तत्वमस्तीति वदन् समागमः शशाङ्केनेति प्राक् प्रतिज्ञातसमागम उक्तप्रकारमतिदिशति \textendash

%\vspace{2mm}

\begin{quote}
{\ssi उदक्स्थो दक्षिणस्थो वा भार्गवः प्रायशो जयी~।\\
 शशाङ्केनैवमेतेषां कुर्यात् संयोगसाधनम्~॥~२३~॥ }
%\vspace{2mm}
\end{quote}

 इतरग्रहापेक्षयोदक्स्थो दक्षिणदिक्स्थो वोभयदिशीत्यर्थः । शुक्रः प्रायश उत्सर्गतो जयलक्षणाक्रान्तत्वेन जयी । कदाचित्पराजयलक्षणाक्रान्तो भवतीति तात्पर्यार्थः ।  एतेषां भौमा \textendash


\newpage

\noindent २३२ \hspace{4cm} सूर्यसिद्धान्तः
\vspace{1cm}


\noindent दिपञ्चताराणां चन्द्रेण सह संयोगसाधनं युतिसाधनमेषामुक्तरीत्या गणकः कुर्यात् । अत्र विशेषार्थकम् ।
%\vspace{2mm}

\begin{quote}
{\qt अवनत्या स्फुटो ज्ञेयो विक्षेपः शीतगोर्युतौ~।}\\
%\vspace{2mm}
\end{quote}

\noindent इत्यर्थं क्वचित् पुस्तके दृश्यते न सर्वत्रेति क्षिप्तं मत्वोपेक्षितम् ।अधिकारस्यापूर्णश्लोकत्वापत्तेश्च । एतक्त्यान्ययोगे नतिसंस्कारनिषेधस्य सिद्धेस्तस्यानुक्तत्वमिति तदनुक्तौ सूर्यग्रक्षणोक्तरीत्या साधारण्येन सर्वत्र तद्विशेषोक्तिरर्थसिद्धेरिति ध्येयम् ~॥~२३~॥\\
\noindent नन्वेषां ग्रहाणां दूरान्तरेण सदोर्ध्वाधरान्तरसद्भावात् परस्परं योगासम्भवेन कथं युतिः सङ्गतेत्यत आह \textendash 

%\vspace{2mm}

\begin{quote}
{\ssi भावाभावाय लोकानां कल्पनेयं ग्रदर्शिता~।\\
 स्वमार्गगाः प्रयान्त्येते दूरमन्योन्यमाश्रिताः~॥~२४~॥ }
%\vspace{2mm}
\end{quote}

 एते ग्रहाः स्वमार्गगाः स्वस्वकक्षास्था अन्योन्यमाश्रिता युतिकाल ऊर्ध्वाधरान्तराभावेन संयुक्ताः सन्तः प्रयान्ति गच्छन्ति । इति दूरं दूरान्तरेण दर्शनादियं ग्रहयुतिकल्पना कल्पनात्मिकावास्तवा प्रदर्शिता पूर्वोक्तग्रन्थेन कथिता । नन्ववस्तुभूता किमर्थमुक्तेत्यतः प्रयोजनमाह । भावाभावायेति । लोकानां भूस्थप्राणिनां भावः शुभफलमभावोऽशुभफलं तस्मै शुभाशुभफलादेशायावस्तुभूतापि युतिरुक्तेति भावः~॥~२४~॥\\
\noindent अथाग्रिमग्रन्थस्यासङ्गतित्वनिरासार्थमधिकाशरसमाप्तिं फक्किकयाह \textendash


\begin{center}
 इति ग्रहयुत्यधिकारः ~।
\end{center}

\newpage

\hspace{3cm} गूढार्थप्रकाशकेन सहितः~। \hfill २३३
\vspace{1cm}


\noindent स्पष्टम् ।
%\vspace{2mm}

\begin{quote}
{\qt रङ्गनाथेन रचिते सूर्यसि द्धान्तटिप्पणे~।\\
ग्रहद्युत्यधिकारोऽयं पूर्णो गूढप्रकाशके~॥ }
%\vspace{2mm}
\end{quote}
 इति श्रीसकलगणकसार्वभौमबल्लालदैवज्ञात्मजरङ्गनाथगणकविरचिते गूढार्थप्रकाशके ग्रहयुत्यधिकारः सम्पूर्णः~॥

\begin{center}
    \rule{7em}{.5pt}
\end{center}



 अथ प्रसङ्गादारब्धो नक्षत्रग्रहयुत्यधिकारो व्याख्यायते~।\\
\noindent तत्र प्रथमं नक्षत्राणां ध्रुकज्ञाममाह \textendash
%\vspace{2mm}

 \begin{quote}
{\ssi  प्रोच्यन्ते लिप्तिका भानां स्वभोगोऽथ दशाहतः~।\\
 भवन्त्यतीतधिष्ण्यानां भोगलिप्ता युता ध्रुवाः~॥~१~॥}
%\vspace{2mm}
\end{quote}

 भानामश्विन्यादिनक्षत्राणामुत्तराषाढाभिजिच्छ्रावणधनिष्ठावर्जितानां लिप्तिका भोगसञ्ज्ञाः कलाः प्रोच्यन्ते समनन्तरमेव कथ्यन्ते । अथानन्तरं स्वभोगः स्वाभीष्टनक्षत्रभोगः कलात्मको वक्ष्यमाणो दशभिर्गुणितः कार्यः । तत्र स्वाभीष्टनक्षत्रगतनक्षत्राणामश्विन्यादीनां भोगलिप्ताः । भभोगोऽष्टशतीलिप्ता इत्युकाष्टशतकलाः प्रत्येकं युताः । अश्विन्याद्यतीतनक्षत्रसङ्ख्यागुणितकलाष्टशतं युतमित्यर्थः । ध्रुवा नक्षत्राणां भवन्ति~॥~९~॥\\
 \noindent अथ प्रतिज्ञाता नक्षत्रभोगलिप्ता उत्तराषाढाभिजिच्छ्रवणधनिष्ठाव्यतिरिक्तानां तेषां ध्रुवकान्नक्षत्रशरांश्चाष्टश्लोकैराह\textendash


{\tiny{2 G}}

\newpage

\noindent २३४ \hspace{4cm} सूर्यसिद्धान्तः 
%\vspace{1cm}

 \begin{quote}
 {\ssi अष्टार्णवाः शून्यकृताः पञ्चषष्टिर्नगेषवः~।\\
अष्टार्था अब्धयोऽष्टागा अङ्गागा मनवस्तथा~॥~२~॥

कृतेषवो युगरसाः शून्यबाणा वियद्रसाः~।\\
स्ववेदाः सागरनगा गजागाः सागरर्तवः~॥~३~॥

मनवोऽथ रसा वेदा वैश्वमाप्यार्धभोगगम्~।\\
आप्यस्यैवाभिजित्प्रान्ते वैश्वान्ते श्रवणस्थितिः~॥~४~॥

त्रिचतुःपादयोः सन्धौ श्रविष्ठा श्रवणस्य तु~।\\
स्वभोगतो वियन्नागाः षट्कृतिर्यमलाश्विनः~॥~५~॥

रन्ध्राद्रयः क्रमादेषां विक्षेपाः स्वादपक्रमात्~।\\
दिङ्मासविषयाः सौम्ये याम्ये पञ्च दिशो नव~॥~६~॥

सौम्ये रसाः खं याम्ये गाः सौम्ये खार्कास्त्रयोदश~।\\
दक्षिणे रुद्रयमलाः सप्तत्रिंशदथोत्तरे~॥~७~॥

याम्येऽध्यर्धत्रिककृता नवसार्धशरेषवः~।\\
उत्तरस्यां तथा षष्टिस्त्रिंशत् षटत्रिंशदेव हि~॥~८~॥

दक्षिणे त्वर्धभागस्तु चतुर्विंशतिरुत्तरे~।\\
भागाः षड्विंशतिः खं च दस्रादीनां यथाक्रमम्~॥~९~॥ }
%\vspace{4mm}
\end{quote}

 अश्विन्यादिनक्षत्राणां क्रमाद्भोगा एते । तत्राश्विन्यां अष्टचत्वारिंशत् कलाः । भरण्याश्चत्वारिंशत् । कृत्तिकायाः कलाः पञ्चषष्टिः । रोहिण्याः सप्तपच्चाशत् कलाः । मृगशिरसोऽष्टपञ्चाशत् । आर्द्रायाश्चत्वारः । अत्राब्धय इत्यत्र गोऽब्धयो गो \textendash


\newpage

\hspace{3cm} गूढार्थप्रकाशकेन सहितः~। \hfill २३५
\vspace{1cm}

\noindent ग्नय इति वा पाठस्त्वयुक्तः । शाकल्यसंहिताविरोधात् । एतेन
%\vspace{2mm}

\begin{quote}
{\qt सौरोक्तरुद्रभस्यांशास्त्यद्रयोऽगाब्धयः कलाः~।}\\
%\vspace{2mm}
\end{quote}

\noindent इति नार्मदोक्तं दशकलोनपञ्चदमभागा मिथुने सर्वजनाभिमतध्रुवको दशकलायुतत्रयोदशभागाः पर्वताभिमतध्रुवकश्च निरस्तः । पुनर्वसोरष्टसप्ततिः । पुष्यस्य षट्सप्तति । आश्लेषायाश्चतुर्दश । तथेति छन्दःपूरणार्थम् । मघायाश्चतुःपञ्चाशत् । पूर्वाफाल्गुन्याश्चतुःषष्टिः । उत्तराफाल्गुन्याः पञ्चाशत् । हस्तस्य षष्टिः । चित्रायाश्चत्वारिंशत् । स्वात्याचतुःसप्ततिः । विशाखाया अष्टसप्ततिः । अनुराधायाश्चतुःषष्टिः । ज्येष्ठायाश्चतुर्दश। अनन्तरं मूलस्य षट् । पूर्वाषाढायाश्चत्वारः । उत्तराषाढाया ध्रुवकमाह\textendash  वैश्वमिति~। उत्तराषाढायोगतारानक्षत्रम् । आप्यार्धभोगगम् । आप्यस्य पर्वाषाढानक्षत्रस्यार्धभोगः । धनूराशेर्विंशतिभागस्तत्र स्थितं ज्ञेयम् । अष्टौ राशयो विंशतिभागा उत्तराषाढाया ध्रुव इत्यर्थः । एतेन पूर्वाषाढायोगतारायाः सकाशादुत्तराषाढायोगतारा विंशतिकलोनसप्तभागान्तरिता । तेन पूर्वाषाढाध्रुवकोऽष्टराशयश्चतुर्दश भागा विंशतिकलोन सप्तभागैर्युत उत्तराषाढाया ध्रुवश्चत्वारिंशत्कलाधिकोक्तध्रव इति पर्वतोक्तमपास्तम् । ब्रह्मसिद्धातविरोधात् । अभिजिद्ध्रुवकमाह\textendash आप्यस्येति~। पूर्वाषाढाया अवसाने धनूराशेर्विंशति कलोनसप्तविंशतिभागेऽभिजिद्योगतारा ज्ञेया । चत्वारिंशत्कलाधिकषड्विंशतिभागाधिका अष्टौ राशयोऽभिजितो ध्रुव इत्यर्थः । एवकारोऽन्ययोगव्यवच्छेदार्थः । ते संहितासम्मतं श्रव \textendash


{\tiny{2 G 2}}

\newpage

\noindent २३६ \hspace{4cm} सूर्यसिद्धान्तः 
\vspace{1cm}


\noindent णपञ्चदशांशस्थानं विंशतिविकलायुतत्रयोदशकलायुतचतुर्दशभागादिकनवराशयो निरस्तम् । श्रवणस्य ध्रुवकमाह\textendash वैश्वान्त इति~। उत्तराषाढाया अवसाने श्रवणयोगतारायाः स्थानं ज्ञेयम् । नवराशयो दशभागाः श्रवणध्रुवक इत्यर्थः । धनिष्ठाया ध्रुवकमाह\textendash त्रिचतुःपादयोरिति~। श्रवणस्य तृतीयचतुर्थचरणयोः क्रमेणान्तादिसन्धौ मकरराशेर्विंशतिभागेश्रविष्ठा धनिष्ठा ज्ञेया । नवराशयो विंशतिभागा धनिष्ठाध्रुव इत्यर्थः । तुकारात् क्षेत्रान्तर्गतधनिष्ठास्थानं कुम्भस्य विंशति कलोनसप्तभागा निरस्तम् । शतताराया भोगमाह\textendash स्वभोगत इति~। धनिष्ठाभोगात् कुम्भस्य विंशतिकलोनसप्तभागावधेरित्यर्थः । शतताराया अशीतिर्भोगः । अतः प्राग्वद्ध्रुवा इति ज्ञापनार्थं स्वभोगत इत्युक्तम् । शततारायाः स्थानं शततारकाध्रुव इति पर्यवमन्नम् । अवशिष्टनक्षत्राणां भोगानाह\textendash षट्कृतिरिति~। पूर्वाभाद्रपदायाः षट्त्रिंशत् कला भोगः । उत्तराभाद्रपदाया द्वाविंशतिः । रेवत्या एकोनाशीतिः । अथ ध्रुवकानयनं यथा । अश्विन्या भोगः ~।~४८~। दशगुणितः~।~४८०~। अतीतनक्षत्राभावाद्भोगयोजनाभावः । अतोऽश्विन्याः कलात्मकोध्रुवः~।~४८०~। राश्याद्यस्तु~।~८~। भरण्या भोगः~।~४०~। दशाहतः~।~४००~। अतीतनक्षत्रस्यैकत्वादष्टशतयुतो भरण्या परिभाषयाराशाद्यो ध्रुवः~।~०~।~२०~। एवमार्द्राभोगः~।~४~। दशहतः ~।~४०~। अतीतनक्षत्राणां पञ्चतया पञ्चगुणिताष्टशतेन~।~४०००~। चतुःसहस्रात्मकेन युतः कलाद्यो ध्रुवः~।~४०~।~४०~। राश्याद्यस्तु । 


\newpage

\hspace{3cm} गूढार्थप्रकाशकेन सहितः~। \hfill २३७ 
\vspace{1cm}


\noindent २~।~७~।~२०~। एवं पूर्वाषाढाया दशगुणितो भोगः~।~४०~। एकोनविंशतिगुणिताष्टशतेन~।~१५२००~। युतः परिभाषया राश्याद्यो ध्रुवः~।~८~।~१४~। शतताराया दशगुणितो भोगः~।~८००~। त्रयोविंशतिगीणिताष्टशतेन~।~९८४००~। युतश्चतुर्विंशति गुणिताष्टशतरूपो~।~१९२००~। जातो ध्रुवो राश्याद्यः~।~१०~।~२०~। पूर्वाभाद्रपदाया दशगुणितो भोगः~।~३६०~। चतुर्विंशतिगुणिताष्टशतेन~।~९९२००~। युतो~।~९९५६०~। जातो ध्रुवो राश्याद्यः~।~१०~।~२६~। उत्तराषाढाभिजिच्छ्रवणधनिष्ठानां स्वभोगस्थानात् पश्चात् स्थितत्वेनोक्तरीत्यसम्भवाद्भिन्नरीत्या ध्रुवका उक्ताः स्वादिस्थानाद्योगतारा यदन्तरकलाभिस्थितास्ता लाघवाद्दशापवर्तिता भोगसञ्ज्ञा उक्ताः । तथा च ब्रह्यसिद्धान्ते ।

%\vspace{2mm}

\begin{quote}
{\qt अष्टौ विंशतिरर्धेन गजाग्निर्व्यर्धखेषवः~।\\
त्रितर्काः सत्रिभागाद्रिरसास्त्यङ्काश्च षट्शतम्~॥

नवाशा नवसूर्याश्च वेदेन्द्राः शरबाणभूः~।\\
खात्यष्टिः खधृतिर्गोऽतिधृतिर्विश्वाश्विनस्तथा~॥

वेदाकृतिर्गोऽदृद्व्यस्ताः क्वब्धिहस्ता युगार्थदृक्~।\\
खोत्कृतिस्त्र्यंशहीनाश्च रसहस्ताः खहस्तिदृक् ~॥

खगोऽश्विनः खदन्ताः षड्दन्ताः शैलगणाग्नयः~।\\
मेषाश्व्यादिमध्यांशाः षडंशोनाः खषड्गुणाः~॥}
\end{quote}

इति । अथ नक्षत्राणां विक्षेपभागानाह\textendash एषामिति~। उक्तध्रुवकसम्बधिनामश्विज्यादिनक्षत्राणां यथाक्रमं क्रमादित्यर्थः। स्वात् स्वकीयापक्रमात् क्रान्त्यग्रात् क्रान्तिवृत्तस्थध्नुवकस्थानादि \textendash


\newpage

 \hspace{4cm} सूर्यसिद्धान्तः \hfill २३८
\vspace{1cm}


\noindent त्यर्थः । विक्षेपा विक्षेपभागा दक्षिणा उत्तरा वा भवन्ति । तत्रोत्तरदिश्यश्विन्यादित्रयाणां दिङ्मासविषयाः क्रमेण दश द्वादश पञ्चेत्थर्थः । दक्षिणदिशि रोहिण्यादित्रयाणां पञ्चदश नव । उत्तरस्यां पुनर्वसोः षड्भागाः । पुष्यस्य खं विक्षेपाभावः । अत्र पञ्चमाक्षरस्य गुरुत्वेन छन्दोभङ्ग आर्षत्वान्न दोषः । दक्षिणस्यामाश्लेषायाः सप्त । उत्तरस्यां मघादित्रयाणां शून्यं द्वादश त्रयोदश । दक्षिणस्यां हस्तचित्रयोरेकादश द्वौ । अनन्तरं खात्या उत्तरदिशि सप्तत्रिंशत् । दक्षिणस्यां विशाखादीनां षण्मासार्धैकत्रयं चत्वारः । नव सार्धपञ्च पञ्च क्रमेण । उत्तरदिशि तथा विक्षेपभागा अभिजितः षष्टिः । श्रवणस्य त्रिंशत् । धनिष्ठायाः षट्त्रिंशत् । एवकारो न्यूनाधिकव्यवच्छेदार्थः । चकारः पूरणार्थः । दक्षिणस्यां तुकारस्तथा । अर्धभागः शततारायाः । तुकारस्तथा । उत्तरस्यां पूर्वाभाद्रपदायाश्चतुर्विंशतिः । तस्यामेव दिशि भागा विक्षेपभागा उत्तराभाद्रपदायाः षड्विंशतिः । रेवत्या विक्षेपाभावः । चकारः पूरणार्थम् ॥~९~॥\\
\noindent अथागस्त्यलुब्धकवह्निब्रह्महृदयताराणां ध्रुवकविक्षेपांस्तदुपपत्तिं श्लोकत्रयेणाह \textendash

%\vspace{2mm}

\begin{quote}
{\ssi अशीतिभागैर्याम्यायामगस्त्यो मिथुनान्तगः~।\\
 विंशे च मिथुनस्यांशे मृगव्याधो व्यवस्थितः~॥~१०~॥

विक्षेपो दक्षिणे भागैः खार्णवैः स्वादपक्रमात्~।\\
हुतभुग्ब्रह्महृदयौ वृषे द्वाविंशभागगौ~॥~११~॥}
\end{quote}
\newpage


\hspace{3cm} गूढार्थप्रकाशकेन सहितः~। \hfill २३९
%\%vspace{1cm}

\begin{quote}
{\ssi अष्टाभिस्त्रिंशता चैव विक्षिप्ता उत्तरेण तौ~।\\
 गोलं बध्वा परीक्षेत विक्षेपं ध्रुवकं स्फुटम्~॥~१२~॥ }
\end{quote}

 स्वकीयात् क्रान्तिविभागस्थानाद्दक्षिणस्यामशीत्यंशैस्तारात्मकोऽगस्त्यो मिथुनान्तगः कर्कादिभागे स्थितः । अगस्त्यनक्षत्रस्य राशित्रयं ध्रुवकः । दक्षिणविक्षेपोऽशीतिरित्यर्थः । मृगव्याधोलुब्धको मिथुनराशेर्विंशतिभागे स्थितः । चकारः समुच्चये । लुब्धकनक्षत्रस्य राशिद्वयं विंशतिभागा ध्रुवक इत्यर्थः । दक्षिणस्यां चत्वारिंशता भागैः परिमितस्तस्य च क्रान्तिर्वृत्तस्थानाद्विक्षेपः ।वृषराशौ वह्निब्रह्महृदयौ द्वाविंशभागस्थितौ वह्निब्रह्महदयनक्षत्रयोर्द्वाविंशतिभागाधिकैकराशिर्ध्रुवकः । तौ वह्निब्रह्महृदयौ ।अष्टाभिस्त्रिंशता । चकारः क्रमार्थे । एवकारो न्यूनाधिकव्यवच्छेदार्थः । उत्तरेणोत्तरस्यामिस्यर्थः । विक्षिप्तौ विक्षेपयन्तौ । वह्नेर्विक्षेपोऽष्टभाग उत्तरः । ब्रह्महृदयस्योत्तरो विक्षेपस्त्रिंशदित्यर्थः । नन्वेते ध्रुवा विक्षेपाश्च कालक्रमेण नियता अनियतावेत्यत  आह\textendash गोलमिति~। गोलं वक्ष्यमाणं बध्वा वंशशलाकादिभिर्निर्बध्य स्फुटं विक्षेपं क्रान्तिसंस्कारयोग्यं धवाभिमुखं ध्रुवकं स्फुटमायनदृक्कर्मसंस्कृतं परीक्षेत । स्वस्वकाले दृग्गोचरसिद्धमङ्गीकुरुत । तथा च क्रान्तिंसंस्कारयोग्यविक्षेपायनसंस्कृतध्रुवकयोरयनांशवशादस्थिरत्वादपि मयेदानीन्तनसमयानुरोधेन लाघवार्थमायनदृक्कर्मसंस्कृता ध्रुवाः क्रान्तिसंस्कारयोग्यविक्षेपाच्च नियता उक्ताः । कालान्तरे गोलयन्त्रेण वेधसिद्धा ज्ञेयाः। नैत इति भावः । गोलयन्त्रेण वेधस्तु गोलबन्धोक्तविधिना गो\textendash



\newpage

\noindent २४० \hspace{4cm} सूर्यसिद्धान्तः
\vspace{1cm}


\noindent लयन्त्रं कार्यम् । तत्र खगोलस्योपरि भगोलमाधारवृत्तस्योपरि विषुवद्वृत्तम् । तत्र यथोक्तं क्रान्तिवृत्तं भगणांशाङ्कितंच बध्वा ध्रुवयष्टिकीलयोः प्रोतमन्यच्चलं भवेधवलयम् । तच्च भगणांशाङ्कितं कार्यम् । ततस्तद्गोलयन्त्रं सम्यग्धुवाभिमुख यष्टिकं जलसमक्षितिजवलयं च यथा भवति तथा स्थिरं कृत्वा रात्रौ गोलगध्यच्छिद्रगतया दृष्ट्या रेवतीतारां विलोक्य क्रान्तिवृत्ते मीनान्ताद्दशकलान्तरितपञ्चाद्भागं रेवतीतारायां निवेश्य मध्यगतयैव दृष्ट्याश्विन्यादेर्नक्षत्रस्य योगतारां विलोक्य तस्या उपरि तद्वेधवलयं निवेश्यम् । एवं कृते सति वेधवलयस्य क्रान्तिवृत्तस्य च यः सम्पातः स मीनान्तादग्रतो यावद्भिरंशैस्तावन्तस्तस्य नक्षत्रस्य ध्रुवांशा ज्ञेयाः । वेधवलये तस्यैव सम्पातस्य योगतारायाश्च यावन्तोऽन्तरेंऽशास्तावन्तस्तस्य विक्षेपांशादक्षिणा उत्तरा वा वेद्याः । अद्य कदम्बप्रोतवेधवलयेन वेधे तु सदा स्थिरा ध्रुवका आयनदृक्कर्मासंस्कृताः परन्तु कदम्बतारयोरभावादशक्यमिति यथोक्तवेधेनैवायनदृक्कर्मसंस्कृता ध्रुवाः शराश्च ध्रुवाभिमुखाः स्फुटाः सिद्धा भवन्तीति दिंक्~॥ ~९~॥ \\
\noindent अथ रोहिणीशकटभेदमाह \textendash

%\vspace{2mm}

\begin{quote}
{\ssi  वृषे सप्तदशे भागे यस्य याम्योंऽशकद्वयात्~।\\
 विक्षेपोऽभ्यधिको भिन्द्याद्रोहिण्याः शकटं तु सः~॥~१३~॥ }
 \end{quote}
%\vspace{2mm}

 वृषराशौ सप्तदशेंऽशे यस्य ग्रहस्य भागद्वयाद्धिको विक्षेपो दक्षिणः स ग्रहो रोहिण्याः शकटं शकटाकारसन्निवेशं


\newpage


\hspace{3cm} गूढार्थप्रकाशकेन सहितः~।\hfill २४१
\vspace{1cm}


\noindent भिन्द्यात् । तन्मध्यगतो भवेदित्यर्थः । तुकाराद्ग्रहविक्षेपो रोहिणीविक्षेपादल्प इति विशेषार्थकः । विक्षेपस्य दक्षिणस्य रोहिणीविक्षेपादधिकत्वे शकटाद्बहिर्दक्षिणभागे ग्रहस्य स्थितत्वेनतद्भेदकत्वाभावात् । अत्र शकटाग्रिमनक्षत्रस्य ध्रुव एकराशिः सप्तदशांशाः । दक्षिणः शरो भागद्वयमिति वेधसिद्धा स्पष्टायुक्तिः ॥ ९३ ॥\\
\noindent अथ ग्रहयोगसाधनार्थं ग्रहयोगसाधनरीत्यतिदेशमाह \textendash

%\vspace{2mm}

\begin{quote}
{\ssi ग्रहवद्युनिशे भानां कुर्यादृक्कर्म पूर्ववत्~।\\
ग्रहमेलकवच्छेषं ग्रहभुक्त्यां दिनानि च~॥~१४~॥ }
%\vspace{2mm}
\end{quote}

 ग्रहवद्द्युनिशे ग्रहाणां यथा दिनरात्रिमाने आक्षदृक्कर्मार्थं कृते तथा दिनमानरात्रिमाने भानां नक्षत्रध्रुवकाणामाक्षदृक्कर्मार्थं गणकः कुर्यात् । तदनन्तरं पूर्ववन्नक्षत्रनित्योदयास्तौ साधयित्वाभीष्टकाले दिनगतशेषाभ्यां नतं कृत्वा विषुवच्छाययाभ्यस्तादित्यादिनेत्यर्थः । दृक्कर्म कुर्यात् । अत्र नक्षत्रध्रुवके पर्वतेनायनदृक्कर्माप्युदाहरणे कृतं तदयुक्तम् । तस्य ध्रुवके स्वतः सिद्धचात् । तदनन्तरं शेषं नक्षत्रग्रहयुतिसाधभं ग्रहध्रुवतुल्यतारूपं ग्रहमेलकवद्ग्रहयोगसाधनरीत्या ग्रहान्तरकला इत्यादिना कार्यम् । ननु तत्र ।

%\vspace{2mm}

\begin{quote}
{\qt ग्रहान्तरकलाः खस्वभुक्तिलिप्तासमाहताः~।}\\
\end{quote}
%\vspace{2mm}

भुक्त्यन्तरेण विभजेदित्युक्तेर्नक्षत्रस्य का गतिर्ग्राह्येत्यत आह\textendash  ग्रहभुक्त्येति~। केवलया ग्रहगत्या ग्रहस्य फलं ग्रहध्रुवान्तररूपग्रहे संस्कार्यं ध्रुवसमो ग्रहो भवति । नक्षत्रस्य पूर्वगत्य \textendash

{\tiny{2 H}}

\newpage


\noindent २४२ \hspace{4cm} सूर्यसिद्धान्तः 
\vspace{1cm}


\noindent भावाद्ध्रुवो यथास्थित इत्यर्थः । ननु तथापि ग्रहनयत्रयुतिकालसाधनं भुक्त्यन्तरासम्भवात् कथं कार्यमिति मन्दाशङ्केत्यत आह\textendash दिनानीति~। अरीष्टसमयाद्विवरमित्वादिना केवलया ग्रहगत्याग्रहनक्षत्रयुतिदिनानि साध्यानि । चः समुच्चये । नक्षत्राणांगत्यभावात्~॥~९४~॥\\
\noindent अथाभीष्टकालाद्ग्रहनक्षत्रयुतिकालस्य गतैष्यत्वमसम्भ्रमार्थं पुनराह \textendash

%\vspace{2mm}

\begin{quote}
{\ssi एष्यो हीने ग्रहे योगो ध्रुवकादधिके गतः~।\\
 विपर्ययाद्वक्रगते* ग्रहे ज्ञेयः समागमः~॥~१५~॥ }
%\vspace{2mm}
\end{quote}

 नक्षत्रध्रुवादुक्ताद्ग्रह आयनदृक्कर्मसंस्कृतग्रह आक्षदृक्कर्मसंस्कृतनक्षत्रध्रुवकात् । दृक्कर्मद्वयसंस्कृतग्रह इति विवेकार्थः । न्यूने सति योगो नक्षत्रग्रहयोगः खाभीष्टसमयाद्भावी । अधिके सति पूर्वं जातः । वक्रगते ग्रहे विपर्ययादुक्तवैपरीत्यात् समागमो नक्षत्रग्रहयोगो ज्ञेयः । हीने ग्रहे गतोऽधिके ग्रह एष्यो योगः ।अत्रोपपत्तिर्नक्षत्रस्य गत्यभावेन सदा स्थिरत्वाद्ग्रहगमनेनैवयोगसम्भवादिति सुगमतरा ॥ १५ ॥ अथाश्विन्यादिनक्षत्रस्यबहुतारात्मकत्वात् कस्यास्ताराया एते ध्रुवका इत्यस्य योगताराया ध्रुवकमित्युत्तरं मनसि घृत्वाश्विन्यादिनक्षत्राणां योगतारां विवक्षुः प्रथममेषां नक्षत्राणां योगतारामाह । 

%\vspace{2mm}

\begin{quote}
{\ssi फाग्लुन्योर्भाद्रपदयोस्तथैवाषाढयोर्द्वयोः~।\\
 विशाखाश्विनिसौम्यानां योगतारोत्तरा स्मृता~॥~१६~॥}
\end{quote}
\noindent\rule{\linewidth}{.5pt}

\begin{center}
 * विपर्ययाद्वक्रगतौ शेत वा पाठः ।
\end{center}

\newpage

\hspace{3cm} गूढार्थप्रकाशकेन सहितः~। \hfill २४३
\vspace{1cm}


 एषामुक्तनक्षत्राणां प्रत्येकं स्वतारासु योत्तरदिक्स्था तारा सा योगतारा गोलतत्त्वज्ञैरुक्ता~॥~९६~॥\\
 \noindent अथान्ययोरनयोराह \textendash
%\vspace{2mm}
%{\setlength{\parindent}{5em
\begin{quote}
{\ssi पश्चिमोत्तरताराया द्वितीया पश्चिमे स्थिता~।\\ 
हस्तस्य योगतारा सा श्रविष्ठायाश्च पश्चिमा~॥~१७~॥ }
%\vspace{2mm}
\end{quote}

 हस्तनक्षत्रं पञ्चतारात्मकं हस्तपञ्चाङ्गुलiसन्निवेशाकारम् । तत्र नैरृत्यदिगाश्रितपश्चिमावस्थितताराया उत्तरदिगवस्थितताराया द्वितीया पूर्वोक्तातिरिक्ता पश्चिमे वायय्याश्रिते स्थिता सा हस्तस्य । योगतारा ज्ञेया । उत्तरतारासन्ना पश्चिमाश्रिता तारा हस्तस्य योगतारेति फलितार्थः । धनिष्ठाया योगतारामाह\textendash श्रविष्ठाया इति~। धनिष्ठायास्तारासु या पञ्चिमदिक्स्था सा तस्या योगतारा । चः समुच्चये~॥~९७~॥\\
\noindent अथान्येषामेषामाह \textendash

%\vspace{2mm}

 \begin{quote}
{\ssi ज्येष्ठाश्रवणमैत्राणं बार्हस्पत्यस्य मध्यमा~।\\
भरण्याग्नेयपित्र्याणां रेवत्याश्चैव दक्षिणा~॥~१८~॥}
%\vspace{2mm}
\end{quote}
 ज्येष्ठाश्रवणानुराधानां पुष्यस्य च प्रत्येकं तारात्रथात्मकत्वान्मध्यमतारा योगतारा स्यात् । भरणीकृत्तिकामघानां रेवत्याः । चः समुच्चये । प्रत्येकं स्वतारासु या दक्षिणदिक्स्था सा योगतारा~॥~९८~॥\\
\noindent  अथान्येषामेषामवशिष्टानां चाह \textendash
%\vspace{2mm}

 \begin{quote}
{\ssi  रोहिण्यादित्यमूलानां प्राची सार्पस्य चैव हि~।\\
यथा प्रत्यवशेषाणां स्थूला स्याद्योगतारका~॥~१९~॥}
\end{quote}
{\tiny{2 H 2}}

\newpage

\noindent २४४ \hspace{3.5cm} सूर्यसिद्धान्तः 
\vspace{2mm}


 रोहिणीपुनर्वसुमूलानामाश्लेषायाश्च प्रत्येकं स्वतारासु पूर्वदिक्स्था सैव योगतारेत्येवह्योरर्थः । प्रत्सवशेषाणामवशिष्टनक्षत्राणामार्द्राचित्रास्वात्यभिजिच्छतताराणां स्वतारासु यात्यन्तंस्थूला महती सा योगतारा स्यात्~॥~१९~॥ \\
\noindent  अथ ब्रह्मसञ्ज्ञकनक्षत्रावस्थानमाह \textendash


\begin{quote}
{\ssi पूर्वस्यां ब्रह्महृदयादंशकैः पञ्चभिः स्थितः~।\\
 प्रजापतिर्वृषान्तेऽसौ सौम्येऽष्टत्रिंशदंशकैः~॥~२०~॥ }
%\vspace{2mm}
\end{quote}

 ब्रह्महृदयस्थानात् पूर्वभागे पञ्चभिरंशैः प्रजापतिस्तारात्मको ब्रह्मा क्रान्तिवृत्ते स्थितः । कुत्रेत्यत आह\textendash वृषान्त इति~। वृषान्तनिकटे । एकराशिः सप्तविंशस्यंशा ब्रह्मध्रुवक इत्यर्थः । अस्य विक्षेपमाह\textendash असाविति~। ब्रह्मा । उत्तरस्यामष्टत्रिंशद्भागैः स्थितः । अष्टत्रिंशद्भागा अस्य विक्षेप इत्यर्थः~॥~२०~॥\\
\noindent  अथापांवत्सापयोस्तारयोरवस्थानमाह \textendash

%\vspace{2mm}

\begin{quote}
{\ssi अपांवत्सस्तु चित्रायामुत्तरेंऽशैस्तु पञ्चभिः~। \\
 बृहत् किञ्चिदतो भागैरापः षड्भिस्तथोत्तरे~॥~२१~॥}
%\vspace{2mm}
\end{quote}

 चित्रायाः सकाशादपांवत्ससञ्ज्ञकस्तारात्मकः पञ्चभिर्भागैरुत्तरस्यां स्थितः । प्रथमतुकारश्चित्राध्रुवतुल्यध्रुवकार्थकः । द्वितीयतुकारश्चित्राविक्षेपस्य दक्षिणभागद्वयात्मकत्वादपांवत्सविक्षेप उत्तरस्त्रिभाग इति स्फुटार्थकः । अतोऽपांवत्सात् किञ्चिदल्पान्तरेण बृहत् स्थूलतारात्मक आपसञ्ज्ञकः । तथापांवत्सात्षड्भिरंशैरुत्तरस्यां स्थितश्चित्राध्रुवक एवापस्य ध्रुवको विक्षेप \textendash


\newpage

\hspace{3cm} गूढार्थप्रकाशकेन सहितः~। \hfill २४५
\vspace{1cm}

उत्तरो नवांशा इत्यर्थः~॥~२१~॥\\
\noindent 
+अथाग्रिमग्रन्थस्यासङ्गतित्वनिरासार्थमधिकारसमाप्तिं फक्किकयाह \textendash 

\begin{center}
 इति नक्षत्रग्रहयुत्यधिकारः । 
\end{center}

\noindent स्पष्टम् । 
%\vspace{2mm}

 \begin{quote}
 {\ssi रङ्गनाथेन रचिते सूर्यसिद्धान्तटिप्पणे~।\\
 ग्रहर्क्षैक्याधिकारोऽयं पूर्णो ग्काढप्रकाशके~॥}
\end{quote}

 इति श्रसिकलगणकसार्वभौमबल्लालदैवज्ञात्मजरङ्गनाथगणकविरचिते गदार्थप्रकाशके नक्षत्रग्रहयुत्यधिकारः पूर्णः~॥ 

\vspace{3mm}

\noindent\rule{\linewidth}{.5pt}
\vspace{3mm}


 अथोदयास्ताधिकारो व्याख्यायते । ननु सूर्येणास्तमनंसहेति प्रागुक्तेर्ग्रहयुत्यधिकारानन्तरं नक्षत्रग्रहयुत्यधिकारात्प्रागेवोदयास्ताधिकारो निरूपणीय इत्यतोऽत्र तत्सङ्गतिप्रदर्शनार्थमादौ तदधिकारं प्रतिजानीते । 

%\vspace{2mm}

\begin{quote}
{\ssi अथोदयास्तमययोः परिज्ञानं प्रर्कीर्त्यते~।\\
दिवाकरकराक्रान्तमूर्तीनामल्पतेजसाम्~॥~१~॥ }
%\vspace{2mm}
\end{quote}

 अथ नक्षत्रयहयुत्यधिकारानन्तरं सूर्यकिरणाभिभूता मूर्तिर्बिम्बं येषां तेषां चन्द्रादिषड्ग्रहाणां नक्षत्राणां च । अत एवाल्पतेजसां न्यूनप्रभावतामुदयास्तसमययोः । अग्रिमकाले सूर्यादधिकासन्निहितसन्निहितत्वसम्भावनया क्रमेणोदयास्तयोः सूर्यान्निसृतस्य यस्मिन् काले यदन्तरेण प्रथमदर्शनं सम्भावितं



\newpage

\noindent २४६ \hspace{3cm} सूर्यसिद्धान्तः
\vspace{1cm}


\noindent स उदयः । सूर्याद्दूरस्थितस्य यस्मिन् काले यदन्तरेण प्रथमादर्शनं सम्भावितं सोऽस्तः । अनेन नित्योदयास्तव्यवच्छेदस्तयोरित्यर्थः । परिज्ञानं सूक्ष्मज्ञानप्रकारः प्रकीर्त्यते । अतिसूक्ष्मत्वेन मयोच्यत इत्यर्थः । तथा च ग्रह इत्युद्देशेऽस्तमनमुद्दिष्टमपि तस्य पूर्वमेव सूर्यासमत्व एव सम्भवात् तद्विलक्षणतयाग्रहयुतिप्रसङ्गेनोक्तम् । नक्षत्रग्रहयुतिस्तु ग्रहयुतिवदिति तदनन्तरमुक्ता । अतः प्रतिबन्धकजिज्ञासापगमेऽवश्यवक्तव्यत्वादस्यावसरष्टङ्गतित्वात् तत्सङ्गत्या नक्षत्रग्रहयुत्यधिकारानन्तरं प्रागुद्दिष्टमस्तमनं तत्प्रसङ्गादुदयश्च प्रतिपाद्यत इति भावः~॥~१~॥\\
\noindent तत्र प्रथमं पञ्चताराणां पश्चिमास्तपूर्वोदयावाह  \textendash

%\vspace{2mm}

\begin{quote}
{\ssi सूर्यादभ्यधिकाः पश्चादस्तं जविकुजार्कजाः~।
 ऊनाः प्रागुदयं यान्ति शुक्रज्ञौ वक्रिणौ तथा~॥~२~॥~}
\vspace{2mm}

 वक्रगती शुक्रबुधौ तथा । सूर्यादधिकौ पश्चिमास्तं गच्छतः । सूर्यादल्पौ पूर्वोदयं प्राप्नुतः । शेषं स्पष्टम्~॥~२~॥\\
 \noindent अथ चन्द्रबुधशुक्राणां पूर्वास्तपश्चिमोदयावाह \end{quote}
%\vspace{2mm}

 \begin{quote}
{\ssi  ऊना विवस्वतः प्राच्यामस्तं चन्द्रज्ञभार्गवाः~।\\
व्रजन्त्यभ्यधिकाः पश्चादुदयं शीघ्रयायिनः~॥~३~॥}
%\vspace{2mm}
\end{quote}
 शीघ्रयायिनः सूर्यगत्यधिकगतय इत्यर्थः । एतेन बुधशुक्रावर्कगत्यल्पगती सूर्यादल्पौ पर्थस्तमधिकौ च पश्चिमोदयं न प्राप्नुत इत्युक्तम् । शेषं स्पष्टम् । अत्रोपपत्तिः । रविगतितोऽल्यगतिर्ग्रहोऽर्कादूनश्चेत् प्राच्यां दर्शनथोग्यो भवितुमर्हति।


\newpage

 \hspace{3cm} गूढार्थप्रकाशकेन सहितः~। \hfill २४७ 
\vspace{1cm}


\noindent यतः सूर्यास्याधिकत्वेन बहुगतित्वाच्चोत्तरोत्तरमधिकवप्रकर्षात् प्रवहवशेन न्यूनस्य पूर्वमुदया दधिकस्यानन्तरमुदयनियमात् । ग्रहस्य क्रान्तिजसंलग्नताकालानन्तरं यावत् सूर्यस्यतादृशः कालस्तावत्पर्यन्तं विप्रकर्षे दर्शनसम्भवात् । एवं यदाल्पगतिः सूर्यादधिकस्तदा प्रवहवशे नार्कस्य पूर्वमुदयानन्तरमुदितग्रहस्य दर्शनासम्भवात् प्रवहवशेनादौ न्यूनानार्कस्यास्तसम्भवादनन्तरमधिकग्रहस्यास्तसम्भ वात् सूर्यास्तानन्तरं पश्चिमभागे ग्रहदर्शभसम्भवे ऽप्यधिकगतिसूर्यस्य पृष्ठस्थितत्वेनोत्तरोत्तरमधिकसन्निकर्षात् पश्चिमा यामदर्शनं सम्भवत्येव । ते तु भौमगुरुशनयः । वक्रत्वे न्यूनगतित्वाद्बुधशुक्रौ चेति । अथार्कगतितोऽधिकगतिग्रहः सूर्यादूनस्तदोक्तरीत्योत्तरोत्तरमधिकसन्निकर्षात् पूर्वस्मिन्नदर्शनं याति । यदा सूर्यादधिकस्तदोक्तरीत्योत्तरोत्तरमधिकविप्रकर्षात् पश्चिमायामुदयः । ते तु शीघ्राश्चन्द्रबुधशुक्रा इत्युपपन्नमुक्तम्~ ॥~३~॥ \\
\noindent अथाभीष्टदिनआसन्ने सूर्योदयास्तकालिकौ सूर्यदृग्ग्रहौ तत्कालज्ञानार्थंकार्यावित्याह \textendash

%\vspace{2mm}

\begin{quote}
{\ssi सूर्यास्तकालिका पश्चात् प्राच्यामुदयकालिकौ~।\\
दिवाकरग्रहौ कुर्याद्दृक्कमाथ ग्रहस्य तु~॥~४~॥ }
%\vspace{2mm}
\end{quote}
 पश्चात् पश्चिमास्तोदयसाधनेऽभीष्टादिन आसन्ने सूर्यग्रहौसूर्यास्तकालिकौ कुर्याद्गणकः । पूर्वास्तोदयसाधने सूर्योदयकालिकौ कुर्यात् । दिनेऽभीष्टकाले कुर्यात् । चकारो विक \textendash


\newpage

\noindent २४८ \hspace{4cm} सूर्यसिद्धान्तः 
\vspace{1cm}


\noindent ल्पार्थकः । अनन्तरं ग्रहस्य दृक्कर्म । आयनाक्षदृक्कर्मद्वयं कुर्यान्। तुकार आक्षदृक्कर्मश्लोकपूर्वार्धोक्तमिति विशेषार्थकः । अत्रोपपत्तिः । पश्चादस्तोदयसाधने पश्चिमायां तद्दर्शनमिति सूर्यास्तकालिकौ सूर्यग्रक्षवशिष्टकालांशसाधनार्थं सूक्ष्मौ । पर्वोदयास्तसाधने पूर्वदिशि तद्दर्शनमिति सूर्योदयकालिकौ सूर्यग्रहावशिष्टकालांशसाधनार्थं सूक्ष्मावन्यकाले तु किञ्चित्स्थूलावपि कृतौ दृक्कर्मसंस्कृतग्रहस्य सूर्यवत् क्षितिजसंलग्नतायोग्यत्वादृक्कर्मसंस्कृतो ग्रहः कार्य इति~॥~४~॥ \\ 
\noindent अथेष्टकालांशानयनमाह । 

%\vspace{2mm}

\begin{quote}
{\ssi ततो * लग्रान्तरप्राणाः कालांशाः षष्टिभाजिताः~।\\
 प्रतीच्यां षड्भयुतयोस्तद्वल्लग्नान्तरासवः~॥~५~॥ }
%\vspace{2mm}


 ततस्तार्भ्या सूर्यदृग्ग्रहाभ्यां लग्नान्तरप्राणाः । भोग्यासूनूनकमकस्याथेत्युक्तप्रकारेणान्तरकालासवः षष्टिभक्ता इष्टाः कालांशा भवन्ति । प्रागुदयास्तसाधने प्रतीच्यां पश्चिमोदयास्तसाधने षड्भयुतयोः षड्राशियुक्तयोः सूर्यदृग्ग्रहयोर्लग्नान्तरासवः । अन्तरासवस्तद्वत् षष्टिभक्ता इष्टकालांशा भवन्तीत्यर्थः । अत्रोपपत्तिः । दृग्ग्रहसूर्याभ्यामन्तरकालो ग्रहस्य सूर्योदयकाले दिनगतं पूर्वोदयास्तनिमित्तमुपधुक्तम् । एवंपश्चिमोदयास्तीयनिमित्तं सूर्यदृग्ग्रहाभ्यामस्तकालासुभिरन्तर' कालः सूर्यास्तकाले ग्रहस्य दिनशेषकाल उपयक्तः । तत्रास्कालानामनुक्तेरुदयासुभिः साधनार्थं सूर्यदृग्ग्रहौ कृतौ स \textendash


\noindent\rule{\linewidth}{.5pt}
\begin{center}
 * द्वयोर्लग्नान्तरप्राणाः इति वा पाठः ।
\end{center}

\newpage

\hspace{3cm} गूढार्थप्रकाशकेन सहितः~। \hfill २४९
\vspace{1cm}


\noindent कालोऽखात्मकः । अहोरात्रासुभिश्चक्रकलातुल्यैश्चक्रांशा लभ्यन्ते तदेष्टासुभिः क इत्यनुपाते प्रमाणफलयोः फलापवर्तनेन हरस्थाने षष्टिः । अतोऽस्तात्मकान्तरकालः षष्टिभक्त इष्टकालांशा इत्युपपन्नमुक्तम् । अत्रेदमवधेयम् । सूर्योदयकालिकाभ्यामर्कदृग्ग्रहाभ्यामानीतेन दिनगतेन पूर्वं चाल्यो दृग्ग्रहः । सूर्यास्तकालिकाभ्यां सषड्भाभ्यामर्कदृग्ग्रहाभ्यामानीतेन दिनशेषेणाग्रे चाल्यः सषड्भो दृग्ग्रहः । क्रमेण ग्रहोदयास्तकाले प्राक्पश्चिमदृग्ग्रहौ भवतः । ताभ्यां सूर्यसषड्भसूर्याभ्यां च क्रमेण पूर्वरीत्यान्तरकालो ग्रहस्य सूर्योदयास्तकाले क्रमेण दिनगतशेषौ नाक्षत्रौ षष्टिभक्तौ कालांशविष्टौ सूक्ष्मौ । अथेष्टकालिकाभ्यगभानीतकालेन पर्वूवच्चालिताभ्यां प्राक्पश्चिमदृग्ग्रहाभ्यां सूर्यसषड्भसूर्याभ्यां चानीतकालो नाक्षत्रोऽपि सूक्ष्मासन्नः । सूर्योदयास्तसम्बन्धाभावात् तदुत्पन्ना कालांशा अपि तथा ।अथ सूर्योदयास्तकालिकाभ्यामानीतैकवार कालात् कालांशा अतिस्थूला उभयत्र कालस्य सावनत्वात् । न हि सावनषष्टिघटीभिश्चक्रपरिपूर्तिर्येन सूक्ष्माः सिद्ध्यन्तीति~॥~५~॥ \\
\noindent अथ यैः कालांशैरुदयोऽस्तो वा भवति तान् विवक्षुः प्रथमं गुरुशनिभौमानांकालांशानाह \textendash

%\vspace{2mm}

\begin{quote}
{\ssi एकादशामरेज्यस्य तिथिसङ्ख्यार्कजस्य च~। \\
अस्तांशा भूमिपुत्रस्य दशसप्ताधिकास्ततः~॥~६~॥}
\end{quote}
{\tiny{2 I}}


\newpage

\noindent २५० \hspace{4cm} सूर्यसिद्धान्तः 
\vspace{1cm}


 तत इष्टकालांशावगमानन्तरमस्तांशाः । अस्तो यैरंशैर्भवतितेंऽशा अस्तोपलक्षणादुदयांशा ज्ञेयाः । अमरेज्यस्य गुरोरेकादश कालांशाः । शनेः पञ्चदश सङ्ख्या कालांशानाम् । चः समुच्चये ।भौमस्य सप्ताधिका दश सप्तदश कास्त्रांशा इत्यर्थः~॥~६~॥
\noindent अथ शुक्रस्याह \textendash

%\vspace{2mm}

\begin{quote}
{\ssi पश्चादस्तमयोऽष्टाभिरुदयः प्राङ्महत्तया~।\\
 प्रागस्तमुदयः पश्चादल्पत्वाद्दशभिर्मृगोः~॥~७~॥}
%\vspace{2mm}
\end{quote}

 शुक्रस्य महत्तया वक्रत्वेन नीचासन्नत्वात् स्थूलबिम्बतयापश्चिमायामस्तोऽष्टाभिः कालांशैः प्राच्यामुदयश्च तैः । नाधिकैः । प्राच्यां शुक्रस्याल्पत्वादणुबिम्बत्वाद्दशभिः कालांशैरस्तं गणकः कुर्यात् । नाल्पैः । पश्चिमायामुदयस्तस्याणुबिम्बस्यदशभिः कालांशैरेव ज्ञेयः~॥~७~॥\\
 \noindent अथ बुधस्याह ।\textendash

%\vspace{2mm}

\begin{quote}
{\ssi एवं बुधो द्वादशभिश्चतुर्दशभिरंशकैः~।
 वक्री शीघ्रगतिश्चार्कात्* करोत्यस्तमयोदयौ ~॥~८~॥}
%\vspace{2mm}
\end{quote}

 वक्री शीघ्रगतिः । चः समुच्चये । बुधः सूर्याद्द्वादशभिश्चतुर्दशभिश्च कालांशैरस्तोदयौ । एवं शुक्ररीत्याः करोति । यश्चादस्तं प्रागुदयं च द्वादशभिः कालांशैर्महाबिम्बतया बुधः करोति । प्रागस्तं पश्चादुदयं च चतुर्दशभिः कालांशैरणुबिम्ब \textendash

%\vspace{3mm}
\end{quote}
\noindent\rule{\linewidth}{.5pt}

\begin{center}
 * वक्रशीघ्रगतिश्चार्कात् इति पाठान्तरम् ।
\end{center}

\newpage


\hspace{3cm} गूढार्थप्रकाशकेन सहितः~।\hfill  २५१
\vspace{1cm}


\noindent त्वाद्बुधः करोतीत्यथेः~॥~८~॥ \\
\noindent अथ प्राक्तेष्टकालांशाभ्यामस्तस्योदयस्य वा गतैष्यत्वज्ञानमाह \textendash
%\vspace{2mm}

\begin{quote}
{\ssi एभ्योऽधिकैः कालभागैर्दृश्या न्यूनैरदर्शनाः~।\\
 भवान्ति लोके खचरा भानुभाग्रस्तमूर्तयः~॥~९~॥ }
%\vspace{2mm}
\end{quote}

 एभ्य एकादशामरेज्यस्येति श्लोकत्रयोक्तेभ्योऽधिकैरिष्टकालांशैर्दृश्या दर्शनयोग्या अभीष्टकाले ग्रहा भवन्ति । तथा चास्तसाधने दृश्यत्व अस्त एष्यः । उदयसाधने दृश्यत्व उदयोगत इति भावः । अल्पैरिष्टकालांशैर्ग्रहा लोके भूलोके अदर्शना न विद्यते दर्दनं दृष्टिगोचरता येषां ते । अदृश्या अभीष्टकाले भवन्ति । नन्वदृश्याः कुतो भवन्तीत्यत आह\textendash भानुभाग्रस्तमूर्तय इति~। सूर्यासन्नत्वेन सूर्यकिरणदीप्त्या ग्रस्ता अभिभूता सूर्यकिरणप्रतिहतलोकगयनाविषया मूर्तिर्बिम्बस्वरूपं येषां त इत्यर्थः । तथा चास्तसाधन अदृश्यत्वेऽस्तो गतः । उदयसाधनेऽदृश्यत्व उदय एष्य इति भावः । अत एव । 

%\vspace{2mm}

\begin{quote}
{\qt उक्तेभ्य ऊनाभ्यधिका यदीष्टाः \\
खेटोदयो गम्यगतस्तदा स्यात्~। \\
अतोऽन्यथा चास्तमयोऽवगम्यः \\}
%\vspace{2mm}
\end{quote}

इति भास्कराचार्योक्तं सङ्गच्छते । अत्रोपपत्तिः । उक्तकालांशतुल्येष्टकालांशे यत्काले ग्रहौ साधितौ तत्काल एव ग्रहस्योदयोऽस्ता वार्ककृतः । उक्तकालांशानां सूर्यसान्निध्यजनिताद्यन्तग्रहादर्शने हेतुत्वप्रतिपादनात् । तथा चेष्टकालांशा उक्तेभ्योऽल्पास्तदा ग्रहस्यास्तङ्गतत्वमेवेत्युदयसाधन इष्टकालांशा


{\tiny{2 I 2}}

\newpage


\noindent २५२ \hspace{4cm} सूर्यसिद्धान्तः
\vspace{1cm}


\noindent उक्रेभ्योऽल्पास्तदेष्टकालादग्रे ग्रहस्योदयः । यदीष्टकालांशाउक्तेभ्योऽधिकास्तदेष्टकालाद्ग्रहस्योदयः पूर्वं जातः । एवमस्त साधन इष्टकालांशा अधिकास्तदेष्टकालादग्रे ग्रहास्तः । यदीष्टकालांशा न्यूनास्तदेष्टकालात् पूर्वं ग्रहास्तो जात इत्युपपन्नमुक्तम्~॥~९~॥ \\
\noindent अथोदयास्तयोर्गतैष्यदिनाद्यानयनमाह \textendash 


\begin{quote}
{\ssi तत्कालांशान्तरकला भुक्त्यन्तरविभाजिताः~। \\
दिनादि तत्फलं लब्धं भुक्तियोगेन वक्रिणः~॥~१०~॥}
%\vspace{2mm}
\end{quote}

 उक्तेष्टकालांशयोरन्तरस्य कलाः सृर्यग्रहयोर्गत्योः कलात्मकान्तरेण भक्ताः । दिनादिकमुदयास्तयोः फलमुदयास्तयोर्गतैष्यदिनाद्यं भवतीत्यर्थः । वक्रगतिग्रहस्य विशेषमाह\textendash लब्धमिति~। वक्रिणो वक्रग्रहस्य भुक्तियोगेन सूर्यग्रहयोः कलात्मकगतियोगेन भक्ताः फलं गतैष्यदिनाद्यं ज्ञेयम् । अचोपपत्तिः । सूर्यग्रहयोर्गत्यन्तरकलाभिरेकं दिनं तदेष्टप्रोक्तकालांशयोरन्तरकलाभिः किमित्यनुपातेनोदयास्तयोरभीष्टकालाद्गतैष्यदिनाद्यवगमः । वक्रग्रहे तु सूर्यग्रहयोर्गतियोगेन प्रत्यहमन्तरवृद्धेर्गतियोगादनुपात उपपन्न इत्युपपन्नमुक्तम्~॥~९०~॥\\
\noindent अथ ग्रहगतिकलयो क्रान्तिवृत्तस्थत्वात् कालांशान्तरस्याहोरात्रवृत्तस्थत्वाच्चानुपातः प्रमाणेच्छयोर्वैजात्येनायुक्त इति मनसि धृत्वा तयोरेकजातित्वसत्पादनार्थं ग्रहगत्योरिच्छाजातीयत्वं वदंस्तदन्तरेणानुपातस्तु युक्त एवेत्याह \textendash

%\vspace{2mm}

\begin{quote}
{\ssi तल्लग्नासुहते भुक्ती अष्टादशतोद्धृते~।\\ 
स्यातां कालगती ताभ्यां दिनादि गतगम्ययोः~॥~११~॥}
\end{quote}
\newpage

\hspace{3cm} गूढार्थप्रकाशकेन सहितः~। \hfill २५३
\vspace{1cm}


 भुक्ती रविग्रहयोर्गती कलात्मके तल्लग्नासुहते कालसाधनार्थं ग्रहस्य यो राश्युदयो गृहीतस्तेनास्वात्मकोदयेन गुणित अष्टादशशतेन भक्ते फले सूर्यग्रहयोः कालांशवत् कालगतीस्याताम् । ताभ्यां गतिभ्यां गतगम्ययोरुदयास्तयोर्दिनादि पूर्वोक्तप्रकारेण साध्यम् । न तु पूर्वोक्तप्रकारेण यथास्थितगतिभ्यां स्थूलत्वापत्तेः । अत्रोपपत्तिः । एकराशिकलाभी राश्युदयासवस्तदा गतिकलाभिः क इत्यनुपातेनाहोरात्रवृत्ते गत्यसवः कलासमा इत्युपपन्नमुक्तम्~॥~९९~॥\\
 \noindent अथ नक्षत्राणांसूर्यसान्निध्यवशादस्तोदयज्ञानार्थं कालांशान् विवक्षुः प्रथममेषामाह \textendash

%\vspace{2mm}

\begin{quote}
{\ssi स्वात्यगस्त्यमृगव्याधचित्राज्येष्ठाः पुनर्वसुः~।\\
 अभिजिद्ब्रह्महृदयं त्रयोदशभिरंशकैः~॥~१२~॥ }
%\vspace{2mm}
\end{quote}
 मृगव्याधो लुब्धकः । त्रयोदशभिः कालांशैर्दृश्यादृश्यानिनक्षत्राणि भवन्ति । शेषं स्पष्टम्~॥~१२~॥ \\
 \noindent अथान्येषामेषामाह \textendash
%\vspace{2mm}

\begin{quote}
{\ssi हस्तश्रवणफाल्गुन्यः श्रविष्ठारोहिणीमघाः~। \\
चतुर्दशांशकैर्दृश्या विशाखाश्विनिदैवतम्~॥~१३~॥}
%\vspace{2mm}
\end{quote}

 फालुानी पूर्वोत्तराफाग्लुनीद्वयम् । अश्विनिदैवतमश्विनोकुमारो दैवतं स्वामी यस्येत्यश्विनीनक्षत्रम् । दृश्या उपलक्षणाददृश्या अपि । लिङ्गपरिणामश्च यथायोग्यं बोध्यः । शेषं स्पष्टम्~॥~१३~॥ \\
\noindent अथान्येषामेषामाह \textendash

\newpage


\noindent २५४ \hspace{4cm} सूर्यसिद्धान्तः 
%\vspace{1cm}

\begin{quote}
{\ssi कृत्तिकामैत्रमूलानि सार्पं रौद्रक्षमेव च~।\\
 दृश्यन्ते पञ्चदशभिराषाढाद्वितयं तथा~॥~१४~॥ }
%\vspace{2mm}
\end{quote}

 कृत्तिकानुराधामूलनक्षत्राणि पञ्चदशभिः कालांशैर्दृश्यन्ते । उपलक्षणान्न दृश्यन्तेऽपि । एवकारो न्यूनाधिकव्यवच्छेदार्थः ।आश्लेषार्द्रा । चः समुच्चये । आषाढाद्वितयं पूर्वोत्तराषाढाद्वयं तथा पञ्चदशकालांशैर्दृश्यन्त इत्यर्थः~॥~९४~॥\\ 
 \noindent अथान्येषामवशिष्टानां चाह \textendash 
%\vspace{2mm}

 \begin{quote}
 {\ssi भरणीतिष्यसौम्यानि सौक्ष्म्यात् त्रिसप्तकांशकैः~। \\
 शेषाणि सप्तदशभिर्दृश्यादृश्यानि भानि तु~॥~१५~॥ }
%\vspace{2mm}
\end{quote}
 तिष्यः पुष्यः सोमदैवतं मृगशिरो नक्षत्रमेतानि नक्षत्राणि सौक्ष्म्यादणुबिम्बत्वात् त्रिःसप्तकांशकैरेकविंशतिकालांशैर्दृश्यादृश्यानि । उदितान्यस्तङ्गतानि च भवन्तीत्यर्थः । शेषाणिपूर्वाधिकारोक्तनक्षत्रेषूक्तातिरिक्तानि शततारापूर्वोत्तराभाद्रपदारेवतीसञ्ज्ञानि । वह्निब्रह्मापांवत्सापसञ्ज्ञानि च सप्तदशभिः कालांशैर्दृश्यादृश्यानि भवन्ति । तुकारो दृश्यादृश्यानीत्यत्र समुच्चयार्थकः~॥~१९५~॥\\
\noindent अथ दिनाद्यानयनार्थमिच्छायाएव प्रमाणजातीयकरणत्वमाह \textendash
%\vspace{2mm}

\begin{quote}
{\ssi अष्टादशशताभ्यस्ता दृश्यांशाः स्वोदयासुभिः~। \\
विभज्य लब्धाः क्षेत्रांशास्तैर्दृश्यादृश्यताथवा~॥~१६~॥}
%\vspace{2mm}
\end{quote}
 दृश्यांशाः कालांशा अष्टादशशतगुणितास्तान् स्वोदयासुभिर्ग्रहराश्युदयासुभिर्भक्ता लब्धाः क्षेत्रांशाः क्रान्तिवृत्तस्यां \textendash  


\newpage


\hspace{3cm} गूढार्थप्रकाशकेन सहितः~। \hfill २५५
\vspace{1cm}


\noindent शास्तैरंशैर्दृश्यादृश्यता । उदयास्तौ प्रकारान्तरेणोक्तरीत्या ज्ञेयौ । कालांशाभ्यां क्षेत्रांशावानीय -तदन्तरकला यथास्थितगत्योरन्तरेण योगेन वा भक्ताः फलमुदयास्तयोर्गतैष्यदिनाद्यं पूर्वागतमेव स्यादित्यर्थः । अत्रोपपत्तिः । राश्युदयासुभिरेकराशिकलास्तदा कालांशकलातुल्यासुभिः का इति क्रान्तिवृत्ते कलास्ताः षष्टिभक्ता अंशा इति पूर्वमेवेच्छास्थाने कालांशा एव धृता लाघवात् । इत्युक्तमुपपन्नम्~॥~६६~॥\\
\noindent ननु ग्रहाणाममुकदिश्यस्तोऽमुकदिश्युदय इत्युक्तम्। तथा नक्षत्राणां नोक्तम्।गत्यभावाद्वियोगयोगासम्मवेन गतैष्यदिनाद्यानयनासम्भवश्चेत्यत आह \textendash
%\vspace{2mm}

 \begin{quote}
{\ssi  प्रागेषामुदयः पश्चादस्तो दृक्कर्म पूर्ववत्~।\\ 
गतैष्यदिवसप्राप्तिर्भानुभुक्त्या सदैव हि~॥~१७~॥}
%\vspace{2mm}
\end{quote}
 एषां नक्षत्राणां प्राच्यामुदयः प्रतीच्यामस्तो गत्यभावादल्पगतिग्रहवत् । एषां नक्षत्राणां दृक्कर्माक्षदृक्कर्म पूर्ववत् पूर्वप्रकारेणकार्यम् । परन्तु श्लोकपूर्वार्धोक्तमिति ध्येयम् । सदा नित्यम्। एवकारात् कदाचिदप्यन्यथा नेत्यर्थः । हि निश्चयेन। रविगत्यागतैष्यदिवसानां लब्धिः स्यात् । नक्षत्रगत्यसम्भवात् । योगे ग्रहगतिवत्~॥~९७~॥\\
 \noindent अथ कतिपयानां नक्षत्राणां सूर्यसान्निध्यवशादस्तो नास्तीत्याह \textendash
%\vspace{2mm}

\begin{quote}
{\ssi अभिजिद्ब्रह्महृदयं स्वातीवैष्णववासवाः~।\\
 अहिर्बुध्न्यमुदक्स्थत्वान्न लुप्यन्तेऽर्करश्मिभिः~॥~१८~॥}
%
\end{quote}
\newpage


\noindent २५६ \hspace{4cm} सूर्यसिद्धान्तः
\vspace{1cm}


 अभिजित् । ब्रह्महृदयम् । अनेनैकदेशस्य ब्रह्मणोऽपि ग्रहणम् । स्वातीश्रवणधनिष्ठाः । अहिर्बुध्न्यमुत्तराभाद्रपदा । एतानि नक्षत्राण्युत्तरदिक्स्थत्वादुत्तरविक्षेपाधिक्यादित्यर्थः । सूर्यकिरणैर्न लुप्यन्ते । अस्त्रं न यान्तीत्यर्थः । अत्रोपपत्तिः| 
%\vspace{2mm}

 \begin{quote}
 {\qt यस्योदयार्कादधिकोऽस्तभानुः \\
प्रजायते सौम्यशरातिदैर्घ्यात्~।\\
तिग्मांशुसान्निध्यवशेन नास्ति \\
धिष्णस्य तस्यास्तमयः कथञ्चित्~॥\\ }
%\vspace{2mm}
\end{quote}

 इति भास्कराचार्योक्ता । परमिदमुक्तमष्टाक्षभायाम् । अन्यथा पूर्वाभाद्रपदाया अपि तथात्वापत्तेरिति दिक्~॥~९८~॥ \\
\noindent अथाग्रिमग्रन्थस्यासङ्गतित्वनिरासार्थमधिकारसमाप्तिं फक्किकयाह \textendash

\begin{center}
 इत्युदयास्ताधिकारः~॥ 
\end{center}

 नक्षत्रग्रहयोरस्तोदयभिरूपणात् साधारण्येनोदयास्माधिकार इत्युक्तम् ।
%\vspace{2mm}

\begin{quote}
{\qt रङ्गनाथेन रचिते सूर्यसिद्धान्तटिप्पणे~।\\
उदयास्ताधिकारोऽयं पूर्णो गूढप्रकाशके~॥}
%\vspace{2mm}
\end{quote}
 इति श्रीसकलगणकसार्वभौमबल्लालदैवज्ञात्मजरङ्गनाथगणकविरचिते गूढार्थाप्रकाशक उदयास्ताधिकारः पूर्णः~॥



\begin{center}
\rule{8em}{.5pt}
\end{center}

\newpage


\hspace{3cm} गूढार्थप्रकाशकेन सहितः~।\hfill २५७
\vspace{1cm}


 अथ भौमादीनां सूर्यसान्निध्योदयास्तासन्ने दीप्त्या सकलबिम्बदर्शनं तथा चन्द्रस्य स्वोदयास्तकाले सकलबिम्बदर्शनं शुक्लत्वेन न भवति । किन्तु बिम्बैकदेश एव शुक्लत्वेन दृश्यत इति भौमादिविसदृशत्वं चन्द्रस्य कुत इत्याशङ्कायाः पूर्वाधिकारे समुपस्थितेस्तदुत्तरभूतशृङ्गोन्नमनाधिकारोऽवश्यमुपस्थित आरब्धो व्याख्यायते । तत्र शृङ्गोन्नतेरुदयकालात् पूर्वकालेऽस्तकालान्तरकाले चासन्नकतिपयदिवसेषु दर्शनात् पूर्वाधिकारेचन्द्रस्य कालांशानुक्त्या तदुदयानुक्तेश्चप्नथममुपस्थितचन्द्रोदयास्तयोः साधनमतिदिशति\textendash 
%\vspace{2mm}

 \begin{quote}
{\ssi उदयास्तविधिः प्राग्वत् कर्तव्यः शीतगोरपि~। \\ 
भागैर्द्वादशभिः पश्चाद्दृश्यः प्राग्यात्यदृश्यताम्~॥~१~॥}
%\vspace{2mm}
\end{quote}
 चन्द्रस्य । अपिशब्दः पूर्वाग्धिकारोक्तैर्ग्रहनक्षत्रैः समुच्चयार्थकः । उदयास्तविधिरुदयास्तयोः साधनप्रकारः प्राग्वत् पूर्वाधिकारोक्तरीत्या गणकेन कार्यः । ननु कालांशानां पूर्वमनुक्तेः कथं तत्सिद्धिरत आह\textendash भागैरिति~। द्वादशभिशैश्चन्द्रः पश्चिमायां दृश्य उदितो भवति। । प्राच्यामदृश्यतामस्तं प्राप्नोति । अत्र पश्चात् प्रागिति पुनरुक्तमपि पूर्वं बुधशुक्रयोः साहचर्येण चन्द्रोदयास्तदिगुक्त्या तत्साहचर्येण चन्द्रस्य पश्चिमास्तर्पूर्वोदयौवर्तेते इति कस्यचिन्मन्दबुद्धेर्भ्रमस्य वारणायेति ध्येयम्~॥~१~॥\\
\noindent अथोदयास्तप्रसङ्गेन स्मृतयोश्चन्द्रनित्यास्तोदययोः साधनं विवक्षुः प्रथमं श्लोकत्रयेणेन्दोर्नित्यास्तसाधनमाह \textendash


{\tiny{2 k}}

\newpage

\noindent २५८ \hspace{4cm} सूर्यसिद्धान्तः 
%\vspace{1cm}

 \begin{quote}
{\ssi  रवीन्द्वोः षड्भयुतयोः प्राग्वल्लग्नान्तरासवः~।\\
 एकराशौ रवीन्द्वोश्च कार्या विवरलिप्तिकाः~॥~२~॥

तन्नाडिकाहते भुक्ती रवीन्द्वोः षष्टिभाजिते~।\\
तत्फ्लान्वितयोर्भूयः कर्तव्या विवरासवः~॥~३~॥

एवं यावत् स्थिरीभूता रवीन्द्वोरन्तरासवः~।\\
तैः प्राणैरस्तमेतीन्दुः शुक्लेऽर्कास्तमयात् परम्~॥~४~॥ }
%\vspace{2mm}
\end{quote}

 शुक्ले शुक्लपक्षाभीष्टदिने सूर्यास्तकाले स्पष्टौ सूर्यचन्द्रौ साध्यौ । चन्द्रस्य दृक्कर्मद्वयं संस्कार्यम् । तत्राक्षदृक्कर्म श्लोक पूर्वार्धोक्तमेव । तयोः सूर्यचन्द्रयोः षड्राशिप्तियुतयोर्लग्नान्तरासवोऽन्तरकालासवः प्राग्वद्भोग्यासूनूनकस्येत्यादिना साध्याः । तौ सषड्भार्कचन्द्रावेकराशावभिन्नराशौ चेत् स्तस्तदा सषड्भयोस्तयोः सूर्यचन्द्रयोरन्तरकलाः कार्याः । चकारो विषयव्यवस्थर्थकः । तयोरसुकलयोर्घटिकाभिरसवः षष्ट्यधिकशतत्रयेण भाज्याः । घटिकाः कला उदयासुगुणिता एकराशिकलाभिर्भक्ता असवस्ते षष्ट्यधिकशतत्रयेण भाज्याः । घटिकाः । आभिः सूर्येन्द्वोर्गती कलात्मके गुण्ये षष्टिभक्ते तत्फलान्वितयोः स्वस्वफलयुक्तयोः सषड्भसूर्यचन्द्रयोर्भूयः पुनर्विवरासवोऽन्तरप्राणाः पूर्वरीत्या कर्तव्याः । एवं तद्घटकाभिः सूर्यास्तकालिकौ सषड्भसूर्यदृक्कर्मससंस्कृतचन्द्रौ प्रचाल्य तयोर्विवरासव इति यावत्स्थिरीभूता अभिन्नास्तावत् साध्याः । तैरभिन्नैरसुभिः सूर्यास्तादनन्तरं चन्द्रोऽस्तं प्राप्नोति । अत्रोपपत्तिः । सूर्यास्तकाले \textendash


\newpage


\hspace{3cm} गूढार्थप्रकाशकेन सहितः~।\hfill २५९
\vspace{1cm}

%
\noindent सषड्भार्को लग्नं दृक्कर्मसंस्कृतश्चन्द्रः षङ्गयुतश्चन्द्रास्तकाले लग्नम्।परन्तु सूर्यास्तकालिकं न स्वास्तकालिकम् । पश्चिमदृग्ग्रहः सूर्यास्तकालिक इति तत्त्वम् । तदन्तरासवः सावनाश्चन्द्रस्यसूक्ष्मा दिनशेषाः । परन्तु परिभाषया नाक्षत्रज्ञानसम्भवान्नाक्षत्राः साध्या इति चन्द्रस्ताभिश्चाधूल्यः स्वास्तकाले सषड्भो लग्नमस्मात् सूर्यास्तकालिकसषड्भसूर्याच्चान्तरासवो नाक्षत्राः सूक्ष्माः अपि भगवतैकरीतिप्रदर्शनार्थं भिन्नकालिकाभ्यां सूर्यचन्द्राभ्यां कथं सूक्ष्मसमयसिद्धिरिति मन्दाशङ्कापनोदार्थं च सषड्भः सूर्योऽपि माधितश्चन्द्रास्तकाले । ताभ्यामन्तरासवो नाक्षत्रा अपिसूर्यास्तकालिकलग्नाग्रहादसूक्ष्मा इत्यसकृत् सूक्ष्मा इत्युक्तमुपपन्नम् । वस्तुतस्तु सावनाभ्युपगमे । 
%\vspace{2mm}

\begin{quote}
{\qt  रवीन्द्वोः षड्भयुतयोः प्राग्वल्लग्नान्तरासवः~।\\ 
तैः प्रार्णैरस्तमेतीन्दुः शुक्लेऽर्कास्तमनात् परम्~॥}
%\vspace{2mm}
\end{quote}

 इत्येक एव सूर्यसिद्धान्ते श्लोकः । श्लोकमध्य एकराशावित्यादि रवीन्द्वोरित्यन्तरासव इत्यन्तं श्लोकद्वयं केनचिन्मन्दमतिना समयोऽसकृदेव साध्य इति शिष्यधीवृद्धिदतन्त्रोक्तं सुबुद्धिम्मन्येनायुक्तमपि युक्तियुक्तं मत्वा निक्षिप्तम् । कथमन्यथा भगवतः सर्वज्ञस्य शुद्धसावनवटोह्लागामन्तरमसकृत्साधनोक्तिःसङ्गच्छते । किञ्च ।
%\vspace{2mm}

\begin{quote}
{\qt एकराशौ रवीन्द्वोश्च कार्या विवरलिप्तिकाः~। }
%\vspace{2mm}
\end{quote}

 इत्यर्धस्य त्रिप्रश्नाधिकारे भोग्यासूनूनकस्येत्यादिश्लोकाकाग्रेऽपेक्षितत्वेनात्रानपेक्षितत्वम् । प्राग्वल्लग्नान्तरासव इत्यनेनैवात्र\textendash 
 %\end{quote}
%

{\tiny{2 K 2}}


\newpage

\noindent २६० \hspace{4cm} सूर्यसिद्धान्तः
\vspace{1cm}


\noindent तत्सिद्धेरिति~। अथ नाक्षत्राभ्युपगमे तु चन्द्रस्य सावनघटीभिश्चालनं स्वास्तकालिकसिद्ध्यर्थमावश्यकं न तु सूर्यस्य प्रयोजनाभावात् । न हि चन्द्रास्तकालसाधितसषड्भसूर्यः सूर्यास्तकालिकं लग्नं येन सूर्यचालनं युक्तम् । अपिच । एकस्य चन्द्रस्यचालनेन पुनरेकवारेणैव सूक्ष्मनाक्षत्रकालसिद्धौ द्वयोश्चालनोक्त्या नाक्षत्रस्यासकृत् क्रियानयनमतत्वं गौरवं सर्वज्ञेन कथमुक्तम् । असकृत्साधनेन सूक्ष्मनाक्षत्रसिद्धौ युक्त्यभावश्च । अतएव ।
%\vspace{2mm}

\begin{quote}
{\qt ज्ञातुं यदा भाभिमता ग्रहस्य \\
तत्कालखेटोदयलग्नलग्ने~। \\
साध्ये तयोरन्तरनाडिका या \\
स्ताः सावनाः स्युर्द्युगता ग्रहस्य~॥}
%\vspace{2mm}
\end{quote}
 इति भास्कराचार्योक्तं सङ्गच्छत इति तत्वम्~॥~४~॥\\
\noindent  अथोदयसाधनमाह \textendash
%\vspace{2mm}

\begin{quote}
{\ssi भगणार्धं रवेर्दत्त्वा कार्यास्तद्विवरासवः~।\\
 तैः प्राणैः कृष्णपक्षे तु शीतांशुरुदयं व्रजेत्~॥~५~॥}
\end{quote}
 कृष्णपक्षे भगणार्धं षड्राशीन् सूर्यस्य दत्वा संयोज्य तुकाराच्चन्द्रस्यादत्त्वेत्यर्थः । तद्विवरासवलयोर्दृक्कर्मसंस्कृतचन्द्रसषड्भसूर्ययोरन्तरासवः प्रागुक्तप्रकारेण साध्याः । तैः साधितैरसुभिश्चन्द्रः सूर्यास्तानन्तरमुदयं गच्छेत् । अत्रोपपत्तिः । सूर्यास्तकाले सषड्भार्कस्य लग्नत्वात् सूर्ये षड्राशियोजनम् । उदयसाधनार्थं । प्राग्दृग्ग्रहस्यापेक्षितत्वाच्चन्द्रो दृक्कर्मसंस्कृतो यथा


 
\newpage
 

\hspace{3cm} गूढार्थप्रकाशकेन सहितः~। \hfill २६१
\vspace{1cm}


\noindent स्थितो न षड्राशियुक्तः । तद्विवरासुभिश्चन्द्रस्य सूर्यास्तानन्तरमुदयः सावनैः । तच्चालितचन्द्रात् सुर्यास्तकालिकसषड्भार्काच्चविवरासवो नाक्षत्राः इति । शृङ्गोन्नतिसाधनार्थं दृश्यकाले सूर्यचन्द्रौ साध्याविति ज्ञापनार्थं चन्द्रस्य नित्योदयास्तावुक्तावन्येषां ग्रहनक्षत्रादीनां प्रयोजनाभावादनुक्तौ चन्द्रोपलक्षणादुक्तौ वा तत्र शुक्लकृष्णपक्षविवेको नेति ध्येयम्~॥~५~॥\\
\noindent अथ प्रकृतं विवक्षुः प्रथमं तदुपयुक्तभुजकोटिकर्णात्मकं क्षेत्रं श्लोकत्रयेणाह \textendash
%\vspace{2mm}

\begin{quote}
{\ssi अर्केन्द्वोः क्रान्तिविश्लेषो दिक्साम्ये युतिरन्यथा~। \\
तज्ज्येन्दुरर्काद्यत्रासौ विज्ञेया दक्षिणोत्तरा~॥~६~॥

मध्याह्नेन्दुप्रभाकर्णसङ्गुणा यदि सोत्तरा~। \\
तदार्कघ्नाक्षजीवायां शोध्या योज्या च दक्षिणा~॥~७~॥

शेषं लम्बज्यया भक्तं लब्धो बाहुः स्वदिद्मुखः~। \\
कोटिशङ्कुस्तयोर्वर्गयुतेर्मूलं श्रुतिर्भवैत्~॥~८~॥}
%\vspace{2mm}
\end{quote}
 सूर्यचन्द्रयोः स्पष्टक्रान्त्योर्दिगैक्येऽन्तरम् । अन्यथा दिग्भेदेयोगः । अत्र क्रान्तिशब्दः क्रान्तिज्यापरो ज्ञेयः । उपपत्त्यविरोधात् । तज्ज्या सा चासौ ज्या च संस्कारसिद्धाङ्कमिताज्येत्यर्थः । अर्काच्चन्द्रो यत्र यस्यां दिशि तद्दिक्का दक्षिणोत्तरावासौ ज्या ज्ञेया । एकदिशि रविक्रान्तितश्चन्द्रक्रान्तेरधिकत्वेसूर्याच्चशन्द्रस्य क्रान्तिदिक्स्थत्वेन ज्या क्रान्तिदिक् ।ऊनत्वेऽर्का क्रान्तिदिग्विपरीतदिक्स्थत्वेन क्रान्तिभिन्नदिक् । भिन्नदिशि \textendash


\newpage


\noindent २६२ \hspace{4cm} सूर्यसिद्धान्तः
\vspace{1cm}


\noindent चन्द्रक्रान्तिदिग्ज्या ज्ञेयेत्यर्थः । सा ज्यामध्माह्नेन्दुप्रभाकर्णसङ्गुणा यत्काले चन्द्रः शृङ्गोन्नत्यर्थं साधितस्तत्काले मध्याह्नच्छायाकर्णवच्छायाकर्णश्चन्द्रस्य साध्यः । स ज्वक्षांशचन्द्रस्पष्टक्रान्त्योरुत्तरदिशि वियोगो दक्षिणदिशि योगस्तदूननवत्यंशज्ययाभक्ता द्वादशगुणितत्रिज्येति । उपपत्त्यनुरोधेन तु मध्याह्नपदं तत्कालपरम् । यत्काले चन्द्रस्तत्काले चन्द्रस्य द्युगतं दिनशेषं वा प्रसाध्य त्रिप्रश्नाधिकारविधiना शङ्कुं प्रसाध्य छायाकर्णः साध्यः । अह्नोऽहोरात्रस्य मध्यं सूर्यास्तः । तत्कालिकचन्द्रश्ञा छायाकर्णोवायमेव भगवदभिप्रेतः । कथमन्यथा चन्द्रस्य शृङ्गोन्नतौ दृक्कर्मद्वयसंस्कारः शृङ्गोन्नतौ शशाङ्कस्येति प्रागुक्तः सङ्गच्छते। दिनार्धातिरिक्तच्छायासाधनार्थमेव दृक्कर्मणोरुपयोगादन्यत्र शृङ्गोन्नतिगणित उपयोगाभावात् । स्यष्टक्रान्त्यैव छायाकर्णसिद्धेः । अत्रापि श्लोकपकर्वार्धोक्तमेवाक्षदृक्कर्म संस्कार्यम् । तेन छायाकर्णेन गुणितेत्यर्थः । सा तादृशी ज्या यद्युत्तरा तदा द्वादशगुणितायामक्षज्यायां शोध्यान्तरिता । तेन द्वादशगुणिताक्षज्याधिका तादृशी ज्या । तदापि विपरीतशोधने न क्षतिः ।यदि दक्षिणा तदा तस्यामेव युक्ता कार्या । चो व्यावस्थार्थकः । शेषं संस्कारजं स्वदेशलम्बज्यया भक्तं फलं भुजः प्राप्तः । स्वदिङ्मुखः स्वशब्देन संस्कारस्तस्य दिक् तस्यां मुखमग्नं यस्यासौ । संस्कारदिक्क इत्यर्थः । भुजस्य कोटिकर्णसापेक्षत्वात् तावाह\textendash कोटिरिति~। शङ्कुर्द्वादशाङ्गुलः कोटिः । तयोर्भुजकोट्योर्वर्गयोर्योगात् पदं कर्णः स्यात् । अत्रोपपत्तिः\textendash


\newpage


\hspace{3cm} गूढार्थप्रकाशकेन सहितः~। \hfill २६३ 
\vspace{1cm}


 \begin{quote}
{\qt स्वाग्रास्वशङ्कुतलयोः समभिन्नदिक्त्वे~।\\
योगोऽन्तरं भवति दोरिनचन्द्रदोष्णोस् \\
तुल्याशयोर्विवरमन्यादशोस्तु योगः~॥
स्पष्टो भुजो भवति चन्द्रभुजाश इन्द्रोः \\
शुद्धे भुजे रविभुजाद्विपरोतदिक्कः~। }
%\vspace{2mm}
\end{quote}

 इति सूक्ष्मभुजसाधनं भास्कराचार्येण सिद्धान्तशिरोमणावुक्तम् । तदुपपत्तिस्तु तट्टीकायां व्यक्ता । अनया रीत्या भवसाधनार्थं क्रान्तिज्ययोरग्रे साध्ये लम्बज्याकोटौ त्रिज्याकर्णस्तदा क्रान्तिज्याकोटौ कः कर्ण इत्यनुपातेन। तत्स्वरूपं तु प्रत्येकं सूर्यचन्द्रयोः सूर्यक्रान्तिज्या त्रिज्यागुणा लम्बज्याभक्ता $\left\{
\begin{array}{lr}\mbox{
सू. क्रां. ज्या.}
&\mbox{ त्रि १ }\\
& \mbox{लं १ }
\end{array}
\right\}$चन्द्रस्पष्टक्रान्तिज्या त्रिज्यागुणा लम्बज्याभक्ता $\left\{
\begin{array}{lr}\mbox{
चं. क्रां.
ज्या.}& \mbox{त्रि १} \\
 & \mbox{लं १ }
\end{array}
\right\}$अनयोः स्वं स्वं शङ्कुतलं संस्कार्यम् । तत्र शृङ्गोन्नत्यर्थं सूर्येण भगवता सूर्योदयास्तकालिकगणितस्यैवाभ्युपगमात् । तत्र सूर्यशङ्कोरभावात् तच्छङ्कुतलाभावाच्च सूर्याग्रैव सूर्यभुजः सिद्धः । चन्द्रस्य तु तदा शङ्कोः सद्भावाच्छङ्कुतलमुत्पद्यते तत्तु लम्बज्याकोटावक्षज्याभुजस्तदा शङ्कुकोटौ को भुज इत्यनुपातेन तात्कालिकचन्द्रोन्नतनतकालसाधितत्रिप्रन्नाधिकारोक्तचन्द्रगहाशङ्कुगुणिताक्षज्या लम्बज्याभक्तेति दक्षिणमेव शङ्कुतलस्वरुपम् $\left\{
\begin{array}{lr}\mbox{
अक्षज्या चं.} & \mbox{श १} \\ 
& \mbox{लं १ }
\end{array}
\right\}$ इदं चन्द्रदक्षिणाग्रायां योज्यम्। चन्द्रस्यदक्षिणो भुजः । चन्द्रोत्तराग्रायां तु हीनं चन्द्रस्योत्तरो भुजः ।चन्द्रोत्तराग्रया हीनमिदं चन्द्रस्य दक्षिणो भुजः । यथा दक्षिणोभुजः $\left\{
\begin{array}{lr}\mbox{
 चं. क्रां. ज्या. त्रि. १ अक्षज्या. चं.} & \mbox{शं. १} \\
 & \mbox{लं १ }
 \end{array}
\right\}$ वा$ \begin{cases}
\mbox{ चं. क्रां. ज्या. त्रि १}
\end{cases}
$


\newpage


\noindent २६४ \hspace{4cm} सूर्यसिद्धान्तः
\vspace{1cm}


{\scriptsize{$
\begin{rcases}\begin{array}{lr}\mbox{
अक्षज्या. च.} & \mbox{शं १}\\
& \mbox{लं १ }
\end{array}
\end{rcases}
$}} उत्तरो भुजः 
{\scriptsize{
$\left\{
\begin{array}{lr}\mbox{
 चं. क्रां. ज्या. त्रि
१ अक्षज्या. चं.} & \mbox{ शं १} \\
{} & \mbox{लं १}
\end{array}
\right\}$}} अयं चन्द्रभुजः सूर्याग्रयैकदिश्यन्तरितो भिन्नदिशि युक्तः स्पष्टःशृङ्गोन्नत्युपयुक्तो भुजः । यथा सूर्यस्य दक्षिणगोले 
{\scriptsize{$\left\{
\begin{array}{lr}\mbox{
सू. क्रां.
ज्या.
त्रि. १ चं. क्रां. ज्या. त्रि. १ अक्षज्या. चं.} & \mbox{ शं १}\\
& \mbox{लं १ }
\end{array}
\right\}$
$\left\{
\begin{array}{lr}\mbox{
सू. क्रां. ज्या. त्रि. १ चं.
क्रां. ज्या. त्रि. १ अक्षज्या. चं } & \mbox{शं १ }\\
{} & \mbox{लं १}
\end{array}
\right\}$}} इदं भुजद्वयं
स्पष्टो भुजो भवति चन्द्रभुजांश इत्युक्तेर्दक्षिणम् । सूर्यभुजस्य न्यूनत्वेन शोध्यत्वात् ।सूर्यभुजस्याधिकत्वे
{\scriptsize{$\left\{
\begin{array}{lr}\mbox{ तु  सू. क्रां. ज्या. त्रि. १ चं. क्रां. ज्या. त्रि १
अक्षज्या. चं} & \mbox{ शं १ }\\
& \mbox{लं १ }
\end{array}
\right\}$
$\left\{
\begin{array}{lr}\mbox{
 सू. क्रां. । ज्या. त्रि. १ चं. क्रां. । ज्या. त्रि १ अक्षज्या. चं.} & \mbox{शं. १ }\\
 & \mbox{लं १ }
 \end{array}
\right\}$ }} इदं भुजद्वयमुत्तरम् । इन्दोः शुद्धे भुजे रविभुजाद्विपरीतदिक्क इत्युक्तेः। योगे तूत्तरोभुजः 
{\scriptsize{$\left\{
\begin{array}{lr}\mbox{
सू क्रां. ज्या. त्रि. १ चं. क्रां. ज्या. त्रि. १
अक्षज्या. चं } & \mbox{शं १}\\
& \mbox{लं १ }
\end{array}
\right\}$}}
सूर्योत्तरगोलेऽपि 
{\scriptsize{$\left\{
\begin{array}{lr}\mbox{ सू क्रां. ज्या. त्रि. १ चं. क्रां. ज्या. त्रि. १
अक्षज्या. चं } & \mbox{शं १}\\
& \mbox{लं १ }
\end{array}
\right\}$
$\left\{
\begin{array}{lr}\mbox{
 सू क्रां. ज्या. त्रि. १ चं. क्रां. ज्या. त्रि. १ अक्षज्या. चं} & \mbox{शं १ }\\
& \mbox{लं १ }
\end{array}
\right\}$ }}
इदं भुजद्वयं दक्षिणम् । अन्तरे तु सूर्यभुजस्य न्यूनत्व उत्तरो भुजः 
{\scriptsize{$\left\{
\begin{array}{lr}\mbox{
सू क्रां. ज्या. त्रि. १ चं. क्रां. ज्या. त्रि. १ अक्षज्या. चं} & \mbox{शं १}\\
 & \mbox{लं १ }
\end{array}
\right\}$ }}
सूर्यभुजस्याधिकत्वे तु
{\scriptsize{$\left\{
\begin{array}{lr}\mbox{
\ सूर्यक्रां. ज्या. त्रि. १ चं. क्रां. ज्या. त्रि. १ अक्षज्या. चं} & \mbox{ शं १}\\
& \mbox{लं १ }
\end{array}
\right\}$ }}
दक्षिणोऽयं भुजः । इन्दोः शुद्धे भुज इत्युक्तत्वात् । अत्र नवसु पक्षेषु प्रथमपक्षे सूर्यचन्द्रक्रान्तिज्ययोरेकदिशयोरन्तरं त्रिज्यागुणितं तत्सू र्यक्रान्तिसम्बद्धं चेत् तेनोनाक्षज्येन्दुशङ्कुधातो लम्बज्याभक्त इति । चन्द्रक्रान्तिसम्बद्धं चेत् तेन युतस्तद्घातो लम्बज्याभक्त इति सिद्घम् । तत्राक्षांशानां दक्षिणत्विनैकदिशि योगार्थं चन्द्रशेषेदक्षिणत्वं सूर्यशेष उत्तरत्वं भिन्नदिशि वियोगार्थं कल्पितम् । युक्तं चैतत् । सूर्यक्रान्त्यधिकत्वे सूर्याच्चन्द्रस्योत्तरत्वात् । शृङ्गो\textendash



\newpage


\hspace{3cm} गूढार्थप्रकाशकेन सहितः~। \hfill २६५
\vspace{1cm}


\noindent न्नतौ चन्द्रस्यैव प्राधान्याच्च । दितीयपक्षे क्रान्तिज्ययोर्भिन्नदिशयोर्योगेन तादृशेन तद्घातमूनं कृत्वा लम्बज्यया भजेदित्यत्रापि योगस्याग्रेऽन्तरार्थमुत्तरदिक्त्वं चन्द्रक्रान्तेरुत्तरत्वेनदक्षिणस्थसूर्याच्चन्द्रस्य सुतरामुत्तरत्वाच्च । तृतीयपक्षे क्रान्तिज्ययोरेकदिशयोरन्तरे सर्यमम्बद्ध एव तादृशे तद्वध ऊन इति वियोगार्थमन्तरस्योत्तरदिक्त्वम् । द्वयोर्दक्षिणगोलस्थत्वेऽप्यधिकसूर्यान्न्यूनचन्द्रस्योत्तरत्वात् । चतुर्थपक्षे भिन्नदिशयोः क्रान्तिज्ययोर्योगे तादृशे तद्वध ऊन इति वियोगार्थं योगस्योत्तरदिक्त्वम् । चन्द्रस्योत्तरोदिक्स्थत्वात् । पञ्चमपक्षे तु चतुर्थपक्षोक्तंतुल्यत्वात् । षष्ठपक्षे क्रान्तिज्ययोर्भिन्नदिशयोर्योगो दक्षिणस्तद्वधे योगार्थं चन्द्रस्य दक्षिणगोलस्थत्वात् । सप्तमपक्षे क्रान्तिज्ययोरेकदिशयोरन्तरं सूर्यसम्बद्धं तदा तद्वधे योज्यमित्यन्तरं दक्षिणम् । द्वयोरुत्तरगोलस्थत्वेऽपि चन्द्रस्य न्यूनत्वेनार्काद्दक्षिणस्थत्वात् । अधिकत्वे तूत्तरं तद्वधे हीनमिति । अष्टमपक्षे क्रान्तिज्ययोरेकदिशयोरन्तरे चन्द्रसम्बद्ध उत्तरे तद्वध ऊनः । चन्द्रस्याधिकत्वेमोत्तरस्थत्वात् । अन्त्यपक्षे तु समदिशयोः क्रान्तिज्ययोरन्तरं सूर्यसम्बद्धं तद्वधे योज्यमिति दक्षिणम् । चन्द्रस्य न्यूनत्वेन दक्षिणस्थत्वादित्युपपन्नं प्रथमश्लोकोक्तम् । अत्र केनचित् क्रान्तिशब्देन चापात्मकक्रान्ती गृहीत्वा तत्संस्कारः कृतस्तस्य ज्या कार्येति व्याख्यातम् । तदुपपत्तिविरुद्धम् । नहि भुजसाधने चापात्मकक्रान्ती प्रयोजकत्वेगोपपन्ने । येन व्याख्योक्ता । नवा क्रान्तिज्यायोगवियोगाभ्यां चापात्मकक्रान्ति


{\tiny{2 L}}

\newpage

\noindent २६६ \hspace{4cm} सूर्यसिद्धान्तः
\vspace{1cm}


\noindent योगवियोगयोर्ज्ये तुल्ये येनोक्तं सङ्गतं स्यात् । अन्यथाक्षांशक्रान्त्यंशसंस्कारांशज्या विनापि क्रान्तिज्याक्षज्ययोः संस्कारेण नतांशज्यायाः साधनापत्तेरिति दिक् । अथायं भुजस्त्रिज्यावृत्त इति लाघवात् तात्कालिके चन्द्रच्छायाकर्णमितवृत्ते स्वेच्छया साधितस्त्रिज्यावृत्तेऽयं भजस्तदा चन्द्रच्छायाकर्णवृत्ते क इत्यनुपातेन क्रान्तिज्यायाः संस्कारमितमाद्यं खण्डं चन्द्रच्छायाकर्णगुणमिति सिद्धम् । त्रिज्यामितपूर्वगुणस्येदानीन्तनत्रिज्यामितहरस्य तुल्यत्वेन द्वयोर्नाशाच्च । अथापरखण्डं चन्द्रशङ्कक्षज्याघातात्मकं चन्द्रच्छाद्याकर्णगुणं त्रिज्याभक्तं कार्यम् । तत्र त्रिज्याद्वादशघातस्य चन्द्रशङ्कुभक्तस्य छायाकर्णत्वाच्छङ्कुत्रिज्यामितयोर्गुणहरयोः प्रत्येकं नाशादक्षज्या द्वादशगुणेत्यपरं खण्डं सिद्धम् । इयोरेकदिशि योगो भिन्नदिश्यन्तरमिति संस्कारो लम्बज्याभक्तो भुजः संस्कारदिक्कः सिद्धः । शङ्कुः कोटिरिति चन्द्रच्छायाकर्णवृत्ते भुजसाधनात् तद्वृत्ते कोटिरपि साध्या । सा तु नियता द्वादश । नियतकोट्यर्थमेव भुजश्चन्द्रच्छायाकर्णवृत्ते साधितः सूर्योदयास्तयोः सूर्यशङ्कोरभावात् सूर्यशङ्कुसंस्काराभावः । तदितरकाल उक्तक्रियया निर्वाहः कोटिभुजयोर्वर्गयोगान्मूलं कर्ण इत्युपपन्नं मध्याह्नेत्यादि श्लोकद्वयोक्तम्~॥~८~॥
\noindent अथ शुक्लानयनमाह \textendash 

%\vspace{2mm}

\begin{quote}
{\ssi सूर्योनशीतगोर्लिप्ताः शुक्लं नवशतोद्धृताः~।\\
 चन्द्रविम्बाङ्गुलाभ्यस्त हृतं द्वादशभिः स्फुटम्~॥~९~॥}
\end{quote}

\newpage



 \hspace{3cm} गूढार्थप्रकाशकेन सहितः~।\hfill २६७
\vspace{1cm}



 सूर्योनितचन्द्रस्य कला नवशतभक्ताः फलं शुक्लं तच्चन्द्रग्रहणाधिकारोक्तप्रकारेणागतचन्द्रबिम्बाङ्गुलैर्गुणितं द्वादशभिर्भक्तं फलं स्फुटं शुक्लं स्यात् । अत्रोपपत्तिः । दर्शान्ते सूर्यचन्द्रयोरन्तराभावादस्मद्दृश्यार्धे चन्द्रगोले सूर्यकिरणप्रतिफलनाभावाच्छौक्ल्याभावः । ततो यथायथार्काच्चन्द्रः पूर्वतोऽन्तरितस्तथातथा चन्द्रगोलास्मद्दृश्यार्धे चन्द्रपश्चिमभागक्रमेण शौक्ल्यवृद्धिः । एवं षड्राश्यन्तरे पौर्णमास्यन्ते चन्द्रगोलास्मद्दृश्यार्धं सम्पूर्णं श्वेतं भवति । इतः षड्राशिकलाभिः खखाष्टदिग्भिर्द्वादशाङ्गुलव्यासबिम्बं श्वेतं तदेष्टेन सूर्योनचन्द्रकलागुणेन किमित्यनुपाते प्रमाणफलयोः फलापवर्तनेन प्रमाणस्थाने नवशतम् । अतः सूर्योनचन्द्रस्य कला नवशतभक्ताः शौक्ल्यमिदं द्वादशाङ्गुलव्यासप्रमाणेन सिद्धम् । अतो द्वादशाङ्गुलप्रमाणेनेदं तदाभिमतच्चन्द्रबिम्बाङ्गुलव्यासप्रमाणेन किमित्यनुपातेनोक्तमुपपन्नम् ।अनेन प्रकारेण त्रिभान्तरे चन्द्रगोलास्मद्दृश्यार्धभर्धं श्वेतं भवतीति सिद्धम् । भास्कराचार्यैस्तु\textendash

%\vspace{2mm}

\begin{quote}
{\qt कक्षाचतुर्थस्तरणेर्हि चन्द्रः कर्णान्तरे तिर्यगिनो यतोऽब्जात्~। \\
पादोनषट्काष्टलवान्तरेऽतो दलं नृदृश्यं दलमस्य शुक्लम्~॥}
%\vspace{2mm}
\end{quote}
इति शृङ्गोन्नतिवासनायामुक्तम् । शृङ्गोन्नत्यधिकारे~।
%\vspace{2mm}

\begin{quote}
{\qt चन्द्रस्य योजनमयश्रवणेन निघ्नो \\
व्यर्केन्दुदोर्गुण इनश्रवणेन भक्तः~। \\
तत्कार्मुकेण सहितः खलु शुक्लपक्षे \\
कृष्णोऽमुना विरहितः शशभृद्विधेयः~॥}
\end{quote}

{\tiny{2 L 2}}

\newpage


\noindent २६८ \hspace{4cm} सूर्यसिद्धान्तः
\vspace{1cm}

इति तदभिप्रेतश्वेतानयनोपयुक्तश्चन्द्रः साधित इत्यलम्~॥~९~॥ \\
\noindent अथ श्लोकचतुष्टयेन शृङ्गोन्नतिपरिलेखामाह \textendash
%\vspace{2mm}

\begin{quote}
{\ssi  दत्वार्कसज्ज्ञ्तिं बिन्दुं ततो बाहुं स्वदिङ्मुखम्~।\\
 ततः पश्चान्मुखीं कोटिं कर्णं कोट्यग्रमध्यगम्~॥~१०~॥

कोटिकर्णयुताद्बिन्दोर्बिम्बं तात्कालिकं लिखेत्~।\\
कर्णसूत्रेण दिक्सिद्धिं प्रथमं परिकल्पयेत्~॥~११~॥

शुक्लकर्णेन तद्बिम्बयोगादन्तर्मुखं नयेत्~।\\
शुक्लाग्रयाम्योत्तरयोर्मध्ये मत्स्यौ प्रसाधयेत्~॥~१२~॥

तन्मध्यसूत्रसंयोगाद्बिन्दुत्रिस्पृग्लिखेद्धनुः~।\\
प्राग्बिम्बं यादृगेव स्यात् तादृक् तत्र दिने शशी~॥~१३~॥ }
%\vspace{2mm}
\end{quote}

 समभूकमावभीष्टस्थाने दिक्साधनं कृत्वा पूर्वूआपरा दक्षिणोत्तरा च रेखा कार्या । तत्र दिक्यम्पातेऽर्कसञ्ज्ञितमर्कसञ्ज्ञासञ्जाता यस्येत्येतादृशमर्कसञ्ज्ञं बिन्दुं चिह्नं दत्वा कृत्वेत्यर्थः । ततो बिन्दोः सकाशाद्भुजं पूर्वसाधितं स्वदिङ्मुखं स्वदिशा दक्षिणोत्तरान्यतरा तदभिमुखं दत्वा भुजाङ्गुलानि गणयित्वा चिह्नंकृत्वा ततो भुजाग्रचिह्नात् पश्चान्मुखीं पश्चिमदिक्समसूत्राभिमुखाग्रां कोटिं द्वादशाङ्गुलात्मिकां दत्वा कर्णं पूर्वसाधितं कोट्यग्रमध्यगं कोट्यग्रचिह्नं मध्यं सूर्यसञ्ज्ञकचिह्नं तयोर्गतं स्पृष्टम् । तदन्तराले कर्णाङ्गुलानि दत्वेत्यर्थः । कोटिकर्णरेखासंयोगे मध्यं प्रकल्प्य तात्कालिकं सूर्यास्तोदयकालिकं चन्द्रस्यसाधितं मण्डलं लिखेत् । तत्र लिखितचन्द्रबिम्बे कर्णसूत्रेण


\newpage


\hspace{3cm} गूढार्थप्रकाशकेन सहितः~। \hfill २६६
\vspace{1cm}


\noindent कर्णरेखया प्रथममादौ दिक्सिद्धिं दिशानिष्पत्तिं परिकल्पयेत् । कुर्यात् । चन्द्रमण्डलं कर्णरेखायां यत्र लग्नं तत्र चन्द्रवृत्ते पूर्वा । कर्णरेखां स्वमार्गेणाग्रे निःसार्य चन्द्रवृत्तपरिधौ यत्र कर्णरेखापरभागे लग्ना तत्र पश्चिमा । तन्मत्स्याभ्यां रेखा दक्षिणोत्तरा चन्द्रवृत्ते यत्र लग्ना तत्र दक्षिणोत्तरेति फलितार्थः । शुक्लं पूर्वसाधितं कर्णेन कर्णरेखामार्गेण तद्बिम्बयोगात्कर्णरेखाचन्द्रमण्डलपरिध्योः सम्पातादपूर्वात् । अन्तर्मखं चन्द्रवृत्तकेन्द्राभिमुखं नयेत् । शुक्लाग्रचिह्नं कुर्यात् । चन्द्रवृतान्तः कर्णरेखायां पश्चिमचिह्नाच्छुक्लाङ्गुलानि गणयित्वा चिह्नं कुर्यादित्यर्थः । शुक्लाग्रयाम्योत्तरयोश्चन्द्रवृत्तान्तर्यत्र शुक्लाग्रचिह्नं यत्र च चन्द्रवृत्तपरिधौ दक्षिणोत्तरयोश्चिह्नं तयोरित्यर्थः । मध्येऽन्तराले । मत्स्यौ प्रत्येकं साधयेत् । शुक्लाग्रदक्षिणचिह्नाभ्यां मत्स्यः शुक्लाग्रोत्तरौचिह्नाभ्यां मत्स्यश्चेति पूर्वोक्तरोत्यामत्स्यौ कुर्यादित्यर्थः । तन्मध्यसूत्रसंयोगात् । तयोर्मत्स्ययोर्मध्यसूत्रं मुखपुच्छस्पृग्गर्भसूत्रं प्रत्येकं तयोर्यत्रचन्द्रमण्डलान्तस्तद्बहिर्वा केन्द्राच्छुक्लाग्रस्य पश्चिमत्वे पर्वभागे संयोगः पूर्वत्वेपश्चिमभागे संयोगः स्वस्वमार्गेण प्रसारितयोस्तयोः सम्पातस्तस्मात्स्थानात् । बिन्दुत्रिस्पृक् । श्यक्लाग्रबिन्दुर्याम्योत्तरयोश्चिह्नबिन्दुरिति बिन्दुत्रितयस्पर्शि धनुर्वृत्तैकदेशात्मकं लिखेत् । सूत्रसम्पातशुक्लाग्रबिन्द्वन्तरालाङ्गुलव्यासार्द्धेन सम्पातस्थानात् बिन्दुत्रयस्पृष्टवृत्तपरिध्येकदेशात्मकं चन्द्रमण्डलान्तश्चापं कुर्यादित्यर्थः । प्राक् पूर्वकाले । लिखितं चन्द्रबिम्बम् । यादृक् । 


\newpage

\noindent २७० \hspace{4cm} सूर्यसिद्धान्तः
\vspace{1cm}


\noindent लिखितचापच्छेदेन यादृशं पश्चिमभागे भवति । तादृशः एवकारस्तद्भिन्ननिरासार्थकः । तस्मिन् दिने । श्टङ्गोन्नतिगणिताश्रयीभूतसन्ध्यासमये चन्द्र आकाशस्थो भवति । अत्रोपपपत्तिः । भुजस्तु सूर्याच्चन्द्रो यावतान्तरेण तद्रूप इति सूर्यस्थानं प्रकल्प्य तस्माद्यथादिग्भूजो देयस्तस्माच्छुक्लपक्षे पश्चिमदिक्स्थस्य चन्द्रस्य शृङ्गोन्नतिर्भवतीति सूर्यचन्द्रयोरूर्ध्वाधरान्तरं कोटिर्दत्ता । सूर्यचन्द्रयोरन्तरं तिर्थक्कर्ण इति कोट्यग्रसूर्यबिम्बान्तराले कर्णो दत्तः । कर्णदानं कोटेः सरलत्वसिद्ध्यर्थम् । तत्र कोटिकर्णयोगे चन्द्रावस्थानात् चन्द्रवृत्तंतन्मध्यत्वेन लिखितम् । कर्णमार्गेण शुक्लदर्शनात् चन्द्रबिम्बे कर्णसूत्रानुरुद्धा पूर्वापरा तदनुरुद्धा दक्षिणोत्तरा च । शुक्लपक्षे चन्द्रपश्चिमभागेऽर्काभिमुखत्वेन शौक्ल्यात् पश्चिमस्थानात्कर्णरेखायां चन्द्रवृत्तान्तः श्वेतं दत्तम् । तत्र चन्द्रमण्डले याम्योत्तरचिह्नावधिकं वृत्तैकदेशरूपं धनुः शुक्लाग्रबिन्दुस्पृष्टं चन्द्राकृतिदर्शथार्थं कार्यम् । अतो बिन्दुत्रयस्पृग्वृत्तस्य केन्द्रज्ञानार्थं प्रागुक्तरीत्या बिन्दुत्रयेभ्यो मत्स्यौ प्रमाध्य तत्सूत्रयुतिः केन्द्रमस्मात् चापं तथैव भवतीति चन्द्राकृतिः प्रत्यक्षा~॥~१३~॥ \\
\noindent ननु यदर्थमयमुद्योगस्तस्याः शृङ्गोन्नतेर्ज्ञानं नोक्तमत आह \textendash
%\vspace{2mm}
%{\setlength{\parindent}{7em}
\begin{quote}
{\ssi कोट्या दिक्साधनात् तिर्यक्सूत्रान्ते शृङ्गमुन्नतम्~। \\
दर्शयेदुन्नतां कोटिं कृत्वा चन्द्रस्य सा कृतिः~॥~१४~॥ }
%\vspace{2mm}
\end{quote}
 कोट्या कोटिरेखया चन्द्रवृत्ते कर्णरेखावत् दिक्साधनात्परिलेखे शुक्लधनुषः कोटिमग्रभागात्मिकामुन्नतामुच्चां कृत्वा


\newpage

\hspace{3cm} गूढार्थप्रकाशकेन सहितः~। \hfill २७१
\vspace{1cm}


\noindent दृष्ट्वा । तिर्यक्सूत्रान्ते । दक्षिणोत्तररेखाया अन्ते । अवसाने । उन्नतमुच्चं शृङ्गं दर्शयेत् । सा परिलेखसिद्धा । आकृतिः स्वरूपम् । चन्द्रस्य । आकाशस्थचन्द्रस्य । भवति । परिलेखसिद्धरूपमाकाशस्थचन्द्रे प्रत्यक्षमित्यर्थः । अत्रोपपत्तिः । यथा' चन्द्रवृत्तेकर्णरेखया चन्द्रदिशस्तथा कोटिरेखया चन्द्रवृत्ते सूर्यदिशस्तयोरन्तरं भुजश्चन्द्रवृत्तपरिणतः । अथ चन्द्रदक्षिणोत्तरयोर्धनुः कोट्योः संलग्नत्वात् सूर्यदक्षिणोत्तराभ्यां कोटिरूपशृङ्गेणनतोन्नते भवतस्तत्र भुजदिक्कं शृङ्गं नतम् । तदितरदिक्कं शृङ्गमुन्नतम् । अत एव भास्कराचार्यैरुक्तम् ।
%\vspace{2mm}

\begin{quote}
{\qt स्यात् तुङ्गशृङ्गं वलनान्यदिक्स्थम्~।}
%\vspace{2mm}
\end{quote}
इति~॥~९४~॥\\
ननु सूर्योनचन्द्रस्य षड्बाधिकत्व उक्तप्रकारेणचन्द्रबिम्बाभ्यधिकं शुक्लमायाति तत् कथं युक्तं व्याघातादित्यतस्तदुत्तरं विशेषं चाह\textendash 
%\vspace{2mm}

\begin{quote}
{\ssi कृष्णे षड्भयुतं सूर्यं विशोध्येन्दोस्तथासितम्~।\\
 दद्याद्वामं भजं तत्र पश्चिमं मण्डलं विधोः~॥~१५~॥ }
%\vspace{2mm}
\end{quote}
 कृष्णपक्षे षड्राशिभिः सहितमर्कं चन्द्राद्विशोध्य । तथा लिप्तानवशतभक्ता इति पूर्वप्रकारेण । असितं श्याममानेयम् । तथा च पूर्वोक्तं शुक्लानयनं शक्लपक्ष एव चन्द्रशौक्ल्यवृद्धिज्ञानार्थम् । कृष्णपक्षे तु शौक्ल्यहासात् कृष्णतावृद्धेः कृष्णानयनं युक्तं न शुक्लानयनम् । अत एव दर्शान्तमासस्य शुक्लकृष्णौ द्वौ पक्षाविति भावः । अथ कृष्णपरिलेखार्थं पूर्वोक्ते विशेषमाह\textendash दद्यादिति~।


\newpage

\noindent २७२ \hspace{4cm} सूर्यसिद्धान्तः
\vspace{1cm}


\noindent तत्र कृष्णपरिलेखविषये वामं र्विपरीतं भुजं प्रागुक्तं दद्यात्। अर्कचिह्नादुत्तरं भुजं दक्षिणतो दक्षिणं भुजमुत्तरतो गणकोदद्यात् । चन्द्रस्य मण्डलं पश्चिमं दर्शयेत् । यथा शुक्लपक्षे चन्द्रमण्डलस्य पश्चिमभागे शौक्ल्यं तथा कृष्णपक्षे चन्द्रमण्डलस्य पश्चिमभागे कृष्णाभिवृद्धिं दर्शयेदित्यर्थः । अत्रोपपत्तिः । कृष्णपक्षारम्भेसूर्यचन्द्रयोः षड्राश्यन्तरम् । ततः षड्राशिपर्यन्तं कृष्णाभिवृद्धिः ।अतः षड्राशियुतसूर्येण वर्जितचन्द्रात् पूर्वप्रकारेण कृष्णानयनं युक्तम् । अथ शुक्लशृङ्गं यत्र नतं तत्र कृष्णशृङ्गमुन्नतं यत्र चोन्नतं तत्र नतम्। अतः कृष्णपरिलेखार्थं भुजो विपरीतो देयः । तदपि कृष्णं पश्चिमभागादेवाभिवृद्धम् । अतः कर्णरेखायां चन्द्रबिम्बान्तः पश्चिमस्थानाद्देयम्। ततः प्राग्वत् कृष्णशृङ्गोन्नतिरिति~॥~१५~॥\\ 
\noindent अथाग्रिमग्रन्थस्यासङ्गतित्वनिरासार्थमधिकारसमाप्तिं फकिकयाह\textendash

\begin{center}
इति शृङ्गोन्नत्यधिकारः~। 
\end{center}

 चन्द्रोदयास्तयोः शृङ्गोन्नतिविषयत्वेनोक्तत्वादस्यामेवान्तर्भावो न स्वतन्त्राधिकारत्वमन्यथा ग्रहोदयास्ताधिकारे तदुक्त्यापत्तेः । एतेन चन्द्रोदयास्तयोः पौर्णमास्यधिकारत्वं पर्वतोक्तं निरस्तम् । तत्संज्ञायां प्रमाणाभावादन्यथामावास्याधिकारस्यैव णुवक्षत्वापपत्तेरिति ध्येयम् । 
%\vspace{2mm}

 \begin{quote}
{\qt  रङ्गनाथेन रचिते सूर्यसिद्धान्तटिप्पणे~।\\
 शृङ्गोन्नत्यधिकारोऽयं पूर्णो गूढप्रकाशके~॥ }
%\vspace{2mm}
\end{quote}
इति श्रीसकलगणकसार्वभौमबल्लालदैवज्ञात्मजखरङ्गनाथगणकविरचिते गूढार्थप्रकाशके शृङ्गोन्नत्यधिकारः पूर्णः~॥ 



\begin{center}
\noindent\rule{8em}{.5pt}
\end{center}

\newpage

 \hspace{3cm} गूढार्थप्रकाशकेन सहितः~। \hfill २७३
\vspace{1cm}


\noindent साम्यमुत्तरोत्तरपरिवर्तान्तरे भवत्येवेत्युपपन्नं क्रान्त्योर्ज्ये इत्यादि श्लोकत्रयम्~॥~११~॥\\
\noindent अथ क्रान्तिसाम्यं पात इति स्पष्टं कथयन् तत्कालज्ञानार्थं साधितक्रान्तिसाम्यसम्बन्धिचन्द्रासन्नार्धराचात्पातकालस्य गतगम्यत्वमाह\textendash
%\vspace{2mm}

\begin{quote}
{\ssi क्रान्त्योः समत्वे पातोऽथ प्रक्षिप्तांशोनिते विधौ~।\\
होनेऽर्धरात्रिकाद्यातो भावी तात्कालिकेऽधिके~॥~१२~॥ }
%\vspace{2mm}
\end{quote}
सूर्यचन्द्रयोः स्पष्टक्रान्त्योः साम्ये स्पष्टः पातः स्यात् । अथानन्तरम् । स्पष्टपातसम्बन्धी साधितचन्द्रः पूर्वानुसन्धानेनापाततोयद्दिनीयो भवति तदासन्नार्धरात्रकाले स्पष्टचन्द्रो मध्यस्पष्टाधिकारोक्तप्रकारेण साध्यः । तस्मादर्धरात्रकालिकाच्चच्छात् प्रक्षिप्तांशोनिते क्रान्तिचापान्तरेण तदर्धेन वा युतोनिते चन्द्रे स्पष्टक्रान्तिसाम्यसम्बद्धसाधितचन्द्रे न्यूने सति तदर्धरात्रकालात् पातकालो गतः । तात्कालिके क्रान्तिसाम्यकालिकसाधितचन्द्रेऽर्धरात्रकालिकचन्द्रादधिके सति तदर्धरात्रकालात् पातकाल एष्य दूत्यर्थः । अत्रोपपत्तिः । यद्यपि स्पष्टक्रान्तिसाम्यसम्बद्धचन्द्रमध्यक्रान्तिसाम्यकालिकचन्द्राभ्यां वक्ष्यमाणप्रकारेण पातकालस्य मध्यक्रान्तिसाम्यकालाद्गतैष्यघट्यादिज्ञानं भवतीति निकटार्धरात्रिकचन्द्रात् तत्साधनं पुनस्तद्गतैष्यकथनं च गौरवम् । अर्धरात्रिकस्पष्टचन्द्रसाधनक्रियाधिक्यात् । तथापि चन्द्रगतेरतिमहत्त्वेनप्रतिक्षणं गतेर्बह्वन्तरेणान्यादृशत्वाद्बहुकालान्तरे बहुकालान्तरितस्यष्टगत्यानीतघट्यात्मकस्यातिस्थूलत्वादासन्नकाले स्वल्पान्त-


{\tiny{2 N 2}}

\newpage
\noindent २७४ \hspace{4cm} सूर्यसिद्धान्तः
\vspace{1cm}


\noindent राच्चामन्नार्धरात्रिकः स्पष्टचन्द्रो ग्रन्थोक्तः सस्पष्टगतिकोऽवश्यमपेक्षितः । अतस्तस्माच्चन्द्रात्स्पष्टक्रान्तिसाम्यसम्बद्धचन्द्रस्यन्यूनाधिकत्वे क्रमेण तदर्धरात्रात्स्पष्टपातो गतैष्य इति सम्यगुक्तम् ।अतएव ।
%\vspace{2mm}

\begin{quote}
{\qt समीपतिथ्यन्तसमीपचालनंविधोस्तु तत्कालजयैव युज्यते~। }
%\vspace{2mm}
\end{quote}
 इति भास्कराचार्योक्तं सङ्गच्छते~॥~९२~॥\\
 अथ स्पष्टपातकालज्ञानमाह \textendash
%\vspace{2mm}

\begin{quote}
{\ssi स्थिरीकृतार्धरात्रेन्द्वोर्द्वयोर्विवरलिप्तिकाः~।\\
 षष्टिघ्नाश्चन्द्रभुक्त्याप्ताः पातकालस्य नाडिकाः~॥~१३~॥ }
%\vspace{2mm}
\end{quote}
 स्थिरीकृतार्धरात्रेन्द्वोः स्यष्टक्रान्तिसाम्यमम्बद्धसाधिता सकृत्क्रिया नियतचन्द्रस्तदासन्नार्धरात्रिकस्यष्टचन्द्रः । तयोरुभयोः । अत्र द्वयोरिति पकर्वपदार्थव्यक्तीकरणाय । अन्यथैकवचनप्रमादाद्व्याकुलतापत्तेः । अन्तरकलाः षष्ट्या गुणिता अर्धरात्रिकचन्द्रस्पष्टकलात्मकगत्या भक्ताः फलम् । पातकालस्यार्धरात्राद्गतैष्यस्पष्टक्रान्तिसाम्यस्य घटिका भवन्ति । अर्धरात्राद्गतैष्यक्रमेणफलघटीभिः पूर्वमुत्तरत्र स्यष्टक्रान्तिमाम्यरूपपातः स्यादित्यर्थः । अत्रोपपत्तिः । चन्द्रस्पष्टगत्या षष्टिसावनघटिकास्तदास्वाभीष्टार्धरात्रकालिकक्रान्तिसाम्यकालिकस्पष्टचन्द्रयोरन्तरकलाभिः का इत्युपपन्नमुक्तम् । साधितसूर्यस्य प्राथमिकचन्द्रगतिग्रहणेम स्थूलत्वादर्धरात्रिकस्पष्टसूर्यादुक्तरीत्या पातकालानयनं स्थूलं नोक्तमिति ध्येयम्~॥~१३~॥\\
 \noindent अथ पातकालस्य स्थित्यर्धानयनमाह \textendash


\newpage

\hspace{3cm} गूढार्थप्रकाशकेन सहितः~। \hfill २७५ 
\vspace{1cm}


 तयोश्चन्द्रसूर्ययोः । तुकारात् क्रान्तिसाम्यकालिकयोः । तुल्यांशुजालसंपर्कात्समकिरणानां जालं समूहस्तयोरन्योन्याभिमुखयोः संपर्कात् । एकीभावापन्नत्वात् । तद्दृक्क्रोधभवः सूर्यचन्द्रयोरन्योन्याभिमुखयो र्दृक्क्रोधो बिम्बकेन्द्रयोर्दृग्रूपयोः क्रोधः परस्पराभिमुखेन दीप्त्याधिक्यं तदुत्पन्नोऽग्निः । प्रवहाहतः प्रवहवायुप्रज्वलितः । लोकाभावाय जनानामशुभफलाय । जायते~॥~३~॥\\
 \noindent अथायं वह्निर्व्यतीपाताख्यो वैधृताख्यो वेत्यत आह \textendash
%\vspace{2mm}

\begin{quote}
{\ssi विनाशयति पातोऽस्मिन्लोकानामसकृद्यतः~।\\
व्यतीपातः प्रसिद्धोऽयं संज्ञाभेदेन वैधृतिः~॥~४~॥}
%\vspace{2mm}
\end{quote}
अस्मिन् क्रान्तिसाम्यकाले । प्रसिद्धः पूर्वश्लोकोक्तस्वरूपः । पाते वह्निः । यतः कारणात् । असकृत् स्वसम्भवेन वारं वारम् ।लोकानां विनाशयति । नाशं करोति । अतः कारणादयंवह्निर्व्यतीपातसंज्ञोऽयमेवाग्निः संज्ञाभेदेन नामान्तरेण वैधृतिसंज्ञः । तथा चोऽभयत्र पाताख्यो वह्निर्भवतीति भावः~॥~४~॥\\
\noindent अथ तत्स्वरूपमाह \textendash
%\vspace{2mm}

 \begin{quote}
{\ssi  स कृष्णो दारुणवपुर्लोहिताक्षो महोदरः~|\\ 
सर्वानिष्टकरो रौद्रो भूयो भूयः प्रजायते~॥~५~॥ }
%\vspace{2mm}
\end{quote}
 स क्रान्तिसाम्यकालोत्पन्न उभयसंज्ञकः पाताख्योऽग्निपुरुषः कृष्णः श्यामः । दारुणवपुः कठिनशरीरः । लोहिताक्ष आरक्तनेत्रः । भहोदरः पृथूदरः । अत एव सर्वानिष्टकरः सर्वलोकानामशुभकारकः । रौद्रः क्षयकारकः । भूयो भृयोऽनेक \textendash


{\tiny{2 M 2 }}

\newpage

\noindent २७६ \hspace{4cm} सूर्यसिद्धान्तः
\vspace{1cm}


\noindent वारम् । प्रजायते । प्रत्येकं क्रान्तिसाम्यकाल उत्पन्नो भवतीत्यर्थः~॥~५~॥\\
\noindent अथ स्पष्टकालज्ञानं विवक्षुः प्रथमं तादृशयोः सूर्यचन्द्रयोः सायनांशयोः क्रान्ती साध्ये इत्याह \textendash
%\vspace{2mm}

 \begin{quote}
{\ssi भास्करेन्द्वोर्भचक्रान्तश्चक्रार्धावधिसंस्थयोः~।\\
 दृक्तल्यसाधितांशादियुक्तयोः स्वावपक्रमौ~॥~६~॥ }
%\vspace{2mm}
\end{quote}
सूर्यचन्द्रयोर्दृक्तुल्यसाधितांशादियुक्तयोः । प्राक् चक्रं चलितं हीने छायार्कात् करणागते । इत्यादिना । दृग्गोचरीभूतं साधितमंशादिकं तेन संलतयोरित्यर्थः । एतेन पूर्वसाधारणोक्तिरपि स्पष्टीकृता क्रान्त्योः सायनोत्पन्नत्वात् । भचक्रान्तर्भचक्रं द्वादशराशयस्तन्मध्ये । संस्थयोः स्थितयोः । ययोर्योगो द्वादशराशयस्तयोरित्यर्थः । चक्रार्धावधिसंस्थयोः । चक्रार्धं राशिषट्कं तदवाध तदन्तः स्थितयोर्ययोर्योगो राशिषट्कं तयोरित्यर्थः । स्वौ स्वकीयौ । अपक्रमौ साध्यौ । सूर्यस्य क्रान्तिः साध्या । चन्द्रस्य विक्षेपसंकृता क्रान्तिः साध्येत्यर्थः~॥~६~॥\\
\noindent अथ साधितक्रान्तिभ्यां स्वकालात् स्पष्टपातकालस्य गतैष्यत्वं विशेषं च श्लोकाभ्यामाह \textendash
%\vspace{2mm}

\begin{quote}
{\ssi अथौजपदगस्येन्दोः क्रान्तिर्विक्षेपसंस्कृता~।\\
 यदि स्यादधिका भानोः क्रान्तेः पातो गतस्तदा~॥~७~॥

ऊना चेत् स्यात् तदा भावी वामं युग्मपदस्य च~।\\
पदान्यत्वं विधोः क्रान्तिर्विक्षेमाच्चेद्विशुध्यति~॥~८~॥}
\end{quote}


\newpage


\hspace{3cm} गूढार्थप्रकाशकेन सहितः~। \hfill २७७
\vspace{1cm}


 अथ सूर्यचन्द्रयोः क्रान्तिसाधनानन्तरम् । चन्द्रस्य विषमपदस्थस्य । विक्षेपसंस्कृता क्रान्तिः । स्पष्टक्रान्तिरित्यर्थः । यदियर्हि । सूर्यस्य विषमसमान्यतरपदस्थस्य । साधितक्रान्तेः सकाशादधिका स्यात् । तदा तर्हि । पातः स्पष्टक्रान्तिसाम्यात्मकः । गतः । साधितक्रान्तिकालात् पूर्वकाले जात इत्यर्थः । चेद्यर्हि। सूर्यक्रान्तेर्विषमपदस्थचन्द्रस्पष्टक्रान्तिर्न्यूना भवति तदा तर्हि स्पष्टक्रान्तिसाम्यरूपपातः । भावी । साधितक्रान्तिकालादुत्तरकालेभवतीत्यर्थः । ननु विषमपदे चन्द्रो न भवति तदा गतैष्यत्वज्ञानं कथं स्यादत आह\textendash वाममिति~। युग्मपदस्य । समपदस्थचन्द्रस्येत्यर्थः । चकारात् स्यष्टक्रान्तिः सूर्यक्रान्तेः सकाशादधिकोना वा स्यात् तर्हीत्यर्थः । वामम् । उक्तगतीष्यक्रमेण वैपरीत्यम् । एष्यगतत्वंपातस्य भवतीत्यर्थः । अथ चन्द्रस्य विशेषमाह\textendash पदान्यत्वमिति~। चन्द्रस्य स्पष्टक्रान्तिक्रियायाम् । चेद्यर्हि । चन्द्रस्यविक्षेपासंस्कृतकेवलक्रान्तिर्विक्षेपात् भिन्नदिक्काद्विशुध्यति हीना भवति । क्रान्तिवर्जितविक्षेपरूपा स्पष्टक्रान्तिर्यदि स्यात् तदेत्यर्थः । पदान्यत्वं राश्यादिचन्द्राधिष्ठितपदभिन्नपदस्थत्वं चन्द्रस्य ज्ञेयम् ।सायनराश्यादिना समपदस्थस्य चन्द्रहस विषमपदस्थत्वम् । सायनराश्यादिना विषमपदस्थस्य चन्द्रस्य समपदस्थत्वं तत्पदसम्बन्धास्पष्टा क्रान्तिर्ज्ञेयेत्यर्थः । अत्रोपपत्ति । विषमपदे क्रान्तिरूपचिता समपदेऽपचिजा । अतः सूर्यक्रान्तेर्विषमपदस्थ इन्दुक्रान्तिरधिका तदाग्रे सुतरामधिकत्वाद्रविक्रान्त्युपचयस्याल्पत्वाच्चन्यूनया रविक्रान्त्या चन्त्रक्रान्तेः समत्वमग्रिमकालेन भवति । अतः


\newpage


\noindent २७८ \hspace{4cm} सूर्यसिद्धान्तः
\vspace{1cm}


\noindent पूर्वकाले चन्द्रक्रान्तेर्न्यूनत्वाद्रविक्रान्त्यपचयस्याल्पत्त्वाच्च तत्क्रान्तिसाम्यं जातमित्यनुमितम् । एवं समपदस्थ इन्दुक्रान्तिरूना तदाग्रे सूर्यक्रान्तेर्न्यूना तदाग्रे सुतरां न्यूनत्वात् तत्साम्याभावः । पूर्वंत्वधिकत्वात् तत्समत्वं जातमिति ज्ञातम्। यदा तु सूर्यक्रान्तेर्विषमपदस्थ इन्दुक्रान्त्यधिकत्वेन तत्क्रान्तिसाम्यं भवति पूर्वंतन्न्यूनत्वे तदभावात् । एवं सूर्यक्रान्तेः समपदस्थेन्दुक्रान्तिरधिका तदाग्रे न्यूनत्वेन तत्साम्यं भवति । अत एव तुल्यत्वे वर्तमान इति । अत्र चन्द्रस्य विक्षेपवृत्तं विषुवद्वृत्ते लग्नं यत्र तत्रस्पष्टक्रान्तेरभावाद्गोलसन्धिः । तस्मात् त्रिभान्तरे विक्षेपवृत्तेऽयनसन्धिः । स्पष्टक्रान्तिस्तदन्तराल उपचितापचितायनसन्धिस्थक्रान्त्यनधिका । यदाचन्द्रक्रान्तिर्मध्यमा शरभिन्नदिक्का शरादल्पा तदा शराच्छोधनेन स्पष्टक्रान्तिर्मध्यमक्रान्तिसम्बन्धपदभिन्नपदसम्बन्धा भवति । अतः\textendash

%\vspace{2mm}

\begin{quote}
{\qt पदान्यत्वं विधोः क्रान्तिर्विक्षेपाच्चेद्विशुध्यति~।}
%\vspace{2mm}
\end{quote}
इति सम्यगुक्तम् । भास्कराचार्योक्तं च ।
%\vspace{2mm}

\begin{quote}
{\qt चक्रे चक्रार्धे च व्ययनांशेऽर्कस्य गोलसन्धिः स्यात्~।\\
एवं त्रिभे च नवभेऽयनसन्धिर्व्ययनभागेऽस्य~॥

अयनांशोनितपाताद्दोः कोटिज्ये लधुज्यकोत्थे ये~।\\
ते गुणसूर्यैरश्वैर्गुणिते भक्ते कृतैः सूर्यैः~॥

अयनांशोनितपाते मृगकर्क्यादिस्थिते द्विषड्रामैः~।\\
कोटिफलयुतविहिनैर्बाहुफलं भक्तमाप्तांशैः~॥

मेषादिस्थेगोलायनसन्धी भास्करस्योनौ~।\\
तौ चन्द्रस्य स्यातां तुलादिषट्कस्थिते तु संत्तुक्तौ~॥}
\end{quote}


\newpage


\hspace{3cm} गूढार्थप्रकाशकेन सहितः~। \hfill २७९
\vspace{1cm}

%

\begin{quote}
{\qt गोलायनसम्भन्तं पदं विधोरत्र धीमता ज्ञेयम्~।\\
 रविगोलवरस्पष्टा स्पष्टा क्रान्तिः खगोलदिक् शशिनः~॥}
%\vspace{2mm}
\end{quote}
इति पदज्ञानम् । अनेनैव प्रकारेण चन्द्रस्तष्टक्रान्तेः पदं ज्ञेयं विक्षेपवृत्तसम्बन्धत्वात् । न साधारणपदज्ञानेन स्पष्टक्रान्तेः क्रान्तिवृत्तसम्बन्धाभावात् । अन्यथा पदज्ञानासम्भवापत्तेः । एतदङ्गीकारे पदान्यत्वमित्याद्यर्धं व्यर्थमपि भगवता तदर्धेनैतादृशं पदं ज्ञापितमन्यथा तदक्त्यापत्तेरिति दिक्~॥~८~॥\\
\noindent अथ गतैष्यकालानयनं विवक्षुः प्रथमं स्यष्टक्रान्तिसाम्यानयनप्रकारंश्लोकत्रयेणाह \textendash
%\vspace{2mm}

\begin{quote}
{\ssi क्रान्त्योर्ज्ये त्रिज्ययाभ्यस्ते परक्रान्तिज्ययोद्धृते~।\\
 तच्चापान्तरमर्धं वा योज्यं भाविनि शीतगौ~॥~९~॥

शोध्यं चन्द्राद्गते पाते तत्सूर्यगतिताडितम्~।\\
चन्द्रभुक्त्या हृतं भानौ लिप्तादि शशिवत् फलम्~॥~१०~॥

तद्वत् शशाङ्कपातस्य फलं देयं विपर्ययात्~।\\
कर्मैतदसकृत् तावद्यावत् क्रान्ती समे तयोः~॥~११~॥ }
%\vspace{2mm}
\end{quote}
 सूर्यचन्द्रयोः साधितक्रान्त्योर्ज्ये कार्ये ते त्रिज्यया गुणिते । परक्रान्तिज्यया| 
%\vspace{2mm}

\begin{quote}
{\qt परमापक्रमज्या तु सप्तरन्ध्रगुणेन्दवः~। }
% \vspace{2mm}
 \end{quote}

इति । पूर्वोक्तपरभक्रान्तिज्ययेत्यर्थः । भक्ते । तयोः फलयोर्धनुषीकार्ये । चन्द्रस्य यदा त्रिज्याधिकं फलं तदोक्तप्रकारेण धनुषोऽसम्भवात् त्रिज्यया नवत्यंशास्तदेष्टज्यया क इत्यनुपातेन धनुः \textendash
%

\newpage

\noindent २८० \hspace{4cm} सूर्यसिद्धान्तः
\vspace{1cm}


\noindent कार्यमथवा त्रिज्यातो यदधिकं तदुत्क्रमधनुषा युक्ताश्चतुः पञ्चाशच्छतकला धनुः स्यादिति ध्येयम् । तयोरन्तरम् । अर्धम् । अन्तरार्धम् । वा विकल्पार्थकः । अथवा विषयव्यवस्थार्थकः । सा तु यदान्तरमल्पं तदान्तरम् । यदा तु बह्वन्तरं तदान्तरार्धं ग्राह्यमिति । भाविनि भविष्यत्याते । चन्द्रे राश्यात्मके । तत्कालात्मकं युक्तं कार्यम् । गते पाते सति । चन्द्राद्धीनं कार्यं चन्द्रः स्यात् । सूर्यसाधनमाह\textendash तदिति~। चन्द्रसम्बन्धिसंस्कृतफलम् । स्पष्टसूर्यगत्या गुणितं स्पष्टचन्द्रगत्या भक्तं फलं कलादिकं चन्द्रवत् । चन्द्रयुतहीनक्रमेण सूर्ये युतहीनं कार्यं सूर्यः स्यात् । चन्द्रपातसाधनमाह\textendash तद्वदिति~। चन्द्रपातस्य फलं कलादिकम् । तद्वत् । चन्द्रफलं पातगत्या गुणितं स्पष्टचन्द्रगत्याभक्तं विपर्ययात् । व्यत्यासात् । देयं संस्कार्यम् । चन्द्रयुतहीनक्रमेण चन्द्रपाते हीनयुतं कार्यम् । चन्द्रपातः स्यात् । उक्तक्रियातिदेशमाह\textendash कर्मेति~। एतत् । उक्तं कर्म गणितक्रियारूपम् । असकृत् । अनेकवारम् । साधितसूर्यात् । सूर्यक्रान्तिं प्रसाध्य साधितचन्द्रपाताभ्यां चन्द्रस्पष्टक्रान्तिं प्रसाध्य ताभ्यां क्रान्तिभ्यां क्रान्त्योर्ज्ये इत्यादिना चापान्तरं तदर्धं वा तत्क्रान्तिभ्यामवगतगतैष्यपातलक्षणवशात् । द्वितीयचन्द्रे हीनयुतं तृतीयचन्द्रः स्यात् । आद्यसुर्यचन्द्रगतिभ्यामवगतसूर्यपातफलं द्वितीयसूर्यपातयोर्यथोक्तं सम्स्कृतं तृतियसूर्यपातौ । एभ्यः सूर्यचन्द्रपातेभ्यः सूर्यचन्द्रक्रान्तिभ्यां साधिताभ्यां चापान्तरं तदर्धं वा तृतियचन्द्रे तत्क्रान्त्यवगतगतैष्यपातवशात् संस्कृतं चतुर्थचन्द्रः स्यात् । आद्य \textendash


\newpage

\hspace{3cm} गूढार्थप्रकाशकेन सहितः~। \hfill २८१
\vspace{1cm}


\noindent सूर्थचन्द्रगत्यवगतस्वफलसंस्कृतौ तृतीयसूर्यपातौ चतुर्थसूर्यपातौस्तः । एवमेभ्यः पञ्चमाश्चन्द्रसूर्यपाता उक्तरीत्या साध्या इत्युत्तरोत्तरं मुहुः साध्या इत्यर्थः । अवधिमाह\textendash तावदिति~। यावद्यदवधि तयोः सूर्यचन्द्रयोः क्रान्ती स्पष्टक्रान्तितुल्ये स्तस्तावत् तदवधि क्रिया कार्येत्यर्थः । अत्रोपपत्तिः । मभ्यमक्रान्तिसाम्यरूपपातकालिकस्पष्टक्रान्तिभ्यां स्यष्टक्रान्तिसाम्यरूपवस्तुभूतकालो गतैष्यत्वेन ज्ञातोऽपि विशेषतस्तत्कालज्ञानार्थं सूर्यचन्द्रयोः क्रान्ती समे स्पष्टे उपपन्ने कार्ये । तत्र मध्यपातकालाद्गतैष्यपातवशादभीष्टकाले चन्द्रसूर्यपातान् प्रसाध्य तयोः क्रान्ती साध्ये । एवं साधितक्रान्त्योर्यदैवातुल्यत्वं तदैव स्यष्टपातः । अथानियमात् प्रथमं पूर्वाग्रिमकाले चन्द्रसाधनार्थं चत्रस्येष्टांशा हीना योज्याश्चेति नियता भागा उक्तप्रकारानीता एवेष्टाः कल्पिताः । तथाहि । सूर्यक्रान्तिज्यातः परक्रान्तिज्ययान्यूनया चतुर्दशशतमितया त्रिज्या तुल्या दोर्ज्या तदेष्टक्रान्तिज्यया केत्यभीष्टदोर्ज्यायाश्चापं सायनसूर्यभुज एव । एवं चन्द्रस्पष्टक्रान्तिज्यातश्चापं सायनसूर्यभुजान्न्यूनमधिकं भवति । क्रान्तिसमत्वाभावात् । यद्यपि न्यूनचतुर्दशशताधिकस्पष्टक्रान्तेरुक्ररीत्या भुजज्यायास्त्रिज्याधिकत्वेन चापाकरणमशक्यं तथापि ।
%\vspace{2mm}

\begin{quote}
{\qt त्रिज्याधिकस्य क्रमचापलिप्ताः खखाब्धिबाणा धनुरुत्क्रमात् स्यात्~। }
%\vspace{2mm}
\end{quote}
इति सिद्धान्तशिरोमण्युक्तवैपरीत्येन त्रिज्यातो यदधिकं तदुत्क्रमचापयुक्ताश्चतुःपञ्चाशच्छतकला इत्यनेन चापोत्पत्तौ न
%

{\tiny{2 N}}


\newpage


\noindent २८२ \hspace{4cm} सूर्यसिद्धान्तः
\vspace{1cm}


\noindent क्षतिः । एतेन चापासम्भवशङ्कया सार्धाष्टविंशत्यंशानां ज्या परमक्रान्तिज्येति स्वायनसन्धिस्थस्पष्टक्रान्तिज्या चेति च निरस्तम् । ग्रन्थे तयोः परमक्रान्तिज्यात्वानुक्तेः । स्पष्टक्रान्तिसाम्यानन्तरमप्युकतरीत्या कर्मान्तरनिवारणानुपपत्तेश्च । क्रान्त्योस्तुल्यत्वेऽपि हरभेदात् तच्चापान्तरसद्भावेन क्रियाकुण्ठनासम्भवात् । न ह्यसकृत्कर्मणि स्वाभीष्टसिद्ध्यनन्तरं कर्मान्तरं सम्भवति । अप्रसिद्धेः स्वरूपव्याघाताच्च । तच्चापयोरन्तरमिष्टांशाश्चन्द्रस्य गतैष्यापातवशाद्धीनयुता अभीष्टचन्द्रो भवति । तदिष्टांशानां बहुत्वे बहुपरिवर्तैरभीष्टसिद्धिरतोऽल्पपरिवर्तैरभीष्टसिद्ध्यर्थं तदर्धमिष्टांशा इति । अथैते चन्द्रस्येष्टांशा इत्येभ्यश्चन्द्रगतिप्रमाणेनैते तदा सूर्यपातगतिभ्यां क इत्यनुपातेन तयोश्चन्द्रकालिकत्वसिद्ध्यर्थमिष्टांशा एते सूर्यस्य सम्स्कृताश्चन्द्रवदभीष्टसूर्यो भवति । पातस्य तु चक्रशुद्धत्वेन विपरीतत्वात् पातेष्टांशाः पातस्य व्यस्तं संस्कार्याअभीष्टपातो भवति । एभ्यः सूर्यचन्द्रयोः स्पष्टक्रान्ती साध्ये। तयोरसमत्वं उक्तरीत्या चन्द्रस्येष्टांशा एतत्साधितचन्द्रे संस्कार्याः न प्रथमचन्द्रे । तत्क्रान्तिजत्वाभावात् । अन्यथा समक्रान्त्यनन्तरमपि तयोरिष्टांशाभावे प्रथमचन्द्रसूर्यपातानां तत्संस्कृतेऽप्यविकारा तत्क्रान्न्त्योर्द्वितीद्यपरिवर्तक्रान्तिममत्वेन कर्मान्तरसम्भवात् क्रियाकुण्ठनत्वानुपपत्तेः । अव्यवहितपूर्वग्रहयोजनेत्वन्त्यकर्मण एव सिद्धेः । कर्मान्तरासम्भवाच्च । सूर्यपातयोरिष्टांशास्तु पूंर्वचन्द्रसूर्यस्पष्टगतिभ्यामेव स्वल्पान्तरात् कार्याः । अव्यवहितपूर्वकाले स्पष्टगत्यज्ञानात् । एवमसकृत् करणेन क्रान्त्योः \textendash


\newpage

\hspace{3cm} गूढार्थप्रकाशकेन सहितः~। \hfill २८३
\vspace{1cm}


\noindent अथ पाताध्यायो व्याख्यायते । तत्र भेदद्वयात्मकपातस्यसम्भवं विवक्षुः प्रथमं वैधृतसंज्ञपातस्य सम्भवमाह \textendash
%\vspace{2mm}

\begin{quote}
{\ssi एकायनगतौ स्यातां सूर्याचन्द्रमसौ यदा~।\\
तद्युतौ मण्डले क्रान्त्योस्तुल्यत्वे वैधृताभिधः~॥~१~॥ }
\end{quote}
%\vspace{2mm}

 सूर्यचन्द्रौ । सूर्याचन्द्रमसौ धाता यथापूर्वमकल्ययदितिं श्रुत्युक्तप्रयोगः । एकायनगतौ । अभिन्नदक्षिणोत्तरान्यतरायनस्थौ भवतस्तत्र यदा यस्मिन् काले तद्युतौ सूर्यचन्द्रयोर्भाद्योर्योगेमण्डले द्वादशराशिमिते सति तदा तयोः क्रान्त्योः समत्वे महापातरूपे वैधृतसंज्ञः पातो भवति~॥~९~॥\\
 \noindent अथ व्यतीपातसंज्ञपातस्य सम्भवमाह \textendash
 
%\vspace{2mm}

\begin{quote}
{\ssi विपरीतायनगतौ चन्द्रार्कौ क्रान्तिलिप्तिकाः~।\\
समास्तद्वा व्यतीपातो भगणार्धे तयोर्युतौ~॥~२~॥}
\end{quote}
%\vspace{2mm}

 चन्द्रार्कौ विपरीतायनगतौ भिन्नायनस्थौ भवतस्तत्र यदा तयोः सूर्यचन्द्रयोर्भाद्योर्योगे भगणार्धे राशिषट्के सति । तयोः क्रान्तिकलास्तुल्या भवन्ति तदा तस्मिन् काले व्यतीपातसंज्ञकः पातो भवति । अत्रोपपत्तिः । समक्रान्तिकालो महापातकालः । तत्र स्पष्टक्रान्त्योरतिवैलक्षण्योपचयापचययोर्नियमाभावाच्च समकालो दुर्लक्ष्य इति मध्यमक्रान्त्योः समत्वकालात् पूर्वमपरत्र वा शरवशेन शरसंस्कृतक्रान्तिसमत्वं भवतीति निश्चित्य वस्तुभूततत्कालज्ञानार्थं प्रथमं तदासन्नकालस्य मध्यमक्रान्तितुल्यस्य ज्ञानमावश्यकं तत्तु सूर्यचन्द्रयोः क्रान्तिसमत्वं भुजतुस्यत्वे सम्भ \textendash

%

{\tiny{2 M}}

\newpage


\noindent २८४ \hspace{4cm} सूर्यसिद्धान्तः
\vspace{1cm}

%
\noindent वति भुजोत्पन्नत्वात् । भुजसमत्वं सूर्यचन्द्रयोः षड्राशिमितयोगे द्वादशदराशिमितयोगे वा षड्राशिमितान्तरेऽन्तराभावे वा कुतएवमिति चेच्छृणु । तत्रान्तराभावे द्वयोस्तुल्यत्वेन भुजसाम्ये विवादाभावः । एवं ड्भान्तरेऽपीतरयोर्विषमपदस्थयोः समपदस्थयोर्वा क्रमेण पदगतैष्ययोस्तुल्ययोर्भुजत्वमित्यविवादः । षड्द्वादशराशियोगे तु तयोर्विषमसमपदस्थत्वात् क्रमेण तुल्यगतैष्यत्वेन भुजतुल्यत्वम् । रविगोलायनसंधिस्थयोस्तु क्रान्तिपरमाभावत्वइति तत्रापि तदन्तरयोगयोः षड्द्वादशराश्योर्यथायोग्यसत्त्वात्क्रान्तिसाम्यं सहजत एव । अत एकायनस्थयोर्भिन्नगोलस्थयोर्द्वादशराशियोग एकगोलायनस्थयोरन्तराभावे क्रान्तिसाम्यम् । एवं भिन्नायनस्थयोरेकगीलस्थयोः षड्राशियोगे गोलभेदस्थयोः षड्राश्यन्तरे क्रान्तिसाम्यमिति युतावित्युपलक्षणादन्तरइत्यपि ज्ञेयम् । न तु तद्युतौ मण्डले भगणार्धे तयोर्युतावित्युक्तेन क्रमेण गोलभेदैक्ययोरन्तरनिरासार्थकोक्तिस्तत्रापि क्रान्तिसाम्यत्वेनानिवार्यत्वात् । अत्रैकायनगताविति विपरीतायनगताविति च स्वरूपोक्तिरनावश्यकीति ध्येयम् । वस्तुतस्तु सूर्यचन्द्रयोर्द्वादशमिते योगेऽन्तरे वा वैधृताख्यं क्रान्तिसाम्यम् । षड्राशिमिते तयोर्योगेऽन्तरे वा व्यतीपाताख्यं क्रान्तिसाम्यमिति तात्पर्योक्तिः । अत एवाग्ने भास्करेन्द्वोरित्याद्युक्तं युक्तमितितत्त्वम्~॥~२~॥\\
\noindent ननु क्रान्त्योः साम्ये कथं पातो भवतीत्यत आह \textendash
%\vspace{2mm}

\begin{quote}
{\ssi तुल्यांशुजालसंपर्कात् तयोस्तु प्रवहाहतः~।\\
तद्दुक्क्रोधभवो वह्निर्लोकाभावाय जायते~॥~३~॥}
\end{quote}
%

\newpage

\hspace{3cm} गूढार्थप्रकाशकेन सहितः~। \hfill २८५
\vspace{1cm}


 
% {\setlength{\parindent}{5em}
\begin{quote}
{\ssi  रवीन्दुमानयोगार्धं षष्ट्या सङ्गुण्य भाजयेत्~।\\
 तयोर्भूक्त्यन्तरेणाप्तं स्थित्यर्थं नाडिकादि तत्~॥~१४~॥ }
%\vspace{2mm}
\end{quote}

 सूर्यचन्द्रयोश्चन्द्रग्रहणाधिकारोक्तप्रकारेण ये बिम्बमानकले स्वस्वगतिकलोत्पन्ने तयोरैक्यस्यार्धं षष्ट्या गुणयित्वा सूर्यंचन्द्रयोः कलात्मकस्पष्टगत्योरन्तरेण भजेत् । यल्लब्धं तद्घटिकादिकं स्थित्यर्धं पातकालात् पूर्वमपरत्र च स्थित्यर्धकालपर्यन्तंपातस्यावस्थानमित्यर्थः । अत्रोपपत्तिः । सुर्यचन्द्रबिम्बकेन्द्रयोरेकद्युरात्रवृत्तस्थत्वे विषुवद्वृत्तादुभ यतस्तुल्यान्तरत्वे वा पातमध्यंकेन्द्रसाम्याद्विषुवद्वृत्तात् क्रान्तिसूत्रस्थो मण्डलपरिधिप्रदेशो य आसन्नः स बिम्बपृष्ठप्रान्तः । दूरस्थस्तु बिम्बाग्रप्रान्तः । याम्योत्तरगमनेन पातस्योक्तेः । तत्र शीघ्रबिम्बाग्रप्रान्तमन्दपृष्ठबिम्बप्रान्तयोस्तथात्वे पातारम्भः ।सूर्यबिम्बाग्रप्रान्तचन्द्रबिम्रपृष्ठप्रान्तयोस्तथात्वे पातान्तः । अत आद्यन्तकालाभ्यां क्रमेण पूर्वोत्तरकालयोश्चन्द्रार्कबिम्बान्तर्गतप्रदेशानां केषामप्युक्तरूपस्थितित्वाभावेन सूर्यचन्द्रयोस्तथात्वाभावात् पाताभाव इत्यादिकालमारभ्यान्तकालपर्यन्तं सू्यचन्द्रयोस्तथात्वात् पातस्थितिः । पातमध्यकाले क्रान्त्यन्तराभावः  पाताद्यन्तकालयोर्मानैक्यार्धतुल्यंक्रान्त्यन्तरम् । तेन तत्तुल्यान्तरस्यापचयकाल उपचयकालश्चाद्यन्तस्थित्यर्धे । तत्र तत्कालानयनं सूर्यचन्द्रगन्यन्तरेण षष्टिघटिकास्तदा मानैक्यखण्डकलाभिः का इत्यनुपातेनोक्तमुपपन्नम् । यद्यपि प्रमाणेच्छयोः समजातित्वाभावादनुपातो \textendash



\newpage

\noindent २८६ \hspace{4cm} सूर्यसिद्धान्तः
\vspace{1cm}


\noindent ऽसङ्गतः क्रान्तेर्दक्षिणोत्तरान्तरस्योपचयापचययोः सूर्यचन्द्रगत्यन्तरस्य पूर्वापरान्तरस्योपचयापचयाभ्यामतिविलक्षणत्वात् । तथापि गणितलाघवार्थं भगवता स्वल्पान्तरत्वेनानुपातो लोकानुकम्पयाङ्गीकृत इत्यदोषः । भास्कराचार्यैस्तु\textendash

%\vspace{2mm}

\begin{quote}
{\qt मानैक्यार्धं गुणितं स्पष्टघटीभिर्विभक्तमाद्येन~।\\
लब्धघटीभिर्मध्यादादिः प्रागग्रतश्च पातान्तः~॥}
%\vspace{2mm}
\end{quote}

इति युक्तमुक्तम् । केचित्तु षष्टिघटिकाभिर्ग्रहान् प्रचाल्य क्रान्तिः स्पष्टा साध्या । प्रत्येकं तयोरन्तरं योगो वा गत्यन्तरीमेति भास्कराभिमतमाहुः~॥~९४~॥\\
\noindent अथ पातस्यादिमध्यान्तकालानाह \textendash

%\vspace{2mm}

\begin{quote}
{\ssi पातकालः स्फुटो मध्यः सोऽपि स्थित्यर्धवर्जितः~।\\ 
तस्य सम्भवकालः स्यात् तत्संयुक्तोऽन्त्यसञ्ज्ञितः~॥~१५~॥} 
%\vspace{2mm}
\end{quote}

 स्थिरीकृतार्धरात्रेत्यादिना स्पष्टः पातकालः क्रान्तिसाम्यस्य काल आनीतो मध्यसञ्ज्ञो ज्ञेयः । स मध्यकाल आनीतीस्थेत्यर्धेन हीनस्तस्य पातस्य सम्भवकाल आरम्भकालः । अपिः समुच्चये । तत्संयुक्तः स्थित्यर्धयुक्तो मध्यकालोऽन्त्यसञ्ज्ञितः पातो भवति । पातस्यान्तकालो भवतीत्यर्थः । अत्रोपपत्तिश्चन्द्रग्रहणस्पर्शमोक्षवत् स्पष्टा । स्वरूपं तु प्राग्व्यक्तीकृतम्~॥~१५~॥\\
 \noindent अथैतज्ज्ञानस्य प्रयोजनं किमित्यतः पातस्थितिकालो मङ्गलकृत्येनिषिद्ध इत्याह\textendash 
 
%\vspace{2mm}

\begin{quote}
{\ssi आद्यन्तकालयोर्मध्यः कालो ज्ञेयोऽतिदारुणः~।\\
प्रज्वलज्ज्वलनाकारः सर्वकर्मसु गर्हितः~॥~१६~॥}
%
\end{quote}
\newpage

\hspace{3cm} गूढार्थप्रकाशकेन सहितः~। \hfill २८७
\vspace{1cm}


 पातस्यारम्भसमाप्तिसमययोरन्तरालवर्ती समयः । अत्यन्तं कठिनः । सर्वेषु मङ्गलकृत्येषु निन्दितो ज्ञेयः । अत्र हेतुगर्भंविशेषणमाह\textendash प्रज्वलज्ज्वलनाकार इति~। देदीप्यमानाग्निस्वरूपः । तथाच कृतं मङ्गलकृत्यं भस्मावशेषं स्थादिति भावः~॥~९६~॥\\
\noindent ननु पातस्य क्रान्तिसाम्यत्वेन सूक्ष्मकालरूपत्वादागतमध्यकाल एव सूक्ष्मः शुभकर्मसु निन्दितो न पातसथित्यात्मकस्थूलकालः क्रान्तिसाम्याभावादित्यत आह \textendash
%\vspace{2mm}

\begin{quote}
{\ssi एकायनगतं* यावदर्केन्द्वोर्मण्डलान्तरम्~।\\
 सम्भवस्तावदेवास्य सर्वकर्मविनाशकृत्~॥~१७~॥ }
%\vspace{2mm}
\end{quote}
 सूर्यचन्द्रयोर्मण्डलान्तरं प्रत्येकं बिम्बैकदेशरूपं यावद्यत्कालपर्यन्तमेकायनगतं तुल्यमार्गस्थितं भवति । तावत् तत्कालपर्यन्तम् । एवकारो न्यूनाधिकव्यवच्छेदार्थकः । अस्य पातस्य । सकलशुभकर्मणामाचरितानां नाशकारी । सम्भव उत्पत्तिः । स्थितिरिति यावत् । न क्रान्तिसाम्यमात्रं स्थितिरलक्ष्यत्वात्। तथाच विषुवद्वृत्तादुभयत एकतो वा चन्द्रार्कबिम्बैकदेशयोःकयोरपि तुल्यान्तरेण यावदवस्थानं केन्द्रावस्थानाभावेऽपिबिम्बसम्बन्धाप्ता तत्स्थितिः । अतएव\textendash
%\vspace{2mm}

\begin{quote}
{\qt तावत् समत्वमेव क्रान्योर्विवरं भवेद्यावत्~।\\ 
मानैक्यार्धादूनं साम्याद्बिम्बैकदेशक्रान्त्योः~॥}
\end{quote}
\noindent\rule{\linewidth}{.5pt}

\begin{center}
 * एककाष्ठागतं इति पाठान्तरम् ।
 \end{center}
 


\newpage

\noindent २८८ \hspace{4cm} सूर्यसिद्धान्तः
\vspace{1cm}


\noindent इति भास्कराचार्योक्तं युक्ततरमिति भावः~॥~१७~॥\\
\noindent नन्वयं केवलं मङ्गलनाशको न शुभकारक इत्यत आह \textendash
%\vspace{2mm}

\begin{quote}
{\ssi स्नानदानजपश्राद्धव्रतहोमादिकर्मभिः*~।\\
 प्राप्यते सुमहच्छ्रेयस्तत्कालज्ञानतस्तथा~॥~१८~॥}
%\vspace{2mm}
\end{quote}
 व्रतं स्वाभिमतदेवताराधनम् । आदिपदाद्धर्मान्तरम् । इत्यादिपुण्यक्रियाभिस्तत्कालकृताभिः सुतरामुत्कृष्टं कल्याणंमनुष्यैर्लभ्यते । तस्य पातस्य स्थित्यादिकालज्ञानात् । तथा समुच्चये । तेन महच्छ्रेयः प्राप्यत इत्यर्थः~॥~१८~॥\\
 \noindent अथ पातविशेषमाह \textendash
%\vspace{2mm}

\begin{quote}
 {\ssi रवीन्द्वोस्तुल्यता क्रान्त्योर्विषुवत्सन्निधौ यदा~। \\
 द्विर्भवेद्द्विस्तदा पातः स्यादभावो विपर्ययात्~॥~१९~॥ }
%\vspace{2mm}
\end{quote}
 यदा यस्मिन् काले विषुवन्निकटे क्रान्त्यभावासन्ने । अत्र चन्द्रस्य स्पष्टक्रान्त्यभावासन्नत्वं ध्येयम् । सूर्यचन्द्रयोः क्रान्त्योः समता भवति । तदा तस्मिंस्तदासन्नकाले स्थूलरूपे क्रान्त्यभावादुभयत्र द्विर्वैधृतव्यतीपातभेदद्वयात्मकः पातः । द्विः प्रत्येकं द्विधा वारद्वयं भवेत् । विपर्ययादुक्तव्यत्यासात् । चान्द्रायणसन्निधिनिकटे तयोः क्रान्त्योस्तुल्यत्व इत्यर्थः । अत्रातुल्यत्वं सूर्यक्रान्तितश्चन्द्रस्पष्टक्रान्तेर्न्यूनत्वमेव नाधिकत्वमितिध्येयम् । अभावः क्रान्तिसाम्यरूपपातस्य तस्मिन् स्थलूकाले किञ्चिन्मितेऽनुत्पत्ति स्यात् । एतेन\textendash
%

\noindent\rule{\linewidth}{.5pt}

\begin{center}
 * -कर्मसु इति वा पाठः ।
\end{center}

\newpage


\hspace{3cm} गूढार्थप्रकाशकेन सहितः~।\hfill २८९
\vspace{2mm}


\begin{quote}
{\qt स्वायनसन्धाविन्दोः क्रान्तिस्तत्कालभास्करक्रान्तेः~।\\
 ऊना यावत्तावत्क्रान्त्योः साम्यं तयोर्नास्ति~॥}
%\vspace{2mm}
\end{quote}

इति भास्कराचार्योक्तं सङ्गच्छते । तत्साधनं तु प्रथमागतचायान्तरादिष्टांशाश्चन्द्रे युता हीना इति प्रत्येकमसकृत्क्रियया द्विधा पातकालस्य ज्ञेयम् । अत्रोपपत्तिः । व्यतीपाते विषुवद्वृत्तादुभयतस्तुल्मान्तरेण सूर्यचन्द्रयोरवस्थितिकालेऽपि पातत्वम् । क्रान्तिसाम्यादेवं वैधृतेऽप्येकाहोरात्रवृत्तस्थत्वकाले पातत्वम् । एवमेव वियोगव्यतीपातवैधृतयोरप्येकाहोरात्रवृत्तस्थत्वे विषुवद्वृत्तादुभयतस्तुल्यान्तरावस्थितौ च पातत्वम् । क्रान्तिसाम्यादियुक्तं गोलसिद्धं चन्द्रगोलसन्धिनिकटप्रत्यक्षम् । अभावोपपत्तिस्तु । चन्द्रस्य स्वायनसन्धौ तरस्पष्टक्रान्तितुल्यं परमं विषुवद्वृत्ताद्दक्षिणोत्तरं गमनं भवत्यस्मादग्रे पृष्ठे वा विक्षेपवृत्ते भ्रमतश्चन्द्रस्य क्रान्तिर्न्यूनैव सम्भवत्यतः स्वायनसन्धिस्थचन्द्रकालिकसूर्यक्रान्तिः स्वायनसन्धिस्थचन्द्रस्पष्टक्रान्तेरधिका तदेष्टचन्द्रक्रान्तेर्न्यूनत्वेनाधिकसूर्येष्टक्रान्त्या समत्वानुत्पत्तिः । सूर्यस्य चन्द्राल्पगमनवत्त्वात् क्रान्त्यपचयस्यापि चन्द्रक्रान्त्यपचयाल्पत्वसम्भवात् । सूर्यक्रान्त्युपचये तु सुतरां तदसम्भव । एवं तत्रस्यसूर्यक्रान्तिर्न्यूना तदापचयाधिक्याच्चन्द्रस्पष्टक्रान्तिस्तत्समा तदुत्तरपूर्वकाले सम्भवति । सूर्यक्रान्त्युपचये तु सुतराम् । तथा च द्वितीयरविगोलसन्ध्यासन्ने चन्द्रपाते स्वायनसन्ध्यासन्ने सूर्ये च तदसम्भवः कियन्तिचिद्दिनानीति यावत्तावदुक्तमन्यत्र तत्सम्भावना भवतीति गोलयुक्त्या फलितम् ।


{\tiny{2 O}}

\newpage



\noindent २९० \hspace{4cm} सूर्यसिद्धान्तः
\vspace{1cm}


अथासम्भवलक्षणेऽपि क्रान्त्यन्तरस्य मात्रैक्यखण्डादल्पत्वे \textendash

%{\setlength{\parindent}{6em}
\begin{quote}
{\qt एकायनगतं यावदर्केन्द्वोर्मण्डलान्तरम्~।}
\end{quote}
इति पूर्वोक्तेन पातसम्भवः । तत्र पातमध्यं तस्मिन्नेव काले स्थित्यर्धं तु रवहिमानयोगार्धमित्युक्तरीत्या मानयोगार्धमिति स्थाने क्रान्त्यन्तरमानैक्यखण्डयोरन्तरं गृहीत्वा साध्यमितिध्येयम्~॥~१९ ~॥\\
\noindent  cfdअथ शुभकार्ये महापातस्य निषिद्धत्वोक्तिप्रसङ्गात् पञ्चाङ्गान्तर्गतयोगान्तर्गतव्यतीपातस्यैव ज्ञानमाह \textendash
%\vspace{2mm}

\begin{quote}
{\ssi  शशाङ्कार्कयुतेर्लिप्ता भभोगेन विभाजिताः~।\\ 
लब्धं सप्तदशान्तोन्योऽ व्यतीपातस्तृतीयकः~॥~२०~॥ }
%\vspace{2mm}
\end{quote}

 अयनांशसंस्कृतयोश्चन्द्रसूर्ययोर्योगस्य राश्यादेः कला अष्टशतेन भक्ताः फलं सप्तदशान्तः । सप्तदशमध्ये षोडशानन्तरं सप्तदशपर्यन्तमित्यर्थः । तदपि व्यतीपातः । अन्य एतदधिकारपूर्वोक्तातिरिक्तः । तृतीय एव तृतीयकः । सूर्यचन्द्रयोगान्तराभ्यां व्यतीपातद्वैविध्यात् । एवमुपलक्षणादुक्तरीत्या फलं षड्विंशत्यनन्तरं सप्तविंशतिस्तदा तृतीया वैधृतिः । तत्सञ्ज्ञपातस्यापि योगान्तराभ्यां द्वैविध्यादिति । अत्रोपपत्तिः । विष्कम्भादिर्व्यतीपातः सप्तदशो योग इति~॥~२०~॥\\
 \noindent अथ प्रसङ्गादेतत्तुल्यनिषिद्धे गण्डान्तभसन्धी विवक्षुस्तयोः स्वरूपज्ञानमाह \textendash
%\vspace{2mm}

\begin{quote}
{\ssi सार्पेन्द्रपौष्ण्यधिष्ण्यानामन्त्याः पादा भसन्धयः~।\\ 
तदग्रभेष्वाद्यपादो गण्डान्तं नाम कीर्त्यते~॥~२१~॥}
%
\end{quote}
\newpage

\hspace{3cm} गूढार्थप्रकाशकेन सहितः~। \hfill २९१
\vspace{1cm}


 आश्लेषाज्येष्ठारेवतीनक्षत्राणामन्त्याश्चतुर्थाश्चरणाः नक्षत्रसन्धयो भवन्ति । तदग्रभेषु तेषामाश्लेषाज्येष्ठारेवतीनक्षत्राणामग्रिमनक्षत्रेषु मघामूलाश्विनीनक्षत्रेष्वित्यर्थः । प्रथमचरणो गण्डान्तं नाम प्रसिद्धमुच्यते । यद्यप्याश्लेषाज्येष्ठारेवतीनक्षत्राणामन्तिमं घटिकाद्वयं मघामूलाश्विनीनक्षत्राणामादिमं घटिकाद्वयमिति चतस्रोऽन्तरघटिका गण्डान्तम् । एतदतिरिक्तो नक्षत्रसन्धिः पूर्वनक्षत्रान्तघटिकोत्तरनक्षत्रादिमघटिकेत्यन्तरालघटिकाद्वयं चन्द्रमण्डलसम्बन्धेन घटिकाः सार्धद्वयमिति संहिताविरुद्धं तथापि सूर्योक्तस्य स्वतः प्रामाण्यान्न क्षतिः| अथवैकवाक्यतार्थं पादशब्दः करनेत्रादिवद्द्विसङ्ख्यावाचकः । घटिका इत्यध्याहारश्च । तथा च द्विसङ्ख्यामिता अन्त्यघटिका नक्षत्रसन्धयः । प्रथमद्विघटिकामितः कालो गण्डान्तमित्यर्थः । अत्रापि गण्डान्तत्वाद्भसन्धिकथनमयुक्तं गण्डान्तस्य तदन्तरालरूपत्वात् तथापि तत्कालस्य निषिद्धत्वोक्तितात्पर्याद्विभागद्वयेनोक्तावपि तदन्तरालकाल उत्तरोत्तरकालस्यातिनिषिद्धत्वसूचनान्न क्षतिः~॥~२१~॥\\
 \noindent अथैतदधिकारोक्तानां तुल्यनिषिद्धत्वमाह \textendash
%\%vspace{2mm}

\begin{quote}
{\ssi व्यतीपातत्रयं घोरं गण्डान्तत्रितयं तथा~।\\
 एतद् भसन्धित्रितयं सर्वकर्मसु वर्जयेत् ॥~२२~॥ }
%\vspace{2mm}
\end{quote}
 व्यतीपातानां त्रयं योगवियोगात्मकौ क्रान्तिसाम्यरूपौ द्वौव्यतीपातौ विषुवत्सन्निधौ क्रान्तिसाम्यान्तरेण व्यतीपातस्तयोरेव \textendash


{\tiny{2 O 2}}

\newpage



\noindent २९२ \hspace{4cm} सूर्यसिद्धान्तः 
\vspace{1cm}


\noindent भेदः । न पृथक् ।पञ्चाङ्गान्तर्गतयोगान्तर्गतव्यतीपातश्चेति त्रयं स्पष्टम् । उपलक्षणाद्वैधृतित्रयमपि । योगवियोगात्मकौ क्रान्तिसाम्यरूपौ द्वौ वैधृतिसञ्ज्ञौ । विषुवत्सन्निधौ क्रान्तिसाम्यान्तरेण । वैधृतिसञ्ज्ञस्तु तयोरन्तर्गतः । न पृथक् । पञ्चाङ्गान्तर्गतयोगान्तर्गतवैधृतियोगश्चेति स्पष्टं त्रयम् । केचित्तु व्यतीपातवैधृतिसञ्ज्ञं व्यतीपातद्वयं सञ्ज्ञाभेदेन वैधृतिरिति पूर्वमुक्तेः पञ्चाङ्गान्तर्गतयोगान्तर्गतव्यतीपातश्चेति व्यतीपातचयमिति यथाश्रतमाहुः । घोरं दुष्टं गण्डान्तत्रयम् । तथा घोरं नक्षत्रसन्धित्रयम् । एतत् पूर्वोक्तं घोरम् । अतः कारणात् सर्वमाङ्गल्यकर्मसु शुभेच्छुरेतद्दुष्टं जह्यादित्यर्थः~॥~२२~॥\\ 
\noindent अथार्कांशपुरुषः शिष्टावशिष्टं स्ववाक्यमुपसंहरति \textendash
%\vspace{2mm}

\begin{quote}
{\ssi इत्येतत् परमं पुण्यं ज्योतिषां चरितं हितम्~।\\
रहस्यं महदाख्यातं किमन्यच्छ्रोतुमिच्छसि~॥~२३~॥ }
%\vspace{2mm}
\end{quote}
 हे मय तुभ्यमिति । एवमेतत् । शृणुष्वैकमना इत्यादि सर्वकर्मसु वर्जयेदित्यन्तम् । ज्योतिषां ग्रहनक्षत्रादीनां चरितं माहात्म्यं गणितादिज्ञानमिति यावत् । हितमिह लोके कीर्तिकरम् । परमं पुण्यं परत्र लोक उत्कृष्टं धर्म्यम् । अतएव महद्रहस्यम् । अतिगोप्यमाख्यातं मया कथितम् । अथ स्वोक्तं युक्तप्रतिपादितमेतस्य मनसि निश्चितार्थं नागतमिति तदधरोष्ठस्फुरणदर्शनादनुमितं चास्मै मत्सङ्कोचेन स्वाशङ्कोद्घाटनाशक्तायैतत्प्रश्नप्रतीक्षावसाने मया युक्त्यापि वक्तव्यमित्याशयेनाह\textendash


\newpage

\hspace{3cm} गूढार्थप्रकाशकेन सहितः~। \hfill २९३
\vspace{1cm}


\noindent किमिति । अतः परं त्वमन्यदुक्तातिरिक्तं किं कतरत् श्रोतुं ज्ञातुभिच्छसि । तथाच मया तुभ्यं पूर्वमुक्तं तत्र यत्र यत्र तव संशयस्तत्र तत्र मत्सङ्कोचमुपेक्ष्य मां प्रति प्रश्नस्त्वया कार्यः । तव समाधानं करिष्यामीति भावः~॥~२३~॥\\
\noindent अथाग्रिमग्रन्थस्यप्रतिपादिताधिकारासङ्गतित्वपरिहारायारब्धाधिकारसमाप्तिंफक्किकयाह\textendash 
%\vspace{2mm}

 \begin{center}
 इति सूर्यसिद्धान्ते पाताधिकारः । 
\end{center}
इति स्पष्टम् । देशभेदं ग्रहगणितमिति दशाधिकारात्मकग्रन्थपूर्वार्धं पाताधिकारसमाप्त्या समाप्तमिति तु पाताधिकारान्तस्थेनेत्येतत् परमं पुण्यमित्यादिश्लोकेनैव सूचितम् । 
%\vspace{2mm}

 \begin{quote}
{\qt  रङ्गनाथेन रचिते मृर्यसिद्धान्तटिप्पणे~।\\
 पाताधिकारः पूर्णोऽयं तद्गूढार्थप्रकाशके~॥
 
सूर्यसिद्धान्तगूढार्थप्रकाशकमिदं दलम्~।\\
रङ्गनाथकृतं दृष्ट्वा लभन्तां गणकाः सुखम्~॥}
\end{quote}

 इति श्रीसकलगणकसार्वभौमबल्लालदैवज्ञात्मजरङ्गनाथगणकविरचिते गूढार्थप्रकाशके पर्वखण्डं परिपर्तिमगमत् । 



\begin{center}
 \noindent\rule{7em}{.5pt}   
\end{center}


\newpage
\noindent २९४ \hspace{4cm} सूर्यसिद्धान्तः 
\vspace{1cm}
%\begin{center}
  % २९४  
%\end{center}


 
% {\setlength{\parindent}{5em}
\begin{quote}
{\qt महादेवं वक्रतुण्डं वाणीं सूर्यं प्रणम्य च~।\\
 कृष्णं गुरुं रङ्गनाथो व्याख्यामुत्तरखण्डकम्~॥}
%\vspace{2mm}
\end{quote}

\noindent अथ मुनीन् प्रति मुनिः खर्यांशपुरुषवचनमनूद्यानन्तरं मयासुरेण सूर्यांशपुरुषः पृष्ठ इत्याह \textendash

%\vspace{2mm}

 \begin{quote}
 {\ssi अथार्कांशसमुद्भूतं प्रणिपत्य कृताञ्जलिः~।\\ 
भक्त्या परमयाभ्यर्च्य पप्रच्छेदं मयासुरः~॥~१~॥ }
\end{quote}

 अथ सूर्यांशपुरुषवचनश्रवणानन्तरं मयासुरो मयनामाश्रोता दैत्यः कृताञ्जलिः । रचितहस्ताग्राञ्जलिपुटः । अर्कांशसमुद्भूतं सूर्यांशोत्पन्नं पुरुषं स्वाध्यापकं गुरुं परमयोत्कृष्टयाभक्त्या । आराध्यत्वेन ज्ञानरूपया । अभ्यर्च्य सम्पूज्य प्रणिपत्य नमस्कृत्य । समुच्चयार्थश्चकारोऽत्रानुसन्धेयः । इदं वक्ष्यमाणंपप्रच्छ पृष्टवान्~॥~१~॥\\
 \noindent अथ किं पप्रच्छेत्यतस्तत्प्रश्नानुवादे प्रथमं तत्कृतं भूप्रश्नमाह\textendash 
 
%\vspace{2mm}

\begin{quote}
{\ssi भगवन् किम्प्रमाणा भूः किमाकारा किभाश्रया~।\\
किंविभागा कथं चात्र सप्तपातालभूमयः~॥~२~॥ }
%\vspace{2mm}
\end{quote}

 हे भगवन् भूर्भूमिः किम्प्रमाणा कियत् प्रमाणं यस्याः सा । किमाकारा कथमाकारः स्वरूपं यस्याः सा । किमाश्रया क आश्रयो यस्याः सा । किंविभागा कथं विभागा विभक्तांशा यस्यासा । अत्र भूम्यां पातालभूमयः पातालविभागरूपा आश्रयाः सप्तसङ्ख्याकाः कथं तिष्ठन्ति । चः समुच्चयार्थः । किमाकारेत्यादौ \textendash


\newpage

\hspace{3cm} गूढार्थप्रकाशकेन सहितः~। \hfill २९५
\vspace{1cm}


\noindent प्रत्येकमन्वेति । अयमभिप्रायः । योजनानि शतान्यष्टावित्यादिनावगतभूमानं पञ्चाशत्कोटिविस्तीर्णेति सर्वजनावगतभूमानाद्भिन्नमिति त्वदुक्तभूमाने संशयात् किम्प्रमाणेति प्रश्नः । अन्यथा पूर्वं भूमानकथनात् प्रश्नवैयर्थ्यापत्तेः । उक्तश्रुतत्वापत्तेश्च । एवं लम्बज्याघ्न इत्यादिना स्पष्टपरिध्यन्तरसम्भावात् सर्वजनावगतादर्शाकारतायां भूमौ तदसम्भवेन भवदभिमतत्वाकाररस्यतरिक्त इति किमाकारेति प्रश्नः । एवं तेन देशान्तराभ्यस्तेत्यादिना ग्रहाणां भूम्यभितो भ्रमणसूचनाद्यदाधारे शेषादौ तेषामभितो भ्रमणासम्भवेनाधरे संशयात् किमाश्रयेति प्रश्रः । निराधाराया अवस्थानासम्भवात् । एतेन सर्वजनावगतभूस्वरूपातिरिक्तभूस्वरूपेणोत्तरार्धश्नावपि प्रसङ्गादुक्तौ सङ्गताविति~॥~२~॥\\
\noindent अथ किमाश्रयेति प्रश्नकारणे भूम्यभितो ग्रहभ्रमणे सूर्यस्योपलक्षणत्वेन प्रश्नावाह \textendash
%\vspace{2mm}

\begin{quote}
{\ssi अहोरात्रव्यवस्थां च विदधाति कथं रविः~।\\
कथं पर्येति वसुधां भुवनानि बिभावयन्~॥~३~॥}
%\vspace{2mm}
\end{quote}
 सूर्यः । अहोरात्रव्यवस्थां दिनरात्र्योर्विवेकं कथं केन प्रकारेण विदधाति करोति । अयं भावः । आदर्शाकारभूम्या मध्येमेरुस्तदभिमतो भूम्युपरि प्रदक्षिणतया सूर्यभ्रमणेन स्वदृश्यविभागे सूर्ये दिनं स्वादृश्यविभागे रात्रिरिति सर्वजनावगताद्भवदभिप्रेतं सूर्यभ्रमणं भिन्नं तर्हि त्वन्मते सूर्यो दिनं रात्रिं च व्यवधायकाव्यवधायकौ विना कथं करोति । अन्ये ग्रहा \textendash


\newpage

\noindent २९६ \hspace{4cm} सूर्यसिद्धान्तः
\vspace{1cm}


\noindent अपि कथं खदिनं खराचिं च कुर्वन्ति। सूर्योपलक्षणत्वादिति। अथ भूम्यभितो भ्रमणाङ्गीकारे भूरेव व्यवधायिकेत्यहोरात्रव्यवस्था युक्तैवेत्यतः प्रश्नान्तरमाह\textendash कथमिति~। सूर्यो भुवनानिवक्ष्यमाणखरूपाणि विभावयन् प्रकाशयन् सन् वसुधां पृथ्वों कथं केन प्रकारेण पर्येति प्रदक्षिणतया भ्रमति। भूमेर्निराधारावस्थानासम्भवेन साधारत्वे भूम्यभितो ग्रंहभ्रमणमाधारेबाधितमिति भावः~॥ ३~॥\\
\noindent प्रश्नावाह \textendash
%\vspace{2mm}

\begin{quote}
{\ssi देवासुराणामन्योन्यमहोराचं विपर्ययात्~।\\
किमर्थं तत् कथं वा स्याद्भानोर्भगणपूरणात्~॥~४~॥}
%\vspace{2mm}
\end{quote}
 पूर्वार्धं व्याख्यातम्। किमर्थं कोऽर्थोऽभिप्रायो यस्य तदित्यहोरात्रविशेषणम्। देवासुरथोर्दिनं रात्रिश्चाभिन्ना कथं नोक्ता व्यत्यासे नियामकाभावादिति भावः। तद्देवासुरयोरहोरात्रं सूर्यस्य द्वादशराशिभोगादितार्थः। कथं कुतः। वाकारः समुच्चये भवति। उभयत्र नियामकाभावादुभयत्र ममसन्देहः। दिनरात्र्योः सूर्यदर्शनादर्शननियामकत्वाद्यत्र सूर्यंषण्मासावधि देवाः पश्यन्ति तत्रासुरा न पश्यन्ति। यत्र देवाः षण्मासावधि न पश्यन्ति तत्रासुराः पश्यन्तीह भगवताबोधनीय इति भावः~॥~४~॥\\
\noindent अथ प्रश्नान्तरं पूर्वोक्तश्लोकद्वयस्यतात्पर्याय प्रश्नं चाह \textendash
%\vspace{2mm}

\begin{quote}
{\ssi पित्र्यं मासेन भवति नाडीषष्ट्या तु मानुषम्~।\\
तदेव किल सर्वत्र न भवेत् केन हेतुना॥~५~॥}
%
\end{quote}
\newpage

\hspace{3cm} गूढार्थप्रकाशकेन सहितः~। \hfill २९७
\vspace{1cm}


 पितृणामिदमहोरात्रं मासेन दर्श वधिकचान्द्रमासेन केन हेतुनेत्यस्य प्रत्येकं समन्वयात् केन कारणेन भवति। अन्यथा प्रश्नानपपत्तेः। सावनघटीषष्ट्या मानुषं मनव्याणामहोरात्रं केन कारणेन भवतीत्यर्थः। न च यथा दिव्यं तदह उच्यत इत्युक्तं तथा पूर्वोक्तेऽप्यमानुषाहोरात्रयोरनुक्तेः प्रश्नावसङ्गताविति वाच्यम्। दिव्यं तदह उच्यत इत्यनेनैव पूर्वाक्तसावनाहोरात्रत्वेन प्रत्यक्षत्वाच्च परिशेषान्मासस्यैव पित्र्याहोरात्रत्वसिद्धेः। ननु तथापि प्रत्यक्षसिद्धमानुषाहोरात्रे प्रश्नोऽनुपपन्न एवेत्यतस्तात्पर्यप्रश्नमाह\textendash तदेवेति~। तन्मानुषाहोरात्रम् । एवकारस्तदन्यनिरासार्थकः। सर्वत्र सर्वलोके किल निश्चयेन केन कारणेन न स्यात्। पितृदेवदैत्यानामप्रत्यक्षमहोरात्रं कथमङ्गीकृतम्। कथं च मानुषाहोरात्रं प्रत्यक्षसिद्धं तेषां माने नोक्तमित्यर्थः~॥~५~॥\\
 \noindent अथाहर्गणादवगतदिनमासवर्षेश्वरेषु तत्प्रसङ्गाद्धोरेश्वरे प्रश्नं पश्चाद्धजन्तोऽतिजवादित्यत्र प्रश्नद्वयं चाह\textendash
 
%\vspace{2mm}


\begin{quote}
  {\ssi दिनाब्दमासहोराणामधिपा न समाः कुतः~।\\
कथं पर्येति भगणः सग्रहोऽयं किमाश्रयः~॥~६~॥}
\end{quote}
%\vspace{2mm}

 दिनवर्षमासहोराणां खामिनोऽभिन्नाः कुतः कस्मान्न भवन्ति। यथा दिनाधिपतित्वं सूर्यादीनां क्रमेण तथा प्रथमादिमासवर्षक्रमेण सूर्यादीनां क्रमेण मासवर्षाधिपत्वं युक्तम। आनयने युक्त्यप्रलिपादनादिति भावः। यद्यपि र्पूवं होरेश्वरानयनं नोक्तमिति तत्प्रश्नोऽसङ्गतस्तथापि लोके प्रसिद्धतरो होरेश्वर\textendash



\newpage


\noindent २९८ \hspace{4cm} सूर्यसिद्धान्तः
\vspace{1cm}


\noindent स्त्वया किमर्थं नोक्त इति तत्प्रश्नतात्पर्यमिति ध्येयम्। भगणो नक्षत्रसमूहः सग्रहो ग्रहसहितः कथं केन प्रकारेण पर्येति भ्रमति। नक्षत्राणि ग्रहाश्च केन प्रयुक्ताः सन्तो भूम्यभितो भ्रमन्तीत्यर्थः। अथैषामन्तरिक्षावस्थानेऽपि प्रश्नमाह\textendash अयमिति~। सग्रहो भगणो दृश्यमानः किमाश्रयः क आधारो यस्येति। विनाधारमन्तरिक्षावस्थानं न सम्भवतीत्यर्थः~॥~६~॥\\
\noindent ननु कक्षा एवाधाराः पूर्वं तत्रैव स्वमार्गगा इत्युक्तेरित्यतः कक्षाणां प्रश्नचतुष्टयमाह\textendash 
%\vspace{2mm}


\begin{quote}
 {\ssi भूमेरुपर्युपर्यूर्ध्वाः किमुत्सोधाः किमन्तराः~।\\
ग्रहर्क्षकक्षाः किम्मात्राः स्थिताः केन क्रमेण ताः~॥~७~॥}
\end{quote}
%\vspace{2mm}

 भूमेः सकाशादूर्ध्वमुच्चा ग्रहर्क्षकक्षा ग्रहनक्षत्राणामाकाशे मार्गाः किमुत्सेधाः कियानुत्सेध उच्चता यासां ताः। भूमेः सकाशाद्ग्रहनक्षत्रमार्गकक्षाः कियदन्तरेण सन्तीत्यर्थः। किमन्तराः कियदन्तरालं यासां ताः। उत्तरोत्तरमुच्चा अपि परस्परं तासां कियदन्तरालमित्यर्थः। किम्मात्राः किमात्मिका।  किंस्वरूपाः किम्प्रमाणा वा। ता ग्रहनक्षत्रकक्षाः केन क्रमेणाधिष्ठिताः सन्ति। पूर्वं कस्तदुत्तरं क इत्यादिक्रमो न ज्ञात इत्यर्थः~॥~७~॥\\
\noindent अथानुभवप्रश्नं तत्प्रसङ्गात् सूर्यकिरणप्रचारप्रश्नं च पूर्वोक्तमानानां प्रश्नद्वयं चाह\textendash 

%\vspace{2mm}


\begin{quote}
 {\ssi ग्रीष्मे तीव्रकरो भानुर्न हेमन्ते तथाविधः~।\\
कियती तत्करप्राप्तिर्मानानि कति किं च तैः~॥~८~॥}
\end{quote}



\newpage


\hspace{3cm} गूढार्थप्रकाशकेन सहितः~। \hfill २९९
\vspace{1cm}


 ग्रीष्मर्तौ सूर्यो यथा तीक्ष्णकिरण उष्णकिरणस्तथाविधस्तादृशो हेमन्ते न भवतीति किम्। सूर्यस्य किरणानां प्राप्तिर्गमनपद्धतिः कियती कियत्प्रमाणा। मानान्युक्तानीति तत्तत्त्वं सम्यङ्ग ज्ञातमित्यर्थः। तैर्मानैः किं प्रयोजनम्। चः समुच्चयार्थः प्रत्येकमन्वेति~॥~८~॥\\
 \noindent अथास्य प्रश्नमुपसंहरति\textendash
 
%\vspace{2mm}


\begin{quote}
 {\ssi एतं मे संशयं छिन्धि भगवन् भूतभावन~।\\
अन्यो न त्वामृते छेत्ता विद्यते सर्वदर्शिवान्~॥~९~॥ }
\end{quote}
%\vspace{2mm}

 हे भगयवन् षड्गुणैश्वर्यसम्पन्न। सर्वबोधकेति तात्पर्यार्थः। भूतभावन भूतस्यातीतकालस्य भावना विचारो यस्य। भूतस्योपलक्षणाद्वर्तमानभविष्यतोरपि कालज्ञेति सिद्धोऽर्थः। त्वं मे मम। एतमुक्तं संशयम्। जात्यभिप्रायेणैकवचनम्। तेन मत्कृतान् प्रश्नानित्यर्थः। छिन्धि छेदय। नन्वहमिदानीमेतद्क्त्यैव वक्तुं न शक्रोव्यन्यास्मात् संशयान् दूरीकुर्वित्यत आह\textendash अन्य इति~। त्वामृते विना। अन्यः सर्वदर्शिवान् सर्वदष्टा। सर्वज्ञ इत्यर्थः। छेत्ता संशयापनोदकः। न विद्यते नास्ति। तथाचैतावत्कालपर्यन्तं यथोक्तं तथान्यदपि कृपया वक्तव्यमिति भावः~॥~९~॥\\
\noindent अथ मनीन् प्रति मनिर्मयासुरोक्तप्रश्नानुवादं कृत्वा सूर्यांशपरुषो मयासुरं प्रति पुनर्वदति स्मेत्याह\textendash 

%\vspace{2mm}


\begin{quote}
 {\ssi इति भक्त्योदितं श्रुत्वा मयोक्तं वाक्यमस्य हि~।\\
रहस्यं परमध्यायं ततः प्राह पुनः स तम्~॥~१०~॥}
\end{quote}



\newpage

\noindent ३०० \hspace{4cm} सूर्यसिद्धान्तः 
\vspace{1cm}


 स सूयाज्ञपुरुषः। इति पूर्व्वाक्तम्। भक्त्याराध्यज्ञानेन। उदितमुत्पन्नम्। मयेन कथितं वचनम् श्रुत्वाकर्ण्य। पनर्द्वितीयवारं ततः पूर्वार्धोक्त्यनन्तरं तं मयासुरं प्रति परं द्वितीयमध्यायं ग्रन्थम्। ग्रन्थस्योत्तरखण्डमित्यर्थः। अस्य ग्रन्थपूर्वखण्डस्य हि निश्चयेन रहस्यं गोप्यत्वेन तत्त्वभूतं प्राह\textendash प्रकर्षेणावददित्यर्थः~॥~१०~॥\\
 \noindent अथ सूर्यांशपुरुषो मयासुरं प्रति यदुक्तं सावधानतया श्रोतव्यमित्याह\textendash
 
%\vspace{2mm}


\begin{quote}
 {\ssi श्टणुथ्वैकमना भूत्वा गुह्यमध्यात्मसञ्ज्ञितम्~।\\
प्रवक्ष्याम्यतिभक्तानां नादेयं विद्यते मम~॥~११~॥ }
\end{quote}
%\vspace{2mm}

 यतः कारणात्। अतिभक्तानामत्यन्तमद्भजनकारकाणां भवादृशां मम सूर्यांशपुरुषस्य। अदेयमदातव्यं वस्तु न विद्यते। अतः कारणादहं त्वां प्रति गुह्यं गोप्यमध्यात्मसञ्ज्ञितमध्यात्मज्ञानसञ्ज्ञं यत् प्रवक्ष्यामि कथयिष्यामि तत् त्वमेकमना एकस्मिन् मदुक्ते मनो विद्यते यस्यासौ भूत्वा श्टणुय्व श्रोत्रद्वारात्ममनःसंयोगेन कुर्वित्यर्थः~॥~९९~॥\\
 \noindent गुह्यं वक्ष्यामि यदुकं तदाह\textendash
 
%\vspace{2mm}


\begin{quote}
 {\ssi वासुदेवः परं ब्रह्म तन्मूर्त्तिः परुषः परः~।\\
अव्यक्तो निर्गुणः शान्तः पञ्चविंशात् परोऽव्ययः~॥~१२~॥ }
\end{quote}
%\vspace{2mm}


 वसत्यस्मिन् जगत् समस्तमसौ वा जगति समस्ते वसतीति वसतेरुणि वासुः। देवनाद्भासनाद्देवः। वासुश्चासौ देवश्चेति वासुदेवः। तथाचोक्तम्\textendash



\newpage

\hspace{3cm} गूढार्थप्रकाशकेन सहितः~| \hfill ३०१
\vspace{1cm}



\begin{quote}
{\qt सर्वत्रासौ समस्तं च वसत्यत्रेति वै यतः~।\\
 अतोऽसौ वासुदेवाख्यो विद्वद्भिः परिगीयते~॥}
%\vspace{2mm}
\end{quote}

इति। न तु वसुदेवस्यापत्यमिति विग्रहः। तस्य जगत्कारणतागिरूपणावसरेऽनुपयोगात्। अस्मत्पक्षे पुनरुपादाने कार्यस्याधारतया कार्ये वोपादानस्यानुस्यूततया वा स उपयुक्त एव। तथाचोक्तं श्रुतौ। ईशावास्यमिदं सर्वमित्यादि। भागवते च। अजनि च यन्मयं तदविमुच्यमियं भृभवेदिति। जीवानामपिब्रह्मात्मकतया तद्वारणाय परमिति सर्वोत्तममित्यर्थकम्। 

%\vspace{2mm}

 \begin{quote}
 {\qt यस्मात् क्षरमितीतोऽहमक्षरादपि चोत्तम~।\\ 
अतोऽस्मि वेदे लोके च प्रथितः पुरुषोत्तमः~॥ }
%\vspace{2mm}
\end{quote}

इति स्मृतेः। तन्मूर्तिस्तस्य वासुदेवस्य मुर्तिरंशः। इदं विशेषणंवक्ष्यमाणस्य सङ्कर्षणस्य। चिन्मूर्तिरिति पाठस्तु प्रामादिकः। वासुदेवः सङ्कर्षण इत्यस्माद्वासुदेवात् सङ्कर्षण इत्यस्यार्थस्यविवक्षितसगप्रतीतेः। अव्यक्त इत्यतीन्द्रिय इत्यर्थः। तथाच श्रुतिः। 

%\vspace{2mm}

\begin{quote}
{\qt नतं विदाथ य दमा जजा न्यद्युशाकमन्तरं बभूव~।\\
 नीहारेण प्रावृता जल्प्या चासुदृप उक्थशा सञ्चरन्ति~\\
न संदृशं तिष्ठति रूपमस्य न चक्षुषा पश्यति कश्चनैनम्~। }
%\vspace{2mm}
\end{quote}

इति। अव्यक्तत्वे हेतुर्निर्गुण इति। शान्तः षडूर्मिरहितत्वात्। पञ्चविंशात् परः। षोडश विकृतयः सप्त प्रकृतिविकृतयो मूलप्रकृतिश्चेति चतुर्विंशतितत्त्वानि। पञ्चविंशस्तु जीवस्तस्मात् पर इत्यर्थः। पञ्चविंशात्मक इति पाठे जगदात्मक इति~॥~१२~॥\\
\noindent शुद्धस्य ब्रह्मणो जगत्कारणत्वासम्भवादाह \textendash


\newpage


\noindent ३०२ \hspace{4cm} सूर्यसिद्धान्तः
\vspace{1cm}
 

\begin{quote}
{\ssi  प्रकृत्यन्तर्गतो देवो बहिरन्तश्च सर्वगः~।\\
सङ्कर्षणोऽयं सृष्ट्वादो तासु वीर्यमवासृजत्~॥~१३~॥ }
\end{quote}
%\vspace{2mm}


 प्रकृत्यन्तर्गतो आयोपहितो बहिरन्तश्च सर्वगो जगदुपादानत्वात्। एतानि सर्वाणि विशेषणानि सङ्कर्षणस्य वासुदेवोशस्यापि वासुदेवात्मकताध्यवसानेन बोध्यानि। वासुदेवांशात्मकः सङ्कर्षणः प्रथमं जलानि निर्माय। तास्वष्मु। वीर्यं शक्तिविशेषम्। अवासृजच्चिक्षेप~॥~१३~॥\\
 \noindent ततः किमत आह \textendash
%\vspace{2mm}

\begin{quote}
{\ssi तदण्डमभवद्वैमं सर्वत्र तमसावृतम्~।\\
तत्रानिरुद्वः प्रथमं व्यक्तीभूतः सनातनः~॥~१४~॥ }
%\vspace{2mm}
\end{quote}

 तत् तच्छक्तिमिलितं जलं हैमं सौवर्णं गोलाकारं सर्वत्र यहिरन्तश्चान्धकारेणावृतमभवत्। अन्धकारसहिताकाशे सुवर्णाण्डमजनीत्यर्थः। तत्र सुवर्णाण्ड आदावनिरुद्धः सनातनी नित्यो वसुदेवांशसङ्कर्षणोंऽशरूपत्वाद्व्यक्तीभूतोऽभिव्यक्तः। न तूत्पन्नः। सत्कार्यंवादाभ्युपगमात्। यथा तिलेभ्यस्तैलं सदेवाभिव्यक्तं नतूत्पन्नम्॥~१४~॥\\
 \noindent अथास्याभिधान्तराणि लोकसुज्ञानार्थमाह \textendash
%\vspace{2mm}

\begin{quote}
{\ssi हिरण्यगर्भो भगवानेष च्छन्दसि पद्यते~।\\
 आदित्यो ह्यादिभूतत्वात् प्रसूत्या सूर्य उच्यते~ १५ ॥ }
%\vspace{2mm}
\end{quote}

 एष सङ्कर्षणांशोऽनिरुद्धो भगवान् षाङ्गुण्यैश्वर्यसम्पन्नच्छन्दसि वेदे हिरण्यगर्भः सुवर्णाण्डमधरूपगर्भे स्थितत्वात् पयते।


\newpage

\hspace{3cm} गूढार्थप्रकाशकेन सहितः~। \hfill ३०३
\vspace{1cm}


\noindent निरूप्यते विदेऽस्य हिरण्यगर्भ इति प्रसिद्धमभिधान्तरमित्यर्थः। हि निश्चयेनादित्यः प्रथममभिव्यक्तत्वादुच्यते। प्रसूत्या। अस्माज्जगतोऽभिव्यक्ततयायमनिरुद्धः सूर्य उच्यते। हिरण्यगर्भः समवर्तताग्रे भूतस्य जातः पतिरेक आसीत्। इति श्रुतिः~॥~१५~॥\\
\noindent अस्य स्वरूपं स्थितिं चाह \textendash
%\vspace{2mm}

\begin{quote}
{\ssi परं ज्योतिस्तमःपारे सूचर्योऽयं सवितेति च~।\\
पर्येति भुवनान्येष भावयन् भूतभावनः~॥~१६~॥ }
%\vspace{2mm}
\end{quote}

 अयमनिरुद्धः सूर्यनामकः सविता। इति नाम्ना। चः समुञ्चये। प्रसिद्धः। तमःपारेऽन्धकारस्य विरामे परमुत्कृष्टं ज्योतिस्तेजोरूपम्। अन्धकारनाशक इति तात्पर्यार्थः। आदित्यवर्णंतमसस्तु पारे इति श्रतिः। एष सविता भूतभावनः प्राणात्पत्तिस्थितिसंहारकारको भुवनानि वक्ष्यमाणानि भावयन् प्रकाशयन् पर्येति सुवर्णाण्डमध्ये सदा भ्रमति~॥~१६~॥\\
 \noindent अथ परंज्योतिरिति पादं' विवृण्वन्नन्यतमस्येतत् स्वरूपं श्लोकाभ्यामाह \textendash
%\vspace{2mm}

\begin{quote}
{\ssi प्रकाशात्मा तमोहन्ता महानित्येष विश्रतः~।\\
ऋचोऽस्य मण्डलं सामान्युस्रामूर्तिर्यजूंषि च~॥~१७~॥

त्रयीमयोऽयं भगवान् कालात्मा कालकृद्विभुः~।\\
सर्वात्मा सर्वगः सूक्ष्मः सर्वमस्मिन् प्रतिष्ठितम्~॥~१८~॥ }
%\vspace{2mm}
\end{quote}
 प्रकाशरूपोऽन्धकारनाशकोऽतएवैष अनिरुद्धाख्यः सूर्योमहान् महत्तत्त्वमिति। एवं विश्रुतो वेदपराणादौ निरुक्तो \textendash



\newpage


\noindent ३०४ \hspace{4cm} सूर्यसिद्धान्तः
\vspace{1cm}


\noindent ऽस्य निरुक्तस्य सूर्यस्य। ॠचः। ॠग्वेदमन्त्रा मण्डलं सामानिसामवेदमन्त्रा उस्त्राः किरणा यजूंषि यजुर्वेदमन्त्रा मूर्तिः स्वरूपम्। चः समुच्चये। अतएवायं निरुक्तो भगवान् षाङ्गुण्यैश्वर्यसम्पन्नः। त्रयीमयो वेदत्रयात्मकः। कालरूपः कालस्य कारणम्। विभुर्जगदुत्पत्तिस्थितिनाशाय समर्थः। अतएव सर्वात्माजगत्स्वरूपः सर्वगः सर्वत्र स्थितो व्यापकः सूक्ष्मोऽव्यापकमूर्तिधारी। अस्मिन् निरुक्तसूर्ये सर्वं जगत् प्रतिष्ठितम्। एतेनव्यापकाव्यापकत्वयोरत्राविरोधः~॥~१८~॥\\
\noindent अथ पर्येति भुवनात्र्येवेत्यर्धं विवृणोति \textendash
%\vspace{2mm}

\begin{quote}
{\ssi रथे विश्वमये चक्रं कृत्वा संवत्सरात्मकम्~।\\ 
छन्दांस्यश्वाः सप्त युक्ताः पर्यटत्येष सर्वदा*~॥~१९~॥ }
%\vspace{2mm}
\end{quote}

 त्रिलोक्यात्मके रथे संवत्सरात्मकं द्वादशमासात्मकं वर्षचक्रंनियोज्य सप्त छन्दांसि गायत्र्युष्णिगनुष्टुब्दुहतीपङ्कित्रिष्टुजगत्योऽश्वाः युक्ताः संयोजिता कृत्वा। छन्दांस्यश्वास्तत्रयुत्क्केतिपाठे सप्ताश्वान् रथे नियोज्येत्यर्थः। सर्वदा नित्यमेषोऽनिरुद्धनामा पर्यटतिभ्रमति~॥~१९~॥\\
\noindent अथास्य स्वरूपं ब्रह्मण उत्पत्तिंचाह \textendash 
%\vspace{2mm}

\begin{quote}
{\ssi त्रिपादममृतं गुह्यं पादोऽयं प्रकटोऽभवत्~।\\ 
सोऽहङ्कारं जगत्सृष्ट्यै ब्रह्माणमसृजत् प्रभुः~॥~२०~॥}
%\vspace{2mm}
\end{quote}

 अस्य वेदात्मनस्त्रिपादं चरणचयममृतं दिवि ज्ञेयम्। अत \textendash


\noindent\rule{\linewidth}{.5pt}

\begin{center}
 * पर्येत्येष वशी सदा इति पाठान्तरम्।
\end{center}

\newpage

\hspace{3cm} गूढार्थप्रकाशकेन सहितः~। \hfill ३०५
\vspace{1cm}


\noindent एव गुह्यमगम्यमिदम्। पादश्चतुर्थश्चरणः। अयं स्थावरजङ्गमात्मकजगद्रूपः प्रकटः प्रत्यक्षोऽभवत्। त्रिपादूर्ध्व उदैत् पुरुषः पादोस्येहाभवत् पुनरिति श्रुतिरपि व्यक्ता। सोऽनिरुद्धनामाप्रभुरुत्पत्तिसमर्थः। अहङ्कारतत्त्वरूपं ब्रह्याणं पुरुषं अगत्सृष्ट्यै जगत्सर्जननिमित्तमसृजदुत्पादयामास ~॥~२०~॥ \\
\noindent अथोत्पादितब्रह्मपुरुषं जगत्सर्जनार्थं नियुज्य स्वयं भ्रमन्नवतिष्ठत इत्याह \textendash
%\vspace{2mm}

\begin{quote}
{\ssi तस्मै वेदान् वरान् दत्त्वा सर्वलोकपितामहम्~।\\
प्रतिष्ठाप्याण्डमध्येऽथ स्वयं पर्येति भावयन्~॥~२१~॥ }
%\vspace{2mm}
\end{quote}
 अथ ब्रह्योत्पादनानन्तरं स्वयमनिरुद्धनामा। तस्मै। उत्पादितब्रह्मपुरुषाय। वरानुत्कृष्टान् वेदोक्तमार्गेण सृष्टिसर्जनार्थं सर्वलोकानां पितामहरूपं तं ब्रह्माणं सुवर्णाण्डमध्ये प्रतिष्ठाप्यविधाय। चोऽत्रानुसन्धेयः। भावयन् प्रकाशयन् सन् पर्येतिभ्रमति~॥~२१~॥ \\
\noindent अथ जातसृष्टोच्छो ब्रह्मा चन्द्रसूर्यावस्मत्प्रत्यक्षावुत्पादयामासेत्याह \textendash
%\vspace{2mm}

\begin{quote}
{\ssi अथं सृष्ट्या मनश्चक्रे ब्रह्माहङ्कारमुर्तिभृत्~।\\
मनसश्चन्द्रमा जज्ञे सूर्योऽक्ष्णोस्तेजतां निधिः~॥~२२~॥ }
%\vspace{2mm}
\end{quote}
 अथाधिकारप्राप्त्यनन्तरम्। अहङ्कारतत्त्वमन्तर्तिधारको ब्रह्मासृष्ट्यां मनोऽन्तःकरण चक्रे करोति स्म। ब्रह्मणोऽहं सृष्टिं करोमीतोच्छा जातेत्यर्भः। अनन्तरं तस्य मनसः सकाशाच्चन्द्रमाजज्ञ उत्पन्नः। चन्द्रो भवत्विति मनसा चन्द्रो जात इत्यर्थः।
%

\newpage

\noindent ३०६ \hspace{3cm} सूर्यसिद्धान्तः
\vspace{1cm}


अक्ष्णोर्नेत्राभ्यां सकाशात् तेजसां निधिराकरभूतः सूर्य उत्पन्नः। चक्षुरिन्द्रियस्य तैजसत्वात्~॥~२२~॥\\
\noindent अथ महाभृतोत्पत्तिमाह \textendash
%\vspace{2mm}

 \begin{quote}
{\ssi  मनसः खं ततौ वायुरग्निरापो धरा क्र्मात्~।\\
गुणैकवृद्ध्या पञ्चैव महाभूतानि जज्ञिरे~॥~२३~॥ }
%\vspace{2mm}
\end{quote}

 मनस आकाशो भवत्वितीच्छयात्मनः खमाकाशं तत आकाशात् क्रमाद्यथोत्तरं वायुरग्निर्जलं पुथिवी। आकाशाद्वायुर्वायोरग्निरग्रेरापोऽड्भ्यः पृथिवीति गुणैकवृड्व्या गुणस्यैकोपचयेनमहाभूतानि पञ्चसङ्ख्याकानि। एवकारामूनाधिकव्यवच्छेदः। जज्ञिरे। उत्पन्नानि। शब्दगुणसहितमाकाशं शब्दस्पर्शगुणद्वयसमेतो वायुः शब्दस्पर्शरूपात्मकगुणत्रयसमेतोऽग्निः शब्दस्पर्शंरूपरसात्मकगुणचतुष्टयममेतं जलं शब्दस्पर्शरूपरसगन्धात्मकगुणपञ्चकसमेता पृथिवीति स्फुटार्थाः~॥~२३~॥\\
\noindent  अथ चन्द्रसूर्ययोः स्वरूपं वदन् पञ्चताराणामुत्पत्तिमाह \textendash
%\vspace{2mm}

\begin{quote}
{\ssi अग्नीषोमौ भानुचन्द्रौ ततस्त्वङ्गारकादयः*~।\\
तेजोभूखाम्बुवातेभ्यः क्रमशः पञ्च जज्ञिरे~॥~२४~॥}
%\vspace{2mm}
\end{quote}

 सूर्यचन्द्रौ प्रागुदितोत्पत्ती अग्निषोमौ सूर्योऽग्निस्वरूपस्तेजोगोलकश्चाक्षुषत्वात्। चन्द्रस्तु सोमस्वरूपः। मद्यस्य सोमवाच्युत्वाज्जलगोलरूपः। अग्नीषेमाविति प्रयोग च्छान्दसिकः। 



\noindent\rule{\linewidth}{.5pt}

\begin{center}
 * भूतावङ्गारकादयः इति वा पाठः।
\end{center}


\newpage


\hspace{3cm} गूढार्थप्रकाशकेन सहितः~| \hfill ३०७
\vspace{1cm}


\noindent ततोऽनन्तरमङ्गारकादयो भौमादयः पञ्च तारा ग्रहास्तेजोभूखाम्बुवातेभ्यः क्रमादुत्पन्नाः। तुकारादुक्तभूतस्य भागाधिक्यमन्यभूतानां च भागसाम्यमित्यर्थः। मङ्गलस्तेजस उत्पन्नोऽतएवायमङ्गारक उच्यते। बुधो भूमितः। बृहस्पतिराकाशात्। शुक्रोजलात्। शनिर्वायोः~॥~२४~॥\\
अथ राशीन् नक्षत्राणि चाह \textendash
%\vspace{2mm}

 \begin{quote}
{\ssi पुनर्द्वादशधात्मानं विभजद्राशिसञ्ज्ञकम्~।\\
नक्षत्ररूपिणं भूयः सप्तविंशात्मकं वशी~॥~२५~॥}
%\vspace{2mm}
\end{quote}

 पुनरनन्तरमात्मानं द्वादशधा द्वादशस्थानेषु राशिसञ्ज्ञकंविभजत्। मनःकल्पितं वृत्तं द्वादशविभागं राशिवृत्तमकरोदित्यर्थः। भूयो द्वितीयवारमात्मानं नक्षत्ररूपिणं सप्तविंशात्मकं विभजत्। मनःकल्पितं तदेव वृत्तं सप्तविंशतिविभागंचाकरोदित्यर्थः। ननु न्यूनाधिकविभागाः कथं न कृता उक्तं-\/-सङ्ख्यायां नियामकाभावादित्यत आह। वशीति। इच्छाविषयं वशं विद्यते यस्येति वशी खतन्त्रेच्छस्य नियोगानर्हत्वात्। स्वेच्छया तत्सङ्ख्याका विभागाः कृता इति भावः। सप्तविंशतिविभागव्यञ्जकानि नक्षत्राणि तारात्मकानि निर्मितानीश्वर्थसिद्धमु~॥~२५~॥\\
\noindent अथ चराचरं जगदकरोदित्याह \textendash
%\vspace{2mm}

\begin{quote}
{\ssi ततश्चराचरं विश्वं निर्ममे द्ववपूर्वकम्~।\\
ऊर्ध्वमध्याधरेभ्योऽथ स्रोतोभ्यः प्रकृतीः सृजन्~॥~२६~॥ }
%\vspace{2mm}
\end{quote}
 ततः स चक्रग्रहसर्जनानन्तरमर्ध्वमध्याधरेभ्यः श्रेष्ठमध्याधमेभ्यः स्त्रोताभ्यो व्यक्तिभ्यः प्रकृतीः सत्त्वरजस्तमोविभेदात्मप्र \textendash


{\tiny{2 Q 2}}


\newpage


\noindent ३०८ \hspace{3cm} सूर्यसिद्धान्तः
\vspace{1cm}


\noindent कृतीः सूजन् विसृजन् देवपूर्वकं देवमनुष्यासुरादिकं विश्वंजगच्चराचरं चेतजाचेतनात्मकं निर्ममे कृतवान्~॥~२६~॥ \\
\noindent अथ रचितपदार्थानामवस्थानं कृतवानित्याह \textendash
%\vspace{2mm}

\begin{quote}
{\ssi गुणकर्मविभागेन सृष्ट्वा प्राग्वदनुक्रमात्~। \\
विभागं कल्पयामास यथास्वं वेददर्शनात्~॥~२७~॥ }
%\vspace{2mm}
\end{quote}
 गुणाः सत्त्वरजस्तमोरूपाः। कर्म ण्कर्वजन्मार्जितं सदसत् कर्म। अनयोर्विभागेनैककिरणात्मकेन प्राग्वच्चन्द्रसूर्यादिप्रागुक्तसृष्टिरित्यनुक्रमात् सृष्ट्वा देवमनुष्यासुरभूमिपर्वतादिकचराचरमर्जनं कृत्वा वेददर्शनाद्वेदोक्तप्रकाराद्यथास्वं यथादेशं वथाकालंविभागमवस्थानविभागं कल्पयामास कृतवान्~॥~२७~॥\\ 
\noindent केषामित्यत आह  \textendash
%\vspace{2mm}

 \begin{quote}
 {\ssi ग्रहनक्षत्रताराणां भूमेर्विश्वस्य वा विभुः~।\\
देवासुरमनुष्याणां सिद्धानां च यथाक्रमम्~॥~२८~॥}
%\vspace{2mm}
\end{quote}

 विभुर्नियोजनसमर्थो ब्रह्या ग्रहनक्षत्रयोर्बिम्बानां पृथिव्यास्त्रैलोक्यस्य। वाकारः समूच्चये। आकाशेऽवस्थानं कृतवान्। तत्र ग्रहनक्षत्राणां यथाकालमनियतावस्थानम्। पृथिव्यास्तु नियतावस्थानम्। पृथिव्यां तु त्रैलोक्यस्य यथार्देशमवस्थानम्। तत्र यथाक्रमं यथायोग्यं देवासुरमनुष्याणां सिद्धानाम्। चः समुच्चये। अवस्थानं यथादेशं कृतवान्~॥~२८~॥ \\
\noindent ननु सर्वत्राकाशस्यसत्त्वाद्ब्रह्माण्डमध्यस्थेन ब्रह्मणा ग्रहनक्षत्राणां भूमेश्चावस्थानं \textendash


\newpage


\hspace{3cm} गूढार्थप्रकाशकेन सहितः~।  \hfill ३०९
\vspace{1cm}


\noindent ब्रह्माण्डवहिराकाशे कृतमथवा ब्रह्माण्डान्तराकाशे कृतमित्यत आह \textendash
%\vspace{2mm}

 \begin{quote}
{\ssi ब्रह्माण्डमेतत् सुषिरं तत्रेदं भूर्भुवादिकम्~।\\
कटाहद्वितयस्यैव सम्पुटं गोलकाकृतिः~॥~२९~॥}
%\vspace{2mm}
\end{quote}
 एतत् प्रागुक्तं ब्रह्मणाधिष्ठितं सुवर्णाण्डं सुषिरमवकाशात्मकं तत्रावकाश इदं जगत् भूर्भुवादिकं भूर्भुवःस्वर्गात्मकमवस्थितं न बाहिः। नन्वण्डस्य गोलाकारत्वेनान्तरवकाशात्मकत्वमसम्भवतीत्यत आह\textendash कटाहद्वितयस्येति~। कटाहोऽर्धगोलाकारं सावकाशं पात्रं तस्य द्वितयं द्वयं समं तस्य। एवकारोन्यूनाधिकव्यवच्छेदकार्थः। सम्पुटमाभिमख्येन मिलितं गोलकाकृतिर्गोलाकारः स्यात्। तथाच न क्षतिः~॥~२९~॥\\
\noindent अथब्रह्माण्डान्तः परिधिं वदंस्तदन्तर्भग्रहादिकमाकाशे यथास्थानंपरिभ्रमतोति श्लोकाभ्यामाह \textendash
%\vspace{2mm}

\begin{quote}
{\ssi ब्रह्माण्डमध्ये परिधिर्व्योमकक्षाभिधीयते~।\\
तन्मध्ये भ्रमणं भानामधोऽधः क्रमशस्तथा~॥~३०~॥

मन्दामरेज्यभूपुत्रसूर्यशुक्रेन्दुजेन्दवः~।\\
परिभ्रमन्त्युधोऽधस्थाः सिद्वविद्याधरा घनाः~॥~३१~॥}
%\vspace{2mm}
\end{quote}
 ब्रह्माण्डान्तः परिधिस्तुल्यवृत्तमानं व्योमकक्षा वक्ष्यमाणाकाशकक्षोच्यते। तन्मध्ये ब्रह्माण्डमध्य आकाशे भानां नक्षत्राणां सर्वेषां सर्वतस्तुल्योर्ध्वान्तरितानां भ्रमणं भवति। तथा तुल्यो \textendash



\newpage


\noindent ३१० \hspace{4cm} सूर्यसिद्धान्तः
\vspace{1cm}


\noindent र्ध्वान्तरेणाधो नक्षत्रेभ्यः शनिबृहस्पत्तिभौमार्कशुक्रबुधचन्द्राअधस्तात् परिभ्रमन्ति। सिद्धा विद्याधराश्चाधस्थाच्चन्द्रादधस्थिता अधोऽधः क्रमेणाकाशे स्थिताः। एषां प्रवहवायाववस्थानाभावाच्चन्द्रवन्न परिभ्रमः~॥~३१~\\
\noindent अथ भूम्यवस्थानमाह \textendash
%\vspace{2mm}

 \begin{quote}
{\ssi मध्ये समन्तादण्डस्य भूगोलो व्योम्नि तिष्ठति~।\\ 
बिभ्राणः परमां शक्तिं ब्रह्मणो धारणात्मिकाम्~॥~३२~॥}
%\vspace{2mm}
\end{quote}
 अण्डस्य ब्रह्माण्डस्य समन्तात् सर्वप्रदेशान्मध्ये मध्यस्थाने केन्द्ररूप आकाशे' भूगोलस्तिष्ठति। नन्वाकाशे निराधारवस्तुनोऽवस्थानासम्भवात् कथमवस्थितो भूमिगोल इत्यतो भूगोलविशेषणमाह\textendash बिभ्राण इति~। ब्रह्मणः परमां शक्तिं धारणात्मिकां निराधारावस्थानरूपां बिभ्राणो धारयम्। तथा च क्षतिः। एतेन भूः किमाकारा किमाश्रयेतिप्रश्रद्वयमुत्तरितम्~॥~३२~॥\\
\noindent अथ कथं चात्र सप्त पातालभूमय इति प्रश्नस्योत्तरमाह\textendash 
%\vspace{2mm}

\begin{quote}
{\ssi तदन्तरपुटाः सप्त नागातुरसमाश्रयाः~।\\
दिव्यौषधिरसोपेता रम्याः पातालभूमयः~॥~३३~॥ }
%\vspace{2mm}
\end{quote}
 तस्य भूगोलस्यान्तरपुटो मध्यस्यपुटा गुहारूपां सप्तातलवितलसुतलादिकाः पातालभूमयः पातालप्रदेशा रम्यामनोहराः सन्ति। ननु भूगोले मनुय्यादिकमस्ति तथा तत्र के सन्तीत्यतस्तद्विशेषणमाह\textendash नागासुरसमाश्रया इति~। वा \textendash



\newpage


\hspace{3cm} गूढार्थप्रकाशकेन सहितः~। \hfill ३११
\vspace{1cm}


\noindent सुकिप्रमुखादयः सर्पा दैत्या एषामाश्रयभूताः। ननु तत्र सूर्यसञ्चाराभावात् तमोमयत्वेन तत्स्थितलोकानां व्यवहारः कथं भवतीत्यतो द्वितीयं विशेषणमाह\textendash दिव्यौषधिरसोपेता इति~। दिव्या या ओषधयः स्वप्रकाशास्तासां रसैर्युक्ताः। तथाच तत्प्रकाशेन व्यवहारो भवति तद्धशेन तल्लोकानां जीवनं च भवतोति भावः~॥~३३~॥\\
\noindent अथ भूगोलमुत्क्का दक्षिणोत्तरभूव्यासाधिकप्रमाणमेरोरवस्थानमाह \textendash
%\vspace{2mm}

 \begin{quote}
{\ssi अनेकरत्ननिचयो जाम्बूनदमयो गिरिः~।\\
भूगालूमध्यगो मेरुरुभयत्र विनिर्गतः~॥~३४~॥  }
%\vspace{2mm}
\end{quote}
 भूगोलमध्यगतः पर्वतो मेर्वाख्योऽनेकरत्ननिचयोऽनेकानिनानाविधानि माणिक्यवज्रादीनि तेषां निचयः समूहो यचासौ। जाम्बूनदमयो जाम्बनदम्। 
%\vspace{2mm}

\begin{quote}
{\qt अम्बूफलामलगलद्रसतः प्रवृत्ता \\
 जम्बनदीरसयुता मृदभूत् सुवर्णम्~। \\
जाम्बूनदं हि तदतः सुरसिद्धसङ्घाः \\
शश्वत् पिबन्त्यमृतपानपराङ्भुखास्ते~॥ }
%\vspace{2mm}
\end{quote}
इति भास्कराचार्योक्तेश्च सुवर्णं तन्मयः स्वर्णघटिंत उभयत्रव्यासान्तरितभपृष्ठप्रदेशाभ्यां विनिर्गतो बहिःस्थितदरण्डाकारस्वर्णाद्रिमध्ये भूगोलः प्रोतोऽस्ति। अतएव भूभृदित्यन्वर्थसञ्ज्ञ इति तात्पर्यार्थः~॥~३४~॥ \\
\noindent अथ मेरोरूर्ध्वाधःप्रदेशयोर्देवादयोऽसुराश्च वसन्तीत्याह \textendash


\newpage


\noindent ३१२ \hspace{4cm} सूर्यसिद्धान्तः
\vspace{1cm}

%

 \begin{quote}
{\ssi उपरिष्टात् स्थितास्तस्य सेन्द्रो देवा महर्षयः~।\\
अधस्तादसुरास्तद्वद्विषन्तोऽन्योन्यमाश्रिताः~॥~३५~॥}
%\vspace{2mm}
\end{quote}
 उपरिष्टात् स्थितास्तस्य सेन्द्रा इन्द्रसहिता देवा इन्द्रादयोदेवा महर्षयः। चेः समुच्चयार्थोऽनुसन्धेयः। स्थिताः। अधस्तान्मेरोरधःप्रदेशे। असुरा दैत्याः। तद्वत्। यथोर्ध्वभागै देवास्तद्वदित्यर्थः। आश्रिता आस्थिताः। ननु देवा असुराश्चैकत्र कथं न स्थिता इत्यात आह\textendash द्विषन्त इति~। अन्योन्यं परस्परं द्वेषं कुर्वन्तः। तथाच देवासुरयोः परस्परं द्वेषसद्भावादेकत्रावस्थानासम्भवेनोत्तमा देवास्तदूर्ध्वभागे स्थिता महर्षयश्च दैत्यभीतास्तत्रैव स्थितास्तदधोभागे तन्निकृष्टा दैत्याः स्थिता इति भावः~॥~३५~॥\\
\noindent अथ भूगोले समुद्रावस्थानमाह \textendash
%\vspace{2mm}

 \begin{quote}
{\ssi ततः समन्तात् परिधिः क्रमेणायं महार्णवः~।\\
मेखलेव स्थितो धात्या देवासुरविभागकृत्~॥~ ३६~॥ }
%\vspace{2mm}
\end{quote}
 दण्डाकारमेरोः सकाशादभितोऽयं प्रत्यक्षो महार्णवो महासमुद्रः क्रमेण निरन्तरालक्रमणपरिधिरूपो भूम्या मेखलेव काञ्चीरूपो देवासुरविभागकृत् देवदैत्ययोर्भूमिगोले विभागयोः परिधिरेखारूप इत्यर्थः। तेन समुद्रादुत्तरं भूगोलस्यार्धं जम्बूद्वीपं देवानां समुद्राद्दक्षिणं समुद्रातिरिक्तंभूमिगोलस्यार्धं षड्द्वीपषट्समुद्रोभयात्मकं दैत्यानामिति सिद्धम्। मेरुदण्डानुरुद्धभूगोलमध्ये परिधिरूपो लवणसमुद्रोऽस्ति। उत्तरगोलार्धं दक्षिणभूगोलार्धान्तर्गतसमुद्रस्य प्रान्त \textendash
%

\newpage


\hspace{3cm} गूढार्थप्रकाशकेन सहितः~। \hfill ३१३
\vspace{1cm}


\noindent परिधिस्पृष्टमिति मेखलायाः कव्यधःस्थितत्वेन तात्पर्यार्थः~॥~३६~॥\\
\noindent अथ समुद्रोत्तरतटे परिधिरूपे जम्बूद्वीपारम्भे चतुर्विभागे चत्वारि नगराणि सन्तोत्याह \textendash
%\vspace{2mm}

 \begin{quote}
{\ssi समन्तान्मेरुमध्याम् तु तुल्यभागेषु तोयधेः~। \\
दीपेषु दिक्षु पूर्वादिनगर्यो देवनिर्मिताः~॥~३७~॥}
%\vspace{2mm}
\end{quote}

 मेरुमध्याद्दण्डाकारमेरोर्मध्यप्रदेशाद्भूगोलगर्भकादिति त्वर्थः। समन्तादभितो सद्वगोलपृष्ठे तोयधेः परिधिरूपसमुद्रस्य तुल्यभागेषु समभागेषु द्वीपेषु जम्बूद्वीपारम्भेषु दिक्षु चतुर्विभागेषु चतुर्दिक्षु पूर्वादिनगर्यो मेरोःपटुर्वदक्षिणपश्चिमोत्तरदिक्कमेण चतुःपुर्यो देवनिर्मिता देवैः कृताः सन्तीति शेषः। समुद्रोत्तरतटे जम्बूद्वीपस्यादिभागरूपे तुल्यान्तरेण चत्वारि नगराणिं भूगोलस्य कल्पितपूर्वादिदिशासु सन्तीति तात्पर्यार्थः~॥~३७~॥ \\
\noindent अथासां नामानि द्वीपोत्थितस्य जम्बूद्वीपादिभागस्थितवर्षाख्यपारिभाषिकविभागेध्वित्यर्थं च श्लोकत्रयेण विशदयति \textendash
%\vspace{2mm}

 \begin{quote}
{\ssi भूवृत्तपादे पूर्वस्यां यमकोटीति विश्रुता~।\\
भद्राश्ववर्षे नगरी स्वर्णप्राकारतोरणा~॥~ ३८~॥

याम्यायां भारते वष लङ्का तद्वन्महापुरी~।\\
पश्चिमे केतुमालाख्ये रोमकाख्या प्रकीर्तिता~॥~३९~॥

उदक् सिडूपुरनिाम कुरुवर्षे प्रकीर्तिता~।\\
तस्यां सिद्वा महात्माने निवसन्ति गतव्यथाः~॥~४०~॥}
\end{quote}
%


\newpage

\noindent ३१४ \hspace{4cm} सूर्यसिद्धान्तः
\vspace{1cm}

भूगोल उभयत्र दण्डाकारो मेरुर्यत्र निर्गतस्तत्स्यानाभ्यांवृत्ताकारसूत्रेणोर्ध्वाधरेण भूगोलमाखण्डद्वयं पूर्वापरं तिर्यग्वृत्ताकारं सूत्रेणोर्ध्वाधो भूमेः खण्डद्वयं तेन भूगोले वप्राकाराश्चत्वारो भूम्यंशास्तत्रोर्ध्वस्थपूर्ववप्रे भूम्यां यः समुद्रपरिधिस्तस्य चतुर्थांशे भद्राश्वसञ्ज्ञकवर्षे पर्वस्मिन्नूर्ध्वआधःशकलसन्धौ सुवर्णघटिताः प्रासादास्तोरणानि च यस्यामेतादृशी पुरी यमकोटीति सञ्ज्ञया विश्रुता विख्याता याम्यायामूर्ध्वशकलद्वयसन्धौ मेरुस्तस्थ दक्षिणत्वात् भारतसञ्ज्ञकवर्षे लङ्कासञ्ज्ञा महानगरो तद्वत् स्वर्णप्राकारतोरणा विश्रुतेत्यर्थः। पश्चिमे पश्चिमशकलाधःस्थशकलसन्धौ केतुमालसञ्ज्ञे वर्षे रोमकसञ्ज्ञानगरी। उक्ता। उदक्। अधःशकलद्वयसन्धौ कुरुसञ्ज्ञकवर्षे सिद्धपुरी नाम नगरी प्रोक्ता। अस्याः पुर्याः सिद्धपूरत्विमन्वर्थमित्याह\textendash तस्यामिति~। सिद्धपुर्यां सिद्धा योगाभ्यासका अस्मदादिभ्यो महानुत्कृष्ट आत्मा येषां ते गतव्यथा दुःखरहितानिरन्तरा वसन्ति~॥~४०~॥ \\
\noindent अथोक्तानां चतुर्णां पुराणां परस्परमन्तरालमव्यवहितं मेरोरासामन्तरं चाह \textendash
%\vspace{2mm}

\begin{quote}
{\ssi भूवृत्तपादविवरास्ताश्चान्योन्यं प्रतिष्ठिताः~।\\
 ताभ्यश्चोत्तरगो * मेरुस्तावानेव सुराश्रयः~॥~४१~॥ }
%\vspace{2mm}
\end{quote}
 ता उक्तनगर्योऽन्योन्यं परस्परं भूवृत्तपादविवरा भूगोलवृत्तपरिधिचतुर्थांशान्तरदिक्स्थाः। मेरुः स चोक्तः सुराश्रयः 
%

\noindent\rule{\linewidth}{.5pt}
\begin{center}
 * वाभ्यश्चोत्तरतो मेरुरिति वा पाठः ।
\end{center}

\newpage


\hspace{3cm} गूढार्थप्रकाशकेन सहितः~।  \hfill ३१५
\vspace{1cm}


\noindent देवैरधिष्ठितस्तावान् भूपरिधिचतुर्थांशान्तरेण स्थितः। एवकारो न्यूनाधिकव्यवच्छेदार्थः। चकारः श्लोकपूर्वार्धेन समुच्चयार्थः~॥~४१~॥\\
\noindent अथ तेषां पुराणां निरक्षत्वमस्तीत्याह \textendash
%\vspace{2mm}

\begin{quote}
{\ssi तासामुपरिगो याति विषुवस्थो दिवाकरः~।\\
न तासु विषुवच्छाया नाक्षस्योन्नतिरिष्यते~॥~४२~॥}
%\vspace{2mm}
\end{quote}
 तासामुक्तनगरीणां विषुतस्थो विषुवहुत्तस्थो यद्दिने समरात्रिन्दिवं तद्दिने यन्मार्गेण भ्रमति तद्विषुवद्वृत्तं तत्रस्यइत्यर्थः। सूर्य उपरिगः सन् धाति भ्रमति। अतः कारणात् तासुनगरीषु विषुवच्छायाक्षभा न भवति तत्र गतानां विषुवद्धृत्ताभिन्नपूर्वापरवृत्तसद्भावात्। तत्रस्थसूर्ये मध्याह्ने छायाभावोपलम्भात्। अतएव तेषु नगरेषु अक्षध्रुवस्योन्नतिरुच्चताक्षांशरूपा नेष्यते नाङ्गीक्रियते।अक्षांशाभावान्निरक्षदेशत्वं तेषां सिद्धमिति भावः~॥~४२~॥\\
\noindent अथ मेरावुक्तपूरीषु क्रमेण लम्बांशाक्षांशाभावावुपपत्त्या प्रतिपिपादयिषुस्तयो प्रथमं ध्रुवस्थितिमाह \textendash
%\vspace{2mm}

 \begin{quote}
{\ssi मेरोरुभयतो मध्ये ध्रुवतारे नभःस्थिते~।\\
 निरक्षदेशसंस्थानामुभये क्षितिजाश्रये~॥~४३~॥}
%\vspace{2mm}
\end{quote}
 मेरोरुभयतो दक्षिणोत्तराग्रयोराकाशस्थिते ध्रुवतारे दक्षिणोन्तरे क्रमेण मध्य आकाशमध्ये भवतः। निरक्षदेशसंस्थानां प्रागुक्तनगरस्थितमनुष्याणामुभये दक्षिणोत्तरे ध्रुवतारे क्षिति \textendash



\newpage


\noindent ३१६ \hspace{4cm} सूर्यसिद्धान्तः
\vspace{1cm}


जाश्रये तद्भूगर्भक्षितिजवृत्तस्थे भवत इत्यर्थः~॥~४३~॥\\
\noindent अथातएव तेय्वक्षांशाभावलम्बांशपरमत्वमित्याह \textendash
%\vspace{2mm}

 \begin{quote}
{\ssi अतो नाक्षोच्छयस्तासु ध्रुवयोः क्षितिजस्थयोः~।\\
 नवतिर्लम्बकांशास्तु मेरावक्षांशकास्तथा~॥~४४~॥}
 \end{quote}
%\vspace{2mm}

 तासूक्तनगरीषु। अत उभये क्षितिजाश्रये इति कारणात्। अक्षोच्छ्रयो ध्रुवौच्च्यं न। तथाच क्षितिजाद वौच्च्यमक्षांशा इति तदभावात् तदभाव इति भावः। तुकारात् तन्नगरीषु ध्रुवयोः क्षितिजस्थयोः सतोर्लम्बांशा नवतिः शून्याक्षांशोननवतेर्लम्बांशत्वात्। खमध्याद वयोः क्षितिजस्य लम्बांशस्वरूपत्वाच्च मेरावक्षांशास्तथा नवतिः। ध्रुवस्य परमोच्चत्वात्। यथा निरक्षदेशेऽक्षांशाभावाल्लम्बांशाः परमास्तथा मेरावक्षांशपरमत्वाल्लम्बांशाभाव इत्यर्थसिद्धम्। एतेन 

\begin{quote}
{\qt  पुरान्तरं चेदिदमुत्तरं स्यात \\
तदक्षविश्लेषलवैस्तदा किम्~।\\
चक्रांशकैरित्यनपातद्यक्त्या\\
युक्तं निरुक्तं परिधेः प्रमाणम्~॥}
\end{quote}
%\vspace{2mm}

 इति भास्कराचार्योक्तं प्रथमप्रश्नस्योत्तरं सूचितम्। स्पष्टपरिधिसाधनं च कल्पितैकमध्यस्थानानुरोधेनापचीयमानं मेरावभावात्मकं नानुपपन्नमिति च सूचितम्~॥~४४~॥\\
\noindent अथाहोरात्रव्यवस्यां चेत्यादिप्रश्नोत्तरं विवक्षुर्देवासुरयोर्दिनारम्भंप्रथममाह \textendash



\newpage


\hspace{3cm} गूढार्थप्रकाशकेन सहितः~। \hfill ३१७
\vspace{1cm}


% {\setlength{\parindent}{5em}
\begin{quote}
 {\ssi मेषादौ देवभागस्थे देवानां याति दर्शनम्~।\\ 
असुराणां तुलादौ तु सूर्यस्तद्भागसृञ्चरः~॥~४५~॥}
\end{quote}
%\vspace{2mm}

 जम्बूद्वीपलवणसमुद्रसन्धौ परिधिवृत्तं भूगोलमध्मे तत्समसूत्रेणाकाशे वृत्तं विषुवद्वृत्तं तत्र क्रान्तिवृत्तं षड्भान्तरेण स्थानद्वये लग्नं तन्मेषतुल्यास्थानं प्रवहवायुना विषुवद्धृत्तमार्गे भ्रमति मेषस्थानात् कर्कादिस्यानं विषुवद्धृत्ताच्चतुर्विंशत्यंशान्तर उत्तरतः। मकरादिस्यानं विषुवद्धृत्ताच्चतुर्विंशत्यंशान्तरे दक्षिणतः। तत् स्वस्थाने प्रवहवायुना भ्रमति। एवं क्रान्तिवृत्तप्रदेशाः स्वस्वस्थाने प्रवहवायुना भ्रमन्ति। तत्र मेषादौ देवभागस्थो जम्बूद्वीपं देवानां देवासुरविभागकृदिति पूर्वोक्तेः। तत्सम्बद्धा मेषादिकन्यान्ता राशय उत्तरगोलः। तत्रस्यः सूर्यो मेषादौ मेषादिप्रदेशे दैवानां मेरोरुत्तराग्रवर्तिनां दर्शनं षण्मासानन्तरप्रथमदर्शनं याति गच्छति। प्राप्नोतीत्यर्थः। विषुवद्वृत्तस्य तत्क्षितिजत्वात्। एवं दैत्यानां मेरोर्दक्षिणाग्रवर्तिनामित्यसुराणामित्युक्तेनैवोक्तम्। तद्भागसञ्चरो दैत्यभागे समुद्रादिदक्षिणविभागस्थास्तुलादिमीनान्ता राशयो दक्षिणगोलस्तत्र सञ्चरो गमनं यस्येत्येताट्टशसूर्यस्तुलादिप्रदेशे तुकारोददर्शनानन्तरं प्रथमदर्शनं प्राप्नोतीत्यर्थः। तेषामपि विषुवद्धृत्तक्षितिजत्वात्~॥~४५~॥\\
\noindent अथ प्रसङ्गाद्ग्री मे तीव्रकर इत्याद्यर्धोक्तप्रश्नस्योत्तरमाह\textendash 



\newpage


\noindent ३१८ \hspace{4cm} सूर्यसिद्धान्तः
\vspace{1cm}


\begin{quote}
  {\ssi अत्यासन्नतया तेन ग्रीष्मे तीव्रकरा रवेः~।\\
देवभागे सुराणां तु हेमन्ते मन्दतान्यथा~॥~४६~॥}
\end{quote}
%\vspace{2mm}


 तेन। उत्तरदक्षिणगोलयोः सूर्यस्योत्तरदक्षिणसञ्चाररूपकारणेनेत्यर्थः। देवभागे अम्बूद्वीपे। अत्यासन्नतया सूर्यस्यात्यन्तनिकटस्यत्वेन ग्रीष्मे ग्रीष्मर्तौ सूर्यस्य तेजोगोलकस्य किरणास्थीक्ष्या अत्युष्णा असुराणां देवभाग इत्यस्य समन्वयाद्दैत्यानां भागे समुद्रादिदक्षिणप्रदेशे हेमन्ते हेमर्तौ तुकारात् सूर्यस्यात्युष्णाः किरणाः सूर्यस्यात्यासन्नत्वात्। अन्यथा सूर्यस्य दूरस्थत्वेन मन्दता किरणानामनुष्णताभावः। देवभागे हेमन्तर्तै कराणां मन्दता। अतएव तत्र शीताधिक्यं दैत्यभागे ग्रीष्मे कराणां मन्दता शीताधिक्यं च। तथाच देवभागे दक्षिणगोले सूर्यस्य दूरस्थत्वमुत्तरगोले निकटस्थत्वं मध्यान्हनतांशानां क्रमेणाधिकाल्पत्वादिति भावः~॥~४६~॥\\
\noindent अथ मेषादौदेवभागस्य इत्युक्तं देवासुराहोरात्रकथनव्याजेन विशदयति\textendash 
%\vspace{2mm}


\begin{quote}
 {\ssi देवासुरा विषुवति क्षितिजस्थं दिवाकरम्~।\\
पश्यन्त्यन्योन्यमेतेषां वामसव्ये दिनक्षपे~॥~४७~॥}
\end{quote}
%\vspace{2mm}

 विषुवति काले देवदैत्याः सूर्यं क्षितिजस्थं पश्यन्ति। विषुवद्धृत्तस्य तयोः स्वस्थानाद्भूगोलमध्यस्यत्वेन क्षितिजत्वात्। एतेषां देवदैत्यानामन्योन्यं परस्परं वामसव्ये अपसव्यसव्ये तत्क्रमेण दिनक्षपे दिवसरात्री भवतः। अयं भावः। देवानां \textendash



\newpage


\hspace{3cm} गूढार्थप्रकाशकेन सहितः~। \hfill ३१९
\vspace{1cm}


भूमेरुत्तरभागः स्वकीयत्वात सव्यं दैत्यानामपसव्यं स्वकीयत्वाभावात्। एवं दैत्यानां भूमेर्दक्षिणभागः स्वकीयत्वात् सव्यं देवानां स्वकीयत्वाभावादपसव्यमतो दैत्यानां वामसव्यभागावुत्तरदक्षिणगोलौ देवानां क्रमेण दिनरात्री। देवानां वामसव्यभागौ दक्षिणोत्तरगोलौ दैत्यानां दिनरात्री। अन्यथान्योन्यं वामसव्ये इत्यनयोः सङ्गतार्थानुपपत्तेः। अतएव पूर्वं मेषादावित्याद्युक्तमिति~॥~४७~॥\\
\noindent अथ पूर्वश्लोकोत्तरार्धस्य सन्दिग्धत्वं शङ्कया दिनपूर्वापरार्धकथनच्छलेन तदर्थं श्लोकाभ्यांविशदयति\textendash
%\vspace{2mm}


\begin{quote}
 {\ssi मेषादावुदितः सूर्यस्त्रीन् राशीनुदगुत्तरम्~।\\
सञ्चरन् प्रागहर्मध्यं पूरयेन्मेरुवासिनाम्~॥~४८~॥

कर्कादीन् सञ्चरंस्तद्धदह्नः पश्चार्धमेव सः~।\\
तुलादींस्त्रीन्मृगादिंश्च तद्वदेव सुरद्विषाम्~॥~४९~॥}
\end{quote}
%\vspace{2mm}

 मेषादौ विषुवद्दृत्तस्थक्रान्तिवृत्तभागे रेवत्यासन्न उदितो दर्शनतां प्राप्तः सूर्य उत्तरं यथोत्तरं क्रमेणेति यावत्। चोन् राशीनुदगुत्तरभागस्यान् मेषवृषमिथुनान् सञ्चरन्नतिक्रामन् सन् मेरुस्थानां देवानां प्रागहर्मध्यं प्रथमं दिनस्यार्ध पूरयेत् पूर्णं करोतीत्यर्थः। मिथुनान्ते सूर्ये मेरुस्थानां मध्याप्तं स्यादिति फलितार्थः। कर्कादीन् राशीन् कर्कसिंहकन्यास्तद्वत् क्रमेणेत्यर्थः। अतिक्रामन् सन् स सूर्यो दिवसस्य पश्चार्धमपरदलम्। एवकारोऽन्ययोगव्यवच्छेदार्थः। पूरयेत्। कन्यान्ते सूर्ये मेरुस्थानां \textendash



\newpage


\noindent ३२० \hspace{4cm} सूर्यसिद्धान्तः
\vspace{1cm}


\noindent सूर्यास्तो भवतीति फलितार्थः। अथ दैत्यानामाह\textendash तुलादीनिति~। सुरद्विषां, मेरोर्दक्षिणाग्रवर्तिनां दैत्यानामित्यर्थः। तुलादींस्त्रीन् राशींस्तुलावृश्चिकधनुराख्यान् मृगादींस्त्रीन् राशीन् मकरकुम्भमीनांस्तद्वत् क्रमेणातिक्रामन् सूर्यः। चकारस्तुलामृगादिक्रमेण पूर्वापरार्धमित्यर्थकः। एवकार उक्तातिरिक्तव्यवच्छेदार्थः। दिनं पूरयतीत्यर्थः। धनुरन्ते सूर्ये दैत्यानां मध्यान्हं मीनान्ते सूर्ये सूर्यास्तो भवतीति फलितार्थः~॥~४९~॥\\
\noindent अथातो देवासुराणमिति प्रश्नस्योत्तरं सिद्धमित्याह\textendash 
%\vspace{2mm}


\begin{quote}
 {\ssi अतो दिनक्षपे तेषामन्योन्यं हि विपर्ययात्~।\\
अहोरात्रप्रमाणं च भानोर्भगणपूरणात्~॥~५०~॥}
\end{quote}
%\vspace{2mm}

 अत उक्तकारणात् तेषां देवदैत्यानामन्योन्यं परस्परं हि निश्चयेन विपर्ययाद्व्यत्यासाद्दिनरात्री स्त इति फलितम्। एतत्फलितार्थस्तु पूर्वं बहुधोक्तं। अथ तत् कथं वा स्यात्। भानोर्भगणपूरणादिति प्रश्नस्याप्युत्तरं फलितमित्याह\textendash अहोरात्रप्रमाणमिति~। सूर्यस्य मेषादिद्वादशराशिभोगाद्देवदैत्यानामहोरात्रमानं भवति। चकारः पूर्वार्धेन समुच्चयार्थकस्तेन द्वयोः पूवोक्तमेकं कारणमिति स्पष्टम्~॥~५०~॥\\ 
\noindent अथ मेषादावुदित इत्यादिश्लोकद्वयस्य फलितार्थ तदुपपत्तिं चाह\textendash
%\vspace{2mm}


\begin{quote}
 {\ssi दिनक्षपार्धमेतेषामयनान्ते विपर्ययात्~।\\
उपर्यात्मानमन्योन्यं कल्पयन्ति सुरासुराः~॥~५१~॥}
\end{quote}



\newpage


\hspace{3cm} गूढार्थप्रकाशकेन सहितः~। \hfill ३२१
\vspace{1cm}


 एतेषां देवदैत्यानाभयनान्तेऽयनसन्धौ विपर्ययाद्व्यात्यासाद्दिनक्षपार्धं दिनार्धं रात्र्यर्धं च भवति। यत्र देवानां मध्याह्नं रात्र्यर्धं तत्र दैत्यानां क्रमेण रात्र्यर्धमध्याह्ने यत्र च दैत्यानां मध्याह्नरात्र्यर्धे तत्र देवानां क्रमेण रात्रार्धमध्याह्ने इति फलितार्थः। अत्र हेतुमाह\textendash उपरीति~। देवदैत्या मेरोरुत्तरदक्षिणाग्रवर्तिनोऽन्योन्यमात्मानमुपरिभागे कल्पयन्त्यङ्गीकुर्वन्ति। वस्तुतो भूमेर्गोलकत्वेन सर्वत्र तुल्यत्वान्निरपेक्षोर्ध्वाधोभागयोरनुपपत्तेः। तथाच देवा दैत्यापेक्षयोर्ध्वस्य मन्यमाना दैत्यानधःस्थानङ्गीकुर्वन्ति। दैत्याश्च देवस्थानापेक्षयोर्धस्यं मन्यमाना देवानधः कुर्वन्तीति तात्पर्यार्थः। एवं च देवदैत्ययोर्विपरीतावस्थानाद्दिनरात्र्योर्वैपरीत्यं युक्तमेवेति भावः~॥~५१~॥ \\
\noindent अथ देवदैत्ययोरूर्ध्वधोरीतिमन्यत्रापि सदृष्टान्तमतिदिशति\textendash 
%\vspace{2mm}


\begin{quote}
 {\ssi अन्येऽपि समसूत्रस्था मन्यन्तेऽधः परस्परम्~।\\
भद्राश्वकेतुमालस्था लङ्कासिद्वपुराश्रिताः~॥~५२~॥}
\end{quote}
%\vspace{2mm}

 अन्ये देवदैत्यभिन्ना भूगोलस्याः। अपिशब्दो दैत्यैः समुच्चयार्थकः। समसूत्रस्या भूव्यासान्तरिता नराः परस्परमधो मन्यन्ते। तुर्यचरणस्तु व्यक्त एव~॥~५२~॥\\ 
%\noindent 
अथोक्तं काल्पनिकमेवेति द्रढयन्नाह\textendash 

%\vspace{2mm}

\begin{quote}
 {\ssi सर्वत्रैव महीगोले स्वस्थानमुपरि स्थितम्~।\\
मन्यन्ते खे यतो गोलस्तस्य क्वोर्ध्वं क्व वाप्यधः~॥~५३~॥}
\end{quote}
%\vspace{2mm}

 भूगोले सर्वत्र सर्वप्रदेशेषु मध्ये स्वस्थानं निजाधिष्ठितस्थान\textendash



\newpage


\noindent ३२२ \hspace{3cm} सूर्यसिद्धान्तः
\vspace{1cm}


\noindent मुर्ध्वस्थितं तदधिष्ठिता मनुष्याः स्वाभिमानेनाङ्गीकुर्युः। अतः कारणद्भूगोले स गोले सर्व एवोर्ध स्थाः। अधःस्यास्तु न भवन्त्येव। सापेक्षतयोर्ध्वाधःस्थत्वं न वस्तुत इति तत्त्वम्। अन्यथाधःसत्वेन पतनशङ्कया भूगोले अनुष्याद्यवस्थानागुपपत्तेः। अत्रकारणमाह\textendash ख इति~। यतः कारणात् खे ब्रह्माण्डाकाशमध्यभागे भूगोलोऽस्ति। तथाच ब्रह्माण्डस्य भूगोलादभितस्तुल्यत्वाद्भूगोले तत्त्वतयोर्ध्वाधोभागादेरसम्भव इति भावः। स्वाभिप्रायं स्पष्टयति\textendash तस्येति~। भूगोलस्याकाशमध्यस्यस्य समन्तादाकाशे क कस्मिन् भाग ऊर्ध्वमूर्ध्वत्यम्। कस्मित्प भागे। वा समुच्चये। अधोऽधस्त्वम्। अपिरूर्ध्वत्वेग समुच्चयार्थकः। तथाच समन्तादाकाशस्य तुल्यत्वेन भूमेरूर्ध्वाधोभागौ निर्वचनीकर्तुमशक्यौयाभ्यामूर्ध्वाधोलोका नियताः स्युरिति। भूमेरूर्ध्वाधोभागाद्यसम्भवादिति भावः~॥~५३~॥ \\
\noindent नन्वियं भूः समादर्शाकारा प्रत्यक्षाकथं गौलाकारेत्यत आह \textendash

%\vspace{2mm}

\begin{quote}
{\ssi अल्पकायमया लोकाः स्वस्थानात् सर्वतोमुखम्~।\\
पश्यन्ति वृत्तामप्येतां चक्राकारां वसुन्धराम्~॥~५४~॥}
%\vspace{2mm}
\end{quote}
 जनाः स्वाधिष्ठितप्रदेशात् सर्वतः सर्वदिक्षु। अभिमुखं वृत्तांगोलाकारामेतां प्रत्यक्षां पृथ्वीं चक्राकारां मण्डलाकारां समां पश्यन्ति। एवकारार्थेऽपिशब्दः। तेन भूमेर्वस्तुतो गोलाकारत्वेऽपि तदाकारेणादर्शनं मुकुराकारतया दर्शनं च न विरद्धम्। अत्र हेतुमाह\textendash अल्पकायतथेति~। हखशरि त्वेनेत्यर्थः। तथाच \textendash


\newpage


\hspace{3cm} गूढार्थप्रकाशकेन सहितः~। \hfill ३२३
\vspace{1cm}


\noindent महतो भूस्तत्पृष्ठस्थस्यमनुव्यस्यातिहस्वस्याल्पदृष्टिप्रचारोद्गोलाकारतया न भासते किन्तु सममण्डलतया भासते। गोलवृत्तशतांशस्य समत्वेन भानात्। अन्यथा प्रथमज्यायाश्चापसमत्वानुपपत्तिरिति भावः~॥~५४~॥ \\
\noindent अथ निरक्षादिदेशेषु मेरुव्यतिरिक्तान्यदेशेषु दिनरात्र्योर्मानं विवक्षुर्मेरोरग्रभागयोर्निरक्षदेशेषु अचक्रभ्रमणमाह \textendash

%\vspace{2mm}

\begin{quote}
{\ssi सव्यं भ्रमति देवानामपसव्यं सुरद्विषाम्~।\\
उपरिष्टाद्भगोलोऽयं व्यक्षे पश्चान्मुखः सदा~॥~५५~॥}
%\vspace{2mm}
\end{quote}
 अयं प्रत्यक्षो भगोलो नक्षत्राधिष्ठितमूर्तगोलो देवानां मेरोरुत्तराग्रवर्तिनां सव्यम्। पूर्वादिक्रममार्गेणेत्यर्थः। भ्रमति भ्रमपरिवर्तं करोतीत्यर्थः। दैत्यानां मेरोर्दक्षिणाग्रवर्तिनामपसव्यं पूर्वादिदिग्व्युत्सममार्गेंण। पूर्वोत्तरपश्चिमदक्षिणक्रमेणंत्यर्थः। नक्षत्राधिष्ठितगोलो भ्रमति। व्यक्षे निरक्षदेशेषु जात्यभिप्रायेणैकवचनम्। उपरिष्टान्मस्तकोर्ध्वमध्यभागे भगोलः पश्चान्मुखः पश्चिमदिगभिमुखः सदा नित्यं परिभ्रमति। भगोलस्य ध्रुवमध्यस्थत्वेन भ्रमणात्। तयोस्तवक्षितिजवृत्तस्यत्वाच्च~॥~५५~॥\\
\noindent अथ निरक्षे दिनरात्र्योर्मानं कथयन्नन्यत्रापि ततोन्येनाधिकं मानं भवतीत्याह \textendash

%\vspace{2mm}

\begin{quote}
{\ssi अतस्तत्र दिनं त्रिंशन्नाडिकं शर्वरी तथा~।\\
हानिवृद्धी सदा वामं सुरासुरविभागयोः~॥~५६~॥}
%
\end{quote}
\noindent\rule{\linewidth}{.5pt}
\begin{center}
 अतो निरक्षे मस्तकोर्ध्वं भगोलो भ्रमतीति कारणात् तत्र 
\end{center}


\newpage


\noindent ३२४ \hspace{4cm} सूर्यसिद्धान्तः
\vspace{1cm}


\noindent निरक्षदेशे त्रिंशन्नाडिकं त्रिंशद्वटीमितं दिनं स्यात्। शर्वरीरात्रिस्तथा त्रिंशद्वटोपरिमिता स्यात्। तत्क्षितिजवृत्तस्य
ध्रुवद्वयसंलग्नतया गोलमध्यस्थत्वाद्दिनरात्र्योस्तुल्यत्वं युक्तमेवेति भावः।सुरासुरविभागयोर्जम्बूद्वीपसमुद्रादिदक्षिणदेशयोः सदा विषुवत्कमणातिरिक्तकाले क्षयवृद्धी दिनरात्र्योः प्रत्येकं वामं व्यस्तं यथा स्यात् तथा ज्ञेयम्। एतदुक्तं भवति। जम्बूद्वीपे दिनहासेरात्रिवृद्धिस्तदा दक्षिणदेशे दिनरात्र्योः क्रमेण वृद्धिहानी। जम्बूद्वीपे दिनवृद्धौ रात्रिहानिस्तदा दक्षिणदेशे दिनरात्र्योः क्रमेण हानिवृद्धी। एवं दक्षिणदेशे हानिवृड्व्योर्जम्बूद्वीपेवृद्धिहानी दिनरात्र्योर्वा योग्ये इति। त्रत्रोपपत्तिः। तत्क्षितिजवृत्तस्य ध्रुवसम्बन्धाभावेन गोलमध्यस्थत्वाभावाद्दिनरात्र्यो सदाविषुवद्दिनव्यमिरिक्तेन तुल्यत्वं किन्तु न्यूनाधिकत्वमहोरात्रस्य षष्टिघटिकात्मकत्वादिति~॥~५६~॥ \\
\noindent अथैतच्छोकोत्तरार्धार्थंश्लोकाभ्यां विशदयति \textendash

%\vspace{2mm}

\begin{quote}
{\ssi मेषादौ तु सदा * वृर्द्धिरुदगुत्तरतोऽधिका~।\\
देवांशे त्वक्षया हानिर्विपरीतं तथासुरे~॥~५७~॥

तुलादौ द्युनिशोर्वामं क्षयवृद्धी तयोरुभे~।\\
देशक्रान्तिवशान्नित्यं तद्विज्ञानं पुरोदितम्~॥~५८~॥}
%\vspace{2mm}
\end{quote}
 मेषादौ षड्भ उदगुत्तरगेले सूर्ये सति। उत्तरतो यथोत्तरं सदा यावदुत्तरगोले देवांशे जम्बूद्वीपेऽधिका यथोत्तर \textendash


\noindent\rule{\linewidth}{.5pt}

\begin{center}
  * मेवादौ प्रत्यहं इति वा पाठः।
\end{center}

\newpage


\hspace{3cm} गूढार्थप्रकाशकेन सहितः~।  \hfill ३२५ 
\vspace{1cm}


मधका वृद्धिर्निरक्षदेशीयदिने तुकाराद्यथोत्तरं सूर्यस्योत्तरगमने यथोत्तरं दिने वृद्धिः परमोत्तरगमना परावर्तते। यथोत्तरं न्यूना वृद्धिरित्यर्थः। क्षयो हानी रात्रेरपचयं । चः समुच्चये। आसुरे समुद्रादिदक्षिणभागे तथा विनरात्र्योः क्षयवृद्धी विपरीतं व्यस्तम्। दिने हानी रात्रौ वृद्धिरित्यर्थः। तुलादौ षड्भे दक्षिणगोले सूर्ये सति तयोर्जम्बूद्वीपसमुद्रादिदक्षिणभागयोर्दिनरात्र्योरुभे द्वे क्षयवृद्धी उपचयापचयौ वामंव्यस्तम। अयमर्थः। जम्बद्वीपे दिनरात्र्योरुत्तरगोलस्यवृद्धिक्षयक्रमेण क्षयवृद्धी स्तः। समुद्रादिदक्षिणभागे दिनरात्र्योर्वृद्धिक्षयौ स्त इति। ननु क्षयवृद्ध्योः कियन्मितत्वमित्यतः पूर्वं विस्तारयति\textendash देशक्रान्तिवशादिति~। तद्विज्ञानं तयोः क्षयवृडोर्ज्ञानं सङ्ख्याज्ञानं नित्यं प्रत्यहं देशक्रान्तिवशात्। देशपलभाक्रान्तिरेतदुभयानुरोधात् पुरा पर्वखण्डस्यष्टाधिकारे \textendash

%\vspace{2mm}

\begin{quote}
{\qt क्रान्तिज्या विषुवद्भाघ्नी क्षितिज्या द्वादशोद्धृता~।\\
त्रिज्यागुणाहोरात्रार्धकर्णाप्ता चरजासवः~॥}
%\vspace{2mm}
\end{quote}
 तत्कार्मुकमित्यनेन दिनरात्र्योरर्ध उक्तम्। तद्द्विगुणं दिनरात्र्योरित्यर्थसिद्धम्। अत्रोपपत्तिः। निरक्षदेशे ध्रुवद्वयलग्नं क्षितिजवृत्तं तत उत्तरभागे स्वस्थानक्षितिजं स्वभूगोलमध्यस्यमुत्तरध्रुवादधो दक्षिणध्रुवाच्चोच्चमित्यत उत्तरगोले निरक्षक्षितिजादधो दक्षिणगोल ऊर्ध्वमिति पञ्चदशघटिका निरक्षदेशदिनार्धं क्षितिजान्तररूपचरेण गोलक्रमेण युतहीनं दिनार्धंरात्र्यर्धं च विपरीतम्। एवं दक्षिणभागेऽभोष्टदेशै क्षितिज \textendash


\newpage

\noindent ३२६ \hspace{4cm} सूर्यसिद्धान्तः
\vspace{1cm}


\noindent मुत्तरध्रुवादुन्नतं दक्षिणध्रुवान्नतमिति निरक्षक्षितिजान्निरक्षक्षितिजं गोलक्रमेणोर्ध्वाध इत्युरन्तरभागाद्व्यस्तम्~॥~५८~॥ \\
\noindent अथोक्तस्यावधिदेशं विवक्षुः प्रथमं तदुपयुक्तानि क्रान्त्यशयोजनान्याह \textendash

%\vspace{2mm}

\begin{quote}
{\ssi भूवृत्तं क्रान्तिभागघ्नं भगणांशविभाजितम्~। \\
अवाप्तयोजनैरर्को व्यक्षाद्यात्युपरिस्थितः~॥~५९~॥}
%\vspace{2mm}
\end{quote}
 भूवृत्तं भूपरिधियोजनमानं प्रागुक्तमभीष्टाक्रान्त्यंशैर्गुणितं द्वादशराशिभोगैः षष्यधिकशतत्रयमितैर्भक्तं लब्धयोजनैः कृत्वा सूर्य उपरि आकाशे स्थितो वर्तमानो दक्षिणत उत्तरतो वा याति गच्छति। क्रान्त्यभावे तु निरक्षदेशोपर्येव परिभ्रमति।अत्रोपपत्तिः। निरक्षदेशान्मेरोरुत्तरदक्षिणाग्राभिमुखं सूर्यः क्रान्त्यंशैर्गच्छति। तद्योजनज्ञानं तु भगणांशैर्मेर्वग्रद्वयनिरक्षदेशस्पृष्टभूपरिधिश्चोजनानि तदा क्रान्त्यंशैः कानीत्यनुपातेनेत्युपपन्नम्~॥~५९~॥\\
\noindent अथ दिनमानायनगोरगतस्यावधिदेशज्ञानंश्लोकाभ्यामाह \textendash

%\vspace{2mm}

 \begin{quote}
{\ssi परमापक्रमादेवं योजनानि विशोधयेत्~।\\
भूवृत्तपादाच्छेषाणि यानि स्युर्योजनानि तैः~॥~६०~॥ 

अयनान्ते विलोमेन देवासुरविभागयोः~।\\
नाद्धोषष्ट्या सकृदहर्निशा यस्मिन् सकृत् तथा~॥~६१~॥}
%\vspace{2mm}
\end{quote}
 परमक्रान्तिभागाच्चतुर्विंशन्मितात्। एवं पूर्वोक्तरीत्या योजनानि जातानि। भूपरिधेः पूर्वोक्तस्य चतुर्थांशात् परिवर्जयेत् ।अवशिष्टानि यानि यत्सङ्ख्यामितानि योजनानि भवन्ति तैर्यो \textendash



\newpage

\hspace{3cm} गूढार्थप्रकाशकेन सहितः~। \hfill ३२७
\vspace{1cm}


\noindent जनैर्देवासुरविभागयोर्निरक्षदेशादुत्तरदक्षिणप्रदेशयोर्यौ देशौतयोरित्यर्थः। अयनान्त उत्तरदक्षिणायनसन्धौ कर्कादिस्थेसूर्ये दक्षिणोत्तरायनसन्धौ मकरादिस्थे सूर्ये विलोमेन व्यत्यासेन सकृदेकवारं नाडीषष्ट्या घटीषष्ट्याहर्दिनमानं भवति।अस्मिन्नेतादृशे देशे तस्मिन्नेवायनसन्ध्यासन्ने सकृदेकवारं तथा षष्टिघटीमिता विलौमेन रात्रिर्भवति। अपिशब्दे दिनेन समुच्चयार्थः। एतदुक्तं भवति। कर्कादिस्ये सूर्ये निरक्षदैशादुत्तरतद्योजनान्तरितदेशे षष्टिघटीमितदिनं तदैव निरक्षदेशादक्षिणतद्योजनान्तीरेतदेशे षष्टिघटीमिता रात्रिः। मकरादिस्थे सूर्ये तादृशोत्तरभागे षष्टिघटीमिता रात्रिर्दक्षिणभागे तादृशे षष्टिमितं दिनमिते। अत्रोपपत्तिः। परमक्रान्तिय जनानि भूवृत्तचतुर्थाशयोजनेभ्यो हीनानि। निरक्षदेशात्तन्मितयोजनान्तरितो यो दक्षिणोत्तरदेशस्तस्मान्मेरोर्दक्षिणोत्तराग्रं क्रमेण परमक्रान्तियोजनान्तरितम्। अतस्तत्र लम्बांशाश्चतुर्विंशतिः पलांशाद्य षट्षष्टिरिति। तद्देशे क्रान्तिवृत्तानुकारं क्षितिजमित्ययनान्ते पञ्चदशघटीमितमहोरात्रवृत्तचतुर्भागखण्डं निरक्षतद्देशक्षितिजयोरन्तरालरूपं चरमत उक्तरीत्या दिनार्धं रात्र्यर्ध वोक्तरीत्या यथायोग्यं 'त्रिंशत्तद्दूगुणंषष्टिघटीमिततन्मानं गणितरीत्योपपन्नम्। युक्तं चैतत्। अयनान्ताहोरात्रवृत्तस्यैकस्य तत्क्षितिजप्रदेश एकत्रैवसंलग्नत्वाद्दूधासंलग्नत्वाभावात् प्रवहभ्रमितसूर्यपरिवर्तपूर्तिषष्टिघटीभिर्दर्शनमदर्शनं यथायोग्यं तद्गोलस्थित्या प्रत्यक्षसिद्धमेवेति~॥~६९~॥ \\
अथो \textendash



\newpage


\noindent ३२८ \hspace{3cm} सूर्यसिद्धान्तः
\vspace{1cm}


\noindent क्तदिनरात्रिमानगणितं तदवधिदेशपर्यन्तं दक्षिणोत्तरभागयोर्नाग्र इत्याह \textendash

%\vspace{2mm}

\begin{quote}
{\ssi तदन्तरेऽपि षष्ट्यन्ते क्षयवृद्धी अहर्निशोः~। \\
परतो विपरीतोऽयं भगोलः परिवर्तते~॥~६२~॥}
%\vspace{2mm}
\end{quote}
 तदन्तरे निरक्षदेशोक्तावधिदेशयोरन्तरालदक्षिणोत्तरविभागदेशे षष्ट्यन्ते षष्टिघटीमध्ये क्षयवृद्धी अपचयोपचयावुक्तरीत्या दिनरात्र्योर्यथायोग्यं भवतः। परतोऽवधिदेशादग्रिमदेशे दक्षिणोत्तरे दैत्यदेवस्थामनिकटेऽयं प्रत्यक्षो भगोलोनक्षत्राद्यधिष्ठितो मूर्तो गोले विपरीतोऽवधिदेशान्तर्गतदेशसम्बन्धी गणितविरुद्धः परिवर्तते भ्रमति। तत्रोक्तरीत्या दिनरात्र्योर्वृद्धिक्षयौ न भवत इत्यर्थः। त्रिज्य धिकाच्चरानयनानुपपत्तेः। चरस्वरूपासम्भवाच्च~॥~६२~॥ \\
\noindent अथ विपरतिगोलस्थितं तं श्लोकाभ्यां प्रदर्शयति \textendash 

%\vspace{2mm}

 \begin{quote}
{\ssi ऊनभूवृत्तपादे तु द्विज्यापक्रमयोजनैः~।\\
धनुर्मृगस्थः सविता देवभागे न दृश्यते~॥~६३~॥ 

तथाचासुरभागे तु मिथुने कर्कटे स्थितः~।\\
नष्टच्छायामहीवृत्तपदि दर्शनमादिशेत्~॥~६४~॥}
%\vspace{2mm}
\end{quote}
 द्विराशिज्यया ये क्रान्त्यंशास्तेषां योजनैः पूर्वागतैर्भूपरिधिचतुर्थांशे हीने कृते सति। तुकारान्निरक्षदेशात् तद्योजनान्तरिते देशे देवभाग उत्तरभागे धनुर्मकरराशिस्थोऽर्कस्तद्देशवासिभिर्न दृश्यते। धनुर्मकरस्थेऽर्के तेषां रात्रिः सदा स्यादि \textendash



\newpage


\hspace{3cm} गूढार्थप्रकाशकेन सहितः~। \hfill ३२९
\vspace{2mm}


\noindent त्यर्थः। असुरभागे निरक्षदेशाद्दक्षिणप्रदेशे। चः समुच्चयार्थः। तुकारात् तद्योजनान्तरितप्रदेशे मिशुने कर्के कर्कराशौ स्थितोऽर्कस्तथा तद्देशवासिभिर्न दृश्यते। नष्टच्छायामहीवृत्तपादे। अभावं प्राप्ता छाया भूच्छाया यत्र तादृशे भूपरिधिचेतुर्थांशे सूर्यस्य दर्शनं सदा कथयेत्। यत्र भूच्छायात्मिका रात्रिर्नास्तितत्र दिनमित्यर्थः। तथाच निरक्षदेशात् तद्योजनान्तरितोत्तरप्रदेशे कर्कमिथुनस्थोऽर्को दृश्यते तद्योजनान्तरितदक्षिणप्रदेशेधनुर्मकरस्योऽर्को दृश्यत इति फलितार्थः। अतएव \textendash

%\vspace{2mm}

\begin{quote}
{\qt त्र्यंशयुङ्गवरसाः पलांशका \\
 यत्र तत्र विषये कदाचन~।\\
दृश्यते न मकरो न कार्मुकं \\
किञ्च कर्किमिथुनौ सदोदितौ।। }
%\vspace{2mm}
\end{quote}

इति भास्कराचार्योक्तं सङ्गच्छते~॥~६४~॥ \\
\noindent अथान्यत्रापि विपरतिस्थितिं श्लोकाभ्यां दर्शयति \textendash 

%\vspace{2mm}

 \begin{quote}
{\ssi एकज्यापक्रमानीतैर्योजनैः परिवर्जितैः~।\\
भूमिकक्षाचतुर्थांशे व्यक्षाछेषैस्तु योजनैः~॥~६५~॥

धनुर्मृगालिकुम्भेषु संस्थितोऽर्को न दृश्यते~।\\
देवभागेऽसुराणां तु वृषाद्ये भचतुष्टये~॥~६६~॥}
%\vspace{2mm}
\end{quote}
 एकराशिज्यायाः क्रान्त्यंशेभ्यो भूपरिधिचतुर्थांशे हीने कृतेसति निरक्षदेशादवशिष्टैर्योजनैः। तुकारादन्तरिते देशे देव भाग उत्तरभागे धनुर्मकरवृश्चिककुम्भराशिषु स्थितः सूर्यस्त \textendash
%


\newpage

\noindent ३३० \hspace{4cm} सूर्यसिद्धान्तः
\vspace{1cm}


द्देशवासिभिर्न दृश्यते। असुराणां दैत्यानां निरक्षदेशात् तद्योजनान्तरितदक्षिणभागे वृषादिके राशिचतुष्टये स्थितोऽर्कस्तद्देशवासिभिर्न दृश्यते। तुकारादुत्तरभागे वृषादिचतुष्टयस्यितोऽर्कस्तद्देशवासिभिर्दृश्यते वृश्चिकादिचतुष्टयस्थितोऽर्कोदक्षिणभागे तद्देशवासिभिर्दृश्यत इत्यर्थः। अतएव \textendash
%\vspace{2mm}

 \begin{quote}
{\qt यत्र साद्रिगजवाजिसम्मितास्तत्र वृश्चिकचतुष्टयं न च~।\\
 दृश्यते च वृषभाच्चतुष्टयं सर्वदा समुदितं हि लक्ष्यते~॥~}
%\vspace{2mm}
\end{quote}
इति भास्कराचार्योक्तं च सङ्गच्छते~॥~६६~॥\\
\noindent अथ शून्यराशिक्रान्त्यानीतयोजनेभ्योऽवगतमेर्वग्रभागो रपि स्थितिवैलक्षण्यमाह \textendash

%\vspace{2mm}

 \begin{quote}
{\ssi मेरौ मेषादिचक्रार्धे देवाः पश्यन्ति भास्करम्~। \\
सकृदेवोदितं तद्वदसुराश्च तुलादिगम्~॥~६७~॥}
%\vspace{2mm}
\end{quote}
 मेरावुत्तराग्रावस्थिता देवा मेषादिचक्रार्धे मेषादिराशियद्वेऽवस्थितमर्कं सकृदेकवारम्। एवकारादनेकवारनिरासनिश्चयः। उदितमदर्शनानन्तरं प्रथमदर्शनविषवं निरन्तरं पश्यन्ति। असुरा मेरुदक्षिणायस्था दैत्याः। चो देवैः समुच्चयार्थः। तुलादिराशिषट्कस्यं तद्वत् सकृदुदितं निरन्तरं पश्यन्ति~॥~६७~॥ \\
\noindent अथ निरक्षदेशादयनसन्धौ कियद्भिर्योजनैरूर्ध्वमर्को भवति तदाह \textendash



\newpage


\hspace{3cm} गूढार्थप्रकाशकेन सहितः~।  \hfill ३३१ 
\vspace{1cm}

%

 \begin{quote}
{\ssi भूमण्डलात् पञ्चदशे भागे देवेऽथ वासुरे~।\\
उपरिष्टाद्वजत्यर्कः सौम्ययाम्यायनान्तगः~॥~ ६८~॥}
%\vspace{2mm}
\end{quote}
 देव उत्तरभागे। अथवासुरे दक्षिणभागे। निरक्षदेशाद्भूपरिधेः पञ्चदशे भागे तत्फलयोजनान्तरिते देशे क्रमेण सौम्ययाम्यायनान्तग उत्तरायणान्तदक्षिणायनान्तस्थितोऽर्क उपरिष्टादूर्ध्वं व्रजति परिभ्रमति। यथा गोलसन्धौ निरक्षदेशे तथात्रभागद्वय इति फलितार्थः। अत्रोपपत्तिः। अयनान्तस्थे परमक्रान्तिश्चतुर्विशत्यंशास्तद्योजनानि \textendash
%\vspace{2mm}

 \begin{quote}
{\qt भूवृत्तं क्रान्तिभागघ्नं भगणांशविभाजितम्~। }
 %\vspace{2mm}
\end{quote}
इति चतुर्विंशतिमितगुणो भगणांशमितहरो गुणेनापवर्त्य हरस्थाने पञ्चदशेति भूमण्डलात् पञ्चदग्रे भाग इत्युक्तमुपपन्नम्~॥~६८~॥ \\
\noindent अथ निरक्षदेशाद्भूपरिधिपञ्चदशभागपर्यन्तं सूर्यस्यदक्षिणोत्तरतो गमनमत्क्का तच्छायागमनं प्रतिपादयति \textendash

%\vspace{2mm}

 \begin{quote}
{\ssi तदन्तरालयोच्छाया याम्योदक् सम्भवत्यपि~।\\
मेरोरभिमुखं याति परतः स्वविभागयोः~॥~६९~॥}
%\vspace{2mm}
\end{quote}
 तदन्तरालयोर्निरक्षदेशात् पञ्चदशभागमध्यस्थितदक्षिणोत्तरदेशयोश्छाया द्वादथालशङ्गुलशङ्कोर्मध्या च्छायाभीष्टकालिकच्छायाग्रं वा दक्षिणाग्रमुत्तराग्रं वा सम्भवति। एतदुक्तं भवति। निरक्षदेशात् पञ्चदशभागान्तरालोत्तरदेशे मध्याह्ननतांशानांदक्षिणत्वे छायाग्रमुत्तरमुन्नतांशानामुत्तरत्वे छायाग्रं दक्षिणम्। एवं निरक्षदेशात् पञ्चदशभागान्तरालस्थितदक्षिणदेशे सूर्य \textendash
%


\newpage


\noindent ३३२ \hspace{4cm} सूर्यसिद्धान्तः 
\vspace{1cm}


\noindent स्योत्तरत्वे छायाग्रं दक्षिणं दक्षिणस्यत्वे छायाग्रमुत्तरमिति। परतः पञ्चदशभागान्तरालदेशे स्वविभागयोर्दक्षिणोत्तरविभागयोर्मेरोरभिमुखमेवाग्रयोः सम्मुखं क्रमेण दक्षिणाग्रमुत्तराग्रंयथा स्यात् तथेत्यर्थः। छाया याति गच्छति भवतीत्यर्थः। अपिशब्दः पर्वार्धार्थेन समुच्चयार्थकः~॥~६९~॥ \\
\noindent अथ कथं पर्येतिभुवनानि विभावयन्निति प्रश्रस्योत्तरं श्लोकाभ्यामाह \textendash

%\vspace{2mm}

 \begin{quote}
{\ssi भद्राश्वोपरिगः कुर्याद्भारते तूदयं रविः~।\\
रात्र्यर्धं केतुमाले तु कुरावस्तमयं तदा~॥~७०~॥

भारतादिषु वर्षेषु तद्वदेव परिभ्रमन्~।\\
मध्योदयार्धरात्र्यस्तकालात् कुर्यात् प्रदक्षिणम्~॥~७१~॥}
%\vspace{2mm}
\end{quote}

 भद्राश्ववर्षोपरिगतः सूर्यो भरतवर्षे स्वोदयं कुर्यात्। तुकाराद्भद्राश्ववर्षे मध्याह्नं कुर्यात्। तदा तस्मिन् काले केतुमालवर्षेऽर्धरात्रं कुरौ कुरुवर्षेऽस्तमयं स्वास्तं कुर्यात्। तुकारादुक्तवर्षयोरन्तराले दिनस्य गतं शेषं वा रात्रेश्च तद्यथायोग्यं कुर्यादित्यर्थः। अतिस्थूलदेशग्रहणे यथाश्रुतमिदं भव्यं किञ्चित्सूक्ष्मदेशग्रहणे तु यमकोटिलङ्कारोमसिद्धपुराण्यन्तर्गतानि तच्छब्दवात्र्यानि ज्ञेयानि। 

%\vspace{2mm}

\begin{quote}
{\qt लङ्कापुरेऽर्कस्य यदोदयः स्यात् \\
तदा दिनार्धं यमकोटिपुर्याम~।\\
अधस्तदा सिद्धपुरेऽस्तकालः \\
स्याद्रोमके रात्रिदलं तदैव~॥}
%
\end{quote}

\newpage


\hspace{3cm} गूढार्थप्रकाशकेन सहितः~। \hfill ३३३ 
\vspace{1cm}

%
\noindent इति भास्कराचार्योक्तं भूगोल उक्तनगराणां भूपरिध्रिचतुर्थांशान्तरत्वात् सङ्गच्छते। अथ भारतादिषु त्रिषु वर्षसञ्ज्ञेषु भारतकेतुभालकुरुवर्षेषु तद्वद्भद्राश्ववर्षोपरिगवत्। एवकारात् तत्र्यूना कव्यवच्छेदः। परिभ्रमन् परिभ्रमेण स्वस्याभिमतस्थानोपरिस्थिति कुर्वन् सूर्यः प्रदक्षिणं यथा स्यात् तथा सत्यक्रमेण स्वस्थानादिक्रमेणेति यावत्। उक्तचतुर्वर्षेषु मध्योदयार्धरात्र्यस्तकालान्मध्या दयार्धरात्र्यस्तसञ्ज्ञान् कालान् कुर्यात्। एतदुक्तं भवति। भारतवर्षोपरि गतेऽर्के भारतकेतुमालकुरुभद्राश्ववर्षेषु क्रमेण मध्याह्नसूर्योदयार्धरात्रास्ताः स्युः। केतुमालवर्षोपरि गतेऽर्के केतुमालकुरुभद्राश्वभारतवर्षेषु क्रमेण मध्याह्नसूर्योदयार्धरात्रास्ताः। कुरुवर्षोपरि गतार्के कुरुभद्राश्वभारतकेहमालवर्षेषु क्रमेणा मध्याह्नसूर्योदयार्धरात्रास्या भवन्तीति~॥~७१~॥ \\
\noindent ननु ग्रहाणां गतिसद्भावात् प्रतिदेशंयाम्योत्तरयोर्ग्रहगमनं प्रतिक्षणं च विलक्षणं भासतां परन्तुनक्षत्राणां गत्यभावात् प्रतिक्षणं भ्रमेणैकत्रावस्यानाभावेऽपिप्रतिदेशमेकरूपावस्थानं कुतो न। एवं ध्रुवयोः परिभ्रमस्याप्यभावात् सदा सर्वत्रैकरूपावस्थानदर्शनापत्तिश्चेत्यत आह \textendash

%\vspace{2mm}

 \begin{quote}
{\ssi ध्रुवोन्नतिर्भचक्रस्य नतिर्मेरुं प्रयास्यतः~।\\
निरक्षाभिमुखं यातुर्विपरीते नतोन्नते~॥~७२~॥}
%\vspace{2mm}
\end{quote}
\noindent मेरुं मेरोरुत्तराग्र दीक्षेणाग्रं वा तदभिमुखं प्रयास्यतोगच्छतः पुरुषस्य ध्रुवोन्नतिः क्रमेणोत्तरदक्षिणयोर्ध्रुवयोरौच्च्यं \textendash
%


\newpage


\noindent ३३४ \hspace{4cm} सूर्यसिद्धान्तः
\vspace{1cm}


\noindent भवति। भचक्रस्य नक्षत्राधिष्ठितगोलमध्यभागवृत्तस्य नतिः क्रमेण दक्षिणोत्तरयोर्नतत्वं भवति। निरक्षदेशाभिमुखं गच्छतः पुरुषस्य नतोन्नते पर्वोक्ते व्यस्ते भवतः। उत्तरभागस्थपुरुषस्य निरक्षाभिमखं गच्छतः पूर्वोक्तस्थानापेक्षयोत्तरध्रुवस्य नतत्वंपूर्वस्थानापेक्षया भचक्रस्योन्नतत्वम्। एवं दक्षिणभागस्यपुरुषस्य निरक्षाभिमुखं गच्छतः पूर्वस्थानापेक्षया दक्षिणध्रुवस्य नतत्वं भचक्रखोन्नतत्वमिति~॥~७२~॥ \\
\noindent अथ कुत एवमित्यतः \textendash
%\vspace{2mm}

\begin{quote}
{\qt कथं पर्येति भगणः सग्रहोऽयं किमाश्रयः~। }
% \vspace{2mm}
\end{quote}
\noindent इति प्रश्नस्योत्तरं भचक्रभ्रमणवस्तुस्थितिमाह \textendash

%\vspace{2mm}

 \begin{quote}
{\ssi भचक्रं बुवयो बद्ध्वमाक्षिप्तं प्रवहानलः~।\\
पर्येत्यजस्रं तन्नडूा ग्रहकक्षा यथाक्रमम्~॥~७३~॥}
%\vspace{2mm}
\end{quote}
 भचक्रं नक्षत्राधिष्ठितमूर्तगोलरूपं ध्रुवयोर्दक्षिणोन्तरस्थिरतारयोर्बद्धं ब्रह्मणा निबद्धं नियतवायुगतिना गोलाकारेण प्रतिबद्धं प्रवहानिलैः प्रवहवाय्वंशैः स्वस्वस्थनस्थैराक्षिप्तं स्वस्वस्थानाभिघातं प्राप्तं सदजस्रं निरन्तरं पर्येति। पश्चिमाभिमुखंभ्रमतीत्यर्थः। ननु नक्षत्रचक्रं वायुना भ्रमति ग्रहास्त्रधोऽधःस्याः सम्बन्धाभावात् कथं भ्रमन्तीत्यत आह\textendash तन्नद्धा इति~। ग्रहाणां शन्यादीनां कक्षा मार्गा वाय्वंशरूपा भचक्रान्तर्गताकाशस्था यथाक्रममधोऽधस्तन्नद्धा महाप्रवहवायुगोलस्थापितभचक्रे वायुसूत्रेण निबद्धा अतो भचक्रेण सह भ्वमन्ति। तत्रस्था ग्रहा अपि भ्रमन्तोति किं चित्रम्। तथाच प्रवहवायुगोलमध्याख्यविषु \textendash
%


\newpage


\hspace{3cm} गूढार्थप्रकाशकेन सहितः~। \hfill ३३५
\vspace{1cm}

%
\noindent वहुत्तपूर्वापरनिरक्षदेशे ध्रुवयोः क्षितिजस्थत्वाद्भाचक्रस्यमस्तकोपरि भ्रमणाच्च मेर्वग्राभिमुखं प्रयातध्रुव उच्चो भवति। तत आसन्नत्वाद्भचक्रं नतं भवति। ततो दूरत्वादिति सर्वं युक्तम्~॥~७३~॥ \\
\noindent अथ पित्र्यं मासेन भवतीति प्रश्नयोरुत्तरमाह \textendash

%\vspace{2mm}

\begin{quote}
{\ssi सकृदुङ्गतमब्दार्धं पश्यन्त्यंर्कं सुरासुराः~।\\
पितरः शशिगाः पक्षं स्वदिनं च नरा भुवि~॥~७४~॥}
%\vspace{2mm}
\end{quote}
 यथा देवदैत्या एकवारमुदितं सूर्यं 'सौरवर्षार्धपर्यंन्तं पश्यन्ति। नरा भूमौ स्वदिनपर्यन्तमर्कं पश्यन्ति 

%\vspace{2mm}

 \begin{quote}
{\qt  पित्र्यं मासेन भवति नाडीषष्ट्या तु मानुषम्~। }
% \vspace{2mm}
\end{quote}
इति सर्वं युक्तमतएव \textendash

%\vspace{2mm}

 \begin{quote}
{\qt विधूर्ध्वभागे पितरो वसन्तः \\
स्वाधः सुधादीधितिमामनन्ति~।\\
पश्यन्ति तेऽर्कं निजमस्तकोर्ध्व \\
दर्शे यतोऽस्माद्युदलं तदैषाम्~॥

भार्धान्तरत्वान्न विधोरधःस्थं \\
तस्मान्निशीथः खलुपौर्णमास्याम्~।\\
कृष्णे रविः पक्षदलेऽभ्युदेति \\
शुक्लेऽस्तमेत्यर्थत एव सिद्धम्~॥}
%\vspace{2mm}
\end{quote}
इति भास्कराचार्येण विस्तीर्योक्तं सङ्गच्छते~॥~७४~॥ \\
\noindent अथ प्रसङ्गादूस्थास्याल्पभगणामधःस्थस्याधिकभगणानां युक्त्या प्रति \textendash



\newpage


\noindent ३३६ \hspace{4cm} सूर्यसिद्धान्तः
\vspace{1cm}


पादनार्थं प्रथमं कक्षाया ऊर्ध्वाधःक्रमेण महदल्पत्वं तत्रस्थभागानां महदल्पप्रदेशत्वं चाह \textendash

%\vspace{2mm}

 \begin{quote}
{\ssi उपरिस्थस्य महती कक्षाल्पाधःस्थितस्य च~।\\ 
महत्या कक्षया भागा महान्तोऽल्पास्तथाल्पया~॥~७५~॥ }
%\vspace{2mm}
\end{quote}
 ऊर्ध्वस्थस्य ग्रहस्य कक्षा वायुवृत्तमार्गरूपा महती महापरिधिप्रमाणा। अधःस्थस्य ग्रहस्य कक्षाल्पाल्पपरिधिप्रमाणा। चो निश्चयार्थे। लघुकक्षाणां महाकक्षान्तर्गमनं महाकक्षाणांचान्तर्गतलघुकक्षात्वेनोर्ध्वाधःस्थयोर्महदल्पपरिधिके कक्षे। अन्यथोक्तस्वरूपानुपपत्तेः। एवं महति वृत्तपरिधौ द्वादशराशिआगानां समत्वेनाङ्कने भागा एकैकभागप्रदैशा महत्या कक्षया कृत्वा महान्तो बहुस्थलात्मका लघुनि वृत्ते तदङ्कनेतथा भागा अल्पया कक्षया कृत्वाल्पा अल्पस्थलात्मकाः क्रमेणैकैकभागप्रमाणमधिकाल्पं न समं चक्रांशपूर्त्यनपपत्तेरितितात्पर्यम्~॥~७५~॥\\
\noindent अथोर्ध्वधःक्रमेण ग्रहभगणभोगकालयोर्महदल्पत्वमाह \textendash

%\vspace{2mm}

\begin{quote}
{\ssi कालेनाल्पेन भगणं भुङ्क्तेऽल्पभ्रमणाश्रितः~।\\
ग्रहः कालेन महता मण्डले महति भ्रमन्~॥~७६~॥}
\end{quote}
%\vspace{2mm}
 अल्पभ्रमणाश्रितः। अल्पभ्रमणं परिधिमानं यस्याः साल्पभ्रमणाधःस्थकक्षा। तत्स्थो ग्रहोऽल्पेन समयेन भगणं द्वादशराश्यात्मकं भुङ्क्तेऽतिक्रमते। महति मण्डले। ऊर्ध्वस्थकक्षायामित्यर्थः। भ्रमम् गच्छन् महता बहुना शमयेन् द्वादश \textendash



\newpage


\hspace{3cm} गूढार्थप्रकाशकेन सहितः~। \hfill ३३७
\vspace{1cm}


\noindent राशीन् भुङ्क्ते। वक्ष्यमाणयोजनगतेरभिन्नत्वात्~॥~७६~॥ \\
\noindent अथातएवोर्ध्वाधःक्रमेण ग्रहयोर्भगणास्तुल्यकालेऽल्पा बहवो भवन्तीति सोदाहरणमाह \textendash 

%\vspace{2mm}

\begin{quote}
{\ssi स्वल्पयातो बहून् भुङ्क्ते भगणान् शीतदीधितिः~। \\
महत्या कक्षया गच्छन् ततः स्वल्पं शनैश्चरः~॥~७७~॥}
%\vspace{2mm}
\end{quote}
 स्वल्यप्रमाणवा कक्षया। तुकारादतिक्रामश्चन्द्रो बहुप्रमाणान् भगणान् बहवारं द्वादश राशीनित्यर्थः। भुङ्क्ते। महाप्रमाणया कक्षया गच्छन् शनिस्ततश्चन्द्रात् स्वल्पं भगणमल्पममाणान् अगणाम्। जात्यभिप्रायेणैकवचनम्। अल्पवारं द्वादशराशीन् भुङ्क्ते। अतएव शनैश्वर इति~॥~७७~॥\\
 अथ \textendash

%\vspace{2mm}

\begin{quote}
 {\qt दिनाब्दमासहोराणामधिपा न समाः कुतः~। }
 \end{quote}
% \vspace{2mm}

इति प्रश्नस्योत्तरं होकाभ्यामाह\textendash


\begin{quote}
 {\ssi मन्दादधःक्रमेण स्युश्चतुर्था दिवसाधिपाः~।\\
वर्षाधिपतयस्तद्वत् तृतीयाश्च प्रकीर्तिताः~॥~७८~॥

ऊर्ध्वक्रमेण शशिनो मासानामधिपाः स्मुताः~।\\
होरेशाः सूर्यतनयादधोऽधःक्रमतस्तथा~॥~७९~॥}
\end{quote}
\noindent

 ऊर्ध्वक्रमेण चतुर्थसङ्ख्याक्रा ग्रहा दिनाधिपतयो वारेश्वरा भवन्ति। यथा शनिरविचन्द्रभौमबुधगुरुशुक्रा इति तत्क्रमः। वर्षस्य षष्ट्यधिकशतत्रयदिनात्मकस्य स्वामिनस्तद्वन्मन्दादधः क्रमेण तृतीयसङ्ख्याका ग्रहा उक्ताः। चः समुच्चयार्थे। तत्कमश्च यथा शनिभौमश्यक्रचन्द्रगुरुसूर्यबुधा इति। चन्द्रात् सकाशा\textendash



\newpage


\noindent ३३८ \hspace{4cm} सूर्यसिद्धान्तः
\vspace{1cm}


\noindent र्ध्वकक्षाक्रमेण ग्रहा मासानां विंशद्दिनात्मकानां स्वामिनः कथिताः। तत्समश्च चन्द्रबुधश्यक्ररविभौमगुरुशनय इति। शनेः सकाशादधःक्रमशः। अधःक्रमेण होरेशाः 

%\vspace{2mm}

\begin{quote}
 {\qt होरेति लग्नं भगणस्य चार्धम्~। }
 \end{quote}
 %\vspace{2mm}

इति पञ्चदशभागात्मकहोराणां दिने द्वादय रात्रौ द्वादशेत्यहोरात्रे चतुर्विंशतिहोराणामित्यर्थः। 

%\vspace{2mm}

 \begin{quote}
 {\qt  होरा सार्धद्विनाडिका। }
%  \vspace{2mm}
\end{quote}
इति षष्टिघटिकात्मकेऽहोरात्रे। चतुर्विंशतिहोराणामित्यन्ये। स्वामिनस्तथा मासेश्वरवदव्यवहिताः कथिताः। यथा तत्क्रमः शनिगरुभौमरविशुक्रबुधचन्द्रा इति। अत्र शनेः सर्वोर्ध्वस्थत्वाच्चन्द्रस्य सर्वाधःस्थत्वात् ताभ्यामधू र्ध्वक्रमः क्रमेणोक्तः। अन्यग्रहस्याधिपत्वाभ्युपगमं विनिगमनाविरहापत्तेः। ननु शनेराद्यावधित्वेन सृष्ट्यादौ दिनवर्षहोराणां स्वामित्वं नवा चन्द्रस्याद्यावधित्वेन सृष्ट्यादौ मासेशत्वं पूर्वखण्डोक्तानीततदीशैर्विरोधापत्तेः। अत्रोपपत्तिः। होरारूपलग्नानां क्रान्तिवृत्तेऽधःक्रमेण मेषादीनां सम्भवादूर्धकक्षातौऽधःक्रमैण होरेशत्वं युक्तम्। एवमहोरात्रे चतुर्विंशतिहोराः सप्ततष्टास्त्रयो होरेशा गताः। चतुर्थो होरेशो द्वितीयदिनप्रारम्भे स एव प्रथमहोरेशत्वाद्वितीयदिनेशः। एवमुत्तरत्रापि। एवमेतद्वारक्रमेण सावनवर्षे त्रयो वारा इति पर्वूवर्षेशादग्रिमवर्षेशोऽधःकक्षाक्रमेण तृतीय उक्तरोन्तरम्। एवं सावनमासे द्वौ वारौ वारक्रमेण मासेश्वरस्याधिकाविति कक्षोर्ध्वक्रमे वारक्रमेणैकात्तरित्त्वात् कक्षो \textendash
%


\newpage


 \hspace{3cm} गूढार्थप्रकाशकेन सहितः~। \hfill ३३९
\vspace{1cm}



\noindent र्ध्वक्रमेण मासेश्वर उत्तरोत्तरमित्युपपन्नं मन्दादित्यादि श्लोकद्वयम्~॥~७९~॥ \\
अथ \textendash

\begin{center}
{\qt ग्रहर्क्षकक्षाः किंमात्राः। }
\end{center}

इति प्रश्नस्योत्तरं विवक्षुः प्रथमं नक्षत्राणां कक्षामानमाह\textendash

%\vspace{2mm}

\begin{quote}
 {\ssi भवेद्भकक्षा तिग्मांशोर्भ्रमणं षष्टिताडितम्~।\\
सर्वोपरिष्टाङ्ग्रमति योजनैस्तैर्भमण्डलम्~॥~८०~॥}
\end{quote}
%\vspace{2mm}

 सूर्यस्य भ्रमणं कक्षापरिधिमानं योजनात्मकम्। 

\begin{center}
{\qt खखार्थैकसुरार्णवाः~।} 
\end{center}

इति वक्ष्यमाणं षष्ट्या गुणितं सन्नक्षत्राणां कक्षा नक्षत्राधिष्ठितगोलस्य मय्यवृत्तं स्यात्। तैर्नक्षत्रकक्षामितैर्योजनैर्भमण्डलं नक्षत्राधिष्ठितगोलमध्यवृत्त सर्वोपरिष्टाच्चन्द्रादिसप्तग्रहेभ्य उपरि दूरं भ्रमति भूगोलादभितः परिभ्रमति। अत्रोपपत्तिः। नक्षत्राणां गत्यभावस्थानेऽप्यत्यूर्ध्वं नक्षत्रमण्डलं तत्र सूर्यगत्या सूर्यकक्षा तदा नक्षत्रगत्यभावेऽप्येककलागतिकल्पनयानुपातान्यथानुपपत्तितया 

\begin{center}
 {\qt कल्प्यो हरो रूपमहारराशेः~।}
\end{center}

इतीच्छाहासे फलवृड्व्यपेक्षितत्वाद्व्यस्तानुपातो लाघवात् सूर्यगतिः षष्टिकलामिता च भगवता कृता। नक्षत्रगतेरभावाच्चेति षष्टिताडितमित्युपपन्नम्~॥~८०~॥\\
\noindent अथ ग्रहकक्षाणां मानज्ञानार्थमाकाशकक्षामानम् \textendash

\begin{center}
{\qt कियती तत्करप्राप्तिः~।} 
\end{center}




\newpage

\noindent ३४० \hspace{4cm} सूर्यसिद्धान्तः
\vspace{1cm}


\noindent इति प्रश्र स्योत्तरमाह\textendash

%\vspace{2mm}

\begin{quote}
{\ssi  कल्पोक्तचन्द्रभगणा गुणिताः शशिकक्षया~।\\
आकाशकक्षा सा ज्ञेया करव्याप्तिस्तथा रवेः~॥~८१~॥}
\end{quote}
%\vspace{2mm}

 कल्पोक्तचन्द्रभगणाः। 

%\vspace{2mm}
% {\setlength{\parindent}{5em}
\begin{quote}
{\qt एते सहस्रगुणिताः कल्पे स्युर्भगणादयः। }
\end{quote}
 %\vspace{2mm}

इत्युक्त्या युगचन्द्रभगणाः सहस्रगुणिताः कल्पचन्द्रभगणा इत्यर्थः। चन्द्रकक्षया खत्रयाब्धिद्विदहना इति वक्ष्यमाणया गुणिता सा तन्मिताकाशकक्षापरिधिरूपा ज्ञेया। धीमतेति शेषः। मन्वनन्ताकाशस्य कथं परिधिरित्यत आह\textendash करव्याप्तिरिति~। सूर्यस्य किरणप्रचारस्तथाकांशकक्षापरिमित इत्यर्थः। तथाच यद्देशावच्छेदेन सूर्यकिरणप्रचारस्तद्देशाछिन्नाकाशगोलस्य ब्रह्माण्डकटाहान्तर्गतस्य परिधिमानं सम्भवत्येवेति भावः। अत्रोपपत्तिः। समनन्तरमेव यद्भगणभक्ता ककक्षा तस्य कक्षा स्यादित्युक्तेर्भगणकक्षाघातः स्वकक्षा सिद्धा। अतश्चन्द्रभगणकक्षयोर्धातः खकक्षातुल्य एवेति दिक्~॥~८१~॥ \\
\noindent अथ ग्रहाणां कक्षानथनं योजनगत्यानयनं चाह\textendash

%\vspace{2mm}

\begin{quote}
 {\ssi सैव यत्कल्पभगणैर्भक्ता तङ्गमणं भवेत्~।\\
कुवासरैर्विभज्याष्ठः सैर्वेषा प्राग्गतिः स्मृता~॥~८२~॥}
\end{quote}
%\vspace{2mm}

 सार्ककरव्याप्तिरूपाकाशकक्षा यत्कस्मभगणैर्यस्य कल्पभगणैर्भक्ता फलं तस्य कक्षा भवेत्। एवकारो निश्चयार्थे। स्वकक्षा कल्परविसावनैर्भक्ता प्राप्तं फलं सर्वेषामुक्तभगणसम्बन्धिनां ग्रहा\textendash



\newpage


\hspace{3cm} गूढार्थप्रकाशकेन सहितः~। \hfill ३४१
\vspace{1cm}


\noindent दीनामज्ञो दिवसस्य दिनसम्बन्धिनीत्यर्थः। प्राग्गतिर्योजनात्मिका कथिता। अत्रोपपत्तिः। कल्पभगणकक्षाधातरूपाकाशकक्षा कल्पभगणभक्ता कक्षा स्यादेव। कल्पे खकक्षानितयोजजानि ग्रहः क्रामतीति कल्परविसावनदिनैराकाशकक्षामितयोजनानि तदैकरविसावनदिनेन कानीत्यनुपातेन पूर्वगतिर्योजनात्मिका प्रत्यहं तुल्येत्युपपन्नम्~॥~८२~॥ \\
\noindent अथ योजनात्मकगतेः कलात्मकगतिं स्वीयामाह \textendash 

%\vspace{2mm}

\begin{quote}
{\ssi भुक्तियोजनजा सङ्ख्या सेन्दोर्भ्रमणसङ्गुणा~।\\
खकक्षाप्ता तु सा तस्य तिथ्याप्ता गतिलिप्तिकाः~॥~८३~॥ }
%\vspace{2mm}
\end{quote}
 गतियोजनोत्पन्ना या सङ्ख्या सा सङ्ख्या चन्द्रस्य भ्रमणसङ्गुणाकक्षया गणिता खकक्षयाप्ताभिमतग्रहस्य कक्षया भक्ता सा फलरूपा तिथ्यप्ता पञ्चदशभक्ता। तुकारात् फलं तस्याभिमतग्रहस्यगतिकला भवन्ति। अत्रोपपत्तिः। कक्षायोजनैश्चक्रकलास्तदामतियोजनैः का इत्यनुपातेन गतिकलाः। तत्रापि चन्द्रकक्षापञ्चदशभक्ताश्चक्रकला इति चक्रकलास्वरूपं धृतमित्युपपन्नम्~॥~८३~॥ \\
\noindent अथ किमुत्सेधा इति प्रश्नस्योत्तरमाह \textendash

%\vspace{2mm}

\begin{quote}
{\ssi कक्षा भूकर्णगुणिता मचीमण्डलभाजिता~।\\
तत्कर्णा भूमिकर्णोना ग्रहोच्चं स्वं दलीकृताः~॥~८४~॥}
%\vspace{2mm}
\end{quote}

 ग्रहाणां योजनात्मिका कक्षा भूकर्णेन योजनानि शतान्यष्टौ भूकर्णो द्विगुणानीत्युक्तभूव्यासेन षोडशशतेन गुणिता भूपरिधिना तदवगतेन भक्ता फलं तस्याः कक्षायाः कर्णावासा \textendash


\newpage


\noindent ३४२ \hspace{4cm} सूर्यसिद्धान्तः
\vspace{1cm}


\noindent भवन्ति। एते भूव्यसिन हीना अर्धिताः सन्तः स्वगृहीतव्याससम्बन्धीग्रहौच्च्यं ग्रहस्योच्चता भूमेः सकाशाद्भवति। अत्रोपपत्तिः। भूपरिधिना भूव्यासस्तदा कक्षायोजनैः क इत्यनुपातेन कक्षाव्यासास्तेऽर्धिताः कक्षाव्यासार्धं भूगर्भकक्षापरिधिप्रदेशान्तरालरूपं भूपृष्ठात् तदन्तरज्ञानार्थं भूव्यासार्धेन हीनं भूपृष्ठात् कक्षौच्च्यं तत्र कक्षाव्यासा भूव्यासोना अर्धिताः कृताः। उभयथासमत्वात्। कक्षौव्या मेव ग्रहीच्च्यं तत्राधिष्ठानादिति। एतेन सिद्धग्रहौच्च्येभ्यः परस्परान्तरज्ञानं सुगममिति। किमन्तरा इतिप्रश्नस्योत्तरं स्वतः सिद्धमेवेति दिक्~॥~८४~॥\\
अथोर्ध्वक्रमेणसिद्धाः कक्षा विवक्षुः प्रथमं चन्द्रस्य कक्षां बुधशीघ्रो  कक्षांचाह \textendash

%\vspace{2mm}

\begin{quote}
{\ssi खत्रयाब्धिद्विदहनाः कक्षा तु हिमदीधितेः~।\\
ज्ञशीघ्रस्याङ्कखद्वित्रिकृतशून्येन्दवस्ततः~॥~८५~॥}
%\vspace{2mm}
\end{quote}
 चन्द्रस्य कक्षा सहस्रगुणितसिद्धरामाः। तुकारादागमप्रामाण्येनाङ्गीकार्या।mअन्यथान्योज्याश्रयापत्तेस्ततश्चन्द्रादूर्ध्वबुधशीघ्रोच्चस्य कक्षा नवखदन्तवेददिशः। यद्यपि बुधशीघ्रोच्चमाकाशे प्रत्यक्षं नेति तत्कक्षोक्तिरयुक्ता तथापि बुधशीघ्रोच्चभगणानीतकक्षायां गत्यनुरोधेन चन्द्रोर्ध्वगायां बुधो भ्रमति। पूर्वं \textendash

\begin{center}
{\qt सूर्यशुक्रेन्दुजेन्दवः~।} 
\end{center}

इति क्रमोक्तेः। अन्यथा भगणैक्यादेककक्षायां रविबुधंशुक्राणामवस्थितौ मण्डलभङ्गापत्तेरिति सूचनार्थमुक्तम्~॥~८५~॥ \\
\noindent अथशुक्रशीघ्रोच्चस्य कक्षां सूर्यबुधशुक्राणामभिन्नां कक्षाचाह \textendash


\newpage


\hspace{3cm} गूढार्थप्रकाशकेन सहितः~। \hfill ३४३
\vspace{1cm}

%

 \begin{quote}
{\ssi शुक्रशीघ्रस्य सप्ताग्निरसाब्धिरसषड्यमाः~।\\
ततोऽर्कबुधशुक्राणां खखार्थैकसुरार्णवाः~॥~८६~॥}
%\vspace{2mm}
\end{quote}
 तदूर्धं शुक्रशीघ्रोच्चस्य कक्षाद्रित्र्यङ्गवेदषड्सपक्षाः शुक्रावस्थानसूचनार्थमुक्ताः। ततस्तदूर्ध्वं सूर्यबुधशुक्राणां भगणैक्यादभिन्ना कक्षा खखपञ्चभूद्रेवाब्धयः। यद्यपि बुधशुक्रयोः सूर्योधस्थत्वात् केवलं सूर्यकक्षैव वक्तुमुचिता तथापि कक्षयैको भगणस्तदा कल्परविसावनदिनैः खकक्षामितयोजनानि तदाहर्गणेनकानीत्यनुपातागतयोजनैः क इत्यनुपातेन सूर्यबुधशुक्राणामभिन्नत्वसिड्व्यार्थं बुधशुक्रयोरप्युक्ता। अन्यथा समत्वानुपपत्तेरिति~॥~८६~॥ \\
\noindent अथ भौमस्य कक्षां चन्द्रमन्द्रोच्चस्य कक्षां चाह \textendash
%\vspace{2mm}

 \begin{quote}
{\ssi कुजस्याप्यङ्कशून्याङ्कषड्वेदैकभुजङ्गमाः~।\\
चन्द्रोच्चस्य कृताष्टाब्धिवसुद्वित्र्यष्टवक्रयः~॥८७~॥}
%\vspace{2mm}
\end{quote}

 भौमस्य। अपिशब्दात् सूर्यादूर्ध्वकक्षा नवखनवषडिन्द्रसर्पाः। चन्द्रमन्दोच्चस्य कक्षा वेदाहिवेदसर्पपक्षराभनागरामाः। इयमप्याकाशे न दृश्या तथापि गतयोजनैश्चन्द्रोच्चज्ञानायोक्ता~॥~८७~॥\\
\noindent अथ गुरुराज्योः कक्षे आह \textendash

%\vspace{2mm}

  \begin{quote}
{\ssi  कृतर्तुमुनिपञ्चाद्रिगुणेन्दुविषया गरोः~।\\
स्वर्भानोर्वेदतर्काष्टद्विशैलार्थखकुञ्जराः~॥~८८~॥}
%\vspace{2mm}
\end{quote}
 बृहस्पतेर्भौमाच्चन्द्रोच्चादूर्ध्वं कक्षा वेदाङ्गमुनिपञ्चस्वररामचन्द्रशराः। राहोः कक्षा वेदाङ्गगजयमसप्तपञ्चाशोतयः। इयम \textendash
%

\newpage


\noindent 44 \hspace{4cm} सूर्यसिद्धान्तः
\vspace{1cm}


\noindent दृश्यापि राहोर्गतियोजनैर्ज्ञनार्थमुक्ता। अत्रापि पातस्य चक्रशुद्धत्वभवधेयम्~॥~८८~॥ \\
\noindent अथ शनेः कक्षां नक्षत्राधिष्ठितमूर्तगोलमध्यकक्षां चाह \textendash

%\vspace{2mm}

 \begin{quote}
{\ssi पञ्चबाणाक्षिनागर्तुरसाद्र र्काः शनेस्ततः~।\\
भानां रविखशून्याङ्गबसुरभन्ध्रशराश्विनः~॥~८९~॥}
%\vspace{2mm}
\end{quote}
 ततो बृहस्पते राहोर्वोर्ध्वं शनेः कक्षा पञ्चपञ्चद्व्यष्टषड्रससप्तार्काः। नक्षत्राणां गोलमध्ये कक्षा शनेरूर्ध्वं द्वादशनवशताष्टनवतितत्त्वमिता। 

\begin{center}
{\qt भवेद्भकक्षा तीक्षणांशोर्भ्रमणं षष्टिताडितम्~।} 
\end{center}

इत्यनेन भकक्षाया द्वादशान्तरितत्वादयुक्तत्वं तथापि सैव यत्कल्पभगणैरित्यनेन सूर्यकक्षाया उक्त्या द्वादशाधोऽवयवस्य निबन्धने त्यागेऽपि भकक्षार्थं भगवता गृहीतत्वाददोषः। एतेनाधोऽवयवस्यार्धन्यूनत्वेन त्यागोऽर्धाभ्यधिकत्वेनोर्ध्वमेकाधिकग्रहणं कक्षानिबन्धेन कृतमिति सूचितम्~॥~८९~॥ 
ननु चन्द्रकक्षायाआगमप्रामाण्येनाङ्गीकारे सर्वकक्षाणामागमप्रामाण्यापत्त्या \textendash

\begin{center}
{\qt सैव यत्कल्पभगणैर्भक्ता तड्भ्रमणं भवेत्~।} 
\end{center}

इति कक्षानयनं व्यर्थम्। अन्यथाकाशकक्षाज्ञानासम्भवापत्तेरित्यत आकाशकक्षैवागमप्रामाण्येनाङ्गीकार्येति वसन्ततिलकयाह \textendash

%\vspace{2mm}

\begin{quote}
{\ssi खव्योमखत्रयखसागरषट्कनागव्योमाष्टशून्ययमरूपनगाष्टचन्द्राः~।}
%
\end{quote}

\newpage


\hspace{3cm} गूढार्थप्रकाशकेन सहितः~। \hfill ३४५
\vspace{1cm}
 
\begin{quote}
{\ssi ब्राह्माण्डसप्युटपरिभ्रमणं समन्तादभ्यन्तरे दिनकरस्य* करप्रसारः~॥~९०~॥}
%\vspace{2mm}
\end{quote}

 वेदाङ्गाष्टाशीतिनखभूसप्तधृतयः प्रयुतगुणिता योजनानिपूर्वार्धोक्तानि। ब्रह्माण्डसम्पुटपरिभ्रमणं ब्रह्माण्डगोलस्य परिधिः। कल्पभगणकक्षाहतित्वेनाकाशकक्षायाः र्पूवं स्यरूपोक्तेरिति न पौनरुक्त्यम्। अभ्यन्तरे ब्रह्माण्डगोलान्तः सूर्यस्याभितः किरणानां प्रसारः सूर्यकिरणप्रचारदेशस्य परिधिस्तत्तुल्यः। एतेन ब्रह्माण्डगोलान्तःपरिधिर्न बाह्य इति सूचितम्~॥~९०~॥\\
\noindent अथाग्रिग्मग्रन्थस्यासङ्गतित्वपरिहारार्थमध्यायसमाप्तिंफक्किकयाह। 


\begin{center}
 इति सूर्यसिद्ध्वान्ते भूगोलाध्यायः। 
\end{center}

 इति भिन्नच्छन्दसा प्रारब्धप्रसङ्गः समाप्त इत्यर्थः। पूर्वखण्डेग्रन्थैकदेशस्याधिकारसञ्ज्ञा कृता। उत्तरखण्डे ग्रन्थैकदेशस्याध्यायसञ्ज्ञा भिन्नप्रसङ्गवशात् कृतेति ध्येयम्। 

%\vspace{2mm}

\begin{quote}
{\qt रङ्गनाथेन रचिते सूर्यसिद्धान्तटिप्पणे~।\\
उत्तरार्धे समाप्तोऽयं भूगोलाध्यायसञ्ज्ञकः~॥ }
%\vspace{2mm}
\end{quote}
 इति श्रीसकलगणकसार्वभौमबल्लालदैवज्ञात्मजरङ्गनाथविरचिते गूढार्थप्रकाशक उत्तरखण्डे भूगोलाध्यायः पूर्णः। 



\noindent\rule{\linewidth}{.5pt}

\begin{center}
  * करप्रचारः। इति वा पाठः।
\end{center}
\newpage

\noindent ३४६ \hspace{4cm} सूर्यसिद्धान्तः
\vspace{1cm}


\begin{center}
 अथ पुनर्मुनीन् श्रोतृन् प्रति श्लोकाभ्यामाह \textendash
\end{center}

%\vspace{2mm}

\begin{quote}
{\ssi अथ गुप्ते शुचौ देशे स्नातः शुचिरलङ्कृतः~।\\
 सम्पूज्य भास्करं भक्त्या ग्रहान् भान्यथ गुह्यकान्~॥~१~॥

पारम्पर्योपदेशेन यथाज्ञानं गुरोर्मुखात्~।\\
आचार्यः शिष्यबोधार्थं सर्वें प्रत्यक्षदर्शिवान्~॥~२~॥}
%\vspace{2mm}
\end{quote}
 अथशब्दो मङ्गलार्थः। द्वितीयोऽथशब्दः पूर्वोक्तानन्तर्यार्थकः। गुप्ते रहसि शुचौ पवित्रे देशे स्थान आचार्यः सूर्याशपुरुषो मयासुराध्यापकः। स्नातः कृतस्नानः शुचिः शुद्धमनाः। अलङ्कृतो हस्तकर्णकण्ठदिभूषणभूषितः। निश्चिन्तत्वद्योतकमिदं विशेषणम्। अन्यथा गृहादिव्यवहारादिव्याकुलतया मनस्थैर्यानुपपत्तेः। भास्करं श्रीसूर्यं खोपजीव्यं भक्त्याराध्यत्वेन ज्ञानरूपया सम्पूज्य नमस्कारस्तुतिविषयं कृत्वाग्रहान् चन्द्रादिग्रहान् सूर्यस्य पृथगुद्देशः प्राधान्यज्ञानार्थम्।।भानि नक्षत्राणि राशींश्च गुह्यकान् यक्षादीन् क्षुद्रदेवताःसम्पूज्य। समुच्चयार्थकश्चोऽत्रानुसन्धेयः। गुरोः सूर्यस्य मुखाद्वदनारविन्दात्। पारम्पर्योपदेशेन सूर्येण मुनीन् प्रत्युक्तं मुनिभिः सूर्यांशपुरुषं प्रत्युक्तमिति परम्परया कथनेन। वस्तुतस्तु। शिय्यस्याग्रहोत्पादनार्थं ज्ञानेतिगोप्यत्वसूचनमेतदुक्त्याकृतम्। कथमन्यथा सूर्याज्ञप्तांशपुरुषो मयासुरं प्रत्यवदद्दूरस्थमुनीन् प्रति कथन उद्यतोऽर्कः स्वांशपुरुषं प्रति कथनेऽनुद्यतः कुतःकारणाभावाच। यथा स्वशक्त्या यादृशं ज्ञानं \textendash



\newpage


\hspace{3cm} गूढार्थप्रकाशकेन सहितः~। \hfill ३४७
\vspace{1cm}


\noindent पूर्वोक्तमवगतं शिय्यबोधार्थं मयासुरस्याभ्रमज्ञानोत्पादनार्थं सर्वं प्रागध्यायोक्तं प्रत्यक्षदर्शिवान् प्रत्यक्षं दर्शितवानित्यर्थः~॥~२~॥\\
\noindent कथं दर्शितवानिति मयासुरं प्रत्युक्तसूर्यांशपुरुषवचनस्यानुवादेसूर्यांशपुरुषो मयासुरं प्रति गोलबन्धोद्देशं तदुपक्रम च श्लोकाभ्यामाह \textendash

%\vspace{2mm}

  \begin{quote}
{\ssi भृभगोलस्य रचनां कुर्यादाश्चर्यकारिणीम्~।\\  
अभीष्टं पृथिवीगोलं कारयित्वा तु दारवम्~॥~३~॥

दण्डं तन्मध्यगं मेरोरुभयत्र विनिर्गतम्~।\\
आधारकक्षाद्वितयं कक्षा वैषुवती तथा~॥~४~॥ }
%\vspace{2mm}
\end{quote}
 भगोलस्य भूगोलादभितः संस्थितस्य नक्षत्राधिष्ठितगोलस्यप्रागध्यायोक्तार्थस्य रचनां स्थितिज्ञानार्थं दृष्टान्तात्मकगोलस्य निर्मितिं सुधीर्गणको गोलशिल्पज्ञः कुर्यात्। ननु त्वदुक्तेन सर्वंज्ञानं भवतीति दृष्टान्तगोलनिबन्धनं व्यर्थमेवेत्यत आह\textendash आश्चर्यकारिणीमिति~। उक्तप्रतीत्युद्भूताह तबुद्धिजनयित्रीं तथाचोक्तेनस्वाधस्तिर्यग्भागभार्लोकावस्थानस्य तद्भागस्यभगोलप्रदेशस्य च भूमेर्निराधारत्वादेश्च ज्ञानं मनसि सप्रतीतिकं न भवत्यतो दृष्टान्तगोले तन्निश्चयसम्भवात् तन्निबन्धनमावश्यकमिति भावः। कथं रचनां कुर्यादित्यत आह\textendash अभष्टिमिति~। भुवो गोलमभीष्टं स्वेच्छाकल्पितपरिधिप्रमाणकं दारवं काष्ठघटितं सच्छिद्रं कारयित्वा काष्ठभिल्पज्ञद्वारा कृत्वेत्यर्थः। मेरोरनुकल्परूपं दण्डकाष्ठं तन्मध्यगं तस्य काष्ठघटितभूगोलस्य मध्ये \textendash


\newpage


\noindent ३४८ \hspace{4cm} सूर्यसिद्धान्तः
\vspace{1cm}

%
\noindent छिद्रमध्ये शिथिलतया स्थितम्। उभयत्र भूगोलस्थव्यासप्रमाणच्छिद्रस्याग्राभ्यां बहिरित्यर्थः। विनिर्गतमेकाग्रादन्यतराग्रावशिष्टदण्डप्रदेशतुल्यं निःसृतम्। उभयाग्राभ्यां तुल्यौ दण्डप्रदेशौ यथा स्यातां तथा कुर्यादित्यर्थः। भगोलनिबन्धनार्थमाधारवृत्तद्वयमाह। आधारकक्षाद्वितयमिति। भगोलनिबन्धनार्थमादावाश्रयार्थं वृत्तयोर्द्वितव मूर्ध्वाधस्तिर्यगवस्थानक्रमेणैकमेकमेवं द्वयमित्यर्थः। भूगोलादुभयतस्तुल्यान्तरेणदण्डप्रदेशयोः प्रोतमेकं वृत्तं कुर्यात्। तत्तुल्यं वृत्तमपरं तदर्धच्छेदेन दण्डप्रोतं कुर्यादिति सिद्धोऽर्थः। एतद्वृत्तद्वयव्यतिरेकेण भूगोलादभितो भगोलनिबन्धनानुपपत्तेः। भगोलनिबन्धनारम्भमाह\textendash कक्षेति~। वैषुवती विषुवसम्बन्धिनो कक्षा वृत्तपरिधिर्विषुवद्वृत्तमित्यर्थः। तथाधारवृत्तद्वयस्यार्धच्छेदेन भगोलमध्यवृत्तानुकल्पेन गणकेन निबद्धमित्यर्थः~॥~४~॥ \\
\noindent अथ मेषादिद्वादशराशीनामहोरात्रवृत्तनिबन्धनमन्यदपि श्लोकपञ्चकेनाह \textendash 

%\vspace{2mm}

\begin{quote}
{\ssi भगणाशाङ्गुलैः कार्या दलितैस्तिस्र एव ताः~।\\
स्वाहोरात्रार्धकेर्णेश्च तत्प्रमाणानुमानतः~॥~५~॥

क्रान्तिविक्षेपभागैश्च दलितैर्दक्षिणोत्तरैः~।\\
स्वैः स्वैरपक्रमैस्तिस्रो मेषादीनामपक्रमात्~॥~६~॥

कक्षाः प्रकल्पयेत् ताश्च कर्कादीनां विपर्ययात्~।\\
तद्वत् तिस्रस्तुलादीनां मृगादीनां विलोमतः~॥~७~॥ 

याम्यगोलाश्रिताः कार्याः कक्षाधाराद्द्वयोरपि~।}
%
\end{quote}

\newpage

\hspace{3cm} गूढार्थप्रकाशकेन सहितः~। \hfill ३४९
\vspace{1cm}




\begin{quote}
{\ssi याम्योदग्गोलसंस्थानां भानामभिजितस्तथा~॥~८~॥ 
 
सप्तर्षीणामगस्त्यस्य ब्रह्मादीनां च कल्पयेत्~।\\
मध्ये वैषुवती कक्षा सर्वेषामेव संस्थिता~॥~९~॥ }
%\vspace{2mm}
\end{quote}
 भगणांशाङ्गुलैः द्वादशराशिभागैः षष्ट्यधिकशतत्रयपरिमिताङ्गुलैः दलितैः समविभागेन खण्डितैरङ्कितैरित्यर्थः। ताः कक्षाः वंशशलाकावृत्तात्मिकास्तिस्रः। त्रिसङ्ख्याकाः। एवकारादङ्कने वृत्ते च न्यूनाधिकव्यवच्छेदः। शिल्पज्ञेन गोलगणितज्ञेन कार्याः। एताः पूर्ववृत्तप्रमाणेन न कार्या इत्यभिप्रायेणाह\textendash स्वाहोरात्रार्धकर्णैरिति~। स्वशब्देन मेषादित्रिकं तस्य प्रतिराश्यहोरात्रवृत्तस्यार्धकर्णो व्यासार्धं द्युज्या ताभिरित्यर्थः। चकारात कार्याः। स्वस्वद्यज्यमितेन व्यासार्धेन मेषादित्रयाणां वृत्तत्रयं कुर्यादित्यर्थः। ननु स्पष्टाधिकारोक्ताहोरात्रार्धकर्णानयने युक्त्यभावात् तैर्वृत्तनिर्माणं कुतः कार्यभित्यत आह\textendash तत्प्रमाणानुमानत इति~। विषुवत्कक्षाप्रमाणानुमानाद्वृत्तत्रयं कार्यम्। यथा विषुवद्वृत्तं पूर्ववृत्तसमम्। तथा तदनुरोधेन मेषान्तवृत्तमल्पं तदनुरोधेन वृषान्तवृत्तमल्पं तदनुरोधेन मिथुनान्तवृत्तमल्पमित्युत्तरोत्तरमल्पव्यासार्धवृत्तम्। तत्त्वहोरात्रवृत्तमिति द्युज्याव्याक्षार्धेन वृत्तनिर्माणं यक्तियुक्तं क्रान्तिज्यावर्गोनात्त्रिज्यावर्गान्मलस्याहोरात्रवृत्तव्यासार्धत्वादिति भावः। वृत्तत्रयं सिद्धं कृत्वा दृष्टान्तगोलेनिबघ्नाति। क्रान्तिविक्षेपभागैरिति। क्रान्तिवृत्तस्य विषुवहत्त \textendash



\newpage


\noindent ३५० \hspace{4cm} सूर्यसिद्धान्तः
\vspace{1cm}


\noindent प्रदेशाद्विक्षिप्तप्रदेशा यैरंशैः। चकारादाधारवृत्तस्थैर्दलितैः समविभागेन खण्डितैरङ्कितैः। दक्षिणोत्तरैर्विषुवद्वृत्तक्रान्तिवृत्तप्रदेशयोर्दक्षिणोत्तरान्तरात्मकैरुक्तलक्षणैः स्वकीयैः स्वराशिसम्बद्धैरपक्रमैः स्पष्टाधिकारानीतक्रान्त्यंशैर्मेषादीनां मेषादिराशित्रयान्तानां मेषान्तवृषान्तमिथुनान्तानामित्यर्थः। तिस्रस्त्रिसङ्ख्याकाः प्राङ्गिर्मिता वृत्तरूपाः कक्षाः। अपक्रमात्।अपशब्दस्योपवर्गत्वात् क्रमादित्यर्थः। प्रकल्पयेत्। शिल्पज्ञगणको विषुवद्वृत्तानुरोधेनाधारवृत्तद्वय उत्तरतो निबन्धयेदित्यर्थः। कर्कादीनामाह\textendash ता इति~। मेषादिकक्षा निबद्धाः कर्कादीनां कर्कसिंहकन्यानामादिप्रदेशानां विपर्ययात् व्यत्यासात्। चकारः समच्चये। तेन प्रकल्पयेदित्यर्थः। मिथुनान्तवृत्तं कर्कादेर्वृषान्तवृत्तं सिंहादेर्मेषान्तवृत्तं कन्यादेरिति फलितम्। तुलादीनामाह\textendash तद्वदिति~। तुलादीनां तुलावृश्चिकधन्विनां तिस्रः। अन्यास्त्रिसङ्ख्याकाः कक्षास्तद्वदेकद्वित्रिराशिक्रान्त्यंशैस्तुलान्तवृश्चिकान्तधनुरन्तानां याम्यगोलाश्रिताः। विषुवद्वृत्ताद्दक्षिणभाग आधारवृत्तद्वये निबद्धाः कार्याः। गणकेनेति शेषः। मकरादीनामाह\textendash मृगादीनामिति~। विलोमत उत्क्रमात् तुलादिसम्बद्धाः कक्षा मकरादीनां भवन्ति। धनुरन्तवृत्तं मकरादेर्वृश्चिकान्तवृत्तं कुम्भादेस्तुलान्तवृत्तं मीनादेरिति फलितम्। ताराणां कक्षानिबन्धनमाह\textendash कक्षाधारादिति~। भानामश्विन्यादिसप्तविंशतिनक्षत्रबिम्बानां याम्योदग्गोलसंस्थानां विषुवद्वृत्ता दक्षिणोत्तरभा \textendash



\newpage


\hspace{3cm} गूढार्थप्रकाशकेन सहितः~। \hfill ३५१
\vspace{1cm}


\noindent गयोर्यथायोग्यमवस्थितानां यन्नक्षत्रध्रुवकस्पष्टक्रान्तिरुत्तरा तन्नक्षत्राणामुत्तरभागावस्थितानां येषां स्पष्टक्रान्तिर्दक्षिणा तेषां दक्षिणभागावस्थितानामित्यर्थः। द्वयोर्दक्षिणोत्तरभागयोः। अपिशब्दो याम्योत्तरनक्षत्रक्रमेण व्यवस्थार्थकः। कक्षाधारात् कक्षाणामाधारवृत्तद्वयात् तयोरित्यर्थः। सप्तम्यर्थे पञ्चमी। कक्षाः स्वस्पष्टक्रान्तिज्योत्पन्नद्युज्याव्यासार्धप्रमाणेन वृत्ताकाराः प्रकल्पयेत्। शिल्पज्ञो निबन्धयेत्। अन्येषामप्याह\textendash अभिजित इति~। अभिजिन्नक्षत्रबिम्बस्य सप्तविंशतिबिम्बानामगस्त्यनक्षत्रबिम्बस्य ब्रह्मसञ्ज्ञकताराद्युक्तलब्धकापांवत्सादिनक्षत्रबिम्बानां चकारौऽनसन्धेयः। तथा कक्षा यथायोग्यं प्रकल्पयेदित्यर्थः। निबन्धनप्रकारमुपसंहरति\textendash मध्य इति~। सर्वासामुक्तकक्षाणां मध्ये तुल्यभागेऽनाधारवृत्तमध्यप्रदेशे। एवकारादन्ययोगव्यवच्छेदः। वैषुवतो कक्षा विषुवसम्बन्धिनी वृत्तरूपा संस्थितावस्थिता भवति। तथा शिल्पकः कक्षां निबन्धयेदित्यर्थः। विषुवद्धृत्तात् स्वस्पष्टक्रान्त्यन्तरेणस्वद्युज्याव्यासार्धप्रमाणेनाहोरात्रःवृत्तमाधारवृत्तयोर्निबन्धयेदिति निष्कृष्टोऽर्थः~॥~९~॥\\ 
\noindent अथ गोले मेषादिराशिसन्निवेशं सार्धश्लोकेनाह \textendash

%\vspace{2mm}

 \begin{quote}
{\ssi तदाधारयुतेरूर्ध्वमयने विषुवद्द्वयम्~।\\
विषुवत्स्थानतो भागैः स्फुटेर्भगणसञ्चरात्~॥~१०~॥

क्षेत्रण्येवमजादीनां तिर्यग्ज्याभिः प्रकल्पयेत्~। }
%\vspace{2mm}
\end{quote}
 तदाधारयुतेस्तद्विषुवद्वृत्तमाधारमाधारवृत्तं तयोय तेः
%


\newpage


\noindent ३५२ \hspace{4cm} सूर्यसिद्धान्तः 
\vspace{1cm}


\noindent सम्पातादूर्ध्वमुपरि। अन्तिमाहोरावाधारवृत्तयोः सम्पातेऽयनेदक्षिणोत्तरायणसन्धिस्थाने भवतः। अत्रोर्धपदसञ्चारादाधारवृत्तमकर्ध्वाधरं ग्राह्यं न तिर्यगुन्मण्डलाकारम्। तेनैतत्फलितम्। विषुवद्वृत्तस्योर्ध्वाधराधारवृत्त ऊर्ध्वंमधश्च सम्पातस्तुत्रोर्ध्वसम्पातान्मकराद्यहोरत्रिवृत्तंचतुर्विंशत्यंशैस्तदाधारवृत्ते दक्षिणतो यत्र लग्नं तत्रोत्तरायणसन्धिस्थानम्। एवमधःसम्पातात् कर्काद्यहोरात्रवृत्तं चतुर्विंशत्यंशैस्तदाधारवृत्तउत्तरतो यत्र लग्नं तत्र दक्षिणायनसन्धिस्थानमिति। अयनाद्विषुवस्य विपरतिस्थितत्वादूर्ध्वशब्दद्योतितविपरोताधःशब्सम्बन्धाद्विषुवद्वयं भवति। तात्पर्यार्थस्तु तिर्यगुन्मण्डलाकाराधारवृत्तविषुवद्वृत्तसम्पातौ पूर्वापरो क्रमेण मेषादितुलादिरूपौ विषवत्स्याने भवत इति। अथ राशिसाकसल्यसन्निवेशमाह\textendash विषुवत्स्थानत इति~। विषुवप्रदेशात स्फटै राशिसम्बन्धिभिस्त्रिंशन्मितैरंशैर्भगणसञ्चराद्राशिसाकल्यसन्निवेशात् तिर्यग्ज्याभिरुक्तवृत्तानुकारातिरिक्तानुकारसूत्रवृत्तप्रदेशैरजादीनां मेषादीनाम्। एवमयनविषुवकल्पनरीत्या तदन्तराले क्षेत्राणि स्थानानि सुधीर्गणकः प्रकल्पयेदङ्कयेत्। तद्यथा पूर्वदिक्स्यविषुवस्थानाद्गोलवृत्तद्वादशांशखण्डप्रदेशेन मेषान्ताहोरात्रवृत्तेपूर्वभागे यत्र स्थानं तत्र मेषान्तस्थानं तस्मात् तदन्तरेण वृषान्ताहोरात्रवृत्ते तदन्तरेण वृषान्तस्थानमस्मादयनसन्धिस्थानंतत्प्रदेशान्तरेण मिथुनान्तस्यानमस्मात् पश्चिमभागे कर्कान्ताहोरात्रवृत्ते तदन्तरेण कर्कान्तस्थानमस्मादपि सिंहान्ताहो \textendash



\newpage


\hspace{3cm} गूढार्थप्रकाशकेन सहितः~। \hfill ३५३
\vspace{1cm}


\noindent रात्रवृत्ते तदन्तरेण सिंहान्तस्थानमस्मादपि तदन्तरेण पश्चिमविषुवस्थानं कन्यान्तस्यानमस्मादपि पूर्वभागे तुलान्ताहोरात्रवृत्ते तदन्तरेण तुलान्तस्यानमस्मादपि वृश्चिकान्ताहोरात्रवृत्ते तदन्तरेण वृश्चिकान्तस्थानमस्मादपि तदन्तरेणायनमन्धिस्थानं धनुरन्तस्थानमस्मात् कुम्भाद्यहोरात्रवृत्ते तदन्तरेण मकरान्तस्थानभस्मादपि मीनाद्यहोरात्रवृत्ते तदन्तरेण कुम्भान्तस्थानं मीनादिस्थानं च। अस्मादपि पूर्वविषुवेमीनान्तस्थानं मेषादिस्थानं च तदन्तरेणेति व्यक्तम्~॥~१०~॥\\ 
\noindent ननुगोले वृत्ते द्वादशराशीनां सत्त्वादन्यथा चक्रकलानुपपत्तेरित्यत्रैकवृत्ताभावात् कथं राश्यङ्कनं राशिविभागानुपपत्तिश्च। अन्तरालभागस्याकाशात्मकत्वादित्यतोवृत्तकथनच्छलेन पूर्वोक्तंस्पष्टयन् सूर्यस्तदृत्ते भगणभोगं करोतीत्याह \textendash

%\vspace{2mm}

\begin{quote}
{\ssi अयनादयनं चैव कक्षा तिर्यक् तथापरा~॥~११~॥
 
क्रान्तिसञ्ज्ञा तया सूर्यः सदा पर्येति भासयन्~। }
\end{quote}
%\vspace{2mm}

 अयनस्थानमारभ्य परिवर्तन तदयनसाआनपर्यन्तं चकारआरम्भसमाप्योर्भिन्नायनस्थाननिरासार्थकः। अपरा गोलआधारवृत्तसमा वृत्तरूपा कक्षा तथा राश्यङ्कमार्गेण। एवकारोऽन्यमार्गव्यवच्छेदार्थकः। तिर्यक्। उक्तवृत्तानुकारविलक्षणानुकारा क्रान्तिंसञ्ज्ञा क्रमणं क्रान्तिः। ग्रहगमनभोगज्ञानार्थं वृत्तं तत्सञ्ज्ञमुपकल्पितम्। अयनविषुवद्वयसंसक्तंक्रान्तिवृत्तं द्वादशराश्यङ्कितं गोले निंबन्धयेदिति तात्पर्यार्थः।
%


\newpage


\noindent ३५४ \hspace{4cm} सूर्यसिद्धान्तः
\vspace{1cm}


\noindent भासयन् भुवनामि प्रकाशयन् सन् स सूर्यः। एतेन चन्द्रादीनां निरासः। सदा निरन्तरं तथा क्रान्तिसञ्ज्ञया कक्षया पर्येति स्वशक्त्या गच्छन् भगणपरिपूर्तिभोगं करोति। सूर्यगत्यनुरोधेन नियतं क्रान्तिवृत्तं कल्पितमिति भावः~॥~११~॥\\
\noindent ननु चन्द्राद्याः क्रान्तिमृत्ते कुतो न गच्छन्तीत्यत आह \textendash

%\vspace{2mm}

\begin{quote}
{\ssi चन्द्राद्याश्च स्वकैः पातैरपमण्डलमाश्रितैः~॥~१२~॥ 
 
ततेऽपकृष्टा दृश्यन्ते विक्षेपान्तेष्वपक्रमात्~। }
%\vspace{2mm}
\end{quote}

 चन्द्रादयोऽर्कव्यतिरिक्ता ग्रहाः स्वकैः स्वीयैः पातै  पाताख्यदैवतैरपमण्डलं क्रान्तिवृत्तमाश्रितैः स्वस्वभोगस्थानेऽधिष्ठितैस्ततः क्रान्तिवृत्तान्तर्गतग्रहभोगस्थानादित्यर्थः। चकाराद्विक्षेपान्तरेणापकृष्टा दक्षिणत उत्तरतो वा कर्षिता भवन्ति। अतः कारणादपक्रमात् क्रान्तिवृत्तान्तर्गतस्वभोगस्थानादित्यर्थः। दक्षिणत उत्तरतो वा विक्षेपान्तेषु गणितागतविक्षेपकलाग्रस्थानेषु भूस्थजनैर्दृश्यन्ते। तथाच क्रान्तिनृत्तं यथा विषुवन्मण्डलेऽवस्यित तथा क्रान्तिवृत्ते पातस्थाने तत्षड्भान्तरस्थाने च लग्नमुक्तपरमविक्षेपकलाभिस्तत्त्रिभान्तरस्थानादूर्ध्वाधःक्रेणदक्षिणोत्तरतो लग्नं च वृत्तं विक्षेपवृत्तं चन्द्रादिगत्यनुरोधेन स्वं स्वं भिन्नं कल्पितं तत्र गच्छन्तीति भावः~॥~१२~॥ \\
\noindent अथ त्रिप्रश्नाधिकारोक्तलग्नमध्यलग्नयोः स्वरूपमाह \textendash 

%\vspace{2mm}

 \begin{quote}
{\ssi उदयक्षितिजे लग्नमस्तं गच्छच्च तद्वशात्~॥~१३~॥
 
लङ्कोदयैर्यथा सिद्धं खमध्योपरि मध्यमम्~।}
%
\end{quote}

\newpage


 \hspace{3cm} गढार्थप्रकाशकेन सहितः~। \hfill ३५५ 
\vspace{1cm}


 उदयक्षितिजे क्षितिजवृत्त पूर्वदिग्देश इत्यर्थः। लग्नंक्रान्तिवृत्तं यत्प्रदेशे प्रवहवायुना संसक्तं तत्प्रदेशो मेषाद्यवधिभोगेनोदयलग्नमुच्यत इत्यर्थः। प्रसङ्गादस्तलग्नखरूपमाह\textendash अस्तमिति~। तद्वशादुदयलग्नानुरोधादस्तुमस्तक्षितिजं क्षितिजवृत्तस्य पञ्चिमदिक्प्रदेज्ञमित्यर्थः। क्रान्तिवृत्तं गच्छतु।यत्प्रदेशेन प्रवहवायुना सल्लग्नं तत्प्रदेशो मेषाद्यवधिभोगेनास्तलग्नमुच्यत इत्यर्थः। तथाच चितिजोर्ध्वं सदा क्रान्तिनृत्तस्य सद्भावादुदयास्तलग्नयोः षड्राश्यन्तरं सिद्धं लङ्कोदयैर्निरक्षदेशीयराश्युदयासुभिः। यथात्रिप्रश्नाधिकारोक्तप्रकारेण यत्सङ्ख्यामितं सिद्धं निष्पन्नम्। मध्यमं मध्यमलग्नं तत् खमध्योपरि खस्यश्याकाशविभागस्य मध्यं मध्यगतदक्षिणोत्तरसूत्रवृत्तानुकारप्रदेशरूपं नतु स्वमध्यं भास्कराचार्याभिमतं खस्वस्तिकं तल्लग्नस्य कादाचित्कत्वेन सदानुत्पत्ते  । तस्योपरि स्थितं कान्तिवृत्तं याम्योत्तरवृत्ते यत्प्रदेशेन लग्नं तत्प्रदेशो मेषाद्यवधिभोगेन मध्यलग्नमुच्यत इति तात्पर्यार्थः~॥~१३~॥ \\
\noindent अथ त्रिप्रश्नाधिकारोक्तान्त्यायाः स्वरूपं स्पष्टाधिकारोक्तचरज्यायाःस्वरूपं चाह \textendash

%\vspace{2mm}

\begin{quote}
{\ssi मध्यक्षितिजयोर्मध्ये या ज्या सान्त्याभिधीयते~॥~१४~॥ 
 
ज्ञेया चरदलज्या च विषवत्क्षितिजान्तरम्~। }
%\vspace{2mm}
\end{quote}
 या उत्तरगोले त्रिज्याचरज्यायुतिरूपा दक्षिणगोले चरज्योनत्रिज्यारूपा त्रिप्रश्नाधिकारोक्ता। अन्त्या सा मध्यं या\textendash

{\tiny{2 x 2}}

\newpage



\noindent ३५६ \hspace{4cm} सूर्यसिद्धान्तः
\vspace{1cm}


\noindent म्योत्तरवृत्तं क्षितिजं स्वाभिमतदेशक्षितिजवृत्तं तयोर्मध्येऽन्तरालेऽहोरात्रवृत्तस्यैकदेशप्रदेशे ज्या। उदयास्तसूत्रयाम्योत्तरसूत्रसम्पातादहोरात्रयाम्योत्तरवृत्तसम्पातावधिसूत्ररूपा ज्यासूत्रामकारा न तु ज्या। अहोरात्रक्षितिजवृत्तसम्पातद्वयबद्धोदयास्तसूत्रस्याहोरात्रवृत्तव्याससूत्रत्वाभावात्। अतएवोत्तरगोलेऽन्त्या त्रिज्याधिका सङ्गच्छते। अभिधीयते गोलज्ञैःकथ्यते। नन्वन्त्योपजीव्यचरज्यैव किंस्वरूपा यया तत्सिद्धिरित्यत आह\textendash  ज्ञेयेति~। 

\begin{center}
{\qt उन्मण्डलं च विषुवन्मण्डलं परिकीर्त्यते~।} 
\end{center}

 इति त्रिप्रश्नाधिकारोक्तेग द्वयोः शब्दयोरेकार्थवाचकत्वात्तिर्यगाधारवृत्तानुकारं स्थिरं निरक्षत्तितिजवृत्तमुन्मण्डलं क्षितिजं स्वाभिमतदेशक्षितिजवृत्तमनयोरन्तरम्। चकारो विशेषार्थकस्तुकारपरस्तेन तदन्तरालस्यिताहोरात्रवृत्तैकदेशस्यार्धज्यारूपमृजुसूत्रमन्तरविशेषात्मकम्। तथाच स्वनिरक्षदेशस्वदेशयोरुदयास्तसूत्रयोरन्तरमूर्ध्वाधरमिति फलितार्थः। चरदलज्या तदन्तरालस्थिताहोरात्रवृन्तैकदेशरूपचराख्यखण्डकस्य। न तु दलमर्धम्। ज्या चरज्येत्यर्थः। गोलज्ञैर्ज्ञातव्या~॥~१४~॥\\
\noindent ननु पूर्वश्लोकद्वयोक्तं क्षितिजस्याज्ञानाद्दुर्बोधमित्यतः श्लोकार्धेनक्षितिजस्वरूपमाह \textendash

%\vspace{2mm}

  \begin{quote}
{\ssi कृत्वोपरि स्वकं स्थानं मध्ये क्षितिजमण्डलम्~॥~ १५~॥ }
% \vspace{2mm}
\end{quote}
 भूगोले स्वकं स्वीयं स्यानं भूप्रदेशैकदेशरूपमपरि सर्वप्रदेशेभ्य ऊर्ध्वं कृत्वा प्रकल्प्य मध्ये तादृशभूगोल ऊर्ध्वाधःख \textendash



\newpage


\hspace{3cm} गूढार्थप्रकाशकेन सहितः~। \hfill ३५७
\vspace{1cm}


\noindent ण्डसन्धौ यद्वृत्तं तत् क्षितिजवृत्तं तदनुरोधेन दृष्टान्तगोलेक्षितिजवृत्तं स्थिरं संसक्तं कार्यमिति भावः~॥~१५~॥\\
\noindent अथैनंदृष्टान्तगोलं सिद्धं कृत्वास्य स्वत एव पश्चिमभ्रमो यथा भवतितथा प्रकारमाह \textendash

%\vspace{2mm}

 \begin{quote}
{\ssi यस्त्रच्छन्नं बहिश्चापि लोकालोकेन वेष्टितम्~।\\
अमृतस्रावयोगेन कालभ्रमणसाधनम्~॥~१६~॥}
%\vspace{2mm}
\end{quote}
 बहिः। गोलोपरीत्यर्थः। गोलाकारेण वस्त्रेण छन्नं छादितं दृष्टान्तगोलम्। चकारा वस्त्रोपरि तत्तद्वृत्तानामङ्कनं कार्थम्। लोकालोकेन वेष्टितं दृश्यादृश्यसन्धिस्थवृत्तेन क्षितिजाख्येन संसक्तम्। अपिः समुच्चये। एतेन क्षितिक्ष वस्त्रच्छन्नं न कार्यं किं तु वस्त्रोपरि क्षितिज गोलसंसक्तं केनापि प्रकारेणस्थिरं यथा भवति तथा कार्यमिति तात्पर्यम्। अमृतस्रावयोगेनैतादृशं गोलं कृत्वा जलप्रवाहाधोघातेन कालभ्रमणसाधनं षष्टिनाक्षत्रघटीभिर्दृष्टान्तगोलस्य भ्रमणं यथा भवति तथा साधनं कारणं कार्यं स्वयंवहगोलयन्त्त्रं कार्यमित्यर्थः।एतदुक्तं भवति। दूयान्तगोलं वस्त्रच्छन्नं कृत्वा तदाधारयष्ट्यग्रे दक्षिणोत्तरभित्तिक्षिप्तनसिकयोः क्षेप्ये। यथा यष्ट्यग्रं ध्रुवाभिमुखं स्यात्। ततो यष्ट्यग्रर्जमार्गगतजलप्रवाहेण पूर्वाभिमुखेन तस्याधः पश्वाद्भागे घातोऽपि यथा स्यात् तथास्यादर्शनार्थमेव वस्त्रच्छन्नमुक्तम। अन्यथा गोलवृत्तान्तरवकाशमार्गेण चलाघातदर्शनभ्रमेण चमत्कारानुत्पत्तेः। आकाशाका \textendash



\newpage


\noindent ३५८ \hspace{4cm} सूर्यसिद्धान्तः
\vspace{1cm}


\noindent रतासम्पादनार्थमपि वस्त्रच्छन्नमुक्तम्। इदं वस्त्रमार्द्रं यथा न भवति तथा चिक्वणंवस्तुना मदनादिना लिप्तं कार्यम्।क्षितिजवृत्ताकारेणाधो गोलो दृश्यो यथा स्यात् तथा परिखारूपा भित्तिः कार्या। परन्तु दक्षिणयष्टिभागस्तत्र शिथिलो यथा भवति। अन्यथा भ्रमणानपपत्तेः। पर्वदिकस्थपरिखाविभागाद्वहिर्जलप्रवाहोऽदृश्यः कार्य इत्यादि स्वबुध्यैवज्ञेयमिति~॥~१६~॥ \\
\noindent अथ यदि जलप्रवाहस्तत्र न सम्भवतितदा कथं स्वयंवहो दृष्टान्तगोलो भवतीत्यतस्तत्खयंवहार्थमुक्तंच गोप्यं कार्थमित्याह \textendash

%\vspace{2mm}

 \begin{quote}
{\ssi तुङ्गबीजसमायुक्तं गोलयन्त्त्रं प्रसाधयेत्~।\\
गोप्यमेतत् प्रकाशोक्तं सर्वगम्यं भवेदिह~॥~१७~॥ }
%\vspace{2mm}
\end{quote}
 दृष्टान्तगोलरूपं यन्त्त्रं तुङ्गबीजसमायुक्तं तुङ्गो महादेवस्तस्य बीजं वीर्यं पारद इत्यर्थः। तेन योजितं सत् प्रसाधयेत्।गणकः शिल्पज्ञः। प्रकर्षेण यथा नाक्षत्रधष्टिघटीभिर्गोलभ्रमस्तथा पारदप्रयोगेण सिद्धं कुर्यादित्यर्थः। एतदुक्तं भवति ।निबद्धगोलवहिर्भृतयष्टिप्रान्तयोर्यथेच्छया स्थानद्वयेस्थानत्रये वा नेमिं परिधिरूपामुत्कोर्य तां तालपत्रादिना चिक्वणवस्तुलेपेनाच्छाद्य तत्र छिद्रं कृत्वा तन्मार्गेण पारदोऽर्धपरिधौपूर्णो देय इतरार्धपरिधौ जलं च देयं ततो मुद्रितच्छिद्रं कृत्वा यष्ट्यग्रे भित्तिस्थनलिकथोः क्षेणे यथा गोलोऽन्तरिक्षोभवति। ततः पारदजलाकर्षितयष्टि स्वयं भमति। तदा \textendash



\newpage


\hspace{3cm} गूढार्थप्रकाशकेन सहितः~। \hfill ३५९
\vspace{1cm}


\noindent श्रितो गोलश्च। एतत्पक्षे वस्त्रच्छन्नमाकाशाकारतासम्पादनार्थमेव चेत् क्रियत इति। नन्वियं स्वयंवहक्रिया व्यक्तानोक्तेत्यत आह\textendash गोप्यमिति~। एतत् स्वयंवहकरणं गोप्यमप्रकाश्यं कुत इत्यत आह\textendash प्रकाशोक्तमिति~। अतिव्यक्ततयोक्तंस्वयंवहकरणमिह भूलोके सर्वगम्यं सर्वजनगम्यं भवेत्। तथा च सर्वज्ञेये वस्तुनि चमत्कारानुत्पत्तेश्चमत्कृत्यर्थं सर्वत्र न प्रकाश्यमित्याशयेन तत्करणं व्यक्तं नोक्तमिति भावः~॥~१७~॥\\
\noindent ननु त्वया गोप्यत्वेनोक्तं मया कथमवगन्तव्यं मादृशैरन्यैश्चकथमवगन्तव्यमित्यतः सार्धश्लोकेनाह\textendash

%\vspace{2mm}

\begin{quote}
{\ssi तस्माद्गुरूपदेशेन रचयेद्गोलमुत्तमम्~।\\
युगे युगे समुच्छिन्ना रचनेयं विवस्वतः~॥~१८~॥ 

प्रसादात् कस्यचिद्भयः प्रादुर्भवति कामतः~। }
%\vspace{2mm}
\end{quote}
 तस्मात् स्वयंवहकरणस्य गोप्यत्वात् गुरूपदेशेन परम्पराप्राप्तगुरोर्निर्व्याजकथनेन गोलं दृष्टान्तगोलमुत्तमं स्वयंवहात्मकं- गणकः कुर्यात्। तथाच मया तुभ्यमुक्ता अन्ये गोप्यत्वेनातिव्यक्ता नोक्तेति भावः। अन्यैः कथं ज्ञेयमिदमित्यत आह\textendash युग इत्यादि~। विवस्वतः सूर्यमण्डलाधिष्ठातुर्जीवविशेषस्येयंस्वयंवहरूपा रचना क्रिया युगे युगे वह्नकाल इत्यर्थः। समुच्छिन्ना लोके लुप्ता कस्यचित् मादृशस्य प्रसादादनुग्रहाद्भूयः वारंवारमिच्छया प्रादुर्भवति व्यक्ता भवतीत्यर्थः। तथाच यथा मत्तस्त्वयावगतं तथान्यस्मान्मादृशादन्यैरवगन्तव्यं \textendash



\newpage


\noindent ३६० \hspace{4cm} सूर्यसिद्धान्तः
\vspace{1cm}


\noindent कालस्य निरवधित्वात् सृवृ रेनादित्वाच्चेति भावः~॥~१८~॥ \\
\noindent अथोक्तस्वयंवहक्रियारीत्या स्वयंवहगोलातिरिक्तान्यस्वयंवहयन्त्त्राणिकालज्ञानार्थं साध्यानि तत्साधनं रहसि कार्यमिति आह \textendash

%\vspace{2mm}

 \begin{quote}
{\ssi कालसंसाधनार्थाय तथा यन्त्त्राणि साधयेत्~॥~१९~॥
 
एकाकी योजयेद्वीजं यन्त्रे विस्मयकारिणि~। }
%\vspace{2mm}
\end{quote}
 तथा यथा स्वयंवहगोलयन्त्त्रं साधितं तद्वदित्यर्थः। कालसंसाधनार्थाय कालस्य दिनगतादेः सूक्ष्मज्ञाननिमित्तं यन्त्त्राणि स्वयंवहगोलातिरिक्तानि स्वयंवहयन्त्त्राणि साधयेत्। गणकः शिल्यादिस्वकौशल्येन कारयेत्। यन्त्त्रे कालसाधके विस्मयकारिणि स्वयंवहरूपतया लोकानामुत्पन्नाश्चर्यस्य कारणभूते बीजं स्वयंवहतासम्पादकं कारणमेकाकी एकव्यक्तिकोऽद्वितीयःसन् योजयेत्। शिल्पज्ञतया स्वयमेव निष्पादयेदित्यर्थः। अन्यथा द्वितीयस्य तञ्ज्यतया तन्मुखात् तद्यन्त्त्रहार्द्दस्यलोकत्रयेण गोचरतायां कदाचित् सम्भावितायां विस्मयानुत्पत्तेः~॥~१९~॥ \\
\noindent अथैषां स्वयंवहयन्त्त्राणां दुर्घटत्वाच्छङ्कादियन्त्त्रेः कालज्ञानं ज्ञेयमित्याह \textendash

%\vspace{2mm}

  \begin{quote}
{\ssi शङ्कुयष्टिधनुश्चक्रैच्छायायन्त्रैरनेकधा~॥~२०~॥ 
  
गुरूपदेशाद्विज्ञेयं कालज्ञानमतन्द्रितैः। }
%\vspace{2mm}
\end{quote}
 शङ्कुयष्टिधनुश्चक्रैः प्रश्विद्धैच्छायायन्त्त्रेच्छायासाधकयन्त्त्रैरनेकधा नानाविधगणितप्रकारैर्गुरूपदेशात स्वाध्यापकस्य नि \textendash



\newpage


\hspace{3cm} गूढार्थप्रकाशकेन सहितः~। \hfill ३६१
\vspace{1cm}


\noindent र्व्याजकथनादतन्द्रितैरभ्रमैः पुरुषैः कालज्ञानं दिनमतादिज्ञानंविज्ञेयं सूक्ष्मत्वेनावगम्यम। एतत् सर्वं सिद्धान्तशिरोमणौ भास्कराचार्यैः स्पष्टीकृतम्। तत्र शङ्कुस्वरूपम् \textendash

%\vspace{2mm}
%{\setlength{\parindent}{6em}
 \begin{quote}
{\ssi समतलमस्तकपरिधि
  र्भमसिद्धो दन्तिदन्तजः शङ्कुः~।
 
तच्छायातः प्रोक्तं
ज्ञानं दिग्देशकालानाम्~॥ }
%\vspace{2mm}
\end{quote}
\noindent इति। यष्टियन्त्त्रं च। 

%\vspace{2mm}
%{\setlength{\parindent}{6em}
\begin{quote}
{\qt त्रिज्याविय्कम्भार्धं 
वृत्तं कृत्वा दिगङ्कितं तत्र~।\\
दत्वाग्रां प्राक् पश्चात्
द्युज्य वृत्तं च तन्मध्ये~॥

तत् परिधौ षष्ट्यङ्के
यष्टिर्नष्टद्युतिस्ततः केन्द्रे~।\\
त्रिज्याङ्गुला निधेया
यष्ट्यग्राग्रान्तरं यावत्~॥

तावत्या मौर्व्या य
द्द्वितीयवृत्ते धनुर्भवेत् तत्र~।\\
दिनगतशेषा नाड्यः
प्राक् पश्चात् स्युः क्रमेणैवम्~॥ }
\end{quote}
\noindent इति। चक्रयन्त्त्रं तु। 

%\vspace{2mm}
%{\setlength{\parindent}{6em}
\begin{center}
{\qt चक्रं चक्रांशाङ्कं }
\end{center}
%


\newpage


\noindent ३६२ \hspace{4cm} सूर्यसिद्धान्तः
\vspace{1cm}

%{\setlength{\parindent}{6em}
\begin{quote}
 {\qt परिधौ श्लथश्टङ्खलादिकाधारम्~।\\
धात्री त्रिम आधारात
कल्प्या भार्धेऽत्र खार्धं च~॥

तन्मध्ये सूक्ष्माक्षं
क्षिप्तार्काभिमुखनेमिकं धार्यम्~।\\
भूमरुन्नतभागा
स्तत्राक्षच्छायया भुक्ताः~॥

तत्खार्धान्तश्च नता
उन्नतलवसङ्गुणं द्युदलम्~।\\
द्यदलोन्नतांशभक्तं
नाड्यः स्थूलाः परैः प्रोक्ताः~॥ }
\end{quote}

\noindent इति। धनुर्यन्त्त्रं तु। 

\vspace{2mm}
{\setlength{\parindent}{6em}
  दलीकृतं चक्रमशन्ति चापम्। }
\vspace{2mm}

\noindent इति। अथ ग्रन्थविस्तरभयादेतेषां निरूपणविस्तरो गणितादिविचारवो पेक्षित इति मन्तव्यम्~॥~२०~॥  \\
 \noindent अथ घटीयन्त्त्रादिभिश्चमत्कारियन्त्त्रैर्वा सर्वोपजीव्यं कालेसूक्ष्मं साधयेदिति कालसाधनमुपसंहरति \textendash

%\vspace{2mm}

 \begin{quote}
{\ssi तोययन्त्त्रकपालाद्यैर्मयूरनरवानरैः~। \\
ससूत्ररेणुगर्भैश्च सम्यक् कालं प्रसाधयत्~॥~२१~॥}
%\vspace{2mm}
\end{quote}
 जलयन्त्त्रं च तत् कपालं च कपालाख्यं जलयन्त्त्रं वक्ष्यमाणांतदाद्यं प्रथमं येषां तैर्यन्त्त्रैर्वालकायन्त्त्रप्रभृतिभिः सापेक्षघटी 



\newpage


\hspace{3cm} गूढार्थप्रकाशकेन सहितः~।  \hfill ३६३ 
\vspace{1cm}


\noindent यन्त्त्रेर्भयूरनरवानरैः। मयूराख्यं स्वयंवहयन्त्त्रंनिरपेक्षं नरयन्त्त्रं शङ्काख्यं छायायन्त्रं पूर्वोद्दिष्टं वानरयन्त्त्रं स्वयंवहं निरपेक्षमेतैः ससूत्ररेणुगर्भैः सूत्रसहिता रेणवो धूलयो गर्भे मध्येयेषां तैः सूत्रप्रोता षष्टिसङ्ख्याका मृद्वटिका मयूरोदरस्थामुखाद् घटिकान्तरेण स्वत एव निःसरन्तोति लोकप्रसिद्व्यातादृशैर्यन्त्त्रैरित्यर्थः। यद्वा सूत्राकारेण रेणवः सिकतांशा गर्भे उदरे यस्यैतादृशं यन्त्त्रं वालुकायन्त्त्रं प्रसिद्धम्। तेन सहितैर्मयूरादियन्त्त्रैर्मयूराद्युक्तयन्त्त्रैर्वालुकायन्त्त्रेणचेति सिद्धोऽर्थः।चकारस्तोययन्त्त्रकपालाद्यैरित्यनेन समुच्चयार्थकः। कालं दिनगतादिरूपं सम्यक् सूक्ष्मं प्रसाधयेत्। प्रकर्षेण सूक्ष्मत्वेनातिसूक्ष्मत्वेनेत्यर्थः। जानीयादित्यर्थः~॥~२१~॥\\
\noindent ननु मयूरादिस्वयंवहयन्त्त्राणि कथं साध्यानीत्यतस्तत्साधनप्रकारा बहवो दुर्गमाश्च सन्तीत्याह \textendash

%\vspace{2mm}

 \begin{quote}
{\ssi पारदाराम्बुसूत्राणि शुल्बतैलजलानि च~।\\
बीजानि पांसवस्तेषु प्रयोगास्तेऽपि दुर्लभाः~॥~२२~॥}
%\vspace{2mm}
\end{quote}
 तेषु मयूरादियन्त्त्रेषु स्वयंवहार्थमेते प्रयोगाः प्रकर्षेणयोज्याः। प्रकर्षस्तु यावदभिमतसिद्धेः। एते क इत्यत आह।पारदाराम्बुसूत्राणीति। पारदयुक्ता आराः। यथाच सिद्धान्तशिरोमणौ। 

%\vspace{2mm}
%{\setlength{\parindent}{6em}
\begin{quote}
{\qt लघुकाष्ठजसमचक्रे
समसुषिराराः समान्तरा नेम्याम्~।}
\end{quote}





\newpage


\noindent ३६४ \hspace{4cm} सूर्यसिद्धान्तः
\vspace{1cm}

%{\setlength{\parindent}{6em}
 \begin{quote}
{\qt किञ्चिद्वक्रा योज्याः
 सुषिरस्यार्धे पृथक् तासाम्~॥

रसपूर्णे तच्चक्रं
द्व्याधाराक्षस्थितं स्वयं भ्रमति~। }
%\vspace{2mm}
\end{quote}
इति। अम्बु जलस्य प्रयोगः। सूत्राणि सूत्रसाधनप्रयोगः।शुल्बं शिल्पनैपुण्यम्। तैलजलानि तैलयुक्तजलस्य प्रयोगः।चकारात् तयोः पृथक् प्रयोगोऽपि। वथाच सिद्धान्तशिरोमणौ। 

%\vspace{2mm}

 \begin{quote}
{\qt उत्कीर्य नेमिमथवा
 परितो मदनेन संलग्नम्~।

तदुपरि तालदलाद्यं
कृत्वा सुषिरे रसं क्षिपेत् तावत्~॥

यावद्रसैकपार्श्वे
क्षिप्तजलं नान्यतो याति~।

पिहितच्छिद्रं तदत-
श्चक्रं भ्रमति स्वयं जलाकृष्टम्~॥

ताम्रादिमयस्याङ्कुश-
रूपनलस्याम्बुपर्णस्य~।

एकं कुण्डजलान्त-
र्द्वितीयमग्रं त्वधोमखं च वहिः~॥

युगपम्भक्तं चेत् कं
नलेन कुण्डाद्वहिः पतति~।}
\end{quote}



\newpage


\hspace{3cm} गूढार्थप्रकाशकेन सहितः~।  \hfill ३६५
\vspace{1cm}


 \begin{quote}
 {\qt नेम्यां बध्वा घटिका
श्चक्रं जलयन्त्त्रवत् तथा धार्यम~॥

नलकप्रच्युतसलिलं
पतति यथा तद्वटीमध्ये~।\\
भ्रमति ततस्तत् सततं
पूर्णघटीभिः समाकृष्टम्~॥

चक्रच्यतं स्वमुदकं
कुण्डे याति प्रणालिकया~। }
%\vspace{2mm}
\end{quote}
इति। बीजान केवलं तुङ्गबीजप्रयागः। पांसवो धूलिप्रयोगास्तैर्युक्ताः प्रयोगाः। अपिशब्दात् प्रयोगेषु सुगमतरा इत्यर्थः। दुर्लभाः साधारणत्वेन मनुष्यै कर्तुमशक्या इत्यर्थः। अन्यथाप्रतिगृहं स्वयंवहानां प्राचर्यापत्तेः। इयं स्वयंवहविद्या समुद्रान्तर्निवासिजनैः फिरङ्ग्याख्यैः सम्पगभ्यस्तेति। कुहकविद्यात्वादत्र विस्तारानुद्योग इति संक्षेपः~॥~२२~॥\\
अथ कपालाख्यंजलयन्त्त्रमाह \textendash

%\vspace{2mm}

 \begin{quote}
{\ssi ताम्रपात्रमव च्छिद्रं न्यस्तं कुण्डेऽमलाम्भसि~। \\
षष्टिर्मज्जत्यहोरात्रे स्फुटं यन्त्त्रं कपालकम्~॥~२३~॥}
%\vspace{2mm}
\end{quote}
 यत् ताम्रघटितं पात्रमध च्छिद्रमधोभागे छिद्रं यस्यतत्। अमलाम्भसि निर्मलं जलं विद्यते यस्मिन् तादृशे कुण्डेबृहद्भाण्डे न्यस्तं धारितं सदहोरात्रे नाक्षत्राहोरात्रे षष्टिःषष्टिवारमेव न न्यूनाधिकं मज्जति। अधश्छिद्रमार्गेण जला \textendash



\newpage


\noindent ३६६ \hspace{4cm} सूर्यसिद्धान्तेः 
\vspace{1cm}


\noindent गमनेन जलपूर्णतया निमग्नं भवति। तत् कपालकं कपालमेव कपालकं घटखण्डानां कपालपदवाच्यात्वात् घटाधस्तनार्धाकारं यन्त्त्रं घटीयन्त्त्रं स्फुटं सूक्ष्मम्। तद्वटनंतु। 

%\vspace{2mm}

 \begin{quote}
{\qt  शुल्बस्य दिग्भिर्विहितं पलैर्यत्  \\
षडङ्गुलोच्चं द्विगुणायतास्यम्~।\\
तदम्भसा षष्टिपलैः प्रपूर्यं \\
पात्रं घटार्धप्रतिमं घटी स्यात्~॥

सत्र्यंशमाषत्रयनिर्मिता या \\
हेन्नः शलाका चतुरङ्गुला स्यात्~।\\
विद्धं तया प्राक्तनमत्रपात्रं \\
प्रपूर्यते नाडिकयाम्बभिस्तत्~॥ }
%\vspace{2mm}
\end{quote}
इति व्याक्तम्। भगवत तु सूक्ष्ममुक्तम्~॥~२३~॥\\
अथशङ्कुयन्त्रं दिवैव कालज्ञानार्थं नान्यदेत्याह \textendash

%\vspace{2mm}

 \begin{quote}
{\ssi नरयन्त्त्रं तथा साधु दिवा च विमले रवौ~।\\
छायासंसाधनैः प्रोक्तं कालसाधनमुत्तमम्~॥~२४~॥}
%\vspace{2mm}
\end{quote}
 विमले मेघादिव्यवधानरूपमलेन रहिते सूर्य एतद्रपेदिने। चकार एवकारार्थस्तेन साभ्रदिनव्यवच्छेदः। नरयन्त्त्रं द्वादशाङ्गुलशङ्कुयन्त्त्रं तथा घटीयन्त्त्रवत्कालसाधकं साधु सूक्षां रात्रौ नेत्यर्थसिद्धम्। ननु शङ्कोश्च्छायावाधकत्वं न कालसाधकत्वं तेन तस्य कथं यन्त्त्रत्वं कालसाधकवस्तुनोयन्त्त्रत्वप्रतिपादनादित्यत आह। छायासंसाधनैरिति। इदं \textendash


\newpage


\hspace{3cm} गूढार्थप्रकाशकेन सहितः~। \hfill ३६७
\vspace{1cm}


\noindent शङ्कुरूपनरयन्त्रं छायायाः सम्यक् सूक्ष्मत्वेन साधनरवगमैःकृत्वा कालसाधनं दिनगतादिकालस्य कारणमुत्तमम्। अन्ययन्त्त्रेभ्योऽस्मान्निरन्तरतयातिश्रेष्ठम्। तथाच छायासाधकत्वेनैव छायाद्वारा शङ्कोः कालसाधकत्वमिति न यन्त्त्रत्वव्याघातः।अतएव साभ्रदिने रात्रौ चानुपयुक्तः। नरस्य छाया यन्त्त्रोपलक्षणत्वात् यष्टिधनुश्चक्राण्यपि तथेति ध्येयम्~॥~२४~॥\\
\noindent अथादित एतदन्तग्रन्थज्ञानस्यैकफलकथनेन विभक्तमपि खण्डद्वयं क्रोडयति \textendash

%\vspace{2mm}

 \begin{quote}
{\ssi ग्रहनक्षत्रचरितं ज्ञात्वा गोलं च तत्त्वतः~।\\
ग्रहलोकमवाप्राति पर्यायेणात्मवान् नरः~॥~ २५~॥ }
%\vspace{2mm}
\end{quote}
 ग्रहनक्षत्राणां चरितं गणितविषयकं ज्ञानं ग्रन्थपूर्वखण्डरूपंगोलं भूगोलभगोलस्वरूपप्रतिपादकग्रन्थं ग्रन्थोत्तरार्धान्तर्गतम्। चकारः समुच्चये। तत्त्वतः वस्तुस्थितिसद्भावेन सार्वविभक्तिकस्तसिरित्येके। ज्ञात्वावगम्य नरः पुरुषः। अचलोकं चन्द्रादिग्रहाणां लोकं तल्लोकाधिष्ठितस्थानं ग्रहोपलक्षणान्नक्षत्राधिष्ठितस्थानमपि ध्येयम्। प्राप्नोति। ननु ग्रहलोकप्राप्त्या कः परुषार्थ इत्यतो मोक्षरूपं परुषार्थफलमाह\textendash पर्यायेणेति। जन्मान्तरेण पुरुष आत्मवानात्मज्ञानी भवति। तथाचात्मज्ञानान् मोक्षप्राप्तिरेवेति भावः~॥~२५~॥ \\
\noindent  अथाग्रिमग्रन्थस्थासङ्गतिपरिहारायारब्धाध्यायसमाप्तिं फक्किकयाह 



\newpage


\noindent ३६८ \hspace{4cm} सूर्यसिद्धान्तः 
\vspace{1cm}

\begin{center}
  इति ज्योतिषोपनिषदध्यायः। 
\end{center}



 इति। यथा वेदे आत्मस्वरूपनिरूपणान्नारायणोपनिषदुच्यते। तथा ज्योतिःशास्त्रे प्रतिपादितानां ग्रहनक्षत्राणामेतद्ग्रन्थैकदेशे स्वरूपादिनिरूपणाज्ज्येतिःशास्त्रसारंज्योतिषोपनिषदुच्यते। तत्सञ्ज्ञोऽध्यायो ग्रन्थैकदेशः सम्पूर्ण इत्यर्थः। 

%\vspace{2mm}

\begin{quote}
 {\qt रङ्गनाथेन रचिते सूर्यसिद्धान्तटिप्पणे~।\\
 ज्योतिषोपनिषत्सञ्ज्ञोऽध्यायः पूर्णोऽपरार्धके~॥ }
 \end{quote}
%\vspace{2mm}

इति श्रीसकलगणकसार्वभौमबल्लालदैवज्ञात्मजरङ्गनाथगणकविरचिते गूढार्थप्रकाशक उत्तरखण्डे ज्योतिषोपनिषदध्यायः पूर्णः~॥


\vspace{6cm}

\begin{center}
    \rule{8em}{.5pt}
\end{center}



\newpage



\hspace{3cm} गूढार्थप्रकाशकेन सहितः~। \hfill ३६९
\vspace{1cm}


 अथ मानानि कति किञ्च तैरित्यवशिष्टप्रम स्योत्तरभूतआरब्धमानाध्यायो व्याख्यायते। तत्रप्रथमेमानानिकतीतिप्रथमप्रश्नस्यात्तरमाह\textendash

%\vspace{2mm}

 \begin{quote}
{\ssi ब्राह्मं दिव्यं तथा पित्र्यं प्राजापत्यं गुरोस्तथा~।\\
सौरं च सावनं चान्द्रमार्क्षं मानानि वै नव~॥~१ ~॥}
%\vspace{2mm}
\end{quote}
 वै निश्चयेन। नवसङ्ख्याकानि कालमानानि। तत्र प्रथमं ब्राह्ममानम्। 

\begin{center}
{\qt  कल्यो ब्राह्ममहः प्रोक्तम्~।} 
\end{center}

\noindent इत्यादि। 

%\vspace{2mm}

 \begin{quote}
{\qt परमायुः शतं तस्य तथाहोरात्रसङ्ख्यया~। }
% \vspace{2mm}
\end{quote}
 इत्यन्तं मध्यमाधिकारे प्रतिपादितम्। द्वितीयं दिव्यंदेवमानम्। 

\begin{center}
  दिव्यं तदह उच्यते। 
\end{center}

\noindent इत्यादि। 

\begin{center}
 तदेषष्टिः षङ्गुणा दिव्यं वर्षम्। 
\end{center}

इत्यन्तं तत्रैव प्रतिपादितम्। तथा तृतीय मानं पित्र्यं पितृणांमानं वक्ष्यमाणम्। प्राजापत्यं मानं वक्ष्यमाणं चतुर्थम्। बृहस्पतेस्तथा मानं वक्ष्यमाणं पञ्चमम्। सौरं चकारात् षष्ठंमानम्। सावनं सप्तमं मानम्। चान्द्रमानमष्टमम्। नाक्षत्रमानं नवमम्। एतान्यपि तत्रैवोक्तानि~॥~१~॥\\
\noindent अथ किञ्चतैरिति द्वितीयप्रश्नस्योत्तरं विवक्षुः प्रथमं व्यवहारोपयक्तमानानि दर्शयति \textendash



\newpage


\noindent ३७० \hspace{4cm} सूर्यसिद्धान्तः 
\vspace{1cm}


 \begin{quote}
{\ssi चतुर्भिर्व्यवहारोऽत्र सौरचान्द्रर्क्षसावनैः~।\\
बार्हस्पत्येन षष्ट्यब्दं ज्ञेयं नान्यैस्तु नित्यशः~॥~२~॥ }
%\vspace{2mm}
\end{quote}

 अत्र मनुष्यलोके सौरचान्द्रनाक्षत्रसावनैश्चतुर्भिर्मानैर्व्यवहारः कर्मघटना। षष्ट्यब्दं प्रभवादिषष्टिवर्षं जात्यभिप्रायेणैकवचनम्। बार्हस्पत्येन बृहस्पतिमानेन बृहस्पतिमध्यमराशिभोगात्मककालेन प्रत्येकं ज्ञेयम्। अन्यैरवशिष्टैर्ब्राह्मपित्र्यप्राजापत्यैः। नित्यशः सदेत्यर्थः। व्यवहारो नास्ति। तुकारात्कादाचित्कत्वेन तैर्व्यवहारः~॥~२~॥ \\
\noindent अथ सौरेण व्यवहारंप्रदर्शयति \textendash

%\vspace{2mm}

\begin{quote}
{\ssi सौरेण द्युनिशोर्वामं षडशीतिमुखानि च~।\\
अयनं विषुवचच्चैव संक्रान्तेः पुण्यकालता~॥~३~॥}
%\vspace{2mm}
\end{quote}
 अहोरात्र्योर्मानं सौरेण ज्ञेयम्। प्रात्यहिकसूर्यगतिभोगादहोरात्रं भवतीत्यर्थः। षडशीतिमुखानि वक्ष्यमाणानि। चः समुच्चये। तेन सौरमानेन ज्ञेयानि। अयनं विषुवत्। चः समुच्चये। संक्रान्तेः पुण्यकालता सूर्यबिम्बकलासम्बद्धा सौरमानेन~॥~३~॥ \\
\noindent अथ षडशीतिमुखमाह \textendash

%\vspace{2mm}

\begin{quote}
{\ssi तुलादिषडशीत्यर्क्रा षडशीतिमुखं क्रमात्~।\\
तच्चतुष्टयमेव स्याद्द्विस्वभावेषु राशिषु~॥~४~॥}
%\vspace{2mm}
\end{quote}
 तुलारम्भात् षडशीतिदिवसानां सौराणां षडशीतिमुखं भवति। तच्चतुष्टयं षडशीतिमुखस्य चतुःसङ्ख्या द्विस्वभावेषुराशिषु चतुर्षु क्रमादेवं वक्ष्यमाणा भवति~॥~४~॥ \\
तदेवाह \textendash



\newpage


\hspace{3cm} गूढार्थप्रकाशकेन सहितः~। \hfill ३७१
\vspace{1cm}


 \begin{quote}
 {\ssi षड्विशे धनुषो भागे द्वाविंशे निमिषस्य च~।\\
मिथुनाष्टादशे भागे कन्यायास्तु चतुर्दश~॥~५~॥}
%\vspace{2mm}
\end{quote}

 धनूराशेः षड्विंशतितमेंऽशे षडशीतिमुखे मीनराशेर्द्धाविंशतितमेंऽशे षडशीतिमुखम्। चकारः ममुच्चयार्थकः प्रत्येकमन्वेति। मिथुनराशेरष्टादशेंऽशे षडशीतिमुखं कन्यायाश्चतुर्दशेभागे षजशीतिमुखम्। अतएव तुलादितः षडशीत्यंशो गणनया येषु राशिषु भवति ते राशथो द्विस्वभावाः षडशीतिमुखसञ्ज्ञाः संक्रान्तिप्रकरणे सांहितिकैरुक्ताः ~॥~५~॥ \\
\noindent अथ षडशीत्यंशगणनया चत्वारि वडथीतिमुखान्युत्क्का भगणांशपूर्त्त्यर्थमवशिष्टांणःषोडशातिपुण्या इत्याह \textendash

%\vspace{2mm}

\begin{quote}
{\ssi ततः शेषाणि कन्याया यान्यहानि तु षोडश~। \\
क्रतुभिस्तानि तुल्यानि पितॄणां दत्तमक्षयम्~॥~६~॥ }
%\vspace{2mm}
\end{quote}
 ततः कन्यादिचतुर्दशभागानन्तरं शेषाणि भगणभागेऽवशिष्टानि कन्याया यान्यहानि सौरभागसमानि षोडशतानि। तुकारात् पूर्वदिनासमानि क्रतुभिर्यज्ञैः समानि। अतिपुण्यानीत्यर्थः। तत्र पितॄणां दत्तं श्राद्धादि कृतमक्षयमनन्तफलदं भवति~॥~६~॥ \\
\noindent अथ राश्यधिष्ठितक्रान्तिवृत्ते चत्वारिस्थानानि पदसन्धिस्थाने विषुवायनाभ्यां प्रसिद्धानीत्याह \textendash

%\vspace{2mm}

 \begin{quote}
{\ssi भचक्रनाभौ विषुवद्द्वितयं समसूत्रगम्~।\\
अयनद्वितयं चैव चतस्रः प्रथितास्तु ताः~॥~७~॥}
\end{quote}


\newpage


\noindent ३७२ \hspace{4cm} सूर्यसिद्धान्तः
\vspace{1cm}


 भचक्रनाभा भगोलस्य ध्रुवद्वयाभ्यां तुल्यान्तरेण मध्यभागेविषुवद्वितयं विषुवद्द्वयं समसूत्रगं परस्परं व्याससूत्रान्तरितंप्रवमध्ये विषुवदृत्तावस्थानात् तदृत्ते क्रान्तिवृत्तभागौ यौ लग्नौ तौ क्रमेण पूर्वापरौ विषुवत्सञ्ज्ञौ मेषतुलाख्यौ चेत्यर्थः। अयनद्वितयमयनद्वयं कर्कमकरादिरूपम्। चः समुच्चये। तेन समसूत्रगं ता विषुवायनाख्याः क्रान्तिवृत्तप्रदेशरूपा भूमयश्चतस्त्रश्चतुःसङ्ख्याकाः प्रथिता गणितादौ पदादित्वेन प्रसिद्धाः। एवकारादन्यराशीनां निरासः। तुकारात् तासांसमसूत्रस्थत्वेऽपि विषुवायनत्वाभावात् पदादित्वेनाप्रसिद्धिरित्यर्थः~॥~७~॥\\ 
\noindent अथावशिष्टानामादिस्वरूपमन्यदप्याह\textendash

%\vspace{2mm}

\begin{quote}
{\ssi तदन्तरेषु संक्रान्तिद्वितयं द्वितयं पुनः~।\\
नैरन्तर्यात् तु संक्रान्तेर्ज्ञेयं विष्णुपदीद्वयम्~॥~८~॥ }
%\vspace{2mm}
\end{quote}
 तदन्तरेषु विषुवाधनान्तरालेषु। अत्रान्तरालानां चतुःस्थाने सद्भावाद्धहुवचनम्। संक्रान्तिद्वितयं द्वितयं पुना राश्यादिभागे ग्रहाणामाक्रमणं वारद्वयं भवति तदन्तराले राश्यादिभागौ द्वौ भवत इत्यर्थः। यथा हि मेषाख्यविषुवकर्काख्यायनयोरन्तराले वृषमिथुनयोरादी। कर्कतुलयोरन्तरालेसिंहकन्ययोरादी। तुलामकरयोरन्तराले वृश्चिकधनुषोरादी। मकरमेषयोरन्तराले कुम्भमीनयोरादी इति। एवं विषुवानन्तरं सक्रमणद्वयमनन्तरमयनं तदनन्तरं संक्रान्ति \textendash


{\tiny{h}}

\newpage


\hspace{3cm} गूढार्थप्रकाशकेन सहितः~। \hfill ३७३
\vspace{1cm}


\noindent त्यादि पौनःपुन्येन ज्ञेयमित्यर्थः। संक्रान्तिद्वयमध्ये प्रथमसंक्रान्तौ विशेषमाह। नैरन्तर्यादिति। निरन्तरतया सम्भूतायाः संक्रान्तेः सकाशाद्विष्णुपदीद्वयं तदन्तराल इति त्वर्थः। अवगम्यं प्रथमसंक्रान्तिर्विष्णुपदसञ्ज्ञा तयोर्द्वयं तदभ्यन्तरे प्रत्येकं भवतीति तात्पर्यार्थः। षडशीतिसञ्ज्ञं द्वितीयसंक्रमणं पूर्वसूचितं तयोरपि द्वयं तदन्तराले भवतीति ध्येयम्~॥~८~॥\\
\noindent अथायनद्वयमाह \textendash

%\vspace{2mm}

\begin{quote}
{\ssi भानोर्मकरसंक्रान्तेः षण्मासा उत्तरायणम्~।\\
कर्कादस्तु तथैव स्यात् षण्मासा दक्षिणायनम्~॥~९~॥}
%\vspace{2mm}
\end{quote}

 सूर्यस्य मकरसंक्रान्तेः सकाशात् षट् सौरमासा उत्तरायणं भवति। कर्कादेः कर्कसंक्रान्तेः सकाशात तथा सूर्यभोगात।एवकारादन्यग्रहनिरासः। षण्मासा । तुकारात् सौराः। दक्षिणायनं भवति~॥~९~॥\\
अथर्तुमासवर्षाण्याह \textendash

%\vspace{2mm}

\begin{quote}
{\ssi द्विराशिनाथा ॠतवस्ततोऽपि शिशिरादयः~।\\
मेषादया द्वादशैते मासास्तैरैव वत्सरः~॥~१०~॥  }
%\vspace{2mm}
\end{quote}
 ततो मकरसंक्रान्तेः सकाशात्। अपिशब्द उत्तरायणावधिना समुच्चयार्थकः। द्विराशिनाथा राशिद्वयस्वामिका राशिद्वयार्कभोगात्मका इत्यर्थः। शिशिरादयः शिशिरवसन्तग्रीष्मवर्षाशरद्धेमन्ता ॠतवः कालविभागविशेषा भवन्ति।एते सूर्यभोगविषयका मेषादयो राशयो द्वादशमासास्तैर्द्वादशभि \textendash




\newpage


\noindent ३७४ \hspace{4cm} सूर्यसिद्धान्तः 
\vspace{1cm}


\noindent र्मासैः। एवकारात्र्यूनाधिकव्यवच्छेदः। वत्सरः सौरवर्षं भवति~॥~१०~॥ \\
\noindent अथ प्रसङ्गात् संक्रान्तौ पुण्यकालानयनमाह \textendash

%\vspace{2mm}

 \begin{quote}  
{\ssi अर्कमानकलाः षष्ट्या गुणिता भुक्तिभाजिताः~।\\
तदर्धनाड्यः संक्रान्तेरर्वाक् पुण्यं तथापरे~॥~११~॥ }
%\vspace{2mm}
\end{quote}
 सूर्यस्य बिम्बप्रमाणकलाः षष्ट्या गुणिताः सूर्यगत्या भक्तास्तस्य फलास्यार्धं तत्सङ्ख्याका घटिका इत्यर्थः। संक्रान्तेः सूर्यस्यराशिप्रवेशकालादित्यर्थः। अर्वाक् पूर्वं पुण्यं स्नानादिधर्मकृत्येपुण्यघटिकाः पुण्यवृद्धिकारिकाः। अपरे संक्रान्त्तुत्तरकाले तथा स्नानादिधर्मकृत्ये पुण्यवृद्धिदा इत्यर्थः। अत्रोपपत्तिः। सूर्यबिम्बकेन्द्रस्य राश्यादौ सञ्चरणकालः संक्रमणकालस्तस्य सूक्ष्मत्वेन दुर्ज्ञेयत्वात् फलकालः कोऽप्यभ्युपेयः स तु राश्यादौबिम्बसञ्चरणरूपोऽङ्गीकृतो बिम्बसम्बन्धात्। अतः सूर्यगत्याषष्टिसावनघटिकास्तदा सूर्यबिम्बकलाभिः का इत्यनुपातानीता बिम्बघटिकां संक्रान्तिकालः स्यूलः प्राङ्गेमिसञ्चरणकालात् पश्चिननेमिसञ्चरणकालपर्यन्तं तदर्धघटिका व्यासार्धघटिका इति संक्रान्तिकालात् ताभिः पूर्वमपरत्र काले प्रागपरनेम्योः क्रमेण सञ्चरणात् पूर्वोत्तरकाले पुण्या इति~॥~११~॥ \\
अथ सौरमुत्क्का क्रमप्राप्तं चान्द्रमानमाह \textendash

%\vspace{2mm}

\begin{quote}
{\ssi अर्काद्विनिसृतः प्राचीं यद्यात्यहरहः शशी~।\\
तच्चान्द्रमानमंशैस्तु ज्ञेया द्वादशभिस्तिथिः~॥~१२~॥}
%\vspace{2mm}
\end{quote}
 सर्यात् समागमं त्यत्क्का विनिर्गतः पृथग्भूतः संश्चन्द्रो \textendash



\newpage


\hspace{3cm} गूढार्थप्रकाशकेन सहितः~। \hfill ३७५
\vspace{1cm}


\noindent ऽहरहः प्रतिदिनं यत्। यत्सङ्ख्यामितं प्राचीं पूर्वां दिशंगच्छति तत् प्रतिदिने चान्द्रमानं तत्तु गत्यन्तरांशमितम्। ननु सौरदिनं सूर्यांशेन यथा भवति तथैतद्रूपैर्भागैः कियद्भिःपूर्णं चान्द्रं दिनं भवतोत्यत आह\textendash अंशैरिति~। भागैस्तुकारात्सूर्यचन्द्रान्तरोत्पन्नैस्तस्य तद्रूपत्वात्। द्वादशभिर्द्वादशसङ्ख्याकैस्तिथिर्ज्ञेया। एकं चान्द्रदिनं ज्ञेयमित्यर्थः। एतदुक्तं भवति। सूर्यचन्द्रयोगाच्चान्द्रदिनप्रवृत्तेः पुनर्योगे माससमाप्तेर्भगणान्तरेण चान्द्रो मासस्त्रिंशचान्द्रदिनात्मकः। अतस्त्रिंशद्दिनैर्भगणांशान्तरं तदैकेन किमिति। द्वादशभागैरेकं चान्द्रदिनम्। 

\begin{center}
 दर्शः भूर्येन्दुसङ्गमः। 
\end{center}

इत्याभिधानाद्दर्शावधिकमासस्य त्रिंशत्तिथ्यात्मकत्वात् तिथिश्चान्द्रदिनरूपेति~॥~१२~॥ \\
अथ चान्द्रव्यवहारमाह\textendash

%\vspace{2mm}

\begin{quote}
{\ssi तिथिः करणमुद्वाहः क्षौरं सर्वक्रियास्तथा।\\
व्रतोपवासयात्राणां क्रिया चान्द्रेण गृह्यते~॥~१३~॥}
%\vspace{2mm}
\end{quote}
 तिथिः प्रतिपदाद्या करणं बवादिकमुद्वाहो विवाहः क्षौरंचौलकर्म। पतदाद्याः सर्वक्रिया व्रतबन्धाद्युत्सवरूपा व्रतोपवासयात्राणां नियमोपवासगमनानां क्रिया करणम्। तथा समुच्चयार्थकः। चान्द्रमानेन गृह्यते। अङ्गीक्रियते~॥~१३~॥\\
\noindent अथ चान्द्रमासं प्रसङ्गात् पितृमानं चाह\textendash

%\vspace{2mm}

\begin{quote}
{\ssi त्रिंशता तिथिभिर्मासश्चान्द्रः पित्र्यमहः स्मृतम्~।\\
निशा च मासपक्षान्तौ तयोर्मध्ये विभागतः~॥~१४~॥}
%
\end{quote}



\newpage


\noindent ३७६ \hspace{4cm} सूर्यसिद्धान्तः
\vspace{1cm}


 त्रिंशता त्रिंशन्मितैस्तिथिभिश्चान्द्रो मासः। पित्र्यं पितृसम्बन्धि। अहो दिनम्। निशा रात्रिः पितृसम्बद्धा। चकारोव्यवस्थार्थकः। तेनोभयं नैकः प्रत्येकं किंतु मिलितं स्मृतमिति।लिङ्गानुरोधेनोभयत्रान्येति। तथाच चान्द्रो मासः। पित्र्याहोरात्रमित्यर्थः फलितः। मासपक्षान्तौ मासान्तो दर्शान्तःपक्षान्तः पूर्णिमान्तः। एतावित्यर्थः। विभागतः क्रमेणेत्यर्थः।तयो पित्र्याहोरात्रयोर्मध्येऽर्धे भवतः। दर्शान्तः पितृणांमध्याह्नं पूर्णिमान्तः पितॄणां मध्यरात्रमित्यर्थः। अर्थात्कृष्णाष्टम्यर्धे दिनप्रारम्भः। शुक्लाष्टम्यर्धे दिनान्त इति सिद्धम्~॥~१४~॥\\
 \noindent अथ क्रमप्राप्तं नक्षत्रमानं प्रसङ्गान्माससञ्ज्ञां चाह \textendash

%\vspace{2mm}

  \begin{quote}
{\ssi भचक्रभ्रमणं नित्यं नाक्षत्रं दिनमुच्यते~।\\
नक्षत्रनाम्ना मासास्तु ज्ञेयाः पर्वान्तयोगतः~॥~१५~॥ }
%\vspace{2mm}
\end{quote}
 नित्यं प्रत्यहं भचक्रभ्रमणं नक्षत्रसमूहस्य प्रवहवायुकृतपरिभ्रमः। नाक्षत्रं नक्षत्रसम्बन्धि दिनं मानज्ञैः कथ्यते। नित्यमित्यनेन चन्द्रभोगनक्षत्रभोगो नाक्षत्रमित्यस्य निरासः। भचक्रभ्रमणानुपपत्तेः\textendash माससञ्ज्ञा महानक्षत्रनान्नेति~। पर्वान्तयोगतः पर्वान्तः पूर्णिमान्तः। तस्य योगात् तत्सम्बन्धात्। नक्षत्रसञ्ज्ञया मासाः। तुकाराच्चान्द्रा अवगम्याः पूर्णिमान्तस्थितचन्द्रनक्षत्रसञ्ज्ञो मासो ज्ञेय इति तात्पर्यार्थः।यथा हि यद्दर्शान्तावधिकश्चात्रो माश्चस्तदभ्यन्तरस्थितपूर्णिमान्तस्थितचन्द्रनक्षत्रसञ्ज्ञः। चित्रासम्बन्धाच्चैत्र । विशाखा \textendash



\newpage


\hspace{3cm} गूढार्थप्रकाशकेन सहितः~। \hfill ३७७ 
\vspace{1cm}


\noindent सम्बन्धाद्वैशाखः। ज्येष्ठासम्बन्धाज्ज्येष्ठः। आषाढासम्बन्धादाषाढः। श्रवणसम्बन्धाच्छ्रावणः। भाद्रपदासम्बन्धाद्भाद्रपदः। अश्विनीसम्बन्धादाश्विनः। कृत्तिकासम्बन्धात्कार्तिकः।मृगशर्षिसम्बन्धान्मार्गशीर्षिः। पुव्यसम्बन्धात् पौषः।मघासम्बन्धान्माधः। फाल्गुनीसम्बन्धात् फाल्गुन इति ~॥~१५~॥ \\ 
\noindent ननुपूर्णिमान्ते तत्तन्नक्षत्राभावे कथं तत्सञ्ज्ञा मासानामुचितत्यत आह \textendash

%\vspace{2mm}

\begin{quote}
{\ssi कार्तिक्यादिषु संयोगे कृत्तिकादि द्वयं द्वयम्~।\\
अन्त्योपान्त्यौ पञ्चमश्च त्रिधा मासत्रयं स्मृतम्~॥~१६~॥ ।। }
%\vspace{2mm}
\end{quote}
 नक्षत्रसंयोगार्थमिति निमित्तसप्तमी। कार्तिक्यादिषु कार्तिकमासादीनां पौर्णमासीय्वित्यर्थः। कृत्तिकादि द्वयं नक्षत्रं कथितं कृत्तिकारोहिणीभ्यां कार्तिक । मृगार्द्राभ्यांमार्गशीर्षः। पुनर्वसुपुय्याभ्यां पौषः। आश्लोषामघाभ्यां माघः|चित्रास्वातीभ्यां चैत्रः। विशाखानुराधाभ्यां वैशाख । ज्येष्ठामूलाभ्यां ज्येष्ठः। पर्वोत्तराषाढाभ्याभाषाढः। श्रवणधनिष्ठाभ्यां श्रावण इति फलितम्। अवशिष्टमासानामाह\textendash अन्त्योपान्त्याविति~। अत्र कर्त्तिकस्यादित्वेन ग्रहादन्त्य आश्विनः।उपान्त्यो भाद्रपदः। एतौ मासौ पञ्चमः फाल्गुनः। चकारः समुच्चय इति मासत्रयं त्रिधा स्थानत्रय उक्तम। रेवत्यश्विनभिरणीति नक्षत्रत्रयसम्बन्धादाश्विनः। शततारापूर्वोत्तराभाद्रपदेति नक्षत्रत्रयसम्बन्धाद्भाद्रपदः। पूर्वोत्तरा \textendash



\newpage


\noindent ३७८ \hspace{4cm} सूर्यसिद्धान्तः
\vspace{1cm}


\noindent फाल्गुनीहस्तेति नक्षत्रत्रयसम्बन्धात् फाल्गुन इति सिद्धम्~॥~१६~॥ 
अथ प्रसङ्गात् कार्तिकादिबृहस्यतिवर्षाण्याह \textendash

%\vspace{2mm}

 \begin{quote}
{\ssi वैशाखादिषु कृष्णे च योगः पञ्चदशे तिथौ~।\\
कार्तिकादीनि वर्षाणि गुरोरस्तेदयात् तथा~॥~१७~॥ }
\end{quote}
%\vspace{2mm}

 यथा पौर्णमास्यां नक्षत्रसम्बन्धेन तत्सञ्ज्ञो मासो भवति।
तथेति समुच्चयार्थकम्। बृहस्पते सूर्यसान्निध्यदूरत्वाभ्यामस्तादुदयाद्वा वैशाखादिषु द्वादशसु मासेषु कृष्णपक्षे पञ्चदशे तिथौ। अमायामित्यर्थः। चकारः पौर्णमासीसम्बन्धात्समुच्चयार्थकः। योगो दिननक्षत्रसम्बन्धः कार्तिकादीनि द्वादश वर्षाणि भवन्ति। वैशाखकृष्णापक्षपञ्चदश्याममारूपायांबृहस्पतेरस्त उदये वा जाते सति तदादि बृहस्पतिवर्षं कृत्तिकादिनक्षत्रसम्बन्धात् कार्तिकसञ्ज्ञम्। एवं ज्येष्ठाषाढश्रावणभाद्रपंदाश्विनकार्तिकमार्गशर्षपौषमाघफाल्गुनचैत्रामासु मृगपुय्यमघापूफाचित्राविशाखाज्येष्ठापूषाश्रवणपूभाश्विनीदिननक्षत्रसम्बन्धान्मार्गशीर्षादीनि भवन्ति। अत्रापि प्रोक्तनक्षत्रद्वयत्रयसम्बन्धः प्रागुक्तो बोध्यः। अनेनेत्युपलक्षणम्। तेनयद्दिने बृहस्पतेरुदयोऽस्तो वा तद्दिने यच्चन्द्राधिष्ठितनक्षत्रं तत्सञ्ज्ञं बार्हस्पत्यं वर्षं भवतीति तात्पर्यम्। संहिताग्रन्थेऽस्तोदयवशाद्धर्षोक्तिः परमिदानीमुदयवर्षव्यवहारो गणकैर्गण्यतें येनोदितेज्य इत्युक्तेरिति~॥~१७~॥
\noindent अथ क्रमप्राप्तं सावनमाह\textendash



\newpage


\hspace{3cm} गूढार्थप्रकाशकेन सहितः~। \hfill ३७९
\vspace{1cm}

%\vspace{2mm}

\begin{quote}
 {\ssi उदयादुदयं भानोः सावनं तत् प्रकीर्तितम्~।\\
सावनानि स्युरेतेन यत्रकालविधिस्तु तैः~॥~१८~॥ }
%\vspace{2mm}
\end{quote}

 सूर्यस्योदयादुदयकालमारभ्याव्यवहितोदयकालपर्यन्तं यत्
कालात्मकं तत् सावनं मानज्ञैरुक्तम्। एतेनोदयद्वयान्तरात्मककालस्य गणनया सावनानि वसुद्व्यष्टाद्रीत्यादिना मध्याधिकारोक्तानि भवन्ति। तद्व्यवहारमाह\textendash यज्ञकालविधिरिति~। यज्ञस्य यः कालस्तस्य गणना तैः सावनैः। तुकारोऽन्यमाननिरासार्थकैवकारपरः~॥~१८~॥\\ 
\noindent अथ व्यवहारान्तरमाह \textendash

%\vspace{2mm}

\begin{quote}
 {\ssi सूतकादिपरिच्छेदो दिनमासाब्दपास्तथा~।\\
मध्यमा ग्रहभुक्तिस्तु सावनेनैव गृह्यते~॥~१९~॥}
%\vspace{2mm}
\end{quote}
 सूतकं जन्ममरणसम्बन्धि। आदिपदग्राह्यं चिकित्सितचान्द्रायणादि। तस्य परिच्छेदो निर्णयः। दिनाधिपमासेश्वरवर्षेश्वराः। तथा समञ्चये ग्रहाणां गतिर्मध्यमा। तुकारात् स्पष्टगतेर्निंरासः। तस्याः प्रतिक्षणंवेंलक्षण्याद्दिनसम्बन्धस्याभावात्। एतेन स्पष्टगत्या स्पष्टग्रहस्य चालनं निरस्तंस्थूलत्वादिति सूचितम्। सावनमानेन। एवकारादन्यमाननिरासः। गृह्यते सुधीभिरङ्गोक्रियते। अत्र बहुवचनानुरोधेन गृह्यत इत्यत्र बहुवचनं ज्ञेयम्~॥~१९~॥\\
\noindent अथ दिव्यमानमाह \textendash


\newpage


\noindent ३८० \hspace{4cm} सूर्यसिद्धान्तः 
\vspace{1cm}

%\vspace{2mm}

\begin{quote}
{\ssi सुरासुराणामन्योन्यमहोरात्रं विपर्ययात्~।\\
यत् प्रोक्तं तद्भवेद्दिव्यं भानोर्भगणपूरणात्~॥~२०~॥ }
%\vspace{2mm}
\end{quote}

 पूर्वार्धं पूर्वं व्याख्यातम्। यदहोरात्रं पूर्वार्धोक्तं सूर्यस्य भगणभोगपूर्तेः प्रोक्तं पूर्वमनेकधा निर्णीतं तदहोरात्रं दिव्यमानं स्यात्~॥~२०~॥
\noindent  अथावशिष्टे प्राजापत्यब्राह्ममाने आह \textendash
%\vspace{2mm}

\begin{quote}
{\ssi मन्वन्तरव्यवस्था च प्राजापत्यमुदाहृतम्~।\\
न तत्र द्युनिशेर्भेदो ब्राह्मं कल्पः प्रकोर्तितम्~॥~२१~॥ }
%\vspace{2mm}
\end{quote}
%\vspace{2mm}

\begin{quote}
 {\qt मन्वन्तरव्यवस्था मन्वन्तरावस्थितिः। }
 \end{quote}
%\vspace{2mm}

\begin{center}
{\qt यगानां सप्ततिः सैका }
\end{center}
इत्यादिना मध्याधिकारोक्तेति चार्थः। प्राजापत्यं मानं मानज्ञैरुदाहृतमुक्तं मनूनां प्रजापतिपुत्रत्वात्। ननु देवपितृमानयोर्दिनरात्रिभेदो यथोक्तस्तथास्मिन् माने दिनरात्रिभेदप्रीःतपादनं कथं नोक्तमित्यत आह\textendash नेति~। तत्र प्राजापत्यमाने द्युनिशोर्दिनरात्र्योर्भेदो विवेकोगरुसौरचन्द्रमानवन्नास्ति। ब्रह्ममानमाह\textendash ब्राह्ममिति~। कल्पो युगसहस्त्रात्मकं प्रागुक्तः। ब्रह्ममानं मानज्ञैरुक्तम्। यद्यपि पूर्वं पित्र्यबार्हस्पत्यमानयोरनुक्तेरत्र तयोरेव निरूपणं युक्तमन्येषां निरूपणं तु पूर्वेक्त्या पुनरुक्तं तथापि पूर्वं गणिताद्युपजीव्यपरिभाषाकथनावश्यकतया गणितप्रवृत्त्यर्थं तेषाममानत्वेननिरूपणादत्र तु विशेषकथनार्थं मानत्वेन पुनस्तेषां निरूपणं



\newpage


\hspace{3cm} गूढार्थप्रकाशकेन सहितः~। \hfill ३८१ 
\vspace{1cm}


\noindent प्रश्रोत्तरत्वेनाक्षतिकरमन्यथा प्रश्नानुपपत्तेरिति दिक्~॥~२१~॥\\
\noindent अथ स्वोक्तमुपसंहरति\textendash 

%\vspace{2mm}

\begin{quote}
 {\ssi एतत् ते परमाख्यातं रहस्यं परमड्भुतम्~।\\
ब्रह्मैतत् परमं पुण्यं सर्वपापप्रणाशनम्~॥~२२~॥ }
\end{quote}
%\vspace{2mm}

 हे परम दैत्यश्रेष्ठ सूर्यभक्तत्वात्। ते तुभ्यमेतदधुनोक्तं परं द्वितीयकथनमाख्यातं निराकाङ्क्षतया सम्पूर्णं कथितम्। पूर्वं सावशेषमुक्तं स्थितमिति त्वया प्रश्नाः कृतास्तदुत्तररूपद्वितीयकथनमिदं निःसन्दिग्धमस्तीति तव संशया नोद्भवन्तीति भावः। ननु मत्प्रश्नं विना पुर्वमेवेदं कथं नोक्तमित्यत आह\textendash रहस्यमिति~। कुत इत्यत आह\textendash अद्भुतमिति~। आकाशस्य ग्रहनक्षत्रादिस्थितिज्ञानसम्पादकत्वादाश्चर्यकरमित्यर्थः । तथाच मत्पूर्वोक्तं त्येन सावधानतया श्रुतं तेनैव त्वदुक्ताः प्रश्नाः कर्तुं शक्यास्तदुत्तरत्वेन द्वितीयं मदुक्तमिति त्वां परीक्ष्य त्वां प्रत्युक्तं रहस्यमिति भावः। नन्वन्यंशास्त्राणां ज्ञानाद्ब्रह्मानन्दावाप्तिरस्यान्नेत्यत आह\textendash ब्रह्मेति~। एतन्मदुक्तं ब्रह्म ब्रह्यसमं तथाचान्यशास्त्राणां ब्रह्मसमत्वाभावेऽपि तञ्ज्ञानाद्ब्रह्मानन्दावाप्तिरस्माद्ब्रह्मस्वरूपाद्ब्रह्मानन्दावाप्तौ किं चित्रमिति भावः। कुत इदं ब्रह्मसममित्यत आह\textendash परमिति~। उत्कृष्टम्। अत्र हेतुभूतं विशेषणद्वयमाह\textendash पुण्यं सर्वपापप्रणाशनमिति~। पुण्यजनकं सर्वपापनाशकम्~॥~२२~॥\\
 \noindent नन्वस्माद्ब्रह्मानन्द\textendash



\newpage


\noindent ३८२ \hspace{4cm} सूर्यसिद्धान्तः
\vspace{1cm}


प्राप्निरुक्ता पूर्वं ग्रहलोकप्राप्तिश्चोक्ता तत्रानयोः किं फलं भवतीत्यत आह\textendash 

%\vspace{2mm}

\begin{quote}
 {\ssi दिव्यं चार्क्षं ग्रहाणां च दर्शितं ज्ञानमुत्तमम्~।\\
विज्ञायार्कादिलोकेषु स्थानं प्राप्नोति शाश्वतम्~॥~२३~॥ }
\end{quote}
%\vspace{2mm}

 आर्क्षं नक्षत्रसम्बन्धि ज्ञानं ग्रहाणां ज्ञानम्। चः समुच्चये। उत्तमं सर्वशास्त्रेभ्य उत्कृष्टम्। अत्र हेतुभूतं विशेषणं दिव्यं स्वर्गलोकोत्पन्नं दर्शितं मया तुभ्यमुपदिष्टं विज्ञाय ज्ञात्वार्कादिलोकेषु सूर्यादिग्रहलोकेषु स्थानमधिष्ठानं प्राप्नोति शाश्वतं नित्यं ब्रह्मसायुज्यरूपं स्थानम्। पूर्वर्धस्थाद्वितीयचकारः समुच्चयार्थकोऽत्रान्वेति। तथाचोभयं फलं क्रमेण भवतीति भावः। यत्त्वेतत् ते परमाख्यातमित्यादिश्लोकः क्वचित् पुस्तकेऽस्माच्छोकात् पूर्वं नास्ति किन्तु माननिरूपणान्तस्यदिव्यं चार्क्षमित्यादिश्लोकान्ते मानाध्यायसमाप्तिं कृत्वारो। 

%\vspace{2mm}

\begin{quote}
 {\qt यथा शिखा मयूराणां नागानां मणयो यथा~।\\
तद्वद्वेदाङ्गशास्त्राणां गणितं मूर्धनि स्थितम्~॥~१~॥

न देयं तत् कृतघ्नाय वेदविप्लावकाय च~।\\
अर्थलुब्धाय मूर्खाय साहङ्काराय पापिने~॥~२~॥

एवंविधाय पुत्रायाप्यदेयं सहजाय च~।\\
दत्तेन वेदमार्गस्य समुच्छेदः कृतो भवेत्~॥~३~॥

व्रजेतामन्धतामिस्त्रं गुरुशिय्यौ सुदारुणम्~।\\
ततः शान्ताय शुचये ब्राह्मणायैव दापयेत्~॥~४~॥}
\end{quote}



\newpage


\hspace{3cm} गूढार्थप्रकाशकेन सहितः~। \hfill ३८३
\vspace{1cm}



\begin{quote}
 {\qt चक्रानपातजो मध्यो मध्यवृत्तांशजः स्फुटः~।\\
कालेन दृक्समो न स्यात् ततो बीजक्रियोच्यते~॥~५~॥

राश्यादिरिन्दुरङ्कघ्नो भक्तो नक्षत्रकक्षया~।\\
शेषं नक्षत्रकक्षायास्त्यजेच्छेषकयोस्तयोः~॥~६~॥

यदल्पं तद्भजेद्भानां कक्षया तिथिनिघ्नया~।\\
बीजं भागादिकं तत् स्यात् कारयेत् तद्धनं रवौ~॥~७~॥

त्रिगुणं शोधयेदिन्दौ जिनघ्नं भूभिजे क्षिपेत्~।\\
दृग्यमघ्नमृणं ज्ञोच्चि खरामघ्नं गरावृणम्~॥~८~॥

ॠणं व्योमनवघ्नं स्याद्दानवेज्यचलोच्चके~।\\
धनं सप्ताहतं मन्दे परिधीनामथोच्यते~॥~९~॥

युग्मान्तोक्ताः परिधयो ये ते नित्यं परिस्फुटाः~।\\
ओजान्तोक्तास्तु ते ज्ञेयाः परबीजेन संस्कृताः~॥~१०~॥

वच्मि निर्बीजकानोजपदान्ते वृत्तभागकान्~।\\
सूर्येन्द्वोर्मनवो दन्ता धृतितत्त्वकलोनिताः~॥~११~॥

बाणतर्का महीजस्य सौम्यस्याचलबाहवः~।\\
वाक्पतेरष्टनेत्राणि त्र्योमशीतांशवो मृगोः~॥~१२~॥

शून्यर्तवोऽर्कपुत्रस्य बीजमेतेषु कारयेत्~।\\
बीजं खाग्न्युद्धृतं शोध्यं परिध्यंशेषु भास्वतः~॥~१३~॥

इनाप्तं योजयेदिन्दोः कुजस्याश्वहतं क्षिपेत्~।\\
विदश्चन्द्रहतं योज्यं सूरेरिन्द्रहतं धनम्~॥~१४~॥

धनं भृगोभ्रुवा निघ्नं रविघ्नं शोधयेच्छनेः~।\\
एवं मान्दां परिध्यंशाः स्फुटा 'स्युर्वच्मि शीघ्रकान्~॥~१५~॥}
\end{quote}

\newpage



\noindent ३८४ \hspace{4cm} सूर्यसिद्धान्तः
\vspace{1cm}




\begin{quote}
 {\qt भौमस्याभ्रगुणाक्षीणि बुधस्याब्धिगणेन्दवः~।\\
 वाणाक्षा देवपूज्यस्य भार्गवस्येन्दुषड्यमाः~॥~१६~॥
 
शनेश्चन्द्राब्धयः शीघ्ना ओजान्ते बीजवर्जिताः~।\\
द्विघ्नं स्वं कुजभागेषु बीजं द्विघ्नन्नणं विदः~॥~१७~॥

अत्त्यष्टिघ्नं धनं सूरेरिन्दुघ्नं शोधयेत् कवेः~।\\
चन्द्रघ्नमृणमार्कस्य स्युरेभिर्दृक्समा ग्रहाः~॥~१८~॥

एतद्वीजं मया ख्यातं प्रीत्या परमया तव~।\\
गोपनीयमिदं नित्यं नोपदेश्यं यतस्ततः~॥~१९~॥

परीक्षिताय शिय्याय गरुभक्ताय साधवे~।\\
देयं विप्राय नान्यस्मै प्रतिकञ्चुककारिणे~॥~२०~॥

बीजं निःशेर्षसिद्धान्तरहस्यं परमं स्फुटम्~।\\
यात्रापाणिगग्रहादीनां कार्याणां शुभसिद्धिदम्~॥~२१~॥ }
\end{quote}
%\vspace{2mm}

इत्यस्य क्वचित् पुस्तके लिखितस्य बीजोपनयनाध्यायस्यान्ते लिखितो दृश्यते तत् तु न समञ्जसम्। उत्तरखण्डे ग्रहगणितनिरूपणाभावात् तन्निरूपणप्रसङ्गनिरूपणीयस्याध्यायस्य लेखनानौचित्यात् स्पष्टाधिकारे तदन्ते वास्य लेखनस्य युक्तत्वाच्च। किञ्च 

\begin{center}
  मानानि कति किं च तैः। 
\end{center}

इति प्रश्नाग्रे प्रश्रानामभावात् प्रश्नोत्तरभूतोत्तरखण्डेऽस्य लेखनमसङ्गतम्। अपिच। उपदेशकाले वीजाभावादग्रेऽन्तरदर्शनमनियतं कथमुपदिष्टमन्यथान्तर्भूतत्वेनैवोक्तः स्यादित्यादिविचारेण केनचिद्धृष्टेन बीजस्यार्षमूललकत्वज्ञापनाया\textendash



\newpage


\hspace{3cm} गूढार्थप्रकाशकेन सहितः~। \hfill ३८५
\vspace{1cm}


\noindent न्तेऽत्र बीजोपनयनाध्यायः प्रक्षिप्त इत्यवगम्य न व्याख्यात इति मन्तव्यम्~॥~२३~॥\\
\noindent अथ मुनीन् प्रति कथितसंवादस्योपसंहारमाह\textendash 

%\vspace{2mm}

\begin{quote}
 {\ssi इत्युत्क्का मयमामन्त्र्य सम्यक् तेनाभिपूजितः~।\\
दिवमाचक्रमेऽर्कांशः प्रविवेश स्वमण्डलम्~॥~२४~॥ }
\end{quote}
%\vspace{2mm}

 सूर्यांशपुरुषो मयासुरमाम त्र्य सम्यक् तत्त्वतो ग्रहादिचरितमुपदिश्य। इति। एतत् ते इत्यादिन्नोकद्वयमुत्क्का कथयित्वा। समुच्चयार्थकश्चोऽनुसन्धेयः। दिवं स्वर्गमाचक्रमे। आक्रमणविषयं चक्रे। ननु सूर्यांशपुरुषस्य तदुपदेशे को वा पुरुषार्थ इत्यत आह\textendash तेनेति~। मयासुरेणाभिपूजितः। गन्धधूपादिनैवेद्यवस्त्रालङ्करणादिभिः पूजाविषयीकृतः। मयद्वारा मर्त्यलोके प्रसिद्धिं सूर्यतुल्यत्वेन प्राप्त इति भावः। ननु स्वर्गेऽपि किं स्थानं गत इत्यत आह\textendash प्रविवेशेति~। स्वमण्डलं सूर्यविम्बं विशति स्माधिष्ठितवान्। अत्रापि समुच्चयार्थाऽनुसन्धेयश्चकारः~॥~२४~॥\\
 \noindent अथ मयासुरावस्थां तात्कालिकीमाह\textendash

%\vspace{2mm}

\begin{quote}
 {\ssi मयोऽथ दिव्यं तञ्ज्ञानं ज्ञात्वा साक्षाद्विवस्वतः~।\\
कृतकृत्यमिवात्मानं मेने निर्धूतकल्मषम्~॥~२५~॥}
\end{quote}
%\vspace{2mm}

 अथ सूर्यांशपुरुषोऽन्तर्धानानन्तरं मयासुरस्तञ्ज्ञानं ग्रहर्क्षस्थित्यादिज्ञानं पूर्वोक्तं दिव्यं स्वर्गस्थं सूर्यात् साक्षादन\textendash



\newpage


\noindent ३८६ \hspace{4cm} सूर्यसिद्धान्तः
\vspace{1cm}


\noindent न्यद्वारेत्यर्थः। सूर्यांशपुरुषस्य सूर्याभिन्नत्वं सदुत्पन्नत्वादत एव भेदेऽपि साक्षादुक्त युक्तम्। ज्ञात्वात्मानं स्वं निर्धूतकल्मषं निवारितपापं कृतकृत्यं सम्पादितकार्यं मेने मन्यते स्म~॥~२५~॥\\
\noindent अथ त्वमिदं ज्ञानं कथं प्राप्तवानिति श्रोतृमुनिभिः पृष्ठो मुनिस्तान् प्रति तत्रत्या अस्मत्प्रभृतय ॠषयो मयं प्रत्येतञ्ज्ञानं पृष्टवन्त इत्याह\textendash 

%\vspace{2mm}

\begin{quote}
 {\ssi ज्ञात्वा तमृषयश्चाथ सूर्यलब्धवरं मयम्~।\\
परिवव्रुरुपेत्याथो ज्ञानं पप्रच्छुरादरात्~॥~२६~॥ }
\end{quote}
%\vspace{2mm}

 अथ मयासुरस्य ज्ञानप्रात्प्यनन्तरमृषयः सूर्यांशपुरुषमयासुरसंवादाश्रितभूमिप्रदेशासन्नभूमिप्रदेशस्था अस्मत्प्रभृतयो मुनयस्तं कृतकृत्यं मयासुरं सूर्यलब्धवरं सूर्यात् प्राप्तो वरो ज्ञानप्रसादो थेनैतादृशं ज्ञात्वा। उप शमीप एत्यागत्य। चः समुच्चये। परिवव्रुः वेष्टितवन्तः। अथो अनन्तरमादरादत्यन्ते  तं ज्ञानं ग्रहादिचरितं पप्रच्छुः पृष्टवन्तः~॥~२६~॥\\
 \noindent अथ मयासुरः स्वज्ञानं तत्पश्नकारकानस्मत्प्रमृतीन् मुनीन् प्रति कथयामासेत्याह\textendash

%\vspace{2mm}

\begin{quote}
 {\ssi स तेभ्यः प्रददौ प्रीतो ग्रहाणां चरितं महत्~।\\
अत्यद्भुततमं लोके रहस्यं ब्रह्मसम्मितम्~॥~२७~॥ }
\end{quote}
%\vspace{2mm}

 मयासुरः प्रीतः सन्तुष्टः सन् तेभ्योऽस्मत्प्रभृतिभ्य ऋषिभ्यो ग्रहाणां स्थित्यादिज्ञानं महदपरिभेयमत एव ब्रह्मसम्मितं



\newpage


\hspace{3cm} गूढार्थप्रकाशकेन सहितः~। \hfill ३८७
\vspace{1cm}


\noindent ब्रह्मतुल्यं लोके भूलोकेऽत्यद्भुततममत्यन्तमाश्चर्यकारकं श्रेष्ठमत एव प्रददौ प्रकर्षेण निर्व्याजतया दत्तवान् कथयामासेत्यर्थः~॥~२७~॥\\
अथ मानाध्यायसमाप्त्या सूर्यसिद्धान्तसमाप्तिं कस्यचित् प्रक्षिप्ताध्यायस्य निवारिकां फक्किकयाह\textendash 

\begin{center}
  इति सूर्यसिद्धान्ते मानाध्यायः। 

 समाप्तश्च सूर्यसिद्धान्तः~॥ 
\end{center}

\noindent स्पष्टम्। 

%\vspace{2mm}

\begin{quote}
 {\qt रङ्गनाथेन रचिते सूर्यसिद्धान्तटिप्पणे~।\\
मानाध्यायोत्तरदले पूर्णे गूढप्रकाशके~॥

भागीरथीतीरसंस्थे शम्भोर्वाराणसौपरे~।\\
बल्लालगणको रुद्रजपासक्तोऽभवद्वधः~॥~१~॥

तस्यात्मजाः पञ्च गणाभिरामा\\
ज्येष्ठः स रामः सकलागमज्ञः~।\\
येनोपपत्तिः स्वधिया नितान्तं\\
प्रकाशितानन्तसुधाकरस्य~॥~२~॥

ततः स कृष्णो जहंगिरसार्व\textendash \\
भौमस्य सर्वाधिगतप्रतिष्ठः~।\\
श्रीभास्करीयं विवृतं तु येन\\
बीजं तथा श्रीपतिपद्धतिः सा~॥~३~॥

गोविन्दसञ्ज्ञस्तु ततस्तृतीय\textendash \\
स्तस्यानजोऽहं गरुलब्धविद्यः~।}
\end{quote}



\newpage


\noindent ३८८ \hspace{4cm} सूर्यसिद्धान्तः
\vspace{1cm}

%\vspace{2mm}

\begin{quote}
{\qt विश्वेशपत्पद्मनिविष्टचेताः\\
काशीनिवासी सकलाभिमान्यः~॥~४~॥

श्रीरङ्गनाथोऽर्कमुखोत्यशास्ते\\
गूढप्रकाशाभिधटिप्पणं सः~।\\
कृत्वा महादेवबधाग्रजोऽथ\\
विश्वेश्वरायार्पितवान् सुवृड्व्यै~॥~५~॥

शके तत्त्वतिथ्यन्मिते चैत्रमासे\\
सिते शम्भुतिथ्यां बधेऽर्कोदयान्मे~।\\
दलाढ्यद्विनाराचनाडीषु जातौ~।\\
मुनीशार्कसिद्धान्तगूढप्रकाशौ~॥~६~॥

गूढप्रकाशकं दृष्ट्वा रङ्गनाथभवं भुवि~।\\
 मनीश्वरस्य सहजं लभन्तां गणकाः सुखम~॥~७~॥}
 \end{quote}
%\vspace{2mm}

 इति श्रीसकलगणकसार्वभौमबल्लालदैवज्ञात्मजरङ्गनाथविरचितः सूर्यसिंद्धान्तगूढार्थप्रकाशकः सम्पूर्णः~॥

\begin{center}
    \rule{8em}{.5pt}
\end{center}



\newpage





\hfill ३८९ 
\vspace{2cm}

\begin{center}
{\huge{ शुद्धिपत्रम्। }}
\vspace{1cm}

\rule{7em}{.5pt}
\end{center}


\begin{longtable}{p{1cm} p{1.5cm} p{4cm} p{4cm}}

पृष्ठम् & पङ्क्तिः &अशुद्धम् & शुद्धम्\\
\vspace{2mm}\\

 १ & १९ & एतस्माद्वत्मन & एतस्मादात्मन \\

 ३ & १४ & सिद्धार्थ & सिद्ध्यर्थ \\

 ५ & १ & अतएव & अतएव पुण्यं\\

 ७ & १४ & त्वां तथ्य मेव & त्वां प्रत्ययं तथ्यमेव \\

 ९ & ७ &  द्वारा त्मनः  & द्वारात्ममनः \\

 १० & ५ & सूर्यः & सूर्यः पूर्वं\\

 ११ & १७ & कालस्तत्वविद्भिः & कालतत्त्वविद्भिः \\

 १५ & १५ & सविशेषण  & सविशेष \\

 १८ & ८ & शतवर्ष & शरीर\\ 

 १८ & २१ & मनोयुगानां & मनोर्युगानां \\

 १९ & २ & सप्तभिः & सप्तभिः सन्धिभिः \\

 १९ & १६ & ग्रहगतिरूप & ग्रहगतिप्रारम्भरूप \\

 २० & २१ & प्रदेशाभ्यां & $\begin{cases}\mbox{ प्रदेशौ खासन्नविषुवद्वृत्त-}\\
\mbox{प्रदेशाभ्यां} \end{cases}$\\

 २६ & १ & यगे & युगे \\

 २७ & १४ & सूयन्दु & सूर्येन्दु \\

 ३२ & ९ & सम्पीड्य & सम्पिण्ड्य \\

 ३२ & १८ & सम्पीड्यै & सम्पियड्यै \\

 ३२ & १९ & मनोर्वमान  & मनोर्वर्त्तमान\\


\newpage




पृष्ठम् & पङ्क्तिः &  अशुद्धम् & शुद्धम्\\
\vspace{2mm}\\

 ३२ & २० & विंशतिर्गता & विंशतिं गतां\\

 ३३ & ८ & सम्पीडित & सम्पिण्डित\\

 ३३ & १३ & दिनावधिककालो & दिनाधिककाले\\

 ३४ & ९ & गताब्दा & सृष्टिगताब्दा\\

 ३५ & १ & स्वचतु & खचतु\\

 ३५ & २० & भवति & गतो भवति\\

 ३६ & ५ & श्चान्द्राधिमासाः & श्चान्द्रमासाः\\ 

 ३९ & १ & द्युत्क्रमेण & द्यत्क्रमेण\\

 ३९ & १७ &  ग्रहायन & ग्रहानयन\\

 ४० & २ & भक्तफलं & भक्तः फलं\\

 ४१ & १९ & सृष्ट्यानीत & सृष्ट्याद्यानीत \\

 ४२ & ५ & सम्पीड्ये & सम्पिण्ड्ये\\

 ४३ & १९ & ४। १७। २५। ४ & ४। १७। २५। ४८\\

 ४४ & १८ & गुणस्थान & गुणस्थैकस्थान\\

 ४४ & २० & गुणरूप & मूलरूप\\

 ४५ & १६ & तन & तेन\\

 ४५ & २० & प्रक्षिपे & प्रक्षिपेद्योजये\\

 ४६ & १८ & तापचितं & तोपचितं\\

 ५२ & १२ & यतं & युतं\\

 ५६ & १२ & प्राक & प्राक्\\

 ५७ & ४ & तयो- & तयोः।\\

 ५७ & ५ & रन्तरं & अनन्तरं\\

 ५७ & १३ & भ्रमस्तान् & भ्रमस्तान् ग्रहान्\\

 ५८ & ११ & दुच्यन्त & दुच्यत\\

 ६० & १० & णार्धस्थो & णार्धस्थः पावो\\



\newpage



पृष्ठम् & पङ्क्तिः & अशुद्धम् & शुद्धम्\\
\vspace{2mm}\\

 ६० & १३ & बध & बुध\\

 ६१ & ४ & बहत्वा & बहत्वा\\

 ६१ & १९ & शीघ्रोच्चबुध & शीघ्रोच्चाद् बुध\\

 ६२ & १७ & कर्ष्यते & कृष्यते \\

 ६२ & १९ & कर्ष्यत & कृष्यत\\

 ६३ & ११ & मर्हतीति & मर्हतीति चेन्न\\

 ६४ & ३ & अथ & अतः \\

 ६४ & ४ & चलनगतिवशात् & चलनवशात्\\

 ६५ & ६ & तथा & यथा\\

 ६५ & १२ & सङ्गत्वत & सङ्गतत्व\\

 ६८ & १२ & श्लोकद्वम् & श्लोकद्वयम्\\

 ६८ & २१ & नागाद्रि & नगाद्रि \\

 ६९ & २० &  पिण्डान & पिण्डान्\\

 ७० & ३ & दिङ्नागा & दिङ्नगा\\

 ७० & ७ & गणाश्वि & गुणाश्वि\\

 ७० & १८ & चतुविश & चतुर्विंश \\

 ७१ & ३ & $\begin{cases}\mbox{मुखासूत्रे सूत्राकारे} \\ \mbox{वृत्ताकाराक्रान्तिः}\end{cases}$ & $\begin{cases}\mbox{मुखं वृत्ताकारसूत्रे क्रान्तिः}\end{cases}$\\
 

 ७१ & ११ & केन्द्र & केन्द्रं\\

 ७१ & १५ & सम्बधेन & सम्बन्धेन\\

 ७२ & ८ & २१ & २९\\ 

 ७३ & ७ & गत & तगम्\\

 ७४ & ६ & सङ्ख्या तत्वा & सङ्ख्यातत्त्वा\\

 ७४ & १२ & परिध्यंशान & परिध्यंशान्\\

 ८२ & ५ & भिस्त & भिस्तु\\

\newpage



पृष्ठम् & पङ्क्तिः &  अशुद्धम् & शुद्धम्\\
\vspace{2mm}\\

 ८३ & ४ & फलकला & मन्दफलकला\\

 ८४ & ८ & शोद्धृताः & शोद्धृता \\

 ८५ & १२ & युतमिति & हीनमिति \\

 ८६ & २ & भुक्तिर्वक्र & शेषं वक्र \\

 ९२ & २२ & युक्तत्वात & युक्तत्वात् \\

 ९४ & ६ & येऽनांश & येऽयनांश \\

 ९७ & १२ & भक्त्या & भुक्त्या \\

 ९९ & १९ & रङ्व्रितृ & रङ्व्रिस्तृ\\

 १०० & १९ & तिथ्यध & तिथ्यर्ध \\

 १०६ & २१ & सूत्राङ्गलै & सूत्राङ्गुलै\\

 ११० & २ & रक्षमा & रक्षभा\\

 १११ & २१ & यगे & युगे \\

 ११२ & १६ & कृत्वा & कृत्वो\\

 ११५ & १६ & मध्या & मध्याह्ना\\

 ११८ & १५ & ज्या प्ता & ज्याप्ता\\

 १२२ &  ७ & व्यासार्थं & व्यासार्धं\\

 १२६ & २२ & तत्संभवात & तत्संभवात् \\

 १२८ & १७ & परिभ्रमन्ति & परिभ्रमति \\

 १३० & १४ & याव ७२ & याव ७(-२)\\

 १३४ & ८ & लम्बज्याप्तो & लम्बज्याघ्नो\\

 १४२ & १ & व्यस्ताव्यस्तै & व्यस्ता व्यस्तै\\

 १४२ & ३ & मेषादि & एते मेषादि \\

 १४४ & १९ & ओभष्ट & अभीष्ट\\

 १४८ & ६ & सम्पीड्या & सम्पिण्ड्या \\

 १४८ & ८ & सम्पीड्यौ & सम्पिण्ड्य\\

\newpage



पृष्ठम् & पङ्क्तिः &  अशुद्धम् & शुद्धम्\\
\vspace{2mm}\\

 १४८ & १६ & धय & र्धयु \\

 १५० & १४ & कालात्मक & कलात्मक \\

 १५२ & २ & तमोलिप्तास्तु & तमो लिप्तास्तु \\

 १६० & २ & ग्राससम्भवः। ग्रहणं & ग्राससम्भवो ग्रहणं \\

 १६३ & २१ & वयम & वयम् \\

 १६४ & २१ & भक्त्यन्तरं & भुक्त्यन्तरं \\

 १६५ & ११ & सूर्यप्रक्षणोक्त & सूर्यग्रहणोक्त \\

 १६८ & १ & च्छेषागता & च्छेषा गता \\

 १७१ & ९ & एवोमचय & एक उपचय \\

 १७२ & ६ & साम्येनावनते & साम्ये नावनते \\

 १७२ & १६-१७ & प्रदेशयोस्त्रिभोन & प्रदेशस्त्रिभोन \\

 १७९ & ४ & सम्भवः। नाभाव & सम्भवो न। अभाव \\

 १७५ & ११ & पर्वविनाडीनां & पर्वान्तनाडीना \\

 १७६ & ६ & साम्येन्तर & साम्येऽन्तर \\

 १७६ & ७ & धीयत & धीयते \\

 १७६ & १९ & सदृग्गतिः & स दृग्गतिः \\

 १७७ & ९ & प्रदेशः। प्राक् & प्रदेशः प्राक् \\

 १७७ & १२ & तदाग्रान्तरेण & तदग्रान्तरेण \\

 १७९ & ९ & सूर्य अ &  सूर्यः। अ \\

 १८१ & २ & पत्तरिति & पत्तेरिति \\

 १८२ & १७ & तिग्मांशो & तिग्मांश्वो \\

 १८२ & १९ & तिग्मांशो & तिग्मांश्वो \\

 १८३ & ३ & षष्ट्यक्षांशा & षट्षष्ट्यक्षांशा \\

 १८३ & २१ & भक्ता & भक्तात् \\

 १८४ & १ & ७३। २७ & ७३१ । २७ \\



\newpage



पृष्ठम् & पङ्क्तिः & अशुद्धम् & शुद्धम्\\
\vspace{2mm}\\

 १८९ & १७ & तयोः। स्पर्श & तयोः स्पर्श \\

 १८९ & २२ & तदा- & तदाऽप्य- \\

 १९२ & ९ & मध्यलम्बनात & मध्यलम्बनात् \\

 १९३ & १ & विधायासकृत् & विधायासकृज् \\

 १९७ & ५ & पश्चिमोत्तरा & पश्चिमत उत्तरा \\

 १९७ & १८ & यथायोग्यौ & यथायोग्यं \\

 १९७ & २१ & ग्रहबिम्ब & ग्राहकबिम्ब \\

 १९९ & १२ & क्षणाः & क्षिणाः \\

 २०१ & २१ & मध्यादिना & मण्यादिना \\

 २०६ & १ & ग्रहाध्वानं & ग्राहकाध्वानं \\

 २१३ & ४ & समशरग्रहयोः & समग्रहयोः \\

 २१३ & १५ & सम्पीड्यान्तर & सम्पिण्ड्यान्तरः \\

 २१४ & ५ & युक्तत्वम्। अत & युक्तत्वमत \\

 २१६ & ७ & पञ्चोनाः & पञ्चोना \\

 २१७ & १५ & यनग्रहमेवा & यनग्रहचिह्नमेवा \\

 २१९ & १ & विषुवद्वृत्तस्यायन &
विषुवद्वृत्तक्रान्तिवृत्तस्यायन \\

 २१९ & ४ & दिनाद्रविभाजित & दिनार्द्धविभाजित \\

 २१९ & १९ & दृक्कर्मादाविद & दृक्कर्मादाविदं \\

 २२० & १५ & दिक्तल्ये & दिक्तुल्ये \\

 २२० & १९ & ग्रहकला & ग्रहकाला \\

 २२१ & १९ & दधाधवधि & दर्धार्धवर्धि \\

 २२५ & ७ & खदर्प & स्वदर्य \\

 २२५ & १४ & छायाकणा & छायाकर्णौ \\

 २२५ & १६ & मूर्धगो & मूर्धगौ \\

२२५ & १६ & दिक्तुल्य & दृक्तुल्य \\

\end{longtable}


\newpage

\begin{longtable}{p{1cm} p{1.5cm} p{4.5cm} p{4.5cm}}

पृष्ठम् & पङ्क्तिः &  अशुद्धम् & शुद्धम्\\
\vspace{2mm}\\

 २२७ & ९ & हस्तावशिष्टौ & हस्ताववशिष्टौ \\

 २२८ & १९ & भिहतः & भिहितः \\

 २३० & १३ & रश्मिवान् & $\begin{cases}\mbox{रश्मिमण्डलो वा रुक्षो वा}\\
\mbox{व्यपगतरश्मिवान् }\end{cases}$\\

 २३० & १४ & समग्रहो & स ग्रहो \\

 २३२ & ६ & नुक्तत्व & युक्तत्व \\

 २३२ & ६ & ग्रक्षणो & ग्रहणो \\

 २३५ & १३ & विंशतिभाग & विंशतिभागा \\

 २३६ & २२ & ४० । ४० & ४०४० \\

 २३७ & १२ & रर्धेन & रर्धोन \\

 २३७ & १६ & ऽदृग्व्यस्ताः & दृघ्घस्ताः \\

 २३७ & १७ & श्च रस & श्वरस \\

 २३८ & ९ & षण्मासार्धैकत्रयं & षण्णां सार्धेकः। त्रयम् \\

 २३९ & १ & विक्षिप्ता उत्तरेण & विक्षिप्तावुन्तरेण \\

 २४१ & ६ & ग्रहयोग & भग्रहयोग \\

 २४५ & १७ & उदयास्तसमययोः & उदयास्तमययोः \\

 २४७ & ३ & यमात्। ग्रहस्य क्रान्तिज & यमाद्ग्रहबिम्बस्य ग्राक्क्षितिज \\

 २४७ & १७ & कालिका & कालिकौ \\

 २8७ & १८ & दिवाकरग्रहौ & दिवा चार्कग्रहौ\\ 

 २४८ & ४ & वशिष्ट & विष्ट \\

 २४८ & ६ & वशिष्ट &  विष्ट \\

 २५१ & ९ & दर्दनं & दर्शनं \\

 २६१ & १४ & कोटि & कोटिः \\

 २३१ & २१ & क्रान्ति & त्क्रान्ति \\

 २६२ & ८ & सूर्यास्तः। तत्कालिकचन्द्रस्य & सूर्यास्तस्तात्कालिकः। चन्द्रस्य \\



\newpage



पृष्ठम् & पङ्क्तिः & अशुद्धम् & शुद्धम्\\
\vspace{2mm}\\

 २६४ & ४ & त्रि १ & त्रि (-१)\\

 २६५ & २२ & ख्योक्ता। & ख्योक्ता युक्ता।\\

 २६६ & २ & शज्या & शज्यां\\

 २६६ & ६ & ज्यायाः & ज्ययोः\\

 २६६ & १७ & निर्वाहः & न निर्वाहः \\

 २६७ & ८ & गुणेन & गणेन \\

 २६८ & ७ & श्रुक्लकर्णेन  & श्रुक्लं कर्णेन \\

 २७० & २० & सा कृतिः & साकृतिः \\

 २७१ & १५ & भजं & भुजं \\

 २७२ & १८ & कारस्यैव & कारत्वस्यैव \\

 २७५ & ११ & याते वज्ञिः & यातो वज्ञिः \\

 २७६ & ५ & दृक्तल्य & दृक्तुल्य \\

 २७६ & १२ & तदवध & तदवधि \\

 २७८ & ७ & तुल्यत्वे & तत्तुल्यत्वे \\

 २८१ & ७ & कालो & पातकालो \\

 २८३ & १८ & अर्धरात्रिक & आर्धरात्रिक \\

 २८४ & १३ & अर्धरात्रिक  & अर्धरात्रिक \\

 २८७ & १७ & सम्बन्धाप्ता तत्स्थितिः & सम्बन्धात् पातस्थितिः \\

 २८९ & १० & निकट & निकटे \\

 २९३ & ८ & देशभेदं & दशभेदं \\

 २९४ & २ & व्याख्यामुत्तर & व्याख्याम्युत्तर \\

 २९४ & ३ & मनू & मनुवा \\

 २९५ & ८ & सूचनाद्यदा & सूचनादा \\

 २९५ & १८ & दभिमतो & दभितो \\
\end{longtable}


\newpage

\begin{center}
 ३९७ 
\end{center}

\begin{longtable}{p{1cm} p{1.5cm} p{4cm} p{4cm}}
पृष्ठम् & पङ्क्तिः & अशुद्धम् & शुद्धम् \\
\vspace{2mm}\\

२९६ & ७ & प्रश्नावाह & $\begin{cases}\mbox{अथ दिव्यं तदह उच्यत।}\\
\mbox{इत्यत्र सुरासुराणामन्योन्य}\\
\mbox{ महृारात्रं विपर्ययादित्यत्र}\\
\mbox{चोत्क्रमेण प्रश्नावाह }\end{cases}$\\

 २९६ & १० & पूवीर्द्धं & पूर्वार्धं पूवार्धे \\

 २९६ & १७ & पश्यन्तीह & पश्यन्तीत्यहं \\

 २९६ & १८ & प्रश्नान्तरं & प्रश्रान्तरे \\

 २९६ & १९ & तात्पर्याय प्रश्नं & तात्पर्यप्रश्रं \\

 २९७ & ५ & पूर्वाक्तेऽप्य & पूर्वोक्ते पित्र्य \\

२९७ & ६ & रात्रत्वेन & $\begin{cases}\mbox{रात्रचान्त्रमासयोस्तदहोरात्र}\\
\mbox{सूचनात। दिव्यमित्यत्र पि-}\\
\mbox{ तॄणामनुक्तेः सूर्यसावनाहो-}\\
\mbox{रात्रस्य मानुषाहोरात्रत्वेन }\end{cases}$\\

 २९७ & १२ & तेषां माने & तेषामपि \\

२९९ & ३ & मानान्युक्तानीति & $\begin{cases}\mbox{मानानि नाक्षत्रसावनचान्द्र-}\\
\mbox{सौरादीनि पूर्वोक्तानि कति}\\
\mbox{  कियन्ति। उपक्रम एव सङ्क्षे-}\\
\mbox{पेण मानान्युक्तानीति }\end{cases}$\\

 ३०० & ६ & अथ & अथ सूर्यांशपुरुषवचनानुवादे \\

 ३०० & ६ & यदुक्तं & मदुक्तं \\

 ३०० & १५ & संयोगेन & संयोगेन प्रत्यक्षं \\

 ३०० & १५ & वक्ष्यामि & वक्ष्यामीति \\

 ३०१ & १६ & सुदृप & सुतृप \\

 ३०१ & १६ & सञ्चरन्ति & सश्चरन्ति \\

 ३०२ & १० & सौवर्णं &  सौवर्णमण्डं\\
\end{longtable}


\newpage

\begin{center}
 ३९८ 
\end{center}

\begin{longtable}{p{1cm} p{1.5cm} p{4cm} p{4cm}}
पृष्ठम् & पङ्क्तिः & अशुद्धम् & शुद्धम्\\
\vspace{2mm}\\

 ३०३ & १४ & तमस्ये & दप्ये \\

 ३०५ & १० & नुत्कृवृान् & नुतकृवृान् वेदान् दत्वा \\

 ३०५ & १६ & स्तेजतां & स्तेजसां \\

 ३०७ & १ & तारा ग्रहा & ताराग्रहा \\

 ३०७ & २१ & त्मप्रकृतीः & त्मकप्रकृतीः \\

 ३०८ & १ & विस्टजन् & निर्मायन् \\

 ३१० & १ & नक्षत्रैभ्यः & नक्षत्रेभ्योऽधोधः क्रमात् \\

 ३१२ & १७ & विभागयोः & पारधि विभागयोरवधि \\

 ३१३ & ६ & गर्भका & गर्भात्मका \\

 ३१३ & १८ & वष & वर्षे \\

 ३१३ & २१ & महात्माने & महात्मानो \\

 ३१४ & २ & भूगोलमाखण्ड & भूगोलस्य खण्ड \\

३१४ & १० &$\begin{cases}\mbox{ चतुर्थांशान्तरदिक्स्था}\\ \mbox{मेरुः स चोक्तः}\end{cases}$ & $\begin{cases}\mbox{चतुर्थांशन्तरालाः प्रोतष्ठि-}\\
\mbox{ताः स्थिताः सन्तीत्यर्थः।}\\
\mbox{चकारः पूर्वोक्तेन समुच्चया-}\\
\mbox{र्थकः। ताभ्य उक्तपुरीभ्यः}\\
\mbox{सकाशादुत्तरग उत्तरदिकस्थो}\\
\mbox{मेरुः पूर्वोक्तः }\end{cases}$\\

 ३१५ & ९ & तत्र गतानां & तन्नगराणां \\

 ३१५ & १३ & क्रमेण & च क्रमेण \\

 ३१६ & २ & मित्याह & $\begin{cases}\mbox{मिति वदन् मेरावक्षांश-}\\
\mbox{परमत्वमित्याह }\end{cases}$\\

 ३१६ & ९ & वयोः क्षितिजस्य & वान्तरस्य \\

 ३१८ & ६ & इत्यस्य & इत्यस्यासन्नतया भाग इत्यस्य \\

 ३१८ & ७ & हेमर्तौ & हेमन्तर्तो \\
\end{longtable}


\newpage

\begin{center}
 ३९९ 
\end{center}

\begin{longtable}{p{1cm} p{1.5cm} p{4cm} p{4cm}}
पृष्ठम् & पङ्क्तिः & अशुद्धम् & शुद्धम्\\
\vspace{2mm}\\

 ३१८ & ९  & मनुष्णता & मत्युष्णता \\

 ३१८ & १९ & वामसव्ये & ये वामसव्ये \\

 ३१८ & १९ & तत्क्रमेण & ते क्रमेण \\

 ३१९ & १ & सव्यं & सव्यमतो \\

 ३१९ & १९ & राशीन् & त्रीम् राशीन् \\

 ३२१ & ६ & मात्मानमुपरिभागे & $\begin{cases}\mbox{मात्मानं स्वमुपरिभाग}\\
\mbox{ऊर्ध्वभागे' }\end{cases}$\\

 ३२१ & ८ & र्ध्वस्यं & र्ध्वस्थत्वं \\

 ३२१ & १५ & दैत्यै & देवदैत्यैः \\

३२१ & १७ &तुर्यचरणस्तु  & $\begin{cases}\mbox{तत्रोदाहरति। भदाश्वकेतु-}\\
\mbox{मालास्था इति। भद्राश्वकेतु-}\\
\mbox{मालशब्दौ स्वान्तर्गतयम-}\\
\mbox{  कोटिरोमकनगरविशेयाभि-}\\
\mbox{धायकौ स्पष्टभूव्यासान्तरस्थ-}\\
\mbox{त्वाङ्गीकारे तु यथाश्रुतं पर-}\\
\mbox{स्परमधोमन्यन्ते तुर्यचरणस्तु }\end{cases}$\\

 ३२२ & २ & भूगोले स गोले सर्व & भूगोले सर्व \\

 ३२२ & ६ & ब्रह्माण्डस्य भूगोला & भूगोला \\

 ३२३ & १४ & मध्यभागे & मध्यभागो \\

 ३२४ & १० & दिनरात्र्योर्वा योग्ये इत। & दिनेरात्रौवायथायोग्यमिति \\

 ३२४ & १६ & त्वक्षया हानि &  च क्षपाहानि \\

 ३२५ & २ & गमना परावर्तने & गमनात् परावर्तते \\

 ३२५ & ३ & क्षयो हानी & क्षयाहानी \\

 ३२५ & १०-११ & पूर्वं विस्तारयति & पूर्वोक्तं स्मारयति \\

 ३२६ & ७ & राशिभोगैः & राशिभागैः \\
\end{longtable}


\newpage

\begin{center}
   ४००
\end{center}

\begin{longtable}{p{1cm} p{1.5cm} p{4cm} p{4.5cm}}

 पृष्ठम् & पङ्क्तिः & अशुद्धम् & शुद्धम्\\
\vspace{2mm}\\

 ३२६ & १३ & दिनमानायन & दिनमानानयन \\

 ३२६ & १८ & निशा यस्मिन् & निशाप्यस्मिन् \\

 ३२७ & २१ & पूर्ति & पूर्तिः \\

 ३२८ & ९ & गोले & गोलो \\

 ३२८ & १० & सम्बन्धी गणित & सम्बन्धिगणित \\

 ३२८ & १२-१३ & स्थितं तं & स्थितिं \\

 ३२८ & १४ & ऊनभूवृत्त & ऊने भूवृत्त \\

 ३२८ & १८ & पूर्वागतै & पूर्वावगतै \\

 ३३१ & १० & इति & इत्यत्र \\

 ३३१ & १० & गुणो भगणांशमितहरो & गुणभगणांशमितहरौ \\

 ३३१ & २० & मुत्तरमुन्नतांशाना & मुत्तरम्। नतांशाना \\

 ३३२ & १ & स्यात्तरत्वे & स्योत्तरस्थत्वे \\

 ३३२ & ३ & मुखमेवाग्रयोः & मुखं मेर्वग्रयोः \\

 ३३२ & १६ & रोमसिद्ध & रोमकसिद्ध \\

 ३३३ & ४  & स्वस्याभिमत & स्वस्वाभिमत \\

 ३३४ & २१ & मध्याख्य' & मध्यस्थ \\

३३५ & ८ & नरा भूमौ  & $\begin{cases}\mbox{तथा पितरचन्द्रोबिम्बगोलो-}\\
\mbox{ र्ध्वस्थिताः पक्षं पञ्चदशति-}\\
\mbox{थिपर्यन्तं पश्यन्ति। नरा भूमौ }\end{cases}$\\

 ३३५ & ८ & पश्यन्ति & पश्यन्त्यतः \\

 ३३६ & ७ & कक्षान्तर्गमनं & कक्षान्तर्गतत्वेन \\

 ३३७ & १७ & ऊर्ध्वक्रमेण & शनेः सकाशादधः कक्षाक्रमेण \\

 ३३८ & १२ & ग्रहृस्याधिपत्वा & ग्रहस्यावधित्वा \\

 ३३८ & १२ & तनु & न तु \\

 ३३९ & १३ & भावस्थानेऽप्यत्यूर्ध्वं & भावाच्छनेरप्यत्यूर्ध्वं\\ 
\end{longtable}


\newpage

\begin{center}
   ४०१
\end{center}

\begin{longtable}{p{1cm} p{1.5cm} p{4cm} p{4cm}}
 पृष्ठम् & पङ्क्तिः &  अशुद्धम् & शुद्धम्\\
\vspace{2mm}\\

 ३४१ & ९ & खकक्षा & स्वकक्षा \\

 २४१ & ११ & खकक्षया & स्वकक्षया \\

 ३४१ & १८ & ग्रहोच्चं & ग्रहोच्च्यम् \\

 ३४२ & ७ & तत्रा & ग्रहस्य तत्रा \\

 ३४२ & २० & मुक्तम् & मुक्ता \\

 ३४३ & ४ & मुक्ताः & मुक्ता \\

 ३४३ & १९ & द्व &  द्वो \\

 ३४४ & ८ & नवतितत्त्वमिता & नवतत्त्वानि। यद्यपि \\

 ३५० & ३ & स्वकीयैः & स्वकीयैः स्वकीयैः \\

 ३५० & ७ & यवर्गतत्यात् & मसर्गत्वात् \\

 ३५१ & ८ & सप्तविंशति & सप्तोर्ध \\

 ३५१ & १४ & शिल्पकः & शिल्पज्ञः \\

 ३५१ & २१ & तयोय तेः & तयोर्युतेः \\

 ३५५ & १२ & स्वमध्यं  & खमध्यं \\

 ३५५ & १९ & विषवत् & विषुवत् \\

 ३६० & १४ & त्रयेण गोचर & श्रवणगोचर \\

 ३७० & ६ & पित्र्य & दिव्यपित्र्य \\

 ३७३ & ९ & तथव & तथैव \\

 ३७४ & ११ & पालकालः & स्थूलकालः \\
\end{longtable}

\begin{center}
    \rule{8em}{.5pt}
\end{center}



 \end{document}