\documentclass[11pt, openany]{book}
\usepackage[text={4.65in,7.45in}, centering, includefoot]{geometry}

\usepackage[table, x11names]{xcolor}
%\include{alias}

\usepackage{fontspec,realscripts}
\usepackage{polyglossia}

\setdefaultlanguage{sanskrit}
\setotherlanguage{english}
%\setmainfont[Scale=1]{Times New Roman}
%\newfontfamily\regular[Scale=1]{Times New Roman}
\defaultfontfeatures[Scale=MatchUppercase]{Ligatures=TeX} 
\newfontfamily\sanskritfont[Script=Devanagari]{Shobhika}
\newfontfamily\englishfont[Language=English, Script=Latin]{Linux Libertine O}
\newfontfamily\ssi[Script=Devanagari, Color=purple]{Shobhika-Bold}
\newfontfamily\qt[Script=Devanagari, Scale=1, Color=violet]{Shobhika-Regular}
%\newfontfamily\bqt[Script=Devanagari, Scale=1, Color=brown]{Shobhika-Regular}
%\newfontfamily\s[Script=Devanagari, Scale=0.9]{Shobhika-Regular}
\newfontfamily\s[Script=Devanagari, Scale=0.9]{Shobhika-Regular}
\newcommand{\devanagarinumeral}[1]{%
	\devanagaridigits{\number\csname c@#1\endcsname}}
\usepackage{fancyhdr}
\pagestyle{fancy}
\renewcommand{\headrulewidth}{0pt}
%\newfontfamily\e[Scale=0.8]{Shobhika-Regular}
\XeTeXgenerateactualtext=1
\usepackage{enumerate}
%\pagestyle{plain}
%\pagestyle{empty}
\usepackage{afterpage}
\usepackage{mathtools}
\usepackage{amsmath}
\usepackage{amssymb}
\usepackage{tikz}
\usepackage{graphicx}
\usepackage{longtable}
\usepackage{multirow}
\usepackage{footnote}
%\usepackage{dblfnote} 
\usepackage{xspace}
%\newcommand\nd{\textsuperscript{nd}\xspace}
\usepackage{array}
\usepackage{emptypage}
\usepackage{hyperref}   % Package for hyperlinks
\hypersetup{
	colorlinks,
	citecolor=black,
	filecolor=black,
	linkcolor=blue,
	urlcolor=black
}

\begin{document} 

\cfoot{}

\begin{center}
{BIBLIOTHECA INDICA;}

\vspace{1cm}


{COLLECTION OF ORIENTAL WORKS }

\vspace{3mm}

{\small{PUBLISHED UNDER THE PATRONAGE OF THE}}\\
\vspace{3mm}

 Hon. Court of Directors of the East India Company,
\vspace{3mm}
 

{\small{AND THE SUPERINTENDENCE OF THE}}
\vspace{3mm}


 ASIATIC SOCIETY OF BENGAL.
\vspace{1cm}


 \rule{8em}{.5pt}

\vspace{1cm}

 THE SŪRYA-SIDDHĀNTA, 
 \vspace{3mm}
 
{\small{AN ANCIENT}} 
\vspace{3mm}

 SYSTEM OF HINDU ASTRONOMY;
\vspace{3mm}

{\small{WITH }}
\vspace{3mm}

 RA\d{N}GANĀTHA'S EXPOSITION, 
\vspace{3mm}

 THE GŪḌHĀRTHA-PRAKĀŚAKA.
 \vspace{3mm}

{\small{ EDITED BY FITZEDWARD HALK, M. A., }}
 \vspace{3mm}
 
{\small{ WITH THE ASSISTANCE OF}}
\vspace{3mm}

{\small{ PANDIT BĀPŪ DEVA ŚĀSTRĪN,}}
\vspace{3mm}

{\small{\textenglish{\emph{Mathematical Professor in the Benares Government College.}}}}




\vfil


  \rule{8em}{.5pt}

CALCUTTA:
\vspace{3mm}

{\small{PRINTED BY C. B. LEWIS, BAPTIST MISSION PRESS }}
\vspace{3mm}

{\small{1859.}}

\end{center}

\newpage


\begin{center}
PREFACE

\rule{4em}{.5pt}
\end{center}
\vspace{0.6cm}

 For the present edition of the  {\textenglish{\emph{Sūrya-siddhānta}}} and commentary nine MSS. have been collated. Of the mere text I had four copies:

{\textenglish{\emph{A}}}. Borrowed from Pandit Rājājī, of Benares: venerable in appearance, and almost without an error; but defective, in the first part, by the whole of the opening chapter and about half of the second.

 {\textenglish{\emph{B}}}. Belonging to Travāḍī Lajjāśaṅkara Pandit, of Benares: generally correct.

 {\textenglish{\emph{C}}}. From the library of the Asiatic Society of Bengal, being No. 81 : of little value : in the Ba\d{n}gālī character.

{\textenglish{\emph{D}}}. Likewise from the library of the Asiatic Society of Bengal, being No. 77 : almost useless : in the same character
with {\textenglish{\emph{C}}}.

 Of the commentary, exhibiting the text at large, I used five
copies, of which three are more or less imperfect; namely :

{\textenglish{\emph{a}}}. It belongs to Pandit Nārāyaṇa Bhaṭṭa, of Benares : moderately correct.

 {\textenglish{\emph{b}}}. Procured from Travāḍī Lajjāśa\d{n}kara Pandit, of Benares : much more faulty in the first part than in the second.

{\textenglish{\emph{c}}}. A MS. belonging to the Asiatic Society of Bengal, being No. 81 : in the first part, nearly related to the corresponding portion of b ; in the latter, wanting all the thirteenth chapter but its beginning and end : of scarcely any worth : in the Ba\d{n}gālī character.

 {\textenglish{\emph{d}}}. Lent to me by Paṇḍyā Indrajī Durlabhajī, of Benares : contains only the first part ; and of thus much the seventh and

\newpage


\begin{center}
 iv
\end{center}


\noindent eighth chapters were not procurable till after they were printed : very old, and extremely correct. 

 {\textenglish{\emph{e}}}. The property of Chintāmaṇi Jos'i, of Gwalior : the latter part only : not available before nearly the whole work was in type : the best MS. to which I have had access. None of the nine MSS. above described was dated. All the variants which they supply, deserving of the slightest account, are given at the foot of the page, or at the end of the volume.

 The original of part of the first chapter of the {\textit{Sūrya-siddhānta}}, and of all the eighth, has already been printed, together with a French interpretation, in the Abbé J. M. F. Guerin's {\textenglish{\emph{Astronomic Indienne}}}, published in Paris in 1847. A translation, in our own language, of the commencement of the work will be found in the {\textenglish{\emph{Asiatic Journal}}} for the months of May and June, 1817.
I have prepared an English version of a considerable share of the {\textenglish{\emph{Sūrya-siddhānta}}} ; and it was my intention to carry the undertaking to an end. Ill health and other circumstances compel me, however, to abandon the design, probably not to be resumed.

\vspace{1cm}

{\textenglish{\emph{Calcutta, Good Friday, 1859.}}}

\newpage


\begin{center}

CONTENTS.

\vspace{3mm}

 \rule{8mm}{.5pt}
\vspace{3mm}
 
 PART I. 
\end{center}

\hfill \textit{Page}

\noindent CHAP. I---Rule"s for finding the mean places of planets,.....  \hfill 1

\hspace{1em} II---Rules for finding the true places of planets,....... \hfill 55 

\hspace{1em} III---Rules for solving Problems concerning the points

\hspace{2em} of the Horizon, the position of places, and time,...... \hfill 102

\hspace{1em} IV---Of Lunar-Eclipses, .................................................... \hfill149

\hspace{1em} V---Of Solar Eclipses, ......................................................\hfill 172 

\hspace{1em} VI---On the Projection of Lunar and Solar Eclipses, ...... \hfill 193

\hspace{1em} VII---On the conjunction of planets,
.............................. \hfill 209

\hspace{1em} VIII---On the conjunction of planets with stars,............. \hfill 233

\hspace{1em} IX---On the rising and setting of planets,......................\hfill 245

\hspace{1em} X---On the Phases of the Moon and the position of

\hspace{2em} the Moon's cusps, ........................................................\hfill 257 

\hspace{1em} XI--Rules for finding the time at which the declin-

\hspace{2em} ations of the Sun and Moon become equal, .............. \hfill 273

\begin{center}
   PART II. 
\end{center}

\hspace{1em} XII---Matters cosmographical, .........................................\hfill 294

\hspace{1em} XIII.---On the construction of the armillary sphere and

\hspace{2em} other astronomical instruments, .................................\hfill 346

\hspace{1em} XIV--Of Measures of time, .............................................. \hfill 369
\vspace{3mm}

\begin{center}
     \rule{8em}{.5pt}
\end{center}


\newpage

\begin{center}

 अनुक्रमणिका ।
\vspace{3mm}

 \rule{7em}{.5pt}
\vspace{3mm}
 
 पूर्वखण्डम् ।
\end{center}

\hfill पृष्ठम् 

 १. मध्यमाधिकारः .. .. .. .. .. .. .. .. .. .. .. .. .. .. \hfill १
 
२. स्पष्टाधिकारः .. .. .. .. .. .. .. .. .. .. .. .. .. .. .. \hfill ५५

३. त्रिप्रश्नाधिकारः .. .. .. .. .. .. .. .. .. .. .. .. .. ..  \hfill १०२

४. चन्द्रग्रहणाधिकारः .. .. .. .. .. .. .. .. .. .. .. .. ..  \hfill १४९

५. सूर्यग्रहणाधिकारः .. .. .. .. ..  .. .. .. .. .. .. .. .. \hfill १७२

६. परिलेखाधिकारः .. .. .. .. .. .. .. .. .. .. .. .. .. .. \hfill १९३

७. ग्रहयुत्यधिकारः .. .. .. .. .. .. .. .. .. .. .. .. .. .. \hfill २०९

८. नक्षत्रग्रहयुत्यधिकारः .. .. .. .. .. .. .. .. .. .. .. ..  \hfill २३३

९. उदयास्ताधिकारः .. .. .. .. .. .. .. .. .. .. .. ..  .. ..\hfill २४५


१०. शृङ्गोन्नत्यधिकारः .. .. .. .. .. .. .. .. .. .. .. .. .. ..\hfill २५७

११. पाताधिकारः .. .. .. .. .. .. .. .. .. .. .. .. .. .. .. .. \hfill २७३ 

\begin{center}

 उत्तरखण्डम् । 
\end{center}


१२. भूगोलाध्यायः .. .. .. .. .. .. .. .. .. .. .. .. .. .. .. \hfill २९४

१३. ज्योतिषोपनिदध्यायः .. .. .. .. .. .. .. .. .. .. .. .. \hfill ३४६

१४. मानाध्यायः .. .. .. .. .. .. .. .. .. .. .. .. .. .. .. .. \hfill ३६९ 

 
\newpage


\begin{center}
 परमात्मने नमः ~।
 \vspace{3mm}
 
गूढार्थप्रकाशकेन सहितः
\vspace{3mm}

सूर्यसिद्धान्तः ~।
\end{center}
\begin{quote}

{\ssi  यत्स्मृत्याभीष्टकार्यस्य निर्विघ्नां सिद्धिमेष्यति ~।\\
नरस्तं बुद्धिदं वन्दे वक्रतुण्डं शिवोद्भवम् ~॥

पितरौ गोजिबल्लालौ जयतोऽम्बाशिवात्मकौ ~।\\
याभ्यां पञ्च सुता जाता ज्योतिःसंसारहेतवः ~॥

सार्वभौमजहांगीरविश्वासास्पदभाषणम् ~।\\
यस्य तं भ्रातरं कृष्णं बुधं वन्दे जगद्गुरुम् ~॥

नानाग्रन्थान् समालोच्य सूर्यसिद्धान्तटिप्पणम् ~।\\
करोमि रङ्गनाथोऽहं तद्गूढार्थप्रकाशकम् ~॥}
\end{quote}
%\vspace{2mm}

%\begin{sloppypar}

 अथ ग्रहादिचरितजिज्ञासून् मुनींस्तत्प्रश्नकारकान् प्रति स्वविदितं यथार्थतत्त्वं सूर्यांशपुरुषमयासुरसंवादं वक्तुकामः  कश्चिदृषिः प्रथममारम्भणीयतत्कथननिर्विघ्नसमाप्त्यर्थं कृतं ब्रह्मप्रणाममङ्गलं शिष्यशिक्षायै निबध्नाति।

%\end{sloppypar}
%\vspace{2mm}
\begin{quote}
  
{\ssi अचिन्त्याव्यक्तरूपाय निर्गुणाय गुणात्मने ~।\\
समस्तजगदाधारमूर्तये ब्रह्मणे नमः ~॥~१~॥}
\end{quote}
%\vspace{2mm}

%\begin{sloppypar}

 ब्रह्मणे बृहत्त्वादपरिच्छिन्नत्वाज्जगद्व्यापकायेश्वराय तस्माद्वा एतस्माद्वात्मन आकाशः सम्भूत इत्यादि श्रुतिप्रतिपाद्यायेत्यर्थः। नमः कायवाक्चेष्टोपलक्षितेन मानसेन्द्रियबुद्धिविशेषेण मत्त \textendash
%\end{sloppypar}

\newpage


\noindent  २ \hspace{4cm} सूर्यसिद्धान्तः 

\vspace{1cm}

%\begin{sloppypar}
\noindent स्त्वमुत्कृष्टस्त्वत्तोऽहमपकृष्ट इत्यादिरूपेण नतोऽस्मीत्यर्थः। ननु व्यापकत्वेनानकाशस्यैव सिद्धिरत आह\textendash समस्तजगदाधारमूर्तय इति~। समस्तस्य स्थावरजङ्गमात्मकस्य जगत उत्पत्तिस्थिति विनाशवत आधारा आश्रयभूता ब्रह्मविष्णुशिवरूपा मूर्तयः स्वरूपाणि यस्य तस्मै ब्रह्मविष्णुशिवात्मकायेत्यर्थः। आकाशस्य तदात्मकत्वाभावान्न सिद्धिरिति भावः। नन्वेतादृशस्य स्वरूप ध्यानं कर्तुं समुचितमित्यत आह। अचिन्त्याव्यक्तरूपायेति। अचिन्त्यश्चासावव्यक्तरूपस्तस्मै। अचिन्त्यो ध्यानाविषयः। अत्र हेतुरव्यक्तरूपः। न व्यक्तं प्रकटं रूपं स्वरूपं यस्य तथा च स्वरूपध्यानासम्भवान्नमस्कार एव समुचित इति भावः। नन्वव्यक्तरूपः कथमित्यत आह\textendash निर्गुणायेति~। निर्गता गुणाः सत्त्वरजस्तमोरूपा यस्मात् तस्मै गुणातीतायेत्यर्थः। तथा च गुणात्मकस्य व्यक्तरूपत्वेनायं तदभावादव्यक्तरूप इति भावः। नन्वेवमस्यारूपित्वमेव फलितं नाव्यक्तरूपित्वमित्यत आह\textendash गुणात्मन इति~। गुणा नित्यज्ञानसुखादय आत्मगुणा आत्मस्वरूपं यस्य तस्मै नित्यज्ञानसुखाय। सत्यं ज्ञानमनन्तं ब्रह्मेति श्रुतेरित्यर्थः। तथा चास्य रूपित्वमसिद्धमिति भावः। साक्षान्निर्गुणाय परम्परया गुणात्मने। कथमन्यथा जगत्कर्तृत्वं सम्भवति।
%\end{sloppypar}
%\vspace{2mm}


\begin{quote}
{\ssi प्रकृतिं स्वामवष्टभ्य विसृजामि पुनः पुनः ~।\\
भूतग्राममिमं कृत्स्नमवशः प्रकृतेर्वशात् ~॥\\
इति भगवदुक्तेरित्यन्ये ~॥~१~॥} अथ स्वोक्तस्य स्वकल्पितत्व\textendash
\end{quote}

\newpage


\hspace{3cm} गूढार्थप्रकाशकेन सहितः~। \hfill ३
\vspace{1cm}
 
%\begin{sloppypar}
\noindent शङ्कावारणाय तत्संवादोपक्रमं विवक्षुः प्रथमं मयासुरेण तपस्तप्तमिति श्लोकाभ्यामाह\textendash
%\end{sloppypar}
%\vspace{2mm}


\begin{quote}
{\ssi अल्पावशिष्टे तु कृते मयनामा *महासुरः ~।\\
रहस्यं परमं पुण्यं जिज्ञासुर्ज्ञानमुत्तमम् ~॥~२~॥

वैदाङ्गमग्र्यमखिलं ज्योतिषां गतिकारणम् ~।\\
आराधयन् विवस्वन्तं तपस्तेपे सुदुश्चरम् ~॥~३~॥}
\end{quote}
%\vspace{2mm}

\begin{sloppypar}
 मयेति नाम यस्यासौ मयाख्यो महादैत्यः कश्चित्। तपोऽभिमतदेवताप्रीतिकरजपहोमध्यानादिना स्वशरीरादिक्लेशनियमरूपं तेपे कृतवान्। दैत्यानां तपश्चरणं पुराणेषु प्रतिपदं सुप्रसिद्धम्। ननु तत्र तेषां तपश्चरणस्य देवताविशेषमभिमतमुद्दिश्य प्रसिद्धेरनेन कं देवमुद्दिश्य तपस्तप्तमित्यत आह\textendash आराधयन्निति। विवस्वन्तं सवितृमण्डलाधिष्ठातारं नारायणं सेवयन्। ननु दैत्यारिमेनं स्वशत्रुं ज्ञात्वाप्ययं कथं स्वाभिमतसिद्धार्थमारराध। न हि स्वशत्रुतः स्वहितसिद्धिरन्यथा शत्रुत्वव्याघात इत्यतस्तपोविशेषणमाह\textendash सुदुश्चरमिति। सुतरां दुःखैरत्यन्तक्लेशैश्चरितुं कर्तुं शक्यमित्यर्थः। तथा च भक्तजनैकवत्सलतया तादृशतपश्चरणसुप्रसन्नो दैत्यानामप्यभिमतं पूरयतीति पुराणेषु शतशः प्रसिद्धम्। अतस्तत्प्रतीत्याराधयन्निति भावः। ननु पुराणेषु दैत्यानां तपश्चरणोक्तिप्रसङ्गे क्वचिदप्यस्यानुक्तेस्तत्तपश्चरणं कथं प्रमाणं ज्ञेयमित्यत आह\textendash अल्पावशिष्ट इति~।
\end{sloppypar}

\noindent\rule{\linewidth}{.5pt}

\begin{center}
* मयो नाम इति पाठान्तरम्।
\end{center}
 
{\tiny{B 2}}

\newpage


\noindent ४ \hspace{4cm} सूर्यसिद्धान्तः 
\vspace{1cm}
 
%\begin{sloppypar}

\noindent कृते कृताख्ये युगचरणे तुकारात् सन्ध्यासन्ध्यांशसहित इत्यर्थः। तेन सन्ध्यासन्ध्यांशसमेतकेवलकृतरूपाभिमतकृतचरणे। न ग्रन्थान्तरोक्तकेवलकृत इति पर्यवसन्नम्। अल्पकालेन सन्ध्यां शान्तर्गतेन शेषिते। समाप्त्यासन्नाभिमतकृतयुगे मयासुरेण तपस्तप्तमित्यर्थः। तथा च साम्प्रतमेव मयासुरेण तपस्तप्तमिति सर्वजनावगतप्रत्यक्षप्रमाणसिद्धं नागमान्तरप्रामाण्यमपेक्षत इति भावः। ननु मयासुरेण किमर्थं तपस्तप्तं न हि प्रयोजनमनुद्दिश्य मन्दोऽपि प्रवर्तत इत्यतो मधासुरविशेषणमाह\textendash जिज्ञासुरिति~। ज्ञायतेऽनेनेति ज्ञानं शास्त्रं ज्ञातुमिच्छुः। तथा च शास्त्रज्ञाननिमित्तं तेन तपस्तप्तमिति भावः। किं तच्छास्त्रमित्यतो ज्ञानविशेषणमाह\textendash ज्योतिषामिति~। प्रवहवायुस्थानां ग्रहनक्षत्राणां गतिकारणम्। ये गत्यर्थास्ते ज्ञानार्था इति गतेः संस्थानचलनमानादिज्ञानस्य कारणं प्रतिपादकं ज्योतिःशास्त्रं जिज्ञासुरिति फलितम्। ननु ज्योतिःशास्त्रज्ञानार्थमयमायासो च युक्तस्तस्य सर्वर्षिज्ञेयत्वेनादुरूहत्वादित्यत आह\textendash अखिलमिति~। समग्रं ज्योतिःशास्त्रमित्यर्थः। तथा चर्षीणां मानुषत्वेनैभ्यो मम ज्ञानमखिलं यथार्थं वा न भविष्यतीति दैत्यबुद्व्या मत्वा निःशेषज्योतिःशास्त्रस्य दुरूहस्य विदिततत्त्वं भगवन्तमप्रतारकं सर्वज्ञं महागुरुं सेवयामासेति भावः। ननु तस्यासुरस्य ज्योतिःशास्त्रप्रवृत्तिर्न युक्ता फलाभावादित्यत आह\textendash वेदाङ्गमिति~। वेदस्याङ्गम्। तथा चाङ्गिनो यत् फलं तदेवाङ्गस्येति मोक्षरूपफलसद्भावादत्र प्रवृत्तिर्युक्तेति भावः।
%\end{sloppypar}

\newpage

 
 \hspace{3cm}गूढार्थप्रकाशकेन सहितः~।  \hfill ५
 \vspace{1cm}

%\begin{sloppypar}
\noindent अत एव पुण्यजनकं पुराणन्यायेत्यादिचतुर्दशविद्यान्तर्गतत्वात्। नन्विदं वेदाङ्गं कुत इत्यत आह\textendash परममिति~।
%\end{sloppypar}
%\vspace{2mm}


\begin{quote}
  {\ssi कालोऽयं भगवान् विष्णुरनन्तः परमेश्वरः ~।\\
तद्वेत्ता पूज्यते सम्यक् पूज्यः कोऽन्यस्ततो मतः ~॥}
\end{quote}
%\vspace{2mm}

\begin{sloppypar}
 इत्युक्तेः कालप्रतिपादकत्वेनोत्कृष्टमतो वेदाङ्गम्। एतेन पुराणादीनां निरास इति भावः। ननु व्याकरणादीनां षण्णां वेदाङ्गत्वादस्मिन्नेव प्रवृत्तिः कथमित्यत आह\textendash अग्र्यमिति~। षण्णां वेदाङ्गानां मध्ये श्रेष्ठम्। कुत इत्यत आह\textendash उत्तममिति~। मुख्याङ्गं नेत्रमित्यर्थः। तथा च नेत्ररहितस्याकिञ्चित्करत्वादिदं ज्योतिःशास्त्रं वेदाङ्गेषु श्रेष्ठमिति भावः। ननु तथाप्येतस्य ज्ञानार्थमेतावानायासो न युक्त इत्यत आह\textendash रहस्यमिति~। विद्या ह वै ब्राह्मणमाजगाम गोपायमाशेवधिष्ठेऽहमस्मि असूयकायानृजवे यताय न मां ब्रूयादवीर्यवती तथा स्याम् इति श्रुत्युक्तेर्गोप्यमित्यर्थः। तथा चास्य शास्त्रस्यादेयत्वेन निश्चितत्वादनेन तत्प्राप्त्यर्थमेतावानप्यायासः कृत इति भावः ~॥~३~॥\\ \noindent ततस्तुष्टोऽर्को मयायेदं दत्तवानित्याह\textendash
\end{sloppypar}
%\vspace{2mm}


\begin{quote}
 {\ssi तोषितस्तपसा तेन प्रीतस्तस्मै वरार्थिने ~।\\
ग्रहाणां चरितं प्रादान्मयाय सविता स्वयम् ~॥~४~॥}
%\vspace{2mm}
\end{quote}
%\begin{sloppypar}
 स्वयं स्वतः प्रीतः सुखरूपः। यद्वा शोभनोऽयं प्रत्यक्षः प्रीतः सन्तुष्टोऽपि सन् सविता सवितृमण्डलमध्यवर्ती। तेन सुदुश्चरेण तपसाराधनेन तोषितः। अत्यन्तं सन्तुष्टः। तस्मै।
%\end{sloppypar}

\newpage


\noindent ६ \hspace{4cm} सूर्यसिद्धान्तः
 \vspace{1cm}
 
%\begin{sloppypar}
\noindent असुराय मयनाम्ने। वरार्थिने वरं स्वाभिमतं ज्योतिःशास्त्रमर्थयते ज्ञातुमिच्छति तस्मै ज्योतिःशास्त्रजिज्ञासवे। ग्रहाणां प्रवहवायुस्थग्रहताराणाम्। चरितं ज्ञानं प्रादात्। प्रकर्षेण साकल्येन यथार्थतत्त्वेनादाद्दत्तवान् ~॥~४~॥\\
\noindent नन्वयं सूर्यः स्वकार्यार्थं शरणागतमपि स्वशत्रुं प्रति कथमिदमुक्तवानित्यतो मयं प्रति साक्षात सूर्येणोक्तस्य वचनस्यानुवादार्थमुद्यतः प्रथमं तत्सङ्गतिप्रदर्शकमेतदाह\textendash
%\end{sloppypar}

\begin{center}
 श्रीसूर्य उवाच ।
\end{center}

\begin{sloppypar}
 इति। तेजःसमूहैर्देदीप्यमानोऽर्को मयासुरं प्रत्यवददित्यर्थः। अन्यथा चतुर्थपञ्चमश्लोकयोः सङ्गत्यनुपपत्तेः। किमुवाचेत्यतस्तद्वचनमनुवदति\textendash
\end{sloppypar}
%\vspace{2mm}


\begin{quote}
 {\ssi विदितस्ते मया भावस्तोषितस्तपसा ह्यहम् ~।\\
दद्यां कालाश्रयं ज्ञानं ग्रहाणां चरितं महत् ~॥~५~॥}
\end{quote}
%\vspace{2mm}

%\begin{sloppypar}
 हे मयासुर ते तव भावो मनोरथो ज्योतिःशास्त्रजिज्ञासारूपः। मया सूर्येण विदितस्त्वदकथितोऽपि स्वतो ज्ञातः। ततः किं न ह्येतावता मम तत्सिद्धिरत आह\textendash अहमिति~। ते इत्यस्यावृत्तेस्ते तुभ्यं ज्ञानं शास्त्रं कालाश्रयं कालप्रधानम्। ग्रहाणां प्रवहवायुस्थानां महदपरिमेयं चरितम्। माहात्म्यम्। ग्रहस्थितिचलनादिप्रतिपादकज्योतिःशास्त्रमिति फलितार्थः। अहं सूर्यमण्डलस्थः। दद्यां दास्यामि। ननु मां दैत्यं प्रतीदं वाक्यं प्रतारकं भविष्यतीत्यतः स्वविशेषणमप्रतारण\textendash
%\end{sloppypar}

\newpage

\hspace{3cm} गूढार्थप्रकाशकेन सहितः ~। \hfill ७
\vspace{1cm}

%\begin{sloppypar}
\noindent पूर्वकतत्कथने हेतुभूतमाह। तोषित इति। हि यतस्तपसात्वत्कृताराधनेनात्यन्तं सन्तुष्टोऽतो दद्यामित्यर्थः। तथा च त्वत्कर्मवश्येन मया भक्तजनवत्सलतया जातिवैरमुपेक्ष्यानुकम्पितप्रह्लादवत् त्वमप्रतार्योऽनुकम्पित इति भावः ~॥~५~॥\\
\noindent ननु सूर्यस्य सदा जाज्वल्यमानतया तत्सन्निधौ श्रवणकालपर्यन्तं मयः स्थातुं कथं शक्तः कथं वानवरतभ्रमस्य तस्य मयसंवादार्थं भ्रमणविच्छेदः सम्भवति। अतो दानासम्भवात् कथं दद्यामित्युक्तमित्यतस्तद्वचनान्तरमनुवदति\textendash
%\end{sloppypar}
%\vspace{2mm}


\begin{quote}
{\ssi न मे तेजःसहः कश्चिदाख्यातुं नास्ति मे क्षणः ~।\\
मदंशः पुरुषोऽयं ते निःशेषं कथयिष्यति ~॥~६~॥}
%\vspace{2mm}
\end{quote}
\begin{sloppypar}
 हे मय ते तुभ्यमयमग्रस्थः पुरुषो निःशेषं सम्पूर्णं ज्योतिःशास्त्रं कथयिष्यति। नन्वयं तथ्यं न वदिष्यतीत्यत आह\textendash मदंश इति~। मम सूर्यस्यांशः सम्बन्धी मदुत्पन्न इत्यर्थः। तथा च मदनुकम्पितं त्वां तथ्यमेव वदिष्यतीति भावः। एतेनाहं स्वांशद्वारा दास्यामीत्यर्थो दद्यामिति पूर्वपद्योक्तस्य प्रकटीकृतः। ननु त्वयैव वक्तव्यमित्यत आह\textendash नेति~। कश्चिदपि जीवो मे सूर्यमण्डलस्थस्य तेजःसहस्तेजोधारको न। तथा च बहुकालं मत्समीपे स्थातुमशक्तस्त्वं कथं मत्तः श्रोष्यसीति भावः। ननु स्वतपः सामर्थ्येनाहं त्वत्समीपे बहुकालं स्थातुं शक्तस्त्वत्तः श्रोत्यामीत्यत आह\textendash आख्यातुमिति~। मे सूर्यमण्डलस्थस्य प्रवहवायुनानवरतं भ्रममाणस्य स्वशक्त्या कदापि अस्थिरस्य कथयितुं
\end{sloppypar}

\newpage


\noindent ८ \hspace{4cm} सूर्यसिद्धान्तः 
\vspace{1cm}

%\begin{sloppypar}
\noindent क्षणः कालो नास्ति। भ्रमणावसानासम्भवेनैकत्र स्थित्यसम्भवात्। तथा च स्थिरस्य तव बहुकालं मत्सङ्गासम्भवान्मत्तः श्रवणमसम्भवि। न हि त्वमपि मत्स्थानमधिष्ठातुं शक्तो येन मत्तः श्रवणं तव सम्भवति। ईश्वरनियोगाभावादिति भावः ~॥~६~॥\\
\noindent अथ सूर्यवचनानुवादमुपसंहरन् सूर्यांशपुरुषमयासुरसंवादोपक्रममाह\textendash
%\end{sloppypar}
%\vspace{2mm}


\begin{quote}
 {\ssi इत्युक्त्वान्तर्दधे देवः समादिश्यांशमात्मनः ~।\\
स पुमान् मयमाहेदं प्रणतं प्राञ्जलिस्थितम् ~॥~७~॥}
%\vspace{2mm}
\end{quote}
\begin{sloppypar}
 देवः सूर्यमण्डलस्यः। इति पूर्वोक्तमुक्त्वा कथयित्वा। आत्मनः स्वस्यांशमग्रस्यमंशपुरुषं समादिश्य त्वं मयं प्रति सकलं ग्रहमाहात्म्यं कथयेत्याज्ञाप्य। विनाज्ञां स मयं प्रति कथं कथयेत्। समुच्चयार्थश्चकारोऽनुसन्धेयः। अन्तर्दधे। अन्तर्धानं सूर्यांशपुरुषमयनेत्रागोचरतां प्राप्तवान्। प्रकृतमाह\textendash स इति~। सूर्याज्ञप्तः सूर्यांशपुरुषो मयासुरं प्रतीदं वक्ष्यमाणमवदत्।ननु नापृष्टो वदेदित्युक्तेर्मयापृष्टोऽयं कथं मयं प्रत्यवददित्यतो मयविशेषणद्वयमाह\textendash प्रणतं प्राञ्जलिस्थितमिति~। प्रकर्षेण भक्तिश्रद्धातिशयेन नतं नम्रं स्वनमस्कारकारकम्। प्रकृष्टो मानसचेष्टाद्योतको योऽञ्जलिः कराग्रयोः सम्पुटीकरणं तत्र चित्तैकाग्र्यणावस्थितम्। एतेनावनतशिरःकरसम्पुटसंयोगः कायिकनमस्कार इति स्पष्टमुक्तम्। तथा च स्वामिन्नहं त्वां नतोऽस्मि मामनुगृहाणेदं कथयेत्युक्तिद्योतकनमस्कारोक्तेर्मयपृष्टोऽयं मयं प्रत्यवददिति भावः ~॥~७~॥\\
 \noindent अथ प्रतिज्ञाततत्संवा\textendash
\end{sloppypar}

\newpage


\hspace{3cm} गूढार्थप्रकाशकेन सहितः~। \hfill ९
\vspace{1cm}
 
%\begin{sloppypar}
\noindent दानुवादे मयं प्रति ज्ञानं वक्तुकामः सूर्यांशपुरुषः सावधानतयामदुक्तं शृणु त्वमित्याह\textendash
%\end{sloppypar}
%\vspace{2mm}


\begin{quote}
  {\ssi शृणुष्वैकमनाः पूर्वं यदुक्तं ज्ञानमुत्तमम् ~।\\
युगे युगे महर्षीणां स्वयमेव विवस्वता ~॥~८~॥}
%\vspace{2mm}
\end{quote}
%\begin{sloppypar}
 हे मय। एकस्मिन्नेव मनो यस्यासौ। अन्यविषयेभ्यो मनः समाहृत्य मदुक्ति मनो ददानस्त्वं तज्ज्योतिःशास्त्रं शृणुष्व। श्रोत्रद्वारात्मनः संयोगेन प्रत्यक्षं कुर्वित्यर्थः। ननु त्वं स्वकल्पितं वदिष्यसीत्यतस्तच्छब्दसम्बन्धमाह\textendash पूर्वमित्यादि~। यदुत्तमं नेत्र रूपं ज्ञानं शास्त्रं ज्योतिःशास्त्रमित्यर्थः। बहुकालान्तरेण पूर्वकाले कदेत्यत आह\textendash युगे युग इति~। प्रतिमहायुगे महामुनीनां तान् प्रतीति तात्पर्यार्थः। सूर्येण स्वयमद्वारकेण साक्षादित्यर्थः। एवकारो यथा त्वां प्रत्यहं द्वारं साक्षात् कथनासम्भवात् तथा तान् प्रत्यहमन्यो वा द्वारमित्यस्य वारणार्थः।तेषां स्वतपः समाजवशीकृतेश्वराणां तत्प्रसादाधिगताप्रतिहतेच्छानां सूर्यमण्डलाधिष्ठानसम्भवात्। उक्तमुपदिष्टम्। तथा च सूर्योक्तं त्वां प्रति कथ्यते न स्वकल्पितमिति भावः ~॥~८~॥\\
 \noindent ननु प्रतियुगं सूर्योक्तस्यैक्याभावात् त्वया किं युगीयशास्त्रमुप
दिश्यते। अन्यथैकदोक्त्या युगे युगे इत्यस्यानुपपत्तेरित्यत आह\textendash
%\end{sloppypar}
%\vspace{2mm}


\begin{quote}
{\ssi शास्त्रमाद्यं तदेवेदं यत् पूर्वं प्राह भास्करः ~।\\
युगानां परिवर्तेन कालभेदीऽत्र केवलम्* ~॥~९~॥ }
\end{quote}
\rule{\linewidth}{.5pt}

\begin{center}
 * केवल इति वा पाठः।
 \end{center}

{\tiny{C}}

\newpage


\noindent १० \hspace{4cm} सूर्यसिद्धान्तः
\vspace{1cm}

%\begin{sloppypar}
\noindent  इदं मया तुभ्यं वक्ष्यमाणं ज्योतिःशास्त्रं तत् सूर्योक्तम्। एवकारात् सूर्योक्ताभिन्नत्वेन त्वां प्रत्यनुवादो न क्वचित् स्वकल्पनान्तरेणेत्यर्थः। आद्यं प्राक्काले सूर्येणोक्तम्। नन्वासन्नयुगीयसूर्योक्तस्यापि पूर्वकालोक्त्याद्यत्वसम्भव इत्यतस्तत्पदापेक्षितमाद्यपदविवरणरूपमाह। यदिति। शास्त्रं सूर्यः प्रथमं यस्मात् पूर्वमनुक्तमित्यर्थः। प्राह प्रकर्षेण विस्तरेण मुनीन् प्रत्युक्तवान्। तथा च प्रथमातिरेके कारणाभावात् प्रथमस्य विस्तृतत्वाच्चानन्तरोक्तं पूर्वोक्ते गतार्थतया संक्षिप्तमुपेक्ष्य प्रथमयुगीयशास्त्रमुपदिश्यत इति भावः। ननु तर्हि अनन्तरयुगीयशास्त्राणां सूर्योक्तानां वैयर्थ्यप्रसङ्ग इत्यत आह\textendash युगानामिति~। महायुगानां परिवर्तेन पुनः पुनरावृत्त्यात्र सूर्योक्तशास्त्रेषु केवलं स्वभिन्नाभावस्तन्मात्रमित्यर्थः। कालभेदः कालकृतमन्तरम्। पूर्वशास्त्रकालादनन्तरशास्त्रकालो भिन्न इत्येषु शास्त्रेषु भेदो न शास्त्रोक्तरीतिभेद इत्यर्थः। तथा च कालवशेन ग्रहचारे किञ्चिद्वैलक्षण्यं भवतीति युगान्तरे तत्तदन्तरं ग्रहचारेषु प्रसाध्य तत्कालस्थितलोकव्यवहारार्थं शास्त्रान्तरमिव कृपालुरुक्तवानिति नानान्तरशास्त्राणां वैयर्थ्यम्। एवञ्च मया वर्तमानयुगीयसूर्योक्तशास्त्रसिद्धग्रहचारमङ्गीकृत्याद्यसूर्योक्तशास्त्रसिद्धं ग्रहचारं च प्रयोजनाभावादुपेक्ष्य तदुक्तमेव त्वां प्रत्युपदिश्यत इति भावः। एवञ्च युगमध्येऽप्यवान्तरकाले ग्रहचारेष्वन्तरदर्शने तत्तत्काले तदन्तरं प्रसाध्य ग्रन्थांस्तत्कालवर्तमानाभियुक्ताः कुर्वन्ति। तदिदमन्तरं पूर्वग्रन्थे बीजमित्याम\textendash
%\end{sloppypar}

\newpage


\hspace{3cm} गूढार्थप्रकाशकेन सहितः ~। \hfill ११
\vspace{1cm}
 
%\begin{sloppypar}
\noindent नन्ति। पूर्वग्रन्थानां लुप्तत्वात् सूर्यर्षिसंवादोऽपीदानीं न दृश्यत इति तदप्रसिद्धिरागमप्रामाण्याच्च नाशङ्क्या ~॥~९~॥\\
\noindent अथ कालभेद इत्यनेनोपस्थितं कालं प्रथमं निरुरूपयिषुस्तावत् कालं विभजते।
%\end{sloppypar}
%\vspace{2mm}


\begin{quote}
 {\ssi लोकानामन्तकृत् कालः कालोऽन्यः कलनात्मकः ~।\\
स द्विधा स्थूलसूक्ष्मत्वान्मूर्तश्चामूर्त उच्यते ~॥~१०~॥}
%\vspace{2mm}
\end{quote}
\begin{sloppypar}
 कालो द्विधा तत्रैकः कालोऽखण्डदण्डायमानः शास्त्रान्तरप्रमाणसिद्धः। लोकानां जीवानामुपलक्षणादचेतनानामपि। अन्तकृद्विनाशकः। यद्यपि कालस्तेषामुत्पत्तिस्थितिकारकस्तथापि विनाशस्यानन्तत्वात्, कालत्वप्रतिपादनाय चान्तकृदित्युक्तम्। अन्तकृदित्यनेनैवोत्पत्तिस्थितिकृदित्युक्तमन्यथानाशासम्भवात्। अत एव।
\end{sloppypar}
%\vspace{2mm}


\begin{quote}
 {\qt कालः सृजति भूतानि कालः संहरति प्रजाः~।}
%\vspace{2mm}
\end{quote}
\begin{sloppypar}
 इत्याद्युक्तं ग्रन्थान्तरे। अन्यो द्वितीयः कालः खण्डकालः। कलनात्मको ज्ञानविषयस्वरूपः। ज्ञातुं शक्य इत्यर्थः। स द्वितीयः कलनात्मकः कालोऽपि द्विधा भेदद्वयात्मकः। तदाह\textendash स्थूलसूक्ष्मत्वादिति~। महत्त्वाणुत्वाभ्याम्। मूर्तः। इयत्तावच्छिन्नपरिमाणः। अमूर्तस्तद्भिन्नः कालस्तत्त्वविद्भिः कथ्यते। चकारो हेतुक्रमेण मूर्तामूर्तक्रमार्थकः। तेन महान् मूर्तः कालोऽणुरमूर्तः काल इत्यर्थः ~॥~१०~॥\\
 \noindent अथोक्तं भेदद्वयं स्वरूपेण प्रदर्शयन् प्रथमभेदं प्रतिपिपादयिषुस्तदवान्तरभेदेषु भेदद्वयमाह\textendash
\end{sloppypar}

{\tiny{C 2}}


\newpage


\noindent १२ \hspace{4cm} सूर्यसिद्धान्तः
\vspace{1cm}

%\begin{sloppypar}
\begin{quote}
{\ssi प्राणादिः कथितो मूर्तस्त्रुट्याद्योऽमूर्तसञ्ज्ञकः*~।\\
षड्भिः प्राणैर्विनाडी स्यात् तत्षष्ट्या नाडिका स्मृता ~॥~११~॥}
\end{quote}
%\vspace{3mm}

 प्राणः स्वस्थसुखासीनस्य श्वासोच्छ्वासान्तर्वर्ती कालो दशगुर्वक्षरोच्चार्यमाण आदिर्यस्यैतादृशः प्राणानन्तर्गतो मूर्तः काल उक्तः। त्रुटिराद्या यस्यैतादृशः काल एकप्राणान्तर्गतस्त्रुटितत्परादिकोऽमूर्तसञ्ज्ञः। अथामूर्तस्य मूर्तादिभूतस्य व्यवहारायोग्यत्वेनाप्रधानतयानन्तरोद्दिष्टस्य भेदप्रतिपादनमुपेक्ष्य मूर्तकालस्य व्यवहारयोग्यत्वेन प्रधानतया प्रथमोद्दिष्टभेदान् विवक्षुः प्रथमं पलघट्यावाह\textendash  षड्भिरिति~। षट्प्रमाणैरसुभिः पानीयपलं भवति पलानां षष्ट्या घटिकोक्ता कालतत्त्वज्ञैः ~॥~११~॥\\
 \noindent अथ दिनमासावाह\textendash
%\end{sloppypar}
%\vspace{2mm}


\begin{quote}
 {\ssi नाडीषष्ट्या तु नाक्षत्रमहोरात्रं प्रकीर्तितम् ~।\\
तत्त्रिंशता भवेन्मासः सावनोऽर्कोदयैस्तथा ~॥~१२~॥}
%\vspace{2mm}
\end{quote}
\begin{sloppypar}
 घटीनां षष्ट्याहोरात्रं नाक्षत्रमुक्तम्। तुकारादहोरात्रस्य नाक्षत्रत्वोक्त्योक्तघट्या अपि नाक्षत्रत्वमुक्तम्। एतत्षष्टिघटीभिर्भचक्रपरिवर्तनात्। नाक्षत्रदिनानां त्रिंशत्सङ्ख्यया मासो नाक्षत्रः। मासानामनेकत्वेन सावनमासस्वरूपमाह\textendash सावन इति~। तथा त्रिंशदहोरात्रैः सूर्योदयसम्बद्धैस्तदवधिकैः सूर्योदयादिसूर्योदयान्तकालरूपैकाहोरात्रमानमापितैरित्यर्थः।
\end{sloppypar}

\noindent \rule{\linewidth}{.5pt}

\begin{center}
*उच्यत इति पाठान्तरम्।
\end{center}

\newpage


\hspace{3cm} गूढार्थप्रकाशकेन सहितः ~। \hfill १३
\vspace{1cm}


\noindent सावनो मासः ~॥~१२~॥\\
\noindent अथ चान्द्रसौरमासनिरूपणपूर्वकं वर्षं वदन् दिव्यदिनमाह\textendash
%\vspace{2mm}

 
\begin{quote}
{\ssi ऐन्दवस्तिथिभिस्तद्वत् सङ्कान्त्या सौर उच्यते ~।\\
मासैर्द्वादशभिर्वर्षं दिव्यं तदह उच्यते ~॥~१३~॥}
%\vspace{2mm}
\end{quote}
\begin{sloppypar}
 तद्वत् त्रिंशता तिथिभिश्चान्द्रो मासस्तत्र दर्शान्तावधिकः पूर्णिमान्तावधिकश्च शास्त्रे मुख्यतया प्रतिपादितः। अत्र शास्त्रे तु दर्शान्तावधिक एव मुख्यः। इष्टतिथ्यवधिकस्तु मासो गौणः। सुङ्क्रान्त्या सङ्क्रान्त्यवधिकेन कालेन सौरो मासो मानज्ञैः कथ्यते। सङ्क्रान्तिस्तु सूर्यमण्डलकेन्द्रस्य राश्यादिप्रदेशसञ्चरणकालः। द्वादशभिर्मासैर्वर्षम्। यन्मानेन मासास्तन्मानेन वर्षं ज्ञेयम्। तद्वर्षं सौरमासस्यासन्नत्वात् सौरम्। अहः। अहोरात्रं दिव्यम्। दिवि भवम्। सौरवर्षं देवानामहोरात्रमानं मानतत्त्वज्ञैः कथ्यत इत्यर्थः ~॥~१३~॥\\
 \noindent ननु देवानां यथाहोरात्रमुक्तं तथा दैत्यनामहोरात्रं कथं नोक्तमित्यतस्तदुत्तरं वदन् देवासुरयोर्वर्षमाह\textendash
\end{sloppypar}
%\vspace{2mm}


\begin{quote}
{\ssi सुरासुराणामन्योऽन्यमहोरात्रं विपर्ययात् ~।\\
तत्षष्टिः षड्गुणा दिव्यं वर्षमासुरमेव च ~॥~१४~॥}
%\vspace{2mm}
\end{quote}
%\begin{sloppypar}
 देवदैत्यानां बहुत्वाद्बहुवचनम्। अन्योऽन्यम्। परस्परम्। विपर्ययात्। व्यत्यासात्। अहोरात्रम्। अयमर्थः। देवानां यद्दिनं तदसुराणां रात्रिः। देवानां या रात्रिस्तदसुराणां दिनम्। दैत्यानां यद्दिनं तद्देवानां रात्रिः। दैत्यानां या रात्रि\textendash
%\end{sloppypar}

\newpage


\noindent १४ \hspace{4cm} सूर्यसिद्धान्तः
\vspace{1cm}
 
%\begin{sloppypar}
\noindent स्तद्देवानां दिनमिति। तथा च देवदैत्ययोर्दिनरात्र्योरेव व्यत्यासाद्भेदो न मानेनेति। तयोरहोरात्रस्यैक्याद्देवाहोरात्रमानकथनेनैव दैत्याहोरात्रमानमुक्तमिति भावः। युगकथनार्थं दिव्यवर्षं परिभाषया सुगममपि विशेषद्योतनार्थं प्रकारान्तरेणाह\textendash तत्षष्टिरिति~। दिव्याहोरात्रषष्टिः। देवर्तुरूपा वर्षर्तुभिः षड्भिर्गुणिता दिव्यमासुरं दैत्यसम्बन्धि। चः समुच्चये। तेन द्वयोरित्यर्थः। वर्षम्। एवकारस्तयोर्दिनरात्र्योर्भेदेन वर्षभेदः स्यादिति मन्दशङ्कानिवारणार्थम् ~॥~९४~॥\\
\noindent अथ कल्पमानं विवक्षुः प्रथमं युगमानमन्यदपि श्लोकाभ्यामाह\textendash
%\end{sloppypar}
%\vspace{2mm}


\begin{quote}
 {\ssi तद्वादशसहस्राणि चतुर्युगमुदाहृतम् ~।\\
सूर्याब्दसङ्ख्यया द्वित्रिसागरैरयुताहतैः ~॥~१५~॥

सन्ध्यासन्ध्यांशसहितं विज्ञेयं तच्चतुर्युगम् ~।\\
कृतादीनां व्यवस्थेयं धर्मपादव्यवस्थया ~॥~१६~॥}
\end{quote}
%\vspace{2mm}

\begin{sloppypar}
 तेषां दिव्यवर्षाणां द्वादशसहस्राणि चतुर्युगम्। चतुर्णां युगानां कृतत्रेताद्वापरकल्याख्यानां समाहारो योगस्तदात्मकं महायुगमित्यर्थः। एतद्द्योतनार्थं चतुरित्युक्तिरन्यथा युगमित्युक्त्या तद्वैयर्थ्यापत्तेः। मानाभिज्ञैरुक्तम्। अथ सौरमानेन तत्सङ्ख्यां विशेषं चाह\textendash सूर्याब्दसङ्ख्ययेति। तद्देवासुरमानेनोक्तं चतुर्युगं द्वादशसहस्रवर्षात्मकं महायुगं सन्ध्यासन्ध्यांशसहितम्। युगचरणस्याद्यन्तयोः क्रमेण प्रत्येकं सन्ध्यासन्ध्यांशाभ्यां युक्तं सदेव सन्ध्यासन्ध्यांशावन्तर्गतौ न पृथक् यत्रैतादृशम्। सौर\textendash
\end{sloppypar}

\newpage


\hspace{3cm} गूढार्थप्रकाशकेन सहितः ~।  \hfill १५
\vspace{1cm}
 
%\begin{sloppypar}
वर्षप्रमाणेन द्वित्रिसागरैः। अङ्कानां वामतो गतिरित्यनेन द्वात्रिंशदधिकैश्चतुःशतमितैः। अयुतेन दशसहस्रेण गुणितैः। खचतुष्कद्वात्रिंशच्चतुर्भिः परिमितं ज्ञेयमित्यर्थः। अथ चतुर्युगान्तर्गतयुगाङ्घ्रीणां विशेषतो मानाश्रवणात् समं स्यादश्रुतत्वादिति न्यायेन प्रत्येकं महायुगचतुर्थांशो मानमिति चतुर्युगमित्यनेन फलितं निषेधति। कृतादीनामिति। कृतत्रेताद्वापरकलियुगानाम्। धर्मपादव्यवस्थया धर्मचरणानां स्थित्या। इयं वक्ष्यमाणा व्यवस्था स्थितिर्ज्ञेया न तु समकालप्रमाणं स्थितिः। अयमर्थः। कृतयुगे चतुश्चरणो धर्म इति तस्य मानमधिकम्। ततस्त्रेतायां धर्मस्य त्रिपादवत्वात् तदनुरोधेन त्रेतामानं न्यूनम्। एवं द्वापरकल्योर्धर्मस्य क्रमेण द्व्येकचरणवत्वात् कृतत्रेतामानाभ्यां क्रमेणोक्तानुरोधान्न्यूनमानम्। न तु समं मानमिति ~॥~१६~॥\\
\noindent अथ सर्वधर्मचरणयोगेन दशमितेन महायुगं भवति तर्हि स्वस्वधर्मचरणैः किमित्यनुपातेन पूर्वोक्तफलितेन कृतादियुगानां मानज्ञानं सविशेषणमाह\textendash
%\end{sloppypar}


\begin{quote}
 {\ssi युगस्य दशमो भागश्चतुस्त्रिद्व्येकसङ्गुणः ~।\\
क्रमात् कृतयुगादीनां षष्ठांशः सन्ध्ययोः स्वकः ~॥~१७~॥}
\end{quote}
%\begin{sloppypar}
 प्रागुक्तदिव्यवर्षद्वादशसहस्रमितस्य युगस्य दशमो भागो दशांश इत्यर्थः। चतुर्द्धा क्रमेण चतुस्त्रिद्व्येकैर्गुणितः। गुणक्रमात् कृतयुगादीनां कृतत्रेताद्वापरकलियुगानां मानं स्यादिति शेषः। ननु मनुग्रन्थे कृतादिमानं दिव्यवर्षप्रमाणेन ~४०००~|~ ३०००~|
%\end{sloppypar}

\newpage


\noindent १६ \hspace{4cm} सूर्यसिद्धान्तः 
\vspace{1cm}

\begin{sloppypar}
\noindent २०००~|~ १०००~| अत्र तु तन्मानं तद्वर्षप्रमाणेन ~४८००~|~ ३६००~|~ २४००~|~ १२००~|~ इति विरोध इत्यत आह\textendash षष्ठ इति। स्वकः स्वसम्बन्धी षष्ठो विभागः सन्ध्ययोराद्यन्तसन्ध्ययोरैक्यकाल इति शेषः। तथा च मदुक्तमानानि ~४८००~|~ ३६००~|~ २४००~|~ १२००~|~ एषां षडंशाः ~८००~|~ ६००~|~ ४००~|~ २००~|~ एते स्वस्वयुगानामाद्यन्तयोः सन्ध्योर्योगा इत्येषामर्धं सन्धिकालः। प्रत्येकमाद्यन्तयोः सन्धिकालः ~४००~|~ ३००~|~ २००~|~ १००~|~ अनेन प्रत्येकं मदुक्तमानं न्यूनीकृतं ग्रन्थान्तरोक्तं केवलं मानं भवति न स्वसन्धिभ्यां सहितम्। यथा कृतादिसन्धिः ४०० कृतमानम् ४०००, कृतान्तसन्धिः ४००~। त्रेतादिसन्धिः ३०० त्रेतामानं ३०००, त्रेतान्तसन्धिः ३००~| द्वापरादिसन्धिः २००, द्वापरमानं २००० द्वापरान्तसब्धिः २००~| कल्यादिसन्धिः १००, कलिमानं १०००, कल्यन्तसन्धिः १००~| एवं च स्वसन्धिभ्यां सहितं मयोक्तं स्वसम्बन्धात् सन्ध्ययोस्तदन्तर्गतत्वाच्चेति न विरोध इति भावः ~॥~१७~॥\\ 
\noindent अथ कल्पमानार्थं मनुमानं तत्सन्धिमानं चाह\textendash
\end{sloppypar}
\begin{quote}

 {\ssi युगानां सप्ततिः सैका मन्वन्तरमिहोच्यते ~।\\
कृताब्दसङ्ख्या तस्यान्ते सन्धिः प्रोक्तो जलप्लवः ~॥~१८~॥}
\end{quote}
\begin{sloppypar}
 युगानां सैका सप्ततिरेकसप्ततिर्महायुगमित्यर्थः। इह मूर्तकाले मन्वन्तरं मन्वारम्भतत्समाप्तिकालयोरन्तरकालमानमित्यर्थः। मूर्तकालमानभेदाभिज्ञैः कथ्यते। तस्य मनोरन्ते विरामे जाते सति कृताब्दसङ्ख्या मदुक्तकृतयुगवर्षमितिः सन्धिः
\end{sloppypar}

\newpage


\hspace{3cm} गूढार्थप्रकाशकेन सहितः~। \hfill १७
 \vspace{1cm}

%\begin{sloppypar}
कालविद्भिः प्रकर्षेण द्वितीयमन्वारम्भपर्यन्तं भूतभाविमन्वोरन्तिमादिसन्धिरूपैककालेन कथितः। तत्स्वरूपमाह\textendash जलप्लव इति। जलपूर्णा सकला पृथ्वी तस्मिन् लोकसंहारकाले भवति ~॥~१८~॥\\
\noindent अथ कल्पप्रमाणं सविशेषमाह\textendash
%\end{sloppypar}


\begin{quote}
 {\ssi ससन्धयस्ते मनवः कल्प ज्ञेयाश्चतुर्दश ~।\\
कृतप्रमाणः कल्पादौ सन्धिः पञ्चदशः स्मृतः ~॥~१९~॥}
\end{quote}
\begin{sloppypar}
 ते एकसप्ततियुगरूपा मनवः स्वायम्भुवाद्याः ससन्धयः स्वस्वसन्धिसहिताश्चतुर्दशसङ्ख्याकाः कल्पकाले ज्ञातव्याः। स्वसन्धियुक्तचतुर्दशमनुभिः कल्पः स्यादित्यर्थः। ननु ग्रन्थान्तरे कल्पमानं युगसहस्रं त्वया तु युगमानमेकसप्ततिगुणं मनुमानं ३०६७२०००० कृताब्द १७२८००० युक्तं ससन्धिमनुमानं ३०८४४८०००, इदं चतुर्दशगुणं कल्पप्रमाणं कृतोनं युगसहस्रमित्यत आह\textendash कृतप्रमाण इति~। कल्पादौ प्रथममन्वारम्भे कृतयुगवर्षमितो मनोश्चतुर्दशत्वेऽप्याद्यः पञ्चदशकः सन्धिः कालज्ञैरुक्तः। तथा च कृतवर्षानन्तरं प्रथममन्वारम्भ इति तद्वर्षयोजनेनाविरोध इति भावः ~॥~१९~॥\\
 \noindent अथ ब्रह्यणो दिनरात्र्योः प्रमाणमाह\textendash
\end{sloppypar}

 
 \begin{quote}
 {\ssi इत्थं युगसहस्रेण भूतसंहारकारकः ~।\\
कल्पो ब्राह्ममहः प्रोक्तं शर्वरी तस्य तावती ~॥~२०~॥}
\end{quote}
\begin{sloppypar}
 इत्थं पूर्वोक्तप्रकारसिद्धेन युगसहस्रेण भूतसंहारकारको ब्राह्मलयात्मकः कल्पकालो ब्राह्मं ब्रह्मणः सम्बन्ध्यहो दिनं कालज्ञैरुक्तम्। तस्य ब्रह्मणस्तावती दिनपरिमिता शर्वरी रात्रिः।
\end{sloppypar}

 {\tiny{D}}
 
 \newpage
 

\noindent १८ \hspace{4cm} सूर्यसिद्धान्तः
\vspace{1cm}
 
%\begin{sloppypar}
कल्पद्वयं तदहोरात्रमिति फलितार्थः ~॥~२०~॥\\ अथ ब्रह्मण आयुःप्रमाणमतीतवयः प्रमाणं चाह\textendash
%\end{sloppypar}

 
 \begin{quote}
 {\ssi परमायुः शतं तस्य तयाऽहोरात्रसङ्ख्यया ~।\\
आयुषोऽर्द्धमितं तस्य शेषकल्पोऽयमादिमः ~॥~२१~॥ }
\end{quote}
\begin{sloppypar}
 परमपरं शृणु पूर्वोक्तं त्वया श्रुतमपरं च वक्ष्यमाणं शृणुत्वम्। यद्वा परमेति दैत्यवरार्थकं सम्बोधनम्। त्वं तस्य ब्रह्मणस्तया पूर्वोक्तयाहोरात्रमित्या कल्पद्वयरूपया शतं शतवर्षपरिमितमायुः शतवर्षधारणकालं जानीहि। एतदुक्तं भवति। अहोरात्रमानात् पूर्वपरिभाषया मासमानं तस्मात् पूर्वोक्तपरिभाषया ब्रह्मणो वर्षमानमेतच्छतसङ्ख्यया ब्रह्मायुरिति। न तु यथा श्रुतार्थेन कल्पशतद्वयमायुः कीटादीनामपि दिनसङ्ख्ययायुषोऽनुक्तेः सुतरां ब्रह्मणः शतदिनात्मकायुषोऽसम्भवात्।
\end{sloppypar}

  
 \begin{quote}
{\ssi निजेनैव तु मानेन आयुर्वर्षशतं स्मृतम्~।}
\end{quote}
\begin{sloppypar}
 इति विष्णुपुराणोक्तेश्च। एतेन परमायुरिति निरस्तम्। ब्रह्मणोऽनियतायुर्दायासम्भवात्। तस्य ब्रह्मण आयुः शतवर्षरूपमस्यार्द्धं पञ्चाशद्वर्षपरिमितमितं गतम्। अयं वर्तमान आदिमः प्रथमः शेषकल्पः शेषायुर्दायस्य ब्रह्मदिवस उत्तरार्द्धस्य प्रथमदिवसो वर्तमान इति फलितार्थः ~॥~२९~॥\\ 
 \noindent अथ वर्तमानेऽस्मिन् दिवसेऽप्येतद्गतमित्याह\textendash
\end{sloppypar}

 
 \begin{quote}
 {\ssi कल्पादस्माच्च मनवः षड् व्यतीताः ससन्धयः ~।\\
वैवस्वतस्य च मनोर्युगानां त्रिघनो गतः ~॥~२२~॥}
\end{quote}
\newpage

\hspace{3cm} गूढार्थप्रकाशकेन सहितः~। \hfill १९
\vspace{1cm}

\begin{sloppypar}
 अस्माद्वर्तमानात् कल्पाद् ब्रह्मदिवसात् षट्सङ्ख्याका मनव एकसप्ततियुगरूपाः ससन्धयः सप्तभिः सन्धिभिः कृतयुगप्रमाणैः सहिता व्यतीता गताः। चकार आयुषोऽर्धमितमिति प्रागुक्तेन समुच्चयार्थकः। वर्तमानस्य सप्तमस्य मनोर्वैवस्वताख्यस्य युगानां त्रिघनस्त्रयाणां घनः स्थानत्रयस्थिततुल्यानां घातः सप्तविंशतिसङ्ख्यात्मको गतः। सप्तविंशतियुगानि गतानीत्यर्थः। चः समुच्चये ~॥~२२~॥\\ 
 \noindent अथ वर्तमानयुगस्यापि गतमेतदिति वदन्नभिमतकालेऽग्रतो वर्षगणः कार्य इत्याह\textendash
\end{sloppypar}

 
 \begin{quote}
  {\ssi अष्टाविंशाद्युगादस्माद्यातमेतत् कृतं युगम् ~।\\
अतः कालं प्रसङ्ख्याय सङ्ख्यामेकत्र पिण्डयेत् ~॥~२३~॥}
 \end{quote}
\begin{sloppypar}
 अष्टाविंशतितमाद्वर्तमानान्महायुगादेतदल्पकालेन पूर्वकाले साम्प्रतं स्थितं कृतं युगं गतम्। अतः कृतयुगान्तानन्तरमभिमतकाले कालं वर्षात्मकं प्रसङ्ख्याय गणयित्वा सङ्ख्यां पञ्चस्थानस्थितां भिन्नामेकत्रैकस्थाने पिण्डयेत् सङ्कलनविषयां कुर्यात्। सर्वेषां गतानां योगं कुर्यादित्यर्थः ~॥~२३~॥\\ 
 \noindent अथ कल्पादितो ग्रहादिभचक्रनियोजनकालं ग्रहगतिरूपमाह\textendash
\end{sloppypar}

 
 \begin{quote}
  {\ssi ग्रहर्क्षदेवदैत्यादि सृजतोऽस्य चराचरम् ~।\\ 
कृताद्रिवेदा दिव्याब्दाः शतघ्ना वेधसो गताः ~॥~२४~॥}
\end{quote}
\begin{sloppypar}
 अस्य वर्तमानस्य ब्रह्मणो ग्रहनक्षत्रदेवदैत्यमानवराक्षसभूपर्वतवृक्षादिकं चराचरं जङ्गमस्थावरात्मकं जगत् सृजतः सृजतीति सृजन् तस्य जगन्निर्मायकस्य शतसङ्ख्यागुणिताश्चतुःसप्त\textendash
\end{sloppypar}

{\tiny{D2}}

\newpage

\noindent २० \hspace{4cm} सूर्यसिद्धान्तः
\vspace{1cm}

%\begin{sloppypar}
त्यधिकचतुःशतसङ्ख्या दिव्याब्दा गताः। एभिर्दिव्यवर्षैर्ग्रहसृष्ट्यादि प्रवहवायुनियोजनान्तं कर्म ब्रह्मणा कृतमिति फलितार्थः ~॥~२४~॥\\ 
\noindent अथ ग्रहपूर्वगत्युत्पत्तौ कारणमाह\textendash
%\end{sloppypar}

%\vspace{2mm}
 
 \begin{quote}
 {\ssi पश्चाद् व्रजन्तोऽतिजवान्नक्षत्रैः सततं ग्रहाः ~।\\
जीयमानास्तु लम्बन्ते तुल्यमेव स्वमार्गगाः ~॥~२५~॥}
%\vspace{2mm}
\end{quote}
\begin{sloppypar}
 पश्चादनन्तरं पुनरावृत्तया पश्चात् पश्चिमदिगभिमुखं नक्षत्रैस्तारकादिभिः सह ग्रहाः सूर्यादयोऽतिजवात् प्रवहवायुसत्वरगतिवशात् सततं निरन्तरं व्रजन्तो गच्छन्तः स्वमार्गगाः स्वकक्षावृत्तस्था जीयमाना नक्षत्रैः पराजिता नक्षत्राणामग्रे गमनात्। अत एव लज्जयेव गुरुभूता इति तात्पर्यार्थः। तुल्यं समम्। एवकारादधिकन्यूनव्यवच्छेदः। लम्बन्ते स्वस्थानात् पूर्वस्मिन् लम्बायमाना भवन्ति। यथा लज्जितः पश्चाद्भवति नाग्रे। तुकारादधोऽधः कक्षाक्रमानुरोधेन शन्यादिग्रहाणां चन्द्रान्तानां गुरुतापचयः शनिरतिगुरुभूतस्तस्मात् किञ्चिन्न्यूनो गुरुस्तस्मादपि भौम इत्यादि यथोत्तरम्। यस्य कक्षा महती तस्य गुरुत्वाधिक्यं यस्य लघ्वी तस्य तदनुरोधेन गुरुताल्पत्वमिति। एतदुक्तं भवति। ब्रह्मणा प्रवहवायौ नक्षत्राधिष्ठितो मूर्तो गोलः स्थापितस्तदन्तर्गताः स्वस्वाकाशगोलस्थाः शन्यादयो नक्षत्राधिष्ठितमूर्तगोलस्थक्रान्तिवृत्तस्थरेवतीयोगतारासन्नरूपमेषादि प्रदेशसमसूत्रस्थाः स्थापिताः। क्रान्तिवृत्तं तु मेषतुलास्थाने विषुववृत्तलग्नं सम्पातात् त्रिभान्तरितक्रान्तिवृत्तप्रदेशाभ्यां
\end{sloppypar}

\newpage

 \hspace{3cm} गूढार्थप्रकाशकेन सहितः~। \hfill २१ 
\vspace{1cm}

%\begin{sloppypar}
\noindent चतुर्विंशत्यंशान्तरेण दक्षिणोत्तरौ मकरकर्कादिरूपौ तदेव द्वादशराश्यात्मकं वृत्तं ग्रहचारभतूम्। विषुवद्वृत्तं तु ध्रुवमध्यस्थं निरक्षदेशोपरिगतम्। तत्र प्रवहवायुना स्वाघातेन मूर्तो नक्षत्रगोलो नाक्षत्रषष्टिघटीभिः परिवर्त्यते। तदन्तर्गतवायुभिस्तदाघातेन वा ग्रहा भ्रमन्त्यपि नक्षत्रगोलस्थितक्रान्तिवृत्तीयमेषादिप्रदेशेन समं न गच्छन्ति वायूनां स्वल्पत्वात् तदाघातस्याप्यल्पत्वाद्बिम्बानां गुरुत्वाच्च। अतस्तत्स्थानाद्ग्रहाणां लम्बनं दृश्यते। अत एव नक्षत्रोदयकाले तेषां द्वितीयदिने नोदयः किन्तु ग्रहो लम्बितप्रदेशेन वायुना तदनन्तरमूर्ध्वमागच्छतीति, अनन्तरमुदयः। लम्बनं तु शन्यादीनां कक्षानुरोधेन गुरुत्वाद्वायूनां तद्घातानां वा कक्षानुरोधेन बह्वल्पत्वात् तुल्यम्। यद्यपि वायोर्ध्रुवानुरोधेन सत्त्वाद्ग्रहावलम्बनं विषुवद्वृत्ते भवितुमुचितं न क्रान्तिवृत्ते। तथा च वक्ष्यमाणक्रान्त्यनुपपत्तिः क्रान्तिवृत्तस्थद्वादशराशिभोगेन वक्ष्यमाणानां भगणानामनुपपत्तिश्च। तथापि वायुनावलम्बितो ग्रहो विषुवन्मार्गगोऽपि तद्विषुवप्रदेशासन्नक्रान्तिवृत्तप्रदेशेन ग्रहाकाशगोल एव स्वसमसूत्रेणाकृष्यत इति नानुपपत्तिः। अत एव स्वमार्गगा इति क्रान्तिवृत्तानुसृतस्वाकाशगोलस्थकक्षामार्गगता इत्यर्थकमुक्तमिति सङ्क्षेपः ~॥~२५~॥\\ \noindent अथात एव ग्रहाणां लोके प्राग्गतित्वं सिद्धमित्यत आह\textendash
%\end{sloppypar}

 
 \begin{quote}
  {\ssi प्राग्गतित्वमतस्तेषां भगणैः प्रत्यहं गतिः ~।\\
परिणाहवशाद्भिन्ना तद्वशाद्भानि भुञ्जते ~॥~२६~॥}
\end{quote}
\newpage

\noindent २२ \hspace{4cm} सूर्यसिद्धान्तः
\vspace{1cm}

\begin{sloppypar}
 अतोऽवलम्बनादेव तेषां ग्रहाणां प्राग्गतित्वं प्राच्यां दिशि गतिर्येषां ते प्राग्गतयस्तद्भावः प्राग्गतित्वं सिद्धम्। लम्बनस्वरूपैव ग्रहाणां पूर्वगतिरुत्पन्ना लोकैः कारणानभिज्ञैः प्रत्यक्षावगततया तच्छक्तिजनिता कल्पितेत्यर्थः। सा कियतीत्यत आह\textendash भगणैः इति~। वक्ष्यमाणभगणैः प्रत्यहं प्रतिदिनं गतिः प्राग्गमनरूपा भगणानां गत्युत्पन्नत्वाद्भगणसम्बन्धिवक्ष्यमाणदिनैः सूर्यसावनैर्ग्रहभगणा लभ्यन्ते तदैकेन दिनेन केत्यनुपाताज्ज्ञेया। ननु ग्रहभगणानां तुल्यत्वाभावात् प्रतिदिनं ग्रहगतिर्भिन्नेति पूर्वं लम्बनरूपा ग्रहगतिरयुक्तोक्ता ग्रहलम्बनस्याभिन्नत्वादित्यत आह\textendash परिणाहवशात् इति~। परिणाहः कक्षापरिधिस्तद्वशात् तदनुरोधादियं ग्रहगतिर्भिन्ना तुल्या। अयमभिप्रायः। ग्रहाणां लम्बनं तुल्यप्रदेशेन परन्तु स्वस्वकक्षायां तत्प्रदेशे तुल्ये याः कलास्ताः गतिकलास्तास्तु महति कक्षावृत्तेऽल्पा लघुकक्षावृत्ते बह्व्यः सर्वकक्षापरिधीनां चक्रकलाङ्कितत्वात्। भगणास्तु गतिवशादेव यस्य कक्षावृत्तं महत् तस्याल्पा यस्य च लघु कक्षावृत्तं तस्य बहवस्तदुत्पन्ना गतिरपि तथेति न विरोधः। नन्वेकरूपगतिं विहाय भिन्नरूपा गतिः कथमङ्गीकृतेत्यत आह\textendash तद्वशाद् इति~। भिन्नगतिवशाद्भानि राशीन् नक्षत्राणि भुञ्जते ग्रहा भुञ्जन्तीत्यर्थः। तथा च ग्रहराश्यादिभोगज्ञानार्थमियमेव गतिरुपयुक्ता नैकरूपेति भावः ~॥~२६~॥\\ 
 \noindent अथ भभोगे विशेषं वदन् वक्ष्यमाणभगणस्वरूपमाह\textendash
\end{sloppypar}
\newpage

\hspace{3cm} गूढार्थप्रकाशकेन सहितः ~। \hfill २३
\vspace{1cm}
 
 
 \begin{quote}
 {\ssi शीघ्रगास्तान्यथाल्पेन कालेन महताल्पगः ~।\\
तेषां तु परिवर्तेन पौष्णान्ते भगणः स्मृतः ~॥~२७~॥ }
\end{quote}
\begin{sloppypar}

 अथशब्दः पूर्वोक्ते विशेषसूचकः शीघ्रगतिग्रहस्तानि भान्यल्पेन कालेन भुनक्त्यल्पगतिग्रहो बहुकालेन भुनक्ति तुल्यराश्यादिभोगो मन्दशीघ्रगतिग्रहयोस्तुल्यकालेन न भवतीति विशेषार्थः। तेषां राशीनां परिवर्तेन भ्रमणेन। तुकाराद्ग्रहादिगतिभोगजनितेन भगणः प्राज्ञैरुक्तः। क्रान्तिवृत्ते द्वादशराशीनां सत्त्वात् तद्भोगेन चक्रभोगसमाप्तेर्यत् स्थानमारभ्य चलितो ग्रहः पुनस्तत् स्थानमायाति स चक्रभोगः परिवर्तसञ्ज्ञोऽपि द्वादशराशिभोगाद्भगण इत्यर्थः। ननु क्रान्तिवृत्ते सर्वप्रदेशेभ्यः परिवर्तसम्भवादत्र कः परिवर्तनादिभूतः प्रदेश इत्यत आह\textendash पौष्णान्त इति~। सृष्ट्यादौ ब्रह्मणा क्रान्तिवृत्ते रेवतीयोगतारासन्नप्रदेशे सर्वग्रहाणां निवेशितत्वात् तदवधितो ग्रहचलनाच्च। पौष्णस्य रेवतीयोगताराया अन्ते निकटे प्रदेशे तथा च रेवतीयोगतारासन्नाग्रिमस्थानमेव आद्यन्तावधिभूतमिति भावः ~॥~२७~॥\\ 
 \noindent ननु परिवर्त्तस्य भगणसञ्ज्ञा त्वयुक्ता त्रयादिराशीनामपि भगणत्वादित्यतः परिभाषाकथनच्छलेन भगणस्वरूपमाह\textendash
\end{sloppypar}
\begin{quote}
 
 {\ssi विकलानां कला षष्ट्या तत्षष्ट्या भाग उच्यते ~।\\
तत्त्रिंशता भवेद्राशिर्भगणो द्वादशैव ते ~॥~२८~॥}
\end{quote}
\begin{sloppypar}
 यथा मूर्तकाले प्राणकाल आदिभूतस्तथा क्षेत्रपरिभाषायां विकलाः सूक्ष्मादिभूतास्तासां षष्ट्यैका कला कलानां षष्ट्या
\end{sloppypar}

\newpage

\noindent २४ \hspace{4cm} सूर्यसिद्धान्तः
\vspace{1cm}
 
\begin{sloppypar}
\noindent भागोंऽशः क्षेत्रपरिभाषाभिज्ञैः कथ्यते। भागत्रिंशता राशिः स्यात्। ते राशयः सकला द्वादश। एवकारस्त्रिचतुरादीनां निरासार्थम्। तथा च साकल्ये गणपदप्रयोगाद्भगणस्य भोगेऽपि भगणव्यवहाराच्च पूर्वोक्तं युक्तमिति भावः ~॥~२८~॥\\  
\noindent अथ भगणान् विवक्षुः प्रथमं सूर्यबुधशुक्राणां भौमगुरुशनिशीघ्रोच्चानां च भगणानाह\textendash
\end{sloppypar}
%\vspace{2mm}

 
 \begin{quote}
  {\ssi युगे सूर्यज्ञशुक्राणां खचतुष्करदार्णवाः ~।\\
कुजार्किगुरुशीघ्राणां भगणाः पूर्वयायिनाम् ~॥~२९~॥}
%\vspace{2mm}
\end{quote}
\begin{sloppypar}
 महायुगे सूर्यबुधशुक्राणां खानां चतुष्कमेकस्थानादिसहस्रस्थानान्तचतुःस्थानस्थितानि शून्यानि ततोऽयुतादिप्रयुतस्थानपर्यन्तं दन्तसमुद्रास्तथा च युगसौरवर्षाणि खाभ्रखाभ्रद्विरामवेदमितानि भगणा द्वादशराशिभोगात्मक परिवर्तानां सङ्ख्या भवन्तीति शेषः। भौमशनिबृहस्पतीनां यानि शीघ्राणि शीघ्रोच्चानि तेषामेतन्मिता भगणाः। चकारः समुञ्चयार्थकोऽनुसन्धेयः। अत्र कक्षाक्रमेण चारक्रमेण वा गुरोः खलमध्यगता भवतीति न तथोद्देशः। स्वतन्त्रस्य नियोगानर्हत्वाद्वा। नन्वाकाश एषां बिम्बाभावादवलम्बनासम्भवेन गत्यभावात् कथं भगणा उक्ता इत्यत आह\textendash पूर्वयायिनाम् इति~। पूर्वगामिनाम्। तथा च तेषामदृश्यरूपाणां पूर्वगतिसद्भावाद्भगणोक्तौ न क्षतिः। एषां स्वरूपादिनिर्णयस्तु स्पष्टधिकारे प्रतिपादयिष्यते ~॥~२९~॥\\ 
 \noindent अथ चन्द्रभौमयोर्भगणानाह\textendash
\end{sloppypar}

\newpage

\hspace{3cm} गूढार्थप्रकाशकेन सहितः ~। \hfill २५
\vspace{1cm}
 
 
 \begin{quote}
 {\ssi इन्दो रसाग्नित्रित्रीषुसप्तभूधरमार्गणाः ~।\\
दस्रत्यष्टरसाङ्काक्षिलोचनानि कुजस्य तु ~॥~३०~॥}
%\vspace{2mm}
\end{quote}
\begin{sloppypar}
 पूर्वश्लोकोक्तभगणा इत्यत्राग्रिमश्लोकेष्वप्यन्वेति। भूधराः सप्त न तु पर्वतस्य धराभिधानत्वादेकसप्ततिः। मार्गणाः शरास्तथा च चन्द्रस्य भगणाः षडग्निदेवपञ्चसप्तसप्तपञ्चमिताः। भौमस्य तुकारादाकाशस्थबिम्बात्मकस्येति पुनरुक्तिभ्रमवारणार्थं दन्ताष्टषडङ्काकृतिमिताः ~॥~३०~॥\\ 
 \noindent अथ बुधशीघ्रोच्चगुर्वोर्भगणानाह\textendash
\end{sloppypar}
%\vspace{2mm}
\begin{quote}
 
  {\ssi बुधशीघ्रस्य शून्यर्तुखाद्रित्र्यङ्कनगेन्दवः ~।\\
बृहस्पतेः खदस्राक्षिवेदषड्वह्नयस्तथा ~॥~३१~॥}
%\vspace{2mm}
\end{quote}
\begin{sloppypar}
 बुधशीघ्रोच्चस्यादृश्यरूपस्य पूर्वगतेर्भगणाः षष्टिसप्ततित्र्यङ्कात्यष्टिमिताः। बृहस्पतेस्तथा बिम्बात्मकस्येति पुनरुक्तिभ्रमवारणाय नखद्विवेदषड्राममिताः ~॥~३१~॥\\ 
 \noindent अथ शुक्रशीघ्रोच्चशन्योर्भगणानाह\textendash
\end{sloppypar}
%\vspace{2mm}
\begin{quote}
 
 {\ssi सितशीघ्रस्य षट्सप्तत्रियमाश्विखभूधराः ~।\\
शनेर्भुजङ्गषट्पञ्चरसवेदनिशाकराः ~॥~३२~॥ }
%\vspace{2mm}
\end{quote}
\begin{sloppypar}
शुक्रशीघ्रोच्चस्यादृश्यरूपस्य पूर्वगतेर्भगणाः षट्सप्तत्रिद्विद्विखसप्त। एतेन भूधरा इत्यस्यैकसप्ततिरेकादश वार्थो निरस्तः। शनेर्बिम्बात्मकस्याष्टषट्षञ्चरसेन्द्रमिताः ~॥~३२~॥\\ 
\noindent अथ चन्द्रस्योच्चपातयोर्भगणानाह\textendash
\end{sloppypar}

{\tiny{E}}

\newpage

\noindent २७ \hspace{4cm} सूर्यसिद्धान्तः
\vspace{1cm}

  
  \begin{quote}
 {\ssi चन्द्रोच्चस्याग्निशून्याश्विवसुसर्पार्णवा यगे ~।\\
वामं पातस्य वस्वग्नियमाश्विशिखिदस्नकाः ~॥~३३~॥}
%\vspace{2mm}
\end{quote}
\begin{sloppypar}
 चन्द्रमन्दोच्चस्य पूर्वगतेरदृश्यरूपस्य भगणा महायुगे रामनखाष्टाष्टवेदमिताः। पातस्य चन्द्रशब्दस्य सन्निहितत्वाच्चन्द्रपातस्यादृश्यरूपस्य वामं पश्चिमगत्या द्वादशराशिभोगात्मकपरिवर्तरूपभगणा महायुग अष्टरामाकृतिरामद्विमिताः। अत्र युगग्रहणं वक्ष्यमाणग्रहोच्चपातभगणसम्बन्धिकल्पकालवारणार्थम्। ग्रहोच्चपातभगणास्तु युगे युगे नोत्पन्ना इत्यस्मिन् युगसम्बन्धिप्रसङ्गेनोक्ताः। मन्दोच्चपातस्वरूपादिनिर्णयस्तु स्पष्टाधिकारे व्यक्तो भविष्यति ~॥~३३~॥\\  
 \noindent अथ युगे नाक्षत्रदिवसांस्तत्स्वरूपावगमाय ग्रहसावनदिनस्वरूपं स्वसङ्ख्याज्ञानहेतुकं चाह\textendash
\end{sloppypar}
%\vspace{2mm}
\begin{quote}
 
 {\ssi भानामष्टाक्षिवस्वद्रित्रिद्विद्व्यष्टशरेन्दवः ~।\\
भोदया भगणैः स्वैः स्वैरूनाः स्वस्वोदया युगे ~॥~३४~॥}
%\vspace{2mm}
\end{quote}
\begin{sloppypar}
भानां नक्षत्राणां स्वतो गत्यभावेऽपि प्रवहवायुना परिभ्रमणात् तत्सङ्ख्यातुल्या भगणाः स्वदिनतुल्याः। अत एवात्र वाममिति पूर्वोक्तस्य युक्तोऽन्वयः। अष्टद्व्यष्टनगाग्निजातिगजदिनमिताः। ननु ग्रहाणामपि प्रवहवायुना परिभ्रमणेनोदयसद्भावात् तेषां दिवसाः कथं ज्ञेया इत्यत आह\textendash भोदया इति~। उदयो यस्मिन्नहनि स्वाद्यन्तावधिरूप इति व्युत्पत्त्योदयशब्देन दिनम्। तथा च भोदयानाक्षत्रदिवसा एत उक्ताः स्वैः स्वैः स्वकीयैः भगणैः प्रागुक्तैर्वर्जिताः सन्तः स्वस्वोदया
\end{sloppypar}

\newpage

\hspace{3cm} गूढार्थप्रकाशकेन सहितः ~। \hfill २७
\vspace{1cm}
 
%\begin{sloppypar}
\noindent निजनिजसावनदिवसा युगे भवन्ति। युग इत्यनेनाभीष्टकाले नाक्षत्रदिवसा ग्रहगतभोगादिना भगणादिनोना ग्रहसावनदिवसा अभीष्टा भवन्ति। परन्तु राशीन् पञ्चगुणितानंशादिकं दशगुणितं कृत्वा घट्यादिस्थाने हीनं कार्यमन्यथा विजातीयत्वादन्तरानुपपत्तेरिति सूचितम्। अत्रोपपत्तिः। यदि ग्रहाणां प्राग्गमनावलम्बनं न स्यात् तर्हि ग्रहोदयनक्षत्रोदययोरेकहेतुत्वान्नाक्षत्रसावनदिवसयोरभेदः स्यात्। अतो ग्रहाणां लम्बनेन नाक्षत्रदिवसेभ्यः सावनदिवसानामन्तरितत्वात् अवलम्बनजभगणान्तरेण युगे नाक्षत्रदिवसेभ्यो ग्रहसावनदिवसा न्यूना भवन्ति। प्रवहेण भगणतुल्यपश्चिमग्रहतुल्यानामकरणादित्युपपन्नं भोदया इत्यादि। अनेनैव भगणसावनयोगो नाक्षत्रदिवसा इत्यप्यर्थसिद्धम् ~॥~३४~॥\\ 
\noindent अथ वक्ष्यमाणचान्द्रदिवसाधिमासयोः सङ्ख्याज्ञानहेतुकं स्वरूपमाह\textendash
%\end{sloppypar}
%\vspace{2mm}

 
 \begin{quote}
 {\ssi भवन्ति शशिनो मासाः सूर्येन्दुभगणान्तरम् ~।\\
रविमासोनितास्ते तु शेषाः स्युरधिमासकाः ~॥~३५~॥}
%\vspace{2mm}
\end{quote}
\begin{sloppypar}
सूर्यचन्द्रभगणयोरन्तरं चन्द्रस्य मासा भवन्ति ते चान्द्रमासा रविमासोनिताः। अत्र प्रथमं तुकारान्वयाद् द्वादशगुणितरविभगणरूपवक्ष्यमाणार्कमासैरूनिताः सन्तः शेषा अवशिष्टा ये चान्द्रमासास्तेऽधिमासा एव भवन्ति नान्ये। अनेन चान्द्रत्वमधिमासानां स्पष्टीकृतम्। अत्रोपपत्तिः। त्रिंशत्तिथ्यात्मकस्य रवीन्दुयुतिकालरूपदर्शान्तावधेश्चान्द्रमासस्य द्वादशराशिमितेन
\end{sloppypar}

{\tiny{E 2}}


\newpage

\noindent २८ \hspace{4cm} सूर्यसिद्धान्तः 
\vspace{1cm}
 
\begin{sloppypar}
\noindent सूर्येन्द्वन्तरेण एव सिद्धिः। कथमन्यथा दर्शान्ते जातस्य मन्दशीघ्रयोः सूर्येन्द्वोर्योगस्य पुनर्दर्शान्ते सम्भवः। द्वादशराश्यन्तरं त्वेकं भगणान्तरमतो भगणान्तरेण चान्द्रो मासः सिद्धः। सौरमासापेक्षया यदन्तरेण चान्द्रमासानामधिकत्वं त एवाधिमासा इति स्वरूपमेव वक्ष्यमाणोपयोगात् परिभाषितम् ~॥~३५~॥\\ 
\noindent अथ वक्ष्यमाणावमसूर्यसावनयोः स्वरूपमाह\textendash
\end{sloppypar}
%\vspace{2mm}
\begin{quote}
 
 {\ssi सावनाहानि चान्द्रेभ्यो द्युभ्यः प्रोज्झ्य तिथिक्षयाः ~।\\  
उदयादुदयं भानोर्भूमिसावनवासराः ~॥~३६~॥}
%\vspace{2mm}
\end{quote}
\begin{sloppypar}
चान्द्रेभ्यो द्युभ्यो वक्ष्यमाणचान्द्रदिवसेभ्यः सकाशादित्यर्थः। सावनाहानि सावनदिनानि प्रोज्झ्य त्यक्त्वावशेषं तिथिक्षयाः। तिथिषु चान्द्रदिनेषु सावनदिनानामवशेषतुल्यः क्षयो न्यूनत्वम्। यद्वा तिथिशब्देन सावनो दिवसस्तस्य चान्द्रदिवसात् क्षय इति स्वरूपमेव वक्ष्यमाणोपयोगात् परिभाषितम्। ननु भोदया भगणैरित्यादिना पूर्वं सर्वेषां सावनदिवसा उक्ता इत्यत्र कस्य ग्राह्या इत्यतः सूर्यसावनस्वरूपकथनच्छलेनोत्तरमाह\textendash उदयादिति~। सूर्यस्योदयकालमारभ्याव्यवहिततदुदयकालपर्यन्तं यः काल स एको दिवसः। इति ये दिवसास्ते भूमिसावनवासराः। भूदिवसा उदयस्य भूसम्बन्धेनावगमात्। सावनदिवसाश्चेत्यर्थः। तथा च निरुपपदसावनभूमिशब्दाभ्यां सूर्यस्य वासरा एव नान्येषां सोपपदत्वाभावादिति भावः ~॥~३६~॥\\ 
\noindent ते कियन्त इत्यतस्तत्प्रमाणं चान्द्रदिनप्रमाणं चाह\textendash
\end{sloppypar}

\newpage

\hspace{3cm} गूढार्थप्रकाशकेन सहितः ~। \hfill २९
\vspace{1cm}

 
 \begin{quote}
{\ssi वसुद्व्यष्टाद्रिरूपाङ्कसप्ताद्रितिथयो युगे ~।\\
चान्द्राः खाष्टखखव्योमखाग्निखर्तुनिशाकराः ~॥~३७~॥}
%\vspace{2mm}
\end{quote}
\begin{sloppypar}
अष्टाश्विगजसप्तभूगोनसप्तपञ्चभूमिता युगे सूर्यसावनदिवसाः। चान्द्रा
दिवसा युगतिथय इत्यर्थः। अशीतिशून्यचतुष्कत्रिखनृपा एते त्रिंशद्भक्ताश्चान्द्रमासा उक्तप्रायाः। अनेन एव चान्द्रदिवसानामुपपत्तिः सूर्यचन्द्रयोर्भगणयोः अन्तररूपचान्द्रमासास्त्रिंशद्गुणिता इति स्पष्टीकृता ~॥~३७~ ॥\\ 
\noindent अथाधिमासावमयोः सङ्ख्यामाह\textendash
\end{sloppypar}
%\vspace{2mm}
\begin{quote}
 
{\ssi षड्वह्नित्रिहुताशाङ्कतिथयश्चाधिमासकाः ~।\\
तिथिक्षया यमार्थाश्विद्व्यष्टव्योमशराश्विनः ~॥~३८~॥}
%\vspace{2mm}
\end{quote}
\begin{sloppypar}
अधिमासकाः प्रागुक्तस्वरूपाः चकाराद्युगे षड्देवरामगोशरेन्दुमितास्तिथिक्षया दिनक्षया अवमानीत्यर्थः अर्थाः पञ्च। एवं द्विशराकृत्यष्ट खतत्वानि ~॥~ ३८~ ॥\\
\noindent ननु सूर्यामासानुक्तेः अधिमाससङ्ख्या कथं ज्ञाता इत्यतो रविमाससङ्ख्यां स्वरूपेण क्वहांश्च आह\textendash
\end{sloppypar}
%\vspace{2mm}
\begin{quote}
 
{\ssi खचतुष्कसमुद्राष्टकुपञ्च रविमासकाः ~।\\
भवन्ति भोदया भानुभगणैरूनिताः क्वहाः ~॥~३९~॥}
%\vspace{2mm}
\end{quote}
\begin{sloppypar}
सूर्यमासा द्वादशगुणितरविभगणानुरूपाः शून्यखाभ्रखवेदधृतिशरमिताः। ननु सावनदिवससङ्ख्या प्रागुक्ता कथमवगतेत्याह\textendash भवन्तीति~। भोदया नाक्षत्रदिवसाः प्रागुक्ताः, सूर्यभगणैः प्रागुक्तैर्वर्जिताः सन्तः क्वहा भूवासरा भवन्ति। भोदया इत्यादि प्रागुक्तेः ~॥~ ३९~ ॥\\ 
\noindent ननु सूर्यादि मन्दोच्चभौमादिपातानां 
\end{sloppypar}

\newpage

\noindent ३० \hspace{4cm} सूर्यसिद्धान्तः
\vspace{1cm}

\begin{sloppypar}
\noindent युगे भगणानुत्पत्तेः कल्पभगणकथनमावश्यकमतस्तत्पङ्क्त्यां प्रागुक्ता एते भगणादयः कल्प एव कथं नोक्ता इत्यत आह\textendash
\end{sloppypar}
%\vspace{2mm}
\begin{quote}
 
  {\ssi अधिमासोनरात्र्यर्क्षचान्द्रसावनवासराः ~।\\
एते सहस्रगुणिताः कल्पे स्युर्भगणादयः ~॥~४०~॥}
%\vspace{2mm}
\end{quote}
\begin{sloppypar}
 एते प्रागुक्ता भगणादयो भगणा आदिर्येषां ते भगणादयः। अधिमासोनरात्र्यर्क्षचान्द्रसावनवासरा अधिमासाः षड्वह्नीत्यादि तिथिक्षया इत्याद्यूनरात्रयोऽवमानि। ऋक्षचान्द्रसावनानां प्रत्येकं वासरसम्बन्धः। नाक्षत्रदिवसा भानामित्यादि। चान्द्रदिवसाश्चान्द्राः खाष्टेत्यादि। सावनदिवसा वसुद्व्यष्टाद्रीत्यादि। अत्र सौरमासा अपि खचतुष्केत्यादि ग्राह्याः। सहस्रगुणिताः कल्पे भगणादय उक्ता भवन्ति युगसहस्रस्य कल्पत्वात्। तथा च लाघवार्थं युग उक्ता इति भावः ~॥~४०~॥\\
 \noindent अथ श्लोकाभ्यां विचन्द्रसूर्यादिग्रहाणां मन्दोच्चभगणान् वदन् पातभगणान् प्रतिजानीते\textendash
\end{sloppypar}
%\vspace{2mm}
\begin{quote}
 
  {\ssi प्राग्गतेः सूर्यमन्दस्य कल्पे सप्ताष्टवह्नयः ~।\\
कौजस्य वेदखयमा बौधस्याष्टर्तुवह्नयः ~॥~४१~॥

खखरन्ध्राणि जैवस्य शौक्रस्यार्थगुणेषवः ~।\\
गोऽग्नयः शनिमन्दस्य पातानामथ वामतः ~॥~४२~॥}
%\vspace{2mm}
\end{quote}
\begin{sloppypar}
 प्राग्गतेः कल्प इत्यनयोः शनिमन्दान्तं प्रत्येकं सम्बन्धः। पूर्वगतेः सूर्यमन्दोच्चस्य कल्पे सप्ताष्टराममिताः शनिपातस्य
\end{sloppypar}

\newpage

\hspace{3cm} गूढार्थप्रकाशकेन सहितः ~। \hfill ३१
\vspace{1cm}

\begin{sloppypar}
\noindent भगणा इति वक्ष्यमाणस्य भगणा इति पदमत्र प्रत्येकमन्वेति। कौजस्य कुजसम्बन्धिनः सूर्यमन्दस्य इत्यस्यैकदेशो मन्दस्येति मन्दोच्चस्येत्यर्थकमत्रान्वेति। तथा च भौममन्दोच्चस्य चतुरधिकं शतद्वयम्। बौधस्य बुधमन्दोच्चस्याष्टषट्त्रिमिताः। जैवस्य गुरुसम्बन्धिनः। अत्र शनिमन्दस्येति वक्ष्यमाणस्यैकदेशो मन्दस्येति मन्दोच्चस्येत्यर्थकमन्वेति एकवृत्तस्थत्वात्। यद्वा आद्यन्तयोर्मन्दस्येत्युक्त्यैव मध्यस्थानामन्वयः सूपपन्न इति। तथा च गुरुमन्दोच्चस्य नवशतं शौक्रस्य शुक्रमन्दोच्चस्य पञ्चत्रिंशदधिकपञ्चशतं शनिमन्दोच्चस्यैकोनचत्वारिंशत्। अथानन्तरं पातानां भौमादिपातानां वामतः पश्चिमगत्या भगणा उच्यन्त इति शेषः ~॥~४२~॥\\ 
\noindent तान् श्लोकाभ्यामाह\textendash
\end{sloppypar}
%\vspace{2mm}
\begin{quote}
 
  {\ssi मनुदस्रास्तु कौजस्य बौधस्याष्टाष्टसागराः ~।\\
कृताद्रिचन्द्रा जैवस्य त्रिखाङ्काश्च भृगोस्तथा ~॥~४३~॥

शनिपातस्य भगणाः कल्पे यमरसर्तवः ~।\\
भगणाः पूर्वमेवात्र प्रोक्ताश्चत्रोच्चपातयोः ~॥~४४~॥}
%\vspace{2mm}
\end{quote}
\begin{sloppypar}

कुजसम्बन्धिनः। तुकारात् पातस्य भौमपातस्य कल्पे भगणाश्चतुर्दशाधिकं शतद्वयम्। बौधस्य बुधसम्बन्धिनः शनिपातस्य इत्यस्यैकदेशः पातस्य इत्यत्रान्वेति। बुधपातस्य द्वादशोना पञ्चशती। जैवस्य गुरुपातस्य चतुःसप्तत्यधिकं शतम्। भृगोः शुक्रस्य तथा सम्बन्धिनश्चकारात् पातस्य शुक्रपातस्येत्यर्थः। अधिकानवशती। शनिपातस्य द्विरसषट्का भगणाः कल्पे भवन्ति। नन्व
\end{sloppypar}

\newpage

\noindent ३२ \hspace{4cm} सूर्यसिद्धान्तः 
\vspace{1cm}

\begin{sloppypar}
\noindent स्मिन् प्रसङ्गे चन्द्रस्योच्चपातयोर्भगणाः कथं नोक्ता इति मन्दाशङ्कापाकरणाय पूर्वोक्तं स्मारयति। भगणा इति। चन्द्रोच्चपातयोश्चन्द्रस्य मन्दोच्चपातयोर्भगणा अत्रास्मिन्नधिकारे पूर्वं ग्रहयुगभगणकथने। एवकारो विस्मरणनिरासार्थकः प्रोक्ताश्चन्द्रोच्चस्य इत्यादिश्लोकेनोक्ताः ~॥~४४~॥\\
\noindent अथाभिमतकाले ग्रहगतभोगानयनं विवक्षुस्तदुपजीव्याहर्गणसाधनार्थं प्रवृत्तग्रहचारकालाद्गताब्दज्ञानोपजीव्यं कृतयुगान्तीयगताब्दज्ञानं श्लोकत्रयेणाह\textendash
\end{sloppypar}
%\vspace{2mm}
\begin{quote}

  {\ssi षण्मनूनां तु सम्पीड्य कालं तत्सन्धिभिः सह ~।\\
कल्पादिसन्धिना सार्धं वैवस्वतमनोस्तथा ~॥~४५~॥

युगानां त्रिघनं यातं तथा कृतयुगं त्विदम् ~।\\
प्रोज्झ्य सृष्टेस्ततः कालं पूर्वोक्तं दिव्यसङ्ख्यया ~॥~४६~॥

सूर्याब्दसङ्ख्यया ज्ञेया कृतस्यान्ते गता अमी ~।\\
खचतुष्कयमाद्र्यग्निशररन्ध्रनिशाकराः ~॥~४७~॥}
%\vspace{2mm}
\end{quote}
\begin{sloppypar}
 षण्मनूनां कालं सौरवर्षात्मकं तत्सन्धिभिः षण्मनूनां कृतयुगप्रमाणैः षड्भिः सन्धिभिः सह सार्धं कल्पादिसन्धिना कृतप्रमाणः कल्पादौ इत्यनेन कल्पप्रारम्भसम्बद्धकृतयुगमितसन्धिना सार्धं सार्धं सम्पीड्यैकीकृत्य तुकारादायुषोऽर्धमितं तस्येत्यस्य निरासः। वैवस्वतमनोः वर्तमानसप्तमवैवस्वताख्यस्य मनोर्युगानां त्रिघनं यातं युगसप्तविंशतिर्गता तथैकीकृत्येदमष्टाविंशतियुगान्तर्गतं तुकारात् साम्प्रतं स्थितं कुतयुगं तथा गतत्वेनैकीकृत्य
\end{sloppypar}

\newpage

\hspace{3cm} गूढार्थप्रकाशकेन सहितः~। \hfill ३३
\vspace{1cm}

\begin{sloppypar}

\noindent ततः सिद्धाङ्कात् सृष्टेः कालं सृष्टिकरणार्थं यः कालो वर्षात्मकस्तं दिव्यसङ्ख्यया दिव्यमानेन पूर्वोक्तं कृताद्रिवेदा दिव्याब्दाः शतघ्ना इत्यनेनोक्तम्। सूर्याब्दसङ्ख्यया सौरवर्षमानेन षष्ट्यधिकशतत्रयगुणितं कृत्वेति तात्पर्यार्थः। एतेन प्रागुक्तैकीकरणं सौरवर्षप्रमाणेन न दिव्यवर्षप्रमाणेनेति व्यक्तीकृतम्। प्रोज्झ्य न्यूनीकृत्य चः समुच्चयार्थोऽनुसन्धेयः। अमी अवशिष्टाब्दाः खाभ्रखाभ्रद्विसप्तत्रिशरातिधृतयः कृतयुगचरणस्यावमाने गता अतीता ज्ञातव्याः। ननु कल्पादस्माच्च मनव इत्यादिपूर्वोक्तसम्पिडितकालोक्त्येदं षण्मनूनामित्यादि पुनरुक्तमाभाति। न च पूर्वं ब्रह्मगतवयः प्रमाणज्ञानार्थमिदानीं च ग्रहसाधनार्थम्। अन्यथा गतब्रह्मवयः प्रमाणद्ग्रहसाधनापत्तेरिति वाच्यम्। ब्रह्मगतवयः प्रमाणादेव ग्रहसाधनस्य युक्तत्वादिष्टापत्तेः। अन्यथा ग्रहचक्रादेर्ब्रह्मोत्पत्तितस्तदवसानपर्यन्तं सत्त्वाद्ब्रह्मदिनाधिककालो गताब्दज्ञानाभावाद्ग्रहसाधनानुपपत्तिरिति चेन्न। इत्थं युगसहस्रेण भूतसंहारकारकः कल्प इत्यनेन ब्रह्मदिनान्ते ग्रहचक्रादिनाशोक्तेः तद्दिनादौ ग्रहचक्रोत्पत्तेश्च ब्रह्मदिवस एव तदादिगताब्दा ग्रहचारोपजीव्या न ब्रह्मगतायुः प्रमाणाब्दाः। ग्रहासत्त्वे ग्रहसाधनापत्तेः। अतः पुनर्गताब्दा ग्रहचारोपजीव्या ब्रह्मदिवसे साधिताः। परन्तु ब्रह्मदिनादितो ग्रहचारप्रवृत्तिकालपर्यन्तं यः सृष्टि विलम्बितकालस्तदूना ब्रह्मदिनादिगताब्दाः सृष्टिगताब्दा ग्रहसाधनोपजीव्या इति तथोक्तम्। अन्यथा सृष्ट्यन्तर्गतकाले ग्रहचारासत्त्वे तत्साधनापत्तेः सृष्टिकालकथ\textendash
\end{sloppypar}

{\tiny{F}}

\newpage

\noindent ३४ \hspace{4cm} सूर्यसिद्धान्तः
\vspace{1cm}
 
\begin{sloppypar}
\noindent नानुपपत्तेश्चेति दिक्। यथा दिव्याब्दस्य सौरवर्षाणि ३६० द्वादशसहस्त्रगुणितानि महायुगम् ४३२०००० इदमेकसप्ततिगुणं मनुमानम् ३०६७२०००० इदं षड्गुणितं षण्मनुमानम् ९८४०३२०००० इदं स्वसन्धिभिः कृतयुगप्रमाणैः सप्तभिरेभिः १२०९६००० युतम् १८५२४१६००० एतत् सप्तविंशतियुग ११६६४०००० सहितम् १९६९०५६००० कृतयुग १७२८००० युक्तं जातानि कल्पगतवर्षाणि १९७०७८४००० सृष्टिदिव्याब्दैः ४७४०० खषडग्निगुणितैरेभिः १७०६४००० हीनं गताब्दा ग्रहचारोपजीव्याः कृतयुगान्ते खचतुष्केत्याद्युपपन्नाः १९५३७२०००० ~॥~४७~॥\\ 
\noindent अथाभीष्टकालेऽहर्गणसाधनं ततो दिनमासाब्दपप्रतिज्ञां वासरेश्वरज्ञानं च श्लोकचतुष्टयेनाह\textendash
\end{sloppypar}
%\vspace{2mm}
\begin{quote}

 {\ssi अत ऊर्ध्वममी युक्ता गतकालाब्दसङ्ख्यया ~।\\
मासीकृता युता मासैर्मधुशुक्लादिभिर्गतैः ~॥~४८~॥

पृथक्स्थास्तेऽधिमासघ्नाः सूर्यमासविभाजिताः ~।\\
लब्धाधिमासकैर्युक्ता दिनीकृत्य दिनान्विताः ~॥~४९~॥

द्विष्ठास्तिथिक्षयाभ्यस्ताश्चान्द्रवासरभाजिताः~।\\
लब्धोनरात्रिरहिता लङ्कायामार्धरात्रिकाः ~॥~५०~॥

सावनो द्युगणः सूर्याद्दिनमासाब्दपास्ततः ~।\\
सप्तभिः क्षयितः शेषः सूर्याद्यो वासरेश्वरः ~॥~५१~॥}
%\vspace{2mm}
\end{quote}
 \begin{sloppypar}
 अतः कृतयुगान्तादूर्ध्वमुपर्यनन्तरमित्यर्थः। अभीष्टकाले यो गतकालस्तस्य सौरवर्षसङ्ख्यया अमी कृतयुगान्तीयसृष्ट्यब्दाः
\end{sloppypar}

\newpage

\hspace{3cm} गूढार्थप्रकाशकेन सहितः ~। \hfill ३५
\vspace{1cm}

\begin{sloppypar}
\noindent खचतुष्केत्यादि पूर्वोक्ता युक्ता अभीष्टकाले सौरगताब्दा भवन्ति। एते मासीकृता द्वादशगुणिता इत्यर्थः। अभीष्टकाले मधुशुक्लादिभिश्चैत्रशुक्लाद्यवधिभूतैः गतैः मासैर्युताः। अत्र गतमासान्तर्गतोऽधिमासश्चेन्न ग्राह्यस्तस्योत्तरमासाह्वयत्वेन तदन्तर्गतत्वात् तन्मासस्य षष्टिदिनात्मकत्वाच्च। ते सिद्वाः पृथक्स्था युगाधिमासगुणिता युगसूर्यमासभक्ताः प्राप्ताधिमासकैर्निरग्रैः सिद्धायुक्ताः। अत्र यदा स्पष्टोऽधिमासः पतित आनयने न लब्धस्तदानयनप्राप्ताधिमासैः सैकैर्युक्ताः। यदा तु स्पष्टोऽधिमासो न पतित आनयने प्राप्तस्तदानयनप्राप्ताधिमासैर्निरेकैर्युक्ताः। अन्यथाभीष्टकालसाधिताहर्गणस्य त्रिंशद्दिनान्तरितत्वापत्तेरिति ध्येयम्। एते सिद्धा दिनीकृत्य त्रिंशता सङ्गुण्येत्यर्थः। दिनान्विता वर्तमानमासस्य शुक्लप्रतिपदादिगततिथिभिर्युक्ता इत्यर्थः। एते द्विष्ठाः स्थानद्वये स्थाप्या एकत्र युगावमैर्गुणिता युगचान्द्रदिनैर्भक्ताच्च प्राप्तावमैर्निरग्रैरपरत्र हीनाः सन्तो लङ्कादेशेऽर्धरात्रकालिकः सावनोऽहर्गणः स्यात्। ततः साधिताहर्गणात् सकाशात् सूर्यात् सूर्यमारभ्य दिनमासाब्दपा वारेश्वरमासेश्वरवर्षेश्वरा भवन्ति। तत्र वासरेश्वरज्ञानमाह\textendash सप्तभिः इति~। अयमहर्गणः सप्तभिः क्षयितो भक्त्वा शेषितः कार्यः। स शेषोऽवशिष्टः सूर्याद्यः सूर्यवारादिको वासरेश्वरो वारस्वामी भवति । तदग्रिमो वर्तमानो वारेश इत्यर्थसिद्धम्।  अत्रोपपत्ति| सौरवर्षाणां मासकरणे सृष्ट्याद्यधिमासान्तकालसम्बन्धिसावयवसौरमासा अव्यवहितपूर्वपतिताधिमासान्तकालादिस्वाभीष्ट\textendash
\end{sloppypar}
{\tiny{F2}}

\newpage

\noindent ३६ \hspace{4cm} सूर्यसिद्धान्तः
\vspace{1cm}

\begin{sloppypar}
\noindent चैत्राद्यन्तकालसम्बन्धि सावयवचान्द्रमासाः तयोर्योगः चैत्रादौ द्वादशगुणितसौरवर्षाणि जातानि कुत इति चेच्छृणु। द्वादशगुणितसौरवर्षाणि सौरवर्षादौ सौरमासा इति तु निर्विवादम् । ते स्वानीताधिमासैः सावयवैर्युताश्चान्द्राः सावयवाः सौरवर्षादौ। एतेऽवयवहीनाः चैत्रादौ निरवयवाश्चान्द्राधिमासाः। अवयवस्य चैत्रादिसौरवर्षाद्यन्तरकालरूपाधिशेषत्वात्। ते निरग्राधिमासोनाश्चैत्रादौ अधिमासोनचान्द्रा द्वादशगुणितमौसौरवर्षरूपा उक्तयोगस्वरूपाः सिद्धाः। कथमन्यथा निरग्राधिमासयोजनेनैषां चैत्रादौ चान्द्रमासमानत्वसम्भवः। एते स्वाभीष्टमासादिकालसिद्ध्यर्थं चैत्रशुक्लादिगतमासैर्युक्ताः। एतेन द्वादशगुणितसौरवर्षमितसौरमासानां चैत्रादिगतचान्द्रमासाः कथं योजिता एकजातित्वाभावादिति दूषणाङ्गीकारो निरस्तः। उक्तरीत्या तत्र चान्द्रमासानामपि सत्त्वादेकजातीयत्वेन योगसम्भवात्। न हि पूर्वयोगोऽस्माभिः कृतो येन विजातीययोगो दूषणं तस्य द्वादशगुणितसौरवर्षरूपत्वेन स्वतः सिद्धत्वात्। अथैषां निरग्राधिमासा योज्या इति सृष्ट्यादिपूर्वपतिताधिमासान्तकालावधि ये सौरमासाः सावयवास्तेभ्यो युगसौरमासैर्युगाधिमासास्तदा एभिः सौरमासैः क इत्यनुपातेन निरग्राधिमासाश्चान्द्रा भवन्ति सौरेभ्यः साधितत्वात्। अथाभीष्टकालेऽधिमासावयवज्ञानार्थं युगचान्द्रमासैर्युगाधिमासास्तदा पूर्वपतिताधिमासान्तकालाभीष्टमासाद्यन्तरस्थितचान्द्रमासैः सावयवैरेभिः क इत्यनुपातेनाधिमासाभावात् तदवयवः सौर आ\textendash
\end{sloppypar}

\newpage

\hspace{3cm} गूढार्थप्रकाशकेन सहितः~। \hfill ३७
\vspace{1cm}

\begin{sloppypar}
\noindent याति चान्द्रात् साधितत्वात्। परन्तु अवयवावयविनोः एकजातित्वासिद्धिरतः तत्सम्पादनार्थमधिमासावयवस्योक्तसौरस्य युगसौरमासैर्युगचान्द्रमासास्तदा उक्तसौराधिमासावयवेन किमित्यनुपातेन युगचान्द्रमासा गुणो युगसौरमासा हर इति तुल्ययोर्गुणहरयोर्युगचान्द्रमासयोर्नाशादिष्टचान्द्रमासानां युगाधिमासा गुणो युगसौरमासा हर इति फलमधिमासावयवश्चान्द्रः। अथ तादृशेष्टसौरचान्द्रमासयोः पृथगज्ञानादधिमासतदवयवयोर्ज्ञानमशक्यमपि एको हरश्चेद्गुणकौ विभिन्नौ इत्यादिरीत्येष्टतादृशसौरचान्द्रमासयोर्योग एवायं ज्ञातो युगाधिमासगुणितो युगसूर्यमासभक्तः फलमधिमासाः। शेषात् तदवयवोऽहर्गणानयनेऽनुपयुक्तः। तत्र केवलाधिमासानामेव न्यूनत्वेन तेषामेव योजनावश्यकत्वात्। अयं सृष्ट्यादित इष्टमासादिपर्यन्तं चान्द्रमासगणः सिद्धः। बहवस्तु द्वादशगुणितसौरवर्षरूपसौरमासानां सौरवर्षादितोऽभीष्टकालपर्यन्तं सौरमासानामज्ञानाज्ज्ञातचैत्रादिगतचान्द्रमासा एव योजिताः परमिष्टसौरमासेष्वधिमासशेषमधिकं तच्चाधिमासानयनेऽधिशेषत्यागेन केवलाधिमास योजने निरन्तरं भवति। अधिमासानयनं च चान्द्रमिष्टसौरमासत्वेनैव अधिशेषाधिकेष्टसौरमासानामङ्गीकारादित्याहुः\textendash तच्चिन्त्यम्~। केवलेष्टसौरमासानीताधिमासानां निरग्राणामधिशेषाधिकसौरेष्टमासेषु योजनेनैव निरन्तरितत्वसिद्धेः। अन्यथाधिशेषगुणितयुगाधिमासेभ्यो युगार्कमासभक्ताप्तफलेनाधिशेषमधिकमायातीति परमासन्नाधिशेषस्याधिकत्वे भवद्रीत्यनुपा\textendash
\end{sloppypar}

\newpage

\noindent ३८ \hspace{4cm} सूर्यसिद्धान्तः 
\vspace{1cm}

\begin{sloppypar}
\noindent तानयनेन एकाधिकाधिकमासलब्ध्या योजितेन चान्द्रमासगण एकाधिकः स्यादिति। अथाभीष्टमासादिसिद्धचान्द्रमासाश्चान्द्रदिनकरणार्थंं त्रिंशद्गुणिता अभीष्टदिने तत्सिद्ध्यर्थं शुक्लादिगततिथयोऽत्र योजिता अभीष्टतिथ्यादौ चान्द्राहर्गणः। युगचान्द्रदिनैर्युगावमानि तदानेन किमित्यनुपातागतावमैः सावयवैः हीनाश्चान्द्राहर्गणस्तिथ्यन्ते सावनोऽहर्गणः यमकोटिदेशे सूर्योदयकाले ग्रहचारस्य प्रवृत्तेस्तदादितो निरवयवाहर्गणसिद्ध्यर्थं तिथ्यन्ततत्कालयोरन्तरमवमावयवरूपं योज्यमतः पूर्वमेवावमावयवोऽनुपयुक्तोऽत्र न गृहीतोऽतश्चान्द्राहर्गणः स्वानीतावमैर्निरग्रैर्हीनोऽहर्गणः सावनो निरवयवो यमकोटिदेशीयसूर्योदयकाले तत्र तद्देशस्याप्रसिद्धतया प्रसिद्धलङ्कादेशार्द्धरात्रस्य तद्रूपस्योक्तिः कृता। सृष्ट्यादावर्कवारसद्भावात् तदाद्या दिनमासवर्षेश्वराः। ग्रहाणां सप्तसङ्ख्यत्वात् सप्ततष्टोऽहर्गणः शेषं गतवारः ~॥~५१~॥\\ 
\noindent अथ प्रतिज्ञातयोर्मासवर्षपयोरानयनमाह\textendash
\end{sloppypar}
%\vspace{2mm}
\begin{quote}

 {\ssi मासाब्ददिनसङ्ख्याप्तं द्वित्रिघ्नं रूपसंयुतम् ~।\\
सप्तोद्धृतावशेषौ तु विज्ञेयौ मासवर्षपौ ~॥~५२~॥}
%\vspace{2mm}
\end{quote}
 \begin{sloppypar}
 अहर्गणाद्विष्ठादेकत्र मासदिनानां सङ्ख्यया त्रिंशता भक्तादाप्तं फलम्। अपरत्र वर्षदिनानां सङ्ख्यया षष्ट्यधिकशतत्रयेण भक्तादाप्तं फलम्। शेषयोरनुपयोगात् त्यागः। क्रमेण फलद्वयं द्वाभ्यां त्रिभिर्गुणितमुभयत्रैकसङ्ख्यायुक्तं सप्तभागहारेण भक्तात् फलत्यागेनावशिष्टौ क्रमेण मासस्वामिवर्षस्वामिनौ ज्ञातव्यौ
\end{sloppypar}

\newpage

\hspace{3cm} गूढार्थप्रकाशकेन सहितः ~। \hfill ३९
\vspace{1cm}

\begin{sloppypar}
\noindent तुकाराद्व्युत्क्रमेण वारेश्वरगणना तत्क्रमेणानयोर्गणना परमत्र वर्तमानेत्यर्थः।अत्रोपपत्तिः| सृष्ट्यादित्रिंशदहोरात्राणामेकः सौरसावनमासस्तस्य सूर्योऽधिपतिर्मासादिदिनेऽर्कस्याधिपतित्वात्। एवं द्वितीयमासादौ भौमस्य दिनाधिपतित्वाद्भौमो द्वितीयमासेश्वर इति प्रतिमासं मासेश्वरयोरन्तरं द्वयम्। त्रिंशद्दिनानां सप्ततष्टतया द्व्यवशेषात्। एवं षष्ट्यधिकशतत्रयाहोरात्राणामेकं सौरसावनवर्षं तस्याधिपोऽर्कः। वर्षादिदिनेऽर्कस्याधिपतित्वात्। एवं द्वितीयसावनवर्षादौ बुधस्य दिनाधिपतित्वाद्बुधो द्वितीयवर्षेश्वर इति प्रतिवर्षं वर्षेश्वरयोरन्तरं त्रयं षष्ट्यधिकशतत्रयदिनानां सप्ततष्टतया त्र्यवशेषात्। तथा च वर्तमानकाले तद्गणनया कियन्तो मासा गताः कियन्ति च वर्षाणि गतानीति ज्ञानार्थमहर्गणस्त्रिंशद्भक्तः फलं गतमासाः षष्ट्यधिकशतत्रयभक्तः फलं गतवर्षाणि। एकमासे द्वौ वारौ तदा गतमासैः क इति गतमासवारा वर्तमानार्थं सैकाः। एवमेकवर्षे त्रयो वारास्तदा गतवर्षैः क इति गतवर्षवारा वर्त्तमानार्थं सैकावाराणां सप्तसङ्ख्यत्वात् सप्ततष्टौ शेषौ सूर्यादिकौ मासवर्षेश्वरौ~॥~५२~॥\\ 
\noindent अथ ग्रहायनमाह\textendash
\end{sloppypar}
%\vspace{2mm}

\begin{quote}

 {\ssi यथास्वभगणाभ्यस्तो दिनराशिः कुवासरैः ~।\\
विभाजितो मध्यगत्या भगणादिर्ग्रहो भवेत् ~॥~५३~॥}
%\vspace{2mm}
\end{quote}
\begin{sloppypar}
 दिनराशिरहर्गणो यथास्वभगणाभ्यस्तो यत्कालिकनिजोक्तभगणैर्गुणितो युगभगणैः कल्पभगणैर्वत्यर्थः। तथा कुवासरै\textendash
\end{sloppypar}
\newpage


\noindent ४० \hspace{4cm} सूर्यसिद्धान्तः
\vspace{1cm}

\begin{sloppypar}
\noindent स्तात्कालिकसावनदिनैर्युगसावनैः कल्पसावनैर्वेति यथायोग्यमित्यर्थः। भक्तः फलं यस्य ग्रहस्य भगणा गुणनार्थं गृहीताः स ग्रहो भगणादिर्भगणराशिभागकलाविकलात्मकभोगात्मकः। मध्यगत्या मध्यगतिमानेन न प्रतिदिनविलक्षणस्फुटगतिप्रमाणेन अग्रे तत्प्रमाणेन ग्रहभोगज्ञानस्योक्तेः। मध्यमो ग्रहः स्यादित्यर्थः। अत्रोपपत्तिः। युगादिसावनैर्युगादिभगणास्तदा एकेन दिनेन केति प्राप्ता मध्यमगतिस्तत एकेन दिनेनेयं गतिस्तदेष्टाहर्गणेन केति रूपयोस्तुल्यत्वेन विकाराजनकत्वाच्चनाशादुपपन्नमानयनम्। यद्यपि युगादिसावनैर्युगादि भगणास्तदेष्टाहर्गणेन किमित्येकानुपातेनानयनमुपपन्नं लाघवात् तथापि मध्यगत्येत्यस्य प्रदर्शनार्थमनुपातद्वयं गुरुभूतमपि प्रदर्शितम् ~॥~५३~॥\\
\noindent अथामुं प्रकारमुच्चपातयोरानयनायातिदिशति\textendash
\end{sloppypar}
%\vspace{2mm}
\begin{quote}

  {\ssi एवं स्वशीघ्रमन्दोच्चा ये प्रोक्ताः पूर्वयायिनः ~।\\
विलोमगतयः पातास्तद्वच्चक्राद्विशोधिताः ~॥~५४~॥}
%\vspace{2mm}
\end{quote}
\begin{sloppypar}

 ये पूर्वयायिनः पूर्वदिग्गतयः स्वशीघ्रमन्दोच्चाः स्वेषां ग्रहाणां शीघ्रोच्चमन्दोच्चा ग्रहबहुत्वेन
शीघ्रोच्चमन्दोच्चयोर्बहुत्वात् बहुवचनम्। प्रोक्ताः पूर्वं भगणोक्त्या कथितास्तेऽप्येवं ग्रहानयनरीत्या साध्याः। ननु पूर्वयायिन एवं साध्यास्तर्हि पश्चिमगतयः पाताः कथं साध्या इत्यत आह\textendash विलोमगतय इति~। पश्चिमगतयः पाता अपि तद्वद्ग्रहानयनरीत्या अत्र चन्द्रोच्चपातौ ग्रहानयनवद्युगकल्पभगणसावनाभ्यां सिद्धौ भवतोऽन्येषामुच्चपातौ
\end{sloppypar}

\newpage

\hspace{3cm} गूढार्थप्रकाशकेन सहितः ~। \hfill ४१
\vspace{1cm}

\begin{sloppypar}
तु कल्पसावनदिनहरेणेति ध्येयम्। ननु तर्हि पूर्वपश्चिमगत्योः को विशेष आनयन इत्यत आह\textendash चक्रात् इति~। आगता राश्यादिपाता द्वादशराशिभ्यः शोध्याः पाता भवन्ति। एतावानेव विशेष इति भावः।  अत्रोपपत्तिः। पूर्वयायिनो मेषवृषमिथुनादिक्रमेण गच्छन्ति पश्चिमगतयस्तु मेषमीनकुम्भेत्याद्युत्क्रमेण गच्छन्ति। तत्रोत्क्रमगणनाया लोकेऽनभ्यासाद्राशिक्रमेण तज्ज्ञानार्थं द्वादशराशिभ्यः शोधिताः पूर्वगतिपङ्क्तिस्था भवन्ति~॥~ ५४~ ॥\\ 
\noindent अथ संवत्सरानयनमाह\textendash
\end{sloppypar}
%\vspace{2mm}
\begin{quote}

 {\ssi द्वादशघ्ना गुरोर्याता भगणा वर्तमानकैः ~।\\
राशिभिः सहिताः शुद्धाः षष्ट्या स्युर्विजयादयः ~॥~५५~॥}
%\vspace{2mm}
\end{quote}
\begin{sloppypar}
 अहर्गणानीतस्य भगणादिकस्य बृहस्पतेर्याता गता भगणा उपरिस्था द्वादशगुणिता वर्तमानकैर्यस्मिन्नधिष्ठितः स वर्तमानस्तत्सहितैरेकयुक्तैरित्यर्थः। राशिभिर्गणितागतराशिभिर्यद्राशौ तिष्ठति तस्य मेषादिसङ्ख्ययेति फलितार्थः। युताः षष्ट्या शुद्धाभागावशेषिताः फलं भागादिकं चानुपयोगात् त्याज्यम्। विजयादयः संवत्सरा वर्तमानसहिता भवन्ति। अत्रोपपत्तिः।
\end{sloppypar}
%\vspace{2mm}
\begin{quote}

  {\ssi मध्यगत्या भभोगेन गुरोर्गौरववत्सराः ~।}
%\vspace{2mm}
\end{quote}
\begin{sloppypar}
 इति लघुवशिष्ठसिद्धान्तोक्तेर्गुरुमध्यमराशिभोगकाल एकः संवत्सर इति सृष्ट्यानीतभगणादिगुरोः सम्पूर्णराशिज्ञानार्थं भगणा द्वादशगुणा वर्तमानराशिसङ्ख्यायुताः षष्टितष्टाः शेषं विजयादिकाः संवत्सरो वर्तमानो भवति। संवत्सराणां षष्टिसङ्ख्य\textendash
\end{sloppypar}

{\tiny{G}}

\newpage

\noindent ४२ \hspace{4cm} सूर्यसिद्धान्तः 
\vspace{1cm}

\begin{sloppypar}
\noindent त्वात्। सृष्ट्यादौ विजयसंवत्सरसद्भावाच्च ~॥~५५~॥\\ 
\noindent अथोक्तमुपसंहरन् लाघवेन ग्रहानयनमाह\textendash
\end{sloppypar}
%\vspace{2mm}
\begin{quote}

  {\ssi विस्तरेणैतदुदितं सङ्क्षेपाद् व्यावहारिकम् ~।\\
मध्यमानयनं कार्यं ग्रहाणामिष्टतो युगात् ~॥~५६~॥}
%\vspace{2mm}
\end{quote}
\begin{sloppypar}
 एतत् षण्मनूनां तु सम्पीड्येत्यादि विस्तरेण गणितक्रियाबाहुल्येनोदितमुक्तं व्यावहारिकं लोकव्यवहारोपयुक्तमिदं ग्रहानयनं संक्षेपादल्पगणितप्रयासाज्ज्ञेयम्। तदाह\textendash मध्यमानयनं इति~।  ग्रहाणां मध्यमानयनं मध्यमानेन गणितमिष्टतो वर्तमानात् त्रेताख्यात् युगान्महायुगस्य चरणात् त्रेतायुगादितो गताब्दैरल्पभूतैरेवोक्तरीत्याहर्गणमानीयोक्तरीत्या मध्यग्रहाः कार्या इत्यर्थः ~॥~५६~॥\\
 \noindent ननु सृष्ट्यादितो ग्रहचारप्रवृत्तेस्तदादित आनीतस्य ग्रहस्य वास्तवत्वेन तत्तुल्योऽयं ग्रहः कथमवगत इत्यत आह\textendash
\end{sloppypar}
\begin{quote}

  {\ssi अस्मिन् कृतयुगस्यान्ते सर्वे मध्यगता ग्रहाः ~।\\
विना तु पातमन्दोच्चान् मेषादौ तुल्यतामिताः ~॥~५७~॥}
\end{quote}
\begin{sloppypar}
अस्मिन्निदानीन्तने कृतयुगस्यावसानमये सर्वे सप्तग्रहाः सूर्यादयो मध्यगता मध्यमा मेषादौ मेषादिप्रदेशे तुल्यतां समानतां गणितागतराश्यादिभोगेनेताः प्राप्ताः। पातमन्दोच्चान्  विना पातमन्दोच्चास्तु न तुल्या न वा मेषादौ। तथा च ग्रहाणां शीघ्रोच्चोनां च भगणपूर्तित्वात् त्रेतादिसमयावगत गतकालादागतराश्यादयः सृष्ट्यादिगतकालावगतराश्या\textendash
\end{sloppypar}

\newpage 

\hspace{3cm} गूढार्थप्रकाशकेन सहितः~। \hfill ४३
\vspace{1cm}
\begin{sloppypar}
दिभिस्तुल्या भगणानां च प्रयोजनाभावादिति भावः ~॥~५७~॥\\ 
\noindent अथोच्चपातयोर्विशेषमाह\textendash
\end{sloppypar}
\begin{quote}

 {\ssi मकरादौ शशाङ्कोच्चं तत्पातस्तु तुलादिगः ~।\\
 निरंशत्वं गताश्चान्ये नोक्तास्ते मन्दचारिणः ~॥~५८~॥}
\end{quote}
\begin{sloppypar}
चन्द्रस्य मन्दोच्चं तदानीं मकरादावस्ति तत्पातश्चन्द्रपातस्तुलादिस्थोऽस्ति। तुकारादतस्तयोस्त्रेतादित आनयनं नवषड्राशियोजनविशेषेण सुगममित्यर्थः। नन्वेवमन्येषामपि यद्राश्यादिस्थत्वं तत्कथनेन तेषामप्यानयनं सुगमं भविष्यतीत्यत आह\textendash निरंशत्वं इति~। अन्येऽवशिष्टा मन्दोच्चपाता ये मन्दचारिणोऽल्पगतय उक्ताः पूर्वं भगणोक्त्या कथितास्ते चकारादस्मिन् कृतयुगान्ते निरंशत्वमंशाभावतां न प्राप्ताः। तथा च तेषां राश्यादिकथने गौरवं मन्दगतित्वादेकदानीताः सहस्रवर्षपर्यन्तमुपयुक्ता भवन्तीति निरन्तरं तत्साधनावश्यकताभावात् तेषामानयनं त्रेतादिगताब्देभ्य उपेक्षितमिति भावः। यदि च तत आनीयन्ते तदा स्वस्वक्षेपयुक्ताः कार्याः। क्षेपकास्तु रविमन्दोच्चं राश्यादिकं ०~।~७~।~२८~।~१२~।~ भौमस्य ३~।~३~।~१४~।~२४~।~ बुधस्य ५~।~४~।~४~।~४८~।~ गुरोः ०~।~९~।~०~।~०~।~ शुक्रस्य ११~।~१३~।~२१~।~०~।~ शनेः ४~।~२०~।~१३~।~१२~।~ भौमपातस्य ९~।~११~।~२०~।~१२~।~ बुधस्य ८~।~११~।~१६~।~४८~।~ गुरोः. ८~।~८~।~५६~।~२४~।~ शुक्रस्य ४~।~१७~।~२५~।~४८~।~ शनिपातस्य ४~।~२०~।~१३~।~१२~।~ एवमिष्टकालादपि ग्रहाः साध्याः स्वस्व\textendash
\end{sloppypar}

{\tiny{G2}}

\newpage

\noindent ४४ \hspace{4cm} सूर्यसिद्धान्तः
\vspace{1cm}

\begin{sloppypar}
\noindent क्षेपयोजनपूर्वम् ~॥~ ५८~ ॥\\
\noindent अथ ग्रहाणां देशान्तरफलानयनार्थं भूपरिधिं स्वोपजीव्यभूव्यासकथनपूर्वकमाह\textendash
\end{sloppypar}
%\vspace{2mm}
\begin{quote}

 {\ssi योजनानि शतान्यष्टौ भूकर्णो द्विगुणानि तु ~।\\
 तद्वर्गतो दशगुणात् पदं भूपरिधिर्भवेत् ~॥~५९~॥}
%\vspace{2mm}
\end{quote}
\begin{sloppypar}
अष्टौ शतानि द्विगुणानि षोडशशतं योजनानि भूकर्णो भुवो भूगोलस्य कर्णो वृत्तपरिधिमध्यभागसूत्रं परिध्यर्धमितचापस्य ज्यारूपं 'द्विगुण इत्यनेन शतान्यष्टौ केन्द्रात् परिधिपर्यन्तमृजुसूत्रस्य मानमिति सूचितम्। कक्षाव्यासार्धस्य कर्णव्यवहारवदस्यापि भूकर्णव्यवहारः। तुकारात् पुराणविरुद्धोऽपि प्रत्यक्षसहकृतागमप्रमाणसिद्धः। अस्मात् परिधिज्ञानमाह\textendash तद्वर्गत इति~। भूव्यासवर्गात् तुल्योर्घातरूपाद्दशगुणान्मूलम्। कस्यायं समद्विघात इति तन्मूलं तत्प्रकारश्च ग्रन्थान्तरे प्रसिद्धः। भूपरिधिः स्यात्।  अत्रोपपत्तिः। गजाग्निवेदराममित ३४३८ त्रिज्यायाः कक्षाव्यासार्द्धत्वाद्द्विगुणत्रिज्यारूपव्यासे चक्रकलातुल्यः परिधिः २१६०० तदेष्टव्यासे क इति गुण २१६०० हरौ ६८७६ हरेणापवर्तितौ हरस्थाने रूपं गुणस्थाने सार्द्धाष्टावयवयुतास्त्रयस्तथा च व्यासोऽनेन गुणितः परिधिर्भवति। तत्र भगवता गुणस्थानकरणार्थं वर्गः कृतः ९~।~५२~।~१२~।~ अत्र स्वल्पान्तराद्दश गृहीताः। वर्गेण वर्गं गुणयेदित्युक्तत्वाद्व्यासवर्गो दशगुणितस्तन्मूलं व्यासो गुणरूपगुणगुणितः सिद्धो भवति। यद्यपि वर्गस्थाने दशग्रहणेन स्थूलमिदमानयनं तथापि परम\textendash
\end{sloppypar}

 \newpage

\hspace{3cm} गूढार्थप्रकाशकेन सहितः~। \hfill ४५
\vspace{1cm}

\begin{sloppypar}
\noindent कारुणिकेन भगवता लोकानुग्रहार्थं गणितलाघवाय अङ्गीकृतम्। वस्तुतो भगवता वेदमङ्गलविश्वरूपमितव्यासस्य ११३८४ परिधिर्गणितागतः प्रत्यक्षेण खखखरसराममितः ३६००० अत्र पूर्वोक्तरीत्यापवर्तने गुणः ३~।~९~।~४४~ पादोनदशावयवयुतं त्रयमस्य वर्गो दशप्रायः ९~।~५९~।~५९~।~ इत्युपपन्नमुक्तम् ~॥~५९~॥\\  
\noindent अथ स्फुटपरिध्यानयनं देशान्तरफलानयनं तत्संस्कारं च श्लोकाभ्यामाह\textendash
\end{sloppypar}
%\vspace{2mm}
\begin{quote}

  {\ssi लम्बज्याघ्नस्त्रिजीवाप्तः स्फुटो भूपरिधिः स्वकः ~।\\
तेन देशान्तराभ्यस्ता ग्रहभुक्तिर्विभाजिता ~॥~६०~॥

कलादि तत् फलं प्राच्यां ग्रहेभ्यः परिशोधयेत् ~।\\
रेखाप्रतीचीसंस्थाने प्रक्षिपेत् स्युः स्वदेशजाः ~॥~६१~॥}
%\vspace{2mm}
\end{quote}
\begin{sloppypar}
 द्वादशपलभयोर्वर्गयोगमूलमक्षकर्णः। अनेन द्वादशगुणिता त्रिज्या भक्ता फलं लम्बज्या। अनया गुणितो भूपरिधिस्त्रिज्यया गजाग्निवेदराममितया भक्तः फलं स्वकः स्वदेशसम्बन्धी स्पष्टो भूपरिधिः स्यात्। ग्रहस्य गतिर्देशान्तराभ्यस्ता स्वरेखादेशस्वदेशयोरन्तरयोजनानि देशान्तरपदवाच्यानि तैर्गुणिता तत्र स्पष्टेन भूपरिधिना भक्ता फलं कलादिकं तत् फलं प्राच्यां स्वरेखादेशात् स्वदेशस्य पूर्वदिग्भागस्थितत्वे ग्रहेभ्यः कलादिस्थाने परिशोधयेद्वर्जयेद्धीनं कुर्यादित्यर्थः। रेखाप्रतीचीसंस्थाने स्वरेखादेशात् पश्चिमदिग्भागस्थिते स्वदेशे ग्रहेभ्यः कलादिस्थाने प्रक्षिपेद्युक्तं कुर्यात्। गणक इति शेषः। ते सिद्धा ग्रहाः स्वदेशजाः स्वदेशीया भवन्ति। पूर्वमहर्गणस्य लङ्कादेशीयत्वेन
\end{sloppypar}

\newpage

\noindent४६ \hspace{4cm} सूर्यसिद्धान्तः
\vspace{1cm}

\begin{sloppypar}
\noindent तदुत्पन्नग्रहाणां लङ्कादेशीयत्वात् । अत्रोपपत्ति । यद्यपि भूमेः कन्दुकाकारत्वेन सर्वत्राभिन्नः परिधिरिति स्फुटपरिध्यसम्भवस्तथापि निरक्षदेशस्य मध्यत्वकल्पनेनोक्तो भूपरिधिस्तद्देशानामेव तदन्यत्र तदनुरोधेन वृत्तानां लघुत्वसम्भवेनोत्तरोत्तरं न्यूनपरिधिः स्वदेशे स्फुटसञ्ज्ञः । एवं नवत्यक्षांशे मेरुस्थाने वडवास्थाने च परिध्यभावः । निरक्षदेशे परम उक्तः परिधिरतो यत्राक्षांशा नवतिः परमास्तत्र लम्बांशाभावः । यत्राक्षांशाभावस्तत्र लम्बांशाः परमा नवतिः । लम्बांशाक्षांशौ तु वक्ष्यमाणस्वरूपौ । तथा च लम्बांशह्रासानुरोधेन परिधेरपि ह्रास इति परमलम्बांशैर्नवतिमितैरुक्तो भूपरिधिस्तदा स्वदेशीयलम्बांशैः क इत्यनुपात उपपन्नोऽपि वृत्ताश्रितांशेभ्योऽनुपातानामसम्भवेन सर्वैरुपेक्षितत्वाच्च ज्यानुपातस्य सर्वैरङ्गीकृतत्वात् प्रमाणस्थाने प्रमाणांशज्या परमा त्रिज्या । इच्छास्थाने इच्छांशानां ज्यालम्बज्येति युक्तमुक्तमुपपन्नं स्पष्टपरिध्यानयनम् । देशान्तरोपपत्तिस्तु लङ्कादेशीयो ग्रहः स्वदेशतः समसूत्रेण यो दक्षिणोत्तरयोर्निरक्षदेश आसन्नस्तत्र कार्यः । तदर्थं लङ्कादेशस्वनिरक्षदेशयोरन्तरयोजनज्ञानमावश्यकम् । एतत् त्वस्मादृशामशक्यमिति परिध्यपचयवत् तदन्तरतापचितं लङ्कोत्तरदक्षिणसूत्रस्थस्वरेखादेशस्वदेशयोरन्तरं स्वपरिधिस्यं गणनया ज्ञातमस्मात् स्वपरिधिनेदमन्तरं योजनात्मकं तदोक्तपरिधिना किमित्यनुपातेन लङ्कास्वनिरक्षदेशयोरन्तरमुक्तपरिधिस्यं ज्ञातम् । ततोऽर्कोदयद्वयान्तरकालेनार्को भूपरिधिं क्रामति तत्र
\end{sloppypar}

\newpage

\hspace{3cm}गूढार्थप्रकाशकेन सहितः ~।\hfill ४७
\vspace{1cm}

\begin{sloppypar}
\noindent ग्रहाः स्वां स्वां गतिं कलात्मिकामतिक्रामन्त्यत उक्तपरिधिना ग्रहगतिकलास्तदा प्राक्सिद्धलङ्कास्वनिरक्षदेशान्तरयोजनैः केत्यनुपातेनोक्तपरिध्योर्गुणहरयोस्तुल्यत्वेन नाशात् स्वरेखादेशस्वदेशयोरन्तरयोजनानि ग्रहगतिगुणितानि स्वपरिधिभक्तानि फलं ग्रहस्यान्तरकलाः । यद्यपि स्वपरिधिना गतिकलास्तदा स्वरेखादेशस्वदेशयोरन्तरयोजनैः केत्येकानुपातैनैव देशान्तरफलमुपपन्नं भवति तथापि निरक्षदेशपदार्थसम्बन्धाभावादिदमुपपन्नं फलं निरक्षदेशीयं कथमित्याग्रहनिरतातिमन्दस्य बोधार्थं गुरुभूतमप्यनुपातद्वयमुक्तम् । तद्धनर्णोपपत्तिस्तु लङ्कादेशात् स्वनिरक्षदेशस्य पूर्वभागस्थितत्वे लङ्कादेशार्द्धरात्रात् स्वनिरक्षदेशार्द्धरात्रमर्वाग्भवति । तदुदयकालात् प्रवहानिलवेगेन पूर्वभागे पूर्वमेवोदयात् । अतोऽग्रिमकालीनग्रहस्य पूर्वकालिकत्वसिद्ध्यर्थं तत्फलं न्यूनं कार्यम् । एवं निरक्षदेशस्य लङ्कातः पञ्चिमस्यत्वे लङ्कोदयानन्तरोदयसद्भावाल्लङ्कार्द्धरात्रादग्रिमकालेऽर्द्धरात्रमतः पूर्वकालिकग्रहस्याग्रिमकालिकत्वसिद्ध्यर्थं तत्फलं योज्यम् । चक्रशोधितपातस्यायं संस्कारो विपरीत इति ज्ञेयम् । स्वनिरक्षदेशस्य लङ्कातः पूर्वापरभागस्थत्वं स्वरेखादेशात् स्वदेशस्य पूर्वापरभागस्थस्यानुरोधेनेति स्वनिरक्षदेशस्वदेशयोर्याम्योत्तरैक्यादर्द्धरात्रयोरभिन्नत्वात् स्वदेशार्द्धरात्रेऽपि स्वनिरक्षदेशार्द्धरात्रकालिका एव ग्रहा अविकृता इति सर्वमुक्तमुपपन्नम् ~॥~६१~॥\\
\noindent अथ रेखास्वरूपं तद्देशांच्च कांश्चिदाह\textendash
\end{sloppypar}

\newpage

\noindent४८ \hspace{4cm} सूर्यसिद्धान्तः
\vspace{1cm}
\begin{quote}

{\ssi राक्षसालयदेवौकःशैलयोर्मध्यसूत्रगाः ~।\\
रोहीतकमवन्ती च यथा सन्निहितं सरः ~॥~६२~॥ }
%\vspace{2mm}
\end{quote}
\begin{sloppypar}
राक्षसालयं लङ्का देवानां गृहरूपः पर्वतो मेरूरनयोर्मध्ये ऋजुसूत्रं तत्र स्थिता देशा रेखाख्या लङ्कादक्षिणसूत्रस्थास्त्वनुपयुक्तास्तत्र मनुष्यागोचरत्वादिति नोक्ताः । ज्ञानार्थमुदाहरति\textendash रोहीतकमिति । यथा रोहीतकं नगरमवन्त्युज्जयिनी सन्निहितं सरः कुरुक्षेत्रम् । चकारस्तथेत्यव्ययपरः । तथान्यानि परस्परं सन्निहिततया ज्ञेयानि ~॥~६२~॥\\
\noindent ननु येन स्वस्थानं रेखापुरात् पूर्वतोऽपरत्र वा कियद्योजनान्तरेणास्तीति न ज्ञायते तेन देशान्तरफलादिकं कथं कार्यमित्यतः श्लोकत्रयेणाह\textendash
\end{sloppypar}
%\vspace{2mm}
\begin{quote}

{\ssi *अतीत्योन्मीलनादिन्दोः पश्चात् तद्गणितागतात् ~।\\
यदा भवेत् तदा प्राच्यां स्वस्थानं मध्यतो भवेत् ॥६३॥

अप्राप्य च भवेत् पश्चादेवं वापि निमोलनात् ~।\\
तयोरन्तरनाडीभिर्हन्याद्भूपरिधिं स्फुटम् ~॥~६४~॥

षष्ट्या विभज्य लब्धैस्तु योजनैः प्रागथार्परैः ~।\\
स्वदेशपरिधिर्ज्ञेयः कुर्याद्देशान्तरं हि तैः ~॥~६५~॥ }
%=\vspace{2mm}
\end{quote}
\begin{sloppypar}
चन्द्रस्य सर्वग्रहणान्तर्गतोन्मीलनकालाद् विना देशान्तरं गणितागताच्चन्द्रग्रहणोक्तप्रकारगणितज्ञानात् । अतीत्य तत्कालस्यातिक्रमणं कृत्वा पञ्चादनन्तरकाले मन्दबोधार्थमिदम् । अन्यथातीत्य पश्चादित्यनयोरेकतरस्य वैयर्थ्यापत्तेः । तच्चन्द्रबि\textendash
\end{sloppypar}

\noindent \rule{\linewidth}{.5pt}

\begin{center}
*अतीत्योन्मीलनादिन्दोर्ट्टक्सिद्धं गणितागतात् । इति वा पाठः ।
\end{center}

\newpage

\hspace{3cm}गूढार्थप्रकाशकेन सहितः ~। \hfill ४९
\vspace{1cm}

\begin{sloppypar}
\noindent स्वस्वोन्मीलनं यदा यदीत्यर्थः । स्यात् तदा तर्हीत्यर्थः । स्वाभिमतस्थानं मध्यतो मध्यरेखादेशात् पूर्वदिशि भवेत् तिष्ठतीत्यर्थः । पश्चात् तदित्यत्र दृक्सिद्धमिति पाठे तु प्रत्यक्षमुन्मीलनमित्यर्थः । अप्राप्य तदतिक्रमणमकृत्वा पूर्वकाल एव । चकाराच्चन्द्रोन्मीलनं यदि स्यात् तर्हि मध्यरेखातः स्वस्थानमित्यर्थः । पश्चात् पश्चिमदिग्भागे भवेत् तिष्ठतीत्यर्थः । ननु चन्द्रस्य स्पर्शमोक्षसम्मीलनोन्मीलनकालेषून्मीलनकाल एव कथं गृहीत इत्यत आह\textendash एवमिति~। वा प्रकारान्तरेण निमीलनाच्चन्द्रसम्मीलनकालात् । एवं चन्द्रग्रहणाधिकारोक्तगणितप्रकारज्ञानादनन्तरकाले सम्मीलनं यदि तर्हि मध्यरेखादेशात् स्वस्थानं पूर्वदिग्भागे तिष्ठति पर्वकाले सम्मीलनं यदि तर्हि मध्यरेखादेशात् स्वस्थानं पश्चिमदिग्भागे तिष्ठतीत्यर्थः । अपिशब्दो निश्चयार्थे । तेनोन्मीलनसन्मीलनकालयोर्भिन्नरीतिव्युदासः । तथा चोन्मीलनग्रहणमुपलक्षणार्थं तत्रापि स्पर्शमोक्षयोर्ग्रहणाद्यन्तरूपयोरनिश्चयत्वसम्भावनयोक्तिमुपेक्ष्य ग्रहणमध्यस्थयोः सम्मीलनोन्मीलनयोर्निश्चयत्वेनोक्तिः कृतेति भावः । अथ देशान्तरयोजनपुरःसरं देशान्तरफलं सिद्धमित्याह\textendash तयोरिति~। प्रत्यक्षोन्मीलनकालगणितागतोन्मीलनकालयोः सम्मीलनकालयोस्तादृशयोर्वान्तरघटीभिर्भूपरिधिं स्पष्टं स्वदेशभूपरिधिं लम्बज्याघ्न इत्याद्यवगतं हन्याद्गुणयेत् तादृशं गुणितस्पष्टपरिधिं षष्ट्या भत्का लब्धैः प्राप्तैर्योजनैः पूर्वभागयोजनैः । अथाथवापरैः पश्चिमविभागस्थितैर्योजनैः स्वदेशपरिधिः स्वदेशस्य परि\textendash
\end{sloppypar}

{\tiny{H}}


\newpage

\noindent ५० \hspace{4cm} सूर्यसिद्धान्तः
\vspace{1cm}

\begin{sloppypar}
\noindent धिरवधिः स्वदेशस्थानमण्डलरूपस्तुकाराद्रेखादेशादन्तरित इत्यर्थः । ज्ञेयो गणकेनेति शेषः । स्वरेखास्वदेशयोरन्तरयोजनानि फलमिति फलितार्थः । तैरन्तरयोजनैर्देशान्तरं तेन देशान्तराभ्यस्तेत्यादिप्रागुक्तप्रकारेण ग्रहाणां देशान्तरफलं कलात्मकं कुर्यद्गणक इति शेषः । हिकारात् तत्संस्कारोऽप्यभिन्नप्रकारत्वादभिन्न इत्यर्थः । अत्रोपपत्तिः । विना देशान्तरसंस्कारं ग्रहगणितं स्वरेखादेशीयं भवति । अतो गणितसाधितोन्मोलनसन्मीलनादिकालाः स्वरेखादेशे सिद्ध्यन्ति । स्वदेशे पूर्वविभागस्ये प्रथमं खस्य सूर्योदयादिकालास्तदनन्तरं रेखाया इति चन्द्रग्रहणस्य सर्वदेशे युगपत् सम्भवात् । गणितागतकालाद्रेखादेशस्थादनन्तरं स्पर्शादिकालो भवति । एवं स्वदेशे पश्चिमविभागस्थे प्रथमं रेखादेशेऽर्कोदयादिकालास्तदनन्तरं स्वदेश इति रेखास्थगणितागतस्पर्शादिकालाद्वध्यात्मकात् पूर्वमेव स्पर्शादिकालो भवति । अतः सम्यगुपपन्नमतीत्येत्यादिसार्द्धश्लोकोक्तम् ।स्वदेशरेखादेशसूर्योदयाद्यवधिकघट्यात्मककालयोरन्तरं देशान्तरघटिकाः सिद्धाः सूर्योद्वयान्तरकालेनार्को भूपरिधिं क्रामतीति षष्टिसावनघटीभिर्भूपरिधियोजनानि स्वदेशीयानि तदा तत्कालान्तररूपदेशान्तरघटीभिः कानीत्यनुपातेन स्वरेखादेशस्वदेशयोरन्तरयोजनानि । ज्ञातेभ्य एभ्यः पूर्वदिशैव देशान्तरं भवति । सूर्यग्रहणस्य सर्वदेशे युगपदसम्भवात् तदुन्मीलनकालादिनोक्तदिशा नैतज्ज्ञानमित्यनुक्तिरिति ध्येयम् ~॥~६५~॥\\
\noindent अथ वारप्रवृआत्तकालज्ञानमाह\textendash
\end{sloppypar}

\newpage

\hspace{3cm}गूढार्थप्रकाशेन सहितः ~। \hfill ५१
\vspace{1cm}
\begin{quote}

{\ssi वारप्रवृत्तिः प्राग्देशे क्षपार्धेऽभ्यधिके भवेत् ~।\\
तद्देशान्तरनाडीभिः पश्चादूने विनिदिशेत् ~॥~६६~॥}
%\vspace{2mm}
\end{quote}
\begin{sloppypar}
 रेखातः पूर्वभागस्थितस्वाभिमतदेशे तद्देशान्तरनाडीभिः पूर्वप्रकारज्ञातदेशान्तरनाडीभिरभ्यधिकेऽर्धरात्रे युक्तार्धरात्र समयेऽर्धरात्रादनन्तरं देशान्तरघटीकाल इत्यर्थः । वारप्रवृत्तिर्वारस्यादिभृतः कालः स्यात् । रेखातः पश्चिमभागस्यदेशे पूर्वप्रकारज्ञातदेशान्तरघटीभिरूनेऽर्धरात्रेऽर्धरात्रात् पूर्वमेव देशान्तरघटीकाले वारप्रवृत्तिं विनिर्दिशेद्गणकः कथयेत् । अत्रोपपत्तिः । यमकोटिसूर्योदयकालो लङ्कार्धरात्रसमयरूपो ग्रहचारप्रवृत्तिरूपः स्वदेशे कदेति रेखातः पूर्वापरभागयोः स्वार्धरात्रकालादनन्तरं पूर्वक्रमेण तदर्धरात्रं देशान्तरघटीभिर्भवति । स्वनिरक्षदेशस्वदेशार्धरात्रयोर्युगपत् सम्भवात् । अत उपपन्नं वारप्रवृत्तिरित्यादि । नन्वेतत्कालज्ञानं किमर्थमुक्तं प्रयोजनाभावादिति चेन्न । अहर्गणोत्पन्नग्रहस्य तात्कालिकत्वात् तत्कालज्ञानेन स्वार्धरात्रसमयस्य तत्कालस्य च यदन्तरं तेन तात्कालिकस्य ग्रहस्य चालने कृते सति स्वार्धरात्रसगये ग्रहः पूर्वसाधित एव भवतीति मन्दप्रत्ययस्यैव प्रयोजनत्वात् । तत्कालज्ञानन ग्रहस्य देशान्तरसंस्काराकरणमिति लाघवाच्च । अत एव समनन्तरमेव ग्रहस्येष्टकालिकत्वसिद्ध्यर्थं चालनोक्तिः सङ्गच्छते । एतेन तत् ततोऽर्धरात्रात् क्षपार्धे निरक्षरात्र्यर्धे पञ्चदशघटिकात्मककाल उत्तरगोलेऽर्कोदयाच्चरघटीमिताग्रिम\textendash
\end{sloppypar}

{\tiny{H2}}

\newpage

\noindent ५२ \hspace{4cm} सूर्यसिद्धान्तः
\vspace{1cm}

\begin{sloppypar}
\noindent काले दक्षिणगोलेऽर्कोदयाच्चरघटीमितपूर्वकाल इति फलितम्। पूर्वपश्चिमदेशयोर्देशान्तरघटीभिरधिकोने काले क्रमेणवारप्रवृत्तिरिति व्याख्यानं लङ्कासूर्योदयकालरूपवारप्रवृत्तिबोधकमपास्तम् । तच्छब्दस्य पूर्वपरामर्शकत्वादर्धरात्रादित्यस्यानुपपत्तेः पञ्चदशघटिकालस्य क्षपार्धशब्देनासिद्धेश्च । श्रीभगवताहर्गणस्य लङ्कायामार्धराचिक इत्यनेन लङ्कार्धरात्रकालिकत्वोक्तेः स्वदेशे तत्कालरूपवारप्रवृत्तिकालज्ञानस्योक्तस्य सङ्गत्यनुपपत्तेः । व्यवहारयोग्यलङ्कासूर्योदयकालवारप्रवृत्तेरत्र सङ्गत्यभावाच्च ~॥~६६~॥\\ 
\noindent अथ ग्रहस्य तात्कालिककरणमाह\textendash
\end{sloppypar}
%\vspace{2mm}
\begin{quote}

{\ssi इष्टनाडीगुणा भुक्तिः षष्ट्या भक्ता कलादिकम् ~।\\
गते शोध्यं यतं गम्ये कृत्वा तात्कालिको भवेत् ~॥~६७~॥}
%\vspace{2mm}
\end{quote}
\begin{sloppypar}
यत्कालिको ग्रहस्तत्कालात् पूर्वमपरत्राभीष्टकाले या इष्टघट्यस्ताभिर्गुणिता ग्रहमध्यगतिः षष्ट्या भक्ता फलं कलादिकं गते गताभीष्टकाले पूर्वकालेऽभीष्टे सतीत्यर्थः । शोध्यं ग्रहे हीनं गम्येऽग्रिमाभीष्टकाले सति ग्रहे युतं कृत्वा गणकेन विधाय तात्कालिकः स्वाभीष्टसामयिको ग्रहो भवेत् । गणकेन ज्ञातो भवेत् । अत्रोपपत्तिः । षष्टिसावनघटीभिर्गतिकलास्तदाभीष्टगतैष्यघटीभिः का इत्यनुपातेनावगतकलात्मकचालनेन ग्रहः क्रमेण युतोनस्तात्कालिको ग्रहो भवति । चक्रशोधितपातस्य विपरीतमिति ज्ञेयम् । चालितस्पष्टग्रहापेक्षया चालितमध्यग्रहः
\end{sloppypar}

\newpage

\hspace{3cm}गूढार्थप्रकाशकेन सहितः ~। \hfill ५३
\vspace{1cm}

\begin{sloppypar}
\noindent स्पष्टः कृतश्चेत् सूक्ष्म इति सूचनार्थमत्र ग्रहचालनमुक्तम् ~॥~६७~॥\\
\noindent अथ चन्द्रस्य परमविक्षेपमानमाह\textendash
\end{sloppypar}
%\vspace{2mm}
\begin{quote}

{\ssi भचक्रलिप्ताशीत्यंशपरमं दक्षिणोत्तरम् ~।\\
विक्षिप्यते स्वपातेन स्वक्रान्त्यन्तादनुष्णगुः ~॥~६८~॥}
%\vspace{2mm}
\end{quote}
\begin{sloppypar}
अनुष्णगुश्चन्द्रः स्वक्रान्त्यन्ताद्विषुवट्टत्तानुकारेणावलम्बितश्चन्द्रः स्वासन्नक्रान्तिवृत्तप्रदेशेनाकृष्यते तथा तत्स्थानात् स्वभोगमितरेवत्यासन्नाद्यवधिकाभीष्टस्थानभूतक्रान्तिवृत्तप्रदेशादपि स्वपातेन चन्द्रपातेन दक्षिणोत्तरं दक्षिणस्यामत्तरस्यां वा तत्सूत्रेण विक्षिप्यते त्यज्यते स्वभोगस्थानक्रान्तिवृत्तप्रदेशे चन्द्रबिम्बं स्थातुं पातेन न दीयते ततोऽपि चन्द्रबिम्बं स्थलान्तरे दक्षिणोत्तरसूत्रेण किञ्चिदन्तरेण त्यज्यत इत्यर्थः । एतेन सूर्यस्य पाताभावात् स्वभोगस्थानीयक्रान्तिवृत्तप्रदेशे बिम्बं भवति न विक्षिप्तमित्यनुष्णगुरित्यनेनापि सूचितम् । परमविक्षेपणं दक्षिणोत्तरमित्यस्य विशेषणान्याह\textendash भचक्रेति~। द्वादशराशिकलानां षट्शताधिकैकविंशतिसहस्रमितानामेषां २१६०० अशीतिभागः खसप्रयमकलामितः परमं यस्य तद्दक्षिणोत्तरमित्यर्थः । चन्द्रस्य परमो विक्षेपः खभमित इति फलितम् । केचिदत्र सूर्यस्य शराभावात् तत्कक्षातो भचक्रस्य पञ्चमंकक्षात्वात् ततोऽपि चन्द्रकक्षाया अष्टमत्वात् तत्र दक्षिणोत्तररूपदिग्द्वये चन्द्रस्य विक्षेपणात् पञ्चाष्टद्विघातरूपाशीत्यंशो भचक्रलिप्तानां
\end{sloppypar}

\newpage

\noindent ५४ \hspace{4cm} सूर्यसिद्धान्तः
\vspace{1cm}

\begin{sloppypar}
\noindent परमचन्द्रविक्षेप इत्युपपत्तिमाहुः ~॥~६८~॥\\
\noindent अथैवं भौमादयोऽपि स्वपातैर्विक्षिप्यन्त इत्येषामपि परमविक्षेपानाह\textendash
\end{sloppypar}
%\vspace{2mm}
\begin{quote}

{\ssi तन्नवांशं द्विगुणितं जीवस्त्रिगुणितं कुजः ~।\\
बुधशुक्रार्कजाः पातैर्विक्षिप्यन्ते चतुर्गुणम् ~॥~६९~॥}
\end{quote}
\begin{sloppypar}
तन्नवांशं तस्य चन्द्रपरमविक्षेपस्य नवभागं त्रिंशतं द्विगुणितं षष्टिकलामितं परमेण तदन्तरेणेत्यर्थः । पातेन गुरुर्दक्षिणोत्तरयोः क्रमेण विक्षिप्यते । भौमः पातेन त्रिगुणितं त्रिंशतं नवतिकलामितपरमान्तरेण विक्षिप्यते । चतुर्गुणं त्रिंशतं विंशत्यधिकशतकलामितपरमान्तरेण बुधशुक्रशनैश्चराः स्वस्वपातैः प्रत्येकं विक्षिप्यन्ते स्वभोगक्रान्तिवृत्तप्रदेशात् त्यज्यन्ते । केचिदत्रापि त्रयस्त्रिंशत्कलाबिम्बाच्चन्द्रान्नवांशद्विगुणेन सत्र्यंशकलासप्तकस्य गुरुबिम्बस्य तद्रूपं विक्षेपणं युक्तमस्माद्भौमस्याधःस्थत्वात् त्रिगुणं परमविक्षेपणमस्मादपि बुधशुक्रयोर्लघुपृथुविम्बयोरधःस्थत्वाच्चतुर्गुणं परमविक्षेपणं तुल्यं नाल्पाधिकमेवं शनेरुच्चकक्षास्थत्वेऽपि मन्दत्वाद्बुधशुक्रविक्षेपणतुल्यं परमविक्षेपणं युक्तमित्युपपत्तिग्राहुः ~॥~६९~॥\\ 
\noindent नन्वेषामत्र कथने का सङ्गतिरित्यतः पूर्वोक्तमुपसंहरन्नाह \textendash
\end{sloppypar}
%\vspace{2mm}
\begin{quote}

{\ssi एवं त्रिघनरन्ध्रार्करसार्कार्का दशाहताः ~।\\
चन्द्रादीनां क्रमादुक्ता मध्यविक्षेपलिप्तिकाः ~॥~७०~॥}
%\vspace{2mm}
\end{quote}
\begin{sloppypar}
एवं पूर्वश्लोकाभ्यां त्रिघनः सप्तविंशती रन्ध्राणि नव द्वादशषट द्वादश द्वांदशैके दशगुणिताः क्रमादुक्ताङ्कक्रमाच्चन्द्रादीनां
\end{sloppypar}

\newpage

\hspace{3cm}गूढार्थप्रकाशकेन सहितः ~। \hfill ५५
\vspace{1cm}

\begin{sloppypar}

\noindent वारक्रमाच्चन्द्रभौमबुधगुरुशुक्रशनीनां विक्षेपकला मध्या अग्रे परमशरकलानामनियतत्वेनोक्तेः । कथिताः । तथा च मध्यत्वेनैषामत्र प्रसङ्गसङ्गस्या कथनमिति भावः ~॥~७०~॥\\ 
\noindent अथ पूर्वापरग्रन्थयोरसङ्गतिनिवारणायाधिकारसमाप्तिं फक्किकयाह\textendash
\end{sloppypar}
%\vspace{2mm}
\begin{quote}

इति श्रीसूर्यसिद्धान्ते मध्यमाधिकारः ~।
%\vspace{2mm}
\end{quote}
\begin{sloppypar}
मयं प्रति सूर्यांशपुरुषेण सूर्योक्तस्यैव कथनादेतदुक्तस्यापि सूर्यसिद्धान्तत्वम् । तत्र मध्यममानेन गणितमधिएक्रियते यस्मित्रेतादृशो ग्रन्थैकदेशः परिपूर्तिमाप्त इत्यर्थः ।
\end{sloppypar}
%\vspace{2mm}
\begin{quote}

{\ssi रङ्गनाथेन रोचते सूर्यसिद्धान्तटिप्पणे ~।\\
मध्याधिकारः पूर्णोऽयं तद् गूढार्थप्रकाशके~॥ }
\end{quote}
\begin{sloppypar}
इति श्रसिकलगणकसार्वभौमबल्लालदैवज्ञात्मजरङ्गनाथग एकविरचिते गूढार्थप्रकाशके मध्यमाधिकारः पूर्णः ॥
\end{sloppypar}

{\setlength{\parindent}{12em}\rule{7em}{.5pt}}
\vspace{2mm}
\begin{sloppypar}
अथ स्पष्टाधिकारो व्याख्यायते । तत्र ग्रहाणां मध्यमातिरिक्तस्यष्टक्रियायां कारणमाह\textendash
\end{sloppypar}
%\vspace{2mm}
\begin{quote}

{\ssi अदृश्यरूपाः कालस्य मूर्तयो भगणाश्रिताः ~।\\
शीघ्रमन्दोच्चपाताख्या ग्रहाणां गतिहेतवः ~॥~१~॥}
%\vspace{2mm}
\end{quote}
\begin{sloppypar}
शीघ्रोच्चमन्दोच्चपातसञ्ज्ञकाः पूर्वोक्तपदार्था जीवविशेषाः सूर्यादिग्रहाणां गतिकारणभूताः सन्ति । ननु कालेनैव ग्रहचलनं भवतीति कालो गतिहेतुर्नैत इत्यत आह\textendash कालस्येति~।
\end{sloppypar}

\newpage

\noindent ५६ \hspace{4cm} सूर्यसिद्धान्तः
\vspace{1cm}

\begin{sloppypar}
\noindent पूर्वप्रतिपादितकालस्य स्वरूपाणि तथा चैषां कालमूर्तित्वेन ग्रहगतिहेतुत्वं नासम्भवतीति भावः । ननु कालस्य घट्यादिमूर्तित्वादेषां तदात्मकत्वाभावात् कथं कालमूर्तित्वमित्यत आह\textendash भगणाश्रिता इति~। भगोलसयक्रान्तिवृत्तानुसृतग्रहगोलस्थक्रान्तिवृत्तप्रदेशाश्रिता राश्यात्मका इत्यर्थः । तथा च ग्रहराश्यादिभोगानां कालवशेनैवोत्पन्नत्वात् तदात्मकानां कालमूर्तित्वमिति भावः । ननु दृश्यन्ते कुतो नेत्यत आह\textendash अदृश्यरूपा इति~। वायवीयशरीरा अव्यक्तरूपत्वादप्रत्यक्षा इति भावः । एवं च ग्रहाणामुच्चादिसद्भावात् स्पष्टक्रियोत्पन्नेति तात्पर्यम् ~॥~१~॥\\
\noindent अथानयोरुच्चपातयोर्मध्य उच्चयोर्गतिहेतुत्वं प्रतिपादयति\textendash
\end{sloppypar}
%\vspace{2mm}
\begin{quote}

{\ssi तद्वातरश्मिभिर्बद्धास्तैः सव्येतरपाणिभिः ~।\\
प्राक् पश्चादपकृष्यन्ते यथासन्नं स्वदिङ् मुखम् ~॥~२~॥ }
%\vspace{2mm}
\end{quote}
\begin{sloppypar}
तेषामुच्चसञ्ज्ञकजीवानां वायुरूपा ये रश्मयो रज्जवस्ताभिर्बद्धा बिम्बात्मकग्रहास्तैरुच्चसञ्ज्ञकजीवैः सव्यवामहस्तैरुच्चबहुत्वेन हस्तबाहुल्याद्बहुवचनं हस्ताभ्यामित्यर्थः । स्वदिङ् मुखं स्वाभिमुखं यथासन्नं ग्रहबिम्बं भवति तथा प्राक् पश्चात् पूर्वपश्चिममार्गाभ्यामित्यर्थः । अपकृष्यन्ते आकर्व्यन्ते । अयमभिप्रायः ।भचक्रगोलस्यक्रान्तिवृत्तानुसृतग्रहाकाशगोलान्तर्गतक्रान्तिवृत्ते कक्षारूपे स्वस्वप्रदेशे ग्रहोच्चपातास्तिष्ठन्ति । तत्र बिम्बव्यासोनकक्षाकारसूत्रं प्रवहवाय्वतिरिक्तवायुरूपं स्वतोगति स्वस्थाने कम्पमानं ग्रहबिम्बव्यासे पूर्वापरे प्रोतमुच्चजीवहस्तद्वयान्तर्गत\textendash
\end{sloppypar}

\newpage


\hspace{3cm} गूढार्थप्रकाशेन सहितः ~। \hfill ५७
\vspace{1cm}

\begin{sloppypar}
\noindent मस्ति । अथ ग्रहबिम्बमुच्चस्थानात पूर्वस्मिन् स्वशक्त्या गच्छदपि वामहस्तस्थितसूत्रेणोच्चस्थानात् पूर्वरूपेण ग्रहस्थानात् पश्चिमरूपेण बृहत्सूत्रावयवात्मकेन स्वस्थानात् पश्चात् स्वाभिमुखमपकृष्यते निरन्तरमुच्चदैवतैः स्वशक्त्या यावत् षड्मान्तरं तयोरन्तरं तन्मार्गेणाकर्षणसम्भवात् पूर्वस्मिन् गच्छद्ग्रहबिम्बं सव्यहस्तस्थितसूत्रेणोच्चस्थानात् पश्चिमरूपेण ग्रहस्थानात् पूर्वरूपेण बृहत्सूत्रावयवात्मकेन स्वस्थानात् पूर्वस्मिन स्वाभिमुखमाकृष्यते स्वशक्त्या निरन्तरं यावदन्तराभावस्तयोरिति ~॥~२~॥\\
\noindent अथात एकैकरूपां पूर्वाधिकारावगतां गतिं त्यक्ता प्रत्यहं विलक्षणं गतिं प्राप्ता ग्रहा इत्यत आह\textendash
\end{sloppypar}
%\vspace{2mm}
\begin{quote}

{\ssi प्रवहाख्यो मरुत् तांस्तु स्वोच्चाभिमुखमीरयेत् ~।\\
पूर्वापरापकृष्टास्ते गतिं यान्ति पृथग्विधाम् ~॥~३~॥}
%\vspace{2mm}
\end{quote}
\begin{sloppypar}
प्रवहाख्यः प्रवहसञ्ज्ञको मरुद्वायुः पश्चिमाभिमुखभ्रमस्तान् तुकारादुच्चानि स्वोच्चाभिमुखं स्वस्य प्रवहभ्रमणेनोच्चं भावप्रधाननिर्देशादुच्चता यस्यां दिशि तत् स्वोच्चं पूर्वदिक् पूर्वभाग एव ग्रहाणां प्रवहभ्रमेणोच्चगमनदर्शनात् तत्सम्मुखं पूर्वदिशीति तात्पर्यार्थः । ईरयेत् पश्चिमाभिमुखभ्रमणसिद्धप्रागुक्तग्रहावलम्बनरूपेण चालयतीत्यर्थः । अतः कारणात् ते ग्रहाः पूर्वापरापकृष्टा उच्चदैवतैः पूर्वपश्चिमदिशोराकृष्टाः पृथद्विधां प्रथमावगतैकरूपभिन्नप्रकारावगतां प्रतिक्षणविलक्षणां गतिं गमनक्रियां यान्ति प्राप्नुवन्ति । अवलम्बनाकर्षणाभ्यां प्रतिदिनं ग्रहाणां 
\end{sloppypar}

{\tiny{I}}

\newpage

\noindent ५८ \hspace{4cm} सूर्यसिद्धान्तः
\vspace{1cm}

\begin{sloppypar}

गतेरन्यादृशत्वं तदनुसारेण ग्रहचारज्ञानं युक्तमिति ग्रहाणां स्पष्टक्रियोत्पन्नेति भावः । यद्वा । ननु वायुरज्जुभिः कथं ग्रहाणामाकर्षणं सम्भवति तद्रज्जूनां विरलतया घनीभूतत्वाभावेनाकर्षणायोग्यत्वादित्यत आह\textendash  प्रवहाख्य इति~। उच्चदेवताहस्तद्वयस्थितकक्षाकारसूत्रं वायुः प्रवहवायुसम्बन्धात् प्रवहसञ्ज्ञो न पश्चिमाभिमुखभ्रमप्रवहात्मकस्तान् ग्रहान् स्वोच्चाभिमुखं स्वोच्चदेवतास्थानसम्मुखमीरयेत् प्रेरयति चालयति । तुकारादुच्चस्थानात् पूर्वस्मिन् ग्रहे वायुः पश्चिमगत्या ग्रहं चालयति पश्चिमस्थे वायुः पूर्वगत्या ग्रहं चालयतीत्यर्थः । तथा च कक्षाकारसूत्रं तदा तदा तथा तथा भ्रमतीति दैवतैराकृष्यत इत्युपचारादुच्यन्त इति भावः । अत एव ग्रहाणां स्पष्टक्रियोत्पन्नेत्याह\textendash  पूर्वापरापकृष्टा इति~। उच्चदैवतैः पूर्वापरदिशयोराकृष्टा ग्रहाः पृथग्विधां मध्यमातिरिक्तप्रकारां गतिं गमनक्रियां यान्ति । अतो न केवलं मध्यक्रियया निर्वाहः ~॥~३~॥\\ 
\noindent अथ प्राक् पश्चादपकृष्यन्त इत्युक्तं विशदयति\textendash
\end{sloppypar}
%\vspace{2mm}
\begin{quote}

{\ssi ग्रहात् प्राग्भगणार्डूस्थः प्राङ्मुखं कर्षति ग्रहम् ~।\\
उच्चसञ्ज्ञोऽपरार्द्वस्थस्तद्वत् पश्चान्मुखं ग्रहम् ~॥~४~॥}
%\vspace{2mm}
\end{quote}
\begin{sloppypar}
ग्रहस्थानात् पूर्वभागस्यराशिषट्कस्थित उच्चसञ्ज्ञो जीवो ग्रहबिम्बं पूर्वदिगभिमुखं स्वाभिमुखं कर्षत्याकर्षति । अपरार्द्धस्थो ग्रहस्थानात् पश्चिमभागस्थराशिषट्कस्थित उच्चसञ्ज्ञो जीव इत्यर्थः । ग्रहबिम्बं पश्चान्मुखं पश्चिमदिगभिमुखं स्वाभिमुखं तद्व
ट्राकर्षतीत्यर्थः ~॥~४~॥\\ 
\noindent अथ पूर्वोक्तसिद्धं फलितमाह\textendash
\end{sloppypar}

\newpage

\hspace{3cm} गूढार्थप्रकाशकेन सहितः ~। \hfill ५९
%\vspace{1cm}
\begin{quote}

{\ssi स्वोच्चापकृष्टा भगणैः प्राङ् मुखं यान्ति यद्ग्रहाः ~।\\
तत् तेषु धनमित्युक्तमृणं पश्चान्मुखेषु तु ~॥~५~॥}
%\vspace{2mm}
\end{quote}
\begin{sloppypar}
स्वोच्चजीवाकर्षिता ग्रहाः पूर्वाभिमुखं भगणै राशिभिर्भगोलस्थक्रान्तिवृत्तानुसृतस्वाकाशगोलान्तर्गतक्रान्तिवृत्ते द्वादशराश्यन्तिके यद्राशिविभागैरित्यर्थः । यद्यत्सङ्ख्यामितं गच्छन्ति तत्तत्सङ्ख्यामितं भागादिकं फलरूपं तेषु पूर्वावगतग्रहराश्यादिभोगेषु धनं योज्यम् । पश्चान्मुखेषु पश्चिमाकर्षितग्रहपूर्वावगतराश्यादिभोगेषु तुकाराद्यत्सङ्ख्याकितं फलरूपं पश्चिमतो गच्छन्ति तदित्यर्थः । ऋणं हीनमिति । एतत् पूर्वैः कथितम् ~॥~५~॥\\ 
\noindent अथ पातानां ग्रहविक्षेपरूपगतिहेतुत्वं प्रतिपादयति\textendash
\end{sloppypar}
%\vspace{2mm}
\begin{quote}

{\ssi दक्षिणोत्तरतोऽप्येवं पातो राहुः स्वरंहसा ~।\\
विक्षिपत्येष विक्षेपं चन्द्रादीनामपक्रमात् ~॥~६~॥}
%\vspace{2mm}
\end{quote}
\begin{sloppypar}
चन्द्रादीनां विरविग्रहाणामपक्रमात् क्रान्तिवृत्तस्यस्पष्टग्रहभोगस्यानाद्दक्षिणोत्तरतो दक्षिणस्यामुत्तरस्यां वा दिशि । अपिशब्दः पूर्वापराभ्यां समुच्चयार्थकः । एष गणितागतः पातः पातराश्यादिभोगस्थानम् । अत्राप्यपिशब्द उच्चेन समुच्चयार्थकोऽन्वेति । एवमुच्चेन पूर्वापरयोः फलान्तरं भवति तथेत्यर्थः । विक्षेपं विक्षेपणं स्वरंहसात्मवेगेन विक्षिपति करोति । विशिष्टवाचकानां पदानां विशेषणवाचकपदसमवधाने विशेष्यमाचार्थत्वात् । चन्द्रादीन् विक्षिपतीति तात्पर्यार्थः । ननूच्चेन स्वाधिष्ठितजीवद्वारा ग्रहाकर्षणं क्रियते तथा पातेनाचेतनत्वाद्वेगा\textendash
\end{sloppypar}

{\tiny{I 2}}

\newpage

\noindent ६० \hspace{4cm} सूर्यसिद्धान्तः
\vspace{1cm}

\begin{sloppypar}
भावेन ग्रहविक्षेपणं कर्तुमशक्यमित्यत आह\textendash  राहुरिति~। पातस्थानाधिष्ठात्री देवता राहुर्जीवविशेषश्चन्द्रपातस्तु दैत्यविशेषो राहुः । रहति त्यजति ग्रहमिति राहुरिति व्युत्पन्ते~॥~६~॥\\ 
\noindent अथैतद्विशदयति\textendash
\end{sloppypar}
%\vspace{2mm}
\begin{quote}

{\ssi उत्तराभिमुखं पातो विक्षिपत्यपरार्डूगः ~।\\
ग्रहं प्राग्भगणार्धस्थो याम्यायामपकर्षति ~॥~७~॥}
%\vspace{2mm}
\end{quote}
\begin{sloppypar}
अपरार्द्धगो ग्रहस्थानात् पश्चिमविभागस्थितभगणार्धात्मकराशिषट्कस्थितो राहुर्ग्रहबिम्बं स्वराश्यादिभोगस्थानीयप्रदेशादुत्तरदिगभिमुखं विक्षिपति विक्षेपान्तरेण त्यजति । प्राग्भगणार्धस्यो ग्रहस्थानात् पूर्वविभागस्थितराशिषट्कमध्यस्थितो दक्षिणस्यां दिश्यपकर्षति विक्षिपति ~॥~७~॥\\ 
\noindent अथ बुधशुक्रयोर्वि
शेषमाह\textendash
\end{sloppypar}
%\vspace{2mm}
\begin{quote}

{\ssi बुधभार्गवयोः शीघ्रात् तद्वत् पातो यदा स्थितः ~।\\
तच्छीघ्राकर्षणात् तौ तु विक्षिप्येते यथोक्तवत् ~॥~८~॥}
%\vspace{2mm}
\end{quote}
\begin{sloppypar}
बुधशुक्रयोः शीघ्रोच्चाज्जात्यभिप्रायेणैकवचनम् । बुधशुक्रयोः पातो जात्यभिप्रायेणैकवचनम् । तद्वत् परार्धपूर्वार्धभगणार्धमध्ये यदा यत्काले स्थितस्तुकारात् तत्काले पाताभ्यामित्यर्थः । तौ बुधशुक्रौ यथोक्तवत् पूर्वार्धपरार्धक्रमेण दक्षिणोत्तरयोर्विक्षिप्येते विक्षेपान्तरेण त्यज्येते । ननूच्चात् तादृगवस्थितपातौ सम्बन्धाभावाद्-बुधशुक्रौ दक्षिणोत्तरयोः कथं त्यजतोऽन्यथा वैयधिकरण्येनातिप्रसङ्गापत्तेरित्यतः कारणमाह\textendash  तच्छीघ्राकर्ष\textendash
\end{sloppypar}

\newpage

\hspace{3cm}गूढार्थप्रकाशकेण सहितः ~। \hfill ६१
\vspace{1cm}

\begin{sloppypar}
\noindent णादिति~। बुधशुक्रयोः शीघ्रोच्चे तयोराकर्षणाभ्यां जात्यभिप्रायेणैकवचनम् । तथा च तदुच्चाभ्यां तादृगवस्थितपातौ तदुच्चजीवौ दक्षिणोत्तरयोस्त्यजत इति पूर्वोक्तरीत्या न्यायसिद्धमतस्तदुच्चसूत्रबहुत्वाद्बुधशुक्रयोस्तथा विक्षेपणं न्यायसिद्धमेवेति भावः । ननु भौमगुरुशनीनामेवं कथं नोक्तमनयोर्वा कथमेतदुक्तं सर्वेषामेकरीतिकथनस्य समुचितत्वात् । किञ्च गुरुभौमशनीनामुच्चदेवताः स्वस्वकक्षास्था इति फलमुपपन्नं भवति बुधशुक्रयोरुच्चदेवतयोः कक्षातो दक्षिणोत्तरयोः स्थितत्वेन पूर्वोक्तरीत्याफलानुपपत्तिर्विलक्षणप्रवहवायुसूत्रस्थदेवतासम्बद्धस्य स्पष्टभूपरिध्याकारत्वेन कक्षाकारत्वाभावात् । विना कक्षाकारतां फलोत्पादनस्य ब्रह्मणोऽप्यशक्यत्वाच्च । न च विलक्षणप्रवहवायुसूत्रं देवतासम्बद्धं ग्रहाकाशगोले कक्षाकारत्वाभावेऽपि कक्षातुल्यं स्थानान्तर इति फलोत्पत्तिर्याम्योत्तरान्तरसत्त्वेऽपि कल्पनयेति वाच्यम् । उच्चदेवतास्थानस्य कक्षातो दक्षिणत्वे तत्षड्भान्तरप्रदिशस्योत्तरत्वावश्यम्भावेनोच्चबुधशुक्रयोरेकदिग्विक्षेपतुल्यत्वनियमानुपपत्तेः । तत् कथमिदं सङ्गतं भगवदुक्तमिति चेत् । अत्रोच्यते । स्वरुच्यासङ्गतार्थमङ्गीकृत्य तद्दूषणोद्घाटनेन भगवदुपालम्भनकर्तू रसनाच्छेदस्तत्तत्त्वार्थप्रकाशेनावश्यं करणीयः । तथाहि स्वशीघ्रोच्चबुधशुक्रयोर्यदन्तरं राश्यात्मकं तद्वत् पातस्तेनान्तरेण युक्तः पूर्वानोतपात इत्यर्थः । यथा बुधशुक्रयोरपरपूर्वार्धक्रमेणस्थितोऽवस्थितस्तुकारात् तथेत्यर्थः । तच्छीघ्राकर्षणात् तादृश पाताभ्यां शीघ्रं वेगेनाकर्षणं तस्मात् पातस्थानाधिष्ठातृदेवताभ्यां
\end{sloppypar}

\newpage

\noindent६२ \hspace{4cm} सूर्यसिद्धान्तः
\vspace{1cm}

\begin{sloppypar}
\noindent स्वहस्तस्थितग्रहसम्बद्धवायुसूत्रस्यातिवेगाकर्षणरचनादित्यर्थः । तौ बुधशुक्रावुक्तवदुत्तरदक्षिणक्रमेण विक्षिप्येते। अत्र पातशब्देन चक्रशोधितपातो बोध्यः । अन्यथा ग्रहोनशीघ्रोच्चरूपकेन्द्रयोजनस्योपपत्तिसिद्धत्वेन शीघ्रोच्चोनग्रहरूपकेन्द्रयोजनोक्त्यनुपपत्तेः । तथा च सर्वग्रहसाधारणं विक्षेपकथनं पातभेददर्शनार्थं बुधशुक्रयोः पृथगुक्तम् । न ह्यन्यस्मिन् पक्ष उच्चयोर्विक्षेपणं प्रतीयते येन प्रागुक्तसर्वविलोपाशङ्कनं शङ्कनीयम् । पातभेदोक्तिकारणं च ।
\end{sloppypar}
\begin{quote}

{\qt ये चात्र पातभगणाः कथिता ज्ञमृग्वोस्\\
ते शीघ्रकेन्द्रभगणैरधिका यतः स्युः ~।\\
स्वल्पाः सुखार्थमुदिताश्चलकेन्द्रयुक्तौ\\
पातौ तयोः पठितचक्रभवौ विधेयौ~॥}
\end{quote}
\begin{sloppypar}
इति भास्कराचार्योक्तमिति दिक् ~॥~८~॥\\
\noindent स्यादेतत् परमुच्चदेवतयोरविशेषात् सूर्यचन्द्रयोः समं फलं कुतो न भवतीत्यत आह\textendash
\end{sloppypar}
\begin{quote}

{\ssi महत्त्वान्मण्डलस्यार्कः स्वल्पमेवापकृष्यते ~।\\
मण्डलाल्पतया चन्द्रस्ततो बह्वपकर्ष्यते ~॥~९~॥}
\end{quote}
\begin{sloppypar}
सूर्यो मण्डलस्य बिम्बस्य महत्त्वाद् गुरुत्ववत्त्वात् स्वल्पमितरग्रहापेक्षयाल्पं परमफलम् । एवकारो निर्धारणेऽपकर्व्यत उच्चजीवेनाकृष्यते । चन्द्रो मण्डलाल्पतया बिम्बस्य लघुत्वेन ततः सूर्यफलाब्दकधिकं परमफलमुच्चजीवेनाकृव्यते~॥~९~॥\\
\noindent अर्थात एव भौमादीनामल्पमूर्तित्वादाभ्यां फलाधिकत्वं सम्भवतीत्याह\textendash
\end{sloppypar}

\newpage

\hspace{3cm}गूढार्थप्रकाशकेन सहितः ~। \hfill ६३
\vspace{1cm}
\begin{quote}

{\ssi भौमादयोऽल्पमूर्तित्वाच्छीघ्रमन्दोच्चसञ्ज्ञकैः ~।\\
दैवतैरपकृष्यन्ते सुदूरमतिवेगिताः ~॥~१०~॥}
%\vspace{2mm}
\end{quote}
\begin{sloppypar}
भौमादयः पञ्च ग्रहा अल्पमूर्तित्वाल्लघुतरबिम्बत्वाच्छीघ्रमन्दोच्चसञ्ज्ञकैः शीघ्रोच्चमन्दोच्चसञ्ज्ञैर्दैवतैः सुदूरमत्यन्तं बह्वपकृव्यन्ते । अत एवातिवेगिता अत्यन्तवेगः सञ्जातो येषां ते बिम्बलघुत्वेनोच्चद्वयाकर्षणेन च बहुपरमफला इत्यर्थः । ननु सूर्यचन्द्रयोः कक्षाकारविलक्षणप्रवहवायुचलनेन फलोत्पादनं युक्तं भौमादीनां तु प्रत्येकमुच्चद्वयसद्भावाद्वायुरश्म्याकर्षणासम्भवेन कक्षाकारप्रवहविलक्षणवायुचलनेन फलोत्पादनार्थमङ्गीकृतं कथं सम्भवति । उच्चद्वयस्थानस्यैकत्वाभावात् । न ह्येकमेव वायुमण्डलं युगपद्विरुद्धगत्योराश्रयं स्वतो भवितुमर्हतीति भौमादीनां शीघ्रमन्दोच्चदेवताद्वयेन तत्सूत्रमार्गेण ग्रहबिम्बाकर्षणस्यैव स्वशक्त्या रचनात् । न वायुमण्डलचलनकल्पनं सूर्यचन्द्रयोरप्येवमेवाङ्गीकारे बाधकाभावाच्च । वायुमण्डलकल्पनं तु तद्वातरश्मीत्युक्तानुपपत्त्यानतिप्रयोजनम् । तद्वातरश्मिभिर्बद्धा इत्यस्य पश्चिमभरमात्मकप्रवहवायौ स्वस्वाकाशगोले समसूचसस्बन्धेन स्थिता इति ग्रहस्थितिस्वरूपोक्त्या समर्थनात् । न हि तदत्र हेतुगर्भं येनानुपपत्तिः शङ्कनीया । उच्चदेवताकल्पनेनाकाशस्यग्रहाणां तथा तथा स्वशक्त्या तदाकर्षणात् फलद्वयसंस्काररूपैकफलोत्पादनं सङ्गच्छते । अत एव सूत्रम् ग्रहबिम्बप्रोतं कक्षाकारमिति कल्यनमपि निरस्तम्। उच्चद्वयात् तुल्यकर्षणेन विरुद्धकर्षणेन च सूत्रमण्डलभङ्गापत्तेरिति~॥~१०~॥\\
\noindent अथैतदुपसंहरति \textendash
\end{sloppypar}

\newpage
\noindent ६४ \hspace{4cm} सूर्यसिद्धान्तः
\vspace{1cm}
\begin{quote}

{\ssi अतो धनर्णं सुमहत् तेषां गतिवशाद्भवेत् ~।\\
आकृष्यमाणास्तैरेवं व्योम्नि यान्त्यनिलाहताः ~॥~११~॥}
%\vspace{2mm}
\end{quote}
\begin{sloppypar}
अथ पूर्वोक्तसुदूराकर्षणप्रतिपादनात् तेषां भौमादीनां गतिवशादाकर्षणोत्पन्नचलनगतिवशात् सुमहदत्यधिकं फलं धनर्णं स्वोच्चापकृष्टेत्यादिना भवति । नन्वाकर्षणोत्पन्नचलनं कथं नप्रत्यक्षत्मित्यत आह\textendash आकृष्यमाणा इति~। तैरुच्चपातदैवतैरेवमुक्तप्रकारेणाकृष्यमाणा आकर्षिता एते भौमादयो व्योम्नि स्वस्वाकाशगोलेऽनिलाहताः पश्चिमाभिमुखानवरतप्रवहवाय्वाघाता यान्ति गच्छन्ति । तथा चावलम्बनोत्पन्नपूर्वगतिर्यथा न प्रत्यक्षा तथा पूर्वगतिविकृत्यात्मकमेतदाकर्षणचलनमनियतं प्रवहवायुभ्रमणप्राबल्यादप्रत्यक्षमिति भावः ~॥~११~॥\\
\noindent अथैवं गतिकारणसञ्चयैर्ग्रहाणां भौमादीनं फलितैका गतिरष्टभेदात्मिकेत्याह\textendash
\end{sloppypar}
%\vspace{2mm}
\begin{quote}

{\ssi वक्रानुवक्रा कुटिला मन्दा मन्दतरा समा ~।\\
तथा शीघ्रतरा शीघ्रा ग्रहाणामष्टधा गतिः ~॥~१२~॥}
%\vspace{2mm}
\end{quote}
\begin{sloppypar}
भौमादिग्रहाणां विरविचन्द्राणामष्टप्रकारा गतिः फलिता। तत्र वक्रेत्यादिसमेत्यन्तं षट्प्रकारा गतिः शीघ्रतरा शीघ्रेति गतिद्वयम् । तथा समुच्चये । आसां स्वरूपज्ञानमग्रे स्फुटम् ~॥~१२~॥\\
\noindent अथैनामष्टधा गतिं भेदद्वयेन क्रोडयति ।
\end{sloppypar}
\begin{quote}

{\ssi तत्रातिशीघ्रा शीघ्राख्या मन्दा मन्दतरा समा ~।\\
ऋज्वीति पञ्चधा ज्ञेया या वक्रा सानुवक्रगा ~॥~१३~॥}
%\vspace{2mm}
\end{quote}
तत्राष्टविधगतिष्वतिशीघ्रेत्यादिसमेत्यन्ता इत्येवं पञ्चधान

\newpage

\hspace{3cm} गूढार्थप्रकाशकेन सहितः~। \hfill ६५
\vspace{1cm}

\begin{sloppypar}
\noindent गतिः । ऋज्वी मार्गी गतिर्ज्ञेया या गतिः सानुवक्रगानुवक्रगमनेन सह वर्तमाना पूर्वश्लोकेऽनुवक्रगतेर्वक्रकुटिलमध्याभिधानादुभयथासन्नत्वाच्च वक्रानुवक्रा कुटिलेति गतिर्वक्रा ज्ञेया तथा च ग्रहाणां मार्गो वक्रेति गतिद्वयम् ~॥~१३~॥\\
\noindent अथ ग्रहाणां स्पष्टक्रियां प्रतिजानीते\textendash
\end{sloppypar}
%\vspace{2mm}
\begin{quote}

{\ssi तत्तद्गतिवशान्नित्यं तथा दृक्तुल्यतां ग्रहाः ~।\\
प्रयान्ति तत् प्रवक्ष्यामि स्फुटीकरणमादरात् ~॥~१४~॥}
%\vspace{2mm}
\end{quote}
\begin{sloppypar}

नित्यं प्रत्यहं तत्तड्गतिवशात् तास्ता गतय एकस्मिन् दिने शीघ्रापरदिनेऽतिशीघ्रेत्यादिना यस्मिन् दिने या गतिस्तत्सम्बन्धानुरोधादित्यर्थः । ग्रहाः सूर्यादयो यथा येन प्रकारेण दृक् तुल्यतां वेधितग्रहसमतां गच्छन्ति तत् तादृशं स्फुटीकरणं स्पष्टक्रियागणितप्रकारमादरादत्यन्ताभिनिवेशादेतेनासङ्गत्वतनिरासः । प्रवक्ष्यामि सूक्ष्मत्वेन कथयामि ~॥~१४~॥\\ 
\noindent अथ तत्र प्रथमं न्यासाधनार्थं ज्यार्धपिण्डान् विवक्षुस्तदानयनं श्लोकाभ्यामाह\textendash
\end{sloppypar}
%\vspace{2mm}
\begin{quote}

{\ssi राशिलिप्ताष्टमो भागः प्रथमं ज्यार्धमुच्यते ~।\\
तत्तद्विभक्तलब्धोनमिश्रितं तद्द्वितीयकम् ~॥~१५~॥

आद्येनैवं क्रमात् पिण्डान् भक्ता लब्धोनसंयुताः ~।\\
खण्डकाः स्युश्चतुर्विंशज्ज्यार्धपिण्डाः क्रमादमी ~॥~१६~॥}
%\vspace{2mm}
\end{quote}
\begin{sloppypar}
एकराशिकलानामष्टादशशतानामष्टमोंऽशस्तत्त्वाश्विमितः प्रथममाद्यं ज्यार्धं सम्पूर्णजीवार्धपिण्डकः कथ्यते तदभिज्ञैः । ततः
\end{sloppypar}

{\tiny{K}}

\newpage


\noindent ६६ \hspace{4cm} सूर्यसिद्धान्तः
\vspace{1cm}

\begin{sloppypar}
प्रथमज्यार्धात् तेन प्रथमज्यार्धेन भक्ताल्लब्धेन हीनमन्यस्याप्रसङ्गात् प्रथमज्यार्धमनेन युक्तं तत् प्रथमज्यार्धं द्वितीयकं ज्यार्धं भवति । द्विगुणप्रथममेकोनम् । तृतीयादीनामानयनार्थमुक्तप्रकारमतिदिशति । आद्येनेति । प्रथमज्यार्धपिण्डेन । एवमुक्तरीत्या क्रमात् सिद्धपिण्डान् भक्ता लब्धैरूनमाद्यं खण्डमनेनयुताः खण्डका असिद्धाव्यवहितसिद्धज्यार्धपिण्डा असिद्धपिण्डा भवन्ति । यथा प्रथमखण्डं २२५ प्रथमभक्तं फलं १ द्वितयिखण्डं ४४९ प्रथमभक्तं फलं द्वयं २ अर्धाधिकावयवस्यैकाधिकत्वेन ग्रहस्य साम्प्रदायिकत्वात् । फलैक्यो३नं प्रथमं २२२ अनेन द्वितीयखण्डो ४४९ युतस्तृतीयं ६७१ एवमिदं प्रथमखण्डभक्तं फलं ३ अनेन पूर्वफलैक्यं ३ युतं जातं ६ सर्वफलैक्यमनेन प्रथमं खण्डं हीनं २१९ अनेन तृतीयं ६७१ युतं चतुर्थं ८९० एवमिदं प्रथमखण्डभक्तं फलं ४ पूर्वलब्धैक्योनप्रथमखण्डरूपं २१९ ज्यान्तररूपखण्डकमनेन ४ हीनं २१५ अनेन चतुर्थं युतं पञ्चमं ११०५ एवमग्रेऽपि । अथोक्तरीत्यासङ्ख्यखण्डानां सम्भवात् खण्डनियममाह\textendash स्युरिति~। एवं चतुर्विंशत्सङ्ख्याका ज्यार्धपिण्डाः कार्या न तदधिका । अत्र\textendash
\end{sloppypar}
\begin{quote}

{\qt एकविंशाच्च विंशाच्च षष्ठात् पञ्चदशादपि ~।\\
सप्तमाद्द्वादशात् सप्तदशान्नार्धोत्तरं मतम् ~॥}
\end{quote}
\begin{sloppypar}

इति ब्रह्मसिद्धान्तोक्तस्थलेऽर्धाधिकावयवस्यैकाधिकत्वेन न ग्रह इति ध्येयम् । गणितस्याविकृतत्वात् सिद्धाः पिण्डाः कथं नोक्ता इत्यत आह\textendash क्रमादिति~। अमी सिद्धीः पिण्डाः क्रमात्\textendash
\end{sloppypar}

\newpage


\hspace{3cm} गूढार्थप्रकाशकेन सहितः ~। \hfill ६७
\vspace{1cm}

\begin{sloppypar}

\noindent समनन्तरमेवोच्यन्ते । अत्रोपपत्तिः । समायां भूमौ वृत्तं भगणकलाङ्कितं तिर्यगूर्ध्वाधरव्यासमितरेखाभ्यां चतुर्भागं कार्यं तत्रोर्ध्वरेखासक्तपरिधिप्रदेशादुभयत्र समविभागं विगणय्य तदग्रयोर्बद्धं सूत्रं वृत्ते द्विगुणविभागमितसम्पूर्णचापस्य सम्पूर्णज्या । अत्र गणित ऊर्ध्वरेखातोऽर्धज्याया एव प्रयोजनात् तदर्धचापस्य तदर्धमर्धज्या । एवं वृत्तचतुर्थांश ऊर्ध्वरेखातोऽभीष्टाशानां चापार्धाकाराणामर्धज्या अभीष्टा गण्याः । तत्र भगवता स्वेच्छया वृत्तचतुर्थांशे त्रिराशिमिते चतुर्विंशज्ज्याः कल्पितास्तज्ज्ञानं तु वृत्ते चक्रकलानामङ्कितत्वात् तत्परिधिव्यासार्धं त्रिराशिज्यान्तिमा । भनन्दाग्निमितपरिधौ खबाणगसूर्यमितो व्यासस्तदा चक्रकलापारधौ क इत्यनुपातेन व्यासानयनम्। यथा चक्रकलाः २१६०० खबाणसूर्यगणाः २७०००००० भनन्दाग्नि ३९२७ भक्ता व्यासः ६८७६ एतदर्धमन्तिमा ज्या ३४३८ अथ वृते चापज्ययोर्विवेके तयोरतुल्यत्वमपि भगवता कोऽपि वृत्तभागः समोऽस्त्यन्यथामलकादौ सर्षपाद्यवस्थानं न स्यादिति मत्वा तद्भागस्य ज्या तत्तुल्यैवेति ।
\end{sloppypar}
%\vspace{2mm}
\begin{quote}

{\qt वृत्तस्य षष्णवत्यंशो दण्डवदृश्यते तु सः ~।} 
%\vspace{2mm}
\end{quote}
\begin{sloppypar}
इति शाकल्योक्तेः । प्रथमज्या चक्रकलाद्वादशांशरूपैकराशिकलानामष्टभागस्तत्त्वाश्विमितः । एतन्मितमेव प्रथमचापमत एतदन्तरेणाभीष्टा ज्याश्चतुर्विंशत् । अथ चतुर्विंशतिजीवानां यथोत्तरमुपचयात् तदन्तररूपखण्डानां यथोत्तरमपचयस्य वृत्ते ज्याङ्कनेन प्रत्यक्षत्वाज्ज्यान्तररूपखण्डानामन्तरं यथोत्त\textendash
\end{sloppypar}
{\tiny{K 2}}

\newpage

\noindent ६८ \hspace{4cm} सूर्यसिद्धान्तः
\vspace{1cm}

\begin{sloppypar}

\noindent रमुपचितमिति द्वाविंशतित्रयोविंशतिचतुर्विंशतिज्यानामन्तरयोरन्तरमिदं परमं खण्डान्तरं सूक्ष्मज्योत्पत्तिप्रकारेणावगतं ~१५~।~१६~।~४८~। अथ त्रिज्ययेदं खण्डकान्तरं तदा प्रथमज्यया किमित्यनुपातेन फलप्रमाणयोः फलेनापवर्त्य प्रमाणस्थाने तत्त्वाश्विनोऽनेन भक्ता प्रथमज्या फलं पूर्वद्वितीयखण्डयोरन्तरम् । अनेन पूर्वखण्डं हीनं द्वितीयं खण्डं भवति । तत्र पूर्वखण्डं प्रथमज्यातुल्यमेव । द्वितीयखण्डं प्रथमज्यायां युतं द्वितीयज्या । एवमस्यास्तत्त्वाश्विभागलब्धं द्वितीयतृतीयखण्डकयोरन्तरमनेन द्वितीयखण्डमूनं तृतीयखण्डमित्यमेन द्वितीयज्या युता तृतीयज्या । एवं चतुर्थाद्याः । तत्र पूर्वमर्धाभ्यधिकग्रहणेनोत्तरत्राधिकान्तरपातसम्भावनया क्वचित् क्वचिदर्धाभ्यधिकावयवस्यैकाधिकत्वेनाग्रह इत्युपपन्नं श्लोकद्वम् ~॥~१६~॥\\
\noindent अथैताः सिद्धाः श्लोकषट्केन कथयन्नुत्क्रमज्यार्धपिण्डज्ञानमाह\textendash
\end{sloppypar}
%\vspace{2mm}
\begin{quote}

{\ssi तत्त्वाश्विनोऽङ्काब्धिकृता रूपभूमिधरर्तवः ~।\\
खाङ्काष्टौ पञ्चशून्येशा बाणरूपगुणेन्दवः ~॥~१७~॥

शून्यलोचनपञ्चैकाश्क्विद्ररूपमुनीन्दवः ~।\\
वियच्चन्द्रातिधृतयो गुणरन्ध्राम्बराश्विनः ~॥~१८~॥

मुनिषड्यमनेत्राणि चन्द्राग्निकृतदस्रकाः ~।\\
पञ्चाष्टविषयाक्षीणि कुञ्जराश्विनगाश्विनः ~॥~१९~॥

रन्ध्रपञ्चाष्टकयमा वस्वद्यङ्कयमास्तथा ~।\\
कृताष्टशून्यज्वलना नागाद्रिशशिवह्नयः ~॥~२०~॥}
\end{quote}
\newpage

\hspace{3cm} गूढार्थप्रकाशकेन सहितः ~। \hfill ६९
\vspace{1cm}
\begin{quote}

{\ssi षट्पञ्चलोचनगुणाश्चन्द्रनेत्राग्निवह्नयः ~।\\
यमाद्रिवह्निज्वलना रन्ध्रशून्यार्णवाग्नयः ~॥~२१~॥

रूपाग्निसागरगुणा वस्वग्निकृतवह्नयः ~।\\
प्रोज्झ्योत्क्रमेण व्यासार्धादुत्क्रमज्यार्धपिण्डकाः ~॥~२२~॥}
%\vspace{3mm}
\end{quote}
\begin{sloppypar}
तथा समुच्चये । एतानुक्तान् क्रमज्यार्धपिण्डान् । उत्क्रमेणोपान्त्यपिण्डादिप्रथमपिण्डान्तं प्रत्येकं व्यासार्धात् त्रिज्यारूपपरमपिण्डात् प्रोज्झ्य न्यूनीकृत्य क्रमेणोत्क्रमज्यार्धपिण्डा भवन्ति । यथा त्रयोविंशतितमं ज्यार्धमुक्तं रूपाग्निसागरगुणा इति वस्वग्निकृतवक्रय इति चरमपिण्डादूनं सप्त प्रथम उत्क्रमज्यार्धपिण्डः । एवं द्धाविंशतितमं चरमाच्छुद्धं द्वितीय उत्क्रमज्यार्धपिण्डः । एवमग्रेऽपीति चतुर्विंशदुत्क्रमज्यार्धपिण्डाः । अत्रोपपत्तिः\textendash ज्याचापयोर्बाणरूपमन्तरमुत्क्रमज्या । यद्यपि पूर्वार्धज्यावद्बाणस्यार्धं न सम्भवतीत्युत्क्रमज्यापिण्डा इति वक्तुमुचितं नोत्क्रमज्यार्धपिण्डा इति । तथापि भगवतानुगतपरिभाषार्थं चापबाह्यशराग्राभावेनोत्क्रमज्यायाः पूर्णशरांशत्वादुत्क्रमज्यार्धमित्युक्तम् । अथ वृत्तचतुर्थांशे सर्वज्याङ्कनेन यदंशानां ज्या त्रिज्यातो हीना तत्कोट्यंशानामुत्क्रमज्येति स्फुटं दृश्यत अत उक्तज्यार्धक्रमेणोत्क्रमज्याज्ञानार्थं व्युत्क्रमेण त्रिज्या शुद्धा उक्तपिण्डा उत्क्रमज्यापिण्डा इत्युपपन्नं प्रोज्झ्येत्यादि ~॥~२२~॥\\
\noindent अथ श्लोकपञ्चकेनोत्क्रमज्यापिण्डान पूर्वोक्तसिद्धान् निबध्नाति\textendash
\end{sloppypar}
\newpage

\noindent ७० \hspace{4cm} सूर्यसिद्धान्तः
\vspace{1cm}
\begin{quote}

{\ssi मुनयो रन्ध्रयमला रसषट्का मुनीश्वराः ~।\\
द्व्यष्टैका रूपषड्दस्राः सागरार्थहुताशनाः ~॥~२३~॥

खर्तुवेदा नवाद्र्यर्था दिङ्नागास्त्र्यर्थकुञ्जराः ~।\\
नगाम्बरवियच्चन्द्रा रूपभूधरःशङ्कराः ~॥~२४~॥

शरार्णवहुताशैका भुजङ्गाक्षिशरेन्दवः ~।\\
नवरूपमहीध्रैका गजैकाङ्कनिशाकराः ~॥~२५~॥

गणाश्विरूपनेत्राणि पावकाग्निगुणाश्विनः ~।\\
वस्वर्णवार्थयमलास्तुरङ्गर्तुनगाश्विनः ~॥~२६~॥

नवाष्टनवनेत्राणि पावकैकयमाग्नयः ~।\\
गजाग्निसागरगुणा उत्क्रमज्यार्धपिण्डकाः ~॥~२७~॥}
%\vspace{3mm}
\end{quote}
\begin{sloppypar}

एत उत्क्रमज्यापिण्डाः पूर्वसिद्धा निबद्धा महीध्रः पर्वतो भुजज्याभावे कोट्युत्क्रमज्यायाः परमत्वाच्छून्यज्योना त्रिज्या परमोत्क्रमज्यापिण्डस्त्रिज्याया उभयत्र परमत्वेनार्थसेद्धमन्त्यपिण्डत्वं वेति ध्येयम् ~॥~२७~॥\\
\noindent अथ प्रसङ्गात् परमक्रान्तिज्यां वदन् क्रान्त्यानयनमाह\textendash
\end{sloppypar}
\begin{quote}
%\vspace{2mm}

{\ssi परमापक्रमज्या तु सप्तरन्ध्रगुणेन्दवः ~।\\
तद्गुणा ज्या त्रिजीवाप्ता तच्चापं क्रान्तिरुच्यते ~॥~२८~॥}
%\vspace{2mm}
\end{quote}
\begin{sloppypar}

त्र्यूनं चतुर्दशशतं १३९७ परमक्रान्तिज्या तुकाराच्चतुविशत्यंशानां वक्ष्यमाणज्यानयनप्रकारसिद्धेत्यर्थः । अभीष्टा ज्या परमक्रान्तिज्यया गुणिता त्रिज्याभक्ता फलस्य वक्ष्यमाणप्रकारेण
\end{sloppypar}

\newpage

\hspace{3cm} गूढार्थप्रकाशकेन सहितः ~। \hfill ७१
\vspace{1cm}

\begin{sloppypar}
\noindent धनुः कान्तिः कलात्मिका तत्त्वज्ञैः कथ्यते । अत्रोपपत्तिः । विषुवट्टत्तात् क्रान्तिवृत्तभागस्य याम्योत्तरस्यान्तरं ध्रुवाभिमुखसूत्रे सूत्राकारे वृत्ताकारक्रान्तिः । तत्र सायनमेषतुलादिस्थाने तयोरन्तराभावात् कर्कमकरादौ तयोः परमान्तरत्वादभीष्टभुजज्यावशात् क्रान्तिरुपपन्नेति त्रिज्यातुल्यभुजज्यया परमक्रान्तिज्या तदेष्टभुजज्यया केत्यनुपातेन फलं ध्रुवाभिमुखसूत्रे तदन्तररूपार्धचापस्यार्धज्या विषुवट्टत्तोर्ध्वाधरमध्यसूत्रात् तच्चापं तदन्तरकलात्मिका क्रान्तिः ~॥~२८~॥\\
\noindent अथ फलानयनार्थं केन्द्रपदाद्भुजकोटिज्ये कार्ये इत्याह\textendash
\end{sloppypar}
%\vspace{2mm}
\begin{quote}

{\ssi ग्रहं संशोध्य मन्दोच्चात् तथा शीघ्राद्विशोध्य च ~।\\
शेषं केन्द्रपदं तस्माद् भुजज्या कोटिरेव च ~॥~२९~॥}
%\vspace{2mm}
\end{quote}
\begin{sloppypar}
ग्रहं राश्यादिकं मन्दोच्चात् प्रागानीतस्वकीयराश्यादिकमन्दोच्चभोगात् संशोध्योनीकृत्य शीघ्रात् प्रागानीतराश्यादि शीघ्रोच्चात् । चः समुच्चये । ऊनीकृत्य शेषं राश्यात्मकं तथोच्चसम्बधेन केन्द्रं मन्दोच्चाद्धीनो ग्रहो मन्दकेन्द्रम् । शीघ्रोच्चाद्धीनो ग्रहः शीघ्रकेन्द्रं भवतीत्यर्थः । तस्मात् केन्द्रात् पदं राशित्रयात्मकं विषमं समं पदं ज्ञेयम् । त्रिराश्यन्तर्गतं चेत् प्रथमं विषमं पदम् । ततः षड्राश्यन्तर्गतं चेत् त्र्यूनं केन्द्रं द्वितीयं समं पदम् । ततो नवराश्यन्तर्गतं चेत् षडूनं तृतीयं विषमं पदम् । ततो नवोन चतुर्थं पदं सममित्यर्थः । तस्मात् पदाद् भुजस्य ज्या कोटिः कोटेर्ज्या चः समुच्चये । एवकारादेकाद्वयं\textendash
\end{sloppypar}

\newpage

\noindent ७२ \hspace{4cm} सूर्यसिद्धान्तः
\vspace{1cm}

\begin{sloppypar}
\noindent साध्यमित्यर्थः ~॥ अत्रोपपत्तिः । उच्चस्थानाभिमुखमुच्चदैवतैर्ग्रहाणामाकर्षणोक्तेरुच्चाद्ग्रहः कियदन्तरेणेति ज्ञानार्थमुच्चहीनो ग्रहः केन्द्रमुच्चग्रहणवशात् तदाख्यम् । तत्र भगवता स्वेच्छया ग्रहादुच्चं यदन्तरेण तत् केन्द्रं कृतम् । उभयथा भुजकोट्योस्तुल्यत्वात् । द्वादशराश्यङ्किते वृत्त उच्चस्थानाच्चतुर्विभागात्मक एकैको भागो राशित्रयात्मकः पदसञ्ज्ञः ।अथोच्चस्थानाद्ग्रहः कस्मिन् पदेऽस्तीति शून्यत्रिषणवोनं केन्द्रं कृतं ज्यानां पदान्तर्गतत्वात् । ग्रहाधिष्ठितपदाद्भुजज्याकोटिज्ययोर्ज्ञानम् ~॥~२९~॥\\
\noindent ननु पदे ग्रहस्य राशिविभागात्मकेनैकत्वाद् भुजकोटिज्ययोरतुल्ययोः साधनं कथमित्यत आह \textendash
\end{sloppypar}
%\vspace{2mm}


\begin{quote}
{\ssi गताद्भुजज्या विषमे गम्यात् कोटिः पदे भवेत्~।\\
युग्मे तु गम्याद्वाहुज्या कोटिज्या तु गताद्भवेत्~॥~३०~॥ }
\end{quote}
%\vspace{2mm}

\begin{sloppypar}
विषमे पदे गताद्ग्रहस्य पदादितो यद्गतं राशिविभागात्मकं प्राग् ज्ञातं तस्मादित्यर्थः । भुजज्या स्यात् । गम्याद्गतोनं त्रिभंग्रहात् पदान्तावधिकमेव्यम् । तस्मात् कोटिः कोटिज्या स्यात् ।युग्मे समे तुकारात् पद एव्याद्भुजज्या गतात् कोटिज्या स्यात् ।तुकारो विशेषद्योतकः । एकस्मादेवोक्तरीत्या द्वयं साधितमित्यर्थः । अत्रोपपत्तिः \textendash विषमपदेग्रहोच्चोर्धाधररेखान्तरानुसारेण फलमुत्पद्यते ततो वृत्तान्तस्तदन्तरमर्धज्या भुजरूपातदर्धचापं तदन्तरांशा वृत्तभावस्या गताः । ऊर्ध्वाधररेखामत्स्यसम्पन्नतिर्यग्रेखाग्रहयोरन्तरसूत्रमर्धज्यापदान्तः कोटिज्या
\end{sloppypar}

\newpage

\hspace{3cm}गूढार्थप्रकाशकेन सहितः~। \hfill ७३
\vspace{1cm}

\begin{sloppypar}

\noindent भुजोत्क्रमज्योनव्यासार्धरेखारूपकोटितुल्यत्वात् । तदर्धचापं भुजांशोन त्रिभमिति गम्यात् कोटिज्या । समपदे ग्रहोर्ध्वाध ररेखान्तरं तिर्यगर्धज्या भुजज्येति तदर्धं चापं यदैष्यं तिर्यग्रेखाग्रहान्तरं तिर्यगर्धज्याकोटितुल्यत्वात् कोटिस्तच्चापं पदगतमित्युपपन्नं गतादित्यादि ~॥~३०~॥\\ 
\noindent अथाभीष्टकलानां ज्यासाधनं श्लोकाभ्यामाह\textendash
\end{sloppypar}
%\vspace{2mm}
\begin{quote}

{\ssi लिप्तास्तत्त्वयमैर्भक्ता लब्धं ज्यापिण्डकं गत~।\\
गतगम्यान्तराभ्यस्तं विभजेत् तत्त्वलोचनैः~॥~३१~॥

तदवाप्तफलं योज्यं ज्यापिण्डे गतसञ्ज्ञके~।\\
स्यात् क्रमज्याविधिरयमुत्क्रमज्यास्वपि स्मृतः~॥~३२~॥ }
%\vspace{2mm}
\end{quote}
\begin{sloppypar}
यस्य राश्यात्मकस्य पदान्तर्गतस्य ज्या कर्तुमिष्टा तस्य कलाः कार्याः । तत्त्वाश्विभिर्भक्ता लब्धं चतुर्विंशज्ज्यापिण्डेषु पूर्वोक्तेषु लब्धसङ्ख्याकः पिण्डो गतो भवति तदग्रिमपिण्ड एष्यः पूर्वं तु स्वरूपोक्त्यर्थं पिण्डानां ज्यार्धेर्त्युक्तिरिदानीं तु तेषामेवार्धत्यागेन ज्यापिण्डत्वोक्तिः । अर्धग्रहणे गणितक्रियायां व्याकुलतापत्तेः । न तु पूर्वपिण्डाद्द्विगुणा गणितक्रियायां ग्राह्या इत्याशयेनार्धानुक्तिर्गौरवात् । भागेऽवशिष्टं तद्गतैष्यपिण्डयोरन्तरेण गुणितं तत्त्वाश्विभिर्भजेत् तस्मात् प्राप्तं यत् कलादिकं फलं तद्गते ज्यापिण्डे युक्तं कार्यम् । उत्कमज्याभीष्टाशकलोनामर्ध- ज्यारूपा क्रमज्या 'भवति । अयमुक्तः प्रकार अन्तमज्यापिण्डेषु कथितः । अमीमृांशकलानामुत्क्रमज्यापिण्डैरुक्तविधिनोत्क्रमज्या
\end{sloppypar}

{\tiny{L}}

\newpage

\noindent ७४ \hspace{4cm} सूर्यसिद्धान्तः
\vspace{1cm}

\begin{sloppypar}
\noindent स्यादित्यर्थः । अत्रोपपत्तिः । तत्त्वाश्विकलाभिरेका ज्या तदाभीष्टकलाभिः केत्यनुपातेन गतज्या ततस्तत्त्वाश्विकलाभिर्गताग्रिमज्यान्तरं लभ्यते तदा शेषकलाभिः केत्यनुपातागतलब्धेन युक्ताभीष्टज्या~॥~३२~॥\\
\noindent अथ ज्यातो धनुरानयनमाह\textendash
\end{sloppypar}
\begin{quote}

{\ssi ज्यां प्रोज्झ्य शेषं तत्त्वाश्विहतं तद्विवरोड्टुतम्~।\\
सङ्ख्या तत्त्वाश्विसंवर्गे संयोज्य धनुरुच्यते~॥~३३~॥}
\end{quote}
\begin{sloppypar}
यस्य धनुः कर्तुमिष्टं तस्मिन्नशुद्धपूर्वं ज्यापिण्डं न्यूनीकृत्य शेषं पञ्चाकृतिगुणं तद्विवरोद्धृतं तयोः शुद्धाशुद्धपिण्डयोरन्तरेण भक्तं फलं शुद्धज्या यतमा ततमसङ्ख्या तत्त्वाश्विनोः संवर्गे घाते संयोज्य सिद्धं धनुः कथ्यते । अत्रोपपतिः । ज्या यतमा शुद्ध्यति ततमायाश्चापकलास्ततमसङ्ख्यागुणिततत्त्वाश्विनः । ज्यान्तरेण तत्त्वाश्विकलास्तदा शेषज्यया केत्यनुपातागतफलयुता इति वैपरीत्येन सुगमतरा~॥~३३~॥\\
\noindent अथ ग्रहाणां मन्दपरिध्यंशान विवक्षुः प्रथमं सूर्यचन्द्रयोराह\textendash
\end{sloppypar}
%\vspace{2mm}
\begin{quote}

{\ssi रवेर्मन्दपरिध्यंशा मनवः शीतगो रदाः~।\\
युग्मान्ते विषमान्ते च नखलिप्तोनितास्तयोः~॥~३४~॥}
%\vspace{2mm}
\end{quote}
\begin{sloppypar}
सूर्यस्य परमाकर्षणोत्पन्नपरमपूर्वापरगमनरूपपरममन्दफलांशानां ज्या परमफलज्या तत्तुल्यव्यासार्धेनोत्पन्नवृत्ते कक्षावृत्तस्थितांशप्रमाणेन येंऽशास्ते मन्दपरिध्यंशाः केन्द्रयुग्मपदान्ते नीचोच्चसमेऽर्के चतुर्दश चन्द्रस्य तत्र ते द्वात्रिंशत् ।
\end{sloppypar}

\newpage

\hspace{3cm}गूढार्थप्रकाशकेन सहितः~। \hfill ७५
\vspace{1cm}

\begin{sloppypar}
\noindent केन्द्रविषमपदान्ते नीचोच्चाभ्यां त्रिभान्तरिते चकारादुक्तामन्दपरिध्यंशा विंशतिकलोनाः सन्तः सूर्यचन्द्रयोर्मन्दपरिध्यंशा भवन्ति~॥~३४~॥\\
\noindent अथ भौमादीनामाह\textendash
\end{sloppypar}
%\vspace{2mm}
\begin{quote}

{\ssi युग्मान्तेऽर्थाद्रयः खाग्नी सुराः सूर्या नवार्णवाः~।\\
ओजे द्व्यगा वस्तुयमा रदा रुद्रा गजाब्धयः~॥~३५~॥}
%\vspace{2mm}
\end{quote}
\begin{sloppypar}
भौमस्य पञ्चसप्ततिः । बुधस्य त्रिंशत् । गुरोस्त्रयस्त्रिंशत् । शुक्रस्य द्वादश । शनेरेकोनपञ्चाशत् । पूर्वोक्तमन्दपरिध्यंशा इति वक्ष्यमाणकुजादीनामिति चात्रान्वेति । एते युग्मपदान्ते । ओजे विषमपदान्ते भौमस्य द्विसप्ततिः । बुधस्याष्टाविंशतिः। गुरोर्द्वात्रिंशत् । शुक्रस्यैकादश । शनेरष्टचत्वारिंशत्~॥~३५~॥\\
\noindent अथ भौमादीनां युग्मपदान्ते शैघ्र्यपरिध्यंशानाह\textendash
\end{sloppypar}
%\vspace{2mm}
\begin{quote}

{\ssi कुजादीनामतः शैघ्र्या युग्मान्तेऽर्थाग्निदस्रकाः~।\\
गुणाग्निचन्द्राः खनगा द्विरसाक्षीणि गोऽग्नयः~॥~३६~॥}
%\vspace{2mm}
\end{quote}
\begin{sloppypar}

भौमादीनामतो मन्दपरिध्यंशकथनानन्तरं शैघ्र्याः शीघ्रपरिध्यंशा युग्मपदान्ते भौमस्य पञ्चत्रिंशदधिकं शतद्वयम् । बुधस्य त्रयस्त्रिंशदधिकं शतम् । गुरोः सप्ततिः । शुक्रस्य द्विषष्ट्यधिकं शतद्वयम् । शनेरेकोनचत्वारिंशत्~॥~३६~॥\\
\noindent अथैतेषां विषमपदान्ते शैघ्र्यपरिध्यंशानाह\textendash
\end{sloppypar}
%\vspace{2mm}
\begin{quote}

{\ssi ओजान्ते द्वित्रियमला द्विविश्वे यमपर्वताः~।\\
खर्तुदस्रा वियद्वेदाः शीघ्रकर्मणि कीर्त्तिताः~॥~३७~॥}
%\vspace{2mm}
\end{quote}
{\setlength{\parindent}{5em}विषमपदान्ते शीघ्रकर्मणि शीघ्रफलसाधनार्थं परिधय}

{\tiny{L 2}}

\newpage

\noindent७६ \hspace{4cm} सूर्यसिद्धान्तः
\vspace{1cm}

\begin{sloppypar}
\noindent उक्ताः । एते शीघ्रपरिधयः कुजादीनामिति पूर्वोक्तमत्रान्वेति। भौमस्य दन्ताश्विनः । बुधस्य दन्तेन्दवः । गुरोर्द्विसप्ततिः । शुक्रस्य षष्ट्यधिकं शतद्वयम् । शनेश्चत्वारिंशत् । अत्र कीर्तिता इत्यनेन युग्मान्ते फलाभावादेव परिधयः कथं सम्भवन्ति । अतो विषमपदान्ते परमफलस्य सत्त्वात् तत्रैव युक्ताः परिधयः शनिमन्दशीघ्रपरिध्योः क्रमेणाधिकन्यूनत्वं च सञ्ज्ञाव्याघातादयुक्तमित्यादि नाशङ्कनीयमागमप्रामाण्यात्\textendash
\end{sloppypar}


\begin{quote}
{\qt श्रुतिर्यत्र प्रमाणं स्याद् युक्तिः का तत्र नारद~।}
\end{quote}
\begin{sloppypar}

इति ब्रह्मसिद्धान्तोक्तेश्चेति सूचितम्~॥~३७~॥\\
अथाभीष्टकेन्द्रसम्बन्धेन परिधिभागानयनमाह\textendash
\end{sloppypar}
\begin{quote}

{\ssi ओजयुग्मान्तरगुणा भुजज्या त्रिज्ययोड्टता~।\\
युग्मवृत्ते धनर्णं स्यादोजादूनाधिके स्फुटम्~॥~३८~॥}
%\vspace{2mm}
\end{quote}
\begin{sloppypar}
भुजज्या यत्परिधिः स्फुटीकर्तुमिष्यते तत्केन्द्रस्य मन्दशीघ्रान्यतरस्य भुजज्यौजयुग्मान्तरगुणा विषमसमपदान्तीयकेन्द्रीयपरिध्योरन्तरेण गुणिता त्रिज्यया भक्ता फलं युग्मवृत्ते केन्द्रयुग्मपदान्तीयपरिधौ । ओजात् केन्द्रीयविषमपदान्तीयपरिधेः सकाशादूनाधिके क्रमेण धनर्णं हीने युक्तमधिके हीनं स्फुटं परिधिमानं स्यात् । अचोपपत्तिः । युग्मपदान्तीयस्थात् प्ररिधेर्विषमपदान्तीयपरिधिर्यावता न्यूनाधिकस्तदन्तरं विषमपदत्वाद् भुजज्ययोपचितमतस्त्रिज्यातुल्यभुजज्ययेदमन्तरं तदेष्टभुजज्यया किमिति फलं युग्मपरिधौ । ओजपरिधेर्न्यूनत्वे ऋण\textendash
\end{sloppypar}

\newpage

\hspace{3cm}गूढार्थप्रकाशकेन सहितः~। \hfill ७७
\vspace{1cm}

\begin{sloppypar}
\noindent मधिकत्वे धनमिति । विषमपदपरिधेरधिकन्यूनयुग्मपरिधावेवर्णधनं कृतमित्युपपन्नम्~॥~३८~॥\\
\noindent अथ भुजकोटिफलानयनं मन्दफलानयनं चाह\textendash
\end{sloppypar}
%\vspace{2mm}
\begin{quote}

{\ssi तद् गुणे भुजकोटिज्ये भगणांशविभाजिते~।\\
तद्भुजज्याफलधनुर्मान्दं लिप्तादिकं फलम्~॥~३९~॥}
%\vspace{2mm}
\end{quote}
\begin{sloppypar}
भुजकोटिज्ये मन्दशीघ्रान्यतरसम्बन्धेन केन्द्रभुजकोटिज्ये तद्गुणे स्वीयस्फुटपरिधिना गुणिते भगणांशैः षष्यधिकशतत्रयेण भक्ते भुजफलकोटिफले भवतः । मन्दकेन्द्रभुजज्योत्पन्नफलस्य धनुः कलादिकं मान्दं फलं भवति । अत्रोपपत्तिः । कक्षास्योच्चस्थानस्थितदेवतया स्वहस्तस्थितसूत्रप्रोतं ग्रहबिम्बं स्वाभिमुखाकर्षणेन कक्षास्थमध्यग्रहस्थानात् परमफलज्यान्तरितस्थान आकर्षणसूत्रमार्गरूपतिर्यक्कर्णमार्गेणाकर्ष्यते । तेन मध्यग्रहस्थानीयकक्षाप्रदेशादन्त्यफलज्याव्यासार्धेनोत्पन्नवृत्ते भगणांशाङ्किते भूमध्यग्रहस्पृग्रेखासक्ततद्वृत्तप्रदेशरूपोच्चस्थानात् केन्द्रान्तरेण कक्षाविपरीतमार्गेण तद्वृत्तपरिधौ ग्रहो भवति । तस्मिन् नीचोच्चवृत्त ऊर्ध्वरेखाग्रहयोस्तिर्यगन्तरसूत्रमर्धज्याकारं परमफलज्यानुरुद्धं भुजफलम् । तस्मिन्नेव वृत्ते व्यासमिततिर्यग्रेखाग्रहयोरन्तरमूर्ध्वाधरमर्धज्याकारं परमफलज्यानुरुद्धं कोटिफलम् । एते तत्र कक्षास्थभुजज्याकोटिज्यावद् भुजकोटिरूपे इति कक्षास्थभगणांशप्रमाणेनैते भुजज्याकोटिज्यारूपे भुजकोटी तदा कक्षास्थभागप्रमाणानुरुद्धप्रागुक्तनीचोच्चपरिधिभागैः केत्यनुपा\textendash
\end{sloppypar}

\newpage

\noindent ७८ \hspace{4cm} सूर्यसिद्धान्तः
\vspace{1cm}

\begin{sloppypar}
\noindent तेन फलवृत्तस्यत्वाद् भुजफलत्कोटिफले । तत्र नीचोच्चपरिधिवृत्तस्थग्रहमध्यसूत्रं कर्णरूपं कक्षावृत्ते यत्र लग्नं तत्र स्पष्टो ग्रहभोगः । नीचोच्चवृत्तमध्यस्पष्टग्रहभोगस्थानयोः कक्षावृत्ते यदन्तरांशमानं तत्फलं तदर्धज्या तिर्यक्सूत्रं मध्यग्रहस्योर्ध्वाधररेखारूपमध्यसूत्रात् स्यष्टग्रहभोगस्थानासक्तं फलज्या । कर्णाग्रे भुजफलं तदा त्रिज्याग्रे किमित्येतदनुपातावगतास्याश्चापं फलम् । तत्र मन्दफलज्या भुजफलरूपा कर्णानुपातोपेक्षया भगवताङ्गीकृता । मन्दकर्णस्य त्रिज्यासन्नत्वेन स्वल्पास्तरेण त्रिज्यातुल्यत्वेनाङ्गीकारात् । तच्चापं मन्दफलमित्युपपन्नं सर्वमुक्तम् । बोधार्थं क्वेद्यकन्यासश्च यथा~॥~३९~॥

\end{sloppypar}
\vspace{3mm}
{\setlength{\parindent}{8em}
अथ शीघ्रफलं श्लोकत्रयेणाह\textendash}
%\vspace{2mm}
\begin{quote}

{\ssi शैघ्र्यं कोटिफलं केन्द्रे मकरादौ धनं स्मृतम्~।\\
संशोध्यं तु त्रिजीवायां कर्कादौ कोटिजं फलम्~॥~४०~॥

तद्वाहुफलवर्गैक्यान्मूलं कर्णश्चलाभिधः~।
त्रिज्याभ्यस्तं भुजफलं चलकर्णविभाजितम्~॥~४१~॥

लब्धस्य चापं लिप्तादिफलं शैघ्र्यमिदं स्मृतम्~।
एतदद्यि कुजादीनां चतुर्थे चैव कर्मणि~॥~४२~॥}
%\vspace{2mm}
\end{quote}
\begin{sloppypar}
शीघ्रसम्बन्धिकोटिफलं मकरादिषड्भे शीघ्रकेन्द्रे त्रिज्यायां योज्यमुक्तम् । कर्कादिषड्भे शीघ्रकेन्द्रे शीघ्रकेन्द्रकोट्युत्पन्नं फलं त्रिज्यायां हीनं कार्यम् । तर्विशेषे । तेन मन्दकर्मण्येतत्क्रियानि\textendash
\end{sloppypar}


\newpage

\hspace{3cm}गूढार्थप्रकाशकेन सहितः~। \hfill ७९
\vspace{1cm}

\begin{sloppypar}
\noindent रामः । कोटिफलसंस्कृतत्रिज्याभुजफलयोर्वर्गयोर्योगान्मूलं शीघ्रसञ्ज्ञः कर्णः । भुजफलं त्रिज्यया गुण्यं शीघ्रकर्णेन भक्तं फलस्य धनुः कलादि । इदं सिद्धं शीघ्रसम्बन्धिफलं कथितम् । भौमादीनामेतच्छीघ्रफलमाद्ये प्रथमे कर्मणि चतुर्थे कर्मणि । चः समुच्चये । कार्यमेवकाराद्वितीयतृतीयकर्मणोर्नेत्यर्थः । अर्थात् तत्र मन्दफलं संस्कार्यमिति सिद्धम् । अत्रोपपत्तिः । मन्दस्पष्टभोगस्थानीयकक्षावृत्तप्रदेशाद्ग्रहबिम्बं शीघ्रोच्चस्थानस्थिततद्देवतया स्वहस्तस्थितसूत्रेण स्वाभिमुखं शीघ्रान्त्यफलज्यान्तरेणाकर्य्यते । तेन मन्दस्पष्टस्थानाच्छीघ्रान्त्यफलज्यया वृत्ते भांशाङ्किते शीघ्रनीचोच्चसञ्ज्ञे पूर्वरीत्या शीघ्रोच्चस्थानाच्छीघ्रकेन्द्रान्तरेण कक्षामार्गवैपरीत्येन ग्रहबिम्बं भवति । तत्र पूर्ववत् कोटिफलभुजफले कोटिभुजौ कक्षास्थतिर्यग्रेखातः शीघ्रनीचोच्चवृत्ततिर्यग्व्यासरेखात्रिज्यान्तरेणेति त्रिज्याकोटिफलयोगो मकरादौ । कर्कादौ कोटिफलोनत्रिज्या शीघ्रनीचोच्चपरिधिस्थग्रहकक्षातिर्यग्रेखयोरन्तरर्जुसूत्ररूपा कोटिः । कोटिमूलमध्ययोरन्तरं कक्षातिर्यग्रेखान्तर्गतं भुजफलतुल्यंभुजो ग्रहभूमध्यस्थसूत्रं तिर्यक् कर्णः । कोटिभुजफलयोर्वर्गयोगमूलं ततः कक्षायां कर्णसूत्रं यत्र लग्नं तत्र स्यष्टो ग्रहभोगः कक्षामध्यसूत्राद्ग्रहसक्तात् स्पष्टभोगस्थानपर्यन्तमर्धज्याकारं सूत्रं शीघ्रफलज्या शीघ्रकर्णाग्रे भुजफलं तदा त्रिज्याग्रे किमित्यनुपातज्ञाता । अस्याश्चापं मन्दस्पष्टस्पष्टग्रहभोगस्यानयोरन्तररूपं शीघ्रफलम् । अथ नोचोच्चवृत्तमध्यज्ञानाय मन्दस्पष्टज्ञानमावश्यकम् । ततः शीघ्र\textendash
\end{sloppypar}

\newpage

\noindent ८० \hspace{4cm} सूर्यसिद्धान्तः
\vspace{1cm}

\begin{sloppypar}
\noindent फलसंस्कारेण स्पष्टज्ञानम् । तत्र स्फुटसाधितमन्दफलसंस्कृतमध्यग्रहो मन्दस्फुटः सूक्ष्म इति पूर्वं मध्यग्रहस्यासन्नस्फुटत्वसिद्ध्यर्थं फलयोः संस्कार आवश्यकस्तत्रापि प्रथमं मन्दफलं शीघ्रफलसंस्कृतान्मध्यग्रहसाधितमन्दफलापेक्षया सूक्ष्ममिति प्रथमं शीघ्रफलसंस्कृतमध्यग्रहान्मन्दफलं शीघ्रफलसंस्कृतमध्यग्रहे संस्कार्यं स्फुटासन्नो भवति~॥~४२~॥\\
\noindent ननु सूर्येन्द्वोः शीघ्रफलाभावात् कथं स्पष्टत्वं भवतीत्यतस्तदुत्तरं वदन्नैतदाद्ये कुजादीनामित्यर्थं स्फुटयति\textendash
\end{sloppypar}
%\vspace{2mm}
\begin{quote}

{\ssi मान्दं कर्मैकमर्केन्द्वोर्भौमादीनामथोच्यते~।\\
शैघ्रं मान्दं पुनर्मान्दं शैघ्रं चत्वार्यनुक्रमात्~॥~४३~॥}
%\vspace{2mm}
\end{quote}
\begin{sloppypar}
सूर्यचन्द्रयोर्मान्दं कर्मैकं तथा चानयोः शीघ्रफलाभावात् केवलेन मन्दफलेनैव स्पष्टत्वम् । एकमित्यनेन सकृन्मान्दं फलं साध्यं मध्यग्रहेणैव मन्दनीचोच्चमण्डलमध्यज्ञानान्न कर्मान्तरापेक्षेत्युपपत्तिः स्पष्टा । अथानन्तरं भौमादीनामुच्यते । प्रागुक्तं स्फुटतया कथ्यते । तदाह\textendash शैघ्र्यमिति~। प्रथमतो मध्यग्रहात् साधितशीघ्रफलं मध्यग्रहे संस्कार्यमस्मान्मन्दफलमस्यैव संस्कार्यमस्मात् पुनर्द्वितीयवारं मन्दफलं साधितं मध्यग्रहे संस्कार्यं मन्दस्पष्टो भवति । अस्मादपि शीघ्रफलं साधितमस्यैव संस्कार्यमेवमनुक्रमाच्चत्वारि कर्माणि भवन्तीति प्रागुक्ततात्पर्यम्~॥~४३~॥\\
\noindent अथात्रापि विशेषमाह\textendash
\end{sloppypar}
%\vspace{2mm}
\begin{quote}

{\ssi मध्ये शीघ्रफलस्यार्धं मान्दमर्धफलं तथा~।\\
मध्यग्रहे मन्दफलं सकलं शैघ्र्यमेव च~॥~४४~॥}
\end{quote}

\newpage


\hspace{3cm} गूढार्थप्रकाशकेन सहितः~। \hfill ८१
\vspace{1cm}

\begin{sloppypar}
मध्यग्रहे स्वसाधितशीघ्रफलस्यार्धं संस्कार्यम् । अस्मात् साधितं मन्दसम्बन्ध्यर्धफलं साधितमन्दफलस्यार्धमित्यर्थः । तथा यस्मात् साधितं तस्यैव संस्कार्यम्। शीघ्रफलार्धसंस्कृते संस्कार्यमिति फलितार्थः । अस्मात् साधितं मन्दफलं सम्पूर्णं मध्यग्रहे संस्कार्यं मन्दस्पष्टो भवति। अस्मात् साधितं शीघ्रफलं सम्पूर्णम् । चः समुच्चये । तेन मन्दस्पष्टे संस्कार्यम् । एवकारादुक्तरीत्या सिद्धो ग्रहः स्पष्टोनान्यथेति । अत्रोपपत्तिः । मन्दफलं स्फुटसाधितं वास्तवं स्फटस्तु मन्दफलसापेक्ष इत्यन्योऽन्याश्रयात् सूक्ष्ममन्दफलसाधनमशक्यमपि भगवता तदासन्नसाधनार्थमर्धस्फुटादेव मन्दफलं साधितं मध्यग्रहसाधितमन्दफलापेक्षया सूक्ष्मम्।अर्धस्फुटस्तु फलद्वयार्धसंस्कृतो मध्यग्रहः । अत्रापि मन्दफलस्यार्धं शीघ्रफलार्धसंस्कृतात् किञ्चित्सूक्ष्मत्वार्थं साधितमित्युपपन्नं मध्ये शीघ्रफलस्येत्यादि~॥~४४~॥\\
\noindent ननु फलयोः संस्कारः कथं कार्य इत्यत आह\textendash
\end{sloppypar}
%\vspace{2mm}
\begin{quote}

{\ssi अजादिकेन्द्रे सर्वेषां शैघ्रे मान्दे च कर्मणि~।\\
धनं ग्रहाणां लिप्तादि तुलादावृणमेव च~॥~४५~॥}
%\vspace{2mm}
\end{quote}
\begin{sloppypar}
सर्वेषां ग्रहाणां शैघ्रे कर्मणि मान्दे कर्मणि चकारः समुच्चये कलात्मकं फलं मेषादिषड्भान्तर्गतकेन्द्रे युतं कार्यं तुलादि षड्भान्तर्गतकेन्द्रे हीन कार्यम् । चकारो व्यवस्थार्थकः । एवकारः फलयोरानयनप्रकारभेदेऽपि धनर्णरातिभेदव्यवच्छेदार्थकः । अत्रोपपत्तिः । पूर्वाकर्षणे ग्रहस्य फलं धनं पश्चादाकर्षण ऋणमिति प्रागुक्तम् । तत्र ग्रहादुच्चपर्यन्तं केन्द्रे गृहीते
\end{sloppypar}

{\tiny{M}}

\newpage

\noindent ८२ \hspace{4cm} सूर्यसिद्धान्तः
\vspace{1cm}

\begin{sloppypar}
\noindent पूर्वाकर्षणे मेषादिकेन्द्रं भवति पश्चादाकार्षणे तुलादिकेन्द्रं भवतीति तथोक्तमुपपन्नम्~॥~४५~॥\\
\noindent अथ ग्रहाणां भुजान्तरफलमाह\textendash
\end{sloppypar}
%\vspace{2mm}
\begin{quote}

{\ssi अर्कबाहुफलाभ्यस्ता ग्रहभुक्तिर्विभाजिता~।\\
*भचक्रकलिकाभिस्त लिप्ताः कार्या ग्रहेऽर्कवत्~॥~४६~॥}
%\vspace{2mm}
\end{quote}
\begin{sloppypar}
स्पष्टा सूर्यादिग्रहगतिः सूर्यस्य भुजफलेन मन्दफलेन कलात्मकेन गुणिता द्वादशराशिकलाभिः षट्शतयुतैकविंशतिसहस्रमिताभिर्भक्ताप्राप्तफलकला ग्रहे सूर्यादिग्रहेऽर्कवत् सूर्यमन्दफलधनर्णवशादित्यर्थः । कार्याः । तुकाराद्धनर्णं संस्कार्याः ।अत्रोपपत्तिः । अहर्गणस्यैकरूपमध्यममानेन सत्त्वात् तदुत्पन्नग्रहाणां मध्यममानेन यदर्धरात्रं तात्कालिकत्वं सिद्धम् । मध्यममानार्धरात्रे तु मध्यमसूर्यमितक्रान्तिवृत्तप्रदेशोऽधो याम्योत्तरवृत्ते भवति । अस्मात् कालात् स्पष्टार्धरात्रं स्पष्टसूर्यमितक्रान्तिवृत्तप्रदेशाधोयाम्योत्तरवृत्तसंयोगरूपं मन्दफलधनर्णक्रमेणानन्तरपूर्वकाले भवति । अतो मन्दफलकलाभोगसम्बन्धिकालेन ग्रहोऽनन्तरपूर्वकालयोश्चाल्यः स्पष्टार्धरात्रसमये भवति । एतेनानेन कर्मणा स्फुटार्धरात्रकालनिग्रहाः क्रियन्ते । सूर्यश्च स्फुटार्धरात्रकालीन एवातः सूर्यस्य नायं संस्कार इति पर्वतोक्तं निरस्तम् ।सूर्यव्यतिरिक्तग्रहा मध्यार्धरात्रे सूर्यंस्तु स्फुटार्धरात्र इत्यत्राहर्गणोत्पन्नत्वेन सर्वेषामेककालिकत्वसिद्ध्या हेत्व\textendash
\end{sloppypar}

\noindent \rule{\linewidth}{.5pt}

\begin{center}
*भ्यचक्रकलिकाभिः स्युर्लिप्ताः कार्या इति वा पाठः ।
\end{center}

\newpage

\hspace{3cm}गूढार्थप्रकाशकेन सहितः~। \hfill ८३
\vspace{1cm}

\begin{sloppypar}
\noindent भावादिति । तत्र मन्दफलकलानां कालस्त्वेकराशिकलाभिः सायनस्पष्टार्काक्रान्तराश्युदयासवो लभ्यन्ते तदा मन्दफलकलाभिः क इत्यनुपातेन ततोऽहोरात्रासुभिर्गतिकलास्तदा फलकलासुभिः का इति फलकला ग्रहे धनर्णमन्दफलवशाद्धनर्णं कार्या इति सिद्धम् । तत्रापि भगवता लोकानुकम्पया स्वल्पान्तरेण नाक्षत्रदिने ग्रहगतिभोगसङ्गीकृत्य चक्रकलापरिवर्तात्मकनाक्षत्राहोरात्रेण गतिकलास्तदा सूर्यमन्दफलकलाभ्रमणेन का इत्येकानुपाताल्लाघवादानीताश्चालनकला इत्युपपन्नम्~॥~४६~॥\\
\noindent अथ स्पष्टगतिं विवक्षुश्चन्द्रस्य प्रथमं विशेषमाह\textendash
\end{sloppypar}
%\vspace{2mm}
\begin{quote}

{\ssi स्वमन्दभुक्तिसंशुद्धा मध्यभुक्तिर्निशापतेः~।\\
दोर्ज्यान्तरादिकं कृत्वा भुक्तावृणधनं भवेत्~॥~४७~॥}
%\vspace{2mm}
\end{quote}
\begin{sloppypar}
ग्रहगतिसाधने वक्ष्यमाणे गतिफलं ग्रहगतेः साधितं तथा चन्द्रगतेश्चन्द्रगतिफलं न साध्यं किन्तु चन्द्रस्य मध्यमगतिः स्वस्य चन्द्रस्य मन्दं मन्दोच्चं तस्य दिनगत्या हीना कार्या तादृशगतेः सकाशाद्दोर्ज्यान्तरादिकं दोर्ज्यान्तरमादिभूतं यस्यैतादृशं गतिफलं वक्ष्यमाणप्रकारे दोर्ज्यान्तरगुणा भुक्तिरित्यादौ दोर्ज्यान्तरादेव गतिफलोत्पत्तेः । सिद्धं कृत्वा चन्द्रमध्यमगतावृणधनं वक्ष्यमाणरीत्या भवति । अत्रोपपत्तिः । वक्ष्यमाणं गतिफलं केन्द्रगत्योपपन्नमित्यनेन सूर्यादिग्रहाणां विचन्द्राणां मन्दोच्चगतेरत्यल्पत्वात् स्वगत्यैव गतिफलमुक्तम् । तत्र चन्द्रस्य तथा
\end{sloppypar}

{\tiny{M 2}}

\newpage

\noindent ८४ \hspace{4cm} सूर्यसिद्धान्तः
\vspace{1cm}

\begin{sloppypar}
\noindent साधने बह्वन्तरपातात् तस्य मन्दोच्चगत्यूनखगप्तिरूपकेन्द्रगतेः फलं साधितं गतिफलं यद्गतेः साध्यं तद्गतावेव संस्कार्यमिति वक्ष्यमाणरीतिष्युदासाय चन्द्रभुक्तावित्युक्तमन्यथा केन्द्रगतेरेव स्फुटत्वं स्यान्न चन्द्रगतेरिति~॥~४७~॥\\
\noindent अथ ग्रहाणां मन्दस्पष्टगतिं वासनासूचनपूर्वगतिफलानयनपूर्विकां श्लोकाभ्यामाह\textendash
\end{sloppypar}
%\vspace{2mm}
\begin{quote}

{\ssi ग्रहभुक्तेः फलं कार्यं ग्रहवन्मन्दकर्मणि~।\\
दोर्ज्यान्तरगुणा भुक्तिस्तत्त्वनेत्रोद् ता पुनः~॥~४८~॥

स्वमन्दपरिधिक्षुणा भगणांशोद्धृताः कलाः~।\\
कर्कादौ तु धनं तत्र मकरादावृणं स्मृतम्~॥~४९~॥}
%\vspace{2mm}
\end{quote}
\begin{sloppypar}
मन्दकर्मणि गतिमन्दफलक्रियानिमित्तमित्यर्थः । ग्रहवद्ग्रहमन्दफलानयनरीत्या परिधिगुणनभगणांशभजनाप्तचापमित्यात्मिकया ग्रहगतेः सकाशात् फलं ग्रहमन्दगतिफलं साध्यम् । यथा ग्रहमन्दफलं केन्द्रभुजज्यातः साधितं तथेदं गतिफलं ग्रहगतेः साध्यमित्यर्थः । तथाहि ग्रहमन्दफलान्तरस्यैकदिनान्तरीयस्य ग्रहगतिमन्दफलत्वाद्भुजज्ययोरेकदिनान्तरयो रन्तरात् फलं मन्दगतिफलं पर्यवसितं तत्र केन्द्रयोरन्तरस्य केन्द्रगतित्वात् तज्जयोरन्तरं तत्त्वाश्विप्रमाणेनोक्तज्यापिण्डान्तरं गतिकलापरिणामितं भवति । तदेवाह\textendash दोर्ज्यान्तरगुणेति~। ग्रहमध्यगतिः केन्द्रगतिरूपा । उच्चगतेरत्यल्पत्वात् । दोर्ज्यान्तरगणा भुजज्यानयनावसरे यज्ज्यापिण्डान्तरं तेन गुणिता
\end{sloppypar}

\newpage



\hspace{3cm} गूढार्थप्रकाशकेन सहितः~। \hfill ८५
\vspace{1cm}

\begin{sloppypar}
\noindent पञ्चाकृतिभिर्भक्ता पुनरनन्तरमित्यर्थः । ग्रहमन्दपरिधिना स्फुटेन गुणिता षष्टियुतशतत्रयेण भक्ता फलं गतिमन्दफलकलाः । यद्यपि गतिज्यातः फलज्यानयनं कृत्वा तच्चापं गतिफलं समुचितम् । तथापि ग्रहगतेस्तत्त्वाश्विभ्यो न्यूनत्वाज्ज्याचापयोस्तुल्यत्वेन तदनुक्तावक्षतिः । चन्द्रस्य तु स्वल्पान्तरात् तत्करणमुपेक्षितम् । मन्दस्पष्टगतिसिद्ध्यर्थं मध्यगतौ फलसंस्कारमाह\textendash कर्कादाविति~। तत्र ग्रहमध्यगतौ पूर्वानीतफलं कर्कादिषड्भान्तर्गतकेन्द्रे धनं मकरादिषड्भान्तर्गतकेन्द्र ऋणमुक्तम् । तुकारान्मन्दस्पष्टगतिः सिद्धा भवतीत्यर्थः । अत्रोपपत्तिः । ऋणफलोपचये पूर्वफलादग्रिमफलमधिकं हीनमिति फलान्तरं गतावृणम् । ऋणफलापचये पूर्वफलादग्रिमफलं न्यूनं युतमिति फलान्तरं गतौ धनम् । धनफलोपचये पूर्वफलादग्रिमफलमधिकं युतमिति फलान्तरं गतौ धनम् । ऋणफलापचयस्तु मकरादितः प्राक् त्रिभे । धनफलोपचयस्तु तुलादितः प्राक् त्रिभ इति कर्कादिकेन्द्रे गतिफलं धनम् । धनफलापचये पूर्वलादग्रिमं फलं न्यूनं हीनमिति फलान्तरं गतावृणम् । धनफलापचयस्तु कर्कादितः प्राक् त्रिभ ऋणफलोपचयस्तु मेषादितः प्राक् त्रिभ इति मकरादिकेन्द्रे गतिफलमृणं सिद्धम्~॥~४९~॥\\
 अथ श्लोकाभ्यां स्पष्टगतिसाधनमाह\textendash
\end{sloppypar}
%\vspace{2mm}
\begin{quote}

{\ssi मन्दस्फुटीकृतां भक्तिं प्रोज्झ्य शीघ्रोच्चभुक्तितः~।\\
तच्छेषं विवरेणाथ हन्यात् त्रिज्यान्त्यकर्णयोः॥५०}
\end{quote}


\newpage

\noindent ८६ \hspace{4cm} सूर्यसिद्धान्तः
\vspace{1cm}
\begin{quote}

{\ssi चलकर्णहृतं भुक्तौ कर्णे त्रिज्याधिके धनम्~।\\
ऋणमूनेऽधिके प्रोज्झ्य भुक्तिर्वक्रगतिर्भवेत्~॥~५१~॥}
%\vspace{2mm}
\end{quote}
\begin{sloppypar}
मन्दस्पष्टां गतिं प्राक् सिद्धां शीघ्रोच्चगतेः पातयित्वा तत्रावशिष्टं त्रिज्यान्त्यकर्णयोस्त्रिराशिज्याद्वितीयशीघ्रकर्णयोर्ग्रन्थान्तरैकवाक्यतार्थं त्रिज्याशब्देन द्वितीयशीघ्रफलकोटिज्या ग्राह्येति ध्येयम्। अन्तरेण गुणयेत्। तत्र यत् सिद्धं तच्छीघ्रकर्णेन द्वितीयेन भक्तं फलं मन्दस्पष्टगतौ द्वितीयशीघ्रकर्णे त्रिज्याधिके गृहीतफलकोटिज्यातोऽधिके सति हीने च सति धनमृणं क्रमेण कार्यं स्पष्टगतिः स्यात् । ननु यदा मन्दस्पष्टगतितो गतिशीघ्रफलमधिकं तदा मन्दस्पष्टगतौ फलमूनं न स्यादिति तत्र स्पष्टगतिज्ञानं कथम् । न चैतदसम्भव इति वाच्यम् । नीचासन्ने ग्रहे फलकोटिज्या शीघ्रकर्णान्तराच्छीघ्रकर्णस्य न्यूनत्वात् फलस्यावश्यं मन्दस्पष्टगत्यधिकत्वसम्भवादित्यत आह\textendash अधिक इति~। मन्दस्पष्टगतिः । अधिके फले पातयित्वा शेषं वक्रगतिर्विपरीतगतिः पश्चिमगतिः स्यात् । तथा च न क्षतिः । अत्रोपपत्तिः\textendash
\end{sloppypar}
%\vspace{2mm}
\begin{quote}

{\qt फलांशखाङ्कान्तरशिञ्जिनीघ्नी\\
द्राक्केन्द्रभुक्तिः श्रुतिहृद्विशोध्या~।\\
स्वशीघ्रभुक्तेः स्फुटखेटभुक्तिः\\
शेषं च वक्रा विपरीतशुद्धौ.~॥}
%\vspace{2mm}
\end{quote}
\begin{sloppypar}
इति सिद्धान्तशिरोमणौ वृद्धवसिष्ठसिद्धान्तोक्तेः सूक्ष्मप्रकारस्तस्योपपत्तिस्तु तट्टीकायां व्यक्ता । तत्र द्राक्केन्द्रभुक्त्यर्थं प्रथमार्धमुक्तम । इयं गतिः फलकोटिज्यया गुण्या कर्णभक्ता
\end{sloppypar}


\newpage




\hspace{3cm}गूढार्थप्रकाशकेन सहितः~। \hfill ८७
\vspace{1cm}

\begin{sloppypar}
\noindent फलं स्वशीघ्रोच्चगतेः शोध्यम् । तत्र प्रथममेव समच्छेदपूर्वकशोधनार्थं शीघ्रोच्चगतेः कर्णो गुणः । तत्रापि शीघ्रोच्चगतेः केन्द्रग्रहगतियोगरूपत्वात् खण्डद्वयं केन्द्रगतावेव फलं हीनं कृतमिति कर्णगुणितकेन्द्रगतिफलकोटिज्यागुणितकेन्द्रगत्योरन्तरं तत्रापि गुणितयोरन्तरेंऽन्तरे वा गुणिते समत्वाल्लाघवाच्च फलकोटिज्याकर्णान्तरेण केन्द्रगतिर्गुणिता कर्णभक्तेति तच्छेषमित्यादिहृतमित्यन्तमुपपन्नम् । अथ फलकोटिज्यातुल्यकर्णे मुख्यप्रकारेण गतेर्मन्दस्पष्टगतितुल्यतया सिद्धत्वात् फलाभावः कर्णस्य न्यूनत्वे फलस्य शीघ्रकेन्द्रगत्यधिकत्वात् तदूने शीघ्रोच्चगतौ शीघ्रकेन्द्रगतिनाशादधिकस्य गतिफलरूपस्य मन्दस्पष्टगतौ हीनत्वं पर्यवसन्नम् । कर्णस्याधिकत्वे पूर्वप्रकारफलस्य शीघ्रकेन्द्रगतितो न्यूनत्वात् तदूने शीघ्रोच्चगतौ यन्नूनं तदधिका मन्दस्पष्टगतिः स्पष्टगतिरिति पर्यवसन्नम् । तदत्र शीघ्रोच्चगतिस्थाने शीघ्रकेन्द्रगतिग्रहणेन फलं गतिफलमेवोत्पन्नं तन्मन्दस्पष्टगतौ फलकोटिज्यातः कर्णस्याधिकन्यूनत्वक्रमेण धनमृणमित्युपपन्नं कर्ण इत्याद्यून इत्यन्तम् । ऋणफलस्य मन्दस्पष्टगतितोऽधिकत्वे विपरीतशोधनाच्छेषं पश्चिमगतिरेष स्पष्टेति सर्वमनवद्यम्~॥~५१~॥\\
\noindent अथ वक्रगत्युपपत्तिमाह\textendash
\end{sloppypar}
%\vspace{2mm}
\begin{quote}
{\ssi दूरस्थितः स्वशीघ्रोच्चाद्ग्रहः शिथिलरश्मिभिः~।\\
सव्येतराकृष्टतनुर्भवेद्वक्रगतिस्तदा~॥~५२~॥}
\end{quote}
{\setlength{\parindent}{5em}
स्वशीघ्रोच्चाद्दूरस्थितस्त्रिभाधिकान्तरितो ग्रहो भौमादिकः}


\newpage


\hspace{3cm}सूर्यसिद्धान्तः
\vspace{1cm}

\begin{sloppypar}
\noindent शिथिलरश्मिभिः शीघ्रोच्चदेवताहस्तस्थितग्रहबिम्बप्रोतरज्जुभिः भव्येतराकृष्टतनुर्देवतायाः सव्यवामभाग आकर्षिता तनुः शरीरं बिम्बरूपं यस्यासौ यदा तदा वक्रगतिः स्यात् । अयं भावः । त्रिभादूनान्तरितो ग्रहो वृत्ताकारसूत्रैरशिथिलैर्दैवतैर्यथाकर्षितुं शक्यते तथा त्रिभाधिकान्तरितो ग्रहो दैवतैर्वृत्ताकारसूत्रैः शिथिलैराकर्षितुं न शक्यतेऽतोऽल्पधनर्णफलस्थाने ग्रहो वक्रीभवति । आकर्षणोत्कर्षाभावेन वृत्तमार्गे वस्तुनो नीचगामित्वसम्भवादिति~॥~५२~॥\\
\noindent अथ यत्केद्रांशेषु गतिफलमृणं मन्दस्पष्टगतितुल्यं भवति तान् वक्रारम्भभागांस्तदन्तभागांश्च विना गतिसाधनप्रकारं ग्रहवक्रतदन्तज्ञानार्थं श्लोकाभ्यामाह\textendash
\end{sloppypar}
%\vspace{2mm}
\begin{quote}

{\ssi कृतर्तुचन्द्रैर्वेदेन्द्रैः शून्यत्र्येकैर्गुणाष्टिभिः~।\\
शररुद्रैश्चतुर्थेषु केन्द्रांशैर्भूसुतादयः~॥~५३~॥

भवन्ति वक्रिणस्तैस्तु स्वैः स्वैश्चक्राद्विशोधितैः~।\\
अवशिष्टांशतुल्यैः स्वैः केन्द्रैरुज्झन्ति वक्रताम्~॥~५४~॥}
%\vspace{2mm}
\end{quote}
\begin{sloppypar}
भौमाद्या ग्रहाश्चतुर्थकर्मसु केन्द्रांशैः शीघ्रकेन्द्रांशैः कृतर्तुचन्द्रैरित्याद्युक्तरूपैः क्रमेण वक्रिणो भवन्ति । स्वकीयैः स्वकीयैस्तैः केन्द्रांशैरुक्ततुल्यैश्चक्राद्द्वादशराशिभागेभ्यः षष्टियुतशतच्चयेभ्यो विशोधितैर्हीनैरवशेषवसमानैः स्वकीयैश्चतुर्थकेन्द्राशैः । तुकारः क्रमार्थे । भौमादयो वक्रत्वं त्यजन्ति । परिवर्ते वारद्वयं भुजतुल्यत्वेन नीचासन्ने मन्दस्पष्टगतितुल्यगतिफलस्य सम्भ\textendash
\end{sloppypar}



\newpage


\hspace{3cm}गूढार्थप्रकाशकेन सहितः~। \hfill ८९
\vspace{1cm}

\begin{sloppypar}
\noindent वादिति~॥~५४~॥\\
\noindent अथ वक्रान्तभागानामतुल्यत्वे कारणान्तरमप्याह\textendash
\end{sloppypar}
%\vspace{2mm}
\begin{quote}

{\ssi महत्त्वाच्छीघ्रपरिधेः सप्तमे मृगुभूसुतौ~।\\
अष्टमे जीवशशिजौ नवमे तु शनैश्चरः~॥~५५~॥}
%\vspace{2mm}
\end{quote}
\begin{sloppypar}
शीघ्रकेन्द्रस्य सप्तमे राशौ शुक्रभौमौ वक्रत्वं त्यजतः । अष्टमे राशौ गुरुबुधौ वक्रत्यजनार्हौ । अत्र शुक्रगुर्वोः पूर्वोद्देश इतरापेक्षयाभ्यर्हितत्वज्ञापकः । नवमे राशौ शनिर्वक्रत्वं त्यजति । तुरेवार्थे । तेन शनिरेव तत्र वक्रत्वं त्यजति नान्ये । अत्र कारणमाह\textendash महत्त्वादिति~। अन्येषां शीघ्रपरिधेः प्रागुक्तस्य महत्त्वाच्छनिशीघ्रपरिधेरधिकत्वात् । तथा च परिध्यधिकत्वेन पूर्वमेव वक्रत्यजनमत एव भौमशुक्रयोर्बुधगुरुभ्यां प्रथमोद्देशः । शनेस्तु सुतरां बुधगुर्वोः शनितः पूर्वोद्देशः । भृगुभूसुतौ जीवशशिजावित्यत्र परिध्यधिकत्वेन शुक्रगुर्वोः प्रथमं केवलमुद्देशो न भागानामल्पत्वक्रम इति भावः । ननु परिध्यधिकत्वे पूर्वपूर्वराशौ वक्रत्यजने कोपपत्तिरिति चेच्छृणु । शून्यगतिसम्बद्धशीघ्रकर्णात् फलांशखाङ्कान्तरेत्यादेर्विलोमविधिना शीघ्रोच्चगतेः फलकोटिज्यास्याः फलज्यास्यास्त्रिज्याभ्यस्तं भुजफलं चलकर्णविभाजितमित्यस्य विलोमविधिना भुजफलमस्मात् तद्गुणे भुजकोटिज्ये भगणांशविभाजिते इत्यस्य विलोमप्रकारेण भुजांशज्ञानार्थं भौमादीनां भुजज्या उत्तरोत्तरमधिकाः शीघ्रपरिधिभ्यो यथोत्तरमपचयवद्भ्यो हरेभ्यो लब्धत्वाद्धराधिकन्यूनत्वाभ्यां फलयोर्न्यूनाधिकत्व\textendash
\end{sloppypar}

{\tiny{N}}
\newpage


\hspace{4cm} सूर्यसिद्धान्तः
\vspace{1cm}

\begin{sloppypar}
\noindent निश्चयात् तासां चापानि भुजभागा यथोत्तरमधिका वक्रारम्भे तदन्ते च तुल्या अत एव तृतीयपदे वक्रान्तत्वाद्भुजभागाः षड्युता यथोत्तरमधिकं शीघ्रकेन्द्रं तेषां वक्रान्ते भवति । वक्रारम्भस्य द्वितीयपदे सम्भवाद्भुजभागहीनाः षड्राशयस्तेषां वक्रारम्भे यथापचितं केन्द्रं भवति । तत् तूक्तरीत्या भौमशुक्रयोः षष्ठराशौ बुधगुर्वोः पञ्चमराशौ शनेश्चतुर्थराशाविति ज्ञेयम् । इदं भगवता विना चक्रशोधनमापाततः शीघ्रकेन्द्रराशिज्ञानाद्वक्रान्तज्ञानं लोकानुकम्पार्थमनतिप्रयोजनमुक्तमिति ध्येयम्~॥~५५~॥\\
\noindent अथ चन्द्रादिग्रहाणां विक्षेपसाधनं श्लोकाभ्यामाह\textendash
\end{sloppypar}
%\vspace{2mm}
\begin{quote}

{\ssi कुजार्किगुरुपातानां ग्रहवच्छीघ्रजं फलम्~।\\
वामं तृतीयकं मान्दं बुधभार्गवयोः फलम्~॥~५६~॥

स्वपातोनाद्ग्रहाज्जीवा शीघ्राद् भृगुजसौम्ययोः~।\\
विक्षेपघ्नान्त्यकर्णाप्ता विक्षेपस्त्रिज्यया विधोः~॥~५७~॥}
%\vspace{2mm}
\end{quote}
\begin{sloppypar}
भौमशनिगुरूणां ये पाता मध्याधिकारावगतास्तेषां शीघ्रजं फलं स्वग्रहसम्बन्धिचतुर्थकर्मस्यशीघ्रफलं पूर्वसिद्धं ग्रहवद्ग्रहे यथा संस्कृतं तथा संस्कार्यम् । ग्रहशीघ्रफलं ग्रहे चेद्युतं तदा तत्पाते तदेव फलं योज्यं चेद्धीनं तदा हीनं कार्यमित्यर्थः । बुधशुक्रयोस्तृतीयकं तृतीयकर्मसम्बन्धि मान्दं फलं तत्पातयोर्विपरीतं संस्कार्यं बुधशुक्रयोर्मन्दफलं धनमृणं चेत् तत्पातयोस्तदेव फलमृणधनं क्रमेण कार्यमित्यर्थः । अनुक्तत्वाच्चन्द्रस्य यथागत एव पातो ज्ञेयः । स्पष्टग्रहात् स्वस्य फलसंस्कृतो यः पातस्तेन ही\textendash
\end{sloppypar}


\newpage


\hspace{3cm} गूढार्थप्रकाशकेन सहितः~। \hfill ९१
\vspace{1cm}

\begin{sloppypar}
\noindent नाद्भुजज्या । बुधशुक्रयोर्विशषमाह\textendash शीघ्रादिति~। शुक्रबुधयोः शीघ्रोच्चात् पातेन हीनाद् भुजज्या न पातोनबुधशुक्राभ्यां भुजज्या । विशेषस्य सामान्यबाधकत्वात । अर्थात् पूर्वोक्तं चन्द्रभौमगुरुशनीनां सिद्धम् । मध्याधिकारोक्तस्वमध्यमविक्षेपकलाभिर्गुण्या चतुर्थकर्मणि यः शीघ्रकर्णस्तेन भक्ता फलं ग्रहाणां विक्षेपकलाः स्फुटा भवन्ति । ननु चन्द्रस्य शीघ्रकर्णासम्भवात् तत्पातोनतद्भुजज्या खभगुणिता केन भाज्येत्यत आह\textendash त्रिज्ययेति~। चन्द्रस्य विक्षेपसाधने तादृशी भुजज्या त्रिज्यया भाज्येत्यर्थः । अत्रोपपत्तिः । यथा विषुवद्वृत्तात् क्रान्तिवृत्तयाम्योत्तरभागौ यदन्तरेण याम्योत्तरसूत्रे सा ध्रुवाभिमुखी क्रान्तिस्तथा क्रान्तिवृत्ताद्विक्षेपवृत्तभागौ यदन्तरेण याम्योत्तरसूत्रे स विक्षेपः कदस्वाभिमुखः । तथाहि । विक्षेपवृत्तानि ग्रहबिम्बाधिष्ठितानि सूर्यव्यतिरिक्तग्रहाणां षण्णां स्वस्वगोले भिन्नानि सूर्यस्य नित्यं क्रान्तिवृत्तस्यत्वमेव तानि क्रान्तिवृत्ते स्वस्वगत्या प्रोतान्येव गच्छन्ति । तत्र विक्षेपक्रान्तिवृत्तसम्पाते पातस्थाने तत्षड्भान्तरप्रदेशे च स्थिते ग्रहबिम्बे वृत्तप्रदेशैक्यादन्तराभावेन ग्रहविक्षेपाभावः । यथा तस्माद्ग्रहबिम्बं गच्छति तथा ग्रहबिम्बक्रान्तिवृत्तस्यचिह्नयोर्याम्यमुत्तरं वान्तरं क्रान्तिवृत्ताद्ग्रहस्य भवति तदेव विक्षेपसञ्ज्ञम् । स च पातात् त्रिभान्तरे ग्रहे मध्याधिकारोक्तः । अन्तराले पातस्थानाद्ग्रहचिह्नं क्रान्तिवृत्ते यदन्तरेण तदन्तरं राश्याद्यात्मकं पातोनग्रहरूपं तद्भुजज्ययानुपातः । त्रिज्याभुजज्यया परमविक्षेपस्तदेष्टया भुजज्यया क इति । एवं चन्द्रस्यैव
\end{sloppypar}

{\tiny{N 2}}
\newpage

\noindent ९२ \hspace{4cm} सूर्यसिद्धान्तः
\vspace{1cm}

\begin{sloppypar}
\noindent त्रिज्याव्यासार्धगोले परमशरस्य गणितागतपातस्य च लक्षितत्वात् । अन्येषां तु परमशराः शीघ्रोच्चदेवताकृष्टग्रहबिम्बाधिष्ठितकल्पितवृत्ते शीघ्रकर्णव्यासार्धे लक्षिताः । कथमन्यथा शीघ्रफलसंस्कारेण ग्रहस्य स्पष्टत्वं युक्तम् । ग्रहबिम्बस्य तत्स्थत्वे तत्पातस्यापि तत्स्थत्वं युक्तम् । ग्रहबिम्बाधिष्ठितवृत्ते ग्रहभोगस्य मन्दस्पष्टत्वेन गणितागतपातान्मन्दस्पष्टाच्छरसाधनमुपपन्नम् । तदुक्तं सिद्धान्तशिरोमणौ\textendash
\end{sloppypar}
%\vspace{2mm}
\begin{quote}

{\qt मन्दस्फुटो द्राक्प्रतिमण्डले हि\\
ग्रहो भ्रमत्यत्र च तस्य पातः~।\\
पातेन युक्ताद्दणितागतेन\\
मन्दस्फुटात् खेचरतः शरोऽस्मात्~॥}
%\vspace{2mm}
\end{quote}
\begin{sloppypar}
इति । तत्र स्पष्टाच्छरसाधनार्थं शीघ्रफलं पाते संस्कृतं शीघ्रफलव्यस्तसंस्कृतस्पष्टग्रहस्य मन्दस्पष्टत्वाद्यथोक्तसंस्कृतपातोने स्पष्टग्रहे पातोनमन्दस्फुटग्रहस्य सिद्धेः । अथ बुधशुक्रपातभगणौ वास्तवौ नोक्तौ। तौ तु शीघ्रकेन्द्रभगणाधिकावतो गणितागतपातयोर्मध्यग्रहोनशीघ्रोच्चरूपशीघ्रकेन्द्रयुतयोर्द्वादशराशिशुद्धयोः पातत्वम् । तत्र पूर्वपातस्य द्वादशशुद्धत्वाच्छीघ्रकेन्द्रं चक्रशुद्धं योज्यमतो लाघवाद्गणितागतपातस्य शीघ्रोच्चोनमध्यग्रहरूपं केन्द्रं योज्यमयं पातो मन्दस्पष्टे मन्दफलसंस्कृतमध्यरूपे हीन इति ग्रहयोर्मध्ययोर्नाशाद्यथागतमन्दफलसंस्कृतं शीघ्रोच्चं पातोनमिति सिद्धम् । तत्रापि मन्दफलं पाते व्यस्तं कृत्वा तदूनं शीघ्रोच्चं कृतं संस्कृतपातपङ्कयां संस्कृतपातयोर्युक्तत्वात । अथैतदानीतविक्षेपः
\end{sloppypar}


\newpage


\hspace{3cm}गूढार्थप्रकाशकेन सहितः~। \hfill ९३
\vspace{1cm}

\begin{sloppypar}
\noindent कर्णव्यासार्धवृत्ते न त्रिज्यावृत्ते स्फुटग्रहस्थान अतः कर्णाग्रेऽयं पूर्वानुपातानीतविक्षेपस्तदा त्रिज्याग्रे क इत्यनुपातेन त्रिज्यागुणः कर्णो हरः पूर्वं त्रिज्याहर इति त्रिज्ययोर्नाशाद् भुजज्या परमविक्षेपगुणिता शीघ्रकर्णभक्तेति सर्वमुक्तमुपपन्नम्~॥~५७~॥\\
\noindent अथ दिनरात्रिमानज्ञानार्थं चरानयनं विवक्षुः प्रथमं तदुपयुक्तां स्पष्टक्रान्तिमाह\textendash
\end{sloppypar}
%\vspace{2mm}
\begin{quote}

{\ssi विक्षेपापक्रमैकत्वे क्रान्तिर्विक्षेपसंयुता~।\\
दिग्भेदे वियुता स्पष्टा भास्करस्य यथागता~॥~५८~॥}
%\vspace{2mm}
\end{quote}
\begin{sloppypar}
यस्य ग्रहस्य स्पष्टक्रान्तिरभीष्टा तस्य ग्रहस्यायनांशसंस्कृतस्य भुजज्यातः परमापक्रमज्येत्यादिना क्रान्तिरयनांशसंस्कृतग्रहगोलदिक्का ज्ञेया । तस्य विक्षेपोऽपि पूर्वोक्तप्रकारेण पातोनगोलदिक्को ज्ञेयः । गोलस्तु मेषादिषट्कमुत्तरस्तुलादिषट्कं दक्षिणः । अथ शरक्रान्त्योरेकदिक्त्वेन क्रान्तिः कलाद्या कलात्मकविक्षेपेण युता तयोर्दिगन्यत्वे क्रान्तिर्विक्षेपेण वियुतान्तरिता शेषदिक्का स्पष्टा क्रान्तिः स्यात् । ननु सूर्यस्य विक्षेपभावात् कथं स्पष्टा क्रान्तिर्ज्ञेयेत्यत आह\textendash भास्करस्येति~। सूर्यस्य यथा गता पूर्वागता क्रान्तिरेव स्पष्टा क्रान्तिः । अत्रोपपत्तिः । विषुवदृत्ताद्ग्रहबिम्बकेन्द्रपर्यन्तं याम्यमुत्तरं वान्तरं स्पष्टक्रान्तिरिति तयोरेकदिक्त्वे तद्योगतुल्यमन्तरं भिन्नदिक्त्वे तदन्तरमितमन्तरमिति। अत्र शरस्य क्रान्तिसंस्कारयोग्यत्वमन्यादिका क्रियालोकश्रमभयात् स्वल्पान्तरत्वाच्चोपेक्षिता भगवता कृपावता ।
\end{sloppypar}


\newpage


\noindent ९४ \hspace{4cm} सूर्यसिद्धान्तः
\vspace{1cm}

\begin{sloppypar}
\noindent अन्यथा शरस्य ध्रुवाभिमुखत्वे भगवदुक्तमायनदृक्कर्म कथमव्याहतं स्यादित्यलम्~॥~५८~॥\\
\noindent अथ दिनरात्रिमानज्ञानार्थमहोरात्रासून् साधयति\textendash
\end{sloppypar}
%\vspace{2mm}
\begin{quote}

{\ssi ग्रहोदयप्राणहता खखाष्टैकोडृता गतिः~।\\
चक्रासवो लब्धयुताः स्वाहोरात्रासवः स्मृताः~॥~५९~॥}
%\vspace{2mm}
\end{quote}
\begin{sloppypar}
ग्रहस्य येऽनांशसंंस्कृतराशेर्वक्ष्यमाणनिरक्षोदयासवस्तैर्गुणिता निजस्फुटगतिः कलाद्याष्टादशशतभक्ता फलेन युताश्चक्रासवः षष्टिघटिकानामसवः षट्शतयुतैकविंशतिसहस्रमिताः स्वस्वग्रहस्याहोरात्रासवः कालतत्त्वज्ञैः कथिताः । अत्रोपपत्तिः । ग्रहः पूर्वगत्या लम्बितः प्रवहेण गतिभोगकालेन भचक्रपरिवर्तानन्तरमुदेत्यतो भचक्रपरिवर्तकालः षष्टिघटिकासुमितो ग्रहगतिकलासम्बद्धास्वात्मककालेनाधिको ग्रहाहोरात्रमस्वात्मकं ना क्षत्रप्रमाणेन भवति । तत्रैकराशिकलाभिर्ग्रहसम्बद्धराश्युदयप्राणास्तदा गतिकलाभिः क इत्यनुपातेन गत्यसव इत्युपपन्नं ग्रहोदयेत्यादि । अनेनैव श्लोकेन ग्रहाणामुदयान्तरकर्मास्तीत्युक्तं भगवता । तथाहि । अनुपातानीतमध्यग्रहाणां नियताहोरात्रमानान्तरकाले सिद्धत्वान्न मध्यरात्रकाले ग्रहाणां सिद्धिः । रविमध्यगत्यसूनां प्रतिराशौ भिन्नत्वेन मध्यमसूर्याहोरात्रमानस्य नियतत्वाभावादतस्त्रैराशिकावगतग्रहा अनियतमध्यार्काहोरात्रमानान्तरेणार्धरात्रे यत्संस्कारेण भवन्ति तदेवोदयान्तरं तत्साधनं भगवता स्वल्पान्तरत्वादुपेक्षितम् । कथमन्यथा गत\textendash
\end{sloppypar}

\newpage

\hspace{3cm}गूढार्थप्रकाशकेन सहितः~। \hfill ९५
\vspace{1cm}

\begin{sloppypar}
\noindent कलासूनां समत्वमुपेक्ष्य गतिकलानामसवो भगवदुक्ताः सङ्गच्छन्ते । उदयान्तरस्य गतिकलासुभेदोत्पन्नत्वात्~॥~५९~॥\\
\noindent अथ चरोपयुक्तां क्रान्तिज्यां द्युज्यां चाह\textendash
\end{sloppypar}
%\vspace{2mm}
\begin{quote}

{\ssi क्रान्तो क्रमोत्कमज्ये द्वे कृत्वा तत्रोत्कमज्यया~।\\
हीना त्रिज्या दिनव्यासदलं तद्दक्षिणोत्तरम्~॥~६०~॥}
%\vspace{2mm}
\end{quote}
\begin{sloppypar}
स्पष्टक्रान्तेः क्रमोत्क्रमज्येक्रमज्योत्क्रमज्ये द्वे अपि प्रसाध्य तत्र
तन्मध्ये क्रान्त्युत्क्रमज्यया त्रिज्या हीना दिनव्यासदलमहोरात्रवृत्तस्य व्यासार्धं द्-युज्येत्यर्थः । तद्दिनव्यासार्धं दक्षिणोत्तरं दक्षिणगोल उत्तरगोले च स्यात् । क्रान्तेर्गोलद्वयेऽपि सत्त्वात् । अपरा क्रान्तिज्यैव । अत्रोपपत्तिः । क्रान्त्यंशानां क्रमज्या क्रान्तिज्या भुजो विषुवदृत्तानुकाराण्यहोरात्रवृत्तान्युभयगोले तदुभयतस्तद्व्यासार्धं द्-युज्या कोटिस्त्रिज्या कर्ण इति गोले प्रत्यक्षम्। त्रिज्यावृत्त उन्मण्डले याम्योत्तरवृत्ते वा प्रत्यक्षम् । तत्र भुजकर्णयोर्वर्गान्तरपदं कोटिरिति क्रान्तिज्यावर्गोनात्त्रिज्यावर्गान्मूलं द्युज्या । तत्रापि भुजोत्क्रमज्यया हीना त्रिज्या कोटिक्रमज्या स्यादिति वृत्ते प्रत्यक्षदर्शनात् क्रान्त्युत्क्रमज्ययोना त्रिज्या द्युज्या स्यादिति लाघवेन वर्गमूलनिरासेनोक्तं भगवता क्रान्तेरित्यादि~॥~६०~॥\\
\noindent अथ चरानयनपूर्वकदिनरात्रिमानसाधनं श्लोकत्रयेणाह\textendash
\end{sloppypar}
%\vspace{2mm}

\begin{quote}
{\ssi क्रान्तिज्या विषुवद्भाघ्नी क्षितिज्या द्वादशोद्धृता~।\\
त्रिज्यागुणाहोरात्रार्धकर्णाप्ता चरजासवः~॥~६१~॥}
\end{quote}
\newpage


\noindent ९६ \hspace{4cm} सूर्यसिद्धान्तः 
\vspace{1cm}
\begin{quote}

 {\ssi तत्कार्मुकमुदक्क्रान्तौ धनहानी पृथक्स्थिते~।\\ 
स्वाहोरात्रचतुर्भागे दिनरात्रिदले स्मृते ~॥~६२~॥

याम्यक्रान्तौ विपर्यस्ते द्विगुणे तु दिनक्षपे~।\\
विक्षेपयुक्तोनितया क्रान्त्या भानामपि स्वके~॥~६३~॥}
%\vspace{3mm}
\end{quote}
\begin{sloppypar}
 क्रान्तिज्या विषुवद्दिनीयमध्यान्हे द्वादशाङ्गुलशङ्कोश्छायया गुण्या द्वादशभक्ता फलं कुज्या स्यात्। सा त्रिज्यया गुणिताहोरात्रार्धकर्णाप्ताहोरात्रवृत्तस्यार्धकर्णेन व्यासदलेन द्युज्यया भक्ताफलं चरजा ज्या चरज्येत्यर्थः। तस्याश्चरज्याया धनुरसवश्चरासवो भवन्ति। स्वाहोरात्रचतुर्भागे स्वस्य चरसम्बन्धिनो ग्रहस्य प्रागुक्ताहोरात्रासवस्तेषां चतुर्थांशे पृथक्स्थिते स्थानद्वयस्थे उत्तरक्रान्तौ सत्यां चरासू धनहानी युतहीनौ कार्यौ तौ क्रमेण दिनरात्रिदले दिनार्धरात्र्यर्धे कालविद्भिरुक्ते। दक्षिणक्रान्तौ सत्यां विपर्यस्ते दिनरात्रिदले यत्र हीनं तद्दिनार्धं यत्र युतं तद्रात्र्यर्धमित्यर्थः। तुकारात् ते दिनरात्र्यर्धे द्विगुणे दिनक्षपे दिनमानरात्रिमाने ग्रहस्य स्तः। उक्तरीत्या नक्षत्राणामपि दिनरात्रिमाने साध्ये इत्याह। विक्षेपेत्यादि। नक्षत्रध्रुवाणमानीतया क्रान्त्या नक्षत्रविक्षेपेणैकभिन्ना दिक्क्रमेण युक्तयान्तरितयोक्तप्रकारेण सिद्धया स्वके नक्षत्रदिनरात्रिमाने साध्ये इत्यर्थः। अत्रोपपत्तिः। द्वादशाङ्गुलशङ्कुः कोटि: पलभा भुजोऽक्षकर्णः कर्णः क्रान्तिज्या कोटि: कुज्या भुजोऽग्राकर्ण इत्यक्षक्षेत्रद्वयं प्रसिद्धम्। तत्र द्वादशकोटौ पलभा भुजः क्रान्तिज्याकौटौ
\end{sloppypar}


\newpage


\hspace{3cm}   गुढार्थप्रकाशकेन सहितः~। \hfill ९७
\vspace{1cm}

\begin{sloppypar}
\noindent को भुज इत्यनुपातेन कुज्या । तत्स्वरूपं तु निरक्षदेशक्षितिजस्वदेशक्षितिजान्तरालस्थिताहोरात्रवृत्तप्रदेशस्य द्युज्याप्रमाणेन ज्येति त्रिज्याप्रमाणेन तज्ज्या चरज्येति द्युज्याप्रमाणेन कुज्या त्रिज्याप्रमाणेन केत्यनुपातेन चरज्या तद्धनुश्वरासवोऽहोराचवृत्तखण्डप्रदेशे निरक्षस्वक्षितिजान्तराल उत्तरगोले स्वक्षितिजस्य निरक्षक्षितिजादधःस्थत्वान्निरक्षक्षितिजयाम्योत्तरवृत्तान्तरालेऽहोरात्रवृत्तचतुर्थांशत्वादहोरात्रसुचतुर्थांशे चरासवो युता दिनार्धं हीना रात्र्यर्धं दक्षिणगोले स्वक्षितिजस्य निरक्षक्षितिजादूर्ध्वस्थत्वाद्धीना दिनार्धं युता रात्र्यर्धमित्युपपन्नं सर्वं क्रान्तिज्येत्यादि~॥~६३~॥\\
\noindent अथ ग्रहस्य नक्षत्रानयनमाह\textendash
\end{sloppypar}
%\vspace{3mm}
\begin{quote}

 {\ssi भभोगोऽष्टशतीलिप्ताः स्वाश्विशैलास्तथा तिथेः~।\\
ग्रहलिप्ता भभोगाप्ता भानि भक्त्या दिनादिकम्~॥~६४~॥}
%\vspace{3mm}
\end{quote}
\begin{sloppypar}
 अष्टशतमिताः कला नक्षत्रभोगः । प्रसङ्गात् तिथिभोगमाह\textendash खाश्विशैला इति~। तिथेर्विंशत्यधिकसप्तशतमिताः कलास्तथा भोग इत्यर्थः । यस्य ग्रहस्य नक्षत्रज्ञानमिष्टं तस्य ग्रहस्य राशयस्त्रिंशङ्गुण्या अंशा योज्यास्ते षष्टिगुणिताः कला योज्या इति परिभाषया कला नक्षत्रभोगभक्ताः फलं ग्रहस्य गतनक्षत्राणि शेषं वर्तमाननक्षत्रस्य गतकलास्तस्मात् तस्य गतदिनाद्यानयनमाह\textendash भुक्त्येति~। ग्रहस्य कलात्मिकया गत्या शेषदिनादिकं गतं भागहरणेन साध्यमेवं शेषोनाद्भोगाद्गतिकला भागेनैव्यदिनादिकं साध्यम् । अत्रोपपत्तिः । भचक्रभोगेन सप्तविंशतिनक्षत्राण्यश्वि\textendash
\end{sloppypar}
{\tiny{o}}

\newpage


\noindent ९८ \hspace{4cm} सूर्यसिद्धान्तः 
\vspace{1cm}

\begin{sloppypar}
\noindent न्यादीनि ग्रहो भुनक्त्यतः सप्तविंशतिनक्षत्राणां चक्रकलाः षट्शतयुतैकविंशतिसहस्त्रमिता भोगस्तदैकनक्षत्रस्य क इत्यनुपातेनाष्टशतकलाभोगः । एवं तिथेश्चान्द्रमासत्रिंशदंशत्वाच्चान्द्रमासस्य सूर्यचन्द्रान्तरैकभगणसिद्धत्वाच्च । त्रिंशत्तिथीनां चक्रकलाभोगस्तदैकतिथेः क इत्यनुपातेन विंशत्यधिकसप्तशतकलाभोगः। अथाष्टशतकलाभिरेकं नक्षत्रं तदा ग्रहकलाभिः किमित्यनुपातेन फलमश्विन्यादीनि ग्रहभुक्तानि शेषकला ग्रहाधिष्ठितनक्षत्रस्य गतं भभोगाद्धीनं तस्यैष्यमाभ्यां ग्रहगत्यैकं दिनं तदाभीष्टकलाभिः किमित्यनुपातेन तस्य गतैष्यदिवसाद्यं भवति । एवं चन्द्राद्दिननक्षत्रं ज्ञेयम्~॥~६४~॥\\
\noindent अथ प्रसङ्गाद्योगानयनमाह\textendash
\end{sloppypar}
%\vspace{2mm}
\begin{quote}

 {\ssi रवीन्दुयोगलिप्ताभ्यो योगा भभोगभाजिताः~।\\
गता गम्याश्च षष्टिघ्ना भुक्तियोगाप्तनाडिकाः~॥~६५~॥}
%\vspace{2mm}
\end{quote}
\begin{sloppypar}
 सूर्यचन्द्रयोगस्य राश्यादिकस्य परिभाषया याः कलास्ताभ्यो योगा विष्कम्भादयो भभोगभाजिता भभोगेन पूर्वोक्तेन विभक्ता भवन्ति । एकैकयोगस्य भभोगमितो भोगः स प्रत्येकं ताभ्योऽपनीय यन्मिताः शुद्धास्तन्मिता योगा गताः । यस्य भोगो न शुद्ध्यति स वर्तमान इत्यर्थः । कला भभोगभक्ता गता योगास्तदग्रिमो वर्तमान इति तात्पर्यम् । तस्य शेषं गतं भोगात् पतितमेष्यं ताभ्यां घटिकाद्यानयनमाह\textendash गता इति~। गता एष्याः । चः समुच्चये । कलाः षष्टिगुणिताः कार्यास्ताभ्यो भुक्तियोगाप्तनाडिका रविचन्द्रकलात्मकगत्योर्योगेन भजनाल्लब्धा घटिका गतैष्या
\end{sloppypar}

\newpage


\hspace{3cm}  गुढार्थप्रकाशकेन सहितः~। \hfill ९९
\vspace{1cm}

\begin{sloppypar}
\noindent भवन्ति । अत्रोपपत्तिः । सूर्यचन्द्रयोगमितस्य ग्रहस्य नक्षत्राणि विष्कम्भादिसञ्ज्ञानि योगात्पन्नत्वाद्योगा अतस्तदानयनं पूर्वोक्तवत् । अत एव सूर्यचन्द्रगतियोगतुल्यद्गत्या षष्टिसावनघटिकास्तदा गतैष्यकलाभिः का इत्यनुपातेन गतैष्यघटिकानयनं युक्तमुक्तम्~॥~६५~॥\\
\noindent अथ प्रसङ्गात् तिथ्यानयनमाह\textendash
\end{sloppypar}
%\vspace{2mm}
\begin{quote}

 {\ssi अर्कोनचन्द्रलिप्ताभ्यस्तिथयो भोगभाजिताः~।\\
गता गम्याश्च षष्टिघ्ना नाड्यो भुक्त्यन्तरोद्धृताः~॥~६६~॥}
%\vspace{2mm}
\end{quote}
\begin{sloppypar}

 पूर्वार्धव्याख्यानं पूर्वश्लोकपूर्वार्धरीत्या ज्ञेयमुत्तरार्धं स्पष्टम् । अत्रोपपत्तिः । तिथिभोगकलाभिरेका तिथिस्तदा सूर्योनचन्द्रकलाभिः का इत्यनुपातेन फलं गततिथयो वर्तमानतिथेर्गतैष्ये शेषशेषोनभोगकले ताभ्यां गत्यन्तरकलाभिरनुपातेन गतैष्यघटिकाः पूर्ववत्~॥~६६~॥\\
\noindent  अथ पञ्चाङ्गावशिष्टं करणानयनं विवक्षुस्तावत् स्थिरकरणान्याह\textendash
\end{sloppypar}
%\vspace{2mm}
\begin{quote}

 {\ssi ध्रुवानि शकुनिर्नागं तृतीयं तु चतुष्पदम्~।\\
किंस्तुघ्नं तु चतुर्दश्याः कृष्णायाश्चापरार्धतः~॥~६७~॥}
\end{quote}

\begin{sloppypar}
 कृष्णपक्षीयायाश्चतुर्दश्यास्तिथेर्द्वितीयार्धाद्द्वितीयार्धमारभ्येत्यर्थः । चकार एवार्थे । तेनान्यतिथेरेतत्तिथिपूर्वार्धस्य च निरासः । स्थिराणि करणानि । तान्याह\textendash शकुनिरिति~। चतुरङ्घ्रितृतीयमनेन शकुनिनागयोः क्रमेणाद्यद्वितीयत्वं सूचितम् ।
तुकारात् कमेण तिथ्यर्धेषु भवन्ति । किंस्तुघ्नं चतुर्थम् । तुर\textendash
\end{sloppypar}

{\tiny{o2}}

\newpage



\noindent १०० \hspace{3cm} सूर्यसिद्धान्तः
\vspace{1cm}

\noindent न्तावधिद्योतकः तेनोक्तातिरिक्तं स्थिरकरणं नास्तीति सूचितम्~॥~६०~॥\\ अथ चरकरणान्याह\textendash
%\vspace{2mm}
\begin{quote}

{\ssi  बवादीनि ततः सप्त चराख्यकरणानि च~।\\
मासेऽष्टकृत्व एकैकं करणानां प्रवर्तते~॥~६८~॥}
%\vspace{2mm}
\end{quote}
\begin{sloppypar}
 ततः स्थिरकरणपूर्त्यनन्तरं बवादीनि चरसञ्ज्ञककरणानि सप्त भद्रान्तानि शुक्लप्रतिपद्वितीयार्धतश्चतुर्थ्यन्तं भवन्तीति चार्थः । ननु पञ्चम्यादितः कानि करणानि भवन्तीत्यत आह\textendash मास इति । चरकरणानां बवादीनां सप्तानां मध्य एकैकमेकमेकं करणं मासे स्थिरकरणकालोनितत्रिंशत्तिथ्यात्मकमासे स्वल्पान्तरान्मासग्रहणम् । अष्टकृत्वोऽष्टवारं प्रवर्तते प्रकर्षेण तिष्ठति भवतीत्यर्थः । तथा च पञ्चम्याद्यर्धादेतानि करणानि पुनः पुनः परिभ्रमन्ति । कृष्णचतुर्दश्याद्यार्धपर्यन्तमिति भावः~॥~६८~॥\\
\noindent ननु स्थिरकरणोक्तावपरार्धत इत्युक्त्या तेषां चतुर्णां तिथ्यर्धभोगेन शुक्लप्रतिपदाद्यर्धपर्यन्तं क्रमेणावस्थानं युक्तं चरकरणानां तु केवलोक्त्या तदनन्तरं कृष्णचतुर्दश्याद्यार्धपर्यन्तमेक एव परिभ्रमोऽस्त्वित्यतस्तदुत्तरं कथयन्नन्यदप्याह\textendash
\end{sloppypar}
%\vspace{2mm}
\begin{quote}

 {\ssi तिथ्यधभोगं सर्वेषां करणानां प्रकल्पयेत्~।\\
एषा स्फुटगतिः प्रोक्ता सूर्यादीनां खचारिणाम्~॥~६६~॥}
%\vspace{2mm}
\end{quote}
\begin{sloppypar}
 सप्तानां चरकरणानां प्रत्येकं तिथ्यन्तश्चासौ भोगश्च तं तिथ्यर्धकालमितावस्थानं प्रकल्पयेत् । एकत्र निर्णीतः शास्त्रार्थो\textendash
\end{sloppypar}

\newpage



\hspace{3cm}  गूढार्थप्रकाशकेन सहितः~। \hfill १०१ 
\vspace{1cm}

\begin{sloppypar}
\noindent ऽपरत्र भवतीति न्यायात् करणत्वेनैषामप्यवस्थानं तत्तुल्यं कुर्यादित्यर्थः । अत एव तिथ्यर्धं करणं स्मृतमित्युक्त्या चान्द्रमासे त्रिंशत्तिथ्यात्मके षष्टिकरणानां सन्निवेशाच्चरकरणानामेव परिभ्रमणे प्रतिमासमनियततिथिभोगकं करणं भवतीति तद्वारणकप्रतिमासनियततिथिभोगककरणकसिद्ध्यर्थं चरकरणानामष्टवारपरिभ्रमणोत्तरमवशिष्टतिथ्योश्चतुर्ष्वर्धेषु स्थिरकरणान्युक्तानीति तात्पर्यम् । तत्रापि कृष्णचतुर्दश्यपरार्धतस्तत्कल्पनं तदिच्छानियामकं स्वतन्त्रेच्छस्य नियोगानर्हत्वात् । अथाग्रिमग्रन्यासङ्गतित्वनिरासार्थमुक्ताधिकारमुपसंहरति\textendash एषेति~। हे मय सूर्यादीनां सप्तग्रहाणामेषा दृश्येत्यादि कल्पयेदित्यन्तं या वार्ता सा स्फुटगतिः स्पष्टगतिः स्पष्टक्रियाज्ञानसम्पादिका प्रोक्ता तुभ्यं मयोक्ता । एतेन स्पष्टाधिकारः परिपूर्तिमाप्त इति सूचितम्~॥~६९~॥
\end{sloppypar}
%\vspace{2mm}
\begin{quote}

  {\qt रङ्गनाथेन रचिते सूर्यसिद्धान्तटिप्पणे~।\\
स्पष्टाधिकारः पूर्णोऽयं तद्गूढार्थप्रकाशके~॥}
%\vspace{2mm}
\end{quote}
\begin{sloppypar}
 इति श्रीसकलगणकसार्वभौमबल्लालदैवज्ञात्मजरङ्गनाथग एकविरचिते गूढार्थप्रकाशके स्पष्टाधिकारः पूर्णः~॥
\end{sloppypar}

\vspace{6mm}
\begin{center}
    \rule{7em}{.5pt}
\end{center}

   
\newpage
   

 

\noindent १०२ \hspace{4cm} सूर्यसिद्धान्तः
\vspace{3mm}

\begin{sloppypar}
\noindent अथ त्रिप्रश्नाधिकारो व्याख्यायते । तत्र विना प्रश्नं गुरोस्तत्प्रतिपादनेच्छानुदयाद्विना च तदिच्छां छात्राणां तज्ज्ञानासम्भवात् त्रयाणां दिग्देशकालानां प्रश्ना इति चिप्रश्नव्युत्पत्तेस्तद्दिग्ज्ञानं श्लोक चतुष्टयेनाह\textendash
\end{sloppypar}
%\vspace{2mm}
\begin{quote}

  {\ssi शिलातलेऽम्बुसंशुद्धे वज्रलेपेऽपिवा समे~।\\
तच शङ्क्वङ्गुलैरिष्टैः समं मण्डलमालिखेत्~॥~१~॥

तन्मध्ये स्थापयेच्छङ्कुं कल्पनाद्वादशाङ्गुलम्~|\\
तच्छायाग्रं स्पुशेद्यत्र वृत्ते पूर्वापरार्धयोः~॥~२~॥

तत्र बिन्दू विधायोभौ वृत्ते पूर्वापराभिधौ~|\\
तन्मध्ये तिमिना रेखा कर्तव्या दक्षिणोत्तरा~॥~३~॥

याम्योत्तरदिशोर्मध्ये तिमिना पूर्वपश्चिमा~|\\
दिङ्मध्यमत्स्यैः संसाध्या विदिशस्तद्वदेव हि~॥~४~॥}
%\vspace{2mm}
\end{quote}
\begin{sloppypar}
 तत्र दिक्साधनोपक्रमे प्रथममम्बुसंशुद्धे जलवत् समीकृते शिलाप्रदेशे । अपिवाथवा तदभावेऽन्यत्र वज्रलेपे चत्वरादौ घुण्टनादिना समस्थाने कृते शङ्कङ्गुलैः शङ्कुस्थाङ्गुलविभागमानगृहीतैरभीष्टमङ्ख्याकाङ्गुलैर्व्यासार्धरूपैर्वृत्तमवक्रमालिखेत् । सर्वतः केन्द्राद्वृत्तपरिधिरेखा तुल्या स्यात् तथेत्यर्थः । ततस्तन्मध्ये तस्य वृत्तस्य केन्द्ररूपमध्ये कल्पनया द्वादशसङ्ख्याकाङ्गुलानि तुल्यानि यस्मिंस्तं द्वादशविभागाङ्कितमित्यर्थः । शङ्कुं समतलमस्तकपरिधिकाष्ठदण्डं स्थापयेत् । ततः पूर्वापरार्धयोर्दिनस्य
\end{sloppypar}

\newpage


\hspace{3cm}  गुढार्थप्रकाशकेन सहितः~। \hfill १०३
\vspace{1cm}

\begin{sloppypar}
\noindent प्रथमद्वितीयभागयोस्तच्छायाग्रं स्थापितशङ्कोश्छायान्तप्रदेशो मण्डलपरिधौ यस्मिन् विभागे स्पृशेत् । दिनस्य प्रथमविभागेऽनुक्षणं छायाह्रासाद्वृत्ते यत्र प्रविशति दिनस्यापरार्धे छायानुक्षणवृद्धेर्वृत्ते यत्र निर्गच्छतीत्यर्थः । तत्र निर्गमनप्रवेशस्थानयोरुभौ द्वौ बिन्दू पूर्वापरसञ्ज्ञौ क्रमेण वृत्ते परिधिरेखायां कृत्वा तन्मध्ये पूर्वापरबिन्द्वन्तरमध्ये तिमिना मत्स्येन रेखा कार्या सा दक्षिणोत्तररेखा भवति । मत्स्यस्तु बिन्द्वन्तरालसूत्रमितेन व्यासार्धेन बिन्दुद्वयकेन्द्रकल्पनेन वृत्तद्वयं निष्पाद्य वृत्तद्वयसंयोगाभ्यां वृत्तद्वयपरिधिविभागाभ्यामन्तर्गतं मत्स्याकारं स्थानं भवति । तत्रैकः संयोगो मुखं बाह्यवृत्तभागसम्मार्जनेनापरसंयोगस्तु पुच्छमितरवृत्तभागद्वयसम्मार्जनेन । मुखपुच्छावध्यृज्वी रेखा दक्षिणोत्तररेखा। तत्र बिन्दोः सव्यं रेखाग्रं दक्षिणा दिक्। पश्चिमबिन्दोः सव्यं रेखाग्रमुत्तरा दिक् । अनन्तरं पूर्ववृत्तं मत्स्यश्च सम्मार्जनीयः । शङ्कुरपि तत्स्थानान्निष्कास्य इति केवला दक्षिणोत्तररेखा स्थितेति तात्पर्यम् । दक्षिणोत्तरदिशोर्मध्यस्थाने तिमिना दक्षिणोत्तररेखामितेन व्यासार्धेन दक्षिणोत्तरस्थानाभ्यां पूर्ववत् प्रत्येकं वृत्तं विधाय पूर्ववत् सिद्धेन मत्स्येनेत्यर्थः । पूर्वपश्चिमा रेखा कार्या । तत्र पूर्वबिन्दोरासन्नं रेखाग्रं पूर्वा पश्चिमबिन्दोरासन्नं रेखाग्रं पश्चिमेति मत्स्यसम्मार्जनेन केवला पूर्वापररेखापि सिद्धा । अथ रेखासंयोगस्थानाद्दिक्साधनोपक्रमोक्तं पूर्ववृत्तमुल्लिखेत् तद्वृत्तपरिधौ यत्र रेखा लग्ना तत्र दिगिति तद्वृत्तमध्यस्य दिक्चतुष्टयं वृत्ते सिद्धम् ।
\end{sloppypar}


\newpage


\noindent १०४ \hspace{4cm} सूर्यसिद्धान्तः 
\vspace{1cm}

\begin{sloppypar}
\noindent तद्वत् । यथा दक्षिणोत्तराभ्यां पूर्वापरा साधिता तत्प्रकारेणेत्यर्थः । एवकारोऽन्यप्रकारनिरासार्थकः । हि निश्चयेन । विदिशः कोणदिशो दिशां पूर्वादिसिद्धदिशां मध्यमत्स्या अव्यवहितदिग्द्वयान्तरोत्पन्ना लघवस्तैः संसाध्या: सम्यक्प्रकारेण साध्याः । रेखावृत्तसंयोगस्थत्वेन ज्ञेयाः । अत्रोपपत्तिः । क्षितिजपूर्वापरवृत्तसंयोगौ पूर्वापरविभागस्थौ पूर्वापरदिशे तत्र पूर्वापरविभागज्ञानं सूर्योदयास्ताभ्यां तत्र क्षितिजे पूर्वापरवृत्तं कुत्र लग्नमिति ज्ञानं तु विषुवद्वृत्तक्रान्तिवृत्तसम्पातस्थसूर्यस्योदयास्तस्थलज्ञानेन विषुवद्वृतस्य पूर्वापरक्षितिजवृत्तसम्पातयोः सम्बद्धत्वात् । अथान्यस्मिन् दिने सूर्यस्योदयास्तावग्रांशान्तरेण याम्योत्तरे भवत इति सूर्योदयास्तस्थानाभामग्रांशान्तरेणोत्तरयाम्ये पूर्वापरस्थानं भवतीति क्षितिजस्य महत्त्वाद्दूरत्वाच्च तद्दानेन पूर्वापरज्ञानमशक्यमतस्तत्सूत्रेण स्वाभीष्टप्रदेशे तज्ज्ञानार्थमभीष्टसमस्थले क्षितिजानुकारं वृत्तं कृतम् । तत्रापि सूर्योदयास्तसमसूत्रेण स्थलज्ञानस्य दुःशकत्वाच्छायार्थं शङ्कुः स्थाप्यः । तथापि सूर्योदये छायानन्त्याद्वृत्तपरिधौ तदग्रस्पर्शाभावः । परन्तु यथा यथा सूर्य ऊर्ध्वं भवति तथा तथा छायाह्रासाद्यत्र छाया वृत्तपरिधौ यदा प्रविशति तत्स्थानात् तात्कालिको वक्ष्यमाणभुजो व्यस्तोऽर्धज्याकारेण देयस्तदुत्क्रमज्या यत्र परिधिप्रदेशे लगति तत्र शङ्कुस्थानस्य पश्चिमा । छायाग्रस्य पूर्वापरसूत्राङ्भुजान्तरेण याम्योत्तरपतनात् सूर्यापरदिशि छायापतनाच्च । एवं दिनापरार्धे सूर्यो यथा यथाधः सञ्चरति
\end{sloppypar}


\newpage


\hspace{3cm} गुढार्थप्रकाशकेन सहितः~। \hfill १०५
\vspace{1cm}

\begin{sloppypar}
\noindent तथा तथा छायावृद्धेः शङ्कुच्छायावृत्तपरिधौ यत्र यदा निर्गच्छति तात्कालिको वक्ष्यमाणभुजो व्यस्तोऽर्धज्याकारेण तत्स्थानाद्देयस्तदुत्क्रमज्या यत्र परिधिप्रदेशे लगति तत्र शङ्कुस्थानस्य पूर्वा । तत्सूत्रं पूर्वापरसूत्रम् । इदं शङ्कोरूपलक्षणत्वेन ज्ञातं तथा छायोपलक्षणेनापि प्रदेशस्य पूर्वापरसूत्रज्ञानम् । तथाहि । छायाग्रं विशति तत्रापरा छायाग्रं निर्गच्छति तत्र पूर्वा । तत्रापि प्रवेशनिर्गमयोरेककालत्वासम्भवाद्यत्कालिकः प्रवेशस्तत्काले छायायाः पश्चिमत्वं तत्र वस्तुभूतं तत्काले निर्गमनस्य पूर्ववासम्भवः । एवं निर्गमकाले निर्गमस्थानस्य पूर्वत्वं वस्तुभूतं तत्काले प्रवेशस्य पश्चिमत्वासम्भवः । एककालिकसिद्ध्यार्थमुभयोरेकतरं चिन्हं चाल्यं तात्कालिकभुजयोरन्तरेण तत्र पूर्वचिन्हं भुजान्तराङ्गुलैरयनदिशि चाल्यम् । पश्चिमचिन्हं वा व्यस्तायनदिशि चाल्यम्। तत्सूत्रं सूत्रमध्यदेशस्य पूर्वापरसूत्रम् । एतन्मध्ये स्थापितशङ्कोश्छायाग्रप्रवेशनिर्गमचिन्हाभ्यां यथोक्तरीत्या भुजदानेन सिद्धपूर्वापरसूत्रेणाभिन्नत्वात् । तदुक्तं सिद्धान्तशिरोमणौ ।
\end{sloppypar}
%\vspace{2mm}
\begin{quote}

 {\qt तत्कालापमजीवयोस्तु विवराद्भाकर्णमित्या हतात्\\
लम्बज्याप्तमिताङ्गुलैरयनदिश्यैन्द्री स्फुटा चालिता~।}
%\vspace{2mm}
\end{quote}
\begin{sloppypar}
 इति । तदेतद्भगवता लोकानुकम्पया स्वल्पान्तरत्वादेकतरबिन्दुचालनं नोक्तं सुखार्थं किञ्चित्स्थूलावेव निर्गमप्रवेशबिन्दू पूर्वापराभिधावुक्तौ । एवञ्चाभीष्टस्थानं प्रवेशनिर्गमसूत्रमध्ये यथा भवति तथानेन प्रकारेण मण्डलकेन्द्रे शङ्कुस्थापनादिना\textendash
\end{sloppypar}


\newpage


\noindent १०६ \hspace{4cm} सूर्यसिद्धान्तः
\vspace{1cm}

\begin{sloppypar}
\noindent भीष्टप्रदेशे पूर्वापरदिशे साध्ये इति । तन्मध्ये दक्षिणोत्तररेखाबिन्दुद्वयोत्पन्नमध्यमत्स्यरेखैवेति । याम्योत्तरमध्ये पूर्वापरा रेखा  तद्दिङ्मध्यमत्स्येनेति याम्योत्तरदिशोरित्यादि सम्यगुक्तम् । ननु पूर्वापरबिन्दुभ्यां मत्स्येन या दक्षिणेत्तररेखा तदग्राभ्यां मत्स्येन रेखा पूर्वापरबिन्दुस्पृष्टैवेति पूर्वं तस्या एव बिन्द्वन्तरत्वेन सिद्धत्वात् पुन: साधनं व्यर्थमन्यथा दक्षिणोत्तररेखाया अप्यसङ्गतत्वापत्तेरिति चेत् सत्यम् । दक्षिणोत्तररेखाशुद्ध्यर्थमेव पूर्वापरबिन्दुस्पृष्टरेखायाः पुनः साधनमिति केचित् । वस्तुतस्तु दक्षिणोत्तरपूर्वापरसूत्रसम्पातरूपाभीष्टस्थानात् केन्द्रात् प्रागुक्तवृत्तस्य वक्ष्यमाणोपयोगित्वेनावश्यकत्वात् तस्य च पूर्वापरबिन्द्वन्तरसूत्राधिकव्याससूत्रत्वाद्विन्द्विन्तररेखा मूलाग्रयोर्वर्धनीया सा तत्र वृत्ते पूर्वापररेखा भवति । तस्या बिन्दोरुपर्यधश्च वक्रत्वं कदाचित् स्यादतः प्रथममेव पूर्णरेखासिद्ध्यर्थं बिन्द्वन्तरसिद्धमत्स्यमुखपुच्छगतरेखाया बिन्द्वन्तराधिकत्वेन तदुत्पन्नमत्स्यरेखाया ऋज्व्याः सुतरामधिकत्वेन पुनः पूर्वापररेखासाधनं युक्ततरमिति तत्त्वम् । एवमेवाव्यवहितदिग्द्वयान्तरोत्पन्नलघुमत्स्यैश्चतुतुर्भिः सूत्रैर्वृत्ते कोणदिशः । तदिदमभीष्टस्थानकेन्द्रक्रमण्डले दिगष्टकं सिद्धम्~॥~४~॥\\
\noindent अथ दिक्सूत्रसम्पातरूपाभीष्टस्थानात् तात्कालिकच्छायाग्रस्थानमाह\textendash
\end{sloppypar}
%\vspace{2mm} 

\begin{quote}

  {\ssi चतुरस्त्रं बहिः कुर्यात् सूर्त्रैर्मध्याद्विनिर्गतैः~।\\
 भुजसूत्राङ्गलैस्तत्र दत्तैरिष्टप्रभा स्मृता~॥~५~॥}
\end{quote}
\newpage




\hspace{3cm} गुढार्थप्रकाशकेन सहितः~। \hfill १०७
\vspace{1cm}

\begin{sloppypar}
मध्यादभीष्टस्थानाद्दिग्रेखासम्पातरूपाद्विनिर्गतैर्निसृतैरष्टदिग्रेखारूपैः। बहिर्दिक्सूत्रसम्पातकेन्द्रवृत्ताद्बहिः । अनेनैव वृत्तकरणं पूर्वमनुक्तं द्योतितम्। अन्यथा बहिरित्यस्यानुपपत्तेः । पूर्ववृत्तग्रहणे तु दिग्रेखासम्पातस्य मध्यत्वानुपपत्तेः । चतुरस्रं कोणरेखाधिकसूत्रकर्णद्वयतुल्यं समचतुर्भुजं कुर्यात् । यथा च तद्दर्शनम् । तत्र चतुरस्रे भुजसूत्राङ्गुलैर्वक्ष्यमाणभुजमितसूत्रस्याङ्गुलैर्निर्गमप्रवेशकालिकैर्दत्तैः पूर्वापरसूत्रादर्धज्यावद्दोयमानैस्तत्र वृत्ते यस्मिन् प्रदेशे भुजायं तत्प्रदेश इष्टप्रभा निर्गमप्रवेशान्यतरकालिकच्छायाग्रमुक्तम् । प्रतीतिस्तु दिक्सूत्रसम्पातस्थ शङ्कुना ज्ञेया । अत्रोपपत्तिः । वक्ष्यमाणभुजस्य छायाग्रपूर्वापरसूत्रान्तरत्वेन प्रतिपादितत्वादिष्टच्छायाग्रमुक्तदिशा ज्ञातं सम्यक् । चतुरस्रकरणं वक्ष्यमाणाग्रामाधकप्राच्यपररेखानुकाररेखाया वृत्तान्तस्तद्बहिर्वा ऋजुत्वसिद्ध्यर्थमिति~॥~५~॥\\
\noindent अथ पूर्वापररेखायाः सञ्ज्ञान्तरमाह\textendash
\end{sloppypar}
%\vspace{2mm}
\begin{quote}

  {\ssi प्राक्पश्चिमाश्रिता रेखा प्रोच्यते सममण्डलम्~।\\
 उन्मण्डलं च विषुवन्मण्डलं परिकीर्त्यते~॥~६~॥}
%\vspace{2mm}
\end{quote}
\begin{sloppypar}
 प्राक्पश्चिमाश्रिता पूर्वपश्चिमसम्बद्धा साधिता रेखा समवृत्तमुच्यते । सैव रेखोन्मण्डलं विषुवन्मण्डलम्। चः समुच्चये। उभयसञ्ज्ञकं कथ्यते । अत्रोपपत्तिः । क्षितिजपूर्वापरवृत्तसंयोगौ पूर्वापरे तत्सूत्रं पूर्वापरसूत्रमिति पूर्वापरवृत्तस्य भूमावूर्ध्वाधरानुकारिवृत्तत्वेनादर्शनाद्रेखाकारतयैव दर्शनाच्च पूर्वापर\textendash
\end{sloppypar}

\newpage


\noindent १०८ \hspace{4cm} सूर्यसिद्धान्तः 
\vspace{1cm}

\begin{sloppypar}
\noindent वृत्तमपि तत्सूत्रम् । पूर्वापरवृत्तस्य सममण्डलत्वेनाभिधानात् तद्रोखासममण्डलसञ्ज्ञोक्ता । अथ स्वनिरक्षदेशक्षितिजवृत्तस्योन्मण्डलाख्यस्य तत्संयोगयोः संलग्नत्वात् तन्मध्यसूत्रत्वेन पूर्वापरसूत्रस्यापि सत्त्वात् पूर्वापरसूत्रमुन्मण्डलसञ्ज्ञम् । एतेनान्यदेशक्षितिजसञ्ज्ञया स्वदेशक्षितिजसञ्ज्ञा सुतरां सिद्धेति पूर्वापरसूत्रस्य क्षितिजवृत्तसञ्ज्ञा द्योतिता । पूर्वापरस्थानयोः क्षितिजवृत्तस्य संलग्नत्वादुल्लिखितवृत्तस्य क्षितिजानुकारित्वाच्च । एवं निरक्षदेशपूर्वापरवृत्तं विषुवन्मण्डलाख्यं पूर्वापरस्थानयोः संलग्नमिति तन्मध्यसूत्रत्वेनापि पूर्वापरसूत्रस्य सिद्धत्वात् पूर्वापरसूत्रं विषुवन्मण्डलसञ्ज्ञं क्रान्तिवृत्तस्य दृग्वृत्तस्य च चलत्वात् कादाचित्कत्वेन पूर्वापरस्थानसंलग्नत्वात् तत्सञ्ज्ञा नो नोक्तेति ध्येयम्~॥~६~॥\\
\noindent अथाग्राज्ञानमाह\textendash
\end{sloppypar}
\begin{quote}

 {\ssi रेखा प्राच्यपरा साध्या विषुवद्भाग्रगा तथा~।\\
 इष्टच्छायाविषुवतोर्मध्यमग्राभिधीयते~॥~७~॥}
%\vspace{2mm}
\end{quote}
\begin{sloppypar}
तस्मिंश्चतुरस्त्रे पूर्वापररेखात उत्तरभागे विषुवद्भाग्रगाक्षभाग्रप्रदेशस्थाक्षभाङ्गुलान्तरितेत्यर्थः। प्राच्यपरा रेखा पूर्वापररेखानुकारा रेखा तथा सर्वतस्तुल्यान्तरेण यथेष्टच्छायाग्ररेखाभुजान्तरेण तथाक्षभान्तरेण कार्या। अनन्तरमिष्टच्छायाविषुवतोरिष्टच्छायाग्ररेखाक्षभाग्ररेखयोरित्यर्थः। मध्यं चतुरस्रेऽङ्गुलात्मकमन्तरालं सर्वतस्तुल्यम्। अग्राकर्णवृत्तागोच्यते। अत्रोपपत्तिः। भुजस्य कर्णवृत्ताग्रा पलभामंस्कारेणाग्र उक्त\textendash
\end{sloppypar}

\newpage

\hspace{3cm} गुढार्थप्रकाशकेन सहितः~। \hfill १०९
\vspace{1cm}

\begin{sloppypar}

\noindent त्वाद्दक्षिणगोले पलभाधिकोत्तरभुजसद्भावेन पलभोनो भुजोऽग्रेति प्राच्यपरसूत्रादुत्तरभागेऽक्षभाग्ररेखा भुजमध्ये भवतीति द्वयो रेखयोरन्तरमग्रा पलभोनभुजरूपा । एवमुत्तरगोल उत्तरभुजस्य पलभाल्पत्वाद्भुजोनपलभाग्रेति पलभारेखा प्राच्यपरसूत्रादुत्तरभागस्या भुजरेखातोऽप्यग्रान्तरेणोत्तरदिशीति द्वयो रेखयोरन्तरं भुजोनपलभारूपं कर्णवृत्ताग्रा । एवं दक्षिणभुजस्य पलभोनाग्रात्वात् पलभायुतो भुजोऽग्रेति प्राच्यपरसूत्राद्भुजाग्रपलभाग्ररेखयो: क्रमेण याम्योत्तरत्वात् तयोरन्तरालं पलभाभुजैक्यरूपमग्रा पलभायाः शङ्कुतलानुकल्पत्वात् सदोत्तरत्वं छायासम्बन्धाद्युक्तम् । गोले शङ्कुतलस्य दक्षिणत्वाद्ग्रहा परदिशि छायासद्भावाच्च । अत एव प्राच्यपरसूत्राद्दक्षिणभागे दक्षिणभुजवशादक्षभाग्ररेखाकल्पन उक्तानुत्पत्या सम्यगुत्तरभागे पूर्वापरसूत्रादिति विषुवद्भाग्रगेत्यत्र व्याख्यातम्~॥~७~॥\\
\noindent अथ प्रसङ्गाञ्ज्ञातच्छायातः कर्णज्ञानं तच्छुद्धिं चाह\textendash
\end{sloppypar}
%\vspace{2mm}
\begin{quote}

  {\ssi शङ्कुच्छायाकृतियुतेर्मूलं कर्णोऽस्य वर्गतः~।\\
 प्रोज्झ्य शङ्कुकृतिं मूलं छाया शङ्कुर्विपर्ययात्~॥~८~॥}
%\vspace{2mm}
\end{quote}
\begin{sloppypar}
 द्वादशाङ्गुलशङ्कुच्छाययोर्वर्गयोगात् पदं छायाकर्ण: स्यात् । अथास्य शुद्धिरूपं छायासाधनमाह\textendash अस्येति~। छायाकर्णस्य वर्गात् शङ्कुवर्गं चतुश्चत्वारिंशदधिकं शतं विशोध्य मूलं छाया । प्रकारान्तरेण छायाकर्णशुद्धिमाह\textendash शङ्कुरिति~। विपर्ययाच्छायासाधनवैपरीत्याच्छायाकर्णवर्गाच्छायावर्गं विशोध्य मूल\textendash
\end{sloppypar}

\newpage


\noindent ११० \hspace{4cm} सूर्यसिद्धान्तः 
\vspace{1cm}

\begin{sloppypar}
\noindent मित्यर्थः । शङ्कुर्द्वादशाङ्गुलमितः स्यात् । अत्रोपपत्तिः । द्वादशाङ्गुलशङ्कुः कोटिरक्षमाभुजस्तत्कृत्योर्योगपदं कर्ण इत्यक्षकर्णः कर्ण इत्याद्यक्षक्षेत्राद्युक्तरीत्योपपन्नम् । ननु दिक्साधनोत्तरमिष्टप्रभाग्राकर्णसाधनं भगवता सर्वज्ञेन किमर्थमुक्तमग्रेऽग्रादीनां स्वतन्त्रतयोक्तत्वात् । न च विना गणितश्रममग्राज्ञानार्थमिदं युक्तमुक्तमिति वाच्यम् । वक्ष्यमाणभुजज्ञानस्याग्रोपजीव्यत्वेन तस्याश्च भुजोपजीव्यत्वेनान्योन्याश्रयात् । गणितज्ञाताग्रायाः पुनः साधनस्य व्यर्थत्वाच । न च भुजसूत्राङ्गुलैर्दत्तैरित्यनेनेष्टच्छायाग्रं ज्ञातमिति न किन्त्वेतदुक्त्या दिक्सूत्रसम्पातस्थशङ्कोर्वृत्तपरिधौ छायाग्रज्ञानात् तत्पूर्वापरसूत्रान्तरे भुजसद्भावाद्विना गणितं भुजोऽपि ज्ञात इति नान्योन्याश्रय इति वाच्यम् । तथापि भगवतः सर्वज्ञस्य निष्प्रयोजनत्वोक्तेरनुचितत्वात् । विना प्रयोजनं मन्दोक्तेरप्यभावाच्च । न हि दिक्साधनेऽग्राभुजादि कमावश्यकं येन तदुक्तिर्युक्तेति । किञ्च कर्णसाधनस्य गणितोक्त्या वक्ष्यमाणकर्णसाधनतुल्यत्वेनात्र कथनमनुचितम् । न हि दिक्साधनार्थं भाकर्णमित्या हतादिति सिद्धान्तशिरोमण्युक्तिवदत्र छायाकर्ण उपयुक्तो येन तदुक्तिर्युक्तेति चतुरस्रमित्यादिश्लोकचतुष्टयमन्येन मन्दबुद्धिना क्षिप्तं न भगवतोक्तमिति चेन्मैवम् । भुजसाधनोपजीव्याग्राया एतदुक्तप्रकारेण सिद्धौ दिशः सम्यक् सिद्धा इति दिक्साधनशुद्ध्यर्थमग्रासाधनम् । प्रकारान्तरेणापि वक्ष्यमाणत्रिज्यावृत्तीयाग्रया त्रिज्या लभ्यते तदानयागतया केत्यनुपातेन साधितकर्णसंवादेन शुद्ध्यवगमार्थं कर्णसाधनं चो\textendash
\end{sloppypar}

\newpage


\hspace{3cm}  गुढार्थप्रकाशकेन सहितः~। \hfill १११
\vspace{1cm}

\begin{sloppypar}
\noindent क्तम्\textendash अनयाग्रया कर्णस्तदा त्रिज्यावृत्तीयाग्रया क इति फलस्य त्रिज्यातुल्यस्यानयनार्थं वा कर्णसाधनमिति केचित् । वस्तुतस्तु मण्डले छायाप्रवेशनिर्गमस्थानस्थितपूर्वापरबिन्द्वोः प्रत्येकं रेखेति रेखाद्वयं सर्वतस्तुल्यान्तरं कार्यं तेनान्तरेणान्यतरो बिन्दुश्चाल्यस्तौ पूर्वापरबिन्दू तद्रेखा मध्यस्थानस्य पूर्वापररेखेति।तत्रोभयबिन्दुरेखयोरन्तराङ्गुलमानं स्वल्पत्वाद्गणयितुमशक्यमतः प्रत्येकरेखे प्राच्यपररेखे प्रकल्प्य तन्मध्यकेन्द्रात् पूर्ववृत्तं प्रत्येकमिति वृत्तद्वयं कुर्यात् । तत्र स्वस्ववृत्ते स्वस्वप्राच्यपररेखा स्पृष्टा कार्या ताभ्यां स्वस्वकालिकौ भुजौ स्वस्ववृत्ते देयौ तदग्रेछायाग्ररेखे स्वस्ववृत्ते कार्ये स्वस्वप्राच्यपरसूत्रात् स्वस्ववृत्त उत्तरभागेऽक्षभाङ्गुलान्तरेण रेखे कार्ये ततः स्वस्ववृत्ते स्वस्वतद्रेखयोरन्तरं स्वस्ववृत्त उभयकालिककर्णवृत्ताग्रे बहुत्वेन गणयितुं शक्ये तदन्तरं पूर्वबिन्द्वोर्याम्योत्तरमन्तरं कर्णवृत्ताग्रासाधनकथनेनानीतं भुजान्तरस्य बिन्द्वोन्तरत्वात् तस्य चाग्रान्तरत्वेन फलितत्वात् । विषुवद्दिने गोलभेदे तु भुजान्तरमग्रायोग इति बिन्द्वोर्याम्योत्तरमन्तरमग्रायोग इति । तेनोक्तरीत्या बिन्दुश्चाल्यस्तत्सूत्रं पूर्वापरसूत्रं स्फुटमित्याशयेन भगवताग्रा निरूपिता तस्याः शुद्ध्यर्थं कर्णोऽपि साधित इति तत्त्वम्~॥~८~॥\\
\noindent अथ पूर्वाधिकारे क्रान्त्याद्यानयनमुक्तं तत् पूर्वाधिकारावगतग्रहात् केवलान्न साध्यमिति श्लोकाभ्यामाह\textendash
\end{sloppypar}
%\vspace{2mm}
\begin{quote}

  {\ssi त्रिंशत्कृत्यो यगे भानां चक्रं प्राक् परिलम्बते~।\\
तद्गुणाद्भूदिनैर्भक्ताद्युगणाद्यदवाप्यते~॥~९~॥}
%\vspace{2mm}
\end{quote}

\newpage



\noindent ११२ \hspace{4cm} सूर्यसिद्धान्तः 
\vspace{1cm}
\begin{quote}

  {\ssi तद्दोस्त्रिघ्ना दशाप्तांशा विज्ञेया अयनाभिधाः~।\\
 तत्संस्कृताङ्ग्रहात् क्रान्तिच्छायाचरदलादिकम्~॥~१०~॥}
%\vspace{2mm}
\end{quote}
\begin{sloppypar}
 भानां चक्रं राशीनां वृत्तं क्रान्तिवृत्तं स्वस्वविक्षेपमितशलाकाग्रप्रोतनक्षत्रगणैर्युक्तमित्यर्थः । युगे महायुगे प्राक्पूर्वविभागे चिंशत्कृत्यस्त्रिंशत्सङ्ख्याका कृतिर्विंशतिः षट्शतमित्यर्थः । परिलम्बते ध्रुवाधारभगोलस्थानात् तद्द्वारमवलम्बते । अत्र परिलम्बत इत्यनेन भचक्रपूर्णभ्रमणाभाव उक्तोऽन्यथा ग्रहभगणप्रसङ्गेन मध्याधिकार एवैतदुक्तं स्यात् । तथा च तद्वारमवलम्बनोक्त्या परावर्त्य यथास्थितं भवतीत्यागतं तत्रापि स्वस्यानात् तथैव पश्चिमतोऽप्यवलम्बत इति सूचितम् । एवञ्च भचक्रं पश्चिमत ईश्वरेच्छया प्रथमतः कतिचिद्भागैश्चलति ततः परावृत्य यथास्थितं भवति ततोऽपि तद्भागैः क्रमेण पूर्वतश्चलति ततोऽपि परावर्त्य यथास्थितमित्येको विलक्षणो भगणः । तेन प्रागित्युपलक्षणम् । पश्चिमावलम्बनानुक्तिस्तु संवादकाले तदभावात् । अत्र त्रिंशत्कृत्वेति पाठः प्रामादिकः ।
\end{sloppypar}
%\vspace{2mm}
\begin{quote}

 {\qt युगे षट्शतकृत्वा हि भचक्रं प्राग्विलम्बते~।}
%\vspace{2mm}
\end{quote}
\begin{sloppypar}
 इति सोमसिद्धान्तविरोधात् । तत्पश्चाच्चलितं चक्रमिति ब्रह्मसिद्धान्तोक्तेश्च । अहर्गणात् तद्गुणात् षट्शतगुणिताद् भूदिनैर्युगीयसूर्यसावनदिनैर्भक्ताद्यत् फलं भगणादिकं प्राप्यते तस्य भगणत्यागेन राश्यादिकस्य भुजः कार्यस्तस्माद्दशाप्तांशा दशभिर्भजनेनाप्तभागास्त्रिगुणिता अयनमञ्ज्ञका ज्ञेयाः । भुजां शास्त्रिगुणिता दशभक्ताः फलमयनांशा इति तात्पर्यार्थः । तत्सं\textendash
\end{sloppypar}

\newpage



\hspace{3cm} गूढार्थप्रकाशकेन सहितः~। \hfill ११३
\vspace{1cm}

\begin{sloppypar}
\noindent स्कृतात् तैरयनांशैर्भचक्रपूर्वापरचलनवशाद्युतहीनाद्ग्रहात् पूर्वापरभचक्रचलनावगमस्त्वयनग्रहस्य षड्भानन्तर्गतान्तरगतत्वक्रमेण क्रान्तिच्छायाचरदलादिकं साध्यम् । न केवलाद्विशेषोक्तेः। छाया वक्ष्यमाणा चरदलं चरं पूर्वाधिकारोक्तम् । आदिशब्दादयनवलनमायनदृक्कर्म सङ्गृह्यते । यद्यपि तत्संस्कृताद्ग्रहात् क्रान्तिरित्येव वक्तव्यमन्येषामत्र तदुपजीव्यत्वाद्ग्रहणं व्यर्थं तथापि क्रान्तिरित्युक्त्या केवलक्रान्तिज्ञानार्थं तत्संस्कृतग्रहात् क्रान्तिः साध्या । पदार्थान्तरोपजीव्यायाः क्रान्तेः साधनं तु केवलादित्यस्य वारणार्थं क्रान्तिमात्रं तत्संस्कृतात् साध्यमिति सूचकं छायाचरदलादिकथनम् । अत्रोपपत्तिः । ईश्वरेच्छया क्रान्तिवृत्तं स्वमार्गे पश्चिमतः सप्तविंशत्यंशैः क्रमोपचितैश्चलितं ततः परावृत्य स्वस्थान आगत्य तत्स्थानात् पूर्वतः सप्तविंशत्यं शैश्चलितम् । तथा च सृष्ट्यादिभूतक्रान्तिविषुवद्वृत्तसम्पाताश्रितक्रान्तिवृत्तप्रदेशे रेवत्यासन्नः प्रागानोतग्रहभोगावधिरूप: स्वस्थानात पूर्वमपरत्र वा क्रान्तिवृत्तमार्गे गतः । विषुवद्वृत्ते तु तद्भागस्य पश्चिमभागः पूर्वभागो वा गतः । सम्पातेतद्वृत्तयोर्याम्योत्तरान्तराभावात् क्रान्त्यभावः । पूर्वसम्पातप्रदेशे तु तयोर्याम्योत्तरान्तरत्वात् क्रान्तिरुत्पन्नातो यथास्थितग्रहभोगात् क्रान्तिरसङ्गतेति सम्पातावधिकग्रहभोगात् क्रान्तिर्युक्ता तत्र सम्पातावधिकग्रहभोगज्ञानार्थं पूर्वसम्पातावधिकः पूर्वाधिकारोक्तो ग्रहभोगो वर्तमानसम्पातपूर्वसम्पाताश्रितक्रान्तिवृत्तप्रदेश्योरन्तरभागैरयनांशाख्यैः पूर्वसम्पातप्रदेशस्य पूर्वपश्चिमाव\textendash
\end{sloppypar}
{\tiny{Q}}

\newpage


\noindent ११४ \hspace{4cm} सूर्यसिद्धान्तः
\vspace{1cm}

\begin{sloppypar}
\noindent स्थानक्रमेण युतहीनो भवति । क्रान्त्युपजीव्यपदार्था अपि वर्तमानसम्पातादुत्पन्ना इति तत्साधनमपि तत्संस्कृतग्रहात्। अथायनांशज्ञानं तु षट्शतभगणेभ्यः पूर्वानुपातरीत्याहर्गणाद्ग्रहभोगो भगणादिकस्तत्र गतभगणमितं परपूर्वभचक्रावलम्बनं गतम् । वर्तमानं त्वारम्भे पश्चिमावलम्बनाद्राशिषट्कान्तर्गते  राश्यादिके पश्चिमावलम्बनमनन्तर्गते पूर्वावलम्बनम् । तत्रापि त्रिभान्तर्गतानन्तर्गतत्वक्रमेण चलनं परावर्तनं चेति भुजः साधितस्ततो नवत्यंशैः सप्तविंशतिभागास्तदा भुजांशैः क इत्यनुपातेन गुणहरौ नवभिरपवर्त्य भुजांशास्त्रिगुणिता दशभक्ता इति सर्वमुपपन्नम्~॥~१०~॥\\
\noindent अथोक्तस्यान्तरस्य प्रत्यक्षसिद्धत्वमिति सार्धश्लोकेनाह\textendash
\end{sloppypar}
%\vspace{2mm}
\begin{quote}

 {\ssi स्फुटं दृक्तुल्यतां गच्छेदयने विषुवद्वये~।\\
प्राक् चक्रं चलितं हीने छायार्कात् करणागते~॥~११~॥\\
 अन्तरांशैरथावृत्य पश्चाच्छेषैस्तथाधिके~।}
%\vspace{2mm}
\end{quote}
\begin{sloppypar}
 अयने दक्षिणोत्तरायणसन्धौ विषुवद्वये गोलसन्धौ चलितं चक्रं दृक्तुल्यतां दृष्टिगोचरतां स्फुटमनायासं गच्छेत् । तत्र प्रत्यक्षस्तस्तन्मितमन्तरं दृश्यत इत्यर्थः । तथा च सृष्ट्यादिकाले रेवतीयोगतारासन्नावधि मेषतुलाद्योः कर्कमकराद्योर्विषुवायनप्रवृत्तेरिदानीं त्वन्यत्र तत्स्वरूपे प्रत्यक्षे इति क्रान्तिवृत्तं चलितमन्यथा तदनुपपत्तेरिति भावः । ननु पूर्वतोऽपरत्र वा चलितमिति कथं ज्ञेयमित्यत आह\textendash प्रागिति~। छायार्काद्य\textendash
\end{sloppypar}


\newpage


\hspace{3cm} गूढार्थप्रकाशकेन सहितः~। \hfill ११५
\vspace{1cm}

\begin{sloppypar}
\noindent द्दिने सूर्यस्यायनदिक्परावर्तनमुदये प्राच्यपरसूत्रस्थत्व वा तस्मिन् दिनेऽन्यस्मिन दिने वा मध्यान्हच्छायातो वक्ष्यमाणप्रकारेण सूर्यः साध्यस्तस्मादित्यर्थः । करणागते प्रागुतप्रकारेणानीतः स्पष्टः सूर्यस्तस्मिन्नित्यर्थः । न्यूने सति । अन्तरांशैः सूर्ययोरन्तररांशैश्चक्रं क्रान्तिवृत्तं प्राक् पूर्वस्मिन् चलितमिति ज्ञेयम् । अथ यद्यधिके सति शेषैः सूर्ययोरन्तरांशैश्चक्रमावृत्यपरिवृत्य पश्चात् पश्चिमाभिमुखं तथा चलितमिति ज्ञेयम् । अत्रोपपत्तिः । छायातो वक्ष्यमाणप्रकारेण सूर्यो वर्तमानसम्पाताद्गणितागतस्तु रेवतीयोगतारासन्नाद्यावधितोऽतस्तयोरन्तरमयनांशास्तत्र क्रान्तिवृत्तस्य पूर्वचलने गणितागतार्काच्छायार्कोऽधिको भवति । पश्चिमचलने तु न्यूनो भवतीति सम्यगुपपन्नम्~॥~११~॥\\
\noindent अथ चराद्युपजीव्यां पलभामाह\textendash
\end{sloppypar}
%\vspace{2mm}
\begin{quote}

 {\ssi एवं विषुवती छाया स्वदेशे या दिनार्धजा~॥~१२~॥
 
दक्षिणोत्तररेखायां सा तत्र विषुवत्प्रभा~।}
%\vspace{2mm}
\end{quote}
\begin{sloppypar}
 स्वाभीष्टदेश एवं विषुवती चलितविषुवद्दिनसम्बंद्धा रेवत्यासन्नस्याप्युपचाराद्विषुवसञ्ज्ञा तद्व्यावर्तकमेवमिति । दिनार्धजामाध्याह्निकी या यन्मिता द्वादशाङ्गुलशङ्कोश्छाया दक्षिणोत्तररेखायां निरोत्तरदक्षिणदेशक्रमेणोत्तरस्यां दक्षिणस्यां प्रभायाः दक्षिणोत्तररेखास्थत्वं विना मध्यासम्भवात् सा तन्मिता तत्र तस्मिन्नभीष्टदेशे विषुवत्प्रभाक्षभा भवति । एतेन द्वादशाङ्गुलशङ्कुः कोटि: पलभा भुजस्तत्कृत्योर्योगपदं कर्ण इत्यक्षकर्णः
\end{sloppypar}

{\tiny{Q2}}

\newpage

\noindent ११६ \hspace{4cm} सूर्यसिद्धान्तः
\vspace{1cm}

\begin{sloppypar}
\noindent कर्ण इत्यक्षक्षेत्रं वक्ष्यमाणोपयुक्तं प्रदर्शितम् । तदा सूर्यस्य विषुवद्वृत्तस्थत्वाद्विषुवत्प्रभेति सञ्ज्ञोक्ता~॥~१२~॥\\
\noindent अथ लम्बाक्षयोरानयनमाह\textendash
\end{sloppypar}
%\vspace{2mm}
\begin{quote}

 {\ssi शङ्कुच्छायाहते त्रिज्ये विषुवत्कर्णभाजिते~॥~१३~॥
 
लम्बाक्षज्ये तयोश्चापे लम्बाक्षौ दक्षिणौ सदा~।}
%\vspace{2mm}
\end{quote}
\begin{sloppypar}
 त्रिज्ये द्विस्थानस्थे शङ्कुच्छायाहते एकत्र द्वादशगुणितापरत्र प्रागुक्तया विषुवत्प्रभया गुणिता विषुवत्कर्णभाजितोभयत्राक्षकर्णेन भक्ता फले क्रमेण लम्बज्याक्षज्ये तयोर्ज्ययोर्धनुषी क्रमेण लम्बाक्षौ सदोभयगोले दक्षिणदिक्स्यौ भवतः। अत्रोपपत्तिः । याम्योत्तरवृत्ते निरक्षस्वदेशपूर्वापरवृत्तयोर्यदन्तरं तदक्षः । याम्योत्तरवृत्ते दक्षिणक्षितिजप्रदेशाद्विषुवद्वृत्तस्य यदन्तरं तल्लम्बः । उभावूर्ध्वगोले स्वपूर्वापरवृत्ताद्दक्षिणौ तज्ज्ये अक्षलम्बज्ये भुजकोटी त्रिज्याकर्ण इत्यक्षक्षेत्रादक्षकर्णकर्णे द्वादशपलभे कोटिभुजौ तदा त्रिज्याकर्णे कावित्यनुपाताभ्यां लम्बाक्षज्ये तद्धनुषी लम्बाक्षावित्युपपन्नम्~॥~१३~॥\\
 \noindent अथ मध्याह्नच्छायातोऽक्षानयनं श्लोकाभ्यामाह\textendash
\end{sloppypar}
%\vspace{2mm}
\begin{quote}

  {\ssi मध्यच्छाया भुजस्तेन गुणिता त्रिभमौर्विका~॥~१४~॥
  
स्वकर्णाप्ता धनुर्लिप्ता नतास्ता दक्षिणे भुजे~।\\
उत्तराश्चोत्तरे याम्यास्ताः सूर्यक्रान्तिलिप्तिकाः~॥~१५~॥

दिग्भेदे मिश्रिताः साम्ये विश्लिष्टाश्चामलिप्तिकाः ~।}
%\vspace{2mm}
\end{quote}
 अभीष्टदिने माध्याह्निकी छाया भुजसञ्ज्ञा ज्ञेया । तेन


\newpage


\hspace{3cm} गूढार्थप्रकाशकेन सहितः~। \hfill ११६
\vspace{1cm}

\begin{sloppypar}
\noindent भुजेन त्रिज्या गुणिता मध्यान्हछायाकर्णेन भक्ता फलस्य धनुः कला नता नतसञ्ज्ञास्ता नतकला दक्षिणे भुजे मध्यान्हच्छायारूपभुजे प्राच्यपरसूत्रमध्याद्दक्षिणदिक्स्थे सति । उत्तरदिक्का उत्तरे भुजे दक्षिणा: । चा विषयव्यवस्थार्थकः । ता नतकलाः सूर्यक्रान्तिकलाः प्रागुक्ताः । दिग्भेदे स्वदिशोर्भिन्नत्वे मिश्रिताः संयुक्ताः साम्येऽभिन्नदिक्त्वे । विश्लिष्टा अन्तरिताः । चो विषयव्यवस्थार्थकः । अक्षकला भवन्ति । अत्रानावश्यकभुजसञ्ज्ञया भगवतोपपत्तिरुक्ता । तथाहि । द्वादशाङ्गुलशङ्कुकोटौ मध्याह्नच्छायाकर्णे वा मध्यच्छायाभुजस्तथा स्वस्वस्तिकान्मध्यान्हकाले सूर्यस्य याम्योत्तरवृत्ते यदन्तरेण नतत्वं ता नतकलास्तज्ज्या नतांशज्या मध्याह्नोन्नतांशज्यारूपशङ्कौ त्रिज्याकर्णे वा भुज इति मध्यान्हच्छायाकर्णे कर्णे मध्यान्हच्छाया भुजस्तदा त्रिज्याकर्णे को भुज इत्यनुपातेन नतज्या तद्धनुरत्र कलात्मकत्वान्नतकलास्ता ग्रहसम्बद्धा इति छायादिग्विपरीतदिक्काः। अथ क्रान्त्यंशाक्षांशयोरेकदिक्त्वे योगेन नतांशा इति दक्षिणा नतकला दक्षिणक्रान्तिकलाभिर्हीना अक्षांशा भवन्ति । क्रान्त्यंशाक्षांशयोर्भिन्नदिक्त्वेऽन्तरेण नतांशा यदि दक्षिणास्तदा क्रान्त्यूनाक्षांशस्य नतत्वादुत्तरक्रान्तियुता अक्षांशाः । यदि तूत्तरास्तदाक्षोनक्रान्तेर्नतत्वान्नतोनोत्तरक्रान्तिरक्ष इति सम्यगुपपन्नम्। केचित् तु भुजग्रहणादभीष्टकाले प्राच्यपरसूत्राच्छायाग्रं यदन्तरेण याम्यमुत्तरं वा भुजस्तं स्वल्पान्तरान्मध्यच्छायां प्रकल्प्य तस्याः कर्णं चानीयोक्तदिशा नतलिप्तास्ता
\end{sloppypar}

\newpage

\noindent ११८ \hspace{4cm} सूर्यसिद्धान्तः 
\vspace{1cm}


अभीष्टक्रान्तिसंस्कृता अक्षांशा भवन्तीत्याहुः~॥~१५~॥\\
अथाक्षात् पलभानयनमाह\textendash
%\vspace{2mm}
\begin{quote}

{\ssi ताभ्योऽक्षज्या च तद्वर्गं प्रोज्झ्य त्रिज्याकृतेः पदम्~॥~१६~॥

लम्बज्यार्कगुणाक्षज्या विषुवद्भाथ लम्बया~।}
%\vspace{2mm}
\end{quote}
\begin{sloppypar}
 ताभ्योऽक्षकलाभ्योऽक्षज्या भवति । चः समुच्चये । अक्षज्यावर्गं त्रिज्यावर्गात् त्यक्त्वा शेषान्मूलं लम्बज्या । अनन्तरमक्षज्या द्वादशगुणा लम्बया लम्बज्यया गुणनस्य भजनसम्बन्धाद्भक्तेत्यर्थसिद्धम् । अक्षभा स्यात् । अत्रोपपत्तिः । अक्षकलानां ज्याक्षज्या तस्यास्त्रिज्याकर्णे भुजत्वात् तद्वर्गोनात् त्रिज्यावर्गान्मूलं लम्बज्या कोटिः । तयाक्षज्या भुजस्तदा द्वादशकोटौ को भुज इत्यनुपातेन विषुवच्छायेति~॥~१६~॥\\
\noindent अथाक्षज्ञाने नतभागेभ्यः क्रान्तिद्वारा सूर्यसाधनं सार्धश्लोकाभ्यामाह\textendash
\end{sloppypar}
%\vspace{2mm}
\begin{quote}

  {\ssi स्वाक्षार्कनतभागानां दिकसाम्येऽन्तरमन्यथा~॥~१७~॥
  
दिग्भेदेऽपक्रमः शेषस्तस्य ज्या त्रिज्यया हता~।\\
परमापक्रमज्या सा चापं मेषादिगो रविः~॥~१८~॥

कर्कादौ प्रोज्झ्य चक्रार्धात् तुलादौ भार्धसंयुतात्~।\\
मृगादौ प्रोजझ्य भगणान्मध्यान्हेऽर्कः स्फुटो भवेत्~॥~१९~॥}
%\vspace{2mm}
\end{quote}
\begin{sloppypar}
 स्वदेशाक्षांशेष्टदिनीयमध्यान्हसूर्यनतांशयोर्भागानां बहुत्वाद्बहुवचनम् । एकदिक्त्वेऽन्तरमन्यदिक्त्वेऽन्यथा योगः कार्यः । शेष उक्तसंस्कारसिद्धोऽङ्कः क्रान्तिः स्यात् । तस्यापक्रमस्य ज्या
\end{sloppypar}

\newpage


\hspace{3cm} गूढार्थप्रकाशकेन सहितः~। \hfill ११९
\vspace{1cm}

\begin{sloppypar}
\noindent त्रिज्यया गुण्या परमक्रान्तिज्यया प्रागुक्तया भक्ता फलस्य धनुर्भागादिकं मेषादिगो मेषादिराशित्रितयान्तर्गतोऽर्क: स्यात् । कर्कादित्रयेऽर्के चक्रार्धात् षड्राशित आगतार्कं त्यत्क्वा शेषं मध्यान्हकाले स्फुटोऽर्क: स्यात् । तुलादित्रितये षड्भयुतादागतार्कात् स्फुटोऽर्को ज्ञेयः । आगतोऽर्कः षड्भयुतः स्फुटोऽर्कः स्यादित्यर्थः । मकरादित्रयेऽर्के द्वादशराशिभ्य आगतार्कं त्यत्क्वा शेषमयनांशसंस्कृतः स्फुटोऽर्कः स्यात् । करणागतज्ञानार्थं व्यस्तायनांशसंस्कृत इत्यर्थसिद्धम् । पूर्वं तत्संस्कृतग्रहात् क्रान्ति: साध्येत्यर्थस्योक्तेः । अत्रोपपत्तिः । एकदिशि क्रान्त्यक्षयोगान्नतं दक्षिणमतोऽक्षोनं क्रान्तिर्दक्षिणा । भिन्नदिशि कान्त्यूनाक्षो नतं दक्षिणमनेनाक्षो हीनः क्रान्तिरुत्तरा । अक्षोनक्रान्तिर्नतं तूत्तरमतोऽक्षयुतं क्रान्तिरुत्तरा । अस्य ज्या क्रन्तिज्या । परमक्रान्तिज्यया त्रिज्याभुजः स्यात् तदानया केतोष्टा सायनार्कभुजज्या तद्धनुः सायनार्कभुजः । भुजस्य चतुर्षु पदेषु तुल्यत्वात प्रथमपदे मेषादित्रये सूर्यस्यैव भुजत्वाद्भुज एव सूर्यः। कर्कादित्रये  द्वितीयपदे षड्भादूनस्यार्कस्य भुजत्वाद्भुजोनषड्भमर्कः । एवं तृतीयपदे तुलादित्रये षड्भेन हीनार्कस्य भुजत्वात् षड्युतो भुजोऽर्कः । चतुर्थपदे मकरादित्रये सूर्योनभगणस्य भुजत्वाद्भुजोनभगणोऽर्क इति सर्वं वैपरीत्यात् सुगमतरम्~॥~१९~॥\\
\noindent अथागतस्फुटसूर्यस्य करणागतस्फुटतुल्यत्वज्ञानभागतस्फुटसूर्यान्मध्यमस्य करणागतमध्यमार्कतुल्यत्वेन विशेषं वक्तुं श्लोकार्धेनाह\textendash
\end{sloppypar}

\newpage


\noindent १२० \hspace{4cm} सूर्यसिद्धान्तः
\vspace{1cm}
\begin{quote}

  {\ssi तन्मान्दमसकृद्वामं फलं मध्यो दिवाकरः~।}
%\vspace{2mm}
\end{quote}
\begin{sloppypar}
 तस्मादागतस्फुटसूर्यान्मान्दं फलं मन्दफलमसकृदनेकवारं वामं व्यस्तं संस्कृतं स्फुटसूर्येऽहर्गणानीत स्फुटसूर्यः स्यात् । अयमर्थः । स्फुटसूर्यं मध्यमं प्रकल्प्य पूर्वमन्दोच्चात् प्रागुक्तरीत्या मन्दफलं धनमृणमानीय स्फुटसूर्य ऋणं धनं कार्यं मध्यमसूर्यः । अस्मादपि मन्दफलं स्पष्टसूर्ये व्यस्तं संस्कृतं मध्यमोऽस्मादपि मन्दफलं स्पष्टे व्यस्तं मध्यमार्क इति यावदविशेषस्तावदसकृत् साध्योऽर्को मध्याऽहर्गणानीतो भवतीति । तथा च मध्यमार्कात् स्फुटार्कसाधन एकवारं मन्दफलसंस्कारः स्फुटार्कान्मध्यार्कसाधने त्वनेकवारं मन्दफलव्यस्तसंस्कार इति विशेषोऽभिहितः । अत्रोपपत्तिः । मध्यमसूर्यादानीतमन्दफलेन संस्कृतो मध्यःस्फुटोऽर्को भवति । अयं वा तेनैव मन्दफलेन व्यस्तं संस्कृतो मध्यो भवति । अत्र स्फुटार्कान्मध्यार्कसाधने मध्यमज्ञानासम्भवात् तदानीतमन्दफलज्ञानमशक्यमतः स्फुटसूर्यं मध्यमं प्रकल्प्यानीतमन्दफलेनाभिमतासन्नेन स्फुटोऽर्को व्यस्तं संस्कृतो मध्यासन्नः । अस्मादपि मन्दफलमभिमतासन्नमपि पूर्वस्मात् सूक्ष्ममिति यावदविशेषे मध्यार्कसाधितं मन्दफलं भवतीति निरवद्यं सर्वमुक्तम्~॥ अथ मध्यान्हे छायाकर्णयोरानयनं विवक्षुः प्रथमं तात्कालिकनतांशज्ञानं कथयंस्तद्भुजकोटिज्ये कार्ये इत्याह\textendash
\end{sloppypar}
%\vspace{2mm}
\begin{quote}

 {\ssi स्वाक्षार्कापक्रमयुतिर्दिक्साम्येऽन्तरमन्यथा~॥~२०~॥

 शेषं नतांशाः सूर्यस्य तद्वाहुज्या च कोटिजा~।}
\end{quote}

दिक्साम्य एकदिक्त्वे स्वदेशाक्षांशमध्यान्हकालिकसूर्यक्रा\textendash

\newpage



\hspace{3cm} गूढार्थप्रकाशकेन सहितः~। \hfill १२१
\vspace{1cm}

\begin{sloppypar}
\noindent न्त्यंशयोर्योगः । अन्यथा अत उक्तादेकदिक्त्वाद्वैपरीत्ये भिन्नदिक्त्व इत्यर्थः । अक्षांशक्रान्त्यंशयोरन्तरं कार्यं शेषं संस्कारोत्पन्नं सूर्यस्य मध्याह्ने नतांशास्तेषां नतांशानां भुजरूपाणां ज्या कोटिजा तदंशा नवतिशुद्धाः कोटिस्तत उत्पन्ना ज्या चः समुच्चये साध्या । अत्रोपपत्तिः । याम्योत्तरवृत्ते सूर्यस्य मध्यान्हे स्वस्वस्तिकादन्तरं नतांशा विषुवद्दृत्तपर्यन्तमक्षांशाः । विषुवद्दृत्तसूर्ययोरन्तरं क्रान्यंशा: । अतो दक्षिणक्रान्तौ क्रान्त्यक्षयोगो नतांशा उत्तरक्रान्तौ क्रासुनाक्षोऽक्षोनक्रान्तिर्वा दक्षिणोत्तरनतांशास्तेषां ज्या दृग्ज्या भुजस्तत्कोटिज्या महाशङ्कुः कोटिस्त्रिज्या कर्ण इति छायाक्षेत्रे तदंशानां भुजत्वात्~॥~२०~॥\\
\noindent अथ छायाकर्णयोरानयनमाह\textendash
\end{sloppypar}
%\vspace{3mm}
\begin{quote}

 {\ssi शङ्कुमानाङ्गुलाभ्यस्ते भुजत्रिज्ये यथाक्रमम्~॥~२१~॥

 कोटिज्यया विभज्याप्ते छायाकर्णावहर्दले~।}
 \end{quote}
%\vspace{2mm}

\begin{sloppypar}
 भुजत्रिज्ये नतांशज्यात्रिज्ये इत्यर्थः । शङ्कोः प्रमाणाङ्गुलानि द्वादश तैर्गुणिते कार्ये । उभयत्र कोटिज्यया नतांशोननवत्यंशानां ज्ययेत्यर्थः । भक्त्वा लब्धे द्वे यथाक्रमं भुजज्यात्रिज्यास्थानीयफलक्रमेण मध्याह्ने छायातत्कर्णौ भवतः । अत्रोपपत्तिः । द्वादशाङ्गुलशङ्कुः कोटिरिष्टच्छायाभुजस्तत्कृत्योर्योगपदं कर्ण इति छायाकर्णः कर्ण इति छायाक्षेत्रे । महाशङ्कुकोटौ दृग्ज्यात्रिज्ये भुजकर्णौ तदा द्वादशाङ्गुलशङ्कुकोटौ कावित्यनुपातेन मध्याह्नकाले छायातत्कर्णौ भवतः । साधकयोस्तात्कालिकत्वादित्युपपन्नम्~॥~२१~॥\\
 \noindent अथ भुजसाधनं विवक्षुः प्रथममग्रां कर्णाग्र आनयति\textendash
\end{sloppypar}

 {\tiny{R}}

\newpage

%\includegraphics[width=1.\theta4167in,height=1.\theta4167in]Images/Inserted/86\theta\thetaculture-removebg-preview (1)(1).jpg

\noindent १२२ \hspace{4cm} सूर्यसिद्धान्तः 
\vspace{1cm}


\begin{quote}
 {\ssi कान्तिज्या विषुवत्कर्णगुणाप्ता शङ्कुजीवया~॥~२२~॥
 
 अर्काग्रा स्वेष्टकर्णघ्नी मध्यकर्णोद्धृता स्वका~।}
 \end{quote}
%\vspace{2mm}

\begin{sloppypar}
 सूर्यक्रान्तिज्या अक्षकर्णगुणिता शङ्कुजीवया शङ्कुर्द्वादशाङ्गुलस्तद्रूपा ज्या तयेत्यर्थः । द्वादशभिरिति फलितम् । भक्ता फलं सूर्यस्याग्रा । उपलक्षणाद्ग्रहस्यापि इयमग्रा स्वाभिमतकालिकछायाकर्णेन गुणिता मध्यकर्णोद्धृता कर्णस्य व्यासस्य मध्यमर्धमिति मध्यकर्ण व्यासार्थं त्रिज्या तयेत्यर्थः । पूर्वीपरप्रथमचरमजघन्यसमानमध्यमध्यमवीराश्चेति सूत्रेण मध्यपदस्य पूर्वनिपातः। भक्ता फलं स्वका स्वकर्णाग्रा स्यात् । अत्रोपपत्तिः। क्रान्तिज्योन्मण्डले कोटिरग्रा क्षितिजे कर्णः कुज्याभुज इत्यक्षक्षेत्रे द्वादशकोटावक्षकर्णः कर्णस्तदा क्रान्तिज्याकोटौ कः कर्ण इत्यनुपातेनाग्रा । त्रिज्यावृत्त इयं कर्णवृत्ते केत्यनुपातेन कर्णवृत्ताग्रेत्युपपन्नम्~॥~२२~॥\\
 \noindent अथ भुजानयनं श्लोकाभ्यामाह\textendash
\end{sloppypar}
%\vspace{2mm}
\begin{quote}

 {\ssi विषुवङ्भायुतार्कग्रा याम्ये स्यादुत्तरोभुजः~॥~२३~॥
 
 विषुवत्यां विशोध्योदग्गोले स्याद्वाहुरुत्तरः~।\\
 विपर्ययाङ्भुजो याम्यो भवेत् प्राच्यपरान्तरे~॥~२४~॥
 
 माध्यान्हिको भुजो नित्यं छाया माध्यान्हिकी स्मृता~।}
 \end{quote}
%\vspace{2mm}

\begin{sloppypar}
 अर्काग्रा सूर्यस्याभीष्टकालिककर्णाग्रा याम्ये दक्षिणगोले विषूवङ्भायुताक्षच्छायया युक्तोत्तरदिक्को भुजः स्यात् । उत्तरगोले विषुवत्यां पलभायां कर्णाग्रां विशोध्य न्यूनीकृत्य शेषमुत्तरदिक्को
\end{sloppypar}

\newpage


\hspace{3cm}गूढार्थप्रकाशकेन सहितः~। \hfill १२३
\vspace{1cm}

\begin{sloppypar}
\noindent भुजः स्यात्। ननु कर्णाग्रा पलभायां यदा न शुद्ध्यति तदा कथं भुजः साध्य इत्यत आह\textendash विपर्ययादिति~। अक्षभां कर्णाग्रायां विशोध्य शेषं दक्षिण भुजः स्यात्। ननु भुजस्य याम्यत्वमुत्तरत्वं वा कस्मादित्यत आह\textendash प्राच्यपरान्तर इति~। पूर्वापरसूत्रादन्तरालप्रदेशे याम्य उत्तरो वा भुजः स्यादित्यर्थः। ननु तथापि द्वितीयावधेरनुक्तत्वादन्तरस्याप्रसिद्धेः पूर्वापरसूत्रात् कस्यान्तरं भुज इत्याशङ्काया उतरं मध्याह्नच्छायास्वरूपकथगच्छलेनाह\textendash माध्याह्निक इति~। मध्यान्हाकालिको भुजः सदा माध्यान्हिकी मध्यान्हकालिकी छायोक्ता । तथा च छायाग्रं प्राच्यपरसूत्राद्याम्यमुत्तरं वा यदन्तरेण स भुज इति व्यक्तीकृतम्। अत्रोपपत्तिः। शङ्कुमूलं प्राच्यपरसूत्राद्याम्यमुत्तरं वा यदन्तरेण स याम्योत्तरो भुजो ग्रहस्य। शङ्कुस्तु ग्रहादवलम्बसूत्र्ं क्षितिजसमसूत्रावधि तत्रायं भुजः शङ्कुतलाग्रयोः संस्कारजः। शङ्कुतलं तु स्वाहोरात्रवृत्तस्थितोदयास्तसूत्राच्छङ्कुमूलं यदन्तरेण तद्दक्षिणम्। अग्रा तु पूर्वापरसूत्रादुदयास्तसूत्रावध्यन्तरमुत्तरदक्षिणगोलक्रमेणोत्तरदक्षिणा । तत्र ग्रहापरदिशि षङ्भान्तरेऽस्माध्यस्तमिति शङ्कुतलमुत्तरमग्रापि व्यस्तदिक्केति तत्संस्कारो भुजो गोले प्रत्यक्षः। स महाशङ्कोरिति महाशङ्कोरयं तदा द्वादशाङ्गुलशङ्कोः क इत्यनुपातेन भुजः पूर्वीपरसूत्राच्छायाग्रावधि। तत्र शङ्कुतलाग्रे द्वादशाङ्गुलशङ्कोः साधिते तत्संस्कारेण भुजः स एव। तत्राप्यग्रा पूर्व साधिता शङ्कुतलं तु द्वादशाङ्गुलशङ्को: पलभा महाशङ्कु: कोटिः शङ्कुतलं भुजो हृतिः कर्ण इत्यक्षक्षेत्रे द्वाद\textendash
\end{sloppypar}

{\tiny{R2}}


\newpage


\noindent १२४ \hspace{4cm} सूर्यसिद्धान्तः
\vspace{1cm}

\begin{sloppypar}
\noindent शकोटौ पलभाभुजस्तदा महाशङ्कुकोटौ को भुज इत्यनुपातेन शङ्कुतलमानीय महाशङ्कोरियं द्वादशाङ्गुलशङ्कोः किमित्यनुपाते गुणहरयोस्तुल्यत्वान्नशेन पलभाया एवावशिष्टत्वात् । सा तूत्तरा दक्षिणगोलेऽग्राया उत्तरत्वादेकदिक्त्वेन पलभाग्रयोर्योग उत्तरो भुजः । उत्तरगोलेऽग्राया दक्षिणत्वेन भिन्नदिक्त्वात् पलभाग्रयोरन्तरं भुजस्तत्र पलभायाः शेषमुत्तरो भुजोऽग्रायाः शेषं दक्षिणो भुजः । मध्याह्ने छायाया भुजरूपत्वान्मध्याह्नकालिको भुजो मध्याह्नच्छायातो सर्वं युक्तम्~॥~२४~॥\\
\noindent अथ याम्योत्तरवृत्तस्थच्छाथाकर्ण मुक्का पूर्वापरतखच्छायाकर्ण प्रकारयेनाह\textendash
\end{sloppypar}
%\vspace{2mm}
\begin{quote}

  {\ssi लम्बाक्षजीवे विषुवच्छायाद्वादशसङ्गुणे~॥~२५~॥

 क्रान्तिज्याप्ते तु तौ कर्णौ सममण्डलगे रवौ~।}
 \end{quote}
%\vspace{2mm}

\begin{sloppypar}
लम्बज्याक्षज्ये क्रमेणाक्षभाद्वादशाभ्यां गुणिते उभयत्र क्रान्तिज्यया भक्ते तुकारात् फले समवृत्तस्थेऽर्के तो दृग्योग्यछायासम्बद्धौ कर्णौ भवन उभयत्र छाथाकर्णः स्यात् । अत्रोपपत्तिः । खमस्तकोपरि पूर्वापरानुकारेण यदृत्तं तत्सममण्डल सञ्जम् । तत्रस्थस्य छायाकानयनम् । पलभाभुजेऽक्षकर्णः कर्णस्तदा क्रान्तिज्याभुजे कः कर्ण इति समशङ्कुः क्रान्तिज्याभुजे समशङ्कुः कुज्यानतद्धृत्याः क्रमेण कर्णकोटित्वात् । अस्माच्छङ्कुःमानाङ्गुलाभ्यस्ते इत्यादिना त्रिज्या द्वादशगुणितानेन भक्ता तत्र ।
\end{sloppypar}


%\includegraphics[width=1.\theta4167in,height=1.\theta4167in]Images/Inserted/86\theta\thetaculture-removebg-preview (1)(1).jpg

\newpage

\hspace{3cm} गूढार्थप्रकाशकेन सहितः~। \hfill १२५
\vspace{1cm}
\begin{quote}

 {\ssi छेदं लवं च परिवर्त्य हरस्य शेषः
 
कार्योऽत्र भागहरणे गुणनाविधिश्च~।}
\end{quote}
%\vspace{2mm}

\begin{sloppypar}
 इत्युक्तेः। पलभयापि गुण्या क्रान्तिज्याक्षकर्णाभ्यां भक्ता । तत्र त्रिज्या द्वादशभुणिताक्षकर्णभक्ता लम्बज्यैव सिद्धातो लम्बज्या पलभागुणिता क्रान्तिज्याभक्ता फलं समवृत्तगतच्छायाकर्णः । अथात्रैव पलभाभुजे द्वादशकोटिरक्षज्याभुजे का कोटिरिति लम्बज्याग्रहणे पलभयोस्तुल्यत्वान्नाशादक्षज्या द्वादशगुणा क्रान्तिज्या भक्ता छायाकर्णः सममण्डलगतः । क्रान्तिज्याया: मदायं कर्ण: सिद्धोन्न हि सर्वदा समवृत्तगतो ग्रह इति समवृत्तगतग्रहस्यैव कर्णः साध्यो नान्यदेति सूचनार्थं सममण्डलगे रवावित्युक्तम्~॥~२५~॥\\
 \noindent ननु ग्रहाधिष्ठिता होरात्रपूर्वापरवृत्तसम्पातादवलम्बरूपममशङ्कोर्गोले प्रत्यक्षसिद्धस्य साधनार्थं समवृत्तस्थत्वाभावेऽपि छायाकर्णः साध्यः । सममण्डलगे रवावित्युक्तिस्तु स्वाधिष्ठिताहोरात्रवृत्तपरा न त्वन्यदा न साध्योऽन्यथालक्ष्यत्वेन प्रकारस्थातिप्रसङ्गापत्तेः । न हि प्रकारे तड्यावर्तकं विशेषणं प्रसिद्धं येन नातिप्रसङ्गः । परन्तु यदा सममण्डलेऽक्षांशाधिकक्रान्त्या ग्रहाधिष्ठितधुरात्रवृत्तानामसम्बन्धस्तदा गोले समशङ्कोरदर्शनात् तत्र कथं तत्साधनमनिवारितमित्यतः सममण्डलगे रवावित्यस्य पूर्वोक्त एवार्थ इत्यभिप्रायं सममण्डलकर्णानयनप्रकारान्तरकथनच्छलेनाह\textendash
\end{sloppypar}
%\vspace{2mm}
\begin{quote}

 {\ssi सौम्याक्षोना यदाक्रान्तिः स्यात् तदा द्युदलश्रवः~॥~२६~॥
 
 विषुवच्छाययाभ्यस्तः कर्णो मध्याग्रयोद्धृतः~।}
\end{quote}

\newpage


%\includegraphics[width=1.\theta4167in,height=1.\theta4167in]Images/Inserted/86\theta\thetaculture-removebg-preview (1)(1).jpg


\noindent १२६ \hspace{4cm} सूर्यसिद्धान्त 
\vspace{1cm}

\begin{sloppypar}
 यदोत्तरा क्रान्तिरक्षादल्पा स्यात् तदा द्युदलश्रवः समवृत्तस्थार्कक्रान्तिसाधितमध्याह्नकर्णः न तु मध्याह्नकालिकः । अक्षभया गुणितो मध्याग्रया गृहीतमध्याह्नकर्णाग्रया भक्तः फलं सममण्डलगतग्रहबिम्बस्य छायाकर्ण: स्यात् । अत्र सौम्येत्यनेन दक्षिणक्रान्तौ तदसाधनं सममण्डलगतग्रहबिम्बस्यादर्शनादिति स्फुटमुक्तम् । अन्यथाक्षाल्पक्रान्तौ दक्षिणगोले समशङ्कोः प्रत्यक्षत्वात् तन्निवारणानुपपत्तेः । अत्रोपपत्तिः । सममण्डलप्रवेशकालिकमध्याह्नाच्छाधाकर्णादवस्तुभूतात् कर्णेन द्वादशाङ्गुलङ्कुस्तदा त्रिज्याकर्णेन क इति मध्यशङ्कुस्तात्कालिकः । द्वादशकोटावक्षभाभुजस्तदा महाशङ्गुकोटौ क इति शङ्कुतलम् । द्वादशयोर्नाशात् पलभात्रिज्याघातो मध्यकर्णभक्त इति । अनेन भुजेन मध्यशङ्कुस्तदाग्राभुजेन क इति समशङ्कुआदशाग्रामध्यकर्णधाता मध्यकर्णपलभाभ्यां भक्तोऽग्राभुजे समशङ्कुतद्धृत्योः कोटिकर्णत्वात् । अस्मात् पूर्वप्रकारेण छायाकानयने द्वादशयोर्नाशान्मध्यकर्णपलभात्रिज्याघातोऽग्रामध्यकर्णाभ्यां भक्त इति तुल्ययोर्मध्यकर्णमितगुणहरयोर्नाशाकरणेन सिद्धम् । स्वतन्त्रेच्छस्य नियोक्तुमशक्यत्वात् तत्रापि भाज्यहरौ त्रिज्ययापवर्त्य हरस्थाने मध्यकर्णगुणिताग्रा त्रिज्याभक्तेति मध्यकर्णाग्रा सिद्धातो मध्याग्रयोद्धृत्त इत्युक्तम् । भाज्यस्थाने तु मध्यकर्णपलभाघात इति दक्षिणगोले ग्रहादर्शनान्न साधितः । उत्तरगोलेऽपि क्रान्तिरक्षाधिका तदा सममण्डलप्रवेशासम्भवान्न साधितः सममण्डलावध्यक्षांशत्वात् । अल्पक्रान्तौ तत्सम्भवात
\end{sloppypar}



%\includegraphics[width=1.\theta4167in,height=1.\theta4167in]Images/Inserted/86\theta\thetaculture-removebg-preview (1)(1).jpg
\newpage

\hspace{3cm} गूढार्थप्रकाशकेन सहितः~। \hfill १२७
\vspace{1cm}


\noindent साधितः। नह्यसिद्धं गोले गणितसाध्यं मानाभावादित्युपपन्नं
सौम्येत्यादि । भास्कराचार्यैस्तु\textendash


\begin{quote}
 {\qt मार्तण्डः सममण्डलं प्रविशति स्वल्पेऽपमे स्वात् पलात्\\
 दृश्यो ह्युत्तरगोल एव स विशन साध्या तदैवास्य भा~।\\
 अप्राप्तेऽपि समाख्यमण्डलमिने यः शङ्कुरुत्पद्यते\\
 नूनं सोऽपि परानुपातविधये नैवं क्वचिदुष्यति~॥}
 \end{quote}
 इत्यनेन तत्रापि साधितः~॥~२६~॥\\
 \noindent अथ स्वाभिमतकर्णेन
 

\noindent स्वस्वकाले भुजार्थं कर्णदृत्ताग्रा साध्येति सूचनार्थं कर्णाग्रामुक्तप्रकारेण पुनरपि मध्यकर्ण इति प्रागुक्तस्य स्फुटोकरणार्थं चाह\textendash

%\vspace{2mm}


\begin{quote}
  {\ssi स्वक्रान्तिज्या त्रिजीवाघ्नी लम्बज्याप्ताग्रमौर्विका~॥~२७~॥

 स्वेष्टकर्णहता भक्ता त्रिज्ययाग्राङ्गुलादिका~।}
\end{quote}

\begin{sloppypar}
 स्वाभिमतकालिकक्रान्तिज्या ज्या त्रिज्यया गुणिता लम्बज्यया भक्ता फलमग्रा ज्यारूपा । लम्बज्याकोटौ त्रिज्याकर्णः क्रान्तिज्याकोटौ कः कर्ण इत्यग्रेत्युपपत्तिः उत्तरार्धं पुनरुक्तं व्याख्यातप्रायम् । यदितु पूर्वोक्तकर्णवृत्ताग्रानयनश्लोके शङ्कुजीवयेत्यस्य शङ्कोः कोटिरूपत्वात् पूर्वसाधितमतांशभुजकोटिज्ययेत्यर्थो मध्यकर्ण इत्यस्य च तात्कालिकमध्यान्हच्छायायाः कर्णस्तदा न पुनरुक्तम् । परन्त्वर्काग्रेत्यस्य तात्कालिकमध्यान्हकालिककर्णाग्रार्थः स्वकेत्यस्य च स्वाभीष्टकालिककर्णाग्रार्थो बोध्यः । एतदुपपत्तिस्तु द्वादशकोटावक्षकर्णः कर्णस्तदा क्रान्तिज्याकोटौ कः कर्ण इति स्वकालिकाग्रा । त्रिज्यावृत्त इयं तदा तात्कालि\textendash
\end{sloppypar}




%\includegraphics[width=1.\theta4167in,height=1.\theta4167in]Images/Inserted/86\theta\thetaculture-removebg-preview (1)(1).jpg

\newpage

\noindent १२८ \hspace{4cm} सूर्यसिद्धान्तः
\vspace{1cm}

\begin{sloppypar}
कमध्यान्हकालिकच्छायाकर्णेन नतांशकोटिज्याभक्तद्वादशत्रिज्याघातात्मकेन केति द्वादशत्रिज्याघातयोर्गुणहरत्वेन तुल्ययोर्नाशादक्षकर्णगुणितक्रान्तिज्या तात्कालिकमध्याह्वनतांशकोटिज्यया भक्तेति । तात्कालिकमध्याह्नच्छायाकर्णेनेयं कर्णाग्रा तदा स्वाभीष्टकालिकच्छायाकर्णन केति खकालिका कर्णाग्रेत्युपपन्ना । सूर्याधिष्ठिताहोरात्रवृत्तयाम्योत्तरवृत्तोर्ध्वसम्पाततात्कालिकमध्यान्हं परानुपातार्थं बोध्यम्~॥~२७~॥\\
\noindent अथ कोणच्छायाकर्णसाधनार्थं कोणशङ्कुदृज्ये श्लोकपञ्चकेनाह\textendash
\end{sloppypar}
\vspace{2mm}


\begin{quote}
 {\ssi त्रिज्यावर्गार्धतोऽग्रज्यावर्गोनाद्द्वादशहतात्~॥~२८~॥
 
 पुनर्द्वादशनिघ्नाच्च लभ्यते यत् फलं बुधैः~।\\
 शङ्कुवर्गार्धसंयुक्तविषुवद्वर्गभाजितात्~॥~२६~॥
 
 तदेव करणीनाम तां पृथक स्थापयेहुधः~।\\
 अर्कघ्नी विषुवच्छायाग्रज्यया गुणिता तथा~॥~३०~॥
 
 भक्ता फलाख्यं तद्वर्गसंयुक्तकरणीपदम्~।\\
 फलेन हीनसंयुक्तं दक्षिणेत्तरगोलयोः~॥~३१~॥
 
 याम्ययोर्विदिशोः शङ्कुरेवं याम्योत्तरे रवौ~।\\
 परिभ्रमन्ति शङ्कोस्तु शङ्कुरुत्तरयोस्तु सः~॥~३२~॥
 
 तत्त्रिज्यावर्गविश्लेषान्मूलं दृग्ज्याभिधीयते~।}
 \end{quote}
%\vspace{2mm}
 
 पूर्वप्रकारानीतैस्तात्कालिकाग्रज्याया न तु कर्णाग्रायाः पूर्वं कर्णस्यैवासिद्धेः । वर्गेण होनात् त्रिज्यावार्गार्धाद्द्वादशगुणात्


\newpage


\hspace{3cm}गूढार्थप्रकाशकेन सहितः। \hfill १२९
\vspace{1cm}

\begin{sloppypar}
पुनर्द्वितीयवारं दादशगुणात्। चः समुच्चये। तेन द्वादशगुणितस्य द्विधा स्थापननिरासाच्चतुश्चत्वारिंशदधिकशतगुणितादित्यर्थः। पृथग्गुणकोक्तिस्तु गुणनसुखार्थम्। शङ्कोर्द्दादशाङ्गुलात्मकस्य वर्गार्धेन द्विसप्रत्या यक्तेन पलभावर्गेण भाजिताद्बुधैर्गणितकर्तृभिर्यत्सङ्ख्यामितं फलं प्राप्यते तत्सङ्ख्यामितं करणी नाम सञ्ज्ञया करणी। तां करणी बुधो गणकः पृथगेकत्र स्थाने स्थापयेत्। ततो द्वादशगुणिता पलभाग्रज्यया पूर्वगृहीतया गुणिता तथा द्विसप्ततियुतेन पलभावर्गेण भक्ता लब्धं फलसञ्ज्ञं तस्य फलस्थ वर्गेण युतायाः करण्या मूलं दक्षिणोत्तरगोलयोः क्रमेण फलेनोनयुतम्। एवमुक्तप्रकारेण सिद्धः शङ्कुः शङ्कोर्गणितकर्तुः सकाशाद्दक्षिणोत्तरे सूर्ये परिभ्रमति सति तुकारः क्रमार्थे क्रमेण याम्ययोरुत्तरयोर्विदिशोराग्नेयनैर्ण्यत्योरीशानीवायव्योः कोणयोरित्यर्थः। द्वितीयतुकारः पूर्वापरदिने विभागक्रमार्थकत्वेन विदिशोरित्यत्रान्वेति तेन दिनपूर्वार्धे आग्नेयैशान्योर्दक्षिणोत्तरक्रमेण दिनापरार्धे नत्यवायव्योदक्षिणेत्तरक्रमेणेति फलितार्थः। स कोणसञ्ज्ञः शङ्कुः स्यात्। कोणशङ्कुत्रिज्ययोर्वर्गान्तरान्मूलं दृग्ज्योच्यते। अत्रोपपत्तिर्बीजैकवर्णमध्यमाहरणेन। तत्र
\end{sloppypar}


\begin{quote}
{\qt यावत्तावत् कल्प्यमव्यक्ताराशे\\
र्मानं तस्मिन् कुर्वतोद्दिष्टमेव~।\\
तुल्यौ पक्षौ साधनीयौ प्रयत्नात\\
त्यत्का क्षिप्ता वापि सङ्गुण्य भक्ता~॥}
\end{quote}
{\tiny{S}}



\newpage

%\includegraphics[width=1.\theta4167in,height=1.\theta4167in]Images/Inserted/86\theta\thetaculture-removebg-preview (1)(1).jpg

\noindent १३० \hspace{4cm} सूर्यसिद्धान्तः 
\vspace{1cm}

\begin{sloppypar}
 इत्युक्तेः समौ पक्षौ साध्यौ तदर्थं कोणशङ्कुमानम् । या १ द्वादशकोटौ पलभाभुजः शङ्कुकोटौ को भुज इति कोणशङ्कुतलम् । या. प. ११२~। अग्रया युतं दक्षिणगोले भुजः । या. प. १ अ -१२१२~। उत्तरगोलेऽग्रयान्तरितं भुजस्तच समवृत्तादुत्तरं शङ्कुतलोनाग्रा भुजः । या. प.१ं अ .१२१२~। समवृत्ताद्दक्षिणेऽयोनं शङ्कुतलं भुजः । या. प. १ं अ. १२१२~। कोणस्य दक्षिणोत्तरपूर्वापरसूत्रमध्यत्वाद्भुजतुल्यसमचतुरस्रे कर्णः स्वस्वस्तिकात कोणस्थसूर्यनतांशानां ज्या दृग्ज्येति भुजवर्गो द्विगुणो दृग्ज्यावर्गो दक्षिणगोले । याव. प. व १ या. प. अ .२४ अव. १४४७२~। उत्तरगोले । याव. पव. १ या. प. अ. २४ं अव १४४७२~। अयं कोणशङ्कु या १ वर्ग याव १ हीनत्रिज्यावर्गरूपदृग्ज्यावर्ग याव १ त्रिव १ सम इति पक्षौ समच्छेदीकृत्य छेदगमे पक्षयोः शोधनार्थं न्यासः ।
\end{sloppypar}
\vspace{2mm}

 दक्षिणगोले 
 $\left\{
\begin{tabular}{@{}l@{}}\mbox{ याव. पव १ या. प. अ २४ अव १४४ }\\
\mbox{याव ७२ या. त्रिव ७२ ।} १\end{tabular}
\right\}$

\vspace{2mm}

 उत्तरगोले 
 $\left\{
\begin{tabular}{@{}l@{}}\mbox{ याव. पब १ था. प. अ २४ं अव १४४ }\\
\mbox {याव ७२ या. त्रिव ७२} १\end{tabular}
\right\}$ अथ ।
\vspace{2mm}


\begin{sloppypar}
 एकाव्यक्तं शोधयेदन्यपक्षाद्रूपाण्यन्यस्येतरस्माच्च पक्षात् ।

इत्युक्तेनाव्यक्तपक्षेऽव्यक्तवर्गस्थाने द्विसप्ततिपलभावर्गयोगो यावत्तावद्वर्गगुणो व्यक्तस्थाने पलभाग्राचतुर्विंशतिघातो यावत्तावद्गुणे दक्षिणगोले धनमुत्तरगोल ऋणम् । रूपपक्षे तु चतुश्चत्वारिंशदधिकशतगुणितेनाग्रावर्गेण हीनो द्विसप्ततिगुणस्त्रिज्यावर्गस्तत्र द्विसप्ततिगुणस्त्रिज्यावर्गश्चतुश्चत्वारिंशदधिकशतगु\textendash
\end{sloppypar}


%\includegraphics[width=1.\theta4167in,height=1.\theta4167in]Images/Inserted/86\theta\thetaculture-removebg-preview (1)(1).jpg

\newpage

\hspace{3cm}  गूढार्थप्रकाशकेन सहितः । \hfill १३१
\vspace{1cm}

\begin{sloppypar}
\noindent णितेन त्रिज्यावर्गार्धेन तुल्यत्वात् तुल्यगुणलाघवार्थं तथैव धृतः । तत्राप्येकदैव गुणनार्थं त्रिज्यावर्गार्धमयावर्गेण हीनं चतुश्चत्वारिंशदधिकशतगुणमिति सिद्धम्। सार्धराशिज्याधिकाग्रायां तु त्रिज्यावर्गार्धेन हीनोऽग्रावर्गश्चतुश्चत्वारिंशदधिकशतगुण मृणम्।
\end{sloppypar}


\noindent अथ। %\hspace{3em} 
\begin{quote}
{\qt अव्यक्तवर्गादि यदावशेषं\\

पक्षौ तदेष्टेन निहत्य किञ्चित्~।\\
क्षेप्यं तयोर्येन पदप्रदः स्या\\
दव्यक्तपक्षोऽस्य पदेन भूयः~॥

व्यक्तस्य पक्षस्य समक्रियैव\\
मव्यक्तमानं खलु लभ्यते तत्~।}
\end{quote}
\begin{sloppypar}

इत्युक्तेः पक्षयोर्मूलार्थमव्यक्तवर्गाङ्केनापवर्तः कार्यः । वर्गाङ्कस्तु द्विसप्ततियुतः पलभावर्गस्तेनाषवर्तितेऽव्यक्तपक्षे प्रथमस्थाने यावत्तावद्वर्गः सिद्धः । द्वितीयस्थाने द्विमितगुणकस्य पृथक्करणादर्कघ्नी विषुवच्छायाग्रज्यया गुणिता तथा भक्ता फलाख्यमित्युक्त्या फलं द्विगुणं यावत्तावङ्गुणं दक्षिणोत्तरगोलक्रमेण धनमृणम् । रूपपक्षेऽपवर्तिते करण्याख्यं सार्द्धराशिज्यातोऽग्रायामूनाधिकायां धनमृणम् । ततोऽपि मूलार्थं पक्षयोरव्यक्ताङ्कार्धरूपफलस्य वर्गो योजितः । तत्राव्यक्तपक्षे योजनपूर्वकमूलग्रहणे प्रथमस्थाने यावत्तावत् । द्वितीयस्थाने फलं दक्षिणोत्तरगोलयोर्धनमृणम् । यथा । या १ फ १ । या १ फ १ । उत्तरगोलेऽव्यक्तस्यर्णत्वं वा । या १ फ १ । उभयथा मध्याव्यक्तनाशसम्भवात् । रूपपक्षे तु मूलग्रहणे तद्वर्गसंयुक्तकरणीपदमिति
\end{sloppypar}

 {\tiny{s2}}


%\includegraphics[width=1.\theta4167in,height=1.\theta4167in]Images/Inserted/86\theta\thetaculture-removebg-preview (1)(1).jpg

\newpage

\noindent १३२ \hspace{4cm} सूर्यसिद्धान्तः 
\vspace{1cm}

\begin{sloppypar}
 सार्धराशिज्यानधिकाग्रायामधिकायां तु करण्यूनस्य फलवर्गस्य मूलम् । तथा च त्रिज्यावर्गार्धतोऽग्रज्यावर्गोनादित्यत्र सार्धंराशिज्याधिकाग्रायामुक्तानुपपत्तावपि ।
\end{sloppypar}
%\vspace{2mm}


\begin{quote}
{\qt यत्र क्वचिच्छुद्धिविधौ यदेह\\
शोध्यं न शुद्ध्योद्विरीतशुद्ध्या~।\\
 विधिस्तदा प्रोक्तवदेव किन्तु\\
योगे वियोगः सुधिया विधेयः~॥}
\end{quote}
\begin{sloppypar}
 इति भास्करोक्तरीत्याग्रज्यावर्गोनादित्यत्राग्रावर्गेणाग्रावर्गाद्वा हीनादित्यर्थद्वयेन क्रमेण न्यूनाधिकाग्रासम्बन्धेन वा न क्षतिरिति ध्येयम् । अथ पुनः समशोधनार्थं पक्षयोर्न्यासः ।दक्षिणगोले
$\left\{
\begin{tabular}{@{}l@{}}\mbox{ या १ फ १}  \\ \mbox{ या ० प १} \end{tabular}
\right\}$  
करण्यूनफलवर्गपदस्य फलतो न्यूनत्वात् तत्पक्षयोरपि न्यासः।
$\left\{
\begin{tabular}{@{}l@{}}या १ फ १ \\ या ० प १\end{tabular}
\right\}$ अत्रैकाव्यक्तमित्यादिना।

 \end{sloppypar}
 
 \begin{sloppypar}
शेषाव्यक्तेनोद्धरेद्रूपशेषं व्यक्तं मानं जायतेऽव्यक्तराशेः ।

इत्यनेन च प्रथमस्थाने पदं फलेन हीनमित्युपपन्नम् ।
द्वितीयस्थाने पदेन हीनं फलमित्यृणं कोणशङ्कुर्भगवतायं नोक्तः । ऋणस्य स्थितिविपरीतत्वात् । न ह्यूर्ध्वगोले स्थितिविपरीतमधोगोलेऽदृश्यमपि दृश्यते येन तत्कथनमावश्यकम् । नाप्यधोगोले दृश्यत्वात् तत्कथनापत्तिः । ऊर्ध्वगालस्थस्य छायासाधकत्वेन साधनात् तत्र छायासम्भवादेवाप्रयोजकत्वात् । उत्तरगोले तु $\left\{
\begin{tabular}{@{}l@{}}\mbox{ या १ फ १ }\\ \mbox{ या १ं फ १ } \end{tabular}
\right\}$  वा  $\left\{
\begin{tabular}{@{}l@{}} \mbox {या १ प १} \\ \mbox {या १ प १} \end{tabular}
\right\}$  प्रथमस्थाने फलेन युतं

\noindent पदमुपपन्नम् । द्वितीयस्थाने फलेनोनं पदमित्यृणत्वान्नोक्तः । छायानुपयुक्तत्वात् । करण्यूनफलवर्गपदस्य फलतो न्यूनत्वात्
\end{sloppypar}

\newpage

%\includegraphics[width=1.\theta4167in,height=1.\theta4167in]Images/Inserted/86\theta\thetaculture-removebg-preview (1)(1).jpg

  
\hspace{3cm}  गूढार्थप्रकाशकेन सहितः~। \hfill १३३
\vspace{1cm}

\begin{sloppypar}
\noindent तत्यक्षयोरपि न्यासः $\left\{\begin{tabular}{@{}l@{}} \mbox { या १ फ १ं} \\ \mbox{या १ं फ १  } \end{tabular}
\right\}$ वा
$\left\{
\begin{tabular}{@{}l@{}}\mbox{या १ प १} \\ \mbox{ या १ प १}
\end{tabular}
\right\}$   अत्र प्रथमस्थाने पदेन युक्तं फलं कोणशङ्कुरुपपन्नः । द्वितीयस्थाने पदेन हीनं फलं कोणशङ्कुरिति तद्द्वयमुपपन्नम् । नन्विदं तत्रोर्ध्वलगोले दिनार्ध एव कोणशङ्कुद्वयं दृश्यत्वाद्भगवता कथमुपेक्षितमिति चेन्न । तत्र त्रिज्यावर्गार्धत इत्यत्र व्यस्तशोधनात् फलेन हीनसंयुक्त पदमित्यत्राप्युत्तरगोल एव हीनसंयुक्तमित्यस्यावृत्त्या फलं पदेन हीनसंयुक्तमित्यर्थसिद्धेर्भगवता तद्द्वयस्यानुपेक्षितत्वात् । समवृत्ताद्दक्षिणस्थत्वे कोणशङ्कुर्दिनपूर्वापरार्धक्रमेणाग्नेय्यां नेरृत्यां वोत्तरस्थत्वेनैशान्यां वायव्यां वा भवतीति सर्वमुपपन्नम् । अत्र बोजक्रियापपादकसूत्राणामुपपत्तिर्विस्तरभीत्या नोक्ता । सा त्वग्रजकृष्णदैवज्ञगुरुचरणरचितायां भास्करीयबीजटीकायां सम्यगुक्तावधेयेति । शङ्कुः कोटिस्त्रिज्याकर्णस्तद्वर्गान्तरपदं दृग्ज्या दृग्वृत्तनतांशानां ज्येति तत्त्रिज्यावर्गविश्लेषान्मूलं दृग्ज्येत्युपपन्नम्~॥~३१~॥\\
\noindent अथैतच्छायाछायाकर्णयोरानयनमाह\textendash
\end{sloppypar}
%\vspace{2mm}


\begin{quote}
 {\ssi स्वशङ्कुना विभज्याप्ते दृक्त्रिज्ये द्वादशाहते~॥~३२~॥
 
 छायाकर्णौ तु कोणेषु यथास्वं देशकालयोः~।}
 \end{quote}
%\vspace{2mm}

\begin{sloppypar}
 कोणीयदृग्ज्यात्रिज्ये द्वादश गुणे दृग्ज्यासम्बन्धिकोण शङ्कुना भक्त्वा लब्धे दृग्ज्यात्रिज्याक्रमेण छायाछायाकर्णौ स्तः । तुकारादेवं कोणेषु चतुर्यु देशकालयोः । यथास्वं स्वमनतिक्रम्येति यथास्वं यथादेशं यथाकालं छायाछायाकर्णौ साध्यौ । अय\textendash
\end{sloppypar}


\newpage


\noindent १३४ \hspace{4cm} सूर्यसिद्धान्तः
\vspace{1cm}

\begin{sloppypar}
\noindent मर्थः । कचिद्देश चतुर्षु कोणेषु क्वत्तिच्च कोणद्वये क्वत्तिच्च दिनार्ध एव कोणद्वय इत्यादि देशकालानुरोधेन यथायोग्यमिति। अत्रोपपत्तिः प्रागुक्ता स्पष्टा च~॥~३२~॥\\
अथ दिकप्रदेशसम्बन्धेन छायाकर्णावुक्त्वा कालसम्बन्धेन सार्धश्लोलाकाभ्यामाह\textendash
\end{sloppypar}
%\vspace{2mm}


\begin{quote}
{\ssi त्रिज्योदक्चरजायुक्ता याम्यायां तद्विवर्जिता~॥~३३~॥

अन्त्या नतोत्क्रमज्योना स्वाहोरात्रार्धसङ्गुणा~।\\
त्रिज्याभक्ता भवेच्छेदो लम्बज्याप्तोऽथ भाजितः ~॥~३४~॥

त्रिभज्यया भवेच्छङ्कुस्तद्वर्गं परिशोधयेत्~।\\
त्रिज्यावर्गात् पदं दृग्ज्या छायाकर्णौ तु पूर्ववत्~॥~३५~॥}
\end{quote}
%\vspace{2mm}

\begin{sloppypar}
उत्तरगोले चरोत्पन्नया ज्यथा चरज्येत्यर्थः । पूर्वचरानयने चरज्यायाश्चरजेति सञ्ज्ञोक्तेः। युक्ता त्रिज्यान्त्या स्यात्। याम्यगोले तया चरज्ययोना त्रिज्यान्त्या स्यात्। नतोत्क्रमज्योना सूर्योदयाद्दिनगतघट्यो दिनशेषघट्यो वा दिनार्धान्तर्गता उन्नतसञ्ज्ञास्ताभिरूनं दिनार्धं नतकालो घट्यात्मकस्तस्यासुभ्यो लिप्तास्तत्त्वयमैरित्यादिविधिना मुनयो रन्ध्रयमला इत्याद्युक्तोत्क्रमव्यापिण्डैर्ज्योत्क्रमज्या । पञ्चदशघट्याधिकनते तु पञ्चदशघट्यूननतस्य क्रमज्याखण्डैः क्रमज्या तया युक्ता त्रिज्योत्क्रमज्या भवति। तया हीनेत्यर्थः। स्वाहोरात्रार्धसङ्गुणा। गृहीतचरज्यासम्बन्ध्यहोरात्रवृत्तव्यासार्धं द्युज्या तया गुणिता त्रिज्यया भक्ता फलं छेदसञ्जः स्यात्। अथानन्तरं छेदो
\end{sloppypar}



%\includegraphics[width=1.\theta4167in,height=1.\theta4167in]Images/Inserted/86\theta\thetaculture-removebg-preview (1)(1).jpg

\newpage

\hspace{3cm} गूढार्थप्रकाशकेन सहितः~। \hfill १३५
\vspace{1cm}

\begin{sloppypar}
\noindent लम्बज्य या गणितस्त्रिज्यया भाज्यः फलमिष्टकाले शङ्कुः स्यात् । तस्य शङ्कोर्वर्गं त्रिज्यावर्गाच्छोधयेत् । शेषस्य मूलं दृग्ज्या । आभ्यां छायाकर्णौ तु पूर्ववत् । पूर्वोक्तरीत्या भवतः । अत्र छायाकर्णौ त्विति कोणच्छायाकर्णसाधनश्लोकान्तर्भागस्य ग्रहणात् तच्छ्लोकोक्तरीत्याभीष्टशङ्कुदृरज्याभ्यां छायाकर्णौ साध्यावित्युक्तम् । अत्रोपपत्तिः । याम्योत्तरवृत्तोर्ध्वभागग्रहाधिष्ठितधुरात्रवृत्तसम्पातात् क्षितिजद्युरात्रवृत्तसम्पातद्वयबद्धोदयास्तसूत्रक्षितिजसम्बद्धयाम्योत्तरवृत्तसूत्रसम्पातपर्यन्तमहोरात्रवृत्ते सूत्रं त्रिज्यानुरुद्धमन्त्या । सा तूत्तरगोले चरज्यायुता त्रिज्या दक्षिणगोले चरज्ययोना त्रिज्या । उन्मण्डलयाम्योत्तरसूत्रावध्यहोरात्रवृत्तयासार्धे त्रिज्यात्वात् । उम्मण्डस्तस्योन्तरदक्षिणक्रमेण क्षितिजादूर्ध्वाधःस्थत्वेन तद्याम्योत्तरसूत्रयोर्मध्ये चरज्यात्वाच्च । ग्रहाहोरात्रवृत्ते याम्योत्तराहोरात्रवृत्तसम्पातादुभयत्र नतघट्यन्तरेण स्थाने तत्सूत्रं नतकालस्य सम्पूर्णज्या । तन्मध्यादूर्ध्वसूत्रं शररूपं नतोत्क्रमज्या । तया हीनान्त्या ग्रहस्थानादहोरात्रवृत्त उदयास्तसूत्रपर्यन्तमृजुसूत्रं त्रिज्यानुरुद्धमिष्टान्त्या । तत्तुल्या याम्योत्तरोर्ध्वव्याससूत्रान्तर्गता सा द्युज्याप्रमाणसाधितेष्टहृतिः । द्युज्यागुणा त्रिज्याभका फलं छेदः । अस्मात् त्रिज्याकर्णे लम्बज्या कोटिस्तदेष्टहृतिकर्णे का कोटिरित्यनुपातेनेष्टशङ्कुं । अस्मादृरज्याच्छायातत्कर्ण उक्तरीत्या सिद्ध्यन्तीत्युक्तमुपपन्नम्~॥~३५~॥\\
\noindent अथ श्लोकत्रयेण छायाकर्णाभ्यां नतकालानयनमाह\textendash
\end{sloppypar}


\newpage

%\includegraphics[width=1.\theta4167in,height=1.\theta4167in]Images/Inserted/86\theta\thetaculture-removebg-preview (1)(1).jpg

\noindent १३६ \hspace{4cm} सूर्यसिद्धान्तः 
\vspace{1cm}


\begin{quote}
  {\ssi अभीष्टछाययाभ्यस्ता त्रिज्या तत्कर्णभाजिता~।\\
 दृग्ज्या तद्वर्गसंशुद्धात् त्रिज्यावर्गाच्च यत् पदम्~॥~३६~॥
 
 शङ्कुः सत्रिभजीवाघ्नः स्वलम्बज्याविभाजितः~।\\
 छेदः स त्रिज्ययाभ्यस्तः स्वाहोरात्रार्द्धभाजितः~॥~३०~॥
 
 उन्नतज्या तया हीना स्वान्त्या शेषस्य कार्मुकम्~।\\
 ऊत्क्रमज्याभिरेवं स्युः प्राक्पश्चार्धनतासवः~॥~३८~॥ }
 \end{quote}
 %\vspace{2mm}

\begin{sloppypar}
 अभीष्टकालिकच्छायया गुणिता त्रिज्या गृहीतच्छायायाश्छायाकर्णेन भक्ता फलं दृग्ज्या । दृग्ज्याया वर्गेण हीनात् त्रिज्यावर्गाद्यत्सङ्ख्यामितं मूलम् । चकारो यत्तदोर्नित्यसम्बन्धात् तच्छब्दपरः । अभीष्ट शङ्कुः । स दृष्टशङ्कुस्त्रिज्यया गुणितः स्वदेशीयलम्बज्यया भक्तः फलं छेदः । स छेदस्त्रिज्यया गुणितो द्युज्यया भक्त उन्नतकालस्य ज्या विलक्षणा । यद्धनुरुन्नतकालो न भवति । तयानीतयोन्नतज्यया हीना स्वान्त्या स्वद्युज्यासम्बद्धचरज्ययावगतान्त्या । अवशेषस्योत्क्रमज्याभिर्मुनयो रन्ध्रयमला इत्याधुक्तोत्क्रमज्या पिण्डैर्धनुः । अवशेषस्य त्रिज्याधिकत्वे तु यदधिकं तस्य क्रमज्यापिण्डैर्धनुश्चतुःपञ्चाशद्युक्तमुत्क्रमधनुर्भवति । एवं प्रकारेण सिद्धाङ्का दिनस्य पूर्वाधापरार्धयोर्नतकालासवो भवन्ति। अत्रोपपत्तिः । पूर्वोक्तव्यत्यासात् सुगमा । तत्र छेदस्त्रिज्यापरिणत इष्टान्त्या तस्या ज्यात्वासम्भवः । अवध्युदयास्तसूत्रस्याहोरात्रवृत्तव्याससूत्रत्वाभावादित्युन्नतज्याकारेण स्वल्पान्तरत्वेन दर्शनादुन्नतज्येत्युक्तम् । अत एव भास्कराचार्यैः । इष्टा\textendash
\end{sloppypar}



%\includegraphics[width=1.\theta4167in,height=1.\theta4167in]Images/Inserted/86\theta\thetaculture-removebg-preview (1)(1).jpg

\newpage

\hspace{3cm}  गूढार्थप्रकाशकेन सहितः~। \hfill १३७
\vspace{1cm}

\begin{sloppypar}
न्त्यकामुन्नतकालमौर्वोतुल्यां प्रकल्पे त्याद्युक्तम् । तद्धनुरसूनामुन्नतकालत्वापत्त्या तया हीनेत्यादिभागस्य व्यर्थत्वापत्तेरिति दिक~॥~३८~॥\\
अथैष्टकालिकाग्रया क्रान्तिज्याद्वारा सूर्यसाधनं सार्धश्लोकेनाह\textendash
\end{sloppypar}


\begin{quote}
{\ssi इष्टाग्राघ्नी तु लम्बज्या स्वकर्णाङ्गुलभाजिता~।\\
 क्रान्तिज्या सा त्रिजीवाघ्नो परमापक्रमोद्धता~॥~३९~॥
 
 तच्चापं भादिकं क्षेत्रं पदैस्तत्र भवो रविः~।}
 \end{quote}
%\vspace{2mm}

\begin{sloppypar}
 इष्टकालिककर्णाग्रया गुणिता लम्बज्या । तुकारादग्रज्याया निरासः । तात्कालिकच्छायायाः कर्णाङ्गुलसङ्ख्याभिर्भक्ता फलं क्रान्तिज्या । सा क्रान्तिज्या त्रिज्यया गुणिता परमक्रान्तिज्यया भक्ता फलस्य धनू राश्यादिकं क्षेत्रं स्थानं भुज इति यावत् । पदैश्चतुर्भिश्चिन्हज्ञातैस्तत्र पदे भव उत्पन्नः । यथोक्तरीत्या कर्कादौ प्रोज्झ्य चक्रार्धेत्याद्युक्त्या सूर्यः स्यात् । अत्रोपपत्तिः। कर्णाग्रे कर्णाग्रा लभ्यते त्रिज्याग्रे केत्यग्रा । त्रिज्याकर्णे लम्बज्या कोटिस्तदाग्राकर्णे का कोटिरित्यनुपातेन त्रिज्ययोस्तुल्ययोर्गुणहरयोर्नाशादिष्टकर्णाग्रागुणितलम्बज्या कर्णभक्ता क्रान्तिज्या । अस्याः सूर्यानयनं प्रागेवोक्तमिति पुनरुक्तत्वात् सुगमतरम्~॥~३८~॥\\
 \noindent अथ भाभ्रमणमाह\textendash
\end{sloppypar}
%\vspace{2mm}


\begin{quote}
  {\ssi इष्टेऽन्हि मध्ये प्राक् पश्चादृते बाहुत्रयान्तरे~॥~४०~॥
  
 नत्स्यद्वयात्तरयुतेस्त्रिस्पृक्सूत्रेण भाभ्रमः~।}
\end{quote}
 {\tiny{T}}


\newpage


\noindent १३८ \hspace{4cm} सूर्यसिद्धान्तः
\vspace{1cm}

\begin{sloppypar}
अभिमते दिवसे मध्ये पूर्वविभागे पश्चिमविभागे बाहुत्रयान्तरे पूर्वापरसूत्राङ्भुजत्रयान्तरे स्थाने धृते। अयमर्थः। पूर्वापरसूत्रस्य मध्यस्थानाङ्भुजाङ्गुलान्तरेण चिन्हमेकं द्वितीयं पूर्वविभागे पूर्वापरसूत्रात् कालान्तरीयभुजाङ्गुलान्तरेण चिन्हं तृतीयं पश्चिमविभागे पूर्वापरसूत्रादितरकालान्तरीयभुजाङ्गुलान्तरण चिह्नम्। एवमेकस्मिन् दिवसे कालत्रये स्वभुजान्तरेण पूर्वापरसूत्राच्चिन्हत्रये कृते सतीति। मत्स्यद्वयान्तरयुतेरव्यवहितचिन्हाभ्यां प्रत्येकं मत्स्यमुत्पाद्येति मत्स्यद्दयस्य प्रत्येकमुखपुच्छगतरूपमध्यसूत्रयो: स्वमार्गानुसारेण प्रसारितयोर्योगो यस्मिन् स्थाने तस्मादित्यर्थः। त्रिस्पृक्सूत्रेण। चिन्हत्रयलग्नतुल्यसूत्रमितेन व्यासार्धेन भाभ्रमश्छायामार्गमण्डलं भवति । प्रथमान्तिमकालान्तर्गतकालिकच्छायाग्रं तद्वृत्तपरिधौ भवतीत्यर्थः। अत्रोपपत्तिः। प्राच्यपरसूत्राङ्भुजान्तरे छायाग्रमिति छायाग्रत्रयं ज्ञात्वा तत्स्पृष्टपरिधिवृत्तस्य मध्यज्ञानार्थमव्यवतचिन्हदयमत्स्याभ्यामव्यवहितचिन्हमध्यस्य दक्षिणेत्तरसूत्रे भवतः। तत्र वृत्तपरिधिप्रदेशेभ्यः केन्द्रस्य तुल्यान्तरत्वेनाव्यवहितचिह्नमध्यस्थानस्थावश्यं परिधिसक्तत्वात् तत्सूत्रमपि केन्द्रे लग्नं भवति। एवं प्रत्येकाव्यवहितचिन्हमध्यसूत्रयोर्योगस्तद्वृत्तकेन्द्रं सिद्धम् । मध्यरेखाज्ञानार्थं मत्स्यद्वयं तत्केन्द्राद्वृत्तं भागत्रयस्प्रृग्भवतीति किं चित्रम्। यद्यपि छायाग्रस्य सूर्यचलनानुरोधेन चलनात् तस्य तु वृत्ताकारासम्भवात् प्रतिक्षणं द्युरात्रवृत्तभेदात्। अन्यथा क्रान्तिभेदानुपपत्तेरित्येकवृत्तपरिधौ छायाग्र\textendash
\end{sloppypar}



%\includegraphics[width=1.\theta4167in,height=1.\theta4167in]Images/Inserted/86\theta\thetaculture-removebg-preview (1)(1).jpg
\newpage

\hspace{3cm}  गूढार्थप्रकाशकेन सहितः~। \hfill १३९
\vspace{1cm}

\begin{sloppypar}
\noindent भ्रमणं न सम्भवति । अत एव भास्कराचार्यैर्भात्रितयाद्भाभ्रमणं न सदित्युक्तम् । तथापि साधितभाग्राणामवश्यमेकवृत्तस्थत्वसम्भवात् तदन्तर्वर्तिनां छायाग्राणां तत्परिधिस्यत्वं स्वल्पान्तरत्वादङ्गीकृत्य भगवता कृपालुना छायाग्रदर्शनं विनापि छायाग्रस्थानज्ञानमन्यकालिकच्छायाग्रस्थानयोदर्शनेनाभीष्टसमये मेघादिनाच्छादिते रवौ राश्यादिसूर्यज्ञानोपजीव्याग्राभुजादिज्ञानार्थमुक्तम् । बहुकालान्तरितभाग्रग्रहणे स्थूलम् । अल्पान्तरिते किञ्चित् सूक्ष्ममिति ध्येयम्~॥~४०~॥\\
\noindent अथ कालज्ञानमुक्त्वा तदुपजीवकफलादेशाद्युपयुक्तलग्नज्ञाविवक्षुस्तदुपयुक्तस्वोदयज्ञानार्थं मेषादित्रयाणां लङ्कोदयासुसाधनपर्वकतन्निबन्धनं श्लोकाभ्यामाह\textendash
\end{sloppypar}
%\vspace{2mm}


\begin{quote}
  {\ssi त्रिभद्युकर्णार्धगुणाः स्वाहोरात्रार्धभाजिताः~॥~४१~॥
  
 क्रमादेकद्वित्रिभज्यास्तच्चापानि पृथक् पृथक्~।\\
 स्वाधोधः परिशोध्याथ मेषाल्लङ्कोदयासवः~॥~४२~॥
 
 खागाष्टयोऽर्थगोऽगैकाः शरत्र्यङ्कहिमांशवः ~।}
 \end{quote}
%\vspace{2mm}

\begin{sloppypar}
एकद्वित्रिभज्याः । एकराशिज्याद्विराशिज्यात्रिराशिज्यास्त्रिराशिद्युज्यया गुण्याः क्रमात् स्वक्रान्तिज्यासम्बन्धिद्युज्याभिर्भाज्याः । फलानां धनूंषि भिन्नभिन्नस्थाने स्थाप्यानि । स्थानद्वयेस्थाप्यानीत्यर्थः । अनन्तरं स्वाधोऽधः । स्वादधोऽध एकराशिज्यासम्बन्धि फलं यथास्थितं ततः प्रथमफलं द्वितीयफलाङ्कि\textendash
\end{sloppypar}

{\tiny{T2}}

%\includegraphics[width=1.\theta4167in,height=1.\theta4167in]Images/Inserted/86\theta\thetaculture-removebg-preview (1)(1).jpg
\newpage


 \noindent १४० \hspace{4cm} सूर्यसिद्धान्तः
\vspace{1cm}


\begin{sloppypar}
\noindent तीयफलं तृतीयफलान्न्यूनीकृत्य पृथगनुक्तौ प्रथमफलं द्वितीयफलान्न्यूनं कृतं सद्दूयोः फलयोर्मार्जनात् तृतीये शोध्यासम्भवः। प्रथमस्य ज्ञानासम्भवश्चेति प्रथमद्वितीययोः पृथक्स्थापनमावश्यकम् । अत एव न त्रिधा पृथगित्युक्तम् । मेषात् । मेषमारभ्य राशित्रयाणां लङ्कोदयासवो भवन्ति । प्रथमफलं मेषस्योदयासवः । द्वितीयोनतृतीयफलं मिथुनस्योदयासव इत्यर्थः। नियतत्वात् तन्मानमाह\textendash खागाष्टय इति~। मेषमानं सप्ततियुतं षोडशशतं वृषमानं पञ्चोनमष्टादशशतं मिथुनमानं पञ्चत्रिंशदधिकमेकोनविंशतिशतमित्यर्थः । अत्रोपपत्तिः । सिद्धान्तशिरोमणौ\textendash
\end{sloppypar}
%\vspace{2mm}


\begin{quote}
 {\qt मेषादिजीवाः श्रुतयोऽपवृत्तै\\
तद्भूमिजे क्रान्तिगुणा भुजाः स्युः~।\\
 तत्कोट्यः स्वद्युनिशाख्यवृत्ते\\
व्यासार्द्धवृत्ते परिणामितानाम्~॥

 चापेषु ताखामसवस्ततो ये\\
तेऽधो विशुद्धा उदया निरक्षे~।}
\end{quote}
%\vspace{2mm}

\begin{sloppypar}
इति । तत्स्वरूपोक्त्या त्रिज्याकर्णे त्रिराशिद्युज्या कोटिस्तदैकद्वित्रिराशिज्याकर्णेषु का इत्यनुपातेन कोटयो द्युज्याप्रमाणेनाहोरात्रवृत्ते तदसुकरणार्थं त्रिज्याप्रमाणेन साध्या इति द्युज्याप्रमाणेनैतास्तदा त्रिज्याप्रमाणेन का इत्यनुपातेन त्रिज्ययोर्गुणहरयो स्तुल्यत्वेन नाशादेकादिराशि ज्यास्त्रि राशिद्युज्यया गुण्याः स्वद्युज्यया भक्ता इत्युपपन्नाः । आसां धनूंव्येकादिरा\textendash
\end{sloppypar}

%\includegraphics[width=1.\theta4167in,height=1.\theta4167in]Images/Inserted/86\theta\thetaculture-removebg-preview (1)(1).jpg

\newpage

\hspace{3cm}  गूढार्थप्रकाशकेन सहितः~। \hfill १४१
\vspace{1cm}

\begin{sloppypar}
शीनामुदयासवस्तत्र प्रत्येकराश्युदयासुज्ञानार्थं स्वाधोऽधः शोधनमित्युपपन्नं त्रिभद्युकर्णार्धगुणा इत्यादि लङ्कोदयासव इत्यन्तम् । अत्र लङ्कापदं निरक्षदेशपरं व्याख्येयम् । सर्वनिरक्षदेशे क्षेत्रसंस्थानस्योक्तस्य तुल्यत्वेनोकरीत्यान्यनिरक्षदेशे तत्सिद्धौ बाधकाभावात् । अन्यथा स्वनिरक्षदेशे तत्साधनार्थं ग्रहवद्देशान्तरसंस्कारकरणापत्तेः । निजोदयकरणार्थं स्वनिरक्षदेशीयानां चरसंस्कारस्य समनन्तरमेवोक्तत्वादिति दिक् । खागाष्टय इत्यादावुक्तप्रकारगणितकर्मैवोपपत्तिः~॥~४२~॥\\
\noindent अथैभ्यः स्वदेशोदयासून् श्लोकार्धेनाह\textendash
\end{sloppypar}
%\vspace{2mm}


\begin{quote}
 {\ssi स्वदेशचरखण्डोना भवन्तीष्टोदयासवः~॥~४३~॥}
 \end{quote}
%\vspace{2mm}

\begin{sloppypar}
 एते सिद्धाः । स्वकीयैर्देशसम्बन्धेन यान्युत्पन्नानि चरखण्डानि चरानयनप्रकारेणैकादिराशीनां चराण्यानीयोक्तरीत्या स्वाधोऽधः शोधितानि मेषादिमिथुनान्तानां राशीनां चरखण्डानि भवन्ति । तैरूनाः सन्त इष्टोदयासवश्चरखण्डसम्बन्धिदेशे मेषादित्रयाणामुदयासवो भवन्तीत्यर्थः । अत्रोपपत्तिः ।
\end{sloppypar}
%\vspace{2mm}


\begin{quote}
 {\qt मेषादेमिथुनान्तो नाडीभिस्तिथिमिताभिरुद्वृत्ते~।\\
 लगति कुजे तदधःस्थे प्रथमं ताभिश्चरोनाभिः~॥}
 \end{quote}
% \vspace{2mm}
 
 \begin{sloppypar}
 इति भास्करोक्त्या प्रत्येकोदयासुज्ञानं प्रत्येकचरेणेति । प्रत्येकचरं तु चरखण्डमित्युपपन्नम्~॥~४३~॥\\
 अथावशिष्टराशीनामुदयानाह\textendash
\end{sloppypar}

\newpage

\noindent १४२ \hspace{3cm} सूर्यसिद्धान्तः
\vspace{1cm}

\begin{quote}
{\ssi व्यस्ताव्यस्तैर्युताः स्वैः स्वैः कर्कटाद्यास्ततस्तयः*~।\\
उत्क्रमेण षडेवैते भवन्तीष्टास्तुलादयः~॥~४४~॥}
\end{quote}

\begin{sloppypar}
ततोऽनन्तरं मेषादिलङ्कोदयासवो व्यस्ता मिथनवृषमेषक्रमेण स्थापिताः स्वैः स्वैर्मेषादिचरखण्डक्रैस्त्रिभिर्व्यस्तैहदयक्रमेण स्थापितैर्युताः कर्कादयस्त्रयः कन्यान्ताः क्रमेण ज्ञातोदयासुमाना भवन्ति। एवं षणामुक्त्वावशिष्टानामुदयासुज्ञानमाह\textendash उत्क्रमेणेति~। एत उक्ता मेषादयः कन्यान्ताः षटमङ्ख्याका. उत्क्रमेण कन्यासिंहकर्काद्युत्क्रमेण । एवकारो मेषवृषादिक्रमनिरासार्थकः। तुलादयः षङ्रशय दृष्टा ज्ञातस्वदेशोदयासुमाना भवन्ति। तथा च कन्योदयस्तुलायाः। सिंहोदयो वृश्चिकस्य। कर्कोदयो धनुषः। मिथुनोदयो मकरस्य । वृषोदयः कुम्भस्य। मेषोदयो मीनस्येति सिद्धम्। अत्रोपपत्तिः\textendash
\end{sloppypar}
%\vspace{2mm}


\begin{quote}
{\qt कन्यान्ताद्धनुषोऽन्तस्तिथिमितनाडोभिरुद्वलये~।\\
लगति कुजे चोर्ध्वस्थे पश्चात् ताभिश्चराढ्याभिः~॥

तद्रहितैः खडताशैः कन्यान्तो वा झषान्तो वा~।\\
चरखण्डैरूनाढ्यास्तेन निरक्षोदयाः स्वदेशे स्युः~॥}
\end{quote}


\begin{sloppypar}
इति भास्करोक्त्या सुगमा~॥~४४~॥\\
\noindent अथाभीष्टकाले ऋणधनलग्नसाधनार्थं गतभोग्यासुनाह~॥ 
\end{sloppypar}


\begin{quote}
{\ssi गतभोग्यासवः कार्या भास्करादिष्टकालिकात्~।\\
स्वोदयासुहता भुक्तभोग्या भक्ताः खवन्हिभिः~॥~४५~॥}
\end{quote}
\noindent \rule{\linewidth}{.5pt}

\begin{center}
* कर्कटाद्याः पुनस्त्रय इति पाठान्तरम्।

+ भवन्तीष्टोदयासव इति वा पाठः।
\end{center}


%\includegraphics[width=1.\theta4167in,height=1.\theta4167in]Images/Inserted/86\theta\thetaculture-removebg-preview (1)(1).jpg

\newpage

 \hspace{3cm} गूढार्थप्रकाशकेन सहितः~। \hfill १४३
\vspace{1cm}

\begin{sloppypar}
 इष्टकाले चालनेन सञ्जातात सूर्याद्गतभोग्यासवः गतासवो भाग्यामसश्च साध्या: । कथं साध्या इत्यत आह\textendash स्वोदयासुहता इति~। भुक्तभोग्याः सूर्याक्रान्तराशेर्ये भुक्तभागाः सूर्यस्य भागाद्यवयवात्मका एते त्रिंशत: शुद्धा भाग्यभागा : । सूर्याक्रान्तराशेः स्वदेशोदयासुभिर्गुणितास्त्रिंशता भक्ता गतासवो भोग्यासवः क्रमेण भवन्ति । अत्रोपपत्तिः । यस्मिन् काले लग्नं साध्यं तस्मिन काले सूर्यः साध्योऽन्यथा तात्कालिकलग्नसिद्धिर्न स्यात् । अथैतदर्थ सूर्याक्रान्तराशेर्भुक्तासवो भोग्यासवश्च साध्याः । सूर्योदयात् तत्कालपर्यन्तं पूर्वाग्रिमकालयोस्तद्राशेर्लग्नत्वात् । अनन्तरं च राश्युदयासुगणनया लग्नज्ञानस्य सुशकत्वाच्च । अतस्त्रिंशद्भागैरुदयासवस्तदा भुक्तभोग्यभागैः क इति भुक्तभोग्यकालासवः । अत्रोदयकालासूनां सम्पातावधिराशिग्रहणेनोत्पन्नत्वात् सूर्योऽयनांशसंस्कृतो ग्राह्यः । अन्यथा सृर्याक्रान्तराशेरुक्तोदयसम्बन्धाभावादसङ्गततापत्तेः । अत एव
\end{sloppypar}
%\vspace{2mm}


\begin{quote}
 {\qt युक्तायनांशादपमः प्रसाध्यः\\
कालौ च खेटात खलु भुक्तभोग्यौ~।}
\end{quote}
%\vspace{2mm}

\begin{sloppypar}
इति भास्कराचार्योक्तं सङ्गच्छते । ननूक्तरीत्यौदयिकार्कादेव भुक्तभोग्यासवः साध्याः सूर्योदयात् तत्कालावधि तद्राशेर्लग्नत्वात् । नहीष्टकाले तद्राशिर्लग्नं येन तद्गतभोग्यासवः साधवः । नापि तात्कालिकार्कात् सूर्योदयावधिकास्ते तात्कालिकार्कस्य सूर्योदयकालिकत्वाभावात् । तत् कथं भगवता सर्वज्ञेन भास्करादिष्टकालिकादित्युक्तमिति चेत् । उच्यते । उदयानां ना\textendash
\end{sloppypar}


%\includegraphics[width=1.\theta4167in,height=1.\theta4167in]Images/Inserted/86\theta\thetaculture-removebg-preview (1)(1).jpg
\newpage

\noindent १४४ \hspace{4cm} सूर्यसिद्धान्त 
\vspace{1cm}

\begin{sloppypar}
\noindent क्षत्रत्वान्नाक्षत्रघट्यो ग्राह्यास्तास्त्वसिद्धाः । सर्वत्र साधितघटीनां सावनत्वात् । तासां नाक्षत्रीकरणमावश्यकमन्यथा तद्गणनानुपपत्तेः । तदर्थ ग्रहोदयप्राणहता इत्याद्युक्त्या षष्टिसावनघटीषु गतिकलोत्पन्नामोऽधिका नाक्षत्रत्वार्थं तदेष्टसावमघटीषु कियदधिकमित्यनुपातेनागतफलयुक्ताः सावनाः कार्याः तत्रागतफलस्य क्षेत्रावयवोदधासुभिरष्टादशशतकलास्तदा गतासुभिः का इत्यनुपातसिद्धाष्टादशशतोदयास्वोर्गुणहरयोस्तुल्यत्वेन नाशादवशिष्टचालनस्वरूपः सूर्ये योजितः । सावनास्त्वविकृता एव स्थिताः । तथा चेष्टकालिकोऽर्को यत्काले लमं तत्कालात् पूर्वगृहीतसावनघट्यो नाक्षत्रा एव भवन्तोति भगवता सम्यगुक्तम् । भास्करादिष्टकालिकादिति । अनेनैवाभिप्रायेण भास्कराचार्य्यैरप्युक्तम् ।
\end{sloppypar}
%\vspace{2mm}


\begin{quote}
 {\qt लग्नार्थमिष्टघटिका यदि सावनास्ता-\\
स्तात्कालिकार्ककरणेन भवेयुरार्क्ष्यः~।\\
 आर्क्ष्योदया हि सदृशीभ्य इहापनेया-\\
स्तात्कालिकत्वमथ न क्रियते यदार्क्ष्यः~॥}
\end{quote}
%\vspace{2mm}

 इति~॥~४५~॥\\
 अथाभीष्टघटिकाभ्य ऋणधनलग्नसाधनं श्लोकाभ्यामाह\textendash
%\vspace{2mm}


\begin{quote}
 {\ssi ओभष्टघटिकासुभ्यो भोग्यासून् प्रविशोधयेत्~।\\
 तद्वत् तदेष्यलग्नासृनेवं यातान् तथोक्रमात्~॥~४६~॥
 
 शेषं चेत् त्रिंशताभ्यस्तमशुद्धेन विभाजितम्~।\\
 भागहोनं च युक्तं च तल्लग्नं क्षितिजे तदा~॥~४७~॥ }
\end{quote} 

\newpage


 \hspace{3cm} गूढार्थप्रकाशकेन सहितः । \hfill १४५
\vspace{1cm}

\begin{sloppypar}
 अभीष्टकाले याः सूर्योदयघटिकास्तासामसुभ्यो भोग्यासून् शोधयेत् । तदनन्तरं तदेव्यलग्नासून् । सूर्याक्रान्तराशेरग्रिमराशय एव्यलग्नानि । तेषामुदयासूनपि तद्वत् क्रमेण शोधयेत् । एवमुक्तरीत्या शेषघटिकासुभ्यो यातान् भुक्तासून् भुक्तराश्युदयासूंश्च व्यस्तक्रमात् तथा शोधयेत् । यो राश्युदयो न शुद्ध्यति सोऽशुद्धस्तेन त्रिंशता गुणितं शेषं भक्तम् । चेदित्यनेन शेषाभावे क्रिया न कार्या शून्यफलसिद्धेरिति सूचितम् । फलेन भागादिना भुक्तसम्बद्धेन हीनं चकारादण्डद्धराशिमङ्ख्यामानं भोग्यसम्बद्धभागादि फलेन युक्तं चकारादन्तिमशुद्धराशिसङ्ख्यामानं तदागतराश्यादिमानसम्बन्धिसम्पातावधिकक्रान्तिवृत्तैकप्रदेशरूपं तदाभीष्टकाले क्षितिजे क्षितिजवृत्तपूर्वविभागे लग्नं समसूत्रसम्बन्धेन लग्नस्वरूपोक्त्याभष्टिकाले तल्लग्नं स्यादित्यर्थः । फलादेशार्थं ग्रहाणां रेवतीयोगतारासन्नावधितो ग्रहात् तत्यङ्गिघस्थलग्नस्यापि फलादेशार्थं तत एव समुचितं ग्रहणमित्यागतलग्नं सम्पातावधिकमयनांशैर्व्यस्तं संस्कुर्यादिति स्वतः सिद्धमिति नोक्तम् । न च पूर्वमेव सूर्यस्यायनांशसंस्कारानुक्त्या लग्नमपि यथास्थितमित्ययनांशव्यस्तसंस्कारोऽनुक्तः सङ्गत इति वाच्यम् । स्थूलत्वाल्लग्नार्थं सूर्येऽयनांशसंस्कार स्तस्य तत्संस्कृताद्ग्रहात् क्रान्तिच्छायाचरदलादिकमित्यत्रादिपदसङ्गृहोतत्वाच्च । अथ भगवतायनांशव्यस्तसंस्कारः कण्ठेन नोक्त दूति लग्नं सम्पातावधिकमेव फलादेशार्थं गृहीतम् । सूर्यस्य तु लग्नार्थमयनांशसंस्कारस्यावश्यकत्वात् । उदयानां सम्पातावधिकत्वादिति चेन्मैवम् ।
\end{sloppypar}

{\tiny{U}}

\newpage


\noindent १४६ \hspace{4cm} सूर्यसिद्धान्तः 
\vspace{1cm}


\begin{quote}
 {\qt भागहीनं च युक्तं च तल्लग्नं क्षितिजे तदा~।}
 \end{quote}
% \vspace{2mm}
 
 \begin{sloppypar}
इत्यर्धस्यावृत्त्याग्रिमश्लोकादिस्यग्राक्पश्चादित्यस्यावृत्त्या च प्राक्पश्चाच्चक्रचलने भागैरयनांशैः क्रमेण हीनं युक्तं लग्नं स्यादित्यर्थेन भगवतः कण्ठोक्तेः सिद्धत्वाच्च । अत्रोपपत्तिः अभीष्टघटिकासुभ्यो भाग्यगतासुशोधने सूर्याक्रान्तराशिर्लग्नं नेति ज्ञातम् । ततोऽग्रिमपश्चाद्राश्युदयशोधने शुद्धो राशिलग्नं नेति ज्ञातम् । ततो यो राश्युदयो न शुध्यति स एव राशिरभीष्टकाले क्षितिजे लग्न इति । तस्य को भागो लग्न इति ज्ञानार्थमशुद्धराश्युदथासुभिस्त्रिंशद्भागास्तदा शेषासुभिः क इत्यनुपातेन भुक्तभोग्यक्रमेण लग्नराशेर्भोग्यभुक्तभागादिकं सिद्धम्। तत्र भाग्यभागास्त्रिंशतः शुद्धा गता भागा लग्नराशेर्भवतीत्यशुद्धराशिसङ्ख्यातो भोग्यभागाः शुद्धा लग्न भवति । भुक्तभागाश्च भुक्तराशिसङ्ख्यायां युक्ता लग्नं भवति । अयनांशव्यस्तसंस्कारो ग्रहपङ्क्तिस्यत्वार्थम् । अन्यथा फलादेशार्थं ग्रहा अयनांशसंस्कृता ग्राह्या इति सर्वं निरवद्यम्~॥~४७~॥\\
\noindent अथ प्रसङ्गाश्लन्मलग्नानयनं लग्नानयनविशेषसुचनार्थमाह\textendash
\end{sloppypar}
%\vspace{2mm}


\begin{quote}
  {\ssi प्रापश्चान्नतनाडीभिस्तस्माल्लङ्कोदयासुभिः~।\\
भानौ क्षयधने कृत्वा मध्यलग्नं तदा भवेत्~॥~४८~॥}
\end{quote}
%\vspace{2mm}

\begin{sloppypar}
 दिनार्धान्तर्गतदिनगतशेषहीनं दिनार्धं क्रमेण प्राक् पश्चिमं नतं रात्र्यर्धान्तर्गतरात्रिशेषगतयुतं दिनार्धं प्राक् पश्चिमं नतं जातकपद्धतौ प्रसिद्धम् । नतघटिकाभिस्तस्मात् तात्कालिक\textendash
\end{sloppypar}

\newpage

\hspace{3cm}गूढार्थप्रकाशकेन सहितः~। \hfill १४७
\vspace{1cm}

\begin{sloppypar}
\noindent सूर्यात् । निरक्षदेशराश्युदयासुभिः पूर्वोक्तप्रकारेण सिद्धराशिभागादिकं प्राक्पश्चिमनतक्रमेण सूर्ये क्षयधने हीनयुते कृत्वा तदाभीष्टकाले मध्यलग्नं दशमलग्नं स्यात् । अयमभिप्रायः । प्राङ्गते नतघट्यसुभ्यः सूर्याक्रान्तराशेर्निरक्षोदयासुभिर्भुक्तासून् विशोध्य तत्पूर्वराशीनां निरक्षोदयासूंश्च विशेध्य शेषं त्रिंशङ्गुणमशुद्धनिरोक्षोदयभक्तं फलेन भागादिना शोधितगृहसङ्ख्यातुल्यराशिभिश्च सूर्यो हीनो मध्यलग्नम् । एवं पश्चिमनते नतघट्यसुभ्यः सूर्याक्रान्तराशेर्निरक्षोदयासुभिर्भोग्यासून् विशोध्य तदग्रिमराशीनां निरक्षोदयासूंश्च विशोध्य शेषं त्रिंशद्गुणमशुद्धनिरक्षोदयभक्तं फलेन भागादिना शोधितग्रहसङ्ख्यातुल्यराशिभिश्च सूर्यो युतो मध्यलग्नम् । एवं भुक्तभोग्यासुभ्योऽल्पकालेऽपीष्टासवस्त्रिंशद्गुणिताः सूर्याक्रान्तराश्युदयभक्ताः फलेन भागादिना हीनयुतोऽर्को मध्यलग्नं स्यात् । अनेन प्रकारेण लग्नमपि साध्यम् । अत्रोपपत्तिः । ऊर्ध्वयाम्योत्तरवृत्ते यः क्रान्तिवृत्तप्रदेशो लग्नस्तन्मध्यलग्नम् । तत्साधनार्थमभीष्टकाले याम्योत्तरवृत्ताद्युरात्रवृत्ते सूर्यो यावता घटीविभागादिना नत: स नतकालः । प्राक्पश्चिमकपालयोः प्राक्पश्चिमसञ्ज्ञः । अर्धरात्रमारभ्य दिनार्धपर्यन्तं प्राक्कपालम् । दिनार्धमारभ्यार्धरात्रपर्यन्तं पश्चिमकपालम् । तत्र प्राङ्नते सूर्यस्य याम्योत्तरवृत्तात् पूर्वस्थत्वेन सूर्यात् पूर्वराशिभाग एव याम्योत्तरवृत्तलग्न इति सूर्यादूनमृणलग्नरीत्या नतघटीभिः साध्यम् । पश्चिमनते तु सूर्यस्य याम्योत्तरवृत्तात् पश्चिमस्थत्वेन सूर्याग्रिम\textendash
\end{sloppypar}

{\tiny{U2}}

\newpage



\noindent १४८ \hspace{4cm} सूर्यसिद्धान्तः 
\vspace{1cm}

\begin{sloppypar}
\noindent राशेर्मध्यलग्नत्वात् सूर्यादधिकक्रमलग्नरीत्या नतघटीभिः साध्यम् । तत्रोद्वृत्ताद्याभ्योत्तरवृत्तस्य पञ्चदशघट्यन्तरेण नियतं सत्त्वान्निरक्षोदयासुभिः साध्यमिति । शेषक्रियोपपत्तिस्त्वतिस्पष्टतरेति संक्षेपः~॥~४८~॥\\
\noindent अथ कालसाधनमाह\textendash
\end{sloppypar}
%\vspace{2mm}


\begin{quote}
  {\ssi भोग्यासूनूनकस्याथ भुक्तासूनधिकस्य च~।\\
सम्पीङ्यान्तरलग्नासूनेवं स्यात् कालसाधनम्~॥~४९~॥}
\end{quote}
%\vspace{2mm}

\begin{sloppypar}
 अथानन्तरं लग्नार्कयोर्मध्ये योऽत्यन्तमूनस्तस्य भोग्यासूनधिकस्य भुक्तासून् सम्पीङ्यैकीकृत्यान्तरलग्नासुन् सूर्यलग्नमध्ये ये लग्नराशयस्तेषामुदयासून् । चः समुच्चये। एकीकृत्यैवमुक्तप्रकारेण कालस्य सिद्धिर्भवति । अत्रोपपत्तिः । ऊनादधिकमग्र एव भवतीत्यूनतुल्यलग्नस्य भोग्यकालोऽन्तरस्थराश्युदययुतोऽधिकतुल्यलग्नस्य भुक्तकालेन युतस्तल्लग्नयोरन्तरवर्ती काल: सिद्धः स्यात्~॥~४९~॥\\
\noindent अथैवं लगार्काभ्यां साधितकालस्य दिनरात्र्यन्तर्गतत्वज्ञानमाह\textendash
\end{sloppypar}
%\vspace{2mm}


\begin{quote}
  {\ssi सूर्यादूने निशाशेषे लोऽर्कादधिके दिवा~।\\
भचक्राधयताद्भानोरधिकेऽस्तमयात् परम्~॥~५०~॥}
\end{quote}
%\vspace{2mm}

\begin{sloppypar}
 सूर्यात् त्रिराश्यन्तर्गतत्वेन न्यूने लग्ने सति पूर्वप्रकारसिद्धः कालो रात्रिशेषे भवति । सूर्यात् षङ्भान्तर्गतत्वेनाधिके लग्ने पूर्वप्रकारसिद्धः कालो दिने स्यात् । षङ्भयुतात् सूर्यादधिके लग्ने लग्नसषड्भसूर्याभ्यामानीतः पूर्वरीत्या कालोऽस्तमयात् सूर्यास्तकालात् परमनन्तरं रात्रावित्यर्थः । एतेन रात्रीष्टकाले गते
\end{sloppypar}

\newpage

 \hspace{3cm} गूढार्थप्रकाशकेन सहितः~। \hfill १४९
\vspace{1cm}

\begin{sloppypar}
\noindent सषङ्भसूर्यालग्नं साध्यमिति सूचितम् । अत्रोपपत्तिः । सूर्योदये सूर्यतुल्यलग्नत्वात् सूर्यादूनाधिके लग्ने क्रमेण रात्रिशेषे दिने च कालः स्यात् । एवमस्तकाले सषङ्भसूर्यस्य लग्नत्वात् तदधिके लग्ने रात्रावेव कालः सिद्ध्येदित्यादि सुगमतरम्~॥~५०~॥\\
\noindent अथाग्रिमग्रन्थस्यासङ्गतित्वनिरासार्थमधिकारसमाप्तिं फक्किकयाह\textendash
\end{sloppypar}
\vspace{4mm}

\begin{center}
{\huge{इति त्रिप्रश्नाधिकारः~।}}
\end{center}
\vspace{4mm}

\begin{sloppypar}
दिग्देशकालानां प्रतिपादनमिदं परिपूर्तिमाप्तमित्यर्थः । दिशां साधनं शिलातल इत्यादि नियतं तत्सम्बन्धन समकोणयाम्योत्तरशङ्कनां साधनान्यपि दिगन्तर्गतान्यनियतानि । पलभालम्बाक्षादिसाधनं देशनिरूपणं नियतम् । अग्राचरादिसाधनमनियतम् । कालसाधनं तद्वशाच्छायादिसाधनं च कालनिरूपणमिति विवेकः ।
\end{sloppypar}
%\vspace{2mm}


\begin{quote}
{\qt रङ्गनाथेन रचिते सूर्यसिद्धान्तटिप्पणे~।\\
त्रिप्रश्नस्याधिकारोऽयं पूर्णो गूढप्रकाशके~॥}
\end{quote}
%\vspace{2mm}

इति श्रीसकलगणकसार्वभौमबल्लालदेवज्ञात्मजरङ्गनाथगणकविरचिते गूढार्थप्रकाशे त्रिप्रश्नाधिकारः पूर्णः~॥

\vspace{4mm}
   
\begin{center}
\rule{7em}{.5pt}
\end{center}
\vspace{4mm}

 अथ चन्द्रग्रहणाधिकारो व्याख्यायते । तच प्रथमं सूर्यचन्द्रयोर्बिम्बयोजनानि तत्स्फुटीकरणं च सार्धश्लोकेनाह\textendash

\newpage


 \noindent १५० \hspace{4cm} सूर्यसिद्धान्तः 
\vspace{1cm}


\begin{quote}
 {\ssi सार्धानि षट्सहस्राणि योजनानि विवस्वतः~। \\ 
विष्कम्भो मण्डलस्येन्दोः सहाशीत्या चतुःशतम् *~॥~१~॥

स्फुटस्वभुक्त्या गुणितौ मध्यभुक्त्योद्धृतौ स्फुटौ~।}
\end{quote}
%\vspace{2mm}

\begin{sloppypar}
षट्सहस्राणि सार्धानि सहस्त्रस्यार्द्धं पञ्चशतं तत्सहवर्तमानानि पञ्चषष्टिशतं योजनानि सूर्यस्य मण्डलस्य गोलरूपबिम्बस्य विष्कम्भो व्यासः । चन्द्रस्य गोलाकारबिम्बस्याशीत्या सहाशीत्यधिकं चतु:शतं योजनानि । तौ व्यासौ स्पष्टया निजगत्या गुणितौ निजमध्यगत्या भक्तौ स्फुटौ स्तः । अत्र गणिते व्यासस्यैव बिम्बव्यवहारोऽभियुक्तानाम् । अत्रोपपत्तिः । त्रिज्यामितकर्णे मध्यमकक्षायां भ्रमणात् तत्र यद्विम्बं व्यासात्मकं तन्मध्यमम्। तत्र स्वल्पान्तरेण मध्यगत्यङ्गीकारान्मध्यगत्येदं तदा स्फुटगत्या किमिति स्पष्टं बिम्बं नीत्ते पृथूच्चेऽणुतरम् । गत्यौः परमाधिकन्यूनत्वात्~॥~१~॥\\
\noindent अथ सूर्यबिम्बं चन्द्रकक्षायां साधयंस्तयोः कालात्मकबिम्बानयनं सार्धश्लोकेनाह\textendash
\end{sloppypar}
%\vspace{2mm}


\begin{quote}
  {\ssi रवेः स्वभगणाभ्यस्तः शशाङ्कभगणाद्धृतः~॥~२~॥
  
शशाङ्ककक्षागुणितो भाजितो वार्ककक्षया †~।\\
विष्कम्भश्चन्द्रकक्षायां तिथ्याप्ता मानलिप्तिकाः~॥~३~॥}
\end{quote}
%\vspace{2mm}


सूर्यस्य विष्कम्भः प्रागुक्तः स्पष्टो व्यासः स्वभगणैः सूर्यभगणैरुक्तैर्गुणितश्चन्द्रभगणैर्भक्तो वाथवा चन्द्रकक्षया वक्ष्यमाणया

\rule{\linewidth}{.5pt}

\begin{center}
 * चतुः शती इति पाठान्तरम् ।
 
† भाजितश्चार्ककक्षया इति पाठान्तरम् ।
\end{center}


\newpage



\hspace{3cm} गूढार्थप्रकाशकेन सहितः~। \hfill १५१
\vspace{1cm}

\begin{sloppypar}
\noindent गुणितः सूर्यकक्षया वक्ष्यमाणया भक्तश्चन्द्रकक्षायां चन्द्राधिष्ठिताकाशगोले सूर्यव्यासः स्पष्टो भवति । ततो व्यासयोजनसङ्ख्या पञ्चदशभक्ता सूर्यचन्द्रयोर्बिम्बव्यासप्रमाणकला भवन्ति । अत्रोपपत्तिः । चक्रकलाभिश्चन्द्रकक्षायोजनानि तदैककलया कानीति चन्द्रकक्षास्थितैककलायां पञ्चदश योजनानि । अतश्चन्द्रस्य स्वकक्षायां स्थितत्वात् स्पष्टचन्द्रबिम्बव्यासयोजनानि पञ्चदशभक्तानि चन्द्रबिम्वव्यासकला भवन्ति । एवं सूर्यकक्षायामेका कला सार्धशतद्वययोजनैरिति स्पष्टसूर्यव्यासस्तैर्भक्तो व्यासकला भवन्ति । तत्र सूर्यस्य लोकैर्दूरान्तराच्चन्द्राकाश इव दर्शनात् प्रत्यक्षतो विविक्तान्तरेण दर्शनाभावाच्च चन्द्रकक्षाप्रमाणेन सूर्यबिम्बव्यासः सूर्यकक्षयायं तदा चन्द्रकक्षया क इत्यनुपातेन गणितार्थमवस्तुभूतः साधितः । न तु वस्तुतश्चन्द्रकक्षायां सूर्यमण्डलावस्थानं सूर्यग्रहणे चन्द्रस्य छादकत्वानुक्तिप्रसङ्गात् । अथ सूर्यस्पष्टव्यासश्चन्द्रभगणभक्तखकक्षारूपचन्द्रकक्षया गुणितः सूर्यभगणभक्तखकक्षारूपसूर्यकक्षया भक्त इति खकक्षारूपगुणहरयोर्नाशात् सूर्यभगणगुणितश्चन्द्रभगणभक्त इति पूर्व कक्षयोरनुकेरयं प्रकारो मुख्यत्वात् प्रथममुक्तस्ततश्चन्द्रकक्षासिद्धसूएयविम्बव्यासः पञ्चदशभक्तः सूर्यबिम्वव्यासकलाः सिद्धा इत्युपपन्नमुक्तम्~॥~३~॥\\
\noindent अथोपयुक्तां भूच्छायां श्लोकाभ्यां साधयति\textendash
\end{sloppypar}
%\vspace{2mm}


\begin{quote}
  {\ssi स्फुटेन्दुभुक्तिर्भुव्यासगुणिता मध्ययोद्धृता~।\\
 लब्धं सूची महीव्यासस्फुटार्कश्रवणान्तरम्~॥~४~॥}
\end{quote}

\newpage





\noindent १५२ \hspace{4cm} सूर्यसिद्धान्तः
\vspace{1cm}


\begin{quote}
{\ssi मध्येन्दुव्यासगुणितं मध्यार्कव्यासभाजितम्~।\\
विशोध्य लब्धं सूच्यां तु तमोलिप्तास्तु पूर्ववत्~॥~५~॥}
\end{quote}
%\vspace{4mm}

\begin{sloppypar}
 स्पष्टा चन्द्रस्य गतिर्भूव्यासेन गुणिता मध्यया चन्द्रगत्या भक्ता फलं सूचीसञ्ज्ञं स्यात् । भूव्यासस्पष्टसूर्यबिम्बव्यासयोरतरं मध्येन चन्द्रबिम्बव्यासेनाशीत्यधिकचतु: शतयोजनेन गुणितं मध्येन सूर्यबिम्बव्यासेन पञ्चषष्टिशतयोजनेन भक्तं फलं सुच्यां प्राक्सिद्धायां न्यूनीकृत्य तुकाराच्छेषं तमः । भू छायारूपं योजनात्मकं भाभावस्तम इति छायायास्तमस्वात् । अस्य कलात्मकं मानमाह\textendash लिप्ता इति~। लन्तस्य पर्वसम्बन्धानुक्तेरुत्तरत्र सम्बन्धस्तुकारेण सुबोधः । अत एव पूर्ववाक्यसमाप्तिस्थं तमः पदमत्र नान्वेति । पूर्ववत् तिथ्याप्ता मानलिप्तिका इति पूर्वोक्तेन भूच्छायाया: कला: कार्याः । अत्रोपपत्तिः ।
\end{sloppypar}


\begin{quote}
{\qt भूव्यासहीनं रविबिम्बमिन्दुकर्णाहतं भास्करकर्णभक्तम्~।\\
 भूविस्तृतिर्लब्धफलेन हीना भवेत् कुभाविस्तृतिरिन्दुमार्गे~॥}
\end{quote}

\begin{sloppypar}
इति सिद्धान्तशिरोमणौ सूक्ष्मप्रकार उक्तः । अस्योपपत्तिस्तट्टीकायां व्यक्ता । तत्र भूव्यासोनस्य रविबिम्बस्य ४९०० स्वल्पान्तराङ्गीकारेण स्पष्टगतिभक्तमध्यगतिगुणितचन्द्रमध्ययोजनकर्णरूपस्पष्टेन्दुयोजनकर्णो गुण: । तादृशसूर्यकर्णो हरः । तत्रैतत्खण्डस्य कलाकरणार्थं त्रिज्यागुणश्चन्द्रकर्णस्तादृशो हर इति चन्द्रस्पष्टमध्यगत्योस्तुल्यगुणहरत्वेन नाशात् त्रिज्यामध्येन्दुयोजनकर्णयोस्त्रिज्यापवर्तनेन हरः पञ्चदश पृथगुक्तः । अग्रेऽवशिष्टौ
\end{sloppypar}


\newpage


 \hspace{3cm} गूढार्थप्रकाशकेन सहितः~। \hfill १५३
\vspace{1cm}

\begin{sloppypar}
\noindent भूव्यासहीनमध्यार्कबिम्बयोजनानां रविस्पष्टगतिमध्यमगतो गुणहरौ । चन्द्ररर्ययोर्मध्ययोजनकर्णावपि क्रमेण गुणहरौ । तत्र कर्णस्थाने लाघवात् तयोर्बिम्बयोजनानि गृहीतानि । यद्यपि सूर्यचन्द्रयोर्मध्ययोजनकर्णानुसारित्वाभावाद्विम्बयोजनग्रहणमनुचितं तथाप्यल्पान्तराङ्गोकारेण तददोषः । इन्दुव्यासार्कव्यासयोर्भूगोलाध्यायोक्तकक्षाभूकर्णगुणिता महोमण्डलभाजिता तत्कर्ण इति । तत्कक्षाव्यासार्धत्वे तु सुतराम् । तत्रापि स्पष्टार्कबिम्बयोजनग्रहणे मध्यार्कयोजनबिम्बं सूर्यस्पष्टगतिगुणितं सूर्यमध्यगतिभक्तमिति सिद्धम् । न चोक्तरीत्या सूर्यस्पष्टमध्यगती गुणहरौ भूव्यासमध्यार्कबिम्बयोजनान्तरस्योत्पन्नौ न केवलं बिबस्येति भूव्यासस्तादृशो महीव्यास इत्यनेन कथं सिद्ध इति वाच्यम् । भगवता स्वल्पान्तरेण महीव्यासस्य यथास्थितस्यैवाङ्गीकारात् । महीव्यासस्फुटार्कश्रवणान्तरमित्युक्त्या मध्यस्थस्फुटपदस्योभयत्रान्मयेनार्कश्रवणसन्निधानेन च सूर्यबिम्बस्फुटरीत्यैव महीव्यासस्य स्फुटत्वसिद्धश्च । अथैतत्खण्डसिद्धं फलं भूव्यासाद्धीनं भूभायोजनानि । तत्र कलाकरणार्थं भूव्यासस्यापरखण्डस्य त्रिज्या गुणः स्पष्टचन्द्रगतिभक्तमध्यगतिगुणितचन्द्रमध्ययोजनकर्णरूपस्पष्टयोजनकर्णो हरः । तत्र त्रिज्यामध्ययोजनकर्णौ गुणहरौ गुणेनापवर्त्य हरस्थाने पञ्चदश चन्द्रस्पष्टमध्यगती गुणहराविति सूच्युक्तोपपन्ना । भूभायाः सूच्यनुकारत्वात् प्रथमखण्डं द्वितीयखण्डे हीनं भूभायोजनात्मिका सा पञ्चदशभक्ता कलादिकेत्युक्तमपपन्नम् । यदि तु भूव्यासहीनं रविबिम्बमित्यादौ मध्यबिम्बा\textendash
\end{sloppypar}

{\tiny{V}}


\newpage



\noindent  १५४ \hspace{4cm} सूर्यसिद्धान्तः 
\vspace{1cm}

\begin{sloppypar}
\noindent नुक्तेः प्रथममेव स्पाष्टार्कबिम्वग्रहणं तदा महीव्यासस्य स्पष्टत्वाप्रसिद्ध्या महीव्यासस्फुटार्कश्रवणान्तरमित्येव यथाश्रुतं सम्यक् । परन्तु तदा भूव्यासोनार्कबिम्बस्य सूर्यमध्यस्पष्टगती हरगुणाववशिष्टौ वाच्यावपि भगवता स्वल्पान्तरत्वादनुक्तौ । न चानुपाते सूर्यचन्द्रयोर्मध्ययोजनकर्णवेव गृहीतौ न स्फुटाविति मध्यस्फुटगती हरगुणावनुत्पन्नौ नोक्ताविति वाच्यम् । चन्द्रस्पष्टयोजनकर्णस्वरूपग्रहणेनोत्पन्नसूच्या अनुक्तत्वापत्तेः । न च चन्द्रकर्णस्य मध्यत्वेन गृहीते बह्वन्तरमतः स्पष्टत्वेन तस्य ग्रहे सूच्युपपन्ना सूर्यकर्णस्य मध्यत्वेन गृहीतेत्यल्पान्तरमिति वाच्यम् । मध्यार्कबिम्बयोजनग्रहणेन स्फुटार्कश्रवणानुपपत्तेः । न चोभयत्राग्टहीते प्रत्येकमल्पान्तरमपि बह्वन्तरमतः एकत्र सूर्यगतिग्रहणमुचितमिति वाच्यम् । विनिगमनाविरहात् । पूर्व सूर्यबिम्बस्यैव सूर्यस्पष्टमध्यगती गुणहरौ न महीव्यासस्य प्रान्त्येतृभयोरिति स्थूलसूक्ष्मविनिगमके तु प्रान्त्ये सूर्यगतिग्रहणस्यौचित्याच्च । अथ महीव्यासस्य प्रथमखण्डस्य चन्द्रगतिग्रहणेन सूच्युक्तावेव द्वितीयखण्डस्य भूव्यासोनस्फुटरविबिम्बस्यार्थात् सूर्यगतिग्रहणं सूचितमिति न क्षतिरिति चेन्न । व्याख्याप्रसङ्गे सूर्यगतिग्रहणे मानाभावात् उपपत्तेरप्रसङ्गाच्च । अन्यथात्रापि चन्द्रगतिग्रहणापत्तेरिति । एतेन चन्द्रमध्यगत्या भूव्यासस्तदा चन्द्रस्पष्टगत्या क इति भूव्यासरूपं खण्डं स्पष्टं सूचीसञ्ज्ञं सूर्यबिम्बप्रमाणेनापरं भूव्यासोनस्फुटरविबिम्बखण्डं तदा चन्द्रबिम्बप्रमाणेन किमिति स्पष्टं द्वितीयं खण्डं तयोः स्पष्टश्योरन्तरं
\end{sloppypar}


\newpage


\hspace{3cm} गूढार्थप्रकाशकेन सहितः~। \hfill १५५
\vspace{1cm}

\begin{sloppypar}
\noindent स्पष्टा भूभेति सर्वमुपपन्नमिति निरस्तम् । उक्तानुपाताभ्यां तयोः स्पष्टत्वसिद्धौ मानाभावात् । स्पष्टत्वस्याप्रसङ्गाच्च । चन्द्रसूर्ययोर्मध्यबिम्बानुपपत्तेश्च । यत् तु भूव्यासस्य स्पष्टत्वं सूचीरूपमनुपपद्यमानं हृदि ज्ञात्वा भूव्यास एव प्रथमखण्डं भूव्यासोनस्पष्टरविबिम्बस्य मध्यकर्णानुपाताभ्यामल्पान्तरेणापवर्तनान्मध्यबिम्बे गुणहरावुत्पाद्य द्वितीयखण्डमुभयोरङ्गुलीकरणं चन्द्रमध्यकर्णेन त्रिज्यामिताः कलास्तदाभ्यां का इत्यनुपाते प्रमाणफलयोः फलावर्तनेन प्रमाणस्थानापन्नपञ्चदशहरेणेति तयोरन्तरं भूभेत्युक्तं ज्ञानराजदैवज्ञैः सिद्धान्तसुन्दरे ।
\end{sloppypar}
%\vspace{2mm}


\begin{quote}
 {\qt इनावनीव्यासवियोगनिघ्नं\\
शशाङ्कबिम्बं रविबिम्बभक्तम्~।\\
फलोनभूव्याससमा कुभासौ\\
शरेन्दुभक्ता कलिकादिका स्यात्~॥}
\end{quote}
\begin{sloppypar}
इति ग्रन्थेन । अत्र सूर्यव्यासः स्फुटार्कबिम्बयोजनात्मको न मध्ययोजनात्मकः । चन्द्रार्कबिम्बे गुणहरौ मध्ययोजनात्मकौ न स्फुटबिम्बयोजनात्मकौ तट्टीकाकृच्चिन्तामण्यभिमतौ । उपजीव्यसूर्यसिद्धान्तविरोधात्। तदुक्तं तदुपपत्त्यापि तदसिद्धेश्च । अत्र यदपि तट्टीकाकृच्चिन्तामण्युक्तं मध्यमस्य भूभाबिम्बस्यानयनं फलाविशेषेण मध्यकर्णावेव गुणहरौ प्रकल्प्योक्तविधिना सिद्धस्य मध्यबिम्बस्य यदि मध्यगत्यन्तरेणेदं स्फुटगत्यन्तरेण किमित्यनुपातेन स्फुटत्वं मूलकृदनुक्तमपि कार्यमिति तद्गत्यन्तरवशेन भूभाया अनुत्पत्त्या न समञ्जसम् । अन्यथा गतिवशेन साधि\textendash
\end{sloppypar}

{\tiny{V2}}

\newpage



\noindent १५६ \hspace{4cm} सूर्यसिद्धान्तः 
\vspace{1cm}


\noindent तार्कचन्द्रबिनवद्गत्यन्तरकलाभ्योऽविकृताभ्य एव भूभायाः साधनापत्तेरिति । तदसत् ।


\begin{quote}
 {\qt स्फुटेन्दुभुक्तिर्भूव्यासगुणिता मध्ययोद्धृता~।}
\end{quote}
\begin{sloppypar}
इति सूर्यसिद्धान्तोक्तयुक्तिसिद्धसूच्यनुक्त्या भूव्यासस्यैवाविकृतस्य ग्रहणादित्यलं परदोषगवेषणापल्लवितेन~॥~५~॥\\
\noindent अथ ग्रहणद्वयसम्भूतिमाह\textendash
\end{sloppypar}
%\vspace{2mm}


\begin{quote}
 {\ssi भानोर्भार्धे महीच्छाया तत्तुल्येऽर्कसमेऽपिवा~।\\
शशाङ्कपाते ग्रहणं कियद्बागाधिकोनके~॥~६~॥}
\end{quote}
%\vspace{2mm}

\begin{sloppypar}
सूर्यात् सकाशात् षड्भान्तरे भूच्छाया सूर्यापरदिक्त्वात् । तत्तुल्ये सषङ्भार्करूपच्छायाक्षेत्रादिना समे चन्द्रपाते । अपिवाथवा सूर्यतुल्ये चन्द्रपाते सूर्यचन्द्रयोः प्रत्येक ग्रहणम् । ननु समत्वाभावेऽपि ग्रहणमित्यत आह\textendash  कियङ्भागेत्यादि~। सषङ्भार्कादर्काद्वा कतिपयैर्भागैरधिक ऊनेऽपि चन्द्रपाते ग्रहणम् । तथा च न क्षतिः । भागाश्चन्द्रग्रहणे द्वादश निश्चयार्थम् । सूर्यग्रहणे तु नतांशषडंशसंस्कारात् सप्तेत्यापाततः । अत्रोपपत्तिः । सषङ्भार्ककेवलार्कान्तरतुल्ये चन्द्रपाते शराभावश्चन्द्रस्य तत्तुल्यत्वात् । तदा चन्द्रो भूच्छायायां भवतीति ग्रहणम् । एवं भरसत्वेऽपि मानैक्यखण्डादल्ये भूच्छायायां मण्डलैकदेशस्य सत्त्वेन ग्रहणम् । एवं शराभावे मानैक्यखण्डान्न्यूनशरे च चन्द्रमण्डलं सूर्यमण्डलस्याच्छादकं भवति परन्तु तत्र शरो नतिसंस्कृतोऽतः सम्यगुक्तमुपपन्नम्~॥~६~॥\\
\noindent ननु तत् कुत्र भवतीत्यतस्तयोर्ग्रहणयोः कालमाह\textendash
\end{sloppypar}


% \includegraphics[width=1.\theta4167in,height=1.\theta4167in]Images/Inserted/86\theta\thetaculture-removebg-preview (1)(1).jpg
\newpage

\hspace{3cm} गूढार्थप्रकाशकेन सहितः~। \hfill १५७
\vspace{1cm}


\begin{quote}
  {\ssi तुल्यौ राश्यादिभिः स्याताममावास्यान्तकालिकौ~।\\
 सूर्येन्दू पौर्णमास्यन्ते भार्धे भागादिकौ समौ~॥~७~॥}
 \end{quote}
%\vspace{2mm}

\begin{sloppypar}
 अमावास्यान्तकालोत्पन्नौ सूर्यचन्द्रौ राण्याद्यवयवैः समौ भवतः । पौर्णमास्यन्ते भागादिकौ तुल्यौ सूर्यचन्द्रौ षड्वान्तरे स्याताम् । तथा चामान्ते स्यर्यचन्द्रयोरेकत्रोर्ध्वाधरान्तरेण सत्त्वात् सूर्यग्रहणम् । पौर्णमास्यन्ते चन्द्रभूभयोरेकत्रावस्थानाच्चन्द्रग्रहणम् । एतेन पूर्वश्लोके शशाङ्कपात इत्यत्र चन्द्रचन्द्रपातौ द्वौ न ग्राह्याविति सूचितम् । एतच्छ्लोकस्य वैयर्थ्यापत्तेः । अत्रोपपत्तिः । अमान्ते सूर्यचन्द्रयोः पूर्वापरान्तराभावेन योगात् तुल्यौ सूर्यचन्द्रौ पूर्णिमान्ते भचक्रार्धान्तरत्वात् षङ्राश्यन्तरौ भागादिसमाविति~॥~७~॥\\
 \noindent अथ पर्वान्ते सूर्यचन्द्रचन्द्रपातानां साधनमाह \textendash
\end{sloppypar}
%\vspace{2mm}


\begin{quote}
  {\ssi गतैष्यपर्वनाडोनां स्वफलेनोनसंयुतौ~।\\
समलिप्तौ भवेतां तौ पातस्तात्कालिकोऽन्यथा~॥~८~॥}
\end{quote}
%\vspace{2mm}

\begin{sloppypar}
 तौ सूर्यचन्द्रौ गतैव्यपर्वनाडीनां यत्कालिकौ सूर्यचन्द्रौ तत्कालाद्गता एष्या वा दर्शान्तपूर्णिमान्तान्यतर घटिकास्तासां स्वफलेन स्वगतिसम्बन्धेन यत् फलम् । इष्टनाडीगुणा भुक्तिः षड्या भक्ताकलादिकम् । इति मध्याधिकारोक्तेनानीतम् । तेन गतैव्यक्रमेणोनयुतौ तत्र समकलौ स्तः । यद्यपि समांशाविति वक्तुं युक्तं तथाप्यन्यतिथ्यन्तीयसाधितौ समकलाविति द्योतनार्थं समकलावित्युक्तम् ।
\end{sloppypar}

%\includegraphics[width=1.\theta4167in,height=1.\theta4167in]Images/Inserted/86\theta\thetaculture-removebg-preview (1)(1).jpg
\newpage

\noindent १५८ \hspace{4cm} सूर्यसिद्धान्तः
\vspace{1cm}

\begin{sloppypar}
\noindent पातः स्वगत्युत्पन्नफलेनान्यथा गतैय्यक्रमेण युतोनस्तात्कालिकः पर्वान्तकालिकः स्यात् । अत्रोपपत्तिश्चालनश्लोकः । तत्र तिथ्यन्ते भागान्तरत्वेन कलादिसाम्यम् । पातस्य चक्रशोधितन्वेने तरग्रहवैपरीत्यम्~॥~८~॥\\
\noindent अथ प्रागुक्तानां बिम्बानां प्रयोजनमाह\textendash
\end{sloppypar}
%\vspace{2mm}


\begin{quote}
  {\ssi छादको भास्करस्येन्दुरधःस्थो धनवद्भवेत्~।\\
भूच्छायां प्राङ्मुखश्चन्द्रो विशत्यस्य भवेदसौ~॥~९~॥}
\end{quote}
%\vspace{2mm}

\begin{sloppypar}
 सूर्यमण्डलस्याच्छादकश्चन्द्रः स्यात् । नन्वाकाशे द्वयोः सत्त्वेन सूर्य एव चन्द्रस्य छादकः कथं न स्थादित्यत आह\textendash अधःस्थ इति~। वक्ष्यमाणकक्षाध्याये सूर्यकक्षातोऽध: कक्षास्थत्वाञ्चन्द्रस्यैवाच्छादकत्वम् । न ह्युर्ध्वस्थश्छादको येन सूर्यश्चन्द्रस्य छादकः। ननु विनैकत्रावस्थानं छादनं न भवत्यत आह\textendash घनवदिति~। यथाधःस्थो मेघः सूर्यस्याच्छादको भवति तथा चन्द्रो भवतीत्यर्थः । प्राङ्मुखः पूर्वाभिमुखो गच्छंश्चन्द्रो भूच्छायां प्रति प्रविशति । अतः कारणादस्य चन्द्रस्यासौ भूभाच्छादिका भवेत्। तथा च सूर्यग्रहणे सूर्यचन्द्रबिम्बयोः प्रयोजनं चन्द्रग्रहणे चन्द्रभूभाबिस्वयोः प्रयोजनमिति भावः । अत्रोपपत्तिः । चन्द्रो दर्शान्ते सूर्यादधो भवतीति चन्द्रः सूर्यस्याच्छादकः । बुधशुक्रयोस्तु मण्डलाल्पत्वान्नाच्छादकत्वम् । चन्द्रस्थाधो ग्रहाभावात् षङ्भान्तरे भूम्या प्रतिबद्धाः सूर्यकिरणाश्चन्द्रगोले न पतन्ति। अतो निप्रभस्य चन्द्रस्य भूभायां प्रवेश इति चन्द्रस्य भूभाच्छादिका~॥~९~॥\\
 \noindent अथ ग्रामानयनमाह\textendash
\end{sloppypar}

\newpage

 \hspace{3cm} गूढार्थप्रकाशकेन सहितः~। \hfill १५९
\vspace{1cm}


\begin{quote}
 {\ssi तात्कालिकेन्दुविक्षेपं छाद्यच्छादकमानयोः~।\\
योगार्धात् प्रोज्झय यच्छेषं तावच्छन्नं तदुच्यते *~॥~१०~॥}
\end{quote}
%\vspace{2mm}


\begin{sloppypar}
 यश्छाद्यते स छाद्यः । सूर्यग्रहणे सूर्यश्चन्द्रग्रहणेचन्द्रः। यश्छादयति स छादकः । सूर्यचन्द्रग्रहणयोः क्रमेण चन्द्रभूमे । तयोः पूर्वानीतमानकलयोरैक्यस्यार्धात् तात्कालिकचन्द्रात् पूर्वोक्तप्रकारेण साधितं विक्षेपं कलादिकं विशोध्य यदवशिष्टं तत्प्रमाणकं छन्नं छादकेन छाद्यस्य यावान्मण्डलप्रदेश आच्छादितस्तावत् प्रदेशात्मकं ग्रासरूपं ग्रहणतत्त्वज्ञैः कथ्यते । अत्रोपपत्तिः । छाद्यच्छादकमण्डलनेमियोगे ग्रहणाद्यन्तरूपे मण्डलकेन्द्रयोरन्तरं स्वबिग्बखण्ड योगरूपम् । बिम्बस्य व्यासमानाकत्वात् । तत् तु ममत्वाल्लाघवाच्च योगार्धरूपं धृतम् । ततो यथाप्रवेशस्तथा ग्रासो भवतीति पर्वान्ते छाद्यच्छादकयोर्विक्षेपान्तरितत्वात् तदूने विक्षेपे मण्डलयोगस्तदन्तरमितः स एव ग्रासः~॥~१०~॥\\
 \noindent अथ सम्पूर्णन्यूनग्रहणज्ञानं ग्रहणाभावज्ञानं चाह\textendash
\end{sloppypar}
%\vspace{2mm}


\begin{quote}
  {\ssi यद्ग्राह्यमधिके † तस्मिन् सकलं न्यूनमन्यथा~।\\
योगार्धादधिके न स्याद्विक्षेपे ग्राससम्भवः~॥~११~॥}
\end{quote}
%\vspace{2mm}


तस्मिन् छन्नमानेऽधिके ग्राह्यमानाधिके यद्यस्मात् कारणाद्ग्राह्यमानमस्ति । अतः कारणात् सकलं सम्पूर्णं ग्रहणं भवति ।

\noindent \rule{\linewidth}{.5pt}

\begin{center}
 * यच्छिष्ट तत् तमश्छन्नमुच्यते इति वा पाठः ।
 
† ग्राह्यमानाधिके इति पाठान्तरम् ।
\end{center}


\newpage

\noindent १६० \hspace{4cm} सूर्यसिद्धान्तः
\vspace{1cm}

\begin{sloppypar}
\noindent अन्यथा ग्राह्यमानान्न्यूने ग्रासे न्यूनं ग्राह्यमानान्तर्गतं ग्रहणं स्थात् । मानैक्यखण्डाद्विक्षेपेऽधिके सति ग्राससम्भवः । ग्रहणं न स्यात् । अत्रोपपत्तिः । ग्राह्यमानादधिके ग्रासे सम्पूर्णग्रहणं न्यूने न्यूनं मानैक्यखण्डादधिके विक्षेपे मण्डलस्पर्शासम्भवाद्ग्रहणाभावः~॥~११~॥\\
\noindent अथ स्थित्यर्धविमर्धे श्लोकाभ्यामाह\textendash
\end{sloppypar}
%\vspace{2mm}


\begin{quote}
  {\ssi ग्राह्यग्राहकसंयोगवियोगौ दलितौ पृथक्~।\\
विक्षेपवर्गहीनाभ्यां तद्वर्गाभ्यामुगे पदे~॥~१२~॥

षष्ट्या सङ्गुण्य सूर्शेन्द्रोर्भुक्त्यन्तरविभाजिते~।\\
स्यातां स्थितिविमर्धे नाडिकादिफले तयोः~॥~१३~॥}
\end{quote}
%\vspace{2mm}

\begin{sloppypar}
 ग्राह्यग्राहकमानयोर्योगान्तरे अर्धिते पृथक् स्थानान्तरे स्थाप्ये । अग्रिमक्रियायां कदाचिदद्धत्वसम्भवे पुनः क्रियार्थमेतयोरावश्यकत्वात् । तद्वर्गाभ्यां योगार्द्धान्तरार्धयोर्वर्गाभ्यां विक्षेपवर्गेण वर्जिताभ्यामुभे द्वे मूले षश्या गुणयित्वा सूर्यचन्द्रयोर्गत्यन्तरकलाभिर्भक्ते तयोर्योगवियोगयोः स्थाने घट्यादिफले क्रमेण स्थित्यर्धविमर्दार्धे भवतः । अत्रोपपत्तिः । ग्रहणारम्भाद्ग्रहणान्तपर्यन्तं यः कालः स स्थितिसञ्जः । तस्य खण्ड एकं ग्रहणारम्भान्मध्यग्रहणपर्यन्तमपरं मध्यग्रहणाद्ग्रहणान्तपर्यन्तम् । तत्र बिम्बनेमिस्पर्शकाले मानैक्यखण्डं कर्णः स्पर्शमोक्षकालिकशरो भुजः स्पर्शमोक्षान्यतरकालिकशराग्रमध्यकालिकशराग्रथोरन्तरं पर्वापरं कोटिरिति तत्खण्डमाधकं क्षेत्रम् । एवं सम्पूर्णग्रहणे सम्मीलनोन्मीलनकालयोरन्तरकालो मर्दस्तत्र
\end{sloppypar}


\newpage



\hspace{3cm}  गूढार्थप्रकाशकेन सहितः~। \hfill १६१
\vspace{1cm}

\begin{sloppypar}
\noindent मध्यग्रहणात् सम्मीलनोन्मीलनकालावधिखण्डे तत्साधकं छाद्यच्छादकमण्डलकेन्द्रयोरन्तरं मानार्धान्तरतुल्यं कर्णस्तात्कालिकशरो भुजः शराग्रयोरन्तरं विक्षेपवृत्ते पूर्वापरं कोटिरिति क्षेत्रम् । सम्मीलनं छाद्यमण्डलस्याच्छादनसमाप्तिः । उन्मीलनं तु छादकमण्डलादाच्छादितसम्पूर्णच्छाद्यमण्डलस्य निःसरणारम्भः । तत्र स्पर्शमोक्षसम्मीलनोन्मीलनकालानामज्ञानान्मध्यकालिकविक्षेपग्रहणम् । भुजकर्णवर्गान्तरपदं कोटिरिति पूर्वश्लोकोक्तमुपपन्नम् । छाद्यच्छादकमण्डलकेन्द्रयो: पूर्वापरान्तराभावे मध्यग्रहणसम्भवाच्छाद्यच्छादकयुतिर्गत्यन्तरकलाभिः षष्टिघटिकास्तदानीतकोटिकलाभिः का इत्यनुपातेन स्थितिमर्दखण्डे । तत्र चन्द्रग्रहणे भूभागतेः सूर्यगत्यनुरोधात् सूर्यगतित्वमित्युपपन्नं द्वितीयश्लोकोक्तम्~॥~१३~॥\\
\noindent अथ स्थित्यर्धविमर्दार्धे असकृत् साध्ये दूति श्लोकाभ्यामाह\textendash
\end{sloppypar}
%\vspace{2mm}


\begin{quote}
  {\ssi स्थित्यर्धनाडिकाभ्यस्ता गतयः षष्टिभाजिताः~।\\
लिप्तादि प्रग्रहे शोध्यं मोक्षे देयं पुनः पुनः~॥~१४~॥

तद्विक्षेपैः स्थितिदलं विमर्दार्धं तथासकृत्~।\\
संसाध्यमन्यथा पाते तल्लिप्तादिफलं स्वकम्~॥~१५~॥}
\end{quote}

\begin{sloppypar}
 सूर्यचन्द्रपातानां गतयः स्थित्यर्धघटीभिर्गुणिताः षष्ट्या भक्ताः फलं कलादि प्रग्रहे स्पर्शस्थित्यर्धनिमित्तं सूर्यचन्द्रयोर्हीनं मोक्षे मोक्षस्थित्यर्धनिमित्तं सूर्यचन्द्रयोर्देय योज्यम् । चन्द्रपाते तलिप्तादिफलं स्थित्यर्धघट्यानीतं कलादि पूर्वफलं स्वकं
\end{sloppypar}


\newpage


\noindent  १६२ \hspace{4cm} सूर्यसिद्धान्तः 
\vspace{1cm}

\begin{sloppypar}
\noindent स्वगत्युत्पन्नमन्यथा विपरीतं प्रग्रहस्थित्यर्धनिमित्तं योज्यं मोक्षस्थित्यर्धनिमित्तं हीनमित्यर्थः । तद्विक्षेपैस्तात्कालिकचन्द्रपाताभ्यामानीतशरकलाभिः । कलानां बहुत्वाद्विक्षेपैरिति बहुवचनम् । विक्षेपाभ्यामित्यर्थः । पुनः पुनः स्थितिदलं कार्यम् । अत्रैक पुनः पदं स्पर्शस्थित्यर्धसम्बद्धं द्वितीयं मोक्षास्थित्यर्धसम्बद्धं पुनः पदम् । तेन स्पर्शस्थित्यधार्थसाधितचन्द्रपाताभ्यामानीतशरेण प्रागुक्तप्रकारेण स्पर्शस्थित्यर्धं संसाध्यम् । मोक्षस्थित्यार्धार्थसाधितचन्द्रपाताभ्यामानीनशरेण पूर्वोक्तरीत्या मोक्षस्थित्यर्धं साध्यमित्यर्थः । तच्चोभयमसकृद्वारं वारं स्पर्शस्थित्यर्धानीतचालनेन मध्यकालिकौ चन्द्रपातावुक्तरीत्या प्रचाल्य तच्छरेण पूर्वोक्तरीत्या स्पर्शस्थित्यर्धमस्मादप्युक्तरीत्या स्पर्शस्थित्यर्धमेवं यावदविशेषः । एवं मोक्षस्थित्यर्धानीतचालनेन मध्यकालिकौ चन्द्रपातावुक्तरीत्या प्रचाल्य तच्छरेण पूर्व्वोक्तरीत्या मोक्षस्थित्यर्धमस्मादप्युक्तरीत्या मोक्षस्थित्यमेवं यावदविशेष इत्यर्थः । ननु स्थित्यर्धविमर्धयोरेकरीत्युक्तेः कथं विमर्दार्धमसकृत् साध्यमिति नोक्तमित्यत आह\textendash विमार्धमिति~। तथा स्पर्शमोक्षस्थित्यर्धसाधनरीत्यामकृद्यावदविशेषस्तावत् स्पर्शमर्दार्धं मोक्षमर्दार्धं च संसाध्यम् । यथा हि स्थित्यर्धनाडिकाभ्यस्ता इत्यत्र विमर्दार्धनाडिकाग्रहात् स्पर्शमर्दार्धमोक्षमार्धे साध्ये । आभ्यां प्रत्येकमसकृत् स्पर्शमर्दार्धमोक्षमर्दार्धे स्फुटे स्तः । अत्रोपपत्तिः । प्रागुक्तं क्षेत्रं  स्पर्शमोक्षसम्मीलनोन्मीलनकालिकशरवशादिति तदज्ञानान्मध्यकालिकशरग्रहणेन स्थूलं स्थित्यर्धं मर्दार्धं चातो मध्यकालात्
\end{sloppypar}


\newpage


\hspace{3cm}  गूढार्थप्रकाशकेन सहितः~। \hfill १६३
\vspace{1cm}

\begin{sloppypar}
तदन्तरेण पूर्वाग्निमकालिकयोस्तेषां सम्भवात् तत्कालचालितचन्द्रपाताभ्यां विक्षेपस्तात्कालिको भवति परं स्थूलः । स्थूलस्थित्यर्धाद्यानीतत्वात् । अतोऽस्मादानीतं स्थित्यर्धादि पूर्वापेक्षया सूक्ष्ममपि स्थूलमित्यसकृत् सूक्ष्ममिति । तत्र सम्मीलनोन्मीलनकालयोराकाशस्पर्शमोक्षसम्भवात् स्पर्शमोक्षमार्धमिति ध्येयम्~॥~१५~॥\\
\noindent अथ मध्यग्रहणस्पर्शमोक्षकालानाह\textendash
\end{sloppypar}
%\vspace{2mm}


\begin{quote}
  {\ssi स्फुटतिथ्यवसाने तु मध्यग्रहणमादिशेत्~।\\
स्थित्यर्धनाडिकाहीने ग्रासो मोक्षस्तु संयुते~॥~१६~॥}
\end{quote}
%\vspace{2mm}

\begin{sloppypar}
स्पष्टतिथ्यन्तकाले तुकारात् तत्पूर्वापरकालनिरासः । मध्यग्रहणं ग्रासोपचयसमाप्तिं कथयेत् । मध्यग्रहणसम्बन्धेन मध्यसूर्यचन्द्रानीतमध्यतिथ्यन्ते तत्सम्भव इति कस्यचिङ्भ्रमस्तद्वारणार्थं स्फुटेति । स्थित्यर्धघटिकाभिरूने तिथ्यन्तकाले ग्रास: स्पर्शः । संयुते स्थित्यर्धघटीभिर्युते तिथ्यन्तकाले मोक्षः । तुकारः स्पर्शमोक्षस्थित्यर्धाभ्यां स्पर्शमोक्षकालाविति विषयव्यवस्थार्थकः । अत्रोपपत्तिः । तिथ्यन्तकाले छाद्यच्छादकयोः पर्वापरान्तराभावाद्योगे मण्डलस्पर्शो यावान् भवति ततः पूर्वाग्रिमकालयोर्न्यून एवातोऽत्र मध्यग्रहणकालः । केचित् तु ।
\end{sloppypar}
\begin{quote}
{\qt पर्वान्तः किल साधितो भवलये सूर्येन्दुचिन्हान्तरात्\\
तस्मिन् बिम्बसमागमो न हि यतश्चन्द्रः शराग्रे स्थितः~।\\
तस्मादायनदृष्टिसंस्कृतविधोरानीततिथ्यन्तके\\
बिम्बैक्यं भवतीति किं न विहितं पूर्वैर्न विद्मो वयम~॥}
\end{quote}


{\tiny{X2}}

\newpage

\noindent १६४ \hspace{4cm} सूर्यसिद्धान्तः
\vspace{1cm}

\begin{sloppypar}
 इत्यनेनात्र मध्यग्रहणं खण्डयन्ति । तन्न । पूर्वापरान्तराभावे योगसत्त्वेन कदम्बसूत्रस्थयोर्याम्योत्तरान्तरस्यैव सत्त्वेन तत्र मध्यग्रहणस्योचितत्त्वात् । अन्यथा ध्रुवसूत्रे समसूत्रे वा योगाभ्युपगमे विनिगमनाविरहापत्तेः । यथागतग्रहयोः कदम्बसूत्रेणैव योगाभ्युपगमात् । दृष्टिप्रत्ययार्थं दृक्कर्मोक्तेः । ग्रहणद्वयस्य स्वत एव दृग्गोचरत्वात् । ग्रहद्वयादर्शनाच्चेत्यादिसङ्क्षेपः । मध्यग्रहणकालात् पूर्वं स्पर्शस्थित्यर्धघटीभिः स्पर्श: । अग्रिमकाले मोक्षस्थित्यर्धघटोभिर्मोक्षः स्थित्यर्धयोस्तदन्तररूपत्वेन सिद्धेः~॥~१६~॥\\
\noindent अथ सम्पूर्णग्रहणे निमीलनोन्मीलनकालावप्याह\textendash
\end{sloppypar}
%\vspace{2mm}


\begin{quote}
  {\ssi तद्वदेव विमर्दार्धनाडिकाहीनसंयुते~।\\
निमीलनोन्मीलनाख्ये भवेतां सकलग्रहे~॥~१७~॥}
\end{quote}
%\vspace{2mm}

\begin{sloppypar}
 सम्पूर्णग्रहणे तद्वत् । यथा स्थित्यर्धोनाधिके तिथ्यन्ते स्पर्शमोक्षौ तथेत्यर्थः । एवकारात् तद्भिन्नरोतिव्युदासः । स्पर्शविमर्दार्धमोक्षविमर्दार्धघटीभ्यां क्रमेणानयुते तिथ्यन्ते क्रमेण निमीलनोन्मीलनसञ्ज्ञे स्याताम् । अत्रोपपत्तिः । मर्दार्धस्य मध्यकालात् तदन्तररूपत्वेन तदूनाधिके तस्मिन् क्रमेण निमीलनोन्मीलने सम्पूर्णग्रहण एव भवतः । न्यूनग्रहणे तत्वरूपव्याघातात् तदभावः~॥~१७~॥\\
 \noindent श्रथेष्टकाल इष्टग्रासज्ञानार्थं कोटिकलानयनमाह\textendash
\end{sloppypar}
%\vspace{2mm}


\begin{quote}
 {\ssi इष्टनाडीविहोनेन स्थित्यर्धेनार्कचन्द्रयोः~।\\
भक्त्यन्तरं समाहन्यात् षष्ट्याप्ताः कोटिलिप्तिकाः~॥~१८~॥}
\end{quote}

\newpage




 \hspace{3cm} गूढार्थप्रकाशकेन सहितः~। \hfill १६५
\vspace{1cm}

\begin{sloppypar}
 सूर्यचन्द्रयोर्गत्यन्तरं कलात्मकं ग्रहणारम्भाद्या दृष्टघटिकाः स्पर्धस्थित्यार्धघट्यनधिकास्ताभिरूनेन स्पर्शस्थित्यर्धेन गुणयेत् । अस्मात् षष्टिविभकप्राप्ताः कोटिकला भवन्ति । अत्रोपपत्तिः | इष्टकाले छाद्यच्छादकमण्डलकेन्द्रयोरन्तरं कर्णस्तत्कालशरो भुजस्तत्कालशराग्यमध्यकालिकशराग्रयोरन्तरं विक्षेपवृत्ते कोटिरिति क्षेत्र दृष्टघट्यूनस्पस्थित्यर्धघटिकानां कलाः कोटि: सिद्धा । पूर्व स्पर्शकालिककोट्याः स्थित्यर्धघटिकानां सिद्धत्वात्~॥~१८~॥\\
 \noindent अथात्र सूर्यग्रहणे विशेषमाह\textendash
\end{sloppypar}
%\vspace{2mm}


\begin{quote}
 {\ssi भानोर्ग्रहे कोटिलिप्ता मध्यस्थित्यर्धसङ्गुणाः~।\\
स्फुटस्थत्यर्धसम्भक्ताः स्फुटाः कोटिकलाः स्मृताः~॥~१९~॥}
\end{quote}
%\vspace{2mm}

\begin{sloppypar}
 सूर्यस्य ग्रहण उक्तप्रकारेण याः कोटिकलाः सूर्यग्रक्षणौक्तस्पष्टस्थित्यर्धानोता मध्यस्थित्यर्धेन सूर्यग्रहणोक्तस्पष्टशरानीतस्थित्यर्धेन सङ्गुणिता: स्फुटस्थित्यर्धेन सूर्यग्रहणाधिकारोक्तेन भक्ता: सत्यः स्पष्टा: कोटिकलाः सूर्यग्रहणतत्त्वज्ञैरुक्ताः । अत्रोपपतिः । सूर्यग्रहणे स्पर्शमोक्षान्यतरमध्यकालयोरन्तरस्य स्थित्यर्धत्वात् तस्य च स्यष्टशरोद्धतस्थित्यर्धलम्बनान्तरैक्यसंस्कारमितत्वात् स्यष्टस्थित्यर्धानुरुद्धा । उक्तरीत्यानीताः कोटिकलाः । अपेक्षिताश्च स्यष्टशरोद्धूतस्थित्यर्धानुरुद्धाः एतत् कोटिसम्बद्धं क्षेत्रम् । स्थित्यर्धक्षेत्रान्तर्गतत्वात् । स्पष्टस्थित्यर्धस्य तूक्तक्षेत्रोत्पन्नत्वाभावात् । अन्यथा स्पष्टशरोङ्भूतस्थित्यर्धस्य लम्बनान्तरैक्यसंस्कारानुक्तिप्रसङ्गः । अतः स्पष्टस्थित्यर्धेनैता आगताः
\end{sloppypar}

\newpage



\noindent  १६६ \hspace{4cm} सूर्यसिद्धान्तः 
\vspace{1cm}


\noindent कोटिकलास्तदा स्पष्टशरोद्भूतक्षेत्रजमध्यमरूपस्थित्यर्धेन का इति स्फुटाः कलाः सिद्धाः~॥~१९~॥\\
\noindent अथाभ्य इष्टग्रासानयनमाह\textendash
%\vspace{2mm}


\begin{quote}
  {\ssi क्षेपो भुजस्तयोर्वर्गयुतेर्मूलं श्रवस्तु तत्~।\\
मानयोगार्धतः प्राज्झ्य ग्रासस्तात्कालिको भवेत्~॥~२०~॥}
\end{quote}
%\vspace{2mm}

\begin{sloppypar}
 क्षेपो विक्षेपो भुजः । कोटिभुजयोः कर्णसापेक्षत्वादाह\textendash तयोरिति । कर्णस्तु तयोः कोटिभुजयोर्वर्गयोगान्मूलं सिद्ध एव । तत् कर्णवर्गात्मकं मूलं ग्राह्यग्राहकमानैक्यार्धाद्विशोध्य शेषं तात्कालिकः कल्पितेष्टकालसम्बन्धी ग्रासोऽवान्तरग्रासः स्यात् । अत्रोपपत्तिः । क्षेत्रं पूर्वं प्रतिपादितम् । स्पर्शकाले मानैक्यखण्डस्य कर्णत्वात् क्षेत्रयोरुभयोर्मध्यकालावधित्वादिष्टकर्णोनं मानैक्यखण्डमिष्टग्रास एव~॥~२०~॥\\
 \noindent अथ मध्यग्रहणानन्तरमिष्टग्रासानयनमाह\textendash
\end{sloppypar}
%\vspace{2mm}


\begin{quote}
  {\ssi मध्यग्रहणतश्चोर्ध्वमिष्टनाडोर्विशोधयेत्~।\\
स्थित्यर्धान्मौक्षिकाच्छषं प्राग्वच्छेषं तु मौक्षिके~॥~३१~॥}
\end{quote}
%\vspace{2mm}

\begin{sloppypar}
मध्यग्रहणकालादूर्ध्वमनन्तरम् । चकारो विशेषार्थकतुकारपरः । इष्टघटिकाः कर्म । मौक्षिकानमोक्षकालसम्बद्धात् स्थित्यर्धात् । न स्पर्शस्थित्यर्धात् । विशोधयेत् । गणक इति कर्त्राक्षेपः । शेषं कोटिलिप्तादिग्रासानयनान्तं गणितकर्म प्राग्वद्भुक्त्यन्तरं समाहन्यादित्युक्तप्रकारेण कुर्यात् । मौक्षिके मोक्षस्थित्यर्धान्तर्गतेष्टकाले तुर्विशेषे ग्रासः शोधमुर्वरितो ग्रासोऽवान्तरग्रासो
\end{sloppypar}

\newpage



\hspace{3cm}गूढार्थप्रकाशकेन सहितः~। \hfill १६७
\vspace{1cm}

\begin{sloppypar}
\noindent भवति । न पूर्ववद्गतः । अत्रोपपत्तिः । मध्यग्रहणात् पूर्वमिष्टकालस्य ग्रहणारम्भावधिकस्य स्पर्शस्थित्यर्धसम्बद्धत्वादागतो ग्रास उपचयात्मकः । नावशिष्टः । अवशिष्टमण्डलय शुद्धत्वेन ग्रस्तत्वासम्भवात् । एवं मध्यग्रहणानन्तरमिष्टकालस्य मोक्षस्थित्यर्धान्तर्गतत्वादुक्तरीत्यानीतो ग्रासोऽपचयात्मकः । न शुद्धबिम्बदर्शनात्मकः । ग्रस्तत्वाभावात्~॥~२१~॥\\
\noindent अथाभीष्टग्रासादिष्टकालानयनं श्लोकाभ्यामाह\textendash
\end{sloppypar}
%\vspace{2mm}


\begin{quote}
  {\ssi ग्राह्यग्राहकयोगार्धाच्छोध्याः स्वच्छन्नलिप्तिकाः~।\\
तद्वर्गात् प्रोज्झ्य तत्कालविक्षेपस्य कृतिं पदम्~॥~२२~॥

कोटिलिप्ता रवेः स्पष्टस्थित्यर्धेनाहता हृताः~।\\
मध्येन लिप्तास्तन्नाङ्यः स्थितिवद्ग्रासनाडिकाः~॥~२३~॥}
\end{quote}
%\vspace{2mm}

\begin{sloppypar}
 छाद्यच्छादकमानैक्यखण्डादभीष्टग्रासकलाः शोध्याः । शेषस्य वर्गादभीष्टग्रासकालिकविक्षेपस्य वर्गं विशोध्य शेषस्य मूलं कोटिकलाः । सूर्यग्रहणे विशेषमाह\textendash रवेरिति~। सूर्यस्य ग्रहण इति शेषः । भानोर्ग्रह इति पूर्वमुक्तेः । उक्तप्रकारेण याः कलास्ता मध्यग्रहणकालस्पर्शमोक्षान्यतरकालयोरन्तररूपेण स्पष्टस्थित्यर्धेन गुण्याः । स्पष्टशरोत्पन्नस्थित्यर्धेन मध्यमेन भक्ताः फलं कोटिकला भवन्ति । स्थितिवत् स्थित्यर्धसाधनरीत्या ।
\end{sloppypar}

\begin{quote}
 {\qt षट्या सङ्गुण्य सूर्येन्द्वोर्भुक्त्यन्तरविभाजिताः~।}
\end{quote}

\begin{sloppypar}
 इत्युक्तेन तासां कोटिकलानां घटिका यास्ता अभीष्टग्राससम्बन्धिघटिकाः स्पर्शमोक्षान्यतरस्थित्यर्धान्तर्गताः क्रमेण मध्य
\end{sloppypar}

\newpage



\noindent १६८ \hspace{4cm} सूर्यसिद्धान्तः 
\vspace{1cm}

\begin{sloppypar}
\noindent ग्रहणाच्छेषागता वा भवन्ति । अत्रोपपत्तिः पूर्वोक्तव्यत्यासात् सुगमतरा । परन्तु स्वाभीष्टग्रासकालिकशरज्ञाने सूक्ष्मम् । तच्छराज्ञाने मध्यकालिकशरग्रहणेन स्थूलम् । अत एव भास्कराचार्यैः कालसाधने तत्कालबाणेन मुहुः स्फुट इत्युक्तमिति विशेषः~॥~२३~॥\\
\noindent अथ वक्ष्यमाणग्रहणपरिलेखोपयुक्तवलनस्यानयनं श्लोकाभ्यामाह\textendash
\end{sloppypar}
%\vspace{2mm}


\begin{quote}
 {\ssi नतज्याक्षज्ययाभ्यस्ता त्रिज्याप्ता तस्य कार्मुकम्~।\\
 वलनांशाः सौभ्ययाम्याः पूर्वापरकपालयोः~॥~२४~॥

 राशित्रययुताद्ग्राह्यात् क्रान्त्यंशैर्दिक्समैर्युताः~।\\
 भेदेऽन्तराज्ज्या वलना सप्तत्यङ्गुलभाजिता~॥~२५~॥}
 \end{quote}
%\vspace{2mm}

\begin{sloppypar}
 यत्कालिकं वलनं कर्तुमिष्टं तात्कालिकं नतं चन्द्रग्रहणे चन्द्रस्य सूर्यग्रहणे सूर्यस्य माध्यम् । तद्यथा स्वोदयात् स्वास्ताद्गतशेषघटिकाः खदिनार्धान्तर्गताः स्वदिनार्धादूनाः क्रमेण पूर्वापरनतघटिका भवन्ति । तन्नतं नवतिगुणं स्वदिनार्धभक्तं नतांशास्तेषां ज्या नतज्येत्यर्थः । खदेशाक्षांशज्यया गुणिता त्रिज्यया भक्ता फलस्य धनुः कलात्मकं षष्टिभक्तं पूर्वापरकपालयोः पूर्वापरनतयोः क्रमेणेत्तरदक्षिणा वलनांशा भवन्ति । यत्कालिकं वलनं तात्कालिकाद्ग्राह्याद्राशित्रययुतात् सायनांशाद्ये क्रान्त्यंशास्तैर्दिक्तुल्यैर्युतास्तेषां ज्या भेदे भिन्नदिक्त्वेऽन्तरात् क्रान्त्यंशवलनांशयोरन्तराज्ज्या सप्तत्यङ्गुलैर्भक्ता शेषदिक्का । अङ्गुलात्मकत्वेन हरस्योद्देशादङ्गुलादिका वलना भवति । अत्रोपपत्तिः ।
\end{sloppypar}

\newpage




 \hspace{3cm} गूढार्थप्रकाशकेन सहितः~। \hfill १६९
\vspace{1cm}

\begin{sloppypar}
\noindent समवृत्तपूर्वापरादिदिग्भ्यः क्रान्तिवृत्तपूर्वापरादिदिशो यावतान्तरेण वलिता उत्तरस्यां दक्षिणस्यां वा वलनांशाः । तदानयनार्थं प्रथमतः समवृत्तानुरुद्धदिग्भ्यो विषुवद्वृत्तदिशो यावतान्तरेण वलिता दक्षिणोत्तरयोस्तदाक्षवलनम् । तथाहि । समप्रोतचलवृत्तं ग्रहचिन्हस्थं समविषुवद्वृत्तयोर्यत्र लग्नं तत्प्रदेशान्नवत्यंशान्तरे स्वस्ववृत्ते प्राच्योरन्तरं वलनं तत्तुल्यमेवेतरदिशामन्तरं पूर्वकपालस्थग्रहे समवृत्तप्राचीतो विषुववृत्तप्राच्या उत्तरत्वादुत्तरम् । पश्चिमकपालस्थे तु समवृत्तप्राचीतो विषुवद्वृत्तप्राच्या दक्षिणत्वाद्दक्षिणम् । तत्र क्षितिजस्थे ग्रहे तदन्तरमक्षांशतुल्यम् । याम्योत्तरवृत्तस्थे ग्रहे तदन्तराभावः । अतस्त्रिज्यातुल्यया नतकालज्ययाक्षज्यातुल्याक्षवलनज्या तदेष्टनतज्यया केत्यनुपातागताक्षज्याया धनुराक्षं वलनमुक्तमुपपन्नम् । द्वितीयं तु विषुवद्वृत्तदिग्भ्यः क्रान्तिवृत्तदिशो यावतान्तरेण वलिता दक्षिणेत्तरयोस्तदायनं वलनम् । तथाहि ध्रुवप्रोतवृत्तं ग्रहचिन्हस्थं विषुवद्वृत्ते यत्रासन्नं लगति तत्स्थानाच्चतुर्थाशान्तरे यत् स्थानं तद्विषुवत्प्राची । तस्या ग्रहचिन्हात् त्रिभान्तरितक्रान्तिवृत्तप्राची यदन्तरेण तदायनं वलनम् । तत्तुल्यमेवेतरदिशामन्तरम् । उत्तरायणस्थे ग्रह उत्तरं दक्षिणायनस्थे ग्रहे दक्षिणम् । तत् त्वयनसन्धावभावात्मकम् । गोलसन्धौ परमक्रान्तितुल्यमतः सत्रिभक्रान्तितुल्यं सत्रिभग्रहगोलदिक्कमित्युपपन्नं राशित्रययुताद्ग्राह्यात् क्रान्त्यंशैरिति । द्वयोर्वलनयोरेकदिक्त्वे समवृत्तप्राचीतः क्रान्तिवृत्तप्राची तद्योगरूपस्फुटवलना\textendash
\end{sloppypar}

{\tiny{Y}}

\newpage





 \noindent १७० \hspace{4cm} सूर्यसिद्धान्तः 
\vspace{1cm}

\begin{sloppypar}
\noindent न्तरेण वलनदिशि भवति । भिन्नदिक्त्वे तु वलनान्तररूपस्फुटवलनान्तरेण श्रेषदिशि भवति । तज्ज्या स्फुटवलनज्या त्रिज्यावृत्ते । अग्रे परिलेख एकोनपञ्चाशन्मितव्यासार्धवृत्ते दानार्थं त्रिज्यावृत्त इयं तदैकोनपञ्चाशन्मितव्यासार्धे केत्यनुपाते प्रमाणेच्छयोरिच्छापवर्तनाद्धरस्थानेऽधोवयवत्यागात् सप्ततिः। अतो दिक्यमैर्युता इत्याद्युपपन्नम्~॥~२५~॥\\
\noindent अथ कलात्मकबिम्बविक्षेपादीनामङ्गुलीकरणमाह\textendash
\end{sloppypar}
%\vspace{2mm}


\begin{quote}
  {\ssi सोन्नतं दिनमध्यर्धं दिनाधीप्तं फलेन तु~।\\
छिन्द्याद्विक्षेपमानानि तान्येषामङ्गुलानि तु~॥~२६~॥}
\end{quote}
%\vspace{2mm}

\begin{sloppypar}
 दिनमानमध्यर्धमर्ध इत्यध्यर्धं स्वार्धयुक्तमित्यर्थः । अभीष्टकालिकोन्नतघटीभिः सहितं दिनार्धेन भक्तं फलेन । तुकारो यद्ग्रहणं तस्य दिनमानोन्नते ग्राह्यो इत्यर्थकः । विक्षेपग्राह्यग्राहकबिम्बमानानि । तानि पूर्वोक्तानि कलात्मकानि । ग्रासादिकमपि ध्येयम् । भजेत् । तुकारात् फलमेषां कलात्मकानामङ्गुलानि भवन्ति । अत्रोपपत्तिः । उदयास्तकाले बिम्बकिरणानां भूमिगोलावरुद्धत्वेनाल्योर्ध्वस्यकिरणानां नयनप्रतिहननानर्हत्वाद्ध्विम्बं व्यक्तत्वान्महद्भासते । तत्राङ्गुलात्मकं बिम्बं कलात्रयात्मैककाङ्गुलप्रमाणेन भवति । स्वमध्यस्थे ग्रहे तु बिम्बस्य सर्वकिरणावरुद्धत्वान्नयनप्रतिघाताच्च सूक्ष्मं बिम्बं भासते । तत्राङ्गुलात्मकं बिम्बं कलाचतुष्टयात्मकैकाङ्गुलप्रमाणेन भवति । तत्रोदयास्तकाले शङ्कोरभावात् खमध्ये तस्य त्रिज्यातुल्यत्वात् त्रि
\end{sloppypar}

\newpage

\hspace{3cm}  गूढार्थप्रकाशकेन सहितः~। \hfill १७१
\vspace{1cm}

\begin{sloppypar}
\noindent ज्यातुल्यशङ्कावुदयकालिकौकाङ्गुलमानस्य कलात्रयस्तैकाङ्गलमुपचयो लभ्यते तदेष्टशङ्कौ क इत्यनुपातेनाभीष्टकाले फलं युक्तम् । त्रयमेकाङ्गुलस्य कलात्मकं मानं भवति । अत एव भास्कराचार्यैरुदयास्तकाले सार्धद्वयं कलाङ्गुलमानमङ्गीकृत्य ।
\end{sloppypar}
%\vspace{2mm}


\begin{quote}
 {\qt त्रिज्याद्धृतस्तात्समयोत्यशङ्कुः\\
सार्धद्वियक्तोऽङ्गुललिप्तिकाः स्युः~।}
\end{quote}
%\vspace{2mm}

\begin{sloppypar}
इत्युक्तम् । तत्र भगवता लोकानुकम्पया स्वल्पान्तरत्वाच्च मध्यान्हेऽपि कलाचतुष्टयात्मकमेकाङ्गुलमङ्गीकृत्य दिनार्धतुल्यपरमोन्नतकाल एवोपचयस्तदेष्टोन्नतकाले क इत्यनुपातागतफलयुक्तं त्रयं कला एकाङ्गुलमानमभीष्टकाले । तत्र दिनार्धभक्तोन्नतकालस्य फलरूपत्वात् त्रयाणां समच्छेदतया योजने त्रिगुणितं दिनार्धं सार्धैकगुणदिनमानरूपमुन्नतकालयुक्तं दिनार्धभक्तमिति सिद्धम् । तत एतत्कलाभिरेकाङ्गुलं तदेष्टकलाभिः किमित्यनुपातेन कलात्मकानामङ्गुलीकरणमुक्तमुपपन्नम्~॥~२६~॥\\
\noindent अथाग्रिमग्रन्थस्यासङ्गतित्वनिरासार्थमधिकारसमाप्तिं फक्किकयाह\textendash
\end{sloppypar}
%\vspace{2mm}

{\setlength{\parindent}{5em}
 इति चन्द्रग्रहणाधिकारः~॥
}

स्पष्टम् ।
%\vspace{2mm}


\begin{quote}
{\qt रङ्गनाथेन रचिते सूर्यसिद्धान्तटिप्पणे~।\\
चन्द्रग्रहाधिकारोऽयं पूर्णो गूढप्रकाशके~॥}
\end{quote}
%\vspace{2mm}


इति श्रीसकलगणकसार्वभौमबल्लालदैवज्ञात्मजरङ्गनाथगणकविरचिते गूढार्थप्रकाशके चन्द्रग्रहणाधिकारः पूर्णः~॥

 

{\tiny{Y2}}

\newpage





\noindent १७२ \hspace{4cm} सूर्यसिद्धान्तः 
\vspace{1cm}

\begin{sloppypar}
 अथ सूर्यग्रहणाधिकारो व्याख्यायते । तत्र यत्यदार्थविशेषप्रयुक्तश्चन्द्रग्रहणाधिकारातिरिक्तः सूर्यग्रहणाधिकारस्तद्विशेषयोरभावस्थानादेवोत्पत्तिनियमात् तयोरभावस्थानकथनव्याजेन तयोरुद्देशमाह\textendash
\end{sloppypar}
%\vspace{2mm}


\begin{quote}
  {\ssi मध्यलग्नसमे भानौ हरिजस्य न सम्भवः~।\\
अक्षोदङ्मध्यभक्रान्तिसाम्येनावनतेरपि~॥~१~॥}
\end{quote}
%\vspace{2mm}

\begin{sloppypar}
 सूर्यऽमावास्यान्तकालिके मध्यलग्नसमे सति दिनमध्यस्थान ऊर्ध्वयाम्योत्तरवृत्ते लग्नः क्रान्तिवृत्तप्रदेशो मध्यलग्नं त्रिप्रश्नाधिकारोक्तम् । तत्तुल्ये मति मध्याह्न इति फलितम् । हरिजस्य लम्बनस्य भूपृष्ठक्षितिजवशाल्लम्बनोत्पत्तेर्लम्बनस्यापि क्षितिजवाचकहरिजशब्देनाभिधानात् सम्भव उत्पत्तिर्न । तत्र लम्बनाभाव इत्यर्थः । अथ मध्यान्ह इति स्फुटोक्त्यपेक्षया मध्यलग्नसम इति वक्रोक्तिः कृपालोर्भगवतो नोचितेत्यग्रिमग्रन्थार्थतत्त्वविचारणयापि मध्यान्हे तदभावानुपपन्तेः साम्प्रदायिकव्याख्यामनादृत्य तत्त्वार्थो व्याख्यायते । लग्नयोरुदयक्षितिजास्तक्षितिजप्रदेशयोः संलग्नक्रान्तिहत्तप्रदेशयोर्मध्यम् । ऊर्ध्वमध्यप्रदेशयोस्त्रिभोनलग्नमित्यर्थः । प्रयोगस्तु मध्यान्ह इतिवत् । तत्तुल्येऽर्के लम्बनस्याभाव इति ।
\end{sloppypar}
%\vspace{2mm}


\begin{quote}
{\qt दर्शान्तलग्नं प्रथमं विधाय न लम्बनं वित्रिभलग्नतुल्ये~।\\
रवौ तदूनेऽभ्यधिके च तत् स्यादेवं धनर्णं क्रमशश्च वेद्यम्~॥}
\end{quote}

इति भास्कराचार्येण स्फुटमुक्तेश्च । नत्यभावस्थानमाह\textendash

\newpage




\hspace{3cm} गूढार्थप्रकाशाकेन सहितः~। \hfill १७३
\vspace{1cm}

\begin{sloppypar}
अक्षेत्यादि~। अक्षांशा उत्तरा ये मध्यभस्य मध्यलग्नस्य क्रान्त्यंशाः । अत्र मध्यस्लग्नशब्देन दशमभावस्त्रिभानलग्नं वा ग्राह्यमुभयपक्षेऽप्यदोषः । अनयोस्तुल्यत्वेऽवनतेर्नतेः। अपिशब्दात् सम्भवः । नाभाव इत्यर्थः । न त्वपिशब्दाल्लम्बनस्यापि तत्राभावः। उत्तरक्रान्त्यक्षयोस्तुल्यत्वे मध्यलग्नतुल्यार्कत्वाभावेऽपि तदभावापत्तेः । अत्रोपपत्तिः । अमावास्यान्तकाले समौ सूर्यचन्द्रौ । तत्र चन्द्रशराभावे भूगर्भान्नीयमानं सूत्रमर्कस्थानावधि चन्द्रं स्पृशत्येवेति भूगर्भे छादकत्वं चन्द्रस्य सूर्यस्य छाद्यत्वं सम्भवति । तत्र मनुष्याणामसत्त्वाद्भूपृष्ठे तेषां सत्त्वाच्च भूपृष्ठान्नीयमानमर्कोपरि सुत्रं चन्द्रे न लगत्येव । किन्तु चन्द्राधिष्ठानगाले चन्द्रचिन्हादूध्र्वं लगति । तत्र यदा चन्द्र आयाति तदा भूपृष्ठे सूर्यस्य चन्द्रश्छादको भवति । यदा तु खमध्ये सूर्यस्तदा भूगर्भसूत्रं भूपृष्ठसूत्रं च सूर्यापरिगमेकमेव चन्द्रे लगतीति भूपृष्ठेऽमान्तकाले चन्द्रश्च्छादको भवति । अत एव भूगर्भपृष्ठसूत्रान्तरं लम्बनम् । भूपृष्ठसूत्रात् सृर्योपरिगाच्चन्द्राधिष्ठानाकाशगोले चन्द्रस्य शरसत्त्वे चन्द्रचिन्हस्य वा लम्बितत्वात् । अत एव भास्कराचार्यैरुक्तम् ।
\end{sloppypar}
%\vspace{2mm}


\begin{quote}
 {\qt दृग्गर्भसूत्रयोरैक्यात् खमध्ये नास्ति लखनम्~।}
 \end{quote}
%\vspace{2mm}

\begin{sloppypar}
 इति । अथ चन्द्राधिष्ठानगोले भूपृष्ठसूत्रमर्केपरि गतं चन्द्राचिन्हादूर्ध्वं, चन्द्रदृग्वृत्ते यदंशैर्लगति तल्लम्बनं दृग्वृत्ताकारक्रान्तिवृत्ते भवति । यदा तु दृग्वृत्ताद्भिन्नं क्रान्तिवृत्तं तदा भूपृष्ठसूत्रं चन्द्राधिष्ठानगोले चन्द्रदृग्वृत्ते चन्द्रादूर्ध्वं यत्र लग्नं
\end{sloppypar}

\newpage


\noindent  १७४ \hspace{4cm} सूर्यसिद्धान्तः
\vspace{1cm}

\begin{sloppypar}
\noindent तत्र चन्द्रगोलस्थक्रान्तिवृत्तयाम्योत्तररूपकदम्बप्रोतवृत्तमानीय चन्द्रगोलस्थक्रान्ति वृत्ते यत्र लग्नं तच्चन्द्रचिह्नयोरन्तरं क्रान्तिवृत्ते पूर्वापरं स्फुटलम्बनकला: कोटिः । चन्द्रस्य क्रान्तिवृत्तानुसारेण गमनात् प्रोतवृत्ते क्रान्तिवृत्तदृग्वृत्तयोरन्तरं याम्योत्तरं कलात्मकं नतिर्भुजः । भूगर्भपृष्ठसूत्रान्तरं दृग्वृत्ते कलात्मकं दृग्लम्बनं कर्णः । दृग्वृत्तस्य कदम्बप्रोतवृत्ताकारत्वे क्रान्तिवृत्ते तयोरन्तराभावाल्लम्वनाभावः । याम्योत्तर्मन्तरं दृग्लम्बनं नतिरेवोत्पन्ना । दृग्वृत्ताकारक्रान्तिवृत्ते तु दृग्लम्बनमेव क्रान्तिवृत्ते तयोरन्तरमिति लम्बनमुत्पन्नं नत्यभावश्च। तथा च दृग्वृत्तस्य कदम्बप्रोतवृत्ताकारत्वे त्रिभोनलग्नस्थानेऽर्को भवति । तद्वृत्तस्य क्रान्तिवृत्तयाम्योत्तरत्वेनोदयास्तलग्नमध्यवर्तित्वेन लग्नस्थानात् त्रिभान्तरितत्वात् । नहि क्रान्तिवृत्ताद्याम्योत्तरान्तरज्ञानार्थं समप्रोतवृत्तमङ्गीकार्यम् । येन दशमभावतुल्यार्के लम्बनाभाव उपपन्नः स्यात् । क्रान्तिवृत्तस्य गोलवृत्तत्वेन समप्रोतवृत्तस्य देशवृत्तत्वेन सम्बन्धाभावात् । अत एव भगवता सर्वज्ञेन नतिसाधनार्थमग्रे दृक्क्षेपः कदम्बप्रोतवृत्ते त्रिभोनलग्नस्यैव साधितः । दृक्क्षेपाभावे त्रिभोनलग्नस्य खमध्यस्थत्वेन तदा तस्य दशमभावतुल्यत्वेन दशमभावनतांशाभावादृकक्षेपाभावः । तदा त्रिभोनलग्नस्य नतांशाभावश्च । नतांशाभावस्त्वक्षांशतुल्योत्तरक्रान्तौ सुखार्थं स्थूलाङ्गीकारे तु दशमभावस्यैव नतांशोन्नतज्ज्ये दृक्क्षेपदृग्गती नतिलम्बनयोः साधनार्थं समनन्तरमेव भगवतोक्तेर्न तु वस्तुरूपे । आयासेन
\end{sloppypar}

\newpage


\hspace{3cm} गूढार्थप्रकाशकेन सहितः~। \hfill १७५
\vspace{1cm}

\begin{sloppypar}
दृकक्षेपसाधनस्योक्तस्य वैयर्थ्यापत्तेरिति सर्वं निरवद्यम् ~॥~१~॥\\ 
अथोद्दिष्टयोरभावस्थानातिरिक्तस्थाने सम्भवात् प्रतिपादनं प्रतिजानीते\textendash
\end{sloppypar}
%\vspace{2mm}


\begin{quote}
  {\ssi देशकालविशेषेण यथावनतिसम्भवः~।\\
लम्बनस्यापि पूर्वान्यदिग्वशाच्च तथोच्यते~॥~२~॥}
\end{quote}
%\vspace{2mm}

\begin{sloppypar}
 देशविशेषेण कालविशेषेणावनतिसम्भवो नतिकालोत्पत्तिर्गालस्थित्या यथा भवति । लम्बनस्यापि समुच्चये त्रिभोनलग्नस्थानात् पूर्वापरदिगनुरोधात् । चकारात् सम्भवो देशकालविशेषेण यथा भवतीत्यर्थः । तथा तत्तुल्येन नतिलम्बने आनयनद्वारा मया कथ्येते~॥~२~॥\\
 \noindent तत्रोपयुक्तामुदयाभिधामाह\textendash
\end{sloppypar}


\begin{quote}
  {\ssi लग्नं पर्वविनाडीनां कुर्यात् स्वैरुदयासुभिः~।\\
तज्ज्यान्त्यापक्रमज्याघ्नी लम्बज्याप्तोदयाभिधा~॥~३~॥}
\end{quote}
\begin{sloppypar}
 स्वैः स्वदेशीयैरुदयासुभी राभ्युदयासुभिः पर्वघटिकानां लग्नं गणकः कुर्यात् । पर्वान्तकालिकं लग्नं साध्यमित्यर्थः । यद्यपि पूर्वं लग्नसाधनं स्वोदयैरेवोक्तमिति स्वैरुदयासुभिरिति व्यर्थं तथापि समनन्तरमेव दशमभावमाधनोक्त्या कस्यचिल्लग्नं व्यक्षोदयैरेवात्र साध्यमिति भ्रमस्य वारणाय पुनरुक्तिः । तस्य लग्नस्थायनांशसंस्कृतस्य ज्या भुजज्या परमक्रान्तिज्यया गुण्या स्वदेशीयलम्बज्यया भक्ता फलमुदयसञ्ज्ञं स्यात् । अत्रोपपत्तिः । लग्नक्रान्तिज्यासाधनार्थं लग्नभुजज्यायाः परमक्रान्तिज्या गुणस्त्रिच्या हरस्ततो लम्बज्याकोटौ त्रिज्याकर्णस्तदा लग्नक्रान्ति\textendash
\end{sloppypar}

\newpage



\noindent १७६ \hspace{4cm} सूर्यसिद्धान्तः 
\vspace{1cm}

\begin{sloppypar}
\noindent ज्याकोटौ कः कर्ण इत्यनुपाते त्रिज्ययोर्नाशाल्लग्नभुजज्या परमक्रान्तिज्यागुणा लम्बज्यया भक्ता फलं लग्नस्याग्रा । इयं भगवतोदयसञ्ज्ञोक्ता लग्नस्योदयसञ्ज्ञत्वात् । उदयसम्बन्धाच्चेत्युक्तमुपपन्नम्~॥~३~॥\\
\noindent अथोपयुक्तां मध्यज्यां सार्धश्लोकेनाह\textendash
\end{sloppypar}
%\vspace{2mm}


\begin{quote}
  {\ssi तदा लङ्कोदयैर्लग्नं मध्यसञ्ज्ञं यथोदितम~।
तत्क्रान्त्यक्षांशसंयोगो दिकसाम्येन्तरमन्यथा~॥~४~॥

शेषं नतांशास्तन्मौर्वी मध्यज्या साभिधीयत~।}
\end{quote}
%\vspace{2mm}

\begin{sloppypar}
 तदा पर्वान्तकाले लङ्कोदयैर्व्यक्षदेशोयराश्युदयैर्यथोदितं पूर्वोक्तप्रकारेण जातकपद्धत्युक्तनतघटोभिर्धनमृणं यथायोग्य मध्यसञ्ज्ञ लग्नं दशमभावात्मकं साध्यम् । अत्र लग्नसम्बन्धेन स्वदेशराभ्युदयासुग्रहणशङ्कावारणाय लङ्कोदयैरित्युक्तम् । तस्य दशमभावस्यायनांशसंस्कृतस्य क्रान्तिः स्वदेशाक्षांशाः । अनयोर्योग एकदिक्त्वे कार्यः । अन्यथा भिन्नदिक्त्वेऽन्तरं तयोरेव शेषं संस्कारजदिक्का नतांशास्तेषां ज्या कार्या सा मध्यलग्नतांशज्या मध्यज्योच्यते तत्सम्बन्धात् । अत्रोपपत्तिः स्पष्टा~॥~४~॥\\
 \noindent अथाभ्यामुपयुक्तं दृक्क्षेपं लम्बनोपयुक्तां दृग्गतिं च सार्धश्लोकेनाह\textendash
\end{sloppypar}
%\vspace{2mm}


\begin{quote}
 {\ssi मध्योदयज्ययाभ्यस्ता त्रिज्याप्ता वर्गितं फलम्~॥~५~॥
 
मध्यज्यावर्गविश्लिष्टं दृक्क्षेपः शेषतः पदम्~।\\
तत्रिज्यावर्गविश्लेषान्मूलं शङ्कुः सदृग्गतिः~॥~६~॥}
\end{quote}
%\vspace{2mm}

\begin{sloppypar}
 पूर्वोक्तमध्यज्या पूर्वानीतोदयाभिधयोदद्यज्यया । अस्या ज्यारूपत्वाज्ज्ययेत्युतम् । गुणिता त्रिज्यया भक्ता फलं वर्गितं वर्गः
\end{sloppypar}
\newpage

\hspace{3cm} गूढार्थप्रकाशकेन सहितः~। \hfill १७७
\vspace{1cm}

\begin{sloppypar}
\noindent सञ्जातो यस्य तत् । फलस्थ वर्ग: कार्य इत्यर्थः । मध्यज्याया वर्गे विश्लिष्टं हीनं वर्गितं फलं कार्यम् । शेषान्मूलं दृकक्षेप: स्यात् । दृक्क्षेपत्रिज्ययोर्यौ वर्गों तथोरन्तरान्मूलं शङ्कुः । स आनीतः शङ्कुर्दृग्गतिसञ्ज्ञो भवति। न तु शङ्कुमात्रम्। अत्रोपपत्तिः। त्रिभोनलग्नस्य दृग्ज्यानयनार्थं क्षेत्रम्। मध्यलग्नदृग्ज्याकर्णस्त्रिभोनलग्नस्य याम्योत्तरवृत्तात् प्रागपरस्थितत्वेन तत्स्वस्वस्तिकान्तरस्थिततदीयदृग्वृत्तप्रदेशांशज्या कोटिः । मध्यलग्नत्रिभोनलग्रान्तरांशज्या क्रान्तिवृत्तस्या भुजः । अत्र भुजानयनं चोदयलग्नस्थक्रान्तिवृत्तप्रदेशः । प्राक्स्वसिकात् तदग्रान्तरेणोत्तरदक्षिणो भवति। एवमस्तलग्नप्रदेशः परस्वस्तिकाद्दक्षिणोत्तरः । तदनुरोधेन च त्रिभोनलग्नप्रदेशक्रान्तिवृत्तीययाम्योत्तरवृत्तरूपतद्वृग्वृत्तं क्षितिजे याम्योत्तरवृत्तक्षितिजसम्पातात् तदाग्रान्तरेण लग्नमवश्यं भवति। अतस्त्रिज्यातुल्यमध्यलग्नदृग्ज्यया लग्नाग्रातुल्यो भुजस्तदाभीष्टतदृग्ज्यया क इत्यनुपातेन स फलसञ्ज्ञः । तद्वर्गोनान्मध्यलग्नदृग्ज्यावर्गान्मूल त्रिभोनलग्नस्य दृग्ज्या दुक्क्षेपाख्या। एतद्वर्गोनात् त्रिज्यावर्गान्मूलं त्रिभोनलग्नशङ्कुदृग्गतिसञ्ज्ञः। अत्रेदमवधेयम्। त्रिप्रनाधिकारोक्तप्रकारेण त्रिभोनलग्नस्य शङ्कुदृग्ज्ये दृग्गतिदृक्क्षेपतुल्ये न भवतः । किन्तु दुग्गतिदृक्क्षेपाभ्यां क्रमेण न्यूनाधिके भवतः सर्वदा धूलीकर्मणानुभवात्। अत आनीतोऽयं दृक्क्षेपस्त्रिभोनलग्नदृङ्मण्डलस्थितोऽपि ग त्रिज्यानुरुद्धः । किन्तु फलवर्गोनत्रिज्यावर्गपदरुपविलक्षणवृत्तव्यासार्धप्रमाणेन सिद्ध इति गम्यते। अतो दृग्ज्या\textendash
\end{sloppypar}

{\tiny{z}}


\newpage



\noindent १७८ \hspace{4cm} सूर्यसिद्धान्तः
\vspace{1cm}

\begin{sloppypar}
\noindent यास्त्रिज्यानुरुद्धत्वेन त्रिज्यावृत्तपरिणतो दृक्क्षेपस्त्रिभोनलग्नस्य दृज्ज्या स्फुटदृक्क्षेपरूपा । अस्यास्तत्त्रिज्यावर्गेत्यादिना दृग्गतिः स्फुटा त्रिभोनलग्नशङ्कुरूपा। एतदनुक्तिः स्वल्पान्तरत्वाद्गणितसुखार्थं कृपालुना कृता । त्रिप्रश्नक्रियागौरवभियैतन्मार्गान्तरं लाघवादुक्तमिति दिक्~॥~६~॥\\
\noindent अथ लाघवाद्दृक्क्षेपदृग्गती गणितसूखार्थं श्लोकार्धेनाह\textendash
\end{sloppypar}
%\vspace{2mm}


\begin{quote}
 {\ssi नतांशबाहकोटिज्ये स्फुटे दृक्क्षेपदृग्गती~।}
 \end{quote}
%\vspace{2mm}


\begin{sloppypar}
दशमभावनतांशानां भुजकोट्योर्नतांशतदूननवतिरूपयोरनयोर्ज्ये क्रमेण दृक्क्षेपदृग्गती अस्फुटे स्थूले। यद्वा स्फुटे प्रागुक्ते दृक्क्षेपदृग्गती विहाय गणितलाघवार्थं दशमभावनतांशभुजकोट्योर्ज्ये तत्स्थानापन्ने ग्राह्ये। यत्तूदयज्याभावे नतांशवाहकोटिज्ये दृक्क्षेपदृग्गती स्फुटे इति। तन्न। उक्तप्रकारेणैतत्सिद्धेस्तत्कथनस्य व्यर्थत्वात्। अत्रोपपत्तिः। त्रिभोनलग्नस्य दशमभावासन्नत्वेन त्रिभोनलग्नं प्रकल्प्य तन्नतांशज्या मध्यज्यावार्थ दशमभावमेव त्रिभोनलग्नं प्रकल्प्य तन्नतांशज्या मध्यज्यारूपा त्रिभोनलग्नदृक्क्षेपः। उन्नतज्याशङ्कुर्दृग्गतिः । इदमतिस्थूलम् । यैस्तु भगवतोक्तं मध्यलग्नं दशमभावपरतया व्याख्यातं तेषां मत एतदुक्तमिति सूक्ष्मम् । प्रयाससाधितदृक्क्षेपदृग्गती प्रागुक्ते सूक्ष्मे अप्यतिस्थुले इति ध्येयम्। भास्कराचार्यैस्तु\textendash
\end{sloppypar}
%\vspace{2mm}


\begin{quote}
 {\qt त्रिभोनलग्नस्य दिनार्धजाते\\
 नतोन्नतज्ये यदि वा सुखार्थम्~।}
\end{quote}

\newpage


\hspace{3cm}  गूढार्थप्रकाशाकेन सहितः~। \hfill १७९
\vspace{1cm}

\begin{sloppypar}
\noindent इति चदुक्तं तदस्मात् सूक्ष्ममिति ध्येयम्~॥ अथ लम्बनोपयुक्तच्छेदकथनपूर्वकं लम्बनानयनं सार्धश्लोकेनाह\textendash 
\end{sloppypar}
%\vspace{2mm}


\begin{quote}
  {\ssi एकज्यावर्गतश्छेदो लब्धं दृग्गतिजीवया~॥~७~॥

 मध्यलग्नार्कविश्लेषज्या छेदेन विभाजिता~।\\
 रवीन्दोर्लम्बनं ज्ञेयं प्राक् पश्चाद्वटिकादिकम *~॥~८~॥}
 \end{quote}
%\vspace{2mm}

\begin{sloppypar}
 एकराशिज्याया वर्गाद्वृग्गतिज्जीवयेति प्रागुक्तदृग्गत्या । दृग्गतेस्त्रिशङ्गरूपत्वेन ज्यारूपत्वानीवयेति स्वरूपप्रतिपादनम्। भागहरणेन लभं छेदसञ्ज्ञ स्यात् । शथ मध्यलग्नम्। त्रिभोनलग्नं दर्शान्तकालिकं न तु दभमभावः। तात्कालिकः सूर्य अनयोरन्तरस्य त्रिभानधिकस्य ज्या छेदेन प्राक्साधितेन भक्ता फलं घटिकादिकं प्राक् पश्चात् त्रिभोनलग्नरूपमध्यलग्नस्थानात् पूर्वापरविभागयोः सूर्यचन्द्रयोस्तुल्यं लम्बनं ज्ञेयम्। अत्रोपपक्तिः।
\end{sloppypar}
%\vspace{2mm}


\begin{quote}
 {\qt त्रिभोनलग्नार्कविशेषशिञ्जिनी\\
 कृताहता व्यासदलेन भाजिता~।\\
 हतात् फलाद्वित्रिभलग्नशङ्कुना\\
 त्रिजीवयान्तं घटिकादि लम्बनम्~॥}
 \end{quote}
%\vspace{2mm}

\begin{sloppypar}
\noindent इति सिद्धान्तशिरोमणौ सूक्ष्मं लम्बनानयनमुक्तम्। तस्योपपत्तिखट्टीकायां सुप्रसिद्धा। मध्यलग्नस्य त्रिभानपरत्वेन व्याख्यानान्मध्यलग्नार्कविश्लेषज्या त्रिभोनलग्नार्कविश्लेषशित्र्जिनीरूपा जाता । इयं चतुर्गुणा त्रिभोनलग्नशङ्कुरूपदृग्गत्या च गुण्या
\end{sloppypar}

\noindent \rule{\linewidth}{.5pt}

 \begin{center}
 * प्राक्पश्चाद्घटिकादि तत् इति वा पाठः ।
\end{center}
 {\tiny{Z2}}
 

\newpage





 \noindent १८० \hspace{4cm} सूर्यसिद्धान्तः 
\vspace{1cm}

\begin{sloppypar}
\noindent त्रिज्यावर्गेण भाज्येति लम्बनानयनप्रकारेण सिद्धम्। तत्र चतुस्त्रिज्यावर्गयोर्गुणहरयोर्गुणापवर्तनेन हरस्थान एकराशिज्यावर्ग: सिद्धः । अत्रापि दूग्गत्येकराशिष्यावर्गौ गुणहरौ गुणेनापवर्त्य हरस्थान एकज्यावर्ग इत्यादिना छेद उपपन्नः। हरस्य छेदाभिधानात् । अतो मध्यलग्नार्केत्याद्युक्तमुपपन्नम् । लम्बनघटीभिरुभयोश्चालनं वक्ष्यमाणगणित आवश्यकमिति सूचनार्थं रवीन्द्वोर्लम्बनमित्युक्तम् । अन्यथा दर्शान्तकाले सूर्यगतभूपृष्ठसूत्राच्चन्द्रकक्षायां चन्द्रचिह्नस्य तद्वटीभिर्लंम्बितत्वाहूयोरुक्त्यनुपपत्तिः । त्रिभानलग्नसमेऽर्के लम्बनाभावात् पूर्वापरविभागे सूर्ये सति लम्बनं भवतीति प्राक् पश्चादित्युक्तम्। अत्रेदमवधेयम्। लम्बनानयने मध्यलग्नस्य त्रिभोनलनेत्यर्थे छेदः पूर्वसाधितसूक्ष्ममदुग्गत्या सूक्ष्मो नतांशेत्यादिगृहीतस्थूलदृग्गत्या स्थूल इति। एवं मध्यलमेत्यस्य दशमभावार्थे तु विपरीतमिति। एतेन मध्यलग्नेत्यस्य दशमभावार्थः। तत्र प्रयाससाधितसूक्ष्मदृग्गत्या सूक्ष्मं लम्बनम्। नतांशेत्थाद्युक्तस्थूलदृग्गत्या स्थूललम्बनमिति साम्प्रदायिकोक्तं निरस्तम् । युक्त्यभावात् । न चात्र मध्यलग्ननरूपदशमभावग्रहेऽपि गोलयुक्त्या प्रतिपादनस्य सत्त्वात् कथमादित्योक्तं मध्यलग्नमिति पदं सार्वजनीनदशमभावप्रत्यायकं त्रिभोनलग्नपरतया हठाङ्व्याख्यातुं युक्तम्।
\end{sloppypar}
%\vspace{2mm}


\begin{quote}
{\qt नतांशबाहुकोटिज्ये स्फुटे दृक्क्षेपदृग्गत~।}
\end{quote}
%\vspace{2mm}

\begin{sloppypar}
\noindent इत्यत्र स्फुटे इत्यनेन भगवतस्तदाशयस्य व्यक्तीकृतवादिति वाच्यम्। तथापि गौरवसाधितदृकक्षेपोनिर्भगवदाशयस्थितत्रि\textendash
\end{sloppypar}

\newpage


\hspace{3cm} गूढार्थप्रकाशकेन सहितः~। \hfill १८१
\vspace{1cm}

\begin{sloppypar}
\noindent भोनलग्नग्रहणं व्यनक्ति । अन्यथा प्रयाससाधितदृकक्षेपस्य वैयर्थ्यापत्तरिति सूधियावलोक्यमित्यलं विस्तरेण~॥~८~॥\\
\noindent अथ मध्यग्रहणकास्तज्ञानार्थं तिथौ लम्बनसंस्कारं तदमकृत् साध्यमिति चाह\textendash
\end{sloppypar}
%\vspace{2mm}


\begin{quote}
{\ssi मध्यलग्नाधिके भानौ तिथ्यन्तात् प्रविशोधयेत्~।\\
धनमूनेऽसकृत् कर्म यावत् सर्वं स्थिरीभवेत्~॥~९~॥}
\end{quote}

\begin{sloppypar}

सूर्ये मध्यलग्नं त्रिभोनलग्नं तस्मादधिके सति तिथ्यान्ताद्दर्शतिथ्यन्तकालादागतं 
लम्बनं शोधयेत्। सूर्ये त्रिभोनलग्नान्न्यूने पति तिथ्यन्तकाले लम्बनं धनं युतं 
कार्यम्। एवं कर्म  गणितमसकृन्मुहुः कार्यम्। अयमर्थः । तिथ्यन्तकालिकः सूर्यो 
लम्बनघटीभिः क्रमेण पूर्वाग्रिमकाले चाल्पो लम्बनसंस्कृततिथ्यन्तेऽर्को भवति । 
तस्माल्लम्बनसंस्कृततिथ्यन्तकाले लग्नदशमभावौ प्रसाध्य पूर्वोक्तरीत्या लम्बनं साध्यम्। 
इदमपि केवलं तिथ्यन्ते संस्कार्योक्तरीत्या लम्बनं केवलं तिथ्यन्ते संस्कार्यम् । 
अस्मादपि लम्बनं तिथ्यन्ते संस्कार्यमित्यसकृदिति। गणितावधिमाह \textendash यावदिति~। 
सर्वं गणितं लम्बनादि यावद्यत्परिवर्तावधि  स्थिरीभवेत्। अविलक्षणं यावदविशेष इत्यर्थः। 
अत्रोपपत्तिः। दर्शान्तकाले रविगतभूपृष्ठसूत्राचन्द्रस्याधौ लम्बितत्वेन 
त्रिभोनलग्नादूने रवौ क्रान्तिवृत्ते 
पूर्वापरान्तराभावेनैकसूत्रस्थितत्वरूपयुतिर्दन्तिकालाल्लम्बनकालेनाग्रे भवति। 
शीघ्रगचन्द्रस्य मन्दगरवितः पृष्ठे स्थितत्वात्। अधिके रवौ चन्द्रस्य पुरः 
स्थि\textendash
\end{sloppypar}

\newpage

 \noindent १७२ \hspace{4cm} सूर्यसिद्धान्तः
\vspace{1cm}

\begin{sloppypar}
\noindent तत्वेन दर्शान्तकालाल्लम्बनकालेन पूर्वं युतिर्भवति । अतो दर्शान्तकालो लम्बनसंस्कृतो मध्यग्रहणकालः स्यात्। युतिकालस्य मध्यग्रहणकालत्वात्। परन्तु तावता लम्बनकालेन सूर्यस्यापि क्रान्तिवृत्ते चलनाल्लम्बनसंस्कृतदर्शान्तकाले रविगतभूपृष्ठसूत्राचन्द्रस्य लम्बितत्वं स्यादेवेति मध्यग्रहणकालस्त्वसिद्धः । न हि सूर्यो धनलम्बन ऋणलम्बने चन्द्रश्च लम्बनकाले स्थिरो येन तयोर्युतिः सङ्गता स्यात्। अतस्तादृशकालात् पुनस्तात्कालिकं लम्बनं प्रसाध्य दर्शान्ते पुनः संस्कार्यम्। मध्यकालः स्यात्। एवं तादृशलम्बनसंस्कृतदर्शान्तेऽपि तयोर्भुपृष्ठसूत्रस्थत्वाभावात् पुनर्लम्बनं माध्यम्। तत्संस्कृतो दर्शान्तो मध्यग्रह इत्यसकृद्विधिना यदा लम्बनं पूर्वलम्बनतुल्यं सिध्यति तदावश्यं तादृशलम्बनसंस्कृतदर्शान्तरूपमध्यग्रहणकाले भूपृष्ठसूत्रे तयोः सन्निवेशः । यतस्तदा सूर्यगतभूपृष्ठसूत्रचन्द्रयोरन्तराभावेन पूर्वागतलम्बनतुल्यलम्बनस्य पुनः सिद्धेः। अन्यथा तुल्यलम्बनानुपपत्तेः । तस्मान्मध्यकालोऽसकृद्यावदविशेषः साध्य इत्युपपन्नं मध्यलग्नेत्यादि~॥~९~॥\\
\noindent अथ नतिसाधनमाह\textendash
\end{sloppypar}
%\vspace{2mm}


\begin{quote}
 {\ssi दृक्क्षेपः शतितिग्मांशोर्मध्यभुक्त्यन्तराहतः~।\\
 तिथिघ्नत्रिज्यया भक्तो लब्धं सावनतिर्भवेत्~॥~१०~॥}
 \end{quote}
%\vspace{2mm}

\begin{sloppypar}
 दृक्क्षेपः प्रागानीतः शोततिग्मांशश्चन्द्रार्कयोर्मध्यगतो कलात्मके तथोरन्तरेण गुणितया त्रिज्यया भक्तः फलं सा देशकालविशेषाभ्यां या गोले सिद्धा भवति सैवात्र गणिते नति\textendash
\end{sloppypar}

\newpage




\hspace{3cm} गूढार्थप्रकाशकेन सहितः~। \hfill १८३
\vspace{1cm}

\begin{sloppypar}
\noindent र्भवेत्। अत्रोपपत्तिः यदा क्रान्तिवृत्तं दृग्वृत्ताकारं तदा नत्यभाव इति प्रागुक्तम् । तत्र त्रोभोनलग्नस्य स्वमध्यस्थत्वेन दृक्क्षेपाभावः । यत्र च षष्ट्यक्षांशास्तत्र देशे त्रिभोनलग्नस्य क्षितिजस्थत्वेन परमा नतिः । परमास्तु नतिकला भूगर्भक्षितिजाद्भूपृष्ठक्षितिजस्य भूव्यावार्धान्तरेणोच्छ्रितत्वाद्गतियोजनैर्गत्यन्तरकला लभ्यन्ते तदा भूव्यासार्धयोजनैः का इत्यनुपातेन तत्र मध्यगतियोजनानां भूव्यासार्धस्य च नियतत्वाङ्भूव्यासार्धेनापवर्तः कृतः । तेन मध्यगत्यन्तरकलानां स्वल्पान्तरेण पञ्चदशांशः परमा नतिकलाः । अत एव षष्टिघटिकानां पञ्चदशांशो घटिकाचतुष्टयं परमं लम्बनं सिद्धम्। आभिस्त्रिज्यातुल्यदृकक्षेपे सूर्यगतभूपृष्ठसूत्राच्चन्द्रस्य दक्षिणेत्तरेणावलम्बनं भवति । अतस्त्रिज्यातुल्यदृक्क्षेपेण मध्यगत्यन्तरपत्र्चदशांशो नतिस्तदेष्टदृक्क्षेपेण केत्यनुपातेन गत्यन्तरगुणो दृक्क्षेपो हरघातेन पञ्चदशगुणितत्रिज्यात्मकेन भक्तो नतिकला इत्युपपन्नम्~॥~१०~॥\\
\noindent अथ प्रकारान्तराभ्यां नतिसाधनं लाघवादाह\textendash
\end{sloppypar}
%\vspace{2mm}

% {\setlength{\parindent}{5em}
\begin{quote}
  {\ssi दृक्क्षेपात् सप्ततिहृताद्भवेद्वावनतिः फलम्~।\\
 अथवा त्रिज्यया भक्तात् सप्तसप्तकसङ्गुणात्~॥~११~॥}
 \end{quote}
%\vspace{2mm}

\begin{sloppypar}
 सप्तत्या भक्तादृक्षेपात् फलं कलादिका नतिः प्रकारान्तरेण भवेत् । अथवा प्रकारान्तरेण सप्तसप्तकसङ्गुणात् सप्तानां सप्तकं सतवार मात्तिर्वर्ग एकोनपञ्चाशदित्यर्थः । तेन गुणितादृक्क्षेपात् त्रिज्यथा भक्ता. फलं कलादिका नतिः। अत्रो\textendash
\end{sloppypar}

\newpage



\noindent १८४ \hspace{4cm} सूर्यसिद्धान्तः 
\vspace{1cm}

\begin{sloppypar}
\noindent पपत्तिः । दृक्क्षेपस्य गत्यन्तरकलामित ७३~।~२७ गुणकपञ्चदशगुणितत्रिज्यामितहरौ ५१५७० प्रथमप्रकारे गत्यन्तरापवर्त्तितो हरस्थाने सप्ततिः। द्वितीयप्रकारे पञ्चदशभिरपवर्त्य गुणस्थाने खल्पान्तरादेकोनपञ्चाशद्धरस्थाने त्रिज्येत्युपपन्नम्~॥~११~॥\\
अथ नतेर्दिग्ज्ञानं स्पष्टविक्षेपं चाह\textendash
\end{sloppypar}
%\vspace{2mm}


\begin{quote}
  {\ssi मध्यज्या दिग्वशात् सा च विज्ञेया दक्षिणोत्तरा~।\\
 सेन्दुविक्षेपदिकसाम्ये युक्ता विश्लेषितान्यथा~॥~१२~॥}
 \end{quote}
%\vspace{2mm}

\begin{sloppypar}
 सावनतिर्मध्यज्याया दिगनुरोधाद्द्क्षिणोत्तरा मध्यज्या चेद्दक्षिणा तदा नतिरपि दक्षिणा चेदुत्तरा तदोत्तरा ज्ञेया । चः समुच्चये। तेन मध्यज्या नतांशदिक्केति । सा दक्षिणोत्तरा नतिश्चन्द्रविक्षेपदिक्समत्वे । तयोरेकदिक्त्व इत्यर्थः । युक्ता विक्षेपेण युतेत्यर्थः। अन्यथा तयोर्भिन्नदिक्त्वे विक्षेपेणान्तरिता शेषदिक्का विक्षेपसंस्कृता नतिः स्पष्टशररूपा स्यात्। अत्र चन्द्रविक्षेपो मध्यग्रहणकालिक इति ध्येयम् । अत्रोपपत्तिः । नतांशदिक्कमध्यज्यावशाद्वृक्षेपस्योत्पन्नत्वात् तदुत्पन्ननतेस्तद्दिक्त्व युक्तमेव। अथ रविगतभूपृष्ठसूत्राच्चन्द्राकाशगोले क्रान्तिवृत्तावधि याम्योत्तरान्तरस्य नतित्वात् क्रान्तिमण्डलाच्चन्द्रबिम्बावधि विक्षेपत्वाद्रविगतभूपृष्ठसूत्राच्चन्द्रबिम्बावधि याम्योत्तरान्तरस्य सूर्यग्रहणोपयुक्तनतिसंस्कृतविक्षेपरूपस्पष्टविक्षेपत्वाद्द्वयोरेकदिशि योगो भिन्नादिश्यन्तरमित्युपपन्नम्~॥~१२~॥\\
\noindent अथ चन्द्रग्रहणाधिकारोक्तमत्रातिदिशति\textendash
\end{sloppypar}

\newpage


\hspace{3cm} गूढार्थप्रकाशकेन सहितः~। \hfill १८५
\vspace{1cm}
 

\begin{quote}
  {\ssi तया स्थितिविमर्दार्धग्रासाद्यं तु यथोदितम्~।\\
प्रमाणं वलनाभीष्टग्रासादि हिमरश्मिवत्~॥~१३~॥}
\end{quote}
%\vspace{2mm}

\begin{sloppypar}
 तथा विक्षेपसंस्कृतया नत्या स्पष्टविक्षेपरूपयेत्यर्थः । स्थित्यर्धंविमर्दार्धग्रासाः । आद्यशब्दात् स्पर्शमोक्षसम्मीलनोन्मीलनं यथोदितं चन्द्रग्रहणे यथोक्तं तथा । तुकारस्तदतिरिक्तरीतिव्यवच्छेदार्थकैवकारपरः । प्रमाणं मतमित्यर्थः । अवशिष्टमप्याह\textendash वलनेत्यादि~। वलनानीष्टग्रासः । आदिशब्दादिष्टग्रासादिष्टकालानयनम् । हिमरश्मिवत् । चन्द्रग्रहणक्तरीत्याकार्यमित्यर्थः । अत्रोपपत्तिरविशेष एव~॥~१३~॥\\
 \noindent अथ स्थित्यर्धविमर्दार्धे च विशेषं श्लोकचतुष्टयेनाह\textendash
\end{sloppypar}
%\vspace{2mm}


\begin{quote}
  {\ssi स्थित्यर्धोनाधिकात् प्राग्वततिथ्यन्ताल्लम्बनं पुनः~।\\
ग्रासमोक्षोद्भवं साध्यं तन्मध्यहरिजान्तरम्~॥~१४~॥

प्राक्कपालेऽधिकं मध्याद्भवेत् प्राग्ग्रहणं यदि~।\\
मौक्षिकं लम्बनं हीनं पश्चार्धे तु विपर्ययः~॥~१५~॥

तदा भोक्षस्थितिदले देयं प्रग्रहणे तथा~।\\
हरिजान्तरकं शोध्यं यत्रैतत् स्थाद्विपर्ययः~॥~१६~॥

एतदुक्तं कपालैक्ये तद्भेदे लम्बनैकता~।\\
स्वे स्वे स्थितिदले योज्या विमार्दार्धेऽपि चोक्तवत्~॥~१७~॥}
\end{quote}
%\vspace{2mm}

\begin{sloppypar}
 चन्द्रग्रहणाधिकारोक्तप्रकारेणासकृत् साधितं स्पर्शस्थित्यर्धं मोक्षस्थित्यर्धं च । तद्यथा । मध्ययहणकालिकस्पष्टशरादुक्तरीत्या स्थित्यर्धघटिकास्ताभिस्तिथ्यन्तकालिकग्रहाः । स्पर्शस्थित्यर्धनि\textendash
\end{sloppypar}

{\tiny{2A}}

\newpage



\noindent १८६ \hspace{4cm} सूर्यसिद्धान्तः 
\vspace{1cm}

\begin{sloppypar}
\noindent मित्तं पूर्वं चाल्याः । माक्षस्थित्यर्धनिमित्तमग्रे चाल्या: । तत्कालयोः प्रत्येकं नतिशरौ प्रसाध्य स्पष्टशरः साध्यः । ततः प्रथमकालिकस्पष्टशरात् स्थित्यर्धममेन पूर्वं तिथ्यन्तकालिकग्रहान् प्रचाल्योक्तरीत्या स्पष्टशरं प्रसाध्य स्थित्यर्धं साध्यम् । एवमसकृत् स्पर्शस्थित्यर्धम् । एवमेव द्वितीयकालिकस्पष्टशरात् स्थित्यर्धमनेनाग्रे तिथ्यन्तकालिकग्रहान् प्रचाल्योक्तरीत्या स्पष्टशरं प्रसाध्य स्थित्यर्धं साध्यम् । एवमसकृन्मोक्षस्थित्यर्धमिति । अथाभ्यां स्पर्शमोक्षस्थित्यर्धाभ्यां क्रमेण हीनयुताद्दर्शान्तकालात् प्राग्वदुक्तरीत्या लम्बनं पुनरसकृद्ग्रासमोक्षोद्भवं स्पर्शमोक्षकालिकं कार्यम्। तथाहि । स्पर्शस्थित्यर्धहीनात् तिथ्यन्तात् तात्कालिकसूर्याल्लग्नदशमभावौ प्रसाध्योक्तरीत्या लम्बनं साध्यम् । तेन स्पर्शस्थित्यर्धोनतिथ्यन्तं संस्कृत्यास्मालम्बनमनेनापि स्पर्शस्थित्यर्धोनतिथ्यन्तं संस्कृत्यास्मालम्बनमेवमसकृत् स्पर्शकालिकं लम्बनम् । एवमेव मोक्षस्थित्यर्धयुतात् तात्कालिकसूर्याल्लग्नदशमभावौ प्रसाध्योक्तरीत्या लम्बनं साध्यम् । तेन मोक्षस्थित्यर्धयुततिथ्यन्तं संस्कृत्यास्माल्लम्बनमनेनापि मोक्षस्थित्यर्धयुततिथ्यन्तं संस्कृत्यास्माल्लम्बनमेवमसकृन्मोक्षकालिकं लम्बनमिति। प्राक्कपाले त्रिभोनलग्नात् पूर्वभागे त्रिभोनलग्नाधिके रवौ मध्यान्मध्यकालिकात् । अग्रोक्तलम्बनस्य विभक्तिविपरिणामादन्वयेन लम्बनात् प्राग्ग्रहणं प्रग्रहणं स्पर्शः स्पर्शकालिकम् । अत्रापि लम्बनमित्यस्यान्वयः । लम्बनं चेदधिकं स्यात् । मौक्षिकं मोक्षकालसम्बन्धि लम्बनं न्यूनं स्यात्। पश्चार्धे त्रिभोनलग्नात् पश्चिमभागे त्रिभोनलग्राद्धीने
\end{sloppypar}


\newpage


 \hspace{3cm} गूढार्प्रकाशकेन सहितः~। \hfill १८७
\vspace{1cm}

\begin{sloppypar}
\noindent रवौ तुकारः समुच्चयार्थकचकारपरः । विपर्यय उक्तवैपरीत्यम्। मध्यकालिकलम्बनात् स्पर्धकालिकं लम्बनं न्यूनं मोक्षकालिकं लम्बनमधिकमित्यर्थः । तदा तर्हि तन्मध्यहरिजान्तरम् । तयोः स्पर्शमोक्षकालिकलम्बनेन प्रत्येकमन्तरं मोक्षस्थित्यर्धे योज्यम् । प्राग्ग्रहणे स्पर्शस्थित्यर्धे तथा देयम्। मोक्षमध्यकालिकलम्बनयोरन्तरं मोक्षस्थित्यर्धे योज्यम् । स्पर्शमध्यकालिकलम्बनयोरन्तरं स्पर्शस्थित्यर्धे योज्यमित्यर्थः । यत्र यस्मिन् काले विपर्यय उक्तवैपरीत्यं प्राक्कपाले मध्यकालिकलम्बनात् स्पर्शकालिकलम्बनं न्यूनं मोक्षकालिकलम्बनमधिकं पश्चिमकपाले तु मध्यकालिकलम्बनात् स्पर्धकालिकलम्बनमधिकं मोक्षकालिकलम्बंन न्यूनं भवतीत्यर्थः । तत्रैतन्मोक्षस्पर्शमध्यकालिकं हरिजान्तरकं लम्बनान्तरं मोक्षस्थित्यर्द्धे मध्यमोक्षकालिकलम्बनयोरन्तरं स्पर्शस्थित्यर्द्धे मध्यस्पर्शकालिकलम्बनयोरन्तरमित्यर्थः । शोध्यं होनं कुर्यात् । एतल्लम्बनान्तरं योज्यं शोध्यं वा कपालैक्ये द्वयोः स्पर्शमध्ययोर्मध्यमौक्षयोर्वैककपाले स्वस्वकालिकत्रिभोनलग्नात् स्वस्वकालिकसूर्य उभयत्राधिके न्यूने वेत्यर्थः । उक्तं कथितम् । तद्भेदे तयोः । स्पर्शमध्ययोर्मध्यमोक्षयोSच भेदे कपालभेदे स्पर्शकालिकत्रिभोनलग्नात् तात्कालिकसूर्यस्याधिक्ये मध्यकालिकत्रिभोनलग्नात् तात्कालिकार्कस्य न्यूनत्वे मध्यकालिकत्रिभोनलग्नात् तात्कालिकार्कस्याधिकत्वे मोक्षकालिकत्रिभोनलग्नात् तात्कालिकार्कस्य न्यूनत्व इत्यर्थः। लम्बनैकता लम्बनैक्यम् । स्पर्शमध्ययोर्भेदे तात्कालिकलम्बनयो\textendash
\end{sloppypar}

{\tiny{2A2}}

\newpage


\noindent १८८ \hspace{4cm} सूर्यसिद्धान्तः
\vspace{1cm}

\begin{sloppypar}
\noindent र्यागः । मध्यमोक्षयोर्भेदात् तात्कालिकलम्बनयोर्योग इत्यर्थः । स्वकीये स्वकीये स्थित्यर्द्धे संयुक्ता कार्या । स्पर्शस्थित्यर्द्धे स्पर्शमध्यकालिकलम्बनयोर्योगो योज्यः । मोक्षस्थित्यर्द्धे मोक्षमध्यकालिकणम्बनयोर्योगो योज्य इत्यर्थः । स्पर्शस्थित्यर्द्धं मोक्षस्थित्यर्धं च स्फुटं भवति । आभ्यां चन्द्रग्रहणोक्तदिशा मध्यग्रहणकालात् पूर्वमपरत्र क्रमेण स्पर्शमोक्षकालौ स्त इत्यर्थसिद्धम्। अथोक्तरीत्या विमर्दार्धेऽपि स्पष्टत्वमतिदिशति\textendash विमार्दार्ध इति~। स्पर्शमर्दार्धमोक्षमर्दार्ध चन्द्रग्रहणाधिकारोक्तरीत्सा स्पष्टशरेण सकृत् साधिते उक्तवत्\textendash
\end{sloppypar}
%\vspace{2mm}


\begin{quote}
 {\qt स्थित्यर्धोनाधिकात् प्राग्वत् तिथ्यन्ताल्लम्बनं पुनः~।}
 \end{quote}
%\vspace{2mm}

\begin{sloppypar}
\noindent इत्याद्युक्तरीत्या स्थित्यर्धस्थाने मर्दार्धग्रहणेन ग्रासमोक्षोद्भवमित्यत्र सम्बीलनोन्मीलनाद्भवमिति ग्रहणेन प्राग्ग्रहणमित्यत्र सम्मीलनग्रहणेन मौक्षिकमित्यत्रोन्मोलनग्रहणेन स्फुटे साध्ये । अपिः समुच्चये । चकारात् ताभ्यां सम्मीलनोलनकालौ मध्यग्रहणकालात् पूर्ववत् साध्यावित्यर्थः । अत्रोपपत्तिः । स्थित्यर्धोनियुतो मध्यग्रहणकालः स्पर्शमोक्षकालः । मध्यकालिकलम्बनसंस्कारात् । सार्शमोक्षकालिकलम्बनसंस्कारस्यापेक्षितत्वाच्च । न हि यः कालो लम्बनसंस्कृतः स्फुटः स स्वभिन्नकालिकलम्बनसंस्कृतः स्फुटः स्यात् सम्बन्धाभावात् । पूर्वं स्पर्शमोक्षकालयोरज्ञानात् । तात्कालिकलम्बनज्ञानाभावाच्च । अतो मध्यकालज्ञानार्थं यथा तिथ्यन्तादसकृल्लम्बनं प्रसाध्य तिथ्यन्ते संस्कृत्य मध्यकालस्तथा स्पर्शमोक्षस्थित्यर्धहीनयुक्ततिथ्यन्तकालाभ्यां स्पर्शमोक्षतिथ्यतरू\textendash
\end{sloppypar}

\newpage




 \hspace{3cm} गूढार्थप्रकाशकेन सहितः~। \hfill १८९
\vspace{1cm}

\begin{sloppypar}
\noindent पाभ्यां प्रत्येकं लम्बनमसकृत् प्रसाध्य खखतिथ्यन्ते संस्कृत्य स्पर्शमोक्षकालौ स्फुटौ तन्मध्यकालयोरन्तरं स्फुटं स्थित्यर्धम् । तत्रर्णलम्बनेन स्पर्धमध्यमोक्षोत्पत्तौ यदा मध्यलम्बनादधिकं स्पर्शलम्बनं मोक्षलम्बनं च न्यूनं तदा स्पर्शस्थित्यर्धोनतिथ्यन्तस्याधिकलम्बनोनितस्य स्पर्शकालत्वान्न्यूनलम्बनोनितस्य तिथ्यन्तस्य मध्यकालत्वात् तयोरन्तरे तिथेः समत्वेन नाशात् स्पर्शस्थित्यर्धं स्पर्शकालिकलम्बनेन युतं मध्यकालिकलम्बनेन हीनमिति लम्बनयोरन्तरं तत्र धनं योज्यम् । एवं मोक्षस्थित्यर्धयुततिथ्यन्तस्य न्यूनलम्बनोनितस्य मोक्षकालत्वान्मध्यमोक्षकालयोरन्तरे पूर्वरीत्या मध्यमोक्षकालिकयोर्लम्बनयोरन्तरं धनं मोक्षस्थित्यर्धे योज्यम् । यदा तु मध्यलम्बनाद्धीनं स्पर्शलम्बनं मोक्षलम्बनं चाधिकं तदा न्यूनलम्बनहीनस्य स्पर्शकालत्वादधिकं लम्बनम् । हीनस्य मध्यकालत्वादुक्तरीत्या तदन्तरे स्पर्शस्थित्यर्धे लम्बनान्तरं हीनम् । एवमधिकलम्बनहीनस्य मोक्षकालत्वान्मध्यमोक्षयोरन्तरे मोक्षस्थित्यर्धे लम्बनान्तरं हीनम् । धनलम्बनेन स्पर्शमध्यमोक्षोत्पत्तौ तु यदा मध्यलम्बनान्न्यूनं स्थर्शलम्बनं मोक्षलम्बनं चाधिकं तदा स्पर्शस्थित्यर्धोनतिथ्यन्तस्य न्यूनलम्बनाधिकस्य स्पर्शकालत्वादधिकलम्बनाधिकस्य तिथ्यन्तस्य मध्यकालत्वात् तयोरन्तरे लम्बनान्तरं स्पर्शस्थित्यर्धे योज्यम् । एवं मोक्षस्थित्यर्धयुततिथ्यन्तस्याधिकलम्बनाधिकस्य मोक्षकालत्वान्मध्यमोक्षयोरन्तरे लम्बनान्तरं मोक्षस्थित्यर्धे पूर्वरीत्या योज्यम् । यदा तु मध्यलम्बनादधिकं स्पर्शलम्बनं मोक्षलम्बनं च न्यूनं तदा\textendash
\end{sloppypar}


\newpage


 \noindent १९० \hspace{4cm} सूर्यसिद्धान्तः
\vspace{1cm}

\begin{sloppypar}
\noindent धिकलम्बनाधिकस्य स्पर्शकालत्वाद्धीनलम्बनाधिकस्य मध्यकालत्वात् तयोरन्तर उक्तरीत्या स्पर्शस्थित्यर्धे लम्बनान्तरं हीनम्। एवं न्यूनलम्बनाधिकस्य मोक्षकालत्वात् तन्मध्यकालान्तरे मोक्षस्थित्यर्धे लम्बनान्तरं हीनमिति सिद्धम् । नन्वयं लम्बनान्तरहीनपक्षो न सङ्गतः । बाधात् । तथाहि ऋणलम्बनस्य क्रमेणापचयात् स्पर्शमध्यमोक्षकालानां यथोत्तरं सम्भवाच्च मध्यकालिकलम्बनात् स्पर्शमोक्षकालिकलम्बनयोः क्रमेण न्यूनाधिकत्वमसिद्धम् । एवं धनलम्बनस्य क्रमेणोपचयान्मध्यलम्बनात् स्पर्शमोक्षकालिकलम्बनयोः क्रमेणाधिकन्यूनत्वमसिद्धम् । न हि कदाचिन्मध्यकालात् स्पर्शमोक्षकालौ क्रमेणाग्रिमपूर्वकालयोः सम्भवतो येनोक्तं युक्तम् । बाधात् । तथा च लम्बनान्तरं योज्यमित्यस्यैवोपपन्नत्वे महतैतावता प्रपञ्चेन ।
\end{sloppypar}
%\vspace{2mm}


\begin{quote}
{\qt हरिजान्तरकं शाध्यं यत्रैतत् स्याद्विपर्ययः~।}
\end{quote}
%\vspace{2mm}

\begin{sloppypar}
इति सर्वज्ञभगवदुक्तं कथं निर्वहतीति चेत् । मैवम् । लम्बनसंस्कृतस्पर्शमोक्षकालयोः स्फुटयोर्वस्तुभूतयोः सर्वदा मध्यकालात् क्रमेण पूर्वोत्तरावश्यंभावित्वेऽपि लम्बनासंस्कृतयोः स्थित्यर्धोनयुततिथ्यन्तरूपस्पर्शमोक्षकालयोः पारिभाषिकत्वेनावास्तवयोः कदाचिन्मध्यकालर्णधनलम्बनाभ्यां स्पर्शस्थित्यर्धमोक्षस्थित्यर्धयोः क्रमेण न्यूनत्वे मध्यकालादग्रिमपूर्वकालयोः क्रमेण सम्भवात् स्फुटो निर्वाहः । परन्त्वृणलम्बने धनलम्बने च मध्यलम्बनात् क्रमेण मोक्षस्पर्शलम्बनयोरधिकत्वासम्भवः। भध्यकालात् पूर्वाग्रिमकालयोक्षस्पर्शयोः पारिभाषिकयोः क्रमेणा\textendash
\end{sloppypar}

\newpage

\hspace{3cm} गूढार्थप्रकाशकेन सहितः~। \hfill १९१
\vspace{1cm}

\begin{sloppypar}
\noindent सम्भवात् । अतः साक्षात् कण्ठोक्तेरभावाद्विपर्यय इत्यनेन विपर्ययविशेषस्यैव विवक्षितत्वम् । पूर्वं तु साधारण्याच्छब्दस्य साधारण्येन व्याख्यानं कृतमित्यदोषः । ननु तथाप्यसकृल्लम्बनसाधने लम्बनस्य स्पष्ट स्पर्शमोक्षकालाभ्यां सिद्धत्वेनर्णलम्बनात् स्पर्शलम्बनं न्यूनं भवत्येव । धनलम्बने मोक्षलम्बनं न्यूनं न भवत्येव। मध्यकालाद्वात्तवस्पर्शमोक्षकालयोः क्रमेणाग्रिमपकर्वकालयोरसम्भवनिर्णयात् । अन्यथा स्थिरलम्बनासम्भवात् । किञ्चासकृल्लम्बनसाधनेन यत्कालात् स्थिरलम्बनं सिद्धं तत्कालस्य स्वक्ष्मस्पर्शमोक्षकालत्वात् स्फुटस्थित्यर्द्धसाधनं व्यर्थम्। तस्य तज्ज्ञानार्थमेवावश्यकत्वात् । न च चन्द्रग्रहणरीत्या स्पर्शमोक्षकालयोर्ज्ञानार्थं स्फुटस्थित्यर्धोक्तिरिति वाच्यम् । गौरवाङ्व्यार्थत्वाद्धरिजान्तरकं शोध्यमित्यस्यानुपपत्तेश्चेति चेन्न । लम्बनयोरसकृत्माधनस्यानङ्गीकारात् । सकृत्साधितलम्बनस्य सान्तरत्वेऽपि भगवता स्वल्यान्तरेणाङ्गीकाराच्च । अत एव लम्बनं पुनरित्यत्र पुनरित्यस्य व्याख्यानमसकृदिति पूर्वमुक्तं न युक्तम् । किन्तु मध्यकालार्थं लम्बनस्य साधनात् स्पर्शमोक्षकालार्थमपि द्वितीयवारं लम्बनं साध्यमिति व्याख्यानम् । पुनरिति वाक्यालङ्करणं वा युक्ततरमिति । अथ यदा स्थूलस्पर्शकालर्णलम्बने धनलम्बने च मध्यकालस्तदा स्पर्शस्थित्यर्धोनतिथ्यन्तस्य लम्बनहीनस्य स्पर्शकालत्वालम्बनाधिकतिथेमध्यकालत्वात् तदन्तरे स्पर्शस्थित्यर्धं तात्कालिकलम्बनयोर्योगेन युक्तभित्युक्तरीत्यापपद्यते । एवं यदा मध्यकालर्णलम्बने स्थूलमोक्षकालश्च धनलम्बने तदा लम्बनही\textendash
\end{sloppypar}

\newpage

\noindent १९२ \hspace{4cm} सूर्यसिद्धान्तः
\vspace{1cm}


\begin{sloppypar}
\noindent नतिथ्यन्तस्य मध्यकालत्वान्मोक्षस्थित्यर्धयुततीथ्यन्तस्य लम्बनाधिकस्य मोक्षकालत्वात् तदन्तरे मोक्षस्थित्यर्धं लम्बनयोगयुक्तमित्युपपन्नम् । न चासकृल्लम्बनसाधनेन सूक्ष्मस्पर्शमोक्षयोः सिद्धौ सकृल्लम्बनाङ्गीकारेणोक्तरीतेः सान्तरत्वात् कथं भगवतः सर्वज्ञस्यास्यां रीत्यामभिनिवेश इति वाच्यम् । असकृल्लम्बनसाधने प्रयासाधिक्यभयाद्भगवता सर्वज्ञेन स्वल्पान्तराङ्गिकाराल्लाघवाच्च चन्द्रग्रहणोक्तरीत्यानुगभार्थं स्फुटस्थित्यर्धसाधनस्यैवोक्तेरिति दिक् । वस्तुतस्तु सूर्योदयाद्यत्र प्राक स्पर्शोऽनन्तरं मध्यकालखदा मध्यलम्बनात स्पर्शलम्बनं सचिभलग्नचतुर्थभावसाधितं कदाचिन्न्यूनं भवति । यत्र चोदयात् पूर्व मध्यः परतो मोक्षस्तत्र कदाचित् सचिभलग्नचतुर्भावानीतमध्यकाललम्बनान्मोक्षकाललम्बनमधिकं भवति । यत्र चास्तात् पूर्व स्थर्शः परतो मध्यरुदा मध्यकाललम्बनाद्राचिसम्बन्धात् स्पर्शकाललम्बनं कदाचिदधिकं भवति । यत्र चास्तात् पूर्वं मध्यकालः परतो मोक्षस्तदापि मध्यकाललम्बनान्मोक्षकाललम्बनं रात्रिसम्बद्धं न्यूनं न भवति । कदाचिदिति । ग्रास्तोदयग्रस्तास्तयोः कदाचिद्विपर्ययसम्भवाद्धरिजान्तरकं शोध्यमित्यस्य नाप्रसिद्धिः । एतेन लम्बनमनकृन्न साध्यं विपर्यय इति विपर्ययविशेष इति चोक्तं समाधानं निरन्तमिति तत्त्वम् । विमर्दार्धेऽप्युक्तरीतिसुल्येति सर्वमुपपन्नम् । भास्कराचार्यैस्तु\textendash
\end{sloppypar}
%\vspace{2mm}


\begin{quote}
 {\qt तिथ्यन्ताद्गणितागतात् स्थितिदलेनोनाधिकाल्लम्बनं\\
 तत्कालोत्थनतीषुसंस्कृतिभवस्थित्यर्धहीनाधिके~।}
 \end{quote}

\newpage




\hspace{3cm} गूढार्थप्रकाशकेन सहितः~। \hfill १९३
\vspace{1cm}



\begin{quote}
 {\qt दर्शान्ते गणितागते धनमृणं यद्वा विधायासकृत्\\
 ज्ञेयौ प्रग्रहमोक्षसञ्ज्ञसमयावेवं क्रमात् प्रस्फुटौ~।\\
 तन्मध्यकालान्तरयोः समाने त्यष्टे भवेतां स्थितिखण्डके च ~।\\
 दर्शान्ततो मर्ददलोनयुक्तात् सम्मीलनोन्मीलनकाल एवम्~॥}
 \end{quote}
 {\setlength{\parindent}{5em}
इत्यनेन भगवदुक्तादतिसूक्ष्ममुक्तमित्यलं पल्लवितेन~॥~१७~॥}

\begin{sloppypar}
अथाग्रिमग्रन्द्वस्यासङ्गतित्वनिरासार्थमधिकारसमाप्तिं फक्किकयाह\textendash
\end{sloppypar}
\vspace{2mm}

\begin{center}
  इति सूर्यग्रहणाधिकारः~॥
\end{center}

\noindent इति स्पष्टम् ।



\begin{quote}
  {\qt रङ्गनाथेन रचिते सूर्यसिद्धान्तटिप्पणे~।\\
 सूर्यग्रहाधिकारोऽयं पूर्णो गूढप्रकाशके~॥}
\end{quote}
\begin{sloppypar}
 इति श्रीसकलगणकसार्वभौमबल्लालदैवज्ञात्मजरङ्गनाथगणकविरचिते गूढार्थप्रकाशके सूर्यग्रहणाधिकारः सम्पूर्णः~॥
\end{sloppypar}
  
\begin{center}
 \rule{8em}{.5pt} 
\end{center}


 अथ परिलेखाधिकारो व्याख्यायते । तत्र तं सप्रयोजनं प्रतिजानीते\textendash



\begin{quote}
  {\ssi न छेधकमृते यस्माद्भेदा ग्रहणयोः स्फुटाः~।\\
 ज्ञायन्ते तत् प्रवक्ष्यामि छेद्यकज्ञानमुत्तमम्~॥~१~॥}
 \end{quote}
%\vspace{2mm}


 यस्मात् कारणाद्ग्रहणयोश्चन्द्रसूर्यग्रहणयोः । द्विवचनेन ग्रहणत्वेन पूर्वाधिकारयोरेकाधिकारत्वं निरस्तम् । भेदाः कस्यां

 {\tiny{2B}}

\newpage



 \noindent १९४ \hspace{4cm} सूर्यसिद्धान्तः 
\vspace{1cm}

\begin{sloppypar}
\noindent दिशि स्पर्शमोक्षौ सम्मीलनोन्मीलने ग्रस्तोंऽश: कियानित्यादिभेदाः । स्फुटा गोलस्थितिसिद्धा वास्तवाः । छेद्यकं गोलस्थितिप्रदर्शकः कल्पितः प्रकारश्छेद्यकपदवाच्यस्तम् । ऋते विना । छेद्यकव्यतिरेकेणेत्यर्थः । न ज्ञायन्ते । तत् तस्मात् कारणात् । ग्रहणभेदज्ञानार्थमित्यर्थः । उत्तमं सूक्ष्मतद्भेदज्ञानसाधकं छेद्यकज्ञानम् । ज्ञायतेऽनेनेति ज्ञानं परिलेखसाधकग्रन्थं सूर्यांशपुरुषोऽहं प्रवक्ष्यामि कथयामि~॥~१~॥\\
\noindent तन प्रथमं वलनवृत्तं लिखेदित्याह\textendash
\end{sloppypar}
%\vspace{2mm}


\begin{quote}
  {\ssi सुसाधितायामवनौ बिन्दुं कृत्वा ततो लिखेत्~।\\
सप्तवर्गाङ्गुलेनादै मण्डलं वलनाश्रितम्~॥~२~॥}
\end{quote}
%\vspace{2mm}

\begin{sloppypar}
आदौ प्रथमं सुसाधितायां जलवत् समीकृतायाभवनौ पृथिव्यामभीष्टस्थाने बिन्दुं वृत्तमध्यज्ञापकचिन्हं कृत्वा ततश्चिन्हात् सप्तवर्गाङ्गुलेनैकोनपञ्चाशदङ्गुलमितेन व्यासार्धेन मण्डलं वृत्तं वलनाश्रितं प्रागुक्तस्फुटवलनमाश्रितं यत्र बलनाश्रयीभूतं वलगदानार्थं वृत्तमित्यर्थः । लिखेत् ग्रहणभेदज्ञानेच्छुर्गणक उल्लिखेत् । अत्रोपपत्तिः प्रागुक्ता~॥~२~॥\\
\noindent अथ द्वितीयातृतीयवृत्ते आह\textendash
\end{sloppypar}
%\vspace{2mm}


\begin{quote}
{\ssi ग्राहग्राहकयोगार्धसम्मितेन द्वितीयकम्~।\\
मण्डलं तत् समासाख्यं ग्राह्यार्धेन तृतीयकम्~॥~३~॥}
\end{quote}
%\vspace{2mm}



 ग्राह्यग्राहकबिम्बमानाङ्गुलयोर्योगार्धमितेनाङ्गुलात्मकव्यासार्धेन द्वितीयमेव द्वितीयकं द्वितीयं वृत्तं लिखेत् । तद्वृतं

\newpage


 \hspace{3cm} गूढार्थप्रकाशाकेन सहितः~। \hfill १९५
\vspace{1cm}
 
\begin{sloppypar}
\noindent समाससञ्ज्ञं योनोत्पन्नत्वात् । तृतीयकं वृत्तं ग्राह्यबिम्बाङ्गुलार्धमितेन व्यासार्धेन लिखेत्। अनोपपत्तिः । ग्रहणे शरस्य मानैक्यखण्डन्यूनत्वाद्विक्षेपो मानैक्यखण्डवृत्त इति विक्षेपदानार्थं मानैक्यखण्डवृत्तलेखनम् । तत् परिधिकेन्द्रग्राहकार्धव्यासार्धवृन्तेन ग्राह्यवृत्तेऽवश्यं योगात् समासमञ्जम्। ग्राह्यवृत्तं तु ग्रहणभेदज्ञानार्थमत्युपयुक्तम्। न हि तद्वृत्तं विना तद्भेदज्ञानं सम्भवति~॥~३~॥\\
\noindent अथ तद्वृत्तेषु दिशाधनातिदेशं स्पर्शमोचवलनदानार्थ स्पर्शमोक्षदिङ्नियमं चाह\textendash
\end{sloppypar}
%\vspace{2mm}


\begin{quote}
  {\ssi याम्योत्तराप्राच्यपरासाधनं पूर्ववद्दिशाम्~।\\
 प्रागिन्दोर्ग्रहणं पश्चान्मोतोऽर्कस्य विपर्ययात्~॥~४~॥}
 \end{quote}
%\vspace{2mm}


\begin{sloppypar}
 दिशामष्टदिशां मध्ये याम्योर्त्तराप्राच्यपरासाधनं पूर्ववत् । शिलातलेऽम्बुसंशुद्ध इत्यादित्रिप्रश्नाधिकारोक्तरीत्या कार्यम्। तथाहि । द्वादशाङ्गुलशङ्कोर्मध्यकेन्द्रस्थापितस्याद्यवृत्ते पूर्वान्हे छायाप्रवेशोऽपराहे छायानिर्गमस्तचिन्हाभ्यां मस्त्यमुत्पाद्य रेखा याम्योतरा सा वृत्तबाह्येऽधिका सम्मार्जनीया । तदितरभागे वृत्तमध्ये पूरणीया वृत्ते याम्योत्तरा रेखा भवति। तदग्रमत्स्यात् पूर्वापरा रेखा सोभयता वृत्तबाह्ये सम्मार्जनीया। सा वृत्ते पूर्वापरा रेखा भवतीति । चन्द्रस्य पूर्वर्दिशि ग्रहणं ग्रहणारम्भः स्पर्श इति थावत् । पश्चिमदिशि मोक्षो ग्रहणान्तः । अर्कस्य विपर्ययात् स्पर्शमुक्ती ज्ञेये। ग्रहणादिरूपस्पर्शः पश्चिमायां ग्रहणानरूपमोक्षः प्राच्यामित्यर्थः । अत्रोपपत्तिः । वृत्ते
\end{sloppypar}

{ \tiny{2 B 2}}




\end{document}