\documentclass[]{article}
\usepackage{lmodern}
\usepackage{amssymb,amsmath}
\usepackage{ifxetex,ifluatex}
\usepackage{fixltx2e} % provides \textsubscript
\ifnum 0\ifxetex 1\fi\ifluatex 1\fi=0 % if pdftex
  \usepackage[T1]{fontenc}
  \usepackage[utf8]{inputenc}
\else % if luatex or xelatex
  \ifxetex
    \usepackage{mathspec}
  \else
    \usepackage{fontspec}
  \fi
  \defaultfontfeatures{Ligatures=TeX,Scale=MatchLowercase}
\fi
% use upquote if available, for straight quotes in verbatim environments
\IfFileExists{upquote.sty}{\usepackage{upquote}}{}
% use microtype if available
\IfFileExists{microtype.sty}{%
\usepackage[]{microtype}
\UseMicrotypeSet[protrusion]{basicmath} % disable protrusion for tt fonts
}{}
\PassOptionsToPackage{hyphens}{url} % url is loaded by hyperref
\usepackage[unicode=true]{hyperref}
\hypersetup{
            pdfborder={0 0 0},
            breaklinks=true}
\urlstyle{same}  % don't use monospace font for urls
\IfFileExists{parskip.sty}{%
\usepackage{parskip}
}{% else
\setlength{\parindent}{0pt}
\setlength{\parskip}{6pt plus 2pt minus 1pt}
}
\setlength{\emergencystretch}{3em}  % prevent overfull lines
\providecommand{\tightlist}{%
  \setlength{\itemsep}{0pt}\setlength{\parskip}{0pt}}
\setcounter{secnumdepth}{0}
% Redefines (sub)paragraphs to behave more like sections
\ifx\paragraph\undefined\else
\let\oldparagraph\paragraph
\renewcommand{\paragraph}[1]{\oldparagraph{#1}\mbox{}}
\fi
\ifx\subparagraph\undefined\else
\let\oldsubparagraph\subparagraph
\renewcommand{\subparagraph}[1]{\oldsubparagraph{#1}\mbox{}}
\fi

% set default figure placement to htbp
\makeatletter
\def\fps@figure{htbp}
\makeatother


\date{}

\begin{document}

{\ldots{}.23\ldots{}. }

{Otherwise: such examples can be solved by methods given before.}

{त्रयोदशगुणो वर्गो निरेकः कः कृतिर् भवेत् ।}

{को वाऽष्टगुणितो वर्गो निरेको मूलदो वद ।। ७९ ।। }

{Solve the equations (1) 13x}{2}{ - 1 = y}{2}{ (2) 8x}{2}{ - 1 =y}{2}

{को वर्गः षडगुणस्त्र्याढ्यो द्वादशाढ्योऽथवा कृतिः । }

{युतो वा पञ्चसप्तत्या त्रिशत्या वा कृतिर् भवेत् ।। ८० ।। }

{In 6x}{2}{ when we add 3, 12, 75 or 300 we get a square each time. Find
the values of x in each case.}

{स्वबुद्ध्यैव पदे ज्ञेये बहुक्षेपविशोधने । }

{तयोर्भावनयाऽऽनन्त्यं रूपक्षेपपदोत्थया ।। ८१ ।। }

{Whatever be the augment, first of all we should find the two roots by
our own efforts. From these we can get by भावना process any number of
solutions.}

{वर्गच्छिन्ने गुणे ह्रस्वं तत्पादेन विभाजयेत् ।। ८२ ।। }

{If we divide the multiplier a by the square of any number and then find
x and y we should divide the x obtained by the root of the square
number. }

{द्वात्रिंशद् गुणितो वर्गः कः सैको मूलदो वद ।। ८३ ।। }

{32x}{2 }{+ 1 is a perfect square, find the value of x. }

{इष्टभक्तो द्विधा क्षेप इष्टेनाढ्यो दलीकृतः । }

{गुणमूलहृतश्चाऽऽद्यो ह्रस्वज्येष्ठे क्रमात् पदे ।। ८४ ।। }

{If the coefficient is a perfect square divide the augment by any number
and write the quotient at two places. From one subtract that number and
to the other add that number, Divide the two results by double the
square root of the coefficient and we got x and y respectively.}

{का कृतिर् नवभिः क्षुण्णा द्विपञ्चाशद्युता कृतिः । }

{को वा चतुर्गुणो वर्गस् त्रयस्त्रिंशद्युता कृतिः ।। ८५ ।। }

{\ldots{}.3\ldots{}.}

\end{document}
