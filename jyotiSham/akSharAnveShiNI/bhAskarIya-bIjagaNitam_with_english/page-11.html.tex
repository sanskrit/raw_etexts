\documentclass[]{article}
\usepackage{lmodern}
\usepackage{amssymb,amsmath}
\usepackage{ifxetex,ifluatex}
\usepackage{fixltx2e} % provides \textsubscript
\ifnum 0\ifxetex 1\fi\ifluatex 1\fi=0 % if pdftex
  \usepackage[T1]{fontenc}
  \usepackage[utf8]{inputenc}
\else % if luatex or xelatex
  \ifxetex
    \usepackage{mathspec}
  \else
    \usepackage{fontspec}
  \fi
  \defaultfontfeatures{Ligatures=TeX,Scale=MatchLowercase}
\fi
% use upquote if available, for straight quotes in verbatim environments
\IfFileExists{upquote.sty}{\usepackage{upquote}}{}
% use microtype if available
\IfFileExists{microtype.sty}{%
\usepackage[]{microtype}
\UseMicrotypeSet[protrusion]{basicmath} % disable protrusion for tt fonts
}{}
\PassOptionsToPackage{hyphens}{url} % url is loaded by hyperref
\usepackage[unicode=true]{hyperref}
\hypersetup{
            pdfborder={0 0 0},
            breaklinks=true}
\urlstyle{same}  % don't use monospace font for urls
\IfFileExists{parskip.sty}{%
\usepackage{parskip}
}{% else
\setlength{\parindent}{0pt}
\setlength{\parskip}{6pt plus 2pt minus 1pt}
}
\setlength{\emergencystretch}{3em}  % prevent overfull lines
\providecommand{\tightlist}{%
  \setlength{\itemsep}{0pt}\setlength{\parskip}{0pt}}
\setcounter{secnumdepth}{0}
% Redefines (sub)paragraphs to behave more like sections
\ifx\paragraph\undefined\else
\let\oldparagraph\paragraph
\renewcommand{\paragraph}[1]{\oldparagraph{#1}\mbox{}}
\fi
\ifx\subparagraph\undefined\else
\let\oldsubparagraph\subparagraph
\renewcommand{\subparagraph}[1]{\oldsubparagraph{#1}\mbox{}}
\fi

% set default figure placement to htbp
\makeatletter
\def\fps@figure{htbp}
\makeatother


\date{}

\begin{document}

{\ldots{}.9\ldots{}.}

{Give the quotients when (1) + 8 is divided by + 4 (2) -8 is divided by
-4(3)-8 is divided by +4 and (4) +8 is divided by -4.}

{कृतिः स्वर्णयोः स्वं स्वमूले धनर्णे । }

{न मूलं क्षयस्यास्ति तस्याकृतित्वात् ।। १३ ।। }

{The square of a positive or negative number is positive. The square
root of a positive number is positive or negative. Negative number has
no square root as it cannot be a square.}

{धनस्य रूपत्रितयस्य वर्गं क्षयस्य च ब्रूहि सखे ममाऽऽशु ।।१४।।}

{Give me quickly the square of +3 and -3.}

{धनात्मकानामधनात्मकानां मूलं नवानां च पृथक् वदाऽऽशु ।।१५।। }

{Give the square root of +9 and -9 separately.}

{२ शून्यषड्विधम् ।}

{Six rules for zero.}

{खयोगे वियोगे धनर्णं तथैव च्युतं शून्यतस्तद्विपर्यासमेति ।।१६।। }

{If zero is added to or subtracted from any number, that number remains
as it is; its positivity or negativity remains the same. But if from
zero something is removed, its sign changes. }

{रूपत्रयं स्वं क्षयगं च खं च । }

{किं स्यात् खयुक्तं वद खच्युतं च ।। १७ ।। }

{If zero is added to (1) +3 (2) -3 or (3) zero, what are the respective
sums? And if from zero (1) +3 (2) -3 or (3) zero is subtracted what will
be the remainder in each case?}

{वधादौ वियत् खस्य खं खेन घाते । }

{खहारो भवेत् खेन भक्तश्च राशिः ।। १८ ।। }

{\ldots{}.2\ldots{}.\\
}

\end{document}
