\documentclass[]{article}
\usepackage{lmodern}
\usepackage{amssymb,amsmath}
\usepackage{ifxetex,ifluatex}
\usepackage{fixltx2e} % provides \textsubscript
\ifnum 0\ifxetex 1\fi\ifluatex 1\fi=0 % if pdftex
  \usepackage[T1]{fontenc}
  \usepackage[utf8]{inputenc}
\else % if luatex or xelatex
  \ifxetex
    \usepackage{mathspec}
  \else
    \usepackage{fontspec}
  \fi
  \defaultfontfeatures{Ligatures=TeX,Scale=MatchLowercase}
\fi
% use upquote if available, for straight quotes in verbatim environments
\IfFileExists{upquote.sty}{\usepackage{upquote}}{}
% use microtype if available
\IfFileExists{microtype.sty}{%
\usepackage[]{microtype}
\UseMicrotypeSet[protrusion]{basicmath} % disable protrusion for tt fonts
}{}
\PassOptionsToPackage{hyphens}{url} % url is loaded by hyperref
\usepackage[unicode=true]{hyperref}
\hypersetup{
            pdfborder={0 0 0},
            breaklinks=true}
\urlstyle{same}  % don't use monospace font for urls
\IfFileExists{parskip.sty}{%
\usepackage{parskip}
}{% else
\setlength{\parindent}{0pt}
\setlength{\parskip}{6pt plus 2pt minus 1pt}
}
\setlength{\emergencystretch}{3em}  % prevent overfull lines
\providecommand{\tightlist}{%
  \setlength{\itemsep}{0pt}\setlength{\parskip}{0pt}}
\setcounter{secnumdepth}{0}
% Redefines (sub)paragraphs to behave more like sections
\ifx\paragraph\undefined\else
\let\oldparagraph\paragraph
\renewcommand{\paragraph}[1]{\oldparagraph{#1}\mbox{}}
\fi
\ifx\subparagraph\undefined\else
\let\oldsubparagraph\subparagraph
\renewcommand{\subparagraph}[1]{\oldsubparagraph{#1}\mbox{}}
\fi

% set default figure placement to htbp
\makeatletter
\def\fps@figure{htbp}
\makeatother


\date{}

\begin{document}

{\ldots{}.16\ldots{}. }

{त्रिसप्तमित्योर्वद मे करण्योर् विश्लेषवर्गं कृतितः पदं च । }

{द्विकत्रिपञ्चप्रमिताः करण्यः स्वस्वर्णगा व्यस्तधनर्णगा वा । }

{तासां कृतिं ब्रूहि कृतेः पदं च चेत् षड्विधं वेत्सि सखे करण्याः ।। ४३ ।।
}

{Find the square of the difference between }{√}{3 and }{√}{7 and from
the square find the root. Give the squares of (1) }{√}{2 + }{√}{3 -
}{√}{5 and (2) - }{√}{2 - }{√}{3 + }{√}{5 and find the roots of the
squares.}

{एकादिसंकलितमितकरणीखण्डानि वर्गराशौ स्युः । }

{वर्गे करणीत्रितये करणीद्वितयस्य तुल्यरूपाणि । }

{करणीषट्के तिसृणां दशसु चतसृणां तिथिषु पञ्चानाम् । }

{रूपकृतेः प्रोज्झ्य पदं ग्राह्यं चेदन्यथा न सत् क्वापि । }

{उत्पत्स्यमानयैवं मूलकरण्याऽल्पया चतुर्गुणया । }

{यासामपवर्तः स्याद् रूपकृतेस्ता विशोध्याः स्युः ।। }

{अपवर्ते या लब्धा मूलकरण्यो भवन्ति ताश्चापि । }

{शेषविधिना न यदि ता भवन्ति मूलं तदा तदसत् ।। ४४ ।। }

{In the expression for a square there are one or more surds combined.
One should know that if the expression has three surds we have to
subtract from the square of the integral number a number equivalent to
two surds. If it has six surds, we have to remove from that a number
equivalent to three;if it has ten then we have to remove a number
equivalent to four and if it has fifteen then the number will be
equivalent to five. if after removal the remainder cannot be possibly a
perfect square then the example is not proper.}

{veri-fication. In the squareroot take smallest surd. Find the number
four times the equivalent of this surd. This must divide the equivalent
numbers which have been removed from the square of the integral number.
And the quotients must}

\end{document}
