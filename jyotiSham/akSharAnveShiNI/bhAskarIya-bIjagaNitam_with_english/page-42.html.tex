\documentclass[]{article}
\usepackage{lmodern}
\usepackage{amssymb,amsmath}
\usepackage{ifxetex,ifluatex}
\usepackage{fixltx2e} % provides \textsubscript
\ifnum 0\ifxetex 1\fi\ifluatex 1\fi=0 % if pdftex
  \usepackage[T1]{fontenc}
  \usepackage[utf8]{inputenc}
\else % if luatex or xelatex
  \ifxetex
    \usepackage{mathspec}
  \else
    \usepackage{fontspec}
  \fi
  \defaultfontfeatures{Ligatures=TeX,Scale=MatchLowercase}
\fi
% use upquote if available, for straight quotes in verbatim environments
\IfFileExists{upquote.sty}{\usepackage{upquote}}{}
% use microtype if available
\IfFileExists{microtype.sty}{%
\usepackage[]{microtype}
\UseMicrotypeSet[protrusion]{basicmath} % disable protrusion for tt fonts
}{}
\PassOptionsToPackage{hyphens}{url} % url is loaded by hyperref
\usepackage[unicode=true]{hyperref}
\hypersetup{
            pdfborder={0 0 0},
            breaklinks=true}
\urlstyle{same}  % don't use monospace font for urls
\IfFileExists{parskip.sty}{%
\usepackage{parskip}
}{% else
\setlength{\parindent}{0pt}
\setlength{\parskip}{6pt plus 2pt minus 1pt}
}
\setlength{\emergencystretch}{3em}  % prevent overfull lines
\providecommand{\tightlist}{%
  \setlength{\itemsep}{0pt}\setlength{\parskip}{0pt}}
\setcounter{secnumdepth}{0}
% Redefines (sub)paragraphs to behave more like sections
\ifx\paragraph\undefined\else
\let\oldparagraph\paragraph
\renewcommand{\paragraph}[1]{\oldparagraph{#1}\mbox{}}
\fi
\ifx\subparagraph\undefined\else
\let\oldsubparagraph\subparagraph
\renewcommand{\subparagraph}[1]{\oldsubparagraph{#1}\mbox{}}
\fi

% set default figure placement to htbp
\makeatletter
\def\fps@figure{htbp}
\makeatother


\date{}

\begin{document}

{\ldots{}.40\ldots{}.}

{pigeons etc for 100 drammas.}

{षड्भक्तः पञ्चाग्रः पञ्चविभक्तो भवेच्चतुष्काग्रः । }

{चतुरुद्ध्रृतस्त्रिकाग्रो द्वयग्रस्त्रिसमुद्धृतः कः स्यात् ।। १३९ ।। }

{What is that number which when divided by 6 leaves 5 as remainder when
divided by 5 leaves 4, when divided by 4 leaves 3 and when divided by 3
leaves 2 as remainder?}

{स्युः पञ्चसप्तनवभिः क्षुण्णेषु हृतेषु केषु विंशत्या । }

{रूपोत्तराणि शेषाण्यवाप्तयश्चापि शेषसमाः ।। १४०।। }

{Find three numbers such that when they are multiplied respectively by
5, 7, 9 and then divided by 20 the quotients will differ by 1 and the
remainders are equal to the quotients. }

{एकाग्रो द्विहृतः कः स्याद्द्विकाग्रस्त्रिसमुद्धृतः । }

{त्रिकाग्रः पञ्चभिर्भक्तस्तद्वदेव हि लब्धयः ।। १४१ ।। }

{A number divided by 2 leaves 1 as remainder divided by 3 leaves 2 and
divided by 5 leaves 3 as remainder. The three quotients when divided by
2, 3, 5 respectively leave remainders 1, 2, 3. Which is that number?}

{कौ राशी वद पञ्चषट्कविहृतावेकद्विकाग्रौ ययोर् }

{द्व्यग्रं त्र्युद्धृतमन्तरं नवहृता पञ्चाग्रका स्याद्युतिः । }

{घातः सप्तहृतः षडग्र इति तौ षट्काष्टकाभ्यांविना }

{विद्वन् कुट्टकवेदिकुञ्जरघटासंघट्टसिंहोऽसि चेत् ।। १४२ ।। }

{There are two numbers. When they are divided by 5 and 6 respectively
they leave 1 and 2 as remainders. By dividing the difference of the
numbers by 3 we get 2 as the remainder. When the sum of the numbers is
divided by 9 we get 5 as the remainder; when the product of the numbers
is divided by 7 we get 6 as the remainder. What are those numbers other
than 6 and 8?\\
}

\end{document}
