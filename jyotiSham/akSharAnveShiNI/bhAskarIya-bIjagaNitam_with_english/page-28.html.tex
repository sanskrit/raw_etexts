\documentclass[]{article}
\usepackage{lmodern}
\usepackage{amssymb,amsmath}
\usepackage{ifxetex,ifluatex}
\usepackage{fixltx2e} % provides \textsubscript
\ifnum 0\ifxetex 1\fi\ifluatex 1\fi=0 % if pdftex
  \usepackage[T1]{fontenc}
  \usepackage[utf8]{inputenc}
\else % if luatex or xelatex
  \ifxetex
    \usepackage{mathspec}
  \else
    \usepackage{fontspec}
  \fi
  \defaultfontfeatures{Ligatures=TeX,Scale=MatchLowercase}
\fi
% use upquote if available, for straight quotes in verbatim environments
\IfFileExists{upquote.sty}{\usepackage{upquote}}{}
% use microtype if available
\IfFileExists{microtype.sty}{%
\usepackage[]{microtype}
\UseMicrotypeSet[protrusion]{basicmath} % disable protrusion for tt fonts
}{}
\PassOptionsToPackage{hyphens}{url} % url is loaded by hyperref
\usepackage[unicode=true]{hyperref}
\hypersetup{
            pdfborder={0 0 0},
            breaklinks=true}
\urlstyle{same}  % don't use monospace font for urls
\IfFileExists{parskip.sty}{%
\usepackage{parskip}
}{% else
\setlength{\parindent}{0pt}
\setlength{\parskip}{6pt plus 2pt minus 1pt}
}
\setlength{\emergencystretch}{3em}  % prevent overfull lines
\providecommand{\tightlist}{%
  \setlength{\itemsep}{0pt}\setlength{\parskip}{0pt}}
\setcounter{secnumdepth}{0}
% Redefines (sub)paragraphs to behave more like sections
\ifx\paragraph\undefined\else
\let\oldparagraph\paragraph
\renewcommand{\paragraph}[1]{\oldparagraph{#1}\mbox{}}
\fi
\ifx\subparagraph\undefined\else
\let\oldsubparagraph\subparagraph
\renewcommand{\subparagraph}[1]{\oldsubparagraph{#1}\mbox{}}
\fi

% set default figure placement to htbp
\makeatletter
\def\fps@figure{htbp}
\makeatother


\date{}

\begin{document}

{\ldots{}.26\ldots{}. }

{Find the values of x satisfying the equations (1) 9x}{2}{ + 52 = y}{2}{
(2) 4x}{2}{ + 33 =y}{2}

{त्रयोदशगुणो वर्गः कस् त्रयोदशवर्जितः । }

{त्रयोदशयुतो वा स्याद् वर्ग एव निगद्यताम् ।। ८६ ।। }

{Find the values of x satisfying the equations (1) 13x}{2}{ -13 =y}{2}{
(2) 13x}{2}{ + 13 =y}{2}{.}

{ऋणगैः पञ्चभिः क्षुण्णः को वर्गः सैकविंशतिः । }

{वर्गः स्याद् वद् चेद् वेत्सि क्षयगप्रकृतौ विधिम् ।। ८७ ।। }

{If you can deal with equations with negative coefficient, find the
values of x from -5x}{2}{ + 21 =y}{2}

{उक्तं बीजोपयोगीदं संक्षिप्तं गणितं किल । }

{अतो बीजं प्रवक्ष्यामि गणकानन्दकारकम् ।। ८८ ।। }

{We have given in short, methods of calculation useful in algebra. Now
we shall give bijaganita that will give joy to the mathematician.}

{७ एकवर्णसमीकरणम् । }

{यावत्तावत्कल्प्यमव्यक्तराशेर्मानं तस्मिन्कुर्वतोद्दिष्टमेव । }

{तुल्यौ पक्षौ साधनीयौ प्रयत्नात्त्यक्त्या क्षिप्त्वा वाऽपि
संगुण्यभक्त्वा ।। }

{एकाव्यक्तं शोधयेदन्यपक्षाद्रुपाण्यन्यस्येतरस्माच्च पक्षात् । }

{शेषाव्यक्ते नोद्धरेद्रूपशेषं व्यक्तं मानं जायते व्यक्तराशेः ।। }

{अव्यक्तानां द्व्यादिकानामपीह यावत्तावद्द्व्यादिनिघ्नं हृतं वा । }

{युक्तोनं वा कल्पयेदात्मबुद्ध्या मानं क्वापि व्यक्तमेवं विदित्वा ।। ८९
।। }

{First of all we assume x for the value of the unknown quantity.
According to the question in order to equate two sides, we have to add
to or subtract from some quantity or to multiply or divide. When the
sides are equated we should get the unknown to one side and take the
absolute numbers to the other side. Dividing the absolute number by the
coefficient of the unknown, the\\
}

\end{document}
