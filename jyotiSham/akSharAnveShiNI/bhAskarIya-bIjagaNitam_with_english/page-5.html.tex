\documentclass[]{article}
\usepackage{lmodern}
\usepackage{amssymb,amsmath}
\usepackage{ifxetex,ifluatex}
\usepackage{fixltx2e} % provides \textsubscript
\ifnum 0\ifxetex 1\fi\ifluatex 1\fi=0 % if pdftex
  \usepackage[T1]{fontenc}
  \usepackage[utf8]{inputenc}
\else % if luatex or xelatex
  \ifxetex
    \usepackage{mathspec}
  \else
    \usepackage{fontspec}
  \fi
  \defaultfontfeatures{Ligatures=TeX,Scale=MatchLowercase}
\fi
% use upquote if available, for straight quotes in verbatim environments
\IfFileExists{upquote.sty}{\usepackage{upquote}}{}
% use microtype if available
\IfFileExists{microtype.sty}{%
\usepackage[]{microtype}
\UseMicrotypeSet[protrusion]{basicmath} % disable protrusion for tt fonts
}{}
\PassOptionsToPackage{hyphens}{url} % url is loaded by hyperref
\usepackage[unicode=true]{hyperref}
\hypersetup{
            pdfborder={0 0 0},
            breaklinks=true}
\urlstyle{same}  % don't use monospace font for urls
\IfFileExists{parskip.sty}{%
\usepackage{parskip}
}{% else
\setlength{\parindent}{0pt}
\setlength{\parskip}{6pt plus 2pt minus 1pt}
}
\setlength{\emergencystretch}{3em}  % prevent overfull lines
\providecommand{\tightlist}{%
  \setlength{\itemsep}{0pt}\setlength{\parskip}{0pt}}
\setcounter{secnumdepth}{0}
% Redefines (sub)paragraphs to behave more like sections
\ifx\paragraph\undefined\else
\let\oldparagraph\paragraph
\renewcommand{\paragraph}[1]{\oldparagraph{#1}\mbox{}}
\fi
\ifx\subparagraph\undefined\else
\let\oldsubparagraph\subparagraph
\renewcommand{\subparagraph}[1]{\oldsubparagraph{#1}\mbox{}}
\fi

% set default figure placement to htbp
\makeatletter
\def\fps@figure{htbp}
\makeatother


\date{}

\begin{document}

{Translator's Preface}

{If we look to the history of Algebra the supreme algebraist of the
twelvth century cannot fail to attract our attention. He is
Bhaskaracharya born in India in A. D. 1114. Although he is well known
for his Lilavati, the introductory chapter of his book on astronomy
written in 1150, his best work is in Bijaganita i.e. Algebra. Lilavati
and this Bijaganita were used for some seven hundred years. The special
merits of the book and the author can be known through the nine stanzas
given at the end (p. 52). Any teacher of mathematics will rejoice to see
in these unstinted love that the author had for the subject.}

{He says}

{उपदेशलवं शास्त्रं कुरुते धीमतो यतः }

{तत्तु प्राष्यैव विस्तारं स्वयमेवोपगच्छति ।। ७ । । }

{जले तैलं खले गुह्यं पात्रे दानमनागपि । }

{प्राज्ञे शास्त्रं स्वयं याति विस्तारं वस्तुशक्तितः । । ८ ।। }

{Whatever partical an intelligent man receives from his teacher that
well received knowledge spreads itself extensively. A drop of oil put in
water, a secret deposited in the ears of a villain or a gift bestowed on
a deserving person spreads. In like manner knowledge spreads in an
intelligent mind by the force of its merits.}

{Hence the author desires that this science should be given to a
deserving pupil.}

{The book is not available in original and so here is presented along
with the original stanzas in Sanskrit a lucid rendering in English. Word
by word translation of a verse is neither simple nor useful and so this
is not an attempt for translation as such but a useful rendering. To
help the reader's grasp an appendix on numerical terms and a glossary of
technical terms is added at the end.}

{S. K. Abhyankar\\
}

\end{document}
