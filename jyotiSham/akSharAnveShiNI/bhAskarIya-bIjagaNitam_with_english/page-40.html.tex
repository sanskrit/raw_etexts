\documentclass[]{article}
\usepackage{lmodern}
\usepackage{amssymb,amsmath}
\usepackage{ifxetex,ifluatex}
\usepackage{fixltx2e} % provides \textsubscript
\ifnum 0\ifxetex 1\fi\ifluatex 1\fi=0 % if pdftex
  \usepackage[T1]{fontenc}
  \usepackage[utf8]{inputenc}
\else % if luatex or xelatex
  \ifxetex
    \usepackage{mathspec}
  \else
    \usepackage{fontspec}
  \fi
  \defaultfontfeatures{Ligatures=TeX,Scale=MatchLowercase}
\fi
% use upquote if available, for straight quotes in verbatim environments
\IfFileExists{upquote.sty}{\usepackage{upquote}}{}
% use microtype if available
\IfFileExists{microtype.sty}{%
\usepackage[]{microtype}
\UseMicrotypeSet[protrusion]{basicmath} % disable protrusion for tt fonts
}{}
\PassOptionsToPackage{hyphens}{url} % url is loaded by hyperref
\usepackage[unicode=true]{hyperref}
\hypersetup{
            pdfborder={0 0 0},
            breaklinks=true}
\urlstyle{same}  % don't use monospace font for urls
\IfFileExists{parskip.sty}{%
\usepackage{parskip}
}{% else
\setlength{\parindent}{0pt}
\setlength{\parskip}{6pt plus 2pt minus 1pt}
}
\setlength{\emergencystretch}{3em}  % prevent overfull lines
\providecommand{\tightlist}{%
  \setlength{\itemsep}{0pt}\setlength{\parskip}{0pt}}
\setcounter{secnumdepth}{0}
% Redefines (sub)paragraphs to behave more like sections
\ifx\paragraph\undefined\else
\let\oldparagraph\paragraph
\renewcommand{\paragraph}[1]{\oldparagraph{#1}\mbox{}}
\fi
\ifx\subparagraph\undefined\else
\let\oldsubparagraph\subparagraph
\renewcommand{\subparagraph}[1]{\oldsubparagraph{#1}\mbox{}}
\fi

% set default figure placement to htbp
\makeatletter
\def\fps@figure{htbp}
\makeatother


\date{}

\begin{document}

{\ldots{}.38\ldots{}.}

{९ अनेकवर्णसमीकरणम् । }

{Equations involving more than one unknown.}

{आद्यं वर्णं शोधयेदन्यपक्षादन्यान्रूपाण्यन्यतश्चाद्यभक्ते । }

{पक्षेऽन्यस्मिन्नाद्यवर्णोन्मितिः स्याद्वर्णस्यैकस्योन्मितीनां बहुत्वे
।। }

{समीकृतच्छेदगमे तु ताभ्यस्तदन्यवर्णोन्मितयः प्रसाध्याः । }

{अन्त्योन्मितौ कुट्टविधेर्गुणाप्ती ते भाज्यतद्भाजकवर्णमाने ।। }

{अन्येऽपि भाज्ये यदि सन्ति वर्णास्तन्मानमिष्टं परिकल्प्य साध्ये । }

{विलोमकोत्थापनतोऽन्यवर्णमानानि भिन्नं यदि मानमेवम । }

{भूयः कार्यः कुट्टकोऽत्रान्त्यवर्णं तेनोत्थाप्योत्थापयेद्व्यस्तमाद्यात्
।। १३४ ।। }

{We ahould get all terms containing first unknown to one side and take
terms containing other unknowns and absolute number to the other side.
Dividing by the coefficient of the first unknown we get its value. This
is called `unmiti' (value). If we get more than one unmiti for one
unknown, by equating them we get the values of other unknowns. If in the
last unmiti we have an unknown, by kuttak we should get the value of
that unknown. The values of unknowns which multiply the dividend and the
divisor in a kuttak are the गुण and लब्धि obtained as the solution of
the kuttak. If in putting any desired values for those, the kuttak
should be solved. The values of the other unknowns can be got by
substitution and inverse process. But if the values are fractional
kuttak should be solved again. By substitution and backward process we
should get the values of x and other unknowns.}

{माणिक्यामलनीलमौक्तिकमितिः पञ्चाष्ट सप्त क्रमा }

{देकस्यान्यतरस्य सप्त नव षट् तद्रत्नसंख्या सखे ।}{\\
}

\end{document}
