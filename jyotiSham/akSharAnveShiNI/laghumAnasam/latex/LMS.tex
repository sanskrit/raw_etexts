\documentclass[11pt, openany]{book}
\usepackage[text={4.65in,7.45in}, centering, includefoot]{geometry}
\usepackage[document]{ragged2e}
\usepackage[table, x11names]{xcolor}
%\include{alias}

\usepackage{fontspec,realscripts}
\usepackage{polyglossia}

\setdefaultlanguage{sanskrit}
\setotherlanguage{english}
%\setmainfont[Scale=1]{Times New Roman}
%\newfontfamily\regular[Scale=1]{Times New Roman}
\defaultfontfeatures[Scale=MatchUppercase]{Ligatures=TeX}
\newfontfamily\sanskritfont[Script=Devanagari]{Shobhika}
\newfontfamily\englishfont[Language=English, Script=Latin]{Linux Libertine O}
\newfontfamily\ab[Script=Devanagari, Color=purple]{Shobhika-Bold}
\newfontfamily\qt[Script=Devanagari, Scale=1, Color=violet]{Shobhika-Regular}
%\newfontfamily\bqt[Script=Devanagari, Scale=1, Color=brown]{Shobhika-Regular}
%\newfontfamily\s[Script=Devanagari, Scale=0.9]{Shobhika-Regular}
\newfontfamily\s[Script=Devanagari, Scale=0.9]{Shobhika-Regular}
\newcommand{\devanagarinumeral}[1]{%
	\devanagaridigits{\number\csname c@#1\endcsname}}
\usepackage{fancyhdr}
\pagestyle{fancy}
\renewcommand{\headrulewidth}{0pt}
%\newfontfamily\e[Scale=0.8]{Shobhika-Regular}
\XeTeXgenerateactualtext=1
\usepackage{enumerate}
%\pagestyle{plain}
%\pagestyle{empty}
\usepackage{afterpage}
\usepackage{amsmath}
\usepackage{amssymb}
\usepackage{tikz}
\usepackage{graphicx}
\usepackage{longtable}
\usepackage{multirow}
\usepackage{footnote}
%\usepackage{dblfnote} 
\usepackage{xspace}
%\newcommand\nd{\textsuperscript{nd}\xspace}
\usepackage{array}
\usepackage{emptypage}
\usepackage{hyperref}   % Package for hyperlinks
\hypersetup{
	colorlinks,
	citecolor=black,
	filecolor=black,
	linkcolor=blue,
	urlcolor=black
}
\begin{document}
\thispagestyle{empty}
\begin{center}
\vspace{1cm}{\huge{\textbf{आनन्दाश्रमसंस्कृतग्रन्थावलिः । }}}\\
\vspace{2mm}
\textbf{ ग्रम्याङ्कः १२३}\\
\vspace{2mm}
\textbf{पारमेश्वरव्याख्यासंवलितं}\\
\vspace{3mm}
{\huge{\textbf{  लघुमानसम् ।}}}\\
\rule{0.3\linewidth}{1.0pt}\\
\vspace{2mm}
\textbf{तदेतत्}\\
\vspace{2mm}
\textbf{ श्रीयुत 'बलवंत दत्तात्रेय आपटे' }\\
\vspace{2mm}
\textbf{इत्येतैः, आनन्दाश्रमस्थपण्डितानां साहाय्येन}\\
\vspace{2mm}
\textbf{संशोधितम् ।}\\
\vspace{2mm}
\textbf{एतत्पुस्तकं}\\
\vspace{2mm}
\textbf{'रावबहाद्दूर' इत्युपपदधारिभिः}\\
\vspace{2mm}
\textbf{गंगाधर बापुराव काळे}\\
\vspace{2mm}
\textbf{जे. पी.}\\
\vspace{2mm}
\textbf{इत्येतैः}\\
\vspace{3mm}
{\huge{\textbf{श्रीमन् 'महादेव चिमणाजी आपटे'}}}\\
\vspace{2mm}
\textbf{इत्यभिधेयमहाभागप्रतिष्ठापिते}\\
\vspace{2mm}
{\huge{\textbf{आनन्दाश्रममुद्रणालये}}}\\
\vspace{2mm}
\textbf{आयसाक्षरैर्मुद्रयित्वा}\\
\vspace{2mm}
\textbf{प्रकाशितम् ।}\\
\vspace{2mm}
\textbf{शालिवाहनशकाब्दाः १८७४ ।}\\
\vspace{2mm}
\textbf{ख्रिस्ताब्दाः (दि. १६-८-१९५२)।}\\
\vspace{2mm}
\textbf{द्वितीयेयमङ्कनावृत्तिः ।}\\
 \vspace{2mm}
\textbf{( अस्य सर्वेऽधिकारा राजशासनानुसारेण स्वायत्तीकृताः   )}\\
\vspace{2mm}
\textbf{मूल्यं चतुर्दशाऽऽणकाः (१४)}
\end{center}

\newpage
\thispagestyle{empty}
\begin{center}
\large{\textbf{श्रीशंकरः शरणम् ।}}\\
    \rule{0.3\linewidth}{1.0pt}
\end{center}
 
\justifying
 मुञ्जालकृतो लघुमानसाख्यः करणग्रन्थः ८५४ तमेऽब्दे लिखितः। तस्य
काश्चित्मविकृतयः महिशूर, त्रिवेंद्रम्, कोचीन, तिरुपति, विजगापट्टन्,
वाराणसी-त्यादि नगरेषु सन्तीति ज्ञायतेऽस्माभिः। कै० पं० सुधाकरद्विवेदिमहाभागैः
स्वर-चितगणकतरङ्गिण्यां (१८१४ शाके) मुञ्जालस्य किंचिद् वृत्तमुल्लिखितम् ।
संमति तु सकलोऽपि ग्रन्थः पारमेश्वरटीकोपेतः प्रायः श ० १३५३ तमेऽब्दे रचितः
सोऽस्माभिः संपादितः मुद्रितश्च । तेन सकले ग्रन्थेऽस्मिन् यत्
प्रतिपादितं तदव-बोद्धुं सुशकम् ।  

\justifying
ग्रन्थस्यास्य विशेष:-- 

(अ) वर्तुलस्य चतुर्षु भागेषु कृतेषु ओजपदान्ते भुजकोटयोः समानता भवति 

(ब) शा० ८५४ तमेऽब्दे अयनांशाः ६ अंश ५० कला संमानिताः ।

(क) 'इन्दूच्चोनार्ककोटिघ्नाः' `फले शशाङ्कतदगत्योः' इति श्लोकद्वयस्य

\noindent
\justifying\renewcommand{\thefootnote}{*}\footnote{ विषयस्य स्पष्टीकरणार्थं श्रीपतिविरचित सिद्धान्तशेखरे
ग्रहयुद्धाध्याये
श्लोक० १-४ तथा च भास्कराचार्यकृतबीजोपनये श्लो०  ८ दृष्टव्याः ।} कल्पना, वटेश्वरसिद्धान्तान्तर्गता इति यल्लय्यार्यः लघुमानसटीकायां
कल्पवल्ल्या-
मुल्लिखति ।\\

\justifying
 कृतज्ञता--- सुविदितभाण्डारकरप्राच्यमन्दिरस्था: प्राचार्य गोडे
महाशयाः,
दक्षिण विश्वविद्यालयस्थाः प्राचार्य कत्रे महाभागाः ,
तिरुपतिस्थग्रन्थभाण्डा-
गारव्यवस्थापका: डॉ ० ए० शंकरन् तथा च त्रिवेन्द्रस्थ पण्डित
विद्याभूषण-
महोपाध्याय विरुदभाजो वेंकटशर्माणः, एभिर्महाभागैरस्मदर्थं यदात्मा
आयाति-
तस्तदर्थं तत्तेषामुपकृतिभराम्विनयावनतेन मूर्ध्नां वहामस्तराम् । इति
शम् ।\\

\vspace{3mm} 
\begin{table}[h!]
    \begin{tabular}{ccc}
     पुण्यपत्तनम्-आनन्दाश्रमः।     &\multirow{2}{*}{$\Bigg\}$}& \textbf{ आपटेकुलसंभूतो दत्तात्रेयात्मजो}, \\
      मि. ज्ये. व. ६-१८६६  इन्दुवासरे &   & \textbf{ बलवन्तशर्मा।}~।\\
    \end{tabular}
\end{table}
\centering
\rule{0.2\linewidth}{1.0pt}

\newpage
\thispagestyle{empty}
\textbf{\large{ लघुमानसम् ।}}\\
\vspace{5mm}
\begin{tabular}{ c  c  c  c  c } 
अधिकारः &  &  श्लोकसंख्या &  &  पृ ०  \\
मध्यमाधिकारः &  & १० &  &  १ \\
स्फुटगत्यधिकारः  &   & ७ &  & ८ \\
प्रकीर्णकाधिकारः &  & ४  &  & ११ \\
त्रिप्रश्नाध्यायः &   & ९  &   &  १४ \\
ग्रहणाध्यायः &   & २० &   & १८ \\
संकीर्णाधिकारः &   &  १० &  & २७ \\
\end{tabular}\\


ओएअआइउफऱऔऐआईऊभभङ
हहहjjjjjरररररर



\vspace{5mm}
\rule{0.2\linewidth}{1.0pt}

\newpage
\thispagestyle{empty}
\begin{center}
\vspace{1cm}
\textbf{ ॐ तत्सद्ब्रह्मणे नमः ।}\\
\vspace{2mm}
\textbf{ मुञ्जालाचार्यकृतं}\\
\vspace{3mm}
{\huge{\textbf{ लघुमानसम् ।}}}\\
\rule{0.3\linewidth}{1.0pt}\\
\vspace{2mm}
\textbf{ पारमेश्वरव्याख्यासंवलितम् ।}\\
\vspace{2mm}
 प्रथमोऽध्यायः ।\\
\vspace{2mm}
 मध्यमाधिकारः ।\\
\vspace{3mm}
\centering\noindent
\hspace{-0.7cm}
{\textbf{ प्रकाशादित्यवत्ख्यातो भारद्वाजो द्विजोत्तमः।\\
 लघ्वपूर्वस्फुटोपायं वक्ष्येऽन्यल्लघुमानसम् ॥ १ ॥}}\\
\vspace{2mm}
\hspace{-1cm}
 येन व्याप्तमिदं विश्वमरुणं तं किरीटिनम्।

 नत्वा मया मानसस्य व्याख्यानं क्रियते लघोः ॥\\

\vspace{2mm}
\justifying
 प्रकाशाख्ये पत्तन आदित्यवत्ख्यातो भारद्वाजगोत्रजो द्विजोत्तमोऽहं
बृहन्मा-
नसादन्यल्लघुमानसं वक्ष्य इति संबन्धः। पूर्वं स्फुटो न विद्यते यत्र
सोऽपूर्वस्फुटः।
लघुश्च सोऽपूर्वस्फुटश्च लध्वपूर्वस्फुटः । लध्वपूर्वस्फुट उपायो
यस्मिन्शास्त्रे तल्ल-
ध्वपूर्वस्फुटोपायम् ॥ १ ॥

\vspace{2mm}
\centering
\hspace{-1.5cm}
{\textbf{ चैत्रादौ वारसंक्रान्तितिथ्यकेंन्दूच्चसध्रुवान्।\\
ज्ञात्वाऽन्यांश्चार्कवर्षादावाजन्म गणयेत्ततः ॥ २ ॥}}\\

\vspace{2mm}
\justifying
 इष्टवर्षे ध्रुवनिबन्धनं
 कृत्वाऽर्कादीनानयेदित्याह-चैत्रादाविति ।  
इष्टवर्षे चैत्रशु-
क्लप्रतिपदि मध्याह्ने वारं संक्रान्तितिथिमर्कमध्यममिन्दुमध्यममिन्दूच्चं
च सध्रुवान्
शकाब्दांश्च ज्ञात्वाऽन्यान्कुजगुरुमन्दमध्यमान्बुधसितयोः शीघोच्चे राहुं
चार्कवर्षा-
दावर्कस्य मध्यभगणान्ते ज्ञात्वा यावज्जीवं गणयेत् ।  सध्रुवान्सतो
ध्रुवानिति
योज्यम्  ।  ध्रुवाः शकाब्दाः । कथमेषामवगमनमिति चेदुच्यते--
आचार्येण निब-
ध्देभ्यो ध्रुवकेभ्यो वक्ष्यमाणविधिनैषामवगमनमिति तानि ध्रुवकमानान्यत्र
दर्श-
यिष्यामः--
\end{center}

\footnotetext{१ क. °भ्धं कृ° । २ क. °च्चे च रा° ।  ३ क. °म इति। ४ क, °बकान्य° ।}

%\thispagestyle{empty}
\afterpage{\fancyhead[CE] {\chead{पारमेश्वरव्याख्यासंवलितं -\rhead{मध्यमाधिकारः}
\afterpage{\fancyhead[CO]{\chead{लघुमानसम् । \lhead{मध्यमाधिकारः }}}
\afterpage{\fancyhead[LE,RO]{\thepage}}
\cfoot{}
\newpage
%%%%%%%%%%%%%%%%%%%%%%%%%%%%%%%%%%%%%%%%%%%%%%%%%%%%%%%%%%%%%%
\renewcommand{\thepage}{\devanagarinumeral{page}}
\setcounter{page}{2}

\centering
\vspace{2mm}
 '' कृतशरवसुमितशाके चैत्रादौ सौरिवारमध्याह्ने ।\\
 राश्यादिरजनृपार्का रविरिन्दुर्भवधृतिद्वियमाः "॥\\
 
  \vspace{2mm}
\justifying
\indent
 कृतशरवसुभिः ८५४ एभिर्मिते शकाब्दे चैत्रशुक्लप्रतिपदि मन्दवासरे
मध्याह्ने राश्यादिरविः अजनृपार्काः ११ राशयः १६ भागाः १२ कलाः। 

\justifying\noindent
इन्दुः भवधृतिद्वियमाः ११-१८-२२।

\renewcommand{\thefootnote}{*}\footnote{ पद्यमिदं क. पुस्तके नास्ति।} 
\vspace{1mm}\centering
सूर्यान्मन्दोच्चांशा वसुतुरगाः पर्वताश्च सत्र्यंशाः।\\
\hspace{1cm}
स्वररवयः खाकृतयो द्विनवभुवोऽशीतिरद्रिजिनाः ॥

\vspace{1mm}
\justifying
 रवेर्मन्दोच्चांशा वसुतुरगाः ७८ ।  इन्दूच्चभागाः सत्र्यंशाः पर्वताः
विंशतिलिप्तासहिताः सप्त भागाः ७-२०। भौमस्य स्वररवयः १२७। बुधस्य खाकृतयः
२२० । जीवस्य द्विनगभुवः १७२। शुक्रस्याशीतिः ८०। मन्दस्याद्रिजिनाः २४७।

\vspace{1mm}
\centering
 द्व्युत्कृतिखानि युगोत्कृतिकराब्धयः खाष्टनवदशत्रिसुराः।\\
 \hspace{-1.8cm}
 गोष्टाविंशतितानाः कुजादयः सूर्यभगणान्ते ।

\vspace{1mm}
\justifying
 कुजो राश्यादि द्वयुत्कृतिखानि २-२६-०। बुधशीघ्रं युगोत्कृतिकराब्धयः
४-२६-४२। गुरुमध्यं खाष्टनव ०-८-९। शुक्रस्य शीघ्रं दशत्रिसुराः
१०-३-३३। मन्दमध्यं गोष्टाविंशतितानाः ९-२८-४९। एवं सूर्यस्य मध्यभगणान्ते कुजादयः।

\vspace{1mm}
\centering
 संक्रान्तितिथिध्रुवकाः शक्रा वसुनवरसेषवो राहोः।\\
 \hspace{-0.75cm}
 कृतयमवसुरसदशका दशाहताः शेषपातांशाः॥

\vspace{1mm}
\justifying
 चैत्रशुक्लप्रतिषदर्कमध्यभगणान्तरे यावत्यस्तिथयः संभवन्ति ताः
संक्रान्ति-
तिथय इत्युच्यन्ते। अत्र संक्रान्तितिथयः शक्राः १४। राहो राश्यादयः
वसुन-
वरसेषवः ८-९-५६। अयं चक्राद् विशुद्ध एव। शेषाणां कुजादीनां पातांशा:
कृतयमेत्यादिनोच्यन्ते। कुजस्य दशगुणितानि कृतानि ४०। बुधस्य दशगुणा
प्रमाः २०। गुरोर्दशगुणा वसवः ८०। शुक्रस्य दशगुणा रसाः ६०।
मन्दस्य
दशगुणा दशकाः १००। अस्मिन् कालेऽयनचलनांशाः प्रदर्श्यन्ते-

\footnotetext{
 १ क. °दिः र°। २ क. सूर्यान्म°। ३ क. २०७। ४ क. २२। ५ क. २-४

\hspace{7cm}६      
\hspace{1cm}४-६

६ के. °ध्यमं स्वा°। ७ क. १०-३३-३। ८ क. °ध्यमं मो°। ९ क. २२-४८-९।

१० क. गणान्तयोरण्तः ।}

\newpage
\centering
 अयनचलनाः षडंशाः पञ्चाशल्लिप्तिकास्तथैकैकम् ।\\ 
 प्रत्यब्दं तत्सहितो रविरुत्तरविषुवदादिः स्यात् ॥ \\
\vspace{2mm}
\justifying
 अयनचलनांशाः ६-५० पुनः प्रत्यब्दमेकैका लिप्तास्तस्मिन् देयाः ।
एतदय-
नचलनं ग्रहस्य दक्षिणोत्तरस्फुटगत्यवगमने ग्रहस्फुटे देयम् । एतद्युंतो
रविरुत्तर-
विषुवदादिसंज्ञितो भवति । उदग्विषुवदाद्यर्क इत्यत्रायं ग्राह्य इत्यर्थः
। अस्य 
ध्रुवस्य सौरिवारे निबद्धत्वात्तदुपरितनादर्कवारात् वारसिद्धिर्भवति ।
ध्रुवाब्दादीनां 
बाहुल्ये सति द्युगण एकद्वित्र्यादिक्षेपेणेष्टवारः साध्यः ।
अथाङ्कलाघवार्थमाचा-
र्योदितैरोभिर्ध्रुवैरन्यानि ध्रुवाणि निबध्य लिख्यन्ते- 

\vspace{2mm}
\centering
 शशिसुरविधुमितशाके चैत्रादौ सूर्यवारमध्याह्ने । \\
 भानो राश्यङ्ककला रुद्रा नागेन्दवोऽब्धिवेदाश्च ॥ \\
 \hspace{0.7cm}
 एकादश रसयमला विषधरबाणाः क्रमाच्छशाङ्कस्य । \\
 रुद्राः खमाग्नेयमलाश्चन्द्रोच्चस्याथ सूर्यभणान्ते ॥ \\
 भौमस्य पङ्क्तयोऽङ्गान्यहिरामाश्चाथ चन्द्रपुत्रस्य ।\\
\hspace{0.9cm}
 रुद्राः फणियमलाः खं जीवस्य द्वौ कृताश्विनौ देवाः ॥ \\
 दस्रा नृपास्त्रिबाणा भृगुषुत्रस्याथ सूर्यतनयस्य । \\
\hspace{2cm}
 अभ्रं शराः क्र(?)बाणा राहोः खं बाणभूमयोऽङ्गशराः ॥ इति । 

\vspace{2mm}
\justifying
 अत्र संक्रान्तितिथय एकादशरविमध्येनैव तत् सिध्यति । अस्य
ध्रुवनिबन्ध-
नस्य सूर्यवारत्वात्तदुपरितनचन्द्रवाराद् वारसिद्धिः स्यात् ।
एवमिष्टवर्षे ध्रुवाणि 
निबध्य तैरर्कादीन् गणयेत् ॥ २ ॥

\vspace{2mm}
\centering
\hspace{-0.75cm}
\textbf{
 ध्रुवादब्दगणो दिग्घ्नः स्वकीयाष्टांशसंयुतः ।\\ 
 \hspace{0.8cm}
 संक्रान्तितिथियुक्तोऽघः स्वषष्ट्यंशविवर्जितः ॥ ३ ॥ \\
 \hspace{-0.75cm}
 त्रिंशच्छिन्नावशेषोनश्चैत्रादितिथिभिर्युतः ।\\
 \hspace{0.3cm}
 त्रिगुणाब्दगतर्तूनो द्युगणो ध्रुववासरात् ॥ ४ ॥ }\\

\vspace{2mm}
\justifying
 अथ द्युगणानयनमाह-- ध्रुवादिति । यस्मिन् वर्षे ध्रुवनिबन्धनं कृतं
तस्माद् 
वर्षांदतीतानब्दान् विन्यस्य तान् दशभिर्निहत्य पुनः पृथगधो
विन्यस्याधःस्थिता-नष्टभिर्विभज्य पूर्णान्येव फलानि गृहीत्वा तान्युपरिस्थितराशौ
प्रक्षिपेत् । पुनस्त-
स्मिन् राशौ ध्रुवपठिताः संक्रान्तितिथीः प्रक्षिप्य पुनः पृथगधो विन्यस्य

\footnotetext{

 १ क. °प्ता अस्मि° । २ क. °रग° । ३ क. °श्यंशक° । ४ क. °स्रो नृ° । ५ क.\\
°मध्यमेनैव हि त । ६ क. °रगतत्दा° । ७ क. °रादिवा° ।
}

\newpage

\justifying
पुनरपि तस्याधो विन्यस्य तेषां त्रयाणां राशीनामधेःस्थितं षष्ट्या विभज्य
पूर्णान्येव फलानि गृहीत्वा तानि मध्यस्थराशेः संशोध्य तं मध्यस्थराशिं त्रिंशता
विभज्य लब्धानि पूर्णान्येव फलानि गृहीत्वा कुत्रचिद् विन्यस्य
हृतशेषमुपरिस्थ-
राशेः संशोध्य तस्मिन् राशौ वर्तमानाब्दे चैत्रप्रतिपदाद्याः सकलास्तिथीः
प्रक्षिप्य 
पुनस्तस्माद्राशेस्रिगुणान् ध्रुवाब्दान् वर्तमानाब्दे गतानृतूंश्च
संशोधयेत् । स रा-
शिर्द्युगणो नाम भवति । तस्मिन् सप्तभिर्हृते शिष्टाद् ध्रुववासरादिवारो
भवति । 
द्युगण एतत्प्रक्षिप्य वा संशोध्य वेष्टवारः साध्यः । यदा पुनरब्दादौ
त्रिगुणाब्दा 
न शोध्यास्तदा त्रिगुणाब्देभ्यस्रिंशच्छिन्नावशेषोनं चैत्रादितिथियुतराशिं
विशोध-
येत् । तत्र शिष्टमृणद्युगणसंज्ञं भवति । तस्मात्सप्तभ्यो विशोध्य
शिष्टादपि वारः 
साध्यः । एकद्विसंख्ये ध्रुवाब्दे चैवमेव संभवति ॥ ३ ॥ ४ ॥ 

\vspace{2mm}
\centering
\hspace{-1.75cm}
\textbf{
\hspace{-0.5cm}
 द्युगणोऽधो दशघ्नोऽब्दयुतः खागाप्तवर्जितः । \\
 अष्टघ्नाब्दोनितोऽर्कांर्शाः प्रक्षेप्योऽब्दाष्टमः कलाः ॥ ५ ॥ }\\
 
\vspace{2mm}
\justifying
 इदानीमादित्यस्य मध्यमानयनमाह-- द्युगण इति । इष्टदिने द्युगणं
विन्यस्य 
पुनः पृथगधो विन्यस्याधःस्थिते दशघ्नान् ध्रुवाब्दान् प्रक्षिप्य पुनस्तं
खागैः 
सप्तत्या विभज्य लब्धानंशानुपरिस्थितराशेर्विशोध्य हृतशेषं षष्ट्या निहत्य
खागैरेव विभज्य लब्धा लिप्ताश्चोपरिस्थभागेष्वेकं लिप्तीकृत्य ततो विशोधयेत्
। पुनरष्टघ्नान् ध्रुवाब्दानुपरिस्थराशेर्विशोध्य ध्रुवाब्दानामष्टमांशेन
तुलिताः कला 
लिप्तासु प्रक्षिपेत् । तत्र दृष्टा अंशा भवन्ति । रवेरुपरिस्थिता
अंशा अधःस्थिता 
लिप्ता इत्यर्थः । पुनः पुनरुपरिस्थितानंशान् त्रिंशता समारोप्य ध्रुवं
प्रक्षिपेत् । 
सोऽर्कस्य मध्यमो भवति । अष्टघ्नाब्दानामशोध्यत्वे
त्वष्टघ्नाब्देभ्योऽर्कांशान् स-
लिप्तान् विशोध्य शिष्टं ध्रुवाच्छोधयेत् । तत्र शिष्टेऽब्दाष्टमः  कलाः
प्रक्षिपेत् 
स रविर्मध्यमो भवति । ऋणद्युगणे तु दशघ्नाब्दाद् ऋणद्युगणं विशोध्य
शिष्टात् 
खागाप्तऋणद्युगणे प्रक्षिप्याष्टघ्नाब्दानपि तस्मिन्नेव प्रक्षिप्य
पुनस्तद्ध्रुवान् संशो-
ध्य शिष्टेऽब्दाष्टमकलाः प्रक्षिपेत् । स रविर्भवति ॥ ५ ॥

\footnotetext{
 १ क. °दायतीतास्ति° । २ क. °ण एकं प्र° । ३ क. °ति । ऋणद्युगणं स° । ४
क \\
शिष्टान् पूर्वघद वारश्च भवति । एकद्वित्रिसं° । ५ क. °शघ्नाब्द° । ६
क. °कौशोः । ७ \\
। क. °ला । ८ क. °स्थभागेभ्यो वि° । ९ क. °न्ति । उप° । १० क षुनरुप°
। ११ क. \\
°मः क° । १२ क. °वान् सं° ।}


\newpage

\centering
\textbf{
 विश्वघ्नो द्युगणो द्विष्ठस्त्रिघ्नाब्धद्युगणोनितः । \\
 \hspace{0.5cm}
 अष्टाङ्गाप्तजिनघ्नाप्तयुतो भागादितः शशी ॥ ६ ॥ }

\vspace{2mm}
\justifying
 चन्द्रमध्यमानयनमाह-- विश्वघ्न इति । द्युगणं विन्यस्य
विश्वैस्त्रयोदशभिर्निहत्य 
तं पुनः पृथगधो विन्यस्याः स्थिते त्रिघ्नान् ध्रुवाब्दान् केवलं द्युगणं च
विशोध्य 
शिष्टमष्टाङ्गैरष्टषष्ट्या विभज्य लब्धान् भागान् गृहीत्वा शिष्टात्
षष्टिघ्नाप्तैरेव 
विभज्य लब्धा लिप्ताश्च गृहीत्वा तान् भागान् सलिप्तानुपरिस्थिते
विश्वघ्नद्युगणे 
प्रक्षिप्य पुनर्जिनैश्चतुर्विंशत्या निहत्य ध्रुवाब्दान् तस्मिन्नेव
प्रक्षिपेत् । स भागा-
त्मकः शशी भवति । पुनः पूर्ववद् भागास्रिंशता समारोप्य ध्रुवं प्रक्षिपेत्
ऋण-
द्युगणे तु विश्वघ्नाद् द्युगणाद् द्युगणं विशोध्य पुनस्तस्मिन्
त्रिघ्नाब्दान् प्रक्षिप्या-
ष्टाङ्गैर्विभज्य लब्धं सावयवं विश्वघ्ने द्युगणे प्रक्षिप्य पुनस्तान्
मार्गात्मिकान् 
ध्रुवाद् विशोध्य शिष्टे जिनघ्नाब्दान् प्रक्षिपेत् । स चन्द्रो भवति
॥ ६ ॥ 

\vspace{2mm}
\centering
\textbf{
 द्युगणाद्द्विगुणाब्दोनाच्चन्द्रोच्चांशो नवोद्धृतः । \\
 \hspace{1cm}
  खवेदध्नाब्दसंयुक्ताः साष्टांशाब्दकलोनिताः ॥ ७ ॥}\\
  
  \vspace{2mm}
  \justifying
 द्युगणाद्द्विगुणिवोन्ध्रुवाब्दान् विशोध्य शिष्टं नवभिर्विभज्य लब्धा
अंशा 
भवन्ति । शिष्टात्षष्टिघ्नान्नवभिर्लब्धाः कला भवन्ति । तेष्वंशेषु
स्ववेदैश्चत्वारिं-
शता गुणितान् ध्रुवाब्दान् प्रक्षिप्य तस्मात्पुनरब्दाष्टमांशकला
अब्दतमानकलाश्च 
विशोधयेत् । तत्र जाताश्चन्द्रोच्चभागा भवन्ति । पुनस्तान्
भागास्त्रिंशता समा-
रोप्य ध्रुवं प्रक्षिपेत् । स चन्द्रतुङ्गो भवति ।
द्युगणान्द्विगुणिताब्दस्याधिकत्वे तु 
द्विगुणाब्दादूद्युगणं विशोध्य
शिष्टान्नवभिराप्तानंशान्सलिप्तान्ध्रुवाद्विशोध्य स्ववेद-
घ्नाब्दान्प्रक्षिपेत । ततः साष्टांशाब्दकला विशोधयेत् । स तुङ्गः ।
ऋणद्युगणे 
तु द्विगणाब्दयुताद्द्युगणान्नबोद्धृतानंशान्सलिप्ताद्ध्रुवाद्विशोध्य
तस्मिन् स्ववेदघ्ना-
ब्दान् प्रक्षिप्य साष्टांशाब्दकला विशोधयेद् । स तुङ्गः । यदा
पुनरब्दाभावस्त-
दाऽब्दोक्तकार्यं विना रविशशितुङ्गानानयेत् ॥ ७ ॥
\footnotetext{
 १ क. °द्विग° । २ क. °घ्नाब्दयु° । ३ क. °दिकः शं° । ४ क. °स्थिता त्रिं
।\\
५ क. °ष्टान् षष्टिध्नादष्टाङ्गैराप्ता लि° । ६ क.°गात्मकाद् ध्रु° । ७
क. °नश्चन्द्रो° । ८ \\
क. °च्चांशान° । ९ क. °ताः । १० क. °ताद् ध्रु° । ११ क. °प्ताद् ध्रु°।
१२ क. \\
°स चन्द्रोच्चः स्यात् । ऋ° । १३ क. °खवेदाब्धौ प्र° । १४ क. °स
चन्द्रोच्चो भवति । \\
य° । १५ क. °ङ्गानुक्तन्यायेनाऽऽ° ।}


\newpage


\centering
\textbf{
 ध्रुवाद्यर्कात्कुजो द्वाभ्यां नृपघ्नाच्चेषुखेषुभिः ।\\
 सप्तघ्नादृतुवेदैर्ज्ञश्चतुर्घ्नरविणा युतः ॥ ८ ॥}
 
\vspace{2mm}
\justifying
 अथार्कमध्यमेन कुजादीनां मध्यमानयनमाह-- ध्रुवाद्यर्कादिति ।
इष्टदिनमध्या-
ह्ने लिप्ताभागराशिसहितमर्कं विन्यस्य राशिस्थानादुपरि ध्रुवाब्दांश्च
विन्यसेत् ।
स ध्रुवाद्यर्क इत्युच्यते । किंंतु
चैत्रशुक्लप्रतिपदर्कवर्षान्तरयोरन्तराले ध्रुवाब्दा
एकहीना भवन्ति । अर्कवर्षादौ कुजादीनां ध्रुवनिबन्धनकरणात् । एवंविधं
ध्रुवा-
अर्कं विन्यस्याब्दान्द्वाभ्यां विभज्य लब्धानब्दानेकत्र विन्यसेत् ।
पुनर्हृतशिष्टमब्दं
द्वादशभिर्निहत्याधःस्थितेषु राशिषु प्रक्षिप्य तान्राशीन्द्वाभ्यां
विभज्य लब्धान्राशी-
न्पूर्वं विन्यस्तस्याब्दस्याधो विन्यसेत् । पुनस्तद्राशिशेषं त्रिंशता
निहत्य भागेषु
प्रक्षिप्य तान्भागान्द्वाभ्यां विभज्य लब्धानंशान्पूर्वानीतराशेरधो
विन्यस्य पुनस्त-
द्भागशेषं षष्ट्या निहत्य लिप्तासु प्रक्षिप्य पुनस्ता लिप्ता द्वाभ्यां
विभज्य
लब्धा लिप्ताः पूर्वानीतभागस्याधो विन्यसेत् । पुनरपि ध्रुवाद्यर्कं
विन्यस्य लिप्ताभागराश्यब्दान् पृथक्पृथङ् नृपैः षोडशभिर्निहत्य लिप्ताः षष्ट्या
समारोप्यांशेषु प्रक्षिप्यांशास्त्रिंशता समारोप्य राशिषु प्रक्षिप्य राशीन्
द्वादशभिः समारोप्याब्देषु प्रक्षिपेत् । पुनस्तानब्दानिषुखेषुभिः
पञ्चोत्तरपञ्चशतैर्विभज्य लब्धानब्दान्पूर्वानीतेष्वब्देषु प्रक्षिपेत् । पुनस्तच्छेषं
द्वादशभिर्निहत्याधःस्थितराशिषु प्रक्षिप्य तान्राशीनिषुखेषुभिर्विभज्य लब्धान्राशीन्पूर्वानीतराशिषु
प्रक्षिप्य पुनस्तच्छेषं त्रिंशता निहत्य भागेषु प्रक्षिप्य तान्भागानिषुखेषुभिर्विभज्य
लब्धान्भागान्पूर्वानीतभागेषु प्रक्षिप्य पुनस्तच्छेषं षष्ट्या निहत्य लिप्तासु
प्रक्षिप्य ता लिप्ता
इषुखेषुभिर्विभज्य लब्धा लिप्ताः पूर्वानीतलिप्तासु प्रक्षिपेत् । एवं
बुधादीनामपि
गुणहरणसमारोपणादयो द्रष्टव्याः । पुनस्तान्लिप्तादीन्षट्यादिना
पूर्ववत्समारोप्य 
तस्मिन् ध्रुवं प्रक्षिपेत् । स इष्टदिनमध्याह्ने कुजमध्यमो भवति ।\\
\indent
 सप्तघ्नादिति । ध्रुवाद्यर्कं विन्यस्य तस्य लिप्तांशान्राश्यब्दान्
पृथक्पृथक्
सप्तभिर्निहत्य पूर्ववत्समारोप्य पुनस्तानब्दादीनृतुवेदैः षट्चत्वारिंशता
पूर्ववद्विभज्य
लब्धानब्दराशिभागलिप्तिकाख्यान्क्रमेणैकत्र विन्यस्य तस्मिंश्चतुर्गुणितं
ध्रुवाद्यर्कं
प्रक्षिप्य ध्रुवं च प्रक्षिपेत् । स ज्ञो बुधः । बुधस्य
शीघ्रोच्चमित्यर्थः ॥ ८ ॥
\footnotetext{
 १ क. °र्कमध्यमं वि° ।}
 

\newpage
\centering
\textbf{
 रूपघ्नाद्भास्करैर्जीवो भूघ्नाच्च रदखेन्दुभिः ।\\
 \hspace{2cm}
 दिग्घ्नात् षड्भिः सितो दिग्घ्नात् त्रिजिनांशेन वर्जितः ॥ ९ ॥}
 
\vspace{2mm}
\justifying
 रूषघ्नादिति । ध्रुवाद्यर्कं रूपेणैकेन निहत्य
भास्करैर्द्वादशभिर्विभज्य लब्धा-
नब्दादीनेकत्र विन्यस्य पुनरपि ध्रुवाद्यर्कं भुवा-- एकेन निहत्य
रदखेन्दुभिर्द्वात्रिंश-
दुत्तरसहस्त्रेण (१०३२) विभज्य लब्धानब्दादीन् पूर्वानीतेष्वब्दादिषु
प्रक्षिप्य
ध्रुवं प्रक्षिपेत् । स गुरुमध्यमो भवति । अत्र   रूपगुणनं    भूगुणनं   च
पादपूरणार्थमेव । भेदाभावात् ।\\
\indent
 दिग्घ्नादिति । ध्रुवाद्यर्कं विन्यस्य तं दशभिर्निहत्य षड्भिर्विभज्य
लब्धानब्दादीनेकत्र विन्यस्य पुनरपि ध्रुवाद्यर्कं दशभिर्निहत्य
त्रिजिनैस्त्रिचत्वारिंशदुत्तरशतद्वयेन (२४३) विभज्य लब्धानब्दादीन् पूर्वानीतेभ्योऽब्दादिभ्यो
विशोध्य
ध्रुवं प्रक्षिपेत् । स सितः शुक्रस्य शीघ्रोच्चमित्यर्थः ॥ ९ ॥

\vspace{2mm}
\centering
\textbf{षड्गुणादयुतेनाऽऽर्किश्चन्द्रघ्नाच्च खवह्निभिः ।\\
 \hspace{0.75cm}
 नखैः पञ्चाङ्गनेत्रैश्च चन्द्रपातो विलोमगः ॥ १० ॥}\\
 
\vspace{2mm}
\justifying
 षड्गुणादिति। ध्रुवाद्यर्कं षड्भिर्निहत्यायुतेन (१००००) विभज्य
लब्धानब्दादीनेकत्र विन्यस्य पुनरपि ध्रुवाद्यर्कं चन्द्रेणैकेन विभज्य
खवह्निभिस्त्रिंशता
विभज्य लब्धानब्दादीन् पूर्वानीताब्देषु प्रक्षिप्य ध्रुवं प्रक्षिपेत्
। स आर्किर्मन्दमध्यमो भवति ।\\
\indent
 नखैरिति । ध्रुवाद्यर्कं नखैर्विंशत्या विभज्य लब्धानब्दादीनेकत्र
विन्यस्य
पुनरपि ध्रुवाद्यर्कं पञ्चाङ्गनेत्रैः पञ्चषष्ट्युत्तरशतद्वयेन (२६५)
विभज्य लब्धानब्दादीन् पूर्वानीताब्दादिषु प्रक्षिप्य पुनस्तान् राश्यादीन्
ध्रुवाच्छोधयेत् । स
चन्द्रपातो राहुरित्यर्थः । विलोमगया ध्रुवाच्छोधनं क्रियते ।
सर्वेषामप्यब्दाः
प्रयोजनाभावान्न संरक्ष्यन्ते । राश्यादय एव संरक्षणीयाः ॥

\vspace{2mm}
\centering
 इति पारमेश्वरे मानसगणितव्याख्याने\\
\vspace{2mm}
 मध्यमविधिः ॥\\
\rule{0.2\linewidth}{1.0pt}

\footnotetext{
 १ क. निहत्य । २ क. °ताब्दादिषु° । ३ क. °गत्वाद् ध्रु ।}

\newpage
\thispagestyle{fancy}
\fancyhf{}
\chead{पारमेश्वरव्याख्यासंवलितं-}
\rhead{[स्फुटगत्यधिकारः]}
\lhead{८}

\centering
 अथ स्फुटगत्यधिकारः ।\\
 
\vspace{3mm}
\textbf{
 ग्रहः स्वोच्चोनितः केन्द्रं तदूर्ध्वाधोऽर्धजो भुजः ।\\
 धनर्णं पदशः कोटी धनर्णर्णधनात्मिका ॥ १ ॥}\\

\vspace{2mm}
\justifying
 अथ स्फुटकरणमाह-- ग्रह इति । ग्रहमध्यमात्स्वमन्दोच्चं विशोध्य
शिष्टं के
न्द्रसंज्ञं भवति । तथा मन्दभुजाफलसंस्कृताद् ग्रहमध्यमात्स्वशीघ्रोच्चं
विशोध्य
शिष्टमपि केन्द्रसंज्ञं भवति । तदूर्ध्वाधोऽर्धजो भुजो धनर्णम् ।
तस्मिन्केन्द्र ऊर्ध्वार्धजो भुगो धनात्मको भवति  ।  अधोऽर्धजो भुज ऋणात्मको भवति  ।  केन्द्रे
तुलादिषड्राशिगते धनात्मको भुजः  ।  मेषादिषड्राशिगत ऋणात्मक  इत्यर्थः ।
पदशः कोटी धनर्णर्णधनात्मिका कोटी पदक्रमाद् धनर्णर्णधनात्मिका भवति
केन्द्रे प्रथमपदगता कोटी धनात्मिका द्वितीयपदगता ऋणात्मिका, तृतीयपदगता च ऋणात्मिका
चतुर्थपदगता धनात्मिका इत्यर्थः । एवं सर्वत्र
भुजाकोट्योर्धनर्ण-
त्वकल्पना वेदितव्या ॥ १ ॥

\vspace{2mm}
\centering
\textbf{
 ओजे पदे गतैष्याभ्यां बाहुकोटी सभेऽन्यथा ।\\
\hspace{1cm}
 चतुस्त्र्येकघ्नराश्यैक्यं बाहुकोट्योः कलांशकाः ॥ २ ॥}\\

\vspace{2mm}
\justifying
 अथ तयोर्भुजाकोट्योर्विभागं जीवाकल्पनां चाऽऽह-- ओज इति । ओजे पदे
गते राश्यादिकं बाहुर्भवति । तस्मिन्नेव पद एष्यं राश्यादिकं कोटिर्भवति
। समेऽ-
न्यथा समपदे गतं कोटिर्भवति । एष्यं भुजा भवतीति। एवं भुजाकोटिविभागः
सर्वत्र द्रष्टव्यः ।\\
\indent
 चतुरिति । भुजाकोट्योरुक्तविधिना राश्यादिकमानीय पृथग्विन्यसेत् ।
पुन-
र्भुजायाः प्रथमराशिं चतुर्भिर्निहत्य द्वितीयराशिं त्रिभिर्निहत्य
तृतीयराशिमेकेन
निहत्य तेषामैक्यं कुर्यात् । यदा पुनर्भुजाया द्वौ राशी कतिचिद् भागा
लिप्ताश्च
संभवन्ति तदा प्रथमराशिं चतुर्भिर्निहत्य द्वितीयं त्रिभिश्च निहत्य भागा
लिप्ताश्च
द्वाभ्यां निहत्य लिप्ता षष्ट्या समारोप्य भागस्थाने प्रक्षिप्य भागस्थानपि
षष्ट्यैव
समारोप्य राशिस्थाने प्रक्षिपेत् । यदा पुनरेको राशिः कतिचिद्भागा
लिप्ताश्च संभवन्ति
तदा प्रथमराशिं चतुर्भिर्निहत्य भागा लिप्ताश्च षड्भिर्निहत्य
लिप्ताः षष्ट्या
समारोप्य भागस्थाने प्रक्षिप्य भागस्थानपि षष्ट्यैव समारोप्य राशिस्थाने
\footnotetext{
१ क. °ब्रामयनं चा ऽऽ° । २ क. तथा । ३ क. °गाल्लि° । ४ क. °माल्लि° ।}

\newpage
\thispagestyle{fancy}
\fancyhf{}
\chead{लघुमानसम् ।}
\rhead{९}
\lhead{[स्फुटगत्यधिकारः]}
\justifying
\noindent
प्रक्षिपेत् । यदा पुनरेकोऽपि राशिर्नास्ति कतिचिद्भागा लिप्ताश्च
संभवन्ति तदा
भागान्सलिप्तानष्टभिर्निहत्य लिप्ताः षष्ट्या समारोप्य भागस्थाने
प्रक्षिप्य भागा-
नपि षष्ठ्यैव समारोप्य राशौ प्रक्षिपेत् । तत्र राशिस्थानगता भागा
भवन्ति ।
भागस्थानगता लिप्ता भवन्ति । लिप्तास्थानगता विलिप्ता भवन्ति । एवं
चतु-
स्येकघ्नराश्यैक्ये कृते यावन्तो भागा विद्यन्ते तावतीर्लिप्ता लिप्तासु
प्रक्षिपेत्
यावत्यः पुनर्लिप्ताः प्राक्संभवन्ति तावतीर्विलिप्ता विलिप्तासु
प्रक्षिपेत् । एवं कृते
भागाद्या भुजाया जीवा भवन्ति । कोट्या अप्येवमेव ।
चतुस्त्र्येकघ्नराश्यैक्यं
कृत्वा भागासमानलिप्ता लिप्तासमानविलिप्ताश्च प्रक्षिपेत् । ताः
कोटिज्या भवन्ति
कलांशका इत्यस्यायमर्थः-- चतुस्त्र्येकघ्ना लिप्तीकृता राशयो
यावन्तस्तावन्तो ।
भागास्तावत्यो लिप्ताश्च जीवारूपेण ग्राह्या इत्यर्थः । अवयवरूपा
भागाद्या द्वि-
घ्नेन गुणकारेण सदा गणितव्याः । राशिस्थानगतानां भागात्मकत्वात् । एवं
भु-
जाकोट्योर्जीवां पृथगानीयैकत्र संरक्षेत् ॥ २ ॥\\

\centering
\textbf{
\hspace{-2cm}
 सूर्याज्जिनाश्विनोऽगाङ्काः शरवेदाः खखेन्दवः ।\\
 द्व्यङ्काः खदन्तास्त्रिरसाश्छेदाः कोट्यर्घसंस्कृताः ॥ ३ ॥}\\

\vspace{2mm}
\justifying
 अथ मन्दफलानयनाय हारकांस्तेषां स्फुटीकरणं चाऽऽह-- सूर्यादिति ।
सूर्यस्य
जिनाश्विनः (२२४) छेदाः हारका इत्यर्थः । चन्द्रस्य अगाङ्काः (९७)।
कुजस्य शरवेदाः (४५)। बुधस्य खखेन्दवः (१००) गुरोर्द्व्यङ्काः (९२) ।
शुक्रस्य खदन्ताः (३२०) । शनेस्त्रिरसाः (६३)। एते कोट्यर्धसंस्कृता:
स्फुटा भवन्ति । एतदुक्तं भवति- अर्कादिमन्दान्तानां ग्रहाणां
मध्यमं विन्यस्य स्वमन्दोच्चं विशोध्य शिष्टाद् भुजज्यां कोटिज्यां च
पूर्वोक्तविधिनाऽऽनीयैकत्र विन्यस्य कोटिज्यां पुनः पृथग् विन्यास्यार्धीकृत्य स्वीये छेदे जिनाश्विन
इत्यादिपठिते संस्कुर्यात् । कोट्या धनत्वे धनं ऋणत्वे ऋणं कुर्यात् । स
स्फुटच्छेदो
भवति । पुनर्भागात्मिकां भुजज्यां विन्यस्य लिप्तीकृत्य स्फुटच्छेदेन
विभज्य लब्धान् भागान् गृहीत्वा शिष्टात् षष्टिघ्नात् स्फुटच्छेदेन विभज्य लब्धा
लिप्ताश्च
गृहीत्वा तान् भागान् सलिप्तान् ग्रहमध्यमे संस्कुर्यात् । भुजाया
ऋणत्वे ऋणं
धनत्वे धनं कुर्यात् । स मन्दस्फुटो ग्रहो भवति ॥ ३ ॥
\footnotetext{
 १ क. °न्यङ्काः । २ क. °ष्टान् ष° । ३ क.°घ्नान् स्फु '। ४ क. °देनैव
वि° ।\\
 २}

\newpage
\thispagestyle{fancy}
\fancyhf{}
\chead{पारमेश्वरव्याख्यासंवलितं -}
\rhead{[स्फुटगत्यधिकारः]}
\lhead{१०}
\centering
\textbf{
 भुजो लिप्तीकृतश्छेदभक्तो ग्रहफलांशकाः । \\
\hspace{1.8cm}
  कोटिर्गतिघ्ना च्छेदाप्तं व्यस्तं गतिकलाः फलम् ॥ ४ ॥ }\\
\vspace{2mm}
\justifying
 तदुक्तं--'' भुजो लिप्तीकृतश्छेदभक्तो ग्रहफलांशकाः '' इति ।
अर्केन्दू मन्द-
स्फुटेनैव स्फुटौ भवतः । कुजादीनामप्येवम् । मन्दस्फुटं कृत्वा
तन्मन्दस्फुटं स्फुटच्छेदं
चैकत्र संरक्षेत् । \\
\indent
 अथ मन्दस्फुटगत्यानयनमाह--कोटिरिति । पूर्वानीतां कोटिज्यां
स्वमध्यगत्या 
निहत्य स्फुटच्छेदेन विभज्य लब्धा लिप्ताः रवमध्यगतौ ऋणधनव्यत्ययेन तं
स्कुर्यात् । कोट्या ऋणत्वे धनं कुर्यात् । कोट्या धनत्वे ऋणं
कुर्यात् । सा मन्दस्फुटगतिर्भवति ।
अर्केन्द्वोः सैव स्फुटा गतिर्भवति । मध्यगत्यवगमनं
तु द्विनद्वये 
ग्रहस्य मध्यममानीय तयोरन्तरं कुर्यात् । तदन्तरं मध्यगतिर्भवति ॥ ४॥ \\

\centering
\textbf{
\hspace{-1cm}
 कुजजीवशनिच्छेदा युगाग्न्यगहता हृताः । \\
 तिथिशैलर्तुभिर्व्यासा मूर्छनेशा ज्ञशुक्रयोः ॥ ५ ॥}\\

\vspace{2mm}
\justifying
 अथ कुजादीनां शीघ्रफलानयनाय तद्व्यासार्धानयनमाह-- कुजेति ।
कुजजीव-
शनीनां मन्दस्फुटच्छेदान् पूर्वानीतान्युगादिभिर्निहत्य
तिथ्यादिभिर्विभज्य लब्धा 
व्यासा भवन्ति । तत्र कुजस्य मन्दस्फुटच्छेदं युगैश्चतुर्भिर्निहत्य
तिथिभिः पञ्चद-
शभिर्विभज्य लब्धान् भागान् गृहीत्वा शिष्टात्षष्टिघ्नात्
पञ्चदशभिर्विभज्य लब्धा 
लिप्ताश्च गृह्णीयात् । ते कुजस्य व्यासा भवन्ति । जीवस्य तु
स्वमन्दस्फुटच्छेदमग्निभिस्त्रिभिर्निहत्य शैलैः सप्तभिर्विभज्य लब्धं सावयवं जीवस्य
व्यासो 
भवति । मन्दस्य च स्वमन्दस्फुटच्छेदमगैः सप्तभिर्निहत्य ऋतुभिः
षड्भिर्विभज्य 
लब्धं सावयवं मन्दस्य व्यासो भवति । एते मध्यव्यासा न स्फुटाः ।
मूर्छनेशा 
ज्ञशुक्रयोः । ज्ञस्य बुधस्य मूर्छना एकविंशतिर्मध्यव्यासः । शुक्रस्य
ईशा एकादश 
मध्यव्यासः । अनयोः सदाऽप्येवमेव मध्मव्यासः । न मन्दस्फुटच्छेदाद्
भेदः 
संभवति ॥ ५ ॥ \\

\centering
\textbf{
 ते दोस्त्र्यंशयुताः शीघ्रच्छेदाः स्युः कोटिसंस्कृताः । \\
 ताराग्रहार्कयोः शीघ्रः शीघ्रोच्चमितरो ग्रहः ॥ ६ ॥ }

\vspace{2mm}
\justifying
 एषां व्यासानां स्फुटकरणं तैः शीघ्रच्छेदानयनं चाऽऽह-- ते
दोस्त्र्यंशयुता इति । 
ते व्यासाः शीघ्रभुजज्यात्र्यंशयुताः स्फुटा भवन्ति । ते स्फुटव्यासाः
शीघ्रको-
\footnotetext{
 १ क. °र्व्यासो मू° । २ क. °य व्यासान° । ३ क. °ज्य लिप्ता विलि° । ४ क.
न्दच्छे° }

\newpage
\thispagestyle{fancy}
\fancyhf{}
\chead{लघुमानसम्।}
\rhead{११}
\lhead{[प्रकीर्णकाधिकारः]}
\justifying
\noindent
टिज्यय्या सकलया संस्कृताः स्फुटाः शीघ्रच्छेदा भवन्ति । एतदुक्तं
भवति--कु-
जादीनां पञ्चानां मन्दस्फुटं पृथग्विन्यस्य तस्मात्स्वशीघ्रोच्चं विशोध्य
भुजज्यां
कोटिज्यां च पूर्ववदानीयैकत्र विन्यस्य पुनर्भुजज्यां पृथग्विन्यस्य
त्रिभिर्विभज्य
लब्धं पूर्वानीते मध्यव्यासे सदा प्रक्षिपेत्स स्फुटव्यासो भवति ।
पुनस्तं स्फुटव्यासं
पृथग्विन्यस्य तस्मिञ्शीघ्रकोटिज्यां सकलां संस्कुर्यात् । कोट्या
ऋणत्वे ऋणं
धनत्वे धनम् । स स्फुटः शीघ्रच्छेदो भवति। पुनः शीघ्रभुजज्यां
लिप्तीकृत्य
शीघ्रच्छेदेन विभज्य लब्धान्भागान्गृहीत्वा शिष्टात्षष्टिघ्नात्तेनैव
च्छेदेन विभज्य
लब्धा लिप्ताश्च गृहीत्वा छेदं संरक्ष्य ताञ्शीघ्रभागोनेकत्र विन्यस्य
पुनस्तान्पृथ-
ग्विन्यस्य मन्दस्फुटग्रहे ऋणं धनं वा कुर्यात् ।   भुजाया ऋणत्वे ऋणं
धनत्वे धनम् । 
स स्फुटग्रहो भवति । कुजबुधगुरुसितमन्दानामेवं स्फुटकरणं
कुर्यादिति ।\\
\indent
 अथ कुजादीनां शीघ्रोच्चकल्पनमाह-- तारेति । ताराग्रहाः कुजादयः
पञ्च
ग्रहाः । तेष्विष्टग्रहमध्यमां अर्कमध्यमयोर्या शीघ्रगतिः सा तस्य
शीघ्रोच्चमितरो
मध्यमः । अतः कुजगुरुमन्दानामर्कमध्यमः शीघोच्चः स्वमध्यमो मध्यमः ।
बुध-
शुक्रयोः स्वमध्यमः शीघ्रोच्चः । अर्कमध्यम स्वमध्यम इति सिद्धं भवति ।
बुधशुक्रयोरर्कमध्यमं स्वमध्यमं परिकल्प्य मन्दस्फुटं शीघ्रस्फुटं च
कुर्यात् ॥६॥

\vspace{2mm}
\centering
\textbf{
\hspace{-1cm}
 व्यासं शीघ्रफलार्कांशभागोनं ग्रहशीघ्रयोः ।\\
 गत्यन्तरघ्नं छेदाप्तं त्यक्त्वा शीघ्रगतेर्गतिः ॥ ७ ॥}\\

\vspace{2mm}
\justifying
 इदानीं कुजादीनां स्फुटगत्यानयनमाह- व्यासमिति । ग्रहस्य
शीघ्रभुजाफलं
द्वादशभिरंशैर्विभज्य लब्धं स्फुटव्यासाद्विशोध्य शिष्टं
स्वमन्दस्फुटमुक्तिशीघ्रोच्च-
भुक्त्योरन्तरेण निहत्य शीघ्रच्छेदेन विभज्य लब्धं
शीघ्रोच्चभुक्तोर्विशोधयेत् ।
शिष्टं ग्रहस्य स्फुटभुक्तिर्भवति ।  यदा यत्फलमशोध्यं स्यात्तदा
तत्फलाच्छीघ्रभुक्तिं
विशोध्य शिष्टं ग्रहस्य वक्रभुक्तिर्भवति ॥ ७ ॥

\centering
 इति पारमेश्वरे मानसगणितव्याख्याने स्फुटगत्यधिकारः ॥\\

\rule{0.2\linewidth}{1.0pt}\\

\vspace{3mm}
\centering
 अथ प्रकीर्णकाधिकारः ।\\

\vspace{2mm}
\centering
\textbf{
 इन्दूच्चोनार्ककोटिघ्ना गत्यंशा विभवा विधोः ।\\
 गुणो व्यर्केन्दुदोःकोट्योरुपपञ्चाप्तयोः क्रमात् ॥ १ ॥}

\footnotetext{
 १ क. °गान्सलिप्ताने° । २ क. °मार्क° । ३ क, °च्चं स्व° । ४ क. °च्चं ।
५ क.\\
 °दा तत् ।}

\newpage
\thispagestyle{fancy}
\fancyhf{}
\chead{पारमेश्वरव्याख्यासंवलितं -}
\rhead{[प्रकीर्णकाधिकारः]}
\lhead{१२}
\justifying
 अथ ग्रहणादीनां दृग्गणितसमीकरणार्थं चन्द्रस्य तद्भुक्तेश्च द्वितीयं
कर्माऽऽह-
इन्दूच्चेति । इन्दूच्चोनार्कस्य कोटिज्यया गुणिता विधोर्विभवा गत्यंशा
गुणकारो 
भवति । व्यर्केन्द्वोर्भुजाकोटिज्ययोर्गुणकार इत्यर्थः । एतदुक्तं
भवति-- स्फुटार्कादि-
न्दूच्चं विशोध्य धनात्मिकामृणात्मिकां वा कोटिज्यामानीय तया
भागात्मिकये-
न्दोः स्फुटगत्यंशानेकादशत्रिरंशैर्हीनात्सलिप्तान्गुणयेत् । तथा
गुणितास्त इन्दू-
च्चहीनार्ककोटिवशाद्धनात्मक ऋणात्मको वा गुणकारो भवतीति । कस्यायं गुण-
कार इत्यत्राऽऽह-- व्यर्केन्दुदोःकोट्योरुपपञ्चाप्तयोरिति ।
चन्द्रस्फुटात्स्फुटार्कं वि-
शोध्य धनात्मिकामृणात्मिकां वा भुजज्यां तथाभूतां कोटिज्यां चाऽऽनीय गुण-
कारेण पृथग्गुणित्वा भुजज्यामेकेन कोटिज्यां पञ्चभिर्विभज्य लब्धद्वयं
लिप्तादिकं 
गृहीत्वा चन्द्रे तद्भुक्तौ च संस्कुर्यात् । तत्कथमित्यत्राऽऽह-

\vspace{2mm}
\centering
\hspace{-1.7cm}
\textbf{
 फले शशाङ्कतद्गत्योर्लिप्ताद्ये स्वर्णयोर्वधे । \\
 ऋणं चन्द्रे धनं भुक्तौ स्वर्णसाम्यवधेऽन्यथा ॥ २ ॥ }\\

\vspace{2mm}
\justifying
 एतदुक्तं भवति -- व्यर्केन्दुभुजज्यागुणकारयोरेकस्य धनत्वेऽन्यस्य
ऋणत्वे सति 
व्यर्केन्दुभुजज्यात् एकेनाऽऽप्तं लिप्तादिकं फलं चन्द्रस्फुटाच्छोधयेत्
। गुणकार-
भुजज्ययोरुभयोर्युगपद् धनत्वे वा ऋणत्वे वा सति तत्फलं 
चन्द्रस्फुटे
प्रक्षिपेत् । 
स स्फुटचन्द्रो भवति । भुक्तिस्तु व्यर्केन्दुकोटिज्यागुणकारयोरेकस्य
धनत्वेऽन्यस्य 
ऋणत्वे सति व्यर्केन्दुकोटिज्यातः पञ्चभिराप्तं लिप्तादिकं फलं
चन्द्रस्फुटभुक्तौ 
प्रक्षिपेत् । कोटिज्यागुणकारयोर्युगपद्धनत्वे वा ऋणत्वे वा सति तत्फलं
चन्द्रस्फु-
टभुक्तेर्विशोधयेत् । सा स्फुटभुक्तिर्भवतीति ॥ २ ॥

\vspace{2mm}
\centering
\hspace{-1.7cm}
\textbf{
 अवन्तिसमयाम्योद्गरेखा पूर्वापराध्वना । \\
 ग्रहगत्यंशषष्ट्यंशो हतो लिप्तास्वृणं धनम् ॥ ३ ॥ }

\vspace{2mm}
\justifying
 अथ देशान्तरं कर्माऽऽह-अवन्तीति । अवन्तिरिति मध्यरेखागतं किमपि
पत्त-
नम् । तत्समदक्षिणोत्तरा लङ्कापूर्वावगाहिनी या रेखा, तद्रेखातः
पूर्वतोऽपरतो 
वा यावद् योजनान्तरे दृष्टाववतिष्टते तानि योजनानि देशान्तरयोजनानीति
प्र-
सिद्धानि । तैर्ग्रहस्य गत्यंशषष्ट्यंशं देशान्तरयोजनैर्निहत्य षष्ट्या
विभज्य लब्धं 
लिप्तादि चन्द्रस्फुटलिप्तासु ऋणं धनं वा कुर्यात् । अन्येषां तु
गत्यंशाभावा-
\footnotetext{
 १ क.°केन्दोर्भु । २ क. °नाप्त° । ३ क. °प्ताद्योः स्व° । ४ क. °ङ्का
मेर्वव° ।}

\newpage
\thispagestyle{fancy}
\fancyhf{}
\chead{लघुमानसम् ।}
\rhead{१३}
\lhead{[प्रकीर्णकाधिकारः]}
\justifying
\noindent
त्स्वस्फुटभुक्तिलिप्तिका देशान्तरयोजनैर्निहत्य षष्ट्या विभज्य लब्धा
विलिप्ता 
ग्रहस्फुटविलिप्तास्वृणं धनं वा कुर्यादिति । समरेखायाः पूर्वस्मिन्  ऋणं  कुर्यात् । अपरस्मिन् धनं कुर्यात् । वक्रिणां व्यत्ययः । अथवा
देशान्तरयोजनैर्ग्रहस्य 
मध्यभुक्तिं निहत्य षष्ट्या विभज्य लब्धं ग्रहमध्यमे संस्कृत्य स्फुटी
कुर्यात् ।
राहोर्मण्डलशोधनात्प्रागेव देशान्तरं कर्म कुर्यात् । चन्द्रतुङ्गस्य च
देशान्तरं 
कर्म कार्यम् ॥ ३ ॥

\centering
\hspace{-0.5cm}
\textbf{
 व्यर्केन्दोस्तिथितिथ्यर्धे ग्रहाद् भान्यनुपाततः । \\
 योगश्चन्द्रार्कसंयोगात्तदाद्यन्तौ स्वभुक्तितः ॥ ४ ॥ }

\vspace{2mm}
\justifying
 तिथिकरणनक्षत्रयोगपरिज्ञानमाह-- व्यर्केन्दोरिति । अर्कहीनं
चन्द्रं लिप्ती- 
कृत्य शन्याश्विपर्वतैर्विभज्य लब्धाः शुक्लप्रतिपदाद्यास्तिथयो [ गता
] भवन्ति । 
पुनस्तमेव लिप्तीकृतं व्यर्केन्दुं शून्यरसाग्निभिर्विभज्य लब्धानि
व्येकानि सिंहा-
दीनि गतानि करणानि भवन्ति । पुनर्लिप्तीकृतं चन्द्रं
व्योमशून्याष्टभिर्विभज्य-
लब्धान्यश्विन्यादीनि नक्षत्राणि भवन्ति । पुनरर्कयुतं चन्द्रं
लिप्तीकृत्य व्योम-
शून्याष्टभिर्विभज्य लब्धा विष्कम्भाद्या गता योगा भवन्ति । शिष्टानि
वर्तमानतिथिकरणर्क्षयोगानामंशा भवन्ति । तदाद्यन्तौ स्वभुक्तितः ।
तेषामाद्यन्तकाल-
परिज्ञान स्वभुक्तितस्रैराशिकेन ज्ञेयम् । वर्तमानतिथिकरणयोर्गता
लिप्ताः षष्ट्या 
निहत्यार्केन्द्वोर्गत्यन्तरेण विभज्य लब्धा वर्तमानतिथिकरणयोर्गता घटिका
भवन्ति । तथैष्यलिप्ताभ्य एष्यघटिका वेदितव्याः । वर्तमाननक्षत्रस्य
गता वा 
एष्या वा लिप्ताः षष्ट्या निहत्य चन्द्रस्फुटभुक्त्या विभज्य लब्धा गता
एष्याश्च 
घटिका भवन्ति । वर्तमानयोगस्य गता एष्याश्च लिप्ताः षष्ट्या
निहत्यार्केन्द्वोर्भुक्तियोगेन
विभज्य लब्धा वर्तमानस्य योगस्य गता एष्याश्च घटिका भवन्ति
। 
सर्वत्र लिप्तात्मिकैव भुक्तिः । एवं त्रैराशिकादाद्यन्तपरिज्ञानम् ।
अनुपाततस्त्रैराशिकेन ।
अन्येषां ग्रहाणामपि चन्द्रवन्नक्षत्रावगतिः ।
तदुक्तं-- ग्रहाद्भानीति  ॥ ४ ॥ 

\vspace{2mm}
\centering
\textbf{
 इति पारमेश्वरे मानसव्याख्याने प्रकीर्णकाधिकारः । \\
 इति पारमेश्वरे मानसव्याख्याने प्रथमोऽध्यायः ॥ १ ॥}
\footnotetext{
 १ क. °हान् भान्वनु° । २ क. तयोश्च° ।}
 
\newpage
\thispagestyle{fancy}
\fancyhf{}
\chead{पारमेश्वरव्याख्यासंवलितं -}
\rhead{[त्रिप्रश्नाधिकारः]}
\lhead{१४}
\centering
\textbf{\large{
 अथ द्वितीयोऽध्यायः । }}\\
 \vspace{2mm}
 अथ त्रिप्रश्नाध्यायः प्रारभ्यते । \\

\vspace{2mm}
\centering
\textbf{
\hspace{-0.5cm}
 नखघ्ना विषुवच्छाया स्वाक्षांशोना त्रिभाजिता । \\
 \hspace{-1cm}
 उदग्विषुवदाद्यर्कभुजाराशिगुणाश्चरे ॥ १ ॥ }\\
 
\vspace{2mm}
\justifying
 अथ त्रिप्रश्नाध्याय उच्यते-- नखघ्नेति । सायनार्को यदा मीनान्तं
गच्छति 
तस्मिन्नहनि मध्याह्ने द्वादशाङ्गुलशङ्कोश्छाया विषुवच्छायेत्युच्यते ।
स्वदेशविषुच्छायाङ्गुलानि
नखैर्विंशत्या गुणितानि प्रथमभुजाराशेर्गुणकारो भवति
। 
पुनस्तानि नखघ्नानि विषुवच्छायाङ्गुलानि स्वाक्षांशेन स्वपञ्चमभागेनोनितानि
द्वितीयभुजाराशेर्गुणकारो भवति । पुनस्तेभ्यो
नखघ्नविषुच्छायाङ्गुलेभ्यस्त्रिभि-
र्भक्तं तृतीयभुजाराशेर्गुणकारो भवति । उदग्विषुवदाद्यर्कभुजाराशिं
गुणाः साय-
नोऽर्क उदग्विषुवदादिसंज्ञितो भवति । तस्यार्कस्य भुजाराशिगुणाः ।
एतदुक्तं 
भवति-इष्टकाले सायनार्कं विन्यस्य भुजाराशीनानीय तद्भुजाराशिषु प्रथमराशिं
प्रथमगुणेन द्वितीयं द्वितीयेन तृतीयं तृतीयेन प्रथग्गुणित्वा तैषामेक्यं
कुर्यात् । 
तत्र जाताश्चरविनाड्यो भवन्ति । एवमन्येषामपि चरविनाडयो ग्राह्याः ॥
१ ॥ 

\vspace{2mm}
\centering
\textbf{
\hspace{-0.8cm}
 वसुभान्यङ्कगोदस्रास्त्रिदन्ताश्च क्रमोत्क्रमात् ।\\
 तत्तच्चरगुणार्धोना मध्यषट्केऽन्यथोदया ॥ २ ॥}

\vspace{2mm}
\justifying
 वसुभानीति । लङ्कायां मेषराशेरुदयप्रमाणविनाड्यो वसुभानि ( २७८ ) । 
ऋषभस्याङ्कगोदस्राः ( २९९ ) । मिथुनस्य त्रिदन्ताः ( ३२३ ) । क्रमोत्कमादेवं
क्रमात्पुनरुत्क्रमात्पुनः क्रमात्पुनरुत्क्रमात् । एवं
लङ्कायामुदयप्रमाणविनाड्यवगतिः ।
स्वदेशे तु तत्तच्चरगुणार्धोना मध्यषट्केऽन्यथोदया इति ।
एतदुक्तं 
भवति-- प्रथमचरगुणार्धहीनानि वसुभानि मीनमेषयोरुदयप्रमाणविनाड्यो
भवन्ति । 
द्वितीयचरगुणार्धहीना अङ्कगोदस्त्रा कुम्भवृषयोरुदयविनाड्यः ।
तृतीयचरगुणा -
र्धहीनास्त्रिदन्ता मृगमिथुनयोरुदयविनाडयः ।
पुनस्तृतीयचरगुणार्धयुतास्त्रिदन्ता-
श्चापकर्कटकयोरुदयविनाड्यः । द्वितीयचरगुणार्धयुता अङ्कगोदस्त्रा
वृश्चिकसिंहयोरुदयविनाड्यः ।
प्रथमचरगुणार्धयुतानि वसुभानि तुलाकन्ययोरुदयविनाडयः ।
एवं स्वदेशोदयविनाड्य आनेतव्या इति ॥ २ ॥ 

\vspace{2mm}
\centering
\textbf{
\hspace{-2.2cm}
 स्वोदयैः प्रश्ननाडीभिर्वर्धितोऽर्कोऽनुपाततः । \\
 लग्नं तद्वद् विवृद्धेऽर्के लग्नतुल्ये तु नाडिकाः ॥ ३ ॥ }
\footnotetext{
 १ क. °षुपदान्यर्क° । २ क. °न्ति । अर्कवदन्ये° ।}

\newpage
\thispagestyle{fancy}
\fancyhf{}
\chead{लघुमानसम् ।}
\rhead{१५}
\lhead{[त्रिप्रश्नाधिकारः]}
\justifying
 अथोदयलग्नानयनं [ तेन ] तेन नाडीकरणं चाऽऽह-- स्वोदयैरिति ।
याव-
तीभिर्नाडिकाभिर्वर्धितोऽर्को लग्नसमो भवति तावत्यो नाडिका इत्यर्थः ।
स्वोद-
यविनाडीभिः प्रश्नविनाडीभिश्चानुपाततस्रैराशिकाद्वर्धितोऽर्को लग्नं भवति
। 
तद्वद्विवृद्धेऽर्के लग्नतुल्ये सति नाडिका भवति । एतदुक्तं भवति--
तत्कालार्केऽयनचलनं
प्रक्षिप्यार्कस्थितराशेरेष्यानंशान् स्वोदयविनाडीभिर्निहत्य
त्रिंशता विभर्ज्ये-
लब्धा विनाडीरिष्टविनाडीभ्यो विशोध्य सायनार्के राशिशेषं प्रक्षिप्य पुनरिष्टविनाडीभ्य
एष्यविनाडीभ्यो यावतां राशीनां विनाड्यो विशोध्याः, 
अर्कस्थितोपरिराशेरारभ्य तावतां राशीनामुदयविनाडीः संशोध्यर्के तावतो 
राशीन् प्रक्षिप्य पुनः शिष्टविनाडीस्त्रिंशता निहत्य
वर्तमानराशेरुदयविनाडीभिर्विभज्य
लब्धानंशान् गृहीत्वा [ शेषात् षष्टिघ्नात्ताभिरेव
विनाडीभिर्विभज्य
लब्धा लिप्ताश्च गृहीत्वा ] तान् भागान् सलिप्तान् तस्मिन्नर्के
प्रक्षिप्यायनचलनं 
विशोधयेत् । सोऽर्क उदयलग्नं भवति । नाडीकरणं तु तत्कालार्क उदयलग्ने
चायनचलनं प्रक्षिप्यार्कस्यैष्यानंशान् स्वोदयविनाडीभिर्निहत्य त्रिंशता
विभज्य 
लब्धा विनाडीर्गृहीत्वा पुनः सायनलग्नस्य गतानंशान् स्वोदयैर्निहत्य
त्रिंशता 
विभज्य लब्धा विनाडीः पूर्वानीतविनाडीषु प्रक्षिप्य
लग्नार्कयोरन्तरालगतराशीनां
विनाडीश्च प्रक्षिपेत् । ता विनाडय उदयादतीतविनाड्यो भवन्तीति
। 
रात्रौ तु षड्राशियुतं सायनमर्कं प्रकल्यास्तमयाद् गता
विनाडीरतीतविनाडीश्च 
प्रकल्प्य सर्वं कुर्यात् । यदा पुनरिष्टविनाडीभ्य एष्यविनाडयो न 
शोध्यास्तदेष्टविनाडीस्त्रिंशता
निहत्य स्वोदयेन विभज्य लब्धान् भागान् सलिप्तांस्तत्कालसायनार्के
प्रक्षिप्यायनचलनं विशोधयेत् । तदुदयलग्नं भवति । 
नाडिकानयने लग्नार्कयोरेकराशिगतत्वे सति तयोर्विवरांशानामेव विनाडिका 
ग्राह्याः ॥ ३ ॥ 

\vspace{2mm}
\centering
\textbf{
  व्यस्तं चरविनाडीभिः खाग्नयः संस्कृता दिनम् ।\\
 मध्याह्नान्नतनाड्यः स्युर्दिनार्धद्युगतान्तरम् ॥ ४ ॥ }

\vspace{2mm}
\justifying
 दिनप्रमाणानयनं नतानयनं चाऽऽह-- व्यस्तमिति । सायनार्कस्य
भुजाराशीं-
श्चरगुणैर्निहत्य चरविनाडीरानीय ताः षष्ट्या समारोप्य नाडीकृत्य स्वाग्निषु
त्रिंशन्नाडीषु व्यस्तं संस्कुर्यात् । सायनार्के मेषादिगे धनं कुर्यात्
। सायनार्के 
\footnotetext{
१ क. °प्य पुनः शिष्टविनाडीभ्यो या° । २ क. °डीर्वि° । ३ क.°ब्धा
लिप्ताश्च द्व° ।}

\newpage
\thispagestyle{fancy}
\fancyhf{}
\chead{पारमेश्वरव्याख्यासंवलितं -}
\rhead{[त्रिप्रश्नाधिकारः]}
\lhead{१६}
\justifying
\noindent
तुलादिग ऋणं कुर्यात् । तद्दिनप्रमाणं भवति । दिनप्रमाणं षष्टितो
विशोध्य शिष्टं
रात्रिमानं भवति । मध्याह्नान्नतनाड्यः स्यु-- मध्याह्नात्प्राक्
पश्चाद्वा जाता
घटिका नतनाडयो भवन्ति । तदानयनमेवमित्याह--
दिनार्धद्युगतान्तरमिति ।
दिनार्धप्रमाणदिनगतनाडयोरन्तरं नतं भवतीत्यर्थः ॥ ४ ॥

\vspace{2mm}
\centering
\textbf{
\hspace{-1.4cm}
 पञ्चघ्नेष्टचरार्धेन पलभाप्तेन संस्कृतात् ।\\
 आद्याच्चरगुणादह्ना दिगूनेन दिनार्धभा ॥ ५ ॥}

\vspace{2mm}
\justifying
 अथेष्टच्छायाया आनयनसाधनभूताया मध्यच्छायाया आनयनमाह---  पञ्चघ्नेति ।
इष्टकाले सायनरवेश्वरविनाडीरानीय पञ्चभिर्निहत्यार्धीकृत्य
पलभायाः
स्वदेशविषुवच्छायाङ्गुलेन विभज्य लब्धमाद्यचरगुणे प्रथमभुजाराशेश्चरगुणे
संस्कुर्यात् । मेषादिगेऽर्के ऋणं कुर्यात् । तुलादिगेऽर्के धनं
कुर्यात् । एवं संस्कृतमाद्यं
चरं गुणं दिगूनेनाह्ना नाडिकादशकहीनाभिर्दिनप्रमाणनाडीभिर्विजजेत् ।
तत्र  लब्धं दिनार्धे मध्याह्ने छायाङ्गुलं भवति । शिष्टात् षष्टिघ्नाद्
दिगूनेनाह्ना
लब्धानि व्यङ्गुलानि भवन्ति । ऋणात्मकस्य पलभाप्तफलस्य द्युचरगुणादधि-- कत्वे [ सति ] तस्मादाद्यचरं गुणं   विशोध्य   शिष्टाद् दिगूनेनाह्ना
   दिनार्धमा
भवति । तन्मध्यच्छायाङ्गुलं व्यङ्गुलसहितमेकत्र संरक्षेत् ॥ ५ ॥

\vspace{2mm}
\centering
\hspace{-1.8cm}
\textbf{
 विदिग्दिननवाभ्यासान्नतकत्यंशको युतः ।\\
 विदिग्दिनशतांशेन गुणोऽसौ व्येकको हरः ॥ ६ ॥}

\vspace{2mm}
\justifying
 इष्टच्छायानयनमाह-- विदिगिति । नाडिकादशकहीना दिनप्रमाणनाड्यो
विदिग्दिनमित्युच्यते । विदिग्दिननाडीर्नवभिर्निहत्य तत्कालनाडीवर्गेण
विभज्य
लब्धं सावयवं विदिग्दिननाडीशतांशेन योजयेत्। स गुणो भवति । विदिग्दिननतांशः
सर्वदाऽवयव एव भवति । अतस्तदवयवो   नतकृत्यंशकस्यावयवे
प्रक्षिप्यते । स गुणकार इत्यर्थः । व्येकको हरः--  असौ गुणकार
एकहीनो
हारक इत्यर्थः । एवं गुणकारहारकावानीयैकत्र संरक्षेत् । कदाचित्
पुनरुदयासन्ने
गुणस्य रूपादल्पत्वाद्धारको न संभवति । अतस्तत्र स्फुटतरो गुण
आनेतव्यः । 
तत्प्रकारस्तु--उदयकाले 'विदिग्दिननवाभ्यासात्'
इत्यादिना गुणमानीय
तद्गुणरूपसंख्ययोर्विवरमिष्टगुणे स्वर्णं कुर्यात् । गुणाद् रूपेऽधिके
धनं कुर्यात् ।
गुणाद् रुपे स्वल्प ऋणं कुर्यात् । स स्फुटगुणः स्यात् । तथा च
कश्चिदाह--
\footnotetext{
 १ क. °तघटिकयो° । २ क.°लस्याऽऽद्यच° । ३ क. °कृत्यांश° । ४ क. °त्कालान-\\
तना° । ५ क. °नशतांशः ।}

\newpage
\thispagestyle{fancy}
\fancyhf{}
\chead{लघुमानसम् ।}
\rhead{१७}
\lhead{[त्रिप्रश्नाधिकारः]}
\centering
\textbf{
\hspace{-2cm}
 रूपोदयोत्थगुणयोर्विवरेण युतो गुणः ।\\
\hspace{1.5cm}
 इष्टस्फुटः स्यादधिके रूपेऽल्पे तु विवर्जितः ॥ इति ॥ ६ ॥\\
 \hspace{-1.5cm}
 तदैक्याच्छङ्कवर्गघ्नान्मध्यच्छायागुणाहतेः ।\\
  कृत्यायुतात्पदं यत्स्यात्तस्माच्छेदाप्तमिष्टभा ॥ ७ ॥}

\vspace{2mm}
\justifying
 तदैक्यादिति । तौ गुणकारहारकौ पृथग् विन्यस्यैकमेकेन संयोज्य शङ्कुव-
र्गेण द्वादशानां वर्गेण निहत्य पुनस्तस्मिन्मध्यच्छायाङ्गुलगुणितस्य
गुणकारस्य
कृतिं प्रक्षिप्य मूली कुर्यात् । पुनस्तन्मूलं छेदेन हारकेण विभजेत्।
तत्र लब्धं
द्वादशाङ्गुलशङ्कोश्छायाङ्गुलं भवति । शिष्टात्षष्टिघ्नाच्छेदेन विभज्य
लब्धं
व्यङ्गुलं भवति। इष्टभा-- इष्टकाले द्वादशाङ्गुलशङ्कोश्छायेत्यर्थः
। मध्यच्छायाभावे
तु गुणहारैक्याच्छङ्कुवर्गघ्नाद् यत्पदं तस्माच्छेदाप्तमिष्टभा
स्यात् । तत्र
मध्याह्ने कर्णो द्वादश एव । पुनस्तच्छायाङ्गुलं त्रयोदशभिर्निहत्य
चतुर्विंशत्या
विभजेत् । तत्र लब्धं पुरुषस्य च्छायापदं भवति ।
शिष्टादष्टघ्नाच्चतुर्विंशत्या
लब्धमङ्गुलं भवति ॥ ७ ॥

\vspace{2mm}
\centering
\textbf{
\hspace{-1.3cm}
 छायार्कवर्गसंयोगान्मूलं कर्णस्ततोऽपि भा ।\\
 इष्टः कर्णः स्वमध्याह्नकर्णान्तरहृतो गुणः ॥ ८ ॥}

\vspace{2mm}
\justifying
 अथेष्टच्छायया घटिकानयनाय कर्णानयनमाह-- छायार्केति । इष्टकाले
पुरु-
षस्य च्छायापदानि चतुर्विंशत्या निहत्य त्रयोदशभिर्विभजेत् । तत्र
लब्धं द्वादशाङ्गुलशङ्कोश्छायाङ्गुलं
भवति । एवंविधिना शङ्कुस्थापनेन वा
द्वादशाङ्गु-
लशङ्कोश्छायाङ्गुलं सावयवमानयेत् । तं वर्गीकृत्य तस्मिन् वर्गेऽर्कानां
द्वादशानां
वर्गं प्रक्षिप्य मूली कुर्यात् । तन्मूलमिष्टकर्णो भवति
मध्यच्छायावर्गार्क-
वर्गयोर्योगस्य मूलं मध्याह्नकर्णो भवति । एवमिष्टकर्णं मध्याह्नकर्णं च
सावय-
वमानयेत् । ततोऽपि भा [ प्रथमं कर्णमानीय तस्मात् ] कर्णादपि
च्छायाऽऽने
तव्या । कर्णवर्गादर्कवर्गं विशोध्य शिष्टस्य मूलं छायेत्यर्थः।
कर्णस्तु--

\vspace{2mm}
\centering
 मध्याह्नकर्णो हाराप्तः स्वयुतः कर्ण इष्टजः ।\\
 \hspace{-1cm}
 कर्णार्कवर्गविवरपदं शङ्कुप्रभा भवेत् ॥

\vspace{2mm}
\justifying
इत्यनेन वेद्यः । निरक्षदेशे तु नखान्यष्टिः सप्त च क्रमाच्चरगुणं
प्रकल्प्येष्टचरार्धा-
त्पञ्चघ्नाद् विंशत्या मध्याह्नभामानीय पुनश्चरं विनेष्टभामानयेत् ।
\footnotetext{
 १ क. °योगम्° ।}

\newpage
\thispagestyle{fancy}
\fancyhf{}
\chead{पारमेश्वरव्याख्यासंवलितं -}
\rhead{[ग्रहणाधिकारः]}
\lhead{१८}
\justifying
 इष्टः कर्ण इति । इष्टकर्णान्मध्याह्नकर्णं विशोध्य शिष्टेनेष्टकर्णं
विभजेत्
तत्र लब्धं गुणो नाम भवति । एवं सावयवं गुणमानयेत् ॥ ८ ॥

\vspace{2mm}
\centering
\textbf{
\hspace{-1.5cm}
 विदिग्दिनशतांशोनगुणकेन विदिग्दिनात् ।\\
 नवाहतात्फलं यत्स्यात्तन्मूलं नतनाडिकाः ॥ ९ ॥}

\vspace{2mm}
\justifying
 विदिग्दिनशतांश गुणाद् विशोध्य तेन गुणेन नवाहतं विदिग्दिनं विभजेत्
तत्र यल्लब्धं तस्य मूलं नतनाडिका भवन्ति । नतनाडिका दिनार्धप्रमाणाद्
वि-
शोध्य शिष्टं दिनस्य गता वा एष्या वा घटिका भवन्ति । अत्र ग्रामे
विषुव-
च्छायादेशान्तरयोजनानि प्रदर्श्यन्ते--\\

\vspace{2mm}\centering
\hspace{-0.5cm}
 \renewcommand{\thefootnote}{*}\footnote{अश्वत्थाख्यो ग्राम: 'आलत्तूर' इति केरलेषु प्रसिद्धः । व्याख्याकृतः
परमेश्वर-
स्वायमा...जिनप्रदेशः ।}
ग्रामादश्वत्थाख्यादस्माद् धृतियोजने तु समरेखा ।\\
\hspace{0.8cm}
 प्राच्यामत्र तु फलभा स्यात्सधृतिव्यङ्गुलाङ्गुलद्वितयम् ॥\\
 इति पारमेश्वरे मानसव्याख्याने त्रिप्रश्नाधिकारः ॥\\
 इति पारमेश्वरे मानसव्याख्याने द्वितीयोऽध्यायः ॥\\

\rule{0.2\linewidth}{1.0pt}\\

\vspace{4mm}
\centering
\textbf{
 अथ तुतीयोऽध्यायः ।}\\

\vspace{2mm}
अथ ग्रहणाध्यायः प्रारभ्यते ।\\

\vspace{2mm}
\textbf{
 ग्रहयोरन्तरे स्वल्पेऽनल्पभुक्तेः पुरःसरः ।\\
\hspace{1.8cm}
 यदाऽल्पगतिरेष्यः स्यात्तदा योगोऽन्यथा गतः ।। १ ।।}\\

\vspace{2mm}
\justifying
 अथ ग्रहणसमागमादिपरिज्ञानार्थानाह--ग्रहयोरिति । यदा
ययोर्ग्रहयोरन्तरे
स्वल्पं भवति तदा तयोः समागमान्वेषणं कार्यम् । तत्र
यदाऽनल्पभुक्तेरधिकभुक्तेः
पुरस्तादल्पगतिग्रहो भवति तदा तयोः समागम एष्यः स्यात् । यदाऽल्पगतेः
पुरस्तादधिकभुक्तिश्चेद्
योगो गतोऽतीत इत्यर्थः । ग्रहयोर्वक्रकाले
त्वधःस्थितः पुरस्ताद्
गतो भवति ॥ १ ॥

\vspace{2mm}
\centering
\textbf{
 युक्त्या भिन्नदिशोर्गत्योरन्तरेणैकदिक्कयोः ।\\
 \hspace{1.3cm}
 ग्रहान्तराद् दिनानि स्युस्तैः समावनुपाततः ॥ २ ॥}

\newpage
\thispagestyle{fancy}
\fancyhf{}
\chead{लघुमानसम् ।}
\rhead{१९}
\lhead{[ग्रहणाधिकारः]}
\justifying
 युक्त्येति । भिन्नदिशयोर्वक्रावक्रगतयोर्ग्रहयोर्गतियोगेनैकदिक्कयोः
स्पष्टगत्योर्व-
क्रगत्योश्च गत्यन्तरेण ग्रहयोरन्तराल्लिप्तीकृता [ द् ] दिनानि
भवन्ति । शिष्टात्
षष्टिघ्नान्नाड्यश्च भवन्ति । इष्टकालसमागमकालयोरन्तरालदिनानीत्यर्थः
। अत्र
गतिर्लिप्तात्मिका स्यात् । तैः समावनुपाततस्त्रैराशिकाद् ग्रहौ समौ
कुर्यात् ।
एतदुक्तं भवति-- पर्वणि मध्याह्ने चन्द्रार्कयोर्मध्यममानीय तस्मिन्
मध्यमे देशान्तरं
भुजाविवरं च कृत्वा स्फुटी कुर्यात् । तौ स्फुटो चन्द्रार्कौ भवतः । 
पुनस्तयोरन्तरं
लिप्तीकृत्य षष्ट्या निहत्य स्फुटगत्यन्तरलिप्ताभिर्विभजेत् । तत्र
लब्धं घटिका-
दिसमागमकालो भवति । मध्याह्नात्पूर्वान्तकाल इत्यर्थः ।
पुनस्तन्नाडिकाभिर-
र्केन्द्वोः स्फुटगती पृथङ् निहत्य षष्ट्या विभज्य लब्धा लिप्ता
मध्याह्नार्के तच्चन्द्रे
चर्णे धनं वा कुर्यात् । एष्यो योगश्चेद् धनं कुर्यात् ।
योगो गतश्चेदृणं
कुर्यात् । तदाऽर्केन्दू समौ भवतः । पुनस्ताः समागमघटिका एकत्र संरक्षेत्
। चन्द्रग्रहणे
तु चन्द्रषड्भयुतार्कयोरन्तरान्नाड्यः साध्याः । एवमन्येषामपि
समागमकालानयनं
समीकरणं च कुर्यादिति । भुजाविवरं तु ग्रहस्य
मध्यभुक्तिकलाषडंशं
रविभुजाफललिप्ताभिर्निहत्य दिक्षड्गुणैः (३६१०) एभिर्विभज्य लब्धा लिप्ता
भानुभुजावशाद् ग्रहमध्य ऋणं धनं वा कुर्यात् । एतदस्माभिः
प्रदर्श्यते--

\vspace{2mm}
\centering
 " भुक्तिलिप्ता षडंशघ्नरविदोःफललिप्तिकाः ।\\
 \hspace{2.5cm}
 दिक्षड्गुणाप्ताः स्वर्णाः स्युर्लिप्ता भानुभुजावशात् " ॥ इति ।

\vspace{2mm}
\justifying
 एतद् भेदबाहुल्याभावात्किलाऽऽचार्येण नोपदिष्टम् ॥ २ ॥

\vspace{2mm}
\centering
\textbf{
 भानोर्बिम्बा रविच्छेदहृता खखकृताचलाः ।\\
 \hspace{0.5cm}
 शशिनः खखभूरामाश्चन्द्रमन्दहरोद्धताः ॥ ३ ॥}

\vspace{2mm}
\justifying
 अथार्कादीनां बिम्बलिप्तानयनमाह-- भानोरिति । खखकृताचलान् (७४००)
रविस्फुटच्छेदेन विभजेत् । तत्र लब्धं लिप्तादि रवेर्बिम्बं भवति ।
रविबिम्बव्या-
सलिप्तेत्यर्थः । खखभूरामान् (३१००) चन्द्रस्य मन्दस्फुटच्छेदेन विभज्य
लब्धं शशिनो बिम्बं भवति । मन्दहर इति मन्दग्रहणं द्वितीयस्फुटे
रूपहारकनिरासार्थम् ॥ ३ ॥
\footnotetext{
 १ क. °रन्तं लि° । २ क. मकालनाडिका । ३ क. °हरेर्धृ°}

\newpage
\thispagestyle{fancy}
\fancyhf{}
\chead{पारमेश्वरव्याख्यासंवलितं -}
\rhead{[ग्रहणाधिकारः]}
\lhead{२०}
\centering
\textbf{
 छायाग्रहः सषड़भोऽर्कस्तन्मण्डलकलामितिः ।\\
 \hspace{0.3cm}
 चन्द्रमार्गे शशिच्छेदहृताः खखगुणोरगाः ॥ ४ ॥}

\vspace{2mm}
\justifying
 भूछायावस्थितिं तद्व्यासानयनं चाऽऽह-- छायेति ।
षड्राशियुतस्तत्कालार्कः
छायाग्रहो भवति । अर्कस्य सप्तमराशौ सदा भूछाया वर्तत इत्यर्थः ।
तन्मण्ड-
लकलामितिस्त्वेवमित्याह-- शशिच्छेदेति । खखगुणोरगान् ( ८३०० )
चन्द्रस्य
स्फुटच्छेदेन विभजेत् । तत्र लब्धं लिप्तादि च्छायाचिम्बं भवति ।
चन्द्रमार्गे
चन्द्रकक्षायामेवं बिम्बप्रमाणमित्यर्थः । भूछायया हि चन्द्रश्छाद्यते
। तरणिस्तु
चन्द्रमसा छाद्यते ॥ ४ ॥

\vspace{2mm}
\centering
\textbf{
 अङ्गानीशा नखाः सूर्या द्वियमा दशताडिताः ।\\
 \hspace{0.8cm}
 स्वशीघ्रच्छेददिग्योगहृता बिम्बानि भूसुतात् ॥ ५ ॥}

\vspace{2mm}
\justifying
 अङ्गानीति । कुजस्म दशताडितान्यङ्गानि ( ६० ) ।  बुधस्य तथा ईशाः
( ११० ) । गुरोस्तथा नखाः (  .०० )। शुक्रस्य तथा सूर्याः ( १२०  ) ।
मन्दस्य तथा द्वियमाः ( २२० ) । एते स्वशीघ्रच्छेददिग्योगहृता दशसहितेन
स्वशीघ्रच्छेदेन हृता भूसुताद् बिम्बानि भवन्ति ॥ ५ ॥

\vspace{2mm}
\centering
\textbf{
 कृतनेत्रभुजङ्गाङ्गदिशो दशहताः क्रमात् ।\\
 \hspace{0.8cm}
 पातभागाः कुजादीनां पातक्षेपा न भास्वतः ॥}

\vspace{2mm}
\justifying
 कृतनेत्रेति। कृतादीनि दशघ्नानि कुजादीनां पातभागा भवन्ति ।
तानि--कु-
जस्य ४०। बुधस्य २०। गुरोः ८०। शुक्रस्य ६०। मन्दस्य १००।
पातक्षेपौ
न भास्वतः--सूर्यस्य पातः क्षेपश्च न भवतः । प्रक्षेपोऽयं
श्लोक इति केचित् ॥

\vspace{2mm}
\centering
\textbf{
 मन्दस्फुटात्स्वपातोनाद् ग्रहाच्छीघ्राद् ज्ञशक्रयोः ।\\
 \hspace{1mm}
 भुजाः षट्कृतिसूर्याष्टिनवाष्ठ्यष्टिहताः क्रमात् ॥ ६ ॥\\
\hspace{5mm}
 चन्द्राद् विक्षेपलिप्ताः स्युस्ताः कुजाद् व्यासताडिताः ।\\
 \hspace{8mm}
 शीघ्रच्छेदाहृताः स्पष्टाः स्वर्णाख्या दक्षिणोत्तराः ॥ ७ ॥}\\

\vspace{2mm}
\justifying
 मन्दस्फुटादिति । शशिकुजगुरुमन्दानां स्वमन्दस्फुटात्स्वपातभागान्
विशोध-
येत् । बुधशुक्रयोस्तु
स्वमन्दभुजाफलविपरीतसंस्कृतास्स्वशीघ्रोच्चात्स्वपातभागान्
विशोधयेत् । अत्र मन्दफलविपरीतसंस्कारः संप्रदायात् सिद्धः । एवं
स्वपातभागान्
विशोध्य शिष्टस्य भुजज्यामानीय षट्कृत्यादिभिर्गुणयेत्। तत्र
चन्द्रस्य
 '' चतुस्रयेकघ्न " इत्यादिविधिनाऽऽनीतां भागात्मिकां भुजज्यामानीय तां
\footnotetext{
 १ क. °शगुणाः क्°र । २ क. °क्षेपौ न° ।}

\newpage
\thispagestyle{fancy}
\fancyhf{}
\chead{लघुमानसम् ।}
\rhead{२१}
\lhead{[ग्रहणाधिकारः]}
\justifying
\noindent
षट्कृत्या  षट्त्रिंशता  गुणयेत् । कुजस्य तथाभूतां भुजज्यां
सूर्यैर्द्वादशभिर्गुणयेत् ।
बुधस्याष्ट्या षोडशभिः, गुरोर्नवभिः, शुक्रस्याष्ट्या षोडशभिः,
मन्दस्याप्यष्ट्या
षोडशभिः । एवं स्वेन स्वेन गुणकारेण गुणिता भागात्मिका भुजज्या
विक्षेपलिप्ता
भवन्ति । चन्द्रस्या ता एव स्फुटा भवन्ति । कुजादीनां तु ता
विक्षेपलिप्ताः स्वेन
स्वेन स्फुटव्यासेन निहत्य स्वशीघ्रच्छदेन विभजेत् । तत्र लब्धाः
स्पष्टा विक्षेप--
लिप्ता भवन्ति । स्फुटा भवन्तीत्यर्थः। स्वर्णाख्या दक्षिणोत्तराः
पातोनभुजाया
धनत्वे दक्षिणविक्षेपो भवति । पातोनभुजाया ऋणत्वे उत्तरविक्षेपो भवति
। एवं
विक्षेपमानयेत् ॥ ६ ॥ ७ ॥

\vspace{2mm}
\textbf{
\centering
 विक्षेपयोः समदिशोरन्तरं भिन्नयोर्युतिः ।\\
 \hspace{3cm}
 बिम्बान्तरं लघुन्यस्मिन् भेदो मानार्धयोगतः ॥ ८ ॥}

\vspace{2mm}
\justifying
 अथ ग्रहणस्य सदसद्भावपरिज्ञानार्थमाह-- विक्षेपयोरिति ।
ययोर्ग्रहयोर्ग्रहणं
निरूप्यते तयोर्द्वयोः समदिशोर्विक्षेपयोरेकदिक्कत्वे विक्षेपयोरन्तरं
कुर्यात् । भिन्न-
दिक्कत्वे विक्षेपयोर्योगं कुर्यात् । तद्विम्बान्तरं
बिम्बमध्ययोरन्तराललिप्तेत्यर्थः । अ-
स्मिन् बिम्बान्तरे मानार्धयोगाद् बिम्बैक्यार्धाद् लघुन्यल्पे सति भेदः
स्याद् ग्रहणं
भवेत् । मानार्धयोगादधिके बिम्बान्तरे सति ग्रहणं न स्यात् ।
सूर्यग्रहणे चन्द्र-
ग्रहणे च चन्द्रविक्षेप एव बिम्बान्तरं भवति ।
अर्कभूछाययोर्विक्षेपाभावात् । किं-
त्वर्कग्रहणे सति संस्कृतविक्षेपो विक्षेपः स्यात् ॥ ८ ॥

\vspace{2mm}
\centering
\textbf{
 ग्रहोनं लग्नमक्षघ्नं लम्षनद्युगतं युतौ ।}
 
 \textbf{
 [लम्बनद्युगतात्पञ्चदशभिर्नतसाधनः ।\\
 \hspace{0.3cm}
 दिनार्धेन नतं मध्यं लम्बनस्फुटपर्वणः] ॥}
 
 \textbf{
\hspace{1cm}
 लम्बनं द्व्यक्षकर्णाप्तं नतोनाहतविंशतेः ॥ ९ ॥  }

\vspace{2mm}
\justifying
 अथ सूर्यग्रहणे पर्वणि संस्कारार्थं लम्बनकालानयनमाह-- ग्रहोनमिति ।
मध्याह्नात्प्रागर्केन्द्वोः समागमश्चेत्पूर्वानीताः समागमकालघटिका
दिनार्धप्रमाणाद्
विशोधयेत् । मध्याह्नतः परस्तात्समागमश्चेत्समागमं घटिकादिनार्धप्रमाणे
प्रक्षिपेत् । एवं कृत उदयात्पर्वान्तघटिका भवन्ति । एवं
दिने समा-
गमश्चेत् । रात्रौ चेत्समागमघटिकाभ्यो दिनार्धं शोधयेत् । तत्र शिष्टा
रात्रेर्गता
गन्तव्या वा नाडयः स्युः। एवं पर्वान्तकालं विज्ञाय तत्पर्वान्त
उभयलग्नमानीय
तस्मादुदयलग्नात्स्फुटार्कं विशोध्य शिष्टस्य राशीनक्षैः पञ्चभिर्गुणयेत्
। भागान्
दशभिर्निहत्य षष्ट्या समारोप्य राशिस्थाने प्रक्षिपेत् । [
राशिस्थाननाड्यात्मक-
\footnotetext{
 १ क. °होनल° ।}

\newpage
\thispagestyle{fancy}
\fancyhf{}
\chead{पारमेश्वरव्याख्यासंवलितं -}
\rhead{[ग्रहणाधिकारः]}
\lhead{२२}
\justifying
\noindent
स्वादंशा दशभिर्निहन्यन्ते ] । तत्र राशिस्थानगता घटिका भवन्ति ।
भागस्था-
नगता विनाडयः । एतद्युतौ ग्रहणे लम्बनद्युगतं भवति। लम्बनानयनाय
द्युग-
तकाल इत्यर्थः । पुनः सर्वदाऽपि पञ्चदश नाडिकः एव दिनार्धं परिकल्प्य
तद्दि-
नार्धलम्बनद्युगतनाडीभ्यां नतनाडीरानीय तं नतं विंशतिनाडिकाभ्यो विशोध्य
शिष्टं तेनैव नतेन नाडिकात्मकेन निहत्य द्व्यक्षकर्णन विभजेत् ।
द्विगुणितवि-
षुवच्छायाकर्णेनेत्यर्थः । तत्र लब्धनाडिकादिलम्बनकालो भवति।
यत्पुनश्चर-
वशात्कल्पितं दिनार्धमिति केनचिद्व्याख्यातं तदसत् । दृक्षेपलग्नेऽपि
लम्बन-
संभवात् ॥ ९ ॥

\vspace{2mm}
\centering
\textbf{
\hspace{-2.3cm}
 प्राक्पश्चाल्लम्बनेनोनयुक्तं दिनगतं स्फुटम् ।\\
 तन्नताक्षांशहीनः प्राक् सूर्यः खार्कोऽन्यथा युतः ॥ १० ॥}

\vspace{2mm}
\justifying
 लम्बनस्य धनर्णविभागमाह प्राक्पश्चादिति । प्राक् पश्चाच्च क्रमेण
लम्बनो-
नयुक्तं दिनगतं स्फुटं भवति । एतदुक्तं भवति-- लम्बनद्युगते
पञ्चदशनाडिकात्म-
काद् दिनार्धादूने सति लम्बनघटिकाः पर्वान्तद्युगतघटिकाभ्यः शोधयेत् ।
लम्ब-
नद्युगते पञ्चदशनाडिकात्मकाद् दिनार्धादिधिके सति लम्बनघटिकाः
पर्वान्ताद्युगते-
घटिकाभ्यो योजयेत्। एतद् ग्रहणमध्यकाल इत्यर्थः ।\\
\justifying
 इदानीमवनत्यानयनाय खार्कमाह-- तन्नतेति । तेन
लम्बनसंस्कृतपर्वान्तद्युगतेन
चरवशात्कल्पितेन दिनार्धेन च नतनाडिकामानीय तां नतनाडिकामक्षैः पञ्चभि-
र्विभज्य लब्धान् राशीन् गृहीत्वा तच्छेषं त्रिंशता निहत्य पञ्चभिर्विभज्य
लब्धान्
भागांश्च गृहीत्वा तच्छेषं षष्ट्या निहत्य पञ्चभिर्विभज्य लब्धा लिप्ताश्च
गृहीत्वा
तान् राश्यादीन् तत्कालार्क ऋणं धनं वा कुर्यात्
लम्बनसंस्कृतपर्वान्तघटिका
दिनार्धादूनाश्चेदृणं कुर्यात् । अधिका चेद्धनं कुर्यात् । प्राक्कपाल
ऋणं पश्चात्क-
पाले धनमित्यर्थः । एवं संस्कृतोऽर्कः खार्कसंज्ञितो भवति ।
सूर्यशब्देन च्छाद्य-
ग्रहो विवक्षितः ।

\vspace{2mm}
\centering
\textbf{
\hspace{-2cm}
 " अङ्गुलानि द्वादशात्र व्यङ्गुलानि त्रयोदश ।\\
 ग्रामेऽक्षकर्णो निर्दिष्टः पूर्वैर्गणितकोविदैः " ॥ १० ॥}
\footnotetext{
 १ क. °तघटिकासु प्रक्षिपेत् । एवं लम्बनसंस्कृतं पर्वान्तद्युगतं स्फुटं
भवतीति । त° ।}

\newpage
\thispagestyle{fancy}
\fancyhf{}
\chead{लघुमानसम् ।}
\rhead{२३}
\lhead{[ग्रहणाधिकारः]}
\centering
\textbf{
\hspace{-0.8cm}
 तदिष्टचरषड्घातपलभाप्तेन संस्कृतात् ।\\
\hspace{2cm}
 पलभोनाह (हृ) तात्खाक्षाद् द्विघ्नात्तत्त्वैर्हृता नतिः ॥ ११ ॥}

\vspace{2mm}
\justifying
 इदानीं नतलिप्तानयनमाह-- तदिष्टेति । तस्य खार्कस्य सायनस्य
चरविना-
डीरानीय ता विनाडीः षड्भिर्निहत्य पलभया विषुवच्छायाङ्गुलेन विभज्य लब्धं
कुत्रचिद विन्यस्य पुनः खाक्षात् पञ्चांशतः पलभाङ्गुलं विशोध्य शिष्टं
तेनैव
पलभाङ्गुलेन निहत्य तस्मिन् पूर्वपलभाप्तं फलं संस्कुर्यात् ।
सायनखार्के मेषा-
दिग ऋणं कुर्यात् । [ सायनखार्के तुलादिगे धनं कुर्यात् ] । एवं
संस्कृतं पलभो-
नाहतं खाक्षं द्वाभ्यां निहत्य तत्वैः पञ्चविंशत्या विभजेत् । तत्र
लब्धा नतिलिप्ता
भवन्ति । यदा पुनः पलभाप्तमृणात्मकं फलं पलभोनाहतखाक्षादधिकं
भवति तदा तस्मात् [ पलभाप्त ] फलात्पलभोनाहतं खाक्षं विशोध्य
शिष्टाद्
द्विध्नाद् तत्त्वैर्लिप्ता भवन्ति। खाक्षो नित्यदक्षिणो भवति । यदा
पलभाप्तमृणात्मकं
फलं फलभोनाहतखाक्षादधिकं भवति तदा नतिरुत्तरा भवति । अन्यथा
सदैव दक्षिणा नतिर्भवति । खार्कवशात् [ तदा ] खाक्षवशाच्च दिग्
विज्ञेयेत्यर्थः ।
विश्लेषे सर्वत्र श्लिष्टस्य दिग् ग्राह्या ॥ ११ ॥

\vspace{2mm}
\centering
\textbf{
 तात्कालिकेन्दुविक्षेपो युक्तो नत्यैकदिक्कया ।\\
 \hspace{2cm}
 हीनोऽन्यथा युतौ स्पष्टच्छादकोऽधः स्थितो ग्रहः ॥ १२ ॥}
 
\vspace{2mm}
\justifying
 अथ नतिविक्षेपयोर्योगविश्लेषेण स्फुटविक्षेपानयनमाह-- तात्कालिकेति
। येन
द्युगतेन खार्कः सावितस्तत्कालजस्य चन्द्रस्य विक्षेपमानीय
तद्विक्षेपनत्योरेकदि-
क्कयोर्योगं कुर्यात् । भिन्नदिक्कयोस्तु विश्लेषं कुर्यात्। तद्युतौ
ग्रहणे स्फुटविक्षेपो
भवति। विश्लेषे शिष्टस्य दिग् ग्राह्या । एवं
नतिविश्लेषयोर्योगविश्लेषाभ्यां
सूर्यग्रहणे स्फुटविशेषः साध्यः । चन्द्रग्रहणे तु पातोनचन्द्रादानीत एव
स्फुटविक्षेपो
भवति । छादकोऽवः स्थितो ग्रह:-- अवःस्थितो ग्रहश्छादकः ।
ऊर्ध्वस्थितश्छाद्यः ।
चन्द्रग्रहणे तु भूछाया छअधाकश्चन्द्रश्छाद्यः ।
ऊर्ध्वाधोविभागस्तु
भानामधः शनैश्चरसुरगुरुभौमार्कशुक्रबूधचन्द्रा इत्यनेन वेद्यम् ॥ १२ ॥
\footnotetext{
 १ क. °ता । २. क.°ता खा° । २ क. °क्षा द्विग्ता त° । ४ क. °नतिलि । ५
क.\\
पश्चाशत प°। ६ क. °त्वर्नतिर्लि° । ७ क. °कयोः । ८ क. °स्पष्टौ छा° ।}

\newpage
\thispagestyle{fancy}
\fancyhf{}
\chead{पारमेश्वरव्याख्यासंवलितं -}
\rhead{[ग्रहणाधिकारः]}
\lhead{२४}
\centering
\textbf{
 बिम्बान्तरकृतिं प्रोज्झ्य मानैक्यार्धकृतेः पदम् ।\\
 \hspace{0.3cm}
 षष्टिघ्नं समदिग्गत्योरन्तराप्तं स्थितेर्दलम् ॥ १३ ॥}

\vspace{2mm}
\justifying
 अथ स्थित्यर्धनाडिकानयनमाह-- बिम्बान्तरेति ।
मानैक्यार्धकृतेर्बिम्बयोगार्ध-
वर्गाद् बिम्बान्तरस्य स्फुटविक्षेपस्य वर्गे विशोध्य शिष्टस्य मूलं
षष्ट्या निहत्य
गत्यन्तरेण विभजेत् । तत्र लब्धं घटिकादिस्थित्यर्धकालो भवति ।
समदिग्गत्यो-
रिति । कुजादीनां समागम एकस्य वक्रगतत्वेऽन्यस्य क्रमगतत्वे च सति
भुक्ति-
योगेन हरणं कार्यम् । बिम्बान्तरं तु विक्षेपयोः समदिशोरित्यादिनोक्तमेव
॥१३॥

\vspace{2mm}
\centering
\textbf{
\hspace{-1cm}
 स्थित्यर्धे चन्द्रविक्षेपकृतेन्द्रांशयुतोनिते ।\\
 \hspace{1.9cm}
 स्पष्टे स्पार्शिकमूनं स्याद् द्युविक्षेपेऽन्यथा महत् ॥ १४ ॥}

\vspace{2mm}
\justifying
 स्थित्यर्ध इति । पूर्वानीतां स्थित्यर्धनाडिकामुभयत्र
विन्यस्यैकस्माच्चन्द्रविक्षे-
पस्य कृतेन्द्रांशं चतुश्चत्वारिंशदधिकशतांशं विशोधयेत् । अन्यस्मिन्
कृतेन्द्रांशं
प्रक्षिपेत् । एतदुक्तं भवति-- विक्षेपलिप्तां विलिप्तीकृत्य
कृतेन्द्रैः (१४४) विभज्य
लन्धं विनाडिकासु संस्कुर्यात् । एवं कृते स्थित्यर्धे स्फुटे भवतः ।
स्पार्शिकमूनं
स्याद् द्यूविक्षेपेऽन्यथा महत्-- पातोनचन्द्रस्य युग्मपदगतत्वे
तयोरूनं स्पर्शस्थित्यर्धं
स्यात् । महन्मोक्षस्थित्यर्धम् । अन्यथौजपदगतत्वे महत्स्पार्शिकमूनं
मोक्षस्थित्पर्धं
भवति । ग्रहचन्द्रग्रहण एवं कृते स्थित्यर्ध एव स्फुटे भवतः ।
सूर्यग्रहणे तु
वक्ष्यमाणसंस्कारयुते ते स्फुटे भवतः ॥ १४ ॥

\vspace{2mm}
\centering
\textbf{
\hspace{-2.3cm}
 तदूनयुक्तमासान्तद्युगते कृतलम्बने ।\\
 स्पर्शमोक्षौ भवेद्भानोर्न लग्नादिन्दुपर्वाणि ॥ १५ ॥}

\vspace{2mm}
\justifying
 तदूनेति । केवलात्पर्वान्तद्युगतात्स्पर्शस्थित्यर्धं विशोध्य शिष्टं
स्पर्शपर्वं
भवति । केवलपर्वान्तद्युगते मोक्षस्थित्यर्धं प्रक्षिपेत् ।
तन्मोक्षपर्व भवति । ते एव
तदूनयुतमासान्तद्युगते इत्युच्येते । पुनः स्पर्शपर्वणि मोक्षपर्वणि
चाऽऽर्कमानीय
लग्नं चाऽऽनीय '' ग्रहोनं लग्नम् " इत्यादिनोभयत्रापि लम्बनमानयेत् ।
तत्र
स्पर्शलम्बनं स्पर्शपर्वणि स्पर्शलम्बनद्युगतवशात्संस्कुर्यात् ।
मोक्षलम्बनं
मोक्षपर्वणि मोक्षलम्बनद्युगतवशात्संस्कुर्यात् । एवं लम्बनसंस्कृते
स्पर्शमोक्षपर्वणी
स्फुटे भवतः । स्पर्शमध्यपर्वान्तरं  मोक्षमध्यपर्वान्तरं  च
स्फुटे
स्थित्यर्धे भवतः । एवं भानोः सूर्यग्रहणे । न लग्नादिन्दुपर्वाणि--
इन्दुग्रहणे
 " ग्रहोनं लग्नम् " इत्यादिनोक्ते लम्बनसंस्कारं नतिसंरकारं च न कुर्यादित्यर्थः ॥ १५ ॥
\footnotetext{
 १ क. °न्तरलिप्ताभिर्वि° ।}

\newpage
\thispagestyle{fancy}
\fancyhf{}
\chead{लघुमानसम् ।}
\rhead{२५}
\lhead{[ग्रहणाधिकारः]}
\centering
\textbf{
\hspace{-0.5cm}
 युतिमध्यनताम्यस्ता पलभा भानुभाजिता ।\\
 प्रागुदग्दक्षिणं पश्चाद्वलनं रदमण्डले ॥ १६ ॥}

\vspace{2mm}
\justifying
 अथाक्षवलनानयनमाह-- युतिमध्येति । युतिमध्यनतेन तत्तद्ग्रहस्य
ग्रहण
मध्यकालजेन नतेन गुणितात्पलभाङ्गुलाद् द्वादशभिराप्तं वलनाङ्गुलं भवति
अत्र नतनाड्यः पञ्चदशनाडिकाभ्योऽधिकाश्चेत्ता नाडीस्त्रिंशता विशोध्य
शिष्टा
नतनाडयो भवन्ति । प्राक्कपाले तदुदग्वलनं भवति । पश्चात्कपाले
दक्षिणवलनं
भवति । रदमण्डले द्वात्रिंशदङ्गुले व्यासमण्डल एतद्वलनं भवतीत्यर्थः ॥
१६ ॥

\vspace{2mm}
\centering
\textbf{
\hspace{-2cm}
 ग्रहणायनयोरल्पमन्तरं द्विघ्नमायनम् ।\\
 वलनं स्यात्तयोर्योगवियोगात्पारमार्थिकम् ॥ १७ ॥}

\vspace{2mm}
\justifying
 अथाऽऽयनवलनानयनं वलनद्वययोगविश्लेषेण स्फुटवलनानयनं चाऽऽह-- ग्रह-
णायनयोरिति । सायनस्य च्छाद्यग्रहस्य मेषादिगतत्वे तद्ग्रहस्य त्रयाणां
राशीनां
चान्तरमानयेत् । सायनग्रहे तुलादिगे सति तद्ग्रहस्य नवानां राशीनां
चान्तर-
मानयेत् । एवमानीतं ग्रहायनान्न्तरं राश्यादिकं द्विघ्नं वलनाङ्गुलं
भवति । राशि-
स्थानस्थमङ्गुलं भवतीत्यर्थः । ग्रहे मृगादिग् उत्तरमेतद्वलनम् ।
ग्रहे कर्क्यादिगे
दक्षिणं वलनम् । एवं सायनग्रहस्याऽऽयनवशाद् दिग् विज्ञेया ।
एतदायनवलनं
स्थूलं स्यात् । दृक्कर्मार्थं तु सूक्ष्मतरमानयेत् । तत्प्रकारस्तु
सायनग्रहस्य कोट्यु-
त्क्रमज्या स्वपञ्चभागहीनस्फुटमायनवलनं भवति । एतदस्माभिः
प्रदर्श्यते--

\vspace{2mm}
\centering
 ग्रहस्योत्क्रमकोटिज्या स्वपञ्चांशेन वर्जिता ।\\
 आयनं वलनं स्पष्टं भवेत्तु रदमण्डले ॥ इति
 
 \vspace{2mm}
 \justifying
 तयोर्योगवियोगात्पारमार्थिकमक्षवलनायनवलनयोरेकदिक्कयोर्योगं कुर्यात् भि
न्नदिक्कयोर्विश्लेषं कुर्यात् । पत्पारमार्थिकं वलनं भवति । स्फुटतरं
वलनमित्यर्थः ॥ १७ ॥

\vspace{2mm}
\centering
\textbf{
\hspace{-1.2cm}
 षडक्षाङ्गुलयष्ट्यग्रे दृङ्मध्यादंशकोऽङ्गुलम् ।\\
 दिग्वृत्तपरिधौ प्राची वलनाग्रे ततोऽपरा ॥ १८ ॥}

\vspace{2mm}
\justifying
 अथ ग्रहयोः समागमे यष्टियन्त्रेण तयोरन्तरांशविज्ञानं वलनवशादिविभागं
चाऽऽह-- षडक्षेति । षट्पञ्चाशदङ्गुलप्रमाणामेकां यष्टिं कृत्वा
तदग्र एकैकाङ्गु-
लान्तराङ्कितां कांचिद् यष्टिं तिर्यङ् निधाय तां यष्टिं दृशोर्मध्ये
विन्यस्य ग्रहानी-

\newpage
\thispagestyle{fancy}
\fancyhf{}
\chead{पारमेश्वरव्याख्यासंवलितं -}
\rhead{[ग्रहणाधिकारः] }
\lhead{२६}
\justifying
\noindent
क्षेत । यथा तिर्यगवस्थितयष्टिपृष्ठे ग्रहलग्नौ दृश्येते । तत्र
ग्रहयोरन्तरे मादन्त्य- 
ङ्गुलानि भवन्ति तावन्तस्तयोरन्तरभागा भवन्ति ।\\ 
\indent
 दिग्वृत्तेत्यादिना परिलेखनेन ग्रहणाकृतिज्ञानमुच्यते । तदर्थं
षोडशाङ्गुलप्र-
माणेन सूत्रेण वृत्तमालिखेत् । तद् रदमण्डलमित्युच्यते । तद्
वृत्तपरिधौ प्राची- 
दिग् वलनाग्रे स्यात् । ततोऽपरा पश्चिमा दिश् भवति । एतदुक्तं
भवति-- तस्मिन् 
वृत्ते पूर्वापरसूत्रं दक्षिणोत्तसूत्रं च कुर्यात् । ते
दृष्ट्यनुासारिणी दिक्सूत्रे भवतः । 
पुनः पूर्वापरसूत्रस्य पूर्वाग्रपरिधिसंपाताद् यथादिशं वलनाङ्गुलं परिधौ
नीत्वा 
तत्र बिन्दुं कृत्वा तद्बिन्दुं प्राचीं दिशं परिकल्प्य
पूर्वापरदक्षिणोत्तरे सूत्रे कुर्यात् । 
ते ग्रहगत्यनुसारिणी दिक्सूत्रै भवत इति ॥ १८ ॥

\vspace{2mm}
\centering
\textbf{
\hspace{-1cm}
 तत्पूर्वापररेखातो विक्षेपान्तरिता परा ।\\
 रेखा मन्दगतेर्मार्गस्तद्वच्छीघ्रगतेरपि ॥ १९ ॥ }

\vspace{3mm}
\justifying
 तत्पूर्वेति । वलनवशात्कल्पिता या पूर्वापररेखा तद्रेखातो
दक्षिणेनोत्तरेण वा 
छाद्यग्रहविक्षेपलिप्तासमानान्तराङ्गुले वलनानुसारिणीं पूर्वापरां रेखां
कुर्यात् । सा 
मन्दगतेश्छाद्यग्रहस्य मार्गो भवति । तस्यां छाद्यग्रहो गच्छतीत्यर्थः
। तद्वच्छी-
घ्रगतेश्छादकग्रहस्यापि स्वविक्षेपान्तरे वलनानुसारिणीं [ पूर्वापरां
] रेखां कु-
र्यात् । सा शीघ्रगतेश्छादकस्य मार्गो भवति । एवं सूर्यग्रहणे ।
चन्द्रग्रहणे तु 
मन्दगतिश्छादकः शीघ्रगतिश्छाद्यः । सूर्यग्रहणे तु रवेर्विक्षेपाभावाद्
वृत्तमध्य-
गता वलनानुसारिणी रेखैव सूर्यस्य मार्गः स्यात् । चन्द्रग्रहणेऽपि सैव
रेखा 
छायाग्रहस्य मार्गः । उभयत्रापि चन्द्रस्य स्वविक्षेपान्तरकृता रेखा
मार्गः स्यात् 
॥ १९ ॥

\vspace{2mm}
\centering
\textbf{
\hspace{-1.7cm}
 वृत्तमध्याद् यथायावद्विक्षेपैर्ग्रहमध्ययोः ।\\ 
 ग्रहयोर्युतिमध्यं स्यात्ततोऽन्यत्र ग्रहान्तरात् ॥ २० ॥ }

\vspace{2mm}
\justifying
 वृत्तमध्यादिति । वृत्तमध्यात्स्वविक्षेपाग्रे मध्यं
ययोस्तयोर्लिखितयोर्युतिमध्यं 
स्यात् । एतदुक्तं भवति-- वलनवशात्कृता या दक्षिणोत्तरा रेखा
तद्रेखा चन्द्रमार्ग-
संपातं मध्यं कृत्वा चन्द्रबिम्बार्धलिप्तासमानाङ्गुलप्रमाणेन सूत्रेण
वृत्तमालिखेत् । 
पुना रदमण्डलमध्यमेव मध्मं कृत्वा सूर्यबिम्बार्धलिप्तासमानाङ्गुलसूत्रेण
सूर्यबिम्बं 
चाऽऽलिखेत् । तत्र सूर्यबिम्बस्य यावद्भागश्चन्द्रबिम्बेन च्छाद्यते
तावद्भाग-
\footnotetext{
 १ क. °नाङ्गुलान्तरे व° । २ क, °क्षेपा ग्र° । ३ क. °न चन्द्रबिम्बमा° ।}

\newpage
\thispagestyle{fancy}
\fancyhf{}
\chead{लघुमानसम् ।}
\rhead{२७}
\lhead{[दृक्कर्माधिकारः]}
\justifying
\noindent
स्त्वदृश्यो भवति । चन्द्रग्रहणे तु सूर्यबिम्बवच्छायाबिम्बमालिखेदिति
। ततोऽ-
न्यत्र ग्रहान्तरात्-- मध्यकालादन्यत्रेष्टकाले ग्रहान्तराद्
ग्रहगत्यन्तरलिप्ताङ्गुलै-
स्वत्कालवलनेन तत्कालविक्षेपेण च युक्तित इष्टग्रासपरिलेखनं द्रष्टव्यम्
। एतद्
ग्रन्थविस्तरभयादस्माभिर्नात्र प्रदर्शितम् ॥ २० ॥

\vspace{2mm}
\centering
 इति पारमेश्वरे मानसव्याख्याने ग्रहणाध्यायस्तृतीयः समाप्तः ॥\\

\vspace{2mm}
\rule{0.2\linewidth}{1.0pt}\\

\vspace{2mm}
\centering
\textbf{\large
 अथ चतुर्थोऽध्यायः ।}\\

 [ अथ दृक्कर्माधिकारः ] ।\\

\vspace{2mm}
\hspace{-0.7cm}
\textbf{
 तिथिघ्नाच्चरसंस्कारात्स्वोदयेनांशकादिकम् ।\\
 स्वर्णं क्षेपवशात्कार्यं ग्रहे षड्भयुतेऽन्यथा ॥ १ ॥}

\vspace{2mm}
\justifying
 अथ चन्द्रादीनामुदयास्तमयकालपरिज्ञानार्थं प्रथमं दृक्कर्माऽऽह--
तिथिघ्नादि-
ति । चरतानांशेत्यादिना वक्ष्यमाणेन विधिना चरसंस्कारमानीय
तत्पञ्चदशभिर्नि-
इत्योभयत्र विन्यस्यैकं ग्रहस्थितराश्युदयविनाडीभिर्विभजेत् । अन्यत्तु
ग्रहस्थित-
राशेः सप्तमराश्युदयविनाडिकाभिर्विभजेत् । तत्र लब्धं द्वयं भागादिकं
क्रमेण
ग्रहस्फुटे षड्राशियुते ग्रहस्फुटे व संस्कुर्यात् । विक्षेपस्य ऋणत्वे
सति ग्रहस्थित-
राश्युदयाप्तं ग्रह ऋणं कुर्यात् । ग्रहस्थितराशेः सप्तमराश्युदयाप्तं
षड्राशियुते
ग्रहे धनं कुर्यात् । विक्षेपस्य धनत्वे तु ग्रहे धनं कुर्यात् ।
षड्राशियुते ग्रह ऋणं
कुर्यात् ॥ १ ॥

\vspace{2mm}
\centering
\hspace{-1.9cm}
\textbf{
 ग्रहस्योत्क्रमकोटिघ्नात्क्षेपाब्ध्यंशात्स्वलग्नभात् ।\\
 क्षेपकोट्योः समान्य(न)त्व स्वर्णं भागाद्यपि क्रमात् ॥ २ ॥}

\vspace{2mm}
\justifying
 अथ द्वितीयं दृक्कर्माऽऽह-- ग्रहस्येति । सायनस्य ग्रहस्य
कोट्युत्क्रमज्यामेके-
त्रिचतुर्घ्नराश्यैक्यमित्युत्क्रमविधिनाऽऽनीय तां भागात्मिकां
स्वविक्षेपचतुर्भागलिप्ता-
भिर्निहत्य स्वलग्नेन ग्रहस्थितराश्युदयेन तत्सप्तमराश्युदयेन च
पृथग्विभज्य लब्धं
भागादिकं क्रमेण प्रथमदृक्कर्मयुते ग्रहस्फुटे
प्रथमदृक्कर्मयुतषड्राशियुते ग्रहे च
संस्कुर्यात् । क्षेपकोट्योरुभयोर्युगपट्टणत्वे वा धनत्वे वा सति तत्फलं
ग्रहे षड्रा-
शियुते ग्रहे च धनं कुर्यात् । क्षेपकोटयोरेकस्य धनत्वेऽन्यस्य ऋणत्वे
सति तत्फलं
ग्रहे षड्राशियुते ग्रहे चर्णे कुर्यात् । एवं दृक्कर्मद्वययुतो ग्रहो
ग्रहस्योदयलग्नं
भवति । दृक्कर्मद्वययुतः षड्राशियुतो ग्रहो ग्रहस्यास्तलग्नं भवति ।
एवं दृक्कर्मद्वयं
ग्रहस्फुटे षड्राशियुते ग्रहे वा यथासंभवं कृत्वा मौढ्यारम्भादि
विद्यात् ॥ २ ॥
\footnotetext{
 १.क. °स्यैकग्र° । २ क.°भाक् । ३ क. °तीयतृतीयं । ४ क. कोट्या उ°। ५\\
क. °कत्र च° ।}

\newpage
\thispagestyle{fancy}
\fancyhf{}
\chead{पारमेश्वरव्याख्यासंवलितं -}
\rhead{[दृक्कर्माधिकारः]}
\lhead{२८}
\centering
\hspace{-1.5cm}
\textbf{
 सूर्याष्टिविश्वरुद्राष्टतिथ्यशघ्नैः खखाग्निभिः ।\\
 प्राग्भोदयाप्तैर्युक्तोनः सूर्योऽस्तार्कः शशाङ्कतः ॥ ३ ॥}

\vspace{2mm}
\justifying
 इदानीमुदयास्तमयपरिज्ञानायास्तार्कानयनमाह-- सूर्येति । सूर्यां
द्वादश ।
अष्टिः षोडश । विश्वे त्रयोदश । रुद्रा एकादश । अष्ट प्रसिद्धाः ।
तिथयः
पञ्चदश । एते चन्द्रादीनां भागाः । एतैः खखाग्नीञ्शतत्रयं निहत्य
प्राग्भोदयेन
ग्रहस्थितराश्युदयविनाडीभिर्वा तत्सप्तमराशिविनाडीभिर्वा
यथाप्राप्ताभिर्विभज्य
लब्ध नंशकादींस्तत्कालस्फुटार्कं द्विधा न्यस्यैकस्मिन्प्रक्षिपेत्
। अन्यस्माच्छोधयेत् । 
तावर्कावस्तार्कसंज्ञौ भवतः । [ प्राग्भोदय इति
वचनमुभयत्रापि तत्कालल-
ग्नस्य हारकत्वसंभवात् ]। हारकविभागस्तु यदा प्राक्कपाले दृश्यमानो
द्रक्ष्यमाणो
वा ग्रहो भवति तदा दृक्कर्मद्वययुतग्रहस्थितराशिविनाडिका हारः । यदा
पुन:
पश्चात्कपाले दृश्यमानो द्रक्ष्यमाणो वा ग्रहो भवति तदा ग्रहस्थितराशेः
सप्तम-
राश्युदयविनाडिका हार इति षड्राशियुते ग्रहे सत्यस्तार्कोऽपि षड्राशियुतः
स्यात् । एतदुक्तं भवति-- यदा प्राक्कपाले दृश्यमानो ग्रहो मूढतां
गच्छति प्राक्कपाले
द्रक्ष्यमाण उदयं गच्छति वा तदा ग्रहस्थितराशौ दृक्कर्मद्वयं
कृत्वा
तस्य ग्रहस्य भागैः
सूर्याष्टीत्यादिपठितैरुदयविनाडीभिश्चास्तार्कद्वयमानयेत् । यदा
तयोरस्ताकयारन्तराले दृक्कर्मद्वययुतो ग्रहो विचरति तदाऽसौ ग्रहोऽस्तं
गतो
भवति । एवं प्राक्कपाले । यदा पुनः पश्चात्कपाले दृश्यमानोऽस्तं
गच्छति पश्चात्कपाले
द्रक्ष्यमाण उदयं गच्छति वा तदा षड्राशियुते ग्रहे
दृक्कर्मद्वयं कृत्वाऽस्तार्कद्वयं
चास्तमयविनाडिकाभिः स्वभागैश्चाऽऽनीय तयोरस्तार्कयो
राशिषट्कं
प्रक्षिपेत् । यदा षड्राशियुतयोरस्तार्कयोरन्तराले षड्राशियुतो
दृक्कर्मद्वयसंस्कृतो
ग्रहो भवति तदाऽसौ ग्रहोऽस्तं गच्छति । द्रष्टुमशक्यो
भवतीत्यर्थः ।
अस्तार्कयोर्बहिर्गतिश्चेदुदयगतो भवति । एवं
तत्कालग्रहतद्विक्षेपतदस्तार्कैरुदयास्तमयकालो
विज्ञेयः । एकेनैवास्तार्केणैव हि प्रयोजनं स्यात् ॥ ३ ॥

\vspace{2mm}
\centering
\hspace{-1cm}
\textbf{
 विक्षेपो भिन्नतुल्याशावलनघ्नः खखाङ्ककैः ।\\
 हृतोंऽशास्तैर्युतोनः सन्ग्रहोऽस्तार्कान्तरेऽस्तगः ॥}
\footnotetext{
 १ क. इतरस्मा° । २ क. °राशिंविं° । ३ क. °राशिर्वि° । ४. क. °भिश्चार्क°
। ५\\
क. °हो विचरति । ६ क. °बतीति ।}

\newpage
\thispagestyle{fancy}
\fancyhf{}
\chead{लघुमानसम् ।}
\rhead{२९}
\lhead{[संकीर्णाधिकारः]}
\justifying
 अथ प्रकारान्तरेण दृकर्माऽऽह-- विक्षेप इति । ग्रहस्योदयेऽस्तमये
वा पार-
मार्थिकं वलनमानीय तद्वलनाङ्गुलं विक्षेपलिप्ताभिर्निहत्य खखाङ्कैः.
(९००)
विभज्य लब्धानंशान्केवल एव ग्रहस्फुटे षड्राशियुते ग्रहस्फुटे वा
संस्कुर्यात् ।
विक्षेपवलनयोर्द्विदिकत्वे धनं कुर्यात् । एकदिक्कत्व ऋणं कुर्यात् ।
एवं कृतो
ग्रहो विकृतदर्शनसंस्कारो भवति । अस्तार्कान्तरेऽस्तग इति वलनसंस्कृतो
ग्रहोऽ-
स्तार्कान्तरगतश्चेदस्तं गतो भवतीत्यर्थः । केचिदिमं श्लोकं विना पठन्ति
। केचित्तु
दृक्कर्मद्वययुते वलनसंस्कारमिच्छन्ति । तदसत् । यतो वलनकर्मणाऽपि
दृक्कर्मद्वयमेव
सिध्यति । अतो वलनकर्म वा दृक्कर्मद्वयं वैकमेव कुर्यात् ।
प्राक्कपालगतो
ग्रहश्चेद्ग्रहस्योसयकाल एतत्कर्म कुर्यात् । पश्चात्कपालगतश्चेद्
ग्रहस्यास्तमुषकाल
एतत्कर्म कुर्यात् । [ अथ संकीर्णाधिकारः ]।

\vspace{2mm}
\centering
\hspace{-0.5cm}
\textbf{
 अगस्त्यास्तोदयाोर्कांशाः सप्तशैलाः स्वराङ्ककाः ।\\
 अष्टघ्नविषुवच्छायाहीना युक्ताः स्वदेशजाः ॥ ४ ॥}

\vspace{2mm}
\justifying
 इदानीमगस्त्यमुनेरुदयास्तमयकालपरिज्ञानार्थमाह-- अगस्त्येति ।
अष्टघ्नविषु-
वच्छायाङ्गुलैर्हीनाः सप्तशैलाः सप्तसप्ततिभागा यावन्तस्तावन्तो भागा भागी-
कृतस्य स्फुटार्कस्य यदा भवन्ति तदाऽगस्त्यस्यास्तमयकालो भवति ।
पुनरष्टघ्न-
विषुवच्छायाङ्गुलयुताः स्वराङ्ककाः सप्तनवतिभागा यावन्तस्तावन्तो भागा
भागीकृतस्य स्फुटार्कस्य यदा भवन्ति तदाऽगस्त्यस्योदयकालो भवति ।
पुण्यत्वा-
देव कालः प्रदर्शितः ॥ ४ ॥

\vspace{2mm}
\centering
\hspace{-1.8cm}
\textbf{
 चरतानांशषड्वर्गविश्लेषेणाक्षमाहतात् ।\\
 स्वविक्षेपादवाप्तेन स्वचरं संस्कृतं स्फुटम् ॥ ५ ॥}

\vspace{2mm}
\justifying
 अथ चरसंस्कारमाह-- चरतानेति । सायनस्य चन्द्रस्यार्कवच्चरविनाडीरानीय
पृथग्विन्यस्य तांस्तानैरेकोनपञ्चाशता विभज्य यल्लब्धं
तच्चरतानांशो
भवति । षड्वर्गः षट्त्रिंशत् । चरतानांशषड्वर्गयोरन्तरेण
विषुवच्छायाङ्गुलगुणितां
स्वविक्षेपलिप्तिकां विभजेत् । तत्र लब्धो
विनाडिकात्मकश्चरसंस्कारो
भवति । तेन संस्कृतं स्वचरं स्फुटं भवति । एतदुक्तं भवति--
विक्षेपरचरयोरुभयोरपि
धनत्वे वा ऋणत्वे वा सति चरसंस्कारविनाड्योर्योगः स्फुटचरं भवति ।
\footnotetext{
 १ क. °योर्भिन्नदिक्त्वे । २ क. °हीनयु° । ३ क, °ना अष्टसप्रतिभा°। ४
क.\\
°तिनवतिभागा° ।}

\newpage
\thispagestyle{fancy}
\fancyhf{}
\chead{पारमेश्वरव्याख्यासंवलितं -}
\rhead{[संकीर्णाधिकारः]}
\lhead{३०}
\justifying
\noindent
विक्षेपचरयोरेकस्य धनत्वेऽन्यस्य ऋणत्वे सति चरसंस्कारचरविनाड्योरन्तरं
स्फुटचरं
भवति । एवं चन्द्रादीनां स्फुटचरमानयेत् । अनेन चरसंस्कारेण
पूर्वोक्तं
दृक्कर्म च क्रियते ॥ ५ ॥

\vspace{2mm}
\centering
\hspace{-0.7cm}
\textbf{
 अन्तरेऽर्केन्दुदिनयोर्विनाड्यः पलभाल्पिकाः ।\\
 यावत्तावद्व्यतीपातो वैधृतस्तु दिवानिशोः ॥ ६ ॥}
 
\vspace{2mm}
\justifying
 अथ व्यतीपातकालपरिज्ञानार्थमाह-- अन्तर इति ।
अर्केन्द्वोर्दिनप्रमाणयो-
रन्तरविनाड्यो यावत्कालं पलभाल्पिकाः पलभाङ्गुलेभ्योऽल्पतरा भवन्ति ताव-
त्कालं व्यतीपातो नाम दोषः स्यात् । वैधृतस्तु दिवानिशो:--
अर्केन्द्वोरेकस्य
दिनप्रमाणमितरस्य रात्रिप्रमाणं चाऽऽनीय
तयोर्दिननिशाप्रमाणयोरन्तरविनाड्यो
यावत्कालं पलभाङ्गुलाल्पिका भवन्ति तावत्कालं वैधृतो नाम दोषः स्यात् ।
अत्र पलभाशब्देनाष्टांशहीनविषुवच्छायाङ्गुलमुच्यते ।
अतस्तस्मादल्पिकात्वं वेद्यम् ।
किंचार्केन्दुदिनप्रमाणयोरेकस्य क्रमाद्वृद्धिमत्त्वेऽन्यस्य क्रमाद्
ह्रासवत्त्वे च सत्येव
व्यतीपातः स्यात् । अन्यथा न व्यतीपातो दोषः । तथा
दिननिशाप्रमाणयोरध्ये-
कस्य क्रमाद् वृद्धिमत्त्वेऽन्यस्य क्रमाद् ह्रासवत्त्वे च सत्येव
वैधृतदोषः । अन्यथा
वैधृताभाव इति च वेदितव्यम् ।

\vspace{2mm}
\centering
 " भिन्नायने तुल्यगोले लाटो व्यत्यासतोऽपरः '' ।

\vspace{2mm}
\justifying
 इत्यनेन साम्यं क्रमाद् वृद्धिह्रासकल्पनयैव भवति । ' दन्तादूने
दोषस्तयो-
भवेद् ' इत्यनेन साम्यं चाष्टांशहीनपलमाकल्पनयैवं भवति ] ॥ ६ ॥

\vspace{2mm}
\centering
\hspace{-0.9cm}
\textbf{
 विहितोभयदृक्कर्म तत्कालेन्दुविलग्नतः ।\\
 शशाङ्कद्युगतं तस्मात्तद्दिनादकंवत्प्रभा ॥ ७ ॥}

\vspace{2mm}
\justifying
 इदानीं चन्द्रच्छायानयनमाह-विहितेति । दृकालचन्द्र औदयिकं
दृक्कर्मद्वयं
कुर्यात् । स चन्द्रो विहितोभयदिक्कर्मा तत्कालेन्दुः पुनस्तत्काले
प्राग्विलग्नं चाऽऽन-
येत् । तद् विलग्नं तस्मिंश्चन्द्रे [ वि ] लग्ने चायनं प्रक्षिप्य
सायनयोस्तयोर्ल-ग्नेन्द्वोरन्तरालघटिकास्त्रिप्रश्नाध्यायोक्तविधिनाऽऽनयेत् । ता घटिकाः
शशाङ्कद्युगतं
भवति । तस्मात्तद्दिनादर्कवत्प्रभा तेन शशाङ्कद्युगतेन तस्य
शशाङ्कस्य
स्फुटदिनप्रमाणेन चार्कवत्प्रभा छाया भवति । एतदुक्तं भवति--
चरतानांशेत्यादिना
चन्द्रस्फुटचरमानीय तेन चरेण दिनदिनार्धप्रमाणावानीय तथा शशाङ्कद्युगतं
\footnotetext{
 १ क. °यनैकदिक्कत्वे ला° । २ क. °चायनचलनं प्र° ।}

\newpage
\thispagestyle{fancy}
\fancyhf{}
\chead{लघुमानसम् ।}
\rhead{३१}
\lhead{[संकीर्णाधिकारः]}
\justifying
\noindent
च लग्नेन्दुभ्यामानीय तैः सर्वैरर्कवत् '' पञ्चघ्नेष्टचरार्धेन ''
इत्यादिविधिना चन्द्र-
च्छायामानयेत् । विपरीतच्छायायां तु च्छाययाऽऽनीता [ नत ] घटिका
दिना-
र्धाद् विशोध्य शिष्टं शशङ्कद्युगतं भवति । तात्कालिकं
कृतदृक्कर्मेन्दुं सूर्यं प्रकल्प्य 
शशाङ्कद्युगतं दिनगतं प्रकल्प्य पूर्ववदुदयलग्नमानयेत् । तत्काललग्नं
भवति । 
पुनस्तत्काललग्नषड्राशियुतार्कयोरन्तरालघटिका आनयेत् । ता रात्रौ गता
घटिका 
भवन्तीति । अत्र प्रथमविशेषविधिना चन्द्रोदये तत्कालेन्दुमानीय
तमिन्दुमूहति-
द्धेन स्थूलेनेन्दुद्युगतेन तद्भोगकालयुतेन तत्कालीकृत्य 
तस्मिन्नौदयिकं
दृक्कर्मद्वयं 
कृत्वा पृथक् संस्थाप्य तत्कालच्छायया नतघटिकामानीय स्फुटतरं शशाङ्कद्युगतं
चाऽऽनयेत् । पुनः पूर्वसिद्धस्थूलद्युगततत्स्फुटद्युगतयोरन्तरालघटिका
आनयेत् । 
तथा निहतां चन्द्रभुक्तिं पृथक्स्थितचन्द्रे युक्तितः स्वमृणं वा कुर्यात्
। पुनः 
स्फुटतरं तत्कालेन्दुमानीय तद्वशाल्लग्नमानयेत् ॥ ७ ॥

\vspace{2mm}
\centering
\textbf{
 द्व्यूग्नाः पक्षादितिथ्यर्धाः सस्वाङ्गांशाः सितासिते । \\
 विक्षेपाद् व्योमधृत्यंशसंस्कृतं वलनं स्फुटम् ॥ ८ ॥ }

\vspace{2mm}
\justifying
 इदानीं शृङ्गोन्नतिज्ञानार्थं चन्द्रबिम्बस्य सितकृष्णाङ्गुलानयनं
वलनानयनं 
चाऽऽह-- द्व्यूना इति । शुक्लपक्षे चन्द्रादर्कं विशोध्य
शिष्टाल्लिप्तीकृताद् व्योमरसाग्निभिर्लब्धानि
पक्षादितिथ्यर्धानि भवन्ति । तानि द्वाभ्यामूनानि
कृत्वा तेषु 
स्वानांशं स्वसप्तमांशं प्रक्षिपेत् । तानि रदमण्डले शुक्लाङ्गुलानि
भवन्ति । कृष्णपक्षे
तु षड्राशिहीनाच्चन्द्रादर्कं विशोध्य शिष्टादुक्तवल्लब्धमसितं
कृष्णाङ्गुलमानं 
भवति । एवं शुल्लकृष्णाङ्गुलमानयेत् । पुनः पूर्ववत्पारमार्थिकं
वलनमानीय 
तस्मिन् विक्षेपाद् व्योमधृत्यंशमशीत्यधिकशतांशं संस्कुर्यात् ।
तत्प्रकारस्तु शुक्लपक्षे
विक्षेपवलनयोस्तुल्यदिकत्वे व्योमधृत्यंशवलनयोर्योगं कुर्यात् ।
भिन्नदिक्त्वे 
व्योमधृत्यंशवलनयोरन्तरं कुर्यात् । कृष्णपक्षे तुल्यदिक्त्वेऽन्तरं
कुर्यात् । भिन्न-
दिकत्वे योगं कुर्यात् । कृष्णपक्षे क्षेपस्य व्यत्यासेन दिगित्यर्थः
। एवं संस्कृतं 
शृङ्गोन्नत्यां स्फुटवलनं भवति । किंत्वर्केन्द्वोर्मध्यभागस्थराशिभागे
ग्रहं परिकल्प्य
तस्य वलनानयनं कार्यम् । द्विसहितपक्षादितिथ्यर्धः ।
गुणितैरेकादशभिर्विक्षेपादाप्तं
व्योमधृत्यंशमिति प्रकल्प्यम् । कृष्णपक्षे
तिथ्यर्धमेष्यं स्यादिति । 
तथा च कश्चिदाह-- 

\vspace{2mm}
\centering
\textbf{
 अर्केन्द्वोर्मध्यभागस्थं कल्पयित्वा ग्रहं ततः । \\
 वलनं साधयेत्स्पष्टं शृङ्गोन्नत्यां यथाविधि ॥}
\footnotetext{ २ क. °ति । पश्चात्कपाले तु नयनघटिका दिनार्धे प्रक्षिप्य दृष्टं
शशाङ्कद्यगतं भवति ।}

\newpage
\thispagestyle{fancy}
\fancyhf{}
\chead{पारमेश्वरव्याख्यासंवलितं -}
\rhead{[संकीर्णाधिकारः]}
\lhead{३२}
 \centering
 \textbf{
 \hspace{-1.3cm}
 तस्मिन् वियुक्तपक्षादितिथ्यर्धगुणितैर्भवैः ।\\
 \hspace{-0.9cm}
 लब्धं विक्षेपतः स्वर्णं कुर्यात्तद्वलनं स्फुटम् ॥\\
 \hspace{-1.5cm}
 वलनक्षेपयोस्तुल्यदिक्त्वे स्वमृणमन्यथा ।\\
 \hspace{-1.2cm}
 कृष्णपक्षे तु तिथ्यर्धमेष्यं स्वर्णत्वमन्यथा ॥\\
 \hspace{-1.5cm}
 इन्दोरेवेह विक्षेपो नार्केन्द्वोर्मध्यगस्य तु ।\\
 \hspace{0.7cm}
 विधीयते हि वलनं प्राग्भागे रदमण्डले ॥ इति ॥ ८ ॥\\
 \hspace{-0.5cm}
 बिम्बापरदिशो भागात्प्राग्वृद्धिः शुक्लकृष्णयोः । \\
 शुक्लान्ताद्बिम्बमध्यस्पृक्छेदाग्रच्छेदनं छिदा ॥ ९ ॥}\\
 
 \vspace{2mm}
 \justifying
 इदानीं शृङ्गोन्नविपरिलेखकरणमाह-- बिम्बापरेति ।
बिम्बस्यापरभागात्प-
श्चिमभागात्प्राग्वृद्धिः शुक्लकृष्णयोर्भवति । एतदुक्तं भवति--
षोडशाङ्गुलसूत्रेण
रदमण्डलमालिख्य पूर्वापरगां दक्षिणोत्तरगां च रेखां कृत्वा
वृत्तपूर्वभागाद्वलनं
नीत्वा वलनानुसारिण्यौ पूर्वापरदक्षिणोत्तररेखे कुर्यात् ।
पुनर्वलनसाधितपूर्वापर-
सूत्रपश्चिमाग्रात्पूर्वतः सितमानं कृष्णमानं वा तत्सूत्रे नीत्वा तत्र
बिन्दुं कृत्वा
पुनर्वलनसाधितदक्षिणोत्तरसूत्राग्रयोश्च बिन्दूं कृत्वा त्रिशर्कराविधानेन
तद्बिन्दुत्रयस्पृग्वृत्तैकदेशमालिखेत् । 
तत्र सितकृष्णबिन्दुतोऽपरभागे
वृत्तद्वयान्तरं
सितकृष्णयोः संस्थानं भवति । तदुक्तम्-- " शुक्लान्ताद्
बिम्बमध्यस्य च्छेदाग्न
च्छेदनं छिदा " इति । बिम्बमध्यस्य च्छेदाग्रं दक्षिणोत्तरसूत्राग्नद्वयं
शुल्कबिन्दुतदग्रणां
छेदनं छिदा छेदयोग्येन सूत्रकाष्ठादिना कुर्यादित्यर्थः ।
शुल्कग्रहणं
कृष्णस्याप्युपलक्षणम् ॥ ९ ॥

\centering
\hspace{-0.5cm}
\textbf{
 मानसाख्यं ग्रहज्ञानं श्लोकषष्ट्या मया कृतम् ॥\\
 भवन्त्योऽयशोभागाः प्रतिकञ्चुककारिणः ॥ १० ॥}

\justifying
 इदानीमुपसंहरणार्थं श्लोकमाह-- मानसाख्यमिति । मानसं नाम ग्रहज्ञानं
ग्रह-
गतिज्ञानसाधनं शास्त्रं श्लोकषष्ट्या श्लोकानां पष्ट्या मया कृतं रचितम्
। अतोऽ-
स्मिन्मानसे ये प्रतिकञ्चुककारिणस्तेऽयशोभागाः केवलमयशोभाजना एव
भवन्ति । एवदुक्तं भवति-- यद्यपि भास्करादिभिः
प्रदर्शितान्मन्दोच्चादेरत्रोकस्य-
मन्दोच्चादेः किञ्चिद्भेदः स्यात्तथाऽपि शास्त्रान्तरानुसारित्वाद्
दृष्ठिसाम्याच्चैतत्सर्वैः
पठितव्यमित्यर्थः ॥ १० ॥

\centering
 व्याख्यानं मानसस्यैतत्सुचिरं तिष्ठतु क्षितौ ।\\
 \hspace{-0.7cm}
 हरिपादाब्जयुगले सततं मानसं च मे ॥\\
\textbf{
 इति पारमेश्वरे मानसव्याख्याने संकीर्णाधिकार-\\
 श्चतुर्थोऽध्यायश्च समाप्तः ॥\\}

\rule{0.2\linewidth}{1.0pt}
\footnotetext{
 १ क. °भाजः प्र° ।}

 \end{document}