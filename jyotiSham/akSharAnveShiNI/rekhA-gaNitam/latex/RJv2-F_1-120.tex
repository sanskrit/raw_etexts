%Default Sanskrit,  No auto transition, \en font for english, so italics displayed.
\documentclass[11pt, openany]{book}
\usepackage[text={4.65in,7.45in}, centering, includefoot]{geometry}
\usepackage[table, x11names]{xcolor}
\usepackage{fontspec,realscripts}
\usepackage{wrapfig}
\usepackage{polyglossia}
\setdefaultlanguage{sanskrit}
\setotherlanguage{english}
\setmainfont[Scale=0.9]{Shobhika}
\newfontfamily\s[Script=Devanagari, Scale=0.9]{Shobhika}
\newfontfamily\regular{Linux Libertine O}
\newfontfamily\en[Language=English, Script=Latin]{Linux Libertine O}
\newfontfamily\ab[Script=Devanagari, Color=purple]{Shobhika-Bold}
\newfontfamily\qt[Script=Devanagari, Scale=0.9, Color=violet]{Shobhika-Regular}
\newcommand{\devanagarinumeral}[1]{%
	\devanagaridigits{\number \csname c@#1\endcsname}} % for devanagari page numbers
%\usepackage[Devanagari, Latin]{ucharclasses}
%\setTransitionTo{Devanagari}{\s}
%\setTransitionFrom{Devanagari}{\regular}
\XeTeXgenerateactualtext=1 % for searchable pdf
\usepackage{enumerate}
\pagestyle{plain}
\usepackage{fancyhdr}
\pagestyle{fancy}
\renewcommand{\headrulewidth}{0pt}
\usepackage{afterpage}
\usepackage{multirow}
\usepackage{multicol}
\usepackage{vwcol}
\usepackage{microtype}
 \usepackage{amsmath}
\usepackage{graphicx}
\usepackage{longtable}
\usepackage{footnote}
\usepackage{perpage}
\MakePerPage{footnote}
\usepackage[para]{footmisc}
\makeatletter
\def\blfootnote{\gdef\@thefnmark{}\@footnotetext}
\makeatother
%\usepackage{dblfnote}
\usepackage{xspace}
\usepackage{array}
\usepackage{emptypage}
\usepackage{hyperref}% Package for hyperlinks
\hypersetup{colorlinks,
citecolor=black,
filecolor=black,
linkcolor=blue,
urlcolor=black}
\begin{document}
\thispagestyle{empty}
\begin{center}
{\en \textbf{\Huge THE REKHĀGAṆITA }\\
{\small OR} \\
{\large GEOMETRY IN SANSKRIT }\\
{\large COMPOSED BY SAMRĀḌ JAGANNĀTHA }\\
{\LARGE Volume II. Books VII\textendash XV. }\\
 UNDERTAKEN FOR PUBLICATION     
 \vspace{1mm}
 
{\small BY} 
\vspace{1mm}

THE LATE 
\vspace{1mm}

\textbf{\LARGE HARILĀL HARSHADARĀI DHRUVA }\\
{\small B.A., L.L.B., D.L.A. ( SWEDEN ), M.R.A.S. \\
( LONDON AND BOMBAY ), \\
CITY JOINT JUDGE AND SESSIONS JUDGE, BAROḌĀ,} \\
\textbf{Edited and carried through the press, with Introduction, \\
and brief notes in English} 
\vspace{1mm}

{\small BY} 
\vspace{1mm}

\textbf{\large KAMALĀŚAṄKARA PRĀṆAŚAṄKARA TRIVEDĪ, B.A.,}
\vspace{1mm}

{\small FELLOW OF THE UNIVERSITY OF BOMBAY, HEAD MASTER, NADIĀD\\
HIGH SCHOOL ( FORMERLY PROFESSOR OF ORIENTAL \\
LANGUAGES, SĀMALADĀS COLLEGE, BHĀVA- \\
NAGAR, AND ACTING PROFESSOR OF \\
ORIENTAL LANGUAGES, ELPHIN- \\
STONE AND DECCAN\\
COLLEGES ). }\\
\rule{1.5in}{0.3pt}
\vspace{1mm}

\textbf{\large 1st Edition}\textemdash 300 COPIES. \\
\rule{1.5in}{0.3pt}\\

\emph{\en ( Registered for copy-right under Act XXV. of 1867 )}\\ 
\rule{1in}{1pt}
\vspace{3mm}

\textbf{{\large Bombay.}}
\vspace{1mm}

{\large GOVERNMENT CENTRAL BOOK DEPŌT.} \\
\rule{0.5in}{0.3pt}\\
1902 \\
\rule{0.5in}{0.3pt}\\
{\large {\bf \en [ All rights reserved ].}} \\
\rule{0.5in}{0.3pt}\\
Price 9 Rupees.\\
{\large Bombay Sanskrit Series No. LXII.}}
\end{center}

\newpage
\thispagestyle{empty}

\vspace*{2in}
\begin{quote}

\centerline{\includegraphics[scale=0.8]{Images/rg-1.png}}

\end{quote}

\newpage
\thispagestyle{empty}
\begin{center}
श्रीः
\vspace{5mm}

\textbf{\Huge रेखागणितम् }
\vspace{7mm}

\textbf{\large सम्राड्जगन्नाथविरचितं }
\vspace{6mm}

( द्वितीयभागात्मकं सप्तमाध्यायमारभ्य पञ्चदशाध्यायपर्यन्तम् ) 
\vspace{6mm}

\textbf{स्वर्गवासिमहाशयध्रुवोपपदेन हर्षदरायात्मजेन हरिलालेन }
\vspace{2mm}

\textbf{संस्करणार्थमङ्गीकृतं }
\vspace{6mm}

त्रिवेद्युपपदधारिणा 
\vspace{6mm}

\textbf{\LARGE प्राणशंकरसूनुना कमलाशंकरेण संशोधितं }
\vspace{6mm}

\textbf{स्वनिर्मिताङ्ग्लभाषाटिप्पण्या च समुपेतम्~।}
\vspace{6mm}

तच्च
\vspace{5mm}

\textbf{\large मुम्बापुरीस्थराजकीयग्रन्थशालाधिकारिणा }
\vspace{6mm}

``निर्णयसागरा''ख्यमुद्रणयन्त्रालये मुद्रयित्वा 
\vspace{6mm}

{\small शाके १८२४ वत्सरे १९०२ ख्रिस्ताब्दे प्राकाश्यं नीतम्~। }
\vspace{2mm}

\rule{0.6in}{0.5pt}\\
\vspace{5mm}

\textbf{प्रथमा आवृत्तिः }
\vspace{1mm}

\rule{0.6in}{0.3pt}\\
\vspace{3mm}

\textbf{मूल्यं ९ रूप्यकाः~। }
\end{center}

\newpage
\thispagestyle{empty}

\vspace*{3.4in}
\begin{quote}

\centerline{
\textbf{इदं पुस्तकं मोहमय्यां निर्णयसागराख्ये मुद्रणालये मुद्रितम्~। }}
\end{quote}

\newpage
\thispagestyle{empty}
\begin{center}
\textbf{INTRODUCTION}
\vspace{1mm}

\rule{1in}{0.5pt}
\end{center}
\vspace{3mm}

 {\en After \,the \,publication \,of \,the \,first \,volume \,and \,a \,major \,portion \,of \,the second volume I received a Ms. of the work in charge  of the Ānandāśrama Library of Poona through my friend,  Prof.~Śrīdhara R.~Bhāṇḍārakar, M.A. It is found to coincide mostly with D. Its \textit{variae lectiones} are given in Appendix II. The various readings of V in Books VII, VIII, and IX are 
given in Appendix I, and those of the remaining books in footnotes. \\

 I \,had \,a \,mind \,to \,give \,a \,rendering \,of this \,volume \,into \,English \,in \,my English notes for the benefit of those readers who do not know Sanskrit. But as the idea did not meet with the approval of one of the Superintendents of the Series, who was consulted on the point, it was given up. The notes are consequently very brief, containing mostly as they do, English equivalents of technical Sanskrit terms. \\

\begin{tabular}{ccc}
Rāipur, & \multirow{3}{*}{$\Bigg \}$}&\\
AHMEDĀBĀD, & &  \hspace{1.5in} \textbf{K. P. TRIVEDI }\\
28th March 1902.& & \\
\end{tabular}}

 

\newpage
\setcounter{page}{1}
\renewcommand{\thepage}{\roman{page}}

\begin{center}
 \textbf{\Large अनुक्रमणिका}
\vspace{1mm}

\rule{1in}{0.5pt}
\end{center}
\begin{center}
\begin{tabular}{p{1.5in} l | p{1.5in} l}
& ~पृष्ठ.& & ~पृष्ठ.\\
\hyperref[ch7]{\textbf{सप्तमोऽध्यायः}} & \textbf{१---२७} & ~~षड्विंशतितमक्षेत्रम् & १८---९\\
~~परिभाषा & १---२ & ~~सप्तविंशतितमक्षेत्रम् & १९\\
~~प्रथमक्षेत्रम् & २---३ & ~~अष्टाविंशतितमक्षेत्रम् & २०---१\\
~~द्वितीयक्षेत्रम् & ३---४ & ~~~~प्रकारान्तरम् & ~,,\\
~~तृतीयक्षेत्रम् & ४---५ & ~~एकोनत्रिंशत्तमक्षेत्रम् & २१\\
~~चतुर्थक्षेत्रम् & ५ & ~~त्रिंशत्तमक्षेत्रम् & ~,,\\
~~पञ्चमक्षेत्रम् & ५ & ~~एकत्रिंशत्तमक्षेत्रम् & २१---२\\
~~षष्ठक्षेत्रम् & ६ & ~~द्वात्रिंशत्तमक्षेत्रम् & २२\\
~~सप्तमक्षेत्रम् & ६---७ & ~~त्रयस्त्रिंशत्तमक्षेत्रम् & २२---३\\
~~~~प्रकारान्तरम् & ७ & ~~चतुस्त्रिंशत्तमक्षेत्रम् & २३---४\\
~~अष्टमक्षेत्रम् & ७---८ & ~~पञ्चत्रिंशत्तमक्षेत्रम् & २४---५\\
~~नवमक्षेत्रम् & ८ & ~~षट्त्रिंशत्तमक्षेत्रम् & २५---६\\
~~दशमक्षेत्रम् & ८---९ & ~~सप्तत्रिंशत्तमक्षेत्रम् & २६\\
~~एकादशक्षेत्रम् & ९ & ~~अष्टत्रिंशत्तमक्षेत्रम् & २६---७\\
~~द्वादशक्षेत्रम् & १० & ~~एकोनचत्वारिंशत्तमक्षेत्रम् & २७\\
~~त्रयोदशक्षेत्रम् & १०---१ & \hyperref[ch8]{\textbf{अष्टमोऽध्यायः}} & \textbf{२८---४३}\\
~~~~प्रकारान्तरम् & ११ & ~~प्रथमक्षेत्रम् & २८\\
~~चतुर्दशक्षेत्रम् & ११---२ & ~~द्वितीयक्षेत्रम् & २८---९\\
~~पञ्चदशक्षेत्रम् & १२ & ~~तृतीयक्षेत्रम् & २९\\
~~षोडशक्षेत्रम् & १२---३ & ~~चतुर्थक्षेत्रम् & ३०---१\\
~~सप्तदशक्षेत्रम् & १३ & ~~पञ्चमक्षेत्रम् & ३१\\
~~अष्टादशक्षेत्रम् & १३---४ & ~~षष्ठक्षेत्रम् & ३१---२\\
~~एकोनविंशतितमक्षेत्रम् & १४---५ & ~~सप्तमक्षेत्रम् & ३२\\
~~विंशतितमक्षेत्रम् & १५---६ & ~~अष्टमक्षेत्रम् & ३२---३\\
~~एकविंशतितमक्षेत्रम् & १६ & ~~नवमक्षेत्रम् & ३३\\
~~द्वाविंशतितमक्षेत्रम् & १७ & ~~दशमक्षेत्रम् & ३३---४\\
~~त्रयोविंशतितमक्षेत्रम् & १७ & ~~एकादशक्षेत्रम् & ३४---५\\
~~चतुर्विंशतितमक्षेत्रम् & १७---८ & ~~द्वादशक्षेत्रम् & ३५\\
~~पञ्चविंशतितमक्षेत्रम् & १८ & ~~त्रयोदशक्षेत्रम् & ३५---६\\
\end{tabular}
\end{center}
\thispagestyle{empty}
\afterpage{\fancyhead[CE,CO]{\thepage}}
\cfoot{}
\newpage
%%%%%%%%%%%%%%%%%%%%%%%%%%%%%%%%%%%%%%%%%%%%%%%%%%%%%%%%%%%%%%
\renewcommand{\thepage}{\devanagarinumeral{page}}
\setcounter{page}{2}

\begin{center}
\begin{tabular}{p{1.5in} l | p{1.5in} l}
& ~पृष्ठ.& & ~पृष्ठ.\\
~~चतुर्दशक्षेत्रम् & ३६ & ~~अष्टादशक्षेत्रम् & ५२\\
~~पञ्चदशक्षेत्रम् & ३६---७ & ~~एकोनविंशतितमक्षेत्रम् & ~,,\\
~~षोडशक्षेत्रम् & ३७---८  & ~~विंशतितमक्षेत्रम् & ५३\\
~~सप्तदशक्षेत्रम् & ३८ & ~~एकविंशतितमक्षेत्रम् & ~,,\\
~~अष्टादशक्षेत्रम् & ३८---९ & ~~द्वाविंशतितमक्षेत्रम् & ५३---४\\
~~एकोनविंशतितमक्षेत्रम् & ३९---४० & ~~त्रयोविंशतितमक्षेत्रम् & ५४\\
~~विंशतितमक्षेत्रम् & ४० & ~~चतुर्विंशतितमक्षेत्रम् & ~,,\\
~~एकविंशतितमक्षेत्रम् & ४०---१ & ~~पञ्चविंशतितमक्षेत्रम् & ५४---५\\
~~द्वाविंशतितमक्षेत्रम् & ४१---२ & ~~षड्विंशतितमक्षेत्रम् & ५५\\
~~त्रयोविंशतितमक्षेत्रम् & ४२ & ~~सप्तविंशतितमक्षेत्रम् & ~,,\\
~~चतुर्विंशतितमक्षेत्रम् & ~,, & ~~अष्टाविंशतितमक्षेत्रम् & ~,,\\
~~पञ्चविंशतितमक्षेत्रम् & ~,, & ~~एकोनत्रिंशत्तमक्षेत्रम् & ~,,\\
~~षड्विंशतितमक्षेत्रम् & ४३ & ~~त्रिंशत्तमक्षेत्रम् & ५६\\
~~सप्तविंशतितमक्षेत्रम् & ~,, & ~~एकत्रिंशत्तमक्षेत्रम् & ~,,\\
\hyperref[ch9]{\textbf{नवमोऽध्यायः}} &\textbf{४४---६०} & ~~द्वात्रिंशत्तमक्षेत्रम् & ~,,\\
~~प्रथमक्षेत्रम् & ४४ & ~~त्रयस्त्रिंशत्तमक्षेत्रम् & ५६---७\\
~~द्वितीयक्षेत्रम् & ~,, & ~~चतुस्त्रिंशत्तमक्षेत्रम् & ५७\\
~~तृतीयक्षेत्रम् & ४४---५ & ~~पञ्चत्रिंशत्तमक्षेत्रम् & ~,,\\
~~चतुर्थक्षेत्रम् & ४५ & ~~षट्त्रिंशत्तमक्षेत्रम् & ५७---८\\
~~पञ्चमक्षेत्रम् & ४५---६ & ~~सप्तत्रिंशत्तमक्षेत्रम् & ५८\\
~~षष्ठक्षेत्रम् & ४६ & ~~अष्टात्रिंशत्तमक्षेत्रम् & ५९---६०\\
~~सप्तमक्षेत्रम् & ~,, & \hyperref[ch10]{\textbf{दशमोऽध्यायः}} & \textbf{६१---१२६}\\
~~अष्टमक्षेत्रम् & ४६---७ & ~~परिभाषा & ६१\\
~~नवमक्षेत्रम् & ४७ & ~~प्रथमक्षेत्रम् & ६१---२\\
~~दशमक्षेत्रम् & ४७---८ & ~~~~प्रकारान्तरम् & ६२---३\\
~~एकादशक्षेत्रम् & ४८ & ~~द्वितीयक्षेत्रम् & ६३---४\\
~~द्वादशक्षेत्रम् & ४८---९ & ~~तृतीयक्षेत्रम् & ६४---५\\
~~त्रयोदशक्षेत्रम् & ४९---५० & ~~चतुर्थक्षेत्रम् & ६५---६\\
~~चतुर्दशक्षेत्रम् & ५० & ~~पञ्चमक्षेत्रम् & ६६\\
~~पञ्चदशक्षेत्रम् & ५०---१ & ~~षष्ठक्षेत्रम् & ६७\\
~~षोडशक्षेत्रम् & ५१ & ~~सप्तमक्षेत्रम् & ६७---८\\
~~सप्तदशक्षेत्रम् & ५१---२ & ~~अष्टमक्षेत्रम् & ६९\\
\end{tabular}
\end{center}

\newpage
\begin{center}
\begin{tabular}{p{1.5in} l | p{1.5in} l}
& ~पृष्ठ.& & ~पृष्ठ.\\
~~नवमक्षेत्रम् & ६९---७० & ~~चत्वारिंशत्तमक्षेत्रम् & ८९\\
~~दशमक्षेत्रम् & ७० & ~~एकचत्वारिंशत्तमक्षेत्रम् & ~,,\\
~~एकादशक्षेत्रम् & ७१ & ~~द्विचत्वारिंशत्तमक्षेत्रम् & ~,,\\
~~द्वादशक्षेत्रम् & ७१---२ & ~~त्रिचत्वारिंशत्तमक्षेत्रम् & ९०\\
~~~~प्रकारान्तरम् & ७२ & ~~चतुश्चत्वारिंशत्तमक्षेत्रम् & ~,,\\
~~त्रयोदशक्षेत्रम् & ७३---४ & ~~~~परिभाषा & ९०---१\\
~~चतुर्दशक्षेत्रम् & ७४ & ~~पञ्चचत्वारिंशत्तमक्षेत्रम् & ९१\\
~~पञ्चदशक्षेत्रम् & ७४---५ & ~~षट्चत्वारिंशत्तमक्षेत्रम् & ९१---२\\
~~षोडशक्षेत्रम् & ७५ & ~~सप्तचत्वारिंशत्तमक्षेत्रम् & ९२\\
~~सप्तदशक्षेत्रम् & ७५---६ & ~~अष्टचत्वारिंशत्तमक्षेत्रम् & ९२---३\\
~~अष्टादशक्षेत्रम् & ७७ & ~~एकोनपञ्चाशत्तमक्षेत्रम् & ९३\\
~~एकोनविंशतितमक्षेत्रम् & ७७---८ & ~~पञ्चाशत्तमक्षेत्रम् & ~,,\\
~~विंशतितमक्षेत्रम् & ७८ & ~~एकपञ्चाशत्तमक्षेत्रम् & ९३---४\\
~~एकविंशतितमक्षेत्रम् & ७९ & ~~द्विपञ्चाशत्तमक्षेत्रम् & ९४---५\\
~~द्वाविंशतितमक्षेत्रम् & ७९---८० & ~~त्रिपञ्चाशत्तमक्षेत्रम् & ९५---६\\
~~त्रयोविंशतितमक्षेत्रम् & ८० & ~~चतुःपञ्चाशत्तमक्षेत्रम् & ९६\\
~~चतुर्विंशतितमक्षेत्रम् & ८१ & ~~पञ्चपञ्चाशत्तमक्षेत्रम् & ९६---७\\
~~पञ्चविंशतितमक्षेत्रम् & ८१---२ & ~~षट्पञ्चाशत्तमक्षेत्रम् & ९७\\
~~षड्विंशतितमक्षेत्रम् & ८२---३ & ~~सप्तपञ्चाशत्तमक्षेत्रम् & ९७---८\\
~~सप्तविंशतितमक्षेत्रम् & ८३ & ~~अष्टपञ्चाशत्तमक्षेत्रम् & ९८---९\\
~~अष्टाविंशतितमक्षेत्रम् & ~,, & ~~एकोनषष्टितमक्षेत्रम् & ९९\\
~~एकोनत्रिंशत्तमक्षेत्रम् & ८३---४ & ~~षष्टितमक्षेत्रम् & ९९---१००\\
~~त्रिंशत्तमक्षेत्रम् & ८४---५ & ~~एकषष्टितमक्षेत्रम् & १००\\
~~एकत्रिंशत्तमक्षेत्रम् & ८५ & ~~द्विषष्टितमक्षेत्रम् & १००---१\\
~~द्वात्रिंशत्तमक्षेत्रम् & ८५---६ & ~~त्रिषष्टितमक्षेत्रम् & १०१\\
~~त्रयस्त्रिंशत्तमक्षेत्रम् & ८६ & ~~चतुःषष्टितमक्षेत्रम् & १०२\\
~~चतुस्त्रिंशत्तमक्षेत्रम् & ८६---७ & ~~~~प्रकारान्तरम् & १०२---३\\
~~पञ्चत्रिंशत्तमक्षेत्रम् & ८७ & ~~पञ्चषष्टितमक्षेत्रम् & १०३\\
~~षट्त्रिंशत्तमक्षेत्रम् & ~,, & ~~~~प्रकारान्तरम् & ~,,\\
~~सप्तत्रिंशत्तमक्षेत्रम् & ८८ & ~~षट्षष्टितमक्षेत्रम् & १०४\\
~~अष्टत्रिंशत्तमक्षेत्रम् & ~,, & ~~सप्तषष्टितमक्षेत्रम् & ~,,\\
~~एकोनचत्वारिंशत्तमक्षेत्रम् & ~,, & ~~अष्टषष्टितमक्षेत्रम् & १०४---५\\
\end{tabular}
\end{center}
 
\newpage
\begin{center}
\begin{tabular}{p{1.4in} l | p{1.6in} l}
& ~पृष्ठ.& & ~पृष्ठ.\\
~~नवषष्टितमक्षेत्रम् & १०५---६ & ~~शततमक्षेत्रम् & १२१---२\\
~~सप्ततितमक्षेत्रम् & १०६ & ~~एकाधिकशततमक्षेत्रम् & १२२\\
~~एकसप्ततितमक्षेत्रम् & ~,, & ~~द्व्यधिकशततमक्षेत्रम् & १२२---३\\
~~द्विसप्ततितमक्षेत्रम् & १०६---७ & ~~त्र्यधिकशततमक्षेत्रम् & १२३\\
~~त्रिसप्ततितमक्षेत्रम् & १०७ & ~~चतुरधिकशततमक्षेत्रम् & ~,,\\
~~चतुःसप्ततितमक्षेत्रम् & ~,, & ~~पञ्चाधिकशततमक्षेत्रम् & १२३---४\\
~~पञ्चसप्ततितमक्षेत्रम् & १०८ & ~~षडधिकशततमक्षेत्रम् & १२४\\
~~षट्सप्ततितमक्षेत्रम् & ~,, & ~~सप्ताधिकशततमक्षेत्रम् & १२४---५\\
~~सप्तसप्ततितमक्षेत्रम् & १०८---९ & ~~अष्टाधिकशततमक्षेत्रम् & १२५\\
~~अष्टसप्ततितमक्षेत्रम् & १०९ & ~~नवाधिकशततमक्षेत्रम् & १२६\\
~~एकोनाशीतितमक्षेत्रम् & ~,, & \hyperref[ch11]{\textbf{एकादशोऽध्यायः}} & \textbf{१२७---५९}\\
~~अशीतितमक्षेत्रम् & ११० & ~~~~परिभाषा & १२७---८\\
~~एकाशीतितमक्षेत्रम् & ~,, & ~~प्रथमक्षेत्रम् & १२८\\
~~~~परिभाषा & ~,, & ~~द्वितीयक्षेत्रम् & १२९\\
~~द्व्यशीतितमक्षेत्रम् & १११ & ~~तृतीयक्षेत्रम् & ~,,\\
~~त्र्यशीतितमक्षेत्रम् & ~,, &  ~~~~प्रकारान्तरम् & १३०\\
~~चतुरशीतितमक्षेत्रम् & १११---२ & ~~चतुर्थक्षेत्रम् & १३०---१\\
~~पञ्चाशीतितमक्षेत्रम् & ११२ & ~~पञ्चमक्षेत्रम् & १३१\\
~~षडशीतितमक्षेत्रम् & ~,, & ~~षष्ठक्षेत्रम् & १३१---२\\
~~सप्ताशीतितमक्षेत्रम् & ११३ & ~~सप्तमक्षेत्रम् & १३२\\
~~अष्टाशीतितमक्षेत्रम् & ११३---५ & ~~अष्टमक्षेत्रम् & १३३\\
~~एकोननवतितमक्षेत्रम् & ११५ & ~~नवमक्षेत्रम् & १३३---४\\
~~नवतितमक्षेत्रम् & ११५---६ & ~~दशमक्षेत्रम् & १३४\\
~~एकनवतितमक्षेत्रम् & ११६ & ~~एकादशक्षेत्रम् & ~,,\\
~~द्विनवतितमक्षेत्रम् & ११६---७ & ~~द्वादशक्षेत्रम् & १३५\\
~~त्रिनवतितमक्षेत्रम् & ११७ & ~~त्रयोदशक्षेत्रम् & ~,,\\
~~चतुर्नवतितमक्षेत्रम् & ११८ & ~~चतुर्दशक्षेत्रम् & ~,,\\
~~पञ्चनवतितमक्षेत्रम् & ११९ & ~~पञ्चदशक्षेत्रम् & १३६\\
~~षण्णवतितमक्षेत्रम् & ~,, & ~~षोडशक्षेत्रम् & ~,,\\
~~सप्तनवतितमक्षेत्रम् & १२० & ~~सप्तदशक्षेत्रम् & १३७\\
~~अष्टनवतितमक्षेत्रम् & ~,, & ~~अष्टादशक्षेत्रम् & ~,,\\
~~एकोनशततमक्षेत्रम् & १२१ & ~~एकोनविंशतितमक्षेत्रम् & १३८\\
\end{tabular}
\end{center}

\newpage
\begin{center}
\begin{tabular}{p{1.4in} l | p{1.4in} l}
& ~पृष्ठ. & & ~पृष्ठ.\\
~~विंशतितमक्षेत्रम् & १३८---९ & ~~दशमक्षेत्रम् & १७२---४\\
~~एकविंशतितमक्षेत्रम् & १३९ & ~~एकादशक्षेत्रम् & १७४---५\\
~~द्वाविंशतितमक्षेत्रम् & १४० & ~~द्वादशक्षेत्रम् & १७५---७\\
~~त्रयोविंशतितमक्षेत्रम् & १४०---२ & ~~त्रयोदशक्षेत्रम् & १७७---८\\
~~चतुर्विंशतितमक्षेत्रम् & १४२---३ & ~~चतुर्दशक्षेत्रम् & १७८---८१\\
~~पञ्चविंशतितमक्षेत्रम् & १४३---४ & ~~पञ्चदशक्षेत्रम् & १८१---२\\
~~षड्विंशतितमक्षेत्रम् & १४४---५ & \hyperref[ch13]{\textbf{त्रयोदशोऽध्यायः}} & \textbf{१८३---२०४}\\
~~सप्तविंशतितमक्षेत्रम् & १४५---६ & ~~प्रथमक्षेत्रम् & १८३\\
~~अष्टाविंशतितमक्षेत्रम् & १४६ & ~~द्वितीयक्षेत्रम् & १८४\\
~~एकोनत्रिंशत्तमक्षेत्रम् & १४६---७ & ~~तृतीयक्षेत्रम् & १८४---५\\
~~त्रिंशत्तमक्षेत्रम् & १४७---८ & ~~चतुर्थक्षेत्रम् & १८५\\
~~एकत्रिंशत्तमक्षेत्रम् & १४८---९ & ~~पञ्चमक्षेत्रम् & १८५---६\\
~~द्वात्रिंशत्तमक्षेत्रम् & १४९ & ~~षष्ठक्षेत्रम् & १८६\\
~~त्रयस्त्रिंशत्तमक्षेत्रम् & १४९---५० & ~~सप्तमक्षेत्रम् & १८६---७\\
~~चतुस्त्रिंशत्तमक्षेत्रम् & १५०---१ & ~~अष्टमक्षेत्रम् & १८७---८\\
~~पञ्चत्रिंशत्तमक्षेत्रम् & १५१---२ & ~~नवमक्षेत्रम् & १८८\\
~~षट्त्रिंशत्तमक्षेत्रम् & १५२---३ & ~~दशमक्षेत्रम् & १८८---९\\
~~सप्तत्रिंशत्तमक्षेत्रम् & १५३---४ & ~~एकादशक्षेत्रम् & १८९---९०\\
~~अष्टत्रिंशत्तमक्षेत्रम् & १५५---६ & ~~द्वादशक्षेत्रम् & १९०\\
~~एकोनचत्वारिंशत्तमक्षेत्रम् & १५६---७ & ~~त्रयोदशक्षेत्रम् & १९१---२\\
~~चत्वारिंशत्तमक्षेत्रम् & १५७---८ & ~~चतुर्दशक्षेत्रम् & १९२\\
~~एकचत्वारिंशत्तमक्षेत्रम् & १५८---९ & ~~पञ्चदशक्षेत्रम् & १९२---३\\
\hyperref[ch12]{\textbf{द्वादशोऽध्यायः}} & \textbf{ १६०---८२} & ~~~~प्रकारान्तरम् & १९४\\
~~प्रथमक्षेत्रम् & १६० & ~~षोडशक्षेत्रम् & १९४---५\\
~~द्वितीयक्षेत्रम् & १६०---२ & ~~सप्तदशक्षेत्रम् & १९५---६\\
~~तृतीयक्षेत्रम् & १६२---३ & ~~अष्टादशक्षेत्रम् & १९६---८\\
~~चतुर्थक्षेत्रम् & १६३---५ & ~~एकोनविंशतितमक्षेत्रम् & १९८---२००\\
~~पञ्चमक्षेत्रम् & १६५---६ & ~~विंशतितमक्षेत्रम् & २००---२\\
~~षष्ठक्षेत्रम् & १६६---७ & ~~एकविंशतितमक्षेत्रम् & २०२---४\\
~~सप्तमक्षेत्रम् & १६७---८ & \hyperref[ch14]{\textbf{चतुर्दशोऽध्यायः}} & \textbf{२०५---१३}\\
~~अष्टमक्षेत्रम् & १६८ & ~~प्रथमक्षेत्रम् & २०५\\
~~नवमक्षेत्रम् & १६९---७० & ~~द्वितीयक्षेत्रम् & २०५---६\\
~~~~प्रकारान्तरम् & १७०---२ & &\\
\end{tabular}
\end{center}

\newpage
\begin{center}
\begin{tabular}{p{1.4in} l | p{1.4in} l}
& ~पृष्ठ. & & ~पृष्ठ.\\
~~तृतीयक्षेत्रम् & २०६---७ & \hyperref[ch15]{\textbf{पञ्चदशोऽध्यायः}} & \textbf{२१४---८}\\
~~चतुर्थक्षेत्रम् & २०७---८ & ~~प्रथमक्षेत्रम् & २१४\\
~~पञ्चमक्षेत्रम् & २०८ & ~~द्वितीयक्षेत्रम् & २१४---५\\
~~षष्ठक्षेत्रम् &२०८  & ~~तृतीयक्षेत्रम् & २१५\\
~~सप्तमक्षेत्रम् & २०९---१० & ~~चतुर्थक्षेत्रम् & २१५---६\\
~~अष्टमक्षेत्रम् & २१०---१ & ~~पञ्चमक्षेत्रम् & २१६---७\\
~~नवमक्षेत्रम् & २११---२ & ~~षष्ठक्षेत्रम् & २१७---८\\
~~दशमक्षेत्रम् & २१२---३  & \\
\end{tabular}
\end{center}

\begin{tabular}{lr}
{\en \textbf{Appendix I.} containing
the \emph{variae lectiones}
of V.} & 1-4\\

{\en \textbf{Appendix II.} containing 
 the \emph{variae lectiones} 
 of the Ms.\;in} &\\
 ~~~~~charge of the Ānandāśrama
 Library, Poona & 5-8\\
 
\textbf{Notes} & 9-15\\

\textbf{Errata} & 16
\end{tabular}

\afterpage{\fancyhead[CE] {रेखागणितम्}}
\afterpage{\fancyhead[CO] {सप्तमोऽध्यायः}}
\afterpage{\fancyhead[LE,RO]{\thepage}}
\cfoot{}
\newpage
%%%%%%%%%%%%%%%%%%%%%%%%%%%%%%%%%%%%%%%%%%%%%%%%%%%%%%%%%%%%%%
\renewcommand{\thepage}{\devanagarinumeral{page}}
\setcounter{page}{1}
\newpage
\thispagestyle{empty}
\phantomsection \label{ch7}
\begin{center}
{\bf \LARGE~॥ अथ सप्तमोऽध्यायः प्रारभ्यते~॥}\\
\vspace{6mm}

{\large \renewcommand{\thefootnote}{१}\footnote{तत्र ऊन० {\en K.}}तत्रैकोनचत्वारिंशत्क्षेत्राणि सन्ति~।}\\

\vspace{6mm}
{\LARGE अत्राङ्कैर्गणितप्रकारा निरूपिताः~॥}
\end{center}
\vspace{1mm}

\begin{enumerate}
\item[१.] अङ्को नाम रूपाणां समुदायः~। तन्मते रूपेऽङ्कत्वाभावः~। अन्ये तु गणनायोग्यमङ्कं वदन्ति तन्मते रूपेऽप्यङ्कत्वमस्ति गणनायोग्यत्वात्~। 
\item[२.]यत्र लघ्वङ्को बृहदङ्कादसकृत् शोधितः सन्\renewcommand{\thefootnote}{२}\footnote{{\en Omitted in K.}} बृहदङ्को निःशेषः स्यात् तदा लघ्वङ्को बृहदङ्कस्यांशोऽस्ति~। बृहदङ्को गुणगुणितलघ्वङ्कतुल्योऽस्ति~। 
\item[३.]यस्य भागद्वयं समानं भवति स समाङ्को ज्ञेयः~। 
\item[४.] यस्य भागद्वयं समानं न भवति स विषमाङ्को ज्ञेयः~। 
\item[५.] समाङ्को यद्येकेन हीनोऽधिको वा भवति सोऽपि विषमाङ्को ज्ञेयः~। 
\item[६.] समाङ्को द्विविधः~। एकः समसमः ८~। एकः समविषमः ६~। 
\item[७] समसमो यथा~। समाङ्कः समेन ह्रियमाणः समा लब्धिः प्राप्यते स समसमः~। 
\item[८.] यः समाङ्कः समेन ह्रियमाणः विषमा लब्धिः प्राप्यते स समविषमो ज्ञेयः~। 
\item[९.] अथ विषमविषमाङ्कलक्षणम्~। विषमाङ्को विषमेण ह्रियमाणः विषमा लब्धिः प्राप्यते स विषमविषमाङ्कः~। यथा नवाङ्कः (९) त्रिभक्तः त्रयं प्राप्यते~। 
\item[१०.] योऽङ्को रूपातिरिक्ताङ्केन निःशेषो न भवति स प्रथमोऽङ्को ज्ञेयः~। यथैकादशाङ्कः~।  
\item[११.] यो रूपातिरिक्ताङ्केन विभागार्हः स योगाङ्को ज्ञेयः~।  
\end{enumerate}

\newpage
 \begin{enumerate}
\item[१२.] यावङ्कौ रूपातिरिक्ताङ्केन भक्तौ निःशेषौ भवतस्तावङ्कौ मिलितसञ्ज्ञौ ज्ञेयौ~। 
\item[१३.] यावङ्कावेकातिरिक्तः कोऽपि हरो निःशेषं न करोति तौ भिन्नाङ्कौ ज्ञेयौ~। 
\item[१४.] योऽङ्कः स्वेनैव गुणितः फलं तस्यैव वर्गो भवति~। 
\item[१५.] योऽङ्कः स्ववर्गेण गुणितः घनसञ्ज्ञो भवति~। 
\item[१६.] गुण्याङ्कगुणकाङ्कयोर्घातो गुणनफलं क्षेत्रफलं भवति~। 
\item[१७.] गुण्यगुणकौ भुजसञ्ज्ञौ भवतः~। 
\item[१८.] क्षेत्रफलं केनचिदङ्केन गुणितं घनफलं भवति~। 
\item[१९.] यत्र प्रथमाङ्को यद्गुणितो द्वितीयाङ्कतुल्यो भवति तद्गुणगुणितस्तृतीयाङ्कश्चतुर्थाङ्क-तुल्यो भवति तदा तेऽङ्काः सजातीया भवन्ति~। 
\item[२०.] क्षेत्रफलघनफले ते सजातीये भवतो ययोर्भुजावेकरूपौ सजातीयौ भवतः~। 
\item[२१.] योऽङ्कः स्वलब्धियोगतुल्यो भवति स पूर्णसञ्ज्ञो ज्ञेयः~। यथा षट्~॥ 
\end{enumerate}
\begin{center}
॥ इति परिभाषा~॥ 
\vspace{6mm}

 \textbf{\large अथ प्रथमं क्षेत्रम्~॥~१~॥}
 \end{center}
 \vspace{2mm}
 
 {\ab ययो राश्योः परस्परं भाजितयोरन्ते रूपं शेषं स्यात् तौ राशी भिन्नसञ्ज्ञौ ज्ञेयौ~। }\\

 यथा \textbf{अबं} बृहद्राशिः कल्पितः~। \textbf{जदं} लघुराशिः कल्पितः~। \textbf{जदं अब}मध्ये मुहुः शोधितं शेषं \textbf{तअं} तत् \textbf{जदा}दूनमवशिष्टम्~। पुनः \textbf{तअं} \textbf{जदा}न्मुहुः शोधितं शेषं \textbf{जवं} तत् \textbf{तआ}दूनं जातम्~। एत \textbf{तअ}मध्ये मुहुः शोधितं शेषं \textbf{कअं} रूपम्~। तस्मात् \textbf{अबजद}राशी भिन्नौ स्तः~। 

\begin{center}
अस्योपपत्तिः~। 
\end{center}

यद्येतौ भिन्नौ न भवतः तदान्यौ राशी कल्पनीयौ~। \textbf{हझ}मुभयो- 


\newpage
\begin{flushleft}
\begin{minipage}[t]{0.6\textwidth}
\noindent रपवर्तनाङ्कः कल्पितः~। \textbf{हझे}नापवर्तितं
\textbf{जदं} निःशेषं भविष्यति~। \textbf{जदं बत}मपि निःशेषं
करिष्यति~। इदमेव \textbf{हझं अब}मपि निःशेषं करोति~। तस्मात् \textbf{तअं} निःशेषं
\end{minipage} 
\hfill
\begin{minipage}[t]{0.3\textwidth}
अ ... क ... त ... ... ब\\
{\color{white}अ} ज ... व ... द\\
{\color{white}अ} हझ ...
\end{minipage}
\end{flushleft}
\vspace{-3mm}

\noindent करिष्यति~। 
मिलितराश्योरपवर्ताङ्कः \textbf{तअं दवं} निःशेषं करोति~। तस्मात् \textbf{हझं दवं} निःशेषं करिष्यति~। पूर्वं \textbf{हझं जदं} निःशेषं चकार~। तस्मात् \textbf{जव}मपि निःशेषं करिष्यति~। \textbf{जवं} च \textbf{तकं} निःशेषं करिष्यति~। तस्मात् \textbf{हझं तक}मपि निःशेषं करिष्यति~। \textbf{तअं} निःशेषं पूर्वं कृतवान्~। तस्मात् \textbf{कअं} रूपं निःशेषं करिष्यति~। इदमशुद्धम्~। यतो रूपं निःशेषं कोऽप्यङ्को न करोति~। इदमेवास्माकमिष्टम्~॥ 
\vspace{2mm}

\begin{center}
 \textbf{\large अथ द्वितीयं क्षेत्रम्~॥~२~॥}
\end{center}

 {\ab तत्र मिलितराश्योरपवर्ताङ्को महदङ्कः कल्प्योऽस्ति येन  भक्तौ मिलितराशी निःशेषौ भवतः~।} 

\begin{flushleft}
\begin{minipage}[t]{0.6\textwidth}
यथा \textbf{अबजदौ} मिलितराशी कल्पितौ~। तत्र यदि \textbf{जदं} न्यूनराशिः 
\textbf{अबं} महद्राशिं निःशेषं करोति तदायमेव  महदङ्कोऽस्ति~। यदि \textbf{जदं अबं}  निःशेषं न
करोति किं च \textbf{बहं} निःशेषं   करोति \textbf{अहं}
शेषं \textbf{जदा}न्न्यूनमवशिष्टम्~। त\textbf{ज्जदं} निःशेषं  न करोति किं तु \textbf{दझं} निःशेषं करोति~।
\end{minipage} 
\hfill
\begin{minipage}[t]{0.3\textwidth}
अ .... ..... ब\\
ज ..... द\\
अ .... ह .... ब\\
ज .... झ .... द\\ 
ब .... त
\end{minipage}
\end{flushleft}
\vspace{-3mm}

\noindent \textbf{जझं} शेषं \textbf{अहा}न्न्यूनमवशिष्टं च भवति~। एवं तावन्निःशेषक्रिया कार्या यावद्रूपातिरिक्तान्याङ्केन निःशेषता भवेत्~। \textbf{जझे}न \textbf{अहं} निःशेषं कृतमिति कल्पितम्~। तदा इदमेव \textbf{जझं} महदङ्को जातः~। अनेनोभौ  निःशेषौ जातौ~। 

\begin{center}
अस्योपपत्तिः~।
\end{center}

\textbf{जझं अहं} निःशेषं करोति~। \textbf{अहं} च \textbf{दझं} निःशेषं करोति~। तस्मात् \textbf{जझं दझ}मपि निःशेषं करिष्यति~। \textbf{जद}मपि निःशेषं करिष्यति~। 

\newpage
\noindent \textbf{जदं हबं} निःशेषं करोति~। तस्मात् \textbf{जझं हबं} निःशेषं करिष्यति~। पूर्वं \textbf{जझं अहं} निःशेषमकरोत्~। तस्मात् \textbf{जझं अब}मपि निःशेषं करिष्यति~।  \\
\vspace{-2mm}

इदं \textbf{जझं} महदङ्कः कुतो जातः~। अत्रोच्यते~। यदि महान् न भवति तदास्मादधिकं \textbf{बत}मुभयोरपवर्तकं कल्पितम्~। इदं \textbf{हबं} निःशेषं करिष्यति~। \textbf{अह}मपि निःशेषं करिष्यति~। \textbf{दझ}मपि च निःशेषं करिष्यति~।  \textbf{जदं} निःशेषमकरोत्~। तस्मा\textbf{ज्जझ}मपि निःशेषं करिष्यति~। कल्पितं च 
\textbf{जझा}दधिकम्~। इदमनुपपन्नम्~। तस्मा\textbf{ज्जझं} विनान्यः कश्चन महदङ्क उभयो राश्योरपवर्ताङ्को\renewcommand{\thefootnote}{१}\footnote{रपवर्तको {\en K.}} न भविष्यति~। इदमेवास्माकमिष्टम्~॥ 
\vspace{2mm}

\begin{center}
\textbf{\large अथ तृतीयं क्षेत्रम्~॥~३~॥ }
\end{center}

{\ab अथ राशिद्वयाधिकमिलितराश्यपवर्तनार्थं  \renewcommand{\thefootnote}{२}\footnote{महदङ्कल्पनं क्रियते~। {\en K.}}महदङ्कः कल्पनीयः~। }\\

\begin{flushleft}
\begin{minipage}[t]{0.7\textwidth}
\hspace{4mm} यथा \textbf{अं बं जं} त्रयो राशयः कल्पिताः~। प्रथमं \textbf{अब}राश्योरपवर्तनार्थं महदङ्को \textbf{दं} कल्पनीय:~। यदि \textbf{दं जं} निःशेषं करोति तदायमेव महदङ्को  ज्ञेयः~। यद्येवं महदङ्को न स्यात्तदा \textbf{हं} महदङ्कः कल्पितः~। अय\textbf{मं बं}\hyperref[f4.1]{$^{\scriptsize{\hbox{३}}}$} निःशेषं 
करोति\hyperref[f4.1]{$^{\scriptsize{\hbox{४}}}$} यो महदङ्क एत-द्द्वयं निःशेषं  करोति \textbf{द}मपि स एवाङ्को निःशेषं करिष्यति तस्मात् \textbf{हं} महदङ्को \textbf{दं} लघ्वङ्कं निःशेषं करिष्यति~। इदं बाधितम्~।\\

\hspace{4mm} यदि \textbf{दं जं} निःशेषं न करोति तदैतद्द्वयनिःशेषकारको महदङ्क उत्पाद्यः~। तत् \textbf{हं} कल्पितम्~। इदं \textbf{दं} निःशेषं करिष्यति~। \textbf{अं ब}मपि निःशेषं करिष्यति~। \textbf{ज}मपि निःशेषं करिष्यति~। तस्माद्राशित्रयनिःशेषकारकोऽयं जातः~। अस्मादन्यो महदङ्को न भविष्यति~। यदि
\end{minipage} 
\hfill
\begin{minipage}[t]{0.2\textwidth}
\hspace{1mm}

अ..........\\
ब ........\\
ज ....\\
द .. ...\\
{\color{white}अ} ह ....\\
अ............\\
ब ............\\
ज ..... \\
द ........ \\
ह ....\\
{\color{white}अ} झ ...
\end{minipage}
\end{flushleft}

\blfootnote{\phantomsection \label{f4.1}
$^{\tiny{\hbox{३}}}${\footnotesize अबं {\en D.}} \hspace{4mm} $^{\tiny{\hbox{४}}}${\footnotesize करिष्यति {\en K.}}}

\newpage
\noindent भवति तदा \textbf{झं} कल्पितम्~। इदं \textbf{अं बं} निःशेषं करोति~। \textbf{दं} निःशेषं करिष्यति~। \textbf{जं} निःशेषं करोति~। तस्मात् \textbf{ह}मपि निःशेषं करिष्यति~। अयं \textbf{हा}दधिकोऽस्ति~। इदमशुद्धम्~। तस्मान्महदङ्को \textbf{हं} भविष्यति~। 

\begin{center}
\textbf{\large अथ चतुर्थं क्षेत्रम्~॥~४~॥ }
\end{center}

 {\ab लघुराशिर्महद्राशेरंशोऽस्ति वा गुणगुणितांशोऽस्ति~। }

\begin{flushleft}
\begin{minipage}[t]{0.62\textwidth}
\hspace{4mm} यथा \textbf{जदं अबां}ऽशो वांऽशा भवति~। यदि \textbf{जदं अबं} निःशेषं करोति तदेदं तस्यांशो भवति~। यदि निःशेषं न करोति तदा \textbf{व}चिह्न\textbf{त}चिह्नोपर्यस्य विभागाः कार्याः~। यदि \textbf{अबजदौ} राशी भिन्नौ स्तस्तदा विभागा रूपमिताः कल्पनीयाः~। यदि मिलितराशयः \,स्युस्तदानयोरपव-
\end{minipage} 
\hfill
\begin{minipage}[t]{0.3\textwidth}
\noindent अ .... .... ब\\
\noindent ज ....... द\\
\noindent अ ......... ब  \\
\noindent ह .... झ\\
\noindent ज .... व .... त .... द
\end{minipage}
\end{flushleft}
\vspace{-3mm}

\noindent र्ताङ्केन\renewcommand{\thefootnote}{१}\footnote{रपवर्तनाङ्केन {\en K.}} \textbf{हझे}न तुल्या विभागाः कार्याः~। तदा प्रत्येकं \textbf{जवं वतं तदं अब}स्यांशा भविष्यन्ति~। योगश्चांशा भविष्यन्ति~॥ 
\vspace{2mm}

\begin{center}
\textbf{\large अथ पञ्चमं क्षेत्रम्~॥~५~॥}
\end{center}

 {\ab राशिद्वयमन्यराशिद्वयस्यैकरूपांशो यदि भवति तदा तयोर्योगो राशिर्भविष्यति~। }\\

 यथा \textbf{अबं जद}स्यांशः कल्पितः~। तथैव \textbf{हझं वत}स्यांशः कल्पितः~। तस्मात् \textbf{अब-हझ}योगो \textbf{जदवत}योगस्य स एवांशो भविष्यति~। 

\begin{center}
अस्योपपत्तिः~। 
\end{center}

\begin{flushleft}
\begin{minipage}[t]{0.65\textwidth}
\textbf{जद}स्य \textbf{क}चिह्नोपरि \textbf{अब}तुल्यविभागाः कार्याः~। \textbf{वत}स्य \textbf{ल}चिह्नोपरि \textbf{हझ}तुल्यविभागाः कार्याः~। तस्मात् \textbf{जक-वल}योर्योगो \textbf{अबहझ}योगतुल्यो भविष्यति~। एवं \textbf{कद-लत}योर्योगोऽपि~। तस्मात् \textbf{जदवत}योर्योगे \textbf{अबहझ}योर्योग एकरूपो भविष्यति~। इदमेवास्माकमिष्टम्~॥
\end{minipage} 
\hfill
\begin{minipage}[t]{0.25\textwidth}
अ ... ब\\
ज ... क ... द\\
ह .... झ\\
व .... ल .... त
\end{minipage}
\end{flushleft}
\vspace{-3mm}

\newpage
\begin{center}
\textbf{\large अथ षष्ठं क्षेत्रम्~॥~६~॥}
\end{center}

 {\ab यदि राशिद्वयं राशिद्वयस्य यावदंशो भवति तदा द्वयोर्योगो राशिद्वययो-गस्य स एव यावदंशो भविष्यति~। }

\begin{flushleft}
\begin{minipage}[t]{0.58\textwidth}
\vspace{-3mm}
\hspace{4mm} यथा \textbf{अबं जद}स्य यावदंशः कल्पितस्तदा \textbf{हझं वत}स्य तावदंशः कल्पनीयः~। तस्मात् \textbf{अबह-झ}योगोऽपि \textbf{जदवत}योगस्य स एव यावदंशो भवि-ष्यति~।
\end{minipage} 
\hfill
\begin{minipage}[t]{0.3\textwidth}
अ ... क ... ब \\
ज ............ द \\
ह ... ल ... झ \\
व ... ... ... त
\end{minipage}
\end{flushleft}

\begin{center}
अस्योपपत्तिः~।
\end{center}

\textbf{अब}स्य \textbf{क}चिह्नोपरि \textbf{जदां}शैस्तुल्या विभागाः कार्याः~। \textbf{हझे ल}चिह्रोपरि \textbf{वतां}शतुल्या विभागाः कार्याः~। \textbf{अकं जद}स्य \textbf{हलं वत}स्य चैकांशो भविष्यति~। तस्मात् \textbf{अकहल}योगो \textbf{जदवत}योगस्य स एवांशो भविष्यति~। पुनः \textbf{अकं कबं हललझ}योरेकरूपमस्ति~। तस्मात् द्वयोर्योगो \textbf{जदवत}योगस्य एकरूपा यावदंशा भविष्यन्ति~। इदमेवास्माकमिष्टम्~॥ 
\vspace{2mm}

\begin{center}
\textbf{\large अथ सप्तमं क्षेत्रम्~॥~७~॥}
\end{center}

{\ab राशिद्वयं तथा भवति यथैकराशिर्द्वितीयराशेरंशो भवति~। अन्यराशिद्वयं तथा भवति यथैकराशिर्द्वितीयराशेरप्येकोंऽशो भवति~। न्यूनं तद्राशिद्वयं पूर्वराशिद्वयमध्ये  चेच्छोध्यते तदा शेषं शेषस्य स एवांशो भविष्यति~।}

\begin{flushleft}
\begin{minipage}[t]{0.6\textwidth}
\hspace{4mm} यथा \textbf{अबं जद}स्यांशः \textbf{अहं जझ}स्य स एवांशोऽस्ति~। \textbf{अहं अबा}च्छोधितं \textbf{जझं जदा}च्छोधितं तदा \textbf{हब}शेषं \textbf{झद}शेषस्य स एवांशो भविष्यति~।
\end{minipage} 
\hfill
\begin{minipage}[t]{0.3\textwidth}
अ .... ह .... ब\\
व .... ज .... झ ... द
\end{minipage}
\end{flushleft}
\vspace{-3mm}

\begin{center}
अस्योपपत्तिः~।
\end{center}

\textbf{हबं जव}स्य सोंऽशः कल्पितः योंऽशः \textbf{अहं जझ}स्यास्ति~। तस्मात् \textbf{अबं वझ}स्य स एवांशो भविष्यति~। \textbf{जद}स्यापि स एवांश आसीत्~। 

\newpage
\noindent \textbf{वझजदे} तुल्ये भविष्यतः~। \textbf{जझ}उभयोः शोध्यते~। तदा \textbf{वजं झद}समानमवशिष्यते~। तस्मात् \textbf{हबं झद}स्य स एवांशो भविष्यति~। इदमेवास्माकमिष्टम्~॥ 

\begin{center}
प्रकारान्तरम्~॥
\end{center}

\begin{flushleft}
\begin{minipage}[t]{0.6\textwidth}
\hspace{4mm} यदि \,\textbf{हबं \,झद}स्य \,स \,एवांशो \,न \,भवति \,तदा कल्पितं \textbf{हबं झत}स्य स एवांशोऽस्ति~। तस्मात् \textbf{अबं}
\end{minipage} 
\hfill
\begin{minipage}[t]{0.3\textwidth}
अ ... ह ... ब \\
व ... ज ... झ ... त ... द
\end{minipage}
\end{flushleft}
\vspace{-3mm}

\noindent \textbf{जत}स्य स एवांशो भविष्यति~। \textbf{अबं जद}स्यापि स एवांश आसीत्~। तस्मात् \textbf{जदजते} समाने भविष्यतः~। इदमशुद्धम्~॥ अस्मदिष्टमेव समीचीनम्~॥
\vspace{2mm}

 \begin{center}
\textbf{\large अथाष्टमं क्षेत्रम्~॥~८~॥}
\end{center}

 {\ab तथा राशिद्वयं चेद्भवति यथैकराशिर्द्वितीयराशेर्यावदंशो भवति~। अनयोः मध्ये तथा राशिद्वयं शोध्यं तत्रैकराशिर्द्वितीयराशेर्यावदंशो भवति~। तदा शेषं शेषस्य तादृक् यावदंशो भविष्यति~। }\\

 यथा \textbf{अबं जद}स्य यावन्तोंऽशा भवन्ति तावन्त एव \textbf{अहं जझ}स्यांशा यदि भवन्ति तदा \textbf{हबं झद}स्य तावन्त एवांशा अवशिष्टा भविष्यन्ति~। 
 
\begin{center}
अस्योपपत्तिः~।
\end{center}

 \textbf{वतं अब}तुल्यं कार्यम्~। इदं \textbf{जदां}शानुसारेण \textbf{क}चिह्ने विभक्तं कार्यम्~। \textbf{अहं ल}चिह्ने \textbf{जझां}शानुसारेण\renewcommand{\thefootnote}{१}\footnote{नुकारेण {\en D}.} विभक्तं कार्यम्~। तदा यावन्तौ \textbf{वककतौ} तावन्तौ \textbf{अललहौ} भविष्यतः~।
\vspace{-8mm}

\begin{flushleft}
\begin{minipage}[t]{0.58\textwidth}
 \textbf{वकं जद}स्यांशस्तथास्ति यथा \textbf{अलं जझ}स्यां-शोऽस्ति~। \textbf{जदं जझा}दधिकमस्ति~। तस्मात् \textbf{वकं अला}दधिकं भविष्यति~।
\end{minipage} 
\hfill
\begin{minipage}[t]{0.32\textwidth}
अ .... ल .... ह .... ब\\
ज ........... झ ...... द\\
व ... म ... क ... न ... त
\end{minipage}
\end{flushleft}

\newpage
\noindent \textbf{वमं अल}तुल्यं कल्पयेत्~। तस्मात् \textbf{मकं} शेषं \textbf{झद}स्य सोंऽशो भविष्यति योंऽशो \textbf{वकं जद}स्यास्ति~। एवं \textbf{लह}तुल्यं \textbf{तनं} कल्पितम्~। \textbf{कनं} शेषं \textbf{झद}स्य स एव भविष्यति \textbf{तकं जद}स्य योऽस्ति~। \textbf{अह}तुल्य\textbf{वमतने} \renewcommand{\thefootnote}{१}\footnote{जझस्य यथा भवतसथा {\en \& c. K.}}\textbf{जझ}स्यांशौ भवतस्तथा \textbf{हब}तुल्य\textbf{मनं झद}स्यांशो भविष्यति~। इदमेवास्माकमिष्टम्~॥ 
\vspace{2mm}

\begin{center}
\textbf{अथ नवमं क्षेत्रम्~॥~९~॥}
\end{center}

{\ab यद्यङ्कद्वयमिष्टाङ्कद्वयस्य तुल्यांशं भवति वा यावदंशतुल्यं भवति तदांशोऽपि अंशस्य स एवांशो भवति य इष्टाङ्क इष्टाङ्कस्यांशो भवति~। }\\

 यथा \textbf{अबं जद}स्यांशोऽस्ति \textbf{हझं वत}स्य स एवांशोऽस्ति~। तस्मात् \textbf{अबं हझ}स्य स एवांशो भविष्यति वा यावदंशा भविष्यन्ति यो \textbf{जदं वत}स्यास्ति~। 

\begin{center}
अस्योपपत्तिः~।
\end{center}

\begin{flushleft}
\begin{minipage}[t]{0.64\textwidth}
\hspace{4mm} यदि \textbf{जद}स्य \textbf{क}चिह्नोपरि \textbf{अब}तुल्यविभागः क्रियते~। \textbf{वत}स्य  \textbf{ल}चिह्नोपरि \textbf{हझ}तुल्यो विभागः क्रियते तदा \textbf{जकं वल}स्य सोंऽशो भवति 
अथवा यावदंशो भवति यथा \textbf{अबं हझ}स्यास्ति~। तस्मात् \textbf{जदं वत}स्य स एवांशो भविष्यति अथवा यावदंशो भविष्यति~। इदमेवास्माकमिष्टम्~॥
\end{minipage} 
\hfill
\begin{minipage}[t]{0.28\textwidth}
अ ... ब\\
ज ... क ... द\\
ह .... झ\\
व ..... ल ..... त
\end{minipage}
\end{flushleft}
\vspace{-1mm}

\begin{center}
\textbf{\large अथ दशमं क्षेत्रम्~॥~१०~॥}
\end{center}

{\ab यद्यङ्कद्वयं अभीष्टाङ्कद्वयस्य गुणगुणितांशतुल्यं भवति तयोर्यदि विनिमयः क्रियते तदा यावदंशा यावदंशानां स एवांशो भवति~। अथवा यावदंशास्तथा भविष्यन्ति यथैको द्वितीयस्य~। }

\newpage
यथा \,\textbf{अबं} \,यावदंशा \,\textbf{जद}स्यास्ति \,\textbf{हझं} \,तावन्त \,एव \,यावदंशा \,\textbf{वत}स्यास्तीति~। तस्मात् \textbf{अबं हझ}स्य स एवांशो भविष्यति अथवा तथा यावदंशा भविष्यन्ति\renewcommand{\thefootnote}{१}\footnote{शो भविष्यति {\en K.}} यथा \textbf{जदं वत}स्यास्ति~। 

\begin{center}
अस्योपपत्तिः~।
\end{center}

\begin{flushleft}
\begin{minipage}[t]{0.64\textwidth}
\hspace{4mm} \textbf{अब}स्य \textbf{क}चिह्रोपरि \textbf{जदां}शतुल्या विभागाः कार्याः~। \textbf{हझ}स्य \textbf{ल}चिह्ने \textbf{वतां}शतुल्या विभागाः कार्याः~। प्रत्येकम् \textbf{अकं कबं} प्रत्येकं \textbf{हललझ}योः स एवांशो भविष्यति वा तथा यावदंशा भविष्यन्ति यथा \textbf{अबं हझ}स्यास्ति~। यथा
\end{minipage} 
\hfill
\begin{minipage}[t]{0.26\textwidth}
अ ... क ... ब\\
ज ... ... ... द\\
ह ... ल ... झ\\
व ............. त
\end{minipage}
\end{flushleft}
\vspace{-3mm}

\noindent \textbf{जदं वत}स्यास्ति~। तस्मात् \textbf{अबं हझ}स्य स एवांशो भविष्यति अथवा तथा यावदंशा भविष्यन्ति यथा \textbf{जदं वत}स्यास्ति~। इदमेवास्माकमिष्टम्~॥
\vspace{2mm}

\begin{center}
\textbf{\large अथैकादशं क्षेत्रम्~॥~११~॥}
\end{center}

{\ab यद्यङ्कद्वयमध्येऽङ्कद्वयमेकनिष्पत्तिरूपं \;शोध्यते \;तदा \;शेषे \;तन्निष्पत्तिरूपे भविष्यतः~। }\\

 यथा \,\textbf{अबजद}योः मध्ये \,\textbf{अहजझे} \,शोध्येते~। \,\textbf{अबजद}योः निष्पत्तिः \,\textbf{अहजझ}तुल्या कल्पिता~। तदा \textbf{हबझद}योर्निष्पत्तिरेतन्निष्पत्तितुल्यैव भविष्यति~।
 
\begin{center}
अस्योपपत्तिः~।
\end{center}
 
\begin{flushleft}
\begin{minipage}[t]{0.64\textwidth}
\hspace{4mm}  यतः \textbf{अबं जद}स्य स एवांशो वा यावदंशोऽस्ति यः \textbf{अहं जझ}स्यास्ति~। तस्मात् शेषं \textbf{हबं झद}स्य स एवांशो वा यावदंशो भविष्यति~। तस्मात् अनयोर्निष्पत्तिः सैव निष्पत्तिर्भविष्यति~। इदमेवास्माकमिष्टम्~॥
\end{minipage} 
\hfill
\begin{minipage}[t]{0.25\textwidth}
अ .... ह ... ब\\
ज .... झ ... द
\end{minipage}
\end{flushleft}

\newpage
\begin{center}
\textbf{\large अथ द्वादशं क्षेत्रम्~॥~१२~॥}
\end{center}

{\ab यावन्तोऽङ्का एकनिष्पत्तौ भवन्ति तेषां मध्ये प्रथमाङ्कयोगस्य द्वितीयाङ्कयोगेन सैव निष्पत्तिर्भविष्यति~। }\\

 यथा \textbf{अब}योर्निष्पत्ति\textbf{र्जद}योर्निष्पत्तितुल्या कल्पिता~। तस्मात् \textbf{अज}योगस्य \textbf{वद}योगेन निष्पत्तिः \textbf{अब}निष्पत्तितुल्या भविष्यति~। 

 \begin{center}
अस्योपपत्तिः~।
\end{center}
 
\begin{flushleft}
\begin{minipage}[t]{0.67\textwidth}
\hspace{4mm} योंऽशो वा यावदंशा \textbf{अं ब}स्यास्ति स एवांशो वा यावदंशा \textbf{जं द}स्यास्ति~।  यदि योगः क्रियते तदा \textbf{अजं बद}स्य स एवांशो वा \,यावदंशो भविष्यति~।  यथा \textbf{अं ब}स्यास्ति~।
\end{minipage} 
\hfill
\begin{minipage}[t]{0.24\textwidth}
अ ... ज .....\\
ब ... द ......
\end{minipage}
\end{flushleft}
\vspace{-3mm}

\noindent तस्मात् \textbf{अज}योग\textbf{वद}योगयोर्निष्पत्तिः \textbf{अब}तुल्या भविष्यति~। इदमेवास्माकमिष्टम्~॥ 
\vspace{2mm}

\begin{center}
\textbf{\large अथ त्रयोदशं क्षेत्रम्~॥~१३~॥}
\end{center}

{\ab यदि चतुर्णाम् अङ्कानां मध्ये प्रथमद्वितीययोर्निष्पत्तिस्तृतीयचतुर्थयोर्निष्पत्ति-तुल्या भवति~। तयोर्यदि विनिमयः क्रियते प्रथमतृतीययोर्निष्पत्तिर्द्वितीयचतुर्थयोर्निष्पत्तितुल्या भविष्यति~। }\\

 यथा \textbf{अब}निष्पत्ति\textbf{र्जद}निष्पत्तितुल्या कल्पिता~। तदा \textbf{अज}निष्पत्ति\textbf{र्बद}निष्पत्तितुल्या भविष्यति~।

\begin{center}
अस्योपपत्तिः~।
\end{center}

\begin{flushleft}
\begin{minipage}[t]{0.75\textwidth}
\hspace{4mm} \textbf{अं ब}स्य स एवांशो वा यावदंशोऽस्ति यो \textbf{जं द}स्यास्ति~। यदानयोर्व्यत्यासः क्रियते तदा \textbf{अं ज}स्य स एवांशो वा यावदंशो भवति यो \textbf{बं द}स्यास्ति~। तस्मात् \textbf{अज}योर्निष्पत्ति\textbf{र्बद}निष्पत्तितुल्या भविष्यति~। इदमेवास्माकमिष्टम्~॥ 
\end{minipage} 
\hfill
\begin{minipage}[t]{0.15\textwidth}
अ ...\\
ब ....\\
ज .....\\
द ......
\end{minipage}
\end{flushleft}

\newpage
\begin{center}
प्रकारान्तरम्~।
\end{center}

 अनेनैव \,प्रकारेण \,योगान्तरयोर्निष्पत्तिनिश्चयः\renewcommand{\thefootnote}{१}\footnote{र्निष्पत्तेर्नि {\en K.}} \,कार्यः~। \,यथा \,\textbf{अबबज}निष्पत्ति\textbf{र्दह}-
\vspace{-3mm}

\begin{flushleft}
\begin{minipage}[t]{0.68\textwidth}
हझनिष्पत्तितुल्या कल्पिता~। 
यद्यनयोर्योगः क्रियते वान्तरं क्रियते  तदा
\textbf{अजजब}योर्निष्पत्ति\textbf{र्दझझह}निष्पत्तितुल्या भविष्यति~। 
\end{minipage} 
\hfill
\begin{minipage}[t]{0.22\textwidth}
अ ... ब ... ज \\
द ... ह ... झ
\end{minipage}
\end{flushleft}


\begin{center}
अस्योपपत्तिः~।
\end{center}

यदि व्यत्यासः क्रियते तदा \textbf{अबदह}निष्पत्ति\textbf{र्बजहझ}निष्पत्तितुल्या भविष्यति~। तस्मात्
\textbf{अजदझ}नयोर्निष्पत्ति\textbf{र्बजहझ}निष्पत्तितुल्या 
भविष्यति~। तस्मात् \textbf{अजबज}निष्पत्ति\textbf{र्दझहझ}-निष्पतितुल्या भविष्यति~।
इदमेवास्माकमिष्टम्~॥ 
\vspace{2mm}

\begin{center}
\textbf{\large अथ चतुर्दशं क्षेत्रम्~॥~१४~॥}
\end{center}

{\ab यत्र द्विप्रकारकाङ्का भवन्ति तत्र यदि प्रथमप्रकारे प्रथमद्वितीययोर्निष्पत्तिर्द्वितीयप्रकारे प्रथमद्वितीयनिष्पत्तितुल्या भवति प्रथमप्रकारे द्वितीयतृतीयनिष्पत्तिर्द्वितीयप्रकारे द्वितीयतृतीयनिष्पत्तिसमाना भवति तत्र यदि मध्यमनिष्पत्तिः त्यज्यते तदा प्रथमप्रकारे आद्यन्तनिष्पत्तिर्द्वितीयप्रकारस्याद्यन्तनिष्पत्तिसमाना भवति~। }

\begin{flushleft}
\begin{minipage}[t]{0.75\textwidth}
\hspace{4mm} यथा \textbf{अबज}म् एकप्रकारकाङ्काः कल्पिताः~। \textbf{दहझं} द्वितीयप्रकारकाङ्काः कल्पिताः~। तत्र \textbf{अब}योर्निष्पत्ति\textbf{र्दह}निष्पत्तितुल्या कल्पिता~। \textbf{बज}योर्निष्पत्ति\textbf{र्हझ}निष्पत्तितुल्या कल्पिता~। तस्मात् \textbf{अज}निष्पत्ति\textbf{र्दझ}निष्पत्तितुल्या भविष्यति~।
\end{minipage} 
\hfill
\begin{minipage}[t]{0.13\textwidth}
अ........\\
ब .......\\
ज ......\\
द .....\\
ह .... \\
झ ...
\end{minipage}
\end{flushleft}
\vspace{-12mm}

\begin{center}
अस्योपपत्तिः~।
\end{center}

यदि निष्पत्त्या\renewcommand{\thefootnote}{२}\footnote{निष्पत्तिविनिमयः {\en K.}} विनिमयः क्रियते तदा \textbf{अद}योर्निष्पत्ति\textbf{र्बह}निष्पत्ति- 

\newpage
\noindent तुल्या भविष्यति~। \textbf{बह}निष्पत्ति\textbf{र्जझ}निष्पत्तितुल्या भविष्यति~। तस्मात् \textbf{अद}निष्पत्ति\textbf{र्जझ}-निष्पत्तितुल्या भविष्यति~। यदि व्यत्यासः क्रियते तदा \textbf{अज}निष्पत्ति\textbf{र्दझ}निष्पत्तितुल्या भविष्यति~। इदमेवास्माकमिष्टम्~॥ 
\vspace{2mm}

\begin{center}
\textbf{\large अथ पञ्चदशं क्षेत्रम्~॥~१५~॥}
\end{center}

{\ab यदि रूपं द्वितीयाङ्कं यावद्वारं निःशेषं करोति तावद्वारं तृतीयाङ्कश्चतुर्थाङ्कं निःशेषं करोति चेत्तत्र विनिमये क्रियमाणे रूपं यावद्वारं तृतीयं निःशेषं करिष्यति तावद्वारं द्वितीयं चतुर्थं निःशेषं करिष्यति~।} \\

\begin{flushleft}
\begin{minipage}[t]{0.6\textwidth}
\hspace{4mm} यथा \textbf{अबं} कल्पितम्~। एनम् एकाङ्कस्तावद्वारं निःशेषं करोति यावद्वारं \textbf{जदं हझं} निःशेषं करोति~। तस्मादेकाङ्को \textbf{जदं} तथा निःशेषं करिष्यति यथा \textbf{अबं हझं} निःशेषं करिष्यति~।
\end{minipage} 
\hfill
\begin{minipage}[t]{0.3\textwidth}
अ ... व ... त ... ब\\
{\color{white}अब} ज ... द \\
ह ... क ... ल ... ज्ञ 
\end{minipage}
\end{flushleft}

\begin{center}
अस्योपपत्तिः~।
\end{center}

\textbf{हझ}मध्ये यावन्ति \textbf{जदा}नि सन्ति तावन्ति \textbf{अब}मध्ये रूपाणि सन्ति~। यावन्तो \textbf{हझ}स्य \textbf{कल}चिह्नोपरि \textbf{जद}तुल्या विभागाः क्रियन्ते तावन्तः \textbf{अब}स्य \textbf{व}चिह्न\textbf{त}चिह्नोपरि \renewcommand{\thefootnote}{१}\footnote{एकाङ्क {\en K.}}रूपाङ्कतुल्या विभागाः कार्याः~। तस्मात् \renewcommand{\thefootnote}{२}\footnote{एकं {\en K.}}रूपं \textbf{जदं} तथा निःशेषं करिष्यति यथा प्रत्ये-कम् \textbf{अववततबा}नि \textbf{हककललझा}न् निःशेषान् करिष्यन्ति~। अपि च सम्पूर्णम् \textbf{अबं} सम्पूर्णं \textbf{हझं} निःशेषं करिष्यति~। इदमेवास्माकमिष्टम्~॥ 
 \vspace{2mm}

\begin{center}
\textbf{\large अथ षोडशं क्षेत्रम्~॥~१६~॥}
\end{center}
 
 {\ab तत्र गुण्यगुणकयोर्घातो वा गुणकगुण्ययोर्घातस्तुल्यो भवति~। }\\

 यथा \textbf{अब}गुणनफलं \textbf{ज}सञ्ज्ञं कल्पितम्~। पुन\textbf{र्बअ}गुणनफलं \textbf{दं} कल्पितम्~। \textbf{जं दं} च मिथस्तुल्यमस्ति~। 

\newpage
\begin{center}
अस्योपपत्तिः~।
\end{center}

रूपं\renewcommand{\thefootnote}{१}\footnote{एकं {\en K.}} \textbf{बं} निःशेषं तथा करोति यथा \textbf{अं जं} निःशेषयति~। यतः \textbf{अं ब}गुणितं \textbf{जं} कल्पितम्~। पुनरेकम् \textbf{अं} तथा निःशेषं करोति यथा \textbf{बं दं} निःशेषयति~। यतो \textbf{बं अ}गुणितं \,\textbf{दं} कल्पितम्~। यदि व्यत्यासः क्रियते \,तदैकं \textbf{बं} तथा निःशेषं करिष्यति यथा 
\vspace{-3mm}

\begin{flushleft}
\begin{minipage}[t]{0.74\textwidth}
\textbf{अं दं} निःशेषं करोति~। एकं \textbf{बं} निःशेषमकरोत् यथा \textbf{अं जं} निःशेषमकरोत्~। तस्मात् \textbf{अं} यावद्वारं \textbf{जं} निःशेषं करोति तावद्वारमेव \textbf{दं} निःशेषं करिष्यति~। तस्मात् \textbf{जं दं} तुल्यं जातम्~। इदमेवास्माकमिष्टम्~॥
\end{minipage} 
\hfill
\begin{minipage}[t]{0.15\textwidth}
अ .. \\
व ...\\
ज .... \\
द .....
\end{minipage}
\end{flushleft}
\vspace{-1mm}

\begin{center}
\textbf{\large अथ सप्तदशं क्षेत्रम्~॥~१७~॥}
\end{center}
 
 {\ab यत्राङ्कद्वयं तृतीयाङ्केन गुण्यते तयोः घातयोः निष्पत्तिस्तदङ्कद्वयनिष्पत्तिर्भवि-ष्यति~।} \\

 यथा \textbf{बं अ}गुणितं \textbf{द}घातः कल्पितः~। पुन\textbf{र्जं अ}गुणितं \textbf{ह}घातः कल्पितः~। \textbf{दह}निष्पत्ति-\textbf{र्बज}निष्पत्तितुल्या जाता~। 
 
\begin{center}
अस्योपपत्तिः~।
\end{center}

\begin{flushleft}
\begin{minipage}[t]{0.75\textwidth}
\hspace{4mm} एकम् \textbf{अं} \,तावद्वारं निःशेषं \,करोति यावद्वारं \,\textbf{बं दं} निःशेषं करोति~। एवं हि एकम् \textbf{अं} तावद्वारं निःशेषं करोति यावद्वारं \textbf{जं हं} निःशेषं करोति~। तस्मात् \textbf{बं दं} तावद्वारं निःशेषं करिष्यति यावद्वारं \textbf{जं हं} निःशेषं करोति~। तस्मात् \textbf{बद}निष्पत्ति\textbf{र्जह}निष्पत्तितुल्या भविष्यति~। यदि व्यत्यासः क्रियते तदा \textbf{बज}निष्पत्ति\textbf{र्दह}निष्पत्तिसमाना भविष्यति~। इदमेवास्माकमिष्टम्~॥ 
\end{minipage} 
\hfill
\begin{minipage}[t]{0.15\textwidth}
अ . .\\
ब ...\\
ज ... .\\
द ... ..\\
ह ... ...
\end{minipage}
\end{flushleft}
\vspace{-1mm}

\begin{center}
\textbf{\large अथाष्टादशं क्षेत्रम्~॥~१८~॥}
\end{center}

{\ab योऽङ्कः अङ्कद्वयेन पृथक् गुण्यते तदा तयोर्द्वयोरङ्कयोर्निष्पत्तिस्तद्द्वयनिष्पत्तिसमाना भविष्यति~। }

\newpage
 यथा \textbf{जं अ}गुणितं घातो \textbf{दं} कल्पितः~। पुन\textbf{र्जं बे}न गुणितं घातश्च \textbf{हं} कल्पितः~। तस्मात् \textbf{अब}निष्पत्ति\textbf{र्दह}निष्पत्तितुल्या भविष्यति~।

\begin{center}
अस्योपपत्तिः~।
\end{center}

\begin{flushleft}
\begin{minipage}[t]{0.7\textwidth}
\hspace{4mm} यतो \textbf{ज}म् \textbf{अ}गुणितं \textbf{दं} जातम्~। \textbf{अं ज}गुणितं तदापि \textbf{दं} भविष्यति~। एवं हि \textbf{जं ब}गुणितं \textbf{हं} जातम्~। \textbf{बं ज}गुणितं तदापि \textbf{हं} भविष्यति~। तस्मात् \textbf{दह}निष्पत्तिः \textbf{अब}निष्पत्तितुल्या भविष्यति~। इदमेवास्माकमिष्टम्~॥
\end{minipage} 
\hfill
\begin{minipage}[t]{0.2\textwidth}
\vspace{-8mm}
अ ...\\
ब ... .\\
ज ... .. \\
द ... ...\\
ह ... ... ..
\end{minipage}
\end{flushleft}
\vspace{-1mm}

\begin{center}
\textbf{\large अथैकोनविंशतितमं\renewcommand{\thefootnote}{१}\footnote{अथोनविं {\en K.}} क्षेत्रम्~॥~१९~॥}
\end{center}

 {\ab यत्र तथा चत्वारोऽङ्का भवन्ति येषु प्रथमद्वितीययोर्निष्पत्तिस्तृतीयचतुर्थयोर्निष्पत्तिसमाना भवति~। तदा प्रथमचतुर्थघातो द्वितीयतृतीयघाततुल्यो भविष्यति~। यदि चत्वारोऽङ्का भवन्ति तत्र प्रथमचतुर्थयोर्घातो द्वितीयतृतीयघाततुल्यश्चेद्भवति तदा प्रथमद्वितीयनिष्पत्तिस्तृतीयचतुर्थनिष्पत्तिसमाना भविष्यति~।}\\ 

 यथा \textbf{अबजद}चत्वारोऽङ्काः सन्ति तत्र \textbf{अब}निष्पत्ति\textbf{र्जद}निष्पत्तितुल्यास्ति~। तस्मात् \textbf{अद}घातो \textbf{बज}घातसमानो भविष्यति~। 
 
\begin{center}
अस्योपपत्तिः~।
\end{center}

\begin{flushleft}
\begin{minipage}[t]{0.67\textwidth}
\hspace{4mm} \textbf{अ}म् \textbf{द}गुणितं घातश्च \textbf{हं} कल्पितः~। \textbf{बं जे}न गुणितं घातो \;\textbf{झं} \;\;कल्पितः~। \;पुनः \;\textbf{अज}घातश्च \;\;\textbf{वं} \;कल्पितः~। तस्मात् \textbf{अं जदा}भ्यां गुणितं घातः \textbf{वं हं} जातः~। तस्मात् \textbf{जद}निष्पत्ति\textbf{र्वह}निष्पत्त्या तुल्या भविष्यति~। पुनः \textbf{अं बं ज}गुणितं \textbf{वं झं} घातः कल्पितः~। तस्मात् \textbf{अब}निष्पत्ति-\textbf{र्वझ}निष्पत्तिसमाना भविष्यति~। \textbf{अब}निष्पत्ति\textbf{र्जद}निष्पत्ति-
\end{minipage} 
\hfill
\begin{minipage}[t]{0.23\textwidth}
\vspace{-8mm}

अ ... ...\\
ब ... .\\
ज ...\\
द ...\\
ह ... ... ....\\
झ ... ... ... ...\\
व ....................
\end{minipage}
\end{flushleft}

\newpage
\noindent समानास्ति~। \textbf{जद}निष्पत्तिश्च \textbf{वह}निष्पत्तिसमानास्ति~। तस्मात् \textbf{वह}निष्पत्ति\textbf{र्वझ}निष्पत्तिस-माना भविष्यति~। तस्मात् \textbf{व}निष्पत्ति\textbf{र्हे}न \textbf{झे}न तुल्या जाता~। तस्मात् \textbf{हझे} समाने जाते~। \\

पुनरपि \textbf{हं झं} समानं कल्पितम्~। तस्मात् \textbf{अब}निष्पत्ति\textbf{र्जद}निष्पत्तितुल्या भविष्यति~। 

\begin{center}
अस्योपपत्तिः~।
\end{center}

पूर्वप्रकारेण \,\textbf{वझ}निष्पत्तिः \,\textbf{अब}निष्पत्तिसमानास्ति~। \,\textbf{वह}निष्पत्ति\textbf{र्जद}निष्पत्तिसमा-नास्ति~। \textbf{वह}निष्पत्ति\textbf{र्वझ}निष्पत्तिर्मिथस्तुल्यास्ति~। कुतः~। \textbf{हझ}योस्तुल्यत्वात्~। अतः \textbf{अबजद}निष्पत्तिर्मिथः समाना भविष्यति~। इदमेवास्माकमिष्टम्~॥ 

\begin{center}
\textbf{अनेन क्षेत्रेणेदमपि सिद्धम्~। }
\end{center}

यदि तादृशास्त्रयोऽङ्का भवन्ति येषु प्रथमद्वितीययोर्निष्पत्तिर्द्वितीयतृतीययोर्निष्पत्तिसमाना भवति तत्र प्रथमतृतीयघातो द्वितीयवर्गतुल्यो भवति~। इदमपि ज्ञातम्~। प्रथमतृतीयघातो यदि द्वितीयवर्गतुल्यो भवति तदा प्रथमद्वितीयनिष्पत्तिर्द्वितीयतृतीयनिष्पत्तितुल्या भवति~॥
\vspace{2mm}

\begin{center}
\textbf{\large  अथ विंशतितमं क्षेत्रम्~॥~२०~॥}
\end{center}

{\ab यत्र लघ्वङ्का एकनिष्पत्तौ तथा भवन्ति यथैतेभ्यो लघ्वङ्कास्तन्निष्पत्तौ न भवन्ति तदैतेऽङ्कास्तस्यामेव निष्पत्तौ  ये बृहदङ्कास्तान् निःशेषान् करिष्यन्ति~। यथाक्रमं लघ्वङ्केषु लघ्वङ्कास्ते महदङ्केषु लघ्वङ्कान्निःशेषान् करिष्यन्ति~। लघ्वङ्केषु ये महदङ्कास्ते महदङ्केषु महदङ्कान्निःशेषान् करिष्यन्ति~। }\\

 यथा \textbf{अबजदे} एकनिष्पत्तौ कल्पिते~। \textbf{हझं वतं} तस्यामेव निष्पत्तौ लघ्वङ्कौ कल्पितौ~। तस्मात् \textbf{हझं अबं} यावद्वारं निःशेषं करिष्यति \textbf{वतं जदं} तावद्वारमेव निःशेषं करिष्यति~। 

\newpage
\begin{center}
अस्योपपत्तिः~।
\end{center}

\textbf{हझ}म् \textbf{अब}स्यांशोऽस्ति वा यावद्गुणितोंऽशोऽस्ति~। यदि यावद्गुणितोंऽशो भवति तदा \textbf{हझ}स्य \textbf{क}चिह्नोपरि \textbf{हककझौ} \textbf{अबां}शतुल्यौ कल्पितौ~। तदैते एवांशा\renewcommand{\thefootnote}{१}\footnote{तदा \textbf{वते} त एवां {\en K.}} \textbf{जद}स्य भवि-ष्यन्ति~। तौ च \textbf{बललतौ} कल्पितौ~। \textbf{हकं वल}स्य तत्प्रमाणं भविष्यति यत्प्रमाणं  \textbf{हझं वत}स्य भवति~। तस्मात् \textbf{हकबलौ \,हझवत}योर्न्यूनौ भविष्यतः~। \textbf{हझवत}योर्निष्पत्तितुल्यौ
\vspace{-3mm}

\begin{flushleft}
\begin{minipage}[t]{0.7\textwidth}
 भविष्यतः~। \textbf{हझवतौ} अस्यामेव निष्पत्तौ न्यूनाङ्कौ कल्पितौ~। इदमशुद्धम्~। तस्मात् \textbf{हझ}म् \textbf{अब}स्यांशो भविष्यति~। तदा \textbf{वतं जद}स्यांशो भवति~। न यावद्गुणितोंऽशः~। तस्मात् \textbf{हझं} यावद्वारम् \textbf{अबं} निःशेषं करिष्यति तावद्वारं \textbf{वतं जदं} निःशेषं करिष्यति~। इदमेवास्माकमिष्टम्~॥ 
\end{minipage} 
\hfill
\begin{minipage}[t]{0.2\textwidth}
अ ब ......\\
ज द .... \\
{\color{white}अ} ह .. क .. झ\\ 
{\color{white}अ} व .. ल .. त
\end{minipage}
\end{flushleft}
\vspace{-1mm}

\begin{center}
\textbf{\large अथैकविंशतितमं क्षेत्रम्~॥~२१~॥}
\end{center}

{\ab ये लघ्वङ्कास्तथैकनिष्पत्तौ यदि भवन्ति यथान्ये तेभ्यो लघ्वङ्कास्तन्निष्पत्तौ न भवन्ति~। तदा तेऽङ्का भिन्ना भवन्ति~। }\\

 यथा \textbf{अबौ} लघ्वङ्कौ एकस्यां निष्पत्तौ कल्पितौ~। एतौ भिन्नौ भविष्यतः~। 
 
\begin{center}
अस्योपपत्तिः~।
\end{center}

\begin{flushleft}
\begin{minipage}[t]{0.75\textwidth}
\hspace{4mm} यदि भिन्नौ न स्तस्तदोभयो\textbf{र्जं} अपवर्तनं कल्पितम्~। \textbf{जं} यावद्वारम् \textbf{अं} निःशेषं करोति तत्फलं \textbf{हं} कल्पितम्~। पुनः \textbf{जं बं} यावद्वारं निःशेषं करोति तत्फलं \textbf{दं} कल्पितम्~। तस्मात् \textbf{जं हदा}भ्यां गुण्यते तदानयोर्घातः \textbf{अं बं} भविष्यति~। तस्मात् \textbf{हद}निष्पत्तिः \textbf{अब}निष्पत्तितुल्या भविष्यति~। एतद्वयं \textbf{हं द}म् \textbf{अब}योर्न्यूनमस्ति~। इदमशुद्धम्~। अस्मदिष्टमेव समीचीनम्~॥ 
\end{minipage} 
\hfill
\begin{minipage}[t]{0.15\textwidth}
अ ...\\
ब ...\\
ज ...\\
ह ... \\
द ...
\end{minipage}
\end{flushleft}

\newpage
\begin{center}
\textbf{\large अथ द्वाविंशतितमं क्षेत्रम्~॥~२२~॥}
\end{center}

{\ab भिन्नाङ्कावल्पौ स्तस्तन्निष्पत्तावन्यावल्पावङ्कौ न भविष्यतः~। }

\begin{flushleft}
\begin{minipage}[t]{0.75\textwidth}
\hspace{4mm} यथा \textbf{अबौ} द्वौ भिन्नाङ्कावल्पौ कल्पितौ~। एतन्निष्पत्तावन्यावङ्कावल्पौ  न भविष्यतः~। यदि अन्यावङ्कौ एतन्निष्पत्तावल्पौ स्यातां तदा \textbf{जदौ} कल्पितौ~। तस्मात् \textbf{जं अं ह}तुल्यं निःशेषं करिष्यति~। \textbf{दं बं ह}तुल्यं निःशेषं करिष्यति~। \textbf{हं अं ज}तुल्यं निःशेषं करिष्यति~। \textbf{हं बं द}तुल्यं निःशेषं करिष्यति~। तस्मात् \textbf{अबौ} मिलिताङ्कौ जातौ~। पूर्वं कल्पितौ तु भिन्नाङ्कौ~। इदं बाधितम्~। अस्मदिष्टमेव समीचीनम्~॥ 
\end{minipage} 
\hfill
\begin{minipage}[t]{0.15\textwidth}
अ ... ..\\
ब .. ..\\
ज ...\\
द ... \\
ह ...
\end{minipage}
\end{flushleft}
\vspace{-1mm}

\begin{center}
\textbf{\large अथ त्रयोविंशतितमं क्षेत्रम्~॥~२३~॥}
\end{center}

{\ab द्वयोर्भिन्नाङ्कयोरेकमङ्कमन्यस्तृतीयोऽङ्को निःशेषं करोति चेत्तदा तृतीयोऽङ्को द्वितीयाङ्केन साकं भिन्नो भविष्यति~। }\\

 यथा \textbf{अबौ} द्वौ भिन्नाङ्कौ कल्पितौ~। \textbf{जं} तृतीयाङ्को यथा \textbf{अं} निःशेषं करिष्यति तथा कल्पितः~। तदा जबाङ्कौ भिन्नौ भविष्यतः\renewcommand{\thefootnote}{१}\footnote{\textbf{जं बा}ङ्काद्भिन्नो भविष्यति {\en K.}}\,।

\begin{center}
अस्योपपत्तिः~।
\end{center}

\begin{flushleft}
\begin{minipage}[t]{0.75\textwidth}
\hspace{4mm} यदि \textbf{जबाङ्कौ} भिन्नौ न भविष्यतः तदोभयोः अपवर्तनार्थं \textbf{दं} कल्पितः~। तस्मात् \textbf{दं जं} निःशेषं करिष्यति~। \textbf{जं अं} निःशेषं करोति~। तस्मात् \textbf{दं अं} निःशेषं करिष्यति~। \textbf{दं ब}मपि निःशेषं करोति~। तस्मात् \textbf{अबौ} मिलिताङ्कौ जातौ~। कल्पितौ भिन्नाङ्कौ~। इत्यशुद्धम्~। तस्मादस्मदिष्टं समीचीनम्~॥
\end{minipage} 
\hfill
\begin{minipage}[t]{0.15\textwidth}
अ .....\\ 
ब ... .. \\
ज ...\\
द ...
\end{minipage}
\end{flushleft}
\vspace{-1mm}
 
\begin{center}
\textbf{\large अथ चतुर्विंशतितम क्षेत्रम्~॥~२४~॥}
\end{center}

{\ab यौ द्वावङ्कौ तृतीयाङ्काद्भिन्नौ स्तस्तयोः घातोऽपि तस्मात् तृतीयाङ्काद्भिन्नो भवति~।}

\newpage
यथा \textbf{अबौ जा}ङ्काद्भिन्नौ कल्पितौ~। \textbf{अब}योर्घातो \textbf{दं} कल्पितः~। तस्मादयं \textbf{दा}ङ्को \textbf{जा}द्भिन्नो भविष्यति~। 

\begin{center}
अस्योपपत्तिः~।
\end{center}

\begin{flushleft}
\begin{minipage}[t]{0.75\textwidth}
\hspace{4mm} यदि \textbf{दजा}वङ्कौ भिन्नौ न भवतस्तदा द्वयोः अपवर्तनाङ्को \textbf{हं} कल्पितः~। \textbf{हा}ङ्को \textbf{दा}ङ्कं \textbf{झ}तुल्यं निःशेषं करिष्यतीति कल्पितः~॥ तस्मात् \textbf{हझ}घातो \textbf{दं} भविष्यति~। \textbf{अं बे}न गुणितं \textbf{दं} जातमस्ति~। तस्मात् \textbf{हअ}निष्पत्ति\textbf{र्बझ}निष्पत्तितुल्या भविष्यति~। \textbf{हं जं} निःशेषं करोति~। तस्मात् \textbf{हं अं} भिन्नाङ्कौ भविष्यतः~। तस्मात् \textbf{हं अं} लघू जातौ~। अस्यां निष्पत्तावन्यौ लघ्वङ्कौ न भवतः~। एतावङ्कौ \textbf{बझौ}
\end{minipage} 
\hfill
\begin{minipage}[t]{0.15\textwidth}
अ ... \\
ब ... \\
ज ... \\
द ......\\
ह ... \\
झ ...
\end{minipage}
\end{flushleft}
\vspace{-3mm}

\noindent निःशेषौ करिष्यतः~। तस्मात् \textbf{हं बं} निःशेषं करिष्यति~। \textbf{जं} निःशेषं करोति~। तस्मात् \textbf{बजौ} मिलिताङ्कौ जातौ~। कल्पितौ च भिन्नाङ्कौ~। इदमशुद्धम्~। तस्मादस्मदिष्टं समीचीनम्~॥ 
\vspace{2mm}

\begin{center}
\textbf{\large अथ पञ्चविंशतितमं क्षेत्रम्~॥~२५~॥}
\end{center}

{\ab यद्येकाङ्को द्वितीयाङ्काद्भिन्नो भवति तदा तस्य वर्गोऽपि द्वितीयाङ्काद्भिन्नो भविष्यति~। }\\

यथा \textbf{अं बा}द्भिन्नं कल्पितम्~। \textbf{ज}म् \textbf{अ}अङ्कस्य वर्गः कल्पितः~। तस्मात् \textbf{जं बा}द्भिन्नं भविष्यति~। 

\begin{center}
अस्योपपत्तिः~।
\end{center}

\begin{flushleft}
\begin{minipage}[t]{0.75\textwidth}
\hspace{4mm} \textbf{द}अङ्क\textbf{अ}अङ्कौ तुल्यौ कल्पितौ~। तस्मात् \textbf{अं दं} च \textbf{बा}द्भिन्नं भविष्यति~। \textbf{अ}अङ्क\textbf{द}अङ्कयोर्घाततुल्यं \textbf{ज}मस्ति~। तस्मात् \textbf{जा}ङ्कोऽपि \textbf{बा}द्भिन्नो भविष्यति~। इदमेवास्माकमिष्टम्~॥
\end{minipage} 
\hfill
\begin{minipage}[t]{0.15\textwidth}
अ.. द..\\
ब... \\
ज..
\end{minipage}
\end{flushleft}
\vspace{-1mm}

\begin{center}
\textbf{\large अथ षड्विंशतितमं क्षेत्रम्~॥~२६~॥}
\end{center}
 
 {\ab यदि द्वावङ्कावन्याभ्यामङ्काभ्यां प्रत्येकं भिन्नौ भवतस्तदाद्याङ्कद्वयघातोऽन्यद्वयाङ्कघाताद्भिन्नो भवति~। }

\newpage
\begin{flushleft}
\begin{minipage}[t]{0.65\textwidth}
\hspace{4mm} यथा \textbf{अं ब}मङ्कद्वयं कल्पितं तथा \textbf{जद}म् अन्याङ्कद्वयं कल्पितम्~। प्रत्येकं \textbf{अं बं जदा}भ्यां भिन्नमस्ति~। \textbf{अब}योर्घातो \textbf{हं} कल्पितः~। \textbf{जद}योर्घातो \textbf{झं} कल्पितः~। तस्मात् \textbf{हझा}वपि मिथो भिन्नौ भविष्यतः~।
\end{minipage} 
\hfill
\begin{minipage}[t]{0.25\textwidth}
अ ... ब ....\\
ह ... .... ...\\
ज ... द ....\\
झ ...........
\end{minipage}
\end{flushleft}

\begin{center}
अस्योपपत्तिः~।
 \end{center}
 
 यतः \textbf{अं बं} प्रत्येकं \textbf{जा}द्भिन्नमस्ति~। तस्मात् \textbf{ह}मपि \textbf{जा}द्भिन्नं भविष्यति~। पुनः \textbf{अं बं} प्रत्येकं \textbf{दा}द्भिन्नमस्ति~। तस्मात् \textbf{ह}मपि \textbf{दा}द्भिन्नं भविष्यति~। तस्मात् \textbf{जं दं} प्रत्येकं \textbf{हा}द्भिन्नं भविष्यति~। तस्मात् \textbf{झ}मपि \textbf{हा}द्भिन्नं भविष्यति~। \renewcommand{\thefootnote}{१}\footnote{इदमेवा {\en K.}}इदमस्माकमिष्टम्~॥ 
\vspace{2mm}
 
\begin{center}
\textbf{\large अथ सप्तविंशतितमं क्षेत्रम्~॥~२७~॥}
\end{center}

{\ab यावङ्कौ भिन्नौ भवतस्तयोर्वर्गावपि भिन्नौ भविष्यतः~। एवं तयोर्घनावपि भिन्नौ भवतः~। }\\

 यथा \textbf{अबौ} भिन्नाङ्कौ कल्पितौ~। अनयोर्वर्गौ \textbf{जदौ} कल्पितौ~। \textbf{हझौ} च घनौ कल्पितौ~। तस्मादनयोर्वर्गौ \textbf{जदौ} मिथो भिन्नौ भविष्यतः~। \textbf{हझौ} घनावपि मिथो भिन्नौ भविष्यतः~।

\begin{center}
अस्योपपत्तिः~।
\end{center}

\begin{flushleft}
\begin{minipage}[t]{0.7\textwidth}
\hspace{4mm} \textbf{अबौ} मिथो भिन्नौ स्तः~। तस्मात् प्रत्येकस्य वर्गोऽपि द्वितीयाद्भिन्नो भविष्यति~। तस्मात् \textbf{अं दा}द्भिन्नं भविष्यति~। \textbf{अ}वर्गो \textbf{जं दा}द्भिन्नं भविष्यति~। प्रत्येकम् \textbf{अं जं बदा}भ्यां भिन्नमस्ति~। तस्मात् \textbf{अज}घातो \textbf{ह}मस्ति \textbf{बद}घातो \textbf{झ}मस्ति \textbf{हझा}वपि मिथो भिन्नौ भविष्यतः~। इदमेवास्माकमिष्टम्~॥
\end{minipage} 
\hfill
\begin{minipage}[t]{0.2\textwidth}
\vspace{-8mm}

अ ... \\
ब .... \\
ज ....\\
द .......\\
ह ........\\
झ ...........
\end{minipage}
\end{flushleft}

\newpage
\begin{center}
\textbf{\large अथाष्टाविंशतितमं क्षेत्रम्~॥~२८~॥}
\end{center}

{\ab यावङ्कौ भिन्नौ भवतस्तयोर्योगोऽपि प्रत्येकाद्भिन्नो भविष्यति~। यदि योगः प्रत्येकाद्भिन्नो भविष्यति तदा तदङ्कयोगयोरन्तरमपि भिन्नं भविष्यति~।} 

\begin{flushleft}
\begin{minipage}[t]{0.6\textwidth}
\hspace{4mm} यथा \textbf{अबबजौ} भिन्नाङ्कौ कल्पितौ~। तस्मात् \textbf{अज}म् \textbf{अबा}द्भिन्नं भविष्यति~।
\end{minipage} 
\hfill
\begin{minipage}[t]{0.25\textwidth}
अ....ब...ज\\
द ...
\end{minipage}
\end{flushleft}
\vspace{-1mm}

\begin{center}
{\textbf अस्योपपत्तिः~। }
\end{center}

यदि \textbf{अज}म् \textbf{अबा}द्भिन्नं न भवति तदोभयोरपवर्तनं \textbf{दं} कल्पितम्~। एतत् \textbf{दं} \textbf{बज}स्या-प्यपवर्तनं करिष्यति~। तस्मात् \textbf{अबबजौ} अभिन्नौ भवतः~। इदमशुद्धम्~॥ \\
\vspace{-1mm}

 अनेनैव प्रकारेण \textbf{अजं बजा}द्भिन्नं भविष्यति~।\\ 
\vspace{-1mm}

 पुनरपि \textbf{अजअबौ} भिन्नौ कल्पितौ तस्मात् \textbf{अबबजा}वपि भिन्नौ भविष्यतः~। 

\begin{center}
अस्योपपत्तिः~।
\end{center}

यदि \textbf{अबबजौ} भिन्नौ न भवतस्तदोभयोरपवर्तनं \textbf{दं} कल्पितम्~। तदा \textbf{द}म् \textbf{अज}स्याप्य-पवर्तनं करिष्यति~। तस्मात् \textbf{अजअबौ} मिलितौ भविष्यतः~। इदमशुद्धम्~। अस्मदिष्टमेव समीचीनम्~॥ 

\begin{center}
पुनः प्रकारान्तरम्~॥ 
\end{center}

\begin{flushleft}
\begin{minipage}[t]{0.65\textwidth}
\hspace{4mm} यदि \textbf{अबबजौ} मिलितौ कल्पितौ तदा \textbf{अजबजा}वपि
\end{minipage} 
\hfill
\begin{minipage}[t]{0.25\textwidth}
अ.....ब....ज
\end{minipage}
\end{flushleft}
\vspace{-3mm}
 
\noindent मिलिताङ्कौ भविष्यतः~। यदि \textbf{अजबजौ} मिलिताङ्कौ न भवतस्तदानयो रूपं विना कोऽप्यपवर्तको न भविष्यति~। \textbf{अब}मपि रूपं विना न कोऽप्यपवर्तयति~। तस्मात् \textbf{अबबजौ} भिन्नौ भविष्यतः~। इदमशुद्धम्~॥ 

\newpage
पुनरपि \textbf{अजबजौ} मिलितौ कल्पितौ \textbf{अबबजा}वपि मिलितौ भविष्यतः~। यदि मिलितौ न स्तस्तदानयो रूपं विनापवर्तको न भविष्यतीति~। \textbf{अज}मपि रूपं विना न कोऽप्यपवर्तयतीति~। इदमशुद्धम्~। इष्टमुपपन्नम्~॥ 
\vspace{2mm}

\begin{center}
\textbf{\large अथैकोनत्रिंशत्तमं क्षेत्रम्~॥~२९~॥}
\end{center}

{\ab योगाङ्कं प्रथमाङ्को निःशेषं करोति~। }

\begin{flushleft}
\begin{minipage}[t]{0.75\textwidth}
\hspace{4mm} यथा \textbf{अं} योगाङ्कः कल्पितः~। \textbf{ब}म् अस्यापवर्तकं कल्पितम्~। यदि \textbf{बं} प्रथमाङ्को भवति तदेष्टमस्माकं समीचीनम्~। यदि \textbf{बं} प्रथमाङ्को न भवति तदा \textbf{ब}स्यापवर्तकं \textbf{जं} कल्पितम्~। अनेनैव प्रकारेण \textbf{जं}
\end{minipage} 
\hfill
\begin{minipage}[t]{0.15\textwidth}
अ ...\\
ब ...\\
ज ...
\end{minipage}
\end{flushleft}
\vspace{-3mm}

\noindent प्रथमाङ्को भविष्यति~। यद्ययं न स्यात्तदान्यः कल्पनीयः~। एवं कोऽप्यस्यापवर्तनाङ्को भविष्यति~। तदेव\renewcommand{\thefootnote}{१}\footnote{तदैवं {\en D}} \textbf{जं} कल्पितम्~। तस्मात् \textbf{ज}म् \textbf{अ}मपि निःशेषं करिष्यति~। इदमेवेष्टम्~॥ 
\vspace{-1mm}

\begin{center}
\textbf{\large अथ त्रिंशत्तमं क्षेत्रम्~॥~३०~॥}
\end{center}

{\ab योऽङ्कः कश्चित्स प्रथमाङ्को भवति~। अथवा तस्यापवर्तकः प्रथमाङ्को भवति~। }

\begin{flushleft}
\begin{minipage}[t]{0.75\textwidth}
\hspace{4mm} यथा \textbf{अं} कल्पितम्~। यदीदं प्रथमाङ्कः स्यात्तदैवमिष्टं जातम्~।
\end{minipage} 
\hfill
\begin{minipage}[t]{0.15\textwidth}
अ....
\end{minipage}
\end{flushleft}
\vspace{-3mm}

\noindent यदि प्रथमाङ्को न भवति तदा योगाङ्को भविष्यति~। योगाङ्कं प्रथमाङ्कः निःशेषं करिष्यत्येव~। इदमेवास्माकमिष्टम्~॥ 
\vspace{2mm}

\begin{center}
\textbf{\large अथैकत्रिंशत्तमं क्षेत्रम्~॥~३१~॥}
\end{center}

{\ab यमङ्कं प्रथमाङ्को निःशेषं न करोति तस्मात् प्रथमाङ्को भिन्नो भवति~। }

\begin{flushleft}
\begin{minipage}[t]{0.75\textwidth}
\hspace{4mm} यथा \textbf{अं} प्रथमाङ्कः कल्पितः~। यमङ्कं प्रथमाङ्को निःशेषं न करोति सोऽङ्को \textbf{बं} कल्पितः~। तस्मात् \textbf{अं बा}द्भिन्नं भविष्यति~।
\end{minipage} 
\hfill
\begin{minipage}[t]{0.15\textwidth}
अ ...\\
ब ...
\end{minipage}
\end{flushleft}

\newpage
\begin{center}
अस्योपपत्तिः~।
\end{center}

 यदि द्वावपि भिन्नौ न स्तस्तदैतयो रूपं विहायान्यः कश्चिदङ्कोऽपवर्तनं करिष्यति~। \textbf{अं} च प्रथमाङ्कः कल्पितः~। इदमशुद्धम्~॥ 
\vspace{2mm}

\begin{center}
\textbf{\large अथ द्वात्रिंशत्तमं क्षेत्रम्~॥~३२~॥}
\end{center}

{\ab प्रथमाङ्को यदि घाताङ्कं निःशेषं करोति तदा प्रथमाङ्कस्तस्य घातस्यैकभुजम् अपि निःशेषं करिष्यति~। }\\

 यथा \textbf{अं} प्रथमाङ्कः कल्पितः~। \textbf{बं} घातफलाङ्कः कल्पितः~। घातफलाङ्कस्य \textbf{जदौ} भुजौ कल्पितौ~। \textbf{अं बं} निःशेषं करोतीति कल्पितम्~। तस्मात् \textbf{अं जं} निःशेषं करिष्यति वा \textbf{दं} निःशेषं करिष्यति~। 
 
\begin{center}
अस्योपपत्तिः~।
\end{center}

\begin{flushleft}
\begin{minipage}[t]{0.75\textwidth}
\hspace{4mm} यदि \textbf{अं जं} निःशेषं करोति तदास्मदिष्टं समीचीनम्~। यदि निःशेषं न करोति तदा \textbf{अजौ} मिथो भिन्नौ भविष्यतः~। पुनः \textbf{अं बं ह}तुल्यं निःशेषं करोतीति कल्पितम्~। तस्मात् \textbf{अं} चेत् \textbf{हे}न गुण्यते तदा \textbf{बं} भविष्यति~। \textbf{जद}घातोऽपि \textbf{बं} भविष्यति~। तस्मात् \textbf{अज}निष्पत्तिः \textbf{दह}निष्पत्त्या तुल्या भविष्यति~। \textbf{अजौ} तथा न्यूनाङ्कौ
\end{minipage} 
\hfill
\begin{minipage}[t]{0.15\textwidth}
अ ...\\
ब ........ \\
ज ... \\
द ..........\\
ह ....
\end{minipage}
\end{flushleft}
\vspace{-3mm}

\noindent स्तो यथास्यां निष्पत्तावन्यौ न्यूनाङ्कौ न भविष्यतः~। तस्मात् \textbf{अं दं} निःशेषं करिष्यति~। इदमेवास्माकमिष्ट्म्~।
 
\begin{center}
\textbf{\large अथ त्रयस्त्रिंशत्तमं क्षेत्रम्~॥~३३~॥}
\end{center}
\vspace{2mm}

 {\ab ज्ञाताङ्कनिष्पत्तौ लघ्वङ्कानामुत्पादनं चिकीर्षितमस्ति~। }\\

 यथा \textbf{अबज}म् अङ्काः कल्पिताः~। एतेऽङ्का यदि मिथो भिन्नाः सन्ति तदास्यां निष्पत्तावेत एवाङ्का लघवो भविष्यन्ति~। यदि मिलिताङ्काः स्युस्तदैतेषामपवर्तको महदङ्को \textbf{दं} कल्पितः~। पुनरिदं कल्पनीयं \textbf{दं अं ह}तुल्यं निःशेषं करोति \textbf{बं झ}तुल्यं निःशेषं करोति \textbf{जं} च \textbf{व}तुल्यं निःशेषं करोति~। तस्मात् \textbf{हं झं व}म् एतेऽङ्कास्तस्यां निष्पत्तौ लघ्वङ्का भविष्यन्ति~। 

\newpage
\begin{flushleft}
\begin{minipage}[t]{0.65\textwidth}
यदि न भवन्ति तदा \textbf{तकलं} तस्यां निष्पत्तौ लघ्वङ्का भविष्यन्ति~। \textbf{तः अं कः बं लं जं म}तुल्यं निःशेषं करोतीति कल्पितम्~। तस्मात् \textbf{मत}घातः \textbf{अं} भविष्यति~। \textbf{दह}घातः \textbf{अ}मस्ति~। तस्मात् \textbf{हत}निष्पत्ति\textbf{र्मद}निष्पत्तिसमाना भविष्यति~। \textbf{हं} च \textbf{ता}दधिकमस्ति~। तस्मात् \textbf{मं दा}दधिकं भविष्यति~। \textbf{अबजं} निःशेषं करिष्यति~। पूर्वमेतेषां निःशेषको बृहदङ्को \textbf{दं} कल्पितः~। इदमशुद्धम्~। तस्मात् \textbf{हं झं वं} विनान्ये लघ्वङ्का अस्यां निष्पत्तौ न भविष्यन्ति~। इदम् एवास्माकमिष्टम्~॥
\end{minipage} 
\hfill
\begin{minipage}[t]{0.2\textwidth}
अ.....\\
ब.......... \\
ज............. \\
ह... द...\\
झ.... \\
व..... \\
त... \\
क...... \\
ल......\\
म...
\end{minipage}
\end{flushleft}
\vspace{-3mm}

\begin{center}
\textbf{\large अथ चतुस्त्रिंशत्तमं क्षेत्रम्~॥~३४~॥}
\end{center}

 {\ab तत्र द्वाभ्यामङ्काभ्यां यो लघ्वङ्को निःशेषको\renewcommand{\thefootnote}{१}\footnote{निःशेषो {\en K.}} भवति तदुत्पादनं चिकीर्षितम् अस्ति~। }\\

 यथा \textbf{अं ब}म् अङ्कद्वयं कल्पितम्~। यद्येतयोर्मध्ये लघ्वङ्को महदङ्कं  निःशेषं करोति तदा महदङ्क एवेष्टः~। यदि न करोत्युभौ\renewcommand{\thefootnote}{२}\footnote{{\en D. inserts} तदा.} च मिथो भिन्नौ भवतस्तदा \textbf{अं ब}गुणितं कार्यम्~। तदा घातफलं \textbf{ज}मिष्टं भविष्यति~। 
 
\begin{center}
अस्योपपत्तिः~। 
\end{center}

\begin{flushleft}
\begin{minipage}[t]{0.75\textwidth}
\hspace{4mm} \textbf{जं अं बं} प्रत्येकं निःशेषं करोतीति प्रकटम् एवास्ति~। यद्यन्यो लघ्वङ्को भवति तत् \textbf{दं} कल्पितम्~। \textbf{अबौ ह}तुल्यं \textbf{झ}तुल्यमेनं निःशेषं करिष्यतः~। तस्मात् \textbf{अह}घातो \textbf{दं} भविष्यति~। तथा \textbf{बझ}घातोऽपि \textbf{दं} भविष्यति~। तस्मात् \textbf{अब}निष्पत्ति\textbf{र्झह}निष्पत्तिसमाना भविष्यति~। \textbf{अबौ} तथा लघ्वङ्कौ स्तो यथास्यां निष्पत्तावन्यौ लघ्वङ्कौ न भविष्यतः~। तस्मात् \textbf{अं झं} निःशेषं करिष्यति~। \textbf{बं हं} निःशेषं करिष्यति~। पुन\textbf{र्ब}म् \textbf{अझा}भ्यां गुणितं \textbf{जं दं} जातम्~। तस्मात् \textbf{अझ}नि-
\end{minipage} 
\hfill
\begin{minipage}[t]{0.15\textwidth}
अ....\\
ब...\\
ज.........\\
द........\\
ह.......\\
झ.......
\end{minipage}
\end{flushleft}

\newpage
\noindent ष्पत्ति\textbf{र्जद}निष्पत्तितुल्या भविष्यति~। तस्मात् \textbf{जं} महदङ्को \textbf{दं} लघ्वङ्कमपि निःशेषं करिष्यति~। इदमशुद्धम्~। तस्मात् \textbf{जा}त् कोऽपि लघ्वङ्को न भविष्यति यं \textbf{अबौ} निःशेषं कुरुतः~। \\
\vspace{-1mm}

 यदि \textbf{अबौ} मिलिताङ्कौ स्तस्तस्मात् \textbf{झहौ} तस्यां निष्पत्तौ लघ्वङ्कौ कल्पितौ~। तस्मात् \textbf{अब}निष्पत्ति\textbf{र्झह}निष्पत्तितुल्या भविष्यति~। \textbf{अह}घातफलमथवा \textbf{बझ}घातफलं च \textbf{जं} कल्पि-तम्~। इदमेवास्माकमिष्टम्~। \\
\vspace{-1mm}

\begin{flushleft}
\begin{minipage}[t]{0.75\textwidth}
\hspace{4mm} \textbf{अबौ जं} निःशेषं कुरुत इति प्रकटमेवास्ति~। अयं लघ्वङ्कः कुतोऽस्ति~। यद्ययं लघ्वङ्को न भवति तदा अस्मात् लघ्वङ्को \textbf{दं} कल्पितः~। अमुम् \;\textbf{अं \,व}तुल्यं \;निःशेषं \;करोति \;\textbf{बं} \;च \;\textbf{त}तुल्यं निःशेषं करोति~। तस्मात् \textbf{अव}घातो \textbf{दं} भविष्यति~। \textbf{बत}घातोऽपि \textbf{दं} भविष्यति~। तस्मात् \textbf{अब}निष्पत्तिः \textbf{तव}निष्पत्तिसमाना भविष्यति~। \textbf{झह}निष्पत्तिसमाना आसीत्~। तस्मात् \textbf{झह}निष्पत्तिः \textbf{तव}निष्पत्तिसमाना भविष्यति~। अस्यां निष्पत्तौ \textbf{झहौ} लघ्वङ्कौ स्तः~।
\end{minipage} 
\hfill
\begin{minipage}[t]{0.15\textwidth}
अ....\\
ब...... \\
झ...\\
ह...\\
ज...........\\
द.........\\
व...त...
\end{minipage}
\end{flushleft}
\vspace{-3mm}

\noindent तस्मात् \,\textbf{झं \,तं} \,निःशेषं \,करिष्यति~। पुन\textbf{र्बं \,झे}न \,गुणितं \,\textbf{जं} \,जातं \,तेन \,गुणितं \,\textbf{दं} जातम्~। \textbf{झत}निष्पत्ति\textbf{र्जद}निष्पत्तितुल्या भविष्यति~। तस्मात् \textbf{जं} महदङ्को \textbf{दं} लघ्वङ्कं निःशेषं करिष्यति~। इदमशुद्धम्~। अस्मदिष्टमेव समीचीनम्~॥ 
\vspace{2mm}

\begin{center}
\textbf{\large अथ पञ्चत्रिंशत्तमं क्षेत्रम्~॥~३५~॥}
\end{center}

{\ab यं लघ्वङ्कमन्यौ कावप्यङ्कौ निःशेषं कुरुतः सोऽङ्कस्ताभ्यामङ्काभ्यां निःशेषितमन्याङ्कं\renewcommand{\thefootnote}{१}\footnote{{\en K. omits} अन्य {\en in} अन्याङ्कं.} निःशेषं करिष्यति~। }\\

 यथा \textbf{वतं} लघ्वङ्कः कल्पितः~। अमुं \textbf{अबजदा}ङ्कौ निःशेषं कुरुतः~। पुनरेतावङ्कौ \textbf{हझा}ङ्कं निःशेषं कुरुतः~। तस्मात् \textbf{वता}ङ्कोऽपि \textbf{हझं} निःशेषं करिष्यति~। 
 
\newpage
 \begin{center}
अस्योपपत्तिः~।
\end{center}

\begin{flushleft}
\begin{minipage}[t]{0.65\textwidth}
\hspace{4mm} यदि \textbf{वता}ङ्को \textbf{हझं} निःशेषं न करोति तस्मिन् \textbf{कझ}मवशिष्टं कल्पितम्~। \textbf{कझं वता}न्न्यूनमवशिष्टम्~। पुनः \textbf{अबजदौ हकं} निःशेषं कुरुतः~। कुतः~। \textbf{वत}निःशेषकर-णात्~। \textbf{वते}न \textbf{हक}स्यापि निःशेषकरणाच्च~। पुनः \textbf{अबजदौ हझं} निःशेषं कुरुतः~। तस्मात् \textbf{कझ}मपि निःशेषं करि-ष्यतः~। \textbf{वतं} लघ्वङ्कम् \textbf{अबजदौ} निःशेषं चक्रतुः~। \textbf{वतं कझा}दधिकम् अस्ति~। इदम् अशुद्धम्~। अस्मदिष्टमेव समीचीनम्~॥
\end{minipage} 
\hfill
\begin{minipage}[t]{0.25\textwidth}
\vspace{1mm}
अ ... ब\\
ज ... द \\
व ...... त \\
ह ......... क ... झ
\end{minipage}
\end{flushleft}
\vspace{-2mm}

\begin{center}
\textbf{\large अथ षट्त्रिंशत्तमं क्षेत्रम्~॥~३६~॥}
\end{center}

{\ab तादृशो लघ्वङ्कः कल्पनीयो यं द्वाभ्यामधिका अङ्का निःशेषं कुर्वन्ति~। }\\

 यथा \textbf{अबजा}स्त्रयोऽङ्काः कल्पिताः~। लघ्वङ्कस्तु \textbf{दं} कल्पितः~। अमुम् \textbf{अबौ} निःशेषं कुरुतः~। यदि \textbf{जा}ङ्कोऽपि \textbf{दं} निःशेषं करोति तदायमेव लघ्वङ्कः सिद्धस्त्रिभिरङ्कैरपि निःशेषो भवति~। 
 
\begin{flushleft}
\begin{minipage}[t]{0.75\textwidth}
\hspace{4mm} अत्रोपपत्तिः प्रकटैव~। यदि \textbf{दा}ङ्को लघुर्न भवति तस्मादन्यो लघ्वङ्को \textbf{हः} कल्पितः~। अमुम् \textbf{अबौ} निःशेषं करिष्यतः~। तस्मात् \textbf{हं दा}ङ्कोऽपि निःशेषं करिष्यति~। \textbf{दं हा}ङ्कादधिकमस्ति~। इदमशुद्धम्~। \\
\vspace{-2mm}

\hspace{4mm} यदि \textbf{जा}ङ्को \textbf{दं} निःशेषं न करोति तदा पुनर्लघ्वङ्को निष्पादनीयो यं \textbf{जदौ} निःशेषं कुरुतः~। सोऽङ्कः \textbf{हं} कल्पितः~। अयं लघ्वङ्को जातः~।  एनम् \textbf{अबजदा} निःशेषं कुर्वन्ति~।
\end{minipage} 
\hfill
\begin{minipage}[t]{0.15\textwidth}
अ... \\
ब.... \\
ज...... \\
द..........\\
ह..........
\end{minipage}
\end{flushleft}

\begin{center}
अस्योपपत्तिः~।
\end{center}

 यस्मात् \textbf{अबौ दं} निःशेषं कुरुतो \textbf{दा}ङ्को \textbf{हं} निःशेषं करोति~। तस्मात् 

\newpage

\noindent \textbf{अबौ ह}मपि निःशेषं करिष्यतः~। \textbf{जा}ङ्कोऽपि \textbf{हं} निःशेषं करिष्यति~। तस्मात् \textbf{हा}ङ्कोऽपि \textbf{अबजै}र्निःशेषो\;\renewcommand{\thefootnote}{१}\footnote{भविष्यति {\en K.}}भवति~। अयं \textbf{हा}ङ्कः कुतो लघुस्तत्र युक्तिः~। यद्ययं
\vspace{-4mm}

\begin{flushleft}
\begin{minipage}[t]{0.75\textwidth}
 लघुर्न भवति तदा \textbf{झा}ङ्को लघुः कल्पितः~। एनम् \textbf{अबजा} निःशेषं कुर्वन्ति तस्मात् \textbf{अबा}वपि निःशेषं कुरुतः~। \textbf{दा}ङ्कोऽपि निःशेषं करिष्यति~। \textbf{जा}ङ्कोऽपि निःशेषं करोति~। तस्मात् \textbf{जदा}वपि निःशेषं करिष्यतः~। तस्मात् \textbf{हा}ङ्कोऽपि निःशेषं करिष्यति~। \textbf{हा}ङ्को \textbf{झा}त् अधिकः~। इदमशुद्धम्~। तस्मादिष्टमस्माकं समीचीनम्~॥
\end{minipage} 
\hfill
\begin{minipage}[t]{0.15\textwidth}
\vspace{-6mm}

अ... \\
ब... \\
ज....\\
द..... \\
ह.......\\
झ......
\end{minipage}
\end{flushleft}
\vspace{-2mm}

\begin{center}
\textbf{\large अथ सप्तत्रिंशत्तमं क्षेत्रम्~॥~३७~॥}
\end{center}

{\ab यमङ्कं यः कश्चनाङ्कः निःशेषं करोति तत्र लब्धिस्तन्नामकांशो भवति~। }

\begin{flushleft}
\begin{minipage}[t]{0.77\textwidth}
\hspace{4mm} यथा \textbf{अं} \textbf{बा}ङ्को निःशेषं करोति~। यावद्वारं \textbf{बा}ङ्को \textbf{अं} निःशेषं करोति तावद्वारं रूपं \textbf{जा}ङ्कं निःशेषं करोतीति कल्पितम्~। तस्मात् यावद्वारं \textbf{ज}म् \textbf{अं} निःशेषं करोति तावद्वारं रूपं \textbf{बा}ङ्कं निःशेषं करि-
\end{minipage} 
\hfill
\begin{minipage}[t]{0.13\textwidth}
अ....\\
ब.... \\
ज......
\end{minipage}
\end{flushleft}
\vspace{-3mm}

\noindent ष्यति~। तस्माद्रूपं \textbf{ब}स्य सोंऽशो भविष्यति योंऽशो \textbf{ज}म् \textbf{अ}अङ्कस्यास्ति~। रूपं \textbf{ब}स्य \textbf{बा}ङ्कनामकोंऽशो जातः~। तदा \textbf{ज}म् \textbf{अ}अङ्कस्य सोंऽशो जातः~। इदमेवास्माकमिष्टम्\renewcommand{\thefootnote}{२}\footnote{वास्मदिष्टम् {\en K.}}\;॥
\vspace{2mm}

\begin{center}
\textbf{\large अथाष्टत्रिंशत्तमं क्षेत्रम्~॥~३८~॥}
\end{center}

{\ab यस्याङ्कस्यांशो यन्नामको भवति तन्नामाङ्कस्तमङ्कं निःशेषं करिष्यति~। }

\begin{flushleft}
\begin{minipage}[t]{0.75\textwidth}
\hspace{4mm} यथा \textbf{अ}अङ्कस्य \textbf{ब}मंशोऽस्ति~। रूपं \textbf{ज}स्य स एवांशोऽस्तीति कल्पितम्~। तस्मात् \textbf{बं ज}नामकं भविष्यति~। रूपं \textbf{जा}ङ्कं तथा निःशेषं करोति यथा 
\end{minipage} 
\hfill
\begin{minipage}[t]{0.15\textwidth}
अ......... \\
ब.... \\
ज........
\end{minipage}
\end{flushleft}

\newpage
\noindent \textbf{बा}ङ्कः \textbf{अं} निःशेषं करोति~। तस्माद्रूपं \textbf{बं} निःशेषं तथा करोति यथा \textbf{जा}ङ्कः \textbf{अं} निःशेषं करोति~। तस्मात् \textbf{जा}ङ्कः \textbf{ब}अंशनामकः \textbf{अं} निःशेषं करिष्यति~। इदमेवास्माकमिष्टम्~॥ 
\vspace{2mm}

\begin{center}
\textbf{\large अथोनचत्वारिंशत्तमं क्षेत्रम्~॥~३९~॥}
\end{center}

 {\ab तत्र यस्य बहवोंऽशाः प्राप्यन्ते तादृशो लघ्वङ्को निष्पादनीयोऽस्ति~। }

\begin{flushleft}
\begin{minipage}[t]{0.65\textwidth}
\hspace{4mm} यथा \textbf{अबजा} अंशाः कल्पिताः~। \textbf{दहझ}नामका अङ्काः कल्पिताः~। तस्मात्तादृशो लघ्वङ्कः कल्पनीयो यं \textbf{दहझा} निःशेषं करिष्यन्ति~। असावङ्को \textbf{वं} कल्पितः~। तस्मात् अयं स लघ्वङ्कोऽस्ति यस्य ते कल्पितांशा लभ्यन्ते~।
\end{minipage} 
\hfill
\begin{minipage}[t]{0.25\textwidth}
अ, $\frac{१}{२}$ द.. \\
\vspace{-4mm}

ब, \,$\frac{१}{२}$ ह... \\
\vspace{-4mm}

ज, $\frac{१}{४}$ झ... \\
{\color{white}अबक} व.......... \\
{\color{white}अबक} त.........
\end{minipage}
\end{flushleft}
\vspace{-5mm}

\begin{center}
अस्योपपत्तिः~।
\end{center}

 यद्ययं लघ्वङ्को न भवति तदा \textbf{तो} लघ्वङ्कः कल्पितः~। कल्पिता अंशाः \textbf{त}लघ्वङ्कस्य भविष्यन्ति~। एतल्लघ्वङ्कनामसदृशा अङ्का \textbf{हदझा} एनं निःशेषं करिष्यन्ति~। लघ्वङ्को \textbf{वा}त् लघुरस्ति~। इदमनुपपन्नम्~। तस्मात् \textbf{व} एवेष्टाङ्कः~। इदमेवास्माकमिष्टम्~॥~३९~॥
\vspace{2mm}
 
\begin{quote}
\qt
श्रीमद्राजाधिराजप्रभुवरजयसिंहस्य तुष्टौ द्विजेन्द्रः \\
श्रीमत्सम्राड् जगन्नाथ इति समभिधारूढितेन प्रणीते~। \\
ग्रन्थेऽस्मिन्नाम्नि रेखागणित इति सुकोणावबोधप्रदात-\\ 
र्यध्यायोऽध्येतृमोहापह इह विरतिं सप्तमः सङ्गतोऽभूत्~॥~७~॥ 
\end{quote}
\vspace{-1mm}

\begin{center}
\textbf{\large इति श्रीजगन्नाथसम्राड्विरचिते रेखागणिते \\
 सप्तमोऽध्यायः समाप्तः~॥~७~॥} \\
\vspace{6mm} 

\rule{0.9in}{0.8pt}
\end{center}

\afterpage{\fancyhead[CE] {रेखागणितम्}}
\afterpage{\fancyhead[CO] {अष्टमोऽध्यायः}}
\afterpage{\fancyhead[LE,RO]{\thepage}}
\cfoot{}
\newpage
%%%%%%%%%%%%%%%%%%%%%%%%%%%%%%%%%%%%%%%%%%%%%%%%%%%%%%%%%%%%%%
\newpage
\thispagestyle{empty}
\phantomsection \label{ch8}
\begin{center}
{\bf \LARGE~॥ अथाष्टमोऽध्यायः प्रारभ्यते~॥}
\vspace{5mm}

\textbf{~॥ तत्र पञ्चविंशतिक्षेत्राणि सन्ति~॥}
\vspace{5mm}

\textbf{\large अथ प्रथमं क्षेत्रम्~॥~१~॥}
\end{center}

 {\ab यावन्तोऽङ्का एकनिष्पत्तौ भवन्ति तेषामाद्यन्तौ भिन्नाङ्कौ चेद्भवतस्तदा तस्यां निष्पत्तौ तान् विनान्ये लघ्वङ्का न भविष्यन्ति~। }\\

 यथा एकस्यां निष्पत्तौ \textbf{अवजदा} लघ्वङ्काः कल्पिताः~। \textbf{अदौ} मिथो भिन्नौ कल्पितौ~। तस्मादस्यां निष्पत्तावेते लघ्वङ्काः सन्ति~। 

\begin{center}
अस्योपपत्तिः~। 
\end{center}

\begin{flushleft}
\begin{minipage}[t]{0.62\textwidth}
\hspace{4mm} यद्येते लघ्वङ्का अस्यां निष्पत्तौ न भवन्ति तदा तस्यां निष्पत्तौ \,तेभ्यो \,लघवोऽन्येऽङ्का \,\textbf{हझवताः} \,कल्पिताः~। तस्मात् \textbf{अद}निष्पत्ति\textbf{र्हत}निष्पत्तिसमाना भविष्यति~। \textbf{अदौ} यौ भिन्नाङ्कौ तावस्यां निष्पत्तौ लघ्वङ्कौ भविष्यतः~।
\end{minipage} 
\hfill
\begin{minipage}[t]{0.3\textwidth}
अ,८. ब,१२. ज,१८. द,२७.\\
ह - - - \\
झ - - - त - - - \\
व - - - 
\end{minipage}
\end{flushleft}
\vspace{-3mm}

\noindent यावन्तोऽङ्का अस्यां निष्पत्तौ भवन्ति तान् \textbf{अदा}वेव निःशेषं करिष्यतः~। तस्मात् \textbf{अं हं} निःशेषं करिष्यति~। \textbf{अं हा}दधिकमस्ति~। इदमेवास्माकमिष्टम्~॥ 
\vspace{2mm}

\begin{center}
\textbf{\large अथ द्वितीयं क्षेत्रम्~॥~२~॥}
\end{center}

{\ab एकनिष्पत्तौ ये लघ्वङ्का भवन्ति तेषामुत्पादनमिष्टमस्ति~। }\\

 यथा \textbf{अब}निष्पत्तौ चतुर्णां लघ्वङ्कानाम् उत्पादनम् इष्टम् अस्ति~। अस्यां निष्पत्तौ \textbf{अबौ} लघ्वङ्कौ कल्पितौ~। \textbf{अ}वर्गः कार्यः~। पुनः \textbf{अब}घातः कार्यः~। पुन\textbf{र्ब}वर्गः कार्यः~। फलानां च \textbf{जदह}सञ्ज्ञा कार्या~। पुनरेतत्त्रयेण \textbf{अं} गुणनीयम्~। \textbf{बह}घातश्च कार्यः~। एतेषां फलानि \textbf{झवतका}नि कल्पितानि~। 

\begin{center}
अस्योपपत्तिः~।
\end{center}

 \textbf{अ}म् \textbf{अबा}भ्यां गुणितं फलं \textbf{जं द}मुत्पन्नम्~। तदा \textbf{अब}निष्पत्तिः

\newpage

\begin{flushleft}
\begin{minipage}[t]{0.55\textwidth}
\textbf{जद}निष्पत्त्या \,तुल्या \,भविष्यति~। \textbf{ब}म् \,\textbf{अबा}भ्यां गुणितं फलं \textbf{दह}सञ्ज्ञं जातम्~। तस्मात् \textbf{दह}नि-ष्पत्तिः \textbf{अब}निष्पत्तितुल्या भविष्यति~। तस्मादेतत्त्रयमेकनिष्पत्तौ भविष्यति~।  पुनः \textbf{अ}म् एतत्त्रय-
\end{minipage} 
\hfill
\begin{minipage}[t]{0.35\textwidth}
अ,२. ब,३.\\
ज,४. द,६. ह,९.\\
झ,८. व,१२. त,१८. क,२७.
\end{minipage}
\end{flushleft}
\vspace{-3mm}

\noindent गुणितं \textbf{झवतं} निष्पन्नं तदप्येकनिष्पत्तौ जातम्~। \textbf{ह}गुणितम् \textbf{अबं}\renewcommand{\thefootnote}{१}\footnote{अं बं {\en K. }} फलं \textbf{तक}सञ्ज्ञं जातम्~। इदम् अपि पूर्वनिष्पत्तौ जातम्~। तस्माच्चत्वारोऽङ्का एकस्यामेव निष्पत्तौ जाताः~। एते लघ्वङ्का ये अस्यां निष्पत्तौ जाताः~। कुतः~। \textbf{अब}योर्भिन्नाङ्कत्वात्~। \textbf{जहौ} \renewcommand{\thefootnote}{२}\footnote{{\en K.} एते ( एतौ ? {\en or} एतयोः? )}एतेषां वर्गौ \textbf{झकौ} घनौ त्रयाणामङ्कानामाद्यन्तौ चतुर्णामप्याद्यन्तौ भिन्नौ भिन्नौ पतितौ~। इदमेवास्माकमिष्टम्~॥ \\

 अनेन क्षेत्रेणेदं सिद्धम्~। ये लघवस्त्रयोऽङ्का एकनिष्पत्तौ भवन्ति तेषामाद्यन्तौ वर्गौ भवतः~। ये लघवश्चत्वारोङ्का एकनिष्पत्तौ भवन्ति तेषामाद्यन्तौ घनौ भवतः~॥ 
\vspace{2mm}

\begin{center}
 \textbf{\large अथ तृतीयं क्षेत्रम्~॥~३~॥}
\end{center}

{\ab  यावन्तो लघ्वङ्का एकनिष्पत्तौ भवन्ति तेषामाद्यन्तौ भिन्नौ भवतः~। }\\

 यथा \textbf{अबजदा} लघ्वङ्काश्चत्वार एकनिष्पत्तौ कल्पिताः~। तत्र \textbf{अदौ} भिन्नौ भवतः\renewcommand{\thefootnote}{३}\footnote{भविष्यतः {\en K.}}\;।
 
\begin{center}
अस्योपपत्तिः~।
\end{center}

\begin{flushleft}
\begin{minipage}[t]{0.57\textwidth}
\hspace{4mm} अस्यां निष्पतौ \textbf{हझौ} लघ्वङ्कौ गृहीतौ~। पुन\textbf{र्व-तकाः} त्रयोऽङ्का लघवो गृहीताः~। पुन\textbf{र्लमनसा}श्च-त्वारो लघ्वङ्कास्तस्यामेव निष्पत्तौ गृहीताः~। तस्मादेते \textbf{अबजद}तुल्या भविष्यन्ति~। \textbf{लसौ} भिन्नौ स्तः~। \textbf{अदा}वपि भिन्नौ भविष्यतः~। इदमेवास्माकमिष्टम्~॥
\end{minipage} 
\hfill
\begin{minipage}[t]{0.34\textwidth}
अ,८. ब,१२. ज,१८. द,२७.\\
ह,२. झ,३. \\
व,४. त,६. क,९.\\
ल,८. म,१२. न,१८. स,२७.
\end{minipage}
\end{flushleft}

\newpage
\begin{center} 
 \textbf{\large अथ चतुर्थक्षेत्रम्~॥~४~॥}
 \end{center}

{\ab  तत्र कल्पितबहुनिष्पत्तिषु लघूनामङ्कानामुत्पादनमिष्टमस्ति~। }\\

 यथा \textbf{अब}निष्पत्ति\textbf{जद}निष्पत्ति\textbf{हझ}निष्पत्तयः कल्पिताः~। प्रत्येकमङ्कद्वयमस्यां निष्पत्तौ लघ्वङ्कं भवति~। अथ \textbf{तं} लघ्वङ्क उत्पाद्यः \textbf{यं बजौ} निःशेषं करिष्यतः~। तथैकोऽङ्को \textbf{व}म् उत्पाद्यो यम् \textbf{अं} तथा निःशेषं करिष्यति यथा \textbf{बं तं} निःशेषं करोति~। पुन\textbf{र्दं कं} तथा निःशेषं करोति यथा \textbf{जं तं} निःशेषं करोति~। पुन\textbf{र्लः} लघ्वङ्क उत्पाद्यो यथा \textbf{लं कहौ} निःशेषं करिष्यतः~। पुन\textbf{र्नसौ} लघ्वङ्कौ उत्पाद्यौ यौ \textbf{वतौ} तथा निःशेषं कुरुतो यथा \textbf{कं लं} निःशेषयति~। \textbf{झं मं} निःशेषं तथा करोति यथा \textbf{हं लं} निःशेषयति~। तस्मात् \textbf{नसलम}अङ्कास्तासु निष्पत्तिषु उत्पन्ना जाताः~। 

\begin{center}
 अस्योपपत्तिः~। 
\end{center}

\begin{flushleft}
\begin{minipage}[t]{0.42\textwidth}
\hspace{4mm}  \textbf{अबौ वतौ} क्रमेण तुल्यं निःशेषं कुरुतः~। \textbf{वतौ नसौ} तुल्यं  निःशेषं  कुरुतः~। तस्मात् \textbf{नसौ अब}निष्पत्तौ भविष्यतः~। \textbf{जदौ तकौ} तुल्यं निःशेषं कुरुतः~। \;पुन\textbf{स्तकौ \;सलौ} \;निःशेषं कुरुतः~। तस्मात् \textbf{सलौ  जद}निष्पत्ति-
\end{minipage} 
\hfill
\begin{minipage}[t]{0.48\textwidth}
अ,२. ब,५. ज,३. द,४. ह,५. झ,६.\\
व,६. त,१५. क,२०. ल,२०. म,२४\\
न,६. स,१५.\\
न,६. स,१५. ल,२०. म,२४\\
ग - - - फ - - - छ - - - ख - - - 
\end{minipage}
\end{flushleft}
\vspace{-3mm}

\noindent तुल्यौ जातौ~। \textbf{हझौ लमौ} तुल्यं निःशेषं करिष्यतः~। तस्मात् \textbf{लमौ हझ}निष्पत्तितुल्यौ भविष्यतः~। तस्मात् \textbf{नसलमा} लघ्वङ्का अस्यां निष्पत्तौ जाताः~। यदि लघ्वङ्का एते न भवन्ति तस्मात् \textbf{गफछखा} लघ्वङ्काः कल्पिताः~। तस्मात् \textbf{अबौ गफौ} तुल्यनिष्पत्तौ भविष्यतः~। पुन\textbf{रबौ} लघ्वङ्कौ अस्यां निष्पत्तौ स्तः~। तस्मादेतौ \textbf{गफं} निःशेषं करिष्यतः~। अनेनैव प्रकारेण \textbf{जदौ फछौ} निःशेषं कुरुतः~। \textbf{हझौ छखौ} निःशेषं कुरुतः~। तस्मात् \textbf{बजौ फं} निःशेषं करिष्यतः~। तं लघ्वङ्कं \textbf{बजौ} निःशेषं करिष्यतः~। तस्मा\textbf{त्तं फं} निःशेषं करिष्यति~। पुन\textbf{स्तक}निष्पत्तिः \textbf{फछ}निष्पत्तितुल्या भविष्यति~। तस्मात् \textbf{कं छं} 

\newpage
\noindent निःशेषं करिष्यति~। \textbf{हं छ}निःशेषमासीत्\renewcommand{\thefootnote}{१}\footnote{निःशेषकमासीत् {\en K.}} तस्मात् \textbf{कहौ छं} निःशेषं करिष्यतः~। \textbf{लः} लघ्वङ्कोऽस्ति यं \textbf{कहौ} निःशेषं करिष्यतः~। तस्मात् \textbf{लं छं} निःशेषं करिष्यति~। \textbf{छं} च लघ्वङ्कोऽस्ति~। इदमशुद्धम्~। तस्मा\textbf{न्नसलमा} एव लघ्वङ्का भविष्यन्ति~। इदमेवेष्टम्~॥ 
\vspace{2mm}

\begin{center}
\textbf{\large अथ पञ्चमं क्षेत्रम्~॥~५~॥}
\end{center}

{\ab घातफलाङ्कस्य घातफलाङ्केन निष्पत्तिस्तद्भुजनिष्पत्त्योर्घातो भविष्यति~। }\\

\begin{flushleft}
\begin{minipage}[t]{0.64\textwidth}
\hspace{4mm} यथा \textbf{अ}घातफलाङ्कस्य \textbf{जदौ} भुजौ कल्पितौ~। \textbf{ब}घात-फलस्य \textbf{हझौ} भुजौ कल्पितौ~। तस्मात् \textbf{अब}योर्निष्पत्तिः \textbf{जहदझ}निष्पत्त्योर्घातो भविष्यति~। अनयोर्निष्पत्त्यो\textbf{र्वतकं} लघ्वङ्का ग्राह्याः~। तस्मात् \textbf{जह}निष्पत्ति\textbf{र्वत}निष्पत्तिस-माना भविष्यति~। \textbf{दझ}निष्पत्ति\textbf{स्तक}निष्पत्तिसमानास्ति~।
\end{minipage} 
\hfill
\begin{minipage}[t]{0.28\textwidth}
अ,६. ब,२०.\\
{\color{white}अब} ल,१२.\\
ज,२. द,३. ह,४. झ,५.\\ 
{\color{white}अ} व,३. त,६. क,१०.
\end{minipage}
\end{flushleft}
\vspace{-3mm}

 \noindent अनयोर्निष्पत्त्योर्घातो \textbf{वक}निष्पत्तिरस्ति~। \textbf{दह}घातो \textbf{लः} कल्पितः~। तस्मात् \textbf{वत}निष्पत्ति-तुल्या \textbf{जह}निष्पत्तिः \textbf{अल}निष्पत्तिसमाना भविष्यति~। \textbf{दझ}निष्पत्तितुल्या \textbf{तक}निष्पत्ति\textbf{र्लब}-निष्पत्तितुल्या भविष्यति~। तस्मात् \textbf{वक}निष्पत्तिर्निष्पत्तिद्वयघातः \textbf{अब}निष्पत्तिसमाना भवि-ष्यति~। इदमेवास्माकमिष्टम्~॥ 
 \vspace{2mm}

\begin{center}
\textbf{\large अथ षष्ठं क्षेत्रम्~॥~६~॥}
\end{center}
 
 {\ab यदि बहवोऽङ्का एकनिष्पत्तौ भवन्ति तत्र यदि प्रथमाङ्को द्वितीयं निःशेषं न करोति तदा कोऽप्यङ्कोऽग्रे निःशेषं न करिष्यति~। }\\

\begin{flushleft}
\begin{minipage}[t]{0.5\textwidth}
\hspace{4mm} यथा \textbf{अबजदह}मेकनिष्पत्तौ कल्पितम्~। \textbf{अं बं} निःशेषं न करोति~। तस्मात् कोऽपि कमपि निःशेषं न करिष्यति~। यदि \textbf{जदह}-निष्पत्तौ \textbf{झवता} लघ्वङ्का गृह्यन्ते
\end{minipage} 
\hfill
\begin{minipage}[t]{0.4\textwidth}
\vspace{1mm}

अ,१६. ब,२४. ज,३६. द,५४. ह,८१ \\
झ,४. व,६. त,९.
\end{minipage}
\end{flushleft}

\newpage
\noindent तदा \textbf{झतौ} भिन्नाङ्कौ भविष्यतः~। \textbf{झं} च यदि रूपं नास्ति तदा \textbf{झव}निष्पत्ति\textbf{र्जद}निष्पत्तेः समानास्ति~। पुन\textbf{र्जं दं} निःशेषं न करोति तस्मात् \textbf{झं वं} निःशेषं न करिष्यति~। रूपं च सर्वं निःशेषं करोति~। पुन\textbf{र्झं तं} निःशेषं न करिष्यति~। तस्मात् \textbf{झत}निष्पत्ति\textbf{र्जह}निष्पत्तिसमाना भविष्यति~। इदमेवास्माकमिष्टम्~॥
\vspace{2mm}
 
\begin{center}
\textbf{\large अथ सप्तमं क्षेत्रम्~॥~७~॥}
\end{center}

{\ab यावन्तोऽङ्का \,एकनिष्पत्तौ \,भवन्ति \,आद्याङ्कोऽन्त्याङ्कं \,निःशेषं \,करोति \,तदा आद्याङ्को द्वितीयाङ्कमपि निःशेषं करिष्यति~। }

\begin{flushleft}
\begin{minipage}[t]{0.63\textwidth}
\hspace{4mm} यथा \textbf{अबजदं} चत्वारोऽङ्का एकनिष्पत्तौ कल्पिताः~। \textbf{अं दं} निःशेषयति तदा \textbf{ब}मपि निःशेषयति~।
\end{minipage} 
\hfill
\begin{minipage}[t]{0.3\textwidth}
अ,२. ब,४. ज,८. द,१६.
\end{minipage}
\end{flushleft}
\vspace{-1mm}

\begin{center}
अस्योपपत्तिः~।
\end{center}

यदि \textbf{बं} निःशेषं न करिष्यति तदान्त्याङ्कमपि निःशेषं न करिष्यति~। इदमेवास्माकमिष्टम्~॥ 
\vspace{2mm}

\begin{center}
\textbf{\large अथाष्टमं क्षेत्रम्~॥~८~॥}
\end{center}

{\ab यावन्तोऽङ्का एकनिष्पत्तावङ्कद्वयमध्यगा\renewcommand{\thefootnote}{१}\footnote{मध्यमा {\en K.}} भवन्ति \renewcommand{\thefootnote}{२}\footnote{तन्निष्पत्तौ {\en K.}}तयोर्निष्पत्तौ यौ द्वावङ्कौ अन्यौ भविष्यतस्तयोरन्तर्गतास्तावन्त 
एवाङ्कास्तन्निष्पत्तौ भविष्यन्ति~। }

\begin{flushleft}
\begin{minipage}[t]{0.6\textwidth}
\hspace{4mm} यथा \textbf{अब}योर्मध्ये \textbf{जदा}वङ्कौ पतितौ~। एते चत्वारः \textbf{अज}निष्पत्तौ जाताः~। \textbf{अब}योर्निष्पत्तौ \textbf{हझा}वन्याङ्कौ कल्पितौ~। अनयोर्मध्ये तथा द्वावङ्कौ पतिष्यतो यथैते चत्वारः \textbf{अज}निष्पत्तौ भविष्यन्ति~।
\end{minipage} 
\hfill
\begin{minipage}[t]{0.3\textwidth}
अ,२. ज,४. द,८. ब,१६.\\
\noindent ब,१. त,२. क,४. ल,८. \\
\noindent ह,३. म,६. न,१२. झ,२४.
\end{minipage}
\end{flushleft}

\newpage
\begin{center}
अस्योपपत्तिः~।
\end{center}

\textbf{अजदबा}नां निष्पत्तौ \textbf{वतकला} लघ्वङ्का गृहीताः~। तस्मात् \textbf{वलौ} भिन्नौ भविष्यतः~। अनयोर्निष्पत्तिः \textbf{अब}निष्पत्तिसमानास्ति~। \textbf{हझ}निष्पत्तेः समानास्ति~। तस्मात् एतौ द्वौ \textbf{हझं} तुल्यं निःशेषं करिष्यतः~। पुनस्तथाङ्कौ \textbf{मनौ} कल्पितौ यथा \textbf{तं मं} निःशेषं करिष्यति \textbf{कं न}मपि निःशेषं करिष्यति~। तस्मात् \textbf{वतकल}निष्पत्तौ \textbf{हमनझा} जाताः~। \textbf{अजदबा}नामपि निष्पत्तौ च जाताः~। इदमेवास्माकमिष्टम्~॥ 
\vspace{2mm}

\begin{center}
\textbf{\large अथ नवमं क्षेत्रम्~॥~९~॥}
\end{center}

 {\ab यौ द्वौ भिन्नाङ्कौ तयोर्मध्यगा यावन्तोऽङ्का एकनिष्पत्तौ सन्ति तदा रूपतद्द्वयान्यतराङ्कयोर्मध्ये तावन्त एवाङ्का एकनिष्पत्तौ भविष्यन्ति~। }
 
\begin{flushleft}
\begin{minipage}[t]{0.6\textwidth}
\hspace{4mm} यथा \textbf{अबौ} द्वौ भिन्नाङ्कौ कल्पितौ~। अनयोर्मध्ये \textbf{जदा}वङ्कौ कल्पितौ~। एते सर्वे एकनिष्पत्तौ सन्ति~। पुन\textbf{र्हझौ} लघ्वङ्कौ \textbf{अज}निष्पत्तौ गृहीतौ~। पुनस्तस्याम् एव निष्पत्तौ \textbf{वतका} लघवस्त्रयोऽङ्का गृहीताः~। एवं \textbf{लमनसा}स्तस्यामेव \,निष्पत्तौ गृहीताः~। तस्मादेतेऽङ्का
\end{minipage} 
\hfill
\begin{minipage}[t]{0.3\textwidth}
अ,८. ज,१२. द,१८. व,२७.\\
ह,२. झ,३.\\
व,४. त,६. क,९.\\
ल,८. म,१२. न,१८. स,२७.
\end{minipage}
\end{flushleft}
\vspace{-3mm}

\noindent \textbf{अजदब}समाना भविष्यन्ति~। \textbf{हं हे}न गुणितं फलं \textbf{वं} जातम्~। पुन\textbf{र्हव}घातो \textbf{लं} जातम्~। तस्माद्रूपं \textbf{हं} निःशेषं करिष्यति~। \textbf{हा}ङ्को \textbf{वं} निःशेषं करिष्यति~। \textbf{वं लं} तुल्यं निःशेषं करिष्यति~। \textbf{अ}मपि निःशेषं करिष्यति~। तस्मात् रूप\textbf{अ}मध्ये च \textbf{हवौ} एकनिष्पत्तौ द्वावङ्कौ पतितौ~। एवं रूप\textbf{ब}योर्मध्ये \textbf{झका}वङ्कौ एकनिष्पत्तौ पतितौ~। इदमेवास्माकमिष्टम्~॥ 
\vspace{2mm}

\begin{center}
\textbf{\large अथ दशमं क्षेत्रम्~॥~१०~॥}
\end{center}

 {\ab अङ्कद्वयस्य प्रत्येकाङ्करूपयोर्मध्ये एकनिष्पत्तौ यावन्तोऽङ्का पतिष्यन्ति तदा तयोरङ्कयोर्मध्येऽपि तावन्त एवाङ्का एकनिष्पत्तौ पतिष्यन्ति~।} 

\newpage
यथा \textbf{अबा}वङ्कौ कल्पितौ~। \textbf{लं} रूपं कल्पितम्~। \textbf{अल}योर्मध्ये \textbf{जदा}वङ्कावेकनिष्पत्तौ पतितौ यथा\renewcommand{\thefootnote}{१}\footnote{तथा {\en K.}} \textbf{लब}योर्मध्ये \textbf{हझा}वङ्कावेकनिष्पत्तौ कल्पितौ~। तदा \textbf{अब}योर्मध्येऽपि द्वावङ्कावेकनिष्पत्तौ पतिष्यतः~। 

\begin{center}
अस्योपपत्तिः~।
\end{center}

\begin{flushleft}
\begin{minipage}[t]{0.6\textwidth}
\hspace{4mm} \textbf{लज}योर्निष्पत्ति\textbf{र्जद}निष्पत्तिसमानास्ति~। \textbf{लः जं ज}तुल्यं निःशेषयति~। तदा \textbf{जः दं ज}तुल्यं निःशेषं करिष्यति~। तस्मात् \textbf{दं ज}स्य वर्गो भविष्यति~। पुन\textbf{र्लः जं} तथा निःशेषं करोति यथा \textbf{दः अं} निःशेषं करोति~। तदा \,\textbf{जद}घातः \,\textbf{अं} \,भविष्यति~। \,एवं \,हि \,\textbf{झः \,ह}वर्गो
\end{minipage} 
\hfill
\begin{minipage}[t]{0.3\textwidth}
\begin{center}
अ,८. त,१२. क,१८. ब,२७.\\
द,४. व,६. झ,९.\\
ज,२. ह,३.\\
ल,१.
\end{center}
\end{minipage}
\end{flushleft}
\vspace{-3mm}

\noindent भविष्यति~। हझघातो \textbf{बं} भविष्यति~। \textbf{जह}घातश्च \textbf{व}मस्ति~। तदा \textbf{दवझा} एकनिष्पत्तौ भविष्यन्ति~। पुन\textbf{र्जहौ व}गुणितौ कार्यौ~। फलं \textbf{तं कं} भवति~। तस्मात् \textbf{अतकबा} एक-निष्पत्तौ भविष्यन्ति~।  कुतः~। \textbf{जं दवा}भ्यां गुणितं फलं \textbf{अं तं दव}निष्पत्तौ जातम्~। \textbf{जह}निष्पत्तावपि जातम्~। पुन\textbf{र्जहौ व}गुणितौ फलौ \textbf{तक}सञ्ज्ञं तस्यामेव निष्पत्तौ जातम्~। पुन\textbf{र्हं वझ}गुणितं \textbf{कं बं} जातं \textbf{वझ}निष्पत्तौ \textbf{जह}निष्पत्तावपि~। इदमेवास्माकमिष्टम्~॥ 
\vspace{2mm}

\begin{center}
\textbf{\large अथैकादशं क्षेत्रम्~॥~११~॥}
\end{center}

 {\ab यौ द्वौ वर्गौ स्तस्तयोर्मध्ये यदि कोऽप्यङ्कस्तादृशो भवति यथैकनिष्पत्तौ त्रयोऽङ्का भवन्ति तदा तयोर्वर्गयोर्निष्पत्तिर्भुजयोर्निष्पत्तिवर्गो भवति~।} 

\begin{flushleft}
\begin{minipage}[t]{0.7\textwidth}
\hspace{4mm} यथा \textbf{अबौ} वर्गौ कल्पितौ~। अनयोर्भुजौ \textbf{जदौ} कल्पितौ~। \textbf{जद}योर्घातः फलं \textbf{ह}सञ्ज्ञं भवति~। तस्मात् \textbf{अह}निष्पत्ति\textbf{र्जद}-निष्पत्तिसमाना भविष्यति~। एवं \textbf{हब}निष्पत्ति\textbf{र्जद}निष्पत्तिसमाना 
\end{minipage} 
\hfill
\begin{minipage}[t]{0.2\textwidth}
अ,४. ह,६. ब,९.\\
ज,२, द,३.
\end{minipage}
\end{flushleft}
\vspace{-3mm}

\noindent भविष्यति~। तस्मात् \textbf{अब}मध्ये \textbf{हं} पतितम्~। तस्मादेकनिष्पत्तौ \textbf{अहबा} जाताः~। \textbf{अब}निष्पत्तिः 

\newpage
\noindent \textbf{अह}निष्पत्तिवर्गतुल्या \textbf{जद}निष्पत्तिवर्गतुल्या च जाता~। इदमेवास्माकमिष्टम्~॥ 
\vspace{2mm}

\begin{center}
\textbf{\large अथ द्वादशं क्षेत्रम्~॥~१२~॥}
\end{center}

{\ab द्वयोर्घनयोर्मध्ये द्वावङ्कौ यदि तथा पततो यथा चतुर्णामङ्कानामेकनिष्पत्तिर्भवति तदा घनस्य स्वघनेन निष्पत्तिर्भुजनिष्पत्तिघनतुल्या भवति~। }

\begin{flushleft}
\begin{minipage}[t]{0.58\textwidth}
\hspace{4mm} यथा \textbf{अबौ} घनौ कल्पितौ~। \textbf{जदौ} च भुजौ कल्पितौ~। \textbf{जदा}भ्यां \textbf{हझवा}स्त्रयोऽङ्का एकनिष्पत्तौ भविष्यन्ति~। तस्मा\textbf{ज्जह}घातः \textbf{अं} भविष्यति~। \textbf{दव}घा-तश्च \textbf{बं} भविष्यति~। पुन\textbf{र्जदौ झ}गुणितौ कार्यौ फलं
\end{minipage} 
\hfill
\begin{minipage}[t]{0.32\textwidth}
\begin{center}
अ,८. त,१२. क,१८. ब,२७.\\
ह,४. झ,६. व,९.\\
ज,२. द,३.
\end{center}
\end{minipage}
\end{flushleft}
\vspace{-3mm}

\noindent \textbf{तकौ} कल्पितौ~। तस्मात् \textbf{अतकबा अत}निष्पत्तौ \textbf{जद}निष्पत्तावपि भविष्यन्ति~। \textbf{अब}निष्प-त्ति\textbf{र्जद}निष्पत्तिघनतुल्या भविष्यति~। इदमेवास्माकमिष्टम्~॥ 
\vspace{2mm}

\begin{center}
\textbf{\large अथ त्रयोदशं क्षेत्रम्~॥~१३~॥}
\end{center}

{\ab येऽङ्का एकरूपनिष्पत्तौ भवन्ति तेषां वर्गा अप्येकरूपनिष्पत्तौ भवन्ति~। तथा घना अप्येकरूपनिष्पत्तौ भवन्ति~। }

\begin{flushleft}
\begin{minipage}[t]{0.49\textwidth}
\hspace{4mm} यथा ~~\textbf{अबजा}स्त्रयोऽङ्का ~~एकनिष्पत्तौ कल्पिताः~। \textbf{दहझा} एतेषां  वर्गाः कल्पिताः~। \textbf{वतका} \,घनाः \,कल्पिताः~। यदि \,\textbf{अं \,बे}न गुण्यते तदा \,फलं \textbf{ल}सञ्ज्ञं \,भवति~। \textbf{बं जे}न
\end{minipage} 
\hfill
\begin{minipage}[t]{0.42\textwidth}
\begin{center}
अ,२. ब,४. ज,८. \\
द,४. ल,८. ह,१६. म,३२. झ,६४ \\
व,८. न,१६. स,३२. त,६४. ग,१२८\\
फ,२५६. क,५१२. 
\end{center}
\end{minipage}
\end{flushleft}
\vspace{-3mm}

\noindent गुणितं \textbf{मं} भवति~। तस्मात् \textbf{दलहमझा} एतेऽङ्का एकनिष्पत्तौ भविष्यन्ति~। तस्मात् \textbf{दह}योर्निष्पत्ति\textbf{र्हझ}निष्पत्तिसमाना भविष्यति~। तस्मात् वर्गा अप्येकनिष्पत्तौ भविष्यन्ति~। पुनरपि \textbf{अं लहा}भ्यां\renewcommand{\thefootnote}{१}\footnote{हलाभ्यां {\en D.}} गुण्यते तदा \textbf{नसे} फले भवतः~। \textbf{जं हमा}भ्यां

\newpage
\noindent गुण्यते तदा फले \textbf{गफे} भवतः~। तस्मात् \textbf{वनसतगफका} एते सप्ताङ्का एकरूपनिष्पत्तौ भविष्यन्ति~। तस्मात् घना अप्येकरूपनिष्पत्तौ भविष्यन्ति~। इदमेवास्माकमिष्टम्~॥ 
\vspace{2mm}

\begin{center}
\textbf{\large अथ चतुर्दशं क्षेत्रम्~॥~१४~॥}
\end{center}

{\ab ययोर्वर्गयोर्मध्ये एको द्वितीयवर्गं यदि निःशेषं करोति  तदा तस्य भुजोऽपि द्वितीयस्य भुजं निःशेषं करिष्यति~। यद्येकाङ्को द्वितीयाङ्कं निःशेषं करोति तदा तस्य वर्गस्तद्वर्गें निःशेषं करिष्यति~। }\\

 यथा \textbf{अ}वर्गः कल्पितः~। अस्य भुजो \textbf{जः} कल्पितः~। द्वितीयो वर्गो \textbf{बः} कल्पितः~। तस्य भुजो \textbf{दः} कल्पितः~। यदि \textbf{अः बं} निःशेषं करोति तदा \textbf{जः दं} निःशेषं करिष्यति~। 

\begin{center}
अस्योपपत्तिः~।
\end{center}

\begin{flushleft}
\begin{minipage}[t]{0.6\textwidth}
\hspace{4mm} \textbf{जं द}गुणितं \textbf{हं} भवति~। \textbf{अहबा जद}निष्पत्तितुल्या जाताः~। आद्योऽन्त्यं  निःशेषं करोति~। तस्मात् \textbf{अः हं} निःशेषं करिष्यति~। तस्मा\textbf{ज्जं दं} निःशेषं करिष्यति~।
\end{minipage} 
\hfill
\begin{minipage}[t]{0.3\textwidth}
अ,४. ह,८. ब,१६.\\
ज,२. द,४. 
\end{minipage}
\end{flushleft}

\textbf{अहौ जदौ} चैकनिष्पत्तौ स्तः~। यदि \textbf{जः दं} निःशेषं करोति तदा \textbf{अः हं} निःशेषं करिष्यति~। तस्मात् \textbf{अः बं} निःशेषं करिष्यति~। \\

 अस्मादिदं निश्चितं यदि वर्गो वर्गं निःशेषं न करोति तदा भुजो भुजं निःशेषं न करिष्यति~। यद्येकाङ्कोऽन्याङ्कं निःशेषं न करोति तदा तस्य वर्गोऽन्याङ्कवर्गं निःशेषं न करिष्यति~॥ 
\vspace{2mm}

\begin{center}
\textbf{\large अथ पञ्चदशं क्षेत्रम्~॥~१५~॥}
\end{center}

 {\ab यद्येको घनो द्वितीयघनं निःशेषं करोति तदा तस्य भुजो द्वितीयभुजं निःशेषं करिष्यति~। यत्रैकाङ्को द्वितीयाङ्कं निःशेषं करोति तदा तस्य घनोऽपि द्वितीयघनं निःशेषं करोति~। }

\newpage
 यथा \textbf{अं} घनः कल्पितः~। \textbf{जं} भुजः कल्पितः~। \textbf{बः} अन्यघनः कल्पितः~। \textbf{द}स्तस्य भुजः कल्पितः~। यदि \textbf{अः बं} निःशेषं करोति तदा \textbf{जः दं} निःशेषं करिष्यति~। 

\begin{center}
अस्योपपत्तिः~।
\end{center}

 \textbf{जदा}भ्यां \textbf{हवझा}स्त्रयोऽङ्का एकनिष्पत्तावुत्पादिताः~। पुन\textbf{र्जदौ व}गुणितौ फलं \textbf{तं क}म्~। तदा \textbf{अतकबा जद}निष्पत्तावुत्पत्स्यन्ते~। \textbf{अं बं} निःशेषं करोति~। तस्मात् \textbf{अः त}मपि निःशेषं करोति\renewcommand{\thefootnote}{१}\footnote{करिष्यति {\en K.}}\;। \textbf{जः द}मपि निःशेषं करिष्यति~। 

\begin{flushleft}
\begin{minipage}[t]{0.6\textwidth}
\hspace{4mm} पुन\textbf{र्जः दं} निःशेषं कुर्यात्~। तदा \textbf{अः तं} निःशेषं करिष्यति~। तस्मात् \textbf{अः बं} निःशेषं करिष्यति~। इदम् एवास्माकमिष्टम्~॥ 
\end{minipage} 
\hfill
\begin{minipage}[t]{0.3\textwidth}
अ,८. त,१६. क,३२. ब,६४.\\
ह,४. व,८. झ,१६.\\
ज,२. द,४.
\end{minipage}
\end{flushleft}

अस्मादिदं निश्चितं यदि घनो घनं निःशेषं न करोति तदा तस्य भुजोऽन्यभुजं निःशेषं न करिष्यति~। यद्येकाङ्कोऽन्याङ्कं निःशेषं न करोति तदा तस्य घनो द्वितीयघनं निःशेषं न करिष्यति~॥ 
\vspace{2mm}

\begin{center}
\textbf{\large अथ षोडशं क्षेत्रम्~॥~१६~॥}
\end{center}

{\ab  ययोः सजातीयघातफलाङ्कयोर्मध्ये यद्येकाङ्कस्तथा पतति\renewcommand{\thefootnote}{२}\footnote{तथा एकाङ्को यदि पतति {\en K.}} यथैतत्त्रयमेकनिष्पत्तौ भवति तदा घातफलयोर्निष्पत्तिर्या भवति सा सजातीयतद्भुजनिष्पत्तिवर्गतुल्या भवति~। }

\begin{flushleft}
\begin{minipage}[t]{0.65\textwidth}
\hspace{4mm}  यथा सजातीयघातफले \textbf{अब}कल्पिते~। \textbf{अ}भुजौ \textbf{जदौ} कल्पितौ~। \textbf{ब}भुजौ  \textbf{हझौ} कल्पितौ~। \textbf{जह}निष्पत्ति\textbf{र्दझ}-निष्पत्तितुल्या भविष्यति~। यदि \textbf{दं ह}गुणितं \textbf{व}मुत्पन्नमिति कल्प्यते तदा  \textbf{अवबा} एकनिष्पत्तौ भविष्यन्ति~।
\end{minipage} 
\hfill
\begin{minipage}[t]{0.28\textwidth}
अ,६. व,१२. ब,२४.\\
ज,२. द,३. ह,४. झ,६.
\end{minipage}
\end{flushleft}

\newpage
\begin{center}
अत्रोपपत्तिः~। 
\end{center}

\textbf{दं जहा}भ्यां गुणितं फले \textbf{अवे} जाते~। अनयोर्निष्पत्ति\textbf{र्जह}निष्पत्तितुल्या भविष्यति~। पुन\textbf{र्हं दझा}भ्यां गुणितं \textbf{वबे} उत्पन्ने~। अनयोर्निष्पत्ति\textbf{र्दझ}निष्पत्तितुल्या भविष्यति~। \textbf{जह}-निष्पत्तितुल्यापि भविष्यति~। \textbf{अब}निष्पत्तिः \textbf{अव}निष्पत्तिवर्गतुल्यास्ति~। \textbf{जह}निष्पत्तिवर्गतु-ल्यापि भविष्यति~। इदमेवास्माकमिष्टम्\renewcommand{\thefootnote}{१}\footnote{इदमेवास्मदिष्टम् {\en K.}}\;॥~१६~॥ 
\vspace{2mm}

\begin{center}
\textbf{\large अथ सप्तदशं क्षेत्रम्~॥~१७~॥}
\end{center}

 {\ab सजातीययोर्घनफलयोर्मध्ये तादृशौ द्वावङ्कौ यदि तथा पततो यथा चतुर्णामङ्कानामेकनिष्पत्तिर्भवति घनफलस्य निष्पत्तिर्घनफलेन या भवति सा सजातीयभुजनिष्पत्तिघनतुल्या भवति~। }

\begin{flushleft}
\begin{minipage}[t]{0.55\textwidth}
\hspace{4mm} यथा \textbf{अबे} सजातीये घनफले कल्पिते~। \textbf{अ}भुजा \textbf{जदहाः} कल्पिताः~। \textbf{ब}भुजा \textbf{झवताः} कल्पिताः~। \textbf{जझ}निष्पत्ति\textbf{र्दव}निष्पत्तितुल्यास्ति~। \textbf{हत}निष्पत्तितुल्याप्यस्ति~। \textbf{जं द}गुणितं \textbf{क}मुत्प-न्नम्~। \textbf{झं \,व}गुणितं \,\textbf{ल}मुत्पन्नम्~। तस्मात् \textbf{कलौ}
\end{minipage} 
\hfill
\begin{minipage}[t]{0.35\textwidth}
अ,३०. न,६०. स,१२०. ब,२४०.\\
क,६. म,१२. ल,२४.\\
ज,२. द,३. ह,५.\\
झ,४. ब,६. त,१०.
\end{minipage}
\end{flushleft}
\vspace{-3mm}

\noindent सजातीयौ घातफलाङ्कौ भविष्यतः~। अनयोर्मध्ये \textbf{मः} अङ्कः पतति तदा \textbf{कमला}स्त्रयोऽङ्का \textbf{जझ}निष्पत्तौ पतिष्यन्ति~। पुन\textbf{र्हतौ म}गुणितौ \textbf{नसा}वुत्पन्नौ~। एतयोर्निष्पत्ति\textbf{र्हत}निष्पत्तितुल्या भविष्यति~। \textbf{जझ}निष्पत्तितुल्यापि भविष्यति~। अनयोर्निष्पत्तिः \textbf{कमल}निष्पत्तितुल्यास्ति~। \textbf{जझ}निष्पत्तितुल्याप्यस्ति~। तस्मात् \textbf{अनसबा}श्चत्वारोऽङ्का \textbf{जझ}निष्पत्तौ भविष्यन्ति~। \textbf{अब}-निष्पत्तिः \textbf{अन}निष्पत्तिघनतुल्यास्ति~। \textbf{जझ}निष्पत्तिघनतुल्या भविष्यति~। इदमेवास्माकम् इष्टम्~॥ 
\vspace{2mm}

\begin{center}
\textbf{\large अथाष्टादशं क्षेत्रम्~॥~१८~॥}
\end{center}

{\ab द्वयोरङ्कयोर्मध्ये कश्चिदङ्कः पतति~। यद्येतेऽङ्का एकनिष्पत्तौ भवन्ति तदा तौ द्वावङ्कौ सजातीयघातफले भविष्यतः~।} 

\newpage
\begin{flushleft}
\begin{minipage}[t]{0.62\textwidth}
\hspace{4mm} यथा \textbf{अब}योर्मध्ये \textbf{जः} कल्पितः~। एते त्रयोऽपि एक-निष्पत्तौ कल्पिताः~। पुनर्लघ्वङ्कावस्यां निष्पत्तौ \textbf{दहौ} ग्राह्यौ~। एतौ \,\textbf{अजौ} तुल्यं \,निःशेषं करिष्यतः~। पुन\textbf{र्दः}
\end{minipage} 
\hfill
\begin{minipage}[t]{0.28\textwidth}
अ,८. ज,१२. ब,१८. \\
द,२. ह,३. झ,४. व,६.
\end{minipage}
\end{flushleft}
\vspace{-3mm}

\noindent \textbf{अं झ}तुल्यं निःशेषं करोति~। \textbf{हः बं व}तुल्यं निःशेषं करोतीत्यपि कल्पितम्~। तस्मात् \textbf{दझ}घातः \textbf{अं} भविष्यति~। \textbf{हव}घातो \textbf{बं} भविष्यति~। तस्मात् \textbf{अबौ} घातौ भविष्यतः~। पुनरपि \textbf{दव}घातो \textbf{ज}मस्ति~। \textbf{हझ}घातोऽपि \textbf{ज}मस्ति~। तस्मात् \textbf{दह}निष्पत्ति\textbf{र्झव}निष्पत्तिसमाना भविष्यति~। तस्मात् \textbf{अबौ} सजातीयघातफले भविष्यतः~। इदमेवास्माकमिष्टम्~॥ 
\vspace{2mm}

\begin{center}
\textbf{अथोनविंशतितमं क्षेत्रम्~॥~१९~॥}
\end{center}

 {\ab द्वयोरङ्कयोर्मध्ये द्वावङ्कौ पततः~। यद्येते चत्वारोऽप्यङ्का एकनिष्पत्तौ भवन्ति तदा तौ द्वावङ्कौ सजातीयघनफलाङ्कौ भविष्यतः~। }\\

 यथा \textbf{अब}योर्मध्ये \textbf{जदौ} पतितौ~। \textbf{अजदबा} एते चत्वारो यद्येकनिष्पत्तौ भवन्ति तदा \textbf{अबौ} सजातीयघनफलाङ्कौ भविष्यतः~। 

\begin{center}
अस्योपपत्तिः~।
\end{center}
\vspace{-3mm}

\begin{flushleft}
\begin{minipage}[t]{0.55\textwidth}
\hspace{4mm} \textbf{हझवा}स्त्रयो \,लघ्वङ्का \,\textbf{अज}निष्पत्तौ \,गृहीताः~। तस्मात् \,\textbf{हवौ} \,सजातीयघातफलाङ्कौ \,भविष्यतः~। \textbf{ह}स्य \,भुजौ \,कलौ  \,कल्पितौ~। \textbf{व}स्य \,भुजौ \,\textbf{मनौ} कल्पितौ~। तस्मात्  \textbf{कम}निष्पत्ति\textbf{र्लन}निष्पत्तिस-माना \;भविष्यति~। \,\textbf{हझ}निष्पत्तिसमानापि \;भवि-
\end{minipage} 
\hfill
\begin{minipage}[t]{0.35\textwidth}
अ,२४. ज,७२. द,२१६. ब,६४८. \\
त,२४. स,७२.\\
ह,१. झ,३. व,९.\\
क,१. ल,१. म,३. न,३.
\end{minipage}
\end{flushleft}
\vspace{-3mm}

\noindent ष्यति~। \textbf{हझव}म् \textbf{अजद}निष्पत्तावस्ति~। तस्मात् \textbf{हझव}म् \textbf{अजदं} तुल्यं निःशेषं करिष्यति~। कल्पितं \textbf{त}तुल्यं निःशेषं करोति~। एवं हि \textbf{हझवा जदब}निष्पत्तौ सन्ति~। तस्मात् \textbf{हझवा जदबं} तुल्यं निःशेषं करिष्यन्ति~। कल्पितं च \textbf{स}तुल्यं 

\newpage
\noindent निःशेषं करोति~। तस्मात् \textbf{हत}घातः \textbf{त}गुणित\textbf{कल}घाततुल्यः \textbf{अं} कल्पितम्~। \textbf{वस}घातफलं \textbf{ब}म्~। तत् \textbf{स}गुणित\textbf{मन}घाततुल्यमस्ति~। तस्मात् \textbf{अबौ} घनफलाङ्कौ जातौ~। पुन\textbf{स्तसौ व}गुणितौ फले \textbf{दबौ} भवतः~। तस्मा\textbf{त्तसौ दब}निष्पत्तौ जातौ~। \textbf{कम}निष्पत्तावपि~। तस्मात् \textbf{अबौ} सजातीयघनफलाङ्कौ जातौ~। इदमेवास्माकमिष्टम्~॥ 
\vspace{2mm}

\begin{center}
\textbf{\large अथ विंशतितमं क्षेत्रम्~॥~२०~॥ }
\end{center}

{\ab तत्र ये त्रयोऽङ्का एकनिष्पत्तौ\renewcommand{\thefootnote}{१}\footnote{एकरूपनिष्पत्तौ {\en K.}} यदि भवन्ति तत्र प्रथमाङ्कौ वर्गो यदि भवति तदा तृतीयाङ्कोऽपि वर्गो भविष्यति~। }

\begin{flushleft}
\begin{minipage}[t]{0.65\textwidth}
\hspace{4mm} यथा \,\textbf{अबजा}स्त्रयोऽङ्का \,एकनिष्पत्तौ \,कल्पिताः~। \,\textbf{अं} वर्गोऽस्ति~। तदा \textbf{ज}मपि वर्गो भविष्यति~। कुतः~। \textbf{दहझा} लघ्वङ्का \,\textbf{अबज}निष्पत्तौ \,गृहीताः~। \,तस्मात् \,\textbf{दझौ} \,वर्गौ भविष्यतः~।  पुनः \textbf{व}म् \textbf{अ}भुजः कल्पितः~। तं \textbf{द}भुजः कल्पितः~। \textbf{कं झ}भुजः कल्पितः~। तस्मात् \textbf{दझ}निष्पत्तिः
\end{minipage} 
\hfill
\begin{minipage}[t]{0.25\textwidth}
अ,१६. य,२४. ज,३६.\\
द,४. ह,६. झ,९.\\
व,४. क,३.\\
त,२. ल,६.
\end{minipage}
\end{flushleft}
\vspace{-3mm}

\noindent \textbf{अज}निष्पत्तिसमाना भविष्यति~। \textbf{दझौ} भिन्नाङ्कौ स्तः~। तस्मादेतौ \textbf{अजं} निःशेषं करिष्यतः~। यदि वर्गो वर्गं निःशेषं करोति तदा भुजो भुजं निःशेषं करिष्यति~। तस्मात् \textbf{तं वं} निःशेषं करिष्यति~। पुनः \textbf{कं लं} तथा निःष्पत्तिः \textbf{कल}निष्पत्तिसमाना भविष्यति~। \textbf{त}वर्ग\textbf{व}वर्गयोर्निष्पत्तिः \textbf{क}वर्ग\textbf{ल}वर्गयोर्निष्पत्तितुल्या भविष्यति~। \textbf{त}वर्गो \textbf{द}मस्ति~। \textbf{व}वर्गः \textbf{अ}मस्ति~। \textbf{क}वर्गः \textbf{झ}मस्ति~। \textbf{दअ}निष्पत्ति\textbf{र्झज}निष्पत्तिसमानास्ति~। तस्मात् \textbf{जं ल}वर्गो भविष्यति~। इदमेवास्माकमिष्टम्~॥ 
\vspace{2mm}

\begin{center}
\textbf{\large अथैकविंशतितमं क्षेत्रम्~॥~२१~॥}
\end{center}

{\ab  ये चत्वारोऽङ्का एकनिष्पत्तौ भवन्ति तेषां मध्ये प्रथमाङ्कश्चेत् घनो भवति तदा चतुर्थाङ्कोऽपि घनो भविष्यति~। }

\newpage
यथा \textbf{अबजदा}श्चत्वारोऽङ्का एकनिष्पत्तौ कल्पिताः~। \textbf{अः} घनः कल्पितः~। तदा \textbf{दो}ऽपि घनो भविष्यति~। 

\begin{center}
अस्योपपत्तिः~।
\end{center}

\begin{flushleft}
\begin{minipage}[t]{0.55\textwidth}
\hspace{4mm} \textbf{हझवता}श्चत्वारो \,लघ्वङ्का \,\textbf{अबजद}निष्पत्तौ ग्राह्याः~। तस्मात् \textbf{हतौ} घनौ भविष्यतः~। \textbf{अ}भुजो \textbf{लं ह}भुजः \textbf{कं त}भुजो \textbf{नं} कल्पितः~। तदा \textbf{हत}निष्पत्तिः \textbf{अद}निष्पत्तिसमानास्ति~। \textbf{हतौ} च भिन्नाङ्कौ स्तः~। तस्मात् \textbf{हतौ} \textbf{अदौ} निःशेषं करि-
\end{minipage} 
\hfill
\begin{minipage}[t]{0.35\textwidth}
अ,६४. ब,९६. ज,१४४. द,२१६.\\
{\color{white}अ} ल,४.\\
ह,८. झ,१२. व,१८. त,२७.\\ 
क,२. न,३. स,६.
\end{minipage}
\end{flushleft}
\vspace{-3mm}

\noindent ष्यतः~। यदि \textbf{हं} घनः \textbf{अ}सञ्ज्ञघनं निःशेषं करोति तदा \textbf{क}भुजो \textbf{ल}भुजं निःशेषं करिष्यति~। पुनः कल्पितं \textbf{नः सं} तथा निःशेषं करोति यथा \textbf{कः लं} निःशेषं करोति~। तस्मात् \textbf{कल}निष्पत्ति\textbf{र्नस}निष्पत्तेः समाना भविष्यति~। \textbf{कल}घनयोर्निष्पत्ति\textbf{र्नस}घनयोर्निष्पत्तिसमाना भविष्यति~। \textbf{क}स्य घनो \textbf{हं ल}घनः \textbf{अं न}घनः \textbf{त}म्~। \textbf{हअ}निष्पत्ति\textbf{स्तद}निष्पत्तिसमानास्ति~। तस्मात् \textbf{दः स}घनो भविष्यति~। इदमेवास्माकमिष्टम्\renewcommand{\thefootnote}{१}\footnote{इदमेवास्मदिष्टम् {\en K.}}\,॥
\vspace{2mm}

\begin{center}
\textbf{\large अथ द्वाविंशतितमं क्षेत्रम्~॥~२२~॥} 
\end{center}

{\ab यावङ्कौ वर्गद्वयनिष्पत्तौ स्तस्तयोर्मध्ये यद्येकाङ्को वर्गो 
भवति तदा द्वितीयाङ्कोऽपि वर्गो भविष्यति~। }\\

 यथा \textbf{अबौ जद}वर्गयोर्निष्पत्तौ कल्पितौ~। यदि \textbf{अः} वर्गो भवति तदा \textbf{ब}मपि वर्गो भविष्यति~। 

\begin{center}
अस्योपपत्तिः~।
\end{center}

\begin{flushleft}
\begin{minipage}[t]{0.75\textwidth}
\hspace{4mm} \textbf{जदौ} वर्गौ स्तः~। अनयोर्मध्ये तथा एकाङ्कः पतिष्यति यथैतत्त्रयमेकनिष्पत्तौ भविष्यति~। एवम् \textbf{अब}योर्मध्ये एकाङ्को भविष्यति~। एते \,त्रयोऽङ्का \,एकनिष्पत्तौ \,पतिष्यन्ति~। \textbf{अः} वर्गोऽस्ति~। तस्मात् \textbf{बः} वर्गो भविष्यति~। इदमेवास्मदिष्टम्~॥
\end{minipage} 
\hfill
\begin{minipage}[t]{0.15\textwidth}
अ,४. ब,९.\\
ज,१६. द,३६
\end{minipage}
\end{flushleft}

\newpage
\begin{center}
\textbf{\large अथ त्रयोविंशतितमं क्षेत्रम्~॥~२३~॥}
\end{center}

 {\ab यौ \,द्वावङ्कौ \,घननिष्पत्तौ \,भविष्यतस्तयोर्मध्ये \,यद्येको \,घनो \,भवति \,तदा \,द्वितीयोऽपि \,घनो \,भविष्यति~। }\\

 यथा \textbf{अबौ जद}घनयोर्निष्पत्तौ कल्पितौ~। तयोर्यदि \textbf{अं} घनस्तदा \textbf{बा}ङ्कोऽपि घनो भविष्यति~।

\begin{center}
अस्योपपत्तिः~। 
\end{center}

\begin{flushleft}
\begin{minipage}[t]{0.65\textwidth}
\hspace{4mm}  \textbf{जदौ} घनौ स्तः~। अनयोर्मध्ये तथा द्वावङ्कौ पतिष्यतो यथैते चत्वारोऽङ्का एकनिष्पत्तौ भविष्यन्ति~। एवं हि \textbf{अब}योर्मध्ये द्वावङ्कौ तथा पतिष्यतो यथैतेऽपि चत्वारोऽङ्कौ 
\end{minipage} 
\hfill
\begin{minipage}[t]{0.25\textwidth}
अ,८. ब,२७.\\
ज,६४. द,२१६.
\end{minipage}
\end{flushleft}
\vspace{-3mm}

\noindent एकनिष्पत्तौ स्युः~। \textbf{अः} घनोऽस्ति~। तस्मात् \textbf{बः} घनो जातः~। इदमेवास्माकमिष्टम्~॥~२३~॥ 
\vspace{2mm}

\begin{center}
\textbf{\large अथ चतुर्विंशतितमं क्षेत्रम्~॥~४~॥}
\end{center}

{\ab यावङ्कौ द्वयोर्वर्गयोर्निष्पत्तौ भवतस्तदैतौ घातफलाङ्कौ सजातीयौ भवतः\renewcommand{\thefootnote}{१}\footnote{भविष्यतः {\en K.}}\;। }\\

यथा \textbf{अबौ जद}वर्गयोर्निष्पत्तौ कल्पितौ~। \textbf{अबौ} सजातीयौ घातफलाङ्कौ भविष्यतः~। 

\begin{center}
अस्योपपत्तिः~। 
\end{center}

\begin{flushleft}
\begin{minipage}[t]{0.65\textwidth}
\hspace{4mm} \textbf{जद}योर्मध्ये तथैकाङ्कः पतिष्यति यथैते त्रयोऽप्येकनिष्पत्तौ भविष्यन्ति~। एवम्  \textbf{अब}मध्येऽपि~। तस्मात् \textbf{अबौ} सजातीयौ घातफलाङ्कौ भविष्यतः~॥
\end{minipage} 
\hfill
\begin{minipage}[t]{0.25\textwidth}
अ,१८. ब,३२.\\
ज,९. द,१६.
\end{minipage}
\end{flushleft}
\vspace{-1mm}

\begin{center}
\textbf{\large अथ पञ्चविंशतितमं क्षेत्रम्~॥~२५~॥}
\end{center}

{\ab यावङ्कौ द्वयोर्घनयोर्निष्पत्तौ स्तस्तदा तावङ्कौ सजातीयघनफलाङ्कौ भविष्यतः~। }
\vspace{-1mm}

\begin{center}
अस्योपपत्तिः~। 
\end{center}
\vspace{-3mm}

\begin{flushleft}
\begin{minipage}[t]{0.65\textwidth}
\hspace{4mm} क्षेत्रन्यासश्च पूर्वोक्तवत् ज्ञेयः~॥
\end{minipage} 
\hfill
\begin{minipage}[t]{0.25\textwidth}
\vspace{-6mm}

अ,१६. ब,५४ \\
ज,८. द,२७ 
\end{minipage}
\end{flushleft}

\newpage
\begin{center}
\textbf{\large अथ षड्विंशतितमं क्षेत्रम्~॥~२६~॥ }
\end{center}

{\ab  यौ घातफलाङ्कौ सजातीयौ भवतस्तौ द्वयोर्वर्गयोर्निष्पत्तौ भवतः~। }\\

 यथा \textbf{अबौ} घातफलाङ्कौ सजातीयौ कल्पितौ~। एतौ द्वयोर्वर्गयोर्निष्पत्तौ भविष्यतः~। 

\begin{center}
अस्योपपत्तिः~।
\end{center}

\begin{flushleft}
\begin{minipage}[t]{0.6\textwidth}
\hspace{4mm} एकाङ्को \textbf{ज}सञ्ज्ञकः \textbf{अब}योर्मध्ये पतिष्यति~। एते त्रयोऽप्यङ्का एकरूपनिष्पत्तौ  भविष्यन्ति~। यदि \textbf{दहझा}-स्त्रयो लघ्वङ्का \textbf{अजब}निष्पत्तौ गृह्यन्ते तदा \textbf{अब}निष्प-
\end{minipage} 
\hfill
\begin{minipage}[t]{0.3\textwidth}
अ,६. ज,१२. ब,२४.\\
 द,१. ह,२. झ,४.
\end{minipage}
\end{flushleft}
\vspace{-3mm}

\noindent त्ति\textbf{र्दझ}वर्गयोर्निष्पत्तिसमाना भविष्यति~। इदमेवास्माकमिष्टम्~॥ 
\vspace{2mm}

\begin{center}
\textbf{\large अथ सप्तविंशतितमं क्षेत्रम्~॥~२७~॥} 
\end{center}

{\ab  यौ घनफलाङ्कौ सजातीयौ भवतस्तौ द्वयोर्घनयोर्निष्पत्तौ भविष्यतः~। }

\begin{center}
अस्योपपत्तिः~।
\end{center}

\begin{flushleft}
\begin{minipage}[t]{0.55\textwidth}
\hspace{4mm} \textbf{जदौ अब}योर्मध्ये पतितौ~। एते चत्वार एकनिष्पत्तौ भविष्यन्ति~।  पुनर्यदि \textbf{हझवता}श्चत्वारोऽङ्का \textbf{अजदबा}नां निष्पत्तौ लघवो गृह्यन्ते तदा अबनि-
\end{minipage} 
\hfill
\begin{minipage}[t]{0.35\textwidth}
अ,१६. ज,२४. द,३६. ब,५४.\\
ह,८. झ,१२. व,१८. त,२७.
\end{minipage}
\end{flushleft}
\vspace{-3mm}

\noindent ष्पत्तिर्हतघनयोर्निष्पत्त्या समाना भविष्यति~। इदमेवास्माकमिष्टम्~॥~२७~॥ 
\vspace{2mm}

\begin{quote}
\qt
श्रीमद्राजाधिराजप्रभुवरजयसिंहस्य तुष्ट्यै द्विजेन्द्रः \\
श्रीमत्सम्राड् जगन्नाथ इति समभिधारूढितेन प्रणीते~। \\
ग्रन्थेऽस्मिन्नाम्नि रेखागणित इति सुकोणावबोधप्रदात- \\
र्यध्यायोऽध्येतृमोहापह इह विरतिं चाष्टमः सङ्गतोऽभूत्~॥~८~॥ 
\end{quote}
\vspace{-1mm}

\begin{center}
इति श्रीजगन्नाथसम्राड्विरचिते रेखागणिते \\
 अष्टमोऽध्यायः समाप्तः~॥~८~॥ \\
\vspace{6mm} 

\rule{0.9in}{0.8pt}
 \end{center}

\afterpage{\fancyhead[CE] {रेखागणितम्}}
\afterpage{\fancyhead[CO] {नवमोऽध्यायः}}
\afterpage{\fancyhead[LE,RO]{\thepage}}
\cfoot{}

\newpage
%%%%%%%%%%%%%%%%%%%%%%%%%%%%%%%%%%%%%%%%%%%%%%%%%%%%%%%%%%%%%%
\newpage
\thispagestyle{empty}
\phantomsection \label{ch9}
\begin{center}
{\bf \LARGE~॥ अथ नवमोऽध्यायः प्रारभ्यते~॥}
\vspace{5mm}

\textbf{~॥ तत्राष्टत्रिंशत् क्षेत्राणि सन्ति~॥  }
\vspace{5mm}

\textbf{\large अथ प्रथमं क्षेत्रम्~॥~१~॥}
\end{center}

{\ab द्वयोः सजातीयघातफलाङ्कयोर्घातो वर्गो भवति~।}\\

यथा \textbf{अबौ} सजातीयघातफलाङ्कौ कल्पितौ~। \textbf{अब}घातो \textbf{जः} कल्पितः~। असौ वर्गो जातः~॥ 

\begin{center}
अस्योपपत्तिः~।
\end{center}

\begin{flushleft}
\begin{minipage}[t]{0.7\textwidth}
\hspace{4mm} यदि \textbf{अ}वर्गो \textbf{दं} कल्पितस्तदा \textbf{अब}निष्पत्ति\textbf{र्दज}निष्पत्तितुल्या भविष्यति~। तत्र प्रत्येकाङ्कयोर्मध्ये एकाङ्कस्तथा पतिष्यति यथा त्रयोऽङ्का एकनिष्पत्तौ पतिष्यन्ति~। \textbf{दं} वर्गोऽस्ति~। तस्मात् \textbf{जं} वर्गोऽपि भविष्यति~। इदमेवेष्टम्~॥
\end{minipage} 
\hfill
\begin{minipage}[t]{0.2\textwidth}
अ,६. ब,५४.\\ 
द,३६. ज,३२४.
\end{minipage}
\end{flushleft}
\vspace{-1mm}

\begin{center}
\textbf{\large अथ द्वितीयं क्षेत्रम्~॥~२~॥}
\end{center}

{\ab ययोरङ्कयोर्घातो वर्गो भवति तावङ्कौ सजातीयघातफलाङ्कौ भविष्यतः~। }\\

 यथा \textbf{अब}योर्घातो \textbf{ज}वर्गः कल्पितः~। एतौ सजातीयघातफलाङ्कौ भविष्यतः~। 

\begin{center}
अस्योपपत्तिः~।
\end{center}
\vspace{-3mm}

\begin{flushleft}
\begin{minipage}[t]{0.7\textwidth}
\hspace{4mm} \textbf{अ}वर्गो \textbf{दः} कल्पितः~। \textbf{दज}वर्गयोर्निष्पत्तिः \textbf{अब}निष्पत्ति-तुल्यास्ति~। एतौ सजातीयघातफलाङ्कौ भविष्यतः~॥ \\
\end{minipage} 
\hfill
\begin{minipage}[t]{0.2\textwidth}
अ,४. ब,९.\\
द,१६. ज,३६
\end{minipage}
\end{flushleft}
\vspace{-3mm}

अनेन क्षेत्रेणेदं निश्चितम्~।\\
\vspace{-1mm}

वर्गो वर्गगुणितो वर्गो भवति~। अवर्गगुणितो वर्गोऽवर्गो भवति~। येन गुणितो वर्गो वर्गो भवति स चाङ्कोऽपि वर्ग एव भविष्यति~। यदि वर्गो न भवति तदा सोऽप्यङ्कोऽवर्ग एव~॥ 
\vspace{2mm}

\begin{center}
\textbf{\large अथ तृतीयं क्षेत्रम्~॥~३~॥}
\end{center}

{\ab  घनवर्गो घनो भवति~। }

\newpage

\begin{flushleft}
\begin{minipage}[t]{0.75\textwidth}
\hspace{4mm}  यथा \textbf{अः} घनः कल्पितः~। अस्य वर्गो \textbf{बः} कल्पितः~। \textbf{जः} भुजः  कल्पितः~। भुजवर्गो \textbf{दः} कल्पितः~। रूप\textbf{अ}प्रमाणयोर्मध्ये \textbf{जदौ} तथा पतितौ यथैते चत्वारोऽङ्का एकनिष्पत्तौ पतिष्यन्ति~। रूप\textbf{अ}प्रमाणनिष्पत्तिः \textbf{अब}निष्पत्तितुल्यास्ति~। तस्मात् \textbf{अब}योर्मध्ये 
\end{minipage} 
\hfill
\begin{minipage}[t]{0.15\textwidth}
अ,८.\\
व,१६. द,४.\\
क,३२. ज,२. \\
ब,६४.\\
\end{minipage}
\end{flushleft}
\vspace{-7mm}

\noindent तथा \textbf{वकौ} पतिष्यतो यथैते चत्वार एकनिष्पत्तौ भविष्यन्ति~। \textbf{अं} घनोऽस्ति~। तस्मात् \textbf{ब}मपि घनो भविष्यति~। इदमेवास्माकमिष्टम्~॥ 
\vspace{2mm}

\begin{center}
\textbf{\large अथ चतुर्थ क्षेत्रम्~॥~४~॥}
\end{center}

 {\ab घनयोर्घातो घनो भवति~। }

\begin{flushleft}
\begin{minipage}[t]{0.7\textwidth}
\hspace{4mm}  यथा \textbf{अबौ} घनौ कल्पितौ~। अनयोर्घातो \textbf{जः} कल्पितः~। असा-वपि घनो भविष्यति~। कुतः~। \textbf{अ}वर्गो \textbf{दः} कृतः~। अयं घनो भविष्यति~। \textbf{अब}घनयोर्निष्पत्ति\textbf{र्दज}निष्पत्तिसमाना भवि-
\end{minipage} 
\hfill
\begin{minipage}[t]{0.2\textwidth}
अ,८. ब,२७.\\
द,६४. ज,२१६.
\end{minipage}
\end{flushleft}
\vspace{-3mm}

\noindent ष्यति~। \textbf{दः} घनोऽस्ति~। तस्मात् \textbf{जो}ऽपि घनो भविष्यति~। इत्यस्माकमिष्टम्~॥
\vspace{2mm}

\begin{center}
\textbf{\large अथ पञ्चमं क्षेत्रम्~॥~५~॥ }
\end{center}

{\ab घनः केनाप्यङ्केन गुणितः सन् घनो भवति तदासावङ्कोऽपि घनो भवति~। }\\

 यथा \textbf{अः} घनो \textbf{ब}गुणितो \textbf{जं} घनो जातः~। तस्मात् \textbf{बः} घनो भविष्यति~। 
 
\begin{center}
अस्योपपत्तिः~।
\end{center}

\begin{flushleft}
\begin{minipage}[t]{0.7\textwidth}
\hspace{4mm} \textbf{अ}प्रमाणस्य वर्गो \textbf{दं} घनो भविष्यति~। \textbf{अब}योर्निष्पत्ति\textbf{र्दज}-घनयोर्निष्पत्तितुल्या भविष्यति~। \textbf{अ}प्रमाणं घनोऽस्ति~। तस्मात् \textbf{बः} घनो भविष्यति~। इदमेवास्माकमिष्टम्~॥ 
\end{minipage} 
\hfill
\begin{minipage}[t]{0.2\textwidth}
अ,८. ब,२७,\\
द,६४. ज,२१६.
\end{minipage}
\end{flushleft}
\vspace{-2mm}

 अनेनेदं निश्चितम्~।\\
\vspace{-1mm}
 
 घनोऽघनगुणोऽघन एव भवति~। यदि घनः केनाप्यङ्केन गुणोऽघनो भवति तदा सोऽप्यङ्कोऽघनो भविष्यति~॥ 

\newpage
\begin{center}
\textbf{\large अथ षष्ठं क्षेत्रम्~॥ ६~॥ }
\end{center}

{\ab यस्याङ्कस्य वर्गो घनो भवति स घनो भविष्यति~।} 

\begin{flushleft}
\begin{minipage}[t]{0.6\textwidth}
\hspace{4mm} यथा \textbf{अं} अङ्कः कल्पितः~। अस्य वर्गो \textbf{बं} घनः कल्पितः~। तस्मात् \textbf{अ}मपि घनो भविष्यति~।
\end{minipage} 
\hfill
\begin{minipage}[t]{0.3\textwidth}
अ,८. ब,६४. ज,५१२.
\end{minipage}
\end{flushleft}
\vspace{-3mm}

\begin{center}
अस्योपपत्तिः~।
\end{center}

यदि \textbf{अं बे}न गुण्यते \textbf{जं} घनो भविष्यति~। \textbf{अब}योर्निष्पत्ति\textbf{र्बज}घननिष्पत्तितुल्या भविष्यति~। तस्मात् \textbf{अं} घनो भविष्यति~। इदमेवास्माकमिष्टम्~॥ 
\vspace{2mm}

\begin{center}
\textbf{\large अथ सप्तमं क्षेत्रम्~॥~७~॥}
\end{center}

{\ab योगाङ्कः केनचिदङ्केन गुणितः सन् घनफलाङ्को भवति~।}

\begin{flushleft}
\begin{minipage}[t]{0.65\textwidth}
\hspace{4mm} यथा \textbf{अं} योगसञ्ज्ञाकः कल्पितः~। एनं \textbf{दः ह}तुल्यं निःशेषं करोति~। तस्मात् \textbf{अं दह}घातफलं भविष्यति~। एतत् \textbf{बे}न गुण्यते तदा \textbf{जं} भविष्यति~। इदं \textbf{जं} घनफलाङ्को
\end{minipage} 
\hfill
\begin{minipage}[t]{0.25\textwidth}
अ,६. ब,७. ज,४२. \\
द,३. ह,२.
\end{minipage}
\end{flushleft}
\vspace{-3mm}

\noindent भविष्यति~। कुतः~। \textbf{दं ह}गुणितं \textbf{अं} जातम्~। पुनः \textbf{अं ब}गुणितं \textbf{जं} जातम्~। तस्मात् \textbf{जः} घनफलाङ्को जातः~। इदमेवास्माकमिष्टम्~॥
\vspace{2mm}

\begin{center}
\textbf{\large अथाष्टमं क्षेत्रम्~॥~८~॥}
\end{center}

{\ab रूपादयोऽङ्का एकनिष्पत्तौ यावन्तः स्युः रूपादेकान्तरितास्तृतीयादयोऽङ्का वर्गाः स्युः~। रूपाद्द्व्यन्तरिताश्चतुर्थादयो घना भवन्ति~। रूपात्पञ्चान्तरिताः सप्तादयो वर्गा घनाश्च भवन्ति~। }

\begin{flushleft}
\begin{minipage}[t]{0.63\textwidth}
\hspace{4mm} यथा रूपादयः \textbf{अबजदहझा} एकनिष्पत्तौ कल्पिताः~। तस्मात्  \textbf{बः} वर्गो भविष्यति~। कुतः~। यतो रूपं \textbf{अं} तथा निःशेषं \,करोति \,यथा \,\textbf{अं बं} \,निःशेषं \,करोति~। तस्मात्
\end{minipage} 
\hfill
\begin{minipage}[t]{0.27\textwidth}
१. अ,३. ब,९. ज,२७.\\
द,८१. ह,२४३. झ,७२९.
\end{minipage}
\end{flushleft}
\vspace{-3mm}

\noindent  \textbf{अ}वर्गो \textbf{बः} भविष्यति~। अनेनैव प्रकारेण \textbf{दं} वर्गो भविष्यति~। पुन\textbf{र्जः} घनोऽस्ति~। 

\newpage
\noindent कुतः~। \textbf{अब}घातोत्पन्नत्वात्~। एवं हि \textbf{झो}ऽपि घनः~। कुतः~। यतो रूपनिष्पत्ति\textbf{र्जे}न तथास्ति यथा \textbf{ज}निष्पत्ति\textbf{र्झे}नास्ति~। तस्मात् \textbf{झः} वर्गो जातः घनोऽपि जातः~। एवमग्रेऽपि~। इदमस्मदिष्टम्~॥ 
\vspace{2mm}

\begin{center}
\textbf{\large अथ नवमं क्षेत्रम्~॥~९~॥}
\end{center}

 {\ab रूपादयोऽङ्का यद्येकनिष्पत्तौ भवन्ति तत्र यदि रूपाद्द्वितीयोऽङ्को वर्गो भवति तदा सर्वेऽङ्का वर्गा भवन्ति~। यदि रूपाद्द्वितीयाङ्को घनो भवति तत्र सर्वे घना भविष्यन्ति~। }

\begin{flushleft}
\begin{minipage}[t]{0.65\textwidth}
\hspace{4mm} यथा \textbf{अबजदा} रूपादयः कल्पिताः~। यदि \textbf{अः} वर्गो भवति \textbf{ब}श्च वर्ग एवास्ति~। तस्मा\textbf{ज्जो}ऽपि वर्गो भविष्यति~। यतो \textbf{अब}योर्निष्पत्तितुल्यास्ति~। एवमग्रेऽपि~।\\
\vspace{-2mm}

\hspace{4mm} पुनरपि यदि \textbf{अः} घनो भवति~। तस्य वर्गो \textbf{बः} घनो भविष्यति~। रूपाच्चतुर्थो \textbf{जः} घन  एवास्ति~। \textbf{दो}ऽपि घनः~।
\end{minipage} 
\hfill
\begin{minipage}[t]{0.25\textwidth}
१. अ,४, ब,१६. \\
{\color{white}अ} ज,६४. द,२५६.\\
१. अ,८. ब,६४. \\
{\color{white}अ} ज,५१२. द,४०९६.
\end{minipage}
\end{flushleft}
\vspace{-3mm}

\noindent  यतः \textbf{जद}निष्पत्तिः \textbf{अब}निष्पत्तितुल्यास्ति~। इदमेवास्माकमिष्टम्~॥
\vspace{2mm}

\begin{center}
\textbf{अथ दशमं क्षेत्रम्~॥~१०~॥}
\end{center}

{\ab  रूपादयो यावन्तोऽङ्का एकनिष्पत्तौ भवन्ति तत्र रूपाद्द्वितीयोऽङ्कश्चेद्वर्गो न भवति तत्र द्वितीयस्थानं द्वितीयस्थानं विना वर्गा न भवन्ति~। यदि च रूपा-द्द्वितीयोऽङ्को घनो न भवति तदा तृतीयतृतीयस्थानं विना घना न भविष्यन्ति~।} 

\begin{flushleft}
\begin{minipage}[t]{0.66\textwidth}
\hspace{4mm} यथा \textbf{अबजदहझा} एकरूपनिष्पत्तौ कल्पिताः~। यदि \textbf{अं} वर्गो न भवति तदा \textbf{ज}मपि वर्गो न स्यात्~। यदि वर्गो भवति तदा \textbf{बज}निष्पत्तिः \textbf{अब}निष्पत्तिसमानास्ति~। तस्मात् जं वर्गश्चेत् अं वर्गो भविष्यति~। इदमशुद्धम्~।
\end{minipage} 
\hfill
\begin{minipage}[t]{0.25\textwidth}
१. अ,२. ब,४. ज,८.\\
द,१६. ह,३२. झ,६४.
\end{minipage}
\end{flushleft}
\vspace{-2mm}

 अनेनैव प्रकारेण हमपि वर्गो न भविष्यति~। \\
\vspace{-2mm}

 पुनरपि यदि अं घनो न भवति तदा बमपि घनो न भविष्यति~। 

\newpage
\noindent यदि \textbf{बं} घनो भवति तदा \textbf{बज}निष्पत्तिः \textbf{अब}निष्पत्तिसमानास्ति~। तस्मात् \textbf{अ}मपि घनो भविष्यति~। इदमशुद्धम्~। एवमग्रेऽपि~। इदमेवास्मदिष्टम्~॥ 
\vspace{2mm}

\begin{center}
\textbf{ अथैकादशं क्षेत्रम्~॥~११~॥}
\end{center}

{\ab रूपादयोऽङ्का \,यद्येकनिष्पत्तौ \,भवन्ति \,तदा \,तेषु \,लघ्वङ्कस्तदङ्कतमाङ्कतुल्यं महदङ्कं निःशेषं करिष्यति~।} 

\begin{flushleft}
\begin{minipage}[t]{0.65\textwidth}
\hspace{4mm} यथा \,\textbf{अबजदहा} \,एकनिष्पत्तौ \,कल्पिताः~। \textbf{जः \,हं} निःशेषं \,करोतीति \,कल्पितम्~। तस्मात् \,\textbf{जः \,हं \,ब}तुल्यं निःशेषं करिष्यति~। कुतः~। \textbf{जदहा}स्त्रयोऽङ्का \,एकनिष्पत्तौ
\end{minipage} 
\hfill
\begin{minipage}[t]{0.25\textwidth}
१. अ,३. ब,९. ज,२७.\\
{\color{white}अ} द,८१. ह,२४३.
\end{minipage}
\end{flushleft}
\vspace{-3mm}

\noindent तथा सन्ति यथा रूपं \textbf{अं बं} च एकनिष्पत्तौ सन्ति~। रूपं \textbf{बं} निःशेषं तथा करोति यथा \textbf{जः हं} निःशेषं करोति~। तस्मात् \textbf{जः हं ब}तुल्यं निःशेषं करिष्यति~। एतदेवेष्टम्~॥
\vspace{2mm}

\begin{center}
\textbf{\large अथ द्वादशं क्षेत्रम्~॥~१२~॥}
\end{center}

{\ab रूपादयोऽङ्का एकनिष्पत्तौ भवन्ति तत्र यदि प्रथमाङ्कोऽन्त्याङ्कं निःशेषं करोति तदा स एवाङ्को रूपाद्द्वितीयाङ्कं निःशेषं करिष्यति~।}

\begin{flushleft}
\begin{minipage}[t]{0.65\textwidth}
\hspace{4mm}  यथा \textbf{अबजदा} एकरूपनिष्पत्तौ कल्पिताः~। \textbf{हं} प्रथमाङ्कः कल्पितः~।  अयं \textbf{दं} निःशेषं करोति~। तस्मात् \textbf{हं अ}म् \,अपि \,निःशेषं \,करिष्यति~। यदि \,\textbf{हं \,अं} \,निःशेषं \,न करोति तदा \textbf{अहौ} भिन्नाङ्कौ भविष्यतः~। अस्यां  निष्पत्तौ  च लघ्वङ्कौ भविष्यतः~। पुन\textbf{र्हः दं झ}तुल्यं निःशेषं करोतीति
\end{minipage} 
\hfill
\begin{minipage}[t]{0.25\textwidth}
१. अ,४. ब,१६.\\
{\color{white}अ} ज,६४. द,२४६.\\
ह,२. त,८. व,३२.  \\
{\color{white}अ} झ,१२८.
\end{minipage}
\end{flushleft}
\vspace{-3mm}

\noindent कल्पितम्~। तस्मात् \textbf{हझ}घातो \textbf{दं} भविष्यति~। \textbf{अज}घातोऽपि \textbf{द}मस्ति~। तस्मात् \textbf{हअ}निष्प-त्ति\textbf{र्जझ}निष्पत्तितुल्या भविष्यति~। \textbf{हअं जझं}\renewcommand{\thefootnote}{१}\footnote{हऔ जझौ {\en K.}} क्रमेण तुल्यं निःशेषं करिष्यति\renewcommand{\thefootnote}{२}\footnote{करिष्यतः {\en K.}}\;। पुन\textbf{र्हं जं व}तुल्यं निःशेषं करोतीति क- 

\newpage
\noindent ल्पितम्~। \,\textbf{हअ}निष्पत्ति\textbf{र्बव}निष्पत्तिसमानास्तीति \,निश्चितम्~। \,तस्मात् \,\textbf{हं \,बं} \,निःशेषं करिष्यति~। \textbf{हं बं त}तुल्यं निःशेषं करिष्यतीति कल्पितम्~। पुन\textbf{र्हअ}निष्पत्तिः \textbf{अत}निष्पत्ति-समानास्तीति कल्पितम्~। तदा \textbf{हः अं} निःशेषं करिष्यति~। इदमशुद्धम्~। अस्मदिष्टं समीचीनम्~॥ 
\vspace{2mm}

\begin{center}
\textbf{\large  अथ त्रयोदशं क्षेत्रम्~॥~३~॥}
\end{center}

 {\ab रूपादयो यावन्तोऽङ्का एकनिष्पत्तौ पतन्ति तेषु यदि रूपाद्द्वितीयोऽङ्कः प्रथमो भवति तेषु मध्ये महदङ्कं तैरङ्कैर्विना कोऽपि निःशेषं न करिष्यति~।}\\ 

 यथा \textbf{अबजद}म् एकरूपनिष्पत्तौ कल्पितम्~। \textbf{अः} प्रथमाङ्कः कल्पितः~। तदा \textbf{दं} महदङ्कम् \textbf{अबजं} हित्वा कोऽपि निःशेषं न करिष्यति~। 

\begin{flushleft}
\begin{minipage}[t]{0.65\textwidth}
\hspace{4mm} यदि करिष्यति तदा \textbf{हः} करिष्यतीति कल्पितम्~। \textbf{हः} प्रथमाङ्को न  भविष्यति~। यदि भविष्यति तदासौ \textbf{अं} निःशेषं करिष्यति~। इदमशुद्धम्~। तस्मात् \textbf{हः} योगाङ्को भविष्यति~। तं प्रथमाङ्को निःशेषं करिष्यति~। स प्रथमाङ्को \textbf{आ}द्भिन्नो भविष्यति~। असौ \textbf{कं} भविष्यतीति कल्पितम्~।
\end{minipage} 
\hfill
\begin{minipage}[t]{0.25\textwidth}
१. अ,५. ब,२५.\\
{\color{white}अ}  ज,१२५. द,६२५\\
ह------ व--- झ---\\
{\color{white}अ} क--- त---
\end{minipage}
\end{flushleft}
\vspace{-3mm}

\noindent \textbf{कं दं} निःशेषं करिष्यति~। तदा \textbf{अ}मपि निःशेषं करिष्यति~। इदमशुद्धम्~। तस्मात्सोऽङ्कः \textbf{अ} एव भविष्यति नान्यः~। कल्पितं च \textbf{हः दं झ}तुल्यं निःशेषं करोति~। तस्मात् \textbf{अज}घातो \textbf{झह}घातसमानो भविष्यति~। \textbf{अह}निष्पत्ति\textbf{र्झज}निष्पत्तितुल्या भविष्यति~। \textbf{अः हं} निःशेषं करोति~। तस्मात् \textbf{झं जं} निःशेषं करिष्यति~। \textbf{झं} च \textbf{अबजा}द्भिन्नमस्ति~। कुतः~। यतो \textbf{हः दं झ}तुल्यं निःशेषं करोति~। \textbf{हं} च \textbf{अबजा}द्भिन्नमस्ति~। पुन\textbf{र्झः} प्रथमाङ्को नास्तीति निश्चितम्~। \textbf{झ}म् \textbf{अं} विना कोऽपि निःशेषं न करोति~। पुन\textbf{र्झः जं व}तुल्यं निःशेषं करोतीति कल्पितम्~। \textbf{वं बं} निःशेषं करोतीति निश्चयः कार्यः~। \textbf{व}म् \textbf{अबा}द्भिन्नमस्ति~। प्रथमाङ्को नास्ति~। \textbf{आ}द्भिन्नोऽङ्कस्तं निःशेषं न करिष्यति~। कल्पितं \textbf{वं बं त}तुल्यं निःशेषं

\newpage
\noindent करिष्यतीति~। निश्चितं \textbf{तं अं} नास्ति~। \textbf{वत}योर्घातो \textbf{ब}मस्ति~। \textbf{अ}वर्गोऽपि \textbf{ब}मस्ति~। तस्मात् \textbf{अव}निष्पत्ति\textbf{स्तअ}निष्पत्तिसमाना भविष्यति~। \textbf{अं वं} निःशेषं करोति~। \textbf{त}म् \textbf{अं} निःशेषं करिष्यति~। इदमशुद्धम्~। अस्मदिष्टं समीचीनम्~॥ 
\vspace{2mm}

\begin{center}
\textbf{\large अथ चतुर्दशं क्षेत्रम्~॥~१५~॥}
\end{center}

{\ab यावन्तः प्रथमाङ्काः कल्प्यन्ते तैर्विनान्येऽपि प्रथमाङ्का भविष्यन्ति~। }

\begin{flushleft}
\begin{minipage}[t]{0.65\textwidth}
\hspace{4mm} यथा \textbf{अबजाः} प्रथमाङ्काः कल्पिताः~। एक इष्ट लघ्वङ्को ग्राह्यो यं \textbf{अबजा} निःशेषं कुर्वन्ति~। स \textbf{हदं} कल्पितम्~। अस्मिन् रूपं संयोज्य \textbf{झदं} कल्पितम्~। यदि \textbf{झदं} प्रथमाङ्को भवति तदास्मादस्मदिष्टं सिद्धम्~। यदि प्रथमाङ्को न भवति
\end{minipage} 
\hfill
\begin{minipage}[t]{0.25\textwidth}
अ,२. ब,३. ज,५. वः\\
{\color{white}अ} हद,३०. झद,३१.\\ 
झ. ह...............द.\\
{\color{white}अब} व----
\end{minipage}
\end{flushleft}
\vspace{-3mm}

\noindent  तदा कोऽपि प्रथमाङ्क एनं निःशेषं करिष्यति~। स च \textbf{वः} कल्पितः~। \textbf{वं} च \textbf{अबज}मध्ये नास्ति~। यद्येतन्मध्ये भवति तदा \textbf{हदं} निःशेषं करिष्यति~। \textbf{दझ}मपि निःशेषं करिष्यति~। तस्मात् \textbf{झहं} रूपमपि निःशेषं करिष्यति~। इदमशुद्धम्~। तस्मात् \textbf{वं अबजा}द्भिन्नः प्रथमाङ्क उपलब्धः~। इदमेवास्माकमिष्टम्~॥ 
\vspace{2mm}

\begin{center}
\textbf{\large अथ पञ्चदशं क्षेत्रम्~॥~१५~॥}
\end{center}

 {\ab कल्पितप्रथमाङ्का यदि कमपि लघ्वङ्कं निःशेषं करिष्यन्ति तदा तं लघ्वङ्कं तदन्यः प्रथमाङ्को निःशेषं न करिष्यति~। }

\begin{flushleft}
\begin{minipage}[t]{0.7\textwidth}
\hspace{4mm} यथा \textbf{अं} लघ्वङ्कः कल्पितः~। \textbf{बजदाः} प्रथमाङ्कास्तं निःशेषं कुर्वन्तीति कल्पितम्~। तदान्ये प्रथमाङ्का एनं निःशेषं न करिष्यन्ति~। यदि करिष्यन्ति तदा \textbf{हः झ}तुल्यं निःशेषं करोतीति कल्पितम्~। तस्मात् \textbf{हझ}घातः \textbf{अ}तुल्यो भविष्यति~। \textbf{बः} प्रथ-
\end{minipage} 
\hfill
\begin{minipage}[t]{0.2\textwidth}
व,२. ज,३. द,५.\\
{\color{white}अब} अ,३०.\\
{\color{white}अ} ह-------\\
{\color{white}अ} झ------
\end{minipage}
\end{flushleft}
\vspace{-3mm}

\noindent माङ्कः \textbf{अं} निःशेषं करोति~।तस्मात्तस्यैकभुजमपि निःशेषं करिष्यति~।

\newpage
\noindent तस्मात् \textbf{हं} निःशेषं न करिष्यति~। \textbf{झं} निःशेषं करिष्यति~। एवं \textbf{जदा}वपि~। तस्मात् \textbf{बजदा झं} निःशेषं करिष्यन्ति~। \textbf{झं आ}त् न्यूनमस्तीत्यशुद्धम्~। अस्मदिष्टं समीचीनम्~॥
\vspace{2mm}

\begin{center}
\textbf{\large अथ षोडशं क्षेत्रम्~॥~१६~॥}
\end{center}

{\ab  त्रयो लघ्वङ्का यद्येकरूपनिष्पत्तौ भवन्ति तदा तेषां मध्ये द्वयोर्द्वयो\renewcommand{\thefootnote}{१}\footnote{{\en K. has one} द्वयोः.}र्योगः तृतीयाङ्कात् भिन्नो भविष्यति~। }

\begin{flushleft}
\begin{minipage}[t]{0.65\textwidth}
\hspace{4mm} यथा \textbf{अबजा} लघ्वङ्का एकनिष्पत्तौ कल्पिताः~। पुन\textbf{र्दह-हझौ} लघ्वङ्कौ अस्यां निष्पत्तौ गृहीतौ~। एतौ भिन्नौ स्तः~।  \textbf{दह}वर्गश्च \textbf{अ}मस्ति~। \textbf{हझ}वर्गो \textbf{ज}मस्ति~।  \textbf{दहहझ}घातो \textbf{ब}म् अस्ति~। प्रत्येकं \textbf{दहदझौ हझा}द्भिन्नौ स्तः~। तस्मात् \textbf{दह}-
\end{minipage} 
\hfill
\begin{minipage}[t]{0.25\textwidth}
अ,९. ब,१२. ज,१६.\\ 
द.... ह.... झ.
\end{minipage}
\end{flushleft}
\vspace{-3mm}

\noindent \textbf{दझ}घातः \textbf{अब}योगतुल्यो \textbf{हझा}द्भिन्नो भविष्यति~। तस्य वर्गादपि भिन्नो भविष्यति~। एवं \textbf{बज}योगः \textbf{आ}द्भिन्नोऽस्ति~। पुन\textbf{र्दहहझौ दझा}द्भिन्नौ स्तः~। \textbf{दहहझ}घातश्च \textbf{दझा}द्भिन्नो भविष्यति~। तद्वर्गादपि भिन्नो भविष्यति~। तस्य वर्गश्च द्विगुण\textbf{दहहझ}घात\textbf{दह}वर्ग\textbf{हझ}वर्गयोग-तुल्यश्चास्ति~। तस्मात्  \textbf{दहहझ}घातो \textbf{दहहझ}घात\textbf{दह}वर्ग\textbf{झह}वर्गयोगाद्भिन्नो भविष्यति~। तस्मात् \textbf{ब}तुल्यो \textbf{दहहझ}घातः \textbf{अज}योगतुल्यात् \textbf{दहहझ}वर्गयोगाद्भिन्नो भविष्यति~। इदम् एवास्माकमिष्टम्~॥ 
\vspace{2mm}

\begin{center}
\textbf{\large अथ सप्तदशं क्षेत्रम्~॥~१७~॥} 
\end{center}
 
 {\ab रूपाद्व्यतिरिक्तौ यौ भिन्नाङ्कौ भवतस्तयोस्तृतीयाङ्कस्तन्निष्पत्तौ न भवति~। }

\begin{flushleft}
\begin{minipage}[t]{0.7\textwidth}
\hspace{4mm} यथा \textbf{अबौ} भिन्नाङ्कौ कल्पितौ~। अनयोर्निष्पत्तौ तृतीयाङ्को न  भवति~। यदि भवति तदा \textbf{ज}स्तृतीयाङ्को तस्यामेव निष्पत्तौ कल्पितः~। तस्मात् \textbf{अब}निष्पत्ति\textbf{र्बज}निष्पत्तितुल्या भविष्यति~। \textbf{अबौ} अस्यां निष्पत्तौ लघ्वङ्कौ स्तः~।
\end{minipage} 
\hfill
\begin{minipage}[t]{0.2\textwidth}
अ,५. ब,८.\\
ज---
\end{minipage}
\end{flushleft}

\newpage
\noindent तस्मात् \textbf{बजं} निःशेषं करिष्यतः~। तस्मात् \textbf{अः बं} निःशेषं करिष्यति~। इदमशुद्धम्~। अस्मदिष्टं समीचीनम्~॥
\vspace{2mm}

\begin{center}
\textbf{\large अथाष्टादशं क्षेत्रम्~॥~१८~॥}
\end{center}

 {\ab तत्र यावन्तोऽङ्का एकरूपनिष्पत्तौ भवन्ति तेषामाद्यन्ताङ्कौ  यदि भिन्नौ भव-तस्तयोर्मध्ये कोऽपि रूपो न भवति तदान्त्याङ्काद्द्वितीयोऽङ्कोऽग्रेऽस्यां निष्पत्तौ नोत्पत्स्यते~।}

\begin{flushleft}
\begin{minipage}[t]{0.6\textwidth}
\hspace{4mm} यथा \textbf{अबजा} एकरूपनिष्पत्तौ कल्पिताः~। \textbf{अजौ} भिन्नौ यदि भवतोऽनयोर्मध्ये कोऽपि रूपो न भवति तदा \textbf{जा}द्द्वितीयोऽङ्कः \textbf{अब}निष्पत्तौ न भविष्यति~। यदि
\end{minipage} 
\hfill
\begin{minipage}[t]{0.3\textwidth}
अ,९. ब,१२. ज,१६\\
{\color{white}अ} द---
\end{minipage}
\end{flushleft}
\vspace{-3mm}

\noindent भवति तदा \textbf{जद}निष्पत्तिः \textbf{अब}निष्पत्तितुल्या कल्पिता~। तस्मात् \textbf{अज}निष्पत्ति\textbf{र्बद}निष्पत्ति-तुल्या भविष्यति~। \textbf{अजौ} लघ्वङ्कौ अस्यां निष्पत्तौ स्तः~। तस्मात् \textbf{अः बं} निःशेषं करिष्यति~। \textbf{ज}मपि निःशेषं करिष्यतीत्यशुद्धम्~। अस्मदिष्टं समीचीनम्~॥ 
\vspace{2mm}

\begin{center}
\textbf{\large अथैकोनविंशातितमं क्षेत्रम्~॥~१९~॥} \end{center}

 {\ab द्वयोर्निष्पत्तौ तृतीयाङ्कनिष्पादनमिष्टमस्ति यदि सम्भवः स्यात्~। }

\begin{flushleft}
\begin{minipage}[t]{0.6\textwidth}
\hspace{4mm} यथा \,\textbf{अबौ} \,अभिन्नाङ्कौ \,कल्पितौ~। \textbf{ब}वर्गो \,\textbf{जः} कल्पितः~। यदि \textbf{अः जं} निःशेषं करोति \textbf{द}तुल्यमिति कल्पितम्~। तस्मात् \textbf{द}स्तृतीयाङ्को भविष्यति~। कुतः~।
\end{minipage} 
\hfill
\begin{minipage}[t]{0.3\textwidth}
अ,४. ब,६. द,९. ज,३६.\\
अ,६. ब,४. द--- ज,१६.
\end{minipage}
\end{flushleft}
\vspace{-3mm}

\noindent \textbf{अद}घातो \textbf{ब}वर्गतुल्य\textbf{ज}समोऽस्ति~। तस्मात् \textbf{अब}निष्पत्ति\textbf{र्बद}निष्पत्तितुल्या भविष्यति~।\\
\vspace{-1mm}

यदि \textbf{अः जं} निःशेषं न करोति तदा तृतीयाङ्कोऽस्यां निष्पत्तौ न  भविष्यति~। यदि भवति तदा \textbf{द}तुल्यः कल्पितः~। तस्मात् \textbf{अद}घातो \textbf{ज}तुल्यो भविष्यति~। तस्मात् \textbf{अं जं} निःशेषं करिष्यति~। इदमशुद्धम्~। अस्मदिष्टं समीचीनम्~॥ 

\newpage
\begin{center}
\textbf{\large अथ विंशतितमं क्षेत्रम्~॥~२०~॥}
\end{center}

{\ab यत्राङ्कत्रयम् एकनिष्पत्तावस्ति तत्र निष्पत्तौ चतुर्थाङ्कोत्पादनमिष्टमस्ति यदि तदुत्पादनं सम्भवति~।} 

\begin{flushleft}
\begin{minipage}[t]{0.7\textwidth}
\hspace{4mm} यथा \;\textbf{अबजा} \;अङ्काः \;कल्पिताः~। \,\textbf{अजौ} \;भिन्नाङ्कौ \;न भवतः~।  तस्मात्  \textbf{बं जे}न गुणितं \textbf{दं} जातम्~। \textbf{अः दं ह}तुल्यं निःशेषं करोतीति कल्पितम्~। तस्मात् \textbf{हः} चतुर्थाङ्को भविष्यति~। यतः \textbf{अह}घातो~। \textbf{बज}घाततुल्योऽस्ति~। \textbf{अब}निष्पत्तिः \textbf{जह}निष्पत्तितुल्या भविष्यति~। 
\end{minipage} 
\hfill
\begin{minipage}[t]{0.2\textwidth}
अ,८. ब,१२.\\
{\color{white}अ} ज,१८. ह,२७.\\ 
द, २१६.
\end{minipage}
\end{flushleft}
\vspace{-3mm}

\begin{flushleft}
\begin{minipage}[t]{0.7\textwidth}
\hspace{4mm} यदि \textbf{अः दं} निःशेषं न करिष्यति तदा चतुर्थाङ्को न भविष्यति~। यदि भविष्यति तदा \textbf{हः} कल्पितः~। तस्मात् \textbf{अह}घातो \textbf{द}तुल्यो भविष्यति~। तस्मात् \textbf{अः दं} निःशेषं करिष्यतीत्यशुद्धम्~। अस्मदिष्टमेव  समीचीनम्~॥ 
\end{minipage} 
\hfill
\begin{minipage}[t]{0.2\textwidth}
अ,२०. ब,३०.\\ 
{\color{white}अ} ज,४५. ह--- \\
द,१३५०.
\end{minipage}
\end{flushleft}
\vspace{-1mm}

\begin{center}
\textbf{\large अथैकविंशतितमं क्षेत्रम्~॥~२१~॥}
\end{center}

{\ab  यावन्तः समाङ्कास्तेषां योगः समाङ्को भवति~।} 

\begin{flushleft}
\begin{minipage}[t]{0.65\textwidth}
\hspace{4mm}  यथा \textbf{अबं बजं जदं} समाङ्काः कल्पिताः~। एतेषां योगः \textbf{अदो}ऽपि समाङ्को भविष्यति~। कुतः~। प्रत्येकस्य समाङ्क-
\end{minipage} 
\hfill
\begin{minipage}[t]{0.25\textwidth}
अ..... ब..... ज.. द
\end{minipage}
\end{flushleft}
\vspace{-3mm}

\noindent स्यार्द्धं भवति~। अर्द्धाङ्कानां योगो योगार्द्धं भवति~। तस्मात् \textbf{अद}स्यार्द्धं जातम्~। इदम् एवास्माकमिष्टम्~॥ 
\vspace{2mm}

\begin{center}
\textbf{\large अथ द्वाविंशतितमं क्षेत्रम्~॥~२२~॥} 
\end{center}

{\ab  समतुल्यविषमाङ्कयोगः समो भवति~। }

\begin{flushleft}
\begin{minipage}[t]{0.5\textwidth}
\hspace{4mm}  यथा \,\textbf{अबं \,बजं \,जदं \,दहं} \,विषमाङ्काः कल्पिताः~। एतेषां योगः समाङ्को भविष्यति~।
\end{minipage} 
\hfill
\begin{minipage}[t]{0.4\textwidth}
अ... ब... ज....... द.......ह
\end{minipage}
\end{flushleft}
\vspace{-3mm}

\noindent कुतः~। यदि प्रत्येकविषमाङ्कात् रूपं पृथक् क्रियते तदा समाङ्कः शेषो भविष्यति~। रूपाणां योग एकः समाङ्को भवि-
 
\newpage
\noindent ष्यति~। समाङ्कानां योगश्च समाङ्क एव भवति~। तस्मात् \textbf{अहं} समाङ्को भविष्यतीत्यस्मा-कमिष्टम्~॥ 
\vspace{2mm}

\begin{center}
\textbf{अथ त्रयोविंशतितमं क्षेत्रम्~॥~२३~॥}
\end{center}

{\ab विषमतुल्यविषमाङ्कयोगः विषमाङ्को भवति~। }

\begin{flushleft}
\begin{minipage}[t]{0.55\textwidth}
\hspace{4mm} यथा \textbf{अबबजजदा} विषमाङ्कतुल्या विषमाङ्काः कल्पिताः~। एतेषां \,योगो \,विषमाङ्को \,भविष्यति~।
\end{minipage} 
\hfill
\begin{minipage}[t]{0.35\textwidth}
अ..... ब...... ज....... ह.. द
\end{minipage}
\end{flushleft}
\vspace{-3mm}

\noindent कुतः~। यदि \textbf{जदा}त् \textbf{दह}तुल्यं रूपं पृथक् क्रियते तदा \textbf{जहं} समाङ्कोऽवशिष्यते~। \textbf{अजं} समाङ्कोऽस्ति~।  कुतः~। समतुल्यविषमाङ्कयोगत्वात्~। तस्मात् \textbf{अह}मपि समाङ्को भविष्यति~। \textbf{दहं} रूपमस्ति~। तस्मात् \textbf{अदं} विषमाङ्को भविष्यति~। इदमेवास्माकमिष्टम्~॥ 
 \vspace{2mm}

\begin{center}
\textbf{\large अथ चतुर्विंशतितमं क्षेत्रम्~॥~२४~॥}
\end{center}
 
 {\ab यदि समाङ्कात् समाङ्कः पृथक्क्रियते तदा शेषः समाङ्को भवति~। }

\begin{flushleft}
\begin{minipage}[t]{0.7\textwidth}
\hspace{4mm} यथा \textbf{अब}समाङ्कात् \textbf{बजं} समाङ्कः पृथक्क्रियते~। तदा \textbf{अजं} समाङ्कोऽवशिष्यते~। कुतः~। यदि \textbf{बजा}र्द्धं \textbf{अबा}र्द्धात् शोध्यते
\end{minipage} 
\hfill
\begin{minipage}[t]{0.2\textwidth}
अ.... ज....ब
\end{minipage}
\end{flushleft}
\vspace{-3mm}

\noindent  तदा \textbf{अजा}र्द्धमवशिष्यते~। तस्मात् \textbf{अज}स्यार्द्धं जातम्~। इदमेवास्माकमिष्टम्~॥
 
 \begin{center}
\textbf{\large अथ पञ्चविंशतितमं क्षेत्रम्~॥~२५~॥}
\end{center}
\vspace{2mm}

{\ab यदि समाङ्कात् विषमाङ्कः पृथक्क्रियते तदा शेषं विषमाङ्को भवति~। }

\begin{flushleft}
\begin{minipage}[t]{0.65\textwidth}
\hspace{4mm}  यथा \textbf{अब}समाङ्कात् \textbf{बज}विषमाङ्कः पृथक्क्रियते~। तदा शेषं \textbf{अजं} विषमाङ्को भविष्यति~। कुतः~। \textbf{बजा}त् \textbf{जदं} रूप-
\end{minipage} 
\hfill
\begin{minipage}[t]{0.25\textwidth}
अ.......ज..द....ब
\end{minipage}
\end{flushleft}
\vspace{-3mm}

\noindent तुल्यं पृथक्क्रियते~। शेषं \textbf{दबं} समाङ्कोऽवशिष्यते~। \textbf{अबा}त् \textbf{दबं} शोध्यम्~। \textbf{अदं} समाङ्कोऽवशिष्यते~। \textbf{जदं} च 

\newpage
\noindent रूपमस्ति~। तस्मात् शेषं \textbf{अजं} विषमाङ्को भविष्यति~। इदमेवास्माकमिष्टम्~॥ 
\vspace{2mm}

\begin{center}
\textbf{\large अथ षड्विंशतितमं क्षेत्रम्~॥~६~॥}
\end{center}

 {\ab विषमाङ्कात् समाङ्कः पृथक्क्रियते तदा शेषं विषमाङ्कोऽवशिष्यते~। }

\begin{flushleft}
\begin{minipage}[t]{0.65\textwidth}
\hspace{4mm}  यथा \textbf{अब}विषमाङ्कात् \textbf{जब}समाङ्कः पृथक्क्रियते तदा \textbf{अजं} \,शेषं  \,विषमाङ्को \,भविष्यति~। \,कुतः~। \,यदि \,\textbf{बद}रूपं 
\end{minipage} 
\hfill
\begin{minipage}[t]{0.25\textwidth}
अ.....ज.....ब..द
\end{minipage}
\end{flushleft}
\vspace{-3mm}

\noindent \textbf{अबे} योज्यते तदा \textbf{अदं} समाङ्को भविष्यति~। \textbf{दज}श्च विषमाङ्कोऽस्ति~। तस्मात् \textbf{अजः} विषमाङ्को भविष्यति~। इदमेवेष्टम्~॥ 
\vspace{2mm}

\begin{center}
\textbf{\large अथ सप्तविंशतितमं क्षेत्रम्~॥~२७~॥} 
\end{center}

{\ab  विषमाङ्कात् विषमाङ्कः पृथक्क्रियते तदा शेषं समाङ्को भविष्यति~। }

\begin{flushleft}
\begin{minipage}[t]{0.7\textwidth}
\hspace{4mm}  यथा \textbf{अब}विषमाङ्कात् \textbf{बज}विषमाङ्कः पृथक्क्रियते~। तत्र \textbf{अजः} शेषं समाङ्कोऽवशिष्यते~। यदि \textbf{अबबज}यो\textbf{र्बद}रूपं पृथक् क्रियते~। शेषः \textbf{अजं} समाङ्कः स्यात्~। इदमेवास्माकमिष्टम्~॥
\end{minipage} 
\hfill
\begin{minipage}[t]{0.2\textwidth}
 अ....ज....द..ब
\end{minipage}
\end{flushleft}
\vspace{-1mm}

\begin{center}
\textbf{\large अथाष्टाविंशतितमं क्षेत्रम्~॥~२८~॥} 
\end{center}

{\ab विषमाङ्कसमाङ्कघातः समाङ्को भवति~। }

\begin{flushleft}
\begin{minipage}[t]{0.7\textwidth}
\hspace{4mm}  यथा \textbf{अं} विषमाङ्को \textbf{बं} समाङ्कः~। अनयोर्घातो \textbf{जः} समाङ्को भविष्यति~। कुतः~। समतुल्यविषमाङ्कयोगः समो भवति~। इदमेवास्माकमिष्टम्~॥
\end{minipage} 
\hfill
\begin{minipage}[t]{0.2\textwidth}
अ... \\
ब...\\
ज.............
\end{minipage}
\end{flushleft}
\vspace{-1mm}

\begin{center}
\textbf{\large अथोनत्रिंशत्तमं क्षेत्रम्~॥~२९~॥}
\end{center}

{\ab  विषमाङ्कयोर्घातो विषमाङ्को भवति~। }

\begin{flushleft}
\begin{minipage}[t]{0.7\textwidth}
\hspace{4mm} यथा \textbf{अब}योर्विषमाङ्कयोर्घातो \textbf{जः} विषमाङ्को भवति~। कुतः~। विषमतुल्यविषमाङ्कयोगो विषमो भवति~। इदमेवेष्टम्~॥
\end{minipage} 
\hfill
\begin{minipage}[t]{0.2\textwidth}
\vspace{-6mm}

अ....\\
ब...... \\
ज....... ..
\end{minipage}
\end{flushleft}

\newpage
 \begin{center} 
\textbf{\large अथ त्रिंशत्तमं क्षेत्रम्~॥~३०~॥}
\end{center}

{\ab  विषमाङ्कः समाङ्कं समतुल्यं निःशेषं करिष्यति~। }

\begin{flushleft}
\begin{minipage}[t]{0.72\textwidth}
\hspace{4mm}  यथा \textbf{अं} विषमाङ्को \textbf{ब}समाङ्कं \textbf{ज}तुल्यं निःशेषं करोति~। तदा \textbf{जं} समाङ्को  भविष्यति~। \\
\vspace{-2mm}

\hspace{4mm} यदा न भविष्यति तदा विषमाङ्को भविष्यतीति कल्पितम्~।
\end{minipage} 
\hfill
\begin{minipage}[t]{0.18\textwidth}
अ...\\
ब.... \\
ज.... 
\end{minipage}
\end{flushleft}
\vspace{-3mm}

\noindent  तस्मात् \textbf{अज}योर्घातो \textbf{ब}तुल्यो विषमाङ्को भवतीत्येतदशुद्धम्~। अस्मदिष्टं समीचीनम्~॥
\vspace{2mm}

\begin{center}
\textbf{\large अथैकत्रिंशत्तमं क्षेत्रम्~॥~३१~॥}
\end{center}

{\ab  विषमाङ्को विषमाङ्कं विषमाङ्कतुल्यं निःशेषं करोति~। }

\begin{flushleft}
\begin{minipage}[t]{0.75\textwidth}
\hspace{4mm}  यथा \textbf{अः बं ज}तुल्यं निःशेषं करोति~। तदा \textbf{जः} विषमाङ्को भविष्यति~। यदि न भविष्यति तदा समाङ्कः कल्पनीयः~। तस्मात्  \textbf{अज}योर्घातो \textbf{ब}तुल्यः समाङ्को भविष्यति~। इदमशुद्धम्~। अस्मदिष्टं समीचीनम्~॥
\end{minipage} 
\hfill
\begin{minipage}[t]{0.15\textwidth}
अ...\\
व...... \\
ज..... 
\end{minipage}
\end{flushleft}
\vspace{-1mm}

\begin{center}
\textbf{ अथ द्वात्रिंशत्तमं क्षेत्रम्~॥~३२~॥}
\end{center}

{\ab विषमाङ्कः समाङ्कं चेन्निःशेषं करोति तदा तस्यार्द्धमपि  निःशेषं करिष्यति~। }

\begin{flushleft}
\begin{minipage}[t]{0.67\textwidth}
\hspace{4mm} यथा \textbf{अः बजं} निःशेषं करोति~। तदा \textbf{बद}तुल्यं \textbf{बजा}-र्द्धमपि निःशेषं करिष्यति~। कुतः~। \textbf{अः बजं हझ}तुल्यं निःशेषं करिष्यतीति कल्पितम्~। तस्मात् \textbf{हझं} समाङ्को भवि-ष्यति~। अस्य अर्धं \textbf{हवं} कल्पितम्~। तस्मात् \textbf{अः बजा}र्धं \textbf{हव}समं निःशेषं करिष्यति~। इदमेवास्माकमिष्टम्~॥
\end{minipage} 
\hfill
\begin{minipage}[t]{0.23\textwidth}
अ...\\ 
ब .....द ...अ \\
ह...व....झ
\end{minipage}
\end{flushleft}
\vspace{-1mm}
 
\begin{center}
\textbf{\large अथ त्रयस्त्रिंशत्तमं क्षेत्रम् ~॥~३३~॥}
\end{center}

{\ab  यो विषमाङ्क इष्टाङ्काद्भिन्नो भवति तदा तद्बिगुणाङ्कादपि भिन्नो भविष्यति~। }

\newpage
 यथा \textbf{अः जदा}द्भिन्नोऽस्ति~। तद्द्विगुणात् \textbf{हजा}दपि भिन्नो भविष्यति~। 

\begin{flushleft}
\begin{minipage}[t]{0.7\textwidth}
\hspace{4mm} यदि न भवति तदा कल्पितं \textbf{ब}म् उभयोरपवर्तनं  करोतीति~। अयं च विषमाङ्कोऽस्ति~। \textbf{जद}मपि निःशेषं करिष्यति~। तस्मात् \textbf{अं जदं} च मिलिताङ्कौ भविष्यतः~। इदमशुद्धम्~। अस्मदिष्टं समीचीनम्~॥
\end{minipage} 
\hfill
\begin{minipage}[t]{0.2\textwidth}
अ...\\
ज.... द.... ह\\ 
ब---
\end{minipage}
\end{flushleft}
\vspace{-1mm}

\begin{center}
\textbf{\large अथ चतुस्त्रिंशत्तमं क्षेत्रम्~॥ ३४~॥ }
\end{center}

 {\ab द्व्यादिद्विगुणोत्तरा अङ्काः समसमाङ्का भविष्यन्ति~॥ }

 यथा \textbf{अः} द्व्यङ्कः कल्पितः~। द्विगुणा \textbf{बजदाः} कल्पिताः~। एते  समाङ्काः सन्तीति प्रकटमेव \,चास्ति\renewcommand{\thefootnote}{१}\footnote{\hangindent=0.3in वास्ति {\en K.}}\;। एतेषामादिः \textbf{अः} \,द्विमितोऽस्ति~। स एव प्रथ-
\vspace{-4mm}

\begin{flushleft}
\begin{minipage}[t]{0.75\textwidth}
 माङ्कः\;। एतस्मादधिकाङ्क एनं कोऽपि निःशेषं न करिष्यति\;।~योऽङ्क एतेष्वन्यतमाङ्कं निःशेषयत्यसावेतेष्वन्यतमाङ्कतुल्यम् एव निःशेषं करिष्यति~। तस्मात् प्रत्येकं समसमाङ्को जातः~। इदमेवेष्टम्~॥
\end{minipage} 
\hfill
\begin{minipage}[t]{0.15\textwidth}
\vspace{-6mm}

अ,२\\
ब,४ \\
ज,८\\ 
द,१६
\end{minipage}
\end{flushleft}
\vspace{-1mm}

\begin{center}
\textbf{\large अथ पञ्चत्रिंशत्तमं क्षेत्रम्~॥~३५~॥ }
\end{center}

{\ab  यस्याङ्कस्यार्द्धं विषमाङ्को भवति स समविषमाङ्कः स्यात्~। }

\begin{flushleft}
\begin{minipage}[t]{0.7\textwidth}
\hspace{4mm} यथा \textbf{अब}स्यार्द्धम् \textbf{अजं} कल्पितम्~। \textbf{अजं अबं} वारद्वयं निःशेषं करोति~। अयं समसमाङ्को न  भविष्यति~। यदि  भवि-
\end{minipage} 
\hfill
\begin{minipage}[t]{0.2\textwidth}
 अ.....ज...ब 
\end{minipage}
\end{flushleft}
\vspace{-3mm}

\noindent ष्यति तदास्यार्द्धं समाङ्को भविष्यति~। तस्मादयं समविषमाङ्को जातः~। इदमेवेष्टम्~॥
\vspace{2mm}

\begin{center}
\textbf{\large अथ षट्त्रिंशत्तमं क्षेत्रम्~॥~३६~॥}
\end{center}

 {\ab योऽङ्को द्व्यादिद्विगुणेषु मध्ये न भवति यस्यार्द्धं विषमाङ्कश्च न भवति सोऽङ्कः समसमः समविषमश्च भवति~।} 

\newpage
\begin{flushleft}
\begin{minipage}[t]{0.65\textwidth}
\hspace{4mm}  यथा \textbf{अब}म्~। अस्यार्धम् \textbf{अजं} कल्पितम्~। अयं सम इति प्रकटमेवास्ति~। अर्धभावात्~। समसमः कुतोऽस्ति~।
\end{minipage} 
\hfill
\begin{minipage}[t]{0.25\textwidth}
अ.....ज.....ब
\end{minipage}
\end{flushleft}
\vspace{-3mm}
 
\noindent  अर्द्धस्य समत्वात्~। समविषमः कुतोऽस्ति~। यतोऽस्यार्द्धार्द्धकरणेनान्त्यार्द्धं रूपं विना विषमो भवति~। स विषमो रूपातिरिक्तोऽस्ति  यतो द्व्यादिद्विगुणाङ्केभ्यो नोत्पन्नोऽस्ति~। स विषमाङ्क एनं कल्पितं समतुल्यं निःशेषं करिष्यति~। इदमेवास्माकमिष्टम्~॥
\vspace{2mm}

\begin{center}
\textbf{\large अथ सप्तत्रिंशत्तमं क्षेत्रम्~॥~३७~॥ }
\end{center}

{\ab यावन्तोऽङ्का एकनिष्पत्तौ भवन्ति प्रथमतुल्यं द्वितीयाद्यदि पृथक् क्रियते अन्त्याच्च पृथक् क्रियते तदा द्वितीयशेषस्य प्रथमाङ्केन तथा निष्पत्तिर्भविष्यति यथान्त्यशेषस्य अबाद्यङ्कयोगेन यथास्ति~। }

\begin{flushleft}
\begin{minipage}[t]{0.58\textwidth}
\hspace{4mm}  यथा \textbf{अबं जदं झवं तन}म् एते एकरूपनिष्पत्तौ सन्तीति कल्पितम्~। \textbf{अब}तुल्यं \textbf{जदा}त् \textbf{दहं} पृथक् कार्यम्~। पुन\textbf{रब}तुल्यं \textbf{मनं तना}त् पृथक् कार्यम्~। तस्मात् \textbf{जहअब}योर्निष्पत्ति\textbf{स्तम}स्य \textbf{झवजदअबा}नां योगेन या निष्पत्तिस्तत्तुल्यास्ति~।
\end{minipage} 
\hfill
\begin{minipage}[t]{0.32\textwidth}
अ..........ब\\
ज.....ह.......द\\
झ.....................व \\
त.......क......ल......म.....न
\end{minipage}
\end{flushleft}
\vspace{-3mm}

\begin{center}
अत्रोपपत्तिः~। 
\end{center}

 \textbf{जद}तुल्यं \textbf{लनं तना}त् पृथक्कार्यं~। \textbf{झव}तुल्यं \textbf{कनं} च पृथक्कार्यम्~।  तस्मात् \textbf{तनकन}योर्नि-ष्पत्तिः \textbf{कनलन}निष्पत्तितुल्यास्ति~। \textbf{लनमन}योरपि निष्पत्तितुल्यास्ति~। \textbf{तककन}योर्निष्पत्तिः \textbf{कललन}निष्पत्तिसमानास्ति~। \textbf{लममन}निष्पत्तितुल्याप्यस्ति~। तस्मात् \textbf{लममन}निष्पत्ति-तुल्य\textbf{जहअब}निष्पत्ति\textbf{स्तम}स्य ~\textbf{कनलनमन}योगतुल्य\textbf{झवजदअब}योगेन ~निष्पत्तिस्तत्तुल्या भविष्यति~। इदमेवेष्टम्~॥ 

\newpage
\begin{center}
\textbf{\large अथाष्टत्रिंशत्तमं क्षेत्रम्~॥~३८~॥}
\end{center}

{\ab  रूपादयोऽङ्का द्विगुणोत्तरा द्विमितनिष्पत्तौ यदि भवन्ति सरूपाणामेतेषां योगः प्रथमाङ्को यदि भवत्यस्य योगस्यान्त्याङ्कस्य च घातः सम्पूर्णाङ्को भवति~। }\\

 यथा रूपादयोऽङ्का \textbf{अबजदा} द्विमितनिष्पत्तौ कल्पिताः~। एतेषां योगो \textbf{ह}तुल्यः प्रथमाङ्कः कल्पितः~। तस्मात् \textbf{हद}योर्घातो \textbf{झव}तुल्यः सम्पूर्णाङ्को भविष्यति~। 
 
\begin{center}
अस्योपपत्तिः~।
\end{center}

\textbf{हा}दयो \textbf{अबजद}निष्पत्तितुल्याः \textbf{तकलमा} अङ्का ग्राह्याः~। तस्मात् \textbf{अद}निष्पत्ति\textbf{र्हम}निष्प-त्तितुल्यास्ति~। तस्मात् \textbf{हद}योर्घातः \textbf{अम}योर्घाततुल्यो भविष्यति~। तस्मात् \textbf{अम}योर्घातो \textbf{झव}तुल्यो \,भविष्यति~। \textbf{अः} \,द्विमितः~। तस्मात् \,\textbf{झवं \,मा}त् \,द्विगुणं \,भविष्यति~। तस्मात्
\vspace{-4mm}

\begin{flushleft}
\begin{minipage}[b]{0.5\textwidth}
 \textbf{मं झव}म् एतयोर्निष्पत्ति\textbf{र्लम}योर्निष्पत्तितुल्या भविष्यति~। पुन\textbf{र्ह}तुल्यं \textbf{कसं त}कात् पृथक् कार्यम्~। पुन\textbf{र्ह}तुल्यं \textbf{वगं झवा}त् पृथक्कार्यम्~। तस्मात् \textbf{तसह}निष्पत्ति\textbf{र्झग}स्य निष्पत्ति\textbf{र्मलतकह}योगेन या भवति तत्तुल्या भविष्यति~। \textbf{तसं ह}तुल्यमस्ति~। तस्मात् \textbf{झग}म् एतदङ्कयोगतुल्यं भविष्यति~।  \textbf{ह}तुल्यं \textbf{गवं} रूप\textbf{अबजद}योगेन तुल्यं भविष्यति~। तस्मात् \textbf{झवं} ~रूप\textbf{अबजदहतकलम}योगतुल्यं ~भवि-
\end{minipage} 
\hfill
\begin{minipage}[b]{0.45\textwidth}
\centering
\includegraphics[scale=0.5]{Images/rg-2.png}
\end{minipage}
\end{flushleft}
\vspace{-4mm}

\noindent ष्यति~। अङ्केषु प्रत्येकं \textbf{झवं} निःशेषं करोति~। तस्मात् \textbf{झव}मेतद्भागतुल्यं भविष्यति~। एतैर्विनान्येन विभागो न लभ्यते~। यदि लभ्यते तदा \textbf{न}विभागः कल्पितः~। अयं \textbf{फ}तुल्यं निःशेषं करोति~। तस्मात् \textbf{फन}योर्घातो \textbf{झवो} भविष्यति~। एवं \textbf{हद}घातो \textbf{झव}तुल्यो भवि- 

\newpage
\noindent ष्यति~। तस्मात् \textbf{हफ}निष्पत्ति\textbf{र्नद}निष्पत्तितुल्या भविष्यति~। \textbf{अबजद}मध्ये \textbf{नो} नास्ति~। तस्मात् \textbf{दं} निःशेषं न करिष्यति~। \textbf{हः फं} निःशेषं न करिष्यति~। \textbf{हः} प्रथमाङ्कोऽस्ति~। तस्मात् \textbf{हफौ} भिन्नाङ्कौ भविष्यतः~। तस्मात् \textbf{फः दं} निःशेषं करिष्यति~। \textbf{अः} प्रथमाङ्कोऽस्ति~। तस्मात् \textbf{द}म् \textbf{अबजं} विना कोऽपि निःशेषं न करिष्यति~। तस्मात् \textbf{फः} तन्मध्ये कोऽपि भविष्यति~। स च \textbf{बः} कल्पितः~। पुन\textbf{र्बद}योर्निष्पत्ति\textbf{र्हल}योर्निष्पत्तितुल्यास्ति~। \textbf{हद}योर्घातो \textbf{बल}योर्घाततुल्यो भविष्यति \textbf{झव}तुल्यश्च~। तस्मात् \textbf{बं ल}तुल्यं \textbf{झवं} निःशेषं करिष्यति~। \textbf{बः झवं न}तुल्यं निःशेषम् अकरोत्~। तस्मात् \textbf{नलौ} एकरूपौ भविष्यतः~। कल्पितौ तु भिन्नौ~। इदमशुद्धम्~। तस्मात् \textbf{झवं} विना कोऽपि विभागो न भविष्यति~। अयं स्वसर्वविभागयोगतुल्यो जातः~। सम्पूर्णाङ्कश्च जातः~। इदमेवास्माकमिष्टम्~॥~३८~॥  

\begin{quote}
\qt
श्रीमद्राजाधिराजप्रभुवरजयसिंहस्य तुष्ट्यै द्विजेद्रः \\
श्रीमत्सम्राड् जगन्नाथ इति समभिधारूढितेन प्रणीते~। \\
 ग्रन्थेऽस्मिन्नाम्नि रेखागणित इति सुकोणावबोधप्रदात- \\
 र्यध्यायोऽध्येतृमोहापह इह विरतिं नन्दतुल्यो गतोऽभूत्~॥~९~॥ 
\end{quote}
\vspace{-1mm}

\begin{center}
\textbf{इति श्रीजगन्नाथसम्राड्विरचिते रेखागणिते \\
नवमोऽध्यायः समाप्तः~॥~९~॥}\\
\vspace{6mm} 

\rule{0.9in}{0.8pt}
\end{center}

\afterpage{\fancyhead[CE] {रेखागणितम्}}
\afterpage{\fancyhead[CO] {दशमोऽध्यायः}}
\afterpage{\fancyhead[LE,RO]{\thepage}}
\cfoot{}

\newpage%%%%%%%%%%%%%%%%%%%%%%%%%%%%%%%%%%%%%%%%%%
\thispagestyle{empty}
\phantomsection \label{ch10}
\begin{center}
{\bf \LARGE~॥ अथ दशमोऽध्यायः प्रारभ्यते~॥}
\vspace{5mm}

\textbf{~॥ तत्र नवोत्तरशतमितानि क्षेत्राणि सन्ति~॥}
\vspace{5mm}

 \textbf{\large  तत्रादौ}\renewcommand{\thefootnote}{१}\footnote{\en D., V. and K. omit this sentence.} \textbf{परिभाषा~।}
\end{center}

\begin{enumerate}
\item[१] रेखाणां क्षेत्रफलस्य घनफलस्य वा यानि प्रमाणानि निःशेषकारकाणि प्राप्यन्ते तानि\renewcommand{\thefootnote}{२}\footnote{प्रमाणं निःशेषकारकं प्राप्यते तदा तानि {\en V.,~D.,~K.}} मिलितप्रमाणान्युच्यन्ते\renewcommand{\thefootnote}{३}\footnote{प्रमाणान्युच्यते {\en J. }}\;। 
\item[२] यानि प्रमाणानि निःशेषाणि न भवन्ति तानि भिन्नप्रमाणानि स्युः~। 
\item[३] यासां रेखाणां वर्गाः केनचित् क्षेत्रफलेन निःशेषा भवन्ति ता रेखा मिलितवर्गाभिधाः स्युः~। 
\item[४] यासां रेखाणां वर्गा एवं न भवन्ति ता रेखा भिन्नवर्गाभिधाः स्युः~। 
\item[५] अथैकेष्टा रेखा\renewcommand{\thefootnote}{४}\footnote{अथैकेष्टरेखा {\en J.}, अथेष्टा रेखा {\en K.}} कल्पनीया तद्व्यतिरिक्ताः कल्पितरेखास्तासु काश्चित्तस्याः सका-शात् केवलभिन्नाः\renewcommand{\thefootnote}{५}\footnote{केवलं भिन्नाः {\en J.}} स्युः काश्चिद्भिन्ना भिन्नवर्गाश्च
स्युः सा रेखा तन्मिलिताश्च रेखास्तस्या वर्गो यत्क्षेत्रफलं तद्वर्गमिलितम्\renewcommand{\thefootnote}{६}\footnote{तत्क्षेत्रफलमिलितवर्गश्च {\en D.}; तत्-
क्षेत्रवर्गमिलितश्च {\en B. }} असौ मूलद\renewcommand{\thefootnote}{७}\footnote{तन्मूलद {\en J.}}राशिरित्युच्यते~। 
\item[६] या रेखा तद्भिन्ना भवति यत्क्षेत्रफलं तद्वर्गाद्भिन्नं भवति यद्रेखावर्गस्तत्क्षेत्रतुल्यो भवति ते \renewcommand{\thefootnote}{८}\footnote{ते वर्गाः करणशब्दवाच्या भवन्ति
{\en J. }}करणीशब्दवाच्या भवन्ति~। 
\end{enumerate}
\begin{center}
~॥ इति परिभाषा~॥ 
\vspace{5mm}

\textbf{\large अथ प्रथमं क्षेत्रम्}\renewcommand{\thefootnote}{९}\footnote{प्रथमक्षेत्रम् {\en V.}}\textbf{\large ~॥~१~॥}
\end{center}

{\ab  बृहल्लघुप्रमाणद्वयमस्ति~। तत्र बृहत्प्रमाणे किञ्चिदधिकमर्द्धं 
शोध्यं यच्छेषं तस्मात् किञ्चिदधिकमर्द्धं पुनः शोध्यमेवं मुहुःकरणेन यदन्तिमं लघुखण्डमुत्पन्नं तल्लघुराशेर्न्यूनं भविष्यति~। }

\newpage
 यथा बृहत्प्रमाणम् \textbf{अबं} कल्पितम्~। लघुप्रमाणं \textbf{जं} कल्पितम्~। पुन\textbf{र्ज}प्रमाणस्य यावद्गुणाः कल्प्या\renewcommand{\thefootnote}{१}\footnote{ग्राह्याः {\en J., V.}} यथा \textbf{अबा}दधिका भवन्ति~। ते च \textbf{लस}सञ्ज्ञकाः कल्प्याः~। पुनः प्रत्येकं \textbf{लमं मनं नसं ज}तुल्यं कल्पितम्~। पुनः \textbf{अबा}त् \textbf{बतं} किञ्चिदधिकमर्द्धं पृथक्कार्यम्~। पुनः \textbf{अता}त् किञ्चिदधिकमर्द्धं \textbf{तकं} पृथक्कार्यम्~। एवं मुहुः कार्यम्~। यावन्तो \textbf{लसे ज}विभागाः सन्ति तावन्त एव \textbf{अबे} यथा विभागा\renewcommand{\thefootnote}{२}\footnote{{\en J. omits} विभागाः} भवन्ति  तावत्पर्यन्तं कार्याः~। ते च \textbf{बततककअ}सञ्ज्ञका भवन्ति~। तस्माच्छेषं \textbf{कअं जा}न्न्यूनं भविष्यति~। 

\begin{center}
अस्योपपत्तिः~। 
\end{center}
\vspace{-10mm}

\begin{flushleft}
\begin{minipage}[b]{0.6\textwidth}
\hspace{4mm} \textbf{अक}स्य तावन्तो घाताः पूर्वतुल्या ग्राह्याः~। ते च \textbf{दह}सञ्ज्ञकाः कल्प्याः~। तस्मात् \textbf{दह}म् \textbf{अबा}न्न्यूनं भविष्यति~। कुतः~। \textbf{दझ}स्य \textbf{अक}तुल्यत्वात्~। \textbf{झवं कता}न्न्यूनमस्ति~। \textbf{वहं तबा}न्नितान्तं न्यूनमस्ति~। पुनः \textbf{अबं सला}न्न्युनमस्ति~। तस्मात् \textbf{दहं सला}त् नितान्तमल्पं भविष्यति~। पुर्न\textbf{र्दझसन}योर्निष्पत्ति\textbf{र्झवनम}-निष्पत्तितुल्यास्ति \textbf{वहमल}योर्निष्पत्तेरपि तुल्यास्ति\hyperref[f4.1]{$^{\scriptsize{\hbox{३}}}$}\;। तस्मात् \textbf{दहसल}निष्पत्ति\textbf{र्दझसन}निष्पत्तितुल्या भविष्यति~। \textbf{दहं सला}न्न्यूनमस्ति~। तस्मात् \textbf{दझ}तुल्यम् \textbf{अकं सन}तुल्यात् \textbf{जा}न्न्यूनं भविष्यति~। इदमेवास्माकमिष्टम्~॥ 
\end{minipage} 
\hfill
\begin{minipage}[b]{0.3\textwidth}
\includegraphics[scale=0.7]{Images/rg-3.png}
\end{minipage}
\end{flushleft}
\vspace{-3mm}

\blfootnote{\phantomsection \label{f73}
$^{\tiny{\hbox{३}}}${\footnotesize \textbf{वहमल}योरपि निष्पत्तेस्तुल्यास्ति {\en J.}}}

\begin{center}
प्रकारान्तरम्~।
\end{center}

{\ab  न्यूनाधिकप्रमाणयोर्मध्ये बृहत्प्रमाणात् कोऽपि विभागः शोध्यः~। पुनः शेषात् तन्निष्पत्तितुल्यो\renewcommand{\thefootnote}{४}\footnote{तुल्यविभागः~{\en J.,~V.}} विभागः शोध्यः~। एतत्त-} 

\newpage
\noindent {\ab च्छेषादपि~। चरमावशिष्टं प्रमाणं लघुप्रमाणान्न्यूनं भविष्यति~।} 

\begin{flushleft}
\begin{minipage}[b]{0.55\textwidth}
\hspace{4mm}  यथा \textbf{गफफछ}योर्निष्पत्तिः कल्पिता~। पुनः \textbf{सनं ज}तुल्यं पृथक्कार्यम्~। \textbf{सननख}योर्निष्पत्तिः \textbf{गफफछ}निष्पत्तितुल्या कार्या~। तस्मात् \textbf{सखं जा}त् स्वल्पं भविष्यति~। \textbf{सखखन}योर्निष्पत्ति\textbf{र्गछ-छफ}योर्निष्पत्तितुल्या भविष्यति~। पुनः \textbf{खन}स्य यावन्तो घाता \textbf{अबा}दधिका \textbf{दहाः} कल्पिताः~। पुनः \textbf{सननम}योर्निष्पत्तिः \textbf{सममल}निष्पत्तिश्च \textbf{गछ-छफ}निष्पत्तितुल्या \;कार्या~। \,एवं \;तावत् \,कार्यं यावत् \textbf{खननममला दह}मध्ये \textbf{खन}तुल्या भवन्ति\;। पुन\textbf{र्नखखस}निष्पत्तिः \textbf{मननस}निष्पत्तितुल्यास्ति~। पुन\textbf{र्नखमन}निष्पत्तिः \textbf{खसनस}निष्पत्तितुल्यास्ति~। \textbf{खस}श्च \textbf{नसा}त् स्वल्पोऽस्ति~। तस्मा\textbf{न्नखं मना}त्
\end{minipage} 
\hfill
\begin{minipage}[b]{0.3\textwidth}
\includegraphics[scale=0.55]{Images/rg-4.png}
\end{minipage}
\end{flushleft}
\vspace{-3mm}

\noindent स्वल्पं भविष्यति~। एवं हि \textbf{मनं लमा}त् स्वल्पं भविष्यति~। तस्मात् \;सम्पूर्णं \;\textbf{खलं \;दहा}दधिकं  भविष्यति~। इदं च \textbf{अबा}दधिकम् अस्ति~। तस्मात् सम्पूर्णः \textbf{खलः अबा}दधिको भविष्यति~। \textbf{सलः} अस्मादत्यधिकोऽस्ति~। पुनः प्रत्येक\textbf{सललम}निष्पत्तिः \textbf{सममन}निष्पत्तिः \textbf{सननख}निष्पत्तिश्च \,\textbf{गफफछ}योर्निष्पत्तितुल्यास्ति~। \,अस्यां \,निष्पत्तौ \,\textbf{अबा}त् \,\textbf{बशं} \,पृथक् कार्यम्~। \textbf{अशा}त् \textbf{शतं अता}त् \textbf{तकं} पृथक्कार्यं यावत् \textbf{अब}विभागाः \textbf{सल}भागसमानास्तस्याम् एव निष्पत्तौ भवन्ति~। तस्मात् \textbf{अकअब}योर्निष्पत्तिः \textbf{खससल}निष्पत्तितुल्या भविष्यति~। पुनः \,\textbf{अकसख}निष्पत्तिः \,\textbf{अबसल}निष्पत्तितुल्या \,भविष्यति~। \,\textbf{अबः \,सला}न्न्यूनोऽस्ति~। तस्मात् \textbf{अकं सखा}न्न्यूनं भविष्यति~। तच्च \textbf{जा}न्न्यूनमस्ति~। तस्मात् \textbf{अकं जा}न्नितान्तं स्वल्पं भविष्यति~। इदमेवेष्टम्~॥ 
\vspace{2mm}
 
\begin{center}
\textbf{\large अथ द्वितीयं क्षेत्रम्}\renewcommand{\thefootnote}{१}\footnote{द्वितीयक्षेत्रम् {\en V.}}\;\textbf{\large ॥~२~॥}
\end{center}

{\ab न्यूनाधिकप्रमाणयोर्मध्येऽधिकप्रमाणान्न्यूनं प्रमाणं शोध्यं }

\newpage

\noindent {\ab तावद्यावच्छेषं \;न्यूनप्रमाणात् \;स्वल्पमवशिष्यते~। पुनर्न्यूनप्रमाणात् \;स्वल्पं शोध्यम्~। पुनस्तच्छेषं तच्छेषाच्छोध्यम्~। एवं मुहुः कार्यम्~। यद्येवं निःशेषं न भवति तदा ते प्रमाणे भिन्ने स्तः~।} 

\begin{flushleft}
\begin{minipage}[b]{0.65\textwidth}
\hspace{4mm}  यथा \textbf{अबजदं} प्रमाणद्वयं तादृशं कल्पितम्~। यद्येते प्रमाणे भिन्ने न भवतस्तदोभयोरपवर्तक\textbf{स्तः} कल्पितः~। पुन\textbf{र्जदं अबा}त्तावच्छोध्यं यथा \textbf{अहं} शेषं \textbf{जदा}न्न्यूनमव-शिष्यते~। पुन\textbf{रहं जदा}च्छोध्यं शेषं \textbf{जझं} तच्च \textbf{अहा}च्छोध्यं शेषम्  \textbf{अव}म्~।  \textbf{हब}म् \textbf{अबा}र्द्धादधिकमस्ति~। \textbf{हवं अहा}र्द्धा-दधिकमस्ति~। अनेन प्रकारेण शेषं \textbf{ता}न्न्यूनं भविष्यति~। तच्च \textbf{अवं} कल्पितम्~। पुन\textbf{स्तः दजं} निःशेषं करोति~। तस्मात् \textbf{हब}म् अपि निःशेषं करिष्यति~। \textbf{अबं} च पूर्वमेव निःशेषमकरोत्~। तस्मा\textbf{दह}मपि निःशेषं करिष्यति~। इदं
\end{minipage} 
\hfill
\begin{minipage}[b]{0.27\textwidth}
\includegraphics[scale=0.7]{Images/rg-5.png}
\end{minipage}
\end{flushleft}
\vspace{-3mm}

\noindent \textbf{झदं} निःशेषं करोति~। \textbf{जदं} च पूर्वमेव निःशेषमकरोत्~। तस्मात् \textbf{जझ}मपि निःशेषं करिष्यति~। इदं \textbf{हवं}  निःशेषं करिष्यति~। \textbf{तं हवं} निःशेषं करिष्यति~। \textbf{अहं} निःशेषमकरोत्~। तस्मा\textbf{दव}मपि निःशेषं करिष्यति~। \textbf{अवं ता}च्च लघुरस्ति~। इदमशुद्धम्~। इष्टं समीचीनम्~॥ 
\vspace{2mm}

\begin{center}
\textbf{\large अथ तृतीयं क्षेत्रम्}\renewcommand{\thefootnote}{१} \footnote{तृतीयक्षेत्रम् {\en V.}}\;\textbf{\large ॥~३~॥}
\end{center}

 {\ab \renewcommand{\thefootnote}{२}\footnote{{\en J. inserts} तत्र {\en before} महत्प्रमाणस्य.}महत्प्रमाणस्य मिलितप्रमाणद्वयनिःशेषकारकस्योत्पादनं 
चिकीर्षितमस्ति~। }\\

 यथा \textbf{अबजद}प्रमाणे मिलिते कल्पिते~। तस्माद्यदि लघुप्रमाणं \textbf{जद}म् \textbf{अबं} निःशेषं करोति तदेदमेवेष्टम्\renewcommand{\thefootnote}{३}\footnote{तदेवमेवे {\en D.}}~। यदि न करोति तदा \textbf{जदा}न्न्यूनं \textbf{अह}मवशिष्टं कल्पितम्~। इदं \textbf{जदं} निःशेषं करिष्यति~। अनेन\renewcommand{\thefootnote}{४}\footnote{अनेनैव {\en J., V.}} प्रका- 

\newpage

\begin{flushleft}
\begin{minipage}[b]{0.6\textwidth}
रेण चरमं तादृशप्रमाणमुत्पन्नं स्यात् यत् स्वोपरिस्थ-प्रमाणानि\hyperref[f76]{$^{\scriptsize{\hbox{१}}}$} निःशेषयिष्यति~। यतो मिलितप्रमाणे स्तः~। तस्मात् कल्पितं \textbf{जझ}म् \textbf{अहं} निःशेषं करोति~। इदं महत्प्रमाणं प्रमाणद्वयमपि निःशेषयति~। यदि इदं महत्प्रमाणं \,न \,भवति\hyperref[f76]{$^{\scriptsize{\hbox{२}}}$} \,तदा \,\textbf{वं} \,महत्प्रमाणं \,कल्पितं\hyperref[f76]{$^{\scriptsize{\hbox{३}}}$} यद्द्वयं निःशेषयति~। तस्मादिदं \textbf{जदं} निःशेषं करि-ष्यति~। \textbf{हब}मपि निःशेषं करिष्यति~। \textbf{अबं} निशेषं करोति स्म~। तस्मात् \textbf{अहं} निःशेषं करिष्यति~। \textbf{झदं} निःशेषं  करिष्यति~। \textbf{जझं} निःशेषं करिष्यति~। \textbf{जझं वा}ल्लघ्वस्ति\hyperref[f76]{$^{\scriptsize{\hbox{४}}}$}\;। इदमशुद्धम्~। अस्मदिष्टं समीचीनम्~॥
\end{minipage} 
\hfill
\begin{minipage}[b]{0.3\textwidth}
\includegraphics[scale=0.65]{Images/rg-6.png}
\end{minipage}
\end{flushleft}

\blfootnote{\phantomsection \label{f76}
$^{\tiny{\hbox{१}}}${\footnotesize यत् स्वोपरिप्रमाणानि {\en D., K., V. }} \hspace{4mm} $^{\tiny{\hbox{२}}}${\footnotesize महत् प्रमाणमिदं न भवति {\en J.}} \hspace{4mm} $^{\tiny{\hbox{३}}}${\footnotesize द्वयोर्निःशेषकारकं महत् प्रमाणं कल्पितम्~{\en J.}} \hspace{4mm} $^{\tiny{\hbox{४}}}${\footnotesize {\en K. has} इदं लघ्वस्ति {\en for} \textbf{जझं वा}त् लघ्वस्ति.}}

 अनेन क्षेत्रेणेदं निश्चितं यत् प्रमाणं प्रमाणद्वयं निःशेषं करोति तत् प्रमाणद्वयनिःशेषकारकं महत् प्रमाणं च निःशेषयति~। 
\vspace{2mm}

\begin{center}
\textbf{\large अथ चतुर्थं क्षेत्रम्~॥~४~॥}
\end{center}

 {\ab बहूनां मिलितप्रमाणानां निःशेषकारकं महत् प्रमाणं\renewcommand{\thefootnote}{५}\footnote{मिलितप्रमाणनिःशेषकारकमहत् प्रमाणं {\en J.}} चिकीर्षितमस्ति~। }

\begin{flushleft}
\begin{minipage}[b]{0.6\textwidth}
\hspace{4mm} यथा \textbf{अबजा} मिलितप्रमाणानि कल्पितानि~। \textbf{अब}निःशेषकारकं महत् प्रमाणं \textbf{दं} कल्पितम्~। यदि \textbf{दः जं} निःशेषं करोति तदिदं महत् प्रमाणं त्रयाणामपि निःशेषकारकमस्ति~। यदिदं महत् प्रमाणं न भवति तदा \textbf{हं} महत् प्रमाणं कल्पितम्~। तदिदम् \textbf{अबं} निःशेषं करिष्यति~। \textbf{द}मपि निःशेषयति~। \textbf{द}श्च लघुरस्ति~। इदमशुद्धम्~॥
\vspace{2mm}
\end{minipage} 
\hfill
\begin{minipage}[b]{0.3\textwidth}
\includegraphics[scale=0.5]{Images/rg-7.png}
\end{minipage}
\end{flushleft}
 
\newpage
\begin{flushleft}
\begin{minipage}[b]{0.58\textwidth}
\hspace{4mm} यदि \,\textbf{दं जं} \,निःशेषं न \,करोति \,तदा \textbf{हं} \,महत् प्रमाणं कल्पितम्~। एतच्चैतद्वयं निःशेषं करोति~। \textbf{हः दं} निःशेषयति~। तदा \textbf{अब}मपि निःशेषं करोति\hyperref[f77]{$^{\scriptsize{\hbox{१}}}$}\;। तस्मादिदं महत् प्रमाणमस्ति यतस्त्रयमपि निःशेषं करोति~। यदीदं न करोति\hyperref[f77]{$^{\scriptsize{\hbox{२}}}$} तदा \textbf{झं} महत् प्रमाणं कल्पितम्~। \textbf{झ}म् \textbf{अबौ} निःशेषं करिष्यति~। \hyperref[f77]{$^{\scriptsize{\hbox{३}}}$}तदा \textbf{झं द}मपि निःशेषं करिष्यति~। पुनः स \textbf{दं जं} निःशेषं करोति~। तदा \textbf{ह}मपि निःशेषं करिष्यति~। इदं च तस्माल्लघ्वस्ति~। इदमशुद्धम्~। अस्मदिष्टं समीचीनम्~॥
\end{minipage} 
\hfill
\begin{minipage}[b]{0.3\textwidth}
\includegraphics[scale=0.7]{Images/rg-8.png}
\end{minipage}
\end{flushleft}
\vspace{-1mm}

\blfootnote{\phantomsection \label{f77}
$^{\tiny{\hbox{१}}}${\footnotesize करिष्यति {\en J., V.}} \hspace{4mm} $^{\tiny{\hbox{२}}}${\footnotesize भवति {\en K., J., V.}} \hspace{4mm} $^{\tiny{\hbox{३}}}${\footnotesize {\en This sentence is omitted in D. and J. They read the next sentence as follows:}\textendash \,पुन\textbf{र्दं जं झं} निःशेषं करोति D. पुन\textbf{र्झं दं जं} निःशेषं करोति {\en J. }}}

\begin{center}
\textbf{\large अथ पञ्चमं क्षेत्रम्~॥~५~॥}
\end{center}

{\ab  मिलितयोः प्रमाणयो\renewcommand{\thefootnote}{४}\footnote{मिलितप्रमाणयो {\en J.}}र्निष्पत्तिर्द्वयोरङ्कयोर्निष्पत्तिुल्या भवति~। }

\begin{flushleft}
\begin{minipage}[b]{0.63\textwidth}
\hspace{4mm} यथा \textbf{अब}प्रमाणे द्वे मिलिते कल्पिते~। \textbf{हं} प्रमाणं तृतीयं कल्पनीयं येन द्वयोरपवर्तः स्यात्~। \textbf{हः अं} यावद्वारं निःशेषयति तत्र लघ्वङ्कः \textbf{जः} कल्पनीयः\hyperref[f77.2]{$^{\scriptsize{\hbox{५}}}$}\;। \textbf{ह}प्रमाणं \textbf{ब}प्रमाणं यावद्वारं निःशेषयति तत्प्रमाणं \textbf{दः} कल्पितः\hyperref[f77.2]{$^{\scriptsize{\hbox{६}}}$}\;। तस्मात् \textbf{हअ}निष्पत्तिः रुप\textbf{ज}निष्पत्तितुल्या भविष्यति~। \textbf{अह}योर्निष्पत्ति\textbf{र्ज}रूपयोर्निष्पत्तितुल्यास्ति~। \textbf{हब}योर्निष्पत्तिः रूप\textbf{द}योर्निष्पत्तितुल्यास्ति~। तस्मात् \textbf{अब}योर्निष्पत्ति\textbf{र्जद}निष्पत्तितुल्या भविष्यति~। एतौ \textbf{जदा}वङ्कौ स्तः~। इदमेवास्माकमिष्टम्~॥
\vspace{4mm}
\end{minipage} 
\hfill
\begin{minipage}[b]{0.28\textwidth}
\includegraphics[scale=0.65]{Images/rg-9.png}
\end{minipage}
\end{flushleft}

\blfootnote{\phantomsection \label{f77.2}
$^{\tiny{\hbox{५}}}${\footnotesize तत् प्रमाणं \textbf{जं} कल्पितं {\en J.}} \hspace{4mm} $^{\tiny{\hbox{६}}}${\footnotesize यावद्वारं \textbf{हं बं} निःशेषं करोति तदङ्कं \textbf{दं} कल्पितम् {\en J.}}}

\newpage
\begin{center}
 \textbf{\large अथ}\renewcommand{\thefootnote}{१}\footnote{अथ {\en is omitted in V.}}  \textbf{\large षष्ठं क्षेत्रम्~॥~६~॥ }
\end{center}

 {\ab ययोर्द्वयोः प्रमाणयोर्निष्पत्तिर्द्वयोरङ्कयोर्निष्पत्तितुल्या भवति\renewcommand{\thefootnote}{२}\footnote{ भविष्यति {\en J.  }} ते मिलितप्रमाणे भवतः~। }\\

 यथा \textbf{अबं}\renewcommand{\thefootnote}{३}\footnote{ \textbf{अबौ} {\en V.}} प्रमाणे कल्पिते~। \textbf{जदा}वङ्कौ कल्पितौ~। \textbf{अब}निष्पत्ति\textbf{र्जद}निष्पत्तितुल्या कल्पिता~। तदा \textbf{अबौ} मिलितौ भविष्यतः~। 

\begin{center}
अस्योपपत्तिः~।
\end{center}
\vspace{-5mm}

\begin{flushleft}
\begin{minipage}[b]{0.55\textwidth}
\hspace{4mm} \textbf{अ}प्रमाणस्य \,\textbf{ज}तुल्या \,विभागाः \,कल्पिताः~। तस्मात् \textbf{ह}प्रमाणमुत्पन्नं जातम्~। पुन\textbf{र्ह}स्य \textbf{द}तुल्या घाता ग्राह्याः~। लब्धाङ्को \textbf{झ}सञ्ज्ञोऽस्ति\hyperref[f78]{$^{\scriptsize{\hbox{४}}}$}\;। तस्मात् \textbf{अह}निष्पत्ति\textbf{र्ज}रूपनिष्पत्तितुल्या भविष्यति~। \textbf{हझ}निष्पत्तिरूप\textbf{द}निष्पत्तितुल्या भविष्यति~। तस्मात् \textbf{अझ}निष्पत्ति\textbf{र्जद}निष्पत्तितुल्या भविष्यति~। \;\textbf{अब}निष्पत्तितुल्यापि \;भविष्यति~। तस्मात् \textbf{बझौ} समानौ भविष्यतः~। \textbf{अझौ} मिलितप्रमाणौ\hyperref[f78]{$^{\scriptsize{\hbox{५}}}$} स्तः~। तस्मात् \textbf{अबौ} मिलितप्रमाणौ भविष्यति~। इदमस्मदिष्टम्\hyperref[f78]{$^{\scriptsize{\hbox{६}}}$}\;।
\vspace{2mm}
\end{minipage} 
\hfill
\begin{minipage}[b]{0.38\textwidth}
\includegraphics[scale=0.6]{Images/rg-10.png}
\end{minipage}
\end{flushleft}
\vspace{-1mm}

\blfootnote{\phantomsection \label{f78}
$^{\tiny{\hbox{४}}}${\footnotesize \textbf{झ}संज्ञकः {\en K.,} \textbf{झ}संज्ञकोऽस्ति {\en V.}} \hspace{4mm} $^{\tiny{\hbox{५}}}${\footnotesize \textbf{अझे} मिलितप्रमाणे {\en J.}} \hspace{4mm} $^{\tiny{\hbox{६}}}${\footnotesize दिष्ट समीचीनम् {\en J.}}}

\begin{center}
 \textbf{\large अथ सप्तमं क्षेत्रम्~॥~७~॥}
\end{center}

{\ab  द्वयोर्मिलितरेखावर्गयोर्निष्पत्तिर्द्वयोरङ्कवर्गयोर्निष्पत्तितुल्या भवति~। यदि रेखा-द्वयवर्गयोर्निष्पत्तिरङ्कवर्गयोर्निष्पत्तितुल्या\renewcommand{\thefootnote}{७}\footnote{र्निष्पत्तेस्तुल्या {\en V.}} भवति तदा ते रेखे मिलिते भवतः~। यद्यङ्कवर्गयोर्निष्पत्ती रेखावर्गतुल्या न भवति तदा ते रेखे भिन्ने ज्ञातव्ये~। }

\newpage
\begin{wrapfigure}{r}{0.45\textwidth}
\vspace{-8mm}
\begin{center}
\includegraphics[scale=0.6]{Images/rg-11.png}
\end{center}
\vspace{-8mm}
\end{wrapfigure}

यथा \textbf{अब}रेखाद्वयं कल्पितम् यदि ते मिलिते रेखे भवतस्तदैतयोर्निष्पत्तिर्द्वयोरङ्कयोर्निष्पत्तितुल्या\renewcommand{\thefootnote}{१}\footnote{र्निष्पत्तेस्तुल्या {\en V.}} भविष्यति~। तौ\renewcommand{\thefootnote}{२}\footnote{{\en J. omits} तौ.} द्वावङ्कौ \textbf{जदौ} कल्पितौ~। \textbf{अब}योर्वर्गयोर्निष्पत्तिः \textbf{अब}-निष्पत्तिवर्गतुल्या भविष्यति~। \textbf{जद}वर्गयोर्निष्प-त्ति\textbf{र्जद}निष्पत्तिवर्गो भविष्यति~। \textbf{जद}निष्पत्तिः \textbf{अब}निष्पत्तितुल्यास्ति~। तस्माद्द्वयो रेखावर्गयोर्निष्पत्तिर्द्वयोरङ्कवर्गयोर्निष्पततितुल्या जाता~। \\

\begin{wrapfigure}{r}{0.3\textwidth}
\vspace{-8mm}
\begin{center}
\includegraphics[scale=0.7]{Images/rg-12.png}
\end{center}
\vspace{-8mm}
\end{wrapfigure}

 पुनरपि \textbf{अब}योर्वर्गयोर्निष्पत्ति\textbf{र्जद}योर्वर्गयोर्निष्पत्तितुल्या कल्पिता~।  \textbf{हझौ जद}स्य भुजौ कल्पितौ~। तस्माद्रेखावर्ग-योर्निष्पत्ती रेखानिष्पत्तिवर्गतुल्या जाता~। \textbf{जद}निष्पत्ति\textbf{र्हझ}-निष्पत्तिवर्गोऽस्ति~। तस्माद्रेखयोर्निष्पत्तिरङ्कयोर्निष्पत्तितुल्या जाता~। तस्मात्ते रेखे मिलिते सम्पन्ने\renewcommand{\thefootnote} {३}\footnote{जाते {\en J.}}\;।\\

पुनरपि रेखावर्गयोर्निष्पत्तिरङ्कद्वयवर्गनिष्पत्तितुल्या न भवति तदा ते रेखे भिन्ने भवतः~। यदि भिन्ने न भवतः तदा मिलिते कल्पिते~। तदा\renewcommand{\thefootnote}{४}\footnote{{\en J.~Omits} तदा.} अनयोर्वर्गनिष्पत्तिरङ्कद्वयवर्ग-नि\renewcommand{\thefootnote}{५}\footnote{र्वर्गयोर्नि}ष्पत्तितुल्या भविष्यति~। इदमशुद्धम्~। अस्मदिष्टं समी-चीनम्~॥ \\

 अनेनेदं निश्चितं रेखे यदि मिलिते स्यातां\renewcommand{\thefootnote}{६}\footnote{{\en J. Omits} स्याताम्.} तयोर्वर्गावपि मिलितौ भवतः~। यदि रेखावर्गौ भिन्नौ तदा रेखे अपि भिन्ने भवतः~। अस्य विलोमता नास्ति~॥ 

\newpage
\begin{center}
\textbf{\large अथाष्टमं क्षेत्रम्~॥~८~॥}
\end{center}

{\ab  यानि \,चत्वारि \,प्रमाणानि \,सजातीयानि \,सन्ति \,तेषु \,प्रथमद्वितीयौ \,यदि मिलितौ स्तस्तदा तृतीयचतुर्थावपि मिलितौ भविष्यतः~। यदा\renewcommand{\thefootnote}{१}\footnote{यदि {\en V.}} प्रथमद्वितीयौ भिन्नौ भवतस्तदा तृतीयचतुर्थावपि भिन्नौ भविष्यतः~।} \\

\begin{wrapfigure}{r}{0.3\textwidth}
\vspace{-8mm}
\begin{center}
\includegraphics[scale=0.6]{Images/rg-13.png}
\end{center}
\vspace{-8mm}
\end{wrapfigure}

 यथा \textbf{अबजदा}श्चत्वारि प्रमाणानि सजातीयानि कल्पितानि\renewcommand{\thefootnote}{२}\footnote{चत्वारः प्रमाणाः सजातीयाः कल्पिताः {\en D., K., V. }}\;। तत्र \textbf{अबौ} यदि मिलितौ स्यातां तदा तौ द्वयोर-ङ्कयोर्निष्पत्तौ स्याताम्~। \textbf{जदा}वप्यङ्कयोर्निष्पत्तौ भविष्यतः~। तदा \textbf{जद}रेखे मिलिते भविष्यतः~।\renewcommand{\thefootnote}{३}\footnote{\en This sentence is omitted in K. and V.} यदि \textbf{अबौ}  भिन्नौ \textbf{जदा}वपि  \,भिन्नौ\renewcommand{\thefootnote}{४}\footnote{{\en J. inserts} तदा {\en after} भिन्नौ.} \,भविष्यतः~। \,कुतः~। \,यदि \,भिन्नौ \,न भवतः मिलितौ भवतस्तदा द्वयोरङ्कयोर्निष्पत्तौ भविष्यतः~। \textbf{अबा}वप्येतादृशौ भविष्यतः~। इदम् अशुद्धम्~। अस्मदिष्टं समी\renewcommand{\thefootnote}{५}\footnote{इष्टमस्मत्समी {\en V.}}चीनम्~॥ \\

 यदि प्रमाणानि रेखा भवन्ति तत्र \textbf{अब}वर्गौ मिलितौ वा भिन्नौ भवतस्तदा \textbf{जदा}वप्येता-दृशौ भविष्यतः~। कुतः~। अनयोर्वर्गयोः सजातीयत्वात्~॥ 
\vspace{2mm}

\begin{center}
\textbf{\large अथ नवमं क्षेत्रम्~॥~९~॥}
\end{center}

{\ab तादृशं रेखा\renewcommand{\thefootnote}{६}\footnote{तादृशरेखा {\en J.}}द्वयमुत्पादनीयं यथेष्टरेखया\renewcommand{\thefootnote}{७}\footnote{इष्टया रेखया {\en K., J., V.}} प्रत्येकं भिन्नं स्यात्~। तयोरेकस्या रेखाया वर्गः कल्पितरेखावर्गाद्भिन्नः स्यात्तथा कल्पनीयो भवति~।} \\

 यथा इष्टरेखा \textbf{अं} कल्पिता~। ययोरङ्कयोर्निष्पत्तिर्वर्गनिष्पत्तितुल्या 

\newpage

\begin{wrapfigure}{r}{0.4\textwidth}
\vspace{-4mm}
\begin{center}
\includegraphics[scale=0.6]{Images/rg-14.png}
\end{center}
\vspace{-8mm}
\end{wrapfigure}

\noindent न भवति तथा द्वावङ्कौ ग्राह्यौ~। तावङ्कौ \textbf{बजौ}  कल्पितौ~। पुनः \textbf{अ}वर्ग\textbf{द}वर्गयोर्निष्पत्तिस्तयोरङ्कयोः निष्पत्तितुल्या कार्या~। तस्मात् \textbf{द}म् \textbf{अ}सञ्ज्ञाद्भिन्नं भविष्यति~। कुतः~। अनयोर्वर्गौ द्वयोरङ्कवर्गनि-ष्पत्तौ न स्तः~। अनयोर्वगौ मिलितौ भविष्यतः~। कुतः\renewcommand{\thefootnote}{१}\footnote{यतः {\en J.}}\;। अनयोर्वर्गनिष्पत्तिः द्वयोरङ्कयोर्निष्पत्तितु-ल्यास्ति~। पुनः \textbf{अद}रेखयोर्मध्ये \textbf{ह}रेखा एकनिष्पत्तौ निष्कास्या~। तस्मादिमे \textbf{अ}रेखा\textbf{ह}रेखे\renewcommand{\thefootnote}{२}\footnote{एते \textbf{अह}रेखे {\en J.}} भिन्ने भविष्यतः~। अनयोर्वर्गावपि भिन्नौ भविष्यतः~। कुतः\renewcommand{\thefootnote}{३}\footnote{यतः {\en J. }}\;। \textbf{अ}वर्ग\textbf{ह}वर्गयोः निष्पत्तिः \textbf{अद}निष्पत्तितुल्यास्ति~। \textbf{अद}निष्पत्तिः \textbf{अह}निष्पत्तिवर्गतुल्यास्ति~। \textbf{अः दा}द्भिन्नोऽस्ति~। तस्मात् \textbf{अह}वर्गावपि भिन्नौ भविष्यतः~। ययोर्वर्गौ भिन्नौ भवतस्तौ मिथोऽपि भिन्नौ भवतः~। इदमेवास्माकमिष्टम्\renewcommand{\thefootnote}{४}
\footnote{इत्येवेष्टम् {\en J. }}\;॥ 
\vspace{2mm}

\begin{center}
\textbf{\large अथ दशमं क्षेत्रम्~॥~१०~॥}
\end{center}

{\ab  एकप्रमाणेन यावन्ति\renewcommand{\thefootnote}{५}\footnote{{\en K. inserts} अन्ये {\en here; J. has} अन्यानि.} प्रमाणानि मिलितानि सन्ति तानि  मिथोऽपि मिलितानि स्युः\renewcommand{\thefootnote}{६}\footnote{भवन्ति {\en J. }}~।}\\

\begin{wrapfigure}{r}{0.5\textwidth}
\vspace{-10mm}
\begin{center}
\includegraphics[scale=0.6]{Images/rg-15.png}
\end{center}
\vspace{-8mm}
\end{wrapfigure}

 यथा \,\textbf{अबौ} \,द्वे \,प्रमाणे \,\textbf{ज}प्रमाणेन मिलिते कल्पिते~। \textbf{अज}प्रमाणयोर्निष्पत्ति-\textbf{र्दहा}ङ्कयोर्निष्पत्तेस्तुल्या कल्पिता~। पुन\textbf{र्जब}-प्रमाणयोः निष्पत्तिः \textbf{झवा}ङ्कनिष्पत्तितुल्या कल्पिता~।\\
 
  अस्यां निष्पत्तौ त्रयो लघ्वङ्का\textbf{स्तकला} ग्राह्याः~। तत्र \textbf{अब}प्रमाणयोर्निष्पत्ति\textbf{स्तला}-ङ्कयोर्निष्पत्तितुल्या भविष्यति~। तस्मादेते द्वे प्रमाणे मिलिते भवतः\renewcommand{\thefootnote}{७}\footnote{भविष्यतः {\en J.}}\;। इदमेवेष्टम्~॥ \\
\vspace{10mm}

{\color{white}अ} 
\newpage

\begin{center}
\textbf{\large अथैकादशं क्षेत्रम्~॥~११~॥}
\end{center}

{\ab  यदि द्वे प्रमाणे मिलिते भवतस्तदा तयोर्योगोऽपि तेन मिलितो भवति\renewcommand{\thefootnote}{१}\footnote{भविष्यति {\en J., V.}} तयोरन्तरमपि ताभ्यां मिलितं भविष्यति~। }\\

\begin{wrapfigure}{r}{0.48\textwidth}
\vspace{-6mm}
\begin{center}
\includegraphics[scale=0.5]{Images/rg-16.png}
\end{center}
\vspace{-8mm}
\end{wrapfigure}

 यथा \;\textbf{अबबजे} ~द्वे \;प्रमाणे ~मिलिते कल्पिते~। अनयोरपवर्तको \textbf{दः} कल्पितः~। तदा \textbf{दो}ऽपि अनयोर्योगस्याप्यपवर्तको भविष्यति\renewcommand{\thefootnote}{२}\footnote{तदानयोर्योगस्यापि \textbf{दो}ऽपवर्तको भवि-ष्यति~। {\en J.}}\;। \\

 यदि \textbf{दः} उभयोर्योगमेकं प्रमाणं च निःशेषं करोति तदा\renewcommand{\thefootnote}{३}\footnote{यदि \textbf{दः} योगं निःशेषं करोति \textbf{द}मुभयोः (~एकं~) प्रमाणं च निःशेषं करोति तदा {\en \& c. J.}} द्वितीयप्रमाणमपि निःशेषं करिष्यति~। इदमेवास्माकमिष्टम्\renewcommand{\thefootnote}{४}\footnote{इदमेवेष्टम् {\en J.}
}\;॥ 
\vspace{2mm}

\begin{center}
\textbf{\large अथ द्वादशं क्षेत्रम्~॥~१२~॥}
\end{center}

{\ab यत्र चतस्रो रेखाः सजातीया भवन्ति तत्र यदि प्रथमरेखावर्गो द्वितीयरेखावर्गप्रथममिलितान्यरेखावर्गयोगतुल्यो भवति तदा तृतीयरेखावर्गश्चतुर्थरेखावर्गतृतीयरेखामिलितान्यरेखावर्गयोगतुल्यो भविष्यति~। यदि प्रथमरेखावर्गो द्वितीयरेखावर्गस्य प्रथमरेखाभिन्नान्यरेखावर्गस्य च योगेन तुल्यो भवति तदा तृतीयरेखावर्गोऽपि चतुर्थरेखावर्गस्य तृतीयरेखाभिन्नान्यरेखावर्गस्य च योगेन तुल्यो भवति~। }\\

 यथा \textbf{अबजदा}श्चतस्रो रेखाः सजातीयाः कल्पिताः~। \textbf{अ}रेखावर्गो 

\newpage

\begin{wrapfigure}{r}{0.44\textwidth}
\vspace{-4mm}
\begin{center}
\includegraphics[scale=0.6]{Images/rg-17.png}
\end{center}
\vspace{-8mm}
\end{wrapfigure}

\textbf{ब}रेखा\textbf{ह}रेखावर्गयोगतुल्यो\renewcommand{\thefootnote}{१}\footnote{\textbf{बह}वर्गयोगतुल्यो {\en J.}}ऽस्तीति कल्पि-तम्~। \textbf{ज}रेखावर्गो \textbf{द}रेखा\textbf{झ}रेखावर्गयोगतुल्यः कल्पितः \textbf{अ}वर्गतुल्यस्य \textbf{बह}योर्वर्गयोगस्य \textbf{ब}व-र्गेण निष्पत्ति\textbf{र्ज}वर्गतुल्य\textbf{झद}वर्गयोगस्य \textbf{द}वर्गेण या निष्पत्तिस्तत्तुल्यास्ति~। पुन\textbf{र्ह}वर्ग\textbf{ब}वर्गयोर्नि-ष्पत्ति\textbf{र्झ}वर्ग\textbf{द}वर्गनिष्पत्तेः\renewcommand{\thefootnote}{२}\footnote{वर्गयोर्निष्पतेः {\en J.}} समानास्ति~। तस्मात् \textbf{हब}निष्पत्तिः \textbf{झद}निष्पत्तिसमाना भविष्यति~। \textbf{बह}निष्पत्ति\textbf{र्दझ}निष्पत्तेः समाना भविष्यति~। तस्मात् \textbf{अह}निष्पत्ति\textbf{र्जझ}निष्पतिसमाना भविष्यति~। तस्मात् यदि \textbf{अहौ} मिलितौ स्तस्तदा \textbf{जझा}वपि \;मिलितौ \;भविष्यतः~। \,यदि \,\textbf{अहौ} भिन्नौ स्तस्तदा \textbf{जझा}वपि भिन्नौ भविष्यतः\renewcommand{\thefootnote}{३}\footnote{भिन्नौ तदा भिन्नौ भविष्यतः {\en J.}}\,। 

\begin{center}
पुनः प्रकारान्तरम्~। 
\end{center}

\begin{wrapfigure}{r}{0.28\textwidth}
\vspace{-10mm}
\begin{center}
\includegraphics[scale=0.6]{Images/rg-18.png}
\end{center}
\vspace{-8mm}
\end{wrapfigure}

\textbf{अबबजदहहझा}श्चतस्रो रेखाः कल्पिताः~। तत्र \textbf{अब}वर्ग-\textbf{बज}वर्गयोर्निष्पत्ति\textbf{र्दह}वर्ग\textbf{झह}वर्गर्निष्पत्तेस्तुल्यास्ति~। तस्मात् \textbf{अब}वर्गस्य निष्पत्तिः \textbf{अब}वर्ग\textbf{बज}वर्गान्तरेण तथास्ति यथा \textbf{दह}वर्गस्य निष्पत्तिः \textbf{दह}वर्ग\textbf{झह}वर्गान्तरेणास्ति~। \textbf{अब}स्य निष्पत्तिः \textbf{अब}वर्ग\textbf{बज}वर्गान्तरभुजेन तथास्ति यथा \textbf{दह}स्य निष्पत्ति\textbf{र्दह}वर्ग\textbf{हझ}वर्गयोरन्तरभुजेनास्ति~। \textbf{अब}म् \textbf{अबबज}-वर्गान्तरभुजेन मिलितं भवति~। तदा \textbf{दहं दह}वर्ग\textbf{हझ}-वर्गान्तरभुजेन मिलितं भविष्यति~। यदि ते भिन्ना भविष्यन्ति तदा एतेऽपि भिन्ना भविष्यन्ति~॥ 

\newpage
 \begin{center}
\textbf{\large अथ त्रयोदशं क्षेत्रम्~॥~१३~॥}
\end{center}
\vspace{2mm}

{\ab न्यूनाधिके\renewcommand{\thefootnote}{१}\footnote{{\en J. has} यत्र {\en in the beginning.}} द्वे रेखे भवतस्तदा लघुरेखावर्गचतुर्थांशतुल्यमेकं क्षेत्रं बृहद्रेखाखण्डोपरि कार्यं यथा\renewcommand{\thefootnote}{२}\footnote{कार्यम्~। परं तथा कार्यं यथा {\en D.,K.,V.}} द्वितीयखण्डोपरि कृतं क्षेत्रं वर्गो भवति~। तत्रेदं क्षेत्रं बृहद्रेखाया द्वे खण्डे यदि मिलिते करिष्यति\renewcommand{\thefootnote}{३}\footnote{करोति {\en J.}} तदा बृहद्रेखावर्गो लघुरेखावर्गस्य 
बृहद्रेखामिलितान्यरेखावर्गस्य च योगेन तुल्यो भविष्यति~। यदि च बृहद्रेखावर्गः पूर्वोक्तरूपो भवति तदा क्षेत्रं बृहद्रेखाया मिलिते द्वे खण्डे करिष्यति~।}\\ 

\begin{wrapfigure}{r}{0.44\textwidth}
\vspace{-6mm}
\begin{center}
\includegraphics[scale=0.6]{Images/rg-19.png}
\end{center}
\vspace{-8mm}
\end{wrapfigure}

 यथा अधिकरेखा \textbf{बजं} कल्पिता लघुरेखा \textbf{अं} कल्पिता~। \textbf{अ}वर्गचतुर्थांशः\renewcommand{\thefootnote}{४}\footnote{अवर्गचतुर्थांशतुल्यं \textbf{बज}रेखाखण्डोपरि {\en \& c. J.}} \textbf{अ}लघुरेखाया अर्धवर्गतुल्योऽस्ति~। एतत्तुल्यं \textbf{बज}रेखाखण्डो-परि क्षेत्रं कार्यं यथा द्वितीयखण्डोपरि शेषक्षेत्रं वर्गरूपं भवति~। तदेयं \textbf{बज}रेखा \textbf{द}चिन्होपरि खण्डिता भविष्यति न त्वर्धिता\renewcommand{\thefootnote}{५}\footnote{{\en J. omits} न त्वर्धिता}\,। यतो \textbf{अ}रेखार्धवर्ग \textbf{बज}रेखार्धवर्गतो न्यूनोऽस्ति तस्मात् \textbf{बदं} महत्खण्डं कल्पितम्~। \textbf{दह}रेखा\textbf{जद}तुल्या पृथक्कार्या~। पुन\textbf{र्बददज}योर्घातः \textbf{अ}वर्गचतु-र्थांशतुल्योऽस्ति~। अयं चतुर्गुणः \textbf{अ}वर्गतुल्यो भवति~। अस्मिन् \textbf{बह}वर्गश्चेद्योज्यते तदा \textbf{बज}वर्गसमानो भवति~। तस्मात्  \textbf{बज}वर्गः \textbf{अ}वर्ग\textbf{बह}वर्गयोर्योगतुल्यो भवति~। तस्माद्यदि \textbf{बददजौ} मिलितौ भवतस्तदा \textbf{बहबजौ} मिलितौ भविष्यतः~। कुतः~। \textbf{बजं जदे}न मिलितमस्ति~। \textbf{जदं जहे}न मिलितमस्ति~। तस्मात् \textbf{बजं जहे}न मिलितं भविष्यति~। पुनरपि यदि \textbf{बजं बहे}न मिलितं स्यात् तदा \textbf{बदं दजे}न मिलितं भविष्यति~। कुतः~। \textbf{बजं हजे}न मिलितमस्ति~। \textbf{हजं दजे}न 

\newpage 
\noindent मिलितं चास्ति~। तस्मात् \textbf{बजं दजे}न मिलितं भविष्यति~। तस्मात् \textbf{बदं दजे}न मिलितं भविष्यति~। इदमेवेष्टमस्माकम्\renewcommand{\thefootnote}{१}\footnote{{\en J. omits} अस्माकम्}\;॥ 
\vspace{2mm}

\begin{center}
\textbf{अथ चतुर्दशं क्षेत्रम्~॥~१४~॥}
\end{center}

{\ab द्वे रेखे\renewcommand{\thefootnote}{२}\footnote{{\en J. has} त्रयोदशक्षेत्रोक्तद्वे रेखे} न्यूनाधिके यदि भवतस्तत्र न्यूनरेखावर्गचतुर्थांशतुल्यं क्षेत्रं बृहद्रेखाखण्डोपरि तथा कार्यं यथा शेषखण्डक्षेत्रं वर्गरूपमवशिष्यते~। तत्क्षेत्रं यद्यधिकरेखायाः खण्डद्वयं भिन्नं करोति तदा महद्रेखावर्गो लघुरेखावर्गस्य महद्रेखाभिन्नान्यरेखावर्गस्य च योगेन तुल्यो भविष्यति~। यदि महद्रेखावर्ग ईदृशो भवति तदा क्षेत्रं तस्या रेखायाः खण्डद्वयं भिन्नं करिष्यति~। }\\

\begin{wrapfigure}{r}{0.46\textwidth}
\vspace{-8mm}
\begin{center}
\includegraphics[scale=0.6]{Images/rg-20.png}
\end{center}
\vspace{-8mm}
\end{wrapfigure}

 उपरितनक्षेत्रेणैव निश्चितं \textbf{बज}रेखावर्गः \textbf{अ}वर्ग\textbf{बह}वर्गयोगतुल्योऽस्ति~। यदि \textbf{बदं दजा}-द्भिन्नं भवति तदा \textbf{बजं बहा}द्भिन्नं भवि-ष्यति~। कुतः~। यदि मिलितं स्यात्\renewcommand{\thefootnote}{३}\footnote{चेत् {\en J.}} तदा \textbf{बददजौ} मिलितौ भविष्यतः~। इदमशुद्धम्~। \\

 पुनरपि यदि \textbf{बजबहौ} भिन्नौ भवतस्तदा \textbf{बददजा}वपि भिन्नौ भविष्यतः~। कुतः~। यदि मिलितौ भवत\renewcommand{\thefootnote}{४}\footnote{भविष्यतः {\en J.}}स्तदा \textbf{बजबहौ} मिलितौ भविष्यतः~। इदमशुद्धम्~। अस्मदिष्टं समीचीनम्~॥ 
 \vspace{2mm}
 
\begin{center}
\textbf{\large अथ पञ्चदशं क्षेत्रम्~॥~१५~॥}
   \end{center}
   
 {\ab यानि समकोणक्षेत्राणि भवन्ति तेषां भुजा यद्यङ्कसञ्ज्ञार्हा भवन्ति\renewcommand{\thefootnote}{५}\footnote{भविष्यन्ति {\en J.}} तदा तत्क्षेत्रमप्यङ्कसञ्ज्ञार्हं भवति~। }

\newpage

\begin{wrapfigure}{r}{0.5\textwidth}
\vspace{-4mm}
\begin{center}
\includegraphics[scale=0.6]{Images/rg-21.png}
\end{center}
\vspace{-8mm}
\end{wrapfigure}

 यथा \textbf{बज}क्षेत्रं कल्पितम्~। \textbf{अबअजौ} तस्य भुजौ कल्पितौ~। \textbf{अब}भुजोपरि \textbf{बदं} समकोणसमचतुर्भुजं क्षेत्रं कार्यम्~। इद-मङ्कसञ्ज्ञार्हं \,भविष्यति~। क्षेत्रं \,चानेन \,मिलितमस्ति~। कुतः~। \textbf{अज}म् \textbf{अद}तुल्येन \textbf{अबे}न मिलितमस्ति~। तस्मात् क्षेत्रमप्यङ्कसञ्ज्ञार्हं भविष्यति~। इदमेवास्माकमिष्टम्\renewcommand{\thefootnote}{१}\footnote{{\en J. omits} अस्माकम्.}\;॥ 
\vspace{2mm}
 
\begin{center}
\textbf{\large अथ षोडशं क्षेत्रम्~॥~१६~॥}
\end{center}

{\ab  यद्यङ्कसञ्ज्ञार्हभुजोपर्यङ्कसञ्ज्ञार्हं क्षेत्रं भवति तदा द्वितीय\renewcommand{\thefootnote}{२}\footnote{तदुत्पन्नद्वितीय {\en J.}}भुजोऽप्यङ्कसञ्ज्ञार्हो भविष्यति~। }\\

\begin{wrapfigure}{r}{0.4\textwidth}
\vspace{-4mm}
\begin{center}
\includegraphics[scale=0.6]{Images/rg-22.png}
\end{center}
\vspace{-8mm}
\end{wrapfigure}

 यथा \textbf{अब}भुजोपरि \textbf{बज}क्षेत्रं कल्पितम्~। \textbf{अज}-भुज उत्पन्नः~। तत्र \textbf{अबो}परि \textbf{बद}समकोणसमचतु-र्भुजं कार्यम्~। तस्मादिदं \textbf{बज}क्षेत्रेण मिलितं भवि-ष्यति~। कुतः~। उभयोरङ्कसञ्ज्ञार्हत्वात्~। तस्मात् \textbf{दअ}म् \;\textbf{अब}तुल्यम् \,\textbf{अजे}न \;मिलितं \,भविष्यति~। तस्मात् \textbf{अज}म् अङ्कसञ्ज्ञार्हं भविष्यति~। इदमस्म-दिष्टम्~। अस्य क्षेत्रं पूर्वोक्तवदस्ति~॥
\vspace{2mm}
 
\begin{center}
\textbf{\large अथ सप्तदशं क्षेत्रम्~॥~१७~॥} 
\end{center}

{\ab  यत् क्षेत्रं चतुर्भिः कोणैः समकोणमस्ति तस्य यदि भुजौ भिन्नौ भवतो भुजवर्गौ च मिलितौ भवतस्तदा तत् क्षेत्रं करणीरूपं भविष्यति~। तस्यैव मध्यक्षेत्रसञ्ज्ञा\renewcommand{\thefootnote}{३}\footnote{{\en J. has} तस्यैव एवनै(यवनै?)र्मध्यक्षेत्रमिति सञ्ज्ञा.} कृता~। यस्या रेखाया वर्ग एतत्क्षेत्रतुल्यो भवति सापि करणीगतैव स्यात्~। इयं रेखा मध्यरेखाभिधाना भवति~। }

\newpage

\begin{wrapfigure}{r}{0.5\textwidth}
\vspace{-4mm}
\begin{center}
\includegraphics[scale=0.65]{Images/rg-23.png}
\end{center}
\vspace{-8mm}
\end{wrapfigure}

यथा क्षेत्रं \textbf{बज}म्~। \textbf{अबअजौ} भुजौ भिन्नौ कल्पितौ~। पुनः \textbf{अब}भुजोपरि \textbf{बद}-समकोणसमचतुर्भुजं क्षेत्रं कार्यम्~। तस्मा-दिदमङ्कसञ्ज्ञार्हं भविष्यति कल्पितक्षेत्राद्भिन्नं च पतिष्यति~। रेखयोर्भिन्नत्वात्\renewcommand{\thefootnote}{१}\footnote{भिन्नरेखात्वात् J.}\;। तस्मात् क्षेत्रं करणीरूपं भविष्यति~। एवं हि यस्या रेखाया वर्गः क्षेत्रतुल्यो भवति तदा सापि रेखा करणीरूपा भविष्यति~। इदम् एवेष्टम्~। पूर्ववत् क्षेत्रं कार्यम्\renewcommand{\thefootnote}{२}\footnote{क्षेत्रं पूर्ववत् कार्यम् {\en J.}}\;॥ \\

 अथ\renewcommand{\thefootnote}{३}\footnote{यदि {\en D., K.}} मध्यरेखाः कदाचित् मिथो मिलिता भवन्ति~। यथा \textbf{अब}रेखा  अङ्कसञ्ज्ञार्हा कल्पिता~। यस्य \,क्षेत्रस्यैकभुजः \textbf{अजं} \,भवति \,द्वितीयश्च \,\textbf{अब}रेखाचतुर्थांशतुल्यो भवति तत्क्षेत्रतुल्यो यस्या रेखाया वर्गो भवति सा रेखा मध्यरेखा भवति~। सैव रेखा \textbf{बज}-क्षेत्रतुल्यो यस्याः रेखाया वर्गो भविष्यति तया मिलिता भवति~। कुतः~। अनयो रेखयोर्वर्गौ रूपस्य चतुर्णां च निष्पत्तौ भविष्यतः~। रूपं चत्वारः वर्गौ स्तः~।  कदाचिन्मध्यरेखा भिन्ना भवन्ति मिलितवर्गाश्च भवन्ति~। कुतः~। यस्या रेखाया वर्गस्तत्क्षेत्रतुल्यो भवति यस्य क्षेत्रस्यैको भुजः \textbf{अजं} द्वितीयश्च \textbf{अबा}र्धतुल्यो भवति तदा सा रेखा मध्या भवति~। अस्या वर्गस्तद्रेखावर्गमिलितो भवति यस्या रेखाया वर्गो \textbf{बज}क्षेत्रतुल्योऽस्ति~। यतोऽनयोर्वर्गौ अवर्गाङ्कद्वयनिष्पत्तौ स्तः~। कदाचित्ता मध्यरेखा भिन्ना तद्वर्गाश्च भिन्ना भवन्ति~। कुतः~। यस्या रेखाया वर्गस्तेन क्षेत्रेण तुल्यो भवति यस्य क्षेत्रस्यैकभुजः \textbf{अबं} द्वितीयभुजः \textbf{अज}रेखाया भिन्नो भवति तस्य वर्गोऽङ्कसञ्ज्ञार्हो भवति सा रेखा मध्या भवति~। सा तद्रेखातो भिन्ना भविष्यति यस्या रेखाया वर्गो \textbf{बज}क्षेत्रतुल्यो भवति~। यतोऽनयोर्वर्गौ भिन्नौ भवतः~। 

\newpage
\begin{center}
\textbf{\large अथाष्टादशं क्षेत्रम्~॥~१८~॥}
\end{center}

{\ab अङ्कसञ्ज्ञार्हरेखोपरि क्षेत्रं कार्यम्~। मध्यरेखावर्गतुल्यं क्षेत्रं चेद्भवति\renewcommand{\thefootnote}{१}\footnote{तच्चेन्मध्यरेखावर्गतुल्यं क्षेत्रं भवति {\en J.}} तदा तदुत्पन्नभुजः करणीरूपो भवति~। तस्य वर्गोऽङ्कसञ्ज्ञार्हो भविष्यति~। }\\

\begin{wrapfigure}{r}{0.45\textwidth}
\vspace{-8mm}
\begin{center}
\includegraphics[scale=0.5]{Images/rg-24.png}
\end{center}
\vspace{-8mm}
\end{wrapfigure}

यथा \textbf{अं} मध्यरेखा कल्पिता \textbf{बज}म् अङ्कसञ्ज्ञार्हरेखा कल्पिता~। \textbf{जद}क्षेत्रम्\renewcommand{\thefootnote}{२}\footnote{{\en J. inserts} \textbf{अ}रेखावर्गतुल्यं {\en after} क्षेत्रं.} \textbf{अ}वर्गतुल्यं कल्पितम्~। पुन\renewcommand{\thefootnote}{३}\footnote{{\en J. omits} पुनर्.}र्यस्य भुजौ भिन्नौ भवतोऽङ्क-सञ्ज्ञार्हौ वर्गौ च भवतस्तत्क्षेत्रं \textbf{हवं} कल्पि-तम्~। \textbf{जदहव}समानक्षेत्रयोः \textbf{ब}कोण\textbf{झ}कोणौ समानौ स्तः~। तदा \textbf{जबहझ}निष्पत्ति\textbf{र्झवबद}-निष्पत्तितुल्या भविष्यति~। \textbf{जबहझौ} मिलितवर्गौ स्तः~। तस्मात् \textbf{झवबदा}वपि मिलितवर्गौ भविष्यतः~। पुन\textbf{र्जद}क्षेत्र\textbf{बद}वर्गौ मिथो भिन्नौ स्तः~। तस्मात् \textbf{जबबदा}वपि मिथो भिन्नौ भविष्यतः~। तस्मात् \textbf{बद}वर्ग एवाङ्कसञ्ज्ञार्हो जातः~। इदमेवेष्टम्~॥ 
\vspace{2mm}

\begin{center}
\textbf{\large अथैकोनविंशतितमं क्षेत्रम्~॥~१९~॥}
\end{center}

{\ab  मध्यरेखामिलिता रेखापि मध्या भवति~। }\\
 
\begin{wrapfigure}{r}{0.56\textwidth}
\vspace{-11mm}
\begin{center}
\includegraphics[scale=0.65]{Images/rg-25.png}
\end{center}
\vspace{-8mm}
\end{wrapfigure}

 यथा \,\textbf{अं} \,मध्यरेखा \,कल्पिता~। एतन्मिलिता \textbf{ब}रेखा कल्पिता~।~अङ्क-सञ्ज्ञार्ह\textbf{दज}रेखोपरि तद्रेखाद्वयवर्ग-तुल्यं \textbf{दह}क्षेत्रं \textbf{दझ}क्षेत्रं कार्यम्~।~एते क्षेत्रे मिलिते भविष्यतः~। \textbf{हजं  जझे}न मिलितं भविष्यति~। \textbf{हज}स्य वर्गोऽङ्कसञ्ज्ञार्होऽस्ति~। 

\newpage

\noindent \textbf{हजजदौ} भिन्नौ स्तः~। तस्मात् \textbf{जझ}मप्येवमेव भविष्यति~। तस्मात् \textbf{दझं} मध्यक्षेत्रं जातम्~। इदमेवेष्टम्~॥ 
\vspace{2mm}

\begin{center}
\textbf{\large अथ विंशतितमं क्षेत्रम्~॥~२०~॥}
\end{center}

{\ab द्वयोर्मध्ययोः क्षेत्रयोरन्तरं करणीरूपं भवति~। }\\

 एको मध्यः \textbf{अबः} कल्पितः~। द्वितीयो मध्यः \textbf{अः} कल्पितः~। अन्तरं \textbf{बं} कल्पितम्~। \textbf{जद}म् अङ्कसञ्ज्ञार्हं कल्पितम्~। अस्योपरि \textbf{अब}तुल्यं क्षेत्रं कार्यम्~। अस्य द्वितीयो भुजो

\begin{center}
\includegraphics[scale=0.7]{Images/rg-26.png}
\end{center}

\noindent \textbf{जहो} भविष्यति~। पुनर्द्वितीयक्षेत्रतुल्यं क्षेत्रं कार्यम्~। तत्र \textbf{जझं} द्वितीयो भुजो भविष्यति~। अनयोर्वर्गावङ्कसञ्ज्ञार्हौ भविष्यतः~। एतौ \textbf{जदा}त् सकाशात्\renewcommand{\thefootnote}{१}\footnote{{\en J. drops} सकाशात्} भिन्नौ भविष्यतः~। \renewcommand{\thefootnote}{२}\footnote{एवं क्षेत्रान्तरं करणीरूपं भविष्यति~। यदि करणीरूपं न भवति {\en J.}}\textbf{हवं} क्षेत्रान्तरं भविष्यति~। इदं च करणीरूपं भविष्यति~। यदि करणीरूपं न भवति तदाङ्कसञ्ज्ञार्हं कल्पितम्~। तदुत्पन्नभुजो \textbf{झहः} अङ्कसञ्ज्ञार्हो भविष्यति~। अस्य वर्गो \textbf{जझ}वर्गश्चाङ्कसञ्ज्ञार्होऽस्ति~। पुन\textbf{र्जझझह}योर्भिन्नत्वात् \textbf{जझझह}योर्घातोऽनयो रेखयोर्वर्गाद्भिन्नो भविष्यति~। तस्मात्\renewcommand{\thefootnote}{३}\footnote{\textbf{जझझह}योर्घातो भिन्नोऽस्ति~। \textbf{जझझह}योर्भिन्नत्वात्~। तस्मात् {\en \& c. D.}} \textbf{जझझह}वर्गौ \textbf{जझझह}योर्द्विगुणघाताद्भिन्नौ भवतः~। तस्मात् सम्पूर्णं मिलितं \textbf{जह}वर्गतुल्यं \textbf{जझझह}अङ्कसञ्ज्ञार्हवर्गयोर्भिन्नं भविष्यति~। तस्मात् तत्करणीरूपं भविष्यति~। कल्पितं चाङ्कसञ्ज्ञार्हम्~। इदमशुद्धम्\renewcommand{\thefootnote}{४}\footnote{इदमनुपपन्नम् {\en J.}}\;। अस्मदिष्टं समीचीनम्~॥ 

\newpage
\begin{center}
\textbf{\large अथैकविंशतितमं}\renewcommand{\thefootnote}{१}\footnote{अथैकविंशं {\en J.}} \textbf{\large क्षेत्रम्~॥~२१~॥}
\end{center}

{\ab  तत्र तादृशमध्यरेखाद्वयोत्पादनं चिकीर्षितम् अस्ति ययोर्मध्यरेखयोः केवलं वर्गावेव मिलितौ भवत एतौ चाङ्कसञ्ज्ञार्हक्षेत्रभुजौ भवतः~।}\\

\begin{wrapfigure}{r}{0.41\textwidth}
\vspace{-10mm}
\begin{center}
\includegraphics[scale=0.65]{Images/rg-27.png}
\end{center}
\vspace{-8mm}
\end{wrapfigure}

 अथ प्रथमं द्वे रेखे \textbf{अब}सञ्ज्ञे कल्पिते~। अनयोर्वर्गावेव केवलमङ्कसञ्ज्ञार्हौ भवतः~। अनयोर्मध्ये \textbf{ज}रेखा मध्यनिष्पत्तिरूपा कल्पिता~। \textbf{द}रेखा चतुर्थ्यस्यां निष्पत्तौ कल्पिता~। \textbf{अब}घातो \textbf{ज}वर्गतुल्यो मध्यक्षेत्रं भविष्यति~। तस्मात् \textbf{जं} मध्यरेखा भवि-ष्यति~। \textbf{अब}निष्पत्तिः \textbf{जद}निष्पत्तितुल्यास्ति~। \textbf{अब}योः केवलं वर्गौ मिलितौ स्तः~। तस्मात् \textbf{जद}योरपि केवलं वर्गौ मिलिष्यतः\renewcommand{\thefootnote}{२}\footnote{मिलितौ स्तः {\en J.}}\;। \textbf{दो}ऽपि मध्यरेखा भविष्यति~। \textbf{जद}योर्घातो \textbf{ब}वर्गतुल्योऽङ्कसञ्ज्ञार्होऽस्ति~। तस्मात् \textbf{जदा}विष्टे मध्ये रेखे जाते~॥ 
\vspace{2mm}

\begin{center}
\textbf{\large अथ द्वाविंशतितमं क्षेत्रम्~॥~२२~॥} 
\end{center}

 {\ab ये द्वे मध्ये रेखे केवलवर्गमिलिते मध्यक्षेत्रस्य द्वौ भुजौ भवतस्तादृशरेखाद्वयस्योत्पादनमिष्टमस्ति~। }\\

\begin{wrapfigure}{r}{0.54\textwidth}
\vspace{-10mm}
\begin{center}
\includegraphics[scale=0.65]{Images/rg-28.png}
\end{center}
\vspace{-8mm}
\end{wrapfigure}

 \textbf{अब}जास्तिस्रो रेखाः केवलवर्गमि-लिताः कल्पिताः~। \textbf{अब}योर्मध्ये \textbf{द}रेखा मध्यनिष्पत्तौ कल्पिता~।  \textbf{अज}योर्निष्प-त्तितुल्या \textbf{दह}निष्पत्तिः कल्पिता~। \textbf{अद}-निष्पत्तितुल्या \textbf{बद}निष्पत्ति\textbf{र्जह}निष्पत्ति-तुल्या भविष्यति~। \textbf{अब}योर्घातो \textbf{द}वर्ग-तुल्योऽस्ति~। तस्मात् \textbf{द}रेखा मध्या भविष्यति\renewcommand{\thefootnote}{३}\footnote{दं मध्यरेखा भविष्यति {\en J.}}\;। 
 
\newpage 
\noindent \textbf{अजौ} \,केवलवर्गमिलितौ\renewcommand{\thefootnote}{१}\footnote{मिलितवर्गौ {\en J.}} \,स्तः~। तस्मात् \,\textbf{दहा}वपि \,केवलर्वर्गमिलितौ$^{\scriptsize{\hbox{{\color{blue}१}}}}$ \,भविष्यतः~। तस्मात् \textbf{ह}रेखा मध्यरेखा\textbf{द}रेखायाः केवलवर्गमिलिता भविष्यति~। \textbf{दह}योर्घातो \textbf{बज}योः घातेन तुल्योऽस्ति~। तस्मात् \textbf{दहा}विष्टमध्यरेखे भविष्यतः~॥ 
\vspace{2mm}

\begin{center}
 \textbf{\large अथ त्रयोविंशतितमं क्षेत्रम्~॥~२३~॥} 
 \end{center}
 
 {\ab यस्य क्षेत्रस्य द्वौ भुजौ मध्यरेखे भवतस्तयोः केवलवर्गौ मिलितौ स्तस्तदा तत् क्षेत्रं केवलमङ्कसञ्ज्ञार्हं भविष्यति वा मध्यसञ्ज्ञकं भविष्यति~। }\\

\begin{wrapfigure}{r}{0.54\textwidth}
\vspace{-8mm}
\begin{center}
\includegraphics[scale=0.65]{Images/rg-29.png}
\end{center}
\vspace{-8mm}
\end{wrapfigure}

  \textbf{बज}क्षेत्रस्य \textbf{अबअजौ} द्वौ भुजौ च मध्यौ \;कल्पितौ~। \;द्वयोर्भुजयोरुपरि \textbf{बदजहौ} समकोणसमचतुर्भुजौ कार्यौ\,। \textbf{झव}रेखाङ्कसञ्ज्ञार्हा कल्पिता~। तस्या उपरि \textbf{बदबजजह}क्षेत्राणां तुल्यं  \textbf{वतकलमन}क्षेत्रत्रयं कार्यम्~। तत्र \textbf{झत-तललना} उत्पन्ना भुजा भविष्यन्ति~। प्रत्येकं \textbf{झतलन}योर्वर्गौ केवलम् अङ्कसञ्ज्ञार्हौ स्तः~। एतौ च मिलितरेखारूपौ स्तः~। \textbf{अबअज}वर्गयोर्मिलितत्वात्~। \textbf{बद}क्षेत्र\textbf{बज}क्षेत्रयोर्निष्पत्ति\textbf{र्दअअज}निष्पत्तितुल्यास्ति~। \textbf{बअ-अह}योरपि  \,निष्पत्तितुल्यास्ति~। \,तदा \,\textbf{बज}क्षेत्र\textbf{जह}क्षेत्रयोरपि \,निष्पत्तितुल्या \,भविष्यति\renewcommand{\thefootnote}{2}\footnote{जाता {\en J.}} तस्मात् \textbf{वतकलमना}नि त्रीणि क्षेत्राणि \textbf{झततललना}स्तिस्त्रो रेखाश्चैकनिष्पत्तौ भविष्यन्ति~। \textbf{झतलन}योर्घात\textbf{स्तल}वर्गतुल्यो \,भविष्यति~। \,\textbf{झतलन}योर्घातो \,\textbf{झत}वर्गेण \,मिलितोऽस्ति~। तस्मात् \textbf{तल}वर्गोऽङ्कसञ्ज्ञार्हो भविष्यति~। यदि \textbf{तलं झव}मिलितं भवति तदा \textbf{कल}क्षेत्र-तुल्यं \textbf{वज}क्षेत्रमङ्कसञ्ज्ञार्हं भविष्यति~। यदि \textbf{तल}रेखा \textbf{झव}रेखातो भिन्ना भवति तदा तत् मध्यक्षेत्रं भविष्यति~। इदमेवेष्टम्~॥ 
 
 \newpage
\begin{center} 
\textbf{\large अथ चतुर्विंशतितमं क्षेत्रम्~॥~२४~॥}
\end{center}

{\ab तत्र तादृशरेखाद्वयस्योत्पादनम् इष्टम् अस्ति ययोः केवलवर्गावङ्कसञ्ज्ञार्हौ मिलितौ भवतोऽधिकरेखावर्गो लघुरेखावर्गस्य महद्रेखामिलितान्यरेखावर्गस्य च योगेन तुल्यो भवेत् तथोत्पादनमिष्टमस्ति~। }\\

तदा द्वावङ्कवर्गराशी कल्प्यौ ययोः अन्तरं वर्गो न भवति~। तौ \textbf{अबबजौ} वर्गौ कल्पितौ~। पुन\textbf{र्दह}रेखाङ्कसञ्ज्ञार्हा कल्पिता~। अस्योपरि \textbf{दझहं} वृत्तार्धं कार्यम्~। तत्र \textbf{दह}वर्ग\textbf{दझ}वर्गयोर्निष्पत्तिः \textbf{अबअज}निष्पत्तितुल्या कल्पिता~। तस्मात् \textbf{दहदझौ} इष्टरेखे भविष्यतः~।

\begin{center}
अस्योपपत्तिः~।
\end{center}

\begin{wrapfigure}{r}{0.4\textwidth}
\vspace{-8mm}
\begin{center}
\includegraphics[scale=0.9]{Images/rg-30.png}
\end{center}
\vspace{-8mm}
\end{wrapfigure}

\textbf{दझं} पूर्णज्या कल्पिता~। \textbf{हझ}रेखा संयोज्या~। तत्र \textbf{दह}वर्ग\textbf{दझ}वर्गयो\renewcommand{\thefootnote}{१}\footnote{\textbf{दहदझ}वर्गयो {\en J.}}र्निष्पत्तिर्द्वयोरङ्कयोर्निष्पत्ति-तुल्यास्ति~। वर्गराश्योर्निष्पत्तौ न स्तः~। तस्मादेतद्रेखाद्वयं केवलमिलितवर्गो भविष्यति~। पुनः \textbf{दह}रेखावर्गोऽङ्कसञ्ज्ञार्होऽस्ति~। तस्मात् \textbf{दझ}मप्येवं भविष्यति~। \;पुन\textbf{र्दह}वर्गो \;\textbf{दझ}वर्ग\textbf{हझ}वर्गयोगतु-ल्योऽस्ति~। तदा \textbf{दह}वर्गस्य \textbf{हझ}वर्गेण निष्पत्तिः तथा भविष्यति यथा \textbf{अबवजा}ङ्कवर्गराश्योर्निष्पत्ति-तुल्या भविष्यति~। तस्मात् \textbf{हझं दहे}न मिलितं भविष्यति~। कुतः~। यतोऽनयोर्वर्गौ द्वयोरङ्कयोर्वर्गयोर्निष्पत्तौ स्तः~। तस्माद्द्वे रेख इष्टे जाते~॥ 
\vspace{2mm}

\begin{center}
\textbf{\large  अथ पञ्चविंशतितमं क्षेत्रम्~॥~२५~॥}
\end{center}

{\ab तादृशरेखाद्वयस्योत्पादनमिष्टमस्ति ययोर्वर्गावङ्कसञ्ज्ञार्हौ भवतः पुनः केवलवर्गौ मिलितौ यथा भवतः~। पुनर्बृहद्रेखावर्गो लघुरेखावर्गस्य महद्रेखाभिन्नान्यरेखावर्गस्य च योगेन तुल्यो भवति~। }

\newpage

\begin{wrapfigure}{r}{0.4\textwidth}
\vspace{-8mm}
\begin{center}
\includegraphics[scale=0.7]{Images/rg-31.png}
\end{center}
\vspace{-8mm}
\end{wrapfigure}

 ययोर्वर्गराश्योर्योगो वर्गो न भवति तौ \textbf{अज-बजौ} राशी कल्पितौ~। पुन\textbf{र्दह}रेखा अङ्कसञ्ज्ञार्हा कल्पिता~। शेषमुपरितनक्षेत्रोक्तवत् कार्यं\renewcommand{\thefootnote}{१}\footnote{{\en D. inserts} प्रकारेण {\en before} कार्यं.} यथा \textbf{दझ}रेखोत्पन्ना भवति~। तस्मात् \textbf{दहदझ}रेखे इष्टे भविष्यतः~। कुतः~। अनयोर्वर्गौ \textbf{अबअजा}ङ्कयोः निष्पत्तौ स्तः~। सा निष्पत्तिः वर्गनिष्पत्तिसदृशी नास्ति~। तस्मात्तौ केवलवर्गमिलितौ भविष्यतः~। \textbf{दह}म् अङ्कसञ्ज्ञार्हमस्ति~। तस्मात् \textbf{दझ}वर्गोऽङ्कसञ्ज्ञार्हो भविष्यति~। \textbf{अबबज}योर्निष्पत्तिः वर्गद्वयनिष्पत्तिर्नास्ति~। \textbf{दहहझ}वर्गौ तस्यां निष्पत्तौ स्तः~।  तस्मात् \textbf{दह}वर्गो \textbf{दझ}वर्गस्य तद्रेखाभिन्नान्यरेखावर्गस्य च योगेन तुल्योऽस्ति~। यथेष्टं कल्पितं तथा सिद्धम्~। अस्य क्षेत्रमुपरितनवद्बोध्यम्\renewcommand{\thefootnote}{२}\footnote{क्षेत्रं पूर्वोक्तमेव बोध्यम्~। {\en J.}}\;॥ 
\vspace{2mm}

\begin{center}
\textbf{\large अथ षड्विंशतितमं क्षेत्रम्~॥~२६~॥}
\end{center}

 {\ab अत्र\renewcommand{\thefootnote}{३}\footnote{तत्र {\en J.}} तथा मध्यरेखाद्वयोत्पादनमिष्टमस्ति ययोर्वर्गौ केवलमिलितौ भवतः~। रेखे चाङ्कसञ्ज्ञार्हैकक्षेत्रस्य भुजौ भवतः~। पुनरधिकरेखावर्गो लघुरेखावर्गस्य मिलितान्यरेखावर्गस्य च योगेन तुल्यो भवति~। }\\

\begin{wrapfigure}{r}{0.4\textwidth}
\vspace{-8mm}
\begin{center}
\includegraphics[scale=0.8]{Images/rg-32.png}
\end{center}
\vspace{-8mm}
\end{wrapfigure}

 \textbf{अब}रेखे तथा कल्पिते यथा \textbf{अ}वर्गो \textbf{ब}रेखाव-र्गस्य \textbf{अ}रेखामिलितान्यरेखावर्गस्य च योगेन तुल्यो भवति~। अनयोः \,मध्ये \,एका \,रेखा \,मध्यनिष्पत्तौ निष्कास्या~। सा \textbf{ज}रेखा कल्प्या~। एताभ्योऽन्या चतुर्थी \,अस्यां \,निष्पत्तौ \,निष्कास्या~। सा \,\textbf{द}रेखा कल्पिता~। तत्र \textbf{जद}रेखे मध्यरेखे जाते~। अनयोः वर्गौ केवलमिलितौ भविष्यतोऽङ्कसञ्ज्ञार्हक्षेत्रस्य च भुजौ भविष्यतः~। अङ्कसञ्ज्ञार्हक्षेत्रस्य च भुजौ भविष्यतः~। अनयो\textbf{र्ज}वर्गो \textbf{द}वर्ग\textbf{ज}मिलित-रेखा- 

\newpage
\noindent वर्गोक्तवर्गयोगतुल्यो भविष्यति~। यत एतौ \textbf{अब}योर्निष्पत्तौ स्तः~। इदमेवेष्टम्~॥ 
\vspace{2mm}

\begin{center}
\textbf{\large अथ सप्तविंशतितमं क्षेत्रम्~॥~२७~॥}
\end{center}

{\ab तत्र तथा मध्यरेखाद्वयमिष्टमस्ति ययोर्वर्गौ केवलमिलितौ स्तोऽङ्कसञ्ज्ञार्हक्षेत्रस्य भुजौ स्तः~। अधिकरेखावर्गो लघुरेखावर्गस्य बृहद्रेखाभिन्नरेखावर्गस्य च योगेन तुल्यो भवति~। }\\

 पुनः \textbf{अब}रेखे तथा कल्प्ये यथा \textbf{अ}वर्गो \textbf{ब}वर्गस्य \textbf{अ}रेखाभिन्नान्यरेखावर्गस्य च योगेन तुल्यो भवति~। शेषं पूर्वोक्तवत् ज्ञेयं~॥ 
\vspace{2mm}

\begin{center}
\textbf{\large अथाष्टाविंशतितमं क्षेत्रम्~॥~२८~॥}
\end{center}

{\ab तत्र तथा मध्यरेखाद्वयोत्पादनमिष्टमस्ति यथा द्वे मध्यरेखे केवलवर्गमिलिते मध्यक्षेत्रस्य च भुजौ भवतोऽधिकरेखावर्गो लघुरेखावर्गस्य च महद्रेखामिलितान्यरेखावर्गस्य च योगेन तुल्यो भवति~। }\\

\begin{wrapfigure}{r}{0.4\textwidth}
\vspace{-8mm}
\begin{center}
\includegraphics[scale=0.8]{Images/rg-33.png}
\end{center}
\vspace{-8mm}
\end{wrapfigure}

 \textbf{अबजा}स्तिस्रो रेखास्तथा कल्प्या यथा\renewcommand{\thefootnote}{१}\footnote{तत्र तथा \textbf{अबजा}स्तिस्रो रेखा कल्प्या यथा {\en J.}} \textbf{अ}वर्गो \textbf{ज}वर्गस्य \,\textbf{अ}रेखामिलितान्यरेखावर्गस्य \,च \,योगेन तुल्यो\renewcommand{\thefootnote}{२}\footnote{\textbf{ज}वर्ग\textbf{अ}रेखामिलितरेखावर्गयोगतुल्यो {\en J.}} \,भवति~। \textbf{अब}मध्ये \,\textbf{द}रेखा \,मध्यनिष्पत्तौ कल्पनीया~। पुन\textbf{र्ह}रेखान्या तथा तुल्या यथा \textbf{दह}-निष्पत्तिः \textbf{अज}निष्पत्तितुल्या भवति\renewcommand{\thefootnote}{३}\footnote{{\en D. inserts the words} तस्या निष्पत्तिः \textbf{अ}रेखया तथा भविष्यति यथा \textbf{अज}रेखयास्ति~। {\en after} भवति~।}\;। तस्मात् \textbf{दहौ} इष्टमध्यरेखे भविष्यतः~॥ 
\vspace{2mm}

\begin{center}
\textbf{\large अथोनत्रिंशत्तमं क्षेत्रम्~॥~२९~॥}
\end{center}

 {\ab द्वे मध्यरेखे केवलवर्गमिलिते मध्यक्षेत्रभुजौ यथा भवतस्तथा कल्पनीये~। पुनरधिकरेखावर्गो लघुरेखावर्गस्य बृहद्रेखाभिन्नान्यरेखावर्गस्य च योगेन तुल्यो यथा भवति\renewcommand{\thefootnote}{४}\footnote{तुल्योऽस्ति {\en J.}}\;। }
 
\newpage
अस्य प्रकारस्त्वनन्तरोक्तक्षेत्रवत् ज्ञेयः~। विशेषस्तु \textbf{अ}वर्गो \textbf{ज}वर्गस्य \textbf{अ}रेखाभिन्नान्य-रेखावर्गस्य च योगेन तुल्योऽस्ति~॥ 
\vspace{2mm}

\begin{center}
\textbf{\large अथ त्रिंशत्तमं क्षेत्रम्~॥~३०~॥}
\end{center}

{\ab तादृशरेखाद्वयोत्पादनमिष्टमस्ति\renewcommand{\thefootnote}{१}\footnote{{\en J. has} तत्र {\en in the beginning.}} ययो रेखयोर्वर्गौ मिथो भिन्नौ स्तो वर्गयोगश्चाङ्कसञ्ज्ञार्हो भवति रेखयोर्घातो द्विगुणो मध्यक्षेत्रं भवति~। }\\

 पुनः \textbf{अबबजौ} द्वे रेखे कल्पिते~। तत्र \textbf{अब}वर्गो \textbf{बज}वर्गस्य \textbf{अब}रेखाभिन्नान्यरेखावर्गस्य च योगेन तुल्यो भवति~। \textbf{अब}रेखोपरि \textbf{अझब}वृत्तार्द्धं कार्यम्~। \textbf{बज}वर्गस्य चतुर्थांशतुल्यं क्षेत्रम् \textbf{अब}रेखाखण्डोपरि तथा कार्यं\renewcommand{\thefootnote}{२}\footnote{{\en A and J. have} यथा {\en after} कार्यं.} शेषखण्डस्य क्षेत्रं यथा\renewcommand{\thefootnote}{३}\footnote{{\en Omitted in A and J. in which it is used before}} वर्गरूपं भवेत्\renewcommand{\thefootnote}{४}\footnote{भवति {\en A., J. }}\;। अस्या \textbf{अब}रेखाया\renewcommand{\thefootnote}{५}\footnote{रेखया {\en J.}} \textbf{ह}चिह्नोपरि विभागद्वयं भविष्यति\renewcommand{\thefootnote}{६}\footnote{करिष्यति {\en D.}}\;।\\

\begin{wrapfigure}{r}{0.45\textwidth}
\vspace{-10mm}
\begin{center}
\includegraphics[scale=0.75]{Images/rg-34.png}
\end{center}
\vspace{-8mm}
\end{wrapfigure}

पुन\textbf{र्ह}चिह्नात् \textbf{हझ}लम्बो निष्कास्यः~। पुनः \textbf{अझझब}रेखे संयोज्ये~। एते इष्टरेखे भविष्यतः~। कुतः~। \textbf{अझझब}योर्निष्पत्तिः \textbf{अहहझ}-योर्निष्पत्तितुल्यास्ति~। \textbf{हझहब}योरपि निष्पत्तितुल्यास्ति~। ~तस्मात् ~\textbf{अझझब}वर्गनिष्पत्तिः \textbf{अहहब}भिन्नरेखयोर्निष्पत्तेस्तुल्यास्तीति~। \,तस्मात् \,\textbf{अझझब}योर्वर्गौ \,भिन्नौ \,भविष्यतः~। अनयोर्वर्गौ \;\textbf{अब}अङ्कसञ्ज्ञार्हवर्गेण \;समानौ \;स्तः~। \;तस्मादनयोर्वर्गयोगोऽप्यङ्कसञ्ज्ञार्हो भविष्यति~। \textbf{अहहब}योर्घातो \textbf{हझ}वर्गतुल्योऽस्ति~। \textbf{बद}वर्गस्य तुल्य आसीत्~। \textbf{बद}वर्गश्च \textbf{बज}वर्गचतुर्थांशोऽस्ति~। तस्मात् \textbf{हझ}वर्गो \textbf{बद}वर्गसमानो  भविष्यति~। पुनः \textbf{अबअझ}-योर्निष्पत्ति\textbf{र्झबझह}योर्निष्पत्तितुल्यास्ति~। 

\newpage
\noindent तस्मात् \textbf{अझझब}घातः \textbf{अबबद}घाततुल्यो भविष्यति~। तस्मात् \textbf{अझझब}द्विगुणघातः \textbf{अब-बज}मध्यक्षेत्रेण समानो भविष्यति~। इदमेवास्माकमिष्टम्~॥ 
\vspace{2mm}

\begin{center}
\textbf{\large अथैकत्रिंशत्तमं क्षेत्रम्~॥~३१~॥}
\end{center}

{\ab तत्र तादृशरेखाद्वयस्योत्पादनमिष्टं ययो रेखयोर्वर्गौ भिन्नौ 
भवतो\renewcommand{\thefootnote}{१}\footnote{भविष्यतः {\en J.}} वर्गयोगश्च मध्यक्षेत्रं भवति~। तयोर्घातो द्विगुणोऽङ्कसञ्ज्ञार्हो भवति~।}\\

\begin{wrapfigure}{r}{0.47\textwidth}
\vspace{-10mm}
\begin{center}
\includegraphics[scale=0.65]{Images/rg-35.png}
\end{center}
\vspace{-8mm}
\end{wrapfigure}

तत्र तथा मध्यरेखे \textbf{अबबजे} कल्पिते~। अनयोर्वर्गौ केवलमिलितौ~। एतावङ्कसञ्ज्ञार्ह-क्षेत्रस्य भुजौ भवतः~। एकस्या वर्गो द्वितीय-रेखावर्गस्य तदन्यभिन्नरेखावर्गस्य  च योगेन समानो भवति तथा कल्पनीयः~। पुनरनयो रेखयोरुपरि पूर्वोक्तप्रकारेण तथा क्षेत्रं\renewcommand{\thefootnote}{२}\footnote{पूर्वक्रमप्रकारेण क्षेत्रं {\en A.}} कार्यं यथा \textbf{अझझबे} इष्टरेखे उत्पन्ने भवतः~। अनयोर्वर्गौ \textbf{अहहब}भिन्नरेखावर्गनिष्पत्तौ स्तस्तस्माद्भिन्नौ जातौ~। अनयोर्वर्गयोगो मध्यक्षेत्रं कुतो जातम्~। यतोऽनयोर्वर्गौ \textbf{अब}मध्यवर्गयोस्तुल्यौ स्तः~। अनयोर्द्विगुणो घातोऽङ्कस-ञ्ज्ञार्हः कथम्~। \textbf{अबबज}\renewcommand{\thefootnote}{३}\footnote{र्होऽस्ति \textbf{अबबज} {\en J.}}घातक्षेत्रस्याङ्कसञ्ज्ञार्हस्य तुल्यत्वात्\renewcommand{\thefootnote}{४}\footnote{र्हतुल्यत्वात् {\en J.}}\;। इदमेवेष्टं क्षेत्रमुपरित-नवत्~॥ 
\vspace{2mm}

\begin{center}
\textbf{\large अथ द्वाविंशतितमं क्षेत्रम्~॥~३२~॥}
\end{center}

{\ab  तत्र तादृशरेखाद्वयोत्पादनमिष्टं ययोर्वर्गौ भिन्नौ स्तः~। तयोर्वर्गयोगो मध्यक्षेत्रं भवति~। तयोर्द्विगुणो घातो द्विगुणप्रथममध्यक्षेत्रं भवति~। तयोर्द्विगुणो घातो द्विगुणप्रथममध्यक्षेत्राद्भिन्नं वा मध्यक्षेत्रं भवति~।} 

\newpage

\begin{wrapfigure}{r}{0.48\textwidth}
\vspace{-8mm}
\begin{center}
\includegraphics[scale=0.7]{Images/rg-36.png}
\end{center}
\vspace{-8mm}
\end{wrapfigure}

तत्र \,द्वे \,मध्यरेखे \,\textbf{अबबजे} \,कल्पिते~। अनयोर्वर्गौ केवलमिलितौ भवतः~। रेखे च मध्यक्षेत्रस्य भुजौ भवतः~। एकस्या वर्गो द्वितीयरेखावर्गस्य प्रथमरेखाभिन्नान्यरेखावर्गस्य च योगेन तुल्यो भवतीति\renewcommand{\thefootnote}{१}\footnote{भविष्यतीति {\en J.}} कल्पिते\renewcommand{\thefootnote}{२}\footnote{कल्प्यते {\en A. }}\;। अनयोरुपरितनप्रकारेणैव \textbf{अझबझे} इष्टरेखे उत्पाद्ये~। अनयोर्वर्गौ भिन्नौ भवतः~। अनयोर्योगो मध्यक्षेत्रतुल्यो भवतीति\renewcommand{\thefootnote}{३}\footnote{भविष्यतीति {\en J.}} पूर्वोक्तप्रकारेणैव ज्ञेयः~। अनयोः \textbf{अझबझ}योर्द्विगुणो घातो \renewcommand{\thefootnote}{४}\footnote{मध्यक्षेत्ररूपो जातः~। \textbf{अबबज}घातरुपस्य मध्यक्षेत्रस्य तुल्यत्वात्~। {\en A.}}मध्यक्षेत्रम्~। कुतः~। \textbf{अबबज}घातमध्यक्षेत्रतुल्योऽस्ति~। ततो मध्यक्षेत्रं प्रथममध्यक्षेत्रात् भिन्नं कुतोऽस्ति~। यस्मा\textbf{दबबजौ} भिन्नौ स्तः~। अनयोर्भिन्नत्वात्~। \textbf{अब}वर्गः \textbf{अबबज}घातश्च  भिन्नो\renewcommand{\thefootnote}{५}\footnote{{\en J. inserts} मिथः {\en before} भिन्नो.} भविष्यति~। इदमेवेष्टम्~। क्षेत्रं पूर्ववत्~॥ 
\vspace{2mm}

\begin{center}
\textbf{\large अथ त्रयस्त्रिंशत्तमं क्षेत्रम्~॥~३३~॥}
\end{center}

{\ab ययोर्भिन्नरेखयोर्वर्गावङ्कसञ्ज्ञार्हौ भवत\renewcommand{\thefootnote}{६}\footnote{भविष्यतः {\en J.}}स्तयोर्योगतुल्या या रेखा सा करणीगता भविष्यति~। इयं रेखा योगजाख्योच्यते~। }\\

\begin{wrapfigure}{r}{0.5\textwidth}
\vspace{-8mm}
\begin{center}
\includegraphics[scale=0.75]{Images/rg-37.png}
\end{center}
\vspace{-8mm}
\end{wrapfigure}

 यथा \;\textbf{अज}रेखा \;\textbf{अबबज}योगोत्पन्ना करणीरूपा भवति\renewcommand{\thefootnote}{७}\footnote{रुपास्ति {\en J.}}\;। तयोर्द्विगुणघातोऽङ्क-सञ्ज्ञार्हवर्गयोगात् भिन्नो भविष्यति~। अनयोर्भिन्नत्वात्~। तस्मात् अस्य \textbf{अज}स्य वर्गो द्वाभ्यां वर्गाभ्यां भिन्नो भविष्यति~। तस्मादियं करणीगता भविष्यति~॥ 
\vspace{2mm}
 
\begin{center}
\textbf{\large अथ चतुस्त्रिंशत्तमं क्षेत्रम्~॥~३४~॥ }
\end{center}

 {\ab ययोर्मध्यरेखयोः केवलवर्गौ मिलितौ भवतोऽङ्कसञ्ज्ञार्हक्षेत्रस्य द्वौ भुजौ भव-तस्तयो रेखयोर्योगतुल्या या रेखा भवति सा करणीरूपा भविष्यति~। इयं प्रथममध्ययोगरेखोच्यते~।}

\newpage
 
\begin{wrapfigure}{r}{0.48\textwidth}
\vspace{-4mm}
\begin{center}
\includegraphics[scale=0.7]{Images/rg-38.png}
\end{center}
\vspace{-8mm}
\end{wrapfigure}

यथा \textbf{अबबज}योगोत्पन्ना \textbf{अज}रेखा कर-णीरूपास्ति~। अनयोर्भिन्नत्वादनयोर्द्विगुण-घातोऽप्यनयोर्वर्गयोगात् भिन्नो भविष्यति~। तस्मात् रेखावर्गो द्विगुणघाताद्भिन्नो  भविष्यति~। तस्मादियं करणीरूपा भविष्यति~॥
\vspace{2mm}

\begin{center}
\textbf{\large अथ पञ्चत्रिंशत्तमं क्षेत्रम्~॥~३५~॥}
\end{center}

 {\ab ये मध्यरेखे केवलवर्गमिलिते मध्यक्षेत्रस्य भुजरूपे स्तस्तदा तयोर्योगतुल्या या रेखा सा करणीरूपा भविष्यति~। इयं च द्वितीयमध्ययोगरेखासञ्ज्ञा ज्ञेया~।} \\

\begin{wrapfigure}{r}{0.4\textwidth}
\vspace{-8mm}
\begin{center}
\includegraphics[scale=0.7]{Images/rg-38-1.png}
\end{center}
\vspace{-8mm}
\end{wrapfigure}

 यथा \textbf{अज}रेखा \textbf{अबबज}योगोत्पन्नास्ति~। \textbf{दह}-रेखाङ्कसञ्ज्ञार्हा कल्पिता~। अस्या  उपरि \textbf{अब}वर्ग-\textbf{बज}वर्गयोगतुल्यं \textbf{दझ}क्षेत्रं कार्यम्~। द्वयोर्द्विगुणघाततुल्यं \textbf{झत}क्षेत्रं च कार्यम्~। तदैते भिन्ने भविष्यतः~। \,रेखयोर्भिन्नत्वात~। \,तस्मात् \;\textbf{दववत}रेखे भिन्ने भविष्यतः~। अनयोर्वर्गावङ्कसञ्ज्ञार्हौ भवि-ष्यतः~। तस्मात् \textbf{दतं} योगरेखा भविष्यति~। \textbf{दह}म् अङ्कसञ्ज्ञार्हरेखा भविष्यति~। तस्मात् \textbf{हत}क्षेत्रं करणीरूपं भविष्यति~। तस्मात् \textbf{अज}रेखा करणीरूपा भविष्यति~। 
\vspace{2mm}

\begin{center}
\textbf{\large अथ षट्त्रिंशत्तमं}\renewcommand{\thefootnote}{१}\footnotetext{षट्त्रिंशं {\en J.}} \textbf{\large क्षेत्रम्~॥~३६~॥}
\end{center}

{\ab यदि द्वयो रेखयोर्वर्गौ भिन्नौ भवतो वर्गयोगश्चाङ्कसञ्ज्ञार्हो भवति तयोः द्विगुणघातो मध्यक्षेत्रसञ्ज्ञको भवति तद्योगतुल्या या रेखा सा करणीरूपा भविष्यति~। इयमधिकरेखासञ्ज्ञा~। }\\

\begin{wrapfigure}{r}{0.5\textwidth}
\vspace{-8mm}
\begin{center}
\includegraphics[scale=0.65]{Images/rg-39.png}
\end{center}
\vspace{-8mm}
\end{wrapfigure}

यथा \textbf{अज}रेखा \textbf{अबबज}योर्योगोत्पन्ना स्यात्~। अस्या विचारः क्षेत्रं च पूर्ववत् ज्ञेयं~॥ 

\newpage
\begin{center} 
\textbf{\large अथ सप्तत्रिंशत्तमं}\renewcommand{\thefootnote}{१}\footnote{सप्तत्रिंशं {\en J.}} \textbf{\large क्षेत्रम्~॥~३७~॥}
\end{center}

{\ab  ययो रेखयोर्वर्गौ भिन्नौ भवतो वर्गयोगश्च मध्यक्षेत्रं भवति द्विगुणघातोऽङ्कसञ्ज्ञार्हो भवति तद्रेखाद्वययोगतुल्या या रेखा भवति सा करणीगता भविष्यति~। अस्या वर्गोऽङ्कसञ्ज्ञार्हरेखामध्यरेखयोर्वर्गयोगतुल्योऽस्ति~। }\\

 यथा \textbf{अबबज}योगोत्पन्ना \textbf{अज}रेखास्ति~। अस्याः क्षेत्रं विचारश्च पूर्ववत्\renewcommand{\thefootnote}{२}\footnote{पूर्वोक्तवत् {\en K.}} ज्ञेयम्~। 
\vspace{2mm}
 
\begin{center}
\textbf{\large अथाष्टत्रिंशत्तमं}\renewcommand{\thefootnote}{3}\footnote{अथाष्टत्रिंशं {\en J.}} \textbf{\large क्षेत्रम्~॥~३८~॥}
\end{center}

{\ab  ययोर्वर्गौ भिन्नौ भवतो वर्गयोगश्च मध्यक्षेत्रं भवति तद्द्विगुणितघातो मध्यक्षेत्रं भवत्यनयोर्वर्गयोगमध्यक्षेत्रं द्विगुणघातमध्यक्षेत्राद्भिन्नं भवति तदा तयो रेखयोर्योगतुल्या या रेखा भवति सा करणीरूपा\renewcommand{\thefootnote}{४}\footnote{रेखा {\en J.}} भवति~। अस्या वर्गो मध्यरेखाद्वयवर्गयोगतुल्यो भवति~। }\\

 यथा \textbf{अज}रेखा \textbf{अबबज}योगोत्पन्नास्ति~। अस्या विचारः क्षेत्रं च पूर्वोक्तवत् ज्ञेयम्~॥ 
\vspace{-2mm}

\begin{center}
\textbf{\large अथैकोनचत्वारिंशत्तमं क्षेत्रम्~॥~३९~॥}
\end{center}

{\ab योगरेखाया योज्यखण्डे एकचिह्ने भवतः~।\renewcommand{\thefootnote}{५}\footnote{{\en A. inserts} यथा योगरेखा \textbf{अजं अबबजे} खण्डे एते \textbf{ब}चिह्ने एव भवतः~।}}\\

\begin{wrapfigure}{r}{0.5\textwidth}
\vspace{-8mm}
\begin{center}
\includegraphics[scale=0.65]{Images/rg-40.png}
\end{center}
\vspace{-8mm}
\end{wrapfigure}

 यद्यन्यस्मिंश्चिह्ने\renewcommand{\thefootnote}{६}\footnote{{\en J. has} तन्न्यूनाधिके यदि {\en \& c.}} भवतस्तदा तच्चिह्नं \textbf{दं} कल्पितम्~। \textbf{अबबज}वर्गयोग\textbf{अददज}वर्ग-योगान्तरमिदम\renewcommand{\thefootnote}{७}\footnote{{\en J. has} इष्टम् {\en for} इदम्.}ङ्कसञ्ज्ञार्हरूपम्~। द्विगुण-\textbf{अबबज}घातद्विगुण\textbf{अददज}घातयोरन्तरं द्वयोर्मध्ययोरन्तररूपमस्ति~। तस्मादन्तरमङ्कसञ्ज्ञार्हं करणीरूपं च भविष्यति~। इदमशुद्धम्~। इष्टं समीचीनम्~॥ 
 
\newpage
\begin{center}
\textbf{\large अथ चत्वारिंशत्तमं क्षेत्रम्~॥~४०~॥}
\end{center}

{\ab प्रथममध्ययोगरेखाया अपि योज्यखण्डे एकचिह्ने भवतो नान्यत्र~। }\\

\begin{wrapfigure}{r}{0.46\textwidth}
\vspace{-8mm}
\begin{center}
\includegraphics[scale=0.42]{Images/rg-41.png}
\end{center}
\vspace{-8mm}
\end{wrapfigure}

 यद्यन्यत्र ~भवतस्तदा ~कल्पितं \;\textbf{द}चिह्रे भवतः~। तत्र \textbf{अबबज}योर्वर्गयोगस्य \textbf{अददज}-योरपि वर्गयोगस्यान्तरं द्वयोर्मध्यमयोरन्तररूपं \,द्वयोः \,सञ्ज्ञार्हयोरन्तररूपस्य \,\textbf{अबबज}द्विगुणघात\textbf{अददज}घातयोरन्तरस्य \,तुल्यम् अस्ति~। इदमशुद्धम्~। अस्मदिष्टं समीचीनम्~॥ 
\vspace{2mm}
 
\begin{center}
\textbf{\large अथैकचत्वारिंशत्तमं क्षेत्रम्~॥~४१~॥}
\end{center}

{\ab द्वितीयमध्ययोगरेखाया योज्यखण्डे एकचिह्ने भवतः~।}\\

\begin{wrapfigure}{r}{0.42\textwidth}
\vspace{-8mm}
\begin{center}
\includegraphics[scale=0.45]{Images/rg-42.png}
\end{center}
\vspace{-8mm}
\end{wrapfigure}

यद्यन्यत्र स्यात्तदा \textbf{द}चिह्नं कल्पितम्~। तत्र \textbf{हझ}रेखाङ्कसञ्ज्ञार्हा कल्पिता~। अस्या उपरि \textbf{अब-बज}योर्वर्गयोगतुल्यं \textbf{झव}क्षेत्रं कार्यम्~। अनयोर्द्विगुणघाततुल्यं \textbf{कतं} क्षेत्रं कार्यम्~। तस्मात् \textbf{हक}-रेखाया \textbf{व}चिह्नोपरि द्वौ विभागौ स्तः~। तस्मादियं योगरेखा भविष्यति~। पुन\textbf{र्हझ}रेखोपरि \textbf{अद-दज}वर्गयोगतुल्यं \textbf{झल}क्षेत्रं कार्यम्~। तत्र \textbf{मक}-क्षेत्रं द्वयोर्घातयोर्द्विगुणतुल्यं भविष्यति~। तस्मात् \textbf{हक}रेखाया \,\textbf{ल}चिह्ने \,विभागद्वयं \,जातम्~। इयं योगरेखा भविष्यति~। इदमशुद्धम्~। अस्मदिष्टं समीचीनम्~॥ 
\vspace{2mm}

\begin{center}
\textbf{\large अथ द्विचत्वारिंशत्तमं क्षेत्रम्~॥~४२~॥}
\end{center}

{\ab अधिकरेखाया अप्येकचिह्ने एव खण्डद्वयं भविष्यति नान्यत्र~।}\\ 

\begin{wrapfigure}{r}{0.45\textwidth}
\vspace{-12mm}
\begin{center}
\includegraphics[scale=0.6]{Images/rg-43.png}
\end{center}
\vspace{-8mm}
\end{wrapfigure}

 यद्यन्यत्र भवति तदा \textbf{द}चिह्नं कल्पितम्~। पूर्वोक्तप्रकारेणैवात्रानुपपत्तिर्ज्ञेया~॥

\newpage
\begin{center}
\textbf{\large अथ त्रिचत्वारिंशत्तमं क्षेत्रम्~॥~४३~॥ }
\end{center}

{\ab अङ्कसञ्ज्ञार्हरेखावर्गमध्यरेखावर्गयोगतुल्यो\renewcommand{\thefootnote}{१}\footnote{अङ्कसञ्ज्ञार्हमध्यरेखावर्गयोगतुल्यो {\en J.}} \;यस्या \;रेखाया \;वर्गो \;भवति तस्या अपि योज्यखण्डे एकचिह्ने भवतः~। }\\

\begin{wrapfigure}{r}{0.4\textwidth}
\vspace{-12mm}
\begin{center}
\includegraphics[scale=0.6]{Images/rg-44.png}
\end{center}
\vspace{-8mm}
\end{wrapfigure}

यद्यन्यत्र भवतस्तदा \textbf{द}चिह्नं कल्पितम्~। पूर्वो-क्तप्रकारेणात्राप्यनुपपत्तिर्ज्ञेया~॥ 
\vspace{2mm}

\begin{center}
\textbf{\large अथ चतुश्चत्वारिंशत्तमं क्षेत्रम्~॥~४४~॥}
\end{center}

{\ab द्वयोर्मध्यरेखयोर्वर्गयोगतुल्यो यस्या रेखाया वर्गो भवति तस्या रेखाया अपि योज्यखण्डे एकचिह्ने एव भविष्यतो नान्यत्र~। }\\

\begin{wrapfigure}{r}{0.4\textwidth}
\vspace{-12mm}
\begin{center}
\includegraphics[scale=0.6]{Images/rg-45.png}
\end{center}
\vspace{-8mm}
\end{wrapfigure}

 यदि भवतस्तदा \textbf{द}चिह्नं कल्पितम्~। पुनः पूर्वो-क्तप्रकारेणात्राप्यनुपपत्तिर्ज्ञेया~॥ 
\vspace{2mm}

\begin{center}
\textbf{\large अथ शेषक्षेत्राणां परिभाषा प्रथमं लिख्यते~॥ }
\end{center}

 योगरेखाया महत्खण्डवर्गो लघुखण्डवर्गस्य बृहद्रेखामिलितान्यरेखावर्गस्य च योगेन तुल्यो भवति पुनर्महत्खण्डं चेदिष्टसञ्ज्ञार्हरेखामिलितं भवति तदा सा प्रथमयोगरेखोच्यते~। \\
\vspace{-2mm}

 यदि तत्र लघुखण्डरेखावर्ग इष्टसञ्ज्ञार्हरेखामिलितो भवति तदा सा द्वितीययोगरेखाभिधा भवति~। \\
\vspace{-2mm}

यदि खण्डद्वयस्य वर्गौ केवलाङ्कसञ्ज्ञार्हौ भवतस्तदा तृतीययोगरेखासञ्ज्ञका भवति~। \\
\vspace{-2mm}

 यदि महत्खण्डवर्गो लघुखण्डवर्गस्य महत्खण्डभिन्नान्यरेखावर्गस्य च योगेन तुल्यो भवति पुनर्महत्खण्डं चेदङ्कसञ्ज्ञार्हं स्यात्तदेयं चतुर्थी योगसञ्ज्ञा रेखा भवति~। \\
\vspace{-2mm}

 यदि च लघुखण्डमङ्कसञ्ज्ञार्हं भवति तदा पञ्चमी योगसञ्ज्ञा रेखा भवति~।

\newpage

 यदि द्वे खण्डे केवलवर्गसञ्ज्ञार्हे भवतस्तदा षष्ठी योगसञ्ज्ञा रेखा भवति~॥ 
\vspace{2mm}

\begin{center}
\textbf{\large अथ पञ्चचत्वारिंशत्तमं क्षेत्रम्~॥~४५~॥}
\end{center}

{\ab तत्र प्रथमयोगरेखोत्पादनमिष्टमस्ति~। }\\

\begin{wrapfigure}{r}{0.46\textwidth}
\vspace{-8mm}
\begin{center}
\includegraphics[scale=0.65]{Images/rg-46.png}
\end{center}
\vspace{-8mm}
\end{wrapfigure}

 तत्र प्रथमं \textbf{अ}रेखा इष्टसञ्ज्ञार्हा कल्प्या~। पुनस्तन्मिलिता \textbf{बज}रेखा कल्पिता~। द्वौ वर्ग-राश्यङ्कौ \textbf{दहदझौ} तथा कल्प्यौ\renewcommand{\thefootnote}{१}\footnote{कल्पितौ {\en D.}} यथानयोर-न्तरं \textbf{झहं} वर्गराशिर्न भवति~। पुन\textbf{र्बज}वर्ग\textbf{जव}-वर्गयोर्निष्पत्ति\textbf{र्दहझह}निष्पत्तितुल्या कल्पिता~। तस्मात् \textbf{बवं} प्रथमयोगरेखा भविष्यति~। 
 
\begin{center}
अस्योपपत्तिः~।
\end{center}

 \textbf{बजं} महत्खण्डम् अङ्कसञ्ज्ञार्हमस्ति~। \textbf{जव}खण्डमस्माद्भिन्नमस्ति~। केवलं मिलितवर्गो भवति~। वर्गश्चाङ्कसञ्ज्ञार्होऽस्ति~। \textbf{बज}वर्ग\textbf{जव}वर्गयोरन्तरं \textbf{त}वर्गतुल्यं भवतीति कल्पितम्~। तस्मात् \textbf{बज}वर्ग\textbf{ज}वर्गयोरन्तरं \textbf{त}वर्गतुल्यं भवतीति कल्पितम्~। यस्मात् \textbf{बज}वर्ग-\textbf{त}वर्गयोर्निष्पत्ति\textbf{र्दहदझ}योर्निष्पत्तितुल्या भविष्यति~। तस्मात् \textbf{तं बजे}न मिलितं भविष्यति~। \textbf{बज}वर्गोऽपि \textbf{जब}वर्ग\textbf{त}वर्गयोगतुल्यो भविष्यति~। 
\vspace{2mm}

\begin{center}
\textbf{\large अथ षट्चत्वारिंशत्तमं क्षेत्रम्~॥~४६~॥}
\end{center}

{\ab तत्र द्वितीययोगरेखोत्पादनमिष्टमस्ति~। }\\

\begin{wrapfigure}{r}{0.47\textwidth}
\vspace{-12mm}
\begin{center}
\includegraphics[scale=0.65]{Images/rg-47.png}
\end{center}
\vspace{-8mm}
\end{wrapfigure}

 प्रथममिष्टसञ्ज्ञार्हा \,\textbf{अ}रेखा \,कल्पिता~। तन्मिलिता \;\textbf{जब}रेखा \;कल्पिता~। \;द्वावङ्कौ पूर्ववत् कल्प्यौ~। \textbf{जबजव}वर्गयोर्निष्पत्ति\textbf{र्झह-दह}निष्पत्तितुल्या कल्पिता~। तस्मात् \textbf{बवं} द्वितीययोगरेखा भविष्यति~। 

\newpage
\begin{center} 
अस्योपपत्तिः~।
\end{center}

 \textbf{जबं} लघुखण्डमङ्कसञ्ज्ञार्हमस्ति~। \textbf{वज}स्य केवलवर्गोऽङ्कसञ्ज्ञार्होऽस्ति~। \textbf{वज}महत्ख-ण्डस्य वर्गो \textbf{जब}वर्गस्य \textbf{बज}मिलितरेखावर्गस्य च योगेन तुल्योऽस्ति~। क्षेत्रं च पूर्ववत् ज्ञेयम्~॥ 
\vspace{2mm}
 
\begin{center}
\textbf{\large अथ सप्तचत्वारिंशत्तमं क्षेत्रम्~॥~४७~॥}
\end{center}

{\ab तत्र तृतीययोगरेखोत्पादनमिष्टम्~। }\\

\begin{wrapfigure}{r}{0.46\textwidth}
\vspace{-8mm}
\begin{flushright}
\includegraphics[scale=0.75]{Images/rg-48.png}
\end{flushright}
\vspace{-8mm}
\end{wrapfigure}

 तत्र प्रथममिष्टस\renewcommand{\thefootnote}{१}\footnote{प्रथममङ्कसं {\en J.}}ञ्ज्ञार्हरेखा \textbf{अ}कल्पिता~। द्वौ वर्गराश्यङ्कौ \textbf{झवझतौ} कल्पितौ~। अनयो-रन्तरं \;\textbf{वतं} \,यथा \;वर्गो \;भवति \,तथा \,न  कार्यौ~। अन्याङ्को \textbf{हं} कल्पितः~। अयं वर्ग-राशिर्नास्ति~। पुनरस्य निष्पत्ति\textbf{र्वते}न वर्गरा-श्योर्निष्पत्तिर्न भवेत्तथा कल्प्या~। पुनः \textbf{अ}रेखा-वर्गनिष्पत्ति\textbf{र्बद}वर्गेण तथा कल्प्या यथा \textbf{ह}स्य निष्पत्ति\textbf{र्झते}नास्ति~। \textbf{बद}वर्गस्य निष्पति\textbf{र्दज}वर्गेण तथास्ति यथा \textbf{झत}निष्पत्ति\textbf{र्वते}नास्ति~। तस्मात् \textbf{बजं} तृतीययोगरेखा जाता~॥ 
 
\begin{center}
अस्योपपत्तिः~।
\end{center}

 \textbf{बज}खण्डे \textbf{अ}रेखाभिन्ने स्तः~। खण्डयोर्वर्गावङ्कसञ्ज्ञार्हौ स्तः~। \textbf{बद}वर्गो \textbf{दज}रेखावर्ग\textbf{बद}-रेखामिलित\textbf{क}रेखावर्गयोगतुल्योऽस्ति~। कुतः~। \textbf{वद}वर्गः \textbf{क}वर्गश्च \textbf{झतझव}निष्पत्तावस्ति~॥
\vspace{-2mm}

\begin{center}
\textbf{\large अथाष्टचत्वारिंशत्तमं क्षेत्रम्~॥~४८~॥}
\end{center}

{\ab तत्र चतुर्थयोगरेखोत्पादनमिष्टमस्ति~। }\\

 प्रथमयोगरेखोक्तप्रकारोऽत्रापि कार्यः~। विशेषस्तु \textbf{दझझहौ} द्वौ वर्ग-

\newpage

\begin{wrapfigure}{r}{0.55\textwidth}
\vspace{-6mm}
\begin{flushright}
\includegraphics[scale=0.7]{Images/rg-49.png}
\end{flushright}
\vspace{-8mm}
\end{wrapfigure}

राशी तथा कल्प्यौ यथैतयोः योगो वर्गराशिर्न भवति~। तस्मात् \textbf{बज}वर्गो \textbf{जव}वर्ग\textbf{व}रेखाभिन्न\textbf{त}वर्गयोः योगतुल्योऽस्ति~। कुतः~। यतो \textbf{वज}-वर्ग\textbf{त}वर्गौ \textbf{दहदझ}योर्निष्पत्तौ स्तः~॥
\vspace{2mm}

\begin{center}
\textbf{\large अथैकोनपञ्चाशत्तमं क्षेत्रम्~॥~४९~॥}
\end{center}

{\ab तत्र पञ्चमयोगरेखोत्पादनमिष्टमस्ति~। }\\

\begin{wrapfigure}{r}{0.55\textwidth}
\vspace{-12mm}
\begin{flushright}
\includegraphics[scale=0.7]{Images/rg-50.png}
\end{flushright}
\vspace{-8mm}
\end{wrapfigure}

तत्र द्वितीययोगरेखोक्तप्रकारोऽत्र कार्यः~। परं च \textbf{दहझह}राशी चतुर्थयोगरेखोक्तवत्कार्यौ~। \\
\vspace{2mm}

\begin{center}
\textbf{\large अथ पञ्चाशत्तमं क्षेत्रम्~॥~५०~॥}
\end{center}

{\ab तत्र षष्ठयोगरेखोत्पादनमिष्टम्~। }\\

\begin{wrapfigure}{r}{0.55\textwidth}
\vspace{-12mm}
\begin{flushright}
\includegraphics[scale=0.7]{Images/rg-51.png}
\end{flushright}
\vspace{-8mm}
\end{wrapfigure}

 तत्र द्वितीयरेखोक्तवत् प्रकारः कार्यः~। द्वावङ्कराशी चतुर्थरेखोक्तवत् कार्यौ~। इदमेवास्माकमिष्टम्~॥ \\
\vspace{2mm}
 
\begin{center}
\textbf{\large अथैकपञ्चाशत्तमं क्षेत्रम्~॥~५१~॥}
\end{center}

तत्रैकक्षेत्रस्यैको \,भुजोऽङ्कसञ्ज्ञार्हो \,भवति \,द्वितीयभुजः \,प्रथमयोगरेखा \,भवति \,तत्र यस्या रेखाया वर्ग एतत्क्षेत्रफलतुल्यो भवति सा योगरेखा भवति~। \\
\vspace{-2mm}

 यथा \textbf{बज}क्षेत्रम्~। एक अङ्कसञ्ज्ञार्हः \textbf{अब}भुजः~। द्वितीयः प्रथमयोगरेखा \textbf{अज}भुजः~। \textbf{अज}स्य \textbf{द}चिह्ने द्वौ विभागौ कल्पनीयौ यथा \textbf{अदं} महत्खण्डं \textbf{दजं} न्यूनखण्डं च\renewcommand{\thefootnote}{१}\footnote{\textbf{दजं} च न्यूनखण्डं J.} कल्पितं भवेत्~। \\
\vspace{-2mm}
 
 पुन\textbf{र्दजं ह}चिह्नेऽर्द्धितं कार्यम्~। पुन\textbf{र्दह}वर्गो \textbf{दज}वर्गचतुर्थांशतुल्यः 

\newpage

\begin{wrapfigure}{r}{0.62\textwidth}
\vspace{-6mm}
\begin{flushright}
\includegraphics[scale=0.58]{Images/rg-52.png}
\end{flushright}
\vspace{-8mm}
\end{wrapfigure}

\textbf{अद}स्यैकखण्डोपरि तथा कार्यो यथा शेषखण्डक्षेत्रं वर्ग-तुल्यमवशिष्यते~। तस्मात् \textbf{अद}-रेखाया \textbf{झ}चिह्नोपरि खण्डद्वयं भविष्यति~। \textbf{अझझदौ} मिलितौ भविष्यतः~। पुन\textbf{र्झवदतहक}-रेखा \textbf{अब}रेखायाः समानान्तराः कार्याः~। पुनः \textbf{अव}क्षेत्रतुल्यं \textbf{सन}क्षेत्रं समकोणसमचतुर्भुजं कार्यम्~। \textbf{वद}क्षेत्रतुल्यं~\textbf{मनं} समकोणसमचतुर्भुजं क्षेत्रं कार्यम्~। \textbf{गख}क्षेत्रं समकोणसमचतुर्भुजं सम्पूर्णं कार्यम्~। \textbf{सन}समकोणसमचतुर्भुजक्षेत्रस्य निष्पत्ति\textbf{र्नग}क्षेत्रेण\renewcommand{\thefootnote}{१}\footnote{क्षेत्रस्य {\en J.}} \textbf{सफफग}निष्पत्तिरूपा \textbf{फननछ}निष्प-त्तिरूप\textbf{नगनम}क्षेत्रनिष्पत्तितुल्यास्ति~। तदा \textbf{नग}क्षेत्रं \textbf{सन}क्षेत्र\textbf{नम}क्षेत्रयोर्मध्ये एकनि\renewcommand{\thefootnote}{२}\footnote{र्मध्येऽप्येकनि {\en J.}}ष्पत्तौ पतिष्यति~। तदा \textbf{अववद}योर्मध्येऽप्येकनिष्पत्तौ पतिष्यति~। \textbf{तह}क्षेत्रं द्वयोर्मध्ये एकनि\renewcommand{\thefootnote}{३}\footnote{र्मध्येऽप्येकनि {\en J.}}ष्प-त्तावासीत्~। \,कुतः~। \,\textbf{अझदह}निष्पत्तिः \,\textbf{दहझद}निष्पत्तितुल्यास्ति~। \,तस्मात् \,\textbf{नगतहौ} समानौ भविष्यतः~। तस्मात् \textbf{वजं गख}तुल्यं भविष्यति~। तस्मादस्य भुजो योगरेखा भविष्यति~। कुतः~। \textbf{अझझदौ अदे}न मिलितावङ्कसञ्ज्ञार्हौ स्तः~। तस्मात् \textbf{अववदौ सन-नम}तुल्यावङ्कसञ्ज्ञार्हौ भविष्यतः~। तस्मात् \textbf{सफफग}वर्गावङ्कसञ्ज्ञाहौ भविष्यतः~। पुनः \textbf{अववदौ} अङ्कसञ्ज्ञार्हौ~। \textbf{तहहल}मध्यक्षेत्राभ्यां भिन्नौ स्तः~। तस्मात् \textbf{सननगौ} भिन्नौ भवि-ष्यतः~। तस्मात् \textbf{सफफगौ} भिन्नौ भविष्यतः\renewcommand{\thefootnote}{४}\footnote{तस्मात् \textbf{सगं} योगरेखा भविष्यति~। {\en D.,K.}}\;। तस्मात् \textbf{बज}तुल्यो यस्या रेखाया वर्गः सा \textbf{सग}रेखा योगरेखा भविष्यति~॥
\vspace{2mm}

\begin{center}
\textbf{\large अथ द्विपञ्चाशत्तमं क्षेत्रम्~॥~५२~॥}
\end{center}

{\ab यस्य क्षेत्रस्यैको भुजोऽङ्कसञ्ज्ञार्हो भवति द्वितीयो भुजो }

\newpage
\noindent {\ab  द्वितीययोगरेखा भवति यस्या रेखाया वर्ग एतत्क्षेत्रतुल्यो भवति सा प्रथममध्ययोगरेखा भविष्यति~। }\\

\begin{wrapfigure}{r}{0.58\textwidth}
\vspace{-10mm}
\begin{flushright}
\includegraphics[scale=0.65]{Images/rg-53.png}
\end{flushright}
\vspace{-8mm}
\end{wrapfigure}

 यथा \textbf{बज}क्षेत्रम् \textbf{अब}म् अङ्कसञ्ज्ञार्हो भुजः \textbf{अजं} द्वितीययोगरेखाभुजश्च कल्प्यः~। \\

 उपरितनप्रकारवत्कार्यम्~। परं च \textbf{अव}क्षेत्र\textbf{वद}क्षेत्रे मिथो मिलिते मध्यक्षेत्रे भविष्यतः~। \textbf{अत}मध्य-क्षेत्रेण च मिलिते भविष्यतः~। \textbf{दककजौ} अङ्कसञ्ज्ञार्हक्षेत्रे भविष्यतः~। तस्मात् \textbf{सनमनौ} मिलितमध्यक्षेत्रे भविष्यतः~। \textbf{नगनख}क्षेत्रे अङ्कसञ्ज्ञार्हे भविष्यतः~। तस्मात् \textbf{सफफगौ} केवलमध्यमिलितवर्गौ अङ्कसञ्ज्ञार्ह\textbf{नग}क्षेत्रस्य भुजौ भविष्यतः~। तस्मात् \textbf{सग}रेखा प्रथममध्ययोगरेखा भविष्यति~॥
\vspace{2mm}

\begin{center}
\textbf{\large अथ त्रिपञ्चाशत्तमं क्षेत्रम्~॥~५३~॥} 
\end{center}

 {\ab एकक्षेत्रस्यैको भुजोऽङ्कसञ्ज्ञार्हरेखा द्वितीयभुजश्च तृतीययोगरेखा भवति\renewcommand{\thefootnote}{१}\footnote{भविष्यति {\en J.}} तदा यस्या रेखाया वर्ग एतत्क्षेत्रतुल्यो भवति सा द्वितीयमध्ययोगरेखा भविष्यति~।} \\

\begin{wrapfigure}{r}{0.58\textwidth}
\vspace{-14mm}
\begin{flushright}
\includegraphics[scale=0.7]{Images/rg-54.png}
\end{flushright}
\vspace{-8mm}
\end{wrapfigure}

 तत्र क्षेत्रं\renewcommand{\thefootnote}{२}\footnote{तत् क्षेत्रं {\en J.}} द्वौ भुजौ चोपरि-तनोक्तवत्कल्प्यं तदुक्तवत्~। कार्यं च~। परं च   \textbf{अववद}क्षेत्रे मध्य-मिलिते भविष्यतः~। \textbf{दककजौ} च मध्यौ भविष्यतः~। अतं  च \textbf{तजा}त् भिन्नं भ-\\

\newpage
\noindent विष्यति~। \,तस्मात् \,\textbf{सननम}क्षेत्रे \,मध्यमिलिते \,भविष्यतः~। \,\textbf{नगनख}क्षेत्रे \,च \,मध्यभिन्ने भविष्यतः~। तस्मात् \textbf{सफफगे} मध्यकेवलवर्गमिलिते भुजौ \textbf{नग}मध्यक्षेत्रस्य भविष्यतः~। तस्मात् \textbf{सगं} द्वितीयमध्ययोगरेखा भविष्यति~। इदमेवेष्टम्~॥ 
\vspace{2mm}

\begin{center}
\textbf{\large अथ चतुःपञ्चाशत्तमं क्षेत्रम्~॥~५४~॥} 
\end{center}

{\ab एकक्षेत्रस्यैको \,भुजोऽङ्कसञ्ज्ञार्हो \,द्वितीयो \,भुज\renewcommand{\thefootnote}{१}\footnote{{\en J. drops} भुजः.}श्चतुर्थी \,योगरेखा\renewcommand{\thefootnote}{२}\footnote{{\en A. has} चतुर्थयोगरेखा.}\;। \,अस्य वर्गतुल्यो भुजोऽधिकरेखास्ति\renewcommand{\thefootnote}{३}\footnote{भविष्यति {\en A.,J.}}\;। }\\

\begin{wrapfigure}{r}{0.58\textwidth}
\vspace{-10mm}
\begin{flushright}
\includegraphics[scale=0.6]{Images/rg-55.png}
\end{flushright}
\vspace{-8mm}
\end{wrapfigure}

 अस्य \,विचारः \,क्षेत्रं \,च \,पूर्व-वत् ज्ञेयम्~। \;विशेषस्तु \;\textbf{अझझदौ} भिन्नौ भविष्यतः~। \textbf{अत}क्षेत्रं \textbf{सनन-म}योगतुल्यमङ्कसञ्ज्ञार्हं भविष्यति~। \textbf{तज}क्षेत्ररूपो \textbf{नगनख}योगो मध्यो भविष्यति~। \,तस्मात् \,\textbf{सफफगौ} भिन्नवर्गौ भविष्यतः~। द्वयोर्वर्गयोगोऽङ्कसञ्ज्ञार्हो भविष्यति~। द्विगुणघातो मध्यो भविष्यति~। तस्मात् \textbf{सग}म् अधिकरेखा भविष्यति~॥
\vspace{2mm}

\begin{center}
\textbf{\large  अथ पञ्चपञ्चाशत्तमं क्षेत्रम्~॥~५५~॥}
\end{center}

{\ab क्षेत्रस्यैकभुजोऽङ्कसञ्ज्ञार्हो \;भविष्यति \;द्वितीयो \;पञ्चमयोगरेखा \;भवति~। एतत्तुल्यो\renewcommand{\thefootnote}{४}\footnote{एतत्क्षेत्रतुल्यो {\en J.}} \,यस्या \,रेखाया \,वर्गः \,सोऽङ्कसञ्ज्ञार्हरेखावर्गमध्यरेखावर्गयोगतुल्यो भवति~।}\\

 अस्यापि प्रकारः\renewcommand{\thefootnote}{५}\footnote{विचारः {\en J.}} क्षेत्रं च पूर्ववत् ज्ञेयम्\renewcommand{\thefootnote}{६}\footnote{बोध्यम्. {\en J.}}\;। परं चात्र \textbf{अझझदौ}

\newpage

\begin{wrapfigure}{r}{0.57\textwidth}
\vspace{-4mm}
\begin{flushright}
\includegraphics[scale=0.6]{Images/rg-56.png}
\end{flushright}
\vspace{-8mm}
\end{wrapfigure}

\noindent भिन्नौ भवतः~। \textbf{अत}क्षेत्ररूपः \textbf{सन-नम}योगो मध्यो भवति~। \textbf{तज}क्षे-त्ररूपो ~~\textbf{नगनख}योगोऽङ्कसञ्ज्ञार्हो भवति~। तस्मात् \textbf{सफफगौ} भिन्न-वर्गौ \,भविष्यतः~। \,अनयोः \,योगो मध्यो भवति\renewcommand{\thefootnote}{१}\footnote{भविष्यति {\en J.}}\;। द्विगुणघातोऽङ्कसञ्ज्ञार्हो भविष्यति~। तस्मात् \textbf{सग}-वर्गोऽङ्कसञ्ज्ञार्हमध्ययोगतुल्यो भविष्यति~॥ 
\vspace{2mm}

\begin{center}
\textbf{\large अथ षट्पञ्चाशत्तमं क्षेत्रम्~॥~५६~॥ }
\end{center}

{\ab क्षेत्रस्यैकभुजोऽङ्कसञ्ज्ञार्हो भवति द्वितीयश्च षष्ठी योगरेखा भवति~। अस्य तुल्यो वर्गो मध्यद्वयवर्गयोगतुल्यो भवति~। }\\

\begin{wrapfigure}{r}{0.6\textwidth}
\vspace{-9mm}
\begin{flushright}
\includegraphics[scale=0.65]{Images/rg-57.png}
\end{flushright}
\vspace{-8mm}
\end{wrapfigure}

 अस्य प्रकारः क्षेत्रं च पूर्ववत्~। अपरम् \textbf{अझझदौ} भिन्नौ भविष्यतः~। \textbf{अत}क्षेत्ररुप\textbf{सननमौ} मध्यौ भवतः~। \textbf{तज}क्षेत्ररूप\textbf{नग-नखौ} \,मध्यौ \,भवतः~। \,पूर्वस्मात् मध्याद्भिन्नौ \;भवतः~। \;तस्मात् \textbf{सफफगौ} \,भिन्नवर्गौ\renewcommand{\thefootnote}{२}\footnote{भिन्नौ वर्गौ {\en J.}} \,भवतः~। अनयोर्वर्गयोगो \,मध्यो \,भविष्यति~। \,द्विगुणघातो \,मध्यो \,भविष्यति~। \,प्रथमाद्भिन्नश्च~। तस्मात् \textbf{स}गवर्गो मध्यद्वययोगतुल्यो भविष्यति~। इदमिष्टम्\renewcommand{\thefootnote}{३}\footnote{इदमेवेष्टम् J.}\;॥ 
\vspace{2mm}
 
\begin{center}
\textbf{\large  अथ सप्तपञ्चाशत्तमं क्षेत्रम्~॥~५७~॥ }
\end{center}

 {\ab अङ्कसञ्ज्ञार्हरेखायां योगरेखावर्गतुल्यं क्षेत्रं भवति~। तदा द्वितीयो भुजः प्रथमयोगरेखा भविष्यति~।} 

\newpage

\begin{wrapfigure}{r}{0.54\textwidth}
\vspace{-3mm}
\begin{flushright}
\includegraphics[scale=0.6]{Images/rg-58.png}
\end{flushright}
\vspace{-8mm}
\end{wrapfigure}

\textbf{अब}योगरेखाया \textbf{ज}चिह्ने द्वे खण्डे कल्पनीये~। पुन\textbf{र्दह}अङ्कसञ्ज्ञार्हरेखायां \textbf{अब}वर्गतुल्यं ~\;\textbf{हझ}क्षेत्रं ~\;कल्प्यम्~। तस्मात् \,\textbf{दह}रेखाया \;द्वितीयो \;भुजः प्रथमयोगरेखा भविष्यति~। \textbf{अज}वर्गो \textbf{हव}क्षेत्रतुल्यो \textbf{जब}वर्ग\textbf{स्तक}क्षेत्रतुल्यः कल्प्यः\renewcommand{\thefootnote}{१}\footnote{कार्यः {\en A.}}\;। शेषं \textbf{लझ}म् \textbf{अजजब}द्वि-गुणघाततुल्यमवशिष्यते~। \textbf{कझं म}चिह्नोपरि अर्द्धं\renewcommand{\thefootnote}{२}\footnote{अर्धितं A.} कार्यम्~। पुन\textbf{र्दह}समानान्तरा \textbf{मन}-रेखा \,कार्या~। \,तत्र \,\textbf{अजजव}वर्गयोगोऽङ्कसञ्ज्ञार्होऽस्ति~। \,तस्मात् \,\textbf{हक}क्षेत्रमङ्कसञ्ज्ञार्हं भविष्यति\renewcommand{\thefootnote}{३}\footnote{अस्ति {\en A.}}\;। \textbf{दक}म् अङ्कसञ्ज्ञार्हम् अस्ति\renewcommand{\thefootnote}{४}\footnote{सञ्ज्ञार्हं भवति {\en J.}}\;। \textbf{दवं वकं} मिलितं भविष्यति~। \textbf{अजजव}घातो मध्योऽस्ति~। तस्मात् \textbf{लझं} मध्यो भविष्यति~। \textbf{कझं} केवलवर्गाङ्कसञ्ज्ञार्हो भविष्यति~। \textbf{दह}भिन्नो भविष्यति~। \textbf{अजजब}वर्गयोगः \textbf{अजजब}द्विगुणघातादधिकोऽस्ति~। तस्मात् \textbf{दकं कझा}दधिकं भविष्यति~। \textbf{अजजब}घातः \textbf{अजजब}वर्गयोर्मध्यनिष्पत्तिरस्ति~। \textbf{कनं दततक}-योर्मध्यनिष्पत्तिर्भविष्यति~। \textbf{कमं दववक}योर्मध्यनिष्पत्तिर्भविष्यति~। \\

पुन\textbf{र्दवकम}निष्पत्तिः \textbf{कमवक}निष्पत्तितुल्यास्ति~। पुनः \textbf{कझ}वर्गचतुर्थांशरूपः \textbf{कम}वर्गो \textbf{दके} कार्यः~। तदा \textbf{दकं व}चिह्ने मिलितविभागं भवति~। तस्मात् \textbf{दक}वर्गः \textbf{कझ}वर्गस्य मिलितान्यरेखावर्गस्य च योगेन तुल्यो भविष्यति~। इदमिष्टम्~।
\vspace{2mm}

\begin{center}
\textbf{\large  अथाष्टपञ्चाशत्तमं क्षेत्रम्~॥~५८~॥} 
\end{center}

 {\ab अङ्कसञ्ज्ञार्हरेखायां प्रथममध्ययोगरेखावर्गतुल्यं क्षेत्रं कार्यं तदा द्वितीयो भुजो द्वितीययोगरेखा भवति~।} 

\newpage

\begin{wrapfigure}{r}{0.48\textwidth}
\vspace{-5mm}
\begin{flushright}
\includegraphics[scale=0.55]{Images/rg-59.png}
\end{flushright}
\vspace{-8mm}
\end{wrapfigure}

 क्षेत्रं प्रकारश्च पूर्ववत् ज्ञेयः~। अत्र \textbf{हकं} मध्यो भविष्यति~। \textbf{अजजब}वर्गयोगो \textbf{हवतक}रूपौ मध्यमिलितौ भवतः~। कुतः~। \textbf{अजजव}योरङ्कसञ्ज्ञार्हत्वात्~। तस्मात् \textbf{दककझौ} केवलवर्गावङ्कसञ्ज्ञार्हौ भविष्यतः~। \textbf{कझ}म् अङ्कसञ्ज्ञार्हमस्ति~। तस्मात् \textbf{दक}वर्गः \textbf{कझ}वर्गमिलितरेखावर्गयोर्योगतुल्यो भविष्यति~। कुतः~। \textbf{दववक}योर्मिलितत्वात्~। तस्मात् \textbf{दझं} द्वितीययोगरेखा भविष्यति~॥ 
\vspace{2mm}

\begin{center}
\textbf{\large अथैकोनषष्टितमं क्षेत्रम्~॥~५९~॥}
\end{center}

{\ab अङ्कसञ्ज्ञार्हरेखायां द्वितीयमध्ययोगरेखावर्गतुल्यं क्षेत्रं कार्यं द्वितीयभुजस्तृतीययोगरेखा भविष्यति~। }\\

\begin{wrapfigure}{r}{0.45\textwidth}
\vspace{-8mm}
\begin{flushright}
\includegraphics[scale=0.55]{Images/rg-60.png}
\end{flushright}
\vspace{-8mm}
\end{wrapfigure}

 क्षेत्रं प्रकारश्च पूर्ववत्~। परं \textbf{हक}म् अत्र मध्यो भविष्यति~। यतः \textbf{अजजब}वर्गौ मध्यमिलितौ स्तः~। \textbf{लझं} मध्यो \textbf{हका}द्भिन्नो भविष्यति~। \textbf{अजजब}योर्भिन्नत्वात्~। तस्मात् \textbf{दक-कझे} \;वर्गावङ्कसञ्ज्ञार्हौ \;भविष्यतः\renewcommand{\thefootnote}{१}\footnote{J. omits भविष्यतः.}\;। \,मिथो भिन्नौ \textbf{दहा}दपि भिन्नौ भविष्यतः~। \textbf{दक}वर्गः \textbf{कझ}मिलितरेखावर्गयोर्योगतुल्यो भविष्यति~। \textbf{दववक}योर्मिलितत्वात्~। तस्मात् \textbf{दझं} तृतीया योगरेखा भविष्यति~॥ 
\vspace{2mm}

\begin{center}
\textbf{\large अथ षष्टितमं क्षेत्रम्~॥~६०~॥ }
\end{center}

 {\ab अङ्कसञ्ज्ञार्हरेखायामधिकरेखाया वर्गतुल्यं क्षेत्रं यद्भवति तदुत्पन्नो द्वितीयभुजश्चतुर्थी योगरेखा भवति~।} 

\newpage

\begin{wrapfigure}{r}{0.45\textwidth}
\vspace{-5mm}
\begin{flushright}
\includegraphics[scale=0.65]{Images/rg-61.png}
\end{flushright}
\vspace{-8mm}
\end{wrapfigure}

अस्य प्रकारः क्षेत्रं च पूर्ववत्~। परम् अत्र \textbf{दववकौ} भिन्नौ भविष्यतः~। \textbf{अजजब}-वर्गयोर्भिन्नत्वात्~। \textbf{हक}म् अङ्कसञ्ज्ञार्हमस्ति~। कुतः~। \textbf{अजजव}योर्वर्गयोगस्याङ्कसञ्ज्ञार्हत्वात्~। \textbf{लझं} मध्यमस्ति~। तस्मात् \textbf{दककझ}योर्वर्गावङ्क-सञ्ज्ञार्हौ भविष्यतः~। \textbf{दक}म् अङ्कसञ्ज्ञार्हम् अस्ति~। अस्य वर्गः \textbf{कझ}वर्ग\textbf{दक}भिन्नरेखा-वर्गयोर्योगतुल्योऽस्ति~। \textbf{दववक}योर्भिन्नत्वात्~। तस्मात् \textbf{दझं} चतुर्थी योगरेखा भविष्यति~॥ 
\vspace{2mm}

\begin{center}
\textbf{\large अथैकषष्टितमं क्षेत्रम्~॥~६१~॥ }
\end{center}

{\ab अङ्कसञ्ज्ञार्हरेखायामङ्कसञ्ज्ञार्हरेखामध्ययोगवर्गतुल्यं क्षेत्रं यदा भवति\renewcommand{\thefootnote}{१}\footnote{अङ्कसञ्ज्ञार्हरेखामध्ययोगवर्गतुल्यं क्षेत्रमङ्कसञ्ज्ञार्हरेखायां यदा भवति {\en J.}} तदा द्वितीयो भुजः पञ्चमी योगरेखा भविष्यति~। }\\

\begin{wrapfigure}{r}{0.45\textwidth}
\vspace{-8mm}
\begin{flushright}
\includegraphics[scale=0.63]{Images/rg-62.png}
\end{flushright}
\vspace{-8mm}
\end{wrapfigure}

 प्रकारः क्षेत्रं च पूर्ववत्~। परमत्र \textbf{दव-वकौ} भिन्नौ भविष्यतः~। \textbf{अजजब}वर्गयोर्भिन्न-त्वात्~। \textbf{हकं} मध्यो भविष्यति~। \textbf{अजजब}वर्ग-योर्मध्यत्वात्~। \textbf{लझ}म् अङ्कसञ्ज्ञार्हं भविष्यति~। तस्मात् \textbf{दककझ}योर्वर्गावङ्कसञ्ज्ञार्हौ भवि-ष्यतः~। \textbf{कझ}म् अङ्कसञ्ज्ञार्हमस्ति~। \textbf{दक}वर्गः \textbf{कझ}वर्गभिन्नरेखावर्गयोगतुल्योऽस्ति~। \textbf{दकव-क}योर्भिन्नत्वात्~॥ तस्मात् \textbf{दझं} पञ्चमी योग-रेखा भविष्यति~॥ 
\vspace{2mm}

\begin{center}
\textbf{\large अथ द्विषष्टितमं क्षेत्रम्~॥~६२~॥ }
\end{center}

{\ab अङ्कसञ्ज्ञार्हरेखायां द्वयोर्मध्ययोर्योगवर्गतुल्यं क्षेत्रं चेत् तदा द्वितीयोत्पन्नभुजः षष्ठी योगरेखा भविष्यति~। }

\newpage

\begin{wrapfigure}{r}{0.48\textwidth}
\vspace{-5mm}
\begin{flushright}
\includegraphics[scale=0.6]{Images/rg-63.png}
\end{flushright}
\vspace{-8mm}
\end{wrapfigure}

प्रकारः क्षेत्रं च पूर्वोक्तवद्बोध्यम्~। परमत्र \textbf{दववकौ} भिन्नौ भविष्यतः~। \textbf{हकं} मध्यं भविष्यति~। \textbf{लझं} \,मध्यं \,भवति~। \,\textbf{हका}त् भिन्नं च~। तस्मात् \textbf{दककझ}वर्गावङ्कसञ्ज्ञार्हौ भविष्यतः~। मिथो भिन्नौ भविष्यतः~। \textbf{दहा}-दपि भिन्नौ भविष्यतः~। \textbf{दक}वर्गः \textbf{कझ}वर्ग-भिन्नरेखावर्गयोगतुल्यो भविष्यति~। तस्मात् \textbf{दझं} षष्ठी योगरेखा भविष्यति~। इदमिष्टम्~॥ 
\vspace{2mm}

\begin{center}
\textbf{\large अथ त्रिषष्टितमं क्षेत्रम्~॥~६३~॥ }
\end{center}

{\ab योगरेखया या रेखा मिलितास्ति\renewcommand{\thefootnote}{१}\footnote{मिलिता भवति {\en J.}} सापि तादृश्येव योगरेखा भवति~। }\\

\begin{wrapfigure}{r}{0.48\textwidth}
\vspace{-6mm}
\begin{flushright}
\includegraphics[scale=0.6]{Images/rg-64.png}
\end{flushright}
\vspace{-8mm}
\end{wrapfigure}

 यथा \textbf{अब}योगरेखाया \textbf{ज}चिह्ने योज्यवि-भागद्वयं कल्पितम्~। तन्मिलिता \textbf{दह}रेखा कल्पिता~। पुनः \textbf{अबदह}निष्पत्तितुल्या \textbf{अज-दझ}निष्पत्तिः कल्प्या\renewcommand{\thefootnote}{२}\footnote{कल्पिता  {\en A.\,J.}}\;। तदा \textbf{जबझहौ} शेषौ तस्यामेव निष्पत्तौ स्तः~।  प्रत्येकं \textbf{अजजबौ दझझहा}भ्यां मिलितौ स्तः~। तथैवाङ्कसञ्ज्ञार्हौ स्तः~। अथवानयोर्वर्गौ मिलिताङ्कसञ्ज्ञार्हौ स्तः~। \textbf{अजजब}निष्पत्ति\textbf{र्दझझह}निष्पत्तितुल्यास्ति~। \textbf{अजजबौ} भिन्नौ स्तः तस्मात् \textbf{दझ-झहा}वपि भिन्नौ भविष्यतः~। यदि \textbf{अज}वर्गो \textbf{जब}वर्ग\textbf{अज}मिलितरेखावर्गयोगतुल्यो भवत्यथवा \textbf{जब}वर्ग\textbf{अज}भिन्नरेखावर्गयोगतुल्यो भवति तदा \textbf{दझ}वर्गो \textbf{झह}वर्ग\textbf{दझ}मिलितरेखा-वर्गयोगतुल्यो वा \textbf{झह}वर्ग\textbf{दझ}भिन्नरेखावर्गयोगतुल्यो भविष्यति~। तस्मात् \textbf{अबं} यादृशी योगरेखा भवति \textbf{दह}मपि तथैव भविष्यति~॥

\newpage
\begin{center}
\textbf{\large अथ चतुःषष्टितमं क्षेत्रम्~॥~६४~॥ }
\end{center}

{\ab मध्ययोगरेखाया \,या \,रेखा \,मिलिता \,भवति \,सा \,तादृश्येव \,मध्ययोगरेखा भवति~। }\\

\begin{wrapfigure}{r}{0.5\textwidth}
\vspace{-6mm}
\begin{flushright}
\includegraphics[scale=0.65]{Images/rg-65.png}
\end{flushright}
\vspace{-8mm}
\end{wrapfigure}

 यथा \;\textbf{अबं} \;प्रथममध्ययोगरेखा \;वा द्वितीयमध्ययोगरेखास्ति~। अस्या \textbf{ज}चिह्ने द्वौ विभागौ कल्प्यौ~। तन्मिलिता \textbf{दह}रेखा कल्पिता~। पुनः \textbf{अबदह}निष्पत्तितुल्या \textbf{अजदझ}निष्पत्तिः कल्प्या~। \textbf{जबझह}नि-ष्पत्तिः कल्प्या~। प्रत्येकं \textbf{अजजवे दझ-झहा}भ्यां मिलिते भविष्यतः~। तथैव मध्ये भविष्यतः~। \textbf{अजजबौ} भिन्नौ स्तः~। तस्मात् \textbf{दझझहा}वपि भिन्नौ भविष्यतः~। \textbf{अज}वर्ग\textbf{अजजब}घातयोर्निष्पत्तिः \textbf{अजजब}निष्पत्तिरूपा इयं \textbf{दझ}वर्ग\textbf{दझझह}घातनिष्पत्तितुल्य\textbf{दझझह}निष्पत्तितुल्यास्ति~। पुनः \textbf{अज}वर्ग\textbf{दझ}वर्ग-योर्निष्पत्तिः \textbf{अजजब}घात\textbf{दझझह}घातनिष्पत्तितुल्यास्ति~। द्वौ वर्गौ मिलितौ स्तः~। तस्मात् घातावपि मिलितौ भविष्यतः~। द्वौ वर्गावङ्कसञ्ज्ञार्हौ वा मध्यौ भवतः~। तदा घातावपि अङ्कसञ्ज्ञार्हौ वा मध्यौ भवतः~। \textbf{अब}योर्मध्ये यादृशी मध्यरेखा भवति \textbf{दह}मपि सैव\renewcommand{\thefootnote}{१}\footnote{तथैव {\en J.}} भविष्यति~। क्षेत्रं च पूर्वोक्तवद्बोध्यम्~॥
\vspace{2mm}

\begin{center}
प्रकारान्तरम्~॥
\end{center}

\begin{wrapfigure}{r}{0.47\textwidth}
\vspace{-10mm}
\begin{flushright}
\includegraphics[scale=0.6]{Images/rg-66.png}
\end{flushright}
\vspace{-8mm}
\end{wrapfigure}

 \textbf{अ}रेखा प्रथममध्ययोगरेखा वा द्वितीयमध्यरेखा \,\;कल्पिता~। \,\,तन्मिलिता \,\;\textbf{ब}रेखा कल्पिता~। \textbf{जद}रेखा अङ्कसञ्ज्ञार्हा कल्पिता~। अस्यां \textbf{दह}क्षेत्रम् \textbf{अ}वर्गतुल्यं कार्यम्~। \textbf{दझ}-क्षेत्रं \textbf{ब}वर्गतुल्यं च कार्यम्\renewcommand{\thefootnote}{२}\footnote{\textbf{ब}वर्गतुल्यं \textbf{दझ}क्षेत्रं कार्यम् {\en J.}}\;। तस्मात् \textbf{जहं} द्वितीययोगरेखा वा तृतीययोगरेखा भवि-ष्यति~। \,\textbf{जझ}म् \,एतन्मिलितं \,भविष्यति~। तस्मात् \textbf{जझ}मपि

\newpage
\noindent तादृश्येव भविष्यति~। \textbf{दझ}तुल्यो यस्य वर्गः स प्रथममध्ययोगो वा द्वितीयमध्ययोगो वा भविष्यति~। यथा \textbf{अ}म्~॥
\vspace{2mm}

\begin{center}
\textbf{\large अथ पञ्चषष्टितमं क्षेत्रम्~॥~६५~॥}
\end{center}

{\ab  अधिकरेखातो या मिलिता रेखा भवति साप्यधिकरेखा~। }\\

\begin{wrapfigure}{r}{0.5\textwidth}
\vspace{-10mm}
\begin{flushright}
\includegraphics[scale=0.65]{Images/rg-67.png}
\end{flushright}
\vspace{-8mm}
\end{wrapfigure}

यथा \;\textbf{अब} \;अधिकरेखाया \;\textbf{ज}चिह्ने विभागद्वयं कृतम्~। \textbf{दहं} तस्या मिलिता कल्पिता~। पुन\textbf{र्दह}रेखायां \textbf{झ}चिह्ने तस्याम् एव \,निष्पत्तौ \,विभागद्वयं \,कार्यम्~। \,तत्र \textbf{अजजब}निष्पत्ति\textbf{र्दझझह}निष्पत्तितुल्या भविष्यति~। \textbf{अजजब}योर्वर्गौ भिन्नौ स्तः~। तस्मात् \textbf{दझझह}योरपि वर्गौ भिन्नौ भविष्यतः~। \textbf{अजजब}योर्वर्गयोर्निष्पत्ति\textbf{र्दझझह}वर्गनिष्पत्तितु-ल्यास्ति~। \textbf{अजजब}वर्गयोगनिष्पत्ति\textbf{र्दझझह}वर्गयोगनिष्पत्तिरस्ति~। तस्मात् योगस्य योगेन तथास्ति यथैकस्य द्वितीयेन~। एको द्वितीयेन मिलितोऽस्ति~। योगो योगेन मिलितो भविष्यति~। \textbf{अजजब}वर्गयोगोऽङ्कसञ्ज्ञार्होऽस्ति~। तस्मात् \textbf{दझझह}वर्गयोगोऽप्यङ्कसञ्ज्ञार्हो भविष्यति~। पुनरपि \textbf{अजजब}द्विगुणघातो मध्योऽस्ति~। तस्मात् \textbf{दझझह}घातो द्विगुणस्तेन मिलितोऽपि\renewcommand{\thefootnote}{१}\footnote{तस्माद् द्विगुणो \textbf{दझझह}घातस्तेन मिलितोऽपि {\en \& c.\;J.}} मध्यो भविष्यति~॥

\begin{center}
पुनः प्रकारान्तरम्~॥
\end{center}

\begin{wrapfigure}{r}{0.33\textwidth}
\vspace{-10mm}
\begin{center}
\includegraphics[scale=0.5]{Images/rg-68.png}
\end{center}
\vspace{-8mm}
\end{wrapfigure}

अधिका \;रेखा \;\textbf{अः} \;कल्पिता\renewcommand{\thefootnote}{२}\footnote{\textbf{अ}म् अधिका रेखा कल्पिता {\en J.}}\;। \;\textbf{बं} \;मिलितरेखा कल्पिता~। अनयोर्वर्गौ \textbf{जदो}परि\renewcommand{\thefootnote}{३}\footnote{अङ्कसञ्ज्ञार्ह\textbf{जदो}परि {\en \& c.\,J.}} कार्यौ~। तस्मात् \textbf{अ}व-र्गात् द्वितीयो \textbf{जह}भुजोत्पन्नो भविष्यति~। इयं चतुर्थी योगरेखास्ति~। \textbf{जझं} च तन्मिलितं भविष्यति~। इदमपि तथैव भविष्यति~। तस्मात् या रेखा \textbf{दझ}वर्गतुल्या भवति साधिका भविष्यति~॥

\newpage
 \begin{center}
\textbf{\large अथ ६६ क्षेत्रम्~॥}
\end{center}

{\ab अङ्कसञ्ज्ञार्हमध्ययोगतुल्यो यस्या रेखाया वर्गो भवति तन्मिलितरेखाया अपि वर्गोऽङ्कसञ्ज्ञार्हमध्ययोगतुल्यो भवति~। }\\

\begin{wrapfigure}{r}{0.43\textwidth}
\vspace{-14mm}
\begin{center}
\includegraphics[scale=0.55]{Images/rg-69.png}
\end{center}
\vspace{-8mm}
\end{wrapfigure}

तस्य प्रकारः क्षेत्रं च पूर्ववत् बोध्यम्~॥\\
\vspace{2mm}

\begin{center}
\textbf{\large अथ ६७ क्षेत्रम्~॥}
\end{center}

{\ab द्वयोर्मध्ययोः योगतुल्यो यस्या रेखाया वर्गोऽस्ति तस्या मिलितरेखाया वर्गोऽपि मध्ययोगतुल्यो भवति\renewcommand{\thefootnote}{१}\footnote{भविष्यति {\en A.}}\;। }\\

\begin{wrapfigure}{r}{0.43\textwidth}
\vspace{-14mm}
\begin{center}
\includegraphics[scale=0.45]{Images/rg-70.png}
\end{center}
\vspace{-8mm}
\end{wrapfigure}

अस्य प्रकारः क्षेत्रं च पूर्वोक्तवत्\renewcommand{\thefootnote}{२}\footnote{पूर्ववत् {\en J.}} ज्ञेयम्~। इदमेवेष्टम्~॥\\

\begin{center}
\textbf{\large अथ ६८ क्षेत्रम्~॥}
\end{center}

{\ab यस्या रेखाया वर्गोऽङ्कसञ्ज्ञार्हक्षेत्रमध्यक्षेत्रयोगसमो भवति सा रेखा योगरेखा वा प्रथममध्ययोगरेखाथवाधिकरेखा भविष्यति वा अस्या वर्गोऽङ्कसञ्ज्ञार्हमध्ययोगतुल्यो
भविष्यति~।}\\ 

\begin{wrapfigure}{r}{0.47\textwidth}
\vspace{-10mm}
\begin{center}
\includegraphics[scale=0.45]{Images/rg-71.png}
\end{center}
\vspace{-8mm}
\end{wrapfigure}

 यथा \textbf{अब}म् अङ्कसञ्ज्ञार्हक्षेत्रं \textbf{जदं} मध्य-क्षेत्रं कल्पितम्~। पुन\textbf{र्हझ}म् अङ्कसञ्ज्ञार्हरेखा कल्पिता~। अस्यां रेखायां \textbf{हव}क्षेत्रं \textbf{वक}क्षेत्रं तत्क्षेत्रद्वयतुल्यं कार्यम्~। तस्मादुत्पन्नो \textbf{हत}-भुजोऽङ्कसञ्ज्ञार्हो भविष्यति~। \textbf{तकं} केवलवर्गोऽङ्कसञ्ज्ञार्हो भविष्यति~। यदि \textbf{हत}रेखा \textbf{तका}दधिका भवति पुन\textbf{र्हत}वर्गः \textbf{तक}वर्ग\textbf{हत}-मिलितरेखावर्गयोगतुल्यः स्यात् तदा \textbf{हक}-रेखा प्रथमयोगरेखा\renewcommand{\thefootnote}{३}\footnote{प्रथमसंयोगरेखा {\en J.}} भविष्यति~। यस्या रेखाया वर्गो \textbf{झक}क्षेत्रतुल्योऽस्ति सा योगरेखा 

\newpage
\noindent भविष्यति~। यदि \textbf{हत}वर्गः \textbf{तक}वर्ग\textbf{हत}भिन्नरेखावर्गयोगतुल्यः\renewcommand{\thefootnote}{१}\footnote{समः} स्यात् तदा \textbf{हक}रेखा चतुर्थयोगरेखा भविष्यति~। यस्या वर्गो \textbf{झक}क्षेत्र\renewcommand{\thefootnote}{२}\footnote{एतत्क्षेत्र {\en J.}}तुल्यः स्यात् साधिकरेखा भविष्यति~। \\

 यदि\renewcommand{\thefootnote}{३}\footnote{पुनर्यदि {\en J.}} \textbf{तक}रेखा \textbf{हत}रेखाया अधिका स्यात्\renewcommand{\thefootnote}{४}\footnote{रेखातोऽधिका भवति \textbf{तक}वर्गश्च {\en J.}} पुन\textbf{स्तक}वर्गो \textbf{हत}वर्ग\textbf{तक}मिलितरेखा\renewcommand{\thefootnote}{५}\footnote{\textbf{हत}रेखा\textbf{तक}रेखामिलितरेखा {\en J.}}व-र्गयोगतुल्यः स्यात् तदा \textbf{हकं} द्वितीय\renewcommand{\thefootnote}{६}\footnote{तुल्यो भवति सा द्वितीय {\en J.}}योगरेखा भविष्यति~। यस्या रेखाया\renewcommand{\thefootnote}{७}\footnote{{\en J. omits} रेखायाः.} वर्गो \textbf{झक}-क्षेत्र\renewcommand{\thefootnote}{८}\footnote{एतत्क्षेत्र {\en J.}}तुल्यः स्यात्\renewcommand{\thefootnote}{९}\footnote{भवति {\en J.}} सा प्रथममध्ययोगरेखा भविष्यति~। पुन\renewcommand{\thefootnote}{१०}\footnote{{\en J. omits} पुनर्.}र्यदि \textbf{तक}वर्गो \textbf{हत}वर्ग\textbf{तक}-भिन्नरेखा\renewcommand{\thefootnote}{११}\footnote{{\en J. Omits} रेखा.}वर्गयोगसमः स्यात् तदा\renewcommand{\thefootnote}{१२}\footnote{तुल्यो भवति तदा~{\en J.}} \textbf{हक}रेखा पञ्चमी योगरेखा भविष्यति~। यस्या वर्गो \textbf{झक}क्षेत्र\renewcommand{\thefootnote}{१३}\footnote{एतत्क्षेत्र {\en J.}}समः स्यात् तस्या वर्गोऽङ्कसञ्ज्ञार्हमध्ययोगसमः स्यात्~। इदमेवेष्टम्~॥ 
\vspace{2mm}

\begin{center}
\textbf{\large अथ ६९ क्षेत्रम्~॥}
\end{center}

{\ab यस्या रेखाया वर्गो मिथो भिन्नयोर्मध्यक्षेत्रयोर्योगेन तुल्यो भवति तदा सा रेखा द्वितीयमध्ययोगरेखा भविष्यत्यथवा तस्या वर्गो मध्यद्वययोगतुल्यो भविष्यति~।}\\

\begin{wrapfigure}{r}{0.4\textwidth}
\vspace{-12mm}
\begin{center}
\includegraphics[scale=0.5]{Images/rg-72.png}
\end{center}
\vspace{-8mm}
\end{wrapfigure}

द्वे मध्यक्षेत्रे \textbf{अबजदे} कल्प्ये~। \textbf{झह}म् अङ्कसञ्ज्ञार्हरेखा कल्पिता~। अस्या उपरि कल्पितक्षेत्रद्वयतुल्यं \,\textbf{हव}क्षेत्रं \,\textbf{वक}क्षेत्रं \,च \,कार्यम्~। \,तस्मादुत्पन्नौ \textbf{हततक}भुजौ मिथो भिन्नौ भविष्यतः~। \textbf{हझ}योरपि भिन्नौ भविष्यतः~। अनयोर्वर्गावङ्कसञ्ज्ञार्हौ भवि-ष्यतः~। अनयोरधिकरेखावर्गो लघुरेखावर्गस्याधिकरेखामिलितरेखाया वा भिन्नरेखाया वर्गस्य योगेन तुल्यो भविष्यति~।

\newpage
\noindent \textbf{हकं} तृतीययोगरेखा वा षष्ठी योगरेखा भविष्यति~। तद्रेखावर्ग एतत् क्षेत्रतुल्य उपरितनोक्तरेखयोरन्यतराया वर्गो भविष्यति~। क्षेत्रं च पूर्ववद्बोध्यम्~। इदमेवेष्टम्~॥
\vspace{2mm}

\begin{center}
\textbf{\large  अथ ७० क्षेत्रम्~॥}
\end{center}

{\ab ये द्वे रेखे भिन्ने भवतस्तयोः केवलवर्गावङ्कसञ्ज्ञार्हौ भवतस्तत्रैकतुल्यं यदि द्वितीयात्पृथक्क्रियते तदा शेषं करणीरूपं भवति~। इयमेवान्तररेखोच्यते~।}\\

\begin{wrapfigure}{r}{0.4\textwidth}
\vspace{-7mm}
\begin{flushright}
\includegraphics[scale=0.45]{Images/rg-73.png}
\end{flushright}
\vspace{-8mm}
\end{wrapfigure}

यथा \textbf{अब}म् \textbf{अजा}त् पृथक् कृतम्~। शेषं \textbf{बजं} करणीरूपमवशिष्टम्~। कुतः\renewcommand{\thefootnote}{१}\footnote{\textbf{अबअज}योर्भिन्नत्वात् {\en A.\,J.}}\;। एते भिन्ने स्तः~। अनयोर्वर्गावङ्कसञ्ज्ञार्हौ तयोर्योगः \textbf{अबअज}-घातद्विगुणमध्यक्षेत्राद्भिन्नोऽस्ति~। तस्मात् स एव वर्गः शेषात् वर्गादपि भिन्नो भविष्यति~। तस्मात् \textbf{बज}वर्गः करणीरूपो भविष्यति~। एवं \textbf{बज}मपि करणीरूपं भविष्यति~॥
\vspace{2mm}

\begin{center}
\textbf{\large  अथ ७१ क्षेत्रम्~॥}
\end{center}

{\ab ययोः मध्यरेखयोः केवलवर्गौ मिलितौ भवतोऽङ्कसञ्ज्ञार्हौ क्षेत्रभुजावनयो रेखयोरन्तरं करणीरूपं भविष्यति~। इदं प्रथममध्यान्तराभिधानम्~॥ }\\

\begin{wrapfigure}{r}{0.4\textwidth}
\vspace{-7mm}
\begin{flushright}
\includegraphics[scale=0.45]{Images/rg-74.png}
\end{flushright}
\vspace{-8mm}
\end{wrapfigure}

 यथा \textbf{अब}म् \textbf{अजा}त् पृथक् कृतम् तदा शेषं \textbf{बजं} करणीरूपमवशिष्टम्~। कुतः~। अनयोर्भिन्नत्वात्~। अनयोर्द्विगुणघातोऽङ्कसञ्ज्ञार्हरूपोऽनयोर्वर्गयोगात् मध्यरूपात् भिन्नो भविष्यति~। तस्मात् द्विगुणघातः शेष\textbf{बज}वर्गादपि भिन्नो भविष्यति~। तस्मात् \textbf{बजं} करणीरूपं भविष्यति~॥ 
\vspace{2mm}

\begin{center}
\textbf{\large अथ ७२ क्षेत्रम्~॥}
\end{center}

{\ab केवलवर्गमिलिते द्वे मध्यरेखे मध्यक्षेत्रस्य भुजौ भवतस्त-}

\newpage
\noindent {\ab दानयोरन्तरं करणीरूपं भविष्यति~। अस्याभिधानं द्वितीयमध्यान्तररेखेति\renewcommand{\thefootnote}{१}\footnote{इयं द्वितीयमध्यान्तररेखोच्यते {\en A.,\,J.}}\;। }\\

\begin{wrapfigure}{r}{0.4\textwidth}
\vspace{-7mm}
\begin{flushright}
\includegraphics[scale=0.6]{Images/rg-75.png}
\end{flushright}
\vspace{-8mm}
\end{wrapfigure}

 यथा \textbf{अब}म् \textbf{अजा}त् पृथक् कृतं शेषं \textbf{बजं} करणीरूपं स्यात्~। पुनः \textbf{दह}म् अङ्कसञ्ज्ञार्हरेखा कल्पिता~। अस्या उपरि \textbf{अबअज}वर्गयोगतुल्यं \textbf{हत}क्षेत्रं कार्यम्~। \textbf{अबअज}घातद्विगुणतुल्यं \textbf{हव}क्षेत्रं कार्यम्~। शेषं \textbf{झत}क्षेत्रं \textbf{बज}वर्गतुल्यमवशिष्यते~। कुतः\renewcommand{\thefootnote}{२}\footnote{{\en J. Omits} कुतः.}\;। \textbf{अबअज}योर्भिन्नत्वात्~। \textbf{हतहवौ} मध्यक्षेत्रे भिन्ने भविष्यतः~। उत्पन्नौ \textbf{दतदव}भुजौ मिथो भिन्नौ भविष्यतः~। वर्गावङ्कसञ्ज्ञार्हौ भविष्यतः~। तस्मात् \textbf{वत}म् अन्तररेखा भविष्यति~। \textbf{झतं} करणीरूपमस्ति~। तस्मात् \textbf{बज}मपि करणीरूपं भविष्यति~॥ 
\vspace{2mm}

\begin{center}
\textbf{\large अथ ७३ क्षेत्रम्~॥}
\end{center}

{\ab तयो रेखयोरन्तरं करणीरूपं भवति\renewcommand{\thefootnote}{३}\footnote{ययो रेखयोर्वर्गौ भिन्नौस्तस्तयो रेखयोरन्तरं करणीरूपं भवति {\en A.,\,J.}} ययोर्भिन्नरेखयोर्वर्गौ भिन्नौ स्तो वर्गयोगोऽङ्कसञ्ज्ञार्हो भवति द्विगुणघातश्च मध्यक्षेत्रतुल्यो भवति~। इयं न्यूनरेखोच्यते~। }\\

 यथा \textbf{अब}म् \textbf{अजा}त् पृथक् कृतम्~। शेषं \textbf{बजं} करणीरूपमवशिष्टम्~। अस्य विचारः क्षेत्रं च पूर्ववत् बोध्यम्~॥ 
\vspace{2mm}

\begin{center}
\textbf{\large अथ ७४ क्षेत्रम्~॥}
\end{center}

{\ab \renewcommand{\thefootnote}{४}\footnote{{\en A. and J. have} ययोः {\en in the beginning and} तयोः {\en for} अनयोः.}द्वयो रेखयोर्वर्गौ भिन्नौ स्तो वर्गयोगो मध्यक्षेत्रतुल्यो भवति द्विगुणघातः चाङ्कसञ्ज्ञार्हो भवति~। अनयोरन्तरं करणीरूपं भवति~। इयमङ्कसञ्ज्ञार्हयोगमध्यरेखोच्यते~। }\\

विचारः क्षेत्रं च पूर्ववत्~॥ 

\newpage
\begin{center}
\textbf{\large अथ ७५ क्षेत्रम्~॥}
\end{center}

{\ab द्वयोर्भिन्नवर्गरेखयोर्वर्गयोगो मध्यक्षेत्रतुल्यो भवति द्विगुणघातः प्रथममध्यक्षेत्राद्भिन्नं मध्यक्षेत्रं भवति~। अनयो रेखयोरन्तरं करणीरूपं भवति~। इयं मध्ययोगजमध्यरेखोच्यते~। }\\

विचारः क्षेत्रं च पूर्ववत्~। इदमेवेष्टम्\renewcommand{\thefootnote}{१}\footnote{{\en Omitted in A. and J.}}\;॥ 
\vspace{2mm}

\begin{center}
\textbf{\large अथ ७६ क्षेत्रम्~॥}
\end{center}

 {\ab अन्तररेखायामेकैव रेखा लगति या तस्याः पूर्वस्वरूपं करोति~। }\\

\begin{wrapfigure}{r}{0.4\textwidth}
\vspace{-8mm}
\begin{flushright}
\includegraphics[scale=0.6]{Images/rg-76.png}
\end{flushright}
\vspace{-8mm}
\end{wrapfigure}

 यद्येवं न भवति तदा \textbf{अब}रेखायां \textbf{बज}रेखा\textbf{बद}-रेखे लग्ने~। ताभ्यां तस्याः पूर्वस्वरूपमेव कृतम् इति कल्पितम्~। \textbf{अजजब}योर्वर्गौ\renewcommand{\thefootnote}{२}\footnote{वर्गयोगः {\en A.,\,J.}} \textbf{अजजब}घात-द्विगुण\textbf{अब}वर्गयोगेन तुल्यौ स्तः\renewcommand{\thefootnote}{३}\footnote{तुल्योऽस्ति {\en A.,J.}}\;। \textbf{अददब}वर्गयोगोऽपि \textbf{अददब}घातद्विगुण\textbf{अब}वर्गयोगेन तुल्योऽस्ति\renewcommand{\thefootnote}{४}\footnote{{\en D. and B have} वर्गावपि......तुल्यौ स्तः.}\;। \textbf{अजजब}वर्ग\textbf{अददब}वर्गयोरन्तरं च द्वयोरङ्कसञ्ज्ञार्हयोरन्तररूपम्~। \textbf{अज-जब}घातद्विगुण\textbf{अददब}घातद्विगुणयोरन्तरं द्वयोर्मध्ययोरन्तररूपं द्वयं\renewcommand{\thefootnote}{५}\footnote{{\en J. omits} द्वयम्.} समानं भविष्यति~। इदमशुद्धम्~। अस्मदिष्टं समीचीनम्~॥ 
\vspace{2mm}
 
\begin{center}
\textbf{\large  अथ ७७ क्षेत्रम्~॥}
\end{center}

 {\ab प्रथममध्यान्तररेखयैकैव रेखा मिलति या\renewcommand{\thefootnote}{६}\footnote{सा {\en J.}} तस्याः प्रथमस्वरूपं करोति~। }\\

\begin{wrapfigure}{r}{0.4\textwidth}
\vspace{-10mm}
\begin{flushright}
\includegraphics[scale=0.5]{Images/rg-77.png}
\end{flushright}
\vspace{-8mm}
\end{wrapfigure}

 यद्येवं न भवति तदा \textbf{अब}रेखायां \textbf{बजबद}-रेखे संलग्ने~। \textbf{अब}स्य प्रथमस्वरूपं कृतम्~। तदा \textbf{अजजब}वर्गयोः \textbf{अददब}वर्गयोश्चान्तरं द्वयोर्मध्ययोः अन्तररू-

\end{document}
