%Default Sanskrit,  No auto transition, \en font for english, so italics displayed.
\documentclass[11pt, openany]{book}
\usepackage[text={4.65in,7.45in}, centering, includefoot]{geometry}
\usepackage[table, x11names]{xcolor}
\usepackage{fontspec,realscripts}
\usepackage{polyglossia}
\setdefaultlanguage{sanskrit}
\setotherlanguage{english}
\setmainfont[Scale=1]{Shobhika}
\newfontfamily\s[Script=Devanagari, Scale=0.9]{Shobhika}
\newfontfamily\regular{Linux Libertine O}
\newfontfamily\en[Language=English, Script=Latin]{Linux Libertine O}
\newfontfamily\ab[Script=Devanagari, Color=purple]{Shobhika-Bold}
\newfontfamily\qt[Script=Devanagari, Scale=1, Color=violet]{Shobhika-Regular}
\newcommand{\devanagarinumeral}[1]{%
	\devanagaridigits{\number \csname c@#1\endcsname}} % for devanagari page numbers
%\usepackage[Devanagari, Latin]{ucharclasses}
%\setTransitionTo{Devanagari}{\s}
%\setTransitionFrom{Devanagari}{\regular}
%\XeTeXgenerateactualtext=1 % for searchable pdf
\usepackage{enumerate}
\usepackage{enumitem}
\pagestyle{plain}
\usepackage{fancyhdr}
\pagestyle{fancy}
\renewcommand{\headrulewidth}{0pt}
\usepackage{afterpage}
\usepackage{multirow}
\usepackage{multicol}
\usepackage{vwcol}
\usepackage{microtype}
 \usepackage{amsmath}
\usepackage{graphicx}
\usepackage{longtable}
\usepackage{footnote}
\usepackage{perpage}
\MakePerPage{footnote}
\usepackage[para]{footmisc}
%\usepackage{dblfnote}
\usepackage{xspace}
\usepackage{array}
\usepackage{emptypage}
\usepackage{hyperref}% Package for hyperlinks
\hypersetup{colorlinks,
citecolor=black,
filecolor=black,
linkcolor=blue,
urlcolor=black}
\begin{document}
\afterpage{\fancyhead[CE] {रेखागणितम्}}
\afterpage{\fancyhead[CO] {दशमोध्यायः}}
\afterpage{\fancyhead[LE,RO]{\thepage}}
\cfoot{}
\newpage
%%%%%%%%%%%%%%%%%%%%%%%%%%%%%%%%%%%%%%%%%%%%%%%%%%%%%%%%%%%%%%
\newpage
\renewcommand{\thepage}{\devanagarinumeral{page}}
\setcounter{page}{109}
\noindent पम् अजजबघातद्विगुणः अददबघातद्विगुणः अनयोरन्तरस्य द्वयोरङ्कसंज्ञार्हयोरन्तररूपस्य समानमस्तीत्यशुद्धम् । इष्टं समीचीनम् । क्षेत्रं च पूर्ववत् ॥ \\
\begin{center}
\textbf{\large अथ ७८ क्षेत्रम् ॥ ७८ ॥ }
\end{center}
\vspace{2mm}

{\ab द्वितीयमध्यान्तररेखामेकैव रेखा मिलिष्यति याऽस्याः
पूर्वस्वरूपं करिष्यति~। }\\

 यद्येवं न भवति तदा कल्पितम् अबरेखया बजबदरेखे मिलिते
अस्याः पूर्वस्वरूपं कुरुतः । पुनर्हझरेखा अङ्कसंज्ञार्हा कल्पिता ।
\begin{vwcol}[widths={0.65,0.35}, sep=.8cm, rule=0pt]
 अस्यां अजजवयोर्वर्गयोगो झकक्षेत्रं कार्यम्~। अबवर्गतुल्यं झवक्षेत्रं च
कार्यम् । शेषं तकक्षेत्रम् अजजबघातद्विगुणतुल्यमवशिष्यते~। द्वयोर्वर्गयोगो मध्यक्षेत्रतुल्योऽस्ति । द्विगुणघातश्च प्रथममध्यक्षेत्राद्भिन्नः मध्यक्षेत्रतुल्योऽस्ति । तदा हककवरेखे मिथो
भिन्ने भविष्यतः । अनयोर्वर्गा\\
\noindent \includegraphics[scale=0.65]{Images/rg-78.png}
\end{vwcol}
\vspace{-3mm}

\noindent वङ्कसंज्ञार्हौ भविष्यतः । तस्मात्
हवम् अन्तररेखा भविष्यति । \\

पुनरपि हझरेखोपरि अददबवर्गयोगझलक्षेत्रं कार्यम् । तस्मात्
तलक्षेत्रम् अददबघातद्विगुणतुल्यं भविष्यति ।
हलरेखालवरेखावर्गौ केवलमङ्कसंज्ञार्हौ भविष्यतः~। 
वङ्कसंज्ञार्हौ भविष्यतः । तस्मात् हवम् अन्तररेखा 
पुनरपि हझरेखोपरि अददबवर्गयोगझलक्षेत्रं कार्यम् । तस्मात्
तलक्षेत्रम् अददबघातद्विगुणतुल्यं भविष्यति ।
हलरेखालवरेखावर्गौभविष्यति । \\

केवलमङ्कसंज्ञार्हौ भविष्यतः । हवमन्तररेखास्ति । तस्मात् हवरेखया
वकरेखावलरेखे संलग्ने । आभ्यामन्तररेखा प्रथमरूपा कृतेत्यशुद्धम् ।
अस्मदिष्टं समीचीनम्~॥\\
\begin{center}
\textbf{\large अथ ७९ क्षेत्रम् ॥ ७९ ॥}
\end{center}
\vspace{2mm}

{\ab न्यूनरेखायामप्येकैव रेखा लगति या तस्याः पूर्वस्वरूपं
करोति । }\\

यद्येवं न स्यात् अबरेखायां बजबदरेखे संलग्ने । पूर्वस्वरूपं
कृतम्। विचारः क्षेत्रं च पूर्ववत् ॥ 


\newpage
\begin{center}
\textbf{\large अथ ८० क्षेत्रम् ॥ ८० ॥}
\end{center} 
\vspace{2mm} 


{\ab अङ्कसंज्ञार्हयुक्तमध्यरेखायामेकैव रेखा लगति याऽस्याः
पूर्वस्वरूपं करोति । }\\

 यद्येवं न स्यात् अबरेखायां बजरेखाबदरेखे संलग्ने । आभ्यां
पूर्वस्वरूपं च कृतम्~। अस्य विचारः क्षेत्रं च पूर्ववत् ज्ञेयम् ॥ \\

\begin{center}
\textbf{\large अथ ८१ क्षेत्रम् ॥ ८१ ॥}
\end{center}
\vspace{2mm}

{\ab मध्ययोगमध्यरेखायामप्येकैव रेखा लगति याऽस्याः पूर्वस्वरूपं करोति । }\\

 अबरेखायां बजबदरेखे संलग्ने पूर्वस्वरूपं कुरुतः । विचारः
क्षेत्रं च पूर्ववत् ॥ \\
\begin{center}
\textbf{ ॥ अथ शेषक्षेत्राणां परिभाषोच्यते ॥ }
\end{center}
\vspace{5mm}

 यद्यन्तररेखयैका रेखा मिलति पूर्वस्वरूपं च करोति तत्र संपूर्णरेखावर्गो लग्नरेखावर्गसंपूर्णरेखामिलितान्यरेखावर्गयोगेन तुल्यो भवति ।
संपूर्णरेखाङ्कसंज्ञार्हरेखा चेद्भवति तदान्तररेखा प्रथमान्तररेखा भवति ।\\

\renewcommand{\thefootnote}{१}\footnote{पूर्वोक्तलक्षणाक्रान्ता यदि लग्नरेखा {\en \& c. J., A. }}यदि लग्नरेखाङ्कसंज्ञार्हा भवति तदेयं द्वितीयान्तररेखा भविष्यति ।\\

यद्यनयोः काप्यङ्कसंज्ञार्हा न भवति तदेयं तृतीयान्तररेखा
भविष्यति । \\

 पुनः संपूर्णरेखावर्गो लग्नरेखावर्गसंपूर्णरेखाभिन्नान्यरेखावर्गयोगेन
तुल्यो \\ \noindent भवति~। \\

 संपूर्णरेखा चाङ्कसंज्ञार्हा भवति तदेयं चतुर्थ्यन्तररेखा स्यात् ।\\
 
यदि लग्नरेखाङ्कसंज्ञार्हा भवति तदा पञ्चम्यन्तररेखा भवति ।\\

यदि काप्यङ्कसंज्ञार्हा न भवति तदा षष्ठ्यन्तररेखा भवति । \\
\begin{center}
\textbf{ ॥ इति परिभाषा ॥ }
\end{center}

\newpage
\begin{center}
\textbf{\large अथ ८२ क्षेत्रम् ॥ ८२ ॥}
\end{center}
\vspace{2mm}

{\ab प्रथमान्तररेखोत्पादनमिष्टम्\renewcommand{\thefootnote}{१}\footnote{°मिष्टमस्ति {\en J.}} । }\\

 प्रथममिष्टरेखाङ्कसंज्ञार्हा \renewcommand{\thefootnote}{२}\footnote{अं {\en A,J. }}अः कल्पिता । तन्मिलिता बजरेखा कल्पिता । 
दहदझौ वर्गराश्यङ्कौ तथा कल्प्यौ यथाऽनयोरन्तरं झहं वर्गो न भवति ।
पुनर्बजवर्गजववर्गयोर्निष्पत्तिर्दहझहनिष्पत्तितुल्या कल्पिता । तस्मात् बवं प्रथमान्तररेखा भविष्यति । \renewcommand{\thefootnote}{३}\footnote{यतो {\en J. }}कुतः । बजरेखाङ्कसंज्ञार्हास्ति । जवरेखा बजरेखया केवलवर्गमिलितास्ति~।  अस्या वर्गोऽङ्कसंज्ञा-
\begin{vwcol}[widths={0.65,0.35}, sep=.8cm, rule=0pt]
र्होऽस्ति । इयं जवरेखा बजरेखातो भिन्नास्ति ।
पुनर्बजवर्गस्य जववर्गेणान्तरं तवर्गः कल्पितः । तस्मात् बजवर्गस्य
तवर्गेण निष्पत्तिर्दहदझवर्गराश्योर्निष्पत्तितुल्यास्ति~। तस्मात् तं बजेन मिलितं
भविष्यति~। बजवर्गो जववर्गतवर्गयोगतुल्यो भविष्यति ॥ \\
\vspace{5mm}

\noindent \includegraphics[scale=0.65]{Images/rg-79.png}
\end{vwcol}
\vspace{2mm}

\begin{center}
\textbf{\large अथ ८३ क्षेत्रम् ॥ ८३ ॥}
\end{center}
\vspace{2mm}

{\ab तत्र द्वितीयान्तररेखोत्पादनमिष्टम् । }\\

तत्राङ्कसंज्ञार्हरेखा अं \renewcommand{\thefootnote}{४}\footnote{कल्पिता {\en A.}}कल्प्या । जवरेखैतन्मिलिता कल्पिता ।
द्वावङ्कौ 
\begin{vwcol}[widths={0.6,0.4}, sep=.8cm, rule=0pt]
पूर्ववत् कल्प्यौ । पुनर्जववर्गबजवर्गयोर्निष्पत्तिर्झहदहनिष्पत्तितुल्या कल्पिता । वबं द्वितीयान्तररेखा भविष्यति । कुतः । जबस्याङ्कसंज्ञार्हत्वात् । जवं केवलवर्गाङ्कसंज्ञार्हरेखास्ति~। जबवर्गो जववर्गतवर्गयोगतुल्योऽस्ति । क्षेत्रं च पूर्ववत् ॥ \\
\vspace{5mm}

\noindent \includegraphics[scale=0.7]{Images/rg-80.png}
\end{vwcol}
\vspace{2mm}

\begin{center}
\textbf{\large अथ ८४ क्षेत्रम् ॥ ८४ ॥}
\end{center}
\vspace{2mm}

{\ab तत्र तृतीयान्तररेखोत्पादनमिष्टम् । }

\newpage
प्रथमाङ्कसंज्ञार्हरेखा अं कल्पिता । द्वौ वर्गराश्यङ्कौ झवझतौ
कल्पितौ यथा तवम् अन्तरं वर्गो न भवति ।
हम्  अन्योऽङ्को-
\begin{vwcol}[widths={0.65,0.35}, sep=.8cm, rule=0pt]
ऽवर्गराशिस्तथा कल्प्यो यथा तस्य निष्पत्तिर्वर्गद्वयनिष्पत्तितुल्या न भवति । पुनर् अवर्गजबवर्गयोर्निष्पत्तिर्हझवयोर्निष्पत्तितुल्या कल्प्या~। पुनर्बजवर्गदजवर्गयोर्निष्पत्तिर्झवतवनिष्पत्तितुल्या कल्प्या । तस्मात् बदं तृतीयान्तररेखा भविष्यति । कुतः । बजजदौ केवलवर्गाङ्कसं\\
\noindent \includegraphics[scale=0.7]{Images/rg-81.png}
\end{vwcol}
\vspace{-4mm}

\noindent ज्ञार्हौ  स्तः
आद्भिन्नौ स्तः । बजवर्गो जदवर्गबजमिलितकवर्गयोगतुल्योऽस्ति ।
यतोऽनयोर्वर्गौ झवझतनिष्पत्तौ स्तः । 

\begin{center}
\textbf{\large अथ ८५ क्षेत्रम् ॥ ८५ ॥ }
\end{center}

{\ab  तत्र चतुर्थ्यन्तररेखोत्पादनमिष्टम् । }\\

 अत्रोपरितनप्रकारवत् । परं द्वौ वर्गराशी दझझहौ तथा कल्प्यौ
यथेतयोर्योगो दहं वर्गराशिर्न भवति । बजवर्गो जववर्गबजभिन्नतवर्गतुल्यो
भविष्यति । कुतः~। \renewcommand{\thefootnote}{१}\footnote{{\en J. inserts} यतः. }बजवर्गतवर्गयोर्निष्पत्तिर्दहदझयोर्निष्पत्तितुल्यास्ति । क्षेत्रं च \renewcommand{\thefootnote}{२}\footnote{{\en J. omits} च, }पूर्ववत् ॥ 
\begin{center}
\includegraphics[scale=0.5]{Images/rg-82.png}
\end{center}

\begin{center}
\textbf{\large अथ ८६ क्षेत्रम् ॥ ८६ ॥}
\end{center}

{\ab तत्र पञ्चम्यन्तररेखोत्पादनमिष्टम् । }\\

\renewcommand{\thefootnote}{३}\footnote{{\en A. and J. have}
द्वितीयान्तररेखोत्पादनप्रकारः.}प्रकारः क्षेत्रं च पूर्वोक्तवत् । परं तु
दझझहौ वर्गराशी तथा कल्प्यौ यथैतयोर्योगो दहं वर्गो न भवति । क्षेत्रं
पूर्ववत् ॥ 
\begin{center}
\noindent \includegraphics[scale=1]{Images/rg-83a.png}
\end{center}

\newpage
\begin{center}
\textbf{\large अथ ८७ क्षेत्रम् ॥ ८७ ॥}
\end{center}

{\ab तत्र षष्ठ्यन्तररेखोत्पादनमिष्टम् ॥ }\\

\renewcommand{\thefootnote}{१}\footnote{{\en A. and J. have} तृतीयान्तररेखोत्पादनप्रकारः.}प्रकारः पूर्ववत् । परं दहझहौ\renewcommand{\thefootnote}{२}\footnote{दहदझौ {\en A,}} वर्गराश्यङ्कौ तथा कल्प्यौ यथैतयोर्योगो वर्गराशिर्न भवति । क्षेत्रं च पूर्ववद्बोध्यम् ॥\\ 
\begin{center}
\includegraphics[scale=0.8]{Images/rg-83.png}
\end{center}

\begin{center}
\textbf{\large अथ ८८ क्षेत्रम् ॥ ८८ ॥}
\end{center}
\vspace{1mm}

{\ab क्षेत्रस्यैको भुजोऽङ्कसंज्ञार्हो भवति द्वितीयो भुजः प्रथमान्तररेखा भवति । यस्या रेखाया वर्ग एतत्क्षेत्रतुल्यो भवति सान्तररेखा भविष्यति । }\\

 यथा बझं क्षेत्रं कल्पितम् । अङ्कसंज्ञार्हरेखा अबं कल्पिता ।
प्रथमान्तररेखा अझम् । अझरेखया झजरेखा तथा योज्या यथा प्रथमरूपा भवति।
पुनर्बजक्षेत्रं संपूर्णं कार्यम् । पुनर्झजरेखा दचिह्नेऽर्द्धिता कार्या ।
पुनर् अजरेखाखण्डोपरि जदवर्गतुल्यो झजवर्गस्य चतुर्थांशस्तथा
कार्यो यथा शेषखण्डक्षेत्रं वर्गतुल्यमवशिष्यते ।
तस्मात् \\
\begin{vwcol}[widths={0.5,0.5}, sep=.8cm, rule=0pt]
\hspace{0.5in}\includegraphics[scale=0.6]{Images/rg-84.png}\\
 \includegraphics[scale=0.5]{Images/rg-84-1.png}\
\end{vwcol}
\vspace{5mm}

\noindent अजरेखाया हचिह्ने \renewcommand{\thefootnote}{३}\footnote{द्वौ विभागौ भविष्यतः {\en A., J.}\\
भा० १५
}विभागो भविष्यति । पुनर् अहरेखादजरेखानिष्पत्तिर्दजरेखाजहरेखानिष्पत्तितुल्या भविष्यति । जहं च
खण्डद्वयमध्ये लघुखण्डमस्ति । तस्मात् जहं
जदाल्लघु भविष्यति । जदं च अहाल्लघु
भविष्यति । पुनर्हचिह्नदचिह्नाभ्यां हकरेखादतरेखे अबरेखासमानान्तरे कार्ये । पुनः
समं समकोणसमचतुर्भुजं बहक्षेत्रतुल्यं कार्यम्~। अस्य कर्णेन सनं समकोणसमचतुर्भुजं हलक्षेत्रतुल्यं कार्यम् । पुनः खगक्षे-\\



\newpage

\noindent त्रस्य रेखाः पूर्णा कार्याः । तदा समसमकोणसमचतुर्भुजस्य निष्पत्तिः खफक्षेत्रेण तथास्ति यथा खफक्षेत्रस्य निष्पत्तिः
सनसमकोणसमचतुर्भुजेनास्ति । कुतः । यत एतद्द्वयं गससफनिष्पत्तौ अस्ति । तदा खफक्षेत्रं द्वयोः समकोणसमचतुर्भुजयोर्मध्ये एकनिष्पत्तौ भविष्यति । तदा बहक्षेत्रहलक्षेत्रस्य मध्येऽपि खफक्षेत्रमेकनिष्पत्तौ भविष्यति । दलक्षेत्रं बहक्षेत्रहलक्षेत्रस्य मध्येऽपि एकनिष्पत्तावासीत्~। तस्मात् दलक्षेत्रखफक्षेत्रे समाने भविष्यतः । पुनर्दवक्षेत्रं च रगक्षेत्रेण समानं भविष्यति~। तस्मात् जवक्षेत्रं
तसशक्षेत्रस्य सनसमकोणसमचतुर्भुजयोगेन समानं भविष्यति । पुनर्बझशेषक्षेत्रं
नमसमकोणसमचतुर्भुजेन समानमवशिष्टं भविष्यति~। अस्य भुजः फगोऽस्ति । तस्मात् फगमन्तररेखा भविष्यति । \\
\vspace{5mm}

\begin{center}
\textbf{\large अस्योपपत्तिः । }
\end{center}
\vspace{5mm}

अजवर्गो जझवर्गस्य अजमिलितरेखावर्गस्य च योगेन समानोऽस्ति । तस्माद्यदि जदवर्गतुल्यो जझवर्गस्य चतुर्थांशः अजरेखाखण्डे तथा कार्यो यथा शेषखण्डक्षेत्रं वर्गतुल्यमवशिष्यते तदा अजरेखाया हचिह्ने मिलिते द्वे खण्डे भविष्यतः\renewcommand{\thefootnote}{१}\footnote{{\en J. has} तस्मात् अजहजे मिलिते जाते । {\en after} भविष्यतः }~। अजरेखा चाङ्कसंज्ञार्हास्ति ।
तस्मात् बहक्षेत्रतुल्यं समं समकोणसमचतुर्भुजं हलक्षेत्रतुल्यं
सनं समकोणसमचतुर्भुजमङ्कसंज्ञार्हे भविष्यतः । तस्मात् गसरेखासफरेखयोर्वर्गावङ्कसंज्ञार्हौ भविष्यतः । झजरेखा अजरेखातो भिन्नास्ति~। तस्मात् दजरेखा जझरेखाया मिलितापि मिलितअहरेखाअजरेखयोर्भिन्ना भविष्यति\renewcommand{\thefootnote}{२}\footnote{{\en J. has} तस्माद्वलक्षेत्रतुल्यं जझक्षेत्रं वहक्षेत्रतुल्यसमकोणसमचतुर्भुजाद्भिन्नं भविष्यति ।
{\en after} भविष्यति ।}~। तस्मात् दलक्षेत्रतुल्यं खफक्षेत्रं बहक्षेत्रतुल्यसम-

\newpage
\noindent समकोणसमचतुर्भुजात् भिन्नं भविष्यति । तस्मात् गसरेखासफरेखे
मिथो भिन्ने भविष्यतः । फगं चान्तररेखा भविष्यति । एवं यस्या
रेखाया वर्गो बझक्षेत्रेण तुल्यो भवति सैवान्तररेखा भविष्यति ॥ \\
\begin{center}
\textbf{अथ ८९ क्षेत्रम् ॥ ८९ ॥}
\end{center}
\vspace{5mm}

{\ab यदि क्षेत्रस्यैको भुजोऽङ्कसंज्ञार्हो भवति द्वितीयभुजो द्वितीयान्तररेखा भवति तदा यस्या रेखाया वर्गोऽनेन क्षेत्रेण तुल्यो भवति सा प्रथममध्यान्तररेखा भवति~। }\\
\vspace{3mm}

\begin{vwcol}[widths={0.6,0.4}, sep=.8cm, rule=0pt]
 अस्य प्रकारः क्षेत्रं च पूर्ववत् । परं च हबक्षेत्रतुल्यं
समसमकोण समचतुर्भुजं हलक्षेत्रतुल्यं सनसमकोणसमचतुर्भुजं चैतद्वयं मिलितमध्यक्षेत्रं भविष्यति । कुतः । अहहजयोर्मिलितरेखात्वात् । पुनर्दलक्षेत्रतुल्यं खफक्षेत्रमङ्कसंज्ञार्हं भविष्यति । तस्मात् गसरेखा सफरेखा चैते मध्यरेखे भविष्यतः । अनयोर्वर्गौ मिलितौ भविष्यतः । एतौ भुजौ अङ्कसंज्ञार्हक्षेत्रस्य भविष्यतः । तस्मात् फगरेखा यस्या वर्गो बझक्षेत्रतुल्योऽस्ति सा प्रथममध्यान्तररेखा भविष्यति ॥ \\
\noindent \includegraphics[scale=0.7]{Images/rg-85.png}\\
\end{vwcol}

\begin{center}
\textbf{\large अथ ९० क्षेत्रम् ॥ ९० ॥}
\end{center}
\vspace{5mm}

{\ab यस्य क्षेत्रस्यैकभुजोऽङ्कसंज्ञार्हो भवति द्वितीयभुजस्तृतीयान्तररेखा भवति तदा यस्या रेखाया वर्ग एतत्क्षेत्रतुल्यो भवति सा द्वितीयमध्यान्तररेखा भवति~।}\\
\vspace{3mm}


 प्रकारः क्षेत्रं च पूर्ववत् । परं च हबक्षेत्रतुल्यं
समसमकोणसम-

\newpage
\begin{vwcol}[widths={0.65,0.35}, sep=.8cm, rule=0pt]
\noindent चतुर्भुजं हलक्षेत्रतुल्यं सनसमकोणसमचतुर्भुजं चैते मिलितमध्यक्षेत्रे भविष्यतः । कुतः~। यतः अहहजौ मिलिते रेखे स्तः । झलं दलक्षेत्रतुल्यमपि खफक्षेत्रं मध्यक्षेत्रपूर्वक्षेत्राभ्यां भिन्नं भविष्यति । तस्मात् गसरेखासफरेखे मध्यरेखे केवलवर्गमिलिते भविष्यतः । एते च मध्यक्षेत्रस्य भुजौ भविष्यतः~। तस्मात् फगरेखावर्गो  \\
\noindent \includegraphics[scale=0.6]{Images/rg-86.png}\\
\end{vwcol}
\vspace{-9mm}

\noindent बझक्षेत्रतुल्योऽस्ति ।\renewcommand{\thefootnote}{१}\footnote{{\en J. has} तस्मात् {\en for} स च.}स च द्वितीयमध्यान्तर\\रेखा भविष्यति ॥ \\


\begin{center}
\textbf{\large अथ ९१ क्षेत्रम् ॥ ९१ ॥}
\end{center}
\medskip

{\ab यस्यैको भुजोऽङ्कसंज्ञार्हो भवति द्वितीयभुजश्चतुर्थान्तररेखा भवति
तदा यस्या रेखाया वर्ग एतत्क्षेत्रतुल्यो
भवति सा न्यूनरेखा भविष्यति । }\\
\vspace{5mm}

\begin{vwcol}[widths={0.65,0.35}, sep=.8cm, rule=0pt]
अस्य प्रकारः क्षेत्रं च पूर्ववत् । परं च अहहजरेखे अपि च हबक्षेत्रहलक्षेत्रतुल्ये समक्षेत्रसनक्षेत्रे भिन्ने
भविष्यतः । अनयोर्योगोऽङ्कसंज्ञार्हो भविष्यति । पुनर्झलक्षेत्रतुल्यं द्विगुणखफक्षेत्रं मध्यो भविष्यति । तस्मात् गससफौ भिन्नवर्गौ भविष्यतः । अनयोर्वर्गयोगोऽङ्कसंज्ञार्हो भविष्यति । अनयोर्द्विगुणो घातो मध्यो 
\noindent \includegraphics[scale=0.45]{Images/rg-87.png}\\
\end{vwcol}
\vspace{-8mm}

\noindent  भविष्यति~। तस्मात् फगरेखावर्गो बझक्षेत्र\\
तुल्यो\renewcommand{\thefootnote}{२}\footnote{°तुल्यो न्युनरेखा भवति {\en D.}} भविष्यति ॥ \\
\begin{center}
\textbf{\large अथ ९२ क्षेत्रम् ॥ ९२ ॥}
\end{center}
\vspace{5mm}

{\ab यस्य क्षेत्रस्यैको भुजोऽङ्कसंज्ञार्हो भवति द्वितीयश्च प-}

\newpage
\noindent {\ab ञ्चम्यन्तररेखा भवति पुनर्यद्रेखावर्ग एतत्क्षेत्रतुल्यो भवति
सा अङ्कसंज्ञार्हयुक्तमध्यरेखा भवति ।}\\
\vspace{3mm}

प्रकारः क्षेत्रं \renewcommand{\thefootnote}{१}\footnote{च पूर्ववत् {\en J} }चोपरितनक्षेत्रवत् । परं च अहहजरेखे अपि च हबक्षेत्रहल-
\begin{vwcol}[widths={0.6,0.4}, sep=.8cm, rule=0pt]
क्षेत्रतुल्ये समक्षेत्रसनक्षेत्रे भिन्ने
भविष्यतः~। अनयोर्योगो मध्यो भविष्यति~। झलक्षेत्रतुल्यं द्विगुणखफक्षेत्रमङ्कसंज्ञार्हं भविष्यति~। तस्मात् गससफौ भिन्नवर्गौ भविष्यतः~। \\ \noindent अनयोर्वर्गयोगो मध्यो भविष्यति~। द्विगुणघातश्चाङ्कसंज्ञार्हो
भविष्यति~। तस्मात् फगवर्गो बझक्षेत्रतुल्योऽस्ति । सोऽङ्कसंज्ञार्हयुक्तमध्यो भविष्यति ॥ \\
\vspace{5mm}

\noindent \includegraphics[scale=0.5]{Images/rg-88.png}\\
\end{vwcol}
\vspace{3mm}

\begin{center}
\textbf{\large अथ ९३ क्षेत्रम् ॥ ९३ ॥}
\end{center}
\vspace{5mm}

{\ab यस्यैकभुजोऽङ्कसंज्ञार्हो भवति द्वितीयश्च षष्ठ्यन्तररेखा
भवति तत्र यस्या रेखाया वर्ग एतत्क्षेत्रतुल्यो भवति सा
मध्ययुक्तमध्यरेखा भवति । }\\
\vspace{3mm}

 क्षेत्रं प्रकारश्च पूर्ववत् । परं चात्र अहहजरेखे हबहलक्षेत्रतुल्ये
समक्षेत्र 
\begin{vwcol}[widths={0.6,0.4}, sep=.8cm, rule=0pt]
सनक्षेत्रे च भिन्ने भविष्यतः । अनयोर्योगो मध्यो भविष्यति । पुनर्झलक्षेत्रतुल्यद्विगुणखफक्षेत्रं मध्यो भविष्यति । प्रथममध्याद्भिन्नो भविष्यति । तस्मात् गससफौ भिन्नवर्गौ भविष्यतः । अनयोर्वर्गयोगो मध्यो भविष्यति । अनयोर्द्विगुणो घातश्च मध्यो भविष्यति । प्रथममध्याद्भिन्नो भवति । \\
\noindent तस्मात् फगरेखावर्गो बझक्षेत्रतुल्योऽस्ति । सा मध्ययुक्तमध्या भविष्यति । इदमेवेष्टम् । \\
\noindent \includegraphics[scale=0.6]{Images/rg-89.png}\\
\end{vwcol}


\newpage
\begin{center} 
\textbf{\large अथ ९४ क्षेत्रम् ॥ ९४ ॥}
\end{center}
\vspace{1mm}

{\ab अङ्कसंज्ञार्हरेखायामन्तररेखावर्गतुल्यं क्षेत्रं कार्यं तदोत्पन्नो
द्वितीयभुजः प्रथमान्तररेखा भविष्यति । }\\

यथान्तररेखा अबं कल्प्या । \renewcommand{\thefootnote}{१}\footnote{या रेखा एतां (एनां {\en A., J.)} पूर्वस्वरुपं करोति {\en K.}}यान्तररेखा अनया मिलित्वा प्रथमरूपं करोति सा बजरेखा कल्पिता । अङ्कसंज्ञार्हरेखा च दहं कल्पिता ।
पुनर्दहरेखोपरि अबवर्गतुल्यं दतक्षेत्रं कार्यम् । तस्मादुत्पन्नो दवभुजः प्रथमान्तररेखा भविष्यति~। उपपत्तिः ।\\

पुनर्दहरेखायाम् अजवर्गतुल्यं दनक्षेत्रं कार्यम्~। बजवर्गतुल्यं च नझक्षेत्रं कार्यम् । तस्मात् तझक्षेत्रं द्विगुण अजजबघातसमानं भविष्यति~। पुनर्वझरेखा कचिह्नेऽर्द्धिता कार्या ।
पुनः कलरेखा दहरेखायाः समानान्तरा कार्या । अजजबवर्गावङ्कसंज्ञार्हौ \\
\begin{center}
\noindent \includegraphics[scale=0.9]{Images/rg-90.png}\\
\end{center}
\renewcommand{\thefootnote}{2}\footnote{भविष्यतः {\en A., K.}}स्तः । ततो दनक्षेत्रनझक्षेत्रे दमरेखामझरेखे अपि मिलिताङ्कसंज्ञार्हे भविष्यतः~। तस्मात् दझरेखा संपूर्णाङ्कसंज्ञार्हा
भविष्यति । अजजबघातो मध्यक्षेत्रतुल्योऽस्ति~। तदा झलक्षेत्रझतक्षेत्रे
अपि मध्यक्षेत्रे भविष्यतः । झववर्गोऽप्यङ्कसंज्ञार्हो भविष्यति~। दहरेखाया दझरेखाया भिन्नो भविष्यति । पुनर् अजजबघातः अजवर्गबजवर्गमध्ये एकनिष्पत्तावस्ति\renewcommand{\thefootnote}{३}\footnote{भविष्यति {\en K., A., J.}} । तस्मात् झलक्षेत्रं दनक्षेत्रनझक्षेत्रमध्ये एकनिष्पत्तौ भविष्यति~। पुनर्दमझकनिष्पत्तिः झकरेखाझमरेखानिष्पत्तितुल्यास्ति~। यदि झकवर्गतुल्यझववर्गचतुर्थांशतुल्यं
क्षेत्रं दझरेखाखण्डे तथा कार्यं यथा शेषखण्डक्षेत्रं वर्गरूपं भवति दझरेखायामचिह्ने मिलिते द्वे खण्डे भविष्यतः~। पुनर्दझरेखावर्गो
झवरेखावर्गस्य दझरेखामिलितरेखावर्गस्य च योगेन तुल्यो भविष्यति । अस्मदिष्टं
समीचीनम् ॥ 

\newpage
\begin{center} 
\textbf{\large अथ ९५ क्षेत्रम् ॥ ९५ ॥}
\end{center}

{\ab अङ्कसंज्ञार्हरेखायां प्रथममध्यान्तररेखावर्गतुल्यं क्षेत्रं कार्यं
तदोत्पन्नो भुजो द्वितीयान्तररेखा भविष्यति ।}\\

प्रकारः क्षेत्रं च पूर्ववत् । परं च दनक्षेत्रनझक्षेत्रे
मध्यमिलिते भविष्यतः~। तस्मात् हझक्षेत्रं मध्यं भविष्यति । दझरेखायाः केवलवर्गोऽङ्कसंज्ञार्हो भविष्यति~। अङ्कसंज्ञार्हा  पुनर्झतक्षेत्रतुल्यो\renewcommand{\thefootnote}{१}\footnote{°तुल्य° {\en A.}} द्विगुणअजजबघातोऽङ्कसंज्ञार्हो  भविष्यति~। तस्मात् झवरेखा 
\begin{center}
\includegraphics[scale=0.7]{Images/rg-91.png}\\
\end{center}
भविष्यति । झदरेखावर्गो झवरेखावर्गस्य हदरेखामिलितरेखावर्गस्य च योगेन तुल्यो भविष्यति । कुतः । दममझयोर्मिलितत्वात् । तस्मात् दवरेखा द्वितीयान्तररेखा भविष्यति ॥\\
\begin{center}
\textbf{\large अथ ९६ क्षेत्रम् ॥ ९६ ॥}
\end{center}

{\ab अङ्कसंज्ञार्हरेखोपरि द्वितीयमध्यान्तररेखावर्गतुल्यं क्षेत्रं
कार्यं तदोत्पन्नभुजस्तृतीयान्तररेखा भविष्यति ।}\\

 अस्य प्रकारः क्षेत्रं च पूर्ववत् । परं च हझक्षेत्रमपि मध्यं भविष्यति दनन-
\begin{vwcol}[widths={0.65,0.35}, sep=.8cm, rule=0pt]
झयोर्मध्ये मिलितत्वात् । दझवर्गः केवलमङ्कसंज्ञार्होऽस्ति । तझक्षेत्रमपि मध्योऽस्ति ।
प्रथममध्याद्भिन्नोऽस्ति । अजजवयोर्भिन्नत्वात् । तस्मात् झवरेखापि केवलवर्गाङ्कसंज्ञार्हा\\ \noindent  भविष्यति । दझाद्भिन्ना भविष्यति । दझवर्गो झववर्गस्य दझमिलितरेखावर्गयोगेन तुल्यो\\ \noindent  भविष्यति । कुतः । दममझयोर्मिलितत्वात्~। तस्मात्  दवं तृतीयान्तररेखा भविष्यति ॥ \\
\noindent \includegraphics[scale=0.7]{Images/rg-92.png}
\end{vwcol}
\vspace{-4mm}

\newpage
\begin{center} 
\textbf{\large अथ ९७ क्षेत्रम् ॥ ९७ ॥}
\end{center}
\vspace{5mm}

{\ab अङ्कसंज्ञार्हरेखायां न्यूनरेखावर्गतुल्यं क्षेत्रं कार्यं तत्रोत्पन्नभुजश्चतुर्थ्यन्तररेखा भविष्यति । }\\


अस्य प्रकारः क्षेत्रं च पूर्ववत् । अजबजवर्गयोर्भिन्नत्वेन
दनक्षेत्रनझक्षेत्रे भिन्ने भविष्यतः । दमरेखामझरेखे \renewcommand{\thefootnote}{१}\footnote{च {\en J. }}अपि भिन्ने 
\begin{vwcol}[widths={0.6,0.4}, sep=.8cm, rule=0pt]
भविष्यतः । द्वयोर्वर्गयोर्योगस्याङ्कसंज्ञार्हत्वेन हझक्षेत्रमप्यङ्कसंज्ञार्हं भविष्यति । दझरेखा चाङ्कसंज्ञार्हा भविष्यति ।
द्विगुणअजजबघातस्य मध्यभावित्वेन तझक्षेत्रमपि मध्यं भविष्यति । वझरेखापि केवलवर्गाङ्कसंज्ञार्हास्ति~। दझवर्गो वझवर्गस्य दझभिन्नरेखावर्गस्य च योगेन तुल्योऽस्ति । कुतः । दममझयोर्भिन्नत्वात् । तस्मात् दवं चतुर्थ्यन्तररेखा भविष्यति~॥ \\
\vspace{5mm}

\noindent \includegraphics[scale=0.75]{Images/rg-94.png}
\end{vwcol}
\vspace{2mm}

\begin{center}
{\large अथ ९८ क्षेत्रम् ।। ९८ ।।}
\end{center}
\vspace{2mm}

{\ab अङ्कसंज्ञार्हरेखायामङ्कसंज्ञार्हरेखायुक्तमध्यरेखावर्गतुल्यं क्षेत्रं
कार्यं तत्रोत्पन्नभुजः पञ्चम्यन्तररेखा भविष्यति । }\\


प्रकारः क्षेत्रं च पूर्ववत् । परं अजबजवर्गयोर्भिन्नत्वेन
दनक्षेत्रनझक्षेत्रे भिन्ने 
\begin{vwcol}[widths={0.65,0.4}, sep=.8cm, rule=0pt]
भविष्यतः । दममझरेखापि भिन्ना
भविष्यति~। द्वयोर्वर्गयोर्योगस्य मध्यभावित्वेन दझं केवलवर्गसंज्ञार्हो भाविष्यति ।
द्विगुणअजबजघातस्याङ्कसंज्ञार्हभावित्वेन झवरेखा अङ्कसंज्ञार्हा भविष्यति~। तस्मात् दझरेखावर्गो झवरेखावर्गस्य दझरेखाभिन्नरेखावर्गस्य च योगेन तुल्यो भविष्यति । दममझयोर्भिन्नत्वात् । दमरेखा पञ्चम्यन्तररेखा भविष्यति ॥ 
\vspace{4mm}

\noindent \includegraphics[scale=0.8]{Images/rg-95.png}
\end{vwcol}
\vspace{-2mm}

\newpage
\begin{center}
\textbf{\large अथ ९९ क्षेत्रम् ॥ ९९ ॥}
\end{center}
\vspace{2mm}

 {\ab अङ्कसंज्ञार्हरेखायां मध्ययुक्तमध्यरेखावर्गतुल्यं क्षेत्रं कार्यं
तत्रोत्पन्नद्वितीयभुजः षष्ठ्यन्तररेखा भविष्यति । }\\

\begin{vwcol}[widths={0.65,0.35}, sep=.8cm, rule=0pt]
 प्रकारः क्षेत्रं च पूर्ववत् । परं च अजबजवर्गयोर्भिन्नभावित्वेन
दनक्षेत्रनझक्षेत्रे भिन्ने भविष्यतः~। दममझरेखापि भिन्ना  भविष्यति । द्वयोर्वर्गयोर्योगस्य मध्यक्षेत्रभावित्वेन तथा द्विगुणअजबजघातस्य मध्यभावित्वेन प्रथममध्याद्भिन्नत्वेन च दझझवरेखे केवलवर्गाङ्कसंज्ञार्हे\\
\noindent \includegraphics[scale=0.6]{Images/rg-96.png}
\end{vwcol}
\vspace{-7mm}

\noindent  भविष्यतः । \renewcommand{\thefootnote}{१}\footnote{{\en  J. adds} मिथो.}भिन्ने च भविष्यतः । \renewcommand{\thefootnote}{२}\footnote{{\en J. Omits this sentence.} }केवलवर्गा-\\वङ्कसंज्ञार्हौ 
भविष्यतः । दझवर्गो झववर्गस्य दझभिन्नरेखावर्गस्य च योगेन तुल्यो भविष्यति ।
दममझायोर्भिन्नत्वात् । तस्मात् दवं षष्ठ्यन्तररेखा भविष्यति~। इदमेवेष्टम् ।। \\
\begin{center}
\textbf{अथ १०० क्षेत्रम् ॥ १०० ॥}
\end{center}
\vspace{3mm}

{\ab अन्तररेखामिलितरेखा तादृश्येवान्तररेखा भवति । }\\

 यथा अजम् अन्तररेखा कल्पिता । दझं मिलितरेखा कल्प्या\renewcommand{\thefootnote}{३}\footnote{{\en J. Omits} कल्प्या. } ।
पुनर् अजरेखायां जबरेखा तथा युक्ता\renewcommand{\thefootnote}{४}\footnote{योज्या {\en A., K., J.} } कार्या यथा पूर्वरूपं करोति ।
पुनर्दझरेखाझहरेखानिष्पत्तिः अजजबनिष्पत्तितुल्या कल्प्या\renewcommand{\thefootnote}{५}\footnote{कल्पिता {\en A., K., J.} }। \\

यदि अबवर्गो बजवर्गस्य अजमिलितरेखाया अथवा भिन्नरेखाया वर्गस्य
\begin{center}
\includegraphics[scale=0.7]{Images/rg-97.png}
\end{center}
योगतुल्यो भवति तदा दहरेखा झहरेखे \renewcommand{\thefootnote}{६}\footnote{सदृशे {\en A., J.}}तादृशे स्तः ।  पुनरपि प्रत्येकं
अबबजौ प्रत्येकदहझहाभ्यां मिलितत्वेन प्रत्येकमङ्कसंज्ञार्हो भवति वां\renewcommand{\thefootnote}{७}\footnote{वर्गोऽङ्कसंज्ञार्हो भवति {\en A.}\\
भा∘ १६} वर्गाङ्कसंज्ञार्हो 

\newpage
\noindent भवति । तदा द्वितीयरेखापि तथैव भविष्यति । तस्मात् अजं यान्तररेखा भवति दझमपि तथैवान्तररेखा भविष्यति ॥ \\
\begin{center}
\textbf{\large अथ १०१ क्षेत्रम् ॥ १०१ ॥}
\end{center}
\vspace{5mm}

{\ab मध्यान्तररेखया या मिलिता रेखा भवति सा मध्यान्तररेखासदृशी भवति~। }\\
\vspace{3mm}

यथा अजं प्रथममध्यान्तररेखा वा द्वितीयमध्यान्तररेखा कल्पिता ।
तद्रेखा मिलिता दझरेखा कल्पिता । पुनर्
अजरेखया 
\begin{vwcol}[widths={0.65,0.35}, sep=.8cm, rule=0pt]
लग्ना जबरेखा तथा कल्प्या
यथा सा अजरेखां पूर्वरूपां करोति । दझझहयोर्निष्पत्तिः अजजबनिष्पत्तितुल्यास्ति ।
प्रत्येकम् अबजबौ दहहझाभ्यां मध्यस्वजातीयेन मिलितौ स्तः । 
\includegraphics[scale=0.7]{Images/rg-98.png}
\end{vwcol}
\vspace{-2mm}

\noindent यादृशो मध्यसजातीयोऽस्ति तावत्तथैव प्रत्येकम् अबबजयोर्मध्योऽस्ति । अबबजौ भिन्नौ स्तः । तस्मात् दहहझावपि भिन्नौ भवेताम् ।
अबवर्गनिष्पत्तिः अबबजघातेन तथास्ति यथा
दहवर्गनिष्पत्तिर्दहहझघातेनास्ति । अबवर्गदहवर्गयोर्निष्पत्तिः
अबबजघातदहझहघातनिष्पत्त्या समानास्ति । अबवर्गदहवर्गौ मिलितौ स्तः । तस्मात् अबबजघातदहहझघातावपि मिलितौ भविष्यतः । \\
\vspace{3mm}

यदि अबबजघातोऽङ्कसंज्ञार्हो भवति तदा दहहझघातोऽप्यङ्कसंज्ञार्हो भविष्यति~। यदि अबबजघातो मध्यो भवति तदा दहहझघातोऽपि मध्यो भविष्यति~। क्षेत्रं\renewcommand{\thefootnote}{१}\footnote{{\en V. inserts} द्वयोर्मध्यान्तररेखयोर्मध्येऽन्तररेखा अजं यथा भवति तथैव
मध्यान्तरं दझमपि भविष्यति.} च पूर्ववत् ।। \\
\begin{center}
\textbf{\large अथ १०२ क्षेत्रम् ॥ १०२ ॥}
\end{center}
\vspace{2mm}

{\ab न्यूनरेखया मिलिता रेखा न्यूना भवति । }\\
\vspace{3mm}

 यथा अं न्यूना रेखा कल्पिता । तन्मिलिता बरेखा कल्पिता । अन- 


\newpage
\begin{vwcol}[widths={0.67,0.33}, sep=.8cm, rule=0pt]
\noindent योर्वर्गतुल्ये क्षेत्रे जदअङ्कसंज्ञार्हरेखायां कार्ये । 
अवर्गतुल्यं क्षेत्रं जदरेखायां यत्तद्द्वितीयो भुजो जहं चतुर्थ्यन्तररेखा भवति । बवर्गतुल्यं क्षेत्रं 
जदरेखायां यत् कृतं तदुत्पन्नो जझभुजो जहमिलितोऽस्ति । तस्मात् जझमपि चतुर्थ्यन्तररेखा 
भवति । तस्माद्यद्रेखावर्गो दझक्षेत्रतुल्यो भवति 
सा बरेखा भवति । इयं न्यूनरेखा भविष्यति ॥ \\
\noindent \includegraphics[scale=0.6]{Images/rg-99.png}
\end{vwcol}

\begin{center}
\textbf{\large अथ १०३ क्षेत्रम् ॥ १०३ ॥}
\end{center}

{\ab अङ्कसंज्ञार्हयुक्तमध्यरेखाया मिलिता रेखा भवति साप्यङ्कसंज्ञार्हयुक्तमध्यरेखा \renewcommand{\thefootnote}{१}\footnote{भवेत् {\en V.} }भवति । }\\

प्रकारः क्षेत्रं च पूर्ववत् ॥ 
\begin{center}
\textbf{\large अथ १०४ क्षेत्रम् ॥ १०४ ॥}
\end{center}

{\ab मध्ययुक्तमध्यरेखया या मिलिता रेखा भवति सापि मध्य  
युक्तमध्यरेखा भवति । }\\

प्रकारः क्षेत्रं च पूर्ववत् ॥ 
\begin{center}
\textbf{\large अथ १०५ क्षेत्रम् ॥ १०५ ॥}
\end{center}

{\ab अङ्कसंज्ञार्हक्षेत्रस्य\renewcommand{\thefootnote}{२}\footnote{क्षेत्रमध्यक्षेत्रयोर्यदन्तरमस्ति {\en K., A., J.} } मध्यक्षेत्रेण यदन्तरमस्ति तत्तुल्यो यस्या रेखाया वर्गो भवति सा रेखान्तररेखा वा न्यूनरेखा \renewcommand{\thefootnote}{3}\footnote{भविष्यति
{\en V.}}भवति । }\\

 यथा अङ्कसंज्ञार्हक्षेत्रं अबम् कल्पितम् । मध्यक्षेत्रम् अदं कल्पितम् ।
अङ्कसंज्ञार्ह-
\begin{vwcol}[widths={0.6,0.4}, sep=.8cm, rule=0pt]
क्षेत्रस्य मध्यक्षेत्रेणान्तरं जबक्षेत्रं कल्पितम्। पुनर्हझम् अङ्कसंज्ञार्हरेखा कल्पिता।
अस्याम् अबक्षेत्रतुल्यं झकक्षेत्रं कार्यम् । तस्यामेव अदक्षेत्रतुल्यं झवक्षेत्रं कार्यम् । तस्मात्
कहरेखा अङ्कसंज्ञार्हा भविष्यति । हवरेखा च केवलवर्गाङ्कसंज्ञार्हा भविष्यति । यदि हक-\\
\vspace{-4mm}

\noindent \includegraphics[scale=0.8]{Images/rg-100.png}
\end{vwcol}


\newpage
\noindent रेखावर्गो हवरेखावर्गस्य हकरेखामिलितरेखावर्गस्य च योगेन तुल्यो
\renewcommand{\thefootnote}{१}\footnote{भवति {\en V. }}भवेत् तदा वकं प्रथमान्तररेखा भविष्यति ।\\

यद्रेखावर्गस्तकक्षेत्रतुल्यजबक्षेत्रसमानो भवति सा अन्तररेखा भवति। यदि हकरेखावर्गो हवरेखावर्गस्य हकरेखाभिन्नरेखावर्गस्य च योगेन तुल्यो भवति तदा वकरेखा चतुर्थी अन्तररेखा भविष्यति । पुनस्तकक्षेत्रतुल्यजबक्षेत्रसमानो यद्रेखावर्गो भवति सा न्यूनरेखा
भविष्यति ॥\\
\begin{center}
\textbf{\large अथ १०६ क्षेत्रम् ॥ १०६ ॥}
\end{center}
\vspace{2mm}

{\ab मध्यक्षेत्रस्याङ्कसंज्ञार्हक्षेत्रेणान्तरतुल्यो यद्रेखावर्गो भवति सा प्रथममध्यान्तररेखा भविष्यति वाङ्कसंज्ञार्हयुक्तमध्यरेखा भविष्यति ।}\\

प्रकारः क्षेत्रं च पूर्ववत्। परं त्वत्र अबं मध्यक्षेत्रंभविष्यति । हकरेखा  केवलवर्गाङ्कसंज्ञार्हा भविष्यति। 
हवरेखा चाङ्कसंज्ञार्हा भविष्यति। \renewcommand{\thefootnote}{२}\footnote{{\en Omitted in K., A., J.}}वकरेखा 
\begin{center}
\noindent \includegraphics[scale=0.8]{Images/rg-101.png}
\end{center}
द्वितीयान्तररेखा वा पञ्चम्यन्तररेखा भविष्यति। जबक्षेत्रतुल्यो यद्रेखावर्गों भवति
स प्रथममध्यान्तररेखा भविष्यति वाङ्कसंज्ञार्हयुक्तमध्यरेखा भविष्यति। \\


\begin{center}
\textbf{\large अथ १०७ क्षेत्रम् ॥ १०७ ॥}
\end{center}
\vspace{2mm}

{\ab मध्यक्षेत्रतद्भिन्नमध्यक्षेत्रान्तरतुल्यो यद्रेखावर्गो भवति
सा द्वितीयमध्यान्तररेखा वा मध्ययुक्तमध्यान्तररेखा भविष्यति।}\\

 प्रकारः क्षेत्रं च पूर्ववत्। परं त्वत्र हवरेखाहकरेखे भिन्नरेखे

\newpage

\noindent मिथो भविष्यतः। अनयोः केवलवर्गाङ्कसंज्ञार्हौ भविष्यतः। वकं तृतीयान्तररेखा 
\begin{vwcol}[widths={0.6,0.4}, sep=.8cm, rule=0pt]
तदा भविष्यति यदा हकरेखावर्गो हवरेखावर्गस्य हकमिलितरेखावर्गस्य च योगेन
तुल्यो भविष्यति। पुनः सैव वकरेखा षष्ठयन्तररेखा तदा भविष्यति यदा हकरेखावर्गो हवरेखावर्गस्य हकभिन्नरेखावर्गस्य च योगेन तुल्यो भवति । तस्मात् यद्रेखावर्गो जबक्षेत्रतुल्यो भवति सा द्वितीयमध्यान्तररेखा वा मध्ययुक्तमध्यरेखा भविष्यति~॥\\
\noindent \includegraphics[scale=0.9]{Images/rg-102.png}
\end{vwcol}
\vspace{2mm}


\begin{center}
\textbf{\large अथ १०८ क्षेत्रम् ॥ १०८ ॥}
\end{center}
\vspace{5mm}

{\ab अन्तररेखा योगरेखा न भवति ।}\\
\vspace{3mm}

यदि भवति तदा कल्पितम् अरेखा अन्तररेखा भवति योगरेखापि। बजम् 
\begin{vwcol}[widths={0.6,0.4}, sep=.8cm, rule=0pt]
अङ्क संज्ञार्हरेखा कल्पिता। अरेखावर्गतुल्यं क्षेत्रं बजरेखायां दजक्षेत्रं कार्यम् । तदोत्पन्नो वदभुजः प्रथमयोगरेखा भविष्यति। कुतः। अरेखाया योगरेखात्वात् । स एवोत्पन्नो बदभुजः प्रथमान्तररेखा भविष्यति। यतः अरेखा अन्तररेखास्ति। ज
तदा कल्पितं बदरेखाया झचिह्ने योज्यखण्डे बझं  \\
\noindent \includegraphics[scale=0.8]{Images/rg-103.png}
\end{vwcol}
\noindent महत्खण्डं कल्पितम् । इदं बझम् अङ्कंसज्ञार्हरेखा भविष्यति । झदं  केवलवर्गाङ्कसंज्ञार्हा रेखा भविष्यति । भवति सा द्वितीयमध्यान्तररेखा वा मध्ययुक्तमध्यरेखा भविष्यति ॥ बदरेखया दहरेखा संलग्ना तथा कल्प्या यथा बदरेखां पूर्वरूपां करोति। तस्मात् बहरेखा अङ्कसंज्ञार्हा रेखा भविष्यति । हदरेखा । शेषं झहरेखा अङ्कसंज्ञार्हा
भविष्यति~। तस्मात् झहरेखा झदरेखया वा दहरेखया सह
केवलवर्गाङ्कसंज्ञार्हा भविष्यति~। तस्मात् दहरेखा वा दझरेखा अन्तररेखा भविष्यति।
अस्या एव दहरेखाया बदझरेखाया वर्गोऽङ्कसंज्ञार्ह आसीत्। इदम
शुद्धम्। अस्मदिष्टं समीचीनम् ॥
केवलवर्गाङ्कसंज्ञार्हास्ति

\newpage
\begin{center}
\textbf{\large अथ १०९ क्षेत्रम् ॥ १०९ ॥}
\end{center}
\vspace{2mm}

{\ab मध्यरेखातः करणीरूपा रेखा \renewcommand{\thefootnote}{१}\footnote{याः तुल्याः सन्ति तासां {\en J. }}बह्वय उत्पत्स्यन्ते तासां
मध्ये क्वापि द्वितीयोत्पन्ना प्रथमानुकारा न भवति ।}\\
\vspace{3mm}

यथा अबरेखा अङ्कसंज्ञार्हा कल्पिता। अस्यां अझरेखा लम्बरूपा
कल्पिता। अजं अझे मध्यरेखा कल्पिता। पुनर् अहक्षेत्रं \renewcommand{\thefootnote}{२}\footnote{पूर्णं {\en V. }}संपूर्णं 
कार्यम्। इदं अहक्षेत्रं मध्यक्षेत्रं न  भविष्यति। कुतः । 
मध्यक्षेत्रतुल्यम् अबरेखायां क्षेत्रं यदि क्रियते तदोत्पन्नभुजवर्गोऽङ्कसंज्ञार्हो भवति । अहक्षेत्रोत्पन्नभुजश्च मध्यरेखास्ति। पुनर्जदरेखावर्गः अहक्षेत्रतुल्योऽस्तीति कल्पितम् । इयं जदरेखा अजरेखासदृशी न भवति।
पुनर्दहक्षेत्रं संपूर्णं कार्य~। 
\renewcommand{\thefootnote}{३}\footnote{पूर्णं {\en V.}} इदं दहक्षेत्रम् अहक्षेत्र-
\begin{vwcol}[widths={0.56,0.44}, sep=.8cm, rule=0pt]
सदृशं न भविष्यति ~। कुतः~। अहक्षेत्रस्योत्पन्नभुजो मध्योऽस्ति~।
दहक्षेत्रस्योत्पन्नभुजो जदमस्ति । पुनर्दहक्षेत्रतुल्यो यद्रेखावर्गो भवति सापि
जदरेखासदृशी न भविष्यति । अजरेखासदृशी अपि न भविष्यति ।
अनेनैव ~~~~~प्रकारेण तद्रेखातो जझरेखातुल्यं पृथक्क्रियते क्षेत्राणि च
क्रियन्ते तदा तादृश्यो \\

\noindent \includegraphics[scale=0.85]{Images/rg-104.png}
\end{vwcol}
\vspace{-1mm}

\noindent बह्व्यो रेखा भविष्यन्ति परं पूर्वानुकारा न भवेयुः । श्रीमद्राजाधिराजप्रभुवरजयसिंहस्य तुष्ट्यै द्विजेन्द्रः श्रीमत्सम्राड् जगन्नाथ इति समभिधारूढितेन प्रणीते । ग्रन्थेऽस्मिन्नाम्नि रेखागणित इति सुकोणावबोधप्रदातर्य्यध्यायोऽध्येतृमोहापह इह विरतिं दिड्मितः संगतोऽभूत् ॥\\

\begin{center}
\textbf{ ॥ इति श्रीसम्राड्जगन्नाथविरचिते रेखागणिते\\
 दशमोऽध्यायः संपूर्णः ॥ १० ॥}
 \vspace{3mm}
 
 \rule{0.5in}{0.3pt}
\end{center}
   

\newpage
\afterpage{\fancyhead[CE] {रेखागणितम्}}
\afterpage{\fancyhead[CO] {दशमोध्यायः}}
\afterpage{\fancyhead[LE,RO]{\thepage}}
\cfoot{}
\newpage
%%%%%%%%%%%%%%%%%%%%%%%%%%%%%%%%%%%%%%%%%%%%%%%%%%%%%%%%%%%%%%
\newpage
\thispagestyle{empty}
\begin{center}
\textbf{\LARGE ॥ अथैकादशोऽध्यायः प्रारभ्यते ॥}
\end{center}
\vspace{3mm}

\begin{center}
\textbf{॥ अस्मिन्नेकचत्वारिंशत् क्षेत्राणि सन्ति ॥}
\vspace{5mm}

\textbf{\large \renewcommand{\thefootnote}{१}\footnote{ {\en Omitted in V .; J. has} अत्र {\en for} तत्र. }तत्रादौ परिभाषा ॥}
\end{center}
\vspace{2mm}

\begin{enumerate}

\item[१] यस्य क्षेत्रस्य \renewcommand{\thefootnote}{२}\footnote{दैर्घ्यविस्तारपिण्डा उपलभ्यन्ते {\en K., A., J.}}दैर्घ्यं विस्तारः पिण्डश्चोपलभ्यते तत् घनक्षेत्रसंज्ञकं भवति~। इदं क्षेत्रं धरातलेषु संपूर्णं भवति ।

\item[२] धरातले शङ्कुरूपा निषण्णा या रेखा भवति तन्मूलात् \renewcommand{\thefootnote}{३}\footnote{निसृताः सर्वतो रेखा {\en J.} }सर्वतो निसृता रेखा यदि मूलयोगेन समकोणमुत्पादयन्ति तदा सा रेखा धरातले लम्बो भवति ।

\item[3] धरातलेऽन्यधरातलं भित्तिवत् संलग्नं यदि भवति तद्योगतो निसृतरेखाभ्यां यदि समकोणो भवति तदा संलग्नं धरातलं लम्बवद्भवति ।

\item[४] ये धरातले उभयतो वर्द्धिते यदि न मिलतस्तदा ते समानान्तरे
भवतः ।

\item[५] येषां घनक्षेत्राणां धरातलानि सजातीयानि संख्यया समानानि
 क्षेत्रफलेनापि समानानि स्युस्तानि समानानि सजातीयानि \renewcommand{\thefootnote}{४}\footnote{{\en J. Omits} भवन्ति.}भवन्ति ।

\item[६]  तेषां\renewcommand{\thefootnote}{५}\footnote{{\en K., J., and A. omit} तेषां. } धरातलानां क्षेत्रफलानि समानानि न भवन्ति \renewcommand{\thefootnote}{६}\footnote{तदा तानि {\en V., J.} }तदैतानि
केवलसजातीयानि भवन्ति ।

\item[७] यस्य धनक्षेत्रस्य द्वे धरातले त्रिभुजे भवतस्त्रीणि धरातलानि समानान्तरभुजचतुर्भुजानि भवन्ति तच्छेदितघनक्षेत्रं भवति ।

\item[८] व्यासोपरि सर्वतो वृत्तभ्रमणेन यद् घनफलमुत्पद्यते तद् गोलक्षेत्रं भवति~।

\item[९] \renewcommand{\thefootnote}{७}\footnote{{\en K. and A. have} एक {\en for} अनेकास्त्र. }अनेकास्त्रधरातलान्निःसृतानि सूच्यग्रधरातलानि यद्येकत्र मिलन्ति तत् क्षेत्रं \renewcommand{\thefootnote}{८}\footnote{सूचीफलकघनं क्षेत्रं {\en D.}}सूचीफलकशङ्कुघनक्षेत्रं भवति।

\end{enumerate}

\newpage
  
\begin{enumerate}
\item[१०] समकोणचतुर्भुजक्षेत्रैकभुजभ्रमणेन\renewcommand{\thefootnote}{१}\footnote{° भुजो निषण्णो यथा भवति तद्भ्रमणेन {\en K., A., J.}} यत् क्षेत्र कूपाकारं भवति तत् समतलमस्तकपरिधिरूपं शङ्कुघनक्षेत्रं भवति ।

\item[११] अस्य क्षेत्रस्य स्थिरभुजो लम्बो भवति ।

\item[१२] समकोणत्रिभुजक्षेत्रस्य समकोणभुजं स्थिरं कृत्वा त्रिभुजभ्रमणेन
 यत् क्षेत्रमुत्पद्यते स शङ्कुर्भवति ।

\item[१३] यदि समकोणसंबंन्धिभुजौ समानौ भवतस्तदा शङ्कुशिरसि स
 मानकोणो भवति ।

\item[१४] यदि स्थिरभुजो द्वितीयभुजादधिको भवति तदा शङ्कुर्न्यूनकोणो भविष्यति~।

\item[१५] यदि स्थिरभुजो न्यूनो भवति तदा शङ्कुरधिककोणो भवति ।

\item[१६] अस्य शङ्कोः स्थिरभुज एव लम्बो भवति ।

\item[१७] \renewcommand{\thefootnote}{२}\footnote{घरातलकोणानां योगजनितकोणो घनकोणो भवति । {\en K., A., J.}}व्यादिधरातलयोगजनितकोणो धनकोणो भवति ।

\item[१८] शङ्कुक्षेत्रसमतलमस्तकशङ्कुक्षेत्रयोः खलम्बव्यासयोर्निष्पत्तिः समाना
 यदि भवति तदा ते क्षेत्रे सजातीये भवतः ।
\end{enumerate}

\begin{center}
\textbf{ ॥ इति परिभाषा ॥}
\end{center}
\vspace{2mm}

\begin{center}
\textbf{\large  \renewcommand{\thefootnote}{३}\footnote{प्रथमक्षेत्रम् {\en V.}}अथ प्रथमं क्षेत्रम् ॥ १ ॥}
\end{center}
\vspace{2mm}

{\ab एकस्याः सरलरेखाया एकं खण्डं धरातले एक पिण्डे
भवितुं नार्हति ।}\\

 यदि भवति तदा अबजं सरला रेखा कल्पिता। अस्या अबखण्डं
धरातले 
\begin{vwcol}[widths={0.6,0.4}, sep=.8cm, rule=0pt]
बजखण्डं पिण्डे कल्पितम् । धरातले तु रेखा वर्द्धयितुं शक्यते ।
अबरेखा धरातले एव दचिह्नपर्यन्तं वर्द्धनीया। अबजरेखाअबदरेखे
एकरूपे भवतः। इदमशुद्धम्। अस्मदिष्टं समीचीनम् ॥\\
\vspace{-2mm}

\noindent \includegraphics[scale=0.75]{Images/rg-105.png}
\end{vwcol}

\newpage
\begin{center}
\textbf{\large \renewcommand{\thefootnote}{१}\footnote{{\en omits} अथ.}अथ द्वितीय क्षेत्रम् ॥ २ ॥}
\end{center}
\vspace{5mm}

{\ab ये द्वे \renewcommand{\thefootnote}{२}\footnote{सरले रेखे {\en V.~३}}सरलरेखे मिथः संपातं कुरुतस्ते एकस्मिन् धरातले
भवतः यत्त्रिभुजं तदप्येकस्मिन् धरातले भवति ।}\\
\vspace{3mm}

 यथा अबजदे द्वे रेखे हचिह्ने \renewcommand{\thefootnote}{3}\footnote{संपातं कुरुत इति कल्पितम् {\en J.}}संपातकारिण्यौ कल्पिते । पुनरनयोः झचिह्न-
 
 \begin{vwcol}[widths={0.7,0.3}, sep=.8cm, rule=0pt] 
वचिह्ने  कल्पिते । झवरेखा संलग्ना कार्या । तस्मात् हझवत्रिभुजमेकधरातले भविष्यति । यदि न भवति तदा कस्यापि भुजस्यैकं खण्डं धरातले भविष्यति । द्वितीयं च पिण्डे। इदमशुद्धम् । ते कल्पिते रेखे त्रिभुजधरातले स्तः । तस्मात्ते रेखे  एकस्मिन् धरातले जाते । इदमेवेष्टम् ॥\\
\noindent \includegraphics[scale=0.8]{Images/rg-106.png}
\end{vwcol}
\vspace{5mm}

\begin{center}
\textbf{\large अथ तृतीयं क्षेत्रम् ॥ ३ ॥}
\end{center}
\vspace{5mm}

{\ab द्वे धरातले यदि मिथः संपातं कुरुत एतयोः संपाते \renewcommand{\thefootnote}{४}\footnote{सरलैका रेखा भविष्यति~{\en J.}}एकैव सरला रेखा भवति ।}\\

 \begin{vwcol}[widths={0.65,0.35}, sep=.8cm, rule=0pt] 
 यथा अबजदमेकं धरातलं हझवतं द्वितीयं धरातलम् । अदभुजतव-
 भुजयोः संपातः कचिह्ने कल्पितः । बजभुजहझभुजयोः संपातः 
लचिह्ने 
\noindent \includegraphics[scale=0.75]{Images/rg-107.png}
\end{vwcol}
\vspace{-7mm}

\noindent कल्पितः । यदि कचिह्नसंपात\renewcommand{\thefootnote}{५}\footnote{{\en J.omits } संपात, }
लचिह्नसंपातयोर्या \\
रेखा लग्ना 
सा धरातलद्वयेप्येका न भवति तदैक-\\
स्मिन् धरातले कमलरेखा कल्पिता । द्वितीयधरा-\\
तले कनलरेखा कल्पिता । एते रेखे सरले स्तः । आभ्यां स्थान-द्वये मिथः संपातः कृतः । इदमशुद्धम् । तस्मात् कलं धरातलद्वये एकैव
\renewcommand{\thefootnote}{६}\footnote{सरलरेखा {\en J.}}योज्यरेखा भविष्यति । इयमेव धरातलद्वयसंपातयोज्यरेखास्ति । \renewcommand{\thefootnote}{७}\footnote{{\en J.omits} अस्माकम्.\\
 भा० १७}इदमेवास्माकमिष्टम् ॥

\newpage
\begin{center}
\textbf{प्रकारान्तरम् ॥}
\end{center}
\vspace{2mm}

कचिह्नलचिह्ने अबजदधरातले स्तः । एकधरातलगतचिह्नद्वये एका रेखा योजयितुं शक्यते । तस्मात् अबजदधरातले कलरेखा योज्या  । पुनरपि कचिह्नलचिह्ने हझवतधरातले स्तः  । अस्मिन्नपि धरातले चिह्नद्वये कलरेखा संयोजितास्ति । द्वयोश्चिह्नयोः सरला एकैव रेखा लगति । तस्मात् कलम् एकैव रेखा धरातलद्वये भविष्यति ॥\\
\begin{center}
\textbf{\large अथ चतुर्थं क्षेत्रम् ॥ ४ ॥}
\end{center}
\vspace{5mm}

{\ab द्वे रेखे यद्येकचिह्ने संपातं कुरुतः संपातचिह्नादेको लम्बो रेखाद्वये भवति तदा यस्मिन् धरातले ते द्वे रेखे स्तस्तत्र स लम्बो लम्ब एव भवति ।}\\
\vspace{3mm}

 यथा जदहझरेखे बचिह्ने कृतसंपाते कल्पिते । अनयोरुपरि अबरेखा लम्बः कल्पितः । पुनर्बजं बहं बदं बझं समानं पृथक् कार्यम् । पुनरबलम्बोपरि वचिह्नं
 \begin{vwcol}[widths={0.65,0.35}, sep=.8cm, rule=0pt] 
  कल्पितम् । पुनर्जवं हवं झवं दवं रेखाः संयोज्याः । तत्र चत्वारि त्रिभुजानि भविष्यन्ति~। तेषां भुजाः कोणाश्च मिथः समाना भविष्यन्ति~। पुनर्जहरेखा दझरेखा च संयोज्या । जबहत्रिभुजदबझत्रिभुजयोरपि भुजौ कोणौ मिथः समानौ भविष्यतः । वजहत्रिभुजस्य वदझत्रिभुजस्य च भुजौ कोणौ च मिथः समानौ भविष्यतः । यस्मिन् धरातले जदहझरेखे स्तस्तस्मिन् तबकरेखा बचिह्नगता कार्या । \\
  \vspace{-3mm}
  
  \noindent \includegraphics[scale=0.85]{Images/rg-108.png}
  \end{vwcol}
  \vspace{-2mm}
  
  \noindent पुनस्तवरेखा कवरेखा च संयोज्या  । बजतत्रिभुजे बदकत्रिभुजे बचिह्नसंपातसन्मुखकोणयोः साम्येन बजतकोणबदककोणयोः साम्येन च बजभुजबदभुजयोः साम्येनापि जतभुजतबभुजौ दकभुजकबभुजयोः समानौ भविष्यतः । वजतत्रिभुजे वदकत्रिभुजे वदवजभुजयोः समानभावित्वेन जतभुजदकभुजयो-

\newpage
\noindent रपि समानभावित्वेन\renewcommand{\thefootnote}{१}\footnote{{\en J. inserts} तथा {\en after} समानभावित्वेन. } बदककोणवजतकोणयोः समानभावित्वेन च वतभुजवकभुजौ समानौ भविष्यतः । वकबत्रिभुजे वतबत्रिभुजे च मिथो भुजयोः साम्येन वबतकोणवबककोणौ समानौ भविष्यतः । तस्मात् वबतकोणवबककोणौ समकोणौ भविष्यतः ।\\

 \renewcommand{\thefootnote}{२}\footnote{एवं तस्मिन्नेव. {\en J.}}अनेनैव प्रकारेण तस्मिन्नेव धरातले बचिह्नगता रेखा कल्प्यते । अबरेखया तस्याः संपातः समकोणो भविष्यति । तस्मात् अबरेखा तत्र धरातले लम्बो भविष्यति । इदमेवेष्टम्  ।।\\
 \begin{center}
\textbf{\large अथ पञ्चमं क्षेत्रम् ॥ ५ ॥}
\end{center}
\vspace{2mm}

{\ab यास्तिस्त्रो रेखा एकस्मिन् चिह्ने संपातं करिष्यन्ति तत्संपातचिह्नात् यो लम्बस्तिसृषु रेखासु पतति तदा ता रेखा एकधरातले भविष्यन्ति ।}\\

 यथा बजं बदं बहं रेखा \renewcommand{\thefootnote}{३}\footnote{बचिह्नसंपतिताः {\en K.,~A.,~J.}}बचिह्ने संपातकारिण्यः कल्पिताः ।
अबरेखा तिसृषु रेखासु लम्बः कल्पितः । यद्येता रेखा \renewcommand{\thefootnote}{४}\footnote{एकधरातले {\en J.}}एकस्मिन् धरातले न भवन्ति तदा यस्मिन् धरातले बजबहे रेखे स्तस्तदन्यत्र धरातले बदरेखा कल्प्या  । यस्मिन् धरातले अबबदरेखे स्तस्ते उभे धरातले \renewcommand{\thefootnote}{५}\footnote{{\en J. Omits} मिथः.}मिथः समानान्तरे न \renewcommand{\thefootnote}{६}\footnote{ स्याताम् {\en J.}}भवेताम् ।  \renewcommand{\thefootnote}{७}\footnote{{\en J.Omits} कुतः. }कुतः । 

\begin{center}
\includegraphics[scale=0.85]{Images/rg-109.png}
\end{center}
 बचिह्ने मिलितत्वात् । तदा बझरेखानयोः संपातरेखा कल्पिता । तस्मात् अबदअबझकोणौ प्रत्येकं समकोणौ भवतः । एकं च \renewcommand{\thefootnote}{८}\footnote{द्वितीयस्य खण्ड°~{\en V.,~J.}}द्वितीयखण्डमस्ति । इदमशुद्धम्~। अस्मदिष्टं समीचीनम् ॥\\
\begin{center}
\textbf{\large अथ षष्ठं क्षेत्रम् ॥ ६ ॥}
\end{center}
\vspace{2mm}

{\ab यौ द्वौ लम्बावेकस्मिन् धरातले भवतस्तौ मिथः समानान्तरौ भवतः ।}

\newpage

यथा अबं जदम् \renewcommand{\thefootnote}{१}\footnote{चैत्रक {\en J.}}एकत्र धरातले द्वौ लम्बौ कल्पितौ । पुनस्तस्मिन्नेव धरातले बदरेखा संयोज्या । अस्यां दहलम्बः कार्यः । अबलम्बे झचिह्नं \renewcommand{\thefootnote}{२}\footnote{कल्पितम् {\en J.}}कल्प्यम्~। दहरेखातो बझतुल्यं दवं पृथक्कार्यम्  । पुनर्झदझवबवरेखाः संयोज्याः  । झबदत्रिभुजे वदबत्रिभुजे झबदवभुजौ समानौ स्तः । बदभुजो द्वयोरेक एवास्ति  । झबदकोणवदबकोणौ \renewcommand{\thefootnote}{३}\footnote{समानौ {\en J.}}समकोणौ स्तः । झदभुजवबभुजौ समानौ भविष्यतः  । \renewcommand{\thefootnote}{४}\footnote{{\en V.omits} पुनर्.}पुनर्झवदत्रिभुजे झवबत्रिभुजे भुजयोः समान-
\begin{vwcol}[widths={0.65,0.35}, sep=.8cm, rule=0pt] 
भावित्वेन झबवकोणझदवकोणौ
समानौ भविष्यतः । झबवकोणः समकोणोऽस्ति । तस्मात् झदवकोणः
समकोणो भविष्यति  । तस्मात् दहरेखा दबदझदजरेखासु लम्बो भविष्यति । एतास्तिस्रो रेखा एकस्मिन् धरातले भविष्यन्ति~। बझअरेखा तस्मिन् धरातलेऽस्ति । तस्मात् अबजदे रेखे एकधरातले जाते । आभ्यां बदरेखया संपातः कृतः । संपाताभ्यन्तरकोणौ समकोणौ जातौ~। तस्मात् अबजदे समानान्तरे जाते ॥\\
\noindent \includegraphics[scale=1]{Images/rg-110.png}
\end{vwcol}

\begin{center}
\textbf{\large अथ सप्तमं क्षेत्रम् ॥ ७ ॥}
\end{center}

{\ab द्वाभ्यां रेखाभ्यां समानान्तराभ्यां यद्येकरेखा संपातं करोति तदेयं रेखा तयोर्द्वयोर्धरातले भविष्यति  ।}\\

यथा हझरेखया अबजदरेखयोः समानान्तरयोः संपातः कृतः ।
तदा हझरेखा अबजदयोर्धरातले भविष्यति ।\\
\begin{center}
\noindent \includegraphics[scale=0.9]{Images/rg-111.png}
\end{center}
\renewcommand{\thefootnote}{५}\footnote{यदि न भवति {\en J.}}यदि हझरेखा तयोर्धरातले न भवति तदा तयोर्धरातले हवझरेखा कल्प्या ।
तस्मात् दझरेखा हवझरेखे सरले वा मूलमिलिते जाते  । इदमशुद्धम्  । अस्मदिष्टं समीचीनम्~॥

\newpage
\begin{center}
\textbf{\large अथाष्टमं क्षेत्रम् ॥ ८ ॥}
\end{center}

{\ab द्वयोः समानान्तररेखयोरेका धरातले लम्बो भवति तदा
द्वितीया रेखापि तस्मिन्नेव धरातले लम्बो भवति  ।}\\

 यथा अबजदरेखयोः समानान्तरयोः अबं लम्बः कल्पितः । तदा
जदोऽपि लम्बो भविष्यति । धरातले बदरेखा  संयोज्या । बदरेखायां दहलम्बश्चानीतः\renewcommand{\thefootnote}{१}\footnote{{\en V} °श्च कार्य: {\en J.}} ।  अबरेखायां झचिह्नं कल्पितम् । बझतुल्यं दवं पृथक्कार्यम् । झदं झवं वबं रेखाः 
\begin{center}
\includegraphics[scale=0.7]{Images/rg-112.png}
\end{center}
संयोज्याः । उपरितनप्रकारेण निश्चितं वदझः समकोणो जातः । दहं दबदझयोः संबन्धिधरातले लम्बो भविष्यति  । अबजदयोर्धरातलेऽपि । तस्मात् जदं दहदबयोर्धरातले लम्बो भविष्यति । \renewcommand{\thefootnote}{२}\footnote{अबं यस्मिन् धरातले {\en \& K., A., J.}}अबमप्यस्मिन्{\en V} °श्च कार्य: {\en J.} धरातले लम्बोऽस्ति । तदा तस्मिन् धरातले जदमपि लम्बो भविष्यति । इदमेवेष्टम् ॥\\
\begin{center}
\textbf{\large अथ नवमं क्षेत्रम् ॥ ९ ॥}
\end{center}
\vspace{2mm}

{\ab \renewcommand{\thefootnote}{३}\footnote{एका रेखा वह्नीनां रेखानां समानान्तरा भवति ता रेखा एकधरातले न भवन्ति तदा {\en K.,A., J.} 
}एकया रेखया या बह्व्यो रेखाः समानान्तरा भवन्ति ताः सर्वा अपि मिथः समानान्तरा भविष्यन्ति ।}\\

 यथा जदं हझम् एते अबरेखातः समानान्तरे कल्पिते  । एतास्तिस्रोऽप्येकधरा-
 \vspace{-2mm}
 
  \begin{vwcol}[widths={0.6,0.4}, sep=.8cm, rule=0pt] 
तले न सन्ति  । वचिह्नात् वतवको द्वौ लम्बौ निष्कासितौ  । तस्मात् जतहकरेखे वतवकरेखयोर्धरातले लम्बौ भविष्यतः  । कुतः~। अबं तस्मिन् धरातले
लम्बोऽस्ति । तत एतौ समानान्तरौ भविष्यतः~। कुतः ।\\
\noindent \includegraphics[scale=0.7]{Images/rg-113.png}
\end{vwcol}
\vspace{-6mm}

\noindent  \renewcommand{\thefootnote}{४}\footnote{एतस्मिन्ने°{\en K., A., J.}}एकस्मिन्नेव
\noindent धरातले लम्बत्वात्  । इदमेवेष्टम् ॥

\newpage
\begin{center}
\textbf{\large अथ दशमं क्षेत्रम् ॥ १० ॥}
\end{center}
\vspace{2mm}

{\ab यदैककोणभुजौ तदन्यकोणभुजयोः \renewcommand{\thefootnote}{१}\footnote{ समानान्तरितौ {\en K., A, J.}}समानान्तरौ भवतः पुनरेतौ एकधरातले न भवतस्तदेतौ कोणौ समानौ भविष्यतः  ।}\\
\vspace{3mm}

 \begin{vwcol}[widths={0.7,0.3}, sep=.8cm, rule=0pt] 
 यथा बकोणहकोणौ कल्पितौ । बअभुजो दहभुजस्य समानान्तरः कल्प्यः  । बजभुजो हझभुजस्य समानान्तरः कल्प्यः  । पुनर्बअहदौ समानौ पृथक् पृथक् कृतौ  ।
एवं बजहझौ समानौ पृथक् कृतौ  । अजं दझम् अदं बहं जझं रेखाः संयोज्याः  । अदं \\
\vspace{-4mm}
\noindent \includegraphics[scale=0.9]{Images/rg-114.png}
\end{vwcol}
\vspace{-14mm}

\noindent जझं प्रत्येकं बहात् समानं \renewcommand{\thefootnote}{२}\footnote{समानान्तरितं {\en A., K., J.}}समानान्तरं चास्ति  ।\\
 एतावपि समानौ \renewcommand{\thefootnote}{३}\footnote{समानान्तरितौ {\en A., K., J.}}समानान्तरौ भविष्यतः  ।
  तदा\\ अजदझावपि समानौ
\renewcommand{\thefootnote}{३}\footnote{समानान्तरितौ {\en A., K., J.}}समानान्तरौ भविष्यतः  ।\\
तस्मात् अबजत्रिभुजदहझत्रिभुजयोर्भुजौ
मिथः\\ समानौ भविष्यतः । बकोणहकोणावपि समानौ भविष्यतः ।
\renewcommand{\thefootnote}{४}\footnote{इदमेवेष्टम् {\en J.} }इदमेवास्माकमिष्टम्~॥\\
\begin{center}
\textbf{\large अथैकादशं क्षेत्रम् ॥ ११ ॥}
\end{center}
\vspace{3mm}

{\ab एकस्मिन् धरातले पिण्डात् \renewcommand{\thefootnote}{५}\footnote{निष्काशन° {\en J.} }लम्बनिष्कासनमिष्टमस्ति  ।}\\
\vspace{3mm}

 यथा अचिह्नात् वजधरातले लम्बो निष्कासितव्यः । तत्र धरातले
बजरेखा कल्पिता । अचिह्नात् बजरेखायाम्
अदलम्बो \renewcommand{\thefootnote}{६}\footnote{निष्काश्य: {\en J.}}निष्कास्यः  । दचिह्नात्तस्मिन्नेव
धरातले दहलम्बो \renewcommand{\thefootnote}{६}\footnote{निष्काश्य:~{\en J.}}निष्कास्यः । अचिह्नात् 
\begin{vwcol}[widths={0.65,0.35}, sep=.8cm, rule=0pt] 
दहोपरि अझलम्बो निष्कास्यः  । अयं धरातले लम्बो भविष्यति  । कुतः  । झचिह्नात्
झवतरेखा तत्र धरातले बजसमानान्तरा कार्या  । तस्मात् बजरेखा अझदत्रिभुजस्य धरातले लम्बो भविष्यति  । तवमपि लम्बो भविष्यति  । तदा अझं धरातले लम्बो भविष्यति  । इदमेवेष्टम् ॥\\
\noindent \includegraphics[scale=0.8]{Images/rg-115.png}
\end{vwcol}
\vspace{2mm}

\newpage
\begin{center}
\textbf{अथ द्वादशं क्षेत्रम् ॥ १२ ॥}
\end{center}

{\ab तत्र धरातले तत्रत्येष्टचिह्नात् \renewcommand{\thefootnote}{१}\footnote{लम्बनिष्कासनं निरूप्यते {\en A. K.} लम्बनिष्काशनं निरूप्यते {\en J.} }लम्बो निष्कास्यः  ।}\\

 यथा अचिह्नात् अबधरातले लम्बः \renewcommand{\thefootnote}{२}\footnote{ कृतः {\en D., A, J.}}कार्यः । पुनरन्यस्मात्
\renewcommand{\thefootnote}{३}\footnote{{\en K., J. and J. have} पिण्डकल्पित°. 
}केल्पितचिह्नात् दबलम्बो धरातले \renewcommand{\thefootnote}{४}\footnote{निष्काश्यः {\en J.}}निष्कास्यः । 
\begin{center}
\includegraphics[scale=0.6]{Images/rg-116.png}
\end{center}
\renewcommand{\thefootnote}{५}\footnote{{\en J., A., and K. insert} यद्ययं लम्बः अचिह्ने पतितस्तदायं लम्बो जातः ।
यदि न पतति तदा {\en after} निष्कास्यः.}अचिह्नात् अजं बदस्य समानान्तरकार्यम् । \renewcommand{\thefootnote}{६}\footnote{इदमेवेष्टम् {\en J.} }इदमेवास्मदिष्टम् ॥\\
\begin{center}
\textbf{अथ त्रयोदशं क्षेत्रम् ॥ १३ ॥}
\end{center}

{\ab एकस्मिन् धरातले द्वौ लम्बौ एकचिह्ने न भवतः  ।}\\

\begin{vwcol}[widths={0.7,0.3}, sep=.8cm, rule=0pt] 
 यथा अबअजौ लम्बौ एकस्मिन् चिह्ने कल्पितौ  । पुनर्दहरेखा
अस्मिन् धरातले लम्बयोर्धरातले\\
\noindent \includegraphics[scale=0.62]{Images/rg-117.png}
\end{vwcol}
\vspace{-8mm}

 \noindent \renewcommand{\thefootnote}{७}\footnote{{\en A. and K.have} रेखा {\en in place of} संपातयोगरेखा.}संपातयोगरेखा कल्पिता । तस्मात् बअदकोण- \\
 जअदकोणौ समानौ भविष्यतः । इत्यशुद्धम् । \\
   अस्मदिष्टं समीचीनम् ॥
\begin{center}
\textbf{\large अथ चतुर्दशं क्षेत्रम् ॥ १४ ॥}
\end{center}

{\ab एका रेखा द्वयोर्धरातलयोर्यदि लम्बरूपा भवति तदा तौ धरातलौ समानान्तरौ भवतः  ।}\\

 यथा जदझतौ द्वौ धरातलौ कल्पितौ  । उभयोरुपरि अबं लम्बः कल्पितः  । 
 
\begin{vwcol}[widths={0.65,0.35}, sep=.8cm, rule=0pt] 
यदि समानान्तरौ न भवतस्तदा कल्पितं कलरेखायां द्वावपि मिलिष्यतः  । अस्य मचिन्हं कल्पितम् । पुनर्मअमबरेखे संयोज्ये  । अबमत्रिभुजे अकोणबकोणौ प्रत्येकं समकोणौ
भविष्यतः  । इदमशुद्धम्  । अस्मदिष्टं समीचीनम्~॥\\
\noindent \includegraphics[scale=0.65]{Images/rg-118.png}
\end{vwcol}
\newpage
\begin{center}
\textbf{\large अथ पञ्चदशं क्षेत्रम् ॥ १५ ॥}
\end{center}
\vspace{5mm}

{\ab यदि द्वयोर्धरातलयोरेकस्मिन् धरातले एकचिह्नात् निःसृते द्वे रेखे स्तस्तदा द्वितीयधरातले एकचिह्नादेव निःसृतरेखयोः समानान्तरे यदि भवतस्तदा ते धरातले अपि मिथः
समानान्तरे भविष्यतः ।}\\
\vspace{3mm}

 यथा बचिह्नहचिह्ने कल्पिते  । बअरेखा हदरेखायाः समानान्तरा 
 बजरेखा हझरेखायाः समानान्तरा कल्प्या  ।
पुनर्बचिह्नात् बवलम्बो हचिह्नस्य धरातले \renewcommand{\thefootnote}{१}\footnote{निष्काश्यः {\en J.}}निष्कास्यः~। पुनरस्मिन्नेव धरातले वतरेखा हदरेखायाः समानान्तरा \renewcommand{\thefootnote}{२}\footnote{निष्काश्या {\en J.}}निष्कास्या~। वकरेखा हझरेखायाः समानान्तरा \renewcommand{\thefootnote}{२}\footnote{निष्काश्या {\en J.}}निष्कास्या । वतवकरेखे बअबजरेखयोः
\begin{vwcol}[widths={0.7,0.3}, sep=.8cm, rule=0pt] 
 समानान्तरे भविष्यतः~। बवरेखा वतवकरेखयोर्लम्बोऽस्ति~। तस्मात् बअबजरेखयोरुपरि लम्बो भविष्यति~। तदा धरातलद्वयेऽपि लम्बो भविष्यति~। तदा द्वे
धरातले समानान्तरे भविष्यतः । इदमेवेष्टम्~॥\\
\noindent \includegraphics[scale=0.7]{Images/rg-119.png}
\end{vwcol}
\vspace{5mm}

\begin{center}
\textbf{\large अथ षोडशं क्षेत्रम् ॥ १६ ॥}
\end{center}
\vspace{5mm}

{\ab ये द्वे समानान्तरे धरातले एकधरातले संपातं कुरुतस्तदा द्वे संपातरेखे समानान्तरे भविष्यतः  ।}\\
\vspace{5mm}

 यथा अबजदधरातलहझवतधरातले द्वे समानान्तरे कलमनधरातले संपातं 
 
 \begin{vwcol}[widths={0.65,0.35}, sep=.8cm, rule=0pt] 
 कुरुत इति कल्पितम्  । तस्मात् कमसंपातरेखा लनसंपातरेखा एते द्वे समानान्तरे भविष्यतः  । यदि न भवतस्तदा सचिह्ने मिलिते कल्पिते~। यदि एते धरातले वर्द्धिते सचिह्ने मिलिष्यतः । इदमशुद्धम्  । अस्मदिष्टं समीचीनम्~॥\\
\noindent \includegraphics[scale=0.65]{Images/rg-120.png}
\end{vwcol}

\newpage
\begin{center}
\textbf{\large अथ सप्तदशं क्षेत्रम् ॥ १७ ॥}
\end{center}
\vspace{5mm}

{\ab यावन्ति धरातलानि समानान्तराणि द्वयो रेखयोः संपातं
कुर्वन्ति तानि रेखयोरेकनिष्पत्तौ संपातं करिष्यन्ति ॥}\\
\vspace{3mm}

यथा हझवतधरातलं कलमनधरातलं सगफछघरातलं चैतानि समानान्त-

 \begin{vwcol}[widths={0.7,0.3}, sep=.8cm, rule=0pt] 
राणि अबरेखाया असबचिह्नेषु
जदरेखाया जशदचिह्नेषु संपातं कुर्वन्तीति कल्पितानि  । पुनर्बजअजबदरेखा योज्याः  ।
बजरेखा कलमनधरातले तचिह्ने संपातं करोति । पुनस्तसरेखा तशरेखा संयोज्या  । तत्र
हवकमाभ्यां अबजत्रिभुजे अजतसरेखयोः संपातः कृतः~। तत्र अजतसरेखे समानान्तरे भविष्यतः~। एवं बदतशरेखे समानान्तरे भविष्यतः~। तस्मात् अससबनिष्पत्तिर्जततबनिष्पत्ति-\\
\noindent \includegraphics[scale=0.8]{Images/rg-121.png}
\end{vwcol}
\vspace{-9mm}

\noindent तुल्या जशशदनिष्पत्तितुल्या च भविष्यति । \renewcommand{\thefootnote}{१}\footnote{इदमेवेष्टम् {\en J.}\\
भा. १८
}इदमिष्टम् ॥\\
\begin{center}
\textbf{\large अथाष्टादशं क्षेत्रम् ॥ १८ ॥}
\end{center}
\vspace{5mm}

{\ab एकस्मिन् धरातले यो लम्बो भवति तत्संसक्तधरातलं तस्मिन् धरातले लम्बो भविष्यति ।}\\
\vspace{3mm}

 यथा अबम् एकस्मिन् धरातले लम्बोऽस्ति । अत्र एकं धरातलं
संलग्नम् । 
 \begin{vwcol}[widths={0.66,0.34}, sep=.8cm, rule=0pt]
उभयोर्धरातलयोर्जदसंपातरेखा उत्पन्ना । अत्र हचिह्नं कल्पितम् । हझलम्बो जदरेखायाः संलग्नधरातले कार्यः । अयं प्रथमधरातले लम्बो भविष्यति । या रेखा अस्मिन् धरातले हचिह्नात् निःसृतास्ताः सर्वा अपि प्रथमधरातले लम्बो भविष्यति । एवं यञ्चिह्नं जदरेखायां भवति  \\
\noindent \includegraphics[scale=0.7]{Images/rg-122.png}
\end{vwcol}
\vspace{-1mm}

\noindent तत्रैतादृशमेव भवति । तस्मात् द्वयोर्धरातलयोः संपातः समकोणो भविष्यति~॥

\newpage
\begin{center}
\textbf{\large अथैकोनविंशतितमं क्षेत्रम् ॥ १९ ॥}
\end{center}
\vspace{5mm}

{\ab द्वे धरातले मिथः संपातं कुरुत एकस्मिन् धरातले च
लम्बरूपे भवतः । अनयोः संपातरेखापि लम्बरूपा भविष्यति ।}\\
\vspace{5mm}

\begin{vwcol}[widths={0.7,0.3}, sep=.8cm, rule=0pt]
यथा अबजदधरातलं हझवतधरातलं च अनयोः संपातरेखा
कलरेखा कल्पिता । यस्मिन् धरातलद्वयं लम्बरूपमस्ति तस्मिन् धरातले यदि कलरेखा लम्बरूपा न भवति तदा लचिह्नात् लमलम्बः अज \\
\noindent \includegraphics[scale=0.7]{Images/rg-123.png}
\end{vwcol}
\vspace{-11mm}

\noindent धरातले  अदसंपातरेखायां\renewcommand{\thefootnote}{१}\footnote{निष्काश्यः {\en K., A., J.} }निष्कास्यः । लनलम्बश्च \\
तझधरातले झवसंपातरेखायां \renewcommand{\thefootnote}{२}\footnote{निष्काश्य: {\en K., A., J.} }निष्कास्यः । एते द्वे \\
लमलनरेखे तस्मिन् धरातले लम्बरूपे भविष्यतः ।\\ इदमशुद्धम् ।
\renewcommand{\thefootnote}{३}\footnote{इष्टं समी चीनम् {\en V.}}अस्मदिष्टं समीचीनम् ॥\\
\begin{center}
 \textbf{\large अथ विंशतितमं क्षेत्रम् ॥ २० ॥}
\end{center}
\vspace{5mm}

{\ab यदा त्रयो धरातलकोणा एकं घनकोणं वेष्टयन्ति तदा
कोणद्वययोगस्तृतीयकोणादधिको भवति ।}\\
\vspace{5mm}

 यथा अबजकोणः अबदकोणो जबदकोणो बधनकोणं वेष्टयन्ति । तदैते 
 \begin{vwcol}[widths={0.65,0.35}, sep=.8cm, rule=0pt]
त्रयः कोणा यदि समाना भवन्ति तदेष्टं प्रकटमेव~। यदि न्यूनाधिके स्तस्तदा अबदकोणः\\ \noindent प्रत्येकशेषकोणादधिको भवतीति कल्पितम्~। तत्र अबदकोणात् अबहकोणः अबजकोणतुल्यः पृथक्कार्यः । पुनर् अबभुजदबभुजयोरुपरि तचिह्नकचिह्ने कल्पिते ।  पुनस्तवकरेखा संयोज्या~। पुनर्बवतुल्यं बझं पृथक्कार्यम्~।\\
\noindent \includegraphics[scale=0.8]{Images/rg-124.png}
\end{vwcol}

    
\newpage

\noindent झत्रिभुजे तबवत्रिभुजे च तबभुज एक एवास्ति । झबभुजवबभुजौ समानौ स्तः~। द्वयोर्भुजयोरन्तर्गतकोणोऽपि समान एव । तदा \renewcommand{\thefootnote}{१}\footnote{तवतझौ.}तझतवौ समानौ भविष्यतः~। तझझकयोर्योगस्तकादधिकोऽस्ति~। तस्मात् झकं वकादधिकं भविष्यति~। तस्मात् झबककोणो वबककोणादधिको भविष्यति । तस्मात् अबजकोणदबजकोणयोर्योगः अबदकोणादधिको भविष्यति~। इदमेवेष्टम् ॥\\
\begin{center}
\textbf{\large अथैकविंशतितमं क्षेत्रम् ॥ २१ ॥}
\end{center}
\vspace{2mm}

{\ab घनकोणं यावन्ति धरातलानि वेष्टयन्ति तेषां योगश्चतुःसमकोणान्न्यूनो भवति ।}\\

 यथा \renewcommand{\thefootnote}{२}\footnote{बघनकोणो {\en K., J.} }बधनकोणं झबहकोणहबवकोणझबवकोणा\renewcommand{\thefootnote}{३}\footnote{° णैर्वेष्टितमस्ति । {\en K., J.}} वेष्टितं कुर्वन्ति । पुनर्हझझवहवरेखाः संयोज्याः । पुनर्हझवत्रिभुजे तचिह्नं कल्पितम् । हतझतवतरेखाः संयोज्याः । सर्वे नवकोणा
हतझत्रिभुजहतवत्रिभुजझतवत्रिभुजेषु \renewcommand{\thefootnote}{४}\footnote{तेषां नवकोणानां {\en V.} }नवकोणानां तेषां योगः षट्समकोणतुल्योऽस्ति । 
\begin{center}
\includegraphics[scale=1.4]{Images/rg-125.png}
\end{center}

तेषु नवकोणेषु द्वौ कोणौ हचिह्ने द्वौ झचिह्ने द्वौ
वचिह्ने स्तस्तेषां योगो हझवत्रिभुजस्य षट्कोणा \renewcommand{\thefootnote}{५}\footnote{भविष्यन्ति {\en V.}}भवन्ति ते च
द्विसमकोणतुल्या भविष्यन्ति । तस्मात् तचिह्नस्य त्रयः कोणाश्चतुःसमकोणतुल्या
\renewcommand{\thefootnote}{६}\footnote{भविष्यन्ति {\en V.} }भवन्ति । षट्कोणा हबझत्रिभुजहबवत्रिभुजझबवत्रिभुजानां तादृशा हचिह्नझचिह्नवचिह्नेभ्यो भवन्ति । तेषां योगः प्रथमषट्कोणयोगादधिको भविष्यति । तस्मात् बचिह्नस्य त्रयः कोणास्तचिह्नकोणत्रयेभ्यो न्यूना भविष्यन्ति । \renewcommand{\thefootnote}{७}\footnote{तचिह्नं च चतुःसमकोणतुल्यमस्ति । तस्मात् वचिह्नं चतुःसमकोणान्न्यूनं जातम् ।
इदमेवेष्टम् । {\en K., A., \& J. in place of the last part.}}तस्मात् चतुर्भ्यः समकोणेभ्यो न्यूना भविष्यन्ति~। इदमेवेष्टम् ॥

\newpage
\begin{center}
\textbf{\large अथ द्वाविंशतितमं क्षेत्रम् ॥ २२ ॥}
\end{center}
\vspace{5mm}

{\ab यदि त्रयो धरातलकोणाः\renewcommand{\thefootnote}{१}\footnote{समकोणाः समभुजा {\en J..}} समानभुजा भवन्ति तेषां प्रत्येकद्वययोगस्तृतीयादधिकोस्ति\renewcommand{\thefootnote}{२}\footnote{° दधिको भवति तदा {\en V.}} चेत् तदा तत्कोणसम्मुखभुजेभ्यस्त्रिभुजो भवितुमर्हति तत्र भुजद्वययोगो तृतीयभुजादधिको भविष्यति ।}\\
\vspace{3mm}

 यथा बहतास्त्रयो धरातलकोणाः कल्पिताः । बअबजहदहझतवतकाः समानभुजाः कल्पिताः । पुनर् अजदझवकतत्कोणसन्मुखभुजाः कल्पिताः । यदि सन्मुखभुजा मिथः समाना भवन्ति तदा भुजद्वययोगस्तृतीयभुजादधिको भविष्यति । यदि न्यूनाधिकास्तदा वकम् अधिकं कल्पितम् । जबरेखातो बचिह्ने जबलकोणो हकोण-
 \begin{center}
 \includegraphics[scale=1.2]{Images/rg-126.png}
 \end{center}
\noindent तुल्यः कार्यः । पुनर्बमं बजतुल्यं पृथक्कार्यम् । पुनर्जमअमरेखे संयोज्ये । तस्मात् जमभुजो दझभुजतुल्यो भविष्यति । अजजमयोगोऽसादधिकोऽस्ति । अमं वकादधिकमस्ति । कुतः । अबमकोणो बकोणहकोणयोगतुल्य स्तकोणादधिकोऽस्ति~। भुजाश्च मिथः समानाः सन्ति । तस्मात् अजजमयोगो वकादधिको भविष्यति~। इदमेवेष्टम् ॥\\
\begin{center}
\textbf{अथ त्रयोविंशतितमं क्षेत्रम् ॥ २३ ॥}
\end{center}
\vspace{5mm}

{\ab तादृशत्रयधरातलकोणेभ्यः पृथक् घनकोणचिकीर्षास्ति येषां धरातलकोणानां योगश्चतुर्भ्यः समकोणेभ्यो न्यूनः स्यात् प्रत्येककोणद्वययोगस्तृतीयकोणादधिकः स्यात् ।}

\newpage
यथा अहतत्रयो धरातलकोणाः कल्पिताः । एषां भुजाः समानाः
कार्याः ।

 \begin{vwcol}[widths={0.65,0.35}, sep=.8cm, rule=0pt]
ते अबअजदहहझतवतकाः कल्पिताः । पुनरेतत्कोणसन्मुखभुजेभ्यो बजदझवकसंज्ञेभ्य एकं त्रिभुजं कार्यम् । तत्त्रिभुजं लमनं कल्पितम्~। तत्र लमभुजो बजतुल्यो मनभुजो दझभुजतुल्यो लनभुजो वकभुजतुल्यश्च कल्पितः। पुनरस्मिन् त्रिभुजे लमनवृत्तं कार्यम्~। अस्य केन्द्रं सचिह्नं कल्पितम् । पुनः सलसमसनरेखाः संयोज्याः ।\\
 \includegraphics[scale=1.1]{Images/rg-127.png}
 \end{vwcol}
 \vspace{-7mm}
 
 \noindent बजं लमतुल्यमस्ति । बअभुजजअभुजौ लसभुजसमभुजतुल्यौ भविष्यतो वा न्यूनौ वाऽधिको भविष्यतः । यदि समानौ स्तस्तदा अकोणो लसमकोण-
\begin{center}
\includegraphics[scale=1]{Images/rg-128.png}
\end{center}
तुल्यो भविष्यति । एवं हकोणो मसनकोणतुल्यो भविष्यति । तकोणश्च नसलकोणतुल्यो भविष्यति~। तदा त्रयाणां कोणानां योगः सकोणत्रयतुल्यो भविष्यति~। तदा चतुर्भिः समकोणैस्तुल्यो भविष्यति । कल्पितं च कोणत्रययोगश्चतुर्यः समकोणेभ्यो न्यूनोऽस्ति । इदमनुपपन्नम् ॥\\
\vspace{5mm}

पुनर्यदि बअभुजजअभुजौ लसभुजसमभुजयोर्न्यूनौ स्तो बजभुजो लमभुजे स्थाप्यस्तदा अकोणो लसमत्रिभुजान्तः पतिष्यति । तस्मात् अकोणो लसमकोणादधिको भविष्यति । एवं हकोणो मसनकोणादधिको भविष्यति । तकोणो नसलकोणादधिको भविष्यति । तस्मात् त्रयाणां कोणानां योगः समकोणचतुष्टयादधिको भविष्यति ।

\newpage
\noindent तस्मात् \renewcommand{\thefootnote}{१}\footnote{प्रत्येकं {\en A.} }प्रत्येककोणानां भुजो व्यासार्द्धादधिको भविष्यति । पुनः सचिह्नात् सफलम्बो वृत्ते शङ्कुवत् कल्प्यः । पुनस्तस्मात् लम्बात् सगं \renewcommand{\thefootnote}{२}\footnote{तथा पृथक्कार्यं यथास्य वर्ग: {\en A., K., J.}}तादृशरेखायास्तुल्यं पृथक्कार्यं यस्या वर्गो \renewcommand{\thefootnote}{३}\footnote{अबवर्गलसवर्गयोर्योगतुल्यो भवति {\en A., K., J.}}लसवर्गयुतः
अबवर्गतुल्यो भवेत् पुनर्गलगमगनरेखाः संयोज्याः । तस्मात् गधनकोण इष्टो भविष्यति । कुतः । \renewcommand{\thefootnote}{४}\footnote{गकोणत्रयाणां तिस्रो भुजाः कल्पितधरातलकोणत्रयसन्मुखभुजैः समानाः । {\en A., J., and K. in place of the sentence marked.} }यतस्त्रयः कोणा ये घनकोणसमाश्लिष्टास्तेषां भुजा इष्टानां त्रयाणां कोणानां भुजैः समानाः सन्ति\renewcommand{\thefootnote}{५}\footnote{गकोणत्रयाणां तिस्रो भुजाः कल्पितधरातलकोणत्रयसन्मुखभुजैः समानाः । {\en A., J., and K. in place of the sentence marked.} }~। एतत्रयाणां
सन्मुखभुजाश्च इष्टकोणत्रयसन्मुखभुजसमानाः सन्ति । तस्मादेते\renewcommand{\thefootnote}{६}\footnote{{\en J. omits} एते.} त्रयः
कोणा इष्टकोणत्रयसमाना भविष्यन्ति । इदमेवेष्टम् ।\\
\vspace{5mm}

 अथ च अकोणो लसमत्रिभुजान्तः कुतः पतति । यतः प्रत्येकं लसभुजमसभुजयोर्बअभुजतुल्यजअभुजतुल्यं पृथक्क्रियते । पुनर्लचिन्हमचिन्हं केन्द्रं कृत्वा बअतुल्यजअतुल्यव्यासार्धं कृत्वा वृत्तद्वयं कार्यम् । एते द्वे वृत्ते त्रिभुजान्तः संपातं करिष्यतः ।
यदि त्रिभुजान्तः संपातं न करिष्यतस्तदा लमभुजतुल्यबजभुजो
बअभुजजअभुजयोगान्न्यूनो न भविष्यति । इदमशुद्धम् । \\
\vspace{5mm}

यदि वृत्तसंपातचिह्ने लचिह्नमचिह्ने च रेखे संयोज्येते तदा बअजत्रिभुजतुल्यं लसनत्रिभुजान्तरेकं त्रिभुजमुत्पन्नं भविष्यति ।
\renewcommand{\thefootnote}{७}\footnote{तस्मादुत्पन्नत्रिभुजमस्तककोणसन्मुखभुजोत्पन्नौ द्वौ कोणौ लकोणमकोणयोर्न्यूनौ भविष्यतो
मस्तककोणः सकोणादधिको भविष्यति~। {\en K., A.}
}तस्मादुत्पन्नत्रिभुजमस्तककोणः सकोणादधिको भविष्यति । मस्तककोणसन्मुखभुजोत्पन्नौ द्वौ कोणौ लकोणमकोणयोर्न्न्यूनौ भविष्यतः ॥\\
\begin{center}
\textbf{\large अथ चतुर्विंशतितमं क्षेत्रम् ॥ २४ ॥}
\end{center}
\vspace{5mm}

{\ab समानान्तरधरातलघनक्षेत्रसन्मुखधरातलानि समानभुजानि भवन्ति ।}\\
\vspace{3mm}

यथा धनक्षेत्रम् अवं कल्पितम् । अजहदधरातलबझवतधरातले

\newpage

\noindent सन्मुखधरातले कल्पिते । अनयोर्भुजाः समाना भविष्यन्ति । \renewcommand{\thefootnote}{१}\footnote{यतः {\en K., A.}}कुतः । \renewcommand{\thefootnote}{२}\footnote{धरातलं झजअबधरातलबहदतधरातलयोः समानान्त रालयोः समानान्तरेणेदं पतितमस्ति । {\en A., K., J.} }अजहदधरातले 
 \begin{vwcol}[widths={0.7,0.3}, sep=.8cm, rule=0pt]
झजअबधरातलवहदतधरातले च समानान्तरिते पतिते स्तः । एवं झवहजधरातलबतदअधरातलेपतिते स्तः । तदा जअसंपातरेखाहदसंपातरेखे समानान्तरे भविष्यतः । अनेनैव प्रकारेण जहसंपातरेखाअदसंपातरेखे मिथः समानान्तरे भविष्यतः~। एवं झवबतसंपातौ समानान्तरौ भविष्यतः~। एवं झबवतसंपातौ समानान्तरौ भविष्यतः । तस्मात् अजहदधरातलबझवतधरातले च समानान्तरसमानभुजे भविष्यतः । इदमेवेष्टम् ॥\\
\noindent \includegraphics[scale=1.1]{Images/rg-129.png}
\end{vwcol}

\begin{center}
\textbf{\large अथ पञ्चविंशतितमं क्षेत्रम् ॥ २५ ॥}
\end{center}

{\ab समानान्तरधरातलस्य घनक्षेत्रस्य मिथः सन्मुखधरातलयोर्मध्यगतसमानान्तरं धरातलं भागद्वयं चेत् करोति तदा अनयोः \renewcommand{\thefootnote}{३}\footnote{{\en K., A., and J. have} भूम्योः {\en instead of} धरातलखण्डयो :. }खण्डयोनिष्पत्तिर्धरातलखण्डयोर्निष्पत्ति-समाना
भविष्यति ।}\\

 यथा अबं घनक्षेत्रं कल्पितम् । अस्य वतअकधरातलबलमनसन्मुखधरातलयोः समानान्तरधरातलेन जदहझेन खण्डद्वयं \renewcommand{\thefootnote}{४}\footnote{कृतमस्तीति~{\en J.}}कृतमिति कल्पितम् । तत्र
 \begin{center}
 \noindent \includegraphics[scale=1]{Images/rg-130.png}
 \end{center}
अजखण्डहबखण्डयोर्निष्पत्तिः \renewcommand{\thefootnote}{५}\footnote{अजधरातलखण्ड योर्निष्पत्त्या तुल्या भविष्यति । {\en K., A,}}अझधरातलखण्डनहधरातलखण्डयोर्निष्पत्तितुल्या भविष्यति~।
\begin{center}
\textbf{ अस्योपपत्तिः । }
\end{center}

अमभुज उभयदिशि सगपर्यन्तं वर्द्धनीयः । हअदिशायां अफं फछं हअतुल्यं पृथक्कार्यम् । हमदिशायां मखं खरं हमतुल्यं पृथक्कार्यम् । क्षेत्रं \renewcommand{\thefootnote}{६}\footnote{पूर्णं {\en J., V.}}संपूर्णं 

\newpage
\noindent कार्यम् । यदि संपूर्णं छझम् अझयावद्धातरूपं हनयावद्धातरूपस्य रझस्य समानं भवति तदा छजं घनक्षेत्रं अजघनक्षेत्रयावद्धातरूपं हबघनक्षेत्रयावद्धातरूपेण जरघनक्षेत्रेण समानं भविष्यति । यदि छझं रझान्न्यूनं भवति तदा छजं घनक्षेत्रं जरघनक्षेत्रान्न्यूनं भविष्यति । यदि अधिकं स्यात्तदा इदमप्यधिकं भवति~। तस्मात् अझनहधरातलखण्डयोर्निष्पत्तिः अजहबघनक्षेत्रखण्डयोर्निष्पत्त्योः समाना भविष्यति~। इदमेवेष्टम् ॥\\
\vspace{3mm}

\begin{center}
\textbf{\large अथ षड्विंशतितमं क्षेत्रम् ॥ २६ ॥}
\end{center}
\vspace{5mm}

{\ab एकरेखैकचिह्नोपरि घनक्षेत्रकोणतुल्यकोणचिकीर्षास्ति ।}\\
\vspace{3mm}

यथा अबरेखातः अचिह्ने तादृशो दघनक्षेत्रकोणतुल्यः\renewcommand{\thefootnote}{१}\footnote{तुल्यकोणचिकीर्षास्ति {\en K.}} कोणः
कर्तव्योऽस्ति~। 
\begin{vwcol}[widths={0.55,0.45}, sep=.8cm, rule=0pt]
यथा जदहं जदझं हदझं धरातलकोणाः वेष्ट यन्ति । तत्र दहरेखायां वचिह्नं
कल्पितम् । पुनर्वचिह्नात् जदझकोणधरातले वतलम्बो निष्कास्यः ।
पुनस्तदरेखा योज्या~। पुनर्बअरेखाया अचिह्ने बअलकोणबअमकोणौ जदझकोणजदतकोणतुल्यौ
कार्यौ । पुनरमरेखाया दततुल्यम् अनं पृथक्कार्यम् । पुनर्नचिह्नात्
नसलम्बो बअलकोणधरातले निष्कास्यः~।
\noindent \includegraphics[scale=1.1]{Images/rg-131.png}
\end{vwcol}
\vspace{-2mm}

\noindent  पुनरस्माल्लम्बात्तवतुल्यं नगं पृथक्कार्यम् । पुनर्गअरेखा संयोज्या । तस्मात् अघनक्षेत्रकोणः अस्माकमिष्टो\renewcommand{\thefootnote}{२}\footnote{अस्मदिष्टो {\en K., A.}} भविष्यति ॥\\
\begin{center}
\textbf{अस्योपपत्तिः ।}
\end{center}
\vspace{5mm}

दजरेखायां कचिह्नं कल्पनीयम् । पुनर्वकरेखा कतरेखा
संयोज्या

\newpage
\noindent पुनर् अबरेखातो दकतुल्यम् अफं पृथक्कार्यम् । पुनर्गफनफरेखे
संयोज्ये । अनं दततुल्यं नगं वततुल्यमस्ति । अनगकोणदतवकोणौ प्रत्येकं समकोणौ स्तः । तस्मात् अगं दवसमानं भविष्यति । पुनरपि वअमकोणजदतकोणौ समानौ कृतौ स्तः । फअभुजअनभुजौ कदभुजदतभुजयोः समानौ स्तः । फनभुजः कतभुजेन समानो भविष्यति~। नगभुजतवभुजौ पूर्व समानावास्ताम् । फनगकोणकतवकोणौ प्रत्येकं समकोणौ स्तः । तस्मात् फणभुजः कवभुजेन समानो जातः~। फअभुजअगभुजौ कदभुजदवभुजयोः समानावास्ताम् । तस्मात् फअगकोणकदवकोणौ समानौ भविष्यतः~। एवं निश्चीयते गअलकोणवदझकोणौ समानौ भविष्यतः । बअलकोणजदझकोणौ समानौ कृतावास्तां । तस्मात् त्रयो धरातलकोणा अघनकोणसंलग्ना दघनक्षेत्रकोणवेष्टकानां त्रयाणां धरातलकोणानां समाना भविष्यन्ति ।
पुनर् अघनकोणो दघनकोणेन समानो भविष्यति । इदमेवेष्टम् ॥\\
\vspace{5mm}

\begin{center}
\textbf{\large अथ सप्तविंशतितमं क्षेत्रम् ॥ २७ ॥}
\end{center}
\vspace{5mm}

{\ab एकरेखायां समानान्तरघनक्षेत्रसजातीयघनक्षेत्रस्य चिकीर्षास्ति ।}\\
\vspace{3mm}

 यथा अबरेखायां \renewcommand{\thefootnote}{१}\footnote{जदघनक्षेत्रसजातीयघनक्षेत्रं कृतम् । {\en K., A.}}जदसमानान्तरधरातलघनक्षेत्र\renewcommand{\thefootnote}{२}\footnote{२ समानजातीय°{\en D,}\\
   मा० १९}सजातीयधनक्षेत्रं कर्त्तव्यमस्ति । 
\begin{center}
 \includegraphics[scale=1.1]{Images/rg-132.png}   
\end{center}
पुनर् अचिह्ने जकोणतुल्यो घनकोणः कार्यः । पुनर्जझजवनिष्पत्तितुल्या अबअकयोर्निष्पत्तिः कार्या । जझजहयोर्निष्पत्तितुल्या अबअतयोर्निष्पत्तिः कार्या । पुनस्तबधरातलं पूर्णं कार्यम् । तचिह्नबचिह्नमचिह्नेभ्यः तफरेखामलरेखाबसरेखा अकरे--

\newpage
\noindent खया तुल्याः समानान्तराश्च कार्याः । पुनः फकफलकसलसरेखाः
संयोज्याः । तस्मात् घनक्षेत्रमिष्टं संपूर्णं भविष्यति । इष्टघनक्षेत्रसजातीयं च भविष्यति । इदमेवेष्टम् ॥ \\
\begin{center}
\textbf{\large अथाष्टाविंशतितमं क्षेत्रम् ॥ २८ ॥}
\end{center}
\vspace{5mm}

{\ab समानान्तरधरातलघनक्षेत्रस्य मिथः \renewcommand{\thefootnote}{१}\footnote{सन्मुखकर्णगतसन्मुखधरातलं {\en K., A, }}सन्मुखधरातलयोः कर्णगतधरातलमर्द्धं करोति । तच्छेदितक्षेत्रद्वयमुत्पादयति च ।}\\
\vspace{3mm}

 यथा अबधनक्षेत्रम् । तअवबसन्मुखधरातलयोर्जदकर्णहझकर्णगतजदहझधरातलेन खण्डद्वयं कृतम् । अतो जाते छेदितक्षेत्रे समाने भविष्यतः ।\\
 \begin{center}
\textbf{अस्योपपत्तिः ।}
\end{center}
\vspace{2mm}

अस्मिन् छेदितक्षेत्रे घनक्षेत्रसन्मुखधरातलानि वेष्टितानि सन्ति । सन्मुखभूत-
\vspace{-2mm}

\begin{vwcol}[widths={0.65,0.35}, sep=.8cm, rule=0pt]
लानि मिथः समानानि सन्ति । कर्णगतधरातलं द्वयोरेकमेवास्ति । त्रिभुजेऽपि
समाने स्तः~। कुतः~। ये धरातले कर्णगतधरातलेनार्द्धिते स्तस्तेषामेते त्रिभुजे अर्द्धरूपे स्तः~। तस्मात् उभे क्षेत्रे समाने स्तः । इदमेवेष्टम् ॥\\
अनेनेदं निश्चितं छेदितक्षेत्रं यदि समानान्तरधरातलपूर्णं क्रियते तदा छेदितघनक्षेत्रं संपूर्णघनक्षेत्रस्यार्द्धं भवति ॥\\
\vspace{5mm}

\noindent \includegraphics[scale=1.3]{Images/rg-133.png}   
\end{vwcol}
\vspace{-2mm}

\begin{center}
\textbf{\large अथैकोनत्रिंशत्तमं क्षेत्रम् ॥ २९ ॥}
\end{center}
\vspace{2mm}

{\ab एकस्मिन् धरातले समानान्तरधरातलघनक्षेत्राणि \renewcommand{\thefootnote}{२}\footnote{{\en D. omits} मुखरेखान्तर्गतानि,}मुखरेखान्तर्गतानि यावन्ति सन्ति तेषां लम्बाश्चेत्समाना भवन्ति तानि घनक्षेत्राणि समानानि भवन्ति ।}\\
\vspace{3mm}

 यथा बहबझे द्वे घनक्षेत्रे अबजद धरातलोपरि कल्पिते । वझ--

\newpage
\begin{vwcol}[widths={0.6,0.4}, sep=.8cm, rule=0pt]
\noindent रेखाकनरेखयोरन्तरे कल्पिते । अनयोर्लम्बौ यदि समानौ भवतस्तदैते घनक्षेत्रे समाने भविष्यतः ।\\
\noindent \includegraphics[scale=0.8]{Images/rg-134.png}   
\end{vwcol}
\vspace{5mm}

\begin{center}
\textbf{अस्योपपत्तिः ।}
\end{center}
\vspace{2mm}

अलच्छेदितघनक्षेत्रं दनच्छेदितघनक्षेत्रं च समानमस्ति । कुतः । अवतत्रिभुजदहझत्रिभुजयोः समानत्वात् । बकलत्रिभुजजमनत्रिभुजे च समाने स्तः । वकलतधरातलं हमनझधरातलं च समानमस्ति ।
\renewcommand{\thefootnote}{१}\footnote{\en Omitted in K., A.}अबकवधरातलं दजमहधरातलं च समानमस्ति । \renewcommand{\thefootnote}{२}\footnote{\en Omitted in K., A.}अबलतधरातलं दझनजधरातलं च समानम् । एतयोः शेषं छेदितघनक्षेत्रे योज्यते । तदा द्वे घनक्षेत्रे मिथः समाने भविष्यतः । इदमेवेष्टम् ॥\\
\begin{center}
\textbf{\large अथ त्रिंशत्तमं क्षेत्रम् ॥ ३० ॥}
\end{center}
\vspace{2mm}

{\ab एकस्मिन् धरातले यावन्ति समानान्तरधरातलानि धन क्षेत्राणि भवन्ति समानलम्बानि च रेखाद्वयान्तर्गतानि न भवन्ति तदैतान्यपि समानानि भवन्ति~।}\\
\vspace{3mm}

\begin{vwcol}[widths={0.6,0.4}, sep=.8cm, rule=0pt]
 यथा बहबझे द्वे घनक्षेत्रे अबजदधरातले कल्पिते । एकस्य मुखं
लहं द्वितीयस्य मुखं सझं कल्पितम् । अनयोर्लम्बौ समानौ स्तः~। तदैतौ समानौ भविष्यतः ।\\
\noindent \includegraphics[scale=0.9]{Images/rg-135.png}   
\end{vwcol}
\vspace{5mm}

\begin{center}
\textbf{ अस्योपपत्तिः ।}
\end{center}
\vspace{2mm}

कसरेखा नचिह्नपर्यन्तं वर्द्धनीया लतरेखा च मचिह्नपर्यन्तं वर्द्धनीया । गहरेखा
वचिह्नपर्यन्तं वर्द्धनीया । पुनर् अमबनदवजफरेखाः संयोज्याः । तदा बवं घनक्षेत्रमुत्पन्नं भविष्यति । अस्य मुखं नवमस्ति । इदं
घनक्षेत्र--

\newpage
\noindent मिष्टक्षेत्रद्वयेन सार्द्धमेकस्मिन् धरातलेऽस्ति । द्वयो
रेखयोरन्तर्गतमस्ति । इदमुत्पन्नं घनक्षेत्रं प्रत्येकं घनक्षेत्रेण समानं भविष्यति ।
इदमेवेष्टम् ॥\\
\vspace{3mm}

\begin{center}
\textbf{\large अथैकत्रिंशत्तमं क्षेत्रम् ॥ ३१ ॥}
\end{center}
\vspace{5mm}

{\ab समानान्तरधरातलघनक्षेत्राणि चेत् समानधरातले भवन्ति समानलम्बानि चेद्भवन्ति निजधरातले लम्बरूपाणि भवन्ति तदा समानानि भवन्ति ।}\\
\vspace{3mm}

 यथा बकझले द्वे घनक्षेत्रे अबजदघरातले हझवतधरातले स्तः । झवरेखा सचिह्नपर्यन्तं वर्द्धनीया । अदतुल्यं वसं पृथक्कार्यम् । वचिह्नोपरि सवगकोणो दअबकोणतुल्यः कार्यः । अबतुल्यं वफं पृथक्कार्यम् । वतअनौ समानलम्बौ दअबघरातले सवगधरातले स्तः । तस्मात् वकोणअकोणौ घनकोणौ समानौ भविष्यतः । 
 \begin{center}
 \includegraphics[scale=1.1]{Images/rg-136.png}   
\end{center} 

 पुनः फसघनक्षेत्रं संपूर्णं कार्यम् । इदं बकघनक्षेत्रतुल्यं
भविष्यति । पुनः सचिह्नात् समरेखा तवरेखायाः समानान्तरा कार्या । हतं
तथा वर्द्धनीयं यथा मचिह्ने मिलति~। तवं तथा वर्द्धनीयं यथा
खचिह्ने मिलति । पुनर्वशखसे घनक्षेत्रे पूर्णे कार्ये । तदा खसफसघनक्षेत्रे समाने भविष्यतः । तस्मात् खसबकघनक्षेत्रे समाने भविष्यतः । झलखसनिष्पत्तिर्वशेन तथास्ति यथा झतखसयोर्निष्पत्तिर्वमेनास्ति । खसफसौ समानौ स्तः । तस्मात् \renewcommand{\thefootnote}{१}\footnote{झलसफयोर्झलबकतुल्ययोर्निष्पत्तिर्वशेन तथास्ति यथा
झलफसघरातलयोर्झलबकधरातलयोस्तुल्ययोर्निष्पत्तिर्वश° {\en J.}}झलफसतुल्यघनक्षेत्रयोर्निष्पत्तिर्झलबकयोर्-
निष्पत्तिरपि वशेन तथास्ति यथा झलफसतुल्यधरात--

\newpage
\noindent लयोर्निष्पत्तिर्झलबकधरातलयोरपि निष्पत्तिर्वशघरातलेनास्ति ।
तदैते घनक्षेत्रे समाने भविष्यतः । इदमेवेष्टम् ॥\\
\begin{center}
\textbf{\large अथ द्वात्रिंशत्तमं क्षेत्रम् ॥ ३२ ॥}
\end{center}
\vspace{2mm}

{\ab समानान्तरघरातलघनक्षेत्राणि समानधरातले चेद्भवन्ति
\renewcommand{\thefootnote}{१}\footnote{\en Omitted in A., and K.}पिण्डाश्च तद्धरातले लम्बरूपा न भवन्ति लम्बाश्च तुल्या भवन्ति तदैतानि समानानि भवन्ति ।}\\
\vspace{3mm}

यथा \renewcommand{\thefootnote}{२}\footnote{{\en A. K. and J. have} झ {\en in place of} र {\en althrough.}}बकरखे बदरतधरातले कल्पिते\renewcommand{\thefootnote}{३}\footnote{{\en A., K. and D. insert} कुतः {\en after}
कल्पिते; {\en V. has also} कुतः {\en on the margin.} } । यदि असबगजफदछलम्बा 
\begin{center}
\includegraphics[scale=1]{Images/rg-137.png}  
\end{center}
बदभूतलात् मके भूतले चेत् निष्कास्या\renewcommand{\thefootnote}{४}\footnote{निष्काश्यन्ते {\en J.}} हसरखवझतछलम्बाः शखे भूतले च \renewcommand{\thefootnote}{४}\footnote{निष्काश्यन्ते~{\en J.}}निष्कास्या उभे क्षेत्रे पूर्णे कार्ये । तदा बकबछे समाने भविष्यतः । एवं हि रखरछे समाने भविष्यतः । बछरछे समाने आस्ताम् । तस्मात् बकरखे अपि समाने भविष्यतः । इदमिष्टम्~॥\\
\begin{center}
\textbf{\large अथ त्रयस्त्रिंशत्तमं क्षेत्रम् ॥ ३३ ॥}
\end{center}
\vspace{5mm}

 {\ab समानान्तरधरातलघनक्षेत्राणां यदि लम्बाः समाना भवन्ति तदा तेषां निष्पत्तिर्धरातलनिष्पत्तितुल्या भवति ।}\\
\vspace{3mm}

 यथा बकझलघनक्षेत्रयोर्बदझते उभे धरातले कल्पिते । पुनर्जदरेखोपरि झतघरातलतुल्यजनधरातलं कार्यम् । अदनं संपूर्णा सरलैकरेखा भवति । पुनर्जसं घनक्षेत्रं पूर्णं कार्यम् । \renewcommand{\thefootnote}{५}\footnote{तस्मात् {\en V.}}यदि जसघन-

\newpage
\begin{center}
\includegraphics[scale=1]{Images/rg-138.png}  
\end{center}
क्षेत्रे बकघनक्षेत्रे समानलम्बे समानधरातले च भवतः \renewcommand{\thefootnote}{१}\footnote{तस्मात् {\en V.}}तदा
जसघनक्षेत्रं झलघनक्षेत्रेण समानं भविष्यति । कुतः । घरातलयोर्लम्बयोश्च
साम्यात् । जसघनक्षेत्रबकघनक्षेत्रयोर्निष्पत्तिर्धरातलयोर्निष्पत्तितुल्या जाता । इदमेवेष्टम् ॥\\
\begin{center}
\textbf{\large अथ चतुस्त्रिंशत्तमं क्षेत्रम् ॥ ३४ ॥}
\end{center}
\vspace{5mm}

{\ab मानान्तरधरातलघनक्षेत्रयोः पिण्डौ स्वस्वधरातलयोर्लम्बरूपौ यदि भवतो घनक्षेत्रे समाने च भवतस्तदा धरातलयोर्निष्पत्तिर्लम्बयोर्विलोमनिष्पत्तितुल्या भवति यदि तयोरेतद्रूपा\renewcommand{\thefootnote}{२}\footnote{ईदृशी {\en K. and A.}} निष्पत्तिः स्यात्तदा ते घनक्षेत्रे समाने भविष्यतः ।}\\
\vspace{3mm}

 यथा अबजदघनक्षेत्रे अवजलयोर्धरातलयोः कल्पिते । वबपिण्डलदपिण्डौ \renewcommand{\thefootnote}{३}\footnote{{\en drops} लम्बरूपौ.}लम्बरूपौ यदि समानौ भवतस्तदैतयोर्घनक्षेत्रयोर्निष्पत्तिर्द्वयोर्धरातलयोर्निष्पत्तितुल्या भविष्यति~।
 \begin{center}
 \includegraphics[scale=1]{Images/rg-139.png}  
 \end{center}
 यदि घनक्षेत्रे समाने \renewcommand{\thefootnote}{४}\footnote{° स्तदातयो° {\en J.}}भवतस्तयोर्धरातलेऽपि समाने भविष्यतस्तदैतयोर्धरातलयोर्निष्पत्तिर्लम्बयोर्विलोमनिष्पत्तितुल्या भविष्यति । यद्येतद्रूपानिष्पत्तिः स्यात्तदा ते

\newpage
\noindent द्वे घरातले समाने भविष्यतः । तस्मात् द्वे घनक्षेत्रे अपि समाने भविष्यतः । यदि वबलदौ लम्बौ समानौ न स्तः \renewcommand{\thefootnote}{१}\footnote{{\en J. inserts} तदा.}लदमधिकं कल्पितम् । तस्मात् वबतुल्यं लगं पृथक्कार्यम्~। लगं तखं जसं कशं बवतुल्यं पृथक्कार्यम् । पुनर्गखं खसं सशं शगं रेखाः संयोज्याः~। तस्मात् अबं जगमुभे घनक्षेत्रे \renewcommand{\thefootnote}{२}\footnote{{\en J. has} समाने.}समानलम्बे भविष्यतः~। तदैतयोर्निष्पत्तिर्धरातलयोर्निष्पत्तिसमाना भविष्यति । यदि कदधरातलकगधरातले जदघनक्षेत्रजगघनक्षेत्रयोर्भूमी कल्पिते अनयोर्लम्बौ समौ भविष्यतः । जदजगयोर्निष्पत्तिः कदकगयोर्निष्पत्तिसमाना भविष्यति लदलगयोरपि निष्पत्तिसमाना भविष्यति ।\\
\vspace{5mm}

यदि अबजदे घनक्षेत्रे समाने \renewcommand{\thefootnote}{३}\footnote{भवतः {\en J. and V.}}भविष्यतस्तदैतयोर्निष्पत्तिर्जगघनक्षेत्रेणैकरूपा भविष्यति । इयम् अवधरातलजलधरातलयोर्निष्पत्तितुल्या भविष्यति । लदरेखाया निष्पत्तिर्लगरेखया वबरेखया चैकरूपास्ति । इयं विलोमनिष्पत्तिर्जाता । यदि अवजलनिष्पत्तितुल्यघनक्षेत्रयोः अबजगयोर्निष्पत्तिर्जदजगनिष्पत्तितुल्यलदलगयोर्निष्पत्तितुल्या भवति तदा उभे घनक्षेत्रे समाने भविष्यतः । इदमेवेष्टम् ॥\\
\vspace{5mm}

\begin{center}
\textbf{\large अथ पञ्चत्रिंशत्तमं क्षेत्रम् ॥ ३५ ॥}
\end{center}
\vspace{5mm}

{\ab समानान्तरधरातले उभे घनक्षेत्रे स्तस्तयोः पिण्डे धरातले लम्बरूपे न भवतस्ते द्वे घनक्षेत्रे समाने भवतस्तदा तयोर्धरातलयोर्निष्पत्तिर्लम्बयोर्विलोमनिष्पत्तितुल्या भवति यद्येतादृशोर्निष्पत्तिर्भवति तदा द्वे घनक्षेत्रे समाने भवतः ।}\\
\vspace{5mm}

यथा अबजदे द्वे घनक्षेत्रे अवजलयोर्धरातलयोः कल्पिते । पुनर्धरातलयोः कोणचिह्नेभ्यः अफझखवरगहलम्बास्तथा जझकछलझतखलम्बाः निष्कास्याः । पुनर् अरजझे द्वे घनक्षेत्रे अबजदयो-

\newpage
\begin{center}
 \includegraphics[scale=1]{Images/rg-140.png}  
 \end{center}
र्घनक्षेत्रयोः समाने संपूर्णे कार्ये । अरजझयोः
क्षेत्रयोर्निश्चयेनेष्टसिद्धमस्ति । तस्मात् अबजदयोर्घनक्षेत्रयोरपि । इष्टमस्माकं निश्चितं भविष्यति । कुतः । धरातललम्बयोः साम्यात् । इदमेवेष्टम् ॥\\
\begin{center}
\textbf{\large अथ षट्त्रिंशत्तमं क्षेत्रम् ॥ ३६ ॥}
\end{center}
\vspace{5mm}

{\ab समानान्तरधरातलघनक्षेत्रयोः सजातीययोर्निष्पत्तिः सजातीयभुजनिष्पत्तिघनतुल्या भविष्यति ।}\\
\vspace{3mm}

 यथा अबजदे घनक्षेत्रे कल्पिते । तत्र जअझतयोर्निष्पत्तिः
कझसतयोर्निष्पत्तितुल्या \renewcommand{\thefootnote}{१}\footnote{हझवतयोर्निष्पत्ति° {\en V.}}हझवतनिष्पत्तितुल्या च कल्पिता ।
पुनर्हझरेखा वर्द्धनीया । वततुल्यं झनं कार्यम् । पुनः कझरेखा वर्द्धनीया
 । सततुल्यं झमं कार्यम् । पुनर्गकफझखलानि घनक्षेत्राणि संपूर्णानि कार्याणि । एषु घनक्षेत्रेषु द्वे घनक्षेत्रे क्रमेणैकैकं विहाय चेद्गृह्यते 
\begin{center}
 \includegraphics[scale=1]{Images/rg-141.png}  
 \end{center} 
तदा तेऽभि\renewcommand{\thefootnote}{२}\footnote{{\en V. has} सन्मुख {\en for} अभिमुख. }मुख\renewcommand{\thefootnote}{३}\footnote{{\en J. has} खधरातलसमानान्तर° '.}खसमानान्तरघरातलेन कृतसंपाते भविष्यतः । खलघनक्षेत्रं जदघनक्षेत्रस्य समानं भविष्यति । तस्मात् अबगकघनक्षेत्रनिष्पत्तिर्झहझननिष्पत्तितुल्या भविष्यति । गकफझघनक्षेत्रनिष्पत्तिः
कझझम-

\newpage
निष्पत्तितुल्या भविष्यति ।
फझघनक्षेत्रजदघनक्षेत्रतुल्यखलघनक्षेत्रयोर्निष्पत्तिः अझझलनिष्पत्तितुल्या भविष्यति । तस्मात् अवजदधनक्षेत्रनिष्पत्तिर्भुजयोर्निष्पत्तेर्घनतुल्यास्ति । इदमेवेष्टम् ॥\\
\begin{center}
\textbf{\large अथ सप्तत्रिंशत्तमं क्षेत्रम् ॥ ३७ ॥}
\end{center}
\vspace{2mm}

{\ab समानकोणधरातलद्वये चेन्निषण्णे द्वे रेखे भवतस्तत्र भुजद्वयरेखासंपातजनितकोणौ द्वितीयरेखाभुजद्वयसंपातजनितकोणाभ्यां यथाक्रमं चेत्समानौ भवतः पुनर्निषण्णरेखातः कस्मादपि चिह्नादेको लम्बो धरातले नेयः पुनर्लम्बनिपातात् कोणपर्यन्तं रेखा कार्या तत्रास्यां रेखायां निषण्णरेखयोत्पन्नौ
कोणौ तदा समानौ भविष्यतः~।}\\
\vspace{3mm} 

यथा अबजं दहझं द्वौ धरातलकोणौ कल्पितौ । तत्र वबहते रेखे तथा निषण्णे कल्पिते यथोत्पन्नः अबवकोण उत्पन्नदहझकोणेन समानो भवति । एवं जबवकोणो झहतक समानो भवति । पुनर्वबरेखाया हतरेखाया कचिह्नलचिह्नाभ्यां कमलम्बलनलम्बौ अबजकोणधरातले दहझकोणघरातले मचिह्ननचिह्नस्थाने पतिताविति कल्पितौ । पुनर्मबनहे द्वे रेखे योजिते । तस्मात् मबवउत्पन्नकोणनहतउत्पन्नकोणौ मिथः समानौ भविष्यतः । 
\begin{center}
 \includegraphics[scale=1]{Images/rg-142.png}  
 \end{center} 
\vspace{5mm}

\begin{center}
\textbf{\large अत्रोपपत्तिः ।}
\end{center}
\vspace{2mm}

बकं \renewcommand{\thefootnote}{१}\footnote{हसतुल्यं {\en J.}\\
 भा० २०}हसं तुल्यं कार्यं यदि बकहलौ समानौ न भवतः । पुनः

\newpage
\noindent सचिह्नात् सगलम्बो हनरेखायां नेयः ।
पुनर्मचिह्नगचिह्नाभ्याम् अबरेखादहरेखयोरुपरि मफगरौ द्वौ लम्बौ नेयौ । पुनर्जबझहरेखयोरुपरि मखगशौ द्वौ लम्बौ नेयौ । पुनः फखरशकफसरकखसशरेखाः संयोज्याः । तस्मात् बकवर्गः कमवर्गबमवर्गयोर्योगेन समानोऽस्ति । मबवर्गस्तु मफवर्गफबवर्गयोर्योगेन समानो भविष्यति । तस्मात् बकवर्गः कमवर्गमफवर्गफबवर्गाणां योगेन समानो भविष्यति । तस्मात् कफम् अबे लम्बो भविष्यति । अनेनैव निश्चितं कखं जवे लम्बो भविष्यति । सरं दहे लम्बो भविष्यति । सशं झहे लम्बो भविष्यति । बफकत्रिभुजे हरसत्रिभुजे बकोणहकोणौ समानौ स्तः । फकोणरकोणौ प्रत्येकं समकोणौ स्तः । बकभुजहसभुजौ मिथः समानौ स्तः~। तदा बफं हरं तुल्यं भविष्यति । फकं रसतुल्यं भविष्यति ।\\
\vspace{7mm}

अनेनैव प्रकारेण निश्चितं बखं हशतुल्यं भविष्यति । तस्मात् बफखत्रिभुजे हरशत्रिभुजे बकोणहकोणयोः साम्यात् कोणयोर्भूजयोः साम्याच्च फखरशौ समानौ भविष्यतः । फखरशभुजयोरुपरितनकोणौ मिथः समानौ भविष्यतः । मफखत्रिभुजे गरशत्रिभुजे पूर्वकोणाः समकोणेभ्यश्चेच्छोध्यन्ते तदा द्वौ कोणौ द्वयोः कोणयोः समानाववशिष्यतः । फखरशभुजौ च समानौ स्तः । तस्मात् फमरगौ समानौ भविष्यतः । फकं च रसतुल्यमस्ति । यदि फकवर्गरसवर्गयोः फमवर्गरगवर्गौ चेच्छोध्येते तदा
मकवर्गगसवर्गौ समानाववशिष्यतः । \renewcommand{\thefootnote}{१}\footnote{पुनरेतौ वर्गौ बकहसवर्गयोः,}पुनर्मकवर्गगसवर्गौ
बकहससमानवर्गयोः शोध्येते तदा शेषं बमवर्गगहवर्गौ समानाववशिष्यतः । \\
\vspace{7mm}

पुनर्निश्चयः कार्यः । बकमत्रिभुजे हसगत्रिभुजे भुजा मिथः समानाः सन्ति । तस्मात् मबवकोणनहतकोणौ समानौ भविष्यतः । इदमेवेष्टम् ॥

\newpage
\begin{center}
\textbf{\large अथाष्टत्रिंशत्तमं क्षेत्रम् ॥ ३८ ॥}
\end{center}
\vspace{5mm}

{\ab यदि मिथो द्वे \renewcommand{\thefootnote}{१}\footnote{समानकोणे द्वे घनक्षेत्रे {\en V., and J.}}घनक्षेत्रे समानकोणे भवत एकघनक्षेत्रस्य त्रयो भुजा एकरूपनिष्पत्तौ यदि भवन्ति द्वितीयघनक्षेत्रस्य त्रयो भुजाः प्रथमभुजत्रयमध्ये renewcommand{\thefootnote}{२}\footnote{मध्यनिष्पत्तिभुज° (मध्यभुजनिष्पत्ति°?) {\en J.}}मध्यनिष्पत्तितुल्याश्चेद्भवन्ति तदा ते द्वे घनक्षेत्रे मिथः समाने भविष्यतः ।}\\
\vspace{3mm}

यथा अबजास्तिस्रो रेखा एकरूपनिष्पत्तौ कल्पिताः । पुनर्दहरेखा अरेखातुल्या कल्पिता । पुनर्दचिह्ने एको धनकोणः कल्प्यः । पुनर्दवभुजो बतुल्यः कार्यः~। दतभुजश्च जतुल्यः कार्यः । पुनर्दकधनक्षेत्रं समानान्तरभुजं पूर्णं कार्यम्~।

\begin{center}
 \includegraphics[scale=1]{Images/rg-143.png}  
 \end{center} 
पुनर्लमरेखा लचिह्नोपरि एकघनकोणो दकोणतुल्यस्तथा कार्यो यथा मलनकोणो हदतकोणतुल्यो भवति । मलझकोणश्च हदवकोणतुल्यो भवति । झलनकोणो वदतकोणतुल्यो भवति । पुनर्लसलगौ बतुल्यो पृथक् कार्यौ । पुनर्लफघनक्षेत्रं पूर्णं कार्यम्~। दकं घनक्षेत्रं
लफघनक्षेत्रं मिथः समानं भविष्यति ।\\
\begin{center}
\textbf{अस्योपपत्तिः ।}
\end{center}
\vspace{5mm}

 यदि दवलससमानभुजौ पिण्डौ कल्पितौ तदा दकं घनक्षेत्रं लफं घनक्षेत्रं हतमगधरातलयोर्निष्पत्तौ भविष्यतः । हतमगौ मिथः समानौ स्तः । कुतः । हदतकोणमलगकोणयोर्मिथः साम्यात् । दहभुजम-

\newpage
\noindent लभुजनिष्पत्तिर्लगभुजदतभुजयोः निष्पत्त्या तुल्यास्ति । तस्मात् द्वे घनक्षेत्रे समाने भविष्यतः । इदमेवेष्टम् ॥\\
\begin{center}
\textbf{\large अथैकोनचत्वारिंशत्तमं क्षेत्रम् ॥ ३९ ॥}
\end{center}
\vspace{5mm}

{\ab यदि द्वयो रेखयोः सजातीयसमानान्तरधरातले घनक्षेत्रे भवतोऽन्ययोर्द्वयो रेखयोः सजातीयसमानान्तरधरातले घनक्षेत्रे यदि भवतो यद्येताश्चतस्रो रेखा एकनिष्पत्तौ भवन्ति तदैतानि घनक्षेत्राण्येकनिष्पत्तौ भविष्यन्ति । यदि घनक्षेत्राण्येकनिष्पत्तौ भवन्ति तदा रेखा अप्येकनिष्पत्तौ भविष्यन्ति ।}\\
\vspace{5mm}

यथा अबजदयोरुपरि अकजले द्वे घनक्षेत्रे सजातीये कल्पिते ।
हझवतयोरुपरि हमवने द्वे अन्ये घनक्षेत्रे कल्पिते ।
पुनरेताश्चतस्रो रेखा एकनिष्पत्तौ कल्पिताः । पुनरबजदनिष्पत्तितुल्या जदरेखा
सरेखानिष्पत्तिः कल्पिता । सरेखागरेखयोर्निष्पत्तिः कल्पिता । हझ--
\begin{center}
 \includegraphics[scale=1]{Images/rg-144.png}  
 \end{center} 
वतनिष्पत्तितुल्या वतफरेखानिष्पत्तिः कल्पिता ।
\renewcommand{\thefootnote}{१}\footnote{{\en J., inserts} तथैव.}फरेखाखरेखयोरपि
निष्पत्तिः कल्पिता । तदा अकजलघनक्षेत्रयोर्निष्पत्तिः अबगरेखानिष्प-

\newpage
\noindent त्तितुल्या भविष्यति । हमवनघनक्षेत्रयोर्निष्पत्तिर्हझखरेखयोर्निष्पत्तितुल्या भविष्यति~। अबगरेखानिष्पत्तिर्हझखरेखानिष्पत्तितुल्यास्ति ।
तस्मादेतानि घनक्षेत्राण्येकनिष्पत्तौ भविष्यन्ति ।\\
\vspace{3mm}

पुनरेतानि घनक्षेत्राण्येकरूपनिष्पत्तौ कल्पितानि । अबजदनिष्पत्तिर्हझरशतुल्या कार्या । रशोपरि रतं घनक्षेत्रं वनघनक्षेत्रवत् कार्यम् । इदमपि हमघनक्षेत्रवत् भविष्यति । अकजलयोर्निष्पत्तिर्हमरतयोर्निष्पत्तितुल्यास्ति । हमवनयोर्निष्पत्तितुल्यासीत् । तस्मात्
वनरते घनक्षेत्रे सगाने भविष्यतः । सजातीये आस्ताम् । तस्मात् वतरेखा रशरेखा समाना जाता । तदैता रेखा एकनिष्पत्तौ भविष्यन्ति । इदमेवास्माकमिष्टम् ॥\\
\vspace{3mm}

\begin{center}
\textbf{\large अथ चत्वारिंशत्तमं क्षेत्रम् ॥ ४० ॥}
\end{center}
\vspace{5mm}

{\ab घनहस्तक्षेत्रस्य मिथः सन्मुखधरातलयोर्भुजानामर्द्धं कार्यमर्द्धचिह्नेषु धरातलद्वयं मिथः संपातकर्तृ\renewcommand{\thefootnote}{१}\footnote{° कारक ° {\en J.}} घनहस्तच्छेदकं कार्यं तदा धरातलयोः संपातरेखाधनहस्तकर्णयोः \renewcommand{\thefootnote}{२}\footnote{अर्धे संपातो भविष्यति.}संपातो भविष्यत्यर्द्धे ।}\\
\vspace{3mm}

यथा अबं घनहस्तः कल्पितः । दहझते द्वे सन्मुखधरातले कल्पिते । द्वयोर्ध-
\begin{vwcol}[widths={0.6,0.4}, sep=.8cm, rule=0pt]
रालयोर्भुजानां कचिह्नलचिह्नमचिह्ननचिह्नेषु तथा सचिह्नगचिह्नफचिह्नखचिह्नेष्वर्द्धं कृतम्~। अर्द्धचिह्नेषु कफधरातललखधरातले संप्राप्ते कल्पिते । द्वयोर्धरातलयोः संपातरेखा रशं कल्पिता । धनहस्तकर्णम् अबं कल्पितम् । तदा अबरशरेखे तचिह्नोपर्यर्द्धे संपातं करिष्यतः ।\\
\noindent  \includegraphics[scale=1]{Images/rg-145.png}  
\end{vwcol}
\vspace{5mm}


\begin{center}
\textbf{अस्योपपत्तिः ।}
\end{center}
\vspace{3mm}

जररअरेखे संयोज्ये । अरलत्रिभुजे जरनत्रिभुजे लकोणनकोणौ

\newpage
\noindent समकोणौ स्तः । एतत्संबन्धिभुजौ समानौ । तदा अरभुजजरभुजौ
समानौ भविष्यतः~। पुनर्लरअकोणनरजकोणौ समानौ भविष्यतः । पुनर् अरनकोण उभयत्र योज्यते । तदा लरअकोणअरनकोणयोर्योगो द्वाभ्यां समकोणाभ्यां तुल्यो नरजकोणनरअकोणयोर्योगेन तुल्यो भविष्यति । तस्मात् जरअसरलैकरेखा स्यात् । पुनर्वशशबरेखे संयोज्ये ।\\
\vspace{5mm}

इदं निश्चितम् । अनयोर्योगोऽपि सरलैकरेखा भविष्यति । जबअवरेखा हतरेखायाः समाने समानान्तरे स्तः । तदा अजवबरेखे मिथः समाने समानान्तरे च भविष्यतः । अबकर्णोऽनयोर्धरातलेऽस्ति । तस्मादियं रेखा रशं छेत्स्यति । अरतत्रिभुजे बशतत्रिभुजे अरभुजबशभुजौ समानौ स्तः । अनयोस्त्रिभुजयोः कोणावपि मिथः समानौ स्तः । तस्मात् अतं तबसमानं भविष्यति । रतं तशसमानं भविष्यति ।
इदमेवास्माकमिष्टम् ॥\\
\vspace{5mm}

\begin{center}
\textbf{\large अथैकचत्वारिंशत्तमं क्षेत्रम् ॥ ४१ ॥}
\end{center}
\vspace{3mm}

{\ab \renewcommand{\thefootnote}{१}\footnote{द्वयो° {\en v.}}ययोश्छेदितक्षेत्रयोः समानलम्बयोरेकस्य भूमिस्त्रिभुजास्ति । द्वितीयस्य भूमी चतुर्भुजा समानान्तरभुजा पूर्वभूमेर्द्विगुणास्ति । \renewcommand{\thefootnote}{२}\footnote{तदैते {\en v.}}तदा ते छेदितक्षेत्रे समाने भविष्यतः ।}\\
\vspace{5mm}

यथा अबजदहझक्षेत्रं वतकलमनं द्वितीयं छेदितक्षेत्रं कल्पितम्
। प्रथमस्य भूमिर्बदचतुर्भुजा द्वितीयस्य भूमिर्नकलत्रिभुजा कल्पिता ।
\begin{center}
\noindent \includegraphics[scale=1]{Images/rg-146.png}  
\end{center}
पुनर्नलचतुर्भुजं समानान्तरभुजं संपूर्णं कार्यम् । इदं
बदचतुर्भुज-

\newpage
\noindent समानं भविष्यति । पुनर्जसं घनक्षेत्रं कगं च संपूर्णं कार्यम् ।
एते द्वे घनक्षेत्रे समाने भविष्यतः । कुतः । भूमिलम्बानां \renewcommand{\thefootnote}{१}\footnote{साम्यात् {\en J. J. drops} अपि समाने.}समत्वात् ।
तदैतयोरर्द्धे छेदितक्षेत्रे अपि समाने भविष्यतः । इदमेवास्माकमिष्टम्
॥\\
\begin{center}
{\small श्रीमद्राजाधिराजप्रभुवरजयसिंहस्य तुष्ट्यै द्विजेन्द्रः\\
 सम्राड् श्रीमज्जगन्नाथ इति समभिधारूढितेन प्रणीते ।\\
 ग्रन्थेऽस्मिन्नाम्नि रेखागणित इति सुकोणावबोधप्रदात-\\
 र्य्यध्यायोऽध्येतृमोहापह इह विरतिं प्राप भूचन्द्रतुल्यः ॥\\
 ॥ इति श्रीसम्राइजगन्नाथविरचिते रेखागणिते\\
 एकादशोऽध्यायः संपूर्णः ॥ ११ ॥}\\

\rule{0.6in}{0.3pt} 

\end{center}

\newpage
\afterpage{\fancyhead[CE] {रेखागणितम्}}
\afterpage{\fancyhead[CO] {द्वादशमोध्यायः}}
\afterpage{\fancyhead[LE,RO]{\thepage}}
\cfoot{}
\newpage
%%%%%%%%%%%%%%%%%%%%%%%%%%%%%%%%%%%%%%%%%%%%%%%%%%%%%%%%%%%%%%
\newpage
\thispagestyle{empty}
\begin{center}
\textbf{\LARGE ॥ अथ द्वादशोऽध्यायः प्रारभ्यते ॥}
\end{center}
\vspace{3mm}


\begin{center}
\textbf{॥ \renewcommand{\thefootnote}{१}\footnote{अत्र {\en V.} }तत्र पञ्चदश क्षेत्राणि सन्ति ॥}
\vspace{5mm}
  
 
\textbf{\renewcommand{\thefootnote}{२}\footnote{{\en V. drops} अथ.}अथ प्रथमं क्षेत्रम् ॥ १ ॥}\\
\end{center}
\vspace{2mm}

{\ab द्वे क्षेत्रे सजातीये द्वयोर्वृत्तयोर्मध्ये यदि स्यातां तदा तयोः क्षेत्रयोर्निष्पत्तिर्वृत्तव्यासवर्गयोर्निष्पत्तितुल्या भवति ।}\\
\vspace{3mm}

 यथा अबजदहक्षेत्रं वतकलमक्षेत्रं च कल्पितम् । बझतनौ व्यासौ कल्पितौ~। पुनर् अझवनबहतमरेखाः संयोज्याः । तदा अबहत्रिभुजे वतमत्रिभुजे अकोणवकोणौ समानौ स्तः । कोणयोः संबन्धिभुजौ सजातीयौ स्तः । अहबकोणतुल्यअझबकोणो वतमतुल्यवनतकोणतुल्यो भविष्यति । तस्मात् अझबत्रिभुजवनतत्रिभुजे
\begin{center}
\noindent \includegraphics[scale=1]{Images/rg-147.png}  
\end{center}
झअबकोणवनतकोणयोः साम्येन झअबकोणनवतकोणयोः समकोणभावित्वेन सजातीये भविष्यतः~। अबवतभुजयोर्निष्पत्तिर्बझतनभुजयोर्निष्पत्तिसमाना भविष्यति ।
अवजदहक्षेत्रवतकलमक्षेत्रयोर्निष्पत्तिः अबवतयोर्निष्पत्तिवर्गतुल्यास्ति । तस्मात्  \renewcommand{\thefootnote}{३}\footnote{{\en V. inserts} तयोः. }द्वयोः
क्षेत्रयोर्निष्पत्तिर्बझतननिष्पत्तिवर्गतुल्या भविष्यति । तस्मात्
बझतनयोर्वर्गनिष्पत्तितुल्या भविष्यति । इदमेवेष्टम् ॥\\
\begin{center}
\textbf{\large \renewcommand{\thefootnote}{४}\footnote{{\en V. drops} अथ.}अथ द्वितीय क्षेत्रम् ॥ २ ॥}
\end{center}
\vspace{2mm}

{\ab वृत्तफलयोर्निष्पत्तिर्व्यासवर्गयोर्निष्पत्तितुल्या  \renewcommand{\thefootnote}{५}\footnote{भवति {\en V.}}भविष्यति ।}

\newpage
यथा अजहववृत्ते कल्पिते । बदझतौ \renewcommand{\thefootnote}{१}\footnote{{\en V. has} तयोः {\en for} तत्क्षेत्रयोः,}तत्क्षेत्रयोर्व्यासौ
कल्पितौ । यदि बदवर्गझतवर्गयोर्निष्पत्तिः अजवृत्तफलहववृत्तफलयोर्निष्पत्तितुल्या न भवति तदा \renewcommand{\thefootnote}{२}\footnote{अजवृत्तफलकल्पितान्यक्षेत्र निष्पत्तितुल्या कल्पिता । तत्क्षेत्रं प्रथमवृत्तफलान्न्यून सक्षेत्रं
कल्पितम् । {\en K., A.}}अज-वृत्तक्षेत्रसक्षेत्रनिष्पत्तितुल्या
कल्पिता । सक्षेत्रं प्रथमवृत्तफलान्न्यूनं कल्पितम् । हववृत्तफलसक्षेत्रयोरन्तरं\renewcommand{\thefootnote}{३}\footnote{° रन्तरतुल्यं {\en K., A.} }
खक्षेत्रं कल्पितम् । पुनर्झहतचापझवतचापे हचिह्नवचिह्नयोर्द्धिते कार्ये । पुनर्झहहततववझरेखाः संयोज्याः । तस्मात् हवक्षेत्रं हववृत्तार्द्धफलादधिकं भविष्यति । पुनश्चत्वारि चापानि
कचिह्नलचिह्नमचिह्ननचिह्नष्वर्द्धितानि कार्याणि । एतेषां चापानां पूर्णज्याः संयोज्याः ।
तस्मात् चापानां मध्ये चत्वारि त्रिभुजान्युत्पद्यन्ते । प्रत्येकं क्षेत्रं \renewcommand{\thefootnote}{४}\footnote{खखण्डार्धा° {\en K., A.}\\
 भा ० २१}खार्द्धादधिकं भविष्यति ।
\begin{center}
\includegraphics[scale=1]{Images/rg-148.png}  
\end{center}

अनेन प्रकारेण त्रिभुजानि तावदुत्पादनीयानि यावच्छेषवृत्तखण्डानि खक्षेत्रात् न्यूनानि भवन्ति । तस्मात् बहुभुजोत्पन्नं क्षेत्रं कमक्षेत्रं सक्षेत्रादधिकं भविष्यति~। पुनर् अजवृत्ते
सफक्षेत्रं कमक्षेत्रसजातीयं कार्यम् । तस्मात् बदवर्गझतवर्गयोर्निष्पत्तिः
सफक्षेत्रकमक्षेत्रयोर्निष्पत्तितुल्या भविष्यति । अजवृत्तफलस्य सक्षेत्रफलस्य च-

\newpage
\noindent निष्पत्तितुल्यासीत् । तस्मात् सफक्षेत्रकमक्षेत्रयोर्निष्पत्तिः अजवृत्तफलस्य सक्षेत्रफलस्य च निष्पत्तिसमाना भविष्यति । पुनः सफक्षेत्रअजवृत्तफलस्य निष्पत्तिः कमक्षेत्रसक्षेत्रनिष्पत्तितुल्यास्ति ।
कमक्षेत्रं सक्षेत्रादधिकमस्ति । तस्मात् सफक्षेत्रफलं अजवृत्तफलादधिकं भविष्यति । इदमशुद्धम् ॥\\
\vspace{5mm}

पुनर्बदवर्गझतवर्गयोर्निष्पत्तिः अजवृत्तक्षेत्रहववृत्तादधिकान्यक्षेत्रनिष्पत्तिसमाना कल्पिता । तस्मात्
झतबदवर्गयोर्निष्पत्तिस्तथास्ति यथा हवादधिकक्षेत्रस्य निष्पत्तिः अजवृत्तफलेनास्ति वा
\renewcommand{\thefootnote}{१}\footnote{हवक्षेत्रस्य न्यूनक्षेत्रनिष्पत्त्या तुल्यास्ति । {\en K., A.} }हववृत्तफलस्य
अजवृत्तफलान्न्यूनक्षेत्रेण निष्पत्तिस्तत्तुल्यास्ति ।
\renewcommand{\thefootnote}{२}\footnote{पूर्ववदेतदप्यनुपपन्नम् । {\en K., A.} }पूर्वप्रकारेणैवेदमप्यशुद्धं कुर्मः । तस्मादस्मदिष्टं समीचीनम् ॥\\
\vspace{5mm}

\begin{center}
\textbf{\large अथ तृतीयं क्षेत्रम् ॥ ३ ॥}
\end{center}
\vspace{5mm}

{\ab त्र्यस्रत्रिफलशङ्कोः खण्डचतुष्टयं कार्यं तत्र\renewcommand{\thefootnote}{३}\footnote{तत्र खण्डद्वयं शङ्कुरूपं समानं सजातीयं कर्त्तव्यमस्ति । {\en K., A., V.}} पुनः खण्डद्वयं
शङ्कुरूपं समानजातीयं कर्त्तव्यमस्ति । तस्यैव शङ्कोः शेषे द्वे
खण्डे छेदितक्षेत्ररूपे शङ्क्वर्धादधिके समाने भवतस्तथा कर्त्तव्यम् ।}\\
\vspace{5mm}


यथा अबजदशङ्कोः अबजत्रिभुजं भूमिः दं मुखं कल्पितम् । पुनस्तस्य षड् भुजा हझतवकलचिह्नेष्वर्द्धिताः कार्याः । पुनर्हझझवहवझततकझकतलवलरेखाः संयोज्याः । एवं कृतेऽस्मदिष्टं सिद्धं भविष्यति ।\\
\vspace{3mm}

\begin{center}
\textbf{\large अस्योपपत्तिः ।}
\end{center}
\vspace{3mm}

 अहवझशङ्कोर्झतकदशङ्कोश्च त्रयो भुजा मिथः समानाः सन्ति ।

\newpage
\noindent कुतः । अनयोर्भुजा वृतच्छङ्कोर्भुजार्द्धमिताः सन्ति । एतानि त्रिभुजानि सजातीयानि भविष्यन्ति । कुतः । केचित्कोणा मिलिताः सन्ति । केचित्कोणाः समानाः सन्ति । कुतः । \renewcommand{\thefootnote}{१}\footnote{{\en K. and A. insert} समाना {\en here.}}एतेषां कोणानां भुजा बृहद्भु-
\begin{vwcol}[widths={0.65,0.35}, sep=.8cm, rule=0pt]
जेभ्यः समानान्तराः सन्ति । तस्मादेतौ शङ्कू मिथः सजातीयौ समानौ च भविष्यतः । बृहच्छङ्कोः सजातीयौ च पतिष्यतः । पुनर्बृहच्छङ्कोरर्द्धे छेदितक्षेत्रे समानलम्बेऽवशिष्यते । तस्मादेतयोर्द्वयोश्छेदितक्षेत्रयोर्झतलवं धरातलमेकमेव भविष्यति । पुनरेकच्छेदितक्षेत्रस्य भूमिर्हवलबचतुर्भुजं समानान्तरभुजं भविष्यति~। 
\noindent \includegraphics[scale=1]{Images/rg-149.png}  
\end{vwcol}
\vspace{-3mm}

\noindent  द्वितीयस्य भूमिर्वलजत्रिभुजं भविष्यति ।  इदं त्रिभुजं हवलबक्षेत्रस्यार्द्धमस्ति । तस्मादुभे छेदितक्षेत्रे अपि समाने भविष्यतः । यस्य
च्छेदितक्षेत्रस्य भूमिर्वलजत्रिभुजमस्ति तत् अहवझशङ्कोरधिकमस्ति । कुतः । एतयोः समभूमिसमलम्बत्वात्~। \renewcommand{\thefootnote}{२}\footnote{तस्मादे° {\en V,}}अस्मादेतच्छेदितक्षेत्रद्वयं बृहच्छङ्कोरर्द्धादधिकं भविष्यति । इदमेवेष्टम्~॥\\
\begin{center}
\textbf{\large अथ चतुर्थ क्षेत्रम् ॥ ४ ॥}
\end{center}
\vspace{3mm}

{\ab त्रिभुजभूमिकयोस्त्रिफलकयोः समानलम्बयोः शङ्कोः प्रत्येकस्य पूर्ववच्छङ्कुद्वयं छेदितक्षेत्रद्वयं च क्रियते तदानयोर्भूभ्योर्निष्पत्तिरनयोश्छेदितक्षेत्रनिष्पत्तितुल्या भविष्यति ।}\\
\vspace{5mm}

 यथा अबजदमेको मनसगं द्वितीयः शङ्कुः कल्पितः । अनयोः शङ्क्वोर्मध्ये उभौ शङ्कू द्वे छेदितक्षेत्रे च पूर्ववत्कार्ये । तदा अबजत्रिभुजमनसत्रिभुजयोर्निष्पत्तिः अबजदशङ्कोश्छेदितक्षेत्रद्वयस्य
मनसगशङ्कोश्छेदितक्षेत्रद्वयेन या निष्पत्तिस्तस्याः समाना भविष्यति ।\\
\begin{center}
\textbf{अस्योपपत्तिः ।}
\end{center}
\vspace{3mm}

बजजलयोर्निष्पत्तिर्नससशयोर्निष्पत्तितुल्यास्ति । तस्मात् जबज-

\newpage

\noindent लनिष्पत्तिवर्गतुल्या
अबजत्रिभुजवलजत्रिभुजनिष्पत्तिर्नससशनिष्पत्तिवर्गतुल्यमनसत्रिभुजरसशत्रिभुजनिष्पत्तिसमाना भविष्यति । \\
\begin{center}
\includegraphics[scale=1.1]{Images/rg-150.png}  
\end{center}

 तदा अबजत्रिभुजमनसत्रिभुजयोर्निष्पत्तिर्वलजत्रिभुजरशसत्रिभुजयोर्निष्पत्तितुल्यास्ति । इयं निष्पत्तिर्यस्य च्छेदितक्षेत्रस्य वलजत्रिभुजं भूमिः पुनर्यस्य च्छेदितक्षेत्रस्य रसशत्रिभुजं भूमिरनयोर्निष्पत्तिसमाना
भविष्यति । \renewcommand{\thefootnote}{१}\footnote{यतोऽनयोर्लम्बाः समानाः सन्ति । {\en K., A.}}कुतः । अनयोर्लम्बसाम्यात् । प्रत्येकं छेदितक्षेत्रस्यार्द्धमस्ति । तस्मादपि यस्य च्छेदितघनक्षेत्रस्य भूमिर्वलजत्रिभुजमस्ति पुनर्यस्य च्छेदितघनक्षेत्रस्य भूमी रसशत्रिभुजमनयोर्निष्पत्तिद्विगुणयोर्निष्पत्तिसमानास्ति । पुनर्द्विगुणयोर्निष्पत्तिः अबजदशङ्कोश्छेदितक्षेत्रद्वयस्य मनसगशङ्कोश्छेदितक्षेत्रद्वयेन या निष्पत्तिस्तस्याः समानास्ति ।
तस्मात् अबजदशङ्कुभूमिमनसगशङ्कुभूम्योर्निष्पत्तिः अबजदशङ्कोश्छेदितक्षेत्रद्वयस्य मनशगशङ्कोश्छेदितक्षेत्रद्वयस्य च या निष्पत्तिस्तस्याः समानास्ति । इदमेवास्माकमिष्टम् ।। \\
\vspace{5mm}

अनेन क्षेत्रेणेदं निश्चितम् । चतुर्णां शङ्कूनां मध्ये प्रत्येकस्य द्वौ शङ्कू द्वे छेदितक्षेत्रे च पूर्ववत् कार्येते । एवमुत्पन्नशङ्कूनां द्वौ शङ्कू द्वे छेदितक्षेत्रे कार्ये~। एवमग्रेऽपि यथेच्छं कार्ये । तदा
प्रत्येकशङ्कुभूमेर्निष्पत्तिर्द्वितीयशङ्कुभूम्या तथा स्यात् यथा प्रथमशङ्कोश्छेदितक्षेत्रयोर्द्वितीयशङ्कोश्छेदितक्षेत्राभ्यामस्ति । एकप्रथमस्य द्वितीयेन निष्पत्तिस्तथा भवति यथा सर्वेषां प्रथमानां योगस्य द्वितीययोगेन सह यथा निष्पत्तिः

\newpage
\noindent स्यात् । तस्मात् अबजभूमेर्निष्पत्तिर्मनसभूम्या तथा भवति यथा प्रथमशङ्कोः सर्वच्छेदितक्षेत्रयोगस्य द्वितीयशङ्कोश्छेदितक्षेत्रयोगेनास्ति ॥ \\
\begin{center}
\textbf{\large अथ पञ्चमं क्षेत्रम् ॥ ५ ॥\\}
\end{center}
\vspace{5mm}

{\ab द्वौ शङ्कू त्रिभुजभूमी समानलम्बौ च यदि भवतस्तदा शङ्क्वोर्निष्पत्तिर्द्वयोर्भूम्योर्निष्पत्तिसमाना भवति । }\\
\vspace{5mm}

 यथा अबजदमनसगौ द्वौ शङ्कू कल्पितौ । यदि अबजभूमिमनसभूम्योर्निष्पत्तिः अबजदमनसगशङ्क्वोर्निष्पत्तिसमाना न स्यात् तदा अबजदशङ्कुनिष्पत्तिमनसगक्षेत्रादन्यन्न्यूनाधिकक्षेत्रनिष्पत्तितुल्य भवतीति कल्पितम् । प्रथमं खक्षेत्रं मनसगशङ्कोर्न्यूनं कल्पितम् । मनसगशङ्कुखक्षेत्रयोरन्तरं झक्षेत्रं कल्पितम् । 
 पुनर्मनसगशङ्कोर्द्वौ शङ्कू द्वे छेदितक्षेत्रे च पूर्वप्रकारेण कृते । प्रत्येकमुत्पन्नशङ्कूनां द्वौ शङ्कू द्वे छेदितक्षेत्रे च \renewcommand{\thefootnote}{१}\footnote{कार्ये {\en K., A.}}कुर्मः~। 
\begin{center}
\includegraphics[scale=1.1]{Images/rg-151.png}  
\end{center}
एवं पुनरप्युत्पन्नशङ्कूनां करणेन यावत् लघुशङ्कनां योगो झक्षेत्रान्न्यूनो भवति तावत्कार्यम्~।

\newpage

\noindent तस्मात् सर्वेषां छेदितक्षेत्राणां योगः खक्षेत्रादधिको भविष्यति । पुनर अबजदशङ्कोः शङ्कुच्छेदितक्षेत्राणि तावन्ति कार्याणि यावन्ति मनसगशङ्कोः शङ्कुच्छेदितक्षेत्राणि कृतानि । तस्मात्
अवजभूमेर्निष्पत्तिर्मनसभूम्या तथा स्यात् यथा अबजदशङ्कोः सर्वच्छेदितक्षेत्रयोगस्य निष्पत्तिर्मनसगशङ्कोश्छेदि \renewcommand{\thefootnote}{१}
\footnote{सर्वच्छेदित° {\en K., A.}}तक्षेत्रयोगेनास्ति~। पुनर् अबजमनसभूम्योर्निष्पत्तिः अबजदशङ्कुखघनक्षेत्रयोर्निष्पत्तितुल्या कल्पितासीत् । तस्मात् अबजदशङ्कोः सर्वच्छेदितक्षेत्रयोगस्य निष्पत्तिर्मनसगशङ्कोः सर्वच्छेदितक्षेत्रयोगेन निष्पत्तिस्तथास्ति यथा अबजदशङ्कोः खघनक्षेत्रेणास्ति । अबजदशङ्कोः सर्वच्छेदितक्षेत्रयोगस्य निष्पत्तिः अबजदशङ्कुना तथास्ति यथा मनसगशङ्कोः सर्वच्छेदितक्षेत्रयोगः खघनक्षेत्रादधिकोऽस्ति । तस्मात् अबजदशङ्कोः
सर्वच्छेदितक्षेत्रयोगः अबजदशङ्कुतोऽधिको भविष्यति । इदमशुद्धम् ॥ \\
\vspace{5mm}

पुनः खक्षेत्रं मनसगशङ्कोरधिकं कल्पितम्~। तस्मात् मनसभूमेर्निष्पत्तिः अबजभूम्या तथा भविष्यति यथा मनसगशङ्कोर्निष्पत्तिः अबजदशङ्कोर्न्यूनक्षेत्रेणास्ति । \\
\vspace{5mm}

उपरितनप्रकारेणैवेदमशुद्धं\renewcommand{\thefootnote}{२}\footnote{वेदमप्यशुद्धम् । {\en K., A.}} करिष्यामः । तस्मादस्मदिष्टं समीचीनि । \\
\vspace{3mm}

\begin{center}
\textbf{\large अथ षष्ठं क्षेत्रम् ॥ ६ ॥}
\end{center}
\vspace{5mm}

{\ab यत् छेदितक्षेत्रमस्ति तस्य त्रयः समानाः शङ्कवस्त्रिभुजभूमिकाः कर्त्तुं शक्यन्ते । }\\
\vspace{3mm}


यथा अबदहझच्छेदितक्षेत्रं जझदभूमौ कल्पितम् । पुनर्बदब- \\

\newpage
\noindent झझहरेखाः संयोज्याः । रेखायोगेन त्रयः समानाः शङ्कवस्त्रिभुजभूमिकाः संपद्यन्ते । \\
\begin{center}
\textbf{ अत्रोपपत्तिः । }
\end{center}
\vspace{5mm}

\begin{vwcol}[widths={0.7,0.3}, sep=.8cm, rule=0pt]
यस्य शङ्कोर्भूमिर्जबदत्रिभुजं मुखं झचिह्नं यस्य च शङ्कोर्वदहत्रिभुजं भूमिर्मुखं झचिह्नमस्ति एतौ शङ्कू समौ स्तः । छेदितक्षेत्रस्य अबहझशङ्कुरवशिष्टः । अझं द्वितीयशङ्कुसमानोस्ति । कुतः । यतो बचिह्नमुभयोर्मुखं कल्पितम् । अनयोर्भूमिश्च अझहहझदत्रिभुजौ कल्पितौ । तस्मात् त्रय उत्पन्नशङ्कवः समाना जाताः । \\
\noindent \includegraphics[scale=1]{Images/rg-152.png}  
\end{vwcol}
\vspace{5mm}

अनेन क्षेत्रेणेदमपि ज्ञातं त्रिभुजभूमिकशङ्कोश्छेदितक्षेत्रं संपूर्णं
चेत् क्रियते तदा शङ्कुश्छेदितक्षेत्रस्य त्र्यंशो भविष्यति ॥ ६ ॥ \\
\begin{center}
\textbf{\large अथ सप्तमं क्षेत्रं ॥ ७ ॥ }
\end{center}
\vspace{5mm}

{\ab त्रिभुजभूमिकौ शङ्कू यदि समानौ भवतस्तदा तयोर्भूम्योर्निष्पत्तिस्तल्लम्बयोर्विलोमनिष्पत्तितुल्या भविष्यति । \renewcommand{\thefootnote}{१}\footnote{यदीदृशी निष्पत्तिस्तदा तौ समानौ स्तः । {\en K., A.}}यदि तयोः शङक्कोर्भूमिनिष्पत्तिर्लम्बयोर्विलोमनिष्पत्तितुल्या भवति तदा तौ समानौ भवतः । }\\
\vspace{3mm}

यथा अबजदशङ्कुहझवतशङ्कू कल्पितौ । अनयोः शङ्क्वोर्द्वे घनक्षेत्रे समानान्तरधरातले बलझगे संपूर्णे कार्ये । एते द्वे घनक्षेत्रे 
\begin{center}
\includegraphics[scale=1.1]{Images/rg-153.png}  
\end{center}

\newpage
\noindent यदि समाने भवतस्तदानयोर्भूम्योर्निष्पत्तिरनयोर्लम्बिविलोमनिष्पत्तेस्तुल्या भविष्यति~। यदि घनक्षेत्रभूम्योर्निष्पत्तिरेतल्लम्बनिष्पत्तेर्विलोमतुल्या भविष्यति तदैते घनक्षेत्रे समाने भविष्यतः । अनयोर्घनक्षेत्रयोर्निष्पत्तिर्मिथस्तथास्ति यथाऽनयोः षडंशस्य परस्परनिष्पत्तिरस्ति । अनयोः षडंशैः कल्पितशङ्कू भवतः । \\
\vspace{3mm}

अथ घनक्षेत्रभूम्योर्निष्पत्तिर्भूम्योरर्द्धस्य निष्पत्तितुल्यास्ति । अनयोर्भूम्योरर्द्धे कल्पितशङ्कू भूमी भवतः ।
अनयोर्धनक्षेत्रलम्बयोर्निष्पत्तिः कल्पितशङ्कुलम्बयोर्निष्पत्तिरस्ति । कुतः । यत एतत्घनक्षेत्रलम्बौ कल्पितशङ्कुलम्बावेकरूपौ स्तः । तस्मात् द्वयोः कल्पितयोः शङ्क्वोरस्मदिष्टं स्पष्टं भविष्यति ॥ \\
\begin{center}
\textbf{\large अथाष्टमं क्षेत्रम् ॥ ८ ॥ }
\end{center}
\vspace{5mm}

{\ab त्रिभुजभूमिकौ द्वौ शङ्कू यदा सजातीयौ भवतस्तदा तयोर्निष्पत्तिः सजातीयभुजनिष्पत्तिघनतुल्या भविष्यति । }\\
\vspace{3mm}

 यथा अबजदशङ्कुहझवतशङ्कू कल्पितौ । यद्यनयोर्बलझगे द्वे घनक्षेत्रे पूर्णे क्रियेते
तदैतयोर्घनक्षेत्रयोर्निष्पत्तिरनयोर्भुजनिष्पत्तिघनतुल्या भविष्यति । यत  
\begin{center}
\includegraphics[scale=1.1]{Images/rg-154.png}  
\end{center}
एतौ सजातीयौ स्तः । कल्पितशङ्कू च घनक्षेत्रयौर्निष्पत्तितुल्यौ स्तः । कल्पितशङ्कुक्षेत्रस्य भुजौ द्वयोर्घनक्षेत्रभुजयोर्निष्पत्तौ स्तः । तस्मादस्मिन् शङ्कुद्वयेऽस्मदिष्टं सेत्स्यति । क्षेत्रं च पूर्ववत् ॥ 

\newpage
\begin{center}
\textbf{\large अथ नवमं क्षेत्रम् ॥ ९ ॥}
\end{center}
\vspace{5mm}

{\ab समतलमस्तकपरिधेः शङ्कुः समतलमस्तकपरिधितृतीयांशो भवति । }\\
\vspace{3mm}

 यदि तृतीयांशो न भवति तदा तृतीयांशान्न्यूनः कल्पितः । तस्मात् समतलमस्तकपरिधिक्षेत्रं त्रिगुणितशङ्कोरधिकं भविष्यति । तञ्च खघनक्षेत्रतुल्यमधिकं कल्पितम् । तत्क्षेत्रस्य शङ्कोश्च भूमिः
अबजदवृत्तं कल्पितम् । अस्मिन् वृत्ते समकोणसमचतुर्भुजं कार्यम् । अस्मिन् समकोणसमचतुर्भुजे समतलमस्तकपरिधिक्षेत्रोच्छ्रायतुल्यं घनक्षेत्रं कार्यम् । इदं तत्क्षेत्रार्द्धादधिकं भविष्यति । \\
\vspace{3mm}

पुनश्चत्वारि चापानि हझवतचिह्नेष्वर्द्धितानि । तेषु पूर्णजीवाः संयोज्याः । उत्पन्नत्रिभुजेषु च्छेदितक्षेत्रं तावदेवोच्छ्रितं कार्यम् । एतानि च्छेदितक्षेत्राणि समतलमस्तकपरिधिक्षेत्रशेषखण्डचतुष्टयेभ्योऽधिकानि भविष्यन्ति । एवं तावच्छेदितक्षेत्राणि कार्याणि यावत् समतलमस्तकपरिधिक्षेत्रशेषखण्डानि खक्षेत्रान्न्यूनानि भवन्ति ।। 
\begin{center}
\includegraphics[scale=1]{Images/rg-155.png}  
\end{center}

 अत्रोपपन्नं\renewcommand{\thefootnote}{१}\footnote{अत्रोत्पन्नानि घनक्षेत्राणि त्रिगुणितशङ्कोरधिकानि भविष्यन्ति । {\en K.,
 A.}} घनक्षेत्रं त्रिगुणितशङ्कोरधिकं भविष्यति । \renewcommand{\thefootnote}{२}\footnote{{\en K. and A. insert} प्रत्येकं {\en here.}}पुनश्छेदितक्षेत्रभूमौ तावदेवोच्छ्रितः सफलकः शङ्कुयोगशङ्कुः\renewcommand{\thefootnote}{३}\footnote{°योगाः शङ्कवः कार्या: {\en K.A,}} कार्यः । \renewcommand{\thefootnote}{४}\footnote{°शङ्कुवच्छेदितक्षेत्रतुल्या भविष्यन्ति {\en K., A.}\\
  भा० २२}एवमुत्पन्नशङ्कुश्छेदितक्षेत्रतुल्यो भविष्यति । एवमुत्पन्नशङ्कुस्त्रिगुणितः सन् 

\newpage
\noindent छेदितक्षेत्रयोगतुल्यो भविष्यति । तानि छेदितक्षेत्राणि कल्पितशङ्कोः त्रिगुणादधिकानि भवन्ति । बः उत्पन्नसफलकशङ्कुः कल्पितशङ्कन्तस्तिष्ठति । अयं कल्पितशङ्कोरधिको भविप्यति । इदमशुद्धम्\renewcommand{\thefootnote}{१}\footnote{नि १ {\en K. and A, insert} अयं सफलकशङ्कुश्च बृहत्शङ्कोरन्तरितोऽस्ति ।.
पूर्ववत् {\en and A. have} समस्तमस्तकपरिधिक्षेत्रं {\en instead of} तत्क्षेत्रं.
} ॥ \\
\vspace{3mm}

 पुनः स शङ्कुः समतलमस्तकपरिधितृतीयांशात् खघनफलक्षेत्रतुल्योऽधिकः कल्पितः । तस्मात् तत् क्षेत्रं त्रिगुणितशङ्कोर्न्यूनं भविष्यति । \\
\vspace{3mm}

पुनः पूर्ववत् कल्पितशङ्क्वन्तरनेनोच्छ्रायेण सफलकशङ्कुस्तथा कार्यो यथा शेषखण्डानि खक्षेत्रान्न्यूनानि भविष्यन्ति । अयं सफलकस्त्रिगुणितः सन् समतलमस्तकपरिधिक्षेत्रादधिको भविष्यति । सास्रशङ्कोर्भूमौ तावदुच्छ्रितं छेदितक्षेत्रं कार्यम् । एतानि च्छेदितक्षेत्राणि त्रिगुणितसास्रशङ्कुतुल्यानि भवन्ति । अयं त्रिगुणसफलकशङ्कुश्च समतलमस्तकपरिधिक्षेत्रादधिकोऽस्ति । तस्मात् छेदितक्षेत्राण्यप्यधिकानि भविष्यन्ति । इदमशुद्धम् । अस्मदिष्टं समीचीनम् ॥ \\
\vspace{3mm}

\begin{center}
\textbf{\large प्रकारान्तरम् ॥ }
\end{center}
\vspace{5mm}

यत् घनक्षेत्रं समतलमस्तकपरिधिक्षेत्रत्र्यंशान्न्यूनं भवति तत् क्षेत्रं शङ्कोरपि न्यूनं भविष्यत्यधिकेऽधिकं च तत् । तत्र प्रथमतः घनक्षेत्रं न्यूनं क्षेत्रं कल्पितम्~। इदं त्रिगुणितं
समतलमस्तकपरिधिक्षेत्रात् खक्षेत्रतुल्यं न्यूनं भविष्यति । \\

\begin{center}
\includegraphics[scale=1]{Images/rg-156.png}  
\end{center}

\newpage
पुनः प्रोक्तवत् समतलमस्तकपरिधिक्षेत्रान्तश्छेदितक्षेत्राणि तावन्ति तथा कार्याणि यथा तत् क्षेत्रं शेषखण्डानि खक्षेत्रान्न्यूनानि भवन्ति~। एतानि छेदितक्षेत्राणि कल्पितन्यूनघनक्षेत्रात् त्रिगुणादधिकानि भविष्यन्ति । पुनः शङ्क्वन्तः सफलकशङ्कुः कार्यश्छेदितक्षेत्रभूमौ~। इदं सफलकशङ्कुक्षेत्रं \renewcommand{\thefootnote}{१}\footnote{{\en K. and A. insert} कल्पित.}शङ्कोर्न्यूनं भविष्यति । इदं 
\begin{center}
\noindent\includegraphics[scale=1]{Images/rg-157.png}  
\end{center}
छेदितक्षेत्राणां तत्त्र्यंशेन तुल्यं भविष्यति ।
स च त्र्यंशो न्यूनघनक्षेत्रादधिकोऽस्ति~। तस्मात् कल्पितघनक्षेत्रं समतलमस्तकपरिधित्र्यंशात् न्यूनमस्ति । शङ्कोर्नितान्तं न्यूनं भविष्यति । पुनरप्यधिकं घनक्षेत्रं कल्पितम् । इदं त्रिगुणितं समतलमस्तकपरिधिक्षेत्रात् खक्षेत्रतुल्यमधिकं कल्पितम्~। \renewcommand{\thefootnote}{२}\footnote{°वृत्तोपरि {\en K., A.} }पुनर्वृत्ते समकोणसमचतुर्भुजं क्षेत्रं कार्यम् । तत्र तत्क्षेत्रोच्छ्रायतुल्यमेकं घनक्षेत्रं कार्यम् । एतत्कल्पितघनक्षेत्रादधिकं वा भविष्यति वा न भविष्यति । यद्यधिकं भवति तदा शक्षेत्रतुल्यमधिकं कल्पितम् । अस्य समतलमस्तकपरिधिक्षेत्रस्य चान्तरं खघनक्षेत्रादधिकं भविष्यति~। पुनः केन्द्रे \renewcommand{\thefootnote}{३}\footnote{चतुर्भुजक्षे त्रकोणेषु {\en K., A.}}खघनक्षेत्रकोणे च रेखाः संयोज्याः । एता वृत्तस्य हझवतचिह्नेषु संपातं करिष्यन्ति । पुनः संपातचिह्नेभ्यो \renewcommand{\thefootnote}{४}\footnote{वृत्तपालिस्पर्श कुर्वत्यः {\en K., A.}}वृत्तपालिपर्यन्तं रेखा निष्कास्याः । एता रेखा तदन्तरार्द्धेभ्योऽधिकाः । कुतः । अबअदरेखे मचिह्ननचिह्नवृत्तपालिसंलग्ने कार्ये । लहकरेखा हचिह्नलग्ना कल्प्या । ते द्वे रेखे लचिह्नकचिह्ने कृतसंपाते कल्पिते । पुनर्हमहनरेखे संयोज्ये । तत्र अमअनरेखे समाने भविष्यतः । हककमरेखे समाने भविष्यतः । अकं कहादधिकमस्ति । कुतः । हस्य समकोर्ण-

\newpage
\noindent त्वात् । कमादप्यधिकं भविष्यति । अकहत्रिभुजं कमहत्रिभुजादधिकं भविष्यति~। अलहत्रिभुजं लहनत्रिभुजादधिकं भविष्यति । तस्मात् अलकत्रिभुजमन्तरार्द्धादधिकं भविष्यति । एवं शेषान्तरार्द्धात् शेषत्रिभुजमधिकं भविष्यति । \\
\vspace{3mm}

अनेनैव प्रकारेण तथा कार्यं यथान्तरक्षेत्राणि खक्षेत्रान्न्यूनानि भविष्यन्ति । शेषं तथा घनक्षेत्रं भविष्यति तथा कल्पितघनक्षेत्रादधिकं\renewcommand{\thefootnote}{१}\footnote{°क्षेत्रत्रिगुणादधिकं° {\en K., A.}} न भविष्यति । इदं समतलमस्तकपरिधिक्षेत्रादधिकमस्ति । पुनरस्य भूमौ त्र्यंशतुल्यः सास्रशङ्कुः कार्यः~। क्षेत्रस्य त्र्यंशो भविष्यति । तस्मादयं कल्पितघनक्षेत्रादधिको न भविष्यति~। अयं च सफलककल्पितशङ्कोरधिकोऽस्ति । तस्मात् \renewcommand{\thefootnote}{२}\footnote{समतलमस्तकपरिधित्र्यंशादधिकतत्क्षेत्र शङ्कोरप्यधिकं भविष्यति~। {\en K., A.}}यद् घनक्षेत्रमधिकं भवति तत्समतलमस्तकपरिधितृतीयांशात् तच्छङ्कोरप्यधिकं भविष्यति । \\
\vspace{3mm}

पुनर्निश्चितं यद् घनक्षेत्रं तु शङ्कुतुल्यं भवति तत्समतलमस्तकपरिधिक्षेत्रत्र्यंशतुल्यमेव भविष्यति ॥ \\
\vspace{3mm}

\begin{center}
\textbf{\large अथ दशमं क्षेत्रभ् ॥ १० ॥ }
\end{center}
\vspace{5mm}

{\ab सजातीयसमतलमस्तकपरिधिक्षेत्रद्वयस्याथवा सजातीयशङ्कुद्वयस्य च निष्पत्तिर्वृत्तयोर्व्यासनिष्पत्तेर्धनतुल्या भवति । }\\
\vspace{5mm}

यथा अबजदहझवतवृत्ते\renewcommand{\thefootnote}{३}\footnote{°वृत्तभूमी समतलमस्तकप क्षेत्रद्वयस्य वा शङ्खद्वयस्य कल्पिते । {\en K., A.}} समतलमस्तकपरिधिक्षेत्रद्वयस्य वा शङ्कुद्वयस्य भूमी कल्पिते । अनयोर्व्यासो बदझतौ कल्पितौ । कलमनौ लम्बौ कल्पितौ । यदि बदझतव्यासनिष्पत्तिघनतुल्या अबजदलशङ्कुहझवतनशङ्क्वोर्निष्पत्तिर्न भवति तदा प्रथमशङ्कुनिष्पत्तिर्द्वितीयान्न्यूनाधिकघनक्षेत्रनिष्पत्तितुल्या भवतीति कल्पितम् ।
\renewcommand{\thefootnote}{४}\footnote{प्रथमं न्यूनघनक्षेत्रं अघनतुल्यं कल्पितम्~। {\en K., A.}}प्रथमं न्यूनघनक्षेत्रं कल्पितम् । \renewcommand{\thefootnote}{५}\footnote{अस्यान्तरं {\en V.}}व्यासान्तरं अघनक्षेत्रम् । पुनर्वृत्तान्तः सम- 

\newpage
\noindent कोणसमचतुर्भुजं\renewcommand{\thefootnote}{१}\footnote{{\en K. and A. insert} हझवतं.} कार्यम् । अस्योपरि प्रथमशङ्कूच्छ्रायतुल्यः शङ्कुः कल्पितः । पुनः शेषाणि चत्वारि चापान्यर्द्धितानि कार्याणि । तेषु पूर्णज्याः संयोज्याः । एतासु शङ्कवः कार्याः । \\
\vspace{5mm}

 एवमनेन प्रकारेण तावच्छङ्कवः कार्याः यावच्छेषखण्डानि \renewcommand{\thefootnote}{२}\footnote{अघनक्षेत्रान्यूनानि स्युः {\en K.,~A.}}अघनक्षेत्रान्न्यूनानि स्युः । तदा एभ्य एकः \renewcommand{\thefootnote}{३}\footnote{सास्रफलकशङ्कुरुत्पद्यते {\en K.,~A.}}सास्रसफलकः शङ्कुरुत्पद्यते । हसझगवफतखं तस्य भूमिर्भविष्यति । \renewcommand{\thefootnote}{४}\footnote{अस्य मस्तकं न मस्तकं भविष्यति
 {\en K., A.}}अस्य मस्तकं कल्पितशङ्कुमस्तकं भविष्यति । अयं शङ्कुः कल्पितन्यूनघनक्षेत्रादधिको भविष्यति । पुनर् अबजदवृत्ते अरबशजतदसक्षेत्रमुत्पन्नशङ्कोर्भूमेः सजातीयं कल्पितम् । एतत्क्षेत्रोपरिकल्पितशङ्कुतुल्यमुख एकः शङ्कुः कार्यः । एतौ द्वौ शङ्कू सजातीयौ भविष्यतः । कुतः । लकबदयोर्निष्पत्तिर्नमझतनिष्पत्तिसमानास्ति । कल्पितशङ्कोः सजातीयत्वात् । तस्मात् लकमननिष्पत्तिर्बकझमनिष्पत्तितुल्या भविष्यति । रकसमनिष्पत्तिसमानापि भविष्यति । तस्मात् बकलत्रिभुजझमनत्रिभुजे सजातीये भविष्यतः । एवं रकलसमनत्रिभुजे अपि सजातीये भविष्यतः । कुतः । कमयोः समकोणत्वात् । अनयोः संबन्धिभुजौ सजातीयौ । तस्मात् बलझनयोर्निष्पत्तिः रलसनयोश्च सैव निष्पत्तिर्भविष्यति~। पुनरपि बकरत्रिभुजझमसत्रिभुजे सजातीये स्तः । बकरकोणझमसकोणयोः समानभावित्वेन । पुनस्तत्संबन्धिभुजयोः सजातीयत्वेन बरझसयोर्निष्पत्तिः सैव भविष्यति । बरलत्रिभुजझसनत्रिभुजयोर्भुजौ मिथः सजातीयौ भविष्यतः । तस्मादेतन्त्रिभुजद्वयं सजातीयं \renewcommand{\thefootnote}{५}\footnote{भविष्यति {\en K., A.}}संत्स्यति । बरकलशङ्कुः झसमनशङ्कुश्चोभौ सजातीयौ भविष्यतः ।
कुतः । अनयोर्वेष्टितत्रिभुजयोः सजातीयत्वात् । एवं वेष्टिताः सर्वेऽपि शङ्कवः सजातीयाः पतिष्यन्ति । प्रत्येकशङ्कोः स्वसजातीयशङ्कुना निष्पत्तिस्तयोः सजातीयभुजयोर्घनतुल्या भविष्यति । बदझतयोर्नि-

\newpage
\noindent ष्पत्तेर्घनतुल्यापि भविष्यति । तस्मात् बदझतनिष्पत्तिघनतुल्या
अब-
\begin{center}
\includegraphics[scale=1]{Images/rg-158.png}  
\end{center}
जदलशङ्क्वन्तःपातिसास्रोत्पन्न\renewcommand{\thefootnote}{१}\footnote{{\en K. and A. insert} शङ्कोर्निष्पत्ति. 
}शङ्कुहझवतनशङ्कन्तः\renewcommand{\thefootnote}{२}\footnote{°न्तर्गतसकलशङ्कुनिष्पत्तितुल्या भविष्यति । {\en K., A.}}पातिसास्रोत्पन्नशङ्कोर्निष्पत्तितुल्या भविष्यति । अबजदलशङ्क्वन्तःपातीयसास्रशङ्कोर्निष्पत्तिः अबजदलशङ्कुना तथा भविष्यति यथा हझवतनान्तशङ्कोः कल्पितन्यूनघनक्षेत्रेणास्ति । अयं हझवतनान्तःपातिसास्रशङ्कुः
कल्पितन्यूनधनक्षेत्राधिकोऽस्ति । तस्मात् अबजदलान्तः \renewcommand{\thefootnote}{३}\footnote{{\en K.,~ and A. have} सफलकशङ्कुः {\en for} °पातिसास्रशङ्क:. }पातिसास्त्रशङ्कुः अबजदलशङ्कोरधिको भविष्यति । इदमशुद्धम् । \\
\vspace{5mm}

पुनर्वदझतनिष्पत्तिर्घनतुल्या प्रथमशङ्कुद्वितीयशङ्क्वधिकघनक्षेत्रनिष्पत्तिः कल्पिता~। तदा झतवदनिष्पत्तिघनतुल्या हझवतनशङ्कुअबजदलशङ्कुन्यूनक्षेत्रयोर्निष्पत्तिर्भविष्यति । पूर्वरीत्या
\renewcommand{\thefootnote}{४}\footnote{इदमप्यनुपपन्नम् । इष्टमस्मत्समीचीनम् । {\en K., A.}}एनमप्यशुद्धं कुर्मः । तदेष्टमस्मत् सेत्स्यति । पुनः समतलमस्तकपरिधिक्षेत्रेष्वपि सेत्स्यति\renewcommand{\thefootnote}{५}\footnote{भविष्यति {\en K., A.}} ॥ \\
\vspace{3mm}

\begin{center}
\textbf{\large अथैकादशं क्षेत्रम् ॥ ११ ॥ }
\end{center}
\vspace{5mm}

{\ab समतलमस्तकपरिधिक्षेत्रयोः समानलम्बयोर्निष्पत्तिस्तयोर्भूमिनिष्पत्तितुल्या भविष्यति । एवं द्वयोः शङ्क्वोरपि निजभूमिनिष्पत्तिसमाना भविष्यति । }

\newpage
\noindent क्षेत्रं पूर्ववत् कल्पनीयम् । यदि अबजदभूमिहझवतभूम्योर्निष्पत्तिर्यस्य शङ्कोर्लम्बः कलमस्ति यस्य च लम्बो मनमस्त्येतयोर्निष्पत्तिसमा यदि न स्यात् तदा
प्रथमशङ्कोर्निष्पत्तिर्द्वितीयशङ्कोर्न्यनघनक्षेत्रेण समानास्तीति कल्पितम् । पूर्ववद्द्वितीयशङ्क्वन्तःपातिसास्रशङ्कुः कल्पितघनक्षेत्रादधिको भवति तथा कार्यः । प्रथमशङ्क्वन्तः पातिसास्रशङ्कुः सजातीयः कार्यः । एतौ समानलम्बौ भविष्यतः । द्वयोः 
\begin{center}
\includegraphics[scale=1]{Images/rg-159.png}  
\end{center}
सास्रशङ्कोर्निष्पत्तिर्बदवर्गझतवर्गनिष्पत्तिसमाना भविष्यति । अबजदवृत्तहझवतवृत्तयोर्निष्पत्तिसमानापि भविष्यति । कललम्बस्य शङ्कोः कल्पितन्यूनघनक्षेत्रनिष्पत्तेरपि\renewcommand{\thefootnote}{१}\footnote{निष्पत्तेः समानापि भविष्यति । {\en V,}} समा भविष्यति । तस्मात्
प्रथमसास्रसफलकशङ्कोर्निष्पत्तिः प्रथमशङ्कुना तथास्ति यथा द्वितीयसास्रशङ्कोर्निष्पत्तिः कल्पितन्यूनघनक्षेत्रेणास्ति । द्वितीयः सास्रशङ्कुः कल्पितघनक्षेत्रादधिकोऽस्ति । तदा प्रथमसास्रशङ्कुः प्रथमशङ्कोरधिको भविष्यति । इदमशुद्धम् ॥ \\
\vspace{5mm}

एवं सा निष्पत्तिर्यदाऽधिकघनक्षेत्रेण भवति तदा साप्यशुद्धैव भविष्यति । तस्मात् शङ्कुद्वयेऽपीष्टमस्माकं समीचीनम् । तदा समतलमस्तकपरिधिद्वयेपीष्टमुपपन्नम् ॥\\
\begin{center}
\textbf{\large अथ द्वादशं क्षेत्रम् ॥ १२ ॥\\}
\end{center}
\vspace{5mm}

{\ab यदि समतलमस्तकपरिधिक्षेत्रे वा शङ्कुद्वये वा समाने }

\newpage
\noindent भवतस्तदा तयोर्भूम्योर्निष्पत्तिर्लम्बनिष्पत्तेर्विलोमा भविष्यति । \renewcommand{\thefootnote}{१}\footnote{ईदृशी निष्पत्तिश्चेत् समानौ भविष्यतः {\en K., A.}}एतद्रूपा निष्पत्तिर्भविष्यति तदा समानौ भवतः ।\\
\vspace{5mm}

यथैकक्षेत्रस्य भूमिः अबजदवृत्तं कल्पिता । कलं लम्बश्च कल्पितः । यद्द्वितीयक्षेत्रभूमी\renewcommand{\thefootnote}{२}\footnote{क्षेत्रस्य {\en V.} } हझवतं कल्पिता । मनं लम्बश्च कल्पितः । यदि द्वौ लम्बौ समानौ भवतो यदा भूमी समाने भविष्यतः । तदास्मदिष्टमुत्पन्नं भविष्यति । यदि समानौ न भवतस्तदा मनलम्बः कललम्बादधिकः कल्पितः । पुनर्मनलम्बात् कनतुल्यं मसं पृथक्कार्यम् । तदा हवभूमौ \renewcommand{\thefootnote}{३}\footnote{मललम्बे च {\en K., A.}}मसलम्बतुल्यशङ्कुरुत्पाद्यः । प्रथमम् अबजदलशङ्कुहझवतनशङ्कु समानौ कल्पितौ । तदानयोः शङ्क्वोर्निष्पत्तिर्हझवतसशङ्कुना एकरूपा भविष्यति । पुनरेकशङ्कोर्निष्पत्तिर्हझवतसशङ्कुना तथास्ति यथा भूमेर्निष्पत्तिर्भूम्यास्ति द्वितीयशङ्कोर्निष्पत्तिर्मनलम्बमसलम्बनिष्पत्तितुल्यास्ति~। तस्मात्
अबजदभूमिहझवतभूम्योर्निष्पत्तिर्मनमसनिष्पत्त्या समाना भविष्यति । मनकलनिष्पत्तेरपि समाना भविष्यति~। 
\begin{center}
\includegraphics[scale=1]{Images/rg-160.png}  
\end{center}
\vspace{5mm}

 पुनर्निष्पत्तय एतद्रूपाः \renewcommand{\thefootnote}{४}\footnote{कल्पिता: {\en K., A,}}कल्प्याः । तदा अबजदलशङ्कहझवतनशङ्क्वोर्निष्पत्तिर्हझवतसशङ्कुना एकरूपा भविष्यति । तस्मादेतौ समानौ भविष्यतः । एवं समतलमस्तकपरिधिक्षेत्रद्वयमपि । इदमेवास्मदिष्टम् ॥ 

\newpage

अथ \renewcommand{\thefootnote}{१}\footnote{च यदिदं कल्पितं {\en K., A.} इदं {\en for} अथ {\en in V.}}यत्कथितं हझवतनशङ्कुहझवतसशङ्कोर्निष्पत्तिर्मनमसनिष्पत्तितुल्यास्ति तदेतदर्थम् ।
मनमसनिष्पत्तिर्झतनझतसशङ्क्वोर्निष्पत्तितुल्या न भवति तदा झतनशङ्कोर्न्यूनाधिकेन केनचित् शङ्कुना
तन्निष्पत्तिः कल्पिता । तदा न्यूनं घनक्षेत्रं कल्पितम् । पुनर्झतसशङ्कोरन्तः सास्रशङ्कुर्यथा भवति तथा कार्यः । \renewcommand{\thefootnote}{२}\footnote{न्यूनघनक्षेत्राधिकः {\en K., A.}}कल्पितघनक्षेत्रादधिको झतनशङ्कुभूमावन्यः शङ्कुः कार्यः । एतयोः
सास्रशङ्क्वोरन्तस्त्रिभुजशङ्कवः तुल्यसंख्याकाः पतिष्यन्ति तदैकस्य स्वसजातीयेन निष्पत्तिस्तथा भविष्यति यथा सर्वेषां निष्पत्तिः सर्वैरपि । यथा हतमनस्य स्वजातीयेन हतमसेन निष्पत्तिर्महनत्रिभुजहमसत्रिभुजनिष्पत्तितुल्यास्ति । पुनर्मनमसयोरपि निष्पत्तिस्तुल्यास्ति । तदैकतरस्य बृहत्सास्रशङ्कोः लघुसास्रशङ्कोश्च निष्पत्तिर्झतनशङ्कुन्यूनघनक्षेत्रनिष्पत्त्या
तुल्या भविष्यति । तस्मात् बृहत्सास्रशङ्कोर्निष्पत्तिः स्वेष्टशङ्कुना तथास्ति
न्यूनसफलकशङ्कुर्न्यूनघनक्षेत्रेण निष्पत्त्या तुल्यास्ति~। न्यूनसफलकशङ्कुर्न्यूनघनक्षेत्रादधिकोऽस्ति । बृहच्छङ्कुः स्वशङ्कोरप्यधिको भविष्यति~। इदमशुद्धम् ॥ \\
\vspace{5mm}

एवमधिकघनक्षेत्रेण या निष्पत्तिर्भविष्यति साप्यशुद्धैव । तस्मात्
मनमसयोर्निष्पत्तिः शङ्क्वोर्निष्पत्तितुल्या भविष्यति ॥ \\
\vspace{3mm}

\begin{center}
\textbf{\large अथ त्रयोदशं क्षेत्रम् ॥ १३ ॥ }
\end{center}
\vspace{5mm}

{\ab एककेन्द्रकवृत्तद्वयस्य मध्य एकं क्षेत्रं तथा कर्त्तुमिच्छास्ति यथास्य भुजा लघुक्षेत्रं न स्पृशन्ति । }\\
\vspace{5mm}

 यथा अबजदवृत्तं लबवृत्तं मकेन्द्रं कल्पितम् । पुनरजव्यासबदव्यासौ द्वयोर्वृत्तयोर्लम्बवत्कृतसंपातौ कल्पितौ । पुनर्वचिह्नात् झवतरेखा वलवृत्तपालिलग्ना निष्कास्या । इयं झवतरेखा अजरेखायाः
समानान्तरा भविष्यति । पुनर् अदचापार्द्धं कार्यम् । \renewcommand{\thefootnote}{३}\footnote{पुनःपुनरर्धितं~{\en V.}\\
 भा ० २३}पुनरर्द्धितं यावत् 

\newpage
\noindent हदचापं झदचापान्न्यूनं भवति । हझरेखा झतरेखायाः समानान्तरा कार्या । इयं वलवृत्ते संपातं न करिष्यति । पुनर्हदपूर्णज्या संयोज्या । पुनर्हदचापतुल्यानि वृत्तचापानि कार्याण्येतेषां पूर्णजीवाः\renewcommand{\thefootnote}{१}\footnote{°जीवा च संयोज्या {\en V.}} च
संयोज्याः । इष्टमस्माकं भविष्यति ।। \\
\begin{center}
\includegraphics[scale=1]{Images/rg-161.png}  
\end{center}
\vspace{5mm}

\begin{center}
\textbf{प्रकारान्तरम् । }
\end{center}
\vspace{2mm}

 केन्द्रोपरि अमबसमकोणः कार्यः । पुनम् अमोपरि अजमं वृत्तार्द्धं कार्यम्~। पुनर् अलरेखोपरि दचिह्नं \renewcommand{\thefootnote}{२}\footnote{कल्पयेत् {\en K., A.} }कल्पितम् । पुनर्मकेन्द्रे मदव्यासार्द्धेन दजतवृत्तं कार्यम् । पुनर् अमबकोणस्यार्द्धं पुनः
पुनस्तावत्कार्यं यावदर्द्धरेखा दजचापे कचिह्ने लगति । सा मकरेखा कल्पिता । \renewcommand{\thefootnote}{३}\footnote{अहरेखा संयोज्या । इयं रेखा झचिह्नपर्यन्तं वर्धिता कार्या {\en K., A.}}इयं रेखा हचिह्नपर्यन्तं वर्द्धिता कार्या । पुनर् अहरेखा योज्या । इयं झचिह्नपर्यन्तं वर्द्धिता कार्या । तदास्मात् अझरेखा वलवृत्तं न लगिष्यति~। कुतः । महस्य मकादधिकत्वात् । मदादप्यधिकत्वात्~। मदं मलादधिकमस्ति । अझचापतुल्यानि वृत्तखण्डानि भविष्यन्ति । यद्येषां पूर्णजीवा योज्यते तदास्माकमिष्टं सेत्स्यति\renewcommand{\thefootnote}{४}\footnote{भविष्यति {\en K, A.}}॥\\ 
\begin{center}
\textbf{\large अथ चतुर्दशं क्षेत्रम् ॥ १४ ॥ }
\end{center}
\vspace{5mm}

{\ab एककेन्द्रकयोर्गोलयोर्मध्ये \renewcommand{\thefootnote}{५}\footnote{{\en K., A. insert} बृहद्गोलान्तः.}एकं बहुधरातलयुक्तं घनक्षेत्रं }


\newpage
\noindent तथा कर्त्तुमिच्छास्ति यथा कल्पितानि धरातलानि लघुगोलं न स्पृशन्ति । पुनर्यद्यन्यगोले एतत्सजातीयघनक्षेत्रमन्यत् \renewcommand{\thefootnote}{१}\footnote{क्रियते {\en for} कुर्मः {\en K., A.} }कुर्मस्तदानयोर्घनक्षेत्रयोर्निष्पत्तिर्द्वयोर्गोल-योर्व्यासनिष्पत्तेर्घनतुल्या भविष्यति । \\
\vspace{5mm}

ययोर्गोलयोरेकं केन्द्रमस्ति तयोः केन्द्रगतमेकं धरातलं कल्पितं तद्धरातलबृहद्वृत्तसंपातादबजदवृत्तमुत्पन्न\renewcommand{\thefootnote}{२}\footnote{कल्पितम् {\en for} उत्पन्नं कार्यम् {\en K., A.}} कार्यम् । लघुवृत्तसंपातात् हझवतवृत्तं कार्यम्~। द्वयोः केन्द्रं कचिह्नं कल्पितम् । पुनर् अजव्यासबदव्यासौ लम्बरूपौ कृतसंपातौ कल्पितौ~। पुनर् अबजदवृत्तमध्ये समानबहुभुजं क्षेत्रं तथा कार्यं यथा हझवतलघुवृत्तं न स्पृशति । तथा बमं मलं लअं भुजाः कल्पिताः~। पुनर्मकरेखा संयोज्या~। सचिह्नपर्यन्तं वर्द्धिता कार्या । लकरेखा च योज्या नचिह्नपर्यन्तं वर्द्धिता । कचिह्नादेको लम्बः अबजदवृत्तधरातले तथा पात्यो यथा \renewcommand{\thefootnote}{३}\footnote{बृहद्गोलाद्वहिर्न गच्छति {\en K., A.} }बृहद्गोलं 
 \begin{vwcol}[widths={0.65,0.35}, sep=.8cm, rule=0pt]
स्पृशति । स लम्बः कगं कल्पितः । पुनरेकं धरातलं   लचिह्ननचिह्नगचिह्नगतं कल्पितम् । पुनरन्यद्धरातलं मगसचिह्नगतं कल्पितम्~। प्रथमधरातलबृहद्गोलयोः संपातात् लगनम्\\ \noindent  अर्द्धवृत्तमुत्पन्नं कल्पितम्~। द्वितीयधरातलमहद्गोलसंपातात् मगंसम् अर्द्धवृत्तमुत्पन्नं कल्पितम्~। पुनर्लगचापं मगचापं प्रत्येकं वृत्तस्य चतुर्थोशो भविष्यति । लगचापस्य लखखफफगखण्डानि कार्याणि । मगचापस्य मररशशग-\\
\noindent \includegraphics[scale=1.1]{Images/rg-162.png}  
\end{vwcol}
\vspace{-6mm}

\noindent खण्डानि कार्याणि । एतानि समानानि कार्याणि~। अबचापस्य \renewcommand{\thefootnote}{४}\footnote{खण्डसमानीत्यर्थः {\en K., A.} }यावन्ति खण्डानि तषां सममररशशगनानीत्यर्थः । पुनर् रखरेखाशफरेखा च संयोज्या । पुनर् रचिह्नात् मससंपातरेखायां रतलम्बो नेयः
 । खचिह्नात् लनसंपातरेखायां खसलम्बो नेयः\renewcommand{\thefootnote}{५}\footnote{कार्यः {\en K., A.}} । एतौ लम्बौ अबजद-

\newpage
\noindent धरातले लम्बौ भविष्यतः । एतौ च समानान्तरौ भविष्यतः समानौ च भविष्यत~। कुतः । मरलखचापयोः साम्यात् । एतौ रतखसौ \renewcommand{\thefootnote}{१}\footnote{द्विगुणरमखलचापयो: {\en V.} }रमखलद्विगुणचापयोः पूर्णजीवयोरर्द्धरूपौ\renewcommand{\thefootnote}{२}\footnote{अर्धौ जातौ {\en K., A.}} जातौ । पुनरेतौ रतखसौ मतलसरेखे समाने पृथक् करिष्यतः~। पुनस्तसरेखा संयोज्या । इयं तसरेखा मलरेखायाः समानान्तरा भविष्यति~। कुतः । कततमयोर्निष्पत्तिः कससलयोर्निष्पत्तिसमानास्ति । तसं मलात् न्यूनं भविष्यति । कुतः । एतौ कतकमयोर्निष्पत्तौ स्तः । रखरेखा तसरेखा
च मिथः समानान्तरे भविष्यतः समाने च भविष्यतः । कुतः । रतरेखा खसरेखा च मिथः समाना समानान्तरा च भवति । तस्मात् रखलमरेखे मिथः समानान्तरे भविष्यतः । रखं लमान्न्यूनं भविष्यति । तस्मात् रमलखचतुर्भुजं एकस्मिन् धरातले भविष्यति । इदं चतुर्भुजं तस्य घनक्षेत्रस्यैकं फलकं भविष्यति । अनेन \renewcommand{\thefootnote}{३}\footnote{लघुवृत्तगोलस्य {\en V.}}लघुवृत्तस्य गोलस्य स्पर्शो न कृतः । कुतः~। अस्य रममललखैः समैस्त्रिभुजैः स्पर्शो न कृतः । पुनश्चतुर्थभुजो रखम् एभ्यो न्यूनोऽस्ति । एवं निश्चीयते रशफखचतुर्भुजमप्येकधरातले भविष्यति । लघुगोलस्पर्शं न-करिष्यति गशफत्रिभुजमपि लघुगोलस्पर्शं न करिष्यति । \\
\vspace{10mm}

 अनेनैव प्रकारेण सर्वचापेषु खण्डेषु \renewcommand{\thefootnote}{४}\footnote{एतद्रूपफलकानि {\en A.}}चैतद्रूपाण्यस्त्राणि कार्याणि
 । तदास्माकमिष्टघनक्षेत्रं पूर्णं भविष्यति । एतद्धनक्षेत्रसजातीयमन्यस्मिन् गोले यदि कार्यं भवेत्तदोभे घनक्षेत्रे शङ्कूनां योगेनोत्पद्येते । कीदृशानां शङ्कूनाम् । येषां भूमिर्घनक्षेत्राणां फलकानि पतिष्यन्ति । शङ्कूनां मुखं च गोलयोः केन्द्रं भविष्यति । यावन्तः शङ्कव एकस्मिन् गोले भवन्ति तावन्त एव द्वितीयगोले \renewcommand{\thefootnote}{५}\footnote{भविष्यन्ति {\en V.}}भवन्ति मिथश्च सजातीयानि भविष्यन्ति । कुतः । वेष्टितधरातलानां सजातीयत्वात् । तस्मादेकगोलस्यैकशङ्कोर्निष्पत्तिर्द्वितीयगोलस्य \renewcommand{\thefootnote}{६}\footnote{खसजातीय° {\en V.}}सजातीयशङ्कुना तथास्ति यथैषां 

\newpage
\noindent सजातीयभुजनिष्पत्तिघनतुल्या स्यात् । एषां भुजा गोलयोर्व्यासार्द्धमिताः सन्ति~। तस्मादनयोर्निष्पत्तिर्व्यासार्द्धनिष्पत्तिघनतुल्या भविष्यति । व्यासार्द्धयोर्निष्पत्तिः व्यासनिष्पत्तितुल्यास्ति । तस्मात् शङ्कूनां निष्पत्तिर्गोलव्यासयोर्निष्पत्तिघनतुल्या भविष्यति । यथैकशङ्कोरेकशङ्कुना निष्पत्तिस्तथा सर्वयोगशङ्कोः सर्वयोगशङ्कुना निष्पत्तिः । सर्वयोगशङ्कुस्तु तदेव घनक्षेत्रमस्ति । तस्माद्घनक्षेत्रयोर्निष्पत्तिर्द्वयोर्व्यासयोर्निष्पत्तिघनतुल्या भविष्यति । इदमेवास्माकमिष्टम् ॥ \\

\begin{center}
\textbf{\large अथ पञ्चदशं क्षेत्रम् ॥ १५ ॥}
\end{center}
\vspace{2mm}

{\ab गोलस्य निष्पत्तिर्गोलेन व्यासयोर्निष्पत्तिघनतुल्या भवति ।}\\

 यथा अजगोलः कल्पितः । बदं व्यासः कल्पितः । द्वितीयो हवगोलो झतं व्यासश्च कल्पितः । यदि बदझतव्यासनिष्पत्तिघनतुल्या \renewcommand{\thefootnote}{१}\footnote{अजगोलहबगोलयोर्निष्पत्तिर्न चेत्त् {\en K.,A.}}गोलयोर्निष्पत्तिर्न भवति तदा अजगोलनिष्पत्तिर्हवन्यूनाधिकगोलेन 
\begin{center} \includegraphics[scale=0.9]{Images/rg-163.png}  
\end{center}



\newpage
\noindent भविष्यतीति कल्पितम् । तदा हवान्न्यूनो अगोलः कल्पितः । पुनर्हवगोलकेन्द्रे अगोलतुल्यः कमगोलः कल्पितः । पुनर्हवक्षेत्रमध्ये बह्वस्रयुक्तं घनक्षेत्रं तथा कार्यं यथा कमगोले स्पर्शं न करोति । पुनर् अजगोलमध्ये एकं क्षेत्रं तद्घनक्षेत्रसजातीयं कल्पितम्~। तस्मात् बदझतनिष्पत्तिघनतुल्या अजगोलस्य घनक्षेत्रस्य हवगोलस्य घनक्षेत्रनिष्पत्तिरस्ति । बदझतनिष्पत्तिघनतुल्या \renewcommand{\thefootnote}{१}\footnote{अजगोलतुल्यकमगोलयोर्निष्पत्तिः {\en K.V.}}अजगोलअगोलयोर्निष्पत्तिः
कल्पितासीत्~। तथा अजकमगोलयोर्निष्पत्तितुल्याप्यस्ति । तस्मात् अजगोलघनक्षेत्रहवगोलघनक्षेत्रयोर्निष्पत्तिः अजकमगोलयोर्निष्पत्तितुल्या भविष्यति । अजघनक्षेत्रस्य निष्पत्तिः अजगोलेन तथा भविष्यति यथा हवगोलघनक्षेत्रस्य निष्पत्तिः कमगोलघनक्षेत्रेणास्ति । कमगोलो हवगोलघनक्षेत्रान्न्यूनोऽस्ति । तस्मात्
अजगोलः अजगोलघनक्षेत्रान्न्यूनो भविष्यति । इदमशुद्धम् ॥ \\
\vspace{5mm}

 पुनर्बदझतनिष्पत्तिघनतुल्या अजगोलहवगोलाधिकयोर्निष्पत्तिः कल्पिता । तस्मात् झतबदनिष्पत्तिघनतुल्या हवगोलस्य अजगोलान्न्यूनगोलेन निष्पत्तिर्भविष्यति~। इदमप्यशुद्धं कुर्मः । तस्मादस्मदिष्टं समीचीनम् । \\
 \begin{center}
{\small  श्रीमद्राजाधिराजप्रभुवरजयसिंहस्य तुष्ट्यौ द्विजेन्द्रः\\
श्रीमत्सम्राड् जगन्नाथ इति समभिधारूढितेन प्रणीते ।\\
ग्रन्थेऽस्मिन्नाम्नि रेखागणित इति सुकोणावबोधप्रदात--\\
र्यध्यायोऽध्येतृमोहापह इह विरतिं द्वादशः संगतोऽभूत् ॥\\
॥ इति द्वादशोऽध्यायः ॥ १२ ॥ }\\
\rule{0.7in}{0.3pt}

\end{center}

\newpage
\afterpage{\fancyhead[CE] {रेखागणितम्}}
\afterpage{\fancyhead[CO] {त्रयोदशमोध्यायः}}
\afterpage{\fancyhead[LE,RO]{\thepage}}
\cfoot{}
\newpage
%%%%%%%%%%%%%%%%%%%%%%%%%%%%%%%%%%%%%%%%%%%%%%%%%%%%%%%%%%%%%%
\newpage
\thispagestyle{empty}
\begin{center}
\textbf{\LARGE ॥ अथ त्रयोदशाध्यायः प्रारभ्यते ॥}
\end{center}
\vspace{3mm}

\begin{center}
\textbf{॥ तत्रैकविंशतिक्षेत्राणि सन्ति ॥}
\vspace{5mm}

 \textbf{\large अथ प्रथमं क्षेत्रम् ॥ १ ॥ }
  \end{center}
  \vspace{2mm}
  
{\ab \renewcommand{\thefootnote}{१}\footnote{यस्या रेखाया {\en V., D.} तथैकरेखाया खण्डद्वयचिकीर्षास्ति यथा
संपूर्णरेखाया निष्पत्तिर्महत्खण्डेन महत्खण्डलघुखण्डयोर्निष्पत्तितुल्या स्यात् तत्र
रेखार्धं महत्खण्डेन युक्तं तद्वर्गः पञ्चगुणितरेखार्द्धवर्गतुल्यो भवति ॥ {\en K., A.} }एकस्या रेखायास्तथा खण्डद्वयं कार्यं यथा संपूर्णरेखाया निष्पत्तिर्महत्खण्डेन तथा स्यात् यथा महत्खण्डस्य च लघुखण्डेनास्ति । अर्द्धरेखा महत्खण्डेन युक्ता कार्या तस्या वर्गः पञ्चगुणितार्द्धरेखावर्गतुल्यो भवति । }\\
\vspace{5mm}


 यथा अबरेखा कल्पिता । अस्या महत्खण्डम् अजं कल्पितम् ।
\renewcommand{\thefootnote}{२}\footnote{अदम् अर्द्धरेखा कल्पिता । अनया अजं.}अदं रेखार्धं कल्पितम् । अर्द्धरेखयानया अजं युतं कृतं तस्मात् जदवर्गः पञ्चगुणितेन अदवर्गेण तुल्यो भविष्यति । \renewcommand{\thefootnote}{३}\footnote{अस्योपपत्तिः {\en K.,~A.}}कुतः ।  जदरेखोपरि जहं समकोणसमचतुर्भुजं कार्यम्~।
\begin{center}
\noindent \includegraphics[scale=0.9]{Images/rg-164.png}  
\end{center}
 अलरेखा निष्कासनीया । क्षेत्रं संपूर्णं कार्यम् । अबरेखोपरि अझं समकोणसमचतुर्भुजं कार्यम् । तजरेखा कचिह्नपर्यन्तं वर्द्धनीया । अबतुल्या अवरेखा अदरेखातुल्याया स अमरेखाया द्विगुणास्ति । तदा अकक्षेत्रं असक्षेत्राद्द्विगुणं भविष्यति । बकक्षेत्रं अबबजघाततुल्यं अजवर्गतुल्यलसक्षेत्रेण समानमस्ति । तस्मात् चतुर्गुणअदवर्गतुल्यं अझसमकोणसमचतुर्भुजं खगरक्षेत्रस्य समानं भविष्यति । यदि अदवर्गो योज्यते तदा सर्वं जहं पञ्चगुणितअदवर्गतुल्यं भविष्यति~। 

\newpage
\begin{center}
 अथ \renewcommand{\thefootnote}{१}\footnote{द्वितीयक्षेत्रम् {\en and so in other places. V.}}द्वितीयं क्षेत्रम् ॥ २ ॥ 
\end{center}
\vspace{2mm}

{\ab पूर्वप्रकारेण अबबजघातः अजवर्गतुल्योऽस्ति । पुनर् अबअजघात उभयोर्युक्तः कार्यः । तदा अबवर्गतुल्यः अदवर्गश्चतुर्गुणः अबअजघाततुल्यद्विगुणअदअजघातअजवर्गयोगस्य तुल्यो भविष्यति । पुनर् अदवर्ग उभयोर्युक्तः कार्यः~। तदा पञ्चगुणित अदवर्गतुल्यो जदवर्गो भविष्यति । इदमेवेष्टम् ॥}\\
\begin{center}
\textbf{\large अथ तृतीयं क्षेत्रम् ॥ ३ ॥ }
\end{center}
\vspace{2mm}

{\ab यस्या रेखाया न्यूनाधिके खण्डे क्रियेते तस्या रेखाया वर्गः पञ्चगुणितैकखण्डवर्गसमो भवति । द्वितीये खण्डे एका रेखा तथा योज्या यथा \renewcommand{\thefootnote}{२}\footnote{प्रथमखण्डद्विगुणतुल्या {\en K., A.}}द्विगुणप्रथमखण्ड-तुल्या भवति । तदा द्वितीयखण्डयोज्यरेखायाश्च निष्पत्तिर्द्वितीखण्डेन तथास्ति यथा द्वितीयखण्डस्य निष्पत्तिर्योगरेखयास्ति । }\\

 यथा दजरेखा कल्पिता । अस्या वर्गो दअखण्डस्य पञ्चगुणितवर्गतुल्यः कल्पितः~।\\

\begin{center}
\noindent \includegraphics[scale=0.9]{Images/rg-165.png}  
\end{center}
जबं योगरेखा कल्पिता । तदा अबरेखा जचिह्नोपरि \renewcommand{\thefootnote}{3}\footnote{ पूर्वोक्तनिष्पत्तेः {\en is omitted in K., A.}}पूर्वोक्तनिष्पत्तेर्भागद्वयं प्राप्स्यति ।
महत्खण्डम् अजं भविष्यति । \\

\begin{center}
\textbf{ अत्रोपपत्तिः । }
\end{center}
\vspace{2mm}

 क्षेत्रं पूर्ववत् पूर्णं कार्यम् । अखक्षेत्रं जहक्षेत्राच्छोध्यम् । तदा शेषं खगरक्षेत्रं चतुर्गुणअदवर्गतुल्यं भविष्यति । अबवर्गतुल्यं भविष्यति । अकक्षेत्रं मजक्षेत्राद्द्विगुणमस्ति । मजमहयोगतुल्यमप्यस्ति ।
शेषं 

\newpage
\noindent लसक्षेत्रम् अजवर्गतुल्यं जझक्षेत्रसमानं भविष्यति । इदं अबबजघातोऽस्ति । ततोऽस्मदिष्टं समीचीनम् ॥\\
\begin{center}
\textbf{\large अथ चतुर्थं क्षेत्रम् ॥ ४ ॥ }
\end{center}
\vspace{2mm}

{\ab यदि जदवर्गात् दअवर्गः शोध्यते तदा शेषं दअअजघातस्य द्विगुणेन अबअजघाततुल्येन अजवर्गयुक्तेन \renewcommand{\thefootnote}{१}\footnote{तुल्यं चतुर्गुणित {\en \& c. D., V.}}तुल्यमवशिष्यते । इदं चतुर्गुणितदअवर्गेण समानं भविष्यति । अबवर्गतुल्यं भविष्यति । पुनर् अबअजवातो द्वयोः शोध्यते तदा
\begin{center}
\noindent \includegraphics[scale=0.7]{Images/rg-166.png}  
\end{center}
शेषः अजवर्गः अबबजघाततुल्यो भविष्यति । ततोऽस्मदिष्टं समीचीनं भविष्यति~। क्षेत्रं पूर्वोक्तवत् ज्ञेयम् ॥ }\\
\begin{center}
\textbf{\large अथ पञ्चमं क्षेत्रम् ॥ ५ ॥ }
\end{center}
\vspace{2mm}

{\ab यस्या रेखाया निष्पत्तिर्महत्खण्डेन महत्खण्डलघुखण्डनिष्पत्त्या तुल्या भवति~। \renewcommand{\thefootnote}{२}\footnote{पुनस्तत्रैव महत्खण्डस्यार्धं चेद्योज्यते {\en K., A.}}पुनर्महत्खण्डस्यार्द्धं लघुखण्डयुक्तं कार्यम् । तदा योगवर्गः \renewcommand{\thefootnote}{३}\footnote{पश्चगुणितमहत्खण्डार्धवर्गसमो भवति {\en K., A.}\\
 भा० २४}पञ्चगुणितेन महत्खण्डार्द्धवर्गेण समो भविष्यति । }\\

 यथा अबरेखा कल्पिता । तस्या महत्खण्डम् अजं कल्पितम् । महत्खण्डस्यार्द्धं दजं कल्पितम् । तस्मात् दबवर्गः पञ्चगुणितजदवर्गसमो भविष्यति । \\
 \begin{center}
\textbf{अस्योपपत्तिः । }
\end{center}
\vspace{2mm}

अबरेखोपरि अहं समकोणसमचतुर्भुजं कार्यम् । बझकर्णः सं-- 

\newpage
\noindent योज्यः । पुनर्दवजतरेखे अझरेखायाःसमानान्तरे निष्कास्ये । क्षेत्रं संपूर्णं\renewcommand{\thefootnote}{१}\footnote{पूर्णं {\en K., A.}} 
 \begin{vwcol}[widths={0.7,0.3}, sep=.8cm, rule=0pt]
कार्यम्~। अददजरेखयोः समानभावित्वेन अफक्षेत्रजफक्षेत्रकगक्षेत्रगतक्षेत्राणि मिथः समानानि भविष्यन्ति~। मलक्षेत्रसवक्षेत्रफखक्षेत्रलतक्षेत्राणि चत्वारि समकोणसमचतुर्भुजक्षेत्राणि समानानि भविष्यन्ति~। अबबजघातो जहक्षेत्रतुल्यः तरसक्षेत्रतुल्योऽपि अजवर्गस्य
मतक्षेत्रतुल्यस्य समो भविष्यति~। चतुर्गुणफखक्षेत्रतुल्योऽपि भविष्यति~।\\
\noindent \includegraphics[scale=0.8]{Images/rg-167.png}  
\end{vwcol}
\vspace{-4mm}
\noindent पुनः फखक्षेत्रमुभयोर्युक्तं कार्यम्~। तदा दगक्षेत्रं दबवर्गतुल्यं पञ्चगुणितफखक्षेत्रं भविष्यति । पञ्चगुणितदजवर्गस्यापि समानं भविष्यति ।\\
\vspace{3mm}

\begin{center}
\textbf{\large अथ षष्ठं क्षेत्रम् ॥ ६ ॥ }
\end{center}
\vspace{5mm}

{\ab अबबजघाततुल्यः अजजबघातजबवर्गयोगोऽस्ति\renewcommand{\thefootnote}{२}\footnote{°योगो द्विगुणदजजबघातेन जबवर्गयुतेन तुल्यो भवति {\en K., A.}} । अयं दजजबघातो द्विगुणो जबवर्गयुतस्तेन तुल्योऽस्ति । अयं अजवर्गतुल्योऽस्ति चतुर्गुणदजवर्गतुल्यो भविष्यति । पुनर्दजवर्ग उभयोर्युक्तः\renewcommand{\thefootnote}{३}\footnote{र्योज्यः {\en K.,~A.}} कार्यः । तदा दजजबघातो द्विगुणो दजवर्गजबवर्गयुतो दबवर्गतुल्यः पञ्चगुणितदजवर्गसमो भविष्यति । \renewcommand{\thefootnote}{४}\footnote{इष्टमिदमेव {\en K.}}इदमेवेष्टम्~॥} \\
\vspace{5mm}

\begin{center}
\textbf{\large अथ सप्तमं क्षेत्रम् ॥ ७ ॥ }
\end{center}
\vspace{5mm}

{\ab रेखाया द्वे खण्डे तथा कार्ये यथा सर्वरेखाया महत्खण्डेन निष्पत्तिर्महत्खण्डलघुखण्डनिष्पत्तितुल्या भवति । पुना रेखायां महत्खण्डतुल्या रेखा योज्या । तत्र \renewcommand{\thefootnote}{५}\footnote{योगोत्पन्न° {\en B.}}योगेनोत्पन्नरेखाया निष्पत्तिः प्रथमरेखया तथा \renewcommand{\thefootnote}{६}\footnote{भवति {\en B.}}भवेत् यथा प्रथमरेखाया निष्पत्तिर्महत्खण्डेनास्ति । }

\newpage
 
 यथा अबरेखाया जचिह्ने तथाविधे खण्डे कृते । अस्याम् अजं महत्खण्डं कल्पितम् । पुनर्महत्खण्डतुल्या अदरेखा योजिता । तदोत्पन्नदबरेखाया अचिह्ने तादृशे खण्डे भविष्यतः । \\
 \begin{center}
\textbf{अस्योपपत्तिः । }
\end{center}
\vspace{3mm}

\begin{sloppypar}
अबस्य निष्पत्तिः अजतुल्यअदरेखया तथास्ति यथा अजनिष्पत्तिर्जबेनास्ति~। तस्मात् दअअबयोर्निष्पत्तिर्बजजअनिष्पत्तितुल्या भविष्यति~। तस्मात् \renewcommand{\thefootnote}{१}\footnote{दबनिष्पत्तिः अबेन {\en V, D.}}दबबअर्निष्पत्तिर्बअअजतुल्यअदनिष्पत्तिसमाना भविष्यति । इदमेवास्मदिष्टम्~। \\
\end{sloppypar}
\vspace{3mm}

पुनरपि न्यूनखण्डतुल्यं महत्खण्डात्पृथक्कार्यम् । तदा महत्खण्डं तस्यामेव निष्पत्तौ \renewcommand{\thefootnote}{२}\footnote{विभक्तं भविष्यति {\en K.,A.}}विभागं प्राप्स्यति । न्यूनखण्डं च महत्खण्डं भविष्यति । यथा दबरेखाया अचिह्ने तस्यामेव निष्पत्तौ उभे खण्डे कल्पिते । महत्खण्डम् अबं कल्पितम् । पुनर्दअरेखातुल्या अजरेखा अबरेखायाः पृथक् कृता । तस्मात् अबरेखाया जचिह्नोपरि तस्यां निष्पत्तौ द्वे खण्डे भविष्यतः । अजरेखा च महत्खण्डं भविष्यति~। \\
\vspace{3mm}

\begin{center}
\textbf{अस्योपपत्तिः । }
\end{center}
\vspace{3mm}

दबअबनिष्पत्तिर्बअअदतुल्यअजनिष्पत्तिः\renewcommand{\thefootnote}{३}\footnote{ निष्पत्तिसमानास्ति । तस्मात् {\en V.} } । तस्मात् दअतुल्यअजस्य अबेन निष्पत्तिर्जबजअनिष्पत्तेः समाना भविष्यति । तस्मात् अबअजयोर्निष्पत्तिः अजजबनिष्पत्तितुल्या भविष्यति । इदमेवेष्टम् ॥\\
\vspace{3mm}

\begin{center}
\textbf{\large अथाष्टमं क्षेत्रम् ॥ ८ ॥ }
\end{center}

{\ab \renewcommand{\thefootnote}{४}\footnote{यस्या रेखायाः {\en K., A.}}यदा रेखायाः स्वमहत्खण्डेन निष्पत्तिर्महत्खण्डलघुखण्डनिष्पत्तितुल्या भवति तदा सर्वरेखाया वर्गो लघुखण्डवर्गयुतः सन् त्रिगुणमहत्खण्डवर्गतुल्यो भविष्यति~। }\\
\vspace{3mm}


यथा अबरेखा कल्पिता । जबन्यूनखण्डं तस्यां निष्पत्तौ कल्पितम् । तदा अबवर्गबजवर्गयोगस्त्रिगुणितअजवर्गेण तुल्यो भविष्यति । 

\newpage
\begin{center}  
\textbf{अस्योपपत्तिः । }
\end{center}
\vspace{3mm}

अबबजवर्गयोगो द्विगुणअबबजघातअजवर्गयोगसमानोऽस्ति ।
तस्मात् अबबजवर्गयोगः त्रिगुणितेन अजवर्गेण तुल्यो भविष्यति ।
इदमेवेष्टम् ॥ \\
\vspace{3mm}

\begin{center}
\textbf{\large अथ नवमं क्षेत्रम् ॥ ९ ॥ }
\end{center}
\vspace{5mm}

\begin{sloppypar}
{\ab या रेखाङ्कसंज्ञार्हा भवति तस्यास्तथा द्वे खण्डे कार्ये यथा \renewcommand{\thefootnote}{१}\footnote{ सर्वरेखामहत्ख° {\en V.} }सर्वमहत्खण्डयोर्निष्पत्तिर्महत्खण्डलघुखण्डयोर्निष्पत्ति तुल्या भवति । तत्र ~~~~प्रत्येकं खण्डमन्तररेखा भविष्यति । }\\
\end{sloppypar}
\vspace{3mm}

 यथा अबरेखा कल्पितमहत्खण्डं च अजं कल्पितम् । पुनर् अदरेखा अबार्द्धतुल्या योज्या । तस्मात् दजवर्गः पञ्चगुणितदअवर्गतुल्यो भविष्यति । तस्मात् दअरेखा दजरेखा च मिथो भिन्ना भविष्यति । अनयोर्वर्गौ केवलमङ्कसंज्ञार्हौ भविष्यतः~। तस्मात् अजम् अन्तररेखा भविष्यति । पुनर्यदि अजवर्गतुल्यं अबरेखोपरि क्षेत्रं \renewcommand{\thefootnote}{२}\footnote{क्रियते {\en K., A.} }कार्यं तदोत्पन्नद्वितीयभुजो जबरेखा भविष्यति । तस्मात् जबरेखाप्यन्तररेखा भविष्यति~। इदमेवास्मदिष्टम् ॥ \\
 \vspace{3mm}
 
 \begin{center}
\textbf{\large अथ दशमं क्षेत्रम् ॥ १० ॥ }
\end{center}
\vspace{5mm}

{\ab समपञ्चास्रक्षेत्रमध्ये त्रयः कोणा यदि समाना भवन्ति
तदा शेषा अपि कोणाः समाना भवन्ति । }\\
\vspace{3mm}

 यथा अबजदहपञ्चभुजं क्षेत्रं कल्पितम् । अजदकोणाः समानाः कल्पिताः~। पुनर्बहबदरेखे संयोज्ये । बहअत्रिभुजे बजदत्रिभुजे अकोणजकोणयोः समानभावित्वेन अकोणजकोणसंबन्धिभुजानां साम्यभावित्वेन तकोणककोणौ समानौ भविष्यतः । एवं बहवदभुजावपि समानौ भविष्यतः । बहदकोणबदहकोणावपि समानौ भविष्यतः । तस्मात् संपूर्णो हकोणः संपूर्णदकोणतुल्यो भविष्यति । 
\newpage
पुनरेवं निश्चीयते बकोणो जकोणतुल्यो भविष्यति । पुनर्जदहकोणाः समानाः 
\begin{vwcol}[widths={0.7,0.3}, sep=.8cm, rule=0pt]
कल्पिताः । जहरेखा च
संयोज्या । तदा बदजत्रिभुजे दहजत्रिभुजे जकोणदकोणयोः साम्यात् जकोणदकोणसंबन्धिभुजयोः साम्येन च गकोणलकोणौ समानौ भविष्यतः । एवं बदजहभुजावपि समानौ भविष्यतः । वकोणमकोणावपि समानौ भविष्यतः~। तस्मात् दझजझभुजावपि समानौ \\
\noindent \includegraphics[scale=0.8]{Images/rg-168.png}  
\end{vwcol}
\vspace{-4mm} 

\noindent भविष्यतः । शेषौ झबझहावपि समानौ भविष्यतः । \renewcommand{\thefootnote}{१}\footnote{पुनः {\en K., A.}}तस्मात् नकोणसकोणावपि समानौ भविष्यतः\renewcommand{\thefootnote}{२}\footnote{जातौ {\en K., A.}} । खकोणतकोणौ समानावास्ताम् । कुतः । अबअहभुजयोः साम्यात्~। तस्मात् सर्वो बकोणः सर्वहकोणतुल्यो जातः ।\\
\vspace{5mm}

एवं निश्चितम्\renewcommand{\thefootnote}{३}\footnote{{\en A. and K. insert} हि {\en after} एवम्. } अकोणो जकोणतुल्यो भविष्यति । इदमेवेष्टम् ॥\\
\vspace{3mm}

\begin{center}
\textbf{\large अथैकादशं क्षेत्रम् ॥ ११ ॥ }
\end{center}
\vspace{5mm}

{\ab वृत्तक्षेत्रान्तः समत्रिभुजस्य भुजवर्गस्त्रिगुणितव्यासार्द्धवर्गतुल्यो भविष्यति । }\\
\vspace{3mm}

यथा अबजं समत्रिभुजं क्षेत्रं \renewcommand{\thefootnote}{४}\footnote{दकेन्द्रजवृत्तान्तः० {\en V.}}दकेन्द्रं अबजवृत्तान्तःपाति कल्पितम् । पुनर्
\begin{vwcol}[widths={0.7,0.3}, sep=.8cm, rule=0pt]
 अदहरेखा हजरेखा च
संयोज्या । तस्मात् अजहचापं वृत्तार्द्धं भविष्यति । अजहचापं वृत्तत्रिभागो भविष्यति~।
जहचापं वृत्तषष्ठांशो भविष्यति । अहवर्गश्चतुर्गुणितअदवर्गतुल्योऽस्ति । अहवर्गः अजजहवर्गयोगतुल्योऽस्ति । अजवर्गअदवर्गयौगेनापि समानो भविष्यति ।\\
\noindent \includegraphics[scale=0.8]{Images/rg-169.png}  
\end{vwcol}

\newpage

\noindent तस्मात् अजअदवर्गयोगश्चतुर्गुण अदवर्गेण समानो भविष्यति ।
तस्मात् अदवर्ग उभयोः शोध्यः । तदा अजवर्गस्त्रिगुणअदवर्गतुल्यो-
ऽवशिष्यते । इदमेवास्माकमिष्टम्~॥\\
\vspace{2mm}

\begin{center}
\textbf{\large अथ द्वादशं क्षेत्रम् ॥ १२ ॥ }
\end{center}
\vspace{2mm}

{\ab वृत्तस्यान्तः समानषड्भुजक्षेत्रमस्ति तथा समानदशभुजमपि क्षेत्रमस्ति तयोः क्षेत्रयोर्भुजयोगस्य समानषड्भुजेन निष्पत्तिस्तथास्ति यथा षड्भुजस्य \renewcommand{\thefootnote}{१}\footnote{दशभुजेनास्ति {\en D.} }दशभुजभुजे नास्ति~।}\\
\vspace{3mm}

यथा अबजवृत्ते दशभुजस्य भुजो बजं कल्पितः । बजभुजो दचिह्नपर्यन्तं वर्द्धनीयः । षड्भुजक्षेत्रभुजतुल्यं जदं \renewcommand{\thefootnote}{२}\footnote{कार्यम् । {\en A., K.}}पृथक्कार्यम् । बदस्य जदेन निष्पत्तिर्दजजबनिष्पत्तिः\renewcommand{\thefootnote}{३}\footnote{०निष्पत्तितुल्या भविष्यति {\en V.}} ।\\
\begin{center}
\textbf{\large अस्योपपत्तिः । }
\end{center}
\vspace{5mm}

अबचापं चतुर्गुणबजचापतुल्यमस्ति । तदा अहबकोणश्चतुर्गुणबहजकोण-
\begin{vwcol}[widths={0.7,0.3}, sep=.8cm, rule=0pt]
तुल्यो भविष्यति । पुनर् अहबकोणो बजहकोणात् द्विगुणोऽस्ति । बजहकोणो दकोणाद्द्विगुणोऽस्ति~। कुतः~। जदजहयोः साम्यात् । तस्मात् अहबकोणश्चतुर्गुणितदकोणतुल्यो भविष्यति । तस्मात् बहजकोणबदहकोणौ
बजहत्रिभुजे बदहत्रिभुजे च समानौ भविष्यतः~। द्वयोस्त्रिभुजयोर्बकोण एक एवास्ति ।
तस्मादुभे त्रिभुजे सजातीये भविष्यतः ।\\
\noindent \includegraphics[scale=0.8]{Images/rg-170.png}  
\end{vwcol}
\vspace{-3mm}

\noindent तस्मात् दबभुजस्य निष्पत्तिर्बहभुजेन 
बहभुजबजभुजनिष्पत्तिसमाना भविष्यति~। बहजदौ समानौ स्तः~। तस्मात् बददजयोर्निष्पत्तिर्दजजबयोर्निष्पत्तिसमाना भविष्यति । इदमेवेष्टम् ॥


\newpage
\begin{center}
\textbf{\large अथ त्रयोदशं क्षेत्रम् ॥ १३ ॥ }
\end{center}
\vspace{2mm}

{\ab वृत्तपञ्चमांशस्य पूर्णजीवावर्गः षष्ठांशपूर्णज्यावर्गदशमांशपूर्णज्यावर्गयोर्योगेन तुल्यो भवति । }\\
\vspace{3mm}

यथा अबदहजवृत्तं बकेन्द्रं कल्पितम् । पञ्चमांशज्या अबं कल्पितम् । पुनर् अवझं व्यासः कल्पितः । वबरेखा संयोज्या । पुनर्वचिह्नात् अबरेखोपरि वतकं 
\begin{vwcol}[widths={0.7,0.3}, sep=.8cm, rule=0pt]
 लम्बो देयः । पुनर् अककबरेखे संयोज्ये । अकरेखोपरि वलमं लम्बो देयः । पुनः कनरेखा संयोज्या~। तदा बमचापं सार्द्धं दशमांशोऽस्ति । बझचापं त्रिगुणदशमांशतुल्यमस्ति । तदा बवझकोणो द्विगुणबवमकोणतुल्यो भविष्यति । अयं बवझकोणो द्विगुणबअवकोणतुल्योऽस्ति । कुतः~। बववअभुजयोः साम्यात् । बवनत्रिभुजे बवअत्रिभुजे\\
 
\noindent \includegraphics[scale=0.8]{Images/rg-171.png}  
\end{vwcol}
\vspace{-3mm}
\noindent बवनबअवकोणौ समानौ स्तः । उभयोर्वबनकोण एक एवास्ति । तस्मादुभे त्रिभुजे सजातीये भविष्यतः । तस्मात् अबबवयोर्निष्पत्तिर्वबबनयोर्निष्पत्तिसमाना भविष्यति~। तस्मात् अबबनयोर्घातो बववर्गतुल्यो भविष्यति । बवं वृत्तषष्ठांशस्य पूर्णजीवास्ति।\\
\vspace{5mm}

पुनरपि वलम् अके लम्बोऽस्ति । तस्मात् अकं लचिह्ने अर्द्धं भविष्यति~। न अनकयोः साम्येन नकअकोणनअककोणौ कनअत्रिभुजे समानौ भविष्यतः~। एवं बकअत्रिभुजे कबअकोणकअबकोणौ समानौ भविष्यतः । कअबकोणो बकअत्रिभुजे कनअत्रिभुजे एक एवास्ति । तस्मादेते त्रिभुजे सजातीये भविष्यतः~। तस्मात् बअभुजनिष्पत्तिः अकभुजेन अकभुजअनभुजयोर्निष्पत्तिसमाना
भविष्यति । तस्मात् न अअबघातः अकवर्गतुल्यो भविष्यति । अकं दशमांशस्य पूर्णजीवास्ति~। अबबनघातः अबअनघातयुक्तः अबवर्ग-

\newpage

\noindent तुल्योऽस्ति । तस्मात् पञ्चांशपूर्णजीवावर्गः षष्ठांशपूर्णजीवावर्गदशमांशपूर्णजीवावर्गयोर्योगतुल्यो जातः । इदमेवास्माकमिष्टम् ॥\\
\vspace{3mm}

\begin{center}
\textbf{\large अथ चतुर्दशं क्षेत्रम् ॥ १४ ॥ }
\end{center}
\vspace{5mm}

{\ab वृत्तान्तः समभुजपञ्चास्रक्षेत्रस्य कोणद्वयसन्मुखजीवयोः  संपातो यदि भवति तत्र पूर्णजीवाया निष्पत्तिर्महत्खण्डेन तथास्ति यथा महत्खण्डस्य निष्पत्तिर्लघुखण्डेनास्ति । महत्खण्डं च पञ्चसमभुजक्षेत्रस्य भुजतुल्यं भविष्यति । }\\
\vspace{5mm}

यथा अबदहजपञ्चसमभुजे अदपूर्णजीवाजबपूर्णजीवयोः संपातो झचिह्ने कल्पितः~। अबझत्रिभुजबजअत्रिभुजे सजातीये भविष्यतः ।  कुतः । बअझकोणबज-
\vspace{-2mm}

\begin{vwcol}[widths={0.7,0.3}, sep=.8cm, rule=0pt]
अकोणयोः साम्यात् । उभयोर्बकोण एक एवास्ति । तस्मात् जबभुजनिष्पत्तिर्बअभुजतुल्य अजभुजेन तथास्ति यथा अजभुजस्य बझभुजेनास्ति । पुनरपि झबअकोणझ अबकोणयोः समानभावित्वेन जझअकोणः द्विगुणझअबकोणतुल्यो भविष्यति । पुनरपि जहदचापं बदचापाद्द्विगुणमस्ति । तेन जअझकोणो झअबकोणाद्द्विगुणो भवति । तस्मात् जझअकोण-\\
\noindent \includegraphics[scale=0.8]{Images/rg-172.png}  
\end{vwcol}
\vspace{-5mm}

\noindent 
जअझकोणौ समानौ भविष्यतः । तस्मात् अजं झजं समानं भविष्यति । तस्मात् बजजझयोर्निष्पत्तिर्जझझबयोर्निष्पत्तिसमाना भविष्यति । झजम् अजसमानमस्ति~। एवम् अदपूर्णजीवा झचिह्ने एतन्निष्पत्तितुल्या भविष्यति ।
इदमेवास्माकमिष्टम् ॥\\
\begin{center}
\textbf{\large अथ पञ्चदशं क्षेत्रम् ॥ १५ ॥ }
\end{center}
\vspace{5mm}

{\ab यदि वृत्तव्यासोऽङ्कसंज्ञार्हो भवति तदा पञ्चसमभुजस्य भुजो न्यूनरेखा भविष्यति । }

\newpage
यथा वृत्तं पञ्चसमभुजं च अबदहजं कल्पितम् । पुनर् अझव्यासबवव्यासौ 
\begin{vwcol}[widths={0.65,0.35}, sep=.8cm, rule=0pt]
निष्कास्यौ । पुनर् अदरेखा संयोज्या । पुनस्तबचतुर्थांशतुल्यं तकं पृथक्कार्यम् । तदा अलतत्रिभुजअमदत्रिभुजे अकोणस्यैकत्वेन लकोणमकोणयोश्च समानभावित्वेन सजातीये भविष्यतः~। तस्मात् अतस्य बततुल्यस्य निष्पत्तिर्लतेन  तथास्ति यथा अदस्य दमेनास्ति । पुनर्बतचतुर्थांशतुल्यतकनिष्पत्तिर्लतेन तथास्ति यथा लदार्द्धस्य दमेनास्ति~। लदार्द्धस्य\\
\vspace{5mm}

\noindent \includegraphics[scale=0.9]{Images/rg-173.png}  
\end{vwcol}
\vspace{-3mm}

\noindent दहार्द्धेनापि । पुनः कलतकयोर्निष्पत्तिस्तथास्ति यथा हदलस्य निष्पत्तिर्दलेनास्ति~। तस्मात् कलवर्गतकवर्गयोर्निष्पत्तिर्हदलवर्गदलवर्गयोर्निष्पत्तितुल्या भविष्यति । अदं पञ्चसमभुजकोणस्य पूर्णजीवास्ति । दहं पञ्चसमकोणभुजोऽस्ति । एतयोर्योगो यदि भवति तदाऽनयोर्दचिह्ने तथा विभागौ भविष्यतो यथा सर्वयोगस्य निष्पत्तिः अदेन अददहनिष्पत्तितुल्या भविष्यति । हदलवर्गः पञ्चगुणितदलवर्गतुल्यो भविष्यति । तस्मात् कलवर्गः पञ्चगुणकतवर्गतुल्यो भविष्यति । बकं पञ्चगुणतकतुल्यमस्ति । तस्मात् बककतयोर्निष्पत्तिर्लककतनिष्पत्तिवर्गतुल्या भविष्यति । तस्मात् लकं बकतकयोर्मध्यनिष्पत्तौ पतितम् । तस्मात् बकवर्गः पञ्चगुणलकवर्गतुल्यो भविष्यति । तस्मात् बककलवर्गौ \renewcommand{\thefootnote}{१}\footnote{पञ्चकस्य रूपस्य च {\en K., A.} }पञ्चरूपयोर्निष्पत्तौ भविष्यतः । \renewcommand{\thefootnote}{२}\footnote{{\en Omitted in K., A.}\\
भा० २५}तदा किं भविष्यति । एते द्वे रेखे भिन्ने भविष्यतः । अनयोर्वर्गौ चाङ्कसंज्ञार्हौ भविष्यतः । बकम् अङ्कसंज्ञार्हमस्ति । अस्य वर्गः कलवर्गबलभिन्नरेखावर्गयोर्योगतुल्योऽस्ति । तदा बलरेखा चतुर्थ्यन्तररेखा भविष्यति । बवबलघाततुल्यो बअवर्गोऽस्ति । तस्मात् बअं न्यूनरेखा भविष्यति । इदमेवेष्टम्~॥


\newpage
\begin{center}
\textbf{\large पुनः प्रकारान्तरम् ॥}
\end{center}
\vspace{3mm}

दझरेखा संयोज्या । इयं रेखा लतरेखायाः समानान्तरा भविष्यति । कुतः~। अदझस्य समकोणत्वात् । अतअझयोर्निष्पत्तिस्तलझदयोर्निष्पत्तितुल्या भविष्यति । तस्मात् लतं दझस्यार्द्धं भविष्यति । इदं किमस्ति । दशसमभुजस्य क्षेत्रस्य भुजार्द्धं भवति । पुनः कनं तकतुल्यं पृथक्कार्यम् । तस्मात् तनं षट्समभुजस्य क्षेत्रस्य भुजार्द्धतुल्यं भविष्यति । लनस्य तचिह्ने एतादृशे खण्डे जाते लनस्य तनेन निष्पत्तिः तनलतनिष्पत्तितुल्यास्ति । तस्मात् लकवर्गः पञ्चगुणतकवर्गतुल्यो भविष्यति । तस्मात् बकवर्गः पञ्चविंशतिगुणतकवर्गतुल्यो भविष्यति~। पञ्चगुणलकवर्गेणापि तुल्यो भविष्यति । पुनः पूर्वप्रकारेण
एतामुपपत्तिं पूर्णां कुर्मः ॥\\
\vspace{3mm}

\begin{center}
\textbf{\large अथ षोडशं क्षेत्रम् ॥ १६ ॥ }
\end{center}
\vspace{5mm}

{\ab गोलान्तश्चतुःफलकः शङ्कुस्तथा कर्त्तव्योऽस्ति यथा प्रतिफलकं त्रिभुजं समभुजं भवति । अस्य गोलस्य व्यासवर्गः शङ्कुभुजस्य सार्द्धवर्गतुल्यः पतिष्यति~। }\\
\vspace{3mm}

यथा गोलव्यासः अबं कल्पितः । अस्योपरि वृत्तार्द्धं कार्यम् । \renewcommand{\thefootnote}{१}\footnote{ व्यासात् {\en V.}}पुनर्व्यासतृतीयांशं जबं पृथक्कार्यम् । जचिह्नात् जदलम्बो निष्कास्यः । अदरेखा संयोज्या । एकमन्यवृत्तं भवति । पुनरस्य वृत्तान्तः कलमं समानत्रिभुजं कार्यम् । वृत्तकेन्द्रं कार्यं यस्य व्यासार्द्धं दजतुल्यं
च झं कल्पितम् । पुनरस्मात्केन्द्रात् हवलम्बो वृत्तधरातले द्वयोर्दिशोः कार्यः । जअतुल्यं झनं पृथक्कार्यम् । पुनः कनमनलनरेखाः संयोज्याः । तस्मात् कलमनशङ्कुरिष्टो भविष्यति ।\\
\begin{center}
\textbf{अस्योपपत्तिः ।}
\end{center}
\vspace{3mm}

अबबजयोर्निष्पत्तिः अददजनिष्पत्तिवर्गतुल्यास्ति । अबं बजात्रिगुणमस्ति । तस्मात् अदवर्गो दजवर्गात्रिगुणो भविष्यति । कझ-

\newpage
\noindent वर्गादपि त्रिगुणो भविष्यति । तस्मात् लकम् अदसमानं भविष्यति ।
\begin{center}
\includegraphics[scale=0.9]{Images/rg-174.png}  
\end{center}

अनेनैव प्रकारेण सर्वे भुजाः कार्याः । पुनरपि कझनत्रिभुजदजअत्रिभुजयोर्द्वौ कोणौ समकोणौ स्तः । कोणसंबन्धिभुजौ च समानौ स्तः । तस्मात् कनम् अदतुल्यं भविष्यति । अनेन प्रकारेण सर्वा रेखाः समाना भविष्यन्ति । तस्मात् सर्वे शङ्कुभुजाः समाना भविष्यन्ति । पुनर्जबतुल्यं झतं पृथक्कार्यम् । तस्मात् नतम् अबतुल्यं भविष्यति । \renewcommand{\thefootnote}{१}\footnote{{\en V. inserta} पुनः {\en here.}}नते वृत्तार्द्धं कार्यम् । तस्योपरि वर्तनं च कार्यम् । तदेदं वृत्तं कचिह्नलचिह्नमचिह्नेषु लगिष्यति । कुतः । झकझलझमलम्बा जदतुल्याः सन्ति । तस्मादयं शङ्कुरिष्टगोलान्तःपाती भविष्यति । अदवर्गअबवर्गयोर्निष्पत्तिः अजअबयोर्निष्पत्तितुल्यास्ति~। तस्मात् गोलव्यासवर्गः शङ्कुभुजस्य सार्द्धतुल्यः पतितः । इदमस्माकमिष्टम् ॥\\
\begin{center}
\textbf{\large अथ सप्तदशं क्षेत्रम् ॥ १७ ॥ }
\end{center}
\vspace{5mm}


{\ab गोलान्तर्घनहस्तसंज्ञं क्षेत्रं \renewcommand{\thefootnote}{२}\footnote{क्रियते {\en K., A.} }कर्त्तुमिच्छास्ति तदा गोलव्यासवर्गो घनहस्तभुजवर्गान्त्रिगुणो भवति । }\\
\vspace{5mm}

यथा अबं व्यासः कल्पितः । जचिह्नेऽस्य तृतीयांशः कार्यः~। अस्योपरि अदबं वृत्तार्द्धं कार्यम् । जदलम्बश्च निष्कास्यः । बदरेखा संयोज्या । \renewcommand{\thefootnote}{३}\footnote{ {\en V. omits} अदरेखा संयोज्या ।.}अदरेखा संयोज्या । बदरेखातुल्या हझरेखा निष्कास्या ।


\newpage

\noindent हझरेखोपरि झतं समकोणसमचतुर्भुजं कार्यम् । पुनर्झतसमकोणसमचतुर्भुजोपरि झलं घनहस्तक्षेत्रं \renewcommand{\thefootnote}{१}\footnote{ कृतम् {\en K., A.}}कार्यम् । इदमिष्टं भविष्यति ।\\
\vspace{2mm}

\begin{center}
\textbf{अस्योपपत्तिः ।}
\end{center}
\vspace{5mm}

हवरेखा सवरेखा च संयोज्या । सवरेखावर्गः सहवर्गहववर्गयोगतुल्योऽस्ति~। हववर्गो झहवर्गझववर्गयोगतुल्योऽस्ति । तस्मात् सववर्गो हझवर्गात्त्रिगुणो
\begin{center}
\includegraphics[scale=0.9]{Images/rg-175.png}  
\end{center}
भविष्यति । वदवर्गात्त्रिगुणोऽपि भविष्यति । अबबजयोर्निष्पत्तिः अबवर्गबदवर्गनिष्पत्तितुल्यास्ति । तस्मात् अबवर्गो बदवर्गात्त्रिगुणो भविष्यति । तस्मात् अबसवौ समानौ भविष्यतः । यदि सवरेखायामर्द्धवृत्तं क्रियते तस्य चेद् \renewcommand{\thefootnote}{२}\footnote{भ्रामणं {\en V.}}भ्रमणं क्रियते तदा हचिह्ने
लगिष्यति । कुतः । सहवं समकोणोऽस्ति । एवं घनहस्तस्य सर्वकोणेषु लगिष्यति । तस्मादयं घनहस्तः अबगोलान्तःपाती भविष्यति । इदमेवास्माकमिष्टम् ॥\\
\begin{center}
\textbf{\large अथाष्टादशं क्षेत्रम् ॥ १८ ॥ }
\end{center}
\vspace{5mm}

{\ab वृत्तान्तरष्टास्रं\renewcommand{\thefootnote}{३}\footnote{अष्टफलकघनक्षेत्रं {\en K., A.}} घनक्षेत्रं कर्त्तुमिच्छास्ति \renewcommand{\thefootnote}{४}\footnote{यथा पतति {\en K., A.}}यथा प्रतिफलकघनहस्ते सर्वभुजानां समत्वात् त्रिभुजं समानभुजं प्रत्यस्रं त्रिभुजं समानभुजं पतत्यस्य गोलस्य व्यासवर्गो घनक्षेत्रभुजवर्गाद्विगुणे पतिष्यति । }\\
\vspace{3mm}

यथा अबं व्यासः कल्पितः । अयं दचिह्नेऽर्द्धितः कार्यः । अजबम् 


\newpage
\noindent अर्द्धं वृत्तं कार्यम् । दजलम्बो निष्कास्यः । जबरेखा च संयोज्या । पुनर्जबतुल्या हझरेखा निष्कास्या । पुनर्हझरेखोपरि हवं समकोणसमचतुर्भुजं कार्यम्~। पुनर्हवरेखा झकरेखा च संयोज्या । एते रेखे तचिह्ने संपातं करिष्यतः । पुनस्तचिह्नात् लमलम्बः समकोणसमचतुर्भुजस्य धरातले उभयतः कार्यः । पुनर् अदतुल्यं नतं तसं च पृथक्कार्यम् । पुनर्हनझनवनकनहसझसवसकसरेखाः संयोज्याः । तस्मात् हनझवकसम् इष्टघनक्षेत्रं भविष्यति ।\\
\begin{center}
\textbf{अत्रोपपत्तिः ।}
\end{center}
\vspace{5mm}

बदजदसमानरेखावर्गयोगतुल्यो बजवर्गोऽस्ति । बजवर्गो हझव-
र्गतुल्योऽस्ति~। 
\begin{center}
\includegraphics[scale=0.9]{Images/rg-176.png}  
\end{center}
हझवर्गो हतझतसमानरेखयोर्वर्गयोगतुल्योऽस्ति । तस्मात् तहं तझं प्रत्येकं दबतुल्यं भविष्यति । पुनस्तवं तकं दबसमानं भविष्यति । तनतसौ दबतुल्यावास्ताम्~। तस्मात् नचिह्ने सचिह्ने समकोणसमचतुर्भुजकोणेषु यावत्यो रेखा लगिष्यन्ति ताः सर्वाः समाना भविष्यन्ति । तदाष्टौ भुजाः समाना भविष्यन्ति~। यदि नसरेखायाम् अबरेखातुल्यायां वृत्तार्द्धं क्रियते तदा \renewcommand{\thefootnote}{१}\footnote{तद्भ्रामणेन {\en D., V.} तदा तत् {\en V., D.}}तद्भ्रमणेन तत्समकोणसमचतुर्भुजकोणेषु लगिष्यति । कुतः । सर्वेषां लम्बानां दजतुल्यत्वात् । तस्मादिदं घनक्षेत्रं गोलान्तर्गतं भविष्यति । अबवर्गो बज-

\newpage
\noindent वर्गाद्द्विगुणोऽस्ति । तदा गोलव्यासवर्गो घनक्षेत्रभुजवर्गाद्द्विगुणो
भविष्यति । इदमेवेष्टम् ॥\\
\begin{center}
\textbf{\large अथैकोनविंशतितमं क्षेत्रम् ॥ १९ ॥ }
\end{center}
\vspace{5mm}

{\ab \renewcommand{\thefootnote}{१}\footnote{वृत्तान्तं० {\en K., A.} }गोलान्तर्विंशतिफलकयुतं क्षेत्रमुत्पादयितुं \renewcommand{\thefootnote}{२}\footnote{इष्यते परंतु प्रतिफलकं {\en \& c. K., A.} इष्टमस्ति । प्रतिफलकं {\en V.}}यथेष्टमस्ति प्र-
तिफलकं त्रिभुजं समानभुजं यथा भवति । यदि गोलव्यासोऽङ्कसंज्ञार्हो भवति तदास्य क्षेत्रस्य भुजो न्यूनरेखा पतिष्यति । }\\
\vspace{3mm}

यथा अबं व्यासः कल्पितः । अस्मात् पञ्चमांशो बजं पृथक् कार्यम् । अबव्यासोपरि अदबम् अर्द्धवृत्तं कार्यम् । पुनर्जदलम्बो निष्कास्यः । बदरेखा च संयोज्या~। पुनरेकं वृत्तं कार्यं यस्य व्यासार्द्धं बदतुल्यं भविष्यति । तद्वृत्तं हझवं कल्पितम् । तद्वृत्तान्तर्हझतवकपञ्चसमभुजं\renewcommand{\thefootnote}{३}\footnote{समानाः~{\en V.}} कार्यम्~। पुनरस्य पञ्चचापानां लमनसगचिह्नेष्वर्द्धं कार्यम् । ततो दशपूर्णजीवाः संयोज्याः । प्रथमपञ्चसमानभुजानां पञ्चकोणेभ्यो वृत्तव्यासार्द्धतुल्याः पञ्च लम्बाः स्थाप्यास्ते च लम्बा हफझखतरवशकतसंज्ञकाः कल्पिताः~। पुनर्दशभुजकोणेषु रेखाः संयोज्याः । तस्मात् लमनसगपञ्चसमानभुजं वृत्तेऽन्यत् क्षेत्रं भविष्यति~। पुनर्दशभुजकोणेभ्यो लम्बमस्तकेषु च दशरेखाः संयोज्याः~। एता रेखाः प्रत्येकं वृत्तान्तः समपञ्चभुजभुजेन तुल्या भविष्यन्ति । पञ्चत्रिभुजानि समभुजान्युत्पन्नानि भविष्यन्ति । एषां भूमिर्वृत्तान्तः पञ्चभुजस्य\\
\begin{center}
\includegraphics[scale=0.9]{Images/rg-177.png}  
\end{center} 

\newpage
\noindent भुजा भविष्यति । पुनस्त्रिभुजानां शीर्षे रेखाः संयोज्याः । एता रेखाः समानाः समानान्तरा वृत्तान्तः पञ्चभुजभुजेन समानाः पतिष्यन्ति । पुनः पञ्चक्षेत्राणि त्रिभुजानि भविष्यन्ति । पुनर्वृत्तकेन्द्रं सचिह्नं कल्पितम् । सचिह्नात् वृत्तोभयदिशि धरातलयोर्लम्बो निष्कास्यः । ततो लम्बात् सखरेखा वृत्तषडंशस्य पूर्णजीवातुल्या पृथक्कार्या । वृत्तदशमांशस्य पूर्णजीवातुल्या खझरेखा पृथक्कार्या । एवं द्वितीयदिशि छसं वृत्तदशमांशपूर्णजीवातुल्यं पृथक्कृतम् । पुनः सहव्यासार्द्धं योजनीयम्~। खफरेखा सहरेखायाः समाना समानान्तरा च योज्या । पुनरुपरितनपञ्चसमभुजकोणझचिह्नयो रेखाः संयोज्याः । तस्मात् पञ्चत्रिभुजान्यन्यान्युत्पद्यन्ते । पुनर्वृत्तान्तः पञ्चसमभुजकोणछचिह्नयो रेखाः संयोज्याः । तस्मादिष्टं क्षेत्रं संपूर्णं भविष्यति । संयुक्ता
रेखाः प्रत्येकं पञ्चसमभुजस्य भुजा भविष्यन्ति~।\\
\vspace{10mm}


सझरेखायाः खचिह्ने एतादृशौ विभागौ जातौ सझरेखाया निष्पत्तिः सखरेखया तथा जाता यथा सखरेखाया निष्पत्तिः खझरेखयास्ति । तस्मात् सझरेखातुल्यछखरेखाझखरेखयोर्घातः सखरेखावर्गतुल्यो भविष्यति । खफरेखावर्गतुल्योऽपि भविष्यति । तस्मात् खफरेखा छखखझरेखयोर्मध्यनिष्पत्तौ पतिष्यति~। यदि छझरेखायामर्द्धं वृत्तं क्रियते तदा फचिह्ने लगिष्यति । पुनः क्षेत्राणां सर्वेषु कोणेषु लगिष्यति~। पुनः सखरेखा अचिह्नेऽर्द्धीकृता । तस्मात् झअरेखावर्गः पञ्चगुणितखअरेखावर्गतुल्यो भविष्यति । छझरेखासखरेखयोर्निष्पत्तिर्झअखअरेखयोर्निष्पत्तितुल्यास्ति । तस्मात् छझरेखावर्गः पञ्चगुणखसरेखावर्गतुल्यो भविष्यति । अबरेखावर्गः पञ्चगुणबदरेखावर्गतुल्य आसीत् । कुतः । एतौ द्वौ अबवर्गबदवर्गौ अबबजयोर्निष्पत्तौ स्तः~। तस्मात् छझरेखा अबतुल्या भविष्यति~। तस्मादिदं क्षेत्रं गोलान्तर्गतं भविष्यति । अस्य भुजः पञ्चसमभुजभुजतुल्योऽस्ति~। तस्मादस्य भुजो न्यूनरेखा भविष्यति । इदमिष्टम् ।

\newpage

पञ्चसमभुजस्य भुजो न्यूनरेखा ततो भवति यतो वृत्तव्यासोऽङ्कसंज्ञार्हो भवति । अत्र तु गोलव्यासोऽङ्कसंज्ञार्होऽस्ति । वृत्तव्यासोऽङ्कसंज्ञार्हो नास्ति । परं तु वृत्तव्यासार्द्धवर्गो गोलव्यासवर्गस्य पञ्चमांशोऽस्ति । तदा वृत्तव्यासः केवलमङ्कसंज्ञार्हो भविष्यति । यस्य वृत्तस्य व्यासोऽङ्कसंज्ञार्हो भवत्यन्यवृत्तव्यासवर्गः केवलमङ्कसंज्ञार्हो भवति तदा
प्रथमव्यासनिष्पत्तिर्द्वितीयवृत्तव्यासेन तथा भवति यथा प्रथमवृत्तान्तः पञ्चसमभुजभुजस्य निष्पत्तिर्द्वितीयवृत्ते पञ्चसमभुजभुजेनास्ति । यदि द्वयोर्व्यासयोर्वर्गौ मिलितौ भवतस्तदा द्वयोर्भुजयोरपि वर्गौ मिलितौ भविष्यतः । तस्मादस्य क्षेत्रस्य पञ्चसमभुजस्य भुजो न्यूनरेखया केवलवर्गमिलितो भविष्यति । न्यूनरेखया या मिलिता रेखा स्यात् सा केवलवर्गमिलिता भविष्यति । तदा सापि न्यूनरेखा भवति । तस्मादस्य क्षेत्रस्य भुजो न्यूनरेखा भविष्यति ॥\\
\vspace{5mm}

\begin{center}
\textbf{\large अथ विंशतितमं क्षेत्रम् ॥ २० ॥ }
\end{center}
\vspace{5mm}

{\ab गोलस्यान्तः समभुजद्वादशफलकं क्षेत्रं कर्त्तुमिच्छास्ति यथा प्रत्येकं फलकः पञ्चसमभुजः समानकोणो भविष्यति । अस्य क्षेत्रस्य भुजोऽन्तररेखा भविष्यति यदि व्यासोऽङ्कसंज्ञार्हो \renewcommand{\thefootnote}{१}\footnote{भवति {\en K., A.}}भविष्यति । }\\
\vspace{5mm}

यथा अबअजे उभे धरातले अगोलान्तर्गतघनहस्तक्षेत्रस्य कल्पिते । एकं धरातलं द्वितीये धरातले लम्बवत् कल्पितं भवति । पुनरेतद्द्वयोर्धरातलयोः सर्वभुजानां वतकलमनसचिह्नेष्वर्द्धं कार्यम् । पुनरेतच्चिह्नेषु मिथः \renewcommand{\thefootnote}{२}\footnote{संपातकर्त्र्यः {\en V.}}संपातकारिण्यः धरातलभुजानां समानान्तरा रेखाः संयोज्याः । प्रत्येकं तफरेखाकफरेखागलरेखानां रचिह्नखचिह्नशचिह्नेषु \renewcommand{\thefootnote}{३}\footnote{द्वौ विभागौ {\en V.}}द्वाविमौ तथा कार्यौ यथा प्रत्येकस्य स्वमहत्खण्डेन तथा निष्पत्तिर्भवति या महत्खण्डस्य लघुखण्डेनास्ति । एतासां महत्खण्डानि फरफखगशसंज्ञानि कल्पितानि । पुनः खरशचिह्नेभ्यः

\newpage

लम्बाः फखरेखातुल्या उभयोर्धरातलयोर्निष्कास्याः । एते लम्बाः खथरसशघाः 
कल्पिताः । पुनर् \renewcommand{\thefootnote}{१}\footnote{अत {\en V.}}अखअघअथथससझझघरेखाः संयोज्याः । तस्मात् तफवर्गतखवर्गयोः अतवर्गतखवर्गयोर्वा योगः अखवर्गतुल्यो भवति । अयं त्रिगुणखफवर्गतुल्योऽस्ति । त्रिगुणखथवर्गस्यापि तुल्योऽस्ति । पुनर् अथवर्गश्चतुर्गुणखथवर्गतुल्योऽस्ति । तस्मात् अथरेखा द्विगुणखफरेखातुल्या भविष्यति । तदा खरतुल्या भविष्यति । थसतुल्यापि भविष्यति । एतत्प्रकारेण निश्चितम् अघरेखा घझरेखा झसरेखा
थसरेखा समाना भविष्यन्ति । तस्मात् अथथससझझघघ\renewcommand{\thefootnote}{२}\footnote{{\en V. has} रघ {\en aftar} घअ.}अभुजाः समाना 
\begin{center}
\includegraphics[scale=0.9]{Images/rg-178.png}  
\end{center} 
भविष्यन्ति । पुनः फझलम्बः अजधरातले खफतुल्यः निष्कास्यः । पुनर्झललखरेखे संयोज्ये । तदा फततुल्यफलरेखाया निष्पत्तिः शघतुल्यखफरेखया कीदृश्यस्ति~। यादृशी झफरेखातुल्यखफरेखाया निष्पत्तिः शलरेखातुल्यतखरेखयास्ति~। फलरेखा शघरेखायाः समानान्तरास्ति । तदा झफरेखा लशरेखायाः समानान्तरा भविष्यति । तस्मात् झलघं सरलैका रेखा भविष्यति । अलझं सरलैका रेखास्ति तस्मात् \renewcommand{\thefootnote}{३}\footnote{अतसझघं {\en V.}}अथसझघं पञ्चसमभुजं एकधरातले
भविष्यति \renewcommand{\thefootnote}{४}\footnote{ यत् {\en V.}\\
भा० २६}यतो झलघरेखाअलझरेखयोर्धरातलमस्ति । तस्मिन् पुनर् असं अरं द्वे रेखे संयोज्ये । तररेखा फचिह्ने एतादृक्खण्डितास्ति यथा सर्वरेखाया महत्खण्डेन निष्पत्तिर्महत्खण्डस्य लघुखण्डेन चास्ति~। अस्या महत्खण्डं तफमस्ति । तस्मात् तरवर्गरफवर्गौ तरवर्गरसवर्गतुल्यौ स्तः । तद्योगः तअवर्गतुल्यस्य तफवर्गत्रिगुणोऽस्ति । पुनस्तअवर्ग उभयोर्योज्यः । तस्मात् तरवर्गरसवर्गतवर्गाणां योगः

\newpage
\noindent असवर्गतुल्यचतुर्गुणतअवर्गसमानो जातः । अझवर्गस्तु चतुर्गुण \renewcommand{\thefootnote}{१}\footnote{अल {\en in V.}}अतवर्गसम आसीत् । तस्मात् असरेखा अझरेखा च समा भविष्यति । तस्मात् अझसअसझकोणौ समानौ भविष्यतः । एवं
निश्चीयते रसझकोणस्तयोः कोणयोः समानो भविष्यति । तस्मात्
पञ्चभुजस्य कोणाः समाना जाताः । इदं पञ्चभुजं क्षेत्रं घनहस्तस्यैकभुजे पतितम् । घनहस्तस्य द्वादशभुजाः सन्ति । यदि प्रत्येकभुजे पञ्चभुजोपरि एतादृशं क्रियते चेत्तदा क्षेत्रं पूर्णं द्वादशास्रं भविष्यति । प्रत्येकफलके पञ्चपञ्चभुजा भवन्ति~।\\
\vspace{10mm}


पुनर्झफरेखा निष्कास्या यथा घनहस्ते कर्णे छचिह्ने\renewcommand{\thefootnote}{२}\footnote{सचिह्ने {\en in V.}} संपातं करोति । तस्मात् फछरेखा घनहस्तकर्णार्द्धं करिष्यति । इयं फछरेखा घनहस्तस्य भुजार्द्धतुल्यास्ति~। पुनश्छसरेखायाः फचिह्नोपर्येतादृशौ विभागौ जातौ सर्वरेखाया महत्खण्डेन निष्पत्तिस्तथास्ति यथा महत्खण्डस्य लघुखण्डेनास्ति । छझवर्गझफवर्गयोगः छझझथवर्गयोगतुल्यश्छथवर्गतुल्योऽपि त्रिगुणछफवर्गसमोऽस्ति । छफं घनहस्तस्य भुजार्द्धमस्ति । घनहस्तकर्णार्द्धं घनहस्तार्द्धस्य त्रिगुणस्य सममस्ति~। या रेखाश्छचिह्नात् पञ्चभुजकोणपर्यन्तं निःसरिष्यन्ति ताः सर्वा अपि समाना भविष्यन्ति । तस्मात् घनहस्तावेष्टको गोल एतत्क्षेत्रावेष्टकोऽपि भविष्यति~। यदि घनहस्तभुजस्योभे खण्डे एतादृशे क्रियेते यथा सर्वभुजस्य महत्खण्डेन यथा निष्पत्तिर्भवति तथा महत्खण्डस्य लघुखण्डेन भवति तदा पञ्चभुजस्य भुजो धनहस्तभुजस्य महत्खण्डं भवेत् । तस्मादियमन्तररेखा भविष्यति । इदमेवास्माकमिष्टम्~॥\\
\vspace{3mm}

\begin{center}
\textbf{\large अथैकविंशतितमं क्षेत्रम् ॥ २१ ॥ }
\end{center}
\vspace{5mm}

{\ab एतन्निश्चयं कर्त्तुमीहामहे । किं तत् । यानि पञ्चक्षेत्राणि गोलान्तर्गतान्युक्तानि यद्येतानि एकगोले भवन्ति तदैतेषां भुजा एकगोले भवितुमर्हन्ति नवेति विचार्यते । }\\

\newpage
यथा अबं गोलव्यासः कल्पितः । व्यासोपरि अझबमर्द्धवृत्तं कार्यम् । अबं हचिह्नेऽर्द्धितं कार्यं जचिह्ने तृतीयांशः कर्त्तव्यः । हझजदलम्बौ निष्कास्यौ । पुनर्बझरेखाअदरेखावदरेखाः संयोज्याः । तदा अदं शङ्कुभुजो भविष्यति । बदं घनहस्तभुजो भविष्यति । वझं अष्टास्रघनक्षेत्रस्य भुजो भविष्यति । पुनर् अतलम्बः अबतुल्यः अबरेखोपरि निष्कास्यः । तहरेखा संयोज्या । पुनः कलरेखा तअरेखायाः समानान्तरा निष्कास्या । तस्मात् तअअहयोर्निष्पत्तिः कललहयोर्निष्पत्तितुल्या भविष्यति । तअं अहाद्द्विगुणमस्ति । कलं लहाद्द्विगुणं भविष्यति । तअवर्गश्चतुर्गुणअहवर्गतुल्योऽस्ति ।
तस्मात् कलवर्गश्चतुर्गुणलहवर्गतुल्यो भविष्यति । कहवर्गतुल्यो अहवर्गः पञ्चगुणलहवर्गतुल्योऽस्ति । अबकलयोर्निष्पत्तिः अहलहयोर्निष्पत्तितुल्यास्ति । तस्मात् 
\begin{center}
\includegraphics[scale=0.9]{Images/rg-179.png}  
\end{center} 
अबवर्गः पञ्चगुणकलवर्गतुल्यो भविष्यति । तस्मात् कलं विंशत्यस्रक्षेत्रस्य व्यासार्द्धं भविष्यति । अबं वहाद्द्विगुणमस्ति । अजं च बजात् द्विगुणमस्ति । तस्मात् जबं जहात् द्विगुणं भविष्यति । तस्मात् हबं अहतुल्यं त्रिगुणहजतुल्यं भविष्यति~। तस्मात् अहवर्गो नवगुणहजवर्गतुल्यो भविष्यति । पञ्चलहवर्गतुल्यश्चासीत् । तस्मात् लहं हजादधिकं भविष्यति । हमं लहतुल्यं पृथक्कार्यम् । मनलम्बो निष्कास्यः
प्रत्येकं लमं मनं च लकतुल्यं भविष्यति । लअं मबतुल्यं भविष्यति । लमं विंशतिफलकक्षेत्रवृत्तस्य व्यासार्द्धतुल्यमस्ति । प्रत्येकम् अलं मबं दशांशस्य पूर्णज्या भविष्यति । पुनर्बनरेखा संयोज्या ।
तदा पञ्चभुजस्य भुजो भविष्यति~। अयं विंशत्यस्रक्षेत्रस्य भुजो जातः । पुनर्दबस्य सचिह्ने द्वौ विभागौ कार्यौ महत्खण्डं बसं कल्पितम् । तत्

\newpage
\noindent द्वादशास्रभुजो भविष्यति । इदं प्रकटमस्ति । अदं गोलान्तर्गतशङ्कुभुजोऽष्टास्रभुजस्य बझभुजादधिकोस्ति । पुनर्बझं बदघनहस्तभुजादधिकमस्ति । बदं विंशत्यस्रभुजाद् बनादधिकमस्ति । तदा बनं द्वादशफलकभुजात् बसादधिकं भविष्यति । कुतः । अजवर्गश्चतुर्गुणबजवर्गतुल्योऽस्ति । दबवर्गस्त्रिगुणबजवर्गेण तुल्योऽस्ति । तस्मात् अजं दबादधिकं भविष्यति । अममत्यधिकं भविष्यति~। प्रत्येकम् अमे दमे च उभे महत्खण्डे मलबसे स्तः । तस्मात् मलतुल्यं मनं बसादधिकं भविष्यति । बसमत्यधिकं भविष्यति । इदमेवेष्टम् ॥\\

\begin{center}
{\small श्रीमद्राजाधिराजप्रभुवरजयसिंहस्य तुष्टौ द्विजेन्द्रः\\
श्रीमत्सम्राड् जगन्नाथ इति समभिधारूढितेन प्रणीते ।\\
ग्रन्थेऽस्मिन्नाम्नि रेखागणित इति सुकोणावबोधप्रदात-\\
र्यध्यायोऽध्येतृमोहापह इह विरतिं विश्वसंख्यो गतोऽयम् ॥\\

 ॥ इति त्रयोदशोऽध्यायः ॥ १३ ॥} \\
 
 \rule{0.7in}{0.3pt}
\end{center}


\newpage
\afterpage{\fancyhead[CE] {रेखागणितम्}}
\afterpage{\fancyhead[CO] {चतुर्दशमोध्यायः}}
\afterpage{\fancyhead[LE,RO]{\thepage}}
\cfoot{}
\newpage
%%%%%%%%%%%%%%%%%%%%%%%%%%%%%%%%%%%%%%%%%%%%%%%%%%%%%%%%%%%%%%
\newpage
\thispagestyle{empty}
\begin{center}
\textbf{\LARGE ॥ अथ चतुर्दशाध्यायः प्रारभ्यते ॥}
\end{center}
\vspace{3mm}

\begin{center}
\textbf{॥ अत्र दश क्षेत्राणि सन्ति ॥}
\vspace{5mm}

\textbf{\large \renewcommand{\thefootnote}{१}\footnote{तत्र {\en V.}}अथ प्रथमं क्षेत्रम् ॥ १ ॥ }
\end{center}
\vspace{2mm}

{\ab वृत्तकेन्द्रात् पञ्चभुजस्य भुजोपरि यो लम्बो भवति स वृत्तषष्ठांशपूर्णजीवादशमांशपूर्णजीवायोगस्यार्द्धं भवति । }\\
\vspace{3mm}

यथा दकेन्द्रोपरि अबजवृत्तं बजं पञ्चभुजस्य भुजो दहलम्बश्च कल्पितः ।
\begin{vwcol}[widths={0.7,0.3}, sep=.8cm, rule=0pt]
अयं लम्बो झपर्यन्तं वर्द्धनीयः । जझरेखा च कार्या~। इयं वृत्तदशमांशपूर्णजीवा जाता । दजं जझादधिकमस्ति । तस्मात् हझं दहान्न्यूनं भविष्यति~। कुतः~। जझस्य
जदान्न्यूनत्वात् । पुनर्दहात् हवं हझतुल्यं पृथक् कार्यम् । जवरेखा संयोज्या~। अदजकोणो जदझकोणाच्चतुर्गुणोऽस्ति । दझजकोणाद्द्विगुणोऽस्ति । जवझकोणादपि द्विगुणोऽस्ति~। \\
\noindent \includegraphics[scale=0.9]{Images/rg-180.png}  
\end{vwcol}
\vspace{-3mm}

\noindent जवझकोणो वदजकोणवजदकोणयोगो बदजकोणाद्द्विगुणोऽस्ति । तस्मात् वजदकोणवदजकोणौ समानौ भविष्यतः । एवं वजभुजवदभुजौ समानौ भविष्यतः~। तस्मात् जझझहयोगो हदसमानो जातः । अयं द्विगुणो द्विगुणहदसमानो भवति~। द्विगुणं हदं दशमांशपूर्णज्याषष्ठांशपूर्णज्यायोगतुल्यमस्ति । तस्मात् हदं षष्ठांशपूर्णज्यादशमांशपूर्णज्यायोगार्द्धं जातम् । इदमेवास्माकमिष्टम् ॥\\
\vspace{3mm}

\begin{center}
\textbf{\large अथ द्वितीयं क्षेत्रम् ॥ २ ॥ }
\end{center}
\vspace{5mm}

{\ab पञ्चसमभुजस्य भुजवर्गोऽस्य कोणसन्मुखपूर्णज्यावर्गोऽनयोर्योगः पञ्चगुणितव्यासार्द्धवर्गतुल्यो भवति । }

\newpage

यथा अबजवृत्तं बजं पञ्चभुजस्य भुजः अजं तत्कोणस्य पूर्णज्या अदझं 

\begin{vwcol}[widths={0.65,0.35}, sep=.8cm, rule=0pt]
व्यासः कल्पितः । जझरेखा संयोज्या । इयं दशमांशपूर्णज्यास्ति । अजवर्गजझवर्गयोगः अझवर्गतुल्यो दझवर्गाच्चतुर्गुणोऽस्ति । पुनर्दझवर्ग उभयोर्योज्यः~।
अयं दझवर्गो जझवर्गयुक्तो जबवर्गसमानोऽस्ति~। तस्मात् अजवर्गबजवर्गयोगः पञ्चगुणितदझवर्गसमानो जातः~। इदमेवास्माकमिष्टम्~॥ \\
\noindent \includegraphics[scale=0.9]{Images/rg-181.png}  
\end{vwcol}

\begin{center}
\textbf{\large अथ तृतीयं क्षेत्रम् ॥ ३ ॥ }
\end{center}
\vspace{5mm}

{\ab यद्येकगोले द्वादशफलकघनक्षेत्रमथ च विंशत्यस्रघनक्षेत्रं चोभे भवेतां तदा द्वादशास्रस्य पञ्चभुजं विंशत्यस्रस्य च त्रिभुजमेते द्वे क्षेत्रे एकवृत्ते \renewcommand{\thefootnote}{१}\footnote{पतिष्यतः {\en K., A.}}भविष्यतः~। }\\
\vspace{3mm}

यथा अबं गोलस्य व्यासः कल्पितः । जदहवझं द्वादशास्रघनक्षेत्रे पञ्चभुजं कल्पितम् । तयकं विंशत्यस्रघनक्षेत्रस्य त्रिभुजं कल्पितम् । दझरेखा कल्पितगोलघनहस्तस्य भुजः कल्पितः । लमरेखा विंशत्यस्रघनक्षेत्रस्य वृत्ते व्यासार्द्धं कल्पितम् । अस्या लमरेखाया नचिह्ने तथाविधं खण्डद्वयं कृतं यथा सर्वरेखाया निष्पत्तिर्महत्खण्डेन भवति तथा महत्खण्डस्य निष्पत्तिर्लघुखण्डेन भवति~। तन्महत्खण्डं लनं कल्पितम् । इदं लनं वृत्तदशमांशस्य पूर्णज्या भविष्यति~। तयरेखावर्गो लमलनयोर्वर्गयोगतुल्यो भविष्यति । लमरेखानिष्पत्तिर्लनरेखया तथास्ति यथा झदनिष्पत्तिर्जदेनास्ति~। पञ्चगुणितलमवर्ग-

\newpage
\begin{center}
\includegraphics[scale=0.9]{Images/rg-182.png}  
\end{center}
\vspace{5mm}

\noindent स्त्रिगुणितझदवर्गतुल्योऽस्ति । यतो लमपञ्चवर्गा झदस्य त्रयो वर्गाश्च पृथक् अबवर्गतुल्याः सन्ति । तस्मात् लमपञ्चवर्गा लनपञ्चवर्गाश्च सर्वेषां योगतुल्यः पञ्चगुणिततयवर्गो भवति । अयं त्रिगुणझदवर्गस्त्रिगुणदजवर्गश्चानयोर्योगतुल्योऽस्ति । यस्मिन् वृत्ते तयकं त्रिभुजं पतति तत् व्यासार्द्धत्रिगुणवर्गतुल्यस्तयवर्गो भवति~। यद्वृत्तान्तर्जदहवझं पञ्चभुजं पतति तत्र पञ्चगुणतदव्यासार्द्धवर्गतुल्यो झददजवर्गयोगोऽस्ति~। यद्वृत्तान्तस्तयकत्रिभुजं पतति पञ्चदशगुणतद्व्यासार्द्धवर्गतुल्यः पञ्चगुणतयवर्गो भवति~। यद्वृत्तान्तर्जदहवझपञ्चभुजं पतति पञ्चदशगुणिततद्व्यासार्द्धवर्गतुल्यस्त्रिगुणो झददजवर्गयोगो भवति । पुनः पञ्चगुणस्तयवर्गस्त्रिगुणझददजवर्गयोगतुल्यो भवति । तस्मात् यस्मिन् वृत्ते तयकत्रिभुजं पतति अथ च यद्वृत्ते जदहवझं पञ्चभुजं पतति द्वयोर्व्यासार्द्धवर्गौ तुल्यौ भवतः । तस्माद् व्यासार्द्धवर्गयोस्तुल्यत्वाद्वृत्तेऽपि तुल्ये जाते । इदमेवास्माकमिष्टम् ॥\\
\vspace{3mm}

\begin{center}
\textbf{\large अथ चतुर्थं क्षेत्रम् ॥ ४ ॥ }
\end{center}
\vspace{5mm}

{\ab द्वादशफलकघनक्षेत्रस्य पञ्चभुजा यस्मिन् वृत्ते पतन्ति  तद्वृत्तकेन्द्रान्निःसृतो लम्बः पञ्चभुजस्य भुजं यदा गच्छति तदा पञ्चभुजस्यैकभुजलम्बयोर्घातस्त्रिंशद्गुणितो द्वादशफलकघनक्षेत्रस्य संपूर्णधरातलतुल्यो भवति ।}

\newpage
\begin{vwcol}[widths={0.7,0.3}, sep=.8cm, rule=0pt]
यथा अबं तद्वृत्तं कल्पितं यस्यान्तर्द्वादशफलकघनक्षेत्रस्य पञ्चभुजक्षेत्रं पतितम् । पञ्चभुजक्षेत्रं च अबजदहं कल्पितम् । झतं लम्बः कल्पितः~। अस्य पञ्चभुजस्य पञ्चत्रिभुजानि भविष्यन्ति\\
\noindent \includegraphics[scale=0.8]{Images/rg-183.png}  
\end{vwcol}
\vspace{-11mm}

\noindent यथैकं तेषां झदजमस्ति । तस्मात्\renewcommand{\thefootnote}{१}\footnote{द्वादशफलक० {\en K., A.} }द्वादशास्रघन\\
क्षेत्रस्य षष्टित्रिभुजानि भविष्यन्ति । झतलम्ब \\
एकभुजेन गुणितस्तदा त्रिभुजद्वयक्षेत्रफलतुल्यो भविष्यति । तस्मात् त्रिंशत्घाताः संपूर्णधरातलतुल्या भविष्यन्ति ।
इदमेवेष्टम् ॥\\
\begin{center}
\textbf{\large अथ पञ्चमं क्षेत्रम् ॥ ५ ॥ }
\end{center}
\vspace{2mm}

{\ab यद्वृत्तान्तर्विंशत्यस्रघनक्षेत्रस्य\renewcommand{\thefootnote}{२}\footnote{{\en K., A. have} फलक {\en for} अस्र.} त्रिभुजं पतति तत्केन्द्रात्  लम्बस्त्रिभुजस्य भुजे यदा गच्छति तदा त्रिभुजैकभुजलम्बघातस्त्रिंशद्गुणो \renewcommand{\thefootnote}{३}\footnote{फलक {\en K., A.} }विंशत्यस्रघनक्षेत्रस्य संपूर्णधरातलतुल्यो भवति ।}\\

यथा अबं तद्वृत्तं कल्पितं \renewcommand{\thefootnote}{४}\footnote{फलक {\en K.,~A.}}यदन्तर्विंशत्यस्रघनक्षेत्रस्य अबजत्रिभुजं पतितम्~। दहं लम्बः कल्पितः । तस्मादस्य त्रिभुजस्य त्रीणि त्रिभुजानि भविष्यन्ति~। 
\begin{center}
\includegraphics[scale=0.9]{Images/rg-184.png}  
\end{center}
तेषु यथैकं  दबजमस्ति । \renewcommand{\thefootnote}{५}\footnote{{\en K., A. have} फलक {\en for} अस्र.}विंशत्यस्रघनक्षेत्रस्य ईदृशानि षष्टित्रिभुजानि पतिष्यन्ति । त्रिभुजस्यैकभुजेन लम्बश्चेद्गुण्यते षष्टित्रिभुजान्तर्गतक्षेत्रद्वयफलतुल्यो भविष्यति । तस्मात् विंशदूघाताः संपूर्णधरातलतुल्या भविष्यन्ति । इदमेवेष्टम्~॥\\
\begin{center}
\textbf{\large अथ षष्ठं क्षेत्रम् ॥ ६ ॥ }
\end{center}
\vspace{2mm}

{\ab द्वादशफलकघनक्षेत्रं विंशतिफलकघनक्षेत्रं च यदैकगोला- }


\newpage
{\ab न्तः पतति । तदैतद्धरातलयोर्निष्पत्तिस्तथा भवति यथा तद्गोलान्तर्घनहस्तभुजनिष्पत्तिर्विंशत्यस्रघनक्षेत्रभुजेनास्ति । }\\
\vspace{5mm}


अबजं तद्वृतं कल्पितं यदन्तर्द्वयोर्घनक्षेत्रयोः पञ्चभुजं त्रिभुजं च पतितम्~। अबं त्रिभुजस्य भुजः कल्पितः । अजं पञ्चभुजस्य भुजः कल्पितः । तरेखा घनहस्तभुजः कल्पितः । पुनर्दहलम्बः अबरेखायां \renewcommand{\thefootnote}{१}\footnote{निष्काश्यः {\en V.}}निष्कास्यः । दझलम्बः अजरेखायां \renewcommand{\thefootnote}{२}\footnote{निष्काश्यः {\en V.}}निष्कास्यः पुनरयं लम्बो वचिह्नपर्यन्तं वर्द्धनीयः । पुनरवरेखा संयोज्या~। इयं वृत्तदशमांशस्य पूर्णज्या भविष्यति । तस्मात् दझं वृत्तषडंशदशमांशपूर्णजीवयो-
\begin{center}
\includegraphics[scale=1]{Images/rg-185.png}  
\end{center}
र्योगार्द्धतुल्यं भविष्यति । द्वयोः पूर्णजीवयोर्योगार्द्धस्य निष्पत्तिः षडंशजीवार्द्धेन तथास्ति यथा षडंशार्द्धजीवानिष्पत्तिर्दशमांशजीवार्द्धेनास्ति । तस्मात्
झददहयोरपीदृश्येव निष्पत्तिर्भविष्यति । एवं तरेखाअजरेखयोरपि निष्पत्तिर्भविष्यति । तस्मात्तरेखाअजरेखानिष्पत्तिर्दझदहरेखानिष्पत्तितुल्या भविष्यति । तस्मात् अजदझघातो दहतरेखयोर्घाततुल्यो भविष्यति । पुनस्त्रिंशद्गुणितैकघातस्त्रिंशद्गुणितद्वितीयघाततुल्यो
भविष्यति । दझअजघातस्त्रिंशद्गुणितो \renewcommand{\thefootnote}{३}\footnote{द्वादशास्त्र० {\en V.}}द्वादशफलकधरातलक्षेत्र-फलतुल्योऽस्ति~। तस्मात् दहरेखातरेखयोर्घातस्त्रिंशद्गुणितस्तद्धरातल\renewcommand{\thefootnote}{४}\footnote{त्रिंशद्गुणः {\en V.}\\
भा० २७} एवास्ति~। दहअबघातस्त्रिंशद्गुणितो विंशत्यस्रघनक्षेत्रधरातलतुल्योऽस्ति । तस्मात्तरेखानिष्पत्तिः अबरेखया तथास्ति यथा
द्वादशास्त्रधरातलक्षेत्रस्य विंशत्यस्रधरातलेनास्ति । इदमेवेष्टम् ॥\\
\begin{center}
\textbf{\large अथ सप्तमं क्षेत्रम् ॥ ७ ॥ }
\end{center}
\vspace{5mm}

{\ab वृत्तान्तर्गतपञ्चभुजक्षेत्रकोणस्य पूर्णजीवायाः पञ्चगुणः}

\newpage  

{\ab षडंशः तद्वृत्तव्यासस्य त्रयश्चतुर्भागाश्चानयोर्घातः पञ्चभुजक्षेत्रफलतुल्यो भवति~। }\\
\vspace{3mm}

यथा अहं वृत्तं कल्पितम् । तन्मध्ये अबकलजं पञ्चभुजक्षेत्रं
कल्पितम् । 
सन्मुखकोणस्य बजपूर्णज्या कल्पिता । अदहव्यासः कल्पितः । दहं झचिह्ने
\begin{center}
\includegraphics[scale=0.8]{Images/rg-186.png}  
\end{center}
अर्द्धितं कार्यम्~। तस्मात् अझं व्यासस्य त्रयश्चतुर्भागा भविष्यन्ति । जतस्य जवं
तृतीयांशः पृथक्कार्यः । तस्मात् बवं बजस्य पञ्चषष्ठांशा भवन्ति । अझनिष्पत्तिः अदेन 
तथास्ति यथा बतनिष्पत्तिः तवेनास्ति । अझतवघातो बतअदघाततुल्योऽस्ति । अयं द्विगुणितअदबक्षेत्रफलतुल्योऽस्ति । दझम् अदस्यार्द्धमस्ति~। तदा बतअझघातः अदबत्रिभुजस्य त्रिगुणक्षेत्रफलतुल्यो भविष्यति । तवअझघातो बतअझघातयुतस्तदा अझबवघातः पञ्चभुजस्य क्षेत्रफलं भविष्यति । इदमेवेष्टम् ॥\\
\begin{center}
\textbf{\large अथाष्टमं क्षेत्रम् ॥ ८ ॥ }
\end{center}
\vspace{2mm}

{\ab द्वादशधरातलविंशतिधरातलक्षेत्रे यदि गोलमध्ये पततस्तदा तद्धरातलयोर्निष्पत्तिर्गोलान्तर्गतघनहस्तभुजविंशतिधरातलक्षेत्रभुजयोर्निष्पत्तितुल्या भवति । }\\
\vspace{5mm}

पञ्चभुजं त्रिभुजं वृत्तं व्यासश्च पूर्वोक्तवत् कल्पनीयः । बजं घनहस्तस्य  भुजः
\begin{vwcol}[widths={0.7,0.3}, sep=.8cm, rule=0pt]
 संयोज्यः । तस्मात् अयं व्यासस्य त्रयश्चतुर्थांशाः भविष्यन्ति । तदा अयस्य बजपञ्चगुणितषष्ठांशजसस्य च घातः पञ्चभुजक्षेत्रफलतुल्योऽस्ति । तस्मात् अयसंज्ञं द्वादशगुणजसेन गुणितं अथवा दशगुणितवजेन चेद्गुण्यते तदा द्वादशधरातलक्षेत्रस्य संपूर्णधरातलफलं भवति । अयसंज्ञं चेत्\\
 \vspace{5mm}
 
\noindent \includegraphics[scale=0.8]{Images/rg-187.png}  
\end{vwcol}

\newpage
\noindent झतेन गुण्यते तदा त्रिभुजक्षेत्रफलद्विगुणं भवति । तस्मात् अयसंज्ञं दशगुणितझतेन गुण्यते तदा विंशतिधरातलक्षेत्रस्य फलं भवति । तस्मात् द्वयोर्धरातलयोर्निष्पत्तिर्जबझतनिष्पत्तितुल्या भवेत् । इदमेवेष्टम्॥\\
\begin{center}
\textbf{\large अथ नवमं क्षेत्रम् ॥ ९ ॥ }
\end{center}
\vspace{2mm}

{\ab इष्टरेखायाः खण्डद्वयं तथा कार्यं यथा सर्वरेखामहत्खण्डयोर्निष्पत्तिर्महत्खण्डलघुखण्डनिष्पत्तितुल्या भवति तदा  सर्वरेखावर्गमहत्खण्डवर्गयोगतुल्यो यस्या रेखाया वर्गो भवति  पुनः सर्वरेखावर्गलघुखण्डवर्गयोगतुल्यो यस्या रेखाया वर्गो  भवति तदाऽनयोरेखयोर्निष्पत्तितुल्या गोलान्तर्गतघनहस्तभुजविंशतिधरातलभुजयोर्निष्पत्तिर्भवति ॥ }\\
\vspace{5mm}

यथा बजरेखा कल्पिता । अस्या दचिह्ने तथा खण्डद्वयं कृतं यथा
संपूर्णरेखा  महत्खण्डयोर्निष्पत्तिर्महत्खण्डलघुखण्डनिष्पत्ति-तुल्या जाता । महत्खण्डं जदं
कल्पितम् । पुनर्जबव्यासार्द्धेन अबं वृत्तं कार्यम् । हरेखात्रिभुजस्य भुजः कल्पितः~। वरेखा पञ्चभुजकोणस्य पूर्णज्या कल्पिता । झरेखा सा रेखा कल्प्या यस्या वर्गो जबवर्गजदवर्गयोगतुल्योऽस्ति । तरेखा च सा रेखा कल्प्या यस्या \\
\begin{center}
\noindent \includegraphics[scale=0.9]{Images/rg-188.png}  
\end{center}
वर्गो जबवर्गबदवर्गयोगतुल्योऽस्ति । लरेखा च जदतुल्या कल्पिता । तत्र हरेखावर्गो बजरेखावर्गान्त्रिगुणोऽस्ति । तरेखावर्गश्च दजरेखावर्गात्रिगुणोऽस्ति । लरेखावर्गादपि त्रिगुणोऽस्ति । तस्मात् हरेखानिष्पत्तिर्बजरेखया तथास्ति यथा तरेखानिष्पत्तिर्लरेखयास्ति । पुनर्हरेखानिष्पत्तिस्तरेखया तथास्ति यथा बजरेखानिष्पत्तिर्लरेखयास्ति । यदि वरेखाया एतादृशं खण्डद्वयं क्रियते यथा संपूर्णरेखाया महत्खण्डेन निष्पत्तिर्महत्ख-

\newpage
\noindent ण्डलघुखण्डयोर्निष्पत्तितुल्या भवति तदास्य महत्खण्डं झतुल्यं भविष्यति । तस्मात्
वरेखाझरेखयोर्निष्पत्तिर्बजरेखालरेखयोर्निष्पत्तितुल्या भविष्यति । हरेखातरेखयोरपि निष्पत्तितुल्यास्ति । तस्मात् वरेखाहरेखयोर्निष्पत्तिर्झरेखातरेखयोर्निष्पत्तितुल्या भविष्यति । इदमेवेष्टम् ॥\\
\begin{center}
\textbf{\large अथ दशमं क्षेत्रम् ॥ १० ॥ }
\end{center}
\vspace{5mm}

{\ab तत्रेष्टरेखायाः खण्डद्वयं तथा कार्यं यथा सर्वरेखानिष्पत्तिर्महत्खण्डेन तथास्ति यथा महत्खण्डलघुखण्डयोरस्ति । ये ये प्रकारा अस्यां रेखायां भवन्ति ते ते प्रकारा एतन्निष्पत्तिविभागगतास्वन्यरेखासु भवन्ति ।}\\
\vspace{3mm}

यथा अबं जचिह्ने एतन्निष्पत्तिसदृशं खण्डद्वयं कल्पितम् । पुनर्महत्खण्डं च अजं कल्पितम् । अन्या रेखा दहं कल्पिता । अस्या झचिह्ने तन्निष्पत्तौ खण्डद्वयं कल्पितम् । पुनर्महत्खण्डं दझं कल्पितम् । अबअजनिष्पत्तिः अजजबयोर्निष्पत्तितुल्यास्ति~। पुनर्दहदझनिष्पत्तिर्दझझहनिष्पत्तितुल्यास्ति । अबबजघातअजवर्गयोर्निष्पत्तिर्दहहझघातदझवर्गनिष्पत्तितुल्यास्ति । चतुगुर्णअबबजघातअजवर्ग-
\begin{vwcol}[widths={0.6,0.4}, sep=.8cm, rule=0pt]
निष्पत्तिश्चतुर्गुणदहहझघातदझवर्गनिष्पत्ति\\
\noindent तुल्यास्ति~। चतुर्गुणअबबजघातअजवर्गयोगनिष्पत्तिः अजवर्गेण तथास्ति यथा चतुर्गुणितदहहझघातदझवर्गयोगस्य निष्पत्तिर्दझ-\\ 
\noindent \includegraphics[scale=1]{Images/rg-189.png}  
\end{vwcol}
\vspace{-2mm}

\noindent वर्गेणास्ति~। अबबजयोगनिष्पत्तिः अजेन तथास्ति यथा दहहझयोगनिष्पत्तिर्दझेनास्ति । तस्मात् द्विगुणअबनिष्पत्तिः अजेन तथास्ति यथा द्विगुणदहनिष्पत्तिर्दझेनास्ति ।
अबअजयोर्निष्पत्तिर्दहदझयोर्निष्पत्तितुल्यास्ति । अबबजनिष्पत्तिर्दहहझनिष्पत्तितुल्यास्ति । तस्मात्
अबदहनिष्पत्तिः अजदझनिष्पत्तितुल्यास्ति । \renewcommand{\thefootnote}{१}\footnote{निष्पत्त्यापि {\en V.}}जबहझनिष्पत्तेरपि


\newpage
\noindent तुल्यास्ति । तस्मात् ये प्रकारा अजजबयोर्भवन्ति ते सर्वे प्रकारा दहहझयोर्भवन्ति । इदमेवेष्टम् ॥\\
\begin{center}
{\small श्रीमद्राजाधिराजप्रभुवरजयसिंहस्य तुष्टौ द्विजेन्द्रः

श्रीमत्सम्राड् जगन्नाथ इति समभिधारूढितेन प्रणीते

ग्रन्थेऽस्मिन्नाम्नि रेखागणित इति सुकोणावबोधप्रदात-

र्यध्यायोऽध्येतृमोहापह इह विरतिं शक्रतुल्यो गतोऽभूत् ॥

 ॥ \renewcommand{\thefootnote}{१}\footnote{{\en V. omits} इति.}इति चतुर्दशोऽध्यायः ॥ १४ ॥} \\
 \rule{0.7in}{0.3pt}
 \end{center}


\newpage
\afterpage{\fancyhead[CE] {रेखागणितम्}}
\afterpage{\fancyhead[CO] {पञ्चदशमोध्यायः}}
\afterpage{\fancyhead[LE,RO]{\thepage}}
\cfoot{}
\newpage
%%%%%%%%%%%%%%%%%%%%%%%%%%%%%%%%%%%%%%%%%%%%%%%%%%%%%%%%%%%%%%
\newpage
\thispagestyle{empty}
\begin{center}
\textbf{\LARGE ॥ अथ पञ्चदशोऽध्यायः प्रारभ्यते ॥}
\end{center}
\vspace{3mm}

\begin{center}
\textbf{॥ अस्मिन्षट् क्षेत्राणि सन्ति ॥}
\vspace{5mm}

\textbf{\large \renewcommand{\thefootnote}{१}\footnote{{\en V. omits} अथ.}अथ प्रथमं क्षेत्रम् ॥ १ ॥ }
\end{center}
\vspace{2mm}

{\ab तत्र व्यासार्द्धस्य तथाविधे द्विखण्डे \renewcommand{\thefootnote}{२}\footnote{अपेक्षिते {\en K., A.}}कर्त्तव्ये यथा व्यासार्द्धस्य महत्खण्डे या निष्पत्तिस्तथामहत्खण्डस्य लघुखण्डेन भवति तदा वृत्तदशमांशस्य पूर्णज्या महत्खण्डं भवति । }\\
\vspace{3mm}

यथा अबरेखाया जचिह्ने तथा खण्डे कृते । बजं महत्खण्डं कल्पितम् । पुनर् अबरेखया सह बदरेखा वृत्तदशमांशस्य पूर्णजीवातुल्या संयोज्या । \renewcommand{\thefootnote}{३}\footnote{{\en V. notices} तदा {\en also.}}तस्मात् अदरेखा बचिह्ने उपरितननिष्पत्तितुल्यविभागा भविष्यति । पुनर्हवरेखा अबरेखातुल्या कल्प्या~। अस्या झचिह्ने उपरि-
\begin{vwcol}[widths={0.55,0.45}, sep=.8cm, rule=0pt]
तननिष्पत्तितुल्ये खण्डे कृते । वझं\\
\noindent बजतुल्यं कल्प्यम् । तदा अदअबयोर्निष्पत्तिर्हववझयोर्निष्पत्तितुल्यास्ति~। \\
\noindent \includegraphics[scale=1]{Images/rg-190.png}  
\end{vwcol}
\vspace{-2mm}

\noindent 
अबबदयोर्निष्पततिर्वझझहयोर्निष्पत्तितुल्यास्ति । तस्मात् अबझहघातो बदवझघाततुल्यो भविष्यति । अबं वहतुल्यमस्ति । तस्मात् वहझहघातो बदवझघाततुल्यो\renewcommand{\thefootnote}{४}\footnote{समो {\en K., A.}} भविष्यति । वहझहघातो वझवर्गतुल्योऽस्ति । तस्मात् वझं बजतुल्यं बदतुल्यं भविष्यति । तस्मात् बजं वृत्तदशमांशस्य पूर्णजीवा भविष्यति । इदमेवास्माकमिष्टम् ॥\\
\begin{center}
\textbf{\large \renewcommand{\thefootnote}{५}\footnote{{\en V. omits} अथ.}अथ द्वितीयं क्षेत्रम् ॥ २ ॥ }
\end{center}
\vspace{5mm}

{\ab घनहस्तक्षेत्रमध्ये यस्य \renewcommand{\thefootnote}{६}\footnote{फलकानि समानानि {\en K., A.}}फलकाः समाना भवन्ति \renewcommand{\thefootnote}{७}\footnote{तादृशशङ्कुचिकीर्षास्ति.}तादृशः 
शङ्कुरुत्पादनीयोऽस्ति~। }


\newpage
\begin{vwcol}[widths={0.6,0.4}, sep=.8cm, rule=0pt]
यथा बझं घनहस्तः कल्पितः । अझझजअजअहजहझहरेखाः संयोज्याः ।
तस्मात् अजझहमस्माकमिष्टं भविष्यति । कुतः । अस्य भुजा घनहस्तभुजानां कर्णा
भविष्यन्ति । इदमिष्टम्~॥\\
\noindent \includegraphics[scale=0.8]{Images/rg-191.png}  
\end{vwcol}
\vspace{5mm}

\begin{center}
\textbf{\large \renewcommand{\thefootnote}{१}\footnote{{\en V. omits} अथ.}अथ तृतीयं क्षेत्रम् ॥ ३ ॥ }
\end{center}
\vspace{2mm}

{\ab यस्य शङ्कोः फलकानां भुजाः समाना \renewcommand{\thefootnote}{२}\footnote{भवन्ति {\en V.}}भविष्यन्ति तस्यान्तरष्टफलकक्षेत्रं कर्त्तुमिच्छास्ति ।} \\
\vspace{3mm}

\begin{vwcol}[widths={0.7,0.3}, sep=.8cm, rule=0pt]
यथा अबजदं शङ्कुः कल्पितः । अस्य षड् अपि भुजा अर्द्धिताः । अर्द्धचिह्नेषु रेखाः
संयोज्याः~। वझलयतहम् अष्टभुजक्षेत्रमुत्पन्नं भविष्यति~। इदमेवास्माकमिष्टम् ॥\\
\noindent \includegraphics[scale=0.8]{Images/rg-192.png}  
\end{vwcol}
\vspace{5mm}

\begin{center}
\textbf{\large \renewcommand{\thefootnote}{३}\footnote{{\en V. omits} अथ.}अथ चतुर्थं क्षेत्रम् ॥ ४ ॥ }\\
\end{center}
\vspace{7mm}

 {\ab घनहस्तक्षेत्रान्तरष्टफलकक्षेत्रं कर्त्तुमिच्छास्ति ।  }\\
 \vspace{3mm}
 
 यथा अबजदहवझछं घनहस्तः कल्पितः । घनहस्तफलककर्णसंपातचिह्नेषु रेखाः संयोज्याः । यतलकमसअष्टफलकक्षेत्रमुत्पन्नं
भविष्यति ।\\
\begin{center}
\textbf{अस्योपपत्तिः ।}
\end{center}
\vspace{5mm}

तचिह्नात् गफरेखा हअरेखायाः समानान्तरा निष्कास्या । रख-

\newpage
\begin{vwcol}[widths={0.7,0.3}, sep=.8cm, rule=0pt]
रेखा च अदरेखा समानान्तरा निष्कास्या । अनेनैव प्रकारेण सर्वभुजेषु रेखाः संयोज्याः । तदैताः रेखाः समाना भविष्यन्ति । एता रेखास्तत्संपातचि-
\noindent \includegraphics[scale=0.9]{Images/rg-193.png}  
\end{vwcol}
\vspace{-15mm}

\noindent ह्नेषु तत्संबन्धिभुजयोश्च \renewcommand{\thefootnote}{१}\footnote{लम्बा भविष्यन्ति {\en V.}}लम्बाश्च भविष्यन्ति ।\\
एतासु द्वे द्वे  रेखे समकोणसंबन्धिभुजा भविष्यन्ति । \\
तस्मादेतत् कर्णाः समाना भविष्यन्ति । एता एव \\ क्षेत्रभुजाः सन्ति । इदमेवेष्टम् ॥\\
\vspace{5mm}

\begin{center}
\textbf{\large \renewcommand{\thefootnote}{२}\footnote{{\en V. omits} अथ.}अथ पञ्चमं क्षेत्रम् ॥ ५ ॥ }
\end{center}
\vspace{5mm}

{\ab अष्टफलकक्षेत्रमध्ये एकं घनहस्तक्षेत्रं कर्त्तुमिच्छास्ति ।}\\
\vspace{3mm}

यथा अबजदहवम् अष्टफलकक्षेत्रं कल्पितम् । त्रिभुजानां केन्द्राण्युत्पादनीयानि~। केन्द्रेषु च रेखाः संयोज्याः । तत्र झवतयकलमनमिष्टं घनहस्तक्षेत्रमुत्पन्नम्~।\\
\begin{center}
\textbf{अस्योपपत्तिः ।}
\end{center}
\vspace{5mm}

यदि केन्द्रेभ्यस्त्रिभुजभुजेषु लम्बा निष्कास्यास्ते सर्वेऽपि लम्बाः समाना भविष्यन्ति । ते लम्बाः समानकोणसंबन्धिभुजा भविष्यन्ति । कुतः । अष्टफलकक्षेत्रस्य फलकद्वयसंबन्धजनितकोणाः खसमाना \renewcommand{\thefootnote}{३}\footnote{भविष्यन्ति {\en K., A.}}भवन्ति । समाप्तकोणस्य
\begin{vwcol}[widths={0.7,0.3}, sep=.8cm, rule=0pt]
  भुजा घनहस्तभुजतुल्या मिथः समाना भविष्यन्ति । तेषां मध्ये चत्वारश्चत्वार एकधरातलवेष्टनं
करिष्यन्ति । यदि केन्द्रेषु कोणचिह्नेषु च रेखाः संयोज्यन्ते तदैता रेखाः समाना भविष्यन्ति ।\\  \noindent समानकोणसंबन्धिभुजा भविष्यन्ति ।\\
\noindent \includegraphics[scale=0.8]{Images/rg-194.png}  
\end{vwcol}

\newpage
\noindent प्रत्येकचतुर्भुजस्य कर्णाः समाना भविष्यन्ति । तस्मात् समचतुर्भुजसमकोणा भविष्यन्ति । तदोत्पन्नं घनहस्तक्षेत्रं भविष्यति । इदमवेष्टेम् ॥\\

\begin{center}
\textbf{\large अथ षष्ठं क्षेत्रम् ॥ ६ ॥ }
\end{center}
\vspace{2mm}

{\ab तत्र विंशतिफलकक्षेत्रमध्ये द्वादशफलकक्षेत्रचिकीर्षास्ति । }\\

यथा अबजहदवझछतयकलं विंशतिफलकक्षेत्रं कल्पितम् । अस्य त्रिभुजानां केन्द्राण्युत्पादनीयानि । तेषु चिह्नानि कार्याणि । तत्र रेखाः संयोज्याः । तस्मादुत्पन्नं क्षेत्रमिष्टं भविष्यति ।\\

\begin{center}
\textbf{अस्योपपत्तिः ।}
\end{center}

 यदि एभ्यः केन्द्रेभ्यो \renewcommand{\thefootnote}{१}\footnote{{\en V. omits} अथ.}लम्बास्त्रिभुजेषु निष्कास्यन्ते । एते लम्बाः
समाना भविष्यन्ति । समकोणसंबन्धिभुजा
भविष्यन्ति । तस्मात् कोणसन्मुखभुजाः \renewcommand{\thefootnote}{२}\footnote{त्रिभुजभुजेषु निष्कास्यन्ते {\en V.}}समाना भविष्यन्ति । तासु पञ्चपञ्चरेखा एकध-
\renewcommand{\thefootnote}{३}\footnote{{\en V. inserts} अपि. }रातले वेष्टनं कुर्वन्ति ।\\


पुनरपि यदि विंशतिफलकक्षेत्रकर्णः सन्मुखकोणगतो भवति । कर्णार्द्धाच्च पञ्चत्रिभुजेषु
लम्बा निष्कास्याः । त्रिभुजानि तथाविधानि कार्याणि येषां कोणाः कर्णशिरःसंभक्ता\renewcommand{\thefootnote}{४}\footnote{०तलवेष्टनं {\en V.}} भवन्ति । एते लम्बाः समानाश्च स्युः । पुनर्यत्र लम्बाः पतन्ति 
\begin{center}
\noindent \includegraphics[scale=0.8]{Images/rg-195.png}  
\end{center}
ततः कर्णोपरि लम्बा निष्कास्याः । तदैते लम्बा
एकस्मिन्नेव चिह्ने पतिष्यन्ति~। तस्मात् पञ्चरेखा याः \renewcommand{\thefootnote}{५}\footnote{संसक्ता {\en V.}\\
भा० २८}केन्द्रसंसक्तास्ता एकस्मिन्नेव धरातले भविष्यन्ति । पुनरपि त्रिभुजकेन्द्राणामन्तराणि लम्बानां संपातचिह्नात् समानानि भविष्यन्ति~। प्रत्येककेन्द्रद्वयान्तरमपि मिथः समानमस्ति । तदा पञ्चसमभुजकोणा अपि समाना भवि-

\newpage
\noindent ष्यन्ति । पञ्चसमभुजक्षेत्रस्य त्रयस्त्रयः कोणा इष्टक्षेत्रस्य कोणाः
स्युः । तस्मादिष्टक्षेत्रस्य कोणा अपि समाना भविष्यन्ति । इदमेवास्माकमिष्टम् ॥\\

\begin{center}
{\small \renewcommand{\thefootnote}{१}\footnote{{\en K., A. have---}\\

शिल्पशास्त्रमिदं प्रोक्तं ब्रह्मणा विश्वकर्मणे ।\\
पारम्पर्य्यवशादेतदागतं धरणीतले ॥\\
तद्विच्छिन्नं महाराजजयसिंहाज्ञया पुनः ।\\
प्रकाशितं मया सम्यग् गणकानन्दहेतवे ॥\\}श्रीमद्राजाधिराजप्रभुवरजयसिंहस्य तुष्टौ द्विजेन्द्रः\\
श्रीमत्सम्राड् जगन्नाथ इति समभिधारूढितेन प्रणीते ।\\

ग्रन्थेऽस्मिन्नाम्नि रेखागणित इति सुकोणावबोधप्रदात-\\

र्यध्यायोऽध्येतृमोहापह इह विरतिं विश्वसंख्यो गतोऽयम् \renewcommand{\thefootnote}{२}\footnote{{\en V. has after this} समाप्तोऽयं ग्रन्थः । शुभं भूयात् । सं० १७८४.\\
युगवसुनगभूवर्षे शुचि शुक्ले युगतिथौ रवेर्वारे ।\\ 
व्यलिखल्लोकमणिः किल सम्राजामाज्ञया पुस्तम् ॥ १ ॥} ॥}\\

\rule{0.7in}{0.3pt}
\end{center}

\newpage
\afterpage{\fancyhead[CE] {}}
\afterpage{\fancyhead[CO] {}}
\afterpage{\fancyhead[LE,RO]{\thepage}}
\cfoot{}

%%%%%%%%%%%%%%%%%%%%%%%%%%%%%%%%%%%%%%%%%%%%%%%%%%%%%%%%%%%%%%
\newpage
\renewcommand{\thepage}{\roman{page}}
\setcounter{page}{1}
\thispagestyle{empty}
\begin{center}
\textbf{\en APPENDIX I.}
\vspace{2mm}

\rule{0.4in}{0.3pt}
\end{center}
\vspace{3mm}

{\en Collation of the Ms. of the Rekhȃgaṇita in the Benares Sanskrit College Library, the one copied by Lokamaṇi under instructions from Jayasiṁha.}
\vspace{5mm}

\begin{center}
{\en  DESIGNATED V.}
\vspace{3mm}

Books VII., VIII., IX.
\end{center}

\begin{center}
\begin{tabular}{lll}

Page 1 & L. 2 & तत्रोनचत्वारिंशत्$^{०}$.\\
~~~~~" & L. 6 & सन् {\en is omitted.}\\
~~~~~" & L. 10 & समानं भागद्वयं {\en for} भागद्वयं समानं.\\
~~~~~" & L. 18 & स विषमविषमः ।\\
Page 3 & L. 5 & {\en and 12} $^{०}$रपवर्त्तकः.\\
~~~~~" & L. 18 & अहशेषं.\\
Page 4 & L. 8&  $^{०}$रपवर्त्तको.\\
~~~~~" &  L. 10& महदङ्ककल्पनं क्रियते.\\
~~~~~" & L. 17& करिष्यति {\en for} करोति.\\
Page 5 & L. 4 & चतुर्थक्षेत्रम्.\\
~~~~~" & L. 11-12& $^{०}$रपवर्त्तनाङ्केन.\\
~~~~~" & L. 16& $^{०}$र्योगो राशियोगस्य स एवांशो भविष्यति.\\
Page 7 & L. 1&  जझमुभयोः.\\
~~~~~" & L. 3 & पुनः प्रकारान्तरम्.\\
~~~~~" & L. 10 & अथाष्टमक्षेत्रम्.\\
Page 8 & L. 4 & जझस्यांशौ यथा भवतस्तथा.\\
~~~~~" & L. 6 & नवमक्षेत्रम्.\\
Page 9 & L. 2-3 & यावदंशो भविष्यति.\\
~~~~~" & L. 13 & अथैकादशक्षेत्रम्.\\
Page 11 & L. 2 & $^{०}$र्निष्पत्तेर्निश्चयः.\\
~~~~~" & L. 12 & अथ {\en is omitted.}\\
~~~~~" & L. 25 & निष्पत्तिविनिमयः.\\
Page 12 & L. 10-11 &  तस्माद्रूपं जदं.\\
\end{tabular}

\end{center}

\newpage
\begin{center}
\begin{tabular}{llp{2in}}

Page 14 & L. 2 & कल्पितम्.\\
~~~~~" & L. 9 &  अथोनविंशति०.\\
~~~~~" & L. 19-20 & झं कल्पितम्.\\
~~~~~" & L. 20 & वं कल्पितम्.\\
~~~~~" & L. 21 & वं हं जातम्. \\
Page 16 & L. 3 & तदा वते त एवां$^{०}$.\\
Page 17 & L. 2 & द्वौ भिन्नाङ्का$^{०}$.\\
~~~~~"& L. 12 & भिन्नाङ्को {\en for} भिन्नो.\\
~~~~~" & L. 14-15 & जं बाङ्काद्भिन्नो भविष्यति.\\
Page 18 & L. 22 & भिन्नं {\en for} भिन्नो.\\
Page 19 & L. 9 & इदमेवास्माक$^{०}$.\\
Page 23 & L. 2 & तं अ.\\
~~~~~" & L. 11 & निःशेषो.\\
~~~~~" & L. 17 & अं बं प्रत्येकं जं निःशेषं.\\
Page 26 & L. 3 & भविष्यति.\\
~~~~~" & L. 11-12 & {\en For} तन्नामकः {\en the Ms. has} हरनामकः {\en on the margin (p. 150 Ms.).}\\
Page 28 & L. 1 & प्रारभ्यते {\en is omitted.}\\
Page 29 & L. 5 & अं बं. \\
~~~~~" & L. 18 & भविष्यतः {\en for} भवतः.\\
Page 30 & L. 1 & अथ चतुर्थं क्षेत्रम्.\\
~~~~~" & L. 5 & तलघ्वङ्कः.\\
~~~~~" & L. 8 & ललघ्वङ्कः.\\
~~~~~" & L. 9 & तथा {\en is omitted.}\\
~~~~~" & L. 11 & लसनमअङ्का$^{०}$.\\
Page 31 &  L. 1 & छनिःशेषकमासीत्.\\
~~~~~" & L. 3-4 & तस्मात् लसनमा. \\
~~~~~" & L. 5 & अथ पञ्चमक्षेत्रम्.\\
~~~~~" & L. 7 & भवति {\en for} भविष्यति.\\
Page 32 & L. 6 & अथ सप्तमक्षेत्रम्.\\
~~~~~" & L. 7 & आद्यङ्को$^{०}$.\\
Page 33 & L. 3 & अबनिष्पत्तिसमास्ति.\\
Page 34 & L. 2 & तथा {\en for} यथा.
\end{tabular}
\end{center}

\newpage
\begin{center}
\begin{tabular}{llp{2.8in}}
Page 35 & L. 4 & घनस्य घनेन निष्पत्ति$^{०}$.  \\
~~~~~" & L. 23 &  $^{०}$निष्पत्तिसमा भविष्यति.\\
Page 36 & L. 1 & वनसतगफकएते.\\
~~~~~" & L. 18 & इदमेवास्माकमिष्टम् {\en after} करिष्यति.\\
~~~~~" & L. 22 & पञ्चदशं क्षेत्रम्.\\
Page 7 & L. 1 & जः भुजः कल्पितः.\\
~~~~~" & L. 7 & करिष्यति for करोति.\\
Page 38 & L. 6 & इदमेवास्मदिष्टम्.\\
~~~~~" & L. 20 & अन्योर्निष्पत्तिः कमनिष्पत्तितुल्या आसीत् । जझनिष्पत्तितुल्याप्यासीत् । कुतः । हं कमाभ्यां गुणितौ अनौ
जातौ । पुनः सबनिष्पत्तिर्मलनिष्पत्तितुल्यास्ति । जझनिष्पत्तितुल्याप्यस्ति । {\en \&~c.}\\
~~~~~" & L. 23 & इदमेवास्मदिष्टम्.\\
Page 39 & L. 5 & करोति । हः जं झतुल्यं निःशेषं करोति इति कल्पितम् । पुनर्दः जं वतुल्यं निःशेषं करोति~। हः बं वतुल्यं निः
शेषं करोतीत्यपि । {\en \& c.}\\
~~~~~" & L. 6 & अबौ सजातीयौ घातौ.\\
Page 40 & L. 1 & हतघातः कलघाततुल्यः.\\
~~~~~" & L. 7 & एकरूपनिष्पतौ.\\
Page 41 & L. 15 & इदमेवास्मदिष्टम्.\\
Page 42 & L.14 & भविष्यतः {\en for} भवतः.\\
Page 43 & & after L. 13 {\en and before} अस्योपपत्तिः L. 14  {\en the Ms. has} यथा अबौ घनफलाङ्कौ सजातीयौ कल्पितौ । एतौ द्वयोर्घनयोर्निष्पत्तौ भविष्यतः ।  \\
~~~~~" & L. 25 & समाप्तः {\en is omitted.}\\
Page 44 & L. 3 & तत्र प्रथमक्षेत्रम्.\\
~~~~~" & L. 12 & अथ द्वितीयक्षेत्रम्.\\
~~~~~" & L. 24 & अथ तृतीयक्षेत्रम्.\\
Page 45 & L. 8 & अथ चतुर्थक्षेत्रम्.\\
~~~~~" & L. 15 & पञ्चमं क्षेत्रम्.\\
Page 46 & L. 1 & अथ षष्ठक्षेत्रम्.\\
~~~~~" & L. 10 & योगसंज्ञाङ्कः {\en for} योगाङ्कः.\\
~~~~~" & L. 26 & {\en after} दं वर्गो भविष्यति, {\en the Ms. has} यतो रूप-

\end{tabular}
\end{center}

\newpage
\begin{center}
\begin{tabular}{llp{2.8in}}

&& निष्पत्तिः बेन तथास्ति यथा बनिष्पत्तिः देनास्ति~। अनेनैव प्रकारेण झः वर्गो भविष्यति । पुनर्जः घनोऽस्ति ।\\
Page 47 & L. 15 & दशमं क्षेत्रम्\\
~~~~~" & L. 17 & $^{०}$श्चेदवर्गो भवति {\en for} $^{०}$श्चेद्वर्गो न भवति.\\
~~~~~" & L. 22-3 & अबनिष्पत्तिसमास्ति ।\\
Page 48 & L. 24-5 & हऔ जझौ क्रमेण तुल्यं निःशेषं करिष्यतः.\\
Page 49 & L. 2 & हः बं निःशेषं करिष्यति.\\
Page 50 & L. 10 &  हदं कल्पितः.\\
Page 51 & L. 3 & इष्टमस्मत्समीचीनम् ।\\
~~~~~" & L. 14-15 & तस्य दझस्य वर्गश्च दहहझघातो द्विगुणः दहवर्गहझवर्गयोगतुल्यश्चास्ति ।\\
Page 52 & L. 15 & अथैकोनविंशं क्षेत्रम्.\\
Page 53 & L. 1 & विंशतितमं क्षेत्रम्.\\
~~~~~" & L. 14 & एकविंशतितमं क्षेत्रम्.\\
Page 54 & L. 5 & विषमतुल्या विषमाङ्काः.\\
Page 55 & L. 17 & अष्टाविंशतितमं क्षेत्रम्.\\
Page 57 & L. 10 &  प्रकटमेवास्ति.\\
~~~~~" & L. 15 & पञ्चत्रिंशत्तमं क्षेत्रम्.\\
Page 59 & L. 18-19 & $^{०}$कहयोगेन तुल्या भविष्यति.\\
Page 60 & L. 18 & {\en The Ms. omits} समाप्तः.
\end{tabular}
\end{center}
\vspace{2mm}

\begin{center}
\rule{0.5in}{0.3pt}
\end{center}

\newpage
\thispagestyle{empty}

\begin{center}
\textbf{{\en APPENDIX II.}}
\end{center}
{\en {\emph The Varioe Lectiones of} the Ms. of the work in charge of the Ȃnandȃśrama, Poona, as compared with the text. The Ms. was received
for collation through Prof. S.R. Bhȃṇdȃrakar.}
\begin{center}
\begin{tabular}{llp{3in}}
Page 1 & L. 2 & {\en The Ms. drops} श्रीलक्ष्मीनृसिंहाय नमः ॥\\
~~~~~" & L. 3-4 & {\en For the first verse} गणाधिपं- {\en the Ms. has two verses} गजाननं गणाधिपं- {\en as found in K.}\\
Page 2 & L. 5 & तदुच्छिन्नं {\en for} तद्विच्छिन्नं.\\
Page 3 & L. 1 & प्रारभ्यते {\en is dropped.}\\
Page 3 & L. 2 & अत्र {\en for} तत्रास्मिन्.\\
~~~~~" & L. 2 & सन्ति {\en after} पञ्चदशाध्यायाः.\\
~~~~~" & L. 2 & शकलानि {\en for} क्षेत्राणि.\\
~~~~~" & L. 3 & {\en The Ms. omits the sentence} तत्र प्रथमा०-प्रदर्श्यन्ते.\\
~~~~~" & L. 5 & बिन्दुर्वाच्यः {\en for} बिन्दुशब्दवाच्यः.\\
~~~~~" & L. 7 & विस्तारदैर्ध्ययोर्यद्भिद्यते {\en for} विस्तारदैर्ध्याभ्यां भिद्यते.\\
~~~~~" & L. 7 & तद् धरातलं तदेव क्षेत्रम् for तद्धरातलक्षेत्रसंज्ञं भवति.\\
~~~~~" & L. 7 & {\en After} भवति {\en the Ms. inserts} तद्द्विविधम् । एकं जलवत् समं द्वितीयं
विषमम्.\\
~~~~~" & L. 8 & एका वक्रा अन्या सरला {\en for} एका सरला अन्या वक्रा.\\
~~~~~" & L. 10-11 & $^{०}$बिन्दुनाच्छाद्यन्ते {\en for} बिन्दुनाच्छादिता इव दृश्यन्ते.\\
~~~~~" & L. 11 & ज्ञेया {\en is omitted.}\\
~~~~~" & L. 12-13 & धरातलमपि समं विषमं च ज्ञेयम् । समं यथा । यत्र बिन्दून् {\en for} अथ धरातल$^{०}$---बिन्दून्.\\
~~~~~" & L. 14 & भवति {\en for} स्यात्.\\
~~~~~" & L. 15 & अन्यथा विषमम् {\en is dropped.}\\
~~~~~" & L. 17 & या सूच्यु० {\en for} सूच्यु० {\en and} स {\en for} सैव.\\
~~~~~" & L. 18 & समकोणः विषमकोणश्च {\en for} समो विषमश्च.\\
~~~~~" & L. 18 & {\en After} विषमकोणश्च {\en the Ms. inserts} अथ समकोणविषमकोणलक्षणम्.
\end{tabular}
\end{center}

\newpage
\begin{center}
\begin{tabular}{llp{3in}} 

Page 3 & L. 19 & भवतः {\en for} स्तः.\\
Page 4 & L. 4 & समकोणस्तु {\en for} इह समकोणः.\\
~~~~~" & L. 4 & सरलकुटिलरेखाभ्यां {\en is dropped.}\\
~~~~~" & L. 8 & तत्र {\en is dropped.}\\
~~~~~" & L. 8 & उच्यते {\en for} भवति.\\
~~~~~" & L. 9 & तच्च {\en is dropped.}\\
~~~~~" & L. 12 & {\en The Ms. agrees with D. for} तस्मादेव {\en \& c. in place of} चक्राकारा {|en \& c.}\\
~~~~~" & L. 14 & वृत्तं क्षेत्रं {\en for} वृत्तक्षेत्रं.\\
Page 5 & L. 1 & मध्यबिन्दु {\en for} बिन्दुः.\\
~~~~~" & L. 2 & भवति {\en for} स्यात्.\\
~~~~~" & L. 4 & केन्द्रगा न भवति {\en for} केन्द्रगा न स्यात्.\\
~~~~~" & L. 11 & तत् त्रिभुजं {\en for} तत्.\\
~~~~~" & L. 12 & येत्रैको० {\en for} यस्यैको०.\\
~~~~~" & L. 12 & न्यूनकोणौ {\en for} न्यूनौ.\\
~~~~~" & L.12 & स्तः {\en is dropped.}\\
~~~~~" & L. 12 & अधिककोणं त्रिभुजं {\en for} अधिककोणत्रिभुजं\\
Page 6 & L. 12 & च {\en is dropped.}\\
~~~~~" & L. 12 & न्यूनकोणं भवेत् {\en for} न्यूनकोणत्रिभुजं स्यात्.\\
~~~~~" & L. 3  & अथ च {\en after} समानं.\\
~~~~~" & L. 3 &  यद्यपि {\en for} अपि.\\
~~~~~" & L. 5 & अथ च {\en after} समानं.\\
~~~~~" & L. 5 & मिथः {\en is dropped.}\\
~~~~~" & L. 6 & आयतं च ज्ञेयम् {\en for} आयतसंज्ञम्.\\
~~~~~" & L. 7 & समं {\en for} च समं\\
~~~~~" & L. 7 & विषमकोणं सम० {\en for} विषमकोणसम$^{०}$.\\
Page 7 & L. 2 & च {\en before} ज्ञेयम्.\\
~~~~~" & L. 6 & {\en The Ms. agrees with D. and K. in its
omission.}\\
Page 8 & L. 8 & यावतः {\en for} यावन्तः.\\
~~~~~" & L. 11 & तस्य {\en for} तत्र.\\
~~~~~" & L. 13 & यत्राल्प$^{०}$ {\en for} यत्र च स्वल्प$^{०}$.\\
~~~~~" & L. 13 & {\en The Ms. inserts} भवति {\en after} ०न्तरं.
\end{tabular}
\end{center}

\newpage
\begin{center}
\begin{tabular}{llp{3in}} 

Page 8 & L. 14 & $^{०}$रेखाद्वयसंयोगं for $^{०}$रेखाद्वयसंयोगः.\\
~~~~~" & L. 19 & प्रथमक्षेत्रम् {\en for} प्रथमं क्षेत्रम्.\\
~~~~~" & L. 20 & तत्र {\en is dropped.}\\
~~~~~" & L. 21 & च {\en is dropped.}\\
Page 9 & L. 2 & बकेन्द्रं.\\
~~~~~" & L. 2 & द्वितीयं {\en is dropped.}\\
~~~~~" & L. 4 & ततः {\en for} तत्र.\\
~~~~~" & L. 5 & जातं समानत्रिभुजम्.\\
~~~~~" & L. 7 & अतो {\en for} यतो.\\
~~~~~" & L. 8 & {\en The Ms. inserts} कुतः {\en before} अजवृत्तस्य.\\
~~~~~" & L. 11 & अथ द्वितीयक्षेत्रम्.\\
~~~~~" & L. 12 & तत्र {\en is dropped.}\\
~~~~~" & L. 14 & कल्पितम् {\en is dropped.}\\
~~~~~" & L. 17-18 & तदेव {\en for} दव.\\
~~~~~" & L. 18 & च {\en is dropped.}\\
~~~~~" & L. 19 & पुनर् {\en is dropped.}\\
Page 10 & L. 1 & दझरेखा समानास्ति ।\\
~~~~~" & L. 2 & तत्र {\en and} अस्ति {\en are dropped.}\\
~~~~~" & L. 3 & च {\en is dropped.}\\
~~~~~" & L. 3 & पुनर् {\en is dropped.}\\
~~~~~" & L. 3 & च {\en and} अस्ति {\en are dropped.}\\
~~~~~" & L. 5 & $^{०}$समाना जातास्तीति.\\
Page 10 & L. 6 & अथ तृतीयक्षेत्रम्.\\
~~~~~" & L. 8 & इति चेत् {\en is dropped.}\\
~~~~~" & L. 10 & निष्कासनीया.\\
\multicolumn{3}{l}{\en Hereafter only material changes are noted, as the Ms. is found to} \\
\multicolumn{3}{l}{\en agree mostly with D.}\\ 
Page 14 & L. 6 & इमौ तु {\en for} इमौ तौ.\\
Page 15 & L. 10 & कार्यम् {\en for} कृतम्.\\
Page 25 & L. 15 & दधिको भवति {\en for} $^{०}$दधिको भवतीति निरूप्यते.\\
Page 35 & L. 7 & {\en The Ms. inserts} तस्मादुक्तमेव सिद्धम् {\en after}
इदमनुपपन्नम्.\\
Page 60 & L. 2 & यथान्येष्ट {\en for} यथेष्ट.\\
२९ & &\\
\end{tabular}
\end{center}
\newpage

\begin {center}
\begin{tabular}{llp{3in}} 

Page 62 & L. 13 & {\en After} $^{०}$णोस्ति, {\en the Ms. reads as under:---} यदा अबं अजं तुल्यं भविष्यति तदा तचिह्नं वचिह्नं भविष्यति दतजं सरलं कारेखा भविष्यति । यदा अबं अजादधिकं स्यात् तदाथवा तचिह्नं वचिह्नं न भविष्यति अथवा अन्यच्चिह्नं भविष्यति । तच्चिह्नं झवरेखोपरि पतिष्यति वा झवरेखाया बहिः पतिष्यति । क्षेत्रत्रयेऽपि {\en \& c.}\\

Page 82 & L. 5-6 & खण्डद्वयं समानं कार्यमथवा खण्डद्वयं च न्यूनाधिकं कार्यं
तदा खण्डद्वयघात$^{०}$ {\en \& c.}\\
Page 108 & L. 13-16 & व्याससूत्रवृत्तपालिसंपातजनितः वृत्तान्तर्गतकोणः सरलरेखोत्पन्नेभ्यः सर्वेभ्यो न्यूनकोणेभ्योऽधिको भवति । लम्बवृत्तपालिसंपातजनितः कोणः सर्वेभ्यो न्यूनकोणेभ्यो
न्यूनो भवति ॥\\
Page 124 & L. 17-18 & तत्र वृत्ताद्बहिर्दूरस्थितैकचिह्नादेका रेखा कर्णानुकारा
वृत्तपालिमात्रलग्ना कार्यो {\en \& c.}\\
Page 134& L. 19& बकोणः संपूर्णखण्डद्वययोगतुल्यदकोणतुल्योऽस्ति {\en for} वकोण उभयोरेक एवास्ति । शेषम् {\en is dropped.}\\
Page 144 & L. 5 & महान् गुणगुणितलघुतुल्यो भवति {\en is dropped.}\\
~~~~~" & L. 7 & लघोर्यावद्धाततुल्यं भवति महान् गुणगुणितलघुतुल्यं भवति
तत्रैको राशिर्द्वितीयराशे० {\en \& c.}\\
Page 147 & L. 16 & द्वितीये {\en for} तृतीयगुणनफले \\
Page 199 & L. 3-4 & पुनस्तगं तनतुल्यं पृथक् कार्यम् । मसं लमतुल्यं.....\\
~~~~~" & L. 8 & मगक्षेत्रं {\en for} सगक्षेत्रं.\\
~~~~~" & L. 9 & हखक्षेत्रं {\en for} सफगक्षेत्रं.\\
~~~~~" & L. 10 &  हबखण्डोपरि {\en for} अहखण्डोपरि.\\

~~~~~" & L. 10 &  हखक्षेत्रं {\en for} अफक्षेत्रं.\\
~~~~~" &  L. 11 & अहद्वितीय० {\en for} हबद्वितीय$^{०}$\\
~~~~~" & L. 11 & मसक्षेत्रं {\en for} हखक्षेत्रं.\\
Page 201 & L. 20 & झहवर्गेणा० {\en for} दहवर्गेणा$^{०}$.\\
{\en Vol. II.} && \\
Page 5 & L. 6 & भवन्ति {\en for} भवति.\\
~~~~~" & L. 15-16 & तदानयोर्योगः राशियोगस्य एवांशो भविष्यति {\en for} तदा
तयोर्योगो राशिर्भविष्यति ।\\

Page 69 & L. 19 & कल्पनीया भवति {\en for} कल्पनीयो भवति.\\
\end{tabular}

\rule{0.5in}{0.3pt}
\end{center}

\afterpage{\fancyhead[CE] {}}
\afterpage{\fancyhead[CO] {}}
\afterpage{\fancyhead[LE,RO]{\thepage}}
\cfoot{}

%%%%%%%%%%%%%%%%%%%%%%%%%%%%%%%%%%%%%%%%%%%%%%%%%%%%%%%%%%%%%%
\newpage
\renewcommand{\thepage}{\Roman{page}}
\setcounter{page}{1}
\thispagestyle{empty}

\begin{center}
\textbf{NOTES.}\\
\vspace{5mm}

BOOK VII.\\
\vspace{3mm}

DEFINITIONS.
\end{center}
\vspace{3mm}

\begin{quote}
 अङ्क = {\en A number.}\\
 रूप = {\en A unit, one.}\\
 बृहदङ्को गुणगुणितलघ्वङ्कतुल्योऽस्ति = {\en The greater number is a multiple
(lit. equal to the less number repeated a number of times) of the less number.}\\
 समाङ्क = {\en An even number.}\\
 विषमाङ्क = {\en An odd number.}\\
 लब्धि = {\en A quotient.}\\
 प्रथमाङ्क = {\en A prime number.}\\
 योगाङ्क = {\en A composite number.}
 मिलितसंज्ञौ = {\en Commensurable.}
 हर = {\en A divisor.}
 भिन्नाङ्क = {\en Incommensurable.}
 समसम = {\en Evenly even.}
 घात = {\en A product.}
 \end{quote}

{\en A} समसम {\en number is defined as one which, when divided by an even number, gives an even quotient. This is not a very
accurate definition. 24 when divided by 8 gives 3 as its quotient, and when divided by 6 gives 4 as its quotient. Is 24
then} समसम {\en according to definition 6 or} समविषम {\en according to definition 8? To make the definitions 6 and 8 accurate, therefore, we should understand} समेन {\en to be equal to} यावत्समेन, {en i.e., all even numbers.}\\
\vspace{5mm}

{\en A} समसम {\en number is thus equal to that which all even
numbers which measure it measure it by even numbers; and
a} समविषम {\en number is one which all even numbers which
measure it measure it by odd numbers.}\\
\vspace{5mm}

{\en A} पूर्ण {\en or perfect number is one which is equal to the sum of}


\newpage

\noindent {\en its measures. Thus the numbers that measure 6 are 1, 2, and 3 and their sum (1+2+3) is 6. The numbers that measure 28 are 1, 2, 4, 7 and 14 and their sum (1+2+4+7+14) is 28. A list of such numbers is given in the Introduction to Vol. I. {\emph Vide Intro. p. 12} foot note.}\\

\noindent {\en Prop. I.}

अपवर्त्तनाङ्क = {\en A common measure.}\\

\noindent {\en Prop. IV.

A small number or quantity is a part of a large number or
of its multiple.\\

\noindent Prop. VI.}

यावदंशः = {\en Parts.

Bil.'s def. of parts is as under:---

When a less number does not measure a greater one, the less is parts of the greater.

The enunciation of Prop. VI. is---

If two numbers are the same parts of two other numbers, then the sum of the first two shall be the same parts of the sum of the second two.

6 and 8 are the same parts of 9 and 12, therefore 14 is the same parts of 21.\\

\noindent Prop. XI.}

निष्पत्ति = {\en Ratio.}\\

\noindent {\en Prop. XXVIIL}

{\en The latter part of the definition seems faulty.} 'तदा तावङ्कावपि भिन्नौ भविष्यतः' {\en should be the reading in place of} 'तदा तदङ्कयोगयो-रन्तरमपि भिन्नं भविष्यति ।'\\

\noindent {\en Prop. XXXVII.}

{\en If one number measures another number, the quotient is a part called by that name (i.e. by the name of the divisor).

Bil.'s enunciation of it is as under:---

'If a number measure any number, the number measured shall have a part after the denomination, of the number measuring. 

The Prop. means that if 3 measure any number, that number}

\newpage
{\en has a third part, if 4 measure any number, that number has a fourth part
and so fourth.}\\

\noindent {\en Prop. XXXVIII.}

{\en Bil.'s enunciation of it is:---

'If a number have any part, the number whereof the part taketh its
denomination shall measure it.'}
\begin{center}
{\en BOOK VIII.}
\end{center}
{\en Prop. XVI.

If between two like superficial numbers there is a mean
proportional number, then the ratio of the products shall be
equal to the square of the ratio of their sides of like proportion.}\\

 सजातीयघातफलाङ्कौ = {\en Products of two numbers which are their sides} (भुजौ) {\en are called} घातफलाङ्कौ {\en and when the sides are in the same ratio, the products are said to be like or similar.\\

6 and 24 have 2 and 3 and 4 and 6 respectively as their sides and 2 and 3 are in the same ratio as are 4 and 6. 6 and 24 are their like superficial or plain numbers.\\

\noindent Prop. XVII.}

 सजातीयघनफले = {\en Solid numbers are those which are products of three numbers. Like solid numbers, 30 and 240, have 2, 3 and 5, and 4, 6 and 10 as their sides and these sides are in the same ratio. Therefore 30 and 240 are similar solid numbers.}
\begin{center}
{\en BOOK IX.}
\end{center}

\noindent {\en Prop. XII.

Page 49} कल्पितम् {\en in L. 3 seems to be improper. It should be} जातम्.\\

\noindent {\en Prop. XXVII.}

{\en Page 55 L. 15. It should be} शेषः अजं जदम् {\en instead of} शेषः अजम्.\\

\noindent {\en Prop. XXXVII.}

{\en If in a certain series of numbers which are in the same ratio a number equal to the second be taken from the first and also from the last, then the ratio of the first remainder to the first}

\newpage
\noindent {\en number shall be equal to that of the second remainder to the sum of all the terms in the series except the last.}

 अबाद्यङ्कयोगेन {\en is the reading of all the Mss. It is equal to the sum of all the terms beginning with अब except the last.}\\

\noindent {\en Prop. XXXVIII.

This Prop. pertains to a perfect number. In a certain series of numbers beginning with unity, in which each succeeding number is double of the preceding one and the terms are in a duplicate ratio, if the sum of the terms be a prime number, then the product of this sum and the last number shall be a perfect number.

1, 2, 4, 8, 16---The sum of this series is 31, a prime number. Then the product of 16 and 31, which is 496, is a perfect number.}
\begin{center}
\vspace{1mm}

BOOK X.
\vspace{2mm}

Definitions.
\end{center}

मिलितप्रमाणानि = {\en Commensurable magnitudes ( lines, superficies and solids ).}

भिन्नप्रमाणानि = {\en lncommensurable magnitudes.}

मिलितवर्गाभिधा रेखाः = {\en Lines commensurable in power.}

भिन्नवर्गाभिधा रेखाः = {\en Lines incommensurable in power.}

मूलदराशिः = {\en Rational. It comprehends}

\begin{enumerate}[itemsep=1pt,parsep=2pt]
\item {\en The line first supposed and set forth,}
\item {\en Lines commensurable to it,}
\item {\en The square on it,}
\item {\en Such superficies as are commensurable to the square.} 
\end{enumerate}

करणी = {\en Surds or irrational. It comprehends}

\begin{enumerate}[itemsep=1pt,parsep=2pt]
\item {\en The line which is incommensurable to the first line supposed and set forth,}
\item {\en The superficies which is incommensurable to the square described on the rational line first supposed and set forth,}
\item {\en The line the square of which shall be equal to the above superficies.}
\end{enumerate}

करणी {\en or} रज्जुकरणी {\en originally meant a cord of reeds used by}

\newpage
\noindent {\en the sacrificial priest to measure the side of a square altar. It then came to mean the side of a square and lastly the square root of a number which cannot be worked out exact, but which can be represented only graphically. Vide Dr. Thebaut's Article on the S'ulva Sutras in the Journal of the Asiatic Society of Bengal 1875, pp. 274-5.\\

\noindent Prop. XV.

If the sides containing a rectangle be rational, the rectangle shall also be rational}

 अङ्कसंज्ञार्ह = {\en rational.}\\

\noindent {\en Prop. XVII.

It teaches what a medial superficies and a medial line are. A rectangle which has its sides commensurable in power only and not in length shall be irrational and is called a medial superficies; and the line the square of which is equal to this figure is irrational and is called a medial line.\\

\noindent Prop. XXXIV.

It teaches the formation of the first bi-medial line. If two
medial lines commensurable in power only and containing a
rational superficies be added together the line thus formed
shall be irrational and is called the first bi-medial line.\\

\noindent Prop. XXXV.

It teaches the formation of the second bi-medial line. If two medial lines commensurable in power only and containing a medial superficies be added together, the whole line is irrational and is called the second bi-medial line.\\

\noindent Prop. XXXVI.}

 अधिकरेखा = {\en A greater line.

If two lines be incommensurable in power, the sum of
their squares be rational and twice their rectangle be a medial superficies, then the whole line formed by these two lines shall be irrational and is called a greater line.}
\begin{center}
Second definitions p. 90.
\end{center}
प्रथमयोगरेखा = {\en The first binominal line.}

\newpage
{\en This and other lines are all explained in the Intro. to Vol. I.pp. 15-19.\\

\noindent Prop. LII.}

प्रथममध्ययोगरेखा = {\en The first bimedial line.\\
\noindent Prop. LXX.}

अन्तररेखा = {\en A residual line.}\\
\noindent{\en Prop. LXXIII.}

न्यूनरेखा = {\en A less line.}
\begin{center}
{\en Third Definitions ( p.~110).}
\end{center}

प्रथमान्तररेखा = {\en The first residual line.

\noindent Prop. LXXXIX.}

 प्रथममध्यान्तररेखा = {\en The first medial residual line.}
\begin{center}
{\en BOOK XI.
\vspace{3mm}

Definitions.}
\end{center}

 पिण्डः = {\en Depth.}

 घनक्षेत्रम् = {\en A solid body.}

 शंकुः = {\en A cone or a pyramid.}

 छेदितघनक्षेत्रम् = {\en A prism.}

 गोलक्षेत्रम् = {\en A sphere.}

 सूचीफलकशङ्कुघनक्षेत्रम् = {\en A pyramid.}

 समतलमस्तकपरिधिरूपं शङ्कुघनक्षेत्रम् or समतलमस्तकशङ्कुक्षेत्रम् = {\en A
\indent cylinder.}

 घनकोणः = {\en A solid angle.}\\

\noindent{\en Prop. XIX.}

 संपातरेखा = {\en Common section.}\\

\noindent{\en Prop. XXIV.}

 समानान्तरधरातलघनक्षेत्रम् = {\en A parallelepiped.}

\noindent{\en Prop. XL.}

 घनहस्तक्षेत्रम् = {\en A parallelepiped.}
\begin{center}
{\en BOOK XII.}
\end{center}
{\en Prop. IIL.}

 त्र्यस्रफलकशङ्कुः = {\en A pyramid having a triangle as its base.

Every pyramid having a triangle as its base may be divided}

\newpage 

\noindent {\en into four parts of which two are pyramids equal and like to one another and the other two are equal prisms greater than half the whole pyramid.\\

\noindent Prop. IV.

If two pyramids of equal altitudes having triangles as their
bases be each divided into two pyramids and two prisms as in the preceding proposition, then the ratio of their bases shall be equal to that of the prisms.\\

\noindent Prop. IX.

A cone} ( शङ्कु ) {\en is a third part of a cylinder} (समतलमस्तकपरिधि )
{\en having the selfsame base} (तल) {\en and altitude} (मस्तकपरिधि ) {\en with it.\\

\noindent Prop. XIV.

Two concentric spheres being given, it is required to inscribe in the greater sphere a solid figure of many sides (i.e. a polyhedron ), the superficies of which shall not touch the less sphere and if a similar polyhedron be inscribed in another sphere, these two polyhedrons shall
be in treble ratio of that in which the diameters of the spheres are.
\begin{center}
BOOK XIII.
\end{center}
\noindent Prop. II.

No enunciation is given for this Prop. and it simply seems to
be an alternative proof of the 1st Prop.\\

\noindent Prop. IV.

For this also no enunciation is given and the Prop. seems to
be an alternative proof of Prop. III.}

\begin{center}
\rule{0.5in}{0.3pt}
\end{center}

\newpage
\begin{center}
ERRATA.
\end{center}
\vspace{3mm}

\begin{center}
\begin{tabular}{lll}
Line. & Incorrect. & Correct.\\ \\

21  & एत & एतत्\\ \\

8 &  कृतवान्  & कृतवत्\\ \\

21 & ०मन्यांकं & ०मन्याङ्क०\\ \\

19 & द्वाविंशतितमं & द्वात्रिंशत्तमं \\ \\
\end{tabular}
\end{center}

\begin{center}
\rule{0.5in}{0.3pt}
\end{center}

\end{document}

