\documentclass[11pt, openany]{book}
\usepackage[text={4.65in,7.45in}, centering, includefoot]{geometry}
\usepackage[table, x11names]{xcolor}
\usepackage{setspace}
\usepackage{multicol}
\usepackage{ragged2e}
\usepackage{multirow}
\usepackage{wrapfig}
\usepackage{fontspec,realscripts}
\usepackage{polyglossia}

\setdefaultlanguage{sanskrit}
\setotherlanguage{english}
\defaultfontfeatures[Scale=MatchUppercase]{Ligatures=TeX}

\newfontfamily\sanskritfont[Script=Devanagari]{Shobhika}
\newfontfamily\englishfont[Language=English, Script=Latin]{Linux Libertine O}
\newfontfamily\q[Script=Devanagari, Scale=0.95, Color=violet]{Shobhika-Regular}
\newfontfamily\ab[Script=Devanagari, Scale=1.05, Color=purple]{Shobhika-Bold}
\newfontfamily\eg[Script=Devanagari, Scale=1.05, Color=red]{Shobhika-Bold}
\newfontfamily\qt[Script=Devanagari, Scale=0.9, Color=violet]{Shobhika-Regular}
\newfontfamily\bqt[Script=Devanagari, Scale=1, Color=black]{Shobhika-Regular}
\newfontfamily\s[Script=Devanagari, Scale=0.9]{Shobhika-Regular}
\newcommand{\devanagarinumeral}[1]{%
 \devanagaridigits{\number\csname c@#1\endcsname}}
\usepackage{fancyhdr}
\pagestyle{fancy}
\renewcommand{\headrulewidth}{0pt}
\usepackage{enumerate}
%\pagestyle{plain}
\usepackage{afterpage}
\usepackage{amsmath}
\usepackage{amssymb}
\usepackage{graphicx}
\usepackage{longtable}
\usepackage{footnote}
\usepackage{dblfnote}
\usepackage{xspace}
%\newcommand\nd{\textsuperscript{nd}\xspace}
\usepackage{array}
\usepackage{emptypage}
\usepackage{hyperref}   % Package for hyperlinks
\hypersetup{
colorlinks,
citecolor=black,
filecolor=black,
linkcolor=blue,
urlcolor=black
}

\XeTeXgenerateactualtext=1 % for searchable pdf

\begin{document}
\begin{center}
श्रीगणेशाय नमः\\
\vspace{1cm}
श्रीभास्कराचार्यप्रणीतम्\\
\vspace{5mm}
\textbf{\Huge बीजगणितम् \\}
\vspace{5mm}
श्रीकृष्णदैवज्ञकृतबीजाङ्कुराख्यया\\
\vspace{0.3cm}
संस्कृतव्याख्यया संवलितम्~।\\
\vspace{1cm}
\textbf{\LARGE सम्पादकः\textemdash \\}
\vspace{0.2cm}
{\Large श्री विहारीलाल वासिष्ठः\\ }
\vspace{0.2cm}
व्याख्याता ज्योतिषे \\
\vspace{10mm}

\includegraphics[scale=0.8]{graphics/Capture.PNG}\\
\vspace{0.3cm}
{\Large श्रीरणवीर-केन्द्रीय-संस्कृतविद्यापीठम्}\\
\vspace{0.2cm}
\textbf{\LARGE जम्मू }
\end{center}
\thispagestyle{empty}
\newpage
%%%%%%%%%%%%%%%%%%%%%%%%
\noindent प्रकाशकः \\

{\indent\doublespacing\large प्राचार्यः \\
\vspace{-4mm}

\indent (श्रीरणवीर-केन्द्रीय-संस्कृतविद्यापीठम् जम्मू\\
\indent केन्द्रीयशिक्षामन्त्रालय-राष्ट्रियसंस्कृतसंस्थान- \\
\indent द्वारा सञ्चालितम्~। )}\\

\vspace{10mm}
सर्वाधिकारः सुरक्षितः\\

\vspace{5mm}
\textbf{\Large प्रथम संस्करणम् १९७७-७८\\}

\vspace{1.5cm}
{\large मूल्यम् :\textendash\  Rs.\,40-00. }\\

{\doublespacing\vspace{1cm}
मुद्रक :\textemdash \\
\indent सिंह प्रिंटिंग प्रैस,\\
\indent नहर रोड कृष्णा नगर, जम्मू तवी}
\thispagestyle{empty}
\newpage
%%%%%%%%%%%%%%%%%%%%%%%%%%%%%%%%%%%%%%%%
\begin{minipage}{0.3\textwidth}
\begin{flushleft}
\includegraphics[scale=0.6]{graphics/Capture.PNG}
\end{flushleft}
\end{minipage}
\begin{minipage}{0.6\textwidth}
\begin{center}
\textbf{\LARGE~॥~श्रीः~॥}\\
\vspace{5mm}
\textbf{\LARGE डा. मुरलीधरपाण्डेयः\\}
\vspace{3mm}
{\large प्राचार्यः\\}
\vspace{3mm}
श्रीरणवीर-केन्द्रीय-संस्कृतविद्यापीठम्\\
\vspace{1mm}
जम्मू~।\\
\vspace{10mm}
\end{center}
\end{minipage}

\begin{minipage}{0.15\textwidth}
\begin{flushleft}
\includegraphics{graphics/Capture1.PNG}
\end{flushleft}
\end{minipage}
\begin{minipage}{0.75\textwidth}
\s\onehalfspacing
 विश्वज्योतिषशास्त्रवित्सु श्रीभास्कराचार्यस्य स्थानं महामहनीयम् अस्ति~। एवम् एव ज्योतिष-त्रिस्कन्धेष्वन्यतमे सिद्धान्ते (गणिते) स्कन्धे
श्रीभास्कराचार्यविरचित-सिद्धान्तशिरोमणिग्रन्थो ज्योतिषग्रन्थेषु शिरोमणीयति~।
 अस्य चत्वारो भागा अपि ग्रन्थचतुष्टयनाम्ना प्रसिद्धाः १ लीलावती २ बीजगणितम् ३ गणिताध्यायः ४~गोला-ध्यायश्च~। एतेषां चतुर्णामपि
 महदुपयोगित्वमाकलय्य शिक्षाधिकारिभिः विभिन्नसंस्कृतपरीक्षासु
 पाठ्यत्वेन निर्धारिताः~। किन्तु अद्यत्वे यथापरे
प्राचीनसंस्कृतग्रन्थाः
 प्रायः आपणे लुप्तदर्शनास्तथैव भास्करीयबीजगणितं दुर्लभग्रन्थोऽपि
 दुर्लभतां गतः~। एतद्भावे बीजगणिताध्येतॄणां तत्तत्परीक्षायां
 प्रविविक्षूणां च सम्मुखे महदसौविध्यमापतितम्~। तत् परिजिहीर्षयेदं
 विद्यापीठमेतत् प्रकाशनमका-मयत~। एतदर्थं सम्पादकत्वं निर्वाहाय ज्योतिषविभागीयाः प्राध्या-पकाः प्रतिभाधनाः श्रीविहारीलालशास्त्रिणः सम्प्रार्थिताः सम्यक् सम्पादनकार्यं न्यभालयन्~।
 यथा साध्यमशुद्धीरपाकृत्य प्रकाशयन् विदुषां पुरस्तादिमं ग्रन्थं
 समुपस्थापयञ्चातीव प्रसीदामि, मुद्रणयन्त्रकृतया साधारण्या त्रुट्या
विषीदामि च~। अस्योपयोगिताविषये प्रकाशनसाफल्यं कृते च गुणैकपक्षपातिनो
 विद्वच्चरणा एव प्रमाणम्~।\\
 \vspace{3mm}
\end{minipage}

{\renewcommand{\arraystretch}{1.2}
\begin{tabular}{lp{2.5cm}c}
 २०३८ वै. सं फा. शु. नवम्यां&& विदुषां वशंवदः\textemdash  \\
\hspace{5mm} (४. ३. ८२) &&मुरलीधरपाण्डेयः\\
&& प्राचार्यः\\
\end{tabular}}

\thispagestyle{empty}
\newpage
%%%%%%%%%%%%%%%%%%%%%%%%%%%%%%%%%%%%%%%%
\begin{center}
    \textbf{\Large द्वित्राः शब्दाः}\\
    \vspace{5mm}
    
\q गं बीजन्धरमव्यक्तं चतुर्वर्गफलप्रदम्~।~~~~ \\
\vspace{1mm}
गणेशं स्वस्तिकाकारं द्वैमातुरमुपास्महे~॥
\end{center}

\s\onehalfspacing
\indent ऐदानीन्तनाः परप्रत्ययनेयमतयः सुमतयोऽपि भारतीयवैदिकगणितस्य 
 रूपान्तरीभूतपाश्चात्यगणितस्याभ्युदयं प्रशंसन्तोऽत्रत्यगणितं
धनर्णादिचिह्नहीनमुदाहरन्तो दृक्पथमवतरन्ति~। तैः सधैर्यमित्थमवधार्य यत्
स्वस्तिकचिह्नं वैदिककालादेव समग्रस्यापि षडङ्गवेदस्य निदानं वर्ततेतरां
सुतराम्~। यस्मिंश्च त्रिस्कन्धज्योतिषस्यापि प्रत्यक्षतः साक्षात्कारो
भवितुं
पार्यते किमु तत्र कैमुतिकन्यायेन धनर्णादिचिह्नानां स्थानं न सेत्स्यति?\\

\vspace{-2mm}
यतो हि स्वस्तिकं $\begin{matrix}
\vspace{-2mm}
\includegraphics[scale=0.4]{graphics/Capture33.png}
\vspace{1mm}
\end{matrix}$~गणेशस्य चिह्नं वरीवर्ति~। तस्य बीजं {\qt 'गं'}
इति
तन्त्रप्रीतिजुषामतिरोहितमेव~। तदेव चतुर्मुखीभूयस्वस्तिकविग्रहस्य
रचनामाविष्करोति~। तद्धि द्विमातृकाभिर्विरचितत्वात् द्वैमातुर इत्युच्यते
ऋतसत्यात्मिके द्वेऽपि मातृके वैदिकतान्त्रिकपौराणिकग्रन्थेषु च प्रसिद्धे स्तः~।
तिर्य्यग्गामिनी ऋतमातृकैव ऋणचिह्नेन ($-$) लक्ष्यते~। ऊर्ध्वगामिनी
सत्यमातृका च ॠणचिह्नीभूत ऋतमध्यस्था धनचिह्नं ($+$) प्रति-मुञ्चति~।
ॠतसत्यात्मिके द्वेऽपि मातृके योगपरिणमनसमकालमेव चलत्वरूपमाश्रित्य
गुणन ($\times$) चिह्नत्वमधिगच्छतः~। गुणनचिह्नरचनसमकालमेव सत्यम्
{\qt 'अम्'} इत्याकारकं बिन्दुं रूपं चाधिगत्य यस्य कस्याप्यङ्कस्य मूर्धानं
समारुह्य
विराजते~। स चैवाङ्कविशेषोऽत्र ऋणरूपेन स्वीक्रियते~। यथा १ं $= -$१~।
तदेव बिन्दुभूतम् {\qt 'अम्'} उपर्यधोगामीभूयभागात्मकं ($\div$) चिह्नं व्यञ्जयति~।
स्वतन्त्ररूपेणाङ्कानां बीजानां वा मध्यगामिबिन्दुरेव गुणचिह्नरूपेण
व्यक्तीभवति~। यथा\textemdash क.प $=$ क $\times$ प स्वस्तिविग्रहस्यैवैका परावरगा पूर्णमातृकावर्गमूलचिह्नं ( $\sqrt{}$ ) द्योतयति~। एवमूर्ध्वगामिनी सत्यमातृकापि तिर्यग्गामिनी $-$ भूय ॠतात्मक ऋणचिह्नीभूतरेखया $-$ यदा समानान्तरी भवति,
तदा समीकरणचिह्नत्वम् ($=$) उपव्रजति~। अतोऽस्माकं भारतीयगणिते
चिह्नानां ज्ञानं प्रागेवासीत्~। अनेन संख्यानार्थकगणधातुतो
निष्पन्नत्वात् गणपतिशब्दस्य गणिताधिष्ठातृदेवत्वमपि सार्थकीभवति~। 
स्वस्तिकचिह्न एव समेषां ग्रहनक्षत्रादीनां विज्ञानमपि गोप्यरूपेण सन्निविष्टमास्ते परञ्च
विस्तरभयादत्र नोपन्यस्यते~।
\thispagestyle{empty}
\newpage
%%%%%%%%%%%%%%%%%%%%%%%%%%%%%%%%%%%%%%%%
विज्जिडविडनाम्नि कस्मिंश्चित् पुण्यधाम्नि शाण्डिल्यगोत्रोत्पन्नस्य
समस्तशास्त्रेष्वप्रतिहत-गतितया पूर्णरूपेण व्युत्पन्नस्य विशेषतोऽत्र
त्रिस्कन्धज्योतिषमर्मज्ञेषु मूर्धन्यस्य लीलावती-बीजगणित-सिद्धान्तशिरोमणि-करणकुतूहलादि-दिव्यसानुबन्ध-प्रबन्ध-लेखन-सम्पन्नस्य\\
पूर्वोक्तचिह्नोपधानमन्तरेणैव
केवलमङ्कोपरिगतेनेकेनैव बिन्दुना (.) ऋणद्योतनमात्रमधिकृत्यानर्गलगणितार्णवावगाहनिर्वाहकौशलतोशलमवलोक्यापि किमाधुनिकान्भावुकान्
महत्तास्पदपदभाक्त्वं न प्रतिभाति\,?\\

\vspace{-4mm}
एतादृशस्य \;ग्रहगणितेऽक्षुण्णतया \;ज्ञानवृद्धस्य \;समस्तज्योतिषजगति \;प्रसिद्धस्य
विद्वद्वरभास्कराचार्यवर्यस्य ~निजप्रतिभप्रसूतं ~{\qt 'बीजगणितं'} ~नामानुपमग्रन्थरत्नस्वरूपं बहुशो मुद्रितम् अप्यतिचिराल्लोके दुर्लभतममनुभूय
श्रीरणवीरकेन्द्रीयसंस्कृतविद्यापीठस्य प्राचार्यपदमलङ्कुर्वाणाः सकलशास्त्रधौरन्धुरेयाः
डा.
मुरलीधरपाण्डेयमहाभागधेया ज्योतिषविभागीयछात्राणां हितभावनया
पाठ्यक्रमनिर्धारितप्रकृतग्रन्थाभावपूर्तिकामनया श्रीकृष्णदैवज्ञकृतया
बीजाङ्कुराख्ययातिविस्तृतव्याख्यया समेतं {\qt 'भास्करीयं बीजगणितम्'} इति नामोपेतं
सद्ग्रन्थं प्रकाशयितुं प्रतिज्ञातवन्तः स्वयमेव
प्रकाशनभारमुद्वहन्तश्चात्रत्यविद्यापीठीयाध्यापकपदे कार्यं कुर्वाणं व्यावहारिकज्योतिषे लब्धबहुमानं रैणोपाभिधानं
ज्योतिषा-	चार्यश्रीचन्द्रमौलिशर्माणं यथास्वमुपसम्पादकरूपं स्वीकुर्वाणं
प्रारूपशोधनायादिष्टवन्तः~।\\

\vspace{-4mm}
दत्तावधानोऽप्येष प्रत्यहं मुद्रणालयं गामं गामं यातायातमिषेण स्वीयजैवगतमुद्राव्ययं कारं ~कारमकाण्डकूष्माण्डभाण्डीभूतकालक्षेपणाभ्यसनव्यसनिनो ~मुद्रणभवनव्यवस्थातुः प्रवञ्चनात्मकव्यवहृतिपाटवकाटवमनुभूय
त्रिंशत्प्रायपृष्ठमुद्रणान्त एव सोत्साहोऽपि निरुत्साहतामभजत्~।\\

\vspace{-4mm}
ततो धैर्यधुरीधुरीणैः सकलकार्यनिर्वहणप्रवीणैः
प्राचार्यपदासीनैरत्यत्रविद्यापीठीयज्योतिषाध्यापकविशिष्टः
सिद्धान्तफलितोभयविधज्योतिषाचार्ययोर्लब्धप्रतिष्ठः
शिक्षाशास्त्रिकक्षावधिकशैक्षणिकविधिनानुशिष्टः
विप्रान्वयेबडूपाभिधयाश्लिष्टः श्रीसुदेशकुमारशर्माभिख्यः पूर्वोक्तावशिष्टग्रन्थविशिष्टस्य प्रचिकाशयिषया समादिष्टः~।
एषोऽमुष्य धार्ष्ट्यवैशिष्टेऽप्यतिशिष्टतया
चतुश्चत्वारिंशदधिकशतावधिकपृष्ठान्तमादिष्टकार्यं
\thispagestyle{empty}
\newpage
%%%%%%%%%%%%%%%%%%%%%%%%%%%%%%%%%%%%%%%%
\noindent सयत्नमत्यवाहयत्~। तस्मिन्नप्युपरते सति मुद्रणकार्यं
हनुमत्पुच्छायमानतामापन्नमेतदनुभूय भूयो धैर्यम् अवधार्य साश्चर्याः प्राचार्यवर्याः सुचिरं रुचिरं
विचार्य भूतं प्रेतेन
योजयेदिति शब्दकदम्बकमवलम्बनं सार्थकीकुर्वन्तोऽङ्गीकृतं सुकृतिनः
परिपालयन्तीति
वाक्यामृतं निर्वहन्तश्च समस्तछात्रवसंसदि सामयिकनयनैपुण्यवेत्तारं
विद्यापीठीयाद्यावधिकप्रतिवर्षमभिनीयमानेष्वभिनेयेषु वेणीसंहारादिरूपकेषु
नायकादिभूमिकायामभिनवाभिनयकौशलेन, विभिन्नक्रीडासु स्फूर्तिप्रदर्शनेन, भाषणप्रतियोगितादिषु
चाबाल्यछात्रोचितपाठरटनाद्यवलम्बनसम्बलेन यत्र तत्र सर्वत्र
प्रथमपुरस्कारविजेतारं
शैक्षणिकपद्धत्या शिक्षाशास्त्रिकक्षीयपरीक्षापारावारमुल्लंघयितारं
साम्प्रतं फलितज्योतिषाचार्यस्य
द्वैतीयीकखण्डेऽध्येतारमकाण्डकूष्माण्डभाण्डायमानानामधमण्डपभेत्तारं
सङ्गोत्तरोपाह्वं श्रीरामदासशर्माणमुपसम्पादकरूपेण
प्रारूपशोधनार्थम् उपनियुज्य सायासं
भगीरथप्रयासं पूर्णतया सफलीकृतवन्तः~।\\

\vspace{-4mm}
एतत्सकलं येषां परोपकारैककर्त्तव्यतापरायणानां राष्ट्रियसंस्कृतसंस्थानशासिपरिषच्छिक्षासमाजकल्याणमन्त्रालयाधिकारिणां च सहानुभूतिततिभिः
सम्पादनकार्यं सुसम्पन्नम्; तेषां समेषामपि धन्यवादपुरस्सरमानृण्यं
सदा सर्वदा सर्वथा साभारं
बहुवारं सप्रमोदञ्चावनतमूर्ध्ना संवहामि~।\\

\vspace{-4mm}
निजाश्रितजनानल्पमनोरथोपकल्पककल्पपादपकल्पतां प्रतिमुञ्चमानानाम्,
अनन्तशास्त्राध्ययनाध्यापनाभ्यसनादिना निर्मलीकृतान्तःकरणानां
श्रीमदुदयददयदलदलनजनि-तापरिमित-विततदिगन्त-विश्रान्तकान्त-कीर्तिचमत्कृतिततिभिः ~तडिल्लतायमानविग्रहानां
राष्ट्रियसंस्कृतसंस्थानस्थप्रमुखनिदेशकपदमलङ्कुर्वाणानामनन्तश्रीभाजां
{\qt डा. मण्डनमिश्र}-महाभागानामनन्तप्रणतिततिपुरस्सरं नैकशो धन्यवादान्निवेदयामि~।\\

\vspace{-4mm}
एवं ~श्रीरणवीरकेन्द्रीयसंस्कृतविद्यापीठीयप्राचार्यपदाधिष्ठितानां ~पूर्वजन्मोपार्जितम-हिम्नां जगत्प्रतिष्ठितानां सच्छ्रीमतां डा. मुरलीधरपाण्डेयमहाभागधेयानां साधिकारिणां सकर्मचारिणां सान्तेवासिनामध्ययनाध्यापनपाटवमाधतां नैकविधविद्यावतां सहचारिणां सहायकसम्पादकानां
सिंहमुद्रणालयाधिकारिणां
च सधन्यवादं कृतज्ञतोपहरन्नेव विरमामि~।
\thispagestyle{empty}
\newpage
%%%%%%%%%%%%%%%%%%%%%%%%%%%%%%%%%%%%%%%%
आशासे यदत्राक्षरच्युतिप्रच्युतिसूक्ष्मस्थूलवर्णानौचित्यप्रभृतिषु नैकशो
दोषेषु सत्स्वपि {\qt 'शूर्पवद्दोषमुत्सृज्य गुणं गृह्णन्ति सज्जनाः'} इत्युक्तिं चरितार्थमद्भिः सद्भिरदोषैर्दोषज्ञैरकम्पयानुकम्पयाक्षरयोजकशोधकानामसान्निध्यप्रातिनिध्यतया क्षन्तव्योऽस्मीति समभ्यर्थनं पुरस्सरं निवेदयति\textemdash \\

\vspace{3mm}
{\bqt\renewcommand{\arraystretch}{1.1}
\begin{flushleft}
\begin{tabular}{lp{2cm}r}
आर्षविहारीविहारः&& श्रीजुषां विदुषां वशंवदः\textemdash\\
श्री रघुनाथ पुरी जम्मू&& विहारीलालशर्मा वासिष्ठः\\
सं २०३८ वै० महाशिवरात्र्याम्&&\\
\end{tabular}
\end{flushleft}}

\thispagestyle{empty}
\newpage
%%%%%%%%%%%%%%%%%%%%%%%%%%%%%%%%%%%%%%%%
\pagestyle{empty}
\begin{center}
    \textbf{\LARGE अशुद्धिपत्रम्}
\end{center}
{\onehalfspacing\s
\begin{longtable}{llrr}
\textbf{\large अशुद्धम्}& \textbf{\large शुद्धम्}&\textbf{\large पृष्ठम्}&\textbf{\large पङ्क्तिः}\\
&&&\\
\endhead
मगङलाचरणम् &मङ्गलाचरणम् &१& २\\
विजनृनगर &विजडविड्-नगर &२ &१०\\
साख्याः& सांख्याः &२& १७\\
ज्ञानरुपाया &ज्ञानरूपाया& ४& १८\\
तावत्सावत्क्रया &तावत्सम्यक्रया& ५ &१४\\
मेवायमारमारम्भ &मेवारम्भो &५& १५\\
प्रश्नोत्तरार्थज्ञानं &प्रश्नोतरार्थज्ञानं गोलज्ञानं च& ५& २६\\
सप्रमाष्टमस्थाने &तृतीयचतुर्थयष्टम &५& २६\\
रुपाणां& रूपाणां &९& १\\
क्षराणयुप& क्षराण्युप &९& १\\
वगत्व& वर्गत्व &२६& १०\\
मूलयोस्त्तसंभवात्& मूलोयस्त्संभवात्& १८& ९\\
लिपिष्वापि& लिपिष्वपि& १९& २\\
मानन्त्तज्ञ्ज्ञापका& मानन्त्याज्ञ्ज्ञापका &१२& ४\\
बुद्देरपि &बुद्धेरपि &१९& ५\\
दकशतकयो &दशकशतकयो &१९& १९\\
योजकुतलो &योजकतुल्ये &२१& २५\\
वियोजयोद्वविधा &वियोगयोर्द्वैविध्य &२२& २१\\
निपेक्षत्वाद् &निरपेक्षत्वाद् &२३& १७\\
हृियते &ह्रियते &२४& १०\\
छायाभेदः& छायाभेदः &२६& ४\\
मतिमद्भि &मतिमद्भिः &२६& ६\\
विवृकल्पताता& विवृतिकल्पलता& २६& १०\\
शड्विद्यस्य& षड्विद्यस्य &२६& ९\\
जायियो &जातिर्ययोः &२७& १७\\
रूपप्वम् &रूपत्वम् &२४& ४\\
धनवव्क्त& धनमव्यक्त& २९& ५\\

\newpage
%%%%%%%%%%%%%%%%%%%%%%%%%%%%%%%%%%%%%%%%

रूपमिदं &रूपमिदं &२९& ९\\
अस्माद्वृपा &अस्माद्रूपा& २९& ११\\
क्षयाव्क्त& क्षयाव्यक्त &२९& १५\\
योगोऽन्तरऽन्तरेषु &योगोऽन्तरेषु &३१& २५\\
विशेषण्यसक्रामत्तो &विशेषणयुसंक्रामतो &३२& १४\\
चतुर्थोताहरणे &चतुर्थोदाहरणे &३३& २०\\
चतुर्थैव &चतुर्धैव &३३& २६\\
आस्ति &अस्ति &३३ &२७\\
पञ्चाशदाढका& पंचाशदाढका& ३४ &१\\
त्रिगुतिं &त्रिगुणितं &३४& ५\\
त्र्यशा &त्र्यंशा &३४& ७\\
स्याद्रूपा &स्याद्रूप &३४& ८\\
चतुर्थाशातत्र्य& चतुर्थाशातय &३४& ११\\
सकरार्थं &संकरार्थं &३४& २३\\
वर्गात्मकवं& वर्गात्मकत्वम् &३४& ३०\\
जत्यो &जात्यो &३५& १\\
भर्वत &भवति &३५& १\\
करणीतुल्येन्येव &करणीतुल्यान्येव &५७& १२\\
तुल्यन्येव& तुल्यान्येव &५७& १२\\
इतरकरीखण्डानि& इतरकरणीखण्डानि &५८& १\\
क १६ं& क १५ं६ &५८& ५\\
क ४८ &क ४९& ५८& ५\\
देन २& पदेन २& ५८& १४\\
स्योतसर्गत &स्योत्सर्गात्& ५८& १९\\
महतो& महती &५८& २१\\
सधितकरणी &साधितकरणी &५८& २४\\
घनत्व& घनत्वं &६०& १३\\
६०& ६०$^{\text{०}}$& ६१& ८\\


\newpage
%%%%%%%%%%%%%%%%%%%%%%%%%%%%%%%%%%%%%%%%
करणीद्वयं क ४० &करणीद्वयं २४ क ४०& ६१& ९\\
एतत्कृतेः २१& एतत्कृते २५& ६१& १७\\
\hspace{2mm} "  & \hspace{2mm} " &६१& १९\\
द्रष्टयम् &द्रष्टव्यम् &६१& २२\\
वर्गकृत्वा &वर्गंकृत्वा &६२& २\\
एर्षायोगः &एषांयोगः &६२& २१\\
चतुर्थांशो &चतुर्थांशो $\dfrac{\text{१}}{\text{४}}$ &६२& २९\\
चतुःषष्यंशः &चतुःषष्ठ्यंशः $\dfrac{\text{९}}{\text{६४}}$ &६२& २९\\
रणीनवकस्य& करणीनवकस्य &६३& १\\
चतुर्गणया &चतुर्गुणया &६४& १५\\
स्थाप्योऽन्पयवर्ग& स्थापयोऽन्त्यवर्गः &६४& २४\\
ष्डानि &खण्डानि &६५& १३\\
बग &वर्ग &६६& ८\\
क १० &क १००& ६८ &९\\
कर्मविंशातिं &कर्मविशतिं& ७०& ८\\
बीजतुष्टम &बीजचतुष्टय &७१& ४\\
तुल्येनाङेन &तुल्येनाङ्गेन& ७१& १५\\
द्विच्यादिकानां& द्वित्र्यादिकानां& ३५& १३\\
तुन्टणां &तुतौनृणाम् &३५& २३\\
मानालिङ्ग &मानलिङ्ग& ३५ &२५\\
वर्गणामेव &वर्गाणामेव &३६& १\\
गुन्य& गुण्य &३६& १९\\
तेषं& तेषां &३७& १९\\
या ८& या ४८& ३८& ३\\
इत्थृणोयो &इत्थृणयो &३८& ४\\

\newpage
%%%%%%%%%%%%%%%%%%%%%%%%%%%%%%%%%%%%%
समद्विधिाता& समद्विधाता& ३८& १३\\
हतिश्व &हतिश्च &३८& १४\\
वर्गश्च &वर्गश्च &३८& १८\\
गुनणज &गुणनज& ३९& १६\\
ऋणयावत् &ॠणंयावत् &४०& २१\\
नह्यास्ति &नह्यस्ति& ४३& ९\\
गणितेऽन्ततस्दा& गणितेऽन्ततस्तदा &४३ &११\\
॥ ३४~॥~&॥ ३५~॥& ४५& ७\\
कवचित् &क्वचित् &४६ &२२\\
स्त्रतत्रैवेति& स्तत्रैनेति& ४७& ३\\
षीड्विधं &षड्विधम् &४८& ४\\
श्रयरूपाणां& श्रयंरूपाणां& ४८& १०\\
भजन &भजनं &४३& १७\\
ऐकैव &एकैव &४३& २२\\
बदृनाम् &बहूनां &५५& १४\\
पृथक्तदर्घे &पृथक्तदर्धे &५५& १६\\
न्यसो &न्यासो &५६& १९\\
रूपाणि १& रूपाणि १०& ५६& २३\\
महत्ती ८ &महती ७& ५६& २५\\
मूलकरण्य &मूलकरण्यः& ५६ &२७\\
पूर्वखण्डत्र्यं &पूर्वखण्डत्र्यमासीदिति &&\\
खण्डत्र्ययं कृतया &खण्डत्र्यंकृतम् &५७& ४\\
शेषं &शेषं ३& १०५& २५\\
रस्यु &स्युः &१०६& ९\\
नुपायः &मुपायः &१०६& १०\\
शेषस्यात् &शेषंस्यात् &१०६& ११\\
द्वादशनं &द्वादशकेन &१०६& २६\\
विपयतौ& विषयता &१०६& २७\\

\newpage
%%%%%%%%%%%%%%%%%%%%%%%%%%%%%%%%%%%%%%%%
भज्यं& भाज्यं &१०८& १०\\
सप्रवशेषो &सप्रावशेषो &१०८& २५\\
दैवज्ञवर &दैवज्ञवर्य&&\\
मूलतत्र्य &मूलंतत्र्य &११०& ४\\
बहनि &बहूनि &१११& २\\
तप्यामेव& तप्यमिव& १११& ९\\
इप्यादि &इत्यादि &११५& २२\\
क्षोतक्षण &क्षेपतक्षण& १२०& १६\\
हरर्स्पणत्वालब्धे& हरर्स्पणत्वाल्लब्धे &१२०& १८\\
जातं कनिष्ठं &जातं धनं कनिष्ठं &१२०& २५\\
मदुक्तयकारेण &मुदुक्तप्रकारेण &१२०& २६\\
दूयं& राशिद्वयं &१२१& १६\\
ज्ये $\dfrac{\mbox{३१}}{\mbox{२}}$ क्षे १$^\text{०}$ &ज्ये $\dfrac{\mbox{३९}}{\mbox{२}}$क्षे १$^\text{०}$& १२२& १८\\
$\dfrac{\mbox{१५ : ३}}{\mbox{२}}$& $\dfrac{\mbox{१५२३}}{\mbox{२}}$& १२२& २१\\
ज्ये १७६६३& ९०४९\textendash\ ३७६६३१९०४९& १२३& २४\\
ज्येष्ठेष्टयोयर्तिः &ज्येष्ठेष्टयोर्युतिः &१२८& ९\\
योगे &योगे &१२८& १४\\
बृहदाशिरूदृिज्यष्ठमृणक्षेपे &बृहद्राशिरूदिष्ठमृष्ठेमृणक्षेपे &१२८& १९\\
ॠणनेपे &ॠणक्षेपे& १२९& ८\\
१& २& १२९& १४\\
पदनन्त्यं &पदानन्त्यं &१३०& २\\
प्राग्व्त् &प्राग्वत् &१३०& २\\
मध्यमहरणविशेषे &मध्यमाहरणविशेषे& १३१& ११\\
यल्लभ्ते\textendash\  &यल्लभ्यते &१३३ &१७\\
मेवावति &मेवावन्ति &१३४& ८\\
रु०० &रूप ३००& १३५& १०\\
कल्पयेदात्मद्वया &कल्पतेदात्कबुद्धया& १३७& ६\\

\newpage
%%%%%%%%%%%%%%%%%%%%%%%%%%%%%%%%%%%%%%%%%%%
या १ रू १०& या १ रु ११०& १३७& १८\\
ॠयंशः &त्र्यंशः & १३८& १७\\
दशहत्वाचस्वगृहं &दज्ञदत्वाद्विगुणीकृत्यदशमुक्तवा&&\\
&दशहत्वाचस्वगृहं &१४३& २०\\
या $\dfrac{\mbox{१}^{\text{०}}}{\mbox{१०}}$& या $\dfrac{\mbox{९}}{\mbox{१०}}$ &१४४& १८\\
कः& क ५$^{\text{०}}$& १४५& ८\\
तुयाकारनपगमे &याकारनपगमे &१४९& २७\\
राशी १००० &राशी १००००& १५४& १०\\
जातवं& जातैवं &१५५& १८\\
रूपतोल्पं &रूपतोऽल्पं &१५९& ८\\
यात्रालोपे &यात्रालोप &१९०& १४\\
तीत्खलमेवेर्त्थात् &तत्खिलमेवेत्यर्थात् &१६२& ४\\
व्यक्तवर्गांवक्त &व्यक्तवर्गाव्यक्तं& १६२& ५\\
गौरवत् &गौरवात् &१६२ &१७\\
व्क्तवर्गेषु &व्यक्तवर्गेषु &१६२& २२\\
भजकः &भाजकः &१६३& १४\\
पधमध्यें &पद्ममध्यें &१६४& ५\\
द्विगुण& द्विगुणं &२६४& ७\\
समच्छेदकृित्य &समच्छेदिकृत्य &२६४& १९\\
मानस्ह &द्विधामानस्य &१६६& १४\\
बीजक्ष &बीजज्ञ &२६७& ४\\
क्षेत्रदर्णन &क्षेत्रदर्शनं &१६९& ७\\
क्षत्रं &क्षेत्रं &१६९& १०\\
स्याद्वर्णास्यैकस्योन्मित्नां &स्याद्वर्णस्यैकस्योन्मितीनां &१९०& ४\\
१& १४& १९२& २७\\
मवति& भवति& १८३& २३\\
वण &वर्णं &१८४& १४\\
वीजक्ष &बीजज्ञ& १८७& २७\\

\newpage
%%%%%%%%%%%%%%%%%%%%%%%%%%%%%%%%%%%%%%
द्वगुणस्ततोअन्यः &द्विगुणस्ततोऽन्यः &१८८& २\\
षष्काष्टकाभ्यां& षटकाष्टकाभ्यां &१९१& १६\\
वेदिञ्जरघटा &वेदिकुञ्जरघटा &१९१& १७\\
संघट्टा &संथट्ट &१९१& १७\\
स्यादित्यदउक्तं& स्यादित्यतउक्तं &१९१& २१\\
तथैतादृशो &तथैतादृशौ &१९१& २१\\
फलक्याष्यं &फलैक्याष्यं &१९२& ३\\
शषैक्ययो &शेशषैक्ययो &१९२& २३
\end{longtable}}
\vspace{1cm}

\begin{center}
{\LARGE \textbf{अनुक्रमणिका}}\\
    \rule{0.2\linewidth}{0.9pt}
\end{center}

\begin{table}[h!]
    \centering
    \begin{tabular}{llr}
१ & \hyperref[ch1]{धनर्णषड्विधम्} & १ \\
२ & \hyperref[ch2]{खषड्विधम्} & १८ \\
३ & \hyperref[ch3]{वर्णषड्विधम्} & २७ \\
४ & \hyperref[ch4]{करणीषड्विधम्} & ४३ \\
५ & \hyperref[ch5]{कुट्टकविवरणम्} & ७० \\
६ & \hyperref[ch6]{वर्गप्रकृतिः} & ११० \\
७ & \hyperref[ch7]{एकवर्णसमीकरणम्} & १३१ \\
८ & \hyperref[ch8]{मध्यमाहरणम्} & १५९ \\
९ & \hyperref[ch9]{अनेकवर्णसमीकरणम्} & १७९ \\
१० & \hyperref[ch10]{अनेकवर्णसमीकरणान्तर्गतं मध्यमाहरणम्} ~~~~~~ & २०३ \\
११ & \hyperref[ch11]{भावितम्} & २२५ \\
 & \hyperref[ch12]{ग्रन्थसमाप्तिः} & २३५ \\
 & \hyperref[ch13]{हस्तलिखितप्रतीनां समाप्तिः} & २३८ \\
    \end{tabular}
\end{table}


\newpage
%%%%%%%%%%%%%%%%%%%%%%%%%%%%%%%%%%%%%%%%
\phantomsection \label{ch1}
\begin{center}
    \textbf{\Huge बीजगणितम्~।}\\
\vspace{0.8cm}
{\LARGE \textbf{१ धनर्णषड्विधम्~।}}\\
\vspace{0.5cm}
{\large ( टीकाकारकृतमङ्गलाचरणम् )}
\end{center}
\begin{quote}
    \q
शिवयोर्भजनातिगौरवाद्यत्सुतलीलाधृतकुंजरास्यरूपम्~।\\
 अपहन्तु ममान्तरं तमस्तत्सततानन्दमयं महो महीयः~॥~१~॥\\
 
 \vspace{-5mm}
 यदीयचरणाम्भोजस्मर्तुः सकलसिद्धयः~। \\
 भवन्ति वशवर्तिन्यः सिद्धेशीं तामहं भजे~॥~२~॥~\\
 
 \vspace{-5mm}
 मिहिरमिव वराहमिहिरं वन्दे संदेहभेदिनं जगताम्~।\\
 ज्योतिश्चक्रविभावनहेतुं जगदेकचक्षुरक्षुद्रम्~॥~३~॥~\\
 
 \vspace{-5mm}
 कविबुधजनमूर्धनि स्फुरन्तं कविबुधसंततसेवनीयपार्श्वम्~। \\
 गणितनिपुणतां प्रवर्तयन्तं प्रणमत भास्करमीप्सितार्थसिद्ध्यै~॥~४~॥\\

\vspace{-5mm}
 कदापि नैव सम्भ्रमः स्थितश्च भौममण्डले~। \\
 अपूर्वमार्गमाश्रयञ्जयत्यपूर्वभास्करः~॥~५~॥\\
\vspace{-2mm}

आसीदसीमगुणरत्ननिधानकुम्भः कुम्भोद्भवा भरणदिग्ललनाललामः~।\\
आशैशवार्जितविशेषकलानुवर्ती श्रीकेशवः सुगणितागमचक्रवर्ती~॥~६~॥\\

\vspace{-5mm}
तस्मादभूद्भुवनभूषणभूतमूर्तिः श्रीमानगण्यगुणगौरवगेयकीर्तिः~।\\
ज्योतिर्विदागमगुरुर्गुरुसंप्रदायः प्रज्ञातशास्त्रहृदयः सदयो गणेशः~॥~७~॥\\

\vspace{-5mm}
भ्रातुः सुतस्तस्य यथार्थनामा नृसिंह इत्यद्भुतरूपशोभः~।
 \end{quote}
\afterpage{\fancyhead[CE] {बीजगणिते~।}}
\afterpage{\fancyhead[CO]{धनर्णषड्विधम्~।}}
\afterpage{\fancyhead[LE,RO]{\thepage}}
\cfoot{}
\newpage
%%%%%%%%%%%%%%%%%%%%%%%%%%%%%%%%%%%%%%%%
\renewcommand{\thepage}{\devanagarinumeral{page}}
\setcounter{page}{2}
\pagestyle{fancy}
\begin{quote}
    \q
अवर्धयद्यो जगतामभीष्टं प्रह्लादमाश्चर्यकरः सुराणाम्~॥~८~॥\\
\vspace{-3mm}

तच्छिष्यो विष्णुनामा स जयति जगतीजागरूकः प्रदिष्टः \\
शिष्टानामग्रगण्यः सुभणितगणिताम्नायविद्याशरण्यः~। \\
यद्वक्त्रोन्मुक्तमुक्ताफलविमलवचोवीचिमालागलन्तः \\
चित्राः सिद्धान्तलेशा जगति विदधतेऽज्ञोऽपि सर्वज्ञगर्वम्~॥~९~॥\\
\vspace{-3mm}

तस्मादधीत्य विधिवत् त्रिस्कन्धं ज्योतिषं गुरोः~। \\
कृष्णो दैवविदां श्रेष्ठस्तनुते बीजपल्लवम्~॥~१०~॥\\
अव्यक्तत्वादिदं बीजमित्युक्तं शास्त्रकर्तृभिः~। \\
तद्व्यक्तीकरणं शक्यं न विना गुर्वनुग्रहम्~॥~११~॥
\end{quote}
\onehalfspacing
\s\noindent अथ शाण्डिल्यगोत्रमुनिवरवंशावतंसबिजन्नगरनिवासिकुम्भोद्भवभूषणदिग्भूषणसकलाग-माचार्यवर्य-श्रीमहेश्वरोपाध्यायतनय-निखिलविद्यावाचस्पति-गणितविद्याचतुरानन-धरणि-\\
तरणिः श्रीभास्कराचार्यः खगगणितरूपसिद्धान्तशिरोमणिं
चिकीर्षुस्तदुपयोगितया तदध्यायभूतं व्यक्तगणितमुक्त्वा तथाभूतमव्यक्तगणितमारभमाणः 
प्रत्यूहव्यूहनिरासाय शिष्टा-चारपरिपालनार्थं मङ्गलमाचरञ्शिष्यशिक्षार्थं
तदुपजातिकया निबध्नाति\textemdash  \\

\begin{center}
   \bqt \textbf{मङ्गलाचरणम्} 
\end{center}
\vspace{-2mm} 
 
\phantomsection \label{1}
\begin{quote}
    \ab
    उत्पादकं यत्प्रवदन्ति बुद्धेरधिष्ठितं सत्पुरुषेण साङ्ख्याः~।\\
 व्यक्तस्य कृत्स्नस्य तदेकबीजमव्यक्तमीशं गणितं च वन्दे~॥~१~॥
\end{quote}
 
 अत्रायमन्वयः~। तदव्यक्तमीशं गणितं च वन्दे~। ईशपक्षे यत्तदोर्लिङ्ग[वि]परिणामेन यदिति स्थाने यं तदिति स्थाने तं चेति बोद्धव्यम्~। अव्यक्तं
प्रधानं साङ्ख्यशास्त्रे जगत्कारणतया प्रसिद्धम्~। ईशं सच्चिदानन्दरूपं
वेदान्तवेद्यं गणितमत्राव्यक्तमेव~। अव्यक्तपदस्यावृत्त्याव्यक्तं
गणितमिति
\newpage
%%%%%%%%%%%%%%%%%%%%%%%%%%%%%%%%%%%%%%%%%%%%
 
\noindent तद्विशेषणस्य विवक्षितत्वात्~। तन्नमस्कारेण च तदधिष्ठात्री देवता
नमस्कृता भवति~। शालग्रामशिलादौ तथा दृष्टत्वात्~। तत्र प्रधानपक्षे किं तदव्यक्तम्~।
साङ्ख्या यद्बुद्धेरुत्पादकं प्रवदन्ति~। बुद्धेस्तत्त्वविशेषस्य महदाख्यस्य~।
उत्पत्तिरत्राभिव्यक्तिः यतस्ते सत्कार्यवादिनः~। ननु प्रधानमचेतनं कथं कार्यमुत्पादयेदित्यत उक्त\hyperref[1]{\textbf{पुरुषेणाधिष्ठितं सत्}} इति~। यथा हि कुलालादिना चेतनेनाधिष्ठितं कपालादि घटाद्युत्पादकं तद्वत् इत्यर्थः~। अत्र साङ्ख्याः सेश्वराः श्रीमद्भगवत्पतञ्जलिमतानुसारिणो ज्ञेयाः निरीश्वराः कपिलमतानुसारिणः पुरुषनिरपेक्षमेव प्रधानमुत्पादकं प्रवदन्ति
तदुक्तमीश्वरकृष्णेन सप्तत्याम्\textemdash
\begin{quote}
    \q
     वत्सविवृद्धिनिमित्तं क्षीरस्य यथा प्रवृत्तिरज्ञस्य~। \\

\vspace{-6.5mm}
 पुरुषविमोक्षनिमित्तं तथा प्रवृत्तिः प्रधानस्य~॥~इति~॥
\end{quote}

\indent ननु तादृशे प्रधाने किं प्रमाणम् इत्यत आह\textendash\, \hyperref[1]{\textbf{कृत्स्नस्य व्यक्तस्यैकबीजम्}} इति~। समस्तस्य व्यक्तस्य कार्यजातस्यैकं बीजम् उपादानम्~। तथा च
वियदादिकार्यजातं सोपा-दानकं कार्यत्वात्~। घटवदित्यनुमानम्~। लाघवसहकृतं
तत्र प्रमाणमिति भावः~। न च ईश्वरेणार्थान्तरता~तस्य
निर्विकारस्यापरिणामितयानुपादानत्वात्~। परिणामित्वेऽपि कथम् अचेतनं चेतनपरिणामं स्यात् इति~। एकमिति पुरुषव्यवच्छेदः, तन्मते पुरुषस्यानुपादानत्वात्~। यतस्ते वदन्ति
पुरुषस्तु पुष्करपलाशवन्निर्लेप इति~। यथा वेदान्तिमते माया-ब्रह्मणी द्वे अपि प्रपञ्चस्योपादाने तद्वदित्यर्थः~। अथेशपक्षे\textendash \,\hyperref[1]{\textbf{साङ्ख्याः}}
सम्यक्ख्यायते ज्ञायत आत्मा यया सा सङ्ख्यात्माकारान्तः करणवृत्तिः सा येषां ते साङ्ख्या आत्मज्ञानिनः~। \hyperref[1]{\textbf{सत्पुरुषेण}} विवेकादिसाधनचतुष्टयसम्पत्तिमता~।
अधिष्ठितमादरनैरन्तर्याभ्यां श्रवणादिविषयीकृतं सन्तं बुद्धेस्तत्त्वज्ञानस्योत्पादकं प्रवदन्ति~। 
ननु तस्याजनकत्वाद्बुद्धिजनकत्वे मानाभाव इत्यत आह\textemdash   {\qt समस्तस्य 
व्यक्तस्य कार्यजातस्यैकम् असाधारणं बीजमुपादानम् इत्यर्थः}~।
{\qt "यतो वा इमानि भूतानि जायन्ते"} इति~।
{\qt "तत्सृष्ट्वा तदेवानुप्राविशत्"} {\qt "इति तस्माद्वा एतस्मादात्मन आकाशः सम्भूतः"}~।
इत्यादिश्रुतयस्तदुपादानत्वे प्रमाणमिति भावः~। 
ननु निर्विकारस्योपादानत्वे
\newpage
%%%%%%%%%%%%%%%%%%%%%%%%%%%%%%%%%%%%%%%%%%%%%%%%%%%%%%%%
\noindent परिणामितया कथमुपादानत्वमिति चेत्~। सत्यम्~। उपादानं द्विविधम्~।
परिणममानं विवर्तमानं चेति~। तत्र परिणामि विक्रियावत्~। 
यथा मृदादि घटादेः~। विक्रियशून्यं विव-र्तमानम्~। यथा शुक्त्यादि रजतादेः~। 
तत्र यद्यपि निर्विकारस्येशस्य परिणाम्युपादानता नोपपद्यते तथापि
विवर्तमानोपादानत्वे काप्यनुपपत्तिरस्तीत्यलं पल्लवितेन~। 
मायाया उपादानत्वपक्षेऽपि विवर्तमानोपादानत्वस्यात्र विवक्षितत्वादेकमित्युक्तम्~। 
अथ गणितपक्षे~।
साङ्ख्याः सङ्ख्याविदो गणकाः सत्पुरुषेण स्वरूपयोग्येनाधिष्ठितमभ्यस्तं
यद्बुद्धेः शिरोमणिवक्ष्यमाणप्रश्नोत्तरार्थादिज्ञानस्योत्पादकं प्रवदन्ति~। 
ननु प्रश्नोत्तरार्थादिज्ञानस्योत्पादकं व्यक्तमेवास्ति\textemdash  
\begin{quote}
    \q\onehalfspacing
    गुणघ्नमूलोनयुतस्य राशेर्दृष्टस्य युक्तस्य गुणार्धकृत्या~।\\
 मूलं गुणार्धेन युतं विहीनं वर्गीकृतं प्रष्टुरभीष्टराशिः~॥~
\end{quote}
 
\noindent इत्यादि~। कुज्योनतद्धृतिहृताकृतशक्रनिघ्नी कुज्यैव यत्फलपदं पलभा
भवेत्सेति~।
\begin{quote}
    \q\onehalfspacing
    द्युज्यापक्रमभानुदोर्गुणयुतिस्तिथ्युद्धृता द्व्याहता \\
 स्यादाद्यो युतिवर्गतो यमगुणात्सप्तामराप्त्योनिताः~। \\
 नागाद्यङ्गदिगङ्ककाः पदमतस्तेनाद्य ऊनो भवेत् \\
 व्यासार्धेष्टगुणाब्धिपावकमिते क्रान्तिज्यकातो रविः~॥

\end{quote}
 
इत्यादिवाक्यतो \;यावत्तावदादिवर्णकल्पनानिरपेक्षैः \;गुणनभजनादिमार्गैः \;क्रियमाणं गणितं व्यक्तमित्युच्यते~। तत्कथम् उच्यते प्रश्नोत्तरार्थज्ञानरूपाया
बुद्धेरुत्पादकम् अव्यक्तम् इत्यत आह \hyperref[1]{\textbf{व्यक्तस्ये}}ति~। व्यक्तस्य यावत्तावदादिवर्णकल्पनानिरपेक्षस्य
गुणघ्नमूलोनयु-तस्य राशेरित्यस्य द्युज्यापक्रमभानुदोर्गुणयुतिस्तिथ्युद्धृता
द्व्याहतेत्याद्यस्य च गणितस्यैकं बीजं मूलमिति यावत्~।
द्युज्यापक्रमेत्यादिगणितप्रकारस्य वर्णकल्पनामूलत्वादिति भावः श्रेयांसि
बहुविघ्नानीत्युक्तत्वान्नमस्कारत्रयमुचितमेव मङ्गलस्य समाप्तिजनकत्वं विघ्नध्वंसजनकत्वं वा
प्रकृतानुपयुक्तत्वाद्ग्रन्थविस्तरभयाच्चूडामण्यादौ विस्तृतत्वाच्च नेह
व्युत्पाद्यते तत्तत एव द्रष्टव्यम्~। ईशस्य समस्तकार्यजनकत्वं वदता तत्प्रणामस्य
\newpage
%%%%%%%%%%%%%%%%%%%%%%%%%%%%%%%%%%%%%%%%%%%%%%%%%%%%%%%%
\noindent ग्रन्थसमाप्तिप्रचयादिरूपं फलं कैमुतिकन्यायेनैव सूचितम्~। 
यतो यो यदिष्टमनिष्टं वा कर्तुं शक्तः स स्वप्रणतस्य तदिष्टं
स्वद्वेष्टुस्तदनिष्टञ्च विदधाति~।
ईशस्तु सर्वं कर्तुं समर्थः स्वप्रणतस्य सर्वमिष्टं विदध्यात्
ग्रन्थसमाप्तिप्रचयरूपं किमुतेति~।
अत्र साङ्ख्यवेदान्तिमतव्युत्पादनं ग्रन्थविस्तरभयान्न कृतं
तत्तत एवावगन्तव्यम्~॥~१~॥\\

\vspace{-4mm}
 इदानीं प्रेक्षावत्प्रवृत्तिहेतुविषयादिचतुष्टयं सङ्गतिं च शालिन्या
दर्शयति\textendash 

\phantomsection \label{2}
\begin{quote}
    \ab
    पूर्वं प्रोक्तं व्यक्तमव्यक्तबीजं \\
 प्रायः प्रश्ना नो विनाव्यक्तयुक्त्या~।\\
 ज्ञातुं शक्या मन्दधीभिर्नितान्तं \\
 यस्मात्तस्माद्वच्मि बीजक्रियां च~॥~२~॥~

\end{quote}
 
अस्यार्थः\textendash \,\hyperref[2]{\textbf{तस्माद्धे}}तोर्बीजस्य यावत्तावदादिवर्णकल्पनादिभिः
क्रियमाणस्य गणितस्य क्रियामितिकर्तव्यतां वच्मि~। \hyperref[2]{\textbf{यस्माद्व्यक्तं}}
वर्णकल्पनानिरपेक्षं गणितं पूर्वं प्रोक्तम्~। ततः किमित्यत आह\textemdash  \hyperref[2]{\textbf{अव्यक्तबीजम्}} इति~। अव्यक्तं बीजगणितं मूलं यस्य~। तथा च पूर्वं प्रोक्तमपि व्यक्तं तावत्सम्यक्तया न ज्ञायते यावद्बीजक्रिया नोपपाद्यते~। तत्किं व्यक्त-ज्ञानार्थम् एवायमारम्भः नेत्याह~। यस्माच्च सुधीभिरप्यव्यक्तयुक्त्या विना प्रश्नं ज्ञातुं प्रायो न शक्या
मन्दधीभिस्तु नितान्तं ज्ञातुमशक्या एवेत्यर्थः~। प्रश्नाश्चात्र सिद्धान्तशिरोमणौ
त्रिप्रश्नाधिकारे वक्ष्यमाणा भाकर्णे खगुणाङ्गुले किल सखे याम्यो भुजस्त्र्यङ्गुल इत्यादयः~। 
परे प्रश्नाध्यायोक्ता इतरे पृच्छकवशादपि ते ज्ञेयाः~। यद्वा तस्माद्व्यक्तं पूर्वं
प्रोक्तमिदानीं बीजक्रियां च वच्मि~। यस्मादव्यक्तयुक्त्या विना प्रश्नाः प्रायो बहुधा
ज्ञातुं नो शक्याः~। तेनैवमुपलभ्यते केचन प्रश्ना व्यक्तयुक्त्यापि ज्ञातुं शक्यन्ते~।
वक्ष्यति च प्रश्नाध्याये\textemdash 
\begin{quote}
    \q
     पाट्या च बीजेन च कुट्टकेन वर्गप्रकृत्या च तथोत्तराणि~। \\
 गोलेन यन्त्रैः कथितानि तेषां बालावबोधे कतिचिच्च वच्मि~॥~इति~॥
\end{quote}

तथा च प्रश्नोत्तरार्थज्ञानसाधनमव्यक्तं च भवति~। यतस्तस्माद्व्यक्तं
पूर्वं प्रोक्तमिदानीं बीजक्रियां वच्मीत्यर्थः~। ननु
प्रश्नोत्तरार्थज्ञानं साधनं द्वयमपि
\newpage 
%%%%%%%%%%%%%%%%%%%%%%%%%%%%%%%%%%%%%%%%%%%%%%%%%%%%%%%% 

\noindent भवत्यतस्तर्हि त्वयोक्तमेतत्कथं व्यक्तं पूर्वप्रोक्तमित्यत
आह\textendash \,\hyperref[2]{\textbf{अव्यक्तबीजम्}} इति~।
अव्यक्तस्य बीजं मूलं तथा च यावद्व्यक्तगणितोक्तभिन्नपरिकर्माष्टकत्रैराशिकादिकं न ज्ञायते तावदव्यक्ते प्रवेशो न भवतीति व्यक्तं पूर्वं प्रोक्तमितिभावः~। तदेवं
व्यक्तसापेक्षतया व्यक्तानन्तरं ग्रहगणितोपयुक्ततया
ग्रहगणितात्प्रागव्यक्तस्यारम्भो युक्त इति सङ्गतिः प्रदर्शिता~। असङ्गतप्रलापो हि प्रेक्षावतामनधेयवचनो भवति~। बीजक्रियां वच्मीति वदतैकवर्णानेकवर्णसमीकरणमध्यमाहरणभावितरूपभेदचतुष्टयभिन्नं गणितं विषयः प्रदर्शितः~। तदुपयुक्ततया धनर्णषड्विधकरणीषड्विधकुट्टकवर्गप्रकृतिचक्रवालान्यपि विषयत्वेन प्रदर्शितानि~। 
विषयस्य \,शास्त्रस्य \,च \,प्रतिपाद्यप्रतिपादकभावः \,सम्बन्धोऽपि \,बीजक्रियां \,वच्मीत्यनेनैव दर्शितः~।
यद्वा ज्ञातेऽपि विषये प्रयोजने च वेदबाह्यैरहेतुकैराधुनिकैः
कल्पितमिदमुत पारम्पर्यागतमिति संशयेन नूतनकल्पितमेवेदं शास्त्रमिति भ्रमेण वा
प्रेक्षावन्तः शिष्टा न प्रवर्तेरन्~। तदर्थं पारम्पर्यलक्षणसम्बन्धकथनमावश्यकम्~। तच्च
बीजगणितस्य प्रश्नज्ञानसाधनत्वं वदताचार्येण कृतमेव~। तथा हि अव्यक्तगणितं प्रश्नज्ञानसाधनत्वाज्ज्योतिषत्वाद्वेदाङ्गत्वाद्ब्रह्मणः
सकाशाद्वसिष्ठादिद्वारा पारम्पर्येणागतमित्युक्तं भवति~। उक्तं च नारदेन {\qt "अस्ति शास्त्रस्यसम्बन्धो वेदाङ्गमिति धातृत"} इति~। आचार्योऽपि गोलाध्याये स्पष्टीकृतवासनायां
वक्ष्यति\textemdash 
\begin{quote}
    \q
    दिव्यं ज्ञानमतीन्द्रियं यदृषिभिर्ब्राह्मं वसिष्ठादिभिः\\
 पारम्पर्यवशाद्रहस्यमवनीं नीतं प्रकाश्यं ततः~।\\
 नैतद्द्वेषिकृतघ्नदुर्जनदुराचाराचिरावासिनां\\
 स्यादायुः सुकृतक्षयो मुनिकृतां सीमामिमामुज्झतः~॥~इति~॥
\end{quote}
 
 प्रयोजनं तु प्रश्नोत्तरार्थज्ञानं गोलज्ञानं चापरं परम्परया जगतः
शुभाशुभफलादेश्च~। यतो वक्ष्यति गोलाध्याये\renewcommand{\thefootnote}{1}\footnote{सिद्धान्तशिरोमणिः गो.अ.}
\begin{quote}
    \q
    ज्योतिःशास्त्रफलं पुराणगणकैरादेश इत्युच्यते\\
 नूनं लग्नबलाश्रितः पुनरयं तत्स्पष्टखेटाश्रयम्~।
\end{quote}
 \newpage 
 %%%%%%%%%%%%%%%%%%%%%%%%%%%%%%%%%%%%%%%%%%%%%%%%%%%%%%%%७
\begin{quote}
    \q
    ते गोलाश्रयिणोऽन्तरेण गणितं गोलोऽपि न ज्ञायते \\
 तस्माद्यो गणितं न वेत्ति स कथं गोलादिकं ज्ञास्यति~॥~इति~॥
\end{quote}
 
नारदोऽपि\textendash \,प्रयोजनं तु जगतः शुभाशुभनिरूपणम् इति~। मुख्यं च शास्त्रप्रयोजनम् एवास्य प्रयोजनम्~। यो ज्योतिषं वेत्ति नरः स
सम्यग्धर्मार्थकामाल्लँभते\renewcommand{\thefootnote}{१}\footnote{कामाँ इति पाठान्तरम्}
यशश्चेति~। इहाधिकारी तु प्रश्नादिजिज्ञासुः पठितव्यक्तश्च~। स च द्विज
एव~। यद्वक्ष्यति सिद्धान्तशिरोमणौ\textendash \\
\vspace{-2mm}

{\q तस्माद्द्विजैरध्ययनीयमेतत्पुण्यं रहस्यं परमं च तत्त्वम्~॥~इति~। }\\
\vspace{-2mm}

\noindent अत्रैवकारस्य \,पाठक्रमेण \,योजने \,ज्योतिषस्यावश्याध्ययनीयता \,प्रतीयते~। \,द्विजैरेवेति योजने द्विजातिरिक्तैरनध्ययनीयता च प्रतीयते~। द्वे अप्यत्र
युक्ते इति~। ननु यद्वेति व्याख्याने अव्यक्तबीजमित्यत्र तत्पुरुषसमासे
व्यक्तस्य कृत्स्नस्य तदेकबीजमिति सर्वग्रन्थविरोधः~। कश्विदव्यक्तभागो व्यक्तस्य
बीजं कश्चिद्व्यक्तभागोऽव्यक्तस्य बीजमिति न विरोध इति चेत्~। न~।
कृत्स्नपदस्योक्तत्वात्~। न च व्यक्तस्य कृत्स्नस्य तदेकबीजमिति बीजस्य
व्यक्तमूलकत्वेऽप्यविरुद्धामिति वाच्यम्~। व्यक्तज्ञानेऽव्यक्तज्ञानमव्यक्तज्ञाने च
व्यक्तज्ञानमिति परस्पराश्रयस्य दुस्तरत्वात्~। मैवम्~। {\qt गङ्गा गङ्गेति यो
ब्रूयाद्योजनानां शतैरपि~। मुच्यते सर्वपापेभ्य} इत्यादौ सर्वशब्दस्येव प्रकृते कृत्स्नपदस्य
बहुत्वपरत्वात्~। इतरथा व्यक्तानन्तरमव्यक्तारम्भानुपपत्तेः~। अत एव कश्चन
व्यक्तभागोऽव्यक्तमूलं कश्चिदव्यक्तभागो व्यक्तमूलमिति विरोधपरिहारो
युक्त एव कृत्स्नपदे सङ्कोचस्यावश्याभ्युपेयत्वान्न हि
व्यक्तोक्तसङ्कलनव्यवकलनादिष्वप्यव्यक्तं मूलमिति केनाप्युररीक्रियते~। 
किं तु गुणघ्नमूलोनेत्यादावेव~।
किञ्च कृत्स्नपदे सङ्कोचाभावेऽपि न कश्चिद्दोषः~। तथाहि यथा
गुणघ्नमूलोनेत्यादिव्यक्तगणितस्याव्यक्तमूलकत्वेऽपि न स्वरूपनिर्वाहाय तदपेक्षा किं
तूपपत्तावेव तद्वदखिलस्यापि व्यक्तस्याव्यक्तमूलकत्वे कुतस्त्यः परस्पराश्रय इत्यलं
पल्लवितेन~॥~२~॥~\\

\vspace{-4mm}
अव्यक्तक्रिया तावदव्यक्तषड्विधाधीना तदपि धनर्णषड्विधाधीनम् अतः प्रथमतस्तदत्र प्रतिपादनीयं तत्रापि व्यवकलनादीनां
सङ्कलनपूर्वकत्वाद्धनर्णसङ्कलनं

\newpage %%%%%%%%%%%%%%%%%%%%%%%%%%%%%%%%%%%%%%%%

\noindent तावदुपजातिकापूर्वार्धेनाह\textemdash  \\

{\bqt धनर्णसङ्कलने करणसूत्रम्\textemdash }
\phantomsection \label{3}
\begin{quote}
    \ab
 योगे युतिः स्यात्क्षययोः स्वयोर्वा धनर्णयोरन्तरमेव योगः~॥~३~॥
\end{quote}

क्षययोरृणयोः स्वयोर्धनयोर्वा योगे कर्तव्ये युतिः स्यात्~। एतदुक्तं
भवति~। ययोर्योगः कर्तव्योऽस्ति तौ रूपात्मकौ करण्यात्मकौ वा राशी
यद्युभावप्यृणगतौ धनगतौ वा भवतस्तदा तयो राश्योर्योगः कार्यः~।
क्रमादुत्क्रमतोऽथ वाङ्कयोग इति व्यक्तगणितोक्तयोगो विधेयः~। 
स एवात्र योगो भवति~। करण्योस्तु योगोऽन्तरं वा योगं करण्योर्महतीं
प्रकल्प्येत्यादि वक्ष्यमाणप्रकारेण विधेयमिति द्रष्टव्यम्~। 
एवं बहूनामपि सजातीययोग उक्तः~।
यत्र त्वेको राशिर्धनमितरश्चर्णं तयोर्योगे कर्तव्ये किं कर्तव्यं
तदाह\textendash \,\hyperref[3]{\textbf{धनर्णयोरन्तरमेव योग}} इति~। व्यक्तरीत्या यदन्तरं सम्पद्यते स एव
धनर्णयोर्योग इत्यर्थः शेषस्य धनर्णवशाद्योगस्यापि धनर्णत्वं ज्ञेयम्~॥~३~॥\\

\vspace{-2mm}
{\bqt उदाहरणम्\textendash }

\phantomsection \label{4}
\begin{quote}
    {\eg
     रूपत्रयं रूपचतुष्टयं च \\
     क्षयं धनं वा सहितं वदाशु~।\\
 स्वर्णं क्षयः\renewcommand{\thefootnote}{1}\footnote{क्षयं} स्वं च पृथक्पृथक्त्वे\renewcommand{\thefootnote}{2}\footnote{पृथङ् मे} \\
 धनर्णयोः सङ्कलनामवैषि~॥~४~॥}
\end{quote}

\hyperref[4]{\textbf{रुपत्रयं रुपचतुष्टयं चे}}ति~। द्वयमित्यृणम् इत्येकम्~। द्वयमपि धनमिति
द्वितीयम्~। आद्यं धनमपरमृणमिति तृतीयम्~। प्रथममृणमितरद्धनमिति
चतुर्थमेवं चत्वार्युदाहरणानि~। \hyperref[3]{\textbf{धनर्णयोरिति}}~। धने चर्णे च धनर्णम्~। धनं चर्णं
धनर्णम्~। धनर्णं च धनर्णं च धनर्णे~। तयोर्धनर्णयोः~। धनयोरृणयोर्धनर्णयोश्चेत्यर्थः~। 
चतुर्थप्रश्नस्य तृतीयेऽन्तर्भूतत्वात्पक्षत्रयमेवोद्दिष्टमिति~॥~४~॥ \\

\vspace{-4mm}
\indent नन्विदं धनमिदमृणमिति वेदं व्यक्तमिदमव्यक्तमित्यादि वा कथमवधेयमित्यत
आह\textendash 
 \newpage
 %%%%%%%%%%%%%%%%%%%%%%%%%%%%%%%%%%%%%%%%%%%%%%%%%%%%%%%%९
\begin{quote}
    \ab
    अत्र रूपाणामव्यक्तानां चाद्याक्षराण्युपलक्षणार्थं\renewcommand{\thefootnote}{1}\footnote{उपलणार्थं} लेख्यानि~।\\
 तथा यान्यृणगतानि\renewcommand{\thefootnote}{2}\footnote{यान्यूणगतानि} तान्यूर्ध्वबिन्दूनि चेति~॥~५~॥
\end{quote}
 
अतिरोहितार्थमिदम्~। यद्यप्यृणत्वादिकमालापत एवावगन्तुं शक्यं तथाप्यालापबहुत्वं ऋणत्वादौ भ्रान्तिः संशीतिर्वा स्यात्~। उपस्थितिलाघवं च
न स्यादित्यूर्ध्वबिन्द्वादिलिखनं युक्ततरम्~। धनर्णत्वं तु व्यवकलनोपपत्तौ
विचारयिष्यामः~। अत्र प्रथमोदाहरणे न्यासः~। ३ं~। ४ं योगे जातं ७ं~।
द्वितीये न्यासः~। ३~। ४ योगे जातम् ७~। तृतीये न्यासः~। ३~। ४ं
\hyperref[3]{\textbf{धनर्णयोरन्तरमेव योग}} इति जातम् १ं~। चतुर्थे न्यासः~। ३ं~। ४ अन्तरमेव योग इति जातम्
१~। अत्रोपपत्तिर्लोकसिद्धैव~। तथाहि देवदत्तस्य मुद्रात्रयमृणमेकमितरदपि
मुद्राचतुष्टयमृणमित्यभिहिते मुद्रासप्तकमृणमस्तीति प्रतीतिरस्त्यागोपालाविपालेभ्यो
व्यवहारसिद्धा~। एवं देवदत्तस्य मुद्रात्रयं धनमेकमन्यदपि मुद्राचतुष्टयं
धनमस्तीत्युक्तेऽस्त्यस्य मुद्रासप्तकं धनमिति विलसति सार्वजनीने व्यवहारः~। अत उक्तं\textendash \,\hyperref[3]{\textbf{योगे युतिः स्यात्क्षययोः स्वयोर्वेति}}~। अथ देवदत्तस्य मुद्रात्रयं धनमस्ति मुद्राचतुष्टयमृणमप्यस्तीत्युक्तेनास्य धनम् अस्ति किन्तूत्तमर्णस्य
मुद्रात्रये दत्त एकैव मुद्रास्यर्णमस्तीति वरीवर्ति सकलजनसाधारणो व्यवहारः~। 
अत उक्तं\textendash \,\hyperref[3]{\textbf{धनर्णयोरन्तरमेव योग}} इति~॥~५~॥~\\

 \vspace{-4mm}
 ननु व्यक्ते भिन्नानाम् अभिन्नानां च सङ्कलनव्यवकलनादि पृथक्पृथगुक्तम्~।
अत्र तु भिन्नानां सङ्कलनव्यवकलनाद्यं च न पृथगभिहितमस्ति तत्कथं कर्तव्यमिति
तदाह\textendash 
\vspace{1mm}

\begin{center}
    \ab
     एवं भिन्नेष्वपीति\renewcommand{\thefootnote}{3}\footnote{विभिन्नेष्वपि}~॥~६~॥
\end{center}
 
अयमर्थः\textendash \,सच्छेदानामपि रूपाणां वर्णानां वा योगार्थं धनर्णत्ववशाद्योगेऽन्तरे वा प्राप्ते योगोऽन्तरं तुल्यहरांशकानामित्यादिना योगोऽन्तरं
वा विधेयमिति~। एवं भिन्नव्यवकलनादिष्वपि बोद्धव्यम्~॥~६~॥
 \newpage %%%%%%%%%%%%%%%%%%%%%%%%%%%%%%%%%%%%%%%%%%%%%%%%%%%%%%%% 

यद्यपि व्यवकलनादीनां सङ्कलनोपजीवकत्वात्तत्प्राथम्येन सङ्कलननिरूपणं
युक्तं न तथा गुणनप्राथम्येन व्यवकलने निरूपणं
युक्तमुपजीव्योपजीवकभावाभावात्तथापि
धनर्णताव्यत्यासमात्रविलक्षणस्य व्यवकलनस्य गुणनापेक्षया
सङ्कलनान्तरङ्गत्वात्खण्डगुणन इष्टोनयुक्तेन गुणेन निघ्न इत्यस्मिन्नपि गुणने
तस्योपजीव्यत्वाच्च गुणनप्राथम्येन तन्निरूपणं युक्तमित्युपजातिकोत्तरार्धेन तदाह\textendash 

\phantomsection \label{7}
\begin{quote}
    \ab
     संशोध्यमानं स्वमृणत्वमेति स्वत्वं क्षयस्तद्युतिरुक्तवच्च~॥~७~॥
\end{quote}

संशोध्यतेऽपनीयते तत् \hyperref[7]{\textbf{संशोध्यमानम्}}~। रूपं वर्णः करणी चेति
त्रिलिङ्गसामान्यं नपुंसकत्वम्~। तद्यदि धनमस्ति तर्हि ऋणत्वमेति~। यदि क्षयोऽस्ति तर्हि
धनत्वमेति~। पश्चादुक्तवत्तद्युतिश्च~। एतदुक्तं भवति\textendash \,ययोरन्तरं
विधेयमस्ति तयोर्मध्ये संशोध्यमानस्य धनर्णताव्यत्यासं कृत्वा योगे युतिः
स्यादित्यादिना तयोर्युतिः कर्तव्या~। तदेव व्यवकलनं फलं भवतीत्यर्थः~। अत्रोपपत्तिः\textendash \,ऋणत्वमिह त्रिधा तावदस्ति~। देशतः कालतो वस्तुतश्चेति~। तच्च वैपरीत्यमेव
यत उक्तमाचार्यैर्लीलावत्यां क्षेत्रव्यवहारे\textendash \,दशसप्तदशप्रमौ
भुजावित्यस्मिन् उदाहरणे~। ऋणगताबाधादिग्वैपरीत्येनेत्यर्थ इति~। तत्रैकरेखा स्थिता द्वितीया
दिग्विपरीता दिगित्युच्यते~। यथा पूर्वदिग्विपरीता पश्चिमा दिक्~। यथा
चोत्तरदिग्विपरीता दक्षिणा दिगित्यादि~। तथा च पूर्वापरदेशयोर्मध्य एकतरस्य धनत्वे कल्पिते
तं प्रति तदितरस्यर्णत्वम्~। यथा पूर्वगतेर्धनत्वकल्पने यदा ग्रहः
पश्चिमगतिर्भवति तदा ग्रहे गतितुल्यकला ऋणं भवति~। अथवा पश्चिमभ्रमस्य धनत्वे
यावद्ग्रहः पूर्वतो गच्छति तावत्पश्चिमभ्रम ऋणमिति दक्षिणोत्तरदेशादिष्वप्येवमेवर्णत्वं बोध्यम्~। एवं पूर्वोत्तरकालयोरप्यन्योन्यमृणत्वं वारप्रवृत्त्यादिषु प्रसिद्धम्~।
एवं यस्मिन्वस्तुनि यस्य स्वस्वामिभावसम्बन्धस्तस्य
तद्धनमिति व्यवह्रियते~। तस्मिन्वैपरीत्यं तु परस्य स्वस्वामिभावसम्बन्धः~।
अतो देवदत्तस्वामिके धने यावद्यज्ञदत्तस्वामिकत्वं तावद्देवदत्तस्यर्णमिति
व्यवह्रियते~। तत्र पूर्वदेशस्य धनत्वं पश्चिमदेशस्य चर्णत्वं प्रकल्प्योपपत्तिरुच्यते~।
सा यथा\textendash \,श्रीविश्वेशितुः शम्भोरानन्दकाननात्पुरन्दरदिशि पञ्चदशसु योजनेषु
स्वर्गतरङ्गिणीतीरविलासि वरीवर्ति किलैकं पत्तनम्~। वरुणदिशि
चाष्टयोजनेष्विन्दीवरदलश्यामलपतङ्गतनयातरङ्गचुम्बिभिः शरच्चन्द्रिकाधवलैः
सुर\textendash\ 

\newpage %%%%%%%%%%%%%%%%%%%%%%%%%%%%%%%%%%%%%%%%%%%%%%%%%%%%%%%%११

\noindent नदीलोलकल्लोलैः स्मृतहरिहरमूर्तिरानन्दलहरीरनुभवञ्जागर्ति तीर्थराजः प्रयागः~।
तयोः तूच्चावचसकलजनव्यवहारसिद्धमस्ति त्रयोविंशतियोजनात्मकमन्तरम्~।
तच्च योगं विना नोपपद्यते~। अतो विजातीययोरन्तरे साध्ये योगः कर्तव्यः
परन्तु स योगः पश्चिमः पूर्वो वा~। तत्र पत्तनात्प्रयागः कस्यां दिशीति
विचारे तावदानन्दकाननात् प्रयागपर्यन्तमष्टयोजनात्मको देशो यथा
पश्चिमस्तथा पत्तनादपि पश्चिमो भवति किं त्वानन्दकाननात् पत्तनपर्यन्तं
पञ्चदशयोजनात्मकमेकं शकलम्~। ततः प्रयागावधि द्वितीयमष्टयोजनात्मकम्~। 
शकलद्वयस्य पश्चिमस्थत्वाज्जातस्त्रयोविंशतियोजनात्मकः पश्चिमो देशः~। एवं
प्रयागात् पत्तनं कस्यां दिशीति विचारे प्रयागादानन्दवनपर्यन्तं देशशकलं विपरीतदिक्कं
भवति~। तथा च यस्मादन्तरं साध्यते तदवधि शकलं विपरीतदिक्कं
भवतीत्यत उक्तं \hyperref[7]{\textbf{संशोध्यमानं स्वमृणत्वमेति स्वत्वं क्षय}} इति~। एवं
धनर्णयोरन्तरे प्रतिपादितम्~। एवं धनयोरपि~। तद्यथा~। एकः किल काशीतः
पूर्वदिग्भागे दशयोजनानि गत इतरोऽपि तस्मिन्नेव सप्तयोजनानि
गतस्तयोश्चान्तरं योजनत्रयं सर्वजनप्रसिद्धम्~। तच्च दशयोजनगात्पश्चिमम्~।
सप्तयोजनगात्पूर्वम्~। इदमपि प्रथमावधिभूतस्य खण्डस्य व्यत्यासे कृते \hyperref[3]{\textbf{धनर्णयोरन्तरमेव योग}} इति योगे च कृते सिध्यति~। एवमृणयोरपि बोध्यम्~। अत उपपन्नं \hyperref[7]{\textbf{संशोध्यमानं
स्वमृणत्वमेति स्वत्वं क्षयस्तद्युतिरुक्तवच्चे}}ति~। अन्यदपि सुधीभिरूहनीयम्~॥~७~॥\\

\vspace{-2mm}
{\bqt उदाहरणम्\textendash \,}
\begin{quote}
    {\eg
 त्रयाद्द्वयं स्वात्स्वमृणादृणं च \\
 व्यस्तं च संशोध्य वदाशु शेषम्~॥~८~॥}
\end{quote}

स्वात्त्रयात् स्वं द्वयमित्येकमृणात् त्रयादृणं द्वयमिति
द्वितीयम् इत्युदाहरणद्वयम्~। व्यस्तत्वे च स्वात्त्रयादृणं द्वयमित्येकमृणात्त्रयात्स्वं द्वयमीति द्वितीयमेवं चत्वार्युदाहरणानि~। तत्र प्रथमे न्यासः ३~। २ संशोध्यमानं २ स्वमृणत्वमेतीति जातम्~। ३~। २ं~। अनयोर्युतिरुक्तवत्~। \hyperref[3]{\textbf{धनर्णयोरन्तरमेव योग}} इति जातम्~। १~। द्वितीये न्यासः ३ं~। २ जातमुक्तवदन्तरम्~। १ं~। तृतीये
न्यासः ३~। २~। \hyperref[7]{\textbf{संशोध्यमानं क्षयः स्वत्वमेती}}त्यादिना जातम् ५~। चतुर्थे न्यासः ३ं~। २ं~।
\afterpage{\fancyhead[CE] {बीजगणिते}}
\afterpage{\fancyhead[CO]{धनर्णषड्विधम्}}
\afterpage{\fancyhead[LE,RO]{\thepage}}
\cfoot{}
\newpage
%%%%%%%%%%%%%%%%%%%%%%%%%%%%%%%%%%%%%%%%%%%%%%%%%%

\noindent \hyperref[7]{\textbf{संशोध्यमानं स्वमृणत्वमेती}}त्यादिना जातम् ५ं~। इदमेव प्रतीत्यर्थं
पूर्वपश्चिमदेशत्वेन योज्यते~। पू. ३ पू. २ संशोध्यमानः पूर्वदेशः
पश्चिमदेशो भवतीति जातं पू. ३ प. २~। अनयो\hyperref[3]{\textbf{र्धनर्णयोरन्तरमेव योग}} इति शेषमन्तरं
पू. १~। अत्रैकस्मादवधेः पूर्वतो योजनद्वयेन त्रयेण च नरौ तिष्ठतः~।
तत्र योजनद्वयगतात् पुंसो योजनत्रयगो योजनमेकं पूर्वतस्तिष्ठतीत्यर्थः~।
अत्रोदाहरणेषु द्वयस्य शोध्यतोक्तेर्योजनद्वयगान्नरादन्तरं ज्ञातव्यम्~। 
अथ द्वितीये प. ३ प. २~। उक्तवदन्तरे जातं प. १~। पश्चिमतो योजनत्रयगतः पश्चिमतो
योजनद्वयगतादेकेन योजनेन पश्चिमतस्तिष्ठतीत्यर्थः~। तृतीये न्यासः पू. ३
प. २~। उक्तवदन्तरे जातं पू. ५~। पश्चिमतो योजनद्वयगतात्पुंसः पूर्वतो
योजनत्रयगः पञ्चभिर्योजनैः पूर्वतस्तिष्ठतीत्यर्थः~। चतुर्थे न्यासः प. ३
पू. २ उक्तवज्जातमन्तरं प. ५~। पूर्वतो योजनद्वयगतात्पश्चिमतो योजनत्रयगः
पञ्चभिर्योजनैः पश्चिमतस्तिष्ठतीत्यर्थः~॥~८~॥\\

\vspace{-2mm}
{\bqt गुणने करणसूत्रम्\textendash }

\phantomsection \label{9}
\begin{quote}
    \ab
    स्वयोरस्वयोः स्वं वधः स्वर्णघाते क्षयः~॥~९~॥
\end{quote}
 
\hyperref[9]{\textbf{स्वयोरस्वयोर्वा वधो}} गुणनम्~। एकस्यापरतुल्यावृत्तिरिति यावत्~। धनं
भवति~। \hyperref[9]{\textbf{स्वर्णघाते क्षयो}} भवति~। एतदुक्तं भवति~। यदा गुण्यो गुणकश्चेति
द्वावपि धनमृणं वा भवतस्तदा तदुत्थं गुणनफलं धनं भवति~। यदा त्वेकतरो\renewcommand{\thefootnote}{1}\footnote{त्वेतकरो}
धनमृणमितरस्तदा तदुत्थं गुणनफलम् ऋणं भवतीति~। अत्र गुणनफलस्य धनर्णत्वमात्रं प्रतिपादितम्~। अङ्कतस्तु व्यक्तोक्ताः सर्वेऽपि गुणनप्रकारा
द्रष्टव्याः~॥~९~॥\\

\vspace{-2mm}
{\bqt उदाहरणम्\textendash }
\begin{quote}
    \eg धनं धनेनर्णमृणेन निघ्नं द्वयं त्रयेण स्वमृणेन किं स्यात्~॥~१०~॥~
\end{quote}
\newpage %%%%%%%%%%%%%%%%%%%%%%%%%%%%%%%%%%%%%%%%

ऋणं धनेनेति चतुर्थमप्युदाहरणं द्रष्टव्यम्~। अत्र गुणकः ३ गुण्यः २~।
अथ प्रथमे न्यासः~। २~। ३~। उक्तवज्जातं गुणनफलं धनम् ६~। द्वितीये
न्यासः\textendash \,२~। ३ं~। अस्व-योर्वधः स्वमिति जातम् ६~।
तृतीये न्यासः २ं~। ३~। स्वर्णघाते क्षय इति जातम् ६ं~। चतुर्थे न्यासः २~। ३ं~। स्वर्णघाते क्षय इति ६ं~। गुण्येन हते गुणके च तदेवेति चूर्णिकया गुण्यत्वगुणकत्वयोः
कामचारः प्रदर्शितः~। ननु स्वयोर्वधः स्वं भवितुमर्हति~।
समजातीयत्वाद्दृष्टचरत्वाच्च~। परमृणयोर्वधः कथं धनं भवितुमर्हति
विजातीयत्वात्~। एवं स्वर्णघातेऽपि क्षयः कथं भवति~। न च
विजातीयत्वादिति वाच्यम्~। वैपरीत्यस्यापि सुवचत्वाद्धनमेव कथं न
स्याद्विनिगमनाविरहात्~। अत्रोच्यते गुण्यस्य गुणकतुल्यावृत्तिर्हि
गुणनफलमिति तावत्प्रसिद्धम्~। तत्र गुणको द्विविधः~। धनमृणं चेति~। तत्र धनगुणके
सति धनस्य, ऋणस्य वा गुण्यस्यावर्तने क्रियमाणे क्रमेण धनमृणं च गुणनफलं
स्यात्~। अतः स्वयोर्वधः स्वम्~। गुणकस्य धनत्वे गुण्यस्यर्णत्वे
त्वृणमिति सिद्धम्~।\\

\vspace{-4mm}
 अथर्णगुणके विचारः\textendash \,तत्रर्णत्वं वैपरीत्यमिति प्रागेव प्रतिपादितम्~।
तथा च ऋण-गुणको नाम विपरीतगुणकः~। गुण्यस्य विपरीतावर्तनकर इति
यावत्~। तथा सति धने गुण्ये गुणनफलमृणम्~। ऋणे गुण्ये गुणनफलं
धनमिति सिद्धम्~। अत्रान्तिमपक्षे स्वयोर्वधः स्वम् इत्युपपन्नम्~।
मध्यमपक्षयोस्तु गुण्यगुणकयोरेकतरस्य धनत्वेऽन्यस्य-र्णत्वे फलमृणमुत्पद्यत इति स्वर्णघाते
क्षय इत्युक्तम्~। यद्वा गणितेनोपपत्तिः प्रदर्श्यते~। धनगुणने तावदविवाद एव~।
ऋणगुणने तु विचारः~। अस्ति तावदिदं सुप्रसिद्धं गुण्यगुणकखण्डाभ्यां
पृथग्गुणितः सहितश्च गुणनफलं भवतीति~। यथा गुण्यः १३५ गुणकः १२~।
अस्य खण्डद्वयम् ४~। ८ एकमिष्टमिष्टोनो राशिरपरं च~। खण्डाभ्यां
पृथग्गुणितो गुण्यः ५४०~। १०८०~। योगे जातं गुणनफलम् १६२०~।
एकमेव कल्पितमिष्टम् ४ं~। एतदूनो \,राशिः १२ \,द्वितीयं \,खण्डम् १६~।
अत्रापि \,पृथक्खण्डद्वयगुणितेन \,सहितेन च गुण्येन गुणनफलेन च भवितव्यम्~।
तत्र खण्डाभ्यां ४ं~। १६ पृथग्गुणितो गुण्यः ५४ं०~। २१६०~।
अनयोर्योगे गुणनफलं नोपपद्यत इति~। गुणनफलान्यथानुपपत्त्या स्वर्णघाते
क्षयो भवतीत्यवगम्यते~। यतस्तथा कृतो ५४ं०~। २१६०~। \hyperref[3]{\textbf{धनर्णयोरन्तरमेव योग}} इति १६२० गुणनफलमुपपद्यते~। अत उक्तं स्वर्णघाते क्षय इति~।
\newpage
%%%%%%%%%%%%%%%%%%%%%%%%%%%%%%%%%%%%%%%%
\noindent एवं गुण्यखण्डे प्रत्येकं गुणकखण्डगुणिते सहिते च गुणनफलं भवति~।
तद्यथा\textendash \,गुण्यः १३५ एतस्य खण्डद्वयं १३०~। ५~। गुणकस्यापि खण्डद्वयं
४~। ८~। गुणकखण्डाभ्यां प्रत्येकं गुणितं गुण्यपूर्वखण्डं १३० जातं ५२०~।
१०४०~। एवमेव प्रत्येकं गुणितं द्वितीयखण्डं ५ जातं २०~। ४०~। सर्वेषां
योगे जातं गुणनफलं १६२०~। एवमेव कृतमभीष्टं खण्डद्वयं गुण्यस्य १४०~। ५ं गुणकस्यापि
१६~। ४ं~। अत्रापि गुणकखण्डाभ्यां प्रत्येकं गुणितं पूर्वखण्डं १४० जातं
२२४०~। ५६ं०~। अनयोर्योगः १६८०~। एवमेव द्वितीयमपि ५ं गुणकखण्डाभ्यां
पृथग्गुणितं ८ं०~। २० अत्रर्णगुणितमृणं सजातीयत्वादृणमेवेति कृते गुणनफलं
नोपपद्यत इति गुणनफलान्यथानुपपत्त्या, ॠणमृणगुणितं धनं भवतीत्यवगम्यते~।
यतस्तथा कृते ८ं०~। २ं०~। गुणनफलं १६२० उपपद्यत इत्यत
उक्तमस्वयोर्वधः स्वमिति~। एवं बुद्धिमद्भिरन्यदप्यूह्यम्~। ननु वर्गस्य
समद्विघातरूपतया गुणनान्तरङ्गत्वाद्भजनानपेक्षत्वाच्च प्रथमतो निरूपणं युक्तम्~।
न च {\qt "भक्तो गुणः शुध्यति" (ली.\,५)} इत्यादिना गुणनप्रकारेण वर्गकरणे
भजनस्योपजीव्यतया तस्यैव प्राथम्येन निरूपणं युक्तम् इति वाच्यम्~।
गुणनादपि पूर्वं तन्निरूपणप्रसङ्गादिति चेन्न~। वर्गकरणप्रकाराणामतिविलक्षणतया
वर्गस्य गुणनं प्रति बहिरङ्गत्वात्~। प्रत्युत वर्गं प्रतिपदस्येव गुणनं
प्रति भजनस्यैवान्तरङ्गत्वाद्वर्गं प्रत्युपजीव्यत्वाच्च
प्रथमतस्तन्निरूपणस्यैवावश्यकत्वात्~॥~१०~॥\\

\vspace{-4mm}
कस्यचिद्गुणनप्रकारस्य भजनसापेक्षत्वेऽपि भजननिरपेक्षतयापि गुणनस्य
सिद्धत्वाद्भजनस्य तु सर्वथा गुणनसापेक्षत्वाद्गुणनानन्तरमेव तन्निरूपणं
युक्तमिति भुजङ्गप्रयातपूर्वार्धशेषशकलेन तदाह\textendash \,

\phantomsection \label{11}
\begin{center}
    \ab भागहारेऽपि चैवं निरुक्तम्\renewcommand{\thefootnote}{1}\footnote{निरूक्तमिति}~॥~११~॥
\end{center}

\hyperref[11]{\textbf{भागहारेऽपि}} गुणनवदेव \hyperref[11]{\textbf{निरुक्तमि}}त्यर्थः~। एतदुक्तं भवति\textendash \,भाज्यभाजकयोरुभयोरपि धनत्वे ऋणत्वे वा लब्धिर्धनमेव~। यद्वा त्वेकतरस्य
धनत्वमृणत्वमितरस्य तदा लब्धमृणम् एवेति~। अत्राप्यङ्कतो भागप्रकारो व्यक्तोक्तो ज्ञेयः~॥
११~॥
\newpage
%%%%%%%%%%%%%%%%%%%%%%%%%%%%%%%%%%%%%%%%

{\bqt उदाहरणम्\textendash \,}
\begin{quote}
    \eg
     रूपाष्टकं रूपचतुष्टयेन धनं धनेनर्णमृणेन भक्तम्~। \\
 ॠणं धनेन स्वमृणेन किं स्याद्द्रुतं वदेदं यदि बोबुधीषि~॥~१२~॥
\end{quote}

 स्पष्टोऽर्थः~। प्रथमे न्यासः\textendash \,८$\div$४ स्वयोर्भागहारः स्वमिति जाता
लब्धिर्धनं २~।
द्वितीये न्यासः ८ं$\div$४ं अस्वयोर्भागहारः स्वमिति जाता लब्धिर्धनमेव २~।
तृतीये न्यासः\textendash \,८ं$\div$४ स्वर्णभागहारे क्षय इति जाता लब्धिः ऋणं २ं, चतुर्थे न्यासः~। ८$÷$४ं~। स्वर्णभागहारे क्षय इति जाता लब्धिः ऋणं २ं~। अत्रोपपत्तिः\textendash \\
\vspace{-4mm}

{\qt भाज्याद्धरः शुध्यति यद्गुणः स्यादन्त्यात्फलं तत्खलु भागहारे}~।\\

\vspace{-4mm}
\noindent इत्युक्तत्वाद्यस्मिन्नङ्के हरगुणिते भाज्यादपनीते शुद्धिर्भवति सा किल
लब्धिः~। तत्र प्रथमे ८$\div$४ धनेन द्वयेन हरे ४ गुणिते ८ भाज्यात्, ८ अपनीते
शुद्धिर्भवतीति धनेन द्वयं २ लब्धिः~। द्वितीयेऽपि ८ं~। ४ं धनद्वयेन हरे
४ं अस्मिन्गुणिते ८ं भाज्यात् ८ं अस्मादपनीयमाने \hyperref[7]{\textbf{संशोध्यमानं क्षयः स्वत्वमेती}}ति \hyperref[3]{\textbf{धनर्णयोरन्तरमेव योगः}} इति च कृते शुद्धिर्भवतीति द्वयं धनमेव लब्धिः २~।
एवं सिद्धम्~। स्वयोरस्वयोर्वा भागहारे स्वमिति~। तृतीये तु ८ं~। ४
धनद्वयेन हरे ४ गुणिते ८ भाज्यादस्मात्, ८ं अपनीते \hyperref[7]{\textbf{संशोध्यमानं स्वमृणत्वमेती}}ति
ॠणयोर्योगे १ं६ शुद्धिर्न स्यादृणगुणिते तु हरे ८ं शुद्धिर्भवतीत्यृणद्वयं
लब्धिः २ं~। एवं चतुर्थेऽपि ८~। ४ं ऋणगुणित एव हरः शुध्यतीति ऋणमेव
लब्धिरिति सिद्धं स्वर्णभागहारे क्षय इति~। अत उक्तं भागहारेऽपि चैवं
निरुक्तमिति~॥~१२~॥\\
\vspace{-2mm}

{\bqt वर्गे मूले च करणसूत्रम्\textendash }

\phantomsection \label{13}
 \begin{quote}
     \ab
     कृतिः स्वर्णयोः स्वं स्वमूले धनर्णे \\
 न मूलं क्षयस्यास्ति तस्याकृतित्वात्~॥~१३~॥~
 \end{quote}
 
 स्वस्य, ऋणस्य वा वर्गः स्वं भवति~। अङ्कतस्तु वर्गप्रकारा व्यक्तोक्ताः
सर्वेऽपि द्रष्टव्याः~। अथ मूलमाह\textendash \,\hyperref[13]{\textbf{स्वमूले धनर्णे}} इति~। स्वस्य धनस्य मूले धनर्णे
\newpage
%%%%%%%%%%%%%%%%%%%%%%%%%%%%%%%%%%%%%%%%

\noindent स्याताम्~। धनस्यैव वर्गस्य, ऋणमपि मूलं भवतीत्यर्थः~। अथात्र
विशेषमाह\textendash \,\hyperref[13]{\textbf{न मूलं क्षयस्यास्तीति}}~। तत्र हेतुमाह\textendash \,\hyperref[13]{\textbf{तस्याकृतित्वादिति}}~। वर्गस्य हि
मूलं लभ्यते~। ऋणाङ्कस्तु न वर्गः~। कथमतस्तस्य मूलं लभ्यते~। ननु, ॠणाङ्कः
कुतो वर्गो न भवति~। न हि राजनिदेशः~। किञ्च यदि न वर्गस्तर्हि वर्गत्वं
निषेद्धुमप्यनुचितमप्रसक्तेः~। सत्यम्~। ऋणाङ्कं वर्गं वदता भवता कस्य स वर्ग इति
वक्तव्यम्~। न तावद्धनाङ्कस्य समद्विघातो हि वर्गः~। तत्र धनाङ्केन
धनाङ्के गुणिते यो वर्गो भवेत् स धनमेव~। स्वयोर्वधः स्वमित्युक्तत्वात्~।
नाप्यृणाङ्कस्य~। तत्रापि समद्विघातार्थमृणाङ्केनर्णाङ्के गुणिते धनमेव वर्गो भवेत्
अस्वयोर्वधः स्वमित्युक्तत्वात्~। एवं सति कमपि तमङ्कं न पश्यामो यस्य वर्गः क्षयो
भवेत्~। न चाप्रसक्तिः~। अङ्कसादृश्याद्भ्रान्त्या वर्गत्वप्रसक्तेः~।
वर्गयुक्तिस्तु गुणनयुक्तिरेव~।
मूले तु व्यस्तविधिरेवोपपत्तिः~॥~१३~॥\\

\vspace{-2mm}
{\bqt वर्गोदाहरणम्\textendash }
\begin{quote}
    \eg
    धनस्य रूपत्रितयस्य वर्गं क्षयस्य च ब्रूहि सखे ममाशु~॥~१४~॥
\end{quote}
 
 स्पष्टोऽर्थः~। प्रथमे न्यासः~। ३ जातो वर्गः ९ स्वम्~। द्वितीये न्यासः\textendash \,३ं जातो वर्गः ९ स्वमेव~। कृतिः स्वर्णयोः स्वमित्युक्तत्वात्~॥~१४~॥\\

\vspace{-2mm}
{\bqt मूलोदाहरणम्\textendash }
\begin{quote}
    \eg
     धनात्मकानामधनात्मकानां मूलं नवानां च पृथग्वदाशु~॥~१५~॥
\end{quote}

 अतिरोहितार्थम्~। [प्रथमे] न्यासः ९ जातं मूलं ३ वा ३ं स्वमूले
धनर्णे इत्युक्तत्वात् द्वितीये न्यासः~। ९ं एषामवर्गत्वान्मूलं नास्ति~। धने
धनपदे वा न कश्चिद्धनर्णत्वकृतो विशेषः~। किन्तु सजातीयत्वमेवेति नात्र तन्निरूपणमिति
ध्येयम्~॥~१५~॥
\newpage
%%%%%%%%%%%%%%%%%%%%%%%%%%%%%%%%%%%%%%%%%%%%%%%%%%%%%%%%%
\begin{quote}
{\qt दैवज्ञवर्यगणसन्ततसेव्यपार्श्वबल्लालसञ्ज्ञगणकात्मजनिर्मितेऽस्मिन्~।\\
बीजक्रियाविवृतिकल्पलतावतारे स्वर्णोद्भवाः समभवन्निति षट्प्रकाराः~॥}
\end{quote}

\begin{center}
इति श्रीसकलगणकसार्वभौमश्रीबल्लालदैवज्ञसुतकृष्णगणकविरचिते\\
बीजविवृतिकल्पलतावतारे धनर्णे\,(र्ण?)\,षड्विधविवरणम्~।\\
(अत्र मूलं मूलश्लोकैः सह ग्रन्थसङ्ख्या दशाधिकशतत्रयम्) \\
\vspace{3.5cm}
\includegraphics[scale=0.8]{graphics/Capture34.png}
\end{center}

\afterpage{\fancyhead[CE]{खषड्विधम्~।}}
\afterpage{\fancyhead[CO]{शून्यषड्विधम्}}
\afterpage{\fancyhead[LE,RO]{\thepage}}
\cfoot{}
\newpage
%%%%%%%%%%%%%%%%%%%%%%%%%%%%%%%%%%%%%%%%%
\thispagestyle{empty}
\phantomsection \label{ch2}
\begin{center}
{\LARGE \textbf{२ खषड्विधम्~।}}
\end{center}

\vspace{2mm}
अथ यथा स्वरूपवर्णादिषड्विधोपयुक्ततया धनर्णषड्विधस्य प्रथमतो निरूपणं
युक्तं तथा खषड्विधस्यापि तद्युक्तम्~। तच्च यद्यपि व्यक्तोक्तशून्यपरिकर्माष्टकेनात्र धनर्णषड्विधेन गतार्थमिति नारम्भणीयं तथापि
यद्यत्र नारभ्येत तर्हि शिष्यैर्व्यक्तोक्तशून्यपरिकर्ममार्गेणैव
शून्यगणितं क्रियेत न तु धनर्णकृतो विशेषोऽनवधानादभ्रमाद्वेति तन्निरासार्थमिह तदारम्भणं
युक्तमेव~। ननु खं हि शून्यमभाव इति यावत्~। तस्य सङ्कलनादिफलस्य सङ्ख्याधर्मत्वात्~। न च सङ्ख्यायाः शून्येन सह सङ्कलनाद्ये कर्तव्ये मा भूच्छून्ये
सङ्कलनादिफलं किन्तु सङ्ख्यायामेव तदस्तित्वमिति वाच्यम्~। एवमपि खचतुर्विधमेव सम्भवेन्न
खषड्विधं वर्गमूलयोस्तदसम्भवात्~। वस्तुतस्तु द्वितीयसङ्ख्याया
अभावात्सङ्कलनादेरप्यसम्भव एव~।
तस्य सङ्ख्याद्वयसाध्यत्वादिति~। अत्रोच्यते\textendash \,{\qt अस्त्वेव शून्यस्यापि सङ्कलनादिसम्भवः}~।
न च द्वितीयसङ्ख्याया अभावात् तदसम्भव इति वाच्यम्~। शून्यसङ्कलनादावपि
द्वितीयस-ङ्ख्यायाः सत्त्वात्~। तद्यथा\textendash \,पञ्चोत्तरशतस्य १०५ विंशत्या २० योगे कर्तव्ये स यथास्थानं कार्यः~। तत्रैकस्यां सङ्ख्यायां दशकस्थाने पञ्च~। इतरस्यां दशकस्थाने
द्वयम् एकस्थाने शून्यमिति~। अस्त्यत्र शून्यसङ्कलनेऽपि सङ्ख्याद्वयम्~। एवं
व्यवकलनादिष्वपि ज्ञेयम्~। एवं {\qt "स्थाप्योऽन्त्यवर्ग"} इत्यादिना वर्गकरणे {\qt "स्थाप्यो घनोऽन्त्यस्य ततोऽन्त्यवर्गः"} इत्यादिना घनकरणे च शून्यवर्गघनयोरपि सम्भवो द्रष्टव्यः~। ननु शून्यं किं सङ्ख्यान्तर्गतमभावो वेति व्युत्पादयन्त्यार्याः~। अस्ति ते
जिज्ञासा यदि तच्छ्रूयताम्~। सविशेषमिदं सङ्ख्याव्युत्पादनम्~। तथा हि\textendash \,इह 
किल सकलचराचरनिर्माता भगवान्परमकारुणिकः
स्वयम्भूः तत्क्रमविशेषविशिष्टवर्णमयानि \,शास्त्राणि \,सृष्ट्वाथाल्पमेधसां \,तदुपस्थितये \,मेधाविनां \,तु
तदुपस्थितिलाघवाय सति विस्मरणेऽन्यनिरपेक्षं तत्स्मरणाय चाश्रुतपरकृतग्रन्थावगमाय च
\newpage
%%%%%%%%%%%%%%%%%%%%%%%%%%%%%%%%%%%%%%%%%%%%%%%%%%%%%%%%
\noindent यथा वर्णज्ञापकलिपीः ससर्ज तथा सङ्ख्योपस्थितिलाघवाय तज्ज्ञापकानङ्कानप्यसृजत्~। तत्र प्रतिवर्णलिपिसर्गे वर्णानामियत्तया तज्ज्ञापकलिपिष्वापि
सास्तीति लिपिषु सङ्केत-ग्रहः सुशकः~। इह तु प्रतिसङ्ख्यमङ्कसर्गे
सङ्ख्यानामानं तत्तज्ज्ञापकाङ्केषु वर्षशतेनाप्यशक्यः सङ्केतग्रहः~। तथा
हि\textendash \,इह कुशाग्रबुद्धेरपि प्रतिदिनं यथाकथञ्चिच्छतपर्यन्तमपि सङ्केतग्रहे
तदेकचित्ततया शतवर्षपर्यन्तमभ्यासेन षट्त्रिंशल्लक्षपर्यन्तं सङ्केतग्रहः स्यान्मेधाविनः
न तु तदधिकसङ्ख्याज्ञापकाङ्केष्विति~। अतः परमकारुणिको भगवान् अतिचतुरो
नवैवाङ्कान् ससर्ज~। यथा १~। २~। ३~। ४~। ५~। ६~। ७~। ८~। ९~।
अथ चाभीष्टस्थानाद्वामक्रमेण द्वितीयतृतीयादिस्थानान्युत्तरोत्तरं
दशगुणानां सङ्ख्यानां सञ्ज्ञाभिर्दशशतादिभिः असङ्केतयत्~। प्रथमस्थानं
चैकगुणसङ्ख्यास्थानत्वादेकसञ्ज्ञया~। तथा सति नवैवाङ्कास्तत्र स्थानसम्बन्धात्~। स्थानानि वा तत्तदङ्कसम्बन्धाद्यथा स्वान्तान्तां सङ्ख्यां ज्ञापयेयुरिति सकलसङ्ख्यावगमः सुगम इति~। यथाभीष्टस्थाने निवेशितोऽयमङ्कः ३ एकगुणायास्त्रित्वसङ्ख्याया ज्ञापको भवति~। ततो वामतो
द्वितीयस्थाने निवेशितः स्वसङ्ख्याया दशकज्ञापको भवति~। यथा दशकद्वयज्ञापकोऽयम् २३~। एवं वामतः तृतीयचतुर्थपञ्चमादि-स्थाननिवेशितोऽङ्क उत्तरोत्तरं
दशगुणानां शतसहस्त्रायुतादीनां यथास्वं ज्ञापको भवति~।
तत्राभीष्टसङ्ख्याया यथासम्भवमेकदशकशताद्यभावे तत्स्थानपूरणार्थमभावद्योतकाङ्कः शून्यसञ्ज्ञको लिपिविशेषो निवेश्यते~। यथाष्टोत्तरशतसङ्ख्याया दशकाभावाद्द्वितीयस्थाने
शून्यनिवेशनं १०८ यथा वाष्टोत्तरसहस्रसङ्ख्यायां दशकशतकयोरभावाद्द्वितीयतृतीयस्थानयोस्तत् १००८~। अनयोरुदाहृतसङ्ख्ययोर्यथाक्रममष्टकशतकयोरष्टकसहस्रकयोरेव वानिवेशे १८ द्वितीयस्थाननिवेशितस्य दशकज्ञापकत्वादष्टादशत्वं प्रतीयेत नाभीष्टसङ्ख्या~। अत एवात्रायुतलक्षादीनामभावेऽपि तत्स्थाने
शून्यं निवेश्यते~। तेन विनाप्यभीष्टसङ्ख्याज्ञापकस्थानपूरणात्~।
अतोऽभीष्टसङ्ख्यायामुत्तरावधिभूताङ्कस्थानाद्दक्षिणस्थानानां पूरकत्वात् तत्रोक्तरीत्या
शून्यनिवेशनमावश्यकम्~। वामस्थानानां त्वपूरकत्वादानन्त्याच्च न तत्तथेति~। नन्वस्ति लिपिषु सव्यक्रमः शिष्टसम्मतो माङ्गलिकत्वादादरणीयश्च तत्कथं तमपहायापसव्यक्रम आदृत इति चेत्~। न~। शतसहस्रायुतलक्षादिसङ्ख्याया उत्तरोत्तरमभ्यर्हितत्वात्
तत्सव्यक्रमस्योचितत्वादेतत्क्रमस्य युक्तत्वात्~। न चाभ्यर्हितसङ्ख्यातः
सव्यक्रमार्थमुत्तरावधितः प्रदक्षिणक्रमेणैव द्वितीयादिस्थानानां सञ्ज्ञास्त्विति वाच्यम्~।
\afterpage{\fancyhead[CE] {बीजगणिते}}
\afterpage{\fancyhead[CO]{शून्यषड्विधम्}}
\afterpage{\fancyhead[LE,RO]{\thepage}}
\cfoot{}
 \newpage
 %%%%%%%%%%%%%%%%%%%%%%%%%%%%%%%%%%%%%%%%

\noindent उत्तरावधेरभावात्~। \,परिच्छिन्नसङ्ख्यासु \,तत्सत्त्वेऽपि \,तस्यानियतत्वात्~। \,प्रथमावधेस्तु नियतत्वात्तत्स्थानमारभ्य स्थानसञ्ज्ञायुक्ततरेत्यलं पल्लवितेन~। तदेवं
शून्यस्याभावत्वेऽपि तत्सङ्कलनादेर्न सङ्ख्याद्वयसाध्यत्वहानिः~। न हि
द्वितीयसङ्ख्याया उभयोर्वा सङ्ख्ययोर्दशकाद्यभावमात्रेण सर्वथा तदभाव इति~। वस्तुतस्तु सङ्ख्याया
दशकाद्यभावे सर्वथाप्यभावे वेत्यभावमात्रे यत् षड्विधं तत्खषड्विधम् उच्यते~।
अन्यथानन्तस्य खहरराशेः खमूलस्य चासम्भवात्~। ननु द्वितीयसङ्ख्यायाः सर्वथाप्यभावे कथं
सङ्कलनादेः सम्भवस्तस्य सङ्ख्याद्वयसाध्यत्वादित्युक्तमेवेति चेत्~। न~।
खसङ्कलनादेरतथात्वात्~। ययोः सङ्ख्यासङ्कलनादिना यस्य सङ्ख्या सम्भवति तयोरन्यतरस्योभयोर्वाभावे तस्य सङ्ख्यायाः सङ्ख्याभावस्य वा खसङ्कलनादिफलत्वात्~। यथा
शरक्रान्तिसङ्ख्ययोर्यथासम्भवं सङ्कलनेन व्यवकलनेन वा स्फुटक्रान्तिसङ्ख्या
भवतीति तयोरन्यतरस्योभयोर्वा भावे स्फुटक्रान्तेः सङ्ख्यायास्तदभावस्य वा यथास्वं
खसङ्कलनव्यवकलनफलत्वम्~। एवं खगुणनादिष्वपि बोध्यम्~। न च वस्तुतः
खषड्विधाभावे किमनेन परिभाषामात्रेणेति वाच्यम्~। अस्ति
महत्प्रयोजनमेतस्याः परिभाषायाः~। तथा हि\textendash \,यदि परिभाषा न विधीयेत तदा क्रान्तिशरयोः 
सत्त्वे तयोरेकभिन्नदिक्त्वे तत्सङ्ख्यासङ्कलनव्यवकलनाभ्यां स्फुटक्रान्तिसङ्ख्या
भवति~। एकस्यैव सत्त्वे तत्सङ्ख्यातुल्या स्फुटक्रान्तिसङ्ख्या भवति द्वयोरभावे
स्फुटक्रान्त्यभाव इति वक्तव्यं स्यात्~। एवं प्रतिपदं साधकसङ्ख्याया अभावे साध्यसङ्ख्यायाः
साधनार्थं पृथग्वचनावश्यकतया ग्रन्थगौरवं स्यात्~। खषड्विधपरिभाषायां
त्वेकभिन्नदिशोः क्रान्तिशरयोः सङ्ख्यासङ्कलनव्यवकलनाभ्यां स्फुटक्रान्तिसङ्ख्या भवतीत्येव
वक्तव्यं स्यात्~। एवं प्रतिपदं तथा सति ग्रन्थलाघवं गणितपरिच्छेदश्च स्यादिति दिक्~।
\vspace{2mm}

\begin{center}
    \Large खसङ्कलनव्यवकलने करणसूत्रम्
\end{center}

\phantomsection \label{16}
\begin{quote}
    \ab
    खयोगे वियोगे धनर्णं तथैव च्युतं शून्यतस्तद्विपर्यासमेति~॥~१६~॥
\end{quote}

अस्यार्थः~। रूपस्य यावत्तावदादिवर्णस्य करण्या वा शून्येन सहयोगे वियोगे
वा कर्तव्ये रूपादिकं धनमृणं वा तथैव भवेत्~। योगवियोगकृतो न कश्चिद्विशेष
इत्यर्थः~। अत्र खयोगो द्विविधः~। खेन योगो रूपादेः खयोग इत्येकः~। खस्य योगो
रूपादिना खयोग इति द्वितीयः~। एवं खवियोगोऽपि द्विविधः~। खेन वियोग इत्येकः~।
खाद्वियोग इति द्वितीयः~। तत्र द्वैधेऽपि खयोगे पूर्वस्मिन्खवियोगे च रूपादिकं
धनमृणं वा
\newpage
%%%%%%%%%%%%%%%%%%%%%%%%%%%%%%%%%%%%%%%%%%%%%%%%%%
\noindent यथास्थितमेव~। खाद्वियोगे विशेषमाह \hyperref[16]{\textbf{च्युतं शून्यत}} इति~। धनमृणं वा
रूपादिकं शून्यतः शोधितं सद्विपर्यासं वैपरीत्यं प्राप्नोति~। धनं
चेच्छून्यतश्च्युतमृणं भवति~। ऋणं चेद्धनं भवतीत्यर्थः~॥~१६~॥\\

\vspace{-2mm}
{\bqt उदाहरणम्\textendash \,}
\begin{quote}
    \eg
    रूपत्रयं स्वं क्षयगं च खं च~। \\
 किं स्यात् खयुक्तं वद खच्युतं च~॥~१७~॥~
\end{quote}
 
खाच्च्युतम् इति पाठः~। धनं रूपत्रयमृणं रूपत्रयं खं चैतत् त्रयमपि पृथक्
पृथक् खयुक्तं किं स्याद्वद~। खेन युक्तं खयुक्तम्~। खे युक्तं
खयुक्तमित्युदाहरणद्वयमपि द्रष्टव्यम्~। एवं खच्युतमित्यत्रापि
तृतीयापञ्चमीतत्पुरुषाभ्यामुदाहरणद्वयं द्रष्टव्यम्~।
खाद्वियोग उदाहरणमाह\textendash \,वद खाच्च्युतं चेति~। अत्र शून्यस्य धनत्व ऋणत्वे
वा न कश्चिद्विशेष इति तस्य धनर्णत्वं नोद्दिष्टम्~। न्यासः ३~। ३ं~। ०~। एतानि
खेन युक्तानि खे युक्तानि खेन च्युतानि चाविकृतान्येव~। ६~। ३ं~। ०~। अथ
खाच्छोधनार्थं न्यासः ३~। ३ं~। ०~। एतानि खाच्छोधितानि जातानि विपर्यस्तानि
३ं~। ३~। ०~। शून्यस्य विपर्यासे न कश्चिद्विशेष इति स न कृतः~।
वस्तुतस्तु खस्य धनर्णत्वं नास्त्येवाभावत्वात्~। न च सङ्ख्यागतं
योजकयोज्यत्वादिकं यथा तदभावे शून्य उपचर्यते
तद्वद्धनर्णत्वमुपचर्यतामिति वाच्यम्~। योजकयोज्यवियोजकवियोज्यगुणकगुण्यभाजकभाज्यत्वधर्माणां फले
विशेषोपलम्भात्तदुपचारस्यावश्यकत्वात्~। सङ्ख्याभावे धनर्णत्वयोस्तु फले
विशेषानुपलम्भात्तदुपचारस्य व्यर्थत्वादिति दिक्~।\\

\vspace{-4mm}
अथ खसङ्कलनव्यवकलनयोरुपपत्तिः~। इह योज्ययोजकयोरुभयोरन्यतरस्य वा
यावा-नुपचयोऽपचयो वा भवति तावानेव तत्सङ्कलनेऽपीति प्रसिद्धम्~। यथा
योज्यः ३ योजकः ४ सङ्कलनफलं ७~। अथवा योजकः ३ सङ्कलनफलं ६~। अथवा
योजकः २ सङ्कलनफलं ५~। योजकः १ फलं ४~। एवं योजकः ० योज्यः ३~।
अत्र योजकसङ्ख्यायां यावानपचयस्तावानेव सङ्कलनफलेऽप्युपलभ्यत इति
योजकतुल्ये योजकापचये सङ्कलनफलेऽपि योजकतुल्येनापचयेन भाव्यम्~। तथा सति
\newpage
%%%%%%%%%%%%%%%%%%%%%%%%%%%%%%%%%%%%%%%%

\noindent योज्यतुल्यमेव सङ्कलनफलं स्यादिति खेन योगेऽविकृतो राशिः~। एवं
योज्यापचयवशादपि सङ्कलनफलापचयाद्योज्यतुल्ये योज्यापचये सङ्कलनफलेऽपि तावतैवापचयेन
भाव्यमिति योजकसङ्ख्यातुल्यमेव सङ्कलनफलं स्यादिति खस्य योगेऽप्यविकृतो
राशिः एवमुभयापचयवशेन शून्ययोः सङ्कलनफलं शून्यमिति द्रष्टव्यम्~। अथ वियोज्यसङ्ख्यायां वियोजकसङ्ख्यातुल्येऽपचये व्यवकलनफलं भवति~। तत्र
वियोजकसङ्ख्याया यावानपचयस्तावानेवोपचयो व्यवकलनफले भवतीति वियोजकतुल्ये वियोजकापचये सति व्यवकलनफले वियोजकतुल्येनोपचयेन भाव्यम् इति
वियोज्यसङ्ख्यातुल्यं व्यवकलनफलं स्यादतः खेन वियोगेऽविकृतो राशिः~। अथ
वियोज्ये यथा यथापचयो भवति तथा तथा व्यवकलनफलेऽप्यस्तीति प्रसिद्धम्~।
यथा वियोज्यः ५ वियोजकः ३ व्यवकलनफलं २~। अथ वियोज्यः \,४~। \,व्यवकलनफलं \,१~। वियोज्यः \,३ \,व्यवकलनफलं \,०~। अथ \,वियोज्यः \,२
अत्रापि व्यवकलनफलेनैकोनेन भाव्यम्~। तथा सति व्यवकलनफलं १ं~। अथ वियोज्यः १ उक्तवद्व्यवकलनफलं २ं~। वियोज्यः ० उक्तवद्व्यवकलनफलेन ३ं भाव्य-मित्युपपन्नम्~। \hyperref[16]{\textbf{च्युतं शून्यतस्तद्विपर्यासमेति}} इति~। एवं योज्ययोजकयोर्वियोजकयोश्च धनत्वं प्रकल्प्य यथा युक्तिरुक्ता तथोभयोरृणत्वमपि प्रकल्प्य द्रष्टव्या एकस्य धनत्वमितरस्यर्णत्वमिति कल्पने
तूपचयापचययोरन्यथात्वेनोपपत्तिर्द्रष्टव्येत्यलं
पल्लवितेन~॥~१७~॥\\

\vspace{-2mm}
{\bqt खगुणादिषु करणसूत्रम्\textendash }

\phantomsection \label{18}
\begin{quote}
    \ab
    वधादौ वियत् खस्य खं खेन घाते\\
 खहारो भवेत्खेन भक्तश्च राशिः~॥~१८~॥
\end{quote}
 
यथा पूर्वं खयोगवियोगयोर्द्वैविध्यम् उक्तं तथा खगुणनभजनयोरपि
द्वैविध्यमस्ति~। खस्येति खेनेति च~। वर्गादिषु तु खस्येत्येक एव प्रकारः
सम्भवति वर्गादिकरणे द्विती-यसङ्ख्यानपेक्षणात्~। तत्र खस्येति प्रकारेष्वाह~। वधादौ वियत्\textendash \,खस्येति~। \hyperref[18]{\textbf{खस्य}} शून्यस्य \hyperref[18]{\textbf{वधादौ}} गुणनभजनवर्गतन्मूलादिषु
कर्तव्येषु वियत् स्यात्~। गुणनफलादिकं शून्यं भवे-दित्यर्थः~। खेनेति~।
गुणनप्रकारेण फलमाह~। \hyperref[18]{\textbf{खं तेन घात}} इति~। खेन शून्येन घाते
\newpage
%%%%%%%%%%%%%%%%%%%%%%%%%%%%%%%%%%%%%%%%%%%%%%%%%%%%%%%%% २३
\noindent कस्यचिदङ्कस्य \,गुणने \,गुणनफलं \,खं \,स्यात्~। अत्र {\qt "खगुणश्चिन्त्यश्च शेषविधौ"} इत्यादि पाटीस्थो विशेषो द्रष्टव्यः~। अन्यथा {\qt "त्रिभज्यकोन्मण्डलशङ्कुघातात्"} इत्यादिना यष्ट्यानयनेन गोलसन्धौ यष्ट्यभावापत्तेरिति
दिक्~। खेनेति भजनप्रकारे फलमाह~। खहारो भवेत् खेन भक्तश्च
राशिः~। इति~। खेन भक्तो राशिः खहारो भवेत्~। खं हारो यस्येति
खहारोऽनन्त इत्यर्थः~। उदाहरणावसरे वक्ष्यति च अयमनन्तो राशिः खहर
उच्यत इति~। अत्रोपपत्तिः~। गुण्यस्यापचयवशाद्गुणनफलस्यापचय इति तावत्
प्रसिद्धम्~। यथा गुणकः १२ गुण्यः ४ गुणनफलं ४८~। अथवा गुण्यः
३ गुणनफलं ३६ वा गुण्यः २ गुणनफलं २४~। वा गुण्यः १
गुणनफलं १२ वा गुण्यः $\begin{matrix}
\vspace{-2mm}
\mbox{{१}}\\
\vspace{-1.5mm}
\mbox{{२}}
\vspace{1mm}
\end{matrix}$ गुणनफलं ६~। वा गुण्यः $\begin{matrix}
\vspace{-2mm}
\mbox{{१}}\\
\vspace{-1.5mm}
\mbox{{४}}
\vspace{1mm}
\end{matrix}$ गुणनफलं ३ वा गुण्यः $\begin{matrix}
\vspace{-2mm}
\mbox{{१}}\\
\vspace{-1.5mm}
\mbox{{१२}}
\vspace{1mm}
\end{matrix}$ गुणनफलं १ इति~। अनयैव युक्त्या गुण्यस्य परमापचये
गुणनफलस्यापि परमापचयेन भाव्यम्~। परमापचये च शून्यतैव पर्यवस्यतीति
शून्ये गुण्ये गुणनफलं शून्यमेवेति सिद्धम्~। यद्वा गुण्य
एकैकापचये गुणनफले गुणकतुल्योऽपचयो भवति~। यथा गुणकः ८ गुण्यः ४
गुणनफलं ३२ एकोनो गुण्यः ३ गुणनफलं २४~। पुनरेकोनो गुण्यः २
गुणनफलं १६ पुनरेकोनो गुण्यः १ गुणनफलं ८ पुनरेकोनो गुण्यः ०
अत्रापि गुणनफले गुणकतुल्येनापचयेन भाव्यम्~। तथा सति गुणनफलं
शून्यतैव सिद्धा~। एवं गुणकापचयवशादपि गुणनफलेऽपचयाद्गुणकस्यापि
शून्यत्वे गुणनफलं शून्यमेवेति सिद्धम्~। ननु गुणकवैलक्षण्यादेकस्मिन्नपि
गुण्ये गुणनफलवैचित्र्यमस्ति तत्कथं शून्ये गुण्ये गुणकवैलक्षण्येऽपि गुणनफलं
शून्यमेवेति चेत्~। न~। अप्रयोजकत्वात्~। अन्यथैकातिरिक्तसङ्ख्याया
वर्गवर्गमूलघनघनमूलादीनां वैलक्षण्यव्याप्तेरेकसङ्ख्याया अपि तेषां वैलक्षण्यापत्तेः~।
वस्तुतस्तु गुणको ह्यावर्तकः~। सति गुण्ये गुण्यस्य गुणकतुल्यावर्तनाद्गुणनफलं भवतीति
गुणकवैचित्र्येऽस्ति गुणनफलवैचित्र्यम्~। इह त्वावर्तनीयस्य गुणनकसहस्रमपि
कमावर्तर्यदिति गुणनफलस्याप्याभाव इति~। एवं भाज्यापचयवशाद्भजनकलापचयोऽस्तीति
भाज्ये शून्ये भजनफलं शून्यमिति पूर्वयुक्त्यैव सिद्धम्~। वर्गादेश्च
द्वितीयसङ्ख्यानिरपेक्षत्वाद्वर्गादिसङ्ख्याया अभावाच्चाभाव इति स्पष्टम्~।
तदेवमुपपन्नम् \hyperref[18]{\textbf{"वधादौ वियत्खस्य खं खेन घाते"}} इति खहरोपपत्तिस्तूदाहरणे वक्ष्यते~॥~१८~॥
 \newpage %%%%%%%%%%%%%%%%%%%%%%%%%%%%%%%%%%%%%%%%

{\bqt उदाहरणम्\textendash \,}
\begin{quote}
   \eg
    द्विघ्नं त्रिहृत्खं खहृतं त्रयं च शून्यस्य वर्गं वद मे पदं च~॥~१९~॥
\end{quote}

अत्र द्वाभ्यां हन्यते तद्विघ्नमिति व्युत्पत्त्या गुण्ये द्वौ हन्तीति
व्युत्पत्त्या शून्ये गुणके च पृथगुदाहरणं द्रष्टव्यम्~। शेषं स्पष्टम्~।
प्रथमे न्यासः~। गुणकः २ गुण्यः ० गुणनफलं वधादौ वियत्खस्येति जातं ०~। 
द्वितीये न्यासः~। गुणकः ० गुण्यः २ खं खेन घात इति जातं ०~। अथ भागहारे
प्रथमोदाहरणे न्यासः~। भाजकः ३ भाज्यः ० वधादौ वियत्खस्येति जातं
भजनफलं ० द्वितीये न्यासः भाजकः ३ भाज्यः ३ खहारो भवेत्खेन भक्तश्च
राशिरिति जातः खहरः ३~। ननु यो राशिर्येन ह्रियते स तद्धरो भवतीति
राशेः खेन हरणे खहरो भवेदिति स्पष्टमेवास्ति~। किन्तु खेन राशौ हृते का
लब्धिरिति प्रश्नस्य किमुत्तरमित्यत आह~। अयमनन्तो राशिः खहर
इत्युच्यत इति~। लब्धिरनन्तेत्युत्तरमिति भावः~। एतस्यानन्तत्वे ह्येषा
युक्तिस्त्वस्ति~। यथा यथा भाजकापचयस्तथा तथा लब्धेरुपचयः~। तथा
सति भाजकाङ्के परमापचिते लब्धेः परीमोपचयेन भाव्यम् लब्धेश्चेदियत्तोच्येत
तर्हि परमत्वं न स्यात्ततोऽप्याधिक्यसम्भवात्~। अतो लब्धेरियत्ताभाव एव
परमत्वम्~। तदेवमुपपन्नं खहरो राशिरनन्त इति~॥~१९~॥\\

\vspace{-4mm}
अथानन्तपदसञ्जातभगवत्स्मृतिः भागवतोत्तमः श्रीभास्कराचार्यः प्रसङ्गेनापि
स्तुतो हरिः कृतार्थतां सम्पादयतीति दृढनिश्चयः खहरराशेरविकारतादृष्टान्तप्रसङ्गेन श्रीभगवन्तमनन्तं स्तौति\textendash \,
\begin{quote}
    \ab
    अस्मिन्विकारः खहरेण राशावपि प्रविष्टेष्वपि निःसृतेषु~। \\
बहुष्वपि स्याल्लयसृष्टिकालेऽनन्तेऽच्युते भूतगणेषु यद्वत्~॥~२०~॥
\end{quote}

उपजातिकेयम् अस्यार्थः प्रलयकाले श्रीभगवत्यनन्तेऽच्युते बहुष्वपि
भूतगणेषु प्रविष्टेषु लीनेष्वपि वा निःसृतेषु देहादिमत्तया भगवतोऽनन्तात्
पृथग्भूतेष्वपि
\newpage
%%%%%%%%%%%%%%%%%%%%%%%%%%%%%%%%%%%%%%%%%%%%%%%
\noindent यद्वद्विकारो नास्ति न हि तेषु प्रविष्टेषु महान्भवति निःसृतेषु वा
लघुर्भवति तथास्मिन्खहरे राशावपि बहुष्वपि राशिषु प्रविष्टेषु निःसृतेषु वा विकारो नास्तीति~।
ननु कथं विकारो नास्ति~। न हीशनिदेशः~। योगे वियोगे वाविकारस्य व्याप्तिसिद्धिः
स्यात्~। सत्यम्~। सर्वत्र योगोऽन्तरं वा समच्छेदत्वे भवति~। प्रकृतेऽपि समच्छेदतां
विधायैव योगोऽन्तरं वा विधेयम्~। समच्छेदता च\textendash 
\vspace{-2mm}

\begin{center}
    \q अन्योन्यहाराभिहतौ हरांशौ
\end{center}
\vspace{-2mm}

इत्यनेन~। तथा च खहरस्य राशेर्हरेण शून्येनापरराशौ गुणिते शून्यमेव भवेत्~।
शून्ययोगवियोगयोश्चाविकृतत्वं पूर्वमेवोक्तम्~। ननु यद्यप्यभिन्नराशिना
योगान्तरयोर्भवत्यविकृतत्वं तथापि भिन्नराशिना योगेऽन्तरे च त्वदुक्तरीत्या भवेदेव
विकारः~। यथा $\begin{matrix}
\vspace{-2mm}
\mbox{{३}}\\
\vspace{-1.5mm}
\mbox{{०}}
\vspace{1mm}
\end{matrix}$~। $\begin{matrix}
\vspace{-2mm}
\mbox{{१}}\\
\vspace{-1.5mm}
\mbox{{३}}
\vspace{1mm}
\end{matrix}$~। अन्योन्यहाराभिहतौ हरांशाविति जातौ तुल्यहरौ $\begin{matrix}
\vspace{-2mm}
\mbox{{९}}\\
\vspace{-1.5mm}
\mbox{{०}}
\vspace{1mm}
\end{matrix}$~। $\begin{matrix}
\vspace{-2mm}
\mbox{{०}}\\
\vspace{-1.5mm}
\mbox{{०}}
\vspace{1mm}
\end{matrix}$~। अनयोर्योगे जातं $\begin{matrix}
\vspace{-2mm}
\mbox{{९}}\\
\vspace{-1.5mm}
\mbox{{०}}
\vspace{1mm}
\end{matrix}$~। अथ यद्युच्येतैकस्य हरेण येन केनचिदङ्केन
वापरराशिहरांशगुणनमात्रेण
तुल्यहरत्वे जाते परतः श्रमो व्यर्थ एव~। प्रकृतेऽपि खहरराशेर्हरणे
शून्येनापरराशि\textendash \,$\begin{matrix}
\vspace{-2mm}
\mbox{{१}}\\
\vspace{-1.5mm}
\mbox{{३}}
\vspace{1mm}
\end{matrix}$\textendash \,हरांशगुणनमा-त्रेण तुल्यहरस्य जातत्वाद्योगेऽन्तरे च नास्त्येव विकार इति~। तर्हि खहरस्य खहरेण योगेऽन्तरे च भवेदेव विकारः~। यथा राशी
$\begin{matrix}
\vspace{-2mm}
\mbox{{३}}\\
\vspace{-1.5mm}
\mbox{{०}}
\vspace{1mm}
\end{matrix}$~। $\begin{matrix}
\vspace{-2mm}
\mbox{{५}}\\
\vspace{-1.5mm}
\mbox{{०}}
\vspace{1mm}
\end{matrix}$~। अनयोस्तुल्यहरत्वाद्योगे जातं $\begin{matrix}
\vspace{-2mm}
\mbox{{८}}\\
\vspace{-1.5mm}
\mbox{{०}}
\vspace{1mm}
\end{matrix}$~। तत्कथं न विकार इति चेत्~। मैवम्~।
अत्रापि फलतो विकाराभावात्~। न हि खेन भक्तेषु त्रिष्वन्यत्फलमष्टसु
भक्तेष्वितरदिति किं तूभयत्राप्यनन्तत्वे न व्यभिचार इति~। यथोदयकाले न्यूनाधिकपरिमाणयोरपि
शङ्कोश्छायानन्त्यं न व्यभिचरति तथा वर्तमानेऽस्मिन्काले भूते भविष्यति
च गतकल्पसङ्ख्याया न्यूनाधिकभावेऽप्यनन्तत्वाव्यभिचारः~।
किञ्चोन्नतांशजीवास्वरूपे शङ्कौ यदि दृग्ज्याभुजस्तदेष्टे द्वादशाङ्गुलादिके शङ्कौ किमिति
त्रैराशिकेन च्छाया सिध्यति~। तत्रोदयकाल उन्नतजीवाया अभावः~। दृग्ज्या च त्रिज्यामिता १२० अत्र
द्वित्रिचतुरङ्गुलादीनां शङ्कूनामुक्तत्रैराशिकेन छायासाधने २४०~। ३६०~। ४८०~।
एतदाद्याः सिध्यन्ति खहराश्छायाः~। न ह्येतासु फलतो वैलक्षण्यमस्ति~। यतस्तस्मिन्नपि काले
\newpage 
%%%%%%%%%%%%%%%%%%%%%%%%%%%%%%%%%%%%%%%%

\noindent न्यूनाधिकपरिमाणानामपि शङ्कूनां छायानन्त्यं न व्यभिचरति~। किं चोदयकाल
एव ३४३८~। १२०~। १००~। ९० आभ्यस्त्रिज्याभ्यः प्राग्वदनुपातेन
द्वादशाङ्गुलशङ्कोश्छायाः $\begin{matrix}
\vspace{-2mm}
\mbox{{४१२५६}}\\
\vspace{-1.5mm}
\mbox{{०}}
\vspace{1mm}
\end{matrix}$~। $\begin{matrix}
\vspace{-2mm}
\mbox{{१४४०}}\\
\vspace{-1.5mm}
\mbox{{०}}
\vspace{1mm}
\end{matrix}$~। $\begin{matrix}
\vspace{-2mm}
\mbox{{१२००}}\\
\vspace{-1.5mm}
\mbox{{०}}
\vspace{1mm}
\end{matrix}$~। $\begin{matrix}
\vspace{-2mm}
\mbox{{१०८०}}\\
\vspace{-1.5mm}
\mbox{{०}}
\vspace{1mm}
\end{matrix}$~। न ह्यासां भेदः सम्भाव्यते~। नहि त्रिज्याभेदप्रयुक्तश्छायाभेदः~। किन्तु
नानात्रिज्याभ्योऽनुपातसिद्धा छाया
तुल्यैवेति सकलगणकानाम् अविवाद इति सर्वमवदातम्~। एवं मतिमद्भिरन्यदप्यूह्यम्~।
शून्यस्य वर्गः ० वर्गमूलं ०~। एवं घनादिष्वपि शून्यतैव~॥~२०~॥

\begin{quote}
{\qt दैवज्ञवर्यगणसन्ततसेव्यपार्श्वबल्लालसञ्ज्ञगणकात्मजनिर्मितेऽस्मिन्~।\\
बीजक्रियाविवृतिकल्पलतावतारे व्यक्तिः क्रमादभवदम्बरषड्विधस्य~॥}
\end{quote}

\begin{center}
इति श्रीसकलगणकसार्वभौमश्रीबल्लालदैवज्ञसुतकृष्णदैवज्ञविरचिते \\ बीजविवृतिकल्पलतावतारे खषड्विधविवरणम्~॥~\\
(अत्र ग्रन्थसङ्ख्या पादोनशतद्वयम् १७५)~।\\
\vspace{1.5cm}
\rule{0.2\linewidth}{0.5pt}
\end{center}
\newpage
%%%%%%%%%%%%%%%%%%%%%%%%%%%%%%%%%%%%%%%%%%%%%%%%%%%%%%%%%%%%%%%%%%%
\phantomsection \label{ch3}
\begin{center}
{\LARGE \textbf{३ वर्णषड्विधम्~।}}\\
\end{center}

अथ यद्यपि करणीषड्विधं
निरूपणीयमित्युक्तत्वादुक्तषड्विधस्यान्तरङ्गत्वमिति प्रथमतस्तन्निरूप्य बहिरङ्गमव्यक्तषड्विधं पश्चान्निरूपणीयमिति युक्तम्~।
तथापि करणीषड्विधस्या-तिकठिनतया तन्निरूपणे प्रयासाबाहुल्यादव्यक्तषड्विधनिरूपणे
च प्रयासलाघवात्सूचीकटाहन्यायेनाव्यक्तषड्विधं प्रथमतो निरूपयति~। तत्र
द्वित्र्यादीनां राशीनामव्यक्तत्वे सञ्ज्ञाभेदमन्तरेण तत्सङ्करः
स्यादतस्तन्निरासार्थमव्यक्तसञ्ज्ञाः शालिन्याह\textendash 

\phantomsection \label{21}
\begin{quote}
    \ab
    यावत्तावत्कालको नीलकोऽन्यो \\

\vspace{-6mm}
\hspace{0.6cm}वर्णः पीतो लोहितश्चैतदाद्याः~। \\

\vspace{-6mm}
अव्यक्तानां कल्पिता मानसञ्ज्ञाः  \\

\vspace{-6mm}
\hspace{0.6cm}तत्सङ्ख्यानं कर्तुमाचार्यवर्यैः~॥~२१~॥
\end{quote}

यावत्तावादित्येकं नाम~। कालकः २ नीलकः ३ पीतः ४ लोहितः ५
\hyperref[21]{\textbf{एतदाद्या}} हरित-श्वेतकचित्रकादयोऽनेकवर्णसमीकरणपठिता वर्णा \hyperref[21]{\textbf{अव्यक्तानाम्}} अज्ञातराशीनां मानसञ्ज्ञा आचार्यवर्यैः कल्पिताः~।
नामकल्पने प्रयोजनमाह\textendash \,\hyperref[21]{\textbf{तत्सङ्ख्यानं कर्तुम्}} इति~। तेषामज्ञातराशीनां सङ्ख्यानं गणनां कर्तुं साधयितुं ज्ञातुमिति यावत्~॥~२१~॥\\

\vspace{-2mm}
{\bqt अव्यक्तसङ्कलनव्यवकलने करणसूत्रम्\textendash \,}

\phantomsection \label{22}
\begin{quote}
    \ab
    योगोऽन्तरं तेषु समानजात्योर्विभिन्नजात्योश्च पृथक्स्थितिश्च~॥~२२~॥
\end{quote}

तेषु वर्णेषु मध्ये रूपेष्वित्यपि द्रष्टव्यम्~। \hyperref[22]{\textbf{समानजात्योः}} समानैका
जातिर्ययोस्तौ
\thispagestyle{empty}
\afterpage{\fancyhead[CE] {बीजगणिते}}
\afterpage{\fancyhead[CO]{वर्णषड्विधम्}}
\afterpage{\fancyhead[LE,RO]{\thepage}}
\cfoot{}
\newpage %%%%%%%%%%%%%%%%%%%%%%%%%%%%%%%%%%%%%%%%
\noindent तथा तयोः समानजात्योः पूर्वोक्तो योगोऽन्तरं च स्यात्~। अत्र स्यादिति
पदमुत्तरदलस्थमन्वेति देहलीदीपन्यायेन~। "पृथक् स्थितिः स्यात्" इति वा पाठः~।
समानजात्योरित्युपलक्षणं समानजातीनामित्यपि द्रष्टव्यम्~। यद्वा बहूनामपि योगे
द्वयोर्योगस्यैव मुख्यत्वाद्युगपत्सर्वयोगस्य कर्तुमशक्यत्वाद्द्विवचनम्~। जातिश्चात्र
रूपत्वम्~। यावत्तावत्त्वम्~। कालकत्वम्~। नीलकत्वम्~। यावत्तावद्वर्गत्वम्~। यावत्तावद्घनत्वम्~। यावत्तावद्वर्गव-र्गत्वं च यावत्तावत्कालकभावितत्वमित्यादिर्योज्ययोजकनिष्ठसकलजातिव्याप्या
योज्ययोजकनिष्ठा च~। न त्वङ्कत्वं वर्णत्वं वा~। अङ्कत्वोक्तौ विशेषणवैयर्थ्यापत्तिः~।
व्यावर्त्याभावात्~। वर्णत्वोक्तौ वर्णकल्पनानर्थक्यप्रसङ्गः~। असङ्करार्थं हि वर्णकल्पना~।
वर्णत्वाजात्या साजात्ये विवक्षिते सङ्कर एव स्यात्~। तस्मादुक्तविधजात्यैव साजात्यं
विवक्षितम्~। यद्वा समानशब्दस्य तुल्यार्थत्वाद्योज्ययोजकयोः स्वस्वनिष्ठसकलजातिभिः
साजात्यं विवक्षितं विभिन्नजात्योश्च~। चस्त्वर्थे~। विभिन्ना जातिर्ययोस्तयोर्वा
योगेऽन्तरे वा क्रियमाणे पृथक् स्थितिश्च~। चोऽवधारणे~। पृथक् स्थितिरेव स्यादित्यर्थः~।
एतदुक्तं भवति~। रूपस्य रूपेण यावत्तावतो यावत्तावता कालकस्य कालकेन कालकवर्गस्य
कालकवर्गेण कालकघनस्य कालकघनेन कालकनीलकभावितस्य तद्भावितेन~। एवं
समानजात्योर्योगेऽन्तरे वा कर्तव्ये योगोऽन्तरं वोक्तवद्भवति~। रूपस्य
यावत्तावता कालकादिभिर्वा यावत्तावतः कालकादिभिर्यावत्तावतो यावद्वर्गेण यावद्घनस्य
यावता तद्वर्गेण वा भावितादिभिर्वा~। एवं विभिन्नजात्योर्योगेऽन्तरे वा कर्तव्ये
पृथक्स्थितिरेव~। अत्रैकपङ्क्ताविति~। द्रष्टव्यम्~। अन्यथा
योगान्तज्ञापकाभावादिति~।
अत्रोपपत्तिस्तु व्यक्ते प्रसिद्धैव~। अन्यथा समच्छेदविधानपूर्वकं
योगान्तरकथनं न स्यात्~। किञ्च विभिन्नजात्योर्योगः किमात्मकः~। यथा राशिद्वयमंशपञ्चकं
चेत्यनयोर्विभिन्नजात्योरपि योगः क्रियेत तर्हि सप्त स्युः~। ते सप्त न राशयो न वा
लवाः~। नहि ग्रहेण राशिद्वयमंशपञ्चकं भुक्तमित्युक्ते ग्रहेण सप्त राशयः सप्त लवा
वा भुक्ता इति कस्यापि प्रतीतिरस्त्युपपद्यते वा किन्तु ग्रहेण
कियद्भुक्तमस्तीति प्रश्ने राशिद्वयमंशपञ्चकं च भुक्तमित्युत्तरस्य सर्वसम्प्रतिपन्नत्वाद्युक्तत्वाच्च
पृथक्स्थितिरेव युक्ता~।
अत्रैव साजात्ये योगो भवत्येव~। यथा\textendash \,राशिद्वयस्य लवाः ६० पञ्चभिर्लवैः ५ योगे जाताः पञ्चषष्टिर्लवाः ६५~। ग्रहेण राशिद्वयमंशपञ्चकं च भुक्तमित्युक्ते
पञ्चषष्टिर्लवा भुक्ता इत्यस्त्येव प्रतीतिः सर्वसम्मतेत्यादि सुधीभिरूह्यम्~॥~२२~॥\\

\vspace{-3mm}
नन्वेवं वर्णेष्वपि साजात्यं विधाय योगोऽस्त्विति चेन्न~।
वर्णमानानामज्ञातत्वात्साजा-
\newpage
%%%%%%%%%%%%%%%%%%%%%%%%%%%%%%%%%%%%%%%%
\noindent त्यविधानस्याशक्यत्वात्~। अत एव तन्मानोत्थापनानन्तरं साजात्येन योगो भवत्येव~। एवमेव वियोगेऽप्युपपत्तिर्द्रष्टव्या~। अत्रोदाहरणानि
भुजङ्गप्रयातेनाह\textendash \,
\begin{quote}
    {\eg
    स्वमव्यक्तमेकं सखे सैकरूपं \\
    धनाव्यक्तयुग्मं विरूपाष्टकं च~। \\
युतौ पक्षयोरेतयोः किं धनर्णे \\
विपर्यस्य चैक्ये भवेत् किं वदाशु~॥~२३~॥}
\end{quote}

एकस्य रूपसहितकं धनमव्यक्तम् इत्येकः~। रूपाष्टकरहितं धनमव्यक्तयुग्मम् इति द्वितीयः~। एतयोः पक्षयोर्युतौ किं फलं स्यात्~। अथ च पक्षयोर्धनर्णे विपर्यस्यैक्ये
किं फलं स्यात् इति~। अत्र पूर्वपक्षमात्रव्यत्यासादुत्तरपक्षमात्रव्यत्यासादुभयपक्षव्यत्यासात् च प्रश्नत्रयम्~। व्यत्यासाभावे \,चैकम् \,इत्युदाहरणचतुष्टयम्~। धनर्णे \,इत्यत्र \,भावप्रधानो निर्देशः~। यद्वाव्यक्ते रूपे इत्यध्याहार्य योजना द्रष्टव्या~। एकमव्यक्तमिदं १ या १ एकं रुपमिदम्~।  रू १ अनयोर्योगे द्वयं न भवति~। भिन्नजातित्वात्~। किन्तु पङ्क्तौ पृथक्
स्थितिरेवेति जात एकः पक्षः~। या १ रू १ एवं धनाव्यक्तयुग्मं या २ अस्माद्रूपाष्टके
शोध्यमाने \hyperref[7]{\textbf{संशोध्यमानं स्वमृणत्वमेती}}ति जातमृणं रूपाष्टकं ८ अनयो\hyperref[3]{\textbf{र्धनर्णयोरन्तरमेव योग}} इति ऋणगताः षट् ६ं न भवन्ति~। किञ्चैकपङ्क्तौ पृथक् स्थितिरेव तथा कृते जातो द्वितीयः पक्षः~। या २ रू ८ं योगार्थमुभयोर्न्यासो $\begin{matrix}
\vspace{-1mm}
\mbox{{या १ रू १}}\\
\vspace{-1mm}
\mbox{{या २ रू ८ं}}
\vspace{1mm}
\end{matrix}$~। अनयोर्योगे कर्तव्ये समानजात्योरेव योग इति~। अव्यक्तव्यक्तेन रूपं रूपेण च संयोज्यम्~। तथा कृते जातं या ३~। रू ७ं आद्यपक्षे धनर्णत्वे विपर्यस्य न्यासः $\begin{matrix}
\vspace{-1mm}
\mbox{{या १ं रू १ं}}\\
\vspace{-1mm}
\mbox{{या २ रू ८ं}}
\vspace{1mm}
\end{matrix}$~। अनयोरुक्तवद्योगे जातं या १ रू ९ं~। द्वितीयपक्षव्यत्यासे न्यासो $\begin{matrix}
\vspace{-1mm}
\mbox{{या १ रू १}}\\
\vspace{-1mm}
\mbox{{या २ं रू ८}}
\vspace{1mm}
\end{matrix}$~। योगे जातं या १ं रू ९ उभयपक्षधनर्णव्यत्यासे न्यासो
$\begin{matrix}
\vspace{-1mm}
\mbox{{या १ं रू १ं}}\\
\vspace{-1mm}
\mbox{{या २ं रू ८}}
\vspace{1mm}
\end{matrix}$~। योगे जातं या ३ं रू ७~। एवं
द्वयोर्भिन्नजातित्वे सत्युदाहरणान्युक्तानि~॥~२३~॥~\\

\vspace{-2mm}
{\bqt उदाहरणम्}
\begin{quote}
    \eg
    धनाव्यक्तवर्गत्रयं सत्रिरूपं क्षयाव्यक्तयुग्मेन युक्तं च किं स्यात्~॥~२५~॥
\end{quote}
\newpage %%%%%%%%%%%%%%%%%%%%%%%%%%%%%%%%%%%%%%%%
त्रिभी रूपैः सहितं धनमव्यक्तवर्गत्रयं सत्रिरूपं याव ३ रू ३ अयं पक्ष
ऋणाव्यक्तयुग्मेनानेन या २ं योज्यः~। इदमव्यक्तयुग्मं न वर्गैः संयुज्यते
नापि रूपैः~। भिन्नजातित्वात्~। तस्मात्पङ्क्तौ पृथक्स्थितिरेव~। तत्र क्रमस्तु\textendash \,आदौ
वर्गधनस्य~। ततो वर्गवर्गस्य~। ततो धनस्य~। ततो वर्गस्य ततोऽव्यक्तस्य ततो
रूपाणामित्यादिः~। तथा स्थितौ जातम्~। याव ३ या २ं रू ३ एवं कालकादिष्वपि बोद्धव्यम्~॥~२४~॥\\ 

\vspace{-2mm}
{\bqt उदाहरणम्}
\begin{quote}
    \eg
    धनाव्यक्तयुग्मादृणाव्यक्तषट्कं सरूपाष्टकं प्रोज्झ्य शेषं वदाशु~॥~२५~॥
\end{quote}

स्पष्टोऽर्थः~। अथ न्यासः~। सरूपाष्टकमृणाव्यक्तषट्कमुक्तवज्जातम्~। या ६ं
रू ८ एतद्धनाव्यक्तयुग्मादस्मात् या २ विशोध्यम्~। तत्र \hyperref[7]{\textbf{संशोध्यमानं स्वमृणत्वमेती}}त्यादिना जातः शोध्यपक्षः~। या ६ रू ८~। एतन्मध्ये व्यक्तमेव सजातीयत्वादव्यक्ते योज्यम्~। रूपाणां पृथक्स्थितिरेवेति तथा कृते जातम्~। या ८ रू ८ं~॥~२५~॥~\\

\vspace{-2mm}
{\bqt अव्यक्तादिगुणने करणसूत्रम्\textendash \,}

\phantomsection \label{26}
\begin{quote}
    \ab
    स्याद्रूपवर्णाभिहतौ तु वर्णो द्वित्र्यादिकानां समजातिकानाम्~॥\\
वधे तु तद्वर्गघनादयः स्युस्तद्भावितं चासमजातिघाते~।\\
भागादिकं रूपवदेव शेषं व्यक्ते यदुक्तं गणिते तदत्र~॥~२६~॥
\end{quote}

अस्यार्थः\textendash \,वर्णगुणनं त्रिधैव सम्भवति~। रूपेण सजातीयवर्णेन विजातीयवर्णेन वा~। तत्र रूपेण गुणने~। \hyperref[26]{\textbf{स्याद्रूपवर्णाभिहतौ तु वर्णः}} इति~। रूपवर्णाभिहतौ तु
वर्णः स्यात्~। अयमर्थः~। रूपेण वर्णे गुणनीये वर्णेन वा रूपे गुणनीयेऽङ्कतस्तु
गुणनफलं भवति~। नाम तु वर्णस्यैव~। अथ सजातीयवर्णेन गुणने समजातिकानां
द्वित्र्यादिकानां वर्णानां वधे तु तद्वर्गघनादयः स्युः~। एतदुक्तं भवति~। यावत्तावता
यावत्तावति गुणिते समजात्योर्द्वयोर्घात इति यावत्तावद्वर्गः स्यात्~। स
चेत्पुनर्यावत्तावता गुण्यते तदा समत्रिघातत्वाद्यावत्तावद्घनः स्यात्~। अयमपि चेत्तेन गुण्यते तदा समचतुर्घातत्वाद्यावत्तावद्वर्गवर्गो भवेत्~। असावपि तेन गुणितश्चेत्पञ्चघातत्वाद्यावद्वर्गघनयोर्घातः~।\\

\vspace{-4mm}
एवं षड्घाते यावद्वर्गघनो यावद्घनवर्गो वा भवेत्~। इत्यादि~। कालकादीनामपि सम- 
\newpage
%%%%%%%%%%%%%%%%%%%%%%%%%%%%%%%%%%%
\noindent द्वित्र्यादिवधे कालकादिवर्गघनादयो ज्ञेयाः~। अथ विजातीयवर्णेन गुणने~।
असमजातिघाते तद्भावितं स्यादिति~। विजातीयवर्णयोर्घाते
तयोर्वर्णयोर्भावितं स्यात्~।
यथा\textendash \,यावता कालके गुणिते यावत्कालकभावितं भवति~। कालकेन नीलके गुणिते
कालकनीलकभावितं भवतीत्यादि~। यावत्कालकभावितं यदि कालकेन गुण्यते तदा
यावत्कालकवर्गभावितं भवति~। इदमपि यदि यावत्तावता गुण्यते तदा
यावद्वर्गकालकवर्गभावितं भवतीत्यादि सुधीभिरूह्यम्~। एवं गुणने विशेषमुक्त्वा
भागादिकमाह\textendash \,शेषं भागादिकं भागवर्गमूलघनमूलादि यद्व्यक्ते गणित उक्तं तदत्र
रूपवज्ज्ञेयम्~। {\qt "भाज्याद्धरः शुध्यति यद्गुणः"} इत्यादिना भजनफलं ज्ञेयम्~। {\qt "समद्विघातः कृतिरुच्यत"} इत्यादिना वर्गो ज्ञेय इत्यादि~। भागादिकानां
गुणनपूर्वकत्वाद्गुणनसञ्ज्ञाविशेषस्य चोक्तत्वात्तत्र कोऽपि विशेषो वक्तव्यो नास्तीति भावः~।
इदमुपलक्षणम्~। अत्रासङ्करार्थं गुणनफलसञ्ज्ञामात्रमुक्तम्~। अङ्कतस्तु गुणनादिकं व्यक्ते
गणिते यदुक्तं तदत्र ज्ञेयमित्यपि द्रष्टव्यम्~॥~२६~॥\\

\vspace{-4mm}
एवम् अत्र {\qt "गुण्यान्त्यमङ्कं गुणकेन हन्यात्"}~। इत्यादिना गुणनफलसिद्धावपि
शिष्य-सौकर्यार्थं {\qt "गुण्यस्त्वधोऽधो गुणखण्डतुल्य"} इत्यादिव्यक्तोक्तखण्डगुणनं
वसन्ततिलकया विशदयति\textendash 

\phantomsection \label{27}
\begin{quote}
    \ab
    गुण्यः पृथग्गुणकखण्डसमो निवेश्यः  \\
तैः खण्डकैः क्रमहतः सहितो यथोक्त्या~।\\
अव्यक्तवर्गकरणीगुणनासु चिन्त्यो\\
व्यक्तोक्तखण्डगुणनाविधिरेवमत्र~॥~२७~॥
\end{quote}

गुणकस्य यावन्ति खण्डानि तावत्सु स्थानेषु पृथक् गुण्यो निवेश्यः~। अत्र
खण्डानि सञ्ज्ञा-भेदेनावगन्तव्यानि~। यथा गुणको या ३ रू २~। अत्र सञ्ज्ञाद्वयाद्गुणकस्य खण्डद्वयम्~। 
यथा वा गुणको याव २ या ३ का ५~। अत्र सञ्ज्ञात्रयाद्गुणकस्य खण्डत्रयमित्यादि~। अथ पृथङ्निवेशितो 
गुण्यस्तैर्गुणकखण्डैः प्रथमस्थाने प्रथमखण्डेन द्वितीयस्थाने द्वितीयेन तृतीयस्थाने तृतीयेनेत्येव क्रमेण
\hyperref[26]{\textbf{स्याद्रूपवर्णाभिहतौ तु वर्णः}} इत्यादिना गुणितः अन्यथोक्त्या पूर्वोक्तप्रकारेण \hyperref[22]{\textbf{योगोऽन्तर-}}
 \newpage %%%%%%%%%%%%%%%%%%%%%%%%%%%%%%%%%%%%%%%%
\noindent \hyperref[22]{\textbf{ऽन्तरं तेषु समानजात्योः}} इत्यादिना \hyperref[3]{\textbf{योगे युतिः स्यात्क्षययोः स्वयोर्वा}} 
इत्यादिना च सहितः~। अत्राव्यक्तगणितेऽव्यक्तवर्गकरणीगुणनासु यथा तथा व्यक्तगुणनासु
वर्गार्थगुणनासु करणीगुणनासु च व्यक्तोक्तखण्डगुणनाविधिरेवं चिन्त्यः~।
एवमन्येऽपि गुणनप्रकारा द्रष्टव्याः~॥~२७~॥\\

\vspace{-2mm}
{\bqt उदाहरणम् }

\phantomsection \label{28}
\begin{quote}
    \eg
    यावत्तावत्पञ्चकं व्येकरूपं यावत्तावद्भिस्त्रिभिः सद्विरूपैः~। \\
सङ्गुण्य द्राग्ब्रूहि गुण्यं गुणं वा व्यस्तं स्वर्णं कल्पयित्वा च\\
विद्वन्~॥~२८~॥
\end{quote}


गुण्ये गुणे वेति व्यस्तस्वर्णमिति च पाठोभेदात्पाठत्रयं प्रसिद्धमस्ति~।
तत्र पूर्वलिखितपाठे तावदियं व्याख्या~। स्वर्णं गुण्यं स्वर्णं गुणकं वा
\hyperref[28]{\textbf{व्यस्तं कल्पयित्वे}}ति~। गुण्ये गुणे वेति पाठे गुण्ये विद्यमानं स्वर्णं यथासम्भवं स्वमृणं यावत्कालकरूपादिव्यस्तं कल्पयित्वेति~। एवं गुणेऽपि~। अथ व्यस्तस्वर्णम् इति पाठे गुण्यं गुणं वा व्यस्तस्वर्णं कल्पयित्वा~। व्यस्तं स्वर्णं यथासम्भवं स्वम् ऋणं च यावदादि यत्र तं तादृशं कल्पयित्वेत्यर्थः अत्र सर्वत्र {\qt "सविशेषणौ हि विधिनिषेधौ
विशेषणमुपसङ्क्रामतो विशेष्ये बाधके सति"} इति न्यायेन स्वर्णत्वयोरेव
व्यस्तत्वविधानं द्रष्टव्यम्~। शेषं स्पष्टम्~। अत्र यथास्थितगुण्यगुणकयोरेकमुदाहरणम्~।
गुण्यमात्रव्यत्यासे द्वितीयम्~। गुणकमात्रव्यत्यासे तृतीयम्~। चकारादुभयव्यत्यासे चतुर्थमिति चत्वार्युदाहरणानि~। अत्र रूपोनं यावत्तावत्पञ्चकं गुण्यो या ५ रू १ं~। रूपद्वययुतं यावत्तावत्त्रयं गुणको या ३ रू २~। \hyperref[27]{\textbf{गुण्यः पृथग्गुणकखण्डसमो निवेश्यः}} इत्यादिना गुणनार्थं न्यासो
{\begin{tabular}{l}
या ३~। या ५ रू १ं~।\\
रू २~। या ५ रू १ं~।
\end{tabular}}
\noindent अत्र यावत्त्रयेण यावत्तावत्पञ्चके गुणितेऽङ्कतः पञ्चदश १५~। अक्षरतस्तु
द्वित्र्यादिकानां समजातिकानां वधे तु तद्वर्गघनादयः स्युः इत्यादिना
जाता यावत्तावद्वर्गाः~। तत्र यावत्तावतो वर्गस्य चाद्याक्षरोपलक्षणपूर्वकं
लिखने सम्पन्नं याव १५~। अथ यावत्त्रयेण क्षयरूपे गुणिते \hyperref[9]{\textbf{स्वर्णघाते क्षयः}} इत्यङ्कतः ३ं~।
अक्षरतस्तु \hyperref[26]{\textbf{रूपवर्णाभिहतौ वर्णः स्यात्}} इति जातो वर्ण एव या ३ं~। एवं प्रथमपङ्क्तौ
जातं याव
\afterpage{\fancyhead[CE] {बीजगणिते}}
\afterpage{\fancyhead[CO]{अव्यक्तषड्विधम्}}
\afterpage{\fancyhead[LE,RO]{\thepage}}
\cfoot{}
\newpage
%%%%%%%%%%%%%%%%%%%%%%%%%%%%%%%%%%%%%%%%%%%%%%%%
\noindent १५ या ३ं~। अथ द्वितीयस्थाने द्वितीयेन गुणकखण्डेन रू २ यावत्पञ्चके गुणितेऽङ्कतो दश १०~। अक्षरतस्तु \hyperref[26]{\textbf{रूपवर्णाभिहतौ वर्णः}}
इति जातो वर्णो या १०~। रूपद्वयेन क्षयरूपे गुणिते \hyperref[9]{\textbf{स्वर्णघाते क्षयः}} इति जातं २ं~।
अत्राक्षरसञ्ज्ञा व्यक्ते प्रसिद्धैव~। नहि व्यक्ते द्वित्र्यादिघाते सञ्ज्ञाभेदोऽस्ति~। रूपं तु
व्यक्तमेव~। अतो रूपस्य रूपेण गुणनेऽक्षरतो रूपमेव~। तथा सति जातं रू २ं~। एवं जातं द्वितीयपङ्क्तौ गुणनफलं या १० रू २ं~। एवमुभयपङ्क्त्योर्न्यासो \;$\begin{matrix}
\vspace{-1mm}
\mbox{{याव १५ या ३ं~।}}\\
\vspace{-1mm}
\mbox{{या ~१० रू २ं~।}}
\vspace{1mm}
\end{matrix}$ \;अत्र \;यथोक्त्या \;सहित \;इति \;\hyperref[22]{\textbf{योगोऽन्तरं \;तेषु समानजात्योः}} इत्यादिना तत्र प्रथमपङ्क्तौ
यावत्त्रयमृणम्~। द्वितीयपङ्क्तौ यावद्दशकं धनम्~। अनयोः साजात्याद्योगे
\hyperref[3]{\textbf{धनर्णयोरन्तरमेव योगः}} इति जातं या ७~। इतरयोर्भिन्नजातित्वात् पृथक् स्थितिरेव~।
तथा कृते जातं गुणनफलं याव १५ या ७ रु २ं~। अथ गुण्येन धनर्णव्यत्यासं \,कृत्वा \,द्वितीयोदाहरणे \,न्यासो \,$\begin{matrix}
\vspace{-1mm}
\mbox{{या ३~। या ५ं रू १~।}}\\
\vspace{-1mm}
\mbox{{रू २~। या ५ं रू १~।}}
\vspace{1mm}
\end{matrix}$ \,गुणकखण्डाभ्यां गुणिते जातं $\begin{matrix}
\vspace{-1mm}
\mbox{{याव १५ं या ३~।}}\\
\vspace{-1mm}
\mbox{{या १०ं रू २~।}}
\vspace{1mm}
\end{matrix}$ यथोक्त्या योगे जातं गुणनफलं याव १५ं या ७ं रू २~। अथ
गुणके धनर्णताव्यत्यासं कृत्वा तृतीयोदाहरणे न्यासो $\begin{matrix}
\vspace{-1mm}
\mbox{{या ३ं~। या ५ ष १ं~।}}\\
\vspace{-1mm}
\mbox{{रू २ं~। या ५ रू १ं~।}}
\vspace{1mm}
\end{matrix}$ गुणने जातं $\begin{matrix}
\vspace{-1mm}
\mbox{{याव १५ं या ३~।}}\\
\vspace{-1mm}
\mbox{{या १ं० रू २~।}}
\vspace{1mm}
\end{matrix}$ यथोक्तयोगे जातं गुणनफलं याव १५ं या ७ं रू २~। अथोभयोर्व्यत्यासे चतुर्थोदाहरणे न्यासो $\begin{matrix}
\vspace{-1mm}
\mbox{{या ३ं~। या ५ं रू १~।}}\\
\vspace{-1mm}
\mbox{{रू २ं~। या ५ं रू १~।}}
\vspace{1mm}
\end{matrix}$ गुणिते जातं $\begin{matrix}
\vspace{-1mm}
\mbox{{याव १५ या ३ं~।}}\\
\vspace{-1mm}
\mbox{{या १० रू २ं~।}}
\vspace{1mm}
\end{matrix}$ यथोक्तयोगे जातं गुणनफलं याव १५ या ७ रू २ं~। 
अत्रोपपत्तिः~। रूपै रूपेषु गुणितेषु रूपाणि भवन्तीति प्रसिद्धम्~। रूपेण वर्णे
गुणिते रूपं वा भवेद्वर्णो वा~। विनिगमनाविरहे सति कथं वर्ण एवेत्युक्तम्~। उच्यते~।
अज्ञातराशिमानं तावच्चतुर्धैव सम्भवति~। रूपसमूहस्तदवयवो रूपं रूपावयवो वेति~। तत्र रूपसमूहत्वमज्ञातराशेरङ्गीकृत्य युक्तिः उच्यते~। अस्ति किञ्चिद्धान्यं सप्ताढकमानेनैकं मानं १~।
इदं सप्तगुणितं जातं ७~। एतस्य गुणनफलस्य रूपात्मकत्वं समूहात्मकत्वं
वेति विचार्यम्~। तत्रास्य रूपात्मकत्वे सप्ताढकधान्यमिदमिति स्यात्~। न
चैतद्युक्तम्~। गुणना-
\newpage
%%%%%%%%%%%%%%%%%%%%%%%%%%%%%%%%%%%%%%%%%%%%
\noindent त्पूर्वमेव सप्ताढकस्य धान्यस्य विद्यमानत्वात्~। गुणनोत्तरं
त्वेकोनपञ्चाशदाढका भाव्याः~। अतः समूहात्मकत्वं वक्तव्यम्~। तथा सति सप्ताढकधान्यसमूहाः
सप्तेत्युपपन्नं \hyperref[26]{\textbf{स्याद्रूपवर्णाभिहतौ तु वर्ण}} इति~। अथाज्ञातराशौ
रूपसमूहावयवत्वमुररीकृत्य युक्तिरुच्यते~। अस्ति सप्ताढकस्य मानस्य त्र्यंशमितं मानम्~। अनेन
मानेनास्ति धान्यमिति १ इदं त्रिगुणितं ३~। अस्य रूपात्मकत्वं आढकत्रयमेव स्यात्~।
तच्चायुक्तम्~। आढकसप्तकस्य त्र्यंशे हि त्रिगुणित आढकसप्तकेन भाव्यम्~। अत एव तस्य समूहावयवात्मकत्वम्~। तथा सति त्रय आढकसप्तकत्र्यंशा इति स्यात्~। एवम् अप्युपपन्नं \hyperref[26]{\textbf{स्याद्रूपवर्णाभिहतौ तु वर्ण}} इति अथ रूपावयवत्वमज्ञातराशेरुररीकृत्योच्यते~।
अस्त्याढकचतुर्थांशमितं मानम्~। एतन्मितं धान्यं प्रस्थमितं १ भवति इदं
त्रिभिर्गुणितं ३~। नेदं रूपात्मकम्~। अस्य रूपात्मकत्व आढकत्रयं स्यात्~। न चैतद्युक्तम्~। तस्माद्रूपावयवात्मकत्वमस्य वक्तव्यम्~। तथा सत्याढकचतुर्थांशास्त्रय इति भवति प्रस्थत्रयम्~। \hyperref[26]{\textbf{रूपवर्णाभिहतौ वर्णः}} इति~। अथाज्ञातराशे रूपत्वे
वर्णरूपयोरभेदाद्गुणनफले वर्णतापि युक्तैव~। न च गुणनफले रूपत्वमेवास्तु~। तस्यापि युक्तत्वादिति
वाच्यम्~। अज्ञातराशे रूपत्वे वर्णरूपत्वेनावगमाभावात्~। अवधृते हि राशे रूपत्वे
गुणनफले रूपत्वमपि युक्तम्~। अत्र तु राशेरज्ञानाद्रूपत्वानवधारणात्~। न चैवं
गुणनफले वर्णत्वमपि कथं स्याद्रूपसमूहत्वादिना राशेरनवगमादिति वाच्यम्~। नहि
रूपवर्णयोर्गुणनफलस्य वर्णत्वे रूपसमूहत्वादिनाप्यवगमो राशेरावश्यकः~। किन्तु तस्य
चतुष्टयसाधारणत्वाच्चतुष्टयान्यतमत्वेनैव राशेरवगमोऽपेक्षितः~। स चास्त्येव~। चतुष्टयान्यस्य
राशेरसम्भवात् अत एव लाघवाद्वर्णत्वपुरस्कारेणैव प्रकृतगुणनफलस्य
वर्णत्वमुक्तमाचार्यैरित्युपपन्नम्~।
\hyperref[26]{\textbf{स्याद्रूपवर्णाभिहतौ तु वर्णः}} इति~। किञ्च रूपं हि व्यक्तसङ्ख्या~। तथा
गुणनेऽङ्कत एव गुणनं स्यान्नाक्षरतः~। न च रूपव्यक्तसङ्ख्ययोरभेदे
सङ्ख्याज्ञापकाङ्कलिखनमेवास्तु किं रूपप्रथमाक्षरलिखनेनेति वाच्यम्~। अङ्कस्य भेदकाभावे वर्णाङ्कसंनिधानेन
कदाचित्सङ्करः स्यादित्यसङ्करार्थं रूपाक्षरलिखनात्~। अत एव सति रेखादिके भेदके
नास्त्येवाक्षरलिखनोपयोगः~। किन्तु शीघ्रोपस्थितये तत्~। एवं यावद्वर्गादीनामपि
रूपगुणनेऽक्षरतो न विकार इत्यादि सुधीभिरन्यदप्यूह्यम्~। अथ समजातिवर्णगुणने तत्र वर्णस्य रूपसमूहत्वमुररीकृत्य युक्तिरुच्यते~। यथाढकसप्तकस्यैकः समूहः १
अनेनैवास्मिन्गुणिते जातं \,१~।
अस्याढकसप्तलक्षणसमूहात्मकत्व \,एकगुणितसमूहस्य \,समूहगुणितसमूहस्य
चाभेदापत्तिः~। न
चात्रेष्टापत्तिः~। एकस्मिन्गुण्ये गुणकभेदाद्गुणनफलभेदस्यावश्यकत्वात्~।
अतो गुणनफलस्य समूहवर्गात्मकत्वं वक्तव्यम्~। तथा सत्येकोनपञ्चाशदाढकाः स्युः~। युक्तं चैतत्~। अतः
\newpage
%%%%%%%%%%%%%%%%%%%%%%%%%%%%%%%%%
\noindent समानजात्योर्द्वयोर्वर्णयोर्वधे तद्वर्गो भवतीत्युपपन्नम्~। एवं
समूहावयवत्वादिकम् अप्यङ्गी-कृत्य युक्तिर्द्रष्टव्या~। एवं त्र्यादीनां समजातिकानां वधे धनादित्वमप्यूह्यम्~। तदेवमुपपन्नं \hyperref[26]{\textbf{द्वित्र्यादिकानां \,समजातिकानां \,वधे \,तु \,तद्वर्गघनादयः \,स्युः}} \,इति~। अथासमजातिघात आढकसप्तकात्मकः समूहः १ आढकपञ्चात्मकोऽन्यः १~। अनयोर्वधे जातं १~। नायमाढकसप्तकात्मकः समूहः~। तस्यैकगुणस्य समूहगुणितस्य चाभेदापत्तेः~। नायं समूहवर्गः~। समूहस्य स्वेन गुणने समूहान्तरेण च गुणने गुणनफलस्याभेदापत्तेः~। अतः समूहयोः वधोऽयमेकः~।
तथा सति पञ्चत्रिंशदाढकाः स्युः~। युक्तं चैतत्~। तस्मादसमजातिघाते
तयोर्घात इत्यक्षरतो भवितुं युक्तम्~। तत्राद्यैर्घातस्य भावितमिति सञ्ज्ञा कृता~।
वधशब्दस्याद्याक्षरलिखने यावदादिवर्गेण सङ्करः स्यात्~। घातशब्दस्याद्याक्षरलिखने
कदाचिद्धनेन सङ्करः स्यात्~। गुणनशब्दप्रथमाक्षरलिखनेऽश्लीलता स्यात्~। हति
शब्दप्रथमाक्षरलिखने कदाचिद्धरितकवर्णभ्रमः स्यादिति~। अथ यद्यपरः कश्चिच्छब्दोऽस्ति
तत्प्रथमाक्षरलिखने सङ्करादिदोषो न स्यात्~। अस्तु तर्हि तल्लिखनं न कदाचित्क्षतिः~।
किन्त्वाचार्येणाद्याक्षरानुरोधाद्भावितमिति सञ्ज्ञा कृतेत्युपपन्नं तद्भावितं
चासमजातिघाते इति~। खण्डगुणनोपपत्तिः स्पष्टैव~॥~२८~॥\\

\vspace{-2mm}
{\bqt भागहारे करणसूत्रम्}

\phantomsection \label{29}
\begin{quote}
    \ab
    भाज्याच्छेदः शुध्यति प्रच्युतः सन्स्वेषु स्वेषु स्थानकेषु क्रमेण~।\\
यैर्यैर्वर्णैर्यैर्गुणो यैश्च रूपैर्भागाहारे लब्धयस्ताः स्युरत्र~॥~२९~॥
\end{quote}

\hyperref[29]{\textbf{छेदो}} हरः \hyperref[29]{\textbf{यैर्यैर्वर्णैर्यै रूपैश्च}} गुणितः \hyperref[29]{\textbf{सन्भाज्यात्स्वेषु स्वेषु स्थानेषु}} यथास्वं समानजातिषु \hyperref[29]{\textbf{प्रच्युतः सञ्शुद्ध्यति}} न शिष्यति \hyperref[29]{\textbf{ता}} अत्र \hyperref[29]{\textbf{लब्धयः स्युः}}~। ते
वर्णास्तानि च रूपाणि लब्धयः स्युरित्यर्थः~। अत्र यैर्गुणितो हरो
भाज्याच्छुध्यति तेष्वधिको लब्धिर्भवतीति द्रष्टव्यम्~। अन्यथा न्यूनगुणोऽपि हरः शुध्यतीति
न्यूना अपि लब्धयः स्युः~। यद्वा भाज्योऽपि शुध्यतीति द्रष्टव्यम्~। ता लब्धय इत्यत्र
तच्छब्दस्य विधीयमानलिङ्गता {\qt "शैत्यं हि यत्सा प्रकृतिर्जलस्य"} इत्यादौ प्रसिद्धा~। {\qt "दैवे युगसहस्रे द्वे ब्राह्मः कल्पौ तु नृणाम्"} इत्यस्य व्याख्यावसरे लिखितं च क्षीरस्वामिना
{\qt "सर्वनाम्नां विधीयामाननूद्यमानलिङ्गग्रहणे कामचारः"}  इति~।
अत्रोदाहरणार्थं पूर्वगुणनफलस्य स्वगुणच्छेदस्य न्यासः~। तत्र भाज्यो याव १५ या ७ रू २ं भाजको या ३
\newpage
%%%%%%%%%%%%%%%%%%%%%%%%%%%%%%%%%%%%%%%%%%
\noindent रू २ं~। अत्र भाज्ये प्रथमतो यावद्वर्गाः सन्ति तेभ्यो यावद्वर्गाणामेव
शोधनं युक्तम्~।
समजातित्वात्~। अत्र हरे तु प्रथमतो यावत्त्रयमस्ति~। तस्य रूपेण गुणने
\hyperref[26]{\textbf{स्याद्रूपवर्णाभिहतौ तु वर्णः}} इति वर्ण एव स्यान्न तद्वर्गः~। यावता गुणनेऽपि
समानजातिघातत्वाद्यद्यपि यावद्वर्गो भवेत्तथाप्यङ्कतस्त्रयमेवेति तच्छोधनेऽपि भाज्येन
यावद्वर्गाणां न शुद्धिः~। अतो यावत्पञ्चकेन भाजके गुणिते पञ्चदश यावद्वर्गा भवेयुः~। तथा सति
शुद्धिर्भवेदिति यावत् पञ्चकेन या ५ छेदोऽयं या ३ रू २ गुणिते याव १५ या १०~।
अस्मिन्भाज्यादस्मात् याव १५ या ७ रू २ं यथास्थानमपनीते जातं या ३ं रू २ं~। यावत्पञ्चकेन गुणितः छेदः शुद्ध इति यावत्पञ्चकं लब्धिः\textendash \,या ५~। अथ
भाज्यशेषे यावत्तावत्त्रयमस्ति~। अतो हरे रूपेण गुणिते तस्माच्छोधिते तस्य शुद्धिः
स्यात्~। परं धनरूपेण गुणेन \hyperref[7]{\textbf{'संशोध्यमानं स्वम् ऋणत्वमेति'}} इति द्वयोरृणत्वाद्योगः
स्यादिति न शुद्धिः~। तस्मादृणरूपेण गुणिते तस्माच्छोधितस्य शुद्धिः स्यादिति ऋणरूपेण रू १ं हरोऽयं या ३ रू २ गुणितो या ३ं रू २ं भाज्यशेषादस्मात् या ३ं रू २ \hyperref[29]{\textbf{च्युतः शुध्यति}} इति रूपमृणं लब्धी रू १ं~। एवं जाता लब्धिः\textendash \,या ५ रू १ं~।
पूर्वगुण्योऽयम्~। अथ द्वितीयोदाहरणे भाज्यो याव १५ं या ७ं रू २ं~। भाजको या ३
रू २ उक्तवज्जाता लब्धिः\textendash \,या ५ं रू १~। अथ तृतीयोदाहरणे भाज्यो या १५ं या
७ं रू २~। भाजको या ३ं रू २ं~। उक्तवल्लब्धिः\textendash \,य ५ रू १ं~। अथ
चतुर्थोदाहरणे भाज्यो या १५ या ७ रू २ं~। भाजको या ३ं रू २ं~। उक्तवल्लब्धिः\textendash \,या ५ं रू
१~। अत्रोपपत्तिः\textendash \,भाज्यराशिस्तावत्कयोश्चिद्गुण्यगुणकयोर्गुणनफलम्~।
भाजकस्तु गुण्यगुणकयोरन्यतरः~। तदितरो लब्धिश्चेति स्थितिरस्ति~।
तत्रास्मिन् गुणनफलेऽस्मिंश्च गुणके सति को गुण्य इत्यस्मिन्गुण्ये सति को वा गुणक इति लब्धिः
प्रश्नार्थः~। तत्र गुणको येन गुणितः सन् गुणनफलसमो भवेत्स गुण्यो वा येन गुणितः सन्
गुणनफलसमः स्यात्स गुणक इति स्पष्टैव युक्तिः~। ननु तथाप्येतावदेव
वक्तव्यं यद्गुणितो हरो भाज्यसमः स्यादिति न तु 
\hyperref[29]{\textbf{प्रच्युतः सञ्शुध्यति}} इति~।
गौरवात्~।
सत्यम्~। असमे समताभ्रमनिबन्धनोऽलब्धौ लब्धिभ्रमः स्यादिति तन्निरासार्थं
\hyperref[29]{\textbf{प्रच्युतः
सञ्शुध्यति}} इत्युक्तम्~। अन्यथा भाज्येऽस्मिन् याव १५ या ७ रू २ सति
हरेऽस्मिन् या ३ रू २ अनेन या ५ रू १ गुणितो याव १ं५ या ७ं रू २ भाज्यसमताभ्रमेण
लब्धिरियं या ५ं रू १ इति भ्रमः स्यात्~। शोधने तु \hyperref[7]{\textbf{संशोध्यमानं स्वमृणत्वमेति}} इत्युभयेषां यावद्वर्गाणां यावतां च धनत्वाद्रूपयोश्चर्णत्वाद्योगे
वैगुण्यं स्यान्न तु शुद्धिरिति लब्धित्वभ्रमो न स्यात्~। ननु विशेषादर्शनं भ्रमं प्रति
हेतुरिति यथा प्रकृते
\afterpage{\fancyhead[CE] {बीजगणिते}}
\afterpage{\fancyhead[CO]{वर्णषड्विधम्}}
\afterpage{\fancyhead[LE,RO]{\thepage}}
\cfoot{}
\newpage
%%%%%%%%%%%%%%%%%%%%%%%%%%%%%%%%%%%%%%%%%%%%
\noindent धनर्णत्वलक्षणविशेषादर्शनाद्भाज्यसमताभ्रमस्तथा शोधनेऽपि
विशेषादर्शनस्य सत्त्वात्कथं न भ्रमः स्यादिति चेन्मैवम्~। तत्र
धनर्णत्वलक्षणविशेषस्य दर्शनमदर्शनं न सम्भाव्यते~। शोधने तु \hyperref[7]{\textbf{संशोध्यमानं स्वमृणत्वमेति}} इति धनर्णत्वाक्षेपान्न  विशेषादर्शनं सम्भवति~। किञ्च भाजको येन गुणितो भाज्यसमो भवेत्तस्य न
शीघ्रमुपस्थितः~। 
शोधने तु भाज्येऽत्र प्रथमतः पञ्चदश यावद्वर्गा
दृश्यन्ते~। भाजके तु यावत्त्रयम्~। तद्यदि यावत्पञ्चकेन गुण्यते तर्हि पञ्चदश
यावद्वर्गा भवेयुः~। तथा सति यावद्वर्गाणां शुद्धिः स्यादित्यस्ति
शीघ्रोपस्थितिः~। एवं भाज्यशेषशुद्धावपीत्यलं पल्लवितेन~॥~१९~॥\\

\vspace{-2mm}
{\bqt वर्गोदाहरणम्\textemdash}
\begin{quote}
    \eg
    रूपैः षड्भिर्वर्जितानां चतुर्णामव्यक्तानां ब्रूहि वर्गं सखे मे~॥~३०~॥
\end{quote}

 स्पष्टोऽर्थः~। रूपषट्कोनमव्यक्तचतुष्टयमिदं या ४ रू ६ं~। वर्गार्थमयमेव
गुण्यो गुणकः चेति न्यासो $\begin{matrix}
\vspace{-1mm}
\mbox{{या ४~। या ४ रू ६ं~।}}\\
\vspace{-1mm}
\mbox{{रू ६ं~। या ४ रू ६ं~।}}
\vspace{1mm}
\end{matrix}$ स्थानद्वयेऽपि गुणने जातं $\begin{matrix}
\vspace{-1mm}
\mbox{{याव १६ या २ं४~।}}\\
\vspace{-1mm}
\mbox{{या २ं४ रू ३६~।}}
\vspace{1mm}
\end{matrix}$ योगे जातो वर्गो याव १६~। या ४ं८ रू ३६~॥~३०~॥\\

\vspace{-2mm}
{\bqt वर्गमूले करणसूत्रम्\textendash }

\phantomsection \label{31}
\begin{quote}
    \ab
    कृतिभ्य आदाय पदानि तेषां द्वयोर्द्वयोश्चाभिहतिं द्विनिघ्नीम्~। \\
शेषात्त्यजेद्रूपपदं गृहीत्वा चेत्सन्ति रूपाणि तथैव शेषम्~॥~३१~॥
\end{quote}

 तेषां वर्गराशिगतव्यक्तानां मध्ये \hyperref[31]{\textbf{कृतिभ्यः पदान्यादाय तेषां}} पदानां परस्परं \hyperref[31]{\textbf{द्वयोर्द्वयोरभिहतिं द्विनिघ्नीं शेषा}}द्विशोधयेत्~। यदि शुद्धिर्भवेत्तदा 
तानि तस्य वर्गस्य पदानि स्युरित्यर्थादुक्तं भवति~। कृत्योरित्यपि
द्रष्टव्यम्~। अथ यदि वर्गराशौ \hyperref[31]{\textbf{रूपाणि सन्ति}} तर्हि \hyperref[31]{\textbf{रूपपदं गृहीत्वा शेषं}} तथैव
द्वयोर्द्वयोश्चाभिहतिं द्विनिघ्नीं शेषात्त्यजेदिति~। रूपेषु सत्सु यदि रूपपदं लभ्यते तदा स
वर्गराशिर्नेत्यर्थादुक्तं
\newpage
%%%%%%%%%%%%%%%%%%%%%%%%%%%%%%%%%%%%%%%%%
\noindent भवति~। अत्रोदाहरणम्~। पूर्वसिद्धवर्गस्य मूलार्थं न्यासो याव ६ या ४ं८ रू ३६~। 
अत्र वर्गराशौ षोडश यावद्वर्गाः षट्त्रिंशद्रूपाणि चेति
वर्गद्वयमस्माद्गृहीते मूले या 
४ रू ६~। अनयोर्द्वयोरभिहतिं या २४ द्विनिघ्नीं या ४८
शेषात्त्यजेदिति \hyperref[7]{\textbf{संशोध्यमानं स्वमृणत्वमेति}}~। इत्यृणयोर्योगे शुद्धिर्न स्यादिति
द्वयोरन्यतरस्वर्णत्वं  कल्प्यते~। तथा सति द्वयोरभिहतिर्द्विगुणिता या ४ं८ \hyperref[7]{\textbf{संशोध्यमानं स्वमृणत्वमेति}} इति धनत्वे \hyperref[3]{\textbf{धनर्णयोरन्तरमेव योगः}} इति शुद्धिः स्यात्~। अतोऽस्य या ४ं रू ६ं
अस्य  वा या ४ं रू ६ वर्गोऽयं याव १६ या ४ं८ रू ३६~। ननु रूपषट्कयुतस्य 
यावत्त्रयस्य या ३ रू ६ वर्गोऽयं याव ९ या ३६ रू ३६ अत्र \hyperref[31]{\textbf{कृतिभ्य आदाय पदानि}} इत्यादिना सर्वेभ्योऽपि मूललाभाच्छेषाभावे द्वयोर्द्वयोरभिहतिं
द्विनिघ्नीं कुतः शोध्येति चेन्न~। यावतां या ३६ मूलाभावात्~। नहि यावदात्मकः कस्यापि वर्गः
सम्भवति यदस्य मूलं सम्भवेदिति सर्वमवदातम्~। अत्रोपपत्तिः~। समद्विघातो हि
वर्गः~। तथा च यस्य वर्गः क्रियते स एव गुण्यो गुणकश्च~। तत्रैकखण्डात्मके
वर्गे कस्यायं समद्विघात इति समद्विघातान्वेषणे मूलावगमः सुगमः~। अथ
खण्डद्वयस्य वर्गार्थं न्यासो $\begin{matrix}
\vspace{-1mm}
\mbox{{या ४~। या ४ रू ६~।}}\\
\vspace{-1mm}
\mbox{{रू ६~। या ४ रू ६~।}}
\vspace{1mm}
\end{matrix}$ अत्र प्रथमपङ्क्तावेकस्य खण्डस्य वर्गः \,खण्डद्वयाभिहतिश्च~। \,द्वितीयपङ्क्तावपि \,खण्डद्वयाभिहतिर्द्वितीयखण्डवर्गश्च~। \,अत्र पङ्क्तिद्वयेऽपि खण्डाभिहतिरस्तीति योगे द्विगुणिताभिहतिः स्यात्~। अतः खण्डद्वयस्य वर्गे खण्डत्रयं भवति खण्डवर्गौ द्विगुणिता खण्डद्वयाभिहतिश्व याव १६ या ४८ रू ३६~। अथ खण्डत्रयवर्गे $\begin{matrix}
\vspace{-1mm}
\mbox{{या ३~। या ३ का ४ नी ५~।}}\\
\vspace{-1mm}
\mbox{{का ४~। या ३ का ४ नी ५~।}}\\
\vspace{-1mm}
\mbox{{नी ५~। या ३ का ४ नी ५~।}}
\vspace{1mm}
\end{matrix}$ अत्र प्रथमपङ्क्तौ प्रथमखण्डवर्गः
प्रथमद्वितीयखण्डाभिहतिः प्रथमतृतीयखण्डाभिहतिश्च~। द्वितीयपङ्क्तौ द्वितीयखण्डवर्गः
प्रथमद्वितीयाभिहतिर्द्वितीयतृतीयाभिहतिश्च~। तृतीयपङ्क्तौ तृतीयखण्डवर्गः
प्रथमतृतीयाभिहतिर्द्वितीयतृतीयाभिहतिश्चेति~। एवं चतुरादिखण्डवर्गेष्वपि~।
तथा च वर्गे क्रियमाणे खण्डानां वर्गा द्वयोर्द्वयोर्द्विगुणाभिहतिश्च स्यात्~।
तस्मात्सुष्ठूक्तं \hyperref[31]{\textbf{कृतिभ्य आदाय}} इत्यादि~। ननु वर्गराशाववश्यं खण्डवर्गा भवन्तीति
कृतिभ्यः पदान्यादायेत्यनेनैव चरितार्थत्वाद्द्वयोर्द्वयोरित्यादि
व्यर्थमिति चेन्न~। तथा सति यत्र राशौ याव ९ या ८ रू ९ एवमस्ति तस्यापि
मूलं या ३ रू ३ स्यात्~। न चैतद्युक्तम्~। यतोऽस्य वर्गोऽयं याव ९ या १८ रू ९~। तस्माद्यदि मूलेषु
\newpage
%%%%%%%%%%%%%%%%%%%%%%%%%%%%%%%%%%%%%%%%%%%%
\noindent गृहीतेषु द्वयोर्द्वयोर्द्विगुणाभिहतिः शिष्यते तर्हि तस्य वर्गत्वमिति
नियमार्थं द्वयोर्द्वयोश्चाभिहतिं द्विनिघ्नीं शेषात्त्यजेदित्युक्तम्~॥~३१~॥\\

\vspace{-2mm}
{\bqt सङ्कलनव्यवकलनोदाहरणमाह\textendash }
\begin{quote}
    \eg
    यावत्तावत्कालकनीलकवर्णांस्त्रिपञ्चसप्तधनम्~।\\
द्वित्र्येकमितैः क्षयगैः सहिता रहिताः कति स्युस्तैः~॥~३२~॥
\end{quote}

धनं त्रिपञ्चसप्त यावत्तावत्कालकनीलकवर्णाः क्षयगैर्द्वित्र्येकमितैस्तैर्यावत्कालकनीलकवर्णैः सहिताः कति स्यू रहिताश्च कति स्युरित्युदाहरणद्वयम्~। अत्र यावत्तावत्कालकनीलकवर्णानां
भिन्नजातित्वात्पृथक्स्थितिरेव~। या ३ का ५ नी ७ एतैः
क्षयगैर्द्वित्र्येकमितैरेतैः\textendash \,या २ं का ३ं नी १ं सहिताः 
\hyperref[3]{\textbf{धनर्णयोरन्तरमेव योगः}} इति \hyperref[22]{\textbf{योगोऽन्तरं तेषु समानजात्योः}} इति जाता या १ 
का २ नी ६~। रहिताश्चेत्तदा संशोध्यमानमृणं धनं भवतीति धनत्वे
साजा-त्याद्योगे जाता या ५ का ८ नी ८~॥~३२~॥\\

\vspace{-2mm}
{\bqt उदाहरणम् }
\begin{quote}
    \eg
    यावत्तावत्त्रयमृणमृणं कालकौ नीलकः स्वं \\
रूपेणाढ्या द्विगुणितमितैस्तैस्तु\renewcommand{\thefootnote}{1}\footnote{स्तेतु} तैरेव निघ्नाः~। \\
किं स्यात्तेषां गुनणजफलं गुण्यभक्तं च किं स्यात्\\
गुण्यस्याथ प्रकथय कृतिं मूलमस्याः कृतेश्च~॥~३३~॥
\end{quote}

स्फुटोऽर्थः~। ऋणं यावत्तावत्त्रयं या ३ं ऋणं कालकौ का २ं धनं नीलको
नी १ रूपेणाढ्या जातो गुण्यो या ३ं का २ं नी १ रू १~। एत एव
द्विगुणा जातो गुणको या ६ं का ४ं नी २ रू २~। अयं गुणनार्थं न्यासः~।

\newpage 
%%%%%%%%%%%%%%%%%%%%%%%%%%%%%%%%%%%%%%%%
\begin{table}[h!]
    \centering\s
    \begin{tabular}{lp{0.2cm}lp{0.2cm}lp{0.2cm}lp{0.2cm}l}
      या ६ं~।&& या ३ं&& का २ं&& नी १&& रू १ \\
का ४ं~।&& या ३ं& &का २ं&& नी १&& रू १ \\
नी २~।&& या ३ं& &का २ं&& नी १&& रू १ \\
रू २~।&& या ३ं&& का २ं&& नी १&& रू १ 
    \end{tabular}
\end{table}

 \hyperref[26]{\textbf{स्याद्रूपवर्णाभिहतौ तु वर्णः}} इत्यादिना गुणनेन जातं पङ्क्तिचतुष्टये
गुणनफलमक्षरतोऽङ्कतश्च~। अत्र यावद्वर्गाधस्तिर्यक्स्थितानां च क्रमेण
साजात्याद्योगे कालकवर्गादपि

\begin{table}[h!]
 \centering\s

\begin{tabular}{llllllll}
याव &१८ &याकाभा& १२& यानीभा &६ं& या& ६ं \\
याकाभा& १२& काव& ८ &कानीभा &४ं &का &४ं\\
यानीभा& ६ं& कानीभा& ४ं& नीव&२& नी& २ \\
या &६ं& का &४ं &नी& २& रू& २
\end{tabular}
\end{table}

\noindent तिर्यगधः स्थितानां कालकनीलक\textendash \,भा ४ं का ४\textendash \,क्रमेण साजात्याद्योगे
नीलकवर्गादपि तिर्यगधःस्थितयोः नी २ साजात्याद्योगेऽन्येषां पृथक्स्थितौ च
जातं गुणनफलं याव १८ या याकाभा २४ यानीभा १ं२ या १ं२ काव ८ कानीभा ८ं का ८ नीव २ नी ४ रू २~। अथेदं गुण्यभक्तं किं स्यादिति भागहारार्थं गुण्यच्छेदस्य गुणनफलस्य न्यासो याव १८ याकाभा २४ यानीभा १ं२ या १ं२ काव ८ कानीभा ८ं का ८ं नीव २ नी ४ रू २~। या ३ं का २ं नी १ रू १~। अत्र \hyperref[29]{\textbf{भाज्याच्छेदः शुध्यति प्रच्युतः सन्}} इत्यादिना लब्धिर्ग्राह्या~। अत्र भाज्ये प्रथमतोऽष्टादश
यावद्वर्गाः  सन्ति भाजके च यावत्त्रयं या ३ं~। अस्मिन्यावत्षट्केन गुणित ऋणमष्टादश
यावद्वर्गा भवन्ति~। एते यदा शोध्यन्ते तदा धनं स्युरिति साजात्याद्योगः
स्यान्न  शुद्धिः~। ऋणयावत्षट्केन हरगुणने तु शुद्धिः स्यात्~। अतोऽनेन या ६ं हरो 
गुणितो जातो याव १८ याकाभा १२ यानीभा ६ं या ६ं~। अस्मिन्यथास्थानं 
भाज्यादपनीते शेषं लब्धिश्च याकाभा १२ यानीभा ६ं या ६ं काव ८ कानीभा 
८ का ८ं नीव २ नी ४ रू २~। या ३ं का २ं नी १ रू १~। लब्धिश्च या 
६ं~। अथ भाज्ये यावत्कालकभावितमस्ति~। ऋणकालकैः का ४ं हरगुणने तस्य 
शुद्धिः स्यादिति लब्धिः का ४ं~। एतद्गुणो भाजको जातो याकाभा १२ काव ८ 
कानीभा ४ं का ४ं~। अस्मिन्भाज्यादपनीते शेषं यानीभा ६ं या ६ं कानीभा ४ं
का ४ं नीव २ नी ४ रू २~। अत्र भाज्ये यावन्नीलकभावितमस्ति~। 
नीलकद्वयेन भाजके गुणिते तस्मादपनीते शुद्धिः स्यादिति लब्धिः\textendash \,नी २~। एतद्गुणो
\afterpage{\fancyhead[CE] {बीजगणिते}}
\afterpage{\fancyhead[CO]{अनेकवर्णषड्विधम्}}
\afterpage{\fancyhead[LE,RO]{\thepage}}
\cfoot{}
\newpage
%%%%%%%%%%%%%%%%%%%%%%%%%%%%%%%%%%%%%%%%%%%%%%%%

\noindent भाजको यानीभा ६ं कानीभा ४ं नीव २ नी २~। अस्मिन्भाज्यादपनीते शेषं या ६ं
का ४ं नी २ रू २~। अथ भाज्ये यावत्षट्कमस्ति~। हरे रूपद्वयगुणिते तस्य
शुद्धिः स्यादिति लब्धी रू २~। रूपद्वयगुणितो हरो या ६ं का ४ं नी २ रू २~।
अस्मिन्भाज्यादपनीते सर्वशुद्धिरिति जाता सम्पूर्णा लब्धिः\textendash \,या ६ं का ४ं नी २ रू २~। अथ गुण्यस्य प्रकथयेति गुण्यस्य स्वेन गुणनार्थं न्यासः\textendash 

\begin{table}[h!]
    \centering\s
    \begin{tabular}{lllllp{0.5cm}lllll}
  या &३ं~।& या& ३ं& का&२ं&& नी& १& रू& १~।\\
 का &२ं~। &या& ३ं& का&२ं&& नी& १& रू &१~। \\
 नी &१~। &या& ३ं& का&२ं&& नी& १& रू &१~। \\
 रू &१~।& या& ३ं& का& २ं&& नी& १& रू &१~।
    \end{tabular}
\end{table}

\noindent उक्तवद्गुणने योगे च जातो वर्गो याव ९ याकाभा १२ यानीभा ६ं या ६ं काव ४ कानीभा ४ं \,का \,४ं \,नीव \,१ \,नी \,२ \,रू \,१~। अथास्याः \,कृतेर्मूलं \,कथयेति \,मूलोदाहरणम्~। अत्र \hyperref[31]{\textbf{कृतिभ्य आदाय पदानि}} इति गृहीतानि पदानि या ३ का 
२ नी १ रू १~। अत्र द्वयोर्द्वयोरभिहतिं द्विनिघ्नीं यथाक्रमं याकाभा १२
यानीभा ६ या ६~। इयं वर्गशेषाच्छोध्येति 
\hyperref[7]{\textbf{संशोध्यमानं स्वमृणत्वमेति}} इति 
यद्यपि यावत्कालकभावितानामृणत्वे \hyperref[3]{\textbf{धनर्णयोरन्तरमेव योगः}} इति भवति शुद्धिस्तथापि यावन्नीलकभावितानां यावतां चर्णत्वे साजात्याद्योगे द्वैगुण्यं स्यान्न शुद्धिः~। 
अतो यावत्तावत्त्रयमृणं मूलं कल्प्यते~। \hyperref[13]{\textbf{स्वमूले धनर्णे}} इत्युक्तत्वात्~। तथा सति द्वयोर्द्वयोरभिहतिर्द्विनिघ्नीं याकाभा १२ यानीभा ६ं या ६ं~। अत्र यद्यपि \hyperref[7]{\textbf{संशोध्यमानं स्वमृणत्वमेति}} इत्यादिना यावन्नीलकभावितानां यावतां च भवति
शुद्धिस्तथापि यावत्कालकभावितानां द्वैगुण्यं स्यान्न शुद्धिः~।
तस्मात् पूर्वस्यामभिहतौ यावन्नीलकभावितानां यावतां च व्यत्यासार्थं
नीलकरूपयोः ऋणत्वं कल्प्यमथवास्यामभिहतौ यावत्कालकभावितानां व्यत्यासार्थं
कालकस्यर्णत्वं कल्प्यमिति द्विधैव गतिरस्ति तथा सति मूलान्येतानि या ३ं का
२ं नी १ रू १~। एतानि वा या ३ का २ नी १ं रू १ं~। उभयेषामपि परस्परं
द्वयोर्द्वयोरभिहतिर्द्विनिघ्नी तुल्यैव~। याकाभा १२ यानीभा ६ं कानीभा ४ं का ४ं
नी २~। अस्याः शोधनेन भवति सर्वशुद्धिरिति द्वयस्यपि पदत्वं सिद्धम्~॥~३३~॥
\newpage
%%%%%%%%%%%%%%%%%%%%%%%%%%%%%%%%%%%%%%%%%%%%%%%%%%%%

\begin{quote}
{\qt दैवज्ञवर्यगणसन्ततसेव्यपार्श्वबल्लालसञ्ज्ञगणकात्मजनिर्मितेऽस्मिन्~।\\
\indent बीजक्रियाविवृतिकल्पलतावतारे वर्णोद्भवाः समभवन्निति षट् प्रकाराः~॥ }\end{quote}

\begin{center}
इति श्रीसकलगणकसार्वभौमश्रीबल्लालदैवज्ञसुतकृष्णगणकविरचिते\\
बीजविवृतिकल्पलतावतारे वर्णषड्विधविवरणं समाप्तम्~॥\\
(अत्र ग्रन्थसङ्ख्या विंशत्यधिकशतत्रयम् ३२०)~।\\
\vspace{1.5cm}
\rule{0.2\linewidth}{0.5pt}
\end{center}
 
\newpage
%%%%%%%%%%%%%%%%%%%%%%%%%%%%%%%%%%%%%%%%%%%%%%%%%%
\phantomsection \label{ch4}
\begin{center}
{\LARGE \textbf{४ करणीषड्विधम्~।}}
\end{center}

\vspace{2mm}
 अथ करणीषड्विधं व्याख्यायते~। अत्रेदमवगन्तव्यम्~। मूलराश्योर्वर्गद्वारा
यत्षड्विधं तत् करणीषड्विधम् इति~। अस्य षड्विधस्य वर्गत्वपुरस्कारेणैव प्रवृत्तेः~। अत एवास्मिन् षड्विधे मूलदराशावपि करणीत्वव्यवहारः~। करणीत्वपुरस्कारेण गणितप्रवृत्तावयं न स्यात् करणीषड्विधमिति सञ्ज्ञा तु करणीराशावेतस्य
गणितस्यावश्यकत्वाद्द्रष्टव्या~। तत्र यस्य राशेर्मूलेऽपेक्षिते निरग्रं मूलं न
सम्भवति स करणी~। न त्वमूलदराशिमात्रम्~। तथा सति द्वित्रिपञ्चषडादिषु सर्वदा
करणीत्वव्यवहारः स्यात्~। अस्तु स इति चेन्न~। तथा सति तत्प्रयुक्तं कार्यं स्यात्~।
यथा\textendash \,अष्टौ द्विसंयुता अष्टादशैव स्युरित्यादि~। नन्वस्तु परिभाषामात्रमिदं तथापि किमनेन करणीषड्विधनिरूपणश्रमेण~। न ह्यस्ति लोके करणीभिर्व्यवहारः~।
किं तु  तदासन्नमूलैरेव~। तत्षड्विधं च रूपषड्विधेनैव गतार्थम्~। किं च कृतेऽपि
करणीगणितेऽन्तस्तदासन्नमूलेनैव व्यवहारस्तद्वरं प्रागेव तदादर इति
चेन्मैवम्~। प्रागेव स्थूलमूलग्रहणे तद्गुणनादावतिस्थूलता स्यात्~। कृते तु सूक्ष्मे
करणीगणिते पश्चात्तदासन्नमूलग्रहणे किञ्चिदेवान्तरं स्यान्न महदित्यस्ति
महान्विशेष इति करणीषड्विधमवश्यमारम्भणीयम्~। तद्यद्यपि
व्यक्तषड्विधान्तरङ्गत्वात्वर्णषड्विधात्प्रागेवारब्धुं युक्तं 
तथाप्येतस्य निरूपणावगमयोः प्रयासगौरवात्सूचीकटाहन्यायेन
वर्णषड्विधानन्तरमप्यारम्भो युक्त एव~।\\

\vspace{-2mm}
 
{\bqt तत्र सङ्कलनव्यवकलनयोः करणसूत्रं वृत्तद्वयम्}\textendash 

\phantomsection \label{34}
{\begin{quote}
\ab
योगं करण्योर्महतीं प्रकल्प्य \\
घातस्य\footnote{पृथक्} मूलं द्विगुणं लघुं च~।
\end{quote}
\thispagestyle{empty}
\afterpage{\fancyhead[CE] {बीजगणिते}}
\afterpage{\fancyhead[CO]{करणीषड्विधम्}}
\afterpage{\fancyhead[LE,RO]{\thepage}}
\cfoot{}
\newpage 
%%%%%%%%%%%%%%%%%%%%%%%%%%%%%%%%%%%%%%%%

{\begin{quote}
 {\ab योगान्तरे रूपवदेतयोस्ते \\
 वर्गेण वर्गं गुणयेद्भजेच्च~॥\\
लघ्व्या हृतायास्तु पदं महत्या \\
सैकं निरेकं स्वहतं लघुघ्नम्~। \\
योगान्तरे स्तः क्रमशस्तयोर्वा \\
पृथक्स्थितिः स्याद्यदि नास्ति मूलम्~॥~३४~॥}  
\end{quote}

\hyperref[34]{\textbf{करण्योर्योगेऽन्तरे}} \,वा कर्तव्ये \,रूपवत् कृतो \,यः करणीयोगः \,सा महती \,करणीति 
कल्पयेत्~। करण्योर्घातस्य मूलं द्विगुणं सा लघुः करणीति कल्पयेत्~।
तयोर्लघुमहत्योः कल्पितकरण्यो रूपवत्कृते ये योगान्तरे ते
प्रथमकरण्योर्योगान्तरे स्तः~। अथ \hyperref[27]{\textbf{अव्यक्तवर्गकरणीगुणनासु चिन्त्यः}} इत्यादिना {\qt 'भाज्याद्धरः शुध्यति'} इत्यादिना च करणीगुणनभजनयोः सिद्धावपि तत्र विशेषमाह\textendash \,\hyperref[34]{\textbf{वर्गेण वर्गं गुणयेद्भजेच्च}} इति~। एतदुक्तं भवति~। करणीगुणने कर्तव्ये यदि रूपाणां गुण्यत्वं गुणकत्वं वा 
स्यात् करणीभजने वा कर्तव्ये यदि रूपाणां भाज्यत्वं भाजकत्वं वा स्यात्तदा
रूपाणां वर्गं कृत्वा गुणनभजने कार्ये~। करण्या वर्गरूपत्वादिति~।
वर्गस्यापि समद्विघाततया गुणनविशेषत्वादुक्तवत् सिद्धिः~। {\qt 'स्थाप्योऽन्त्यवर्गो
द्विगुणान्त्यनिघ्नाः'} इत्यादिना व्यक्तोक्तप्रकारेण वा करणीवर्गस्यापि सिद्धिः स्यात्किन्तु
\hyperref[34]{\textbf{वर्गेण वर्गं गुणयेद्भजेच्च'}} इत्युक्तत्वाद्द्विगुणान्त्यनिघ्ना
इत्यत्र चतुर्गुणान्त्यनिघ्ना  इति द्रष्टव्यम्~। मूलज्ञानार्थं तु सूत्रं वक्ष्यति~। अथ प्रकारान्तरेण
लघ्व्या करण्या  हृताया महत्याः करण्या यत्पदं तदेकत्र \hyperref[34]{\textbf{सैक}}मपरत्र \hyperref[34]{\textbf{निरेकमु}}भयमपि वर्गितं लघुकरणीगणितं च क्रमेण करण्योर्योगान्तरे स्तः~। अत्र लघ्व्या महत्या भागे
यदि  भिन्नता स्यात्तदा मूलालाभे मूलार्थं यथासम्भवमपवर्तो द्रष्टव्यः~। अनया
युक्त्या महत्या हृताया लघ्व्याः पदेन रूपं युतोनं वर्गितं  च महतीघ्नं
योगान्तरे स्त  इति ज्ञेयम्~। अत्र करण्योर्मध्ये याङ्कतो लघुः सा लघुर्याङ्कतो महती सा महतीति ज्ञेयम्~। ननु पूर्वसूत्रोक्ता करणयोर्योगो महती घातस्य मूलं
द्विगुणं लघुरिति~। अत्र लघ्व्या अमहत्येति व्याख्येयम्~। अथ च महत्या लघ्व्या इति व्याख्येयम्~। अन्यथा करण्योः साम्येऽनेन सूत्रेण योगान्तरसिद्धिर्न
स्यादिति~। अत्र द्वयोर्मध्य एकया भक्तायाः परकरण्याः पदस्य रूपेण योगान्तरयोर्वगौ
भाजककरणीघ्नौ योगान्तरे स्त इति वक्तुं साधीयः~। ननु पूर्वसूत्रे घातस्य मूलमित्यत्र
पदग्रहणमुक्तम्~।
\newpage
%%%%%%%%%%%%%%%%%%%%%%%%%%%%%%%%%%%%%%%%%%%%%%%%%%%%%%%%%%%%%%%%%%%%

\noindent द्वितीयसूत्रेऽपि लघ्व्याः हृताया महत्याः पदमित्यत्र तदुक्तम्~। तत्र यदि
पदं न लभ्यते तर्हि योगान्तरे कथं कर्तव्ये इत्यत आह \hyperref[34]{\textbf{पृथक्स्थितिः स्याद्यदि नास्ति मूलम्}} इति~। स्पष्टोऽर्थः~॥~३४~॥\\

\vspace{-2mm}
{\bqt उदाहरणम्\textendash }

\phantomsection \label{35}
\begin{quote}
    {\eg 
    द्विकाष्टमित्योस्त्रिभसङ्ख्ययोश्च \\
    योगान्तरे ब्रूहि सखे\footnote{पृथक्} करण्योः~।\\
त्रिसप्तमित्योश्च चिरं विचिन्त्य \\
चेत्षड्विधं वेत्सि सखे करण्याः~॥~३५~॥}
\end{quote}

 स्पष्टोऽर्थः~। प्रथमोदाहरणे न्यासः\textendash \,क २ क ८~। अनयोर्योगो महती १०~। करण्योर्घातस्य १६ मूलम् ४~। द्विगुणं ८ लघुः~। क्रमेण
लघुमहत्योर्न्यासः\textendash \,लकृ ८ मक १०~। अनयोर्योगान्तरे रूपवत्कृते १८~। २ \hyperref[35]{\textbf{द्विकाष्टमित्योः}}
करण्योर्योगोऽष्टादश १८~। अन्तरं द्वयं २~। यो हि द्विकाष्टकयोर्मूलयोगः स एवाष्टादशानां मूलम्~। यत्तु द्विकाष्टयोर्मूलान्तरं तदेव द्विकमूलमित्यर्थः~।
अथात्र द्वितीयसूत्रेण योगान्तरे लघ्व्या २ हृताया महत्याः ८ लब्धं ४~। अस्य पदं सैकं निरेकं च 
३~। १~। द्वयोरपि वर्गो ९~। १~। लघु\textendash \,२\textendash \,घ्नौ च १८~। २ क्रमेण जाते त 
एव योगान्तरे~। अथ द्वितीयोदाहरणे न्यासः\textendash \,क ३ क २७~।अनयोर्योगो महती 
क ३०~। घातस्य ८१ मूलं ९ द्विगुणं १८ लघुः~। अनयोर्योगान्तरे ४८~। 
१२~। अथ द्वितीयप्रकारेण~। लघ्व्या हृताया महत्या लब्धं ९~। अस्य पदं ३
सैकं निरेकं च ४~। २ स्वहतं १६~। ४ लघु\textendash \,३ं\textendash \,गुणं ४८~। १२ जाते त एव 
योगान्तरे~। अथ तृतीयोदाहरणे न्यासः\textendash\ क ३ क ७~। अनयोर्योगो महती १०~। 
करण्योर्घातः २१~। अस्य मूलाभावात् \hyperref[34]{\textbf{पृथक्स्थितिः स्याद्यदि नास्ति मूलम्}} इति जाता पृथक्स्थितिः~। योगे क ३ क ७~। अन्तरे क ३ क ७~। अत्रोपपत्तिः\textendash \,करण्योर्मूलयोगो यस्य मूलं स किल करणीयोगः~। स तु मूलयोर्युतिवर्ग एव~।
\newpage 
%%%%%%%%%%%%%%%%%%%%%%%%%%%%%%%%%%%%%%%%%%%%%%%%%%%%%%%%
\noindent कथमन्यथा तस्य मूलं मूलयुतिः~। एवं करण्योर्मूलान्तरं यस्य मूलं तत्किल
करण्यन्तरम्~। तत्तु मूलान्तरवर्ग एव~। कथमन्यथा तस्य मूलं मूलान्तरं स्यात्~।
तत्र करण्यौ हि मूलवर्गौ~। अतः करण्योर्मूले गृहीत्वा तयोर्युतिवर्गः कर्तव्यः~।
स एव करणीयोगः स्यात्~। एवं करणीमूलान्तरवर्गः करण्यन्तरं स्यात्~। परं करण्या
मूलं न लभ्यते~। अतोऽन्यथा यतितव्यम्~। अत्र किल युतिवर्गोऽन्तरवर्गो वा
साध्यः~। स तु वर्गयोगोपलम्भे सुबोधः~। वर्गयोगस्तु करणीयोग एव~।
करण्योर्वर्गरूपत्वात्~।\\

\vspace{-4mm}
 ननु वर्गयोगावगमेऽपि कथं युतिवर्गोऽन्तरवर्गो वा
सुबोधस्तयोर्वैलक्षण्यादिति चेत् उच्यते~। वर्गयोगो द्विगुणितघातेन युक्तो युतिवर्गो भवति~। यथा राशी ३~। ५~। अनयोर्वर्गयोगः ३४~। द्विगुणितघातेनानेन ३० युतो ६४ जातो युतिवर्गः~। वा
राशी ३~। ७~। अनयोर्वर्गयुतिः ५८~। द्विघ्नघातेन ४२ युतो १०० जातो 
युति\textendash \,१०\textendash \,वर्गः~। एवं सर्वत्र~। तथा वर्गयोगो द्विघ्नघातेन हीनोऽन्तरवर्गो
भवति~। यथा राशी ४~। २~। अनयोर्वर्गयोगः २०~। द्विघ्नघातेन १६ हीनो जातो\textendash \,४\textendash \,ऽन्तर\textendash \,२\textendash \,वर्गः ४~। वा राशी ३~। ८~। अनयोर्वर्गयोगः ७३~। द्विघ्नघातेन ४८ हीनो जातो\textendash \,२५\textendash \,ऽन्तरं ५ वर्गः~। एवं सर्वत्र~। तस्मात् वर्गयोगो 
द्विघ्नघातयुतो युतिवर्गो भवति द्विघ्नघातेन हीनोऽन्तरवर्गो भवतीति सिद्धम्~। अत्र मूलयोर्वर्गयोगः करणीयोग एव~। असौ करणीद्वयमूलघातेन द्विघ्नेन
योज्यो युतिवर्गार्थं वियोज्यश्चान्तरवर्गार्थम्~। तत्र यः करणीमूलयोर्घातः स एवं
करणीघातमूलम्~। अतः सुष्ठूक्तं \hyperref[34]{\textbf{योगं करण्योर्महतीं प्रकल्प्य घातस्य मूलं द्विगुणं लघुं च~। योगान्तरे रूपवदेतयोस्ते}} इति~।\\

\vspace{-4mm}
 ननूपपत्त्या विना वर्गयोगो द्विघ्नघातेन युतो हीनो वा
युतिवर्गोऽन्तरवर्गो वा 
भवतीत्येतदेव कथं[चितं] क्कचिदृर्शनं त्वप्रयोजकम्~। अन्यथा
चतुर्गुणो राशिघातो युतिवर्गो भवतीत्यपि सुवचम्~। तस्यापि क्वचित्तथा दर्शनात्~।
तथाहि\textendash \,राशी २~। २~। अनयोर्घातः ४ चतुर्गुणः १६~। अयं जातो युति\textendash \,४\textendash \,वर्गः १६~। 
वा राशी ३~। ३ अनयोर्घातश्चतुर्गुणः ३६~। अयमेव युति\textendash \,६\textendash \,वर्गश्च ३६~। 
वा राशी ४~। ४~। अनयोर्घातः १६~। चतुर्गुणः ६४~। अयमेव युति\textendash \,८\textendash \,वर्गः
६४ इत्यादिषु~। तस्मात्क्वचिद्दर्शनमप्रयोजकं क्वचिद्व्यभिचारस्यापि सम्भवात्~। अतो वर्गयोगो द्विघ्नघातयुतोनो युतिवर्गोऽन्तरवर्गश्च भवतीत्यत्र
युक्तिर्वक्तव्येति चेत्सत्यम्~। इयमुपपत्तिरेकवर्णमध्यमाहरणान्ते
\hyperref[131]{\textbf{वर्गयोगस्य यद्राश्योर्युतिवर्गस्य चान्तरम्~।}}
\newpage
%%%%%%%%%%%%%%%%%%%%%%%%%%%%%%%%%%%%%%%%%%%%%%%%%%%%%%
\noindent \hyperref[131]{\textbf{द्विघ्नघातसमानं स्यात्}} इत्यत्र~। तथा {\qt 'राश्योरन्तरवर्गेण द्विघ्नो घातः समन्वितः~। वर्गयोगसम स स्यात्'} इत्यत्राप्याकर एव स्फुटीभविष्यति~। विवरिष्यते
चास्माभिस्तत्रैवेति नेह निरूप्यते~। अथ वर्गयोर्य एवं मूलघातः स एव
घातमूलमित्यत्र युक्तिरुच्यते~। वर्गघातो हि चतुर्घातः~। वर्गस्य
समद्विघातरूपत्वात्~। 
एवमेकस्य समराशिद्वयस्येतरस्य च समराशिद्वयस्य घात इति चतुर्घातो वर्गघातः~। यथा राशी ३~। ५~। अनयोर्वर्गघातार्थं घातवर्गार्थं वा राशिचतुष्टयेन
भाव्यम्~। ३~। ३~। ५~। ५~। अत्रघातद्वयमेवं ९~। २५ एवं वा १५~। १५~। राश्योर्घातौ
राशिवर्गौ वा~। अत्र वर्गयोः ९~। २५ घाते २२५ घातयोर्वा १५~। १५ 
समयोर्घाते २२५ पूर्वचतुष्कस्य घातोऽस्ति~। अतो वर्गघातस्य घातवर्गस्य
चाभेदाद्यदेव घातवर्गस्य मूलं तदेव वर्गघातस्यापि~। तत्र घातवर्गस्य मूलं घात एव
भवेदिति वर्गघातस्यापि मूलं घात एव~। अत उपपन्नं य एव मूलघातः स एव 
घातमूलमिति~।\\

\vspace{-4mm}
 अथ द्वितीयसूत्रोपपत्तिः~। अत्रापि करण्योर्मूलयुतिवर्गो मूलान्तरवर्गो
वा साध्योऽस्ति~। करण्योस्तु मूलं न लभ्यते~। अतः करणीद्वयं तथापवर्तनीयं यथा मूलं लभ्येत~। परं तथा मूललाभेऽपि तयोर्युतिवर्गोऽन्तरवर्गो वा करण्यपवर्तेनापवर्तितः
स्यात्~। यतोऽपवर्तितकरण्या मूलमपवर्ताङ्कमूलेनापवर्तितं स्यात्~। एवं
द्वितीयकरण्या अपि~। तयोर्मूलयोर्युतिरप्यपवर्तमूलेनैवापवर्तिता स्यात्
युतेर्वर्गस्त्वपवर्तमूलवर्गेणापवर्तितः स्यात्~। अपवर्तमूलवर्गस्त्वपवर्त एव~। अतो युतिवर्गोऽन्तरवर्गो वापवर्ताङ्केन गुणनीय इति 
युक्तिरस्ति~। अथापवर्तो विचारणीयः~। करण्याः केनापवर्तेन मूललाभः
स्यादिति~। तत्र करण्यङ्केनैव करण्या अपवर्ते रूपमेव स्यात्तस्य चावश्यं मूललाभः~।
तत्र यदि महत्याः करण्या अपवर्तः क्रियते तदा लघ्व्याः करण्या अपवर्तो न स्यात्~। अत आचर्येण लघ्व्याः करण्या अपवर्तः कृतः~। तथा सति जातं लघुस्थाने रूपम्~। 
महत्यति लघ्व्यापवर्त्य ततो मूलं च ग्राह्यमत उक्तं \hyperref[34]{\textbf{लघ्व्या हृतायास्तु पदं महत्याः}} इति~। इदमपवर्तितमहत्याः पदम्~। अपवर्तितलघ्व्यास्तु रूपमेव पदम्~। अनयोर्युतावन्तरे वा कर्तव्ये महतीपदं सैकं निरेकं वा भवेत्~।
द्वितीयपदस्य रूपत्वात्~। अत उक्तम् \hyperref[34]{\textbf{सैकं निरेकम्}} इति~। एवं जाता मूलयुतिर्मूलान्तरं च~। अथानयोर्वर्गो विधेयः~। अत उक्तम् \hyperref[34]{\textbf{स्वहतम्}} इति~। एवं जातो युतिवर्गोऽन्तरवर्गश्च~। परमपवर्तित एव~। अतोऽपवर्तनेन लघुकरण्या द्वयमेतद्गुणनीयम्~। अत
\newpage
%%%%%%%%%%%%%%%%%%%%%%%%%%%%%%%%%%%%%%%%%%%%%%%%%%%%%%%%%%%%%%%%%%%
\noindent उक्तम् \hyperref[34]{\textbf{लघुघ्नम्}} इति~। इदमुपलक्षणम्~। येनापवर्ते करण्योर्मूले लभ्येते
तेनापवर्त्य करण्योर्मूले ग्राह्ये~। तयोर्युतिवर्गोऽन्तरवर्गो वापवर्ताङ्केन
गुणितः सन् भवेदेव करण्योर्योगान्तरं चेत्यादि सुधीभिरूह्यम्~। अथ \hyperref[34]{\textbf{वर्गेण वर्गं गुणयेत्}} इत्यत्रोपपत्तिः~। 
इह हि करणीषड्विधेन तन्मूलयोरेव षड्विधं साध्यते~। यथा
द्विकाष्टमित्योः करण्योर्योगस्य दशत्वे सत्यपि मूल-योगार्थं तस्याष्टादशत्वमेव
निरूपितमित्यादि~। 
तद्वदिहापि करण्या द्व्यादिगुणत्वं तथा सम्पादनीयं \,यथा \,तत्पदं \,द्व्यादिगुणं \,भवति~। 
तत्र \,द्व्यादिभिरेव \,करणीगुणने \,तत्पदं द्व्यादिगुणं न भवति किन्तु 
द्व्यादिवर्गेण तद्गुणने~। यथा राशिः ४~। एतस्य द्विगुणत्वेऽभीप्सिते
चेदस्य वर्गो १६ द्विगुणः ३२ क्रियते तर्ह्यस्य पदं द्विगुणो राशिर्न भवति~। राशि\textendash \,४\textendash \,वर्गे १६ द्विवर्गेण ४ गुणिते तु ६४ तत्पदं ८ भवति द्विगुणो राशिः~। एवं 
त्र्यादिगुणत्वेऽपि द्रष्टव्यम्~। अत उपपन्नम् \hyperref[34]{\textbf{वर्गेण वर्गं गुणयेत्}} इति~। एवं  भजनेऽप्युपपतिर्द्रष्टव्या~। अस्ति चाचार्येण पाट्यामुक्तं {\qt वर्गे कृती घनविधौ तु घनौ विधेयौ हारांशयोरपि पदे च पदप्रसिद्ध्यै} इति~। उपपादितं चास्माभिस्तद्व्याख्यावसरे~॥~३५~॥\\

\vspace{-2mm}
{\bqt गुणनोदाहरणम्\textendash }

\phantomsection \label{36}
\begin{quote}
    \eg 
    द्वित्र्यष्टसङ्ख्यागुणकः करण्योर्गुण्यस्त्रिसङ्ख्या च सपञ्चरूपा~। \\
वधं प्रचक्ष्वाशु विपञ्चरूपे गुणेऽथवा त्र्यर्कमिते करण्यौ~॥~३६~॥
\end{quote}

 अत्र पञ्चरूपसहिता \hyperref[36]{\textbf{त्रिसङ्ख्या}} करणीगुण्या~। \hyperref[36]{\textbf{गुणकस्तु द्वित्र्यष्टसङ्ख्याः}} करण्या पञ्चरूपोने त्र्यर्कमिते करण्यौ वा~। अत्र गुणकद्वयादुदाहरणद्वयं
ज्ञेयम्~। अथ प्रथमोदाहरणे न्यासो गुणकः क २ क ३ क ८~। गुण्यो रू ५ क ३
\hyperref[34]{\textbf{वर्गेण वर्गं गुणयेत्}} इति करण्या वर्गरूपत्वाद्रूपाणामपि वर्गे कृते जातो
गुण्यः क २५ क ३~। यथा खण्डैः पृथग्गुणितः सहितश्च गुण्यो गुणनफलं भवति तथा खण्डयोगेनापि गुणितो भवत्येवेति प्रसिद्धम् अतो गुणके द्विकाष्टमित्योः
करण्योर्योगे कृते जातो गुणकः क १८ क ३~। गुण्यः पृथग्गुणकखण्डसमो निवेश्य
इति गुणनार्थः-
\newpage
%%%%%%%%%%%%%%%%%%%%%%%%%%%%%%%%%%%%%%%%%%%%
\noindent $\begin{matrix}
\vspace{-1mm}
\mbox{{क १८~। क २५ क ३~।}}\\
\vspace{-1mm}
\mbox{{क ~३~। क २५ क ३~।}}
\vspace{1mm}
\end{matrix}$ गुणनेन जातं क ४५० क ५४ क 
७५ क ९~। करणीनवकस्य मूलं लभ्यत इति मूले गृहीते जीतं गुणनफलं रू ३ 
क ४५० क ५४ क ७५~। अथ द्वितीयोदाहरणे न्यासो गुणकः रू ५ं क ३ 
क १२~। गुण्यः क २५ क ३~। अत्र गुणके त्र्यर्कमितयोः करण्योर्योगे जातं 
क २७~॥~३६~॥\\

\vspace{-2mm}
{\bqt विशेषसूत्रं वृत्तम्\textendash }

\phantomsection \label{37}
\begin{quote}
 \ab 
    क्षयो भवेच्च क्षयरूपवर्गश्चेत्साध्यतेऽसौ करणीत्वहेतोः~। \\
ऋणात्मिकायाश्च तथा करण्या मूलं क्षयो रूपविधानहेतोः~॥~३७~॥
\end{quote}

 क्षयरूपाणां वर्गस्तर्हि क्षयो भवेत्~। असौ
क्षयरूपवर्गश्चेत्करणीत्वनिमित्तं साध्यते~। न मूलक्षयोऽस्यास्तीत्यस्यापवादमाह\textendash \,\hyperref[37]{\textbf{ॠणात्मिकायाः}} इति~।
ऋणात्मिकायाः करण्या मूलं तर्हि क्षयो भवेच्चेन्मूलं रूपविधाननिमित्तं साध्यत इति~।
अत्रोपपत्तिः~। 
अत्र किल रूप-वर्गः करणीगुणनार्थं क्रियते~। स यद्यपि धनमेव तथापि तस्य
मूलमृणमेव~। \hyperref[13]{\textbf{स्वमूले धनर्णे}} इत्युक्त्वात्~। करणीयोगेन च मूलयुतिवर्गः साध्यते~। 
तत्र क्षयरूपवर्गकरण्या यदि धनत्वं कल्प्यते तदान्यया धनकरण्या सह
योगः स्यात्~। 
तस्य च मूलं मूलयुतिरेव~। भाव्यं च मूलान्तरेण~। \hyperref[3]{\textbf{धनर्णयोरन्तरमेव योगः}} इत्युक्तत्वात्~। तस्मात्करण्या ऋणसञ्ज्ञा मूलस्यर्णत्वबोधार्थमेव कृता~। बालावबोधार्थमिदमुदाह्रियते रू ३ रू ७ं~। अनयोर्युति\textendash \,४ं\textendash \,वर्गः १६ तावदयम्~। स च 
करण्या धनत्वे कल्पिते सति न सिध्यति~। यथा उदाहृतरूपयोः करण्योः क ९ 
क ४९~। \hyperref[34]{\textbf{योगं करण्योर्महतीं प्रकल्प्य}} इत्यादिना जातो योगः क १००~। न 
ह्ययं युतिवर्गस्तस्मादृणत्वं कल्प्यते~। तस्माद्यदि करणीयोगादिकं न
साध्यते तदा क्षय-रूपवर्गो धनमेव~। अत्र करणीत्युपलक्षणम्~। यत्र
वर्गयोगात्करणीयोगवद्युतिवर्गादिकं साध्यते तत्र क्षयरूपवर्गः क्षय एव कल्पनीय इति ध्येयम्~। अलमतिविस्तरेण~। 
प्रकृतमनुसरामः~। गुणकः रू ५ं क ३ क १२~। करणीयोगः क २७~।
रूपवर्गः क्षयः क २५~। एवं जातो गुणकः क २ं५ क २७~। गुण्यः क २५ क
\newpage
%%%%%%%%%%%%%%%%%%%%%%%%%%%%%%%%%%%%%%%%%%%%%%%%%%%%%%%%
\noindent ३~। गुणनार्थं न्यासः $\begin{matrix}
\vspace{-1mm}
\mbox{{क २५ं~। क २५ क ३~।}}\\
\vspace{-1mm}
\mbox{{क २७~। क २५ क ३~।}}
\vspace{1mm}
\end{matrix}$ गुणनाज्जातं क ६२ं५ क ७ं५ क ६७५ क ८१~। प्रथमचतुर्थ्योः करण्योर्मूले रू २ं५
रू ९~। अनयोर्योगो रू १ं६~। इतरकरण्योरन्तरं क ३००~। एवं जातं गुणनफलं 
रू १ं६ क ३००~। अथ भजनोदाहरणम्~। पूर्वगुणनफलस्य स्वगुणच्छेदस्य
न्यासः\textendash \,
क ९ क ४५० क ७५ क ५४~। भाजके द्विकाष्टमित्योः करण्योर्योगे जातो भाजकः $\begin{matrix}
\vspace{-1mm}
\mbox{{क २ क ३ क ८}}\\
\vspace{-1mm}
\mbox{{क ३ क १८~~~}}
\vspace{1mm}
\end{matrix}$~। अथ \hyperref[29]{\textbf{भाज्याच्छेदः शुध्यति}} इत्यादिना लब्धिर्ग्राह्या~। अत्र
भाज्ये  प्रथमतः करणीनवकमस्ति~। भाजके त्रिगुणिते तच्छुध्येदिति
भाजकस्त्रिभिर्गुणितः 
क ९ क ५४~। अस्य शोधनेन प्रथमचतुर्थ्योर्भाज्यकरण्योः शुद्धिः~। अतो
लब्धिः क ३~। अथ भाज्यशेषं क ४५० क ७५ पुनर्भाजके पञ्चविंशतिगुणे क ७५ 
क ४५० भाज्यशेषाद्यथासम्भवमपनीते शुद्धिरस्तीति जाता लब्धिः क २५ ~। एतस्या 
मूलं लभ्यत इति गृहीतं मूलं रू ५~। एवं जाता लब्धिः रू ५ क ३~। अथ 
द्वितीयोदाहरणे भाज्यः क २५ं६ क ३००~। भाजकः क २ं५ क ३ क 
१२~। करण्योर्योगे जातो भाजकः क २ं५ क २७~। अत्र पूर्वगुण्येनानेन क २५
लब्ध्या भाव्यम्~। अतस्त्रिगुणो भाजकः \hyperref[7]{\textbf{संशोध्यमानं स्वमृणत्वमेति}} इति
जातः क ७५ क ८ं१~। अत्र भाज्यभाजकगतयोर्धंनर्णकरण्योरन्तरं न भवितं
मूलाभावात्~। अतो भाज्यभाजकधनकरण्योः क ३०० क ७५ ऋणकरण्योश्च क २५ं६ क ८ं१
योगे जातं भाज्यशेषं क ६७५ क ६२ं५~। अस्मात्पञ्चविंशतिगुणे भाजके क ६२ं५ क ६७५
अपनीते शुद्धिरस्तीति जाता लब्धिः क ३ क २५~। मूले गृहीते जाता लब्धिः
त्रिसङ्ख्या च सपञ्चरूपेति रू ५ क ३~॥~३७~॥\\

\vspace{-3mm}
 अत्र द्वितीयोदाहरणे भाजकः कियद्गुणो \hyperref[29]{\textbf{भाज्याच्छेदः शुध्यति}} इति
दुःखबोधमतः परमकारुणिकैराचार्यैः शिष्यबोधार्थमुपायान्तरमुपजातिकाद्वयेन
निरूप्यते\textendash 

\phantomsection \label{38}
\begin{quote}
    \ab 
    धनर्णताव्यत्ययमीप्सितायाश्छेदे करण्या असकृद्विधाय~। \\
तादृक्छिदा भाज्यहरौ निहन्यादेकैव यावत्करणी हरे स्यात्~॥\\
भाज्यास्तया भाज्यगताः करण्यो लब्धाः करण्यो यदि
\end{quote}
\newpage
%%%%%%%%%%%%%%%%%%%%%%%%%%%%%%%%%%%%%%%%%%%
\begin{quote}
    \ab 
    योगजाः स्युः~।\\
विश्लेषसूत्रेण पृथक् च कार्या यथा तथा प्रष्टुरभीप्सिताः स्युः~॥~३८~॥
\end{quote}

\hyperref[38]{\textbf{छेदे ईप्सिताया}} एकस्याः \hyperref[38]{\textbf{करण्या}} धनर्णताविपर्यासं कृत्वा तादृशेन
च्छेदेन यथास्थितौ \hyperref[38]{\textbf{भाज्यहरौ}} गुणयेत्~। एवं कृते करणीनां यथोक्त्या योगे च कृते
भाज्यभाजकौ स्तः~। अथास्मिन्नपि भाजके यदि द्व्यादीनि करणीखण्डानि
स्युस्तदात्रापि एकस्याः करण्या धन-र्णताविपर्यासं कृत्वा तादृशभाजकेन
पूर्वगुणनसम्पन्नौ भाज्यभाजकौ गुणयेत्~। तत्रापि यथासम्भवं करणीयोगे कृते तौ
भाज्यभाजकौ स्तः~। एवमसकृत्तावद्विधेयं यावद्भाजके एकैव करणी भवेत्~। अथ
सम्पन्नया भाजककरण्या सम्पन्नभाज्यकरण्यो रूपवदेव भाज्या~। यल्लभ्यते
ता लब्धिकरण्यो भवन्ति~। अथ यदि लब्धाः करण्यो योगजाः स्युर्न पुनः
प्रष्टुरभीप्सितास्तदा वक्ष्यमाणविश्लेषसूत्रेण तथा पृथक्कार्या यथा
प्रष्टुरभीप्सिताः स्युः~। द्वितीयोदाहरणे भाज्यः क २५ं६ क ३०० भाजकः क
२ं५ क २७~। अत्र पञ्चविंशतिकरण्या ऋणत्वव्यत्यासं कृत्वा जातो हरः क २५
क २७ अनेन हरेण यथास्थितौ भाज्यहारौ गुणयेदिति गुणनार्थं न्यासः\\
\vspace{-3mm}

$\begin{matrix}
\vspace{-1mm}
\mbox{{क २५~। क २५ं६ क ३००~। क २५~। क २ं५~। क २७~।}}\\
\vspace{-1mm}
\mbox{{क २७~। क २५ं६ क ३००~। क २ं७~। क २ं५~। क २७~।}}
\vspace{1mm}
\end{matrix}$\\
\vspace{-2mm}

\noindent भाज्ये गुणिते जातं क ६४ं०० क ७५०० क ६९ं१२ क ८१०० प्रथमचतुर्थ्योर्द्वितीयतृतीययोश्च योगे जातं भाज्ये करणीद्वयं क १०० क १२~। भाजके गुणिते जातं 
क ६२ं५ क ६७५ क ६७ं५ क ७२९~। अत्रापि प्रथमचतुर्थ्योर्द्वितीयतृतीययोश्च योगे
जातं  क ४ क ०~। एवं हरे जाता करण्येकैव क ४~। अनया भाज्यकरण्यौ क १०० क 
१२ भक्ते लब्धिः क २५ क ३~। एवं पूर्वोदाहरणेऽपि न्यासः~। भाज्यः क ९
क ४५० क ७५ क ५४~। भाजकः क १८ क ३ं~। अत्र च्छेदे त्रिमितकरण्या 
ऋणत्वं प्रकल्प्य तादृशच्छेदेनानेन क १८ क ३ं भाज्यभाजकयोर्गुणनार्थं
न्यासः\textendash \\
\vspace{-3mm}

$\begin{matrix}
\vspace{-1mm}
\mbox{{क ९ क ४५० क ७५ क ५४~। क १८। क १८ क ३~। क १८~।}}\\
\vspace{-1mm}
\mbox{{क ९ क ४५० क ७५ क ५४~। क ~३ं~। क १८ क ३~। क ~३ं~।}}
\vspace{1mm}
\end{matrix}$\\
\vspace{-2mm}

\noindent भाज्ये गुणिते जातं क १६२ क ८१०० क १३५० क ९७२ क २ं७ क १३ं५० क २२ं५ क १६ं२~। अत्र तुल्ययोर्धनर्णकरण्योर्योगेन शुद्धौ सत्यां शेषं करणीचतुष्टयं
\newpage
%%%%%%%%%%%%%%%%%%%%%%%%%%%%%%%%%%%%%%%%%%%%%%%%%%%%%%%

\noindent क ८१०० क २२ं५ क ९७२ क २ं७~। अत्र प्रथमद्वितीययोस्तृतीयचतुर्थ्योश्च 
योगे जातं भाज्ये करणीद्वयं क ५६२५ क ६७५~। एवं भाजके गुणिते जातं 
क ३२४ क ५ं४ क ५ं४ क ९ं~। अत्रानयोः क ५ं४ क ५४ योगे जाता 
शुद्धिः~। इतरयोः ३२४~। ९ं योगे जाता करणी २२५~। एवं हरकरण्येकैव 
जाता क २२५~। अनया भाजककरण्या हृते लब्धिः क २५ क ३~। एवं लब्धा 
करणी यदि योगजा स्यात्तदा विश्लेषसूत्रेण पृथक्कार्या~। तत्रोदाहरणम्~।
भाज्यः क ९ क ४५० क ७५ क ५४ भाजकः क २५ क ३~। अत्र भाजके त्रिमितकरण्या 
ऋणत्वं प्रकल्प्य तादृशहरेण भाज्यहरयोर्गुणनार्थं न्यासः\textendash \, \\
\vspace{-3mm}

$\begin{matrix}
\vspace{-1mm}
\mbox{{क २५~। क ९ क ४५० क ७५ क ५४~। क २५~। क २५ क ३~।}}\\
\vspace{-1mm}
\mbox{{क ~३ं~। क ९ क ४५० क ७५ क ५४~। क ~३ं~। क २५ क ३~।}}
\vspace{1mm}
\end{matrix}$\\
\vspace{-2mm}

\noindent भाज्ये गुणिते क २२५ क ११२५० क १८७५ क १३५० क २७ं क १३ं५० क २२ं५ क १६ं२~। अत्र धनर्णकरणीनां साम्यान्नाशे शेषकरण्यः क २ं७~। १८७५ 
क ११२५० क १६ं२~। आस्वनयोः क २७ क १८७५ अनयोश्च क ११२५० क 
१६ं२ योगे जातं भाज्ये करणीद्वयं \,क \,१४५२ \,क \,८७१२~। एवं \,हरे \,गुणिते \,जातं \,क \,६२५ \,क \,७५ \,क \,७ं५ क ९ं~। अत्रापि तुल्ययोर्धनर्णकरण्योर्नाशे परयोः क 
६२५ क ९ं योगे जातैकैव भाजककरणी क ४८४~। अनया भाज्यकरण्योर्भजने जाता
लब्धिः क ३ क १३~। अत्र किल द्वित्र्यष्टसङ्ख्यागुणकः
करण्योर्गुण्यस्त्रिसङ्ख्या च सपञ्चरूपा~। अनयोर्वधो भाज्य-त्वेनोदाहृतः~।
तयोरेकतरेणास्य भजनेऽन्यतरो लब्धिः स्यात्~। प्रकृते तु सपञ्चरूपया
त्रिसङ्ख्यया ह्रियतेऽतो द्वित्र्यष्टकरणीभिः फलेन 
भाव्यम्~। उक्तरीत्या त्वियं लब्धिः क १८ क ३~। एतन्मध्ये इयं क ३ 
अभीष्टा~। इतरत्करणीद्वयमपेक्षितम्~। अत इयं योगकरणी क १८ पृथक्कार्या~॥~३८~॥ \\

\vspace{-2mm}
{\bqt विश्लेषसूत्रं वृत्तम्\textendash }

\phantomsection \label{39}
\begin{quote}
    \ab 
    वर्गेण योगकरणी विहृता विशुध्येत्\\
    खण्डानि तत्कृतिपदस्य यथेप्सितानि~। \\
कृत्वा तदीयकृतयः खलु पूर्वलब्ध्या \\
क्षुण्णा भवन्ति पृथगेवमिमाः करण्यः~॥~३९~॥ 
\end{quote}
\newpage
%%%%%%%%%%%%%%%%%%%%%%%%%%%%%%%%%%%%%%%%%%%%%%

\hyperref[39]{\textbf{योगकरणी}} येन \hyperref[39]{\textbf{वर्गेण विहृता}} सती \hyperref[39]{\textbf{विशुध्येत्तत्कृतिपदस्य यथेप्सितानि 
खण्डानि कृत्वा तदीयकृतयः पूर्वलब्ध्या क्षुण्णाः पृथक्करण्यो 
भवन्ति~।}} सा चासौ कृतिश्चेति कर्मधारयो द्रष्टव्यः~। एतदुक्तं भवति~।
योगकरणी येन वर्गेण विहृता सती निःशेषा भवेत्तस्य वर्गस्य मूलं ग्राह्यम्~।
तस्य  खण्डानि प्रष्टुर्यावन्त्यभीष्टानि तावन्ति कृत्वा तेषां खण्डानां वर्गाः
कर्तव्याः~। ते पूर्वलब्ध्या क्षुण्णाः~। वर्गेण विहृतायां योगकरण्यां या लब्धिः सा
पूर्वलब्धिः~। 
तथा गुणितास्ते वर्गाः पृथक्करण्यो भवन्ति~। प्रकृतोदाहरणे योगकरणी क १८
इयमनेन वर्गेण ९ विहृता सती शुध्यति~। लब्धिश्च २~। वर्गस्य ९ पदं ३~। अस्य
खण्डे  १~। २~। अनयोर्वगौ १~। ४~। पूर्वलब्ध्या २ गुणितौ २~। ८~। 
जाते करणीखण्डे क २ क ८~। एवं पूर्वकरण्या क ३ सह जाता
द्वित्र्यष्टसङ्ख्या लब्धिकरण्यः~। एवं प्रष्टुर्यदि खण्डत्रयमभीष्टं स्यात्तर्हि
वर्गपदस्यास्य ३ खण्डत्रयं १~। १~। १~। एभ्यः पूर्ववज्जातीनि करणीखण्डानि २~। २~। २~। एतासामपि 
करणीनां योगे करणी सैव भवति क १८~। एवं प्रष्टुरिच्छावशादन्यान्यपि 
खण्डानि कार्याणि~। एवमन्यत्रापि द्रष्टव्यम्~।\\

\vspace{-4mm}
 अथ \hyperref[38]{\textbf{धनर्णताव्यत्ययमीप्सितायाः}} इत्यत्र युक्तिः~।
तुल्येनाङ्केनापवर्तितयोर्गुणितयोर्वा भाज्यभाजकयोः फले वैषम्याभाव इति
तावत्प्रसिद्धम्~। तत्र हरकरणी यथैका भवति 
तथा भाज्यभाजकौ गुणनीयावपवर्त्यौ वा~। तथा सति भजनं सुगमं स्यात्~।
तत्रापवर्ते विचारगौरवम् अस्ति~। यथा भाजककरण्योः केनापवर्ते कृत एकैव करणी
स्यात् इति विचारणीयम्~। पुनस्तेनाङ्केन भाज्यकरणीनामपर्वतः सम्भवति न वेति विचारणीयमिति~। अतः केनचिद्भाज्यभाजकौ गुणनीयौ~। तत्र भाजकतुल्यो गुणकः कृतः~। तथा
सति भाजकगुणने वर्गत्वात्खण्डवर्गौ खण्डाभिहतिद्वयं च स्यात्~। तत्र
वर्गरूपयोः करणीखण्डयोर्मूललाभादवश्यं तयोर्योगे एकैव करणी स्यात्~। परं
खण्डाभिहतिर्द्वयावशिष्टं स्यात्~। \,अत \,आचार्येणैकस्या \,गुणककरण्याः \,धनर्णताव्यत्यास \,उक्तः~। \,तथा \,सति \,खण्डवधयोर्मध्य एकस्य धनत्वमितरस्यर्णत्वमिति तयोर्योगे नाशः स्यात्~।
एवं हरे त्वेकैव करणी स्यात्~। हरस्य गुणितत्वाद्भाज्यगुणनमावश्यकमित्युपपन्नं
धनर्णताव्यत्ययमित्यादि~। एवं त्र्यादिखण्डेष्वप्यूह्यम्~। तत्र
खण्डबाहुल्याद्युगपत्तन्नाशो न भवतीत्यसकृदित्युक्तम्~। अथ
विश्लेषसूत्रोपपत्तिः सा च करणीयोगद्वितीयसूत्रव्यत्यासेन यथा 
करण्यौ करण्यो वा केनचिदपवर्त्य तन्मूलयुतिवर्गोऽपवर्ताङ्केन गुणितः सन्
योगकरणी
\newpage
%%%%%%%%%%%%%%%%%%%%%%%%%%%%%%%%%%%%%%%%%%%%%%%%%
\noindent भवति~। तथा च या या योगकरणी सा सा युतिवर्गापवर्ताङ्कयोराहतिः~। अतः सा 
वर्गेण विहृता विशुध्येदेव~। लब्धिस्त्वपवर्ताङ्क एव स्यात्~। येन वर्गेण विहृता 
विशुध्येत्स युतिवर्ग एव~। तस्य पदं मूलयुतिः स्यात्~। युतेः
खण्डान्यपवर्तितकरणीनां मूलानि स्युः~। तेषां वर्गा अपवर्तितकरण्यः स्युः~। 
एता अपवर्तगुणिता यथास्थितकरण्यः स्युः~। अपवर्ताङ्कस्तु पूर्वलब्धिरेव~। अतः 
सुष्ठूक्तम् \hyperref[39]{\textbf{वर्गेण योगकरणी विहृता विशुध्येत्}} इत्यादि~॥~३९~॥\\

\vspace{-2mm}
{\bqt करणीवर्गादेरुदाहरणम्\textendash}
\begin{quote}
    \eg 
    द्विकत्रिपञ्चप्रमिताः करण्यस्तासां कृतिं द्वित्रिकसङ्ख्ययोश्च~। \\
षट्पञ्चकद्वित्रिकसंमितानां पृथक् पृथङ्मे कथयाशु विद्वन्~। \\
अष्टादशाष्टद्विकसंमितानां कृती कृतीनां च सखे पदानि~॥~४०~॥
\end{quote}

 स्पष्टोऽर्थः~। पूर्वोदाहरणे करण्यः क २ क ३ क ५~। वर्गस्य
समद्विघातरूपत्वादयमेव गुण्यो गुणकश्चेति गुणनार्थं न्यासः\textendash
\vspace{-1mm}

\begin{table}[h!]
    \centering\s
    \begin{tabular}{lllllllll}
        क २~।& क २& क ३& क ५~।& &क ४ &क ६ &क १०~। \\
क ३~।& क २& क ३& क ५~।& गुणिते जातं& क ६ &क ९ &क १५~। \\
क ५~। &क २ &क ३ &क ५~। &&क १०& क १५ &क २५~।
    \end{tabular}
\end{table}

\vspace{-3mm}
\noindent अत्रासां क ४ क ९ क २५ मूलानि २~। ३~। ५~। एषां योगः रू १०~। अन्यासां करणीनां मध्ये द्वयोर्द्वयोस्तुल्ययोर्योगे जाताश्चतुर्गुणाः
करण्यः क २४ क ४० क ६०~। एवं जातो वर्गो रू १० क २४ क ४० क ६०~। अथवा 
{\qt 'स्थाप्योऽन्त्यवर्गो द्विगुणान्त्यनिघ्नाः'} इत्यादिना वर्गो विधेयः~। तत्र करणीवर्गे चतुर्गुणान्त्यनिघ्ना इति बोध्यम्~। \hyperref[34]{\textbf{वर्गेण वर्गं गुणयेत्}} इत्युक्तत्वात्~।
न्यासः क २ क ३ क ५ {\qt 'स्थाप्योऽन्त्यवर्गः'} इत्यादिना जातानि वर्गखण्डानि क ४ 
क २४ क ४० क ९ क ६० क २५~। अत्र वर्गाणां मूलानि गृहीत्वा 
२~। ३~। ५ ऐक्यं च कृत्वा जातो वर्गो रू १० क २४ क ४० क ६०~। 
अथ द्वितीयोदाहरणे क २ क ३ {\qt 'स्थाप्योऽन्त्यवर्गः'} इत्यादिना क ४ क 
२४ क ९~। वर्गयोर्मूलैक्ये कृते जातो वर्गो रू ५ क २४~। अथ तृतीयोदा-
\newpage
%%%%%%%%%%%%%%%%%%%%%%%%%%%%%%%%%%%%%%
\noindent हरणे न्यासः क ६ क ५ क २ क ३~। उक्तवज्जातानि वर्गखण्डानि क 
३६ क १२० क ४८ क ७२ क २५ क ४० क ६० क ४ क २४ 
क ९~। आसु वर्गरूपाभ्यः करणीभ्यो मूलानि गृहीत्वा योगं च कृत्वा जातो 
वर्गो रू १६ क १२० क ४८ क ७२ क ४० क ६० क २४~। अथ 
चतुर्थोदाहरणे न्यासः क १८ क ८ क २~। उक्तवज्जातानि वर्गखण्डानि क 
३२४ क ५७६ क १४४ क ६४ क ६४ क ४~। सर्वेषां वर्गरूपत्वाज्जातानि 
मुलानि १८~। २४~। १२~। ८~। ८~। २~। एषां योगे जातो वर्गो रू 
७२~। यद्वा प्रथमत एव लाघवार्थं करणीयोगं कृत्वा पश्चाद्वर्गः कार्यः~।
यथा\textendash \,क १८ क ८ क २ द्विकाष्टमित्योर्योगः क १८~। पुनरस्याः क १८ पूर्वकरण्या 
क १८ योगे जाता करणी क ७२~। अस्या वर्गे जाता करणी ५१८४~। 
अस्या मूलं जातो वर्गो रू ७२~। एवमुदाहृतकरणीनां खण्डगुणनेनापि वर्गाः 
साध्याः~। एवं खण्डद्वयस्याभिहतिरित्यादिप्रकारद्वयेनापि वर्गाः साध्याः~॥~४०~॥\\

\vspace{-2mm}
{\bqt करणीमूले सूत्रं वृत्तद्वयम्\textendash \,}
\phantomsection \label{41}
\begin{quote}
    {\ab 
    वर्गे करण्या यदि वा करण्योः \\
    तुल्यानि रूपाण्यथवा बहूनाम्~। \\
विशोधयेद्रूपकृतेः पदेन \\
शेषस्य रूपाणि युतोनितानि ॥\\
पृथक्तदर्धे करणीद्वयं स्यात् \\
मूलेऽथ बह्वी करणी तयोर्या~। \\
रूपाणि तान्येव मतोऽपि\footnote{कृतानि} भूयः \\
शेषाः करण्यो यदि सन्ति वर्गे~॥~४१~॥}
\end{quote}

\hyperref[41]{\textbf{वर्गे करण्यास्तुल्यानि करण्योर्वा}} तुल्यानि \hyperref[41]{\textbf{बहूनां}} करणीनां वा तुल्यानि
\hyperref[41]{\textbf{रूपाणि रूपकृतेः शोधयेत्}}~। अत्र रूपग्रहणं योगवियोगयोः \hyperref[34]{\textbf{योगं करण्योर्महतीं प्रकल्प्य}} इत्यादि प्रकारस्य व्यावृत्त्यर्थम्~। शेषस्य पदेन \hyperref[41]{\textbf{रूपाणि पृथक् युतोनितानि}} कृत्वा \hyperref[41]{\textbf{तदर्धे}} कार्ये~। \hyperref[41]{\textbf{मूले तत्करणीद्वयं}} भवति~। \hyperref[41]{\textbf{यदि}} पुनर्वर्गे \hyperref[41]{\textbf{शेषाः करण्यः}}
\newpage 
%%%%%%%%%%%%%%%%%%%%%%%%%%%%%%%%%%%%%%%%
\noindent \hyperref[41]{\textbf{सन्ति}} तर्हि तयोर्मूलकरण्योर्मध्येऽल्पा मूलकरणी या महती तानि \hyperref[41]{\textbf{रूपाणि}} प्रकल्प्यातो 
रूपेभ्यो भूयोऽप्येवं करणीतुल्यानि रूपाणि रूपकृतेर्विशोधयेदित्यादिना
पुनरपि मूलकरणीद्वयं स्यात्~। पुनरपि यदि शेषाः करण्यो भवेयुस्तदैवमेव पुनः कुर्यात्~। अत्र महती रूपाणीत्युपलक्षणम्~। क्वचिन्महती मूलकरण्यल्पा तु रूपाणीत्यपि
द्रष्टव्यम्~। वक्ष्यति चाचार्यः चत्वारिंशदशीतिरित्युदाहरणावसरे~। अथ च महती
रूपाणीत्युपलक्षणं तेन क्वचिदल्पापीति~। अथ पूर्वसिद्धवर्गस्य मूलार्थं न्यासो रू १० क २४ क ४० क ६०~। अत्र रूपकृतेः १०० एककरणीतुल्यरूपशोधने शेषस्य पदाभावः~। करणीत्रितयस्य तुल्यरूपाणि तु न शुध्यन्ति~। अतः करणीद्वयतुल्यरूपाणि शोध्यानि~। करणीद्वयं त्वभीष्टम्~। इदं क २४ क ४० इदं वा क २४ क ६० इदं वा क ४० क ६०~। तत्र प्रथमकरणीद्वयं विशोध्य मूलं साध्यते~। रूपकृतेः १०० करणीद्वयं २४~। ४० तुल्यरूपाणि विशोध्य शेषं ३६ अस्य पदं ६ अनेन रूपाणि १० युतोनितानि १६~। ४ अर्धे ८~। २~। वर्गेऽन्यापि करण्यस्ति क ६०~। अतो 
महती मूलकरणी रूपाणि ८~। एषां वर्गः ६४~। अस्माच्छेषकरणीतुल्यरूपाणि 
६० विशोध्य शेषस्य ४ पदेन २ रूपाणि ८ युतोनितानि कृत्वा १०~। ६ 
अर्धे ५~। ३~। एवं जाता मूलकरण्यः क २ क ३ क ५~। एवं द्वितीयतृतीयकरणीद्वययोः प्रथमशोधनेनाप्येता एव मूलकरण्यो भवन्ति~। अथ
द्वितीयोदाहरणे न्यासो रू ५ क २४~। रूपकृतेः २५ करणीतुल्यरूपाणि २४ विशोध्य शेषस्य १ मूलेन १ रूपाणि ५ युतोनितानि ६~। ४ तदर्धे ३~। २ जाते मूलकरण्यौ 
क २ क ३~। अथ तृतीयोदाहरणे न्यासो रू १६ क १२० क ७२ क ६० क 
४८ क ४० क २४~। रूपकृतेः २५६ करणीत्रितयस्यास्य १२०~। ७२~। ४८ 
तुल्यानि रूपाणि विशोध्य शेषस्यास्य १६ पदेन ४ रूपाणि १६ युतोनितानि २०~। १२ तदर्धे १०~। ६~। अनयोरल्पा मूलकरणी क ६ 
महती रूपाणि १०~। एषां कृतेः १०० करणीद्वयं ६०~। २४ अपास्य शेषस्य १६ 
पदेन ४ रूपाणि १० युतोनितानि १४~। ६ तदर्धे ७~। ३~। अनयोरल्पा ३ 
मूलकरणी~। महती ८ रूपाणि~। एषां कृतेः ४९ करणी ४० तुल्यानि रूपाण्यपास्य 
शेषस्य ९ पदेन ३ रूपाणि ७ युतोनितानि १०~। ४ तदर्धे ५~। २ जाते मूलकरण्यौ क ५ क २~। एवं जाताः सर्वा मूलकरण्यः क ६ क ३ क ५ क २~। अथ चतुर्थोदाहरणे न्यासः रू ७२ क ०~। रूपकृतेः ५१८४ करणी ० विशोध्य शेषस्य ५१८४ पदेन ७२ रूपाणि ७२ युतोनितानि १४४~। ०~। तदर्धे ७२~। ०~। एवं जाता मूलकरणी क ७२~। नन्वियं
\newpage
%%%%%%%%%%%%%%%%%%%%%%%%%%%%%%%%%%%%%%%%
\noindent कृतिः रु ७२ अष्टादशाष्टद्विकसंमितानां करणीनाम्~। तत्कथमस्या मूलं
दिसप्ततिकरण्य 
इति चेदुच्यते~। इयं तासामेव युतिकरणी क ७२~। अतः प्रतीत्यर्थं
विश्लेषसूत्रेण 
पृथक्क्रियते~। यथा\textendash \,इयं योगकरणी ७२ वर्गेणानेन ६ विहृता लब्धिः २~। 
कृतिपदं ६ पूर्वं खण्डत्रयं कृतम् ३~। २~। १~। एषां कृतयः ९~। ४~। १
पूर्वलब्ध्या २ 
गुणिता जाताः पृथक्करण्यः १८~। ८~। २~। अत्रोपपत्तिः\textendash \,करणीवर्गस्तावदेवं
भवति {\qt स्थाप्योऽन्त्यवर्गश्चतुर्गुणान्त्यनिघ्ना} इत्यादिना~। तत्र
प्रथमस्थाने प्रथमकरणीवर्गः~। ततः प्रथमकरणीद्वितीयादिकरणीघातश्चतु-र्गुणाः~। ततो द्वितीयकरणी वर्गः द्वितीयकरणीतृतीयादिकरणीघाताश्चतुर्गुणाः~।
एवमग्रेऽपि तृतीयकरणीवर्गादि~। एवं यावन्ति करणीखण्डानि तावतामवश्यं
वर्गाः स्युः~। वर्गत्वात्तेभ्योऽवश्यं मूललाभ तानि च मूलानि करणीतुल्यान्येव~। 
तथा च वर्गराशौ यो रूपगणः स एव मूलकरणीयोगः~। परं रूपरीत्या न करणीरीत्या~। 
यदि तु करणीरीत्यैव करणीयोगो ज्ञायेत तदा \hyperref[39]{\textbf{वर्गेण योगकरणी विहृता विशुध्येत्}} इत्यादिना पृथक्करणं सुलभम्~। प्रकृते तु रूपरीत्या करणीयोग इत्यन्यथा
यतितव्यम्~। तत्रेदं प्रसिद्धम्~। \hyperref[131]{\textbf{चतुर्गुणस्य घातस्य युतिवर्गस्य चान्तरम्~। राश्यन्तरकृतेस्तुल्यम्}} इति~। इदमेकवर्णमध्यमाहरणे मूल एव स्फुटीभविष्यति~। विवरिष्यते चास्माभिस्तत्रैव~। अत्र तु यानि रूपाणि स करणीयोगः~। अतो रूपवर्गः
करणीयुतिवर्गः~। वर्गराशौ कानिचित्करणीखण्डानि
प्रथमकरणीद्वितीयादिकरणीघाताः चतुर्गुणाः~। तेषां 
योगे प्रथमकरण्याः शेषकरणीयो-गस्य च घातश्चतुर्गुणः स्यात्~। युतिवर्गोऽपि
प्रथमकरण्याः शेषकरणीयोगराशेश्चास्ति~। अतस्तयोरन्तरे प्रथमकरण्याः शेषकरणीयोगस्य चान्तरवर्गः
स्यात्~। अत उक्तम्\textendash \,\hyperref[41]{\textbf{वर्गे करण्या यदि वा करण्योस्तुल्यानि रूपाण्यथवा बहूनाम्~। विशोधयेद्रूपकृतेः}} इति~। एवं ज्ञातोऽन्तरवर्गः~। तस्य मूलं
प्रथमकरण्याः शेषकरणीयोगस्य चान्तरम्~। रूपाणि तु तयो-रेव योगः~। योगान्तरे
च ज्ञाते {\qt योगोऽन्तरेणोनयुतोर्द्धितः} इति सङ्क्रमणसूत्रेण तयोर्ज्ञानं सुलभम्~। तदिदमुक्तम्\textendash \,{\qt शेषस्य पदेन रूपाणि पृथग्युतोनितानि तदर्धं करणीद्वयं स्यात्} इति~। एवं जाता
प्रथमकरणी अवशिष्टकरणीयोगश्च~। अत्र मूले करणीद्वयमागतम्~। तत्र का वा
प्रथमकरणी~। को वा शेषकरणीयोगः~। तत्र करणीयोगे महत्वस्यैककरण्यां
स्वल्पत्वस्य चौचित्याल्लघुकरणी प्रथमा~। महती तु शेषकरणीयोगः~। अथ
द्वितीयादिकरणीयोगाद्द्वितीयकरणीतृतीयादिकरणीघाताच्चतुर्गुणाच्चोक्तवद्द्वितीयकरणी
पृथक्कार्या~। अत उक्तं बह्वी करणी तयोर्यानि रूपाणि तानीति~। एवं
तृतीयादिकरणीनामपि पृथक्करणम्~। इदमत्रावधेयम्~। मूले बह्वी 
करणी तयोर्यानि रूपाणि तानि त्वन्यत्र क्वचिल्लघुकरणीरूपाणि~।
लघुकरण्या
\newpage
%%%%%%%%%%%%%%%%%%%%%%%%%%%%%%%%%%%
\noindent अपि शेषकरणीयोगत्वसम्भवात्~। यत्र ह्येका करणी महती इतरकरणीखण्डानि 
चातिलघूनि तत्र शेषकरणीयोगः पूर्वकरणीतो लघुरपि स्यादेव~। यथा करण्यः क 
१० क ३ क २~। अत्रेतरकरणीयोगः पूर्वकरण्या लघुरस्ति~। अत्र
प्रतीत्यर्थमुदाहरणं 
क १३ क ७ क ३ क २~। {\qt स्थाप्योऽन्तवर्गश्चतुर्गुणान्त्यनिघ्ना} इत्यादिना
जातो वर्गः क १६९ क ३६४ क १६ क १०४ क ४८ क ८४ क ५६ क 
९ क २४ क ४~। वर्गरूपाणां मूलानि १३~। ७~। ३~। २~। एषां योगः २५~। 
एवं जातो वर्गो रू २५ क ३६४ क १५६ क १०४ क ८४ क ५६ क २४~। 
अत्रेयं रूपकृतिः ६२५~। अत्र {\qt चतुर्गुणान्त्यनिघ्ना} इत्यादिना
चतुर्गुणप्रथमकरणीगुणितं करणीत्रितयमेवास्तीति चतुर्गुणघातत्वात्तदेव शोध्यम्~। अतो
रूपकृतेः 
६२५ करणीत्रितयमेतत् ३६४~। १५६~। १०४ अपास्य शेषस्य १ पदेन १ रूपाणि 
युतोनितानि २६~। २४ अर्धे १३~। १२~। अत्र लघुः प्रथमकरणीति वक्तुमनुचितमुदाहृतकरणीषु तस्या अभावात्~। नापि महती रूपाणीति~। तस्याः
शेषकरणीयोगत्वाभावात्~। अतोऽत्र लघुरेव रूपाणि १२१ एषां कृतिः १४४
उक्तवच्चतुर्घातरूपं 
करणीद्वयम् ८४~। ५६ अपास्य शेषस्य ४ पदेन २ युतोनितानि रूपाणि १४~। १० 
अर्धे ५~। ७~। अत्रापि पूर्ववन्महती मूलकरणी ७ लघ्वी ५ रूपाणि ५~। एषां 
कृतेः २५ करणीम् २४ अपास्य शेषस्य पदेन १ युतोनितानि रूपाणि ६~। ४~। 
तदर्धे ३~। २~। एवं जाताः सर्वा मूलकरण्यः क १३ क ७ क ३ क २~। तस्मान्महतीरूपाणीति न नियमः~। यत्तु महती रूपाणीत्युक्तं तद्बहूनामैक्ये
सङ्ख्याबाहुल्यस्योत्सर्गात्~। वस्तुतस्तु करण्याः प्राथमिकत्वं काल्पनिकमिति यैव करणी
पृथककर्तुं शक्यते सैव कार्या~। तत्र लघुकरणीखण्डानां शोधनेन लघुः
पृथग्भवति~। बृहत्खण्डशोधनेन महती पृथग्भवति~। तत्र यद्यपि बृहत्खण्डानां
शोधनेन महती पृथग्भवति तथापि साधितमूलकरणीद्वयमध्येऽस्या न महत्वनियमः~।
इतरकरणीयोगरूपाया द्वितीयमूलकरण्या अपि महत्वसम्भवात्~। लघुखण्डशोधने तु
लघुः पृथग्भवति~। साधितकरणीद्वयमध्येऽप्यस्ति तस्या लघुत्वनियमः~।
इतरकरणीयोगरूपाया द्वितीयमूलकरण्या लघुत्वासम्भवात्~। अतो लघुखण्डकशोधनपूर्वकं मूलग्रहणे लघुर्मूलकरणी महती रूपाणीति नियमो द्रष्टव्यः~।
बृहत्खण्डशोधनपूर्वकं मूलग्रहणे त्वनियमः~। अथ च महती रूपाणीत्युपलक्षणम्~।
तेन क्वचिदल्पापीति प्रस्तुत्योदाह्रियते~। चत्वारिंशदशीतिद्विशतीतुल्याः
करण्यश्चेत्~। सप्तदशरूपयुक्ता इति वर्गेऽपि लघुखण्डशोधनपूर्वकं 
मूलग्रहणे लघुर्मूलकरणी महती रूपाणीति नियमस्य न भङ्गोऽस्ति~। तथाहि\textendash\ 
\newpage
%%%%%%%%%%%%%%%%%%%%%%%%%%%%%%%%%%%%%%%%%%%%%%
\noindent उदाहृतवर्गन्यासः रू १७ क ४० क ८० क २००~। अत्र रूपकृतेः २८९ लघुकरणीद्वयं ४०~। ८० अपास्य शेषस्य १६९ पदेन १३ रूपाणि १७ युतोनितानि ३०~। ४ अर्धे १५~। २~। अत्र लघुर्मूलकरणी २~। महती रूपाणि १५~। एषां कृतेः 
२२५ करणी २०० अपास्य शेषस्य २५ मूलेन ५ रूपाणि १५ युतोनितानि २०~। 
१० अर्धे १०~। ५~। एवं जाता मूलकरण्यस्ता एव क १० क ५ क २~। अतः 
शिष्याणां गणितसौकर्यार्थं लघूखण्डशोधनपूर्वकं मूलं ग्राह्यमिति नियमो
वक्तुमुचितः~। 
अन्यथा लघुर्महती वा मूलकरणीति व्याकुलता स्यादिति~। शोध्यकरणीनियमं
त्वग्रे वक्ष्यति~। एकादिसङ्कलितमितकरणीखण्डानीत्यादिना~॥~४१~॥\\

\vspace{-2mm}
{\bqt अथ वर्गगतर्णकरण्या मूलानयनार्थं सूत्रं वृत्तम्\textendash }

\phantomsection \label{42}
\begin{quote}
{\color{white}अ} \hspace{-10mm} {\ab ऋणात्मिका चेत्करणी कृतौ स्याद्धनात्मिकां तां परिकल्प्य साध्ये~।} \\
{\color{white}अ} \hspace{-10mm} {\ab मूले करण्यावनयोरभीष्टा क्षयात्मिकैका सुधियावगम्या~॥~४२~॥}
\end{quote}

 यदि \,वर्गे \,\hyperref[42]{\textbf{करणी \,ऋणात्मिका \,स्या}}त्तर्हि \,\hyperref[42]{\textbf{तां \,धनात्मिकां \,परिकल्प्य \,मूले \,करण्यौ साध्ये~। अनयो}}र्मूलकरण्योर्मध्येऽ\hyperref[42]{\textbf{भीष्टैका}} करणी \hyperref[42]{\textbf{सुधिया क्षयात्मिका}} ज्ञेया~। अत्र सुधि-येति हेतुगर्भमुक्तम्~। तेन वर्गे यद्येकैव क्षयकरणी भवति तदैवैकस्या मूलकरण्याः क्षय-त्वम्~। यदि द्व्यादयो भवन्ति तदैकस्या
द्वयोर्बहूनां वा मूलकरणीनां युक्त्या यथा सम्भवति तथा क्षयत्वं कल्प्यम्~। यत्र वर्गे सर्वा अपि धनकरण्यस्तत्रापि सर्वासामपि मूलकरणीनां पक्षे क्षयत्वमवगन्तव्यमिति~।
अत्रोपपत्तिः~। य एव ऋणकरणीवर्गः स एव धनकरणीवर्गः~। परमृणकरणीवर्गे करण्यृणात्मिका परत्र धनात्मिकेत्येव विशेषः~। तथा सति वर्गे करणी ऋणात्मिका धनात्मिका वा भवतु मूलं त्वङ्कतः सममेवोचितम्~। उक्तविधिना रूपकृतेः क्षयकरणीशुद्धौ तु संशोध्यमानमृणं धनं स्यादिति योग एव स्यात्~। रूपवर्गाद्धनकरणीशुद्धौ
संशोध्यमानं स्वमृणं स्यादित्यन्तरं स्यात्~। अन्तरे च मूलाङ्कसिद्धिरुक्तैव~। अतो
धनात्मिकां तां परिकल्प्येत्युक्तम्~। परमेवं धनवर्गस्येव पदं स्यात्~। अत
उक्तं क्षयात्मिकैकेति~॥~४२~॥\\

\vspace{-2mm}
{\bqt उदाहरणम्\textendash }
\newpage
%%%%%%%%%%%%%%%%%%%%%%%%%%%%%%

\begin{quote}
    \eg 
\hspace{-1cm} त्रिसप्तमित्योर्वद मे करण्योर्विश्लेषवर्गं कृतितः पदं च~। \\
    
    \vspace{-7mm}
\hspace{-1cm} द्विकत्रिपञ्चप्रमिताः करण्यः स्वस्वर्णगा व्यस्तधनर्णगा वा~। \\

\vspace{-7mm}
\hspace{-1cm} तासां कृतिं ब्रूहि कृतेः पदं च चेत्षड्विधं वेत्सि सखे करण्याः~॥~४३~॥
\end{quote}

 अत्र मूलग्रहण एव विशोषोक्तेर्यद्यपि सिद्धं वर्गमुद्दिश्य मूलप्रक्ष एवोचितस्तथापि
यदि कश्चिद्ब्रूयाद्वर्गे क्षयकरणी न सम्भवत्येवेति तं प्रति
त्रिसप्तमित्योः करणयोर्विश्लेषवर्गं ब्रूहीत्यादिवर्गप्रश्नो द्रष्टव्यः~। शेषं स्पष्टम्~। न्यासः क ३ं क ७ वा न्यासः क ७ं क ३ अनयोर्वर्गः सम एव रू १० क ८ं४~। अत्र वर्गे ॠणकरण्या यथास्थितत्वे उक्त-वद्वर्गपदाभावः~। तथा हि\textendash \,रूपकृतेः १०० करणीं ८ं४ अपास्य शेषं १८४~। अस्य
पदाभावान्नोक्तवन्मूलसिद्धिः~। अतः क्षयकरणीं धनात्मिकां परिकल्प्य मूलं ग्राह्यम्~। 
तथा सति रूपकृतेः १०० करणीमपास्य शेषं १६ अस्य पदेन ४ रूपाणि १० 
युतोनितानि १४~। ६ अर्धे ७~। ३ जाते मूलकरण्यौ क ७ क ३ अनयोरेकाभीष्टा 
क्षयात्मिकेति जाते मूलकरण्यौ क ७ क ३ं वा क ७ं क ३~। अथ द्वितीयोदाहरणे 
न्यासः क २ क ३ क ५ं~। व्यस्तधनर्णत्वेन तृतीयोदाहरणे न्यासः~। क २ं क ३ क ५~। अनयोः पक्षयोर्जातो वर्गः सम एव रू १० क २४ क ४ं० क ६ं०~। अत्राप्यृणत्वे यथास्थित उक्तवन्मूलाभावः~। तस्मात् 
\hyperref[42]{\textbf{ऋणात्मिका चेत्करणी कृतौ स्याद्धनात्मिकां तां परिकल्प्य साध्ये}} इति कृते करण्योः 
४०~। ६० तुल्यानि रूपाणि १०० रूपकृतेः १०० अपास्य शेषं ० अस्य पदेन ० 
रूपाणि युतोनितानि १०~। १० अर्धे ५~। ५ अनयोरेकस्यामृणत्वमवश्यं कल्प्यम्~। 
अन्यथा वर्गे क्षयकरणी न स्यादिति~। तत्र मूलकरण्याः क्षयत्वमितरस्या
धनत्वं च प्रकल्प्य तावदुदाहरणं लिख्यते~। क ५ं इयं मूलकरणी~। शेषकरणीरूपाणि ५~। एतेषां कृतेः २५ करणीं २४ अपास्य शेषस्य १ पदेन रूपाणि ५ युतोनितानि ६~। ४ अर्धे जाते मूलकरण्यौ क ३ क २~। अत्रोभयोर्धनत्वमेव युक्तम्~। एकस्याम् 
ऋणत्वे वर्गे शेषकरण्या क २४ धनत्वं न स्यात्~। तयोश्चतुर्गुणघातात्मकत्वात्~। 
अस्याः उभयोः क्षयत्वे यद्यपि शेषकरण्याः सम्भवति धनत्वं तथापि
पूर्वकरण्योः क्षयत्वं न स्यात्~। पूर्वमूलकरण्या क ५ं चतुर्गुणया क २ं०
गुणितयोरनयोर्मूलकरण्योः 
क ३ क २ धनत्वात् क ४० क ६० एवं जातं पदं क ५ं क ३ क २~।
अथ मूलकरण्या धनत्वं प्रकल्प्योदाहरणम्~। मूलकरणी क ५ शेषा ५ रूपाणि
रूपकृतेः २५ शेषकरणीं २४ अपास्य पूर्ववज्जाते मूलकरण्यौ क ३ क २ अत्रोभयोः
\newpage
%%%%%%%%%%%%%%%%%%%%%%%%%%%%%%%%%%%%%%%%
\noindent क्षयत्वमेव युक्तम्~। एकस्या एव क्षयत्व उक्तयुक्त्या शेषकरण्याः क २४ धनत्वं न स्यात्~। उभयोर्धनत्वं उक्तयुक्त्या पूर्वकरण्योः क ४० क ६० क्षयत्वं न स्यात्~। एवं वा जातं पदं क ५ क ३ क २ तस्मादुक्तं सुधियेति~। एवमनयोः क २४ क 
४० अनयोर्वा क २४ क ६० प्रथमतः शोधनेनापि पदद्वयं द्रष्टव्यम्~।
नन्वृणकरण्या धनत्वकल्पनं विनैवास्ति मूलसिद्धिः~। यथाहि\textendash \,क २ क ३ क ५ं
वा क २ं क ३ं  क ५ {\qt स्थाप्योऽन्त्यवर्गः} इत्यादिना जातो वर्गः क ४ क २४ क ४ं० क ९ क ६ं० 
क २५ \hyperref[13]{\textbf{स्वमूले धनर्णे}} इति वर्गकरणीनां मूलानि रूः २ रू ३ रू ५ं६ वा रू २ं रू ३ं रू ५ उभयेषामपि योगः सम एव रू ० एवं जातो वर्गः रू ० क २४ क ४ं० क ६०~।
अत्र रूपकृतेः ० करणीद्वयं क ४ं० अपास्य शेषस्यः १६ पदेन ४ रूपाणि ०
युतोनितानि  ४।४ अर्धे २।२ एका मूलकरणी क २ अपरा २ रूपाणि~। एतत्कृतेः ४ शेषकरणीं क ६ं० अपास्य शेषस्य ६४ पदेन ८ रूपाणि २ युतोनितानि ६~। 
१० अर्धे ३~। ५ जाता मूलकरण्याः क २ क ३ क ५~। अथ यदि परा मूलकरणी क २ आद्या क २ रूपाणि~। एतत्कृतेः ४ शेषकरणीं ६ं० अपास्य शेषस्य ६४ पदेन ८ रूपाणि 
२ युतोनितानि १०~। ६ अर्धे ५~। ३ एवं जाता मूलकरण्यः क २ क ३ क 
५~। अथवा रूपकृतेः ० करणीद्वयं क ४ं० क ६ं० अपास्य शेषस्य १०० 
पदेन १० रूपाणि ० युतोनितानि १०।१० अर्धे ५।५ अनयोराद्या मूलकरणी क ५ 
परा ५ रूपाणि~। एतत्कृतेः २५ शेषकरणीं २४ अपास्य शेषस्य १ पदेन १ 
रूपाणि ५ युतोनितानि ६।४ अर्धे ३।२ एवं जाता मूलकरण्यः क ५ क २ 
क ३~। अथ यदि परा मूलकरणी क ५ आद्या ५ रूपाणि १ एतत्कृतेः २५ शेषकरणीं क २४ अपास्य शेषस्य १ पदेन १ रूपाणि ५ युतोनितानि ६~। ४ अर्धे ३~। २ एवं जाता मूलकरण्यः क ५ क ३ क २~। एवमनयोरपि क २४ क ६ं० शोधने पदद्वयं  द्रष्टव्यम्~। एवं वर्गकरण्या धनत्वकल्पनं विनैव मूलसिद्धावपि स्वयमशुद्धं वर्गं कृत्वा तस्योक्तवन्मूलं नायातीति \hyperref[42]{\textbf{ऋणात्मिका चेत्करणी कृतौ स्याद्धनात्मिकां तां परिकल्प्य साध्य}} इति विशेषमभिधाय पुनर्मूलकणीनां मध्ये
क्षयत्वकल्पनेऽनुगमाभावात्सुधियेत्यादि यदुक्तं तद-युक्तं
विशेषज्ञानामाचार्याणामिति चेदुच्यते~। 
विस्मृतगुडरसस्य पित्तोपहतरसनस्य गुडं भक्षयतस्तिक्तं रसमनुभवतो
देवदत्तस्य  मधुरोऽयं गुड इति यथार्थवादिनि सर्वज्ञेऽपि यथा
भ्रान्तत्वनिश्चयस्तथाचार्ये तवापि 
स युक्त एव~। ननु कथमिदमवगन्तव्यम्~। शृणु तर्हि~। एता हि मूलकरण्यः
\newpage
%%%%%%%%%%%%%%%%%%%%%%%%%%%%%%%%%%%%%%%%%%%
\noindent क २ क ३ क ५ं  एता वा क २ं क ३ं क ५ एतासामासन्नमूलानि गृहीत्वा 
तदैक्यं च कृत्वा कृते वर्गं वर्गकरणीनामेवादौ वर्गं  कृत्वा
पश्चादासन्नमूलानि 
गृहीत्वा कृते योगे तुल्यतयैव भाव्यम्~। करणीषड्विधस्य
स्वमूलषड्विधार्थं प्रवृत्तेः~। अन्यथा \hyperref[34]{\textbf{योगं करण्यो}}रित्यादिना कृतः करणीयोगो नोपपद्येत~।
तत्रासामासन्नमूलानि $\begin{matrix}
\vspace{-1mm}
\mbox{{१}}\\
\vspace{-1mm}
\mbox{{२५}}
\vspace{1mm}
\end{matrix}$~। $\begin{matrix}
\vspace{-1mm}
\mbox{{१}}\\
\vspace{-1mm}
\mbox{{४४}}
\vspace{1mm}
\end{matrix}$~। $\begin{matrix}
\vspace{-1mm}
\mbox{{२ं}}\\
\vspace{-1mm}
\mbox{{१४}}
\vspace{1mm}
\end{matrix}$~। $\begin{matrix}
\vspace{-1mm}
\mbox{{१ं}}\\
\vspace{-1mm}
\mbox{{वा २५}}
\vspace{1mm}
\end{matrix}$~। $\begin{matrix}
\vspace{-1mm}
\mbox{{१ं}}\\
\vspace{-1mm}
\mbox{{४४}}
\vspace{1mm}
\end{matrix}$~। $\begin{matrix}
\vspace{-1mm}
\mbox{{२}}\\
\vspace{-1mm}
\mbox{{१४}}
\vspace{1mm}
\end{matrix}$~। एषां योगो धनं ०~। ५५~। ऋणं वा ०~। ५५~। अनयोर्वर्गस्तुल्य एव ०~। ५० धनम्~। अथाचार्यैः प्रथमतः कृतस्य करणीवर्गस्यास्य रू १० क २४ क ४ं० क ६ं० करणीनामासन्नमूलानि $\begin{matrix}
\vspace{-1mm}
\mbox{{४}}\\
\vspace{-1mm}
\mbox{{५४}}
\vspace{1mm}
\end{matrix}$~। $\begin{matrix}
\vspace{-1mm}
\mbox{{६ं}}\\
\vspace{-1mm}
\mbox{{१९}}
\vspace{1mm}
\end{matrix}$~। $\begin{matrix}
\vspace{-1mm}
\mbox{{७ं}}\\
\vspace{-1mm}
\mbox{{४५}}
\vspace{1mm}
\end{matrix}$~। रूपेषु १० संयोज्य जातो वर्गः~। स एव ०~। ५०~। अथ यदि त्वत्कृतस्य 
वर्गस्य रू ० क २ क ४ं० क ६ं० आसन्नमूलानां तेषामेव योगः क्रियते 
तदायं स्यात् ९ं~। अयमशुद्धो वर्गः~। ऋणत्वादप्यशुद्धिः~। न त्वृणं वर्गः
सम्भवतीति निरूपितं न मूलं क्षयस्यास्ति तस्याकृतित्वादित्यत्र~। अथ यदि 
मूलस्य सावयवत्वादस्मिन्वर्गे तव न स्फुटा प्रतीतिरस्ति तर्हीदमुदाहरणं रू
३ रू ७ं~। एतेषां योगस्य रू ४ं वर्गेणानेन रू १६ भाव्यम्~। अथात्र 
त्वदुक्तरीत्या यदि करणीवर्गः क्रियते तदैतावान्न भवति~। तथा हि\textendash \,क ९ क ४ं९~। अत्र {\qt स्थाप्योऽन्त्यवर्गः} इत्यादिना जातो वर्गः क ८१ क १७६४ क २४०१~। अत्राचार्योक्तमार्गेण मूलानि रू ९ रू ४ं२ रू ४९~। एषां योगे भवति वर्गः स एव रू १६~। त्वदुक्तमार्गेण मूलग्रहणे जातानि मूलानि रू ९ 
रू ४ं२ रू ४ं९~। एषां योगः रू ८ं२~। नह्ययं वर्गः सम्भवति~। ननु तर्हि 
\hyperref[13]{\textbf{स्वमूले धनर्णे}} इत्यस्य का गतिः~। शृणु तर्हि~। मूलग्रहणे हि \hyperref[13]{\textbf{स्वमूले धनर्णे}} इत्युक्तम्~। प्रकृते तु मूलकरणीवर्गे करणीनां रूपजातित्वेन स्थापनमस्ति~। न तु 
मूलं गृह्यते~। अत एव कृतेष्वपि रूपेषु करणीवर्ग इत्येव
व्यवहरोऽस्ति न तु  करणीवर्गमूलमिति~। एवं रूपाणामपि करणीजातित्वेन स्थापने सति वर्गविधानं नास्ति~। अत एव क्षयरूपाणां करणीत्वेन स्थापने क्षयकरण्य एव स्थाप्यन्ते~। वर्गविधाने तु क्षयत्वं कथं स्यात्~। तस्पाद्रूपकरण्योर्भिन्नजातित्वप्रयुक्तः
सङ्ख्याभेदो न तु वास्तवः~। यथा वराटकजात्या विंशतिः २० काकिणी जात्यैकः १
पणजात्या चतुर्थांशो द्रम्मजात्या चतुःषष्ट्यंशः स्थाप्यते~। न ह्यासां
\newpage
%%%%%%%%%%%%%%%%%%%%%%%%%%%%%%%%%%%%%%%%%%%%
\noindent सङ्ख्यानां फलतो भेदोऽस्ति~। अत एव रूपत्रयस्य करणीनवकस्य वा वर्गो
रूपनवकमेव~। रूपकरण्योः फलतो भेदे सम एव वर्गः कथं स्यात्~।
तस्मात्सुष्ठूक्तम् \hyperref[42]{\textbf{ऋणात्मिका चेत्करणी कृतौ स्यात्}} इत्यादि~। ननु मूलकरणीनां क्षयत्वकल्पने कोऽनुगमः~। 
श्रुणु~। \hyperref[44]{\textbf{वर्गे करणीत्रितये करणीद्वितयस्य तुल्यरूपाणि}} इत्यादि
वक्ष्यमाणप्रकारेण 
शोध्यकरणीनां नियमे तासां धनत्वमेव प्रकल्प्य मूलकरण्यौ साध्ये~। तत्र या
मूलकरणी तस्या धनत्वम् ऋणत्वं \,वा प्रकल्प्य \,तया चतुर्गुणया \,यथास्थितधनर्णताकाः \,शोधितकरण्यो भाज्याः~। भजने यादृश्यः करण्यो धनमृणं वा लभ्यन्ते तादृश्यः 
शेषकरण्यो ज्ञेया इत्यादि मतिमद्भिरन्यदप्यूह्यमित्यलं पल्लवितेन~॥~४३~॥\\

\vspace{-4mm}
 अथ \hyperref[41]{\textbf{वर्गे करण्या यदि वा करण्यो}}रित्याद्युक्तेरनियमेन करणीशोधने सति
मूलाशुद्धिः स्यादिति करणीवर्गे करणीसङ्ख्यानियमपूर्वकं शोध्यकरणीनियमं
गीतिद्वयेनार्याद्वितयेन च निरूपयति\textendash \,

\phantomsection \label{44}
\begin{quote}
    \ab
    एकादिसङ्कलितमितकरणीखण्डानि वर्गराशौ स्युः~। \\
 वर्गे करणीत्रितये करणीद्वितयस्य तुल्यरूपाणि~। \\

\vspace{-5mm}
 करणीषट्के तिसृणां दशसु चतसृणां तिथिषु च पञ्चानाम्~। \\
 रूपकृतेः प्रोज्झ्य पदं ग्राह्यं चेदन्यथा न सत् क्वापि~। \\

\vspace{-5mm}
 उत्पत्स्यमानयैवं मूलकरण्याल्पया चतुर्गुणया~। \\
 यासामपवर्तः स्याद्रूपकृतेस्ता विशोध्याः स्युः~। \\

\vspace{-5mm}
 अपवर्ते या लब्धा मूलकरण्यो भवन्ति ताश्चापि~। \\
 शेषविधिना न यदि ता भवन्ति मूलं तदा तदसत् ॥ ४४ ॥
\end{quote}

अत्र द्वितीयगीतौ तिथिषु पञ्चानामिति बहवः पठन्ति तत्र तिथिषु च
पञ्चानामिति पठनीयम्~। अन्यथा छन्दोभङ्गात्~। अत्रैकादिसङ्कलितमितकरणीखण्डानि वर्गराशौ स्युः इत्यनेनैककरण्या वर्ग एका करणीद्वयोः करण्योर्वर्गे करणीत्रितयं
स्यादित्यादि निरूपितं तच्च प्रत्यक्षविरुद्धमतः स्वयमेव तदर्थं विवृणोति~। करणीवर्गराशौ
\newpage
%%%%%%%%%%%%%%%%%%%%%%%%%%%%%%%%%%%%%%%%
\noindent रूपैरवश्यं भवितव्यम्~। एककरण्या वर्गे रूपाण्येव~। द्वयोः सरूपैका करणी~। तिसृणां तिस्रश्चतसृणां षट् पञ्चानां दश षण्णां पञ्चदश~। ततो द्व्यादीनां
करणीनां वर्गेष्वेकादिसङ्कलितमितानि करणीखण्डानि यथाक्रमं स्युः~। अथ
यद्युदाहरणे तावन्ति न भवन्ति तदा संयोज्य योगकरणीं विश्लेष्य वा तावन्ति
कृत्वा मूलं ग्राह्यमित्यर्थः~। वर्गे करणीत्रितये करणीद्वितयस्य
तुल्यरूपाणीत्यादि स्पष्टार्थमिति~। अत्र करणीवर्गे राशी रूपैरवश्यं
भवितव्यम्~। एककरण्या \,वर्गे \,रूपाण्येव~। द्वयोः \,सरूपैकेत्यार्या \,कल्पयित्वा \,सूत्रमध्ये पठन्ति तदशुद्धम्~। करणीतिसृणां तिस्र
इत्यादेरग्रिमग्रन्थस्यानन्वयात्~। नह्येकमेव वाक्यं
श्लोकचूर्णिकात्मकमिति रीतिरस्ति~। पूर्वार्धे छन्दोभङ्गाच्च~। सङ्कलितं च
{\qt सैकपदघ्नपदार्धमथैकाद्यङ्कयुतिः किल सङ्कलिताख्या} इत्युक्तं पाट्याम्~।
तस्मान्मूले यद्येतावत्प्रभृतीनि करणीखण्डानि
\vspace{-2mm}

\begin{table}[h!]
    \centering\s\renewcommand{\arraystretch}{1.3}
    \begin{tabular}{r}
        २~। ३~। ४~। ५~। ६~। ७~। ८~। ९~। १०~। ११~। १२~। १३~। १४~। १५~। \\
१~। ३~। ६~। १०~। १५~। २१~। २८~। ३६। ४५। ५५।६६।७८।९१।१०५~।
    \end{tabular}
\end{table}
\vspace{-4mm}

\noindent तदा वर्गराशौ तावत्प्रभृतीनि करणीखण्डानि~। शेषं किञ्चिन्मया
व्याख्यायते\textendash \,उत्पत्स्यमान-येति~। अत्राल्पयेत्युपलक्षणम्~। यत्र \,महती \,मूलकरणी, अल्पा \,रूपाणि \,तत्र \,महत्या चतुर्गुणया यासामपवर्तः स्यात्ता एव
विशोध्याः~। आचार्यमते त्वल्पत्वं 
पारिभाषिकम्~। यतोऽस्य \,सूत्रस्योदाहरणे \,यां मूलकरणीं \,रूपाणि \,प्रकल्प्यान्ये करणीखण्डे \,साध्येते सा महतीत्यर्थं इति व्याकरिष्यति~। पुनर्नियमान्तरमाह\textendash \,\hyperref[44]{\textbf{अपवर्त}} इति~। \hyperref[44]{\textbf{अल्पया}} क्वचिन्महत्या वा \hyperref[44]{\textbf{चतुर्गुणया}}पवर्ते कृते याः करण्यो
\hyperref[44]{\textbf{लब्धास्ता}} एव \hyperref[44]{\textbf{मूलकरण्यो भवन्ती}}ति वस्तुस्थितिः~। अथ \hyperref[44]{\textbf{यदि शेषविधिना}} मूलेऽथ बह्वी करणी तयोर्येत्यादिना वा \hyperref[44]{\textbf{न भवन्ति तदा तन्मूलमसत्}} इति~। अत्राल्पयेत्युपलक्षणमिति यद्व्याख्यातं तद्बृहत्खण्डशोधनपूर्वकं मूलग्रहणे~। लघुखण्डशोधनपूर्वकं मूलग्रहणे त्वल्पयेत्येव~। अत्रोपपत्तिः~। यत्रैकैव करणी तत्र {\qt स्थाप्योऽन्त्यवर्ग} इति वर्ग एव स्यात्~। तस्य च
मूललाभाद्रूपाण्येव स्युः~। यत्र तु करणीद्वयं तत्र {\qt स्थाप्योऽन्त्यवर्गः}
इत्येककरण्या वर्गः~। तदुत्तरं {\qt चतुर्गुणान्त्यनिघ्नाः} इति शेषमेकैव
चर्तुर्गुणान्त्यनिघ्नीति~। एवं यत्र करणीत्रयं तत्र {\qt स्थाप्योऽन्त्यवर्ग}
इत्येककरण्या वर्गः
\newpage
%%%%%%%%%%%%%%%%%%%%%%%%%%%%%%%%%%%%
\noindent तदुत्तरं {\qt चतुर्गुणान्त्यनिघ्नाः} इति शेषकरणीद्वयं चतुर्गुणान्त्यनिघ्नम्~। ततोऽयं 
त्यक्त्वेति शेषं करणीद्वयम्~। तत्रापि {\qt स्थाप्योऽन्त्यवर्गः }इति
द्वितीयकरण्या वर्गः~। चतुर्गुणान्त्यनिघ्नी चापरा~। एवं यत्र करणीषट्कं
तत्र {\qt स्थाप्योऽन्त्यवर्गः} इति 
प्रथमकरण्या वर्गः~। ततः पञ्च शेषकरण्यश्चतुर्गुणान्त्यनिघ्न्य\, इति \,पञ्च \,करणीखण्डानि \,पुनरन्त्यत्यागे \,द्वितीयकरण्या वर्गः~।
शेषाश्चतस्रश्चतुर्गुणान्त्यनिघ्न्य इति चत्वारि खण्डानि~। पुनरन्त्यत्यागे
तृतीयक-रण्या वर्गः~। शेषास्तिस्रश्चतुर्गुणान्त्यनिघ्न्य इति त्रीणि
खण्डानि पुनरन्त्यत्यागे चतुर्थकरण्या वर्गः~। ततः शेषं करणीद्वयं
चतुर्गुणान्त्यनिघ्नमिति खण्डद्वयम्~। पुनरन्त्यत्यागे पञ्चमकरण्या वर्गः~। शेषा करणी चतुर्गुणान्त्यनिघ्नीत्येकं खण्डम्~। पुनरन्त्यं त्यक्त्वा
षष्ट्या वर्गः~। एवं वर्गे जाताः षण्णामपि करणीनां वर्गाः~। तेषां मूलानि
मूलकरणीतुल्यानि रूपाणि स्युः~। अतस्तेषां योगः करणीवर्गे रूपाणि~।
करणीखण्डानि तु प्रथमं पञ्च \,ततश्चत्वारि \,ततस्त्रीणि \,ततो \,द्वे \,तत \,एकम् इति \,व्यस्तम् एकाद्येकोत्तराणि \,भवन्ति~। तस्मादेकोनपदसङ्कलितमितकरणीखण्डानि भवन्ति~। प्रथमखण्डस्य वर्गत्वेनैव स्थाप-नात्~। अतो \,द्व्यादीनां \,वर्ग \,एकादिसङ्कलितमितकरणीखण्डानीत्युक्तम्~। अनयैव युक्त्या वर्गे
करणीत्रितय इत्यादि बोध्यम्~। यतो रूपाणि करणीयोगस्तस्य वर्गो युतिवर्गः~।
तत्र प्रथमकरण्याः पृथक्करणे प्रथमकरणीशेषकरणीपञ्चकघातश्चतुर्गुणः शोध्योऽन्तरवर्गार्थम्~। अत्र तु युक्तिः प्रागेवोक्ता~। अतः षण्णां करणीनां वर्गे प्रथमकरण्याः पृथक्करणे
प्रथमकरणीशेषकरणीपञ्चकघातश्चतुर्गुणोऽन्तरवर्गार्थं युतिवर्गः शोध्यो
भवतीति करणीषट्कवर्गे पञ्चैव करण्यः शोध्यः~। तदिदमुक्तं तिथिषु पञ्चानामिति~। यतः करणीषट्कवर्गे पञ्चदशैव करणीखण्डानि भवन्ति~। एवं करणीपञ्चकवर्गे
प्रथमकरण्याः पृथक्करणे प्रथमकरणी शेषकरणीचतुष्टयघातश्चतुर्गुणः शोध्य इति
चतस्र एव करण्यः शोध्याः~। तदिदम् उक्तं दशसु \,चतसृणाम् इति~। एवं \,करणीषट्के \,तिसृणां \,वर्गे \,करणीत्रितये \,करणीद्वितयस्य तुल्यरूपाणीत्याद्यपि बोध्यम्~।
एवमपि यदा केनचिद्धृष्टेनोक्तनियमपूर्वकं यथामूलमायाति तथा रूपाणि करणीश्च
कल्पयित्वा यदि पृच्छ्यते तदा तदुदाहरणं खिलमखिलं वेति ज्ञानार्थम् उक्तम् अल्पया
चतुर्गुणया यासामपवर्तः स्यादिति~। अत्राल्पयेति प्रथमकरणीं लक्ष्यते~। यतः प्रथमकरणीशेषकरणीघातश्चतुर्गुणः शोध्यतेऽतो याः
शोधितास्तासामाद्यया चतुर्गुणयापवर्तः स्यादेव~। यद्यपवर्तो न
स्यात्तदोदाहरणस्य खिलत्वं स्फुटमेव~। अथ यदि धृष्टतरेण
प्रथमशोध्यकरण्यस्ता एव तादृश्योऽन्या वा स्थापिताः परतस्तु या काश्चन
युक्त्या स्थापितास्तदा तदुदाहरणस्य खिलत्वाखिलत्वज्ञानार्थमुक्तमपवर्ते या
लब्धा
\newpage
%%%%%%%%%%%%%%%%%%%%%%%%%
\noindent इत्यादि~। यतश्चतुर्गुणान्त्यनिघ्ना \,इत्यत्र \,चतुर्गुणप्रथमकरणीशेषं \,करणीघातोऽस्ति \,तत्र चतुर्गुणप्रथमकरण्यापवर्ते मूलकरण्य एव लभ्या इति
मूलकरण्योऽपवर्तादेव ज्ञाताः~। यदि तु शेषविधिना ता न भवन्ति तदा
शेषकरणीनां दुष्टत्वात्तदुत्पन्नं मूलमपि दुष्टमित्युपपन्नम्~॥~४४~॥\\

\vspace{-2mm}
{\bqt उदाहरणम्}
\begin{quote}
    \eg 
    वर्गे यत्र करण्यो दन्तैः ३२ सिद्धैः २४ गजैः ८ मिता विद्वन्~। \\
रूपैर्दशभिरुपेताः किं मूलं ब्रूहि तस्य स्यात्~॥~४५~॥
\end{quote}

स्पष्टोऽर्थः~। न्यासो \,रू \,१० \,क \,३२ \,क \,२४ \,क \,८ \,अत्र \,वर्गकरण्या \,इत्यादिनैव \,मूलग्रहणे करणीत्रितयशोधनं विना शेषस्य पदाभावात्~। रूपकृतेः १००
करणीत्रय-तुल्यरूपाणि \,विशोध्य \,शेषस्य \,३६ \,पदेन \,६ \,रूपाणि \,१० \,युतोनितानि \,१६~। ४ \,तदर्धे जाते मूलकरण्यौ क ८ क २ तदिदं पदमसत्~। यतोऽस्य वर्गोऽयं रू १० क ६४ अत उक्तं  वर्गे करणीत्रितये करणीद्वितयस्य तुल्यरूपाणीत्यादि~। एवं येषां
करणीखण्डानां योगे रूपकृतेः शोधिते शेषस्य पदं लभ्यते तादृशानि
करणीखण्डानि कल्पयित्वोदाहरणानि~॥~४५~॥\\

\vspace{-2mm}
{\bqt उदाहरणम्\textendash}

\phantomsection \label{46}
\begin{quote}
    \eg 
    वर्गे यत्र करण्यस्तिथिविश्वहुताशनैश्चतुर्गुणितैः~। \\
तुल्या दशरूपाढ्याः किं मूलं ब्रूहि तस्य स्यात्~॥~४६~॥
\end{quote}

 स्पष्टोऽर्थः~। न्यासो रू १० क ६० क ५२ क १२~। रूपकृतेः १०० उक्तनियमेन करणीद्वयं ५२~। १२ अपास्य शेषस्य ३६ पदेन ६ रूपाणि १० युतोनितानि 
१६~। ४ अर्धे ८~। २ अनयोरल्पा मूलकरणी २ महती रूपाणि ८ तत्कृतेः ६४ 
करणीं ६० अपास्य शेषस्य ४ पदेन २ रूपाणि ८ युतोनितानि १०~। ६ तदर्धे
\newpage
%%%%%%%%%%%%%%%%%%%%%%%%%%%%%%%%%%%%%%%
\noindent ५~। ३ पदं जातं मूलं क २ क ३ क ५ तदिदमसत्~। यतोऽस्य वर्गोऽयं रू १० क २४ क ४० क ६० अत उक्तमल्पया चतुर्गुणया यासामपवर्तः स्यादिति~। अत्राल्पया २ चतुर्गुणया ८ शोधितकरण्योः ५२~। १२ अपवर्तो न भवतीत्यशुद्धं पदम्~॥~४६~॥\\

\vspace{-2mm}
{\bqt उदाहरणम्\textendash}
\begin{quote}
    \eg 
     अष्टौ षट् पञ्चाशत् षष्टिः करणीत्रयं कृतौ यत्र~। \\
 रूपैर्दशभिरुपेतं किं मूलं ब्रूहि तस्य स्यात्~॥~४७~॥
\end{quote}

 अत्र करणीत्रितयं कृतौ सखे यत्रेति केचित्पठन्ति तदशुद्धम्~। मात्राधिक्येन 
च्छन्दो-भङ्गात्~। स्फुटोऽर्थः~। न्यासो रू १० क ८ क ५६ क ६० अत्र
करणीत्रितये करणीद्वितयस्येति नियमात्~। करणीद्वयं ८~। ५६~। शोधनेन जाते
मूलकरण्यौ ८~। २ अत्राल्पया २ चतुर्गुणया ८ शोधितकरण्योः ८~। ५६ अपवर्तः 
सम्भवतीत्यल्पा मूलकरणी २ महतीरूपाणि ८ पुनरेतेभ्य उक्तवज्जातं करणीद्वयं ५~। ३~। अत्राप्यल्पया ३ चतुर्गुणया १२ शोधितकरण्या ६० अपवर्तः सम्भवतीति जातं मूलं क २ क ३ क ५ तदिदमप्यसत्~। यतोऽस्य वर्गोऽयं रू १० क २४ क ४० क ६०~। अत उक्तमपवर्ते या लब्धा इत्यादि~। अत्राल्पया २ चतुर्गुणया ८ शोधितकरण्योः ८~। ५६ अपवर्तेन लब्धे १~। ७ शेषविधिना त्वन्ये मूलकरण्यौ ५~। ३~॥~४७~॥\\

\vspace{-2mm}
{\bqt उदाहरणम्\textendash \,}
\begin{quote}
    \eg 
     चतुर्गुणाः सूर्यतिथीषु रुद्रनागर्तवो यत्र कृतौ करण्यः~। \\
 सविश्वरूपा वद तत्पदं ते यद्यस्ति बीजे पटुताभिमानः~॥~४८~॥\\
\end{quote}
\vspace{-4mm}

 अत्र रुद्रा इति पाठे नागर्तवश्चतुर्गुणा इति न प्रतीयतेऽतो रुद्रनागर्तव
इति पाठः साधीयान्~। स्फुटोऽर्थः~। न्यासो रू १३ क ४८ क ६० क २० क 
४४ क २४ क ३२~। अत्र करणीषट्के तिसृणामिति नियमपूर्वकं मूलं नायातीति
\newpage
%%%%%%%%%%%%%%%%%%%%%%%%%%%%%%%%%%%%%%
\noindent नायं वर्गः~। यदि तु नियमं विहाय मूलं गृह्यते तर्ह्यसत्~।
तथाहि\textendash \,रूपकृतेः
१६९ करणीं ४८ अपास्योक्तवज्जातं मूले करणीद्वयं १२~। १ पुनर्महतीरूपाणीति
तत्कृतेः १४४ क ६० क २० अपास्योक्तवज्जातं मूले करणीद्वयं १०~। २ 
पुनरपि महतीरूपाणीति तत्कृतेः १०० क ४४ क ३२ क २४ अपास्योक्तवज्जातं
करणीद्वयं ५~। ५ एवं जातं मूलं क १ क २ क ५ क ५ तदिदमसत्~।
यतोऽस्य वर्गोऽयं रू १३ क ८ क २० क २० क ४० क ४ं० क १०~। अत्र 
शतमितकरण्या मूललाभात्तन्मूलं १० रूपेषु १३ प्रक्षिप्य जातानि रूपाणि २३~।
समकरण्योर्योगे जाता चतुर्गुणिता ८०~। १६०~। एवं जातो वर्गो रू २३ क ८
क ८० क १६०~। अस्माद्वर्गान्मूलग्रहणे खण्डत्रयमेवायाति~। अस्ति च मूले 
करणीचतुष्टयमिति योगकरणी विश्लेष्या~। ननु प्रथमं \hyperref[41]{\textbf{वर्गे करण्या यदि वा करण्यो}}रित्यादिना नियमं विनैव मूलग्रहणमुक्तमिदानीं तं तं नियमं विना मूलग्रहणे 
सदसदित्युच्यते तत्कथं प्रथमत एव नियमपूर्वकं मूलग्रहणं नोक्तमित्यत आह\textendash \,यैरस्य मूलनियनस्य नियमो न कृतस्तेषामिदं दूषणमिति प्रथमं सर्वसाधारण्येन
मूलग्रहणमुक्तमिदानीं तावन्मात्रेण मूलग्रहणे मूलाशुद्धिरिति स्वयं विशेष
उक्त इति  भावः~। ननूद्दिष्टवर्गेभ्य उक्तविधिना तु मूलं न लभ्यतेऽथ यदि
तादृशवर्गाणां मूलापेक्षा स्यात्तदा किं विधेयमित्यत आह\textendash \,एवंविधे वर्गे करणीनामासन्नमूलकरणेन
मूलान्यानीय रूपेषु प्रक्षिप्य मूलं वाच्यमिति~। तद्रूपसङ्ख्याकाः करण्यो
मूलमित्यर्थः~। शेषं स्पष्टम्~॥~४८~॥\\

\vspace{-2mm}
{\bqt उदाहरणम्\textendash}
\begin{quote}
    \eg 
     चत्वारिंशदशीतिद्विशतीतुल्याः करण्यश्चेत्~। \\
 सप्तदशरूपयुक्तास्तत्र कृतौ किं पदं ब्रूहि~॥~४९~॥
\end{quote}

 अत्र चतुर्थचरणे 'यत्र कृतौ तत्र किं पदं ब्रूहि' इति पाठेऽसावुद्गीतिर्ज्ञेया~। 
अशीतिरिति रेफान्तः पाठो न युक्तः~। स्पष्टोऽर्थः~। न्यासो रू १७ क ४० क 
८० क २०० अत्र लघुखण्डशोधनपूर्वकं मूलग्रहणे महत्येव रूपाणीति प्रागेव
प्रतिपादितम्~। अथ बृहत्खण्डशोधनपूर्वकमूलग्रहणे महती रूपाणीत्युक्तविधिना
यद्यपि मूलं नायाति तथापि नासौ वर्ग इति वक्तुमनुचितम्~। किं त्वल्पा
रूपाणीति
\newpage
%%%%%%%%%%%%%%%%%%%%%%%%
\noindent प्रकल्पनेऽपि यदि मूलं न लभ्यते तदैवावर्गत्वं युक्तम्~। प्रकृते तु
रूपकृतेः २८९  कर-णीद्वयं २००~। ८० अपास्य शेषस्य ९ पदेन ३ रूपाणि १७ युतोनितानि २०~। १४ अर्धे १०~। ७ जाते मूलकरण्यौ~। अत्राल्पया ७ चतुर्गुणया २८ शोधितकरण्योः
२००~। ८० अपवर्तो न भवतीत्येतावता न मूलाशुद्धिः~। किन्त्वल्पा रूपाणीति 
प्रकल्पने महत्या चतुर्गुणया अपवर्तासम्भवे, तस्मात्पारिभाषिकेऽल्पमहत्वे,
न स्वरूपेण~। 
अत एवाचार्येण \hyperref[46]{\textbf{वर्गे यत्र करण्यस्तिथिविश्वहुताशनैश्चतुर्गुणितैः}} इत्यस्मिन्नुदाहरणे 
निरूपितम्~। या मूलकरणी रूपाणि प्रकल्प्यान्ये करणीखण्डे साध्येते सा महती
प्रकल्प्येत्यर्थ इति~। एवं कृतपरिभाषाया प्रकृतेऽल्पा १० मूलकरणी महती ७ रूपाणि~। एतत्कृतेः ४९ करणीं ४० अपास्योक्तवज्जाते मूलकरण्यौ ५~। २~। एवं जातं मूलं क १० क ५ क २~। अस्य सर्वनियमपूर्वकत्वाच्छुद्धता भवति~। तस्य वर्गः स एव रू १७ क २०० क ८० क ४०~। एवं मतिमद्भिरन्यदप्यूह्यम्~।

\begin{quote}
    \qt दैवज्ञवर्यगणसन्ततसेव्यपार्श्वबल्लालसञ्ज्ञगणकात्मजनिर्मितेऽस्मिन्~। \\
बीजक्रियाविवृतिकल्पलतावतारे व्यक्तिः क्रमेण करणीभवषड्विधस्य~॥
\end{quote}

\begin{center}
इति श्रीसकलगणकसार्वभौमश्रीबल्लालदैवज्ञसुतकृष्णगणकविरचिते \\
बीजविवृतिकल्पलतावतारे करणीषड्विधविवरणम्~।\\

( अत्र मूलश्लोकैः सह ग्रन्थसङ्ख्या पञ्चनवत्यधिकपञ्चशतानि ५९५~।\\
 एवं चतुर्षु षड्विधेषु जाता ग्रन्थसङ्ख्या पुरन्दरशतानि १४००~। )\\

\vspace{10mm}
    \includegraphics[scale=0.8]{graphics/Capture34.png}
\end{center} 
\newpage
%%%%%%%%%%%%%%%%%%%%%%%%%%%%%%%%%%%%
\phantomsection \label{ch5}
\begin{center}
{\LARGE \textbf{५ कुट्टकविवरणम्~।}}
\end{center}

\vspace{2mm}
एवं सामान्यतोऽव्यक्तक्रियोपयुक्तं षड्विधचतुष्टयमुक्त्वानेकवर्णसमीकरणप्रक्रियोपयुक्तं
कुट्टकमाह\textendash \,भाज्यो हारः क्षेपक इत्यादिना~।\\

\vspace{-4mm}
 ननु नेह कुट्टकस्यारम्भो युक्तः~। पाट्यां तस्य निरूपितत्वात्~। न च 
{\qt 'अन्त्योन्मितौ कुट्टविधैर्गुणाप्ती~। ते भाज्यतद्भाजकवर्णमाने'}
इत्यनेकवर्णप्रक्रियोपयुक्तत्वात्तस्यारम्भोऽत्र युक्त इति वाच्यम्~।
उपयुक्तत्वाविशेषाद्भिन्नाभिन्नपरिकर्मादित्रैराशिकादिकमप्यत्रारभ्येत~। अथ {\qt 'पाट्या च बीजेन च कुट्टकेन वर्गप्रकृत्या च तथोत्तराणि'} इति
प्रश्नाध्याये कुट्टकस्य पृथङ्निर्देशात्
\begin{quote}
    \q
    परिकर्मविंशतिं यः सङ्कलिताद्यां पृथग्विजानाति~।\\
अष्टौ च व्यवहाराञ्छायान्तान्भवति गणकः सः ~॥~

\end{quote}

 इति ब्रह्मगुप्तादिपाटीगणितारम्भे पाटीस्वरूपकथनेऽनिर्देशाच्च न तस्य
व्यक्तान्तर्भूतत्वमिति व्यक्ते तदारम्भो नावश्यक इति चेदिहाप्यनन्तर्भूतत्वाविशेषादनारम्भ एव युक्त इति~। अत्रोच्यते~। {\qt 'त्रुट्यादिप्रलयान्तकालकलनामानप्रभेदः
क्रमाच्चारश्च द्युसदां द्विधा च गणितम्'} इति सिद्धान्तलक्षणकथने द्विविधगणितमुक्तं
व्यक्तम् अव्यक्तसञ्ज्ञम् इति~। सिद्धान्तपाठाधिकारिनिरूपणे च गणितस्य
द्वैविध्यश्रवणादभ्युपेयमेव तद्द्वैविध्यम्~। 
परं कुट्टकस्य कुत्रान्तर्भाव इत्यस्ति संशयः~। तत्र पाटीस्वरूपकथने
तदनिर्देशाद्व्यक्ते तस्यानावश्यकत्वाच्च न तत्रान्तर्भूतिः किं
त्वव्यक्तेऽनेकवर्णप्रक्रियायां
तस्यावश्यकत्वात्तत्रैवान्तर्भावः~। अनेकवर्णमध्यमाहरणे {\qt 'वर्गाद्यं चेत्तुल्यशुद्धौ कृतायां पक्षस्यैकस्योक्तवद्वर्गमूलम्~। वर्गप्रकृत्या परपक्षमूलम्'} इत्यावश्यकत्वाद्वर्गप्रकृतेरिव व्यक्ते 
तदभिधानं त्वव्यक्तमार्गानपेक्षत्वादव्यक्तगणितानभिज्ञानां तज्ज्ञानार्थं
यथा {\qt 'बाले}
\thispagestyle{empty}
\afterpage{\fancyhead[CE] {बीजगणिते~।}}
\afterpage{\fancyhead[CO]{कुट्टकः}}
\afterpage{\fancyhead[LE,RO]{\thepage}}
\cfoot{}
\newpage
%%%%%%%%%%%%%%%%%%%%%%%%%%%%%%%%%%%%%

\noindent {\qt मरालकुलमूलदलानि सप्त'} इत्याद्युदाहरणजातस्यैकवर्णमध्यमाहरणविषयस्य
विनैवाव्यक्तमार्गं सुखेन ज्ञानार्थं {\qt 'गुणघ्नमूलोनयुतस्य'} इत्यादेः~। {\qt 'पाट्या च बीजेन च कुट्टकेन'} इति पृथङ्निर्देशस्तु तदतिशयार्थः~। यथा प्रमाणप्रमेयेत्यादिन्यायसूत्रे प्रमेयान्तर्गतत्वेऽपि प्रमाणादीनां पृथङ्निर्देशस्तथा
वा बीजचतुष्टयनिरपेक्षतयैव प्रश्नोत्तरार्थज्ञानहेतुत्वाद्वा~। तदेवं
युक्तोऽत्र कुट्टकारम्भः~। तत्र कुट्टको नाम गुणकः~।
हिंसावाचकशब्दैर्गुणनाभ्युपगमात्~। योगरूढ्या गुणकविशेषश्चायम्~।
कश्चिद्राशिर्येन गुणित उद्दिष्टक्षेपयुतोन उद्दिष्टहरेण भक्तः सन्निःशेषो
भवेत्स गुणकः कुट्टक इति पूर्वेषां व्यपदेशात्~। तत्र कुट्टकज्ञानार्थं
प्रथमविधेयमुद्देशखिलत्वं च शालिन्या निरूपयति\textendash 

\phantomsection \label{50}
\begin{quote}
    \ab 
    भाज्यो हारः क्षेपकश्चापवर्त्यः केनाप्यादौ सम्भवे कुट्टकार्थम्~। \\
येन च्छिन्नौ भाज्यहारौ न तेन क्षेपश्चैतद्दुष्टमुद्दिष्टमेव~॥~५०~॥
\end{quote}

 कश्चिद्राशिर्येन गुणित उद्दिष्टक्षेपयुतोन उद्दिष्टहरेण भक्तः सन् निःशेषो भवति 
तस्य गुणकस्य कुट्टक इति सञ्ज्ञेत्युक्तं प्राक्~। अत्रागता लब्धिर्लब्धिसञ्ज्ञैव~। 
हरो हरसञ्ज्ञ एव~। क्षेपोऽपि क्षेपसञ्ज्ञ एव~। अन्वर्थसञ्ज्ञाश्चैताः~। यो राशिर्गुण्यते 
तस्य भाज्य इति सञ्ज्ञा~। भजनयोगात्~। अस्य कुट्टकस्य ज्ञानार्थमादौ स भाज्यो 
हारः क्षेपकश्च केनापि तुल्येनाङ्केनापवर्त्यः~। भाज्यहारक्षेपा एकेनैवापवर्त्या इत्यर्थः~। 
कस्मिन्सति~। अपवर्तनसम्भवे सति~। अपवर्तनं नाम निःशेषभजनम्~। 
तच्चैकातिरिक्तेनाभिन्नेन द्रष्टव्यम्~। अन्यथा सति सम्भव इति विरुध्येत~।
एकेन भिन्नेन वा केनचिदङ्केन सर्वत्रापवर्तनसम्भवात्~। \hyperref[51]{\textbf{तौ भाज्यहारौ दृढसञ्ज्ञकौ स्तः}} इत्यस्य विवरणे 'दृढा' इत्यन्वर्थसञ्ज्ञा~। पुनर्नापवर्तन्ते न क्षीयन्त इत्यर्थ इति 
व्याख्यातवद्भिः श्रीगणेशदैवज्ञचरणैरप्युक्त एवायम् अर्थः~।
यत्त्वर्धेनापवर्त्येत्यादि क्वचिद्दृश्यते तद्द्विगुणत्वादिपरम्~। भाज्यहारक्षेपाणामपवर्तनसम्भवे
सत्यवश्यमपवर्त्या एव~। 
अन्यथा कुट्टकसिद्धिर्न सम्भवतीत्यर्थसिद्धम्~। उद्देशस्य
खिलत्वज्ञानार्थमाह\textendash \,येनेति~। 
येनाङ्केन भाज्यहारौ छिन्नावपवर्तितौ तेनैवाङ्केन क्षेपश्चेन्न
च्छिन्नोऽपवर्तितो न स्यात्तदा तदुद्दिष्टं पृच्छकेन पृष्टं दुष्टमेव~। अयं भाज्यो येन केनापि गुणितस्तेन क्षेपेण 
युतोनस्तेन हरेण भक्तः सन् कदाचिदपि निःशेषो न भवेदि-त्यर्थः~॥~५०~॥\\
\vspace{-4mm}

अथापवर्ताङ्कं कुट्टकेतिकर्तव्यतां चोपजातिकात्रयेणाह\textendash 
\newpage
%%%%%%%%%%%%%%%%%%%%%%%%%%%%%%%%%%%%%%%%%%%

\phantomsection \label{51}
\begin{quote}
    \ab 
 परस्परं भाजितयोर्ययोर्यः शेषस्तयोः स्यादपवर्तनं सः~। \\
तेनापवर्तेन विभाजितौ यौ तौ भाज्यहारौ दृढसञ्ज्ञकौ स्तः ॥\\

\vspace{-5mm}
मिथो भजेत्तौ दृढभाज्यहारौ यावद्विभाज्ये भवतीह रूपम्~। \\
फलान्यधोऽधस्तदधो निवेश्यः क्षेपस्तथान्ते खमुपान्तिमेन~। \\

\vspace{-5mm}
 स्वोर्ध्वे हतेऽन्त्येन युते तदन्त्यं त्यजेन्मुहुः स्यादिति राशियुग्मम्~। \\
 ऊर्ध्वो विभाज्येन दृढेन तष्टः फलं गुणाः स्यादपरो हरेण~॥~५१~॥
\end{quote}

 ययो राश्योः परस्परं भाजितयोः सतोर्यः शेषोऽङ्कः स तयोरपवर्तनं स्यात्~। 
तेन तौ निःशेषं भज्येते एव~। एतदुक्तं भवति~। हरेण भाज्ये भक्ते यच्छेषं स
हरो  भाजनीयः~। तच्छेषेणापि भाज्यशेषं तेनापि हरशेषमिति पुनः पुनः परस्परभजने 
क्रियमाणे यद्यन्ते रूपं शेषं स्यात्तदा तौ नावर्तेते एव~। रूपस्यैव
शेषत्वात्~। तेनापवर्ते भाज्यहारक्षेपाणामविकार एव~। यदा तु शून्यं शेषं
स्यात्तदा हरीभूतं यत्प्राक् शेषमधः स्थापितं तदेव भाज्यहरयोरपवर्तनं स्यात्~। शेषो ह्यपवर्ताङ्कः~। तस्मादमितमशेषोऽङ्क एवापवर्तनाङ्कः~। शून्यं शेषमिति तु शेषाभावपरम्~।
अन्यथापवर्तनं नाम निःशेषभजनमिति विरुध्येत~। तत्रापि शून्यशेषत्वात्~। एवं
ज्ञातेनापवर्ताङ्केन यौ भाज्यहारौ विभाजितौ तौ दृढसञ्ज्ञकौ स्तः~। तेनैव क्षेपोऽप्यपवर्त्यः~। \hyperref[50]{\textbf{भाज्यो हारः क्षेपकश्चापवर्त्य}} इत्युक्तत्वात्~। सोऽपि दृढसञ्ज्ञः स्यात्~। 'दृढा' इत्यन्वर्थसञ्ज्ञा~। पुनर्नापवर्तन्ते न क्षीयन्त इत्यर्थः~। 'दृढौ' इति सञ्ज्ञां वदता कृतेऽप्यपवर्ते यावदन्यदपवर्तनं सम्भवति तावदपवर्तनीयाविति ज्ञापितम्~।
पुनरपवर्तनं च स्वकल्पिताङ्केनापवर्ते  कृते~। अन्यथा परस्परं भाजितयोरित्यादिना ज्ञातेनापवर्ताङ्केनापवर्ते
पुनरपवर्तनासम्भवात्~। अथ तौ दृढभाज्यहारावुक्तवन्मिथः परस्परं तावद्भजेत्~।
तावत्कथम्~। यावद्विभाज्ये भाज्यस्थाने रूपं भवति~। इहैतेषु परस्परभजनेष्वागतानि
फलान्यधोऽधो  निवेश्यानि~। फलं च फले च फलानि च फलानि~। द्वन्द्वैकशेषः~। एकमेव फलं लब्ध्वा यदि रूपं शेषं स्यात्तदा तदेकमेव फलं स्थाप्यम्~। द्वे चेत्तर्हि
द्वे स्थाप्ये~।  बहूनि चेत्तर्हि बहूनि स्थाप्यानीत्यर्थः~। तेषां फलानां वल्लीवदधोऽधः
स्थापितानामधः  क्षेपो निवेश्यः~। दृढ इति पूर्वानुवृतिः~। तथेति पदाद्वा दृढत्वं
क्षेपस्यावगन्तव्यम्~। अस्मिन्पक्षे तथेति पदस्य नाग्रेऽन्वयः~। तथा तेषामप्यधोऽन्ते खं
निवेश्यम्~। एवं वल्ली जायते~। तत उपान्तिमेनाङ्केन स्वोर्ध्वे स्वोर्ध्वस्थितेऽङ्के
हतेऽन्त्येनाङ्केन युते च सति तदन्त्यं त्यजेत्~। इति मुहुरुपान्तिमेन स्वोर्ध्वे हतेऽन्त्येन
युते तदन्त्यं
\newpage
%%%%%%%%%%%%%%%%%%%
\noindent त्यजेदिति पुनः पुनः कृते राशियुग्मं स्यात्~।
तत्रोर्ध्वराशिर्दृढेन विभज्येन तष्टः
सन्फलं भवेत्~। फलं नाम लब्धिः~। अपरोऽधस्तनो राशिर्दृढेन हरेण तष्टः
सन्गुणः स्यात्~। तक्षू त्वक्षू तनूकरणे~। कर्मणि क्तः~।
तष्टस्तनूकृतः 
कृशीकृतोऽवशेषित इति यावत्~। भवत्वावशेषितराशिर्ग्राह्यो न तु
लब्धमित्यर्थः~। 
तेन गुणेन दृढभाज्ये गुणिते दृढक्षेपयुतोने दृढहरेण भक्ते शेषं न स्यादिति~।
उद्दिष्टेष्वपि भाज्यहारक्षेपेषु ते एव गुणलब्धी स्त इत्यर्थसिद्धमविशेषात्~॥~५१~॥

\phantomsection \label{52}
\begin{quote}
    \ab 
    एवं तदैवात्र यदा समास्ताः स्युर्लब्धयश्चेद्विषमास्तदानीम्~। \\
यथागतौ लब्धिगुणौ विशोध्यौ स्वतक्षणाच्छेषमितौ तु तौ स्तः~॥~५२~॥
\end{quote}

 एवं तदैव स्यात्~। यदात्र परस्परभजने ता आगता लब्धयः समाः स्युर्द्वे 
चतस्रः षडित्यादयः~। यदि तु ता लब्धयो विषमाः स्युरेका तिस्रः पञ्च 
वेत्यादयस्तदानीमुक्तप्रकारेण यथागतौ तौ लब्धिगुणौ स्वतक्षणच्छोध्यौ
शेषतुल्यौ तौ लब्धिगुणौ स्तः~। तक्ष्यते तनूक्रियतेऽनेनेति तक्षणः~। तक्ष्णोतीति तक्षणम्
इति वा~।
स्वश्चासौ तक्षणश्च स्वतक्षणस्तस्मात्~। गुणो दृढहाराच्छोध्यो
लब्धिर्दृढभाज्याच्छोध्येत्यर्थः~।\\

\vspace{-4mm}
\hyperref[50]{\textbf{भाज्यो हारः क्षेपकश्चापवर्त्यः}} इत्यत्र तावदियं
युक्तिः\textendash \,अनपवर्तितयोर्ययोर्भाज्यभाज-कयोर्यादृशी लब्धिस्तयोः केनचिदेकेनाङ्केन गुणितयोरपर्तितयोर्वा तादृगेव
लब्धिरिति तु  प्रसिद्धम्~। प्रकृते तु कल्पितभाज्यः केनचिद्गुणकेन गुणितो
धनर्णक्षेपयुतः सन्भाज्यः 
 स्यात्~। भाजकस्तु यथास्थित एव~। तथा चात्र भाज्यस्य खण्डद्वयम्~।
गुणगुणितकल्पितभाज्य एक क्षेपो द्वितीयम्~। अनयोर्योगे भाज्ये सिद्धे
भाज्यभाजकयोरपवर्ते 
कृतेऽपि नास्ति लब्धिवैलक्षण्यम्~। तस्माद्येन भाजकोऽपवर्तितस्तेन 
खण्डद्वययोगलक्षणो भाज्योऽप्यपवर्त्यः~। तत्र
योगापवर्तनेऽपवर्तितखण्डकयोर्योगे 
वा तुल्यैव स्यात्~। भाज्यभाजकौ २७~। १५ त्रिभिरपवर्ते जातौ ९~। ५~।
यद्वा 
भाज्यखण्डे ९~। १८ अनयोस्त्रिभिः अपवर्ते ३~। ६ योगे च जातः स
एवापवर्तितभाज्यः 
९~। एवमन्यादृगपि  खण्डद्वयं बहूनि वा खण्डानि विधायपवर्त्य
तद्योगेऽपवर्तितभाज्ये 
एव स्यात्~। तस्माद्भाजकस्यापवर्तने गुणगुणितकल्पितभाज्योऽपवर्त्यः
क्षेपोऽप्यपवर्त्यः~। 
तत्र यद्यपि गुणस्याज्ञातत्वाद्गुणगुणितभाज्यस्याप्यज्ञाने
तस्यापवर्तनमशक्यं तथापि
\newpage
%%%%%%%%%%%%%%%%%%%%%%%%%%%%%%%%%%%%%%%%
\noindent कल्पितभाज्येऽपवर्तिते पश्चाद्गुणकेन गुणिते गुणगुणितकल्पितभाज्यलक्षणो
भाज्यखण्ड एवापवर्तितः स्यात्~। गुणितस्यापवर्तनेऽपवर्तितस्य वा
गुणनेऽविशेषात्~। तथा च कल्पितभाज्यो येन गुणेन गुणितः सन् भाज्यखण्डं
भवत्यपवर्तितभाज्योऽपि 
तेनैव गुणेन गुणितः सन्नपवर्तितं भाज्यखण्डं भवेत्~। अपवर्तितक्षेपश्च
द्वितीयम्~। 
तदेवं भाज्यहारक्षेपाणाम् अनपवर्तितानामपवर्तितानां च
गुणलब्ध्योरविशेषाल्लाघवाच्च 
\hyperref[50]{\textbf{भाज्यो हारः क्षेपकश्चापवर्त्यः}} इत्युक्तम्~। अपवर्तनम् आवश्यकं न वेति \hyperref[51]{\textbf{मिथो भजेत्तौ दृढभाज्यहारौ}} इत्यादेरुपपत्तौ विचारयिष्यते~।\\

\vspace{-4mm}
 अथ खिलत्वोपपत्तिः~। इह भाज्यभाजकयोरपवर्ते यद्यपि न
तल्लब्धेर्वैचित्र्यं  तथापि शेषस्य तदस्त्येव~। अपवर्तितयोः शेषमपवर्ताङ्केन गुणितं
सदनपवर्तितयोः शेषं  स्यात्~। यथा भाज्यभाजकौ २१।१५ त्रिभिरपवर्तितौ ७।५~। 
अत्रैकगुणे भाज्ये स्वस्वहरभक्ते सति शेषे ६।२ द्विगुणिते भाज्ये स्वस्वहरभक्ते सति शेषे १२।४ त्रिगुणितस्य
शेषे ३।१ चतुर्गुणितस्य ९।३ पञ्चगुणितस्य ०।० षडादिभिर्गुणने पुनस्तान्येव
शेषाणि स्युः~। तस्मादत्र गुणकमात्रेऽपवर्तितहरेऽस्मिन् ५ शेषं ०~। १~। २~।
३~। ४ एभ्योऽन्यन्न स्यात्~। अन-पवर्तितहरे १५ तु शेषं ०~। ३~। ६~। ९~। १२ एभ्योऽन्यन्न स्यात्~। अत्र सर्वेषामपि शेषाणामेकादिगुणितापवर्ताङ्करूपत्वादपवर्तः
स्यादेव~। अथ क्षेपविचारः~। तत्र शून्यशेषे गुणके क्षेपाभाव एकादिगुणितहरतुल्ये वा
क्षेपे शून्यं शेषं स्यान्नान्यस्मिन्क्षेपे~। तथा च हरस्यापवर्तनसम्भवे
क्षेपस्य सुतराम् अपवर्तनसम्भवः~। अथान्यशेषेषु सकलगुणकेषु शेषतुल्य
ऋणशेषक्षेपशेषोनहरतुल्ये धनक्षेपे वैकादिगुणितहरयुत्योरुभयोर्वा शून्यं
शेषं स्यात्~। नान्यस्मिन्क्षेपे~। अत्र शेषतुल्यस्य शेषोनहरतुल्यस्य वा
क्षेपस्योक्तशेषेष्वेवान्तर्भावादपवर्तः स्यादेव~। एवं केवलस्यापवर्तसम्भवे
हरयुतस्य क्षेपस्य सुतरामपवर्तसम्भवः~। तदेवं न कम् अपि तादृशं क्षेपं पश्यामो
यो भाज्यहरापवर्ताङ्केन 
नापवर्तेत~। तस्माद्यत्र क्षेपेऽपवर्तो न स्यात् तादृशक्षेपे शून्यशेषता
कथमपि न स्यात्~। शून्यशेषक्षेपाणामुक्तरीत्या नियतत्वादित्यलं पल्लवितेन~।
तस्मात् \hyperref[50]{\textbf{येन च्छिन्नौ भाज्यहारौ न तेन क्षेपश्चैतद्दुष्टमुद्दिष्टमेव}} इति सुष्ठूक्तम्~।
अथापवर्ताङ्कज्ञानार्थं 
युक्तिः~। अपवर्ताङ्कश्चात्रापवर्ताङ्केषु महान्ज्ञातव्यो येनापवर्तितयोः
पुननापवर्तः स्यात्~।
अनेनापवर्तितयोर्दृढत्वोक्तेः~। अथ तज्ज्ञानार्थमुपायः~।
तत्र भाज्यभाजकयोस्तुल्यत्वे 
तन्मित एव महानपवर्ताङ्क इति मन्दैरप्यवगम्यते~। तयोर्वेलक्षण्ये तु स
विचारपदवीमारोढुमर्हति~। तत्र द्वयोः मध्ये यः १९५ लघुस्ततोऽधिकोऽपवर्ताङ्को
नैव स्यात्~। तेनाङ्केन लघोरपवर्तनस्य बाधितत्वात्~। लघुतुल्यस्तु स्यात्~।
यदि
\newpage
%%%%%%%%%%%%%%%%%%%%%%%%%%%%%%%%%%%%%%%%
\noindent लघुना महति भक्ते शेषं न स्यात्~। निःशेषभजनरूपत्वात्तस्य~। यदि च शेषं २६ स्यात्तदा न लघुतुल्योऽपवर्ताङ्कः~। किन्त्वधिकस्य
बाधितत्वाल्लघोरपि। 
लघुर्महानपवर्ताङ्कः स्यात्~। तत्रापि विचारः~। अत्र हि महतो राशेः 
खण्डद्वयम्~। यावल्लघुना भक्तं तावदेकं १९५ शेषतुल्यमपरं २६~। एवं सति 
लघुतो न्यूनIङ्केषु मध्ये यः शेषतः २६ अधिकस्तस्य नास्त्येवापवर्तकत्वम्~। 
तेन यथाकथञ्चिल्लघोः १९५ अपवर्ते लघुराशिभक्तस्याधिकराशिखण्डस्य १९५ 
अप्यपवर्तः स्यान्न तु शेषतुल्यद्वितीयखण्डस्य २६~। तथा च लघुतः १९५
 न्यूनाङ्केषु यदि महानपवर्ताङ्कः स्यात्तर्हि शेषतुल्यः २६ तथा च लघुः
स्यात्~। परं शेषेण २६ लघुराशौ १९५ भक्ते यदि शेषं न स्यात्तथा 
सति शेषतुल्याङ्केन लघोरपवर्तनस्य जातत्वाल्लघुभक्तस्याधिकराशिखण्डस्य १९५
शेषतुल्यद्वितीयखण्डस्य २६ अप्यपवर्तः स्यात्~। यदि तु शेषं स्यात्तर्हि
पूर्वशेषतः २६ न्यून एव महानपवर्ताङ्कः स्यान्नाधिकः~। अधिकस्य
बाधितत्वात्~। अथ तत्रापि विचारः~। लघुराशेर्हि खण्डद्वयं १८२।१३ यावत्पूर्वशेषेण भक्तं तावदेकं १८२ द्वितीयशेषतुल्यं द्वितीयं १३~। एवं सति
पूर्वशेषान्न्यूनाङ्केषु 
यो द्वितीयशेषादधिकः स्यान्न स्यादयमपवर्ताङ्कः~। तेन यथाकथञ्चित्पूर्वशेषस्य
२६ अपवर्ते शेषभक्तलघुखण्डकस्य १८२ अपवर्तः स्यान्न
द्वितीयशेषतुल्यद्वितीयखण्डस्य १३~। तथा सति लघुराशेरनपवर्तनाल्लघुभक्तस्याधिकराशिखण्डस्य 
१९५ अप्यनपवर्तः न स्यात्~। तस्मात्पूर्वशेषतः २६ न्यूनाङ्केषु यदि 
महानपवर्ताङ्कः स्यात्तर्हि द्वितीयशेषतुल्यः १३ एव स्यात्~। परं
द्वितीयशेषेण १३ पूर्वशेषे २६ भक्ते यदि शेषं न स्यात्~। यतस्तथा सति पूर्वशेषस्य 
२६ अपवर्तस्य जातत्वात् २६ तद्भक्तस्य लघुराशिखण्डस्य १८२ अथ च
द्वितीयशेषतुल्यद्वितीयखण्डस्य २६ अप्यपवर्तः स्यात्~। तथा सति लघुराशेः 
१९५ अपवर्तनस्य जातत्वाल्लघुभक्तस्याधिकराशिखण्डस्य १९५ अप्यपवर्तः
स्यात्~। 
पूर्वशेषतुल्यस्य द्वितीयखण्डस्य २६ अप्यपवर्तोऽनुपदम् एव परं यदीति
ग्रन्थेन प्रतिपादित इत्यधिकराशेरप्यपवर्तः स्यादेव~। यदि च द्वितीयशेषेण
पूर्वशेषे भक्ते शेषं स्यात्तर्ह्यनयैव युक्त्या तृतीयशेषतुल्यो महानपवर्ताङ्कः स्यात्~। एवम् अनयोपपत्त्या पूर्वपूर्वशेष उत्तरोत्तरेण येन शेषेण भक्ते शेषं न स्यात्तच्छेषं
महानपवर्ताङ्कः स्यात्~। तदेवमुपपन्नम् \hyperref[51]{\textbf{परस्परं भाजितयोर्ययोर्यः शेषस्तयोः स्यादपवर्तनं सः}} इति~।
\newpage
%%%%%%%%%%%%%%%%%%%%%%%%%%%%%%%%%%%%%%%%%5
अथ \hyperref[51]{\textbf{मिथो भजेत्तौ दृढभाज्यहारौ}} इत्यादावुपपत्तिः~। क्षेपाभावे तावच्छून्येन 
भाज्ये गुणिते हरभक्ते शेषं न स्यादिति शून्यमेव गुणो लब्धिश्च~। यदि वा 
हरतुल्ये गुणे गुणहरयोस्तुल्यत्वान्नाशे भाज्यतुल्या लब्धिः स्याच्छेषं
च न स्यात्~। एवं द्व्यादिगुणितहरतुल्ये गुणे हरेण गुणहरयोरपवर्ते गुणस्थाने 
द्व्यादयः स्युरिति द्व्यादिगुणितभाज्यतुल्या लब्धिः स्याच्छेषं च न स्यात्~। 
तस्मात् क्षेपाभावे शून्यमिष्टाहतहरो वा गुणः~। लब्धिस्तु
शून्यम् इष्टाहतभाज्यो वेति~। 
एवमत्र हरतुल्यो गुणोपचयो भाज्यतुल्यो लब्ध्युपचयः सर्वत्र~। अत एव
वक्ष्यति~। \hyperref[59]{\textbf{इष्टाहतस्वस्वहरेण युक्ते ते वा भवेतां बहुधा गुणाप्ती}} इति~। अथ
सत्यपि क्षेपे हरतुल्ये  द्वयादिगुणितहरतुल्ये वा तस्मिन्पूर्वोक्त एव
शून्यादिको गुणः स्यात्~। सति हि पूर्वोक्तगुणके क्षेपवशादेव शेषं स्यात्~। क्षेपोऽपि
यद्येकादिर्गुणितहरतुल्यः स्यात्तर्हि शेषं कुतः स्यात्~। तस्मादेतादृशे
क्षेपे सत्यपि पूर्वोक्त एव गुणः~। लब्धौ तु हरभक्ते क्षेपे यल्लभ्यते तावदधिकं स्यात् 
धनक्षेपे~। ऋणक्षेपे तु तावन्न्यूनं स्यात्~। अत एव वक्ष्यति
\hyperref[58]{\textbf{क्षेपाभावोऽथवा यत्र क्षेपः शुध्येद्धरोद्धृतः~। ज्ञेयः शून्यं गुणस्तत्र क्षेपो हरहृतः
फलम्}} इति~। अथान्यथा क्षेपे भाज्यखण्डद्वयेनोपपत्तिः~। हरेण यावद्भाज्यं तावदेकं
शेषमपरम्~। यथा भाज्यभाजकौ १६~। ७ उक्तवज्जाते भाज्यखण्डे १४~। २~। अत्र पूर्वखण्डस्य हरेण निःशेषभजनाद्येन केनापि गुणकेन गुणितस्यापि स्यादेव तस्य निःशेषभजनम्~।
अथोद्दिष्टक्षेपः परखण्डेन भक्तः सन्यदि शुध्येत्तर्ह्यत्र या लब्धिः स एव
गुणकः स्यात् परं वियोगे~। यतस्तेन गुणकेन गुणितस्य भाज्यापरखण्डस्य 
क्षेपसमत्वनियमात् क्षेपवियोगे
नाशः स्यादेव~। अथ यदि न शुध्येत्तह्यशक्यो गुणकावगमः~। अतोऽन्यथा
यतितव्यम्~। 
भोजकेन भाज्ये भक्ते यदि रूपं शेषं स्यात्तर्हि द्वितीयखण्डमपि रूपं स्यात्~। 
तथा सति येन केनापि क्षेपेण तस्य गुणने क्षेपसमत्वनियमात् उक्तयुक्त्या 
क्षेपसम एव गुणः परं वियोगे~। योगे तु क्षेपोनहरो गुणः~। यतस्तेन गुणितं भाज्यपरखण्डं क्षेपो न हरसमं स्यादस्य च क्षेपयोगे हरसमता स्यादिति हरेण
निःशेषभजनं स्यादेव~। लब्धिस्तु केवलभाज्ये हरभक्ते या स्यात्सैव गुणगुणिता सती
गुणितभाज्यं स्यात्~। 
परं वियोगे योगे तु तादृशी सैका~। परखण्डस्य
शुद्ध्याभावाद्धरतुल्यशेषत्वाच्च~। अथ यदि 
भाज्ये हरेण भक्ते रूपं शेषं न स्यात्तर्हि गुणकावगमो दुर्गमः~। अतो 
भाज्यशेषेण हरं भजेत्~। अत्र च हरो भाज्यः~। भाज्यशेषं भाजकः~। 
अत्रापि यदि रूपं शेषं स्यात्तर्हि क्षेपतुल्यो गुणो वियोगे~। योगे तु
क्षेपोनहरो गुणः
\newpage
%%%%%%%%%%%%%%%%%%%%%%%%%%%%%%%%%%%%%5
पूर्ववल्लब्धिश्च~। उक्तयुक्तेरविशेषात्~। अत्रापि यदि रूपं शेषं न
स्यात्तर्हि नास्ति गुणकानुगमः सुगमः~। तस्मादस्यापि शेषेण हरीभूतं शेषं भजेत्~। तत्र
यदि रूपं शेषं स्यात्तर्हि तस्मिन् भाज्य उक्तयुक्त्या
क्षेपाङ्कतुल्यश्च~। क्षेपोनहरतुल्यश्च गुणः स्यात् वियोगयोगयोः~।
अत्रापि रूपाधिके शेषे गुणो दुर्गमः~। तस्मात्परस्परभजने सति 
कुत्र-चिद्रूपं शेषमपेक्षितम्~। तच्च सत्यपवर्तनसम्भवे भाज्यभाजकयोरनपवर्ते
कथं स्यात्~। किं तु तत्रापवर्ताङ्कतुल्यं शेषं
स्यात् परस्परभजनेऽन्त्यशेषस्यैवापवर्ताङ्कत्वात्~। 
कृते तु अपवर्ते शेषमप्यपवर्ताङ्केनापवर्तितं स्यात्~। अन्त्यशेषं
त्वपवर्ताङ्कतुल्यम्~। तच्चेदपवर्ताङ्केनापवर्तितं
स्याद्रूपमेवान्त्यशेषं स्यादिति जातं भाज्यभाजकयोरपवर्तस्यावश्यकत्वम्~। ननु यद्यप्युपान्तिमशेषतुल्ये भाज्ये पूर्वशेषेण भक्ते रूपं
शेषं स्यात् इति ज्ञातस्तस्मिन् गुणस्थाप्युद्दिष्टभाज्ये~। कथं गुणकसिद्धिरिति
चेत्~। व्यस्तविधिना तम् अवगच्छ~। तथा हि\textendash \,भाज्यभाजकक्षेपाः

\vspace{-2mm}
\begin{table}[h!]
    \centering\s
    \begin{tabular}{l}
        भ १२११ क्षे २१~।\\
         ह ४९७ 
    \end{tabular}
\end{table}
\vspace{-2mm}

\noindent अत्र परस्परं भाजितयोर्भाज्यभाजकयोरन्त्यशेषं ७~। अनेनापवर्तिता
भाज्यहरक्षेपाः~।

\vspace{-2mm}
\begin{table}[h!]
    \centering\s
    \begin{tabular}{l}
    भा १७३ क्षे ३~। \\
        ह ७१
    \end{tabular}
\end{table}
\vspace{-2mm}

\noindent अत्र दृढयोरेतयोर्भाज्यभाजकयोः परस्परभजनाल्लब्धिशेषयोर्वल्ल्यौ।

\vspace{-2mm}
\begin{table}[h!]
    \centering\s
    \begin{tabular}{ccr}
       ल & &शे\\
२& &३१\\
२& &६\\
३& &४\\
२& &१
    \end{tabular}
\end{table}
\vspace{-2mm}

\noindent क्रमेण भाज्यभाजकाश्च जाताः

\vspace{-2mm}
\begin{table}[h!]
    \centering\s
    \begin{tabular}{crcrcrcr}
        भा &१७३~।& भा &७१~।& भा &३१~।& भा &९\\
 ह &७१~।& ह& ३१~।& ह& ९~।& ह& ४
    \end{tabular}
\end{table}
\newpage
%%%%%%%%%%%%%%%%%%%%%%%%%%%%%%%%%%%%%%
\noindent अत्रान्त्यभाज्ये खण्डद्वयम्~। यावद्धरभक्तं तावदेकं शेषमपरम्~। एवं खण्डे
८~। १~। उक्तयुक्त्या वियोगे जातः क्षेपसमो गुणः ३ केवलभाज्यलब्धिर्गुणगुणिता सती
लब्धिः स्यादिति प्रकृतेऽन्त्यभाज्यलब्धिः २ गुणेनानेन ३ गुणिता लब्धिश्च ६~।
तदिदमुक्तं~। 
\hyperref[51]{\textbf{मिथो भजेत्तौ दृढभाज्यहारौ यावद्विभाज्ये भवतीह रूपम्~। फलान्यधोऽधस्तदधो
निवेश्यः क्षेपः}} इति~। 

\vspace{-2mm}
\begin{table}[h!]
    \centering\s
    \begin{tabular}{cc}
      फ&\\
२&\\
२&\\
३&\\
२&\\
३& क्षे\\
    \end{tabular}
\end{table}
\vspace{-2mm}

एवमत्रान्त्यो जातो गुणः~। अन्त्येन हतः स्वोर्ध्वो लब्धिश्चेति जातम्~।

\vspace{-2mm}
\begin{table}[h!]
    \centering\s
    \begin{tabular}{rl}
     २&\\
२&\\
३&\\
 ल ६ &\\
गु ३& \hspace{-4mm} क्षे
   \end{tabular}
\end{table}  
\vspace{-2mm}

अथास्मिन्नेव क्षेपेऽस्मात्पूर्वभाज्येऽस्मिन्

\vspace{-2mm}
\begin{table}[h!]
    \centering\s
    \begin{tabular}{ll}
भा &३१\\
ह &९
\end{tabular}
\end{table}
\vspace{-2mm}

\noindent गुणो विचार्यते~। अत्राप्युक्तवत्खण्डे २७~। ४ अत्र पूर्वखण्डं येन केनापि
गुणितं हरभक्तं निःशेषं स्यादेव~। अतोऽपरखण्डादेव गुणविचारो युक्तः~। अतो जातौ
भाज्यभाजकौ ४~। ९। अत्रान्त्यभाज्यभाजकयोर्व्यत्यासोऽस्तीति गुणलब्ध्योरपि
व्यत्यासमात्रम्~। 
तत्र युक्तिः~। भाज्ये ९ गुणेन ३ गुणिते २७ क्षेपेण ३ वियुक्ते २४ हरेण ४
भक्ते
\newpage
%%%%%%%%%%%%%%%%%%%%%%%%%%%%%%%%%%%%%%%%
\noindent सति लब्धिः ६ भवति~। अतो व्यस्तविधिना लब्ध्या ६ हरे ४ अस्मिन् 
गुणिते २४ क्षेप\textendash \,३\textendash \,युते २७ भाज्य\textendash \,९\textendash \,भक्ते लब्धो गुणः ३~। तदेवं पर्यवस्यति~। अयं
भाज्यः ४ तस्य लब्ध्या ६ गुणितः २४ तेन क्षेपेण ३ युतः २७ स्वहरेणानेन ९ भक्तः सञ्शुध्यतीत्यन्त्यभाज्यलब्धिरेव ६ अत्र गुणका लब्धिश्चान्त्यभाज्यगुणः ३~। एवं वल्ल्यां जातं

\vspace{-2mm}
\begin{table}[h!]
    \centering\s
    \begin{tabular}{ll}
        &२\\
&२\\
&३\\
गु& ६\\
ल &३
    \end{tabular}
\end{table}
\vspace{-2mm}

\noindent परमत्र भाज्ये पूर्वखण्डलब्धिर्गुणगुणिता सती स्यात्~। गुणश्चात्र
वल्ल्यामुपान्तिमः ६ पूर्वखण्डलब्धिश्च तदूर्ध्वं तिष्ठति ३~। अत
उपान्तिमेन स्वोर्ध्वे हते जाता पूर्वखण्डलब्धिः १८ द्वितीयखण्डलब्धिश्च वल्ल्यामन्त्या ३ अतस्तया युता पूर्वखण्डलब्धिः १८ 
अस्मिन्भाज्ये सकला लब्धिः स्यात् २१~। एवं जातं वल्ल्यां

\vspace{-2mm}
\begin{table}[h!]
    \centering\s
    \begin{tabular}{lr}
       &२\\
&२\\
ल &२१\\
गु& ६\\
ल &३
    \end{tabular}
\end{table}
\vspace{-2mm}

\noindent अस्मिन्भाज्ये गुणलब्ध्योः सिद्धत्वादधःस्थलब्धेः प्रयोजनाभावादपगमे जातं वल्ल्यां

\vspace{-2mm}
\begin{table}[h!]
    \centering\s
    \begin{tabular}{lr}
     &२\\
&२\\
ल &२१ \\
 गु &६
    \end{tabular}
\end{table}
\vspace{-2mm}

\noindent तदिदमुक्तं \hyperref[51]{\textbf{उपान्तिमेन स्वोर्ध्वे हतेऽन्त्येन युते तदन्त्यं त्यजेत्'}} इति~। एकस्मिन्भाज्ये ३१~। ९ व्यस्तविधिना जातौ लब्धिगुणौ २१~। ६ योगेऽथ 
तदूर्ध्वभाज्येऽस्मिन्
\newpage
%%%%%%%%%%%%%%%%%%%%%%%%%%%%%%%%
\begin{table}[h!]
    \centering\s
    \begin{tabular}{cl}
      भा &७१ \\
ह &३१
    \end{tabular}
\end{table}
\vspace{-2mm}

\noindent तस्मिन्नेव क्षेप\textendash \,३\textendash \,गुणो विचार्यते~। अत्राप्युक्तवत्खण्डे ६२~। ९
पूर्वखण्डं पृथक्संस्थाप्य जातौ भाज्यहरौ ९~। ३१ अत्राप्यनुपदं
प्रदर्शितयोर्भाज्यभाजकयोर्व्यत्यासाल्लब्धिगुणव्यत्यासमात्रम्~। व्यस्तविधेस्तुल्यत्वात्~। तथा जातं वल्ल्यां

\vspace{-2mm}
\begin{table}[h!]
    \centering\s
    \begin{tabular}{lr}
       & २\\
&२\\
गु& २१\\
ल &६
    \end{tabular}
\end{table}
\vspace{-2mm}

\noindent अत्रापि पूर्वखण्डलब्धिर्गुणगुणिता स्यात्~। गुणोऽत्राप्युपान्तिमः~। तदूर्ध्वे च पूर्वखण्डलब्धिः २~। अत उपान्तिमेन स्वोर्ध्वे हते जाता पूर्वखण्डलब्धिः ४२ इयं द्वितीयखण्डलब्ध्यात्मकेनान्त्येन ६ युता जाता सम्पूर्णा लब्धिः ४८~। एवं जातं वल्ल्यां 
\vspace{-2mm}

\begin{table}[h!]
       \centering\s
       \begin{tabular}{lr}
            &२\\
        ल&४८\\
        गु &२१
       \end{tabular}
\end{table}
\vspace{-2mm}

\noindent अत्राप्यधःस्थलब्धेः प्रयोजनाभावादपगमे जातं
\vspace{-2mm}

\begin{table}[h!]
    \centering\s
    \begin{tabular}{lr}
        &२\\
ल &४८\\
गु& २१
    \end{tabular}
\end{table}
\vspace{-2mm}

\noindent एवास्मिन्भाज्ये ७१~। ३१ व्यस्तविधिना जातौ वियोगे लब्धिगुणौ ४८~। २१~।
अथ तदूर्ध्वे भाज्ये मुख्येऽस्मिन् १७३ गुणविचारः~। अत्राप्युक्तवत्खण्डे १४२~। ३१ कृत्वा जातौ भाज्यभाजकौ~। अत्राप्यनुपदं
सिद्धगुणयोर्भाज्यभाजकयोर्व्यत्यासाल्लब्धिगुणयोः 
क्षेपस्य च व्यत्यासे जातौ क्षेपयोगे लब्धिगुणौ २१~। ४८~। जातं
वल्ल्यां
\newpage
%%%%%%%%%%%%%%%%%%%%%%%%%%%%%%%%%%%%%%
\begin{table}[h!]
    \centering\s
    \begin{tabular}{lr}
       &२\\
गु& ४८\\
ल &२१
    \end{tabular}
\end{table}
\vspace{-2mm}

\noindent अत्रापि पूर्वखण्डलब्ध्यर्थमुपान्तिमेन ४८ स्वोर्ध्वे २ हते ९६
सकललब्ध्यर्थमन्त्येन 
२१ युते ११७ जातं वल्ल्यां
\vspace{-2mm}

\begin{table}[h!]
    \centering\s
    \begin{tabular}{lr}
       ल &११७ \\
गु& ४८ \\
ल &२१
    \end{tabular}
\end{table}
\vspace{-2mm}

\noindent अधःस्थलब्धेः प्रयोजनाभावादपगमे जातं
\vspace{-2mm}

\begin{table}[h!]
    \centering\s
    \begin{tabular}{lr}
       ल &११७\\
गु& ४८
    \end{tabular}
\end{table}
\vspace{-2mm}

\noindent तदेवं मुख्यभाज्येऽस्मिन्
\vspace{-2mm}

\begin{table}[h!]
    \centering\s
    \begin{tabular}{cl}
       भा &१७३ क्षे ३ \\
 ह& ७१
    \end{tabular}
\end{table}
\vspace{-2mm}

\noindent क्षेपयुतौ जातौ लब्धिगुणौ ११७।४८~। तदिदमुक्तं \hyperref[51]{\textbf{मुहुः स्यादिति राशियुग्मम्}} इति~। अत्र विनान्त्यभाज्यं सर्वेषु भाज्येषु पूर्वखण्डलब्धिसाधने
गुणस्योपान्तिमत्वादुपान्तिमेन स्वोर्ध्वे हत इति~।
सकललब्धिसाधनार्थमुत्तरखण्डलब्ध्यात्मकेऽन्त्येन युते, 
इति च वक्तव्यम्~। अन्त्यभाज्ये तु
गुणस्यान्तिमत्वादुत्तरखण्डलब्धेरभावाच्च~। 
अन्त्येन हते स्वोर्ध्वे, इत्येव वक्तव्यं स्यादत आचार्येण तदन्तेऽपि
शून्यनिवेशनमुक्तम्~। हतस्तथा कृते सर्वत्रोपान्तिमेन \hyperref[51]{\textbf{स्वोर्ध्वे हतेऽन्त्येन युते तदन्त्यं त्यजेत्}} इत्यनुगमः स्यात्~। एवं सिद्धौ लब्धिगुणौ 
\vspace{-2mm}

\begin{table}[h!]
    \centering\s
    \begin{tabular}{lr}
         ल &११७\\
गु &४८
    \end{tabular}
\end{table}
\newpage
%%%%%%%%%%%%%%%%%%%%%%%%%%%%%%%%%%%%%%%%%%5
\noindent अत्र हरतुल्ये गुणोपचये भाज्यतुल्यो लब्धेरुपचयो भवतीत्युक्तं प्राक्~।
तयैव युक्त्या हरतुल्ये गुणापचये भाज्यतुल्यो लब्धेरपचयः स्यात्~। अतो हराधिके
गुणे यथासम्भवमेकादिगुणो हरस्तस्मादपनेयः~। स लघुतरो गुणः स्यात्~। एवमेव 
तल्लब्धिश्च~। अत उक्तम् \hyperref[51]{\textbf{ऊर्ध्वो विभाज्येन दृढेन तष्टः फलं गुणः
स्यादपरो हरेण}} इति~। उक्तयुक्त्यैव वक्ष्यति~। \hyperref[55]{\textbf{गुणलब्ध्योः समं ग्राह्यं धीमता तक्षणे फलम्}} इति~। न हि गुणस्यैकगुणहरतुल्यापचये द्विगुणभाज्यतुल्यो लब्धेरपचयः
सम्भवतीत्यादि~। नन्वेवं सिद्धयोर्मुख्यभाज्यस्य लब्धिगुणयोर्योगजत्वं
वियोगजत्वं वा कथमवगन्तव्यमन्त्योपान्तिमादिषु भाज्येषु गुणस्य
योगजवियोगजत्वयोरननुगमादिति 
चेदुच्यते~। अन्त्ये भाज्ये क्षेपतुल्यो वियोगजो गुण इत्युक्तम् असकृत्~। अतो
व्यस्तविधिना योगजो गुणः स्यात् उपान्तिमभाज्ये~। पुनरतो व्यस्तविधिना
तृतीयभाज्ये वियोगजो गुणः स्यात्~। एवं चतुर्थे योगजः पञ्चमे वियोगज
इत्यादिनान्त्यभाज्यादारभ्य समभाज्ये योगजो विषमभाज्ये तु वियोगजो गुणः स्यात्~। 
तत्र मुख्यभाज्यस्य विषमता समता वा परस्परभजनलब्धीनां विषमतया समतया वा 
नियता भवति~। तस्मात्परस्परभजने यदि लब्धयः समास्तदा योगजौ लब्धिगुणौ 
यदि विषमास्तदा वियोगजौ लब्धिगुणौ मुख्यभाज्ये स्याताम्~। तत्र
वियोगजयोर्लब्धिगुणयोर्वक्ष्यमाणत्वादत्र योगजयोरेव प्रतिपादनं युक्तम्
\hyperref[52]{\textbf{एवं तदैवात्र यदा समास्ताः स्युर्लब्धयः}} इति~। विषमलब्धिषु पुनर्वियोगजौ लब्धिगुणौ सिध्यतः~।
अपेक्षितौ च योगजौ अत उक्तम् \hyperref[52]{\textbf{स्युर्लब्धयश्चेद्विषमास्तदानीं यथागतौ लब्धिगुणौ विशोध्यौ~। स्वतक्षणाच्छेषमितौ तु तौ स्तः}} इति~। वियोगजो गुणो 
हराच्छुद्धः सयोगजो भवेदित्यत्र युक्तिः प्रागुक्ता~। अथवान्यथोच्यते~। यो भाज्यो येन
गुणेन गुणितः स्वहरेण भक्तो निःशेषः स्यात्स तद्गुण-खण्डाभ्यां पृथग्गुणितः 
पृथग्भाजकेन भक्तः शुध्येदेव~। लब्धियोगश्च लब्धिः स्यात्~। यदा तु
पृथग्गुणितयोर्मध्य एकतरो हरेण भक्तः स्यात्तदा परोऽपि हरभक्तस्तावतैव शेषेण न्यूनः
स्यात् कथमन्यथा पृथग्गुणितयोर्योगो हरभक्तः शुध्येत्~। तत्र भाज्यो
हरतुल्यगुणेन गुणितो हरभक्तः शुध्येदेव~। गुणहरयोस्तुल्यत्वात्तत्र भाज्यतुल्या
लब्धिश्च~। अत्र गुणहरयोस्तुल्यत्वाद्भाजकखण्डे एव गुणखण्डे~। तत्रैकखण्डेन भाज्ये
गुणिते हरभक्ते यावच्छेषं तावदेवापरखण्डगुणे भाज्ये न्यूनं स्यात्~। यथा
\newpage
%%%%%%%%%%%%%%%%%%%%%%%%%%%%%%%%%%%%%%
\begin{table}[h!]
    \centering\s
    \begin{tabular}{lr}
        भा &१७\\
ह &१९
    \end{tabular}
\end{table}
\vspace{-2mm}

\noindent हरतुल्यगुण\textendash \,१५\textendash \,गुणितो भाज्यः २५५ हरेण १५ भक्तो लब्धिश्च १७~। अथ 
गुणखण्डाभ्यां १~। १४ पृथग्गुणितः १७~। २३८~। अत्र प्रथमे हरभक्ते शेषं 
२~। अत्र द्वयमधिकमिति तावता क्षेपेण वियोगे निःशेषभजनं भवति लब्धिश्च~। 
अपरखण्डे तु तावति २ क्षिप्ते २४० हरेण भक्ते निःशेषभजनं भवति 
लब्धिश्च १६~। अथवा गुणखण्डाभ्यां २।१३ पृथग्गुणितः ३४।२२१ एको हरभक्तः 
शेषं ४ एतच्छुद्धौ ३० गुणः १२ लब्धिश्च २~। परत्र २२१ तावत्येव ४ 
क्षिप्ते २२५ निःशेषभजनादपरखण्ड\textendash \,१३\textendash \,गुणो लब्धिश्च १५~। अथवा 
गुणखण्डाभ्यां ३।१२ पृथग्गुणितः ५१।२०४~। अत्राद्यः षडूनः परश्च षड्युतः 
शुध्यतीति षट्क्षेपे योगवियोगजौ गुणौ गुणखण्डे एव १२।३ भाज्यखण्डे एव
तल्लब्धी च १४।३~। अत उपपन्नम्~। \hyperref[52]{\textbf{यथागतौ लब्धिगुणौ विशोध्यौ 
स्वतक्षणात्}} इति~। अत एव वक्ष्यति \hyperref[54]{\textbf{योगजे तक्षणाच्छुद्धे गुणाप्ती स्तो
वियोगजे}} इति~। तदेवं \hyperref[51]{\textbf{मिथो भजेत्तौ दृढभाज्यहारौ}} इत्यादिना \hyperref[52]{\textbf{स्वतक्षणाच्छेषमितौ तु तौ स्तः}} इत्यन्तेन गुणलब्धिसाधनमुपपन्नम्~।
स्यादेतत्~। आचार्येण कुट्टकार्थं यदपवर्तनावश्यकत्वमुक्तं तत्कथम्~। अनपवर्ते
तदसिद्धेरिति चेत्~। तथा हि यथापवर्तसम्भवे सत्यपवर्ते कृते परस्परभजने रूपं शेषं 
स्यादस्मिंश्च क्षेपगुणिते क्षेपसमतया वियोगे शुद्धिः स्यादिति~। यथा
क्षेपतुल्यो गुणस्तथानपवर्ते परस्परभजनेऽपवर्ताङ्कमितेऽन्त्यशेषे क्षेपगुणिते
क्षेपतुल्यता न स्यादिति न क्षेपतुल्यो गुणः~। सत्यम्~। तथाप्यन्त्यशेषेण क्षेपे भक्ते 
यल्लभ्यते तावति गुणे क्षेपतुल्यं शेषं स्यादिति~। तस्य गुणत्वे
बाधकाभावात्~। 
न च यत्रान्त्यशेषेण क्षेपो न शुध्यति तत्र कथं गुणः स्यादिति वाच्यम्~। 
तत्र खिलत्वस्य निरूपितत्वादाचार्योक्तत्वाच्च~। न च यथापवर्ते
\hyperref[51]{\textbf{याव-द्विभाज्ये भवतीह रूपम्}} इत्यनुगमः सुवचोऽस्ति~। न तथानपवर्ते 
यावद्विभाज्येऽमुकं भवेदित्यनुगमः सुवचोऽस्तीति~। क्रियावतारो न स्यादिति वाच्यम्~।
यावद्विभाज्ये शून्यं न भवेदित्यनुगमस्य सुवचत्वात्~। अथवा 'यावद्विभाज्ये भवतीह शून्यम्' इति वक्तव्यम्~। अन्त्यहरेण क्षेपे भक्ते यल्लभ्यते
तदन्त्यफलादेशेन निवेश्य तदधः शून्यं निवेश्यमिति च वक्तव्यम्~। यतोऽत्रान्त्यभाज्यः
शून्यमन्त्यहरस्त्वपवर्ताङ्कः~। अत शून्यमेव गुण इति तदधः स्थाप्यम्~।
\newpage
%%%%%%%%%%%%%%%%%%%%%%%%%%%%%%%%%%%%
\noindent शून्यगुणान्त्यलब्धिः \;क्षेपतक्षणलाभाढ्या \,लब्धिरिति \;सा \,लब्धिस्थाने \;स्थाप्येति \,युक्तं भवति~। न च लाघवार्थमपवर्त इति वाच्यम्~। 
अनपवर्तितयोरपवर्तितयोश्च हरभाज्ययोः परस्परभजने लब्धिसाम्यात्~।
अपवर्तितयोर्लघुत्वाल्लाघवमिति चेन्न~। 
अनपवर्तितयोः पर-स्परभजनस्यापवर्ताङ्कज्ञानार्थमावश्यकतया
प्रत्युतापवर्तितयोः परस्परभजनयोर्गौरवात्~। 
न च सकलगुणलाभार्थमपवर्तनावश्यकत्वम्~। तथा हि\textendash \,व्यस्तविधिना लब्धिगुणसिद्धौ \hyperref[51]{\textbf{ऊर्ध्वो विभाज्येन दृढेन तष्टः फलं गुणः
स्यादपरो हरेण}} इत्यनेन भवति लघुर्गुणो लब्धिश्च~। अनपवर्तिताभ्यां तक्षणे
तद्द्वयं न स्यात्~। \hyperref[59]{\textbf{इष्टाहतस्वस्वहरेण युक्ते}} इत्यत्र गुणेनेष्टाहतहरो
लब्धाविष्टाहतभाज्यश्च क्षेपावुक्तौ~। तत्रानपवर्तितहरतुल्ये तादृशभाज्यतुल्ये च
क्रमेण गुणहरयोः क्षेपेऽवान्तरगुणलब्ध्यवगमश्च न स्यादिति वाच्यम्~।
भवत्वपवर्तितयोस्तक्षणत्वं क्षेपत्वं च~। तथापि गुणलब्ध्योः प्रागेव सिद्धतया \hyperref[51]{\textbf{मिथो भजेत्तौ दृढभाज्यहारौ}} इति कुट्टकार्थमपवर्तानावश्यकत्वात्~। न च नोक्तौ
वापवर्तावश्यकतेति वाच्यम्~। \hyperref[50]{\textbf{भाज्यो हारः क्षेपकश्चापवर्त्यः केनाप्यादौ सम्भवे
कुट्टकार्थम्}} इत्यत्र {\qt समेन केनाप्यपवर्त्य हारभाज्यौ भजेद्वा} इत्यत्रेव {\qt मिथो हराभ्यामपवर्तिताभ्यां यद्वा} इत्यत्रेव च वाकारश्रवणात्~। \hyperref[51]{\textbf{यावद्विभाज्ये भवतीह रूपम्}} इति रूपशेष एव कुट्टकविधानाच्च~। किं च भाज्यशेषेण क्षेपे निःशेषभक्ते या 
लब्धिः सा वियोगे गुण इत्यस्य क्षेपे परस्परभजनं सर्वत्र 
नावश्यकमित्यस्ति लाघवम्~। तथा हि\textendash \,
\vspace{-2mm}

\begin{table}[h!]
    \centering\s
    \begin{tabular}{l}
        भा २१ क्षे १\\
~ह १३
    \end{tabular}
\end{table}
 \vspace{-2mm}

\noindent अत्र भाज्ये हरेण भक्ते शेषं ८~। अनेन क्षेपे १६ भक्ते लब्धिजातो 
वियोगजो गुणः २~। {\qt गुणगुणिता भाज्यलब्धिर्लब्धिश्चे}ति जाता लब्धिः २~। 
आचार्योक्तप्रकारे तु \hyperref[51]{\textbf{मिथो भजेत्तौ}} इत्यादिना वल्लीयं
\vspace{-2mm}

\begin{table}[h!]
    \centering\s
    \begin{tabular}{l}
     १\\
 १\\
 १
       \end{tabular}
\end{table}
\newpage
%%%%%%%%%%%%%%%%%%%%%%%%%%%%%%%%%%
\begin{table}[h!]
    \centering\s
    \begin{tabular}{r}
१\\
१\\
१६\\
०
   \end{tabular}
\end{table}
\vspace{-2mm}

\noindent \hyperref[51]{\textbf{उपान्तिमेन स्वोर्ध्वे हतेऽन्त्येन युत}} इत्यादिना जातं राशिद्वयं
\vspace{-2mm}

\begin{table}[h!]
    \centering\s
    \begin{tabular}{r}
       १२८\\
८०
       \end{tabular}
\end{table}
\vspace{-2mm}

\noindent \hyperref[51]{\textbf{ऊर्ध्वो विभाज्येन दृढेन तष्टः}} इत्यादिना जातौ लब्धिगुणौ तावेव २~। २ अथवा
\vspace{-2mm}

\begin{table}[h!]
    \centering\s
    \begin{tabular}{l}
    भा २१ क्षे १५ \\
~ह १३
       \end{tabular}
\end{table}
\vspace{-2mm}

\noindent अत्र भाज्यशेषेण ८ भक्तः क्षेपो न शुध्यत्यतो भाज्यशेषेण भक्तो हरः~। एवं
जातं लब्धिद्वयं १~। १ द्वितीयशेषं च ५~। अनेन भक्तः क्षेपः शुध्यतीति
लब्धेर्गुणं ३ अन्ते तदधः शून्यं च निवेश्य जाता वल्ली
\vspace{-2mm}

\begin{table}[h!]
    \centering\s
    \begin{tabular}{r}
    १\\
१\\
३\\
०
      \end{tabular}
\end{table}
\vspace{-2mm}

\noindent \hyperref[51]{\textbf{उपान्तिमेन स्वोर्ध्वे हते}} इत्यादिना जातं राशिद्वयं
\vspace{-2mm}

\begin{table}[h!]
    \centering\s
    \begin{tabular}{l}
       ल ६\\
गु ३
    \end{tabular}
\end{table}
\vspace{-2mm}

\noindent लब्धिसमत्वाज्जातौ योगजौ लब्धिगुणावस्मत्पक्षे~। आचार्यप्रकारे तु वल्ली
\newpage
%%%%%%%%%%%%%%%%%%%%%%%%%%%
\begin{table}[h!]
    \centering\s
    \begin{tabular}{r}
    १\\
१\\
१\\
१\\
१\\
१५\\
०
     \end{tabular}
\end{table}
\vspace{-2mm}

\noindent उक्तवज्जातं राशिद्वयं $\begin{matrix}
\vspace{-1mm}
\mbox{{१२०}}\\
\vspace{-1mm}
\mbox{{७५}}
\vspace{1mm}
\end{matrix}$ तक्षणे जातं $\begin{matrix}
\vspace{-1mm}
\mbox{{१५}}\\
\vspace{-1mm}
\mbox{{१०}}
\vspace{1mm}
\end{matrix}$ लब्धिविषमत्वात् स्वतक्षणाच्छोधने जातौ लब्धिगुणौ योगजा तावेव ६~। ३~। एवमस्मत्पक्षेऽस्ति लाघवम्~। तदेवमपवर्तावश्यकत्वे गौरवमेवेति प्रतिभाति~।
अत्रोच्यते प्रकारान्तरेणापवर्ताङ्कोपस्थितौ तेनापवर्ते कृते
भाज्यभाजकयोर्लघुत्वादस्त्येव कुट्टके लाघवम्~। किं चाविदुषामाचार्योक्तप्रकारे 
यथास्ति गणितसौकर्यं न तथान्यप्रकारे~। अन्यप्रकारे ह्यनपवर्तितयोर्भाज्यहरयोः 
परस्परभजनादिना गुणलब्धिसाधनमपवर्तितयोस्तु तक्षणत्वं क्षेपत्वं चेत्यनुसन्धानेऽस्ति 
गौरवम्~। किं च नायमारम्भो लौकिकगणितफलकः~। किं तु ग्रहगणितफलकः~। 
तत्र हि विकलाशेषाद्ग्रहानयने विकलाशेषं शुद्धिः~। षष्टिर्भाज्यः कुदिनानि हार इति 
प्रकल्प्य या लब्धिस्ता विकला यो गुणस्तत्कलाशेषमित्यादिरस्ति प्रकारः~। 
वक्ष्यति च~। \hyperref[67]{\textbf{कल्प्याथ शुद्धिर्विकलावशेषं षष्टिश्च भाज्यः कुदिनानि
हारः~। तज्जं फलं स्युर्विकला गुणस्तु लिप्ताग्रमस्माच्च कलालवाग्रम्~। एवं
तदूर्ध्वं च}} इति~। तत्रर्णक्षेपस्य विकलाद्यग्रस्यानियतत्वात् प्रति-प्रश्न ततस्ततो
विकलाद्यग्रात्कुट्टकप्रकरणेऽस्ति भूयान्प्रयासः~। अतः सुखार्थं स्थिरकुट्टको वक्ष्यते~।
\hyperref[66]{\textbf{क्षेपं विशुद्धिं परिकल्प्य रूपं पृथक्तयोर्ये गुणकारलब्धी~। अभीप्सितक्षेपविशुद्धिनिध्ने
स्वहारतष्टे  भवतस्तयोस्ते}} इति~। एतादृशः स्थिरकुट्टकस्त्वपवर्त एव सम्भवति~। 
अनपवर्ते रूपक्षेपस्याभावात्~। यद्यप्यनपवर्तेऽप्यपवर्ताङ्कतुल्यक्षेपेण सम्भवति
स्थिरकुट्टकस्तथापि यद्यप्यपवर्ताङ्कक्षेप एते गुणाप्ती तर्ह्यभीष्टक्षेपे
क इति त्रैराशिकेऽपवर्ताङ्को हारः स्यात्~। रूपक्षेपात्त्रैराशिके तु गुणनमात्रमित्यस्ति लाघवम्~।
यद्वा सुधियः
\newpage
%%%%%%%%%%%%%%%%%%%%%%%%%%%%%%%%%%%
\noindent साधयन्तु यथाकथञ्चित्~।
अज्ञानुग्राहकैराचार्यैरवधानलाघवायापवर्तावश्यकत्वमुक्तमिति न कोऽपि दोष इत्यलं पल्लवितेन~॥~५२~॥\\

\vspace{-4mm}
तदेवं भाज्यहराक्षेपाणामपवर्तसम्भवेऽपवर्तं कृत्वैव कुट्टकः कार्यो
भाज्यहारयोरेवापवर्तसम्भवे खिलत्वं चेति प्रतिपादितम्~। अथ क्षेपभाज्ययोरेव
क्षेपभाजकयोरेव 
वापवर्तसम्भवे किं कार्यं तदाह\textendash

\phantomsection \label{53}
\begin{quote}
    \ab 
    भवति कुट्टविधेर्युतिभाज्ययोः समपवर्तितयोरपि वा गुणः~। \\
 भवति यो युतिभाजकयोः पुनः स च भवेदपवर्तनसङ्गुणः~॥~५३~॥~
\end{quote}
 
\hyperref[53]{\textbf{युतिः}} क्षेपः~। \hyperref[53]{\textbf{युतिभाज्ययोः समपवर्तितयोः}} सतोरपि \hyperref[51]{\textbf{मिथो भजेत्तौ 
दृढभाज्यहारौ}} इति यथोक्तात्कुट्टकविधेर्वा \hyperref[53]{\textbf{गुणः}} स्यात्~। \hyperref[53]{\textbf{अपिः}}
समुच्चये~। \hyperref[53]{\textbf{वा}}
प्रकारान्तरे~। क्षेपभाज्ययोरपवर्तनसम्भवेऽप्यपवर्तनमकृत्वापि गुणः
सिध्यति~। यद्वा 
तयोरपवर्तितयोः सतोरपि यथोक्तकुट्टकविधिना स एव गुणः स्यादित्यर्थः~। तेन
गुणेन भाज्यं सङ्गुण्य क्षेपेण संयोज्य हरेण विभज्य लब्धिरत्र ज्ञेया~। 
\hyperref[53]{\textbf{भवति}} य इति पुनर्विशेषे~। \hyperref[53]{\textbf{युतिभाजकयो}}स्त्वपर्तनसम्भवे
सत्य\hyperref[53]{\textbf{पवर्तितयोः}} सतोर्यथोक्तकुट्टकविधिना यो गुणो भवति स च भवेत्~। परमपवर्तनसङ्गुणः 
सन्ननपवर्तितयोरपि गुणसिद्धिर्भवति चकारात्~। यद्वा
अपिवाशब्दसामर्थ्यात् अध्याहारेण 
योजना~। सा यथा\textendash \,युतिभाज्ययोः समपवर्तितयोर्या लब्धिर्भवति~। 
अपि वा युतिभाजकयोस्त्वपवर्तितयोर्यो गुणो भवति सा लब्धिः~। स च गुणोऽपवर्तनसङ्गुणः
सन् भवेत्~। लिङ्गविपरिणामेन लब्धिरपवर्तनसङ्गुणा सती भवेदिति योज्यम्~। 
युतिभाज्ययोः समपवर्तितयोर्लब्धिरपवर्ताङ्केन गुण्या~। गुणकस्तु यथागत एव~।
युतिभाजकयोस्त्वपवर्तितयोर्गुणोऽपवर्ताङ्केन गुण्यः~। लब्धिर्यथागता
वेत्यर्थः~। 
अत्र यद्वेत्यादिना व्याख्यातोऽर्थो युक्ततरोऽस्ति~। परं न तथायं शब्दलभ्यः~।
आचार्याणाम् अपि नायमर्थोऽभिप्रेतः किं तु प्रथमः~। यतः \hyperref[61]{\textbf{शतं हतं येन युतं नवत्या}} इत्युदाहरणे ते वक्ष्यन्ति~। अत्र लब्धिर्न ग्राह्येति~। गुणघ्नभाज्ये  क्षेत्रयुते हरभक्ते लब्धिश्चेति च गुणनभजनाल्लब्धिश्चेति च~।
\newpage 
%%%%%%%%%%%%%%%%%%%%%%%%%%%%%%%%%%%%%%%%
अत्रोपपत्तिः~। येभ्यो भाज्यहारक्षेपेभ्यो \hyperref[51]{\textbf{मिथो भजेत्तौ दृढभाज्यहारौ}}
इत्यादिना  ये गुणाप्ती स्यातां तेषु भाज्यादिषु ते गुणाप्ती
पूर्वोक्तयुक्क्त्योपपन्ने एव~। अपि 
च भाज्यभाजकयोर्यथास्थितयोः केनाप्येकेन गुणितयोर्भक्तयोर्वा नास्ति फले
भेद  इति तु प्रसिद्धतरम्~। प्रकृते तु भाज्यस्य खण्डद्वयम्~। गुणगुणितः
कल्पितभाज्य  एकं क्षेपोऽपरम्~। हर एव हरः~। एषु त्रिष्वेकस्यापि गुणनेऽभीष्टे
त्रयाणामपि गुणनमावश्यकम्~। उक्तयुक्तेरेव~। तत्र गुणगुणितकल्पितभाज्यस्य गुणने
प्रकारत्रयं  सम्भवति~। गुणम् एवादौ सङ्गुण्य तादृशेन गुणेन कल्पितभाज्यो गुण्य इत्येकः 
प्रकारः~। कल्पितभाज्यमेवादौ सङ्गुण्य पश्चाद्यथास्थितेन गुणकेन तं
गुणयेदिति द्वितीयः~। गुणगुणितं कल्पितभाज्यं गुणयेदिति तृतीयः प्रकारः~। अथ 
भाज्यादित्रयमपवर्त्य कुट्टकेन येन गुणाप्ती साधिते ते अपवर्तितेष्वेव
भाज्यादिषु युक्ते अपेक्षिते तत्तूद्दिष्टभाज्यादिषु~।
अतोऽपवर्तितभाज्यादिकम् अपवर्ताङ्केन गुणयेत्तदुद्दिष्टभाज्यादिकं भवति~। येभ्यः कुट्टकः कृतस्तेषु गुणितेषु भक्तेषु
वा फलभेदो नास्तीति जाते ते एव गुणाप्ती उद्दिष्टभाज्यादिष्वपीति~। अथ यत्र 
भाज्यक्षेपावेवापवर्तितौ न हरस्तत्रापि तदुत्थे गुणाप्ती तेषु युक्ते एव~।
अपेक्षिते तूद्दिष्टभाज्यादिषु~। तत्र हरस्तूद्दिष्ट एवास्ति~। भाज्यक्षेपौ
त्वपवर्ताङ्कगुणितावुद्दिष्टौ भवतः~। परं हरोऽप्यपवर्ताङ्केन गुण्यः~। भाज्यस्य
गुणितत्वात्~। 
गुणिते च हरे न स्यादुद्दिष्टहरः~। तथा सत्युद्दिष्टभाज्यक्षेपयोरेव
गुणाप्तिसिद्धिर्नोद्दिष्टहरे~। अतोऽत्र हरो न गुणनीयः~। परं भाज्यशकलयोर्गुणनेन
भाज्यमात्रस्य गुणनाल्लब्धिरपि प्रकृतेऽपवर्ताङ्कगुणिता सती भवेत्~। अत उक्तं
युतिभाज्ययोः समपवर्तितयोर्या लब्धिः सापवर्तसङ्गुणा गुणस्तु यथागत एवेति~।
अथ यत्र भाजकक्षेपावेवापवर्त्य कुट्टकः कृतस्तत्रापि ये सिद्धे गुणलब्धी ते तेष्वेव
भवतः~। 
अपेक्षिते तूद्दिष्टभाज्यादिषु~। प्रकृते ते कल्पितभाज्यस्तूद्दिष्ट
एवास्ति~। हरक्षेपौ 
त्वपवर्ताङ्केन गुणितावुद्दिष्टौ भवतः~। परं क्षेपलक्षणभाज्यखण्डस्य
गुणितत्वादपरमपि भाज्यखण्डं गुण-नीयम्~। परखण्डं च गुणगुणितः कल्पितभाज्यः~। 
अतोऽसावपवर्ताङ्केन गुण्यः~। अस्य गुणनं तु त्रेधा सम्भवतीत्युक्तम्~।
तत्र कल्पितभाज्यस्य गुणने उद्दिष्टकल्पितभाज्यो न स्यात् अतो गुण एव गुणनार्हो
भवति~। अत उक्तम्\textendash \,\hyperref[53]{\textbf{भवति यो युतिभाजकयोः पुनः स च भवेदपवर्तनसङ्गुणः}} इति~। अथ यत्र क्षेपमात्रमपवर्त्य कृट्टकः क्रियते तस्मिन्क्षेपे
ते गुणाप्ती युक्ते~। अथ स क्षेपस्तेनापवर्ताङ्केन गुणितः सन्नुद्दिष्टक्षेपो
भवति~।
\newpage
%5555%%%%%%%%%%%%%%%%%%%%%%%%%%%%%%%%%%
\noindent परं भाज्यखण्डस्य गुणितत्वादपरं भाज्यखण्डं गुणनीयम्~। हरोऽपि गुणनीयः 
गुणिते च गुणे भाज्यखण्डमपि गुणितं भवतीति गुणकोऽपवर्ताङ्केन गुण्यः~। 
एवं जातं भाज्यखण्डयोः गुणनम्~। हरस्य गुणने तु नोद्दिष्टहरसिद्धिरिति 
भाज्यमात्रस्य गुणनाल्लब्धिरपवर्ताङ्कगुणिता स्यात्~। अतः
क्षेपमात्रस्यापवर्तने ये गुणलब्धी तयोरपवर्ताङ्कगुणने सत्युद्दिष्टगुणाप्ति-सिद्धिः~।
अपवर्ताङ्कश्चात्रोद्दिष्टक्षेपतुल्यः~। स्वेन स्वस्य सदापवर्तनसम्भवात्~। अतोऽपवर्तितक्षेपोऽपि
रूपमेव~। अनयैवोपपत्त्या वक्ष्यति\textendash \,\hyperref[66]{\textbf{क्षेपं विशुद्धिं परिकल्प्य रूपं पृथक्तयोर्ये
गुणकारलब्धी~। अभीप्सितक्षेपविशुद्धिनिघ्ने स्वहारतष्टे भवतस्तयोस्ते}} इति~। अथ यत्र हारभाज्यावेवापवर्त्य \,कुट्टकः \,क्रियते \,तत्र \,सिद्धे \,ये गुणाप्ती ते \,अपवर्तितयोरेव युक्ते~। उद्दिष्टसिद्ध्यर्थं त्वपवर्ताङ्केन गुणने क्षेपगुणस्याप्यावश्यकतया
नोद्दिष्टक्षेपसिद्धिरत  एव तत्र खिलत्वमुक्तम्~। अत एव त्रयाणाम् अपवर्तनसम्भवेऽपि यदि हरभाज्यावेवापवर्त्य लब्धिगुणौ साध्येते तदा नोद्दिष्टसिद्धिः~। अत एव भाज्यमात्रस्य
वापवर्तनेन सिद्धाभ्यां लब्धिगुणाभ्यां नोद्दिष्टसिद्धिरित्यादि
सुधीभिरूह्यम्~। \\

 \vspace{-4mm}
 अथ ऋणक्षेपे ऋणभाज्ये वा सति विशेषमनुष्टुभाह\textendash
 
\phantomsection \label{54}
\begin{quote}
    \ab 
     योगजे तक्षणाच्छुद्धे गुणाप्ती स्तो वियोगजे~। \\
 धनभाज्योद्भवे तद्वद्भवेतामृणभाज्यजे~॥~५४~॥~
\end{quote}
 
\hyperref[54]{\textbf{योगजे}} धनक्षेपजे ये गुणाप्ती ते स्वतक्षणाच्छुद्धे वियोगजे भवतः~। गुणो दृढहरात् शुद्धः 
संल्लब्धिर्दृढभाज्याच्छुद्धा सती ऋणक्षेपे भवतीत्यर्थः~। एवं
धनभाज्योद्भवे गुणाप्ती तद्वत्स्वतक्षणाच्छुद्धे ऋणभाज्यजे भवतः~। अत्रोत्तरार्धे 'ऋणभाज्योद्भवे तद्वद्भवेतामृणभाजके' 
इत्यपि पाठः क्वचिद्दृश्यते~। अस्यार्थः\textendash \,योगजे गुणाप्ती
स्वतक्षणाच्छुध्ये वियोगजे 
भवतः~। तद्वदृणभाज्योद्भवे भवतः तद्वदृणभाजकेऽपि गुणाप्ती भवतः~।
क्षेपभाज्यहाराणामन्यतमे ऋणे सति पूर्वसिद्धे गुणाप्ती स्वतक्षणाच्छोध्ये इत्यर्थः~।
एवं त्रयाणामप्यृणत्वे त्रिवारं स्वतक्षणाच्छोध्ये इत्यर्थः~। अयम् अपपाठः~। नहि
भाजकस्य धनत्वे ऋणत्वे वास्ति कश्चिदङ्कतो विशेषो येनोपायान्तरमारभ्येत~। किन्तु
धनर्णताव्यत्यासमात्रं लब्धेः~। भाज्यस्य तु धनत्वे ऋणत्वे च क्षेपयोगे
क्रियमाणेऽस्त्यङ्कतोऽपि 
विशेष इति तस्यर्णत्व उपायान्तरमारम्भणीयमेव~। आचार्यस्याप्यनभिमत एवायं
\newpage %%%%%%%%%%%%%%%%%%%%%%%%%%%%%%%%%%%%%%%%

\noindent पाठः~। यतो \hyperref[63]{\textbf{अष्टादश गुणाः केन दशाढ्या वा दशोनिताः~। शुद्धं भागं प्रयच्छन्ति क्षयगैकादशोद्धृताः}} $\begin{matrix}
\vspace{-1mm}
\mbox{{भा १८ क्षे १०}}\\
\vspace{-1mm}
\mbox{{ह १ं१ ~~~~~}}
\vspace{1mm}
\end{matrix}$~। अत्र भाजकस्य धनत्वे कृते गुणल्लब्धी ८~। १४~। ऋणेऽपि भाजके एवं किन्तु लब्धिर्ऋणगता कल्प्या
भाजकस्यर्णत्वात्~। ८~। १ं४ इति वक्ष्यति~।
अस्मिन्पाठेऽर्थाशुद्धिरप्युदाहरणविवरणावसरे प्रतिपादयिष्यते~। 
वस्तुतस्तूत्तरार्धमनपेक्षितमेव~। पूर्वार्धेनैव गतार्थत्वात्~।
तथाहि\textendash\ योगजे गुणाप्ती
वियोगजे भवत इति हि तदर्थः~। तत्र भाज्यक्षेपयोर्धनत्वे ऋणत्वे वा ये
गुणाप्ती ते योगजे~। यत उभयोर्धनत्वे ऋणत्वे वा \hyperref[3]{\textbf{योगे युतिः स्यात् क्षययोः स्वयोर्वां}} इति नास्ति कश्चिदङ्कतो विशेषः~। यदा
पुनर्भाज्यक्षेपयोरन्यतरस्यर्णत्वं तदा \hyperref[3]{\textbf{धनर्णयोरन्तरमेव योगः}} इत्युक्तत्वादन्तरे क्रियमाणे
भवत्यङ्कतोऽपि विशेष इति तदर्थमुपायान्तरमारम्भणीयम्~। तदर्थमुक्तं 
{\qt 'स्वतक्षणाच्छुद्धे वियोगजे भवतः'} इति~। अस्मात्पूर्वार्धादतिरिक्तः को
वार्थ उत्तरार्धेन प्रतिपाद्यते येन तदपेक्षितं स्यात्~। अयमर्थो \hyperref[62]{\textbf{यद्गुणाक्षयगषष्टिरन्विता}} इत्युदाहरणे \hyperref[54]{\textbf{धनभाज्योद्भवे तद्वद्भवेतामृणभाज्यजे}} इति 
मन्दावबोधनार्थं मयोक्तम्~। अन्यथा योगजे
तक्षणाच्छुद्धेरित्यादिनैव तत्सिद्धेरिति 
वदताचार्येणैव प्रतिपादयिष्यते~। तस्मात्सिद्धान्तान्तर्गतबीजमूलसूत्रे
पूर्वार्धमात्रम्~। द्वितीयमर्धं तु तद्विवरणरूपेऽस्मिन्बीजगणिते
बालावबोधार्थमुक्तमतस्तत्पृथग्गणनां 
नार्हति~। अतः कुट्टकसूत्रेष्वनुष्टुभां चतुष्टयमेव न सार्धं तत्~।
अनुष्टुप्त्रयमेका च गाथेति कल्पनस्यान्यायत्वात्~। अनुपपत्तेरभावादित्यलं पल्लवितेन~। \\

\vspace{-4mm}
 सूत्रोपपत्तिस्तु \hyperref[52]{\textbf{यथागतौ लब्धिगुणौ विशोध्यौ स्वतक्षणाच्छेषमितौ तु तौ स्तः}} इत्यस्योपपत्तिनिरूपणावसर एव निरूपिता~। अथ क्षेपे हरमात्राद्भाज्यमात्राद्वा हरभाज्याभ्यां वान्यूने क्वचिद्विशेषमुत्तरार्धेनाह\textendash

\phantomsection \label{55}
\begin{quote}
    \ab 
     गुणलब्ध्योः समं ग्राह्यं धीमता तक्षणे फलम्~॥~५५~॥~
\end{quote}
 
 \hyperref[51]{\textbf{ऊर्ध्वो विभाज्येन दृढेन तष्टः फलं गुणः स्यादपरो हरेण}} इत्यत्र 
गुणलब्धिसम्बन्धिनि तक्षणे क्रियमाणे सत्युभयत्र तक्षणस्य फलं तुल्यमेव
\newpage
%%%%%%%%%%%%%%%%%%%%%%%%%%%%%%%%%%%%%%%%%%
\noindent ग्राह्यम्~। केन धीमता बुद्धिमता~। हेतुगर्भम् इदम्~। तथाहि\textendash \,उभयत्र तक्षणे क्रियमाणे यत्राल्पं तक्षणफलं लभ्यते तत्तुल्यम् एवान्यत्रापि ग्राह्यम्~। नन्वधिकं प्राप्तमप्यस्योपपत्तिः \hyperref[51]{\textbf{ऊर्ध्वो विभाज्येन दृढेन तष्टः फलं गुणः स्यादपरो हरेण}} इत्यस्य युक्तिनिरूपणे निरूपिता~। अत्र पुस्तकेषु \hyperref[55]{\textbf{गुणलब्ध्योः समं ग्राह्यम्}} इत्यादिश्लोकार्धस्य \hyperref[54]{\textbf{योगजे तक्षणाच्छुद्धे}} इत्यतः प्राक् पाठो दृश्यते~। स तु 
लेखकदोषज इति प्रतिभाति~। पुस्तकपाठक्रमस्वीकारे तु \hyperref[55]{\textbf{गुणलब्ध्योः समं ग्राह्यम्}} इत्यत्र प्रकारान्तरार्थं प्रवृत्तस्य \hyperref[56]{\textbf{हरतष्टे धनक्षेपे}} इत्येतस्य सूत्रस्य 
व्यवधानं स्यात्~। उदाहरणक्रमविरोधश्च स्यात्~। लीलावतीपुस्तकेषु
पुनरस्मल्लिखितक्रम एवास्ति~। युक्तश्चायमिति प्रतिभाति~। \\

\vspace{-4mm}
 अथात्र गुणलब्ध्योस्तक्षणे फलयोरतुल्यता यथा न भवति तथा प्रकारान्तरमनुष्टुभाह\textendash

\phantomsection \label{56}
\begin{quote}
    \ab 
     हरतष्टे धनक्षेपे गुणलब्धी तु पूर्ववत्~। \\
 क्षेपतक्षणलाभाढ्या लब्धिः शुद्धौ तु वर्जिता~॥~५६~॥~
\end{quote}
 
 यत्र क्षेत्रो हरादधिकस्तत्र हरेण क्षेपस्तक्ष्यः~। तष्टक्षेपमेव क्षेपं
प्रकल्प्य पूर्ववद्गुणलब्धी साध्ये~। तत्र गुणो यथागत एव लब्धिस्तु क्षेपतक्षणलाभाढ्या कार्या~। क्षेपस्य तक्षणमवशेषणं तत्र यो लाभः फलं तेनाढ्या 
युक्ता~। एवं धनक्षेपे शुध्दौ ऋणक्षेपे तु हरतष्टे कृते सति पूर्ववद्योगजे
तक्षणाच्छुद्धे गुणाप्ती स्तो वियोगजे इत्युक्तप्रकारेण ये गुणाप्ती
स्तस्तत्र लब्धिः क्षेपतक्षणलाभेन वर्जिता कार्या~। यदा तु भाज्यादन्यूने
हरान्न्यूने क्षेपे गुणलब्ध्योस्तक्षणे क्वचित्फलवैषम्यं स्यात्तत्रैकस्य
सूत्रस्याप्रवृत्तेः \hyperref[55]{\textbf{गुणलब्ध्योः समं ग्राह्यम्}} इत्यादिनैव तक्षणफलं ग्राह्यम्~। यथा $\begin{matrix}
\vspace{-1mm}
\mbox{{भा ३ क्षे ३}}\\
\vspace{-1mm}
\mbox{{~~~~~ ह ४}}
\vspace{1mm}
\end{matrix}$ अत्रोक्तवज्जातं राशिद्वयं $\begin{matrix}
\vspace{-1mm}
\mbox{{ल ३}}\\
\vspace{-1mm}
\mbox{{गु ३}}
\vspace{1mm}
\end{matrix}$ अत्र गुणतक्षणे किञ्चिन्न लभ्यते~। लब्धितक्षणे त्वेकः प्राप्यते स न ग्राह्यः~। 
एवं क्षेपस्य हरेण तक्षणेऽपि भाज्यादन्यूनतया यदि क्वचित्फलवैषम्यं 
स्यात्तत्रापि गुणलब्ध्योः समं ग्राह्यमित्यादिनैव तक्षणफलं ग्राह्यम्~। यथात्र~। $\begin{matrix}
\vspace{-1mm}
\mbox{{भा ३ क्षे ७}}\\
\vspace{-1mm}
\mbox{{ह ४ ~~~~}}
\vspace{1mm}
\end{matrix}$ एतादृशस्थले फलयोर्यथा
\newpage %%%%%%%%%%%%%%%%%%%%%%%%%%%%%%%%%%%%%%%%

\noindent वैषम्यं न भवति तथा प्रकारान्तरं न दृश्यते~। अत्रोपपत्तिः~। 
क्षेपस्यात्र खण्डद्वयं कृतम्~। एकादिगुणहरतुल्यमेकं शेषम् अपरम्~। तत्र
शेषमिते क्षेपे यः साधितो गुणस्तेन गुणेन भाज्ये गुणिते तेन क्षेपेण युते हरेण 
भक्ते च शेषं न स्यात्~। अथोद्दिष्टक्षेपार्थमपरखण्डमपि योज्यम्~। तेनापि 
युक्ते तस्मिन्भाज्ये हरभक्ते शेषं नैव स्यात्~।
तस्यैकादिगुणहरतुल्यत्वात्~। 
किन्तु हरेण तस्मिन्क्षेपखण्डे भक्ते यल्लभ्यते तावल्लब्धावधिकं स्यात्~।
एवमृणक्षेपे तावदेव न्यूनं स्यादित्युपपन्नम्~। \\

\vspace{-4mm}
अथ भाज्येऽपि हरादधिके विशेषमाहानुष्टुभा\textendash

\phantomsection \label{57}
\begin{quote}
    \ab 
    अथवा भागहारेण तष्टयोः क्षेपभाज्ययोः~। \\
 गुणः प्राग्वत्ततो लब्धिर्भाज्याद्धतयुतोद्धृतात्~॥~५७~॥~
\end{quote}
 
 यत्र भाज्यक्षेपौ हरादधिकौ तत्र पूर्ववद्वा क्षेपमात्रतक्षणेन वा
गुणाप्ती साध्ये~। अथवा भाज्यक्षेपौ द्वावपि हरेण तक्ष्यौ~। तष्टयोः क्षेपभाज्ययोः 
प्राग्वदेव गुणाप्ती साध्ये~। तत्र गुण एव ग्राह्यो न लब्धिः~। कथं तर्हि
लब्धिर्ज्ञयेति~। तदाह\textendash \,\hyperref[57]{\textbf{भाज्याद्धतयुतोद्धृतात्}} इति~। हतश्चासौ
युतश्च हतयुतः, स  चासावुद्धृतश्चेति हतयुतोद्धृतस्तस्मात्~। गुणेन गुणितात्क्षेपेण
युताद्भाजकेन भक्तादुद्दिष्टाद्भाज्याद्या लब्धिर्भवति सा ज्ञेयेत्यर्थः~। अस्त्यत्र
लब्धिज्ञाने प्रकारान्तरमपि~। 
तथाहि\textendash\  भाज्यतक्षणलाभो गुणेन गुणनीयः~। पश्चात्क्षेपतक्षणलाभेन संस्कार्यः~। 
संस्कृतेन तेन गणितागता लब्धिः संस्कार्या सा लब्धिर्भवतीति~। गौरवादिदमुपेक्षितमाचार्यैः~। अत्रोपपत्तिः~। यया क्षेपस्य खण्डद्वयं कृत्वा
पूर्वमुपपत्तिः प्रदर्शिता तथात्र भाजस्यापि खण्डद्वयेनोपपत्तिर्ज्ञेया~। \\

\vspace{-4mm}
 अथ क्षेपाभाव एकादिगुणहरसमे वा क्षेपे विशेषमनुष्टुभाह\textendash

\phantomsection \label{58}
 \begin{quote}
     \ab 
      क्षेपाभावोऽथवा यत्र क्षेपः शुध्येद्धरोद्धृतः~। \\
 ज्ञेयः शून्यं गुणस्तत्र क्षेपो हरहृतः फलम्~॥~५८~॥~
 \end{quote}
\newpage
%%%%%%%%%%%%%%%%%%%%%%%%%%%%%%%%%%%%%%
 स्पष्टोऽर्थः~। उपपत्तिरपि कुट्टकोपपत्तिप्रारम्भ एवोक्ता~। \\
\vspace{-4mm}

अथ गुणलब्ध्योरनेकत्वमुपजातिकापूर्वार्धेनाह\textendash
 
\phantomsection \label{59}
\begin{quote}
    \ab 
    इष्टाहतस्वस्वहरेण युक्ते ते वा भवेतां बहुधा गुणाप्ती~॥~५९~॥~
\end{quote}

 स्वस्य स्वस्य हरः स्वस्वहरः~। इष्टेनाहतश्चासौ
स्वस्वहरश्चेष्टाहतस्वस्वहरः~। तेन  युक्ते गुणाप्ती बहुधा भवेताम्~। इष्टे गुणितं
हरं गुणे प्रक्षिपेत्तेनैवेष्टेन गुणितं भाज्यं लब्धौ च प्रक्षिपेत्~। एवमेते
गुणाप्ती इष्टवशाद्भवत इत्यर्थः~। अन्योपपत्तिः \hyperref[51]{\textbf{मिथो भजेत्तौ दृढभाज्यहारौ}} 
इत्यस्योपपत्तिकथनोपक्रम एव प्रदर्शिता~। \\

\vspace{-4mm}
 अथोक्तसूत्राणां क्रमेणोदाहरणानि शिष्यबोधार्थं निरूपयति~। \\

\vspace{-4mm}
तेषु यत्र त्रयाणामप्यपवर्तः सम्भवति लब्धयश्च समास्तादृशमुदाहरणं
रथोद्धतया तावदाह\textendash
\begin{quote}
    \eg 
    एकविंशतियुतं शतद्वयं यद्गुणं गणक पञ्चषष्टियुक्~।\\
 पञ्चवर्जितशतद्वयोद्धृतं शुद्धिमेति गुणकं वदाशु तम्~॥~६०~॥

\end{quote}
 
\noindent स्पष्टोऽर्थः~। न्यासः $\begin{matrix}
\vspace{-1mm}
\mbox{{भा २२१ क्षे ६५}}\\
\vspace{-1mm}
\mbox{{~~~~~~ ह १९५}}
\vspace{1mm}
\end{matrix}$ अत्रापवर्ताङ्कज्ञानार्थं भाज्ये
२२१ हरेण १९५ भक्ते शेषं २६ अनेन पुनर्हरे भक्ते शेषं १३~। अनेनापि 
पुनः पूर्वशेषे २६ भक्ते शेषाभावः~। अतः परस्परं भाजितयोरन्त्यशेषमिदं १३~।
इदमेव तयोरपवर्तनम्~। अनेन तौ निःशेषं भज्येते एव~। अनेनापवर्तिता
भाज्यहारक्षेपा जाता दृढाः $\begin{matrix}
\vspace{-1mm}
\mbox{{भा १७ क्षे ५}}\\
\vspace{-1mm}
\mbox{{ह १५ ~~~~}}
\vspace{1mm}
\end{matrix}$ अनयोर्दृढभाज्यहारयोः परस्परं
भक्तयोर्लब्धमधोऽधस्तदधः 
क्षेपस्तदधः शून्यं निवेश्यमिति जाता वल्ली $\begin{matrix}
\vspace{-1mm}
\mbox{{१}}\\
\vspace{-1mm}
\mbox{{७}}\\
\vspace{-1mm}
\mbox{{५}}\\
\vspace{-1mm}
\mbox{{०}}
\vspace{1mm}
\end{matrix}$

\newpage 
%%%%%%%%%%%%%%%%%%%%%%%%%%%%%%%%%%%%%%%%

\noindent अत्रोपान्तिमेन ५ स्वोर्ध्वे ७ हते ३५ अन्त्येन ० युते ३५ अन्त्यं ०
त्यजेदिति जातं $\begin{matrix}
\vspace{-1mm}
\mbox{{१}}\\
\vspace{-1mm}
\mbox{{३५}}\\
\vspace{-1mm}
\mbox{{५}}
\vspace{1mm}
\end{matrix}$ पुनरुपान्तिमेन ३५ स्वोर्ध्वे १ हते ३५ अन्त्येन ५ युते ४० अन्त्यं ५ त्यजेदिति जातं राशिद्वयं $\begin{matrix}
\vspace{-1mm}
\mbox{{४०}}\\
\vspace{-1mm}
\mbox{{३५}}
\vspace{1mm}
\end{matrix}$ एतौ दृढभाज्यहाराभ्यामाभ्यां $\begin{matrix}
\vspace{-1mm}
\mbox{{१७}}\\
\vspace{-1mm}
\mbox{{१५}}
\vspace{1mm}
\end{matrix}$ तष्टौ शेषे $\begin{matrix}
\vspace{-1mm}
\mbox{{६}}\\
\vspace{-1mm}
\mbox{{५}}
\vspace{1mm}
\end{matrix}$ जातौ
क्रमेण लब्धिगुणौ~। \hyperref[59]{\textbf{इष्टाहतस्वस्वहरेण युक्ते ते वा भवेतां बहुधा गुणाप्ती}} इत्युक्तत्वात् अनयोर्लब्धिगुणयोः स्वतक्षणादिष्टगुणं क्षेप इत्येकमिष्टं
प्रकल्प्य जातौ 
लब्धिगुणौ वा २३।२० द्विकेनेष्टेन वा ४०।३५ त्रिकेणेष्टेन वा ५७।५०~।
एवम् इष्टवशाल्लब्धिगुणयोरानन्त्यं ज्ञेयम्~। तेन तेन गुणेनोद्दिष्टभाज्ये \,गुणिते \,क्षेपेण \,युते \,च \,सति \,सा \,लब्धिः \,शेषाभावश्च \,भवतीत्यर्थः~। अत्रापि क्षेपभाज्यावेव क्षेपभाजकावेव वापवर्त्य 
यद्वा प्रथमतः क्षेपभाज्यौ पश्चात् क्षेपभाजकौ चापवर्त्याथवादौ
त्रीनपवर्त्य पश्चात्क्षेपभाजकावप्यपवर्त्य चतुर्धा कुट्टको द्रष्टव्यः~। तत्र
त्रयाणामपवर्तसम्भवे सति 
यदि द्वावेवापवर्त्य कुट्टकः क्रियते तदा न सकलगुणलाभः~। अथ चतुर्षु 
प्रकारेषु भाज्यादीनां क्रमेण न्यासः~। 
\begin{table}[h!]
    \centering\s
    \begin{tabular}{llll}
     भा १७ क्षे ५ &भा २२१ क्षे ५ &भा १७ क्षे १ &भा १७ क्षे १ \\
~ह १९५ &ह ~१५ &ह ~३९& ह ३ \\
क्रमेण लब्धिगुणौ& ७ ~७४ &७ ~६& \\
 &८० ~५ &१६ ~१ & \\
    \end{tabular}
\end{table}

\noindent प्रथमे लब्धेः ७ अपवर्ताङ्केन १३ गुणने जातो लब्धिगुणौ ९१~। ८०~। द्वितीये
गुणकं ५ अपवर्तेन १३ सङ्गुण्य जातौ लब्धिगुणौ ७४~। ६५~। तृतीये क्षेपभाज्यापवर्तेन १३ लब्धिं ७ सङ्गुण्य क्षेपभाजकापवर्तेन ५ गुणं १६ सङ्गुण्य
जातौ 
लब्धिगुणौ ९१~। ८०~। चतुर्थे क्षेपभाजकापवर्तेन ५ गुणं १ सङ्गुण्य जातौ
लब्धिगुणौ ६~। ५~। एवं यथासम्भवं सर्वत्र प्रकारा ऊह्याः~।
वक्ष्यमाणस्थिरकुट्टकप्रकारेण गुणलब्धिसाधनं सर्वत्र बोध्यम्~। \\

\vspace{-4mm}
 अथ त्रयाणामनपवर्ते भवति कुट्टविधेरिति सूत्रस्य स्वतन्त्रमुदाहरणं योगजे
तक्षणाच्छुद्ध इत्यस्य च क्रमेणोदाहरणद्वयमुपजातिकयाह\textendash

\phantomsection \label{61}
\begin{quote}
    \eg 
     शतं हतं येन युतं नवत्या विवर्जितं वा विहृतं त्रिषष्ट्या~। \\
 निरग्रकं स्याद्वद मे गुणं तं स्पष्टं पटीयान्यदि कुट्टकेऽसि~॥~६१~॥
\end{quote}
 
\newpage
%%%%%%%%%%%%%%%%%%%%%%%%%%%%%%%%%%%%%%%%%%
 शतं येन  गुणेन हतं नवत्या युतं त्रिषष्ट्या विहृतं निरग्रकं स्यात्तं 
गुणमाशु वद~। अथ वियोग उदाहरणं विवर्जितं वेति~। शतं येन हतं नवत्या 
विवर्जितं त्रिषष्ट्या विहृतं निरग्रकं स्यात्तं गुणं च वद~। यदि त्वं 
कुट्टके पटीयान् पटुतरोऽसि~। न्यासः $\begin{matrix}
\vspace{-1mm}
\mbox{{भा १०० क्षे ९०}}\\
\vspace{-1mm}
\mbox{{ह ६३ ~~~~~~}}
\vspace{1mm}
\end{matrix}$~। अत्र हरभाज्ययोः परस्परं भक्तयोः शेषं १~।
अत इदमपवर्तनम्~। अनेनापवर्तनेऽर्थादनपवर्त एव~। अत्र प्राग्वद्वल्ली $\begin{matrix}
\vspace{-1mm}
\mbox{{१}}\\
\vspace{-1mm}
\mbox{{१}}\\
\vspace{-1mm}
\mbox{{१}}\\
\vspace{-1mm}
\mbox{{२}}\\
\vspace{-1mm}
\mbox{{२}}\\
\vspace{-1mm}
\mbox{{१}}\\
\vspace{-1mm}
\mbox{{९०}}\\
\vspace{-1mm}
\mbox{{०}}
\vspace{1mm}
\end{matrix}$ जातं राशिद्वयं $\begin{matrix}
\vspace{-1mm}
\mbox{{२४३०}}\\
\vspace{-1mm}
\mbox{{१०५३}}
\vspace{1mm}
\end{matrix}$ स्वस्वहारेण तक्षणे कृते जातौ लब्धिगुणौ $\begin{matrix}
\vspace{-1mm}
\mbox{{३०}}\\
\vspace{-1mm}
\mbox{{१८}}
\vspace{1mm}
\end{matrix}$~। यद्वा
भाज्यक्षेपौ दशभिरपवर्त्य न्यासः $\begin{matrix}
\vspace{-1mm}
\mbox{{भा १० क्षे ९}}\\
\vspace{-1mm}
\mbox{{ह ६३ ~~~~}}
\vspace{1mm}
\end{matrix}$ पूर्ववद्वल्ली $\begin{matrix}
\vspace{-1mm}
\mbox{{०}}\\
\vspace{-1mm}
\mbox{{६}}\\
\vspace{-1mm}
\mbox{{३}}\\
\vspace{-1mm}
\mbox{{९}}\\
\vspace{-1mm}
\mbox{{०}}
\vspace{1mm}
\end{matrix}$ पूर्ववद्राशिद्वयं $\begin{matrix}
\vspace{-1mm}
\mbox{{२७}}\\
\vspace{-1mm}
\mbox{{१७१}}
\vspace{1mm}
\end{matrix}$ तक्षणे जातं $\begin{matrix}
\vspace{-1mm}
\mbox{{७}}\\
\vspace{-1mm}
\mbox{{४५}}
\vspace{1mm}
\end{matrix}$ लब्धयो विषमा इति स्वतक्षणाभ्यामाभ्यां $\begin{matrix}
\vspace{-1mm}
\mbox{{१०}}\\
\vspace{-1mm}
\mbox{{६३}}
\vspace{1mm}
\end{matrix}$ शोधितौ जातौ लब्धिगुणौ $\begin{matrix}
\vspace{-1mm}
\mbox{{३}}\\
\vspace{-1mm}
\mbox{{१८}}
\vspace{1mm}
\end{matrix}$~। अत्र लब्धिर्न ग्राह्या~। किन्तु गुणघ्नभाज्ये क्षेपयुते हरभक्ते लब्धिः ३०~। यद्वापवर्तेन १० गुणिता लब्धिः ३ इयं जाता ३०~। एवं जातौ तावेव
लब्धिगुणौ $\begin{matrix}
\vspace{-1mm}
\mbox{{३०}}\\
\vspace{-1mm}
\mbox{{१८}}
\vspace{1mm}
\end{matrix}$ इदमत्रावधेयम्~। इष्टाहतस्वस्वहरेणेति क्षेपे कर्तव्ये यदि प्रथमत
उत्पन्नयोर्लब्धिगुणयोः क्रियते तदा यादृशाभ्यां भाज्यहराभ्यामुत्पन्नौ लब्धिगुणौ
तावेवेष्टगुणौ
 \newpage %%%%%%%%%%%%%%%%%%%%%%%%%%%%%%%%%%%%%%%%

\noindent क्षेपौ भवतः~। यथात्रैकेनेष्टेन $\begin{matrix}
\vspace{-1mm}
\mbox{{१३}}\\
\vspace{-1mm}
\mbox{{८१}}
\vspace{1mm}
\end{matrix}$ पश्चाल्लब्धिरपवर्ताङ्केन १० गुण्या~।
एवं जातावेकेनेष्टेन लब्धिगुणौ $\begin{matrix}
\vspace{-1mm}
\mbox{{१३०}}\\
\vspace{-1mm}
\mbox{{८१}}
\vspace{1mm}
\end{matrix}$~। एवं युतिभाजकमात्रापवर्तेऽपि क्षेपानन्तरमपवर्ताङ्केन गुणो गुणनीयः~। एवमेव युतिभाज्ययोर्युतिभाजकयोश्चापवर्ते लब्धिगुणौ गुणनीयौ~। यदि तु
स्वोद्दिष्टसिद्धयोर्लब्धिगुणयोः क्षेपः क्रियते तदोद्दिष्टभाज्यहरावेवेष्टगुणौ क्षेपौ
भवतः~। यथात्र लब्धिगुणौ $\begin{matrix}
\vspace{-1mm}
\mbox{{३०}}\\
\vspace{-1mm}
\mbox{{१८}}
\vspace{1mm}
\end{matrix}$ एकेनेष्टेन क्षेपौ $\begin{matrix}
\vspace{-1mm}
\mbox{{१००}}\\
\vspace{-1mm}
\mbox{{६३}}
\vspace{1mm}
\end{matrix}$ स्वस्वक्षेपयुतौ जातौ लब्धिगुणौ तावेव $\begin{matrix}
\vspace{-1mm}
\mbox{{१३०}}\\
\vspace{-1mm}
\mbox{{८१}}
\vspace{1mm}
\end{matrix}$~। यत्र तु त्रयाणामप्यपवर्तः क्रियते तत्रेष्टगुणयोर्दृढभाज्यहारयोरेव
यदा कदापि क्षेपत्वं सम्भवतीत्यादि सुधीभिः सर्वत्रौह्यम्~। अथवा हरक्षेपौ नवभिरपवर्त्य
न्यासः $\begin{matrix}
\vspace{-1mm}
\mbox{{भा १०० क्षे १०}}\\
\vspace{-1mm}
\mbox{{ह ७ ~~~~~~~}}
\vspace{1mm}
\end{matrix}$~। पूर्ववद्वल्ली $\begin{matrix}
\vspace{-1mm}
\mbox{{१४}}\\
\vspace{-1mm}
\mbox{{३}}\\
\vspace{-1mm}
\mbox{{१०}}\\
\vspace{-1mm}
\mbox{{०}}
\vspace{1mm}
\end{matrix}$ जातं राशिद्वयं $\begin{matrix}
\vspace{-1mm}
\mbox{{४३०}}\\
\vspace{-1mm}
\mbox{{३०}}
\vspace{1mm}
\end{matrix}$ तक्षणे जातं $\begin{matrix}
\vspace{-1mm}
\mbox{{३०}}\\
\vspace{-1mm}
\mbox{{२}}
\vspace{1mm}
\end{matrix}$ हरक्षेपापवर्ताङ्केन ६ गुणं सङ्गुण्य वा  तावेव $\begin{matrix}
\vspace{-1mm}
\mbox{{३०}}\\
\vspace{-1mm}
\mbox{{१८}}
\vspace{1mm}
\end{matrix}$~। अथवा भाज्यक्षेपौ हरक्षेपौ चापवर्त्य न्यासः $\begin{matrix}
\vspace{-1mm}
\mbox{{भा १० क्षे १}}\\
\vspace{-1mm}
\mbox{{ह ७ ~~~~}}
\vspace{1mm}
\end{matrix}$~। पूर्ववद्वल्ली $\begin{matrix}
\vspace{-1mm}
\mbox{{१}}\\
\vspace{-1mm}
\mbox{{२}}\\
\vspace{-1mm}
\mbox{{१}}\\
\vspace{-1mm}
\mbox{{०}}
\vspace{1mm}
\end{matrix}$ जातं राशिद्वयं $\begin{matrix}
\vspace{-1mm}
\mbox{{३}}\\
\vspace{-1mm}
\mbox{{२}}
\vspace{1mm}
\end{matrix}$ अत्र भाज्यक्षेपापवर्तेन १० लब्धिं सङ्गुण्य हरक्षेपापवर्तेन ९ गुणं च सङ्गुण्य वा जातौ लब्धिगुणौ तावेव $\begin{matrix}
\vspace{-1mm}
\mbox{{३०}}\\
\vspace{-1mm}
\mbox{{१८}}
\vspace{1mm}
\end{matrix}$ एकेनेष्टोनोक्तवल्लब्धिगुणौ $\begin{matrix}
\vspace{-1mm}
\mbox{{१३०}}\\
\vspace{-1mm}
\mbox{{८१}}
\vspace{1mm}
\end{matrix}$ द्विकेन वा $\begin{matrix}
\vspace{-1mm}
\mbox{{२३०}}\\
\vspace{-1mm}
\mbox{{१४४}}
\vspace{1mm}
\end{matrix}$~। अत्र प्रथमन्यासे
तृतीयन्यासे च हरतष्टे धनक्षेप इत्यपि प्रकारः सम्भवति~। अथवा भागहारेण तष्टयोः
क्षेपभाज्ययोः सत्यपि~। 
अथ द्वितीयोदाहरणे न्यासः $\begin{matrix}
\vspace{-1mm}
\mbox{{भा १०० क्षे ६०}}\\
\vspace{-1mm}
\mbox{{ह ६३ ~~~~~}}
\vspace{1mm}
\end{matrix}$ \hyperref[54]{\textbf{योगजे तक्षणाच्छुद्धे गुणाप्ती
स्तो वियो-}}
\newpage
%%%%%%%%%%%%%%%%%%%%%%%%%%%%%%%%%%%%%
\noindent \hyperref[54]{\textbf{गजे}} इत्युक्तत्वाद्योगजौ लब्धिगुणौ $\begin{matrix}
\vspace{-1mm}
\mbox{{३०}}\\
\vspace{-1mm}
\mbox{{१८}}
\vspace{1mm}
\end{matrix}$ स्वतक्षणाभ्यामाभ्यां शोधितौ जातौ
नवतिवियोगे लब्धिगुणौ $\begin{matrix}
\vspace{-1mm}
\mbox{{७०}}\\
\vspace{-1mm}
\mbox{{४५}}
\vspace{1mm}
\end{matrix}$~। एवं सर्वेष्वपि प्रकारेषु बोध्यम्~। अत्रापि
क्षेपवशादानन्त्यम्~॥~६१~॥\\

\vspace{-1mm}
अथ \hyperref[54]{\textbf{धनभाज्योद्भवे तद्वत्}} इत्यस्योदाहरणद्वयं रथोद्धतयाह\textendash

\phantomsection \label{62}
\begin{quote}
    \eg 
    यद्गुणाक्षयगषष्टिरन्विता वर्जिता च यदि वा त्रिभिस्ततः~। \\
स्यात्त्रयोदशहृता निरग्रका तं गुणं गणक मे पृथग्वद~॥~६२~॥~
\end{quote}

क्षेपस्य धनत्वेनैकम् ऋणत्वेन द्वितीयम् इत्युदाहरणद्वयम्~। शेषं स्पष्टम्~। न्यासः $\begin{matrix}
\vspace{-1mm}
\mbox{{भा ६ं० क्षे ३}}\\
\vspace{-1mm}
\mbox{{ह १३ ~~~~}}
\vspace{1mm}
\end{matrix}$ ~वल्ली $\begin{matrix}
\vspace{-1mm}
\mbox{{४}}\\
\vspace{-1mm}
\mbox{{१}}\\
\vspace{-1mm}
\mbox{{१}}\\
\vspace{-1mm}
\mbox{{१}}\\
\vspace{-1mm}
\mbox{{१}}\\
\vspace{-1mm}
\mbox{{३}}\\
\vspace{-1mm}
\mbox{{०}}
\vspace{1mm}
\end{matrix}$ जातं $\begin{matrix}
\vspace{-1mm}
\mbox{{६९}}\\
\vspace{-1mm}
\mbox{{१५}}
\vspace{1mm}
\end{matrix}$ तक्षणे जातं $\begin{matrix}
\vspace{-1mm}
\mbox{{९}}\\
\vspace{-1mm}
\mbox{{२}}
\vspace{1mm}
\end{matrix}$~। लब्धयोः विषमा इति स्वतक्ष-णाभ्यां \;$\begin{matrix}
\vspace{-1mm}
\mbox{{६०}}\\
\vspace{-1mm}
\mbox{{१३}}
\vspace{1mm}
\end{matrix}$ \;विशोध्य जातौ लब्धिगुणौ \;$\begin{matrix}
\vspace{-1mm}
\mbox{{५१}}\\
\vspace{-1mm}
\mbox{{११}}
\vspace{1mm}
\end{matrix}$ \;धनभाज्ये धनक्षेपे च~। धनभाज्योद्भवे तद्वदित्युक्तत्वात्स्वतक्षणशुद्धौ जातावृणभाज्ये धनक्षेपे च लब्धिगुणौ $\begin{matrix}
\vspace{-1mm}
\mbox{{९}}\\
\vspace{-1mm}
\mbox{{२}}
\vspace{1mm}
\end{matrix}$~। अत्र भाज्यभाजकयोर्विजातीययोर्भागहारेऽपि चैवं
निरुक्तमित्युक्तत्वाल्लब्धेरृणत्वं ज्ञेयं $\begin{matrix}
\vspace{-1mm}
\mbox{{९ं}}\\
\vspace{-1mm}
\mbox{{२}}
\vspace{1mm}
\end{matrix}$~। पुनरेतौ स्वतक्षणाभ्यामाभ्यां $\begin{matrix}
\vspace{-1mm}
\mbox{{६०}}\\
\vspace{-1mm}
\mbox{{१३}}
\vspace{1mm}
\end{matrix}$ शोधितौ जातावृणभाज्यक्षेपयोर्लब्धिगुणौ $\begin{matrix}
\vspace{-1mm}
\mbox{{५१}}\\
\vspace{-1mm}
\mbox{{११}}
\vspace{1mm}
\end{matrix}$~। अत्रापि हरभाज्ययोर्विजातीयत्वाल्लब्धेरृणत्वमिति जातौ $\begin{matrix}
\vspace{-1mm}
\mbox{{५ं१}}\\
\vspace{-1mm}
\mbox{{११}}
\vspace{1mm}
\end{matrix}$~। 
 
\newpage %%%%%%%%%%%%%%%%%%%%%%%%%%%%%%%%%%%%%%%%

\noindent अत्रेदमवधेयम्\textendash \,प्रथमतो भाज्यभाजकक्षेपाणां धनत्वमेव प्रकल्प्य लब्धिगुणौ साध्यौ~। 
अथ यद्युद्दिष्टभाज्यक्षेपयोर्धनत्वमृणत्वं वा स्यात्तदा
साधितगुणाप्तिभ्यामेवोद्दिष्टसिद्धिः~। यदा तु भाज्यक्षेपयोरन्यतरस्य धनत्वमृणत्वमितरस्य तदा यथागतौ
लब्धिगुणौ स्वतक्षणाभ्यां शोध्यौ ताभ्यामुद्दिष्टसिद्धिः~। हरस्य धनत्व
ऋणत्वे वा न 
कश्चित्कुट्टके विशेषः~। उक्तरीत्या गुणाप्त्योर्धनत्वमेव~।
भाज्यभाजकयोर्मध्य 
एकस्यैव ऋणत्वे लब्धिमात्रस्यर्णत्वं ज्ञेयम्~। भागहारेऽपि चैवं
निरुक्तमित्युक्तत्वादिति सङ्क्षेपः~। एवमेकवारशोधनेनैवोद्दिष्टसिद्धिर्भवति~। यत्तु भाज्ये
ऋणगते 
स्वतक्षणाच्छोधनमेकं क्षेप ऋणगते पुनर्द्वितीयमित्युक्तं तद्बालबोधार्थम्~। 
अयमर्थं आचार्येणैव विवृतः~। \hyperref[54]{\textbf{धनभाज्योद्भवे तद्वद्भवेतामृणभाज्यजे}} इति
मन्दावबोधार्थं मयोक्तमन्यथा \hyperref[54]{\textbf{योगजे तक्षणाच्छुद्धे}} इत्यादिनैव तत्सिद्धेः~। यतो
धनर्णयोगो वियोग एव~। अत एव भाज्यभाजकक्षेपाणां धनत्वमेव प्रकल्प्य 
गुणाप्ती साध्ये~। ते योगजे भवतस्ते स्वतक्षणाभ्यां शुद्धे वियोगजे 
कार्ये इत्यादिना~। एवमृणभाज्येऽप्यप्रयासेनैव कुट्टकसिद्धौ
सत्यामप्यन्यैर्वृथा प्रयासः कृत इत्याह {\qt 'भाज्ये भाजके वा ऋणगते परस्परभजनाल्लब्धय ऋणगताः स्थाप्या इति किं प्रयासेन'} इति~। अत्र क्षेपस्यर्णत्वे धनत्वे वोपान्तिमेन स्वोर्ध्वे हत इत्यादिकरणे धनर्णत्वावधानेन प्रयासगौरवं
द्रष्टव्यम्~। न केवलं प्रयासः~। अपि तु लब्धो व्यभिचारोऽपि~। तथाहि\textendash \,प्रकृतोदाहरणे न्यासः $\begin{matrix}
\vspace{-1mm}
\mbox{{भा ६ं०}}\\
\vspace{-1mm}
\mbox{{ह १३}}
\vspace{1mm}
\end{matrix}$ क्षे ३ उक्तवद्वल्ली $\begin{matrix}
\vspace{-1mm}
\mbox{{४ं}}\\
\vspace{-1mm}
\mbox{{१ं}}\\
\vspace{-1mm}
\mbox{{१ं}}\\
\vspace{-1mm}
\mbox{{१ं}}\\
\vspace{-1mm}
\mbox{{१ं}}\\
\vspace{-1mm}
\mbox{{३}}\\
\vspace{-1mm}
\mbox{{०}}
\vspace{1mm}
\end{matrix}$ जातं राशिद्वयं $\begin{matrix}
\vspace{-1mm}
\mbox{{६ं९}}\\
\vspace{-1mm}
\mbox{{१५}}
\vspace{1mm}
\end{matrix}$ तक्षणे जातं $\begin{matrix}
\vspace{-1mm}
\mbox{{९ं}}\\
\vspace{-1mm}
\mbox{{२}}
\vspace{1mm}
\end{matrix}$ लब्धिवैषम्यात्स्वतक्षणशुद्धौ जातौ लब्धिगुणावृणभाज्ये
धनक्षेपे च $\begin{matrix}
\vspace{-1mm}
\mbox{{५ं१}}\\
\vspace{-1mm}
\mbox{{११}}
\vspace{1mm}
\end{matrix}$~। अत्र लब्धौ व्यभिचारः~। यतोऽनेन ११ भाज्येऽस्मिन् ६ं० गुणिते ६६ं० क्षेप\textendash \,३\textendash \,हते ६५ं७ लब्धिः ५ं० शेषं च ७~। नन्वत्र शेषसत्त्वाद्गुणोऽपि व्यभिचारी~। तत्कथ-
\newpage
%%%%%%%%%%%%%%%%%%%%%%%%%%%%%%%%%%%%%%%%
\noindent मुक्तं लब्धौ व्यभिचारः स्यादिति~। सत्यम्~। न ह्यत्र
लब्धावेवेत्यवधारणमस्ति~। किं 
तु लब्धावित्युपलक्षणम्~। तेन गुणेऽपि व्यभिचारः स्यादित्यर्थः~।
लब्धिकाले व्यभिचारनिश्चयाल्लब्धौ व्यभिचारः स्यादित्युक्तमिति~। नन्वत्र नास्ति
व्यभिचारः~। तथाहि\textendash \,अत्रोक्तवज्जातं राशिद्वयं $\begin{matrix}
\vspace{-1mm}
\mbox{{६ं९}}\\
\vspace{-1mm}
\mbox{{१५}}
\vspace{1mm}
\end{matrix}$ तक्षणे जातौ लब्धिगुणौ $\begin{matrix}
\vspace{-1mm}
\mbox{{९ं}}\\
\vspace{-1mm}
\mbox{{२}}
\vspace{1mm}
\end{matrix}$ अनेन २ 
भाज्येऽस्मिन्  ६० गुणिते १२ं०  क्षेप\textendash \,३\textendash \,युते ११ं७ हरभक्ते
लब्धिरियं  ९ं इति चेन्न~। तत्किं विषमलब्धिष्वपि स्वतक्षणाच्छोधनमपाकर्तुमुद्यतोऽसि~। तथा सति भाज्यभाजकक्षेपाणां घनत्वे लब्धीनां विषमत्वे च व्यभिचारस्तावत्स्यात्~। यथास्मिन्नेवोदाहरण उक्तवल्लब्धिगुणौ $\begin{matrix}
\vspace{-1mm}
\mbox{{९}}\\
\vspace{-1mm}
\mbox{{२}}
\vspace{1mm}
\end{matrix}$ अनेन २ भाज्ये ६० गुणिते १२० क्षेप\textendash \,३\textendash \,युते १२३
हर\textendash \,१३\textendash \,भक्ते निःशेषता न स्यात्~। अथ यद्युच्येत धनविषमलब्धिषु स्वतक्षणाच्छोधनमावश्यकं 
न त्वृणलब्धिष्विति चेन्न~। व्यभिचारस्तावत्स्यात्~। यथास्मिन्नेवोदाहरणे
हरमात्रस्यर्णत्व उक्तवजातौ लब्धिगुणौ $\begin{matrix}
\vspace{-1mm}
\mbox{{९ं}}\\
\vspace{-1mm}
\mbox{{२}}
\vspace{1mm}
\end{matrix}$ अनेन २ भाज्ये ६० गुणिते १२० क्षेप\textendash \,३\textendash \,युते १२३ हरभक्ते निरग्रताया अभावात्~। किं च समलब्धिष्वप्यस्ति व्यभिचारसम्भवः~। यथाष्टादश गुणाः केनेत्यनुपदवक्ष्यमाणोदाहरणे~। तथाहि\textendash \, $\begin{matrix}
\vspace{-1mm}
\mbox{{भा १८ क्षे १०}}\\
\vspace{-1mm}
\mbox{{ह १ं१ ~~~~}}
\vspace{1mm}
\end{matrix}$ अत्र वल्ली $\begin{matrix}
\vspace{-1mm}
\mbox{{१ं}}\\
\vspace{-1mm}
\mbox{{१ं}}\\
\vspace{-1mm}
\mbox{{१ं}}\\
\vspace{-1mm}
\mbox{{१ं}}\\
\vspace{-1mm}
\mbox{{१०}}\\
\vspace{-1mm}
\mbox{{०}}
\vspace{1mm}
\end{matrix}$ जातं राशिद्वयं $\begin{matrix}
\vspace{-1mm}
\mbox{{५०}}\\
\vspace{-1mm}
\mbox{{३ं०}}
\vspace{1mm}
\end{matrix}$ तक्षणे $\begin{matrix}
\vspace{-1mm}
\mbox{{१४}}\\
\vspace{-1mm}
\mbox{{८ं}}
\vspace{1mm}
\end{matrix}$ अत्र गुणेन ८ं भाज्ये १८ गुणिते १४ं४
क्षेप\textendash \,१०\textendash \,युते १३ं४ हर\textendash \,१ं१\textendash \,भक्ते लब्धिः १२ शेषं २ं इत्यू-ह्यम्~। अत्र समलब्धिषु
हरस्यर्णत्वे सति विषमलब्धिषु भाज्यस्यर्णत्वे सति वा पूर्वेषां कुट्टके व्यभिचार इति निष्कर्षः~॥~६२~॥\\

\vspace{-4mm}
अथ भाजकस्यर्णत्वेऽनुष्टुभोदाहरणमाह\textendash

\phantomsection \label{63}
\begin{quote}
    \eg 
     अष्टादश गुणाः केन दशाढ्या वा दशोनिताः~। \\
 शुद्धं भागं प्रयच्छन्ति क्षयगैकादशोद्धृता~॥~६३~॥
\end{quote}

\newpage
%%%%%%%%%%%%%%%%%%%%%%%%%%%%%%%%%%%%%%%%%%%%%

\noindent अष्टादशेति छेदः~। स्पष्टमन्यत्~। न्यासः $\begin{matrix}
\vspace{-1mm}
\mbox{{भा १८ क्षे १०}}\\
\vspace{-1mm}
\mbox{{ह १ं१ ~~~~}}
\vspace{1mm}
\end{matrix}$ वल्ली $\begin{matrix}
\vspace{-1mm}
\mbox{{१}}\\
\vspace{-1mm}
\mbox{{१}}\\
\vspace{-1mm}
\mbox{{१}}\\
\vspace{-1mm}
\mbox{{१ं}}\\
\vspace{-1mm}
\mbox{{१०}}\\
\vspace{-1mm}
\mbox{{०}}
\vspace{1mm}
\end{matrix}$ राशिद्वयं $\begin{matrix}
\vspace{-1mm}
\mbox{{५०}}\\
\vspace{-1mm}
\mbox{{३०}}
\vspace{1mm}
\end{matrix}$ तक्षणे जातं $\begin{matrix}
\vspace{-1mm}
\mbox{{१४}}\\
\vspace{-1mm}
\mbox{{८}}
\vspace{1mm}
\end{matrix}$ त्रयाणां धनत्वे जातावेतौ लब्धिगुणौ~।
हरमात्रस्यर्णत्वेऽप्येतावेव लब्धिगुणौ किन्तु लब्धिमात्रमृणं भागहारेऽपि चैवं निरुक्तत्वात्~। एवमृणहरे जातौ लब्धिगुणौ $\begin{matrix}
\vspace{-1mm}
\mbox{{१ं४}}\\
\vspace{-1mm}
\mbox{{८}}
\vspace{1mm}
\end{matrix}$~। अथर्णक्षेपे \hyperref[54]{\textbf{योगजे तक्षणाच्छुद्धे}} इत्यादिना जातौ $\begin{matrix}
\vspace{-1mm}
\mbox{{४}}\\
\vspace{-1mm}
\mbox{{३}}
\vspace{1mm}
\end{matrix}$~। अत्र हरस्य धनत्वे ऋणत्वे वा लब्धिगुणावेतावेव~। किन्तु हरस्यर्णत्वे लब्धेर्ऋणत्वं ज्ञेयम्~। अत्र 
सर्वत्र ऋणत्वनिमित्तं यत्स्वतक्षणाच्छोधनं तद्भाज्यक्षेपयोरेकतरस्यैव
ऋणत्वे~। नान्यथा~। 
तथा भाज्यभा-जकयोरेकतरस्यैवर्णत्वे लब्धेर्ऋणत्वं न त्वन्यथेति निष्कर्षः~।
केचित् 'ऋणभाज्योद्भवे तद्वद्भवेतामृणभाजके' इति पाठं कल्पयित्वा भाजकर्णत्वेऽपि
स्वतक्षणाच्छोधनं 
कुर्वन्ति~। तदसदिति प्रतिभाति'~। यथास्मिन्नुदाहरणे त्रयाणां धनत्वे
जातौ लब्धिगुणौ $\begin{matrix}
\vspace{-1mm}
\mbox{{१४}}\\
\vspace{-1mm}
\mbox{{८}}
\vspace{1mm}
\end{matrix}$~। \\
\vspace{-1mm}

अथ हरमात्रस्यर्णत्वे स्वतक्षणाभ्यां शोधितौ जातौ $\begin{matrix}
\vspace{-1mm}
\mbox{{४}}\\
\vspace{-1mm}
\mbox{{३}}
\vspace{1mm}
\end{matrix}$~। अनेन ३
भाज्येऽस्मिन् १८ गुणिते ५४ क्षेप\textendash \,१०\textendash \,युते ६४ हर\textendash \,१ं१\textendash \,भक्ते लब्धिरियं ५ं~। शेषं च ९~। तस्मादिदमसत्~। यद्युच्येत भाजको वा यादृश उद्दिष्टस्तादृशस्यैव तक्षणत्वमिति स्वतक्षणाभ्यामाभ्यां $\begin{matrix}
\vspace{-1mm}
\mbox{{१८}}\\
\vspace{-1mm}
\mbox{{११}}
\vspace{1mm}
\end{matrix}$ शोधितौ जातौ लब्धिगुणौ $\begin{matrix}
\vspace{-1mm}
\mbox{{४}}\\
\vspace{-1mm}
\mbox{{३}}
\vspace{1mm}
\end{matrix}$~। नात्र कोऽपि दोष 
इति~। न~। \hyperref[7]{\textbf{संशोध्यमानं स्वमृणत्वमेति}} इत्यादिना शोधने कृते जातौ गुणः १ं९~। 
सोऽयमसत्~। न च तक्षणस्यर्णत्वे तक्ष्यस्याप्यृणत्वमिति प्रथमतो गुणस्य
८ं ऋणत्वे 
संशोध्यमानमृणं धनं भवतीत्यादिना जातोऽस्मदुक्त एव गुणः ३ं~। न 
ह्यसावसदिति वाच्यम्~। तत्किं बीजान्तरमधीतवानसि~। न ह्यस्मिन्बीज
ईदृशोऽर्थः 
कस्मिन्नपि सूत्रे प्रतिपादितोऽस्ति~। अथास्त्वाचार्याभिप्रायज्ञः स्वतः
कल्पको वा
\newpage
%%%%%%%%%%%%%%%%%%%%%%%%%%%%%%%%%%%%%%%%%%%%%%%
\noindent भवान्~। इदं तु पृच्छ्यते~। अधोराशिर्धनहरेण तष्टः सन् योगजो गुणो भवेदुत क्षयहरेण तष्टः सन्~। तत्र क्षयतक्षणे भवन्मते गुणस्यापि क्षयत्वम्~। न ह्यस्य ८ं योगजत्वमस्ति~। भजने निरग्रताया अभावात्~। अस्य ८ं अयोगजत्वे \hyperref[54]{\textbf{योगजे तक्षणाच्छुद्धे}} इति सूत्रं कथं
प्रवर्तेत येन त्वदभिमतो गुणः ३ं सिध्येत्~। धनतक्षणे तु गुणस्य धनत्वे ८
\hyperref[7]{\textbf{संशोध्यमानं स्वमृणत्वमेती}}ति क्षयत्वे जाते ८ं तक्षणस्य ११ धनत्वे ऋणत्वे
च शोधनेन जातौ क्रमेण गुणौ ३~। १९~। अनयोर्दुष्टत्वं स्पष्टमेवेत्यलं
पल्लवितेन~॥~६३~॥~\\

\vspace{-4mm}
अथ \hyperref[55]{\textbf{गुणलब्ध्योः समं ग्राह्यम्}} इति \hyperref[56]{\textbf{हरतष्टे धनक्षेपे}} इति \hyperref[57]{\textbf{अथवा भागहारेण तष्टयोः}} इति चैतेषामुदाहरणमनुष्टुभाह\textendash
\begin{quote}
    \eg 
    येन सङ्गुणिताः पञ्च त्रयोविंशतिसंयुताः~।\\
 वर्जिता वा त्रिभिर्भक्ता निरग्राः स्युः स को गुणः~॥~६४~॥
\end{quote}

स्पष्टोऽर्थः~। न्यासो $\begin{matrix}
\vspace{-1mm}
\mbox{{भा ५ क्षे २३}}\\
\vspace{-1mm}
\mbox{{ह ३ ~~~~}}
\vspace{1mm}
\end{matrix}$ प्राग्वद्वल्ली $\begin{matrix}
\vspace{-1mm}
\mbox{{१}}\\
\vspace{-1mm}
\mbox{{१}}\\
\vspace{-1mm}
\mbox{{२३}}\\
\vspace{-1mm}
\mbox{{०}}
\vspace{1mm}
\end{matrix}$ राशिद्वयं $\begin{matrix}
\vspace{-1mm}
\mbox{{४६}}\\
\vspace{-1mm}
\mbox{{२३}}
\vspace{1mm}
\end{matrix}$~। अत्र तक्षणेऽधोराशौ सप्त लभ्यन्ते~। ऊर्ध्वराशौ तु नव~। ते नव न ग्राह्याः~। \hyperref[55]{\textbf{गुणलब्ध्योः समं ग्राह्यं धीमता तक्षणे फलम्}} इत्यतः सप्तैव ग्राह्या इति जातौ
लब्धिगुणौ $\begin{matrix}
\vspace{-1mm}
\mbox{{११}}\\
\vspace{-1mm}
\mbox{{२}}
\vspace{1mm}
\end{matrix}$ योगजौ~। अनयोः स्वस्वतक्षणाच्छोधने जातौ वियोगजौ लब्धिगुणौ $\begin{matrix}
\vspace{-1mm}
\mbox{{६ं}}\\
\vspace{-1mm}
\mbox{{१}}
\vspace{1mm}
\end{matrix}$~। वियोगे
धनलब्ध्यपेक्षा चेत् तर्हि \hyperref[59]{\textbf{इष्टाहतस्वस्वहरेण युक्त}} इत्यादिना द्विकेनेष्टेन जातौ लब्धिगुणौ $\begin{matrix}
\vspace{-1mm}
\mbox{{४}}\\
\vspace{-1mm}
\mbox{{७}}
\vspace{1mm}
\end{matrix}$~। एवं सर्वत्र~। अथवा \hyperref[56]{\textbf{हरतष्टे धनक्षेपे}} इति न्यासः $\begin{matrix}
\vspace{-1mm}
\mbox{{भा ५ क्षे २}}\\
\vspace{-1mm}
\mbox{{ह ३ ~~~}}
\vspace{1mm}
\end{matrix}$ वल्ली $\begin{matrix}
\vspace{-1mm}
\mbox{{७}}\\
\vspace{-1mm}
\mbox{{१}}\\
\vspace{-1mm}
\mbox{{१}}\\
\vspace{-1mm}
\mbox{{२}}\\
\vspace{-1mm}
\mbox{{०}}
\vspace{1mm}
\end{matrix}$ राशि-

\newpage
%%%%%%%%%%%%%%%%%%%%%%%%%%%%%%%%%%
\noindent द्वयं $\begin{matrix}
\vspace{-1mm}
\mbox{{४}}\\
\vspace{-1mm}
\mbox{{२}}
\vspace{1mm}
\end{matrix}$ एतौ योगजौ लब्धिगुणौ~। तक्षणशोधनेन जातौ वियोगजौ $\begin{matrix}
\vspace{-1mm}
\mbox{{१}}\\
\vspace{-1mm}
\mbox{{१}}
\vspace{1mm}
\end{matrix}$~। अत्र \hyperref[56]{\textbf{क्षेपतक्षणलाभाढ्या लब्धिः शुद्धौ तु वर्जिता}} इति क्षेपतक्षणलाभेन ७
योगजलब्धि\textendash \,४\textendash \,युता ११ शुद्धौ तु लब्धिः १ वर्जिता ३ जातौ तावेव लब्धिगुणौ $\begin{matrix}
\vspace{-1mm}
\mbox{{११~। ६ं}}\\
\vspace{-1mm}
\mbox{{२~। १}}
\vspace{1mm}
\end{matrix}$~। \hyperref[57]{\textbf{अथवा भागहारेण तष्टयोः}} इति न्यासो $\begin{matrix}
\vspace{-1mm}
\mbox{{भा २ क्षे २}}\\
\vspace{-1mm}
\mbox{{ह ३ ~~~}}
\vspace{1mm}
\end{matrix}$ वल्ली $\begin{matrix}
\vspace{-1mm}
\mbox{{०}}\\
\vspace{-1mm}
\mbox{{१}}\\
\vspace{-1mm}
\mbox{{२}}\\
\vspace{-1mm}
\mbox{{०}}
\vspace{1mm}
\end{matrix}$ राशिद्वयं $\begin{matrix}
\vspace{-1mm}
\mbox{{२}}\\
\vspace{-1mm}
\mbox{{२}}
\vspace{1mm}
\end{matrix}$~। अत्रापि जातः पूर्व एव गुणः~। लब्धिस्तु \hyperref[57]{\textbf{भाज्याद्धतयुतोद्धृतात्}} इति गुण\textendash \,२\textendash \,गुणितो भाज्यः ५ जातः १० क्षेप\textendash \,२३\textendash \,युतः ३३ हरेण ३ भक्तो जाता लब्धिः
 सैव ११~। अथवा मदुक्तप्रकारेण लब्धिः~। गुणेन २ भाज्यतक्षणलाभो १ गुणितः
 २ क्षेपतक्षणलाभेन ७ संस्कृतः गणितागतलब्ध्या च संस्कृतः ११ जाता सैव
लब्धिः~। एवं सर्वत्र~॥~६४~॥~\\

\vspace{-4mm}
 अथ \hyperref[58]{\textbf{क्षेपाभावोऽथवा यत्र क्षेपः शुध्येद्धरोद्धृतः}} इत्यनयोरुदाहरणे रथोद्धतयाह\textendash
\begin{quote}
    \ab 
     येन पञ्च गुणिताः खसंयुताः पञ्चषष्टिसहिताश्च  तेऽथवा~। \\
 स्युस्त्रयोदश हृता निरग्रकास्तं गुणं गणक कीर्तयाशु मे~॥~६५~॥~
\end{quote}

 स्पष्टोऽर्थः~। उदाहरणद्वयेऽपि न्यासो $\begin{matrix}
\vspace{-1mm}
\mbox{{भा ५ क्षे ०}}\\
\vspace{-1mm}
\mbox{{ह १३ ~~~}}
\vspace{1mm}
\end{matrix}$~। $\begin{matrix}
\vspace{-1mm}
\mbox{{भा ५ क्षे ६५}}\\
\vspace{-1mm}
\mbox{{ह १३ ~~~~}}
\vspace{1mm}
\end{matrix}$~। प्रथमे क्षेपाभावोऽस्ति~। द्वितीये क्षेपो हरोद्धृतः शुध्यतीत्युभयत्रापि शून्यमेव गुणः~। क्षेपो हरहृतः फलमिति द्वयोरपि लब्धी ०~। ५~। एवं जातौ लब्धिगुणौ $\begin{matrix}
\vspace{-1mm}
\mbox{{०~। ५}}\\
\vspace{-1mm}
\mbox{{०~। ०}}
\vspace{1mm}
\end{matrix}$~। \hyperref[59]{\textbf{इष्टाहतस्वस्वहरेण युक्ते}} इत्यादिनैकेनेष्टेन १ जातौ $\begin{matrix}
\vspace{-1mm}
\mbox{{५~। १०}}\\
\vspace{-1mm}
\mbox{{१३~। १३}}
\vspace{1mm}
\end{matrix}$~। एवमिष्टवशादानन्त्यम्~। अथवात्र प्रथमप्रकारेण \hyperref[56]{\textbf{हरतष्टे धनक्षेपे}} इत्यनेन च गुणाप्ती साध्ये~॥~६५~॥~\\

\vspace{-4mm}
 अथ ग्रहगणिते विशेषोपयुक्तं
स्थिरकुट्टकमुपजातिकोत्तरपूर्वार्धाभ्यामाह\textendash
\newpage
%%%%%%%%%%%%%%%%%%%%%%%%%%%%%%%%%%%%
\phantomsection \label{66}
\begin{quote}
    \ab 
    क्षेपं विशुद्धिं परिकल्प्य रूपं पृथक्तयोर्ये गुणकारलब्धी~। \\
 अभीप्सितक्षेपविशुद्धिनिघ्ने स्वहारतष्टे भवतस्तयोस्ते~॥~६६~॥~
\end{quote}
 
\hyperref[66]{\textbf{क्षेपं}} धनक्षेपम्~। विशुद्धिमृणक्षेपं \hyperref[66]{\textbf{रूपं परिकल्प्य तयो}}र्धनर्णक्षेपयोः 
\hyperref[66]{\textbf{पृथग्गुणकारलब्धी}} ये स्यातां ते अभीप्सितक्षेपविशुद्धिगुणिते \hyperref[66]{\textbf{स्वहारतष्टे च तयोः}} क्षेपविशुद्ध्योस्ते गुणाप्ती \hyperref[66]{\textbf{भवतः}}~। एतदुक्तं भवति~। \hyperref[51]{\textbf{मिथो भजेत्तौ दृढभाज्यहारौ}} इत्यादिना फलान्यधोऽधो निवेश्य तदधः क्षेपस्थाने रूपं निवेश्यान्ते 
खं च निवेश्योपान्तिमेन स्वोर्ध्वे हत इत्यादिना धनक्षेपे ऋणक्षेपे च गुणलब्धी 
पृथक्पृथक्साध्ये~। अथाभीप्सितक्षेपो यदि धनम् अस्ति तर्हि धनक्षेपजे
गुणाप्ती अभीप्सितक्षेपेण गुणनीये~। यदि त्वभीप्सितक्षेपः क्षयोऽस्ति तर्हि ऋणक्षेपजे
गुणाप्ती अभीप्सितेनर्णक्षेपेण गुणनीये~। पश्चात्स्वस्वहारेण पूर्ववत्तक्ष्ये~। ते
उद्दिष्टगुणाप्ती स्तः~। अत्र मन्दविश्वासार्थमुदाहरणं प्रदर्शयति~। प्रथमोदाहरणे
दृढभाज्यहारयो रूपक्षेपयोर्न्यासो $\begin{matrix}
\vspace{-1mm}
\mbox{{भा १७ क्षे १}}\\
\vspace{-1mm}
\mbox{{ह १५ ~~~}}
\vspace{1mm}
\end{matrix}$~। अत्रोक्तवद्गुणाप्ती ७~। ८~। एते अभीष्टपञ्चगुणिते ३५~। ४०~। स्वहारतष्टे
जाते ५~। ६~। ते एव गुणाप्ती~। अथ रूपशुद्धौ गुणाप्ती ८~। ९ एते पञ्चगुणे 
४०~। ४५~। स्वहारतष्टे प्रथमोदाहरणे शुद्धिजे गुणाप्ती १०~। ११ एवं 
सर्वत्रेति~। स्पष्टोऽर्थः~। सविस्तरं तु {\qt 'लिप्ताग्रं शशिनः
खखाभ्रगगनप्राणर्तुभूभिर्हृतम्'} इत्यादिना निबद्धः स्थिरकुट्टको गोलाध्याये दर्शितः~।
{\qt 'स्थिरकुट्टकोपपत्तिस्तु भवति कुट्टविधेर्युतिभाज्ययोः'} इत्यस्योपपत्तौ प्रदर्शिता~। अथवा रूपक्षेपे
यद्येते गुणाप्ती तर्हि 
स्वाभीष्टक्षेपे के इति त्रैराशिकेनोपपत्तिर्द्रष्टव्या~। ननु किमर्थमयं
स्थिरकुट्टक उक्तः~। 
नहि प्रतिप्रश्नं तावेव भाज्यभाजकौ येन कृते स्थिरकुट्टके लाघवं
स्यादित्यत 
आह\textendash\ अस्य ग्रहगणिते महानुपयोग इति~। अयमर्थः~। यद्यपि लौकिकेषु कुट्टकप्रश्नेषु प्रतिप्रश्नं भाज्यभाजकभेदान्न स्थिरकुट्टकोपयोगोऽस्ति~।
तथापि ग्रहगणिते विविधक्षेपेषु 
तावेव भाज्यभाजकौ भवत इति तत्रास्त्येव स्थिरकुट्टकोपयोग इति~॥~६६~॥~\\

\vspace{-4mm}
 अथ यदि कश्चिद्ब्रूयाद्ग्रहगणिते स्थिरकुट्टकोपयोगः कुत्रास्ति~।
तदर्थमुपदेशव्याजेन तत्स्थलमुपजातिकोतरार्धेनोपजातिकया च दर्शयति\textendash
\newpage
%%%%%%%%%%%%%%%%%%%%%%%%%%%%%%%%%%

\phantomsection \label{67}
\begin{quote}
    \ab 
    कल्प्याथ शुद्धिर्विकलावशेषं षष्टिश्च भाज्यः कुदिनानि हारः~।\\
 तज्जं फलं स्युर्विकला गुणस्तु लिप्ताग्रमस्माच्च कलालवाग्रम्\\
 एवं तदूर्ध्वं च तथाधिमासावमाग्रकाभ्यो दिवसा रवीन्द्वोः~॥~६७~॥~
\end{quote}
 

अस्यार्थः स्वयमेव विवृणोति~। {\qt "ग्रहस्य
विकलाशेषात्तद्ग्रहाहर्गणयोरानयनम्~। तत्र 
षष्टिर्भाज्यः कुदिनानि हारः~। विकलावशेषं शुद्धिरिति प्रकल्प्य गुणाप्ती
साध्ये~। 
तत्र लब्धिर्विकलाः स्युः~। गुणस्तु कलावशेषम्~। एवं कलावशेषं शुद्धिं
प्रकल्प्य~। तत्र लब्धिः कलाः~। गुणो भागशेषम्~। भागशेषं शुद्धिस्त्रिंशद्भाज्यः
कुदिनानि 
हारस्तत्र फलं भागो गुणो राशिशेषम्~। द्वादशभाज्यः कुदिनानि हारः~।
राशिशेषं 
शुद्धिस्तत्र फलं गतराशयः~। गुणो भगणशेषम्~। कल्पभगणा भाज्यः कुदिनानि 
हारः~। भगणशेषं शुद्धिः~। तत्र फलं गतभगणाः~। गुणोऽहर्गणः स्यात् इति~। 
अस्योदाहरणानि प्रश्नाध्याये~। एवं कल्पाधिमासा \;भाज्यः~। रविदिनानि \;हारः~।
अधिमासशेषं \;शुद्धिः~। फलं \;गताधि-मासाः~। गुणो गतरविदिवसाः~। एवं युगावमानि
भाज्यः~। चन्द्रदिवसा हरः~। अवमशेषं शुद्धिः~। फलं गतावमानि~। गुणो 
गतचान्द्रदिवसाः"} इति~। अत्राचार्यव्याख्याने युगावमानि भाज्य
इत्यत्र 
कल्पशब्दस्थाने युगेति लिखने लेखकभ्रमजं द्रष्टव्यम्~। यद्वा न केवलं
कल्पजैर्भगणं 
कुदिनमधिमासावमादिभिर्ग्रहाहर्गणाद्यानयने विकलाशेषादेस्तदानयनं किं तु
युगजैरपि 
कुविदाद्यैस्तत्साधने तदुत्पन्नाद्विकलाशेषाद्युगजभाज्यभाजकेभ्योऽपि
तत्साधनं भवतीति 
सूचनाय युगावमानीत्युक्तम्~। एवं यथासम्भवं युगचरणजैरपि
कुदि-नादिभिर्ग्रहादिसाधने 
तादृशभाज्यभाजकेभ्यः कुट्टको ज्ञेयः~। अत एव सूत्रे कुदिनानि हार
इत्येवोक्तं न 
तु कल्पकुदिनानीति~। अत्र मन्दप्रतीत्यर्थं कल्पकुदिनानि १९ ग्रहभगणाः
कल्पे 
कल्पिताः ९~। अहर्गणः १३~। अत्र कल्पकुदिनैः कल्पभगणास्तदाहर्गणतुल्यैः
किम् इति त्रैराशिकेन १९~। ९~। १३ {\qt 'द्युचरचक्रहतो दिनसञ्चयः क्वहहृतो
भगणादि फलं ग्रह'} इत्यनेन सिद्धौ भगणादिग्रहः ६~। १~। २६~। ५०~। ३१ विकलापर्यन्तम्~। 
विकला शेषं च ११~। अस्माद्विलोमगत्या ग्रहोऽहर्गणश्चानीयते \hyperref[67]{\textbf{कल्प्याथ शुद्धिर्विकलावशेषम्}} इत्यादिना~। अत्र कुट्टकार्थं न्यासो $\begin{matrix}
\vspace{-1mm}
\mbox{{भा ६० क्षे १ं१}}\\
\vspace{-1mm}
\mbox{{ह १९ ~~~~~}}
\vspace{1mm}
\end{matrix}$ ~वल्ली $\begin{matrix}
\vspace{-1mm}
\mbox{{३}}\\
\vspace{-1mm}
\mbox{{६}}\\
\vspace{-1mm}
\mbox{{११}}\\
\vspace{-1mm}
\mbox{{०}}
\vspace{1mm}
\end{matrix}$

\newpage
%%%%%%%%%%%%%%%%%%%%%%%%%%%%%%%%%%%%%%%%%%%%%%%%%

\noindent जातं राशिद्वयं $\begin{matrix}
\vspace{-1mm}
\mbox{{२०९}}\\
\vspace{-1mm}
\mbox{{६६}}
\vspace{1mm}
\end{matrix}$ तक्षणे जातौ लब्धिगुणौ $\begin{matrix}
\vspace{-1mm}
\mbox{{२९}}\\
\vspace{-1mm}
\mbox{{९}}
\vspace{1mm}
\end{matrix}$ \hyperref[54]{\textbf{योगजे तक्षणाच्छुद्धे}} इति 
जातौ लब्धिगुणौ $\begin{matrix}
\vspace{-1mm}
\mbox{{३१}}\\
\vspace{-1mm}
\mbox{{१०}}
\vspace{1mm}
\end{matrix}$ ऋणक्षेपे अत्र लब्धिः ३१ विकलाः~। गुणाः कलाशेषं १०~। 
इदमृणक्षेपं १ं०~। अथ कलानयनार्थं कुट्टके न्यासो $\begin{matrix}
\vspace{-1mm}
\mbox{{भा ६० क्षे १ं०}}\\
\vspace{-1mm}
\mbox{{ह १९ ~~~~}}
\vspace{1mm}
\end{matrix}$~। उक्तवज्जातौ
लब्धिगुणौ ५०~। १६~। अत्र लब्धिः कलाः ५०~। गुणो भागशेषं १६~।
पुनर्भागशेषं शुद्धिरिति लवार्थं कुट्टके न्यासो $\begin{matrix}
\vspace{-1mm}
\mbox{{भा ३० क्षे १ं६}}\\
\vspace{-1mm}
\mbox{{ह १९ ~~~~}}
\vspace{1mm}
\end{matrix}$~। अत्राप्युक्तवल्लब्धिगुणौ २६~। 
१७~। अत्र लब्धिर्भागाः २६ गुणो राशिशेषं १७~। राशिशेषं शुद्धिरिति
राशिज्ञानार्थं 
न्यासो $\begin{matrix}
\vspace{-1mm}
\mbox{{भा १२ क्षे १ं७}}\\
\vspace{-1mm}
\mbox{{ह १९ ~~~~}}
\vspace{1mm}
\end{matrix}$~। अत्राप्युक्तवल्लब्धिगुणौ १~। ३~। अत्र लब्धिमितो
राशिः १~। गुणो भगणशेषं ३~। भगणशेषं शुद्धिः~। कल्पभगणाः ९ भाज्यः
कल्पकुदिनानि १९ हर इति न्यासो $\begin{matrix}
\vspace{-1mm}
\mbox{{भा ९ क्षे ३ं}}\\
\vspace{-1mm}
\mbox{{ह १९ ~~~}}
\vspace{1mm}
\end{matrix}$~। अत्राप्युक्तवल्लब्धिगुणौ ६~। १३~।
अत्र लब्धिर्गतभगणाः ६~। गुणोऽहर्गणः १३~। एवं
मन्दप्रतीत्यर्थमिष्टान्कल्पसौरदिवसान्कल्पाधिमासांश्च
प्रकल्प्याधिमासशेषाद्गताधिमाससौरदिवसा दर्शनीयाः~।
एवमवमाग्राद्गतावमचान्द्रदिवसाश्च~। अस्त्यत्र ग्रहगणिते स्थिरकुट्टकस्य महत्प्रयोजनम्~। तथाहि~।
विकलाग्राद्ग्रहानयने 
षष्टिर्भाज्यः~। कल्पकुदिनानि हार इति भाज्यभाजकौ नियतावेव~। विकलाशेषमृणक्षेपः स त्वनियतः~। अत्र स्थिरकुट्टककरणे प्रतिप्रश्नं
दीर्घवल्लीसम्भूतयोर्लब्धिगुणयोः 
साधनेऽस्ति गौरवम्~। स्थिरकुट्टके तु रूपमृणक्षेपं प्रकल्प्य लब्धिगुणौ
स्थिरौ कृत्वा 
तत्तद्विकलाशेषेण तयोर्गुणने सति स्वस्वहारेण तक्षणे च सति
स्वाभीप्सितलब्धिगुणसिद्धिरित्यतिलाघवमस्ति~। अत उक्तमस्य ग्रहगणिते महानुपयोग इति~।\\

\vspace{-4mm}
 अथ \hyperref[67]{\textbf{कल्प्याथ शुद्धिर्विकलावशेषम्}}~। इत्यादावुपपत्तिः~। अत्र {\qt 'द्युचरचक्रहतो दिन-सञ्चयः'} इत्यादिना ग्रहानयनेऽहर्गणः १३ कल्पभगणैः ९ गुणितः ११७
कल्पकुदिनैः 
१९ भक्तो लब्धं गतभगणाः ६ शेषं भगणशेषं २ तद्द्वादशगुणं ३६
कुदिनैर्भक्तं 
लब्धं १ राशयः~। शेषं राशिशेषं १७ तत्त्रिंशता सङ्गुण्य ५१०
कुदिनैर्भक्तं लब्धमंशाः 
२६ शेषमंशशेषं १६ तत् षष्ट्या सङ्गुण्य ९६० कुदिनैर्भक्तं लब्धं कलाः
५० शेषं
\newpage
%%%%%%%%%%%%%%%%%%%%%%%%%%%%%%%%%%
\noindent कलाशेषं १० तत्पुनः षष्ट्या सङ्गुण्य ६०० कुदिनैर्भक्तं लब्धं विकलाः ३१ 
शेषं विकलाशेषं ११~। \\

\vspace{-4mm}
 अथ व्यस्तविधिना विकलाशेषाद्ग्रहानयनम्~। तत्र युक्तिः~। अत्र कलाशेषे 
१० षष्ट्या गुणिते ६०० कुदिनभक्ते यच्छेषं तद्विकलाशेषम् ११~।
तच्चेत् षष्टिगुणितात् कलाशेषात् अपनीयते ५८९ तदा तत्कुदिनभक्तं निःशेषं स्यात्~। लब्धिश्च
विकलाः स्युः~। परमत्र कलाशेषस्याज्ञाने षष्टिगुणितस्य सुतरामज्ञानादुक्तविधिर्न
सिध्यति~। अत्र षष्ट्या गुणिते कलाशेषं कलाशेषेण वा गुणिता षष्टिः समैव~।
गुण्यगुणकयोरभेदात्~। 
तस्मात्षष्टिकलाशेषेण गुणिता विकलाशेषेणोना कुदिनभक्ता निःशेषा
स्याल्लब्धिस्तु विकलाः स्युः~। प्रकृते षष्टिर्विकलाशेषं च ज्ञायते~। केवलं कलाशेषं न
ज्ञायते~। तज्ज्ञानार्थमुपायः~। षष्टिर्येन गुणिता सती विकलाशेषेणोना कुदिनभक्ता
निःशेषा भवेत्तदेव कलाशेषं स्यात्~। अयम् अर्थश्च कुट्टकस्य विषयः~। षष्टिः केन गुणिता
विकलाशेषेण रहिता कुदिनभक्ता निःशेषा स्यादिति प्रश्ने पर्यवसानात्~। अत्र
यो गुणस्तदेव कलाशेषम् उक्तयुक्तेः~। या लब्धिस्ता विकला उक्तयुक्तेरेव~। अत
उपपन्नम् \hyperref[67]{\textbf{कल्प्याथ शुद्धिर्विकलावशेषं षष्टिश्च भाज्यः कुदिनानि हारः~। तज्जं फलं 
स्युर्विकला गुणस्तु लिप्ताग्रम्}} इति~। अथ कला-शेषात् कलाज्ञानम्~। तत्र
भागशेषे षष्ट्या गुणिते कुदिनैर्भक्ते लब्धिः कला भवन्ति~। शेषं च कलाशेषम्~। अत 
उक्तयुक्त्या षष्टिर्भागशेषेण गुणिता कलाशेषेणोना कुदिनभक्ता निःशेषा
स्याल्लब्धिश्च 
कलाः स्युः~। तत्र भागशेषरूपस्य गुणकस्याज्ञानादयमर्थः कुट्टकस्यैव विषयः~। 
षष्टिः केन गुणिता कलाशेषेणोना कुदिनभक्ता निःशेषा स्यादिति प्रश्ने
पर्यवसानात्~। 
अत्र यो गुणः स एव भागशेषम्~। या लब्धिस्ताः कलाः~। उक्तमस्माच्च
कलालवाग्रमिति~। अथ भागशेषाद्भागज्ञानम्~। तत्र राशिशेषे त्रिंशता गुणिते
कुदिनैर्भक्ते लब्धिरंशा भवन्ति~। शेषं च भागशेषम्~।
अत्राप्युक्तयुक्त्यैव त्रिंशत्केन 
गुणिता भागशेषोणाः कुदिनैर्भक्ता निःशेषाः स्युरिति कुट्टकविषयतास्ति~।
अत्र यो 
गुणः स एव राशिशेषः स्याद्या लब्धिस्त एव भागाः स्युः~। अथ राशिशेषाद्राशिज्ञानम्~। तत्र भगणशेषे द्वादशगुणिते कुदिनैर्भक्ते लब्धी राशयः~। शेषं च
राशिशेषम्~। अत्रापि द्वादश न गुणिता राशिशेषोनाः कुदिनैर्भक्ता
निःशेषाः स्युरिति 
कुट्टकविषयतास्ति~। अत्र यो गुणस्तदेव भगणशेषं या लब्धिस्त एव गतराशयः 
स्युः~। अथ भगणशेषाद्गतभगणाहर्गणयोर्ज्ञानम्~। तत्र कल्पभगणा
अहर्गणगुणिताः
\newpage
%%%%%%%%%%%%%%%%%%%%%%%%%%%%%%%%%%%%%%%%%
\noindent कुदिनैर्भक्ता लब्धिर्गतभगणा भवन्ति~। शेषं च भगणशेषम्~। अतोऽत्रापि स कल्पभगणाः केन गुणिता भगणशेषोनाः कुदिनैर्भक्ता निरग्रकाः स्युरिति
कुट्टकविषयतास्ति~। अत्र यो गुणः स एवाहर्गणः~। या लब्धिस्त एव गतभगणाः~। उक्तम् \hyperref[67]{\textbf{एवं तदूर्ध्वं च}} इति~। एवं कल्पसौरदिवसैः कल्पाधिमासास्तदेष्टसौरैः कियन्त 
इति त्रैराशिकेन कल्पाधिमासेष्विष्टसौरदिवसैर्गुणितेषु
कल्पसौरैर्भक्तेषु या लब्धिस्ते
 गताधिमासाः~। यच्छेषं तदधिमासशेषम्~। अतोऽत्रापि कल्पाधिमासाः कैर्गुणिता
 अधिमासशेषोनाः कल्पसौरदिनैर्भक्ता निःशेषाः स्युरित्यस्ति कुट्टकविषयता~।
अत्र  यो गुणस्त एवेष्टसौरदिवसाः~। या लब्धिस्त एव गताधिमासाः~। एवं
कल्पचान्द्रैः
 कल्पावमानि तदेष्टचान्द्रैः कियन्तीत्यनुपातेन
कल्पावमेष्विष्टचान्द्रैर्गुणितेषु कल्पचान्द्रैर्भक्तेषु
 या लब्धिस्तानि गतावमानि भवन्ति~। शेषं चावमाग्रम्~।
अतोऽवमाग्राद्व्यस्तविधिना गतावमचान्द्राणामानयनमुक्तयुक्त्या कुट्टकेन सिध्येदेव~। अत उक्तम् \hyperref[67]{\textbf{तथाधिमासावमाग्रकाभ्यां दिवसा रवीन्द्वोः}} इति~। अत्रेदमवधेयम्~। विकलाशेषाद्ग्रहानयने
 विकलाशेषमृणक्षेपः षष्टिर्भाज्यः कल्पकुदिनानि हार इति प्रकल्प्य
कुट्टकेन यौ
 लब्धिगुणौ ताविष्टहतस्वस्वहरेण युक्तौ न विधेयौ~। योजने~। योजने हि
लब्धिः
 षष्टितोऽधिका स्याद्गुणश्च कुदिनतोऽधिकः स्यात्~। न चैतत् सम्भवति~। यतो
लब्धिर्विकला \,गुणश्च \,कलाशेषम्~। नहि \,विकलाः \,षष्टितोऽधिकाः \,सम्भवन्ति~। न \,वा
कलाशेषं
 कुदिनतोऽधिकं सम्भवति~। कुदिनानां हरत्वात्~। अनयैव युक्त्या
भगणशेषपर्यन्तं
 गुणलब्ध्योः क्षेपो न देयः~। भगणशेषाद्गतभगणाहर्गणयोरानयने तु
क्षेपदाने यत्र
 बाधकं न स्यात्तत्र तादृशः क्षेपो देयः~। तस्माद्विकलाशेषाद्ग्रहानयने
राश्यादिर्ग्रहो
 नियत एव~। गतभगणाहर्गणयोस्त्वनियतत्वमिति सिद्धम्~।
एवमधिमासावमाग्राभ्यां
 सौरचान्द्रदिनानयनेऽप्यनियतत्वम्~। मतिमद्भिरन्यदप्यूह्यम्~। अलं
पल्लवितेन~॥~६७~॥~\\

\vspace{-4mm}
 एवमेकस्मिन् गुणके सति राशिज्ञानम् अभिधायाथ द्व्यादिषु गुणेषु सत्सु
राशिज्ञानम् उपजात्याह\textendash

\phantomsection \label{68}
\begin{quote}
\ab
     एको हरश्चेद्गुणकौ विभिन्नौ तदा गुणैक्यं परिकल्प्य भाज्यम्~। \\
 अग्रैक्यमग्रं कृत उक्तवद्यः संश्लिष्टसञ्ज्ञः स्फुटकुट्टकोऽसौ~॥~६८~॥~
\end{quote}

 चेदेको हरः स्याद्गुणकौ तु विभिन्नौ स्तः~। गुणकावित्युपलक्षम्~। तेन
\newpage
%%%%%%%%%%%%%%%%%%%%%%%%%%%%%%%%%%
\noindent त्र्यादयो वा गुणकाः स्युः~। एकस्यैव राशेः पृथक्पृथग्द्वौ गुणकौ
त्रयश्चतुरादयो वा गुणकाः स्युः~। सर्वत्र हरस्त्वेक एव स्यात् तदा तेषां द्व्यादीनां
गुणकानामैक्यं भाज्यं परिकल्प्यो-द्दिष्टं यदग्रैक्यं तदग्रमृणक्षेपं
प्रकल्प्यार्थाद्धरमेव हरं प्रकल्प्योक्तवद्यः
 कृतः स्फुटकुट्टकोऽसौ संश्लिष्टसञ्ज्ञः स्यात्~।
संश्लिष्टस्फुटकुट्टकः~। अन्वर्थसञ्ज्ञेयम्~।
 तथाहि\textendash \,कुट्टको गुणकः~। संश्लिष्टानामेकीभूतानामग्राणां सम्बन्धी स्फुटो विविक्तः
 कुट्टकः संश्लिष्टकुट्टकः~। स एव राशिः स्यादित्यर्थात्सिद्धम्~। अत्र
लब्धिर्न ग्राह्या~।
 अत्र हि यथोद्दिष्टैर्गुणकैः च पृथग्गुणिते राशौ हरतष्टे सति या आगता
लब्धयस्तदग्राणां 
 चैक्ये हरतष्टे सति या लब्धयस्तासामैक्यं तदत्र कुट्टके
लब्धिरूपमुत्पद्यते~।
 प्रयोजनाभावात्तन्न ग्राह्यम्~। अत्रोपपत्तिः~। यथा गुण्यं भाज्यं
कल्पयित्वा कुट्टकेन
 गुणकः सिध्यति तथा गुणकं भाज्यं प्रकल्प्य कुट्टकेन यो गुणः स गुण्य एव
सिध्यति। अत एव पूर्वसूत्रे \hyperref[67]{\textbf{षष्टिश्च भाज्यः}} इत्याद्युक्तम्~। तत्र
यथैकेन गुणकेन
 गुणितो राशिर्हरभक्तो यच्छेषं तेनोनितः स हरभक्तः शुध्यति तथान्यैरपि
गुणकैः
 पृथक्पृथग्गुणितो हरभक्तो यानि  शेषाणि तैर्यथास्वं रहितो हरभक्तः
शुध्येदेव~। युक्तेस्तुल्यत्वात्~। तत्र सर्वत्र यद्येक एव हर स्यात्तर्हि यथा
पृथग्गुणितः स्वस्वशेषोनो
 हरभक्तः शुध्यति तथा पृथग्गुणितो युक्तश्च शेषैक्येनोनो हरभक्तः
शुध्येदेव~। तत्र
 गुणकैः पृथग्गुणितो युक्तश्चेद्गुणकयोगेनैव गुणितः स्यात्~। अतो
गुणकयोग
एवात्र गुणः~। शेषयोग एव शेषम्~। यथा दश १० द्व्यादिभिः २~। ३~। ४
गुणिताः २०~। ३०~। ४०~। हर\textendash \,१९\textendash \,भक्ताः पृथक्पृथक् लब्धयः १~। १~। २ शेषाणि च
१~। ११~। २~। एतैर्यथास्वमूनाः १९~। १९~। ३८ हरभक्ताः शुध्यन्ति~। एवं
गुणैक्येन ९
 गुणिता दश ९० शेषैक्येन १४ रहिता ७६ एकोनविंशत्या भक्ताः शुध्यन्ति~।
 लब्धिश्च लब्धियोग एव ४~। अतो गुणकयोगस्य गुणकत्वाद्गुणकयोगो भाज्यः~।
 अग्रैक्यं शुद्धिर्हर एव हरः~। अत्र कुट्टके यो गुणः सिध्येत्स
गुण्यराशिरेवेत्युपपन्नम् \hyperref[68]{\textbf{एको हरश्चेद्गुणकौ विभिन्नौ}} इत्यादि~॥~६८~॥~\\
 
\vspace{-4mm}
 अत्रोदाहरणमुपजात्याह\textendash
\begin{quote}
    \eg 
     कः पञ्चनिघ्नो विहृतस्त्रिषष्ट्या सप्तावशेषोऽथ स एव राशिः~। \\
 दशाहतः स्याद्विहृतास्त्रिषष्ट्या चतुर्दशाग्रो वद राशिमानम्~॥~६९~॥
\end{quote}
\newpage
%%%%%%%%%%%%%%%%%%%%%%%%%%%%%%%%%%%%%%%%%%%%%%%%%%%%%%%%%

\noindent स्पष्टोऽर्थः~। अत्रोक्तवन्न्यासो $\begin{matrix}
\vspace{-1mm}
\mbox{{भा १५ क्षे २ं१}}\\
\vspace{-1mm}
\mbox{{ह ६३ ~~~~~}}
\vspace{1mm}
\end{matrix}$~। पूर्ववज्जातो गुणः
१४~। अयमेव 
राशिः~। अन्यदप्युदाहरणं गोलाध्याये {\qt 'ये याताधिकमासहीनदिवसाः'} इति~।
बहुगुणकोदाहरणम् अपि तत्रैव {\qt 'चक्राग्राणि ग्रहाग्रकाणि'} इत्यादिश्लोकद्वयेन, अत्र
भगणराश्यादीनां 
शेषेष्वहर्ग-णस्य क्रमेण गुणकाः~। कल्पभगणाः १ द्वादश गुणास्ते २
षष्ट्यधिकशतत्रय\textendash \,३६०\textendash \,गुणास्ते ३ खखनृपाक्षि\textendash \,२१६००\textendash \,गुणास्ते ४ खखखतर्कनन्दतरणि\textendash \,१२९६०००\textendash \,गुणास्ते ५~। एवमन्येऽपि गुणका ऊह्याः~। अत्र गुणैक्यं भाज्यं प्रकल्प्य यो गुणः सिध्येत्स एवाहर्गणः~॥~६९~॥

\begin{quote}
 {\qt दैवज्ञवर्यगणसन्ततसेव्यपार्श्वबल्लालसञ्ज्ञगणकात्मजनिर्मितेऽस्मिन्~। \\
बीजक्रियाविवृतिकल्पलतावतारे युक्तेर्विविक्तिरिति कुट्टकसिद्धिहेतोः~॥}
\end{quote}

\begin{center}
     इति श्रीसकलगणकसार्वभौमश्रीबल्लालदैवज्ञसुतश्रीकृष्णदैवज्ञविरचिते \\
 बीजविवृतिकल्पलतावतारे कुट्टकविवरणम्~॥~५~॥~\\
 अत्र मूलश्लोकैः सह सङ्ख्या ८००~।\\
\vspace{1cm}
\rule{0.2\linewidth}{0.5pt}
\end{center}
\newpage
%%%%%%%%%%%%%%%%%%%%%%%%%%%%%%%%%%%%%%%%%%
\thispagestyle{empty}
\phantomsection \label{ch6}
\begin{center}
{\LARGE \textbf{६ वर्गप्रकृतिः~।}}
\end{center}

 एवमनेकवर्णप्रक्रियोपयुक्तं
कुट्टकमुक्त्वेदानीमनेकवर्णमध्यमाहरणोपयुक्तां वर्गप्रकृतिं निरूपयति~। तत्र प्रथमतस्तत्स्वरूपं शालिन्याह\textendash

\phantomsection \label{70}
\begin{quote}
    \ab 
    इष्टं ह्रस्वं तस्य वर्गः प्रकृत्या क्षुण्णो युक्तो वर्जितो वा स येन~। \\
मूलं दद्यात्क्षेपकं तं धनर्णं मूलं तच्च ज्येष्ठमूलं वदन्ति~॥~७०~॥~
\end{quote}

 अनेकवर्णमध्यमाहरणे पक्षयोः समीकरणानन्तरमेकपक्षस्य पदे गृहीते सति 
द्वितीयपक्षे यदि सरूपोऽव्यक्तवर्गः स्यात्~। यथा\textendash \,काव १२ रू १~। तत्र
पूर्वपक्षतुल्यतया द्वितीयपक्षेणापि मूलदेन भाव्यम्~। अस्ति चात्र कालकवर्गो
द्वादशगुणः 
सरूपश्च~। अतो यस्य वर्गो द्वादशगुणः सरूपः सन्वर्गो भवेत्तदेव
कालकमानमित्यर्थात्सिध्यति~। यच्चात्र पदं तत्पूर्वपक्षपदसममुभयपक्षयोस्तुल्यत्वात्~।
सविस्तरं तु 
तत्रैव प्रतिपादयिष्यते~। वर्गः प्रकृतिर्यत्रेति वर्गप्रकृतिः~। यतोऽस्य
गणितस्य 
यावदादिवर्गः प्रकृतिः~। यद्वा यावदादिवर्गेषु प्रकृतिभूतादङ्कादिदं गणितं
प्रवर्तत 
इति वर्गप्रकृतिः~। अत्र यावद्वर्गादिषु प्रकृतिभूतो योऽङ्कः स
प्रकृतिशब्देनोच्यते~। 
स चाव्यक्तवर्गगुणक एव~। अतोऽत्र पदसाधने वर्गस्य यो गुणः स प्रकृतिशब्देन
व्यवह्रियते~। आदाविष्टं पदं प्रकल्प्य तस्य वर्गः प्रकृतिगुणो येनाङ्केन
युक्तो 
वर्जितो वा मूलं दद्यात्तमङ्कं क्रमेण धनमृणं च क्षेपकं वदन्त्याचार्याः~।
तन्मूलं 
ज्येष्ठमूलमिति वदन्ति~। प्रथमतो यदिष्टं पदं प्रकल्पितं तच्च ह्रस्वमिति
वदन्ति~। 
अन्वर्थांश्चैताः सञ्ज्ञाः~। यत्र तु क्षेपवियोगात्कुत्रचिज्ज्येष्ठपदं
ह्रस्वपदादल्पं भवति 
तत्रापि भावनया ह्रस्वपदादधिकमेव भवति~॥~७०~॥~\\

\vspace{-4mm}
 एवमेकेषु ह्रस्वज्येष्ठक्षेपकेषु ज्ञातेष्वनेकत्वार्थमुपायं शालिनीत्रयेणाह\textendash 
\newpage
%%%%%%%%%%%%%%%%%%%%%%%%%
 \fancyhead[CE] {बीजगणिते}
\fancyhead[CO]{वर्गप्रकृतिः}
\fancyhead[LE,RO]{\thepage}
\cfoot{}

\phantomsection \label{71}
 \begin{quote}
     \ab 
      ह्रस्वज्येष्ठक्षेपकान्न्यस्य तेषां तानन्यान्वाधो निवेश्य क्रमेण~।\\
 साध्यान्येभ्यो भावनाभिर्बहूनि मूलान्येषां भावना प्रोच्यतेऽतः~॥~\\
 वज्राभ्यासो ज्येष्ठलघ्वोस्तदैक्यं ह्रस्वं लघ्वोराहतिश्च प्रकृत्या~। \\
 क्षुण्णा ज्येष्ठाभ्यासयुग्ज्येष्ठमूलं तत्राभ्यासः क्षेपयोः क्षेपकः स्यात्~॥~\\
 ह्रस्वं वज्राभ्यासयोरन्तरं वा लघ्वोर्घातो यः प्रकृत्या विनिघ्नः~। \\
 घातो यश्च ज्येष्ठयोस्तद्वियोगो ज्येष्ठं क्षेपोऽत्रापि च क्षेपघातः~॥~७१~॥~
 \end{quote}

प्रथमसिद्धान् ह्रस्वज्येष्ठक्षेपकान्पङ्क्तौ विन्यस्य \hyperref[71]{\textbf{तेषामधस्तानन्यान्वा ह्रस्वज्येष्ठक्षेपकान्}} क्रमेण निवेश्य तेभ्यः पङ्क्तिद्वयस्थापितेभ्यो ह्रस्वज्येष्ठक्षेपकेभ्यो यतो भावनाभिर्बहूनि मूलानि साध्यान्यत एषां भावना प्रोच्यते~। अन्यान्वेत्यत्र तस्यामेव प्रकृताविति ज्ञेयम्~। यद्यपि भावनाभिः क्षेपा अपि बहवो भवन्ति तथापि नास्ति नियमः~। रूपक्षेपपदजासु भावनासु व्यभिचारात्~। अतः क्षेपा बहवः साध्या इति नोक्तम्~। इष्टक्षेपे सिद्धे तेषामनुद्देश्यत्वाच्च~। तत्र भावना द्विविधा~। समासभावनान्तरभावना चेति~। तत्र पदयोर्महत्त्वेऽपेक्षिते समासभावनामाह\textendash \,\hyperref[71]{\textbf{वज्राभ्यासो ज्येष्ठलघ्वोः}} इत्यादिना~। ज्येष्ठलघ्वोर्यौ वज्राभ्यासौ तदैक्यं ह्रस्वं
स्यात्~। वज्राभ्यासो नाम तिर्यग्गुणनम्~। वज्रस्य तिर्यक्प्रहारस्वभावत्वात्~।
तस्मादूर्ध्वकनिष्ठेनाधःस्थं ज्येष्ठं गुणनीयमधःस्थकनिष्ठेनोर्ध्वस्थं ज्येष्ठं
गुणनीयम्~। तयोरैक्यं ह्रस्वं स्यात्~। लघ्वोराहतिः प्रकृत्या गुणिता ज्येष्ठयोर्वधेन युक्ता
ज्येष्ठमूलं स्यात्~। क्षेपयोरभ्यासः क्षेपकः स्यादिति~। \\

\vspace{-4mm}
 अथ पदयोर्लघुत्वेऽभीप्सितेऽन्तरभावनाम् आह\textendash \,\hyperref[71]{\textbf{ह्रस्वं वज्राभ्यासयोरन्तरं वा}} इति~। 
वज्राभ्यासयोरन्तरं वा ह्रस्वं स्यात्~। ऐक्यापेक्षया विकल्पः~। अत्र  यः
प्रकृत्या गुणितो लघ्वोर्घातो यश्च केवलो ज्येष्ठयोर्घातस्तद्वियोगो ज्येष्ठं स्यात्~। अत्रापि 
क्षेपघातः क्षेपकः स्यात्पूर्ववदेव~। अत्र प्रथमसूत्रोपपत्तिः स्पष्टतरा~।
अथ भावनोपपत्तिरुच्यते~। तत्रासङ्क-रार्थमाद्यद्वितीयादिपदप्रथमाक्षरोपलक्षणपूर्वकं
बीजक्रिया लिख्यते~। 
यथा\textendash \,कनिष्ठज्येष्ठ-क्षेपाणां पङ्क्त्योर्न्यासः $\begin{matrix}
\vspace{-1mm}
\mbox{{आक १ ~आज्ये १ ~आक्षे १}}\\
\vspace{-1mm}
\mbox{{द्विक १ ~द्विज्ये १ ~द्विक्षे १}}
\vspace{1mm}
\end{matrix}$ अथ \hyperref[72]{\textbf{इष्टवर्गहृतः क्षेपः क्षेपः स्यात्}} इति वक्ष्यमाणसूत्रोक्तेन \hyperref[72]{\textbf{क्षेपः क्षुण्णः क्षुण्णे तदा}}
\newpage
%%%%%%%%%%%%%%%%%%%%%%%%%%%%%%%%%%

\noindent \hyperref[72]{\textbf{पदे}} इत्यनेन प्रकारेण परस्परज्येष्ठमिष्टं प्रकल्प्य पङ्क्त्योर्जाताः कनिष्ठज्येष्ठक्षेपाः~। 
\vspace{-2mm}

\begin{table}[h!]
    \centering\s
    \begin{tabular}{rp{0.1cm}rp{0.1cm}r}
          द्विज्ये ० आक १&& द्विज्ये ० आज्ये १&& द्विज्येव ० आक्षे १~। \\
 आज्ये ० द्विक १&& द्विज्ये ० आज्ये १&& आज्येव ० द्विक्षे १~। 
    \end{tabular}
\end{table}
\vspace{-2mm}

 अत्रोर्ध्वपङ्क्तौ \,द्वितीयज्येष्ठवर्गगुणित \,आद्यक्षेपोऽस्ति~। तत्र \,द्वितीयज्येष्ठवर्गोऽन्यथा 
साध्यते~। द्वितीयकनिष्ठवर्गः प्रकृतिगुणो द्वितीयक्षेपयुतो जातो
द्वितीयज्येष्ठवर्गः~। 
द्विक प्र १ द्विक्षे १ अनेन गुणित आद्यक्षेपो जातः खण्डद्वयात्मकः
क्षेपः~। द्विकव ० प्र ० आक्षे १ द्विक्षे ० आक्षे १~। अत्र प्रथमखण्ड आद्यक्षेपोऽन्यथा साध्यते~। ज्येष्ठवर्गे हि 
खण्डद्वयम् अस्ति~। प्रकृतिगुणः कनिष्ठवर्ग एकम्~। क्षेपोऽपरम्~। तत्र
ज्येष्ठवर्गात्प्रकृतिगुणे 
कनिष्ठवर्गे शोधिते क्षेप एवावशिष्यते~। अत आद्यकनिष्ठवर्गः प्रकृतिगुण 
आद्यज्येष्ठवर्गादपनीतो जात आद्यः क्षेपः~। आकव ० प्र १ं आज्येव १~। अयं
प्रकृतिगुणेन द्वितीयकनिष्ठवर्गेण गुणितः सन्प्रकृतक्षेपाद्मखण्डं भवेदिति
जातमाद्यं खण्डं 
खण्डद्वयात्मकम्~। द्विकव ० प्र ० आकव प्र? द्विकव ० प्र आज्येव?~। \\
\vspace{-4mm}

 अत्र प्रथमखण्डे प्रकृत्या वारद्वयं गुणनाज्जातं प्रकृतिवर्गेण गुणनम्~।
तथा सति जातं प्रथमखण्डम्~। द्विकव ० आकव ० प्रव १ं~। एवमूर्ध्वपङ्क्तौ जातः
खण्डत्रयात्मकः क्षेपः~। \\
\vspace{-4mm}

\hspace{4mm} द्विकव ० आकव ० प्रव १ं~। द्विकव ० प्र आज्येव १ द्विक्षे ० आक्षे १~। \\
\vspace{-4mm}

अनयैव युक्त्या द्वितीयपङ्क्तावपि जातः खण्डत्रयात्मकः क्षेपः~। \\
\vspace{-4mm}

\hspace{4mm} द्विकव ० आकव ० प्रव १ं आकव ० प्र ० द्विज्येव १ द्विक्षे ० आक्षे १~। \\
\vspace{-4mm}

एवं पङ्क्तिद्वये जाता कनिष्ठज्येष्ठक्षेपाः~। 
\vspace{-2mm}

\begin{table}[h!]
    \centering\s
    \begin{tabular}{rp{0.1cm}r}
        द्विज्ये ० आक १ &&द्विज्ये ० आज्ये १ \\
 आज्ये ० द्विक १ &&द्विज्ये ० आज्ये १ 
    \end{tabular}
\end{table}
\vspace{-2mm}

\hspace{4mm} द्विकव ० आकव ० प्रव १ं द्विकव ० प्र ० आज्येव १ द्विक्षे ० आक्षे १ \\
\vspace{-5mm}

\hspace{4mm} द्विकव ० आकव ० प्रव १ं आकव ० प्र ० द्विज्येव १ द्विक्षे ० आक्षे १ \\
\vspace{-4mm}

 अत्र ज्येष्ठलघ्वोरेकोऽभ्यास ऊर्ध्वपङ्क्तौ कनिष्ठम्~। अपरोऽभ्यासो
द्वितीयपङ्क्तौ कनिष्ठम्~। ज्येष्ठं तूभयत्र ज्येष्ठाभ्यासरूपमेकमेव~। अत्र प्रत्येकं
वज्राभ्यासस्य कनिष्ठत्वक- 
\newpage
%%%%%%%%%%%%%%%%%%%%%%%%%%%%%%%%%%%%%
\noindent ल्पने \,क्षेपो \,महान् \,स्यादित्याचार्यैरन्यथा \,यतितम्~। तद्यथा~। वज्राभ्यासयोगः कनिष्ठं कल्पितम्~। द्विज्ये ० आक १ आज्ये ० द्विक १~। अस्य वर्गः~। द्विज्येव ० आकव १ द्विज्ये ० आक ० आज्ये ० द्विक २ आज्येव ० द्विकव १~। प्रकृतिगुणः~। द्विज्येव ० आकव ० प्र १ द्विज्ये ० आक ० आज्ये ० द्विक ० प्र २ आज्येव ० द्विकव ० प्र १~। \\

\vspace{-4mm}
 अयं केन क्षेपेण युतः सन्मूलदः स्यादिति विचार्याते~। तत्रास्य
खण्डद्वयम्~। एकैकवज्राभ्यासज्येष्ठवर्गतुल्यमेकम्~। शेषमपरं तत्र कनिष्ठवर्गः
प्रकृतिगुणाः क्षेपयुतो ज्येष्ठवर्गः स्यादिति जातौ पङ्क्तिद्वये ज्येष्ठवर्गौ~।\\ 

\vspace{-2mm}
\noindent {\small द्विज्येव ० आकव ० प्र १ द्विकव ० आकव ० प्रव १ं द्विकव ० प्र ० आज्येव १ द्विक्षे ० अक्षे १~। \\

\vspace{-4mm}
\noindent आज्येव ० द्विकव ० प्र १ द्विकव ० आकव ० प्रव १ं आकव ० प्र ० द्विज्येव १ द्विक्षे ० अक्षे १~।} \\

\vspace{-2mm}
 पङ्क्तिद्वयेऽपि ज्येष्ठाभ्यासलक्षणस्य ज्येष्ठस्य तुल्यत्वादेतौ
ज्येष्ठवर्गावपि तुल्यावेव~। तृतीयोऽयमपि~। द्विज्येव ० आज्येव १~। अथ
वज्राभ्यासयोगरूपकल्पितकनिष्ठस्य वर्गात् प्रकृतिगुणादस्मात्~। द्विज्येव ० आकव ० प्र १ द्विज्ये ० आक ० आज्ये ० द्विक ० प्र २ आज्येव ० द्विकव ० प्र १~। ज्येष्ठवर्गद्वयेऽपि पृथक्पृथगपनीते शेषं तुल्यमेव~। द्विज्ये ० आक ० आज्ये ० द्विक ० प्र २ आकव ० द्विकव ० प्रव १ आक्षे ० द्विक्षे १ं~। \\

\vspace{-4mm}
 इदं शोधितेन ज्येष्ठवर्गेण पुनर्यदि योज्येते तर्हि कल्पितकनिष्ठवर्गः प्रकृतिगुणो यथा स्थितः स्यात्~। अथायमपि ज्येष्ठवर्गः~। द्विज्येव ० आज्येव १ शोधितेन सम इति~। अनेन योगे जातः कल्पितकनिष्ठवर्गप्रकृतिगुणः~। द्विज्येव ० आज्येव १ द्विज्ये ० आक ० आज्ये ० द्विक ० प्र २ आकव ० द्विकव ० प्रव १ आक्षे ० द्विक्षे १ं~। \\

\vspace{-4mm}
 अस्मात्क्षेपघातेन युक्तात् \hyperref[31]{\textbf{कृतिभ्य आदाय पदानि}} इत्यादिना पदमिदं 
द्विज्ये ० आज्ये १ आक ० द्विक ० प्र १ लभ्यत इत्युपपन्नं \hyperref[71]{\textbf{लघ्वोराहतिश्च प्रकृत्या क्षुण्णा ज्येष्ठाभ्यासयुग्ज्येष्ठमूल}}मिति~। एवं\,\,वज्राभ्यासयोरन्तरं\,कनिष्ठं\,\,प्रकल्प्योक्तयुक्त्यान्तरभावनोपपत्तिरपि\,\,द्रष्टव्या~। एवं खण्डक्षोदेन बहुविधा
उपपत्तयः सन्ति~। ग्रन्थविस्तरभयान्न लिख्यन्ते~॥~७१~॥


\end{document}