\documentclass[11pt, openany]{book}
\usepackage[text={4.65in,7.45in}, centering, includefoot]{geometry}

\usepackage[table, x11names]{xcolor}
\usepackage{fontspec,realscripts}
\usepackage{polyglossia}

\usepackage{enumerate}
\pagestyle{plain}
\usepackage{fancyhdr}
\pagestyle{fancy}
\renewcommand{\headrulewidth}{0pt}
\usepackage{afterpage}
\usepackage{multirow}
\usepackage{amsmath}
\usepackage{amssymb}
\usepackage{graphicx}
\usepackage{longtable}
\usepackage{footnote}
\usepackage{perpage}
\MakePerPage{footnote}
%\usepackage{bigfoot}
%\DeclareNewFootnote[para]{default}
%\usepackage{dblfnote}
\usepackage{xspace}
%\newcommand\nd{\textsuperscript{nd}\xspace}
\usepackage{array}
\usepackage{emptypage}

\usepackage{hyperref}% Package for hyperlinks
\hypersetup{
colorlinks,
citecolor=black,
filecolor=black,
linkcolor=blue,
urlcolor=black
}

\usepackage[Devanagari, Latin]{ucharclasses}

\setdefaultlanguage{english}
\setotherlanguage{sanskrit}
\setmainfont[Scale=1]{Times New Roman}

\newfontfamily\s[Script=Devanagari, Scale=0.9]{Shobhika}
\newfontfamily\regular{Times New Roman}

\newcommand{\devanagarinumeral}[1]{%
	\devanagaridigits{\number \csname c@#1\endcsname}} % for devanagari page numbers

\setTransitionTo{Devanagari}{\s}
\setTransitionFrom{Devanagari}{\regular}

\XeTeXgenerateactualtext=1 % for searchable pdf

\begin{document}
\thispagestyle{empty}

\begin{center}
{\LARGE \textbf{आनन्दाश्रमसंस्कृतग्रन्थावलिः~।} }\\
\vspace{-1mm}
\rule{0.5\linewidth}{0.5pt}\\
\vspace{2mm}

{\large \textbf{ग्रन्थाङ्कः ९९}}
\vspace{2mm}

\textbf{\huge भास्करीयबीजगणितम्~।}\\
\vspace{-1mm}
\rule{0.3\linewidth}{0.7pt}\\
\vspace{2mm}

{\large \textbf{कृष्णदैवज्ञविरचितनवाङ्कुरव्याख्यासहितम्~। }}
\vspace{2mm}

एतत्पुस्तकं
\vspace{2mm}

{\large \textbf{आपटेकुलोत्पन्नेन विष्णुसूनुना दत्तात्रेयेणानन्दा-}}
\vspace{2mm}

{\large \textbf{श्रमस्थपण्डितसाहाय्येन संशोधितम्~।}}
\vspace{2mm}

तच्च
\vspace{2mm}

बी. ए. इत्युपपदधारिभिः
\vspace{2mm}

{\Large \textbf{विनायक गणेश आपटे}}
\vspace{2mm}

इत्येतैः
\vspace{2mm}

पुण्याख्यपत्तने
\vspace{2mm}

{\large \textbf{महादेव चिमणाजी आपटे}}
\vspace{2mm}

\textbf{इत्यभिधेयमहाभागप्रतिष्ठापिते}
\vspace{2mm}

{\LARGE \textbf{आनन्दाश्रममुद्रणालये}}
\vspace{2mm}

आयसाक्षरैर्मुद्रयित्वा
\vspace{2mm}

प्रकाशितम्~।\\
\vspace{-2mm}
\rule{0.1\linewidth}{0.5pt}\\
\vspace{2mm}

शालिवाहनशकाब्दाः १८५२
\vspace{2mm}

ख्रिस्ताब्दाः १९३०\\
\vspace{-2mm}
\rule{0.1\linewidth}{0.5pt}\\
\vspace{2mm}

(अस्य सर्वेऽधिकारा राजशासनानुसारेण स्वायत्तीकृताः )~। 
\vspace{2mm}

मूल्यं रूपकद्वयम् (२)~।
\end{center}

\newpage
\thispagestyle{empty}

\begin{center}
{\Large \textbf{प्रास्ताविकं किञ्चित्~।}}\\
\rule{0.2\linewidth}{0.5pt}
\end{center}

\begin{sloppypar}
अथैतद्विदांकुर्वन् त्वत्र भवन्त आनन्दाश्रमसंस्कृतग्रन्थावलीग्राहका अनुग्राहका महा-भागाः\;। यद्बीजगणितं नाम गणितग्रन्थशिरोमणेः सिद्धान्तशिरोमणेः द्वितीयोऽध्याय उच्यत इति~। सोऽयं सिद्धान्तशिरोमणिर्भारतवर्षीयसुप्रथितकुशाग्रधिषणज्योतिर्विद्वर-भास्कराचार्यैर्व्यरचि~। त इमे भास्कराचार्याः स्वजनुषा सिद्धान्तशिरोमणिग्रन्थावतारेण च महीमण्डलीं कदालमकार्षुरिति जिज्ञासायां स्वस्वीयग्रन्थावतारकालनिर्णयार्थं स्वग्रन्थे प्रथमे गोलापरपर्याये प्रश्नाध्याये स्वैरेवोट्टङ्कितं सर्वोत्कृष्टं साधनमुपलभ्यते~। तद्यथा\textendash 

\begin{quote}
{\color{violet}"रसगुणपूर्णमही\textendash \,(१०३६)\textendash \,समशकनृपसमयेऽभवन्ममोत्पत्तिः~।\\
रसगुण\textendash \,(३६)\textendash \,वर्षेण मया सिद्धान्तशिरोमणी रचितः~॥} इति~।
\end{quote}

एतावता प्रबन्धेन~। नयनर्षिखेन्दु\textendash \,(१०७२)\textendash \,मिते शाकेऽसौ सिद्धान्तशिरोमणिरात्मा-नमाससादेति निश्चिते सति पूर्वनिर्दिष्टश्लोकान्तर्गतो मयेति शब्दो भास्कराचार्यानेव स्पृश-तीति साधयितुं बीजगणिताभिधे द्वितीयेऽध्याये प्रमाणं दृश्यते~। तद्यथा\textendash 

\begin{quote}
{\color{violet}आसीन्महेश्वर इति प्रथितः पृथिव्याम्\\
{\color{white}अ}~ आचार्यवर्यपदवीं विदुषां प्रयातः~।\\
लब्ध्वावबोधकलिकां तत एव चक्रे\\
{\color{white}अ}~ तज्जेन बीजगणितं लघु भास्करेण~॥} इति~। (पृ.\,२०३)~।
\end{quote}

बीजनवाङ्कुराभिधायाष्टीकायाः कर्त्रा कृष्णदैवज्ञेनापि टीकायाः प्रारम्भेऽयमेवार्थः किञ्चिदधिकेनार्थान्तरेण संयोज्याभिहितोऽस्ति~। तथा हि\textendash \\

{\color{violet}'शाण्डिल्यगोत्रमुनिवरवंशावतंसजबिडनगरनिवासी कुम्भोद्भवभूषणदिग्भूषणसकला-गमाचार्यवर्यश्रीमहेश्वरोपाध्यायतनयनिखिलविद्यावाचस्पतिर्गणितविद्याचतुराननो~धरणि-तरणिः श्रीभास्कराचार्यः'} इति (पृ.\,२)~। उपरिनिर्दिष्टे ग्रन्थसंदर्भे टीकाकारेण ग्रन्थकार-वसतिस्थानत्वेन यद् 'बिडनगरम्' उक्तं तदेव बिडनगरमाचार्यैरपि स्ववसतिस्थानत्वेन सूचितम्~। तथा चोक्तं {\color{violet}सिद्धान्तशिरोमणिग्रन्थस्य गोलाध्याये}\textendash \,

\begin{quote}
{\color{violet}आसीत् सह्यकुलचलश्रितपुरे त्रैविद्यविद्वज्जने~।\\
नानासज्जनधाम्नि विज्जडबिडे शाण्डिल्यगोत्रो द्विजः~॥} इति~।
\end{quote}

अत्रानिर्दिष्टविशेषनाम पुरं वसतिस्थानत्वेनाभिहितम्~। तथापि तद् 'विज्जडबिडे' इत्यनेन विशेषितम्~। तस्य विशेषणस्यायमर्थः\textendash \,विदो विद्वांसः~। जडा मन्दाः~।
\end{sloppypar}

\afterpage{\fancyhead[RE,LO]{{\small{}}}}
\afterpage{\fancyhead[CE,CO]{[{\small{\thepage}}]}}
\afterpage{\fancyhead[LE,RO]{{\small{}}}}
\cfoot{}

\newpage
\renewcommand{\thepage}{\devanagarinumeral{page}}
\setcounter{page}{2}

\begin{sloppypar}
\noindent विद्याहीना इति यावत्~। तैरुभयैर्बिडे निबिडे व्याप्तं एतादृशे पुर आसीदिति~।~तत्र बिडशब्दं प्रयुञ्जानेनाचार्येण सामान्यतो निर्दिष्टं पुरं श्लेषविधया बिडमिति ध्वनितं दृश्यते~। किञ्च विज्जडव्याप्तत्वादेतत् पुरं लोकैर्विज्जडबिडमिति नाम्ना व्यवह्रियत~इत्यपि सूचितम्~। यत् कृष्णदैवज्ञेनाचार्यवसतिस्थानत्वेन बिडनगरं स्पष्टमुल्लिखितं साम्प्रतम् अहमदगरात् पूर्वस्यां दिशि चत्वारिंशत्क्रोशान्तरे विद्यमानं यद्बिडाभिधं पुरं ह्येतदेव तत् स्यादिति प्रथमं मनस्यापतति~। \renewcommand{\thefootnote}{$\star$}\footnote{हैद्राबादमहामण्डले मोङ्गलराज्ये यः सम्प्रति बीडनामा महान्ग्रामः श्रूयते स एवाचार्याणां निवास इति केचिन्निश्चिन्वन्ति~। तत्र सह्यकुलाचलाश्रितपुरे...बिज्जलबिडे...इति श्लोकं प्रमाणत्वेनोपन्यस्य स बीडाभिधो ग्रामो जिनधर्मीयबिज्जलराजसत्तायामासीदित्येतद्विषये किञ्चिदैतिहासिकं निर्दिश्य तत्र बिज्जलसम्बन्धः साधितः~। सह्याचलाश्रितत्वं च यया कया च विधयारोपितम्~। तथा चेतिहासप्रसिद्धजैनराजविशेषवाचिनि तत्कालप्रचलितव्यावहारिकभाषास्थबिज्जलशब्दे विशेषतो भरो दत्तः~। सह्यकुलाचलाश्रितेति विशेषणे च दुर्लक्षं कृतं दृश्यते~। अस्माकं पुरतो यानि पञ्चषाणि पुरातनानि हस्तलिखितानि मुद्रितानि च गोलाध्यायपुस्तकानि आसते तत्र 'विज्जडबिडे' इति स्पष्टं  पाठो वर्तते, न तु 'बिज्जलबिडे' इति~। मरीचिटीकायां तथा पाठः स्यात्परं लेखकप्रमाद एव सः~। आचार्यैस्तु 'विज्जडबिडे' इत्येव पठितमिति वयं प्रतीमः~। अस्मत्पुरःस्थितविज्जडबिडे इति पाठानुरोधात्~। न च बिज्जलेतिपाठानुसारेण विज्जडेतिपाठ एव प्रमादकृतः किं न स्यादिति वाच्यम्~। यथा तव मते बिडशब्दः केवलं लौकिकः प्रयुक्तस्तथैव यदि बिज्जलशब्दो व्यावहारिकभाषास्थः प्रयुक्तो भवेच्चेदज्ञस्यापि बिडशब्दादिव ततोऽर्थबोधोदयावश्यंभावेन तत्र प्रमादासम्भवात्को हि नाम जडोऽपि तादृशसुबोधस्थले विज्जडेतिदुर्बोधपाठं कल्पयेत्~। मम तु मते विज्जडेतिशब्दस्य गीर्वाणवाणीप्रयुक्तत्वेन तत्र च मध्यमज्ञस्यापि बोधानुदयसम्भवात्तदनुसारेणैव तादृशदुर्बोधपाठकल्पना युज्यते~। किञ्च यथाकथ-ञ्चिद्बिज्जलेति पाठस्याचार्यभिप्रेतत्वेऽपि कोल्हापूरनगरात्किञ्चिदन्तरे सह्याचलपादनिकटेऽधुनापि विद्यमानो बिडग्रामोऽसावपि बिज्जलराजसत्ताधोगतोऽभूदिति सम्प्रत्युपलब्ध शिलालेखादवसीयते~। तथा चाचार्यैः स्वनि-वासत्वेनोद्दिष्टो बिडग्रामः स एव किं न स्यात्~। युक्तं चैतत्~। बिडाभिधग्रामद्वयस्य बिज्जलराजसत्ताश्रितत्वेन स्वनिवासभूतबिडग्रामविषये लोकानां यः सन्देहस्तदपाकरणार्थं सह्यकुलाचलाश्रितेति विशेषणं दत्तम्~। तच्च विशेषणं करवीरक्षेत्रनिकटवर्तिबिडग्रामे साक्षाद्विद्यत इति तेन विशेषणेन तद्विडं व्यावर्त्येदमेव बिडं स्वनिवासत्वेनोपढौकितमाचार्यैरिति~। किञ्च गङ्गाश्रितोऽयं ग्राम इत्युक्तौ यदि स ग्रामः पञ्चक्रोशान्तर्गतो भवति तदैव लक्षणयापि यथाकथञ्चित्सोक्तिः प्रमाणपदवीमारोहति लोके नान्यथा~। सत्येवं सह्याद्विप्रकृष्टे बीडे तादृशीं सह्याश्रितोक्तिं लोकैरसह्यां ग्रन्थकारः कथं प्रयुञ्ज्यात्कथमिव च सा विश्वसनीया भवेत्~। न ह्याचार्यः स्वनिवासभूतग्रामप्रदेशमावेदयितुं प्रवृत्तोऽसन्निहितं सह्यं निर्दिश्य सन्देहे जनान्पातयेत्~। एवं च यद्यपि स एव बीडाभिधो ग्राम अचार्याणां सत्यतया निवासः स्यात् परन्तु तद्विषये सबलप्रमाणाभावात्केचिन्मतं ग्राह्यं धर्तुं न युज्यत इत्यपरे प्राहुः~।}तथाप्याचार्यान्ववायस्य तत्प्रणीतग्रन्थस्य च तेन पुरेण साकं सम्बन्धस्य निदर्शकं गृहक्षेत्रादिकं न किञ्चित्तत्रावलोक्यते~। नापि वा तत्रत्यपुरातनकथाकोविदग्रामवृद्धमुखात्तत्सम्बन्धः श्रूयते~। किञ्चैतद्बिडनगरं न सह्याचलम् आश्रयतीति कथमेतदाचार्यवसतिस्थानत्वेनावगन्तुं युज्येत~। तस्माद्भास्कराचार्याणां निवा-सस्थानं विज्जलबिडं त्वेतस्माद्बिडनगरादन्यदेव किमपि स्यादिति तर्कयामि~।
\end{sloppypar}

\newpage

\begin{sloppypar}
\noindent तद्विज्जलबिडं कस्मिन्देशे वर्तते सम्प्रति च केन नाम्ना तदाख्यायत इत्येतन्निश्चयार्थं संशोधकैर्यतितव्यं भवति~।\\

नवाभ्रतिथि\textendash \,(१९०९)\textendash \,परिमिते शकेऽकबरराजशासनेनाचार्यविरचितग्रन्थस्य पार-सीकभाषायां भाषान्तरं कृतम्~। तत्र ग्रन्थे भास्कराचार्याणां जन्मस्थानं 'बेदर' इत्यभिहितम्~। तद्बेदरं सोलापूरपुरात्प्राच्यां दिशि पञ्चाशत्क्रोशान्तरे मोङ्गलराष्ट्रे वर्तते~। परन्तु तदपि न सह्यकुलाचलाश्रितं भवतीति पारसीकभाषास्थग्रन्थोऽपि नात्र विषये प्रमाणत्वेन धर्तुं शक्यते~। एवं च बिडं वा विज्जडबिडं वा नगरं कस्मिन्देश आसीदिति याथातथ्येन प्रतिपादनं सर्वथेदानीं दुःशकं जातमस्ति~। तथाप्येतद्विषयेऽप्रत्यक्षमपि किञ्चित्प्रमाणमुपलब्धुं~योग्यं भवति~। खानदेशाभिधे प्रदेशे 'चाळिसगांव' सञ्ज्ञकमहाग्रामान्नैर्ऋत्यां दिशि पञ्चक्रोशा-न्तरे 'पाटण' इत्यभिध एकः खेटकग्रामो वर्तते~। तत्र भवानीदेव्याः सुन्दरे मन्दिर एकः शिलालेखोऽस्ति~। स च शिलालेख आचार्याणां नप्त्रा चङ्गदेवेन भभव\textendash \,(११२८)\textendash \,मित-शकादनन्तरमुत्कीर्णोऽस्ति~। तस्मिञ्छिलोत्कीर्णे लेखे भास्कराचार्यप्रणीतग्रन्थाध्ययनाध्या-पनपरम्पराप्रचारार्थं मठसंस्थापनादिकं निर्दिश्य तन्मध्ये जैत्रपालनृपः स्वपितरं लक्ष्मी-धरनामानं पाटणग्रामान्निमन्त्र्य नीतवानित्युल्लेखः कृतोऽस्ति~। एतस्मिन् लेखे प्राक्तन-सन्तानपरम्परानर्न्तगतानां षण्णां पुरुषाणां नामानि कथयित्वानन्तरं भास्कराचार्य-सकाशादग्रिमसन्तानपरम्परायाः परिचयं ददानाः केचिच्छ्लोका लिखिताः~। ते यथा\textendash 

\begin{quote}
{\color{violet}लक्ष्मीधराख्योऽखिलसूरिमुख्यो वेदार्थवित्तार्किकचक्रवर्ती~।\\
क्रतुक्रियाकाण्डविचारसारविशारदो भास्करनन्दनोऽभूत्~॥~२१~॥\\
सर्वशास्त्रार्थदक्षोऽयमिति मत्वा पुरादतः~।\\
जैत्रपालेन यो नीतः कृतश्च विबुधाग्रणीः~॥~२२~॥\\    
तस्मात् सुतः सिङ्घणचक्रवर्ती दैवज्ञवर्योऽजनि चङ्गदेवः~।\\    
श्रीभास्कराचार्यनिबद्धशास्त्रविस्तारहेतोः कुरुते मठं यः~॥~२३~॥ \\
भास्कररचितग्रन्थाः सिद्धान्तशिरोमणिप्रमुखाः~।\\    
तद्वंश्यकृताश्चान्ये व्याख्येया मन्मठे नियमात्~॥~२४~॥} इति~।
\end{quote}

अत्र जैत्रपालेन राज्ञा लक्ष्मीधरः पुरादतः (पाटणग्रामात्) नीतः अथ च लक्ष्मीधर-सूनोश्चङ्गदेवस्य सिङ्घणराजसमाश्रयोऽभूदित्युक्तम्~। जैत्रपालोऽयं यादववंशीयो राजा सन् स देवगिर्याभिधे नगरे त्रयोदशोत्तरैकादशशत\textendash \,(१११३)\textendash \,मितशकसमयादारभ्य रदनो-त्तररुद्र\textendash \,(११३२)\textendash \,परिमितशकपर्यन्तं राज्यं कुर्वन्नासीत्~। तदनन्तरं च तत्पुत्रः सिङ्घणः शकसमयं दशनेशसङ्ख्य\textendash \,(११३२)\textendash \,मारभ्य बाणाधिकद्विगुण-
\end{sloppypar}

\newpage

\begin{sloppypar}
\noindent दन्तेश\textendash \,(११६९)\textendash \,सङ्ख्यशककालपर्यन्तमवर्ततेति सर्वेषां सत्यतया मान्यमस्ति~। भास्क-राचार्यकृतकरणकुतूहलाभिधग्रन्थरचनाकालश्च तत्रैव शरदिगिन्दु\textendash \,(११०५)\textendash \,युतः शकः प्रदत्तोऽस्ति~। तस्मादाचार्याणां जीवितकालमर्यादा सा नाम शरदिगिन्दुपर्यन्ता~धर्त-व्यैव भवति~। तदानीं च तेषां वयोमानमेकोना सप्ततिराप्ता~। पश्चाच्च कियतीभिः अल्पभिरेव शराद्भिस्ते निवृत्ता बभूवुरिति ग्रहीतुं प्रतिबन्धकं नास्ति~। तथा सति 'आचार्याणां लक्ष्मीधराख्यः सुतोऽयं त्रयोदशोत्तरैकादशसङ्ख्य\textendash \,(१११३)\textendash \,शकम् आरभ्य द्वात्रिंशदुत्तरैकादश\textendash \,(११३२)\textendash \,मितशकाङ्कपर्यन्तं राज्यं कुर्वतो जैत्रपालस्य राज्ञः समाश्रये तिष्ठति स्म' इति शिलालेखगतं विधानमाचार्योपरतिकालविषयकेणानुमानेन सह सङ्गच्छते~। तथाप्येतच्छिलालेखे 'पुरादतः' इति शब्देनोल्लिखितः पाटणाह्वयो ग्रामः, अयमेव बिडनगरमिति सनिश्चयं वक्तुं न पार्यते~। लक्ष्मीधरस्य तत्पाटणपुरं निवास-स्थानमभूदिति निर्विवादमस्तु~। परं च तावता तत्पितुर्भास्कराचार्यस्यापि तदेव निवासस्थानमासीदित्यनुमानकरणं कया विधया योग्यं भवेत्~। पाटणाह्वयो ग्रामः सह्यकुलाचलमाश्रयतीति कथं कथमपि महतार्थाकर्षणेन वक्तुं पार्येत~। यतः सह्याचल-शाखास्तं सन्निधयन्ति~। तथापि तावता पाटणविज्जडबिडशब्दावेकस्यैव द्वे नामनी इत्यनुमानं कर्तुं कारणं न दृश्यते~।\\

आचार्यपितृव्यवंशीयेनानन्तदेवेन शके वेदवेदमहमहीसङ्ख्ये (११४४) उत्कीर्ण एकः शिलालेख उपलभ्यते~। स च 'चाळिसगांव' इत्यभिधग्रामादुत्तरस्मिन्पञ्चक्रोशान्तरे~सन्नि-हिताया गिरणानद्याः समीपे 'बहाळ' इति ग्रामोऽस्ति, तत्रत्ये सारजा-(शारदा)-देव्या देवालये दृश्यते~। एतस्मात् कारणाच्चाळिसगांवाभिधग्रामस्य सविधप्रदेश आचार्यवंशजमण्डली वसन्त्यासेति सिध्यति~। तथा चाचार्याणां वसतिस्थानं याथार्थ्येन निश्चेतुं 'चाळिसगांव-पाटण-बहाळ' इत्यादिस्थलेष्वितोऽप्यधिकः संशोधनप्रयत्न आवश्यक इति अस्माकं प्रतिभाति~।\\

आचार्यकृतग्रन्थाध्ययनपरम्पराया अव्याहतं प्रचलनस्य व्यवस्था यादववंशीयस्य राज्ञ आश्रयेण सञ्जातासीदित्युपरिकथितमेवास्ति~। तद्यादववंशराज्यं शून्यवेदसूर्यसङ्ख्ये~(१२४०) शकसमये नामशेषं भूत्वा दक्षिणदेशे यावनी सत्ता प्रवृत्ताभूत्~। अर्थादेव तदारभ्य मठस्य राजाश्रयो नष्टो जातः स्यादिदं स्पष्टमस्ति~। तथापि सिद्धान्तशिरोमणिग्रन्थप्रशंसा सर्वत्रैव प्रथितासीदिति समये समये विद्यानुरागिभिः तद्ग्रन्थसंरक्षणतदर्थसिद्धान्तमननयोरतीव मन आबध्य टीकां विरच्य सिद्धान्तान् विशदान् सुलभावबोधांश्च कर्तुं प्रयतितम् इत्युपलब्धटीकाग्रन्थावलोकनेन वक्तुं पार्यते~। परन्तु तादृशग्रन्थाध्ययनपरम्परायाः खण्डि-तत्वात् सिद्धान्तशिरोमणेः सर्वेष्वेवाध्यायेषु विदुषां मनः सममवस्थातुं न शक्नोति स्म~। प्रथमोऽध्यायः पाटीग-
\end{sloppypar}

\newpage

\begin{sloppypar}
\noindent णितापरपर्याया लीलावती सुगमा सती बालबोधे व्यवहारे चातीवोपयुक्तेति तदुपरि बहुभिः टीका व्यरचिषत~। तासु सर्वासु पुरातनीषु पुरातनी टीका 'गणितामृतलहरी' नाम्नी नवाग्न्यग्नीन्दुमिते (१३३९) शके व्यरचि~। अर्थाद्यादवराज्यसमाप्त्यनन्तरं शतेन संवत्सरैर्लिखितेति यावत्~। एतच्छतके टीकाग्रन्थो विरचितः क्वापि नालक्ष्यते~। अर्थादेतस्मिञ्छतके सिद्धान्तशिरोमणिग्रन्थाध्ययनपरम्पराया उच्छिन्नत्वात् स दुर्बोधः संवृत्तः~। ततश्च तदुपरि टीकाकरणमावश्यकं सञ्जातम्~। तदा च प्रथमभागात् परत्र टीकाकृत्कश्चिदपि नोपलब्ध इति दृश्यते~। तथापि प्रारम्भे राजाश्रयमन्तरापि यथा भास्कराचार्यैः केवलं स्वकर्तव्यताबुद्धिप्रेरितैर्भूत्वा गणितविषये मननं कृतमथ च सिद्धान्तशिरोमणिसमं निखिलविद्वद्वृन्दवन्द्यं ग्रन्थं निर्माय परमं यशः सम्पादितं तद्वत् टीकाकारैरपि स्वस्वबुद्धिवैभवानुसारेण स्थले स्थले यथोपपन्नमध्ययनं यथोपपन्नभागे च टीकाकरणक्रमः प्रचालितः~। राजाश्रयाभावादेतादृशां विद्याव्यासङ्गिनां टीकाकरणजनितप्रयासमूल्यं धनमादरो वा यत्किञ्चिदपि लभ्यं नासीत्~। परन्तु मुख्यतः श्रमसाफल्यकारित्वं परमेश्वरायत्तम्~। स च नित्यं जागरूक इत्यस्मत्परिश्रमानवेक्षत एव~। किञ्च 'ब्राह्मणेन निष्कारणं षडङ्गो वेदोऽध्येयो ज्ञेयश्च' इति महाभाष्यकारोक्तेः, {\color{violet}'कर्मण्येवाधिकारस्ते मा फलेषु कदाचन~। मा कर्मफलहेतुर्भूर्मा तेऽसङ्गोऽस्त्वकर्मणि'}~। इति भगवदुक्तेश्च ब्राह्मणैः स्वकर्तव्यताबुद्धिप्रेरितैर्भूत्वा वेदशास्त्रादिविद्यायाः संरक्षणपुनरुज्जीवनाद्यर्थं कष्टकरणं व्रतमनादिकालप्रवृत्तं तत्तथैवेदानीमपि तादृशैस्तैः प्रक्रान्तम्~। तेन नवत्रित्रिमहीमित\textendash \,(१३३९)\textendash \,शकसमये रामकृष्णाख्यो ज्योतिर्विद्वरः सिद्धान्तशिरोमणिप्रथमाध्यायोपरि टीकाकर्तोदगात्~। तथा शकेऽग्निरसवेदरसासङ्ख्ये (१४६३) सूर्यदासः प्रथमद्वितीयाध्यायटीकाकर्तोदयंसीत्~। अथ च तेन बीजभाष्यं व्यरचि~। सोऽयं सूर्यदासः स्वपितुर्नाम ज्ञानराज इत्यावेद्य लिखति\textendash 

\begin{quote}
{\color{violet}तत्सूनुः सूर्यदासः सुजनविधिविदां प्रीतये बीजभाष्यम्~।\\
चक्रे सूर्यप्रकाशं स्वमतिपरिचयादादितः सोपपत्तिः~॥~३~॥} इति~।
\end{quote}

अत्र 'स्वमतिपरिचयात्,' इत्येतत्पदनिवेशात्, बीजभाष्यविरचनप्रवृत्तिकारणत्वेन यदुल्लिखितं साम्प्रतं गणितशास्त्राध्ययनाध्यापनपरम्परा विच्छिन्नेति, तस्यैवार्थस्य द्विगुणं बलं लब्धं भवति~। अर्थात् गुरुमुखं विना बीजसमग्रन्थोपरि टीकाविरचनं तत्रापि तत्सोपपत्तिकमिति यन्महत्कार्यं तत्सूर्यदासेन सम्पादितमिति सूर्यदासविषये कस्यापि मनसि कुतूहललहरयोऽवश्यं समुल्लसेयुरेव~।\\

सूर्यदासादनन्तरं स्थूलमानत एकस्मिन् संवत्सरशतके व्यतीते बीजगणितोपरि टीकालेखनप्रयत्नः कृष्णदैवज्ञेन कृतः~। स एवाद्यास्माभिः 'बीजनवाङ्कुरा' इति
\end{sloppypar}

\newpage

\begin{sloppypar}
\noindent ग्रथितः संमुद्र्यानन्दाश्रममुद्रणालये विदुषां पुरतः प्रस्थाप्यते~। एतद्ग्रन्थविरचनासमये त्वतीव गणितशास्त्राध्ययनपरम्परा लुप्ता भूत्वा ज्ञानवृक्षः शुष्कतां गत आसीत्~। एवं सत्यपि तादृशस्य दुरूहस्य सिद्धान्तग्रन्थस्य याथातथ्येन तात्पर्यार्थं वारं वारं सूक्ष्मया बुद्ध्या विचार्य महता परिश्रमेण मनस्याकलय्य तदुपरि सोपपत्तिकं टीकाविरचनमिति यत्तच्छुष्कवृक्षस्य नूतनाङ्कुरोत्पादनमिवातीव दुष्करमभूत्~। तथापि तत्कार्यं कृष्णदैवज्ञेन विदुषां पारं नीतम्~। तदानीं चैतद्विदुषो दुःसाध्यकार्यकारिणः परिश्रमस्य कौतुककारी तु दूरे, परं सामान्यतोऽवगन्तापि समाजे कश्चिन्नासीदिति दृश्यते~। दक्षिणापथे सर्वत्र यावनसत्ताप्रभावस्य पूर्णं प्रसराद्बहवो जना द्विजा अपि यवनभाषाध्ययने तदधीनतया यवनसेवाकरणे च व्यापृताः सन्त आयुर्व्ययं कुर्वाणा आसन्~। तेन कारणेनेह लोक आश्रयदातुरथवा गुणग्राहिणः कस्यचिन्मानवदेहधारिणो लिप्सायाः कल्पनां परित्यज्यानेन मत्परिश्रमेण सर्वाधारः सर्वज्ञः परमेश्वरः सन्तुष्यतीति तच्चरणकमलयोरेव प्रस्तुतग्रन्थशिरोमणिसमर्पणेन स्वमनसः समाधानकरणं ग्रन्थकारस्यावश्यकं सञ्जातम्~। तदुक्तं ग्रन्थकारेणैव\textendash 

\begin{quote}
{\color{violet}यद्भास्करेण निजधामगुणातिरेकात् \\
{\color{white}अ}~ सम्पादितं सगुणवर्गघनं हि बीजम्~।\\
तत्कृष्णभूमिमधिगम्य विचारवारि-\\
{\color{white}अ}~ संसिक्तमङ्कुरजनुष्यभवत्समर्थम्~॥~६~॥\\
यैर्यैः श्रमैर्विरचितोऽस्ति नवाङ्कुरोऽसौ\\
{\color{white}अ}~ तेषामभिज्ञ इह कः परमात्मनोऽन्यः~।\\
इत्थं विचिन्त्य जगदीश तवैव तुष्ट्यै\\
{\color{white}अ}~ सर्वज्ञ ते चरणयोर्निहितस्ततोऽयम्~॥~७~॥} इति~।
\end{quote}

सिद्धान्तशिरोमणिग्रन्थस्य प्रणेता श्रीभास्कराचार्यः, तदन्तर्गतबीजाध्यायोपरि सोपपत्तिकनवाङ्कुराख्यटीकाकर्ता कृष्णदैवज्ञश्चेत्यनयोः सम्बन्धी समग्रो विचारः स्वकृतभारतीयज्योतिःशास्त्रे कैलासनिवासिबालकृष्णतनूजशङ्करदीक्षितैः कृतो वरीवर्ति~। स वाचकैरवश्यं प्रेक्षणीयः~। अस्माकमपि मुख्यतः स एवाधारभूतोऽस्ति~।\\

दक्षिणदेशस्थितमहंमदीयराष्ट्रस्य ये स्वतन्त्राः पञ्च विभागाः सम्पन्नास्तेषां मध्ये विदर्भ-(वह्राड)-देशाधिपतिर्यवनोऽपि सन्न जात्या यवनोऽपि तु धर्मतः~। जात्या तु हिन्दुरेवेति तदन्तःकरणगत आर्यधर्मसम्बन्ध्यादरभावः प्रणष्टुं न शशाक~। एतस्मात्तत एव चार्यफलज्योतिषोपरिगतश्रद्धावशाच्चैतद्राज्ये केषाञ्चिदार्यदैवज्ञानां कदा कदा राजाश्रयो लभ्योऽभूत्~।
\end{sloppypar}

\newpage

\begin{sloppypar}
कृष्णदैवज्ञेन स्वकृतटीकोपसंहारार्थं चत्वारः श्लोका लिखिताः~। तत्र स्वगुरुर्विष्णुः, तद्गुरुर्नृसिंहः, तद्गुरुस्तत्पितृव्यो ग्रहलाघवकर्ता गणेशदैवज्ञः, तद्गुरुस्तत्पिता केशव-दैवज्ञ इत्येदं स्वविद्यागुरुपरम्परामुल्लिख्य, स्वपिता बल्लालः, तत्पिता त्रिमल्लः,~तत्पिता रामाभिधः, तत्पिता चिन्तामणिरेवं स्वपितृपरम्परा निर्दिष्टास्ति~। तन्मध्ये कृष्णदैव-ज्ञपितृपरम्परान्तर्गतरामाभिधदैवज्ञस्योपर्युल्लिखितविदर्भाधिपस्येमादशाहस्य~स्वल्पोऽपि समाश्रयोऽभूत्~। परन्तु दिल्लीपतिनाकबरेण तद्विदर्भराष्ट्रं यदा विलयं नीतं तत~आर-भ्यैतद्वंशजस्य राजाश्रयः समाप्तिमगात्~। तथाप्यकबरस्तत्पुत्रो जहांगिरश्चैतयोरत्यन्तो-ग्रयवनधर्माभिमानाभावान्मुहूर्तज्योतिषविषये विश्वासात् चार्यज्योतिर्विदां तन्निकट आश्रयो लभ्योऽभूत् इत्येतद्विषये प्रमाणं लेखाः सन्ति~। तदनुसारेण विद्वत्ताकीर्तिं समाकर्ण्य जहांगिरेण दिल्लीश्वरेण स्वीयराजासमायामाश्रयं दत्त्वा कृष्णदैवज्ञः प्रवेशितः स्यादिति कृष्णद्वैवज्ञभ्रातू रङ्गनाथस्योक्तेर्ज्ञायते~। सूर्यसिद्धान्तोपरि गूढार्थप्रकाशिकानाम्नी टीकानेन रङ्गनाथेन लिखिता~। तत्र प्रारम्भे स उक्तवान्\textendash 

\begin{quote}
{\color{violet}'पितरौ गोजिबल्लालौ जयतोऽम्बाशिवात्मकौ~।\\
याभ्यां पञ्च सुता जाता ज्योतिःसंसारहेतवः~॥~२~॥\\
सार्वभौमजहांगीरविश्वासास्पदभाषणम्~।\\
यस्य तं भ्रातरं वन्दे बुधं कृष्णं जगद्गुरुम्~॥~३~॥} इति\\
{\color{white}अ} \hfill (आनन्दाश्रमस्थहस्तलिखितग्रन्थः, अनुक्रमाङ्कः २०५०)
\end{quote}

रङ्गनाथसुतं मुनीश्वरं प्रति जहांगीरात्मजशहाजानस्याश्रयोऽस्ति स्म~। शहाजानस्य राजकार्यकर्मणि (शासनसमये) उत्तरार्धे, तथाग्रे औरंगजेबराजस्य राज्यकरणकाले च दिल्लीपतेरुग्रतरः स्वधर्माभिमानो जागरूको भूत्वा हिन्दुजातीयज्योतिर्विदां राजा-श्रयोऽनशत्~। तथापि तस्मिन्नेव समये श्रीशिवराजभूपते राज्यप्रस्थापनस्य जातत्वात् आर्य-ज्यौतिषिकान् प्रत्यार्यधर्मीयराजाश्रयो लब्धः~। तेन टीकायाः स्थाने स्वतन्त्रः करणग्रन्थः स्वात्मानमलभत~। तन्नाम करणकौस्तुभ इति~। स चेतः पूर्वमेवानन्दाश्रमग्रन्थावल्ल्यां संमुद्र्य समावेशितोऽस्ति~।\\

एतदनन्तरं नवाङ्कुराटीकास्थोपपत्तिविषये टीकामहत्त्वे च किञ्चिद्वक्तव्यं प्रसङ्गागतम्~। परन्तु बीजोपरिगतसूर्यदासकृतटीका नाद्यापि मुद्रिता~। तत्प्रकाशनवेलायां तुलनात्मक-विचारसरण्या नवाङ्कुरचर्चाकरणं वरम्~। \\

बीजगणिताध्यायगतनवाङ्कुरटीकायाः संस्करणार्थं सहायभूतानि यानि प्रत्यन्तराणि आसंस्तेषु घसञ्ज्ञकं पुस्तकं डॉ.\;नरहर गोपाळ सरदेसाई, एतेषां ग्रन्थसङ्ग्रहालयस्थं वर्तते~।
\end{sloppypar}

\newpage

\begin{sloppypar}
\noindent तथा गसञ्ज्ञकं भाण्डारकरप्राच्यविद्यासंशोधनमन्दिरान्तर्गतम्~। चसञ्ज्ञकं च काशीस्थित महाविद्यालयान्तर्गतं ज्ञेयम्~। तान्येतानि पुस्तकानि सहायार्थं यैः परहितप्रवणान्तः करणैः प्रदत्तानि तेषां संस्थाधिकारिणां चोपकारभराञ्छिरसा वहामः~।~ग्रन्थमुद्रणार्थम् आनन्दाश्रमग्रन्थसङ्ग्रहालयस्थं कसञ्ज्ञकं पुस्तकमेव मुख्यत उपयुक्तमभूत्~। मुद्रणार्थं प्रत्यन्तरकरणकार्यं वे.\;शा.\;सं.\;मराठेशास्त्री इत्येतैः सम्पादितम्~। स्थले स्थले विशिष्ट-संशोधनकर्मणि 'आनन्दाश्रमस्थप्रधानसंशोधकमारुलकरोपाह्वशङ्करशास्त्रिभिः'~अतीव साहायकमाचरितम्~। कुट्टकं मध्यमाहरणं चैतत्प्रकरणद्वयस्य~याथातथ्येन तात्पर्यार्था-वगमविषये प्रो.\;त्र्यम्बक भिकाजी हर्डीकर' इत्यभिधेयमहाशयानामतिमात्रं साहाय्यं सञ्जातमित्यवश्यं निवेदनीयम्~। एभिर्हर्डीकरोपाह्वमहाशयैराधुनिकबीजगणितपद्धत्या~सह एतद्बीजगणितसरण्यास्तुलनां कृत्वा कियानपि प्रास्ताविकभागो लिखितुं समारब्धः~। परन्तु सांसारिककतिपयान्यकार्यव्यासक्तान्तःकरणतया समारब्धः स~न~विस्तृतां~गतः किन्तु यथावस्थित एवावस्थितः~। ग्रन्थगतगद्यपद्यादेर्विभजनं नवाङ्कुराटीकाकारपरिगृही-तविभागसरणिमनुरुध्य कृतमस्ति~। तेन टीकारहस्यार्थस्य मनस्यवतरणं सुलभं भवेत्~। यत्र क्वचन स्थले टीकाया आवश्यकता न प्रतिभाता तत्र टीकाकृता वासनाभाष्योपरि सर्वात्मना विश्वासो दत्तः~। तस्य च वासनभाष्यस्येदानीं चक्षुःपुरतोऽवस्थितेरभावाद्बीजगणितान्तर्गतः कियानपि भागः स्थले स्थले टीकाविरहितावस्थामनुभवति~। सेयं न्यूनता मूलग्रन्थं वासनाभाष्येणालङ्कृत्य तत्सहाये न सूर्यदासकृतबीजभाष्याख्यटीकोट्टङ्कनकरणेन दूरं गता भविष्यति~। तदिदमपि कार्यं यदा कदाचनानन्दाश्रमसंस्था सिद्धिं नेष्यतीत्याशास्ते\textendash
\vspace{6mm}

\hspace{2mm} \textbf{\large विनायक गणेश आपटे}\\

\begin{center}
\rule{0.2\linewidth}{0.5pt}\\
\vspace{-4mm}

\rule{0.2\linewidth}{0.5pt}
\end{center}
\end{sloppypar}

\newpage
\thispagestyle{empty}

\begin{center}
\textbf{ॐ तत्सद्ब्रह्मणे नमः~।}\\
\vspace{5mm}
\textbf{\Huge बीजगणितम्~।}\\
\vspace{1mm}
\rule{0.3\linewidth}{0.5pt}\\
\vspace{5mm}
\textbf{\Large नवाङ्कुरव्याख्यासहितम्~।}\\
\vspace{8mm}
\textbf{\large धनर्णषड्विधम्~।}
\end{center}
\vspace{-1mm}

\begin{quote}
{\color{violet}शिवयोर्भजनातिगौरवाद्यत् सुतलीलाधृतकुञ्जरास्यरूपम्~।\\
अपहन्तु ममान्तरं तमस्तत् सततानन्दमयं महो महीयः~॥~१~॥}
\vspace{1mm}

{\color{violet}यदीयचरणाम्भोजस्मर्तुः सकलसिद्धयः~।\\
भवन्ति वशवर्तिन्यः सिद्धेशीं तामहं भजे~॥~२~॥ }
\vspace{1mm}

{\color{violet}मिहिरमिव वराहमिहिरं वन्दे सन्देहभेदिनं जगताम्~।\\
ज्योतिश्चक्रविभावनहेतुं जगदेकचक्षुरक्षुद्रम्~॥~३~॥}
\vspace{1mm}

{\color{violet}कविबुधजनमूर्धनि स्फुरन्तं कविबुधसन्ततसेवनीयपार्श्वम्~।\\
गणितनिपुणतां प्रवर्तयन्तं प्रणमत भास्करमीप्सितार्थसिद्ध्यै~॥~४~॥}
\vspace{1mm}

{\color{violet}कदापि नैव सम्भ्रमः स्थितश्च भौममण्डले~।\\
अपूर्वेमार्गमाश्रयन् जयत्यपूर्वभास्करः~॥~५~॥}
\vspace{1mm}

{\color{violet}आसीदसीमगुणरत्ननिधानकुम्भः कुम्भोद्भवाभरणदिग्ललनाललामः~।\\
आशैशवार्जितविशेषकलानुवर्ती श्रीकेशवः सुगणितागमचक्रवर्ती~॥~६~॥}
\vspace{1mm}

{\color{violet}तस्मादभूद्भुवनभूषणभूतमूर्तिः श्रीमानगण्यगुणगौरवगेयकीर्तिः~। \\
ज्योतिर्विदागमगुरुर्गुरुसम्प्रदायः प्रज्ञातशास्त्रहृदयः सदयो गणेशः~॥~७~॥}
\vspace{1mm}

{\color{violet}भ्रातुः सुतस्तस्य यथार्थनामा नृसिंह इत्यद्भुतरूपशोभः~। \\
अवर्धयद्यो जगतामभीष्टं प्रह्लादमाश्चर्यकरः सुराणाम्~॥~८~॥}
\vspace{1mm}

{\color{violet}तच्छिष्यो विष्णुनामा स जयति जगतीजागरूकः प्रद्दिष्टः \\
शिष्टानामग्रगण्यः सुभणितगणिताम्नायविद्याशरण्यः~।\\
यद्वक्त्रोन्मुक्तमुक्ताफलविमलवचोवीचिमालागलन्तः\\    
चित्राः सिद्धान्तलेशा जगति विदधतेऽज्ञेऽपि सर्वज्ञगर्वम्~॥~९~॥}
\end{quote}

\afterpage{\fancyhead[RE,LO]{{\small{}}}}
\afterpage{\fancyhead[CE]{{\small{\textbf{बीजगणितम् ।}}}}}
\afterpage{\fancyhead[CO]{{\small{\textbf{नवाङ्कुरव्याख्यासहितम् ।}}}}}
\afterpage{\fancyhead[LE,RO]{{\small{\textbf{\thepage}}}}}
\cfoot{}

\newpage
\renewcommand{\thepage}{\devanagarinumeral{page}}
\setcounter{page}{2}

\begin{quote}
{\small {\color{violet}तस्मादधीत्य विधिवत् त्रिस्कन्धं ज्योतिषं गुरोः~।\\
कृष्णो दैवविदां श्रेष्ठस्तनुते बीजपल्लवम्~॥~१०~॥}
\vspace{1mm}

{\color{violet}अव्यक्तत्वादिदं बीजमित्युक्तं शास्त्रकर्तृभिः~।\\
तद्व्यक्तीकरणं शक्यं न विना गुर्वनुग्रहम्~॥~११~॥}}
\end{quote}

\begin{sloppypar}
{\small अथ शाण्डिल्यगोत्रमुनिवरवंशावतंसज-बिडनगरनिवासि-कुम्भोद्भवभूषण-दिग्भूषण-सकला-गमाचार्यवर्य\,-\,श्रीमहेश्वरोपाध्यायतनय-\,निखिलविद्यावाचस्पति\,-\,गणितविद्याचतुराननधरणितरणिः श्रीभास्कराचार्यः खगगणितरूपसिद्धान्तशिरोमणिं चिकीर्षुस्तदुपयोगितया तदध्यायभूतं व्यक्त-गणितमुक्त्वा तथाभूतमव्यक्तगणितमारभमाणः प्रत्यूहव्यूहनिरासाय शिष्टाचारपरिपालनार्थं मङ्गलम् आचरन् शिष्यशिक्षार्थं तदुपजातिकया निबध्नाति\textendash }

\phantomsection \label{1.1}
\begin{quote}
{\large \textbf{{\color{purple}उत्पादकं यत्प्रवदन्ति बुद्धेः अधिष्ठितं सत्पुरुषेण साङ्ख्याः~।\\
व्यक्तस्य कृत्स्नस्य तदेकबीजम् अव्यक्तमीशं गणितं च वन्दे~॥~१~॥}}}
\end{quote}

अत्रायमन्वयः~। तदव्यक्तमीशं गणितं च वन्दे~। ईशपक्षे यत्तदोर्लिङ्ग[वि]परिणामेन यदितिस्थाने ये तदितिस्थाने तं चेति बोद्धव्यम्~। अव्यक्तं प्रधानम्~। साङ्ख्यशास्त्रे जगत्कारणतया प्रसिद्धम्~। ईशं सच्चिदानन्दरूपं वेदान्तवेद्यम्~। गणितमत्राव्यक्तमेव~। अव्यक्तपदस्यावृत्त्याव्यक्तं गणितमिति तद्विशेषणस्य विवक्षितत्वात्~। तन्नमस्कारेण च तदधिष्ठात्री देवता नमस्कृता भवति~। शालग्रामशिलादौ तथा दृष्टत्वात्~। तत्र प्रधानपक्षे किं तदव्यक्तम्~। साङ्ख्या यद्बुद्धेरुत्पादकं प्रवदन्ति~। बुद्धेस्तत्त्वविशेषस्य महदाख्यस्य~। उत्पत्तिरत्राभिव्यक्तिः~। यतस्ते सत्कार्यवादिनः~। ननु प्रधानमचेतनं कथं कार्यमुत्पादयेदित्यत उक्तम्\textendash \,\hyperref[1.1]{\textbf{पुरुषेणाधिष्ठितं सदिति~।}} यथा हि कुलालादिना चेतनेनाधिष्ठितं कपालादि घटा-द्युत्पादकं तद्वदित्यर्थः~। अत्र साङ्ख्याः सेश्वराः श्रीमत्भगवत्पतञ्जलिमतानुसारिणो ज्ञेयाः~। निरीश्वरा हि कपिलमतानुसारिणः पुरुषनिरपेक्षमेव प्रधानमुत्पादकं प्रवदन्ति~। तदुक्तम् {\color{violet}ईश्वरकृष्णेन सप्तत्याम्\textendash }

\begin{quote}
{\color{violet}वत्सविवृद्धिनिमित्तं क्षीरस्य यथा प्रवृत्तिरज्ञस्य~।\\
पुरुषविमोक्षनिमित्तं तथा प्रवृत्तिः प्रधानस्य~॥} इति~।
\end{quote}

ननु तादृशे प्रधाने किं प्रमाणमित्यत आह\textendash \,\hyperref[1.1]{\textbf{कृत्स्नस्य व्यक्तस्यैकबीजमिति~।}} समस्तस्य व्यक्तस्य कार्यजातस्यैकं बीजमुपादानम्~। तथा च वियदादिकार्यजातं सोपादानकं कार्य-त्वात्~। घटवदित्यनुमानं लाघवसहकृतं तत्र प्रमाणमिति भावः~। न चेश्वरेणार्थान्तरता~। तस्य निर्विकारस्यापरिणामितयानुपादानत्वात्~। परिणामित्वेऽपि कथमनेतनं चेतनपरिणामं स्यादिति~। एकमिति पुरुषव्यवच्छेदः~। तन्मते पुरुषस्यानुपादा-
\end{sloppypar}

\newpage

\begin{sloppypar}
\noindent नत्वात्~। यतस्ते वदन्ति पुरुषस्तु पुष्करपलाशवन्निर्लेप इति~। यथा वेदान्तिमते मायाब्रह्मणी द्वे अपि प्रपञ्चस्योपादाने तद्वदित्यर्थः~। अथेशपक्षे\textendash \,\hyperref[1.1]{\textbf{साङ्ख्याः}} सम्यक् ख्यायते ज्ञायत आत्मा यया सा सङ्ख्यात्माकारान्तःकरणवृत्तिः सा येषां ते साङ्ख्या आत्मज्ञानिनः~। \hyperref[1.1]{\textbf{सत्पुरुषेण}} विवेकादिसाधनचतुष्टयसम्पत्तिमता~। \hyperref[1.1]{\textbf{अधिष्ठितम्}} आदरनैरन्तर्याभ्यां श्रवणादिविषयीकृतं सन्तं \hyperref[1.1]{\textbf{बुद्धेः}} तत्त्वज्ञानस्य \hyperref[1.1]{\textbf{उत्पादकं प्रवदन्ति}}~। ननु तस्याजनकत्वाद्बुद्धिजनकत्वे मानाभाव इत्यत आह\textendash \,समस्तस्य व्यक्तस्य कार्यजातस्यैकमसाधारणं बीजमुपादानमित्यर्थः~।~{\color{violet}"यतो वा इमानि भूतानि जायन्ते"} इति~। {\color{violet}"तत्सृष्ट्वा तदेवानुप्राविशत्"} इति~। {\color{violet}"तस्माद्वा एतस्मादात्मन आकाशः सम्भूतः"}~। इत्यादिश्रुतयस्तदुपादानत्वे प्रमाणमिति भावः~।~ननु निर्विकारस्योपादानत्वे परिणामितया कथमुपादानत्वमिति चेत्~। सत्यम्~। उपादानं द्विविधम्~। परिणाममानं विवर्तमानं चेति~। तत्र परिणामि विक्रियावत्~। यथा मृदादि घटादेः~। विक्रियाशून्यं विवर्तमानम्~। यथा शुक्त्यादि रजतादेः~। तत्र यद्यपि निर्विकारस्येशस्य परिणाम्युपादानता नोपपद्यते तथापि विवर्तमानोपादानत्वे न काप्यनुपपत्तिरस्तीत्यलं पल्लवितेन~। मायाया उपादानत्वपक्षेऽपि विवर्तमानोपादानत्वस्यात्र विवक्षितत्वादेकमित्युक्तम्~। अथ गणितपक्षे~। साङ्ख्याः सङ्ख्याविदो गणकाः सत्पुरुषेण स्वरूपयोग्येनाधिष्ठितमभ्यस्तं यद्बुद्धेः शिरोमणिवक्ष्यमाणप्रश्नोत्तरार्थादिज्ञानस्योत्पादकं प्रवदन्ति~। ननु प्रश्नोत्तरार्थादिज्ञानस्योत्पादकं व्यक्तमेवास्ति\textendash

\begin{quote}
{\color{violet}गुणघ्नमूलोनयुतस्य राशेः दृष्टस्य युक्तस्य गुणार्धकृत्या~।\\
मूलं गुणार्धेन युतं विहीनं वर्गीकृतं प्रष्टुरभीष्टराशिः~॥}
\end{quote}

इत्यादि~। कुज्योनतद्धृतिहृताकृतशक्रनिघ्नी कुज्यैव यत्फलपदं पलभा भवेत्सेति~।

\begin{quote}
{\color{violet}द्युज्यापक्रमभानुदोर्गुणयुतिस्तिथ्युद्धृता द्व्याहता \\
स्यादाद्यो युतिवर्गतो यमगुणात् सप्तामराप्त्योनिताः~।\\
नागाद्र्यङ्गदिगङ्ककाः पदमतस्तेनाद्य ऊनो भवेत्\\
व्यासार्धेष्टगुणाब्धिपावकमिते क्रान्तिज्यकातो रविः~॥}
\end{quote}

इत्यादिवाक्यतो यावत्तावदादिवर्णकल्पनानिरपेक्षैर्गुणनभजनादिमार्गैः क्रियमाणं गणितं व्यक्तमित्युच्यते~। तत्कथमुच्यते प्रश्नोत्तरार्थज्ञानरूपाया बुद्धेरुत्पादकमव्यक्तमित्यत आह\textendash \;\hyperref[1.1]{\textbf{व्यक्तस्येति}\;।} व्यक्तस्य यावत्तावदादिवर्णकल्पनानिरपेक्षस्य गुणघ्नमूलोनयुतस्य, राशेरित्याद्यस्य द्युज्यापक्रमभानुदोर्गुणयुतिस्तिथ्युद्धृता द्व्याहतेत्याद्यस्य च गणितस्यैकं बीजं मूलमिति यावत्~। द्युज्यापक्रमेत्यादिगणितप्रकारस्य वर्णकल्पनामूलत्वादिति~भावः~। श्रेयांसि बहुविघ्नानीत्युक्तत्वान्नमस्कारत्रयमुचितमेव मङ्गलस्य समाप्तिजनकत्वं विघ्नध्वंस-जनकत्वं वा प्रकृतानुपयुक्तत्वाद्ग्रन्थविस्तरभयाच्चूडामण्यादौ विस्तृतत्वाच्च नेह व्युत्पाद्यते तत्तत एव 
\end{sloppypar}

\newpage

\begin{sloppypar}
\noindent द्रष्टव्यम्~। ईशस्य समस्तकार्यजनकत्वं वदता तत्प्रणामस्य ग्रन्थसमाप्तिप्रचयादिरूपं फलं कैमुतिकन्यायेनैव सूचितम्~। यतो यो यदिष्टमनिष्टं वा कर्तुं शक्तः स स्वप्रणतस्य तदिष्टं स्वद्वेष्टुस्तदनिष्टं च विदधाति~। ईशस्तु सर्वं कर्तुं समर्थः स्वप्रणतस्य सर्वमिष्टं विदध्यात् ग्रन्थसमाप्तिप्रचयरूपं किमुतेति~। अत्र साङ्ख्यवेदान्तिमतव्युत्पादनं ग्रन्थविस्तरभयान्न कृतं तत्तत एवावगन्तव्यम्~॥~१~॥\\

{\small इदानीं प्रेक्षावत्प्रवृत्तिहेतुविषयादिचतुष्टयं सङ्गतिं च शालिन्या दर्शयति\textendash }

\phantomsection \label{1.2}
\begin{quote}
{\large \textbf{{\color{purple}पूर्वं प्रोक्तं व्यक्तमव्यक्तबीजं \\
प्रायः प्रश्ना नो विनाव्यक्तयुक्त्या~।\\
ज्ञातुं शक्या मन्दधीभिर्नितान्तं \\
यस्मात्तस्माद्वच्मि बीजक्रियां च~॥~२~॥}}}
\end{quote}

अस्यार्थः\textendash \,\hyperref[1.2]{\textbf{तस्मात्}} हेतोः \hyperref[1.2]{\textbf{बीज}}स्य यावत्तावदादिवर्णकल्पनादिभिः क्रियमाणस्य गणि-तस्य \hyperref[1.2]{\textbf{क्रियाम्}} इतिकर्तव्यतां \hyperref[1.2]{\textbf{वच्मि}}~। \hyperref[1.2]{\textbf{यस्माद्व्यक्तं}} वर्णकल्पनानिरपेक्षं गणितं \hyperref[1.2]{\textbf{पूर्वं प्रोक्तम्}}~। ततः किमित्यत आह\textendash \,\hyperref[1.2]{\textbf{अव्यक्तबीजम्}} इति~। अव्यक्तं बीजगणितं बीजं मूलं यस्य~। तथा च पूर्वं प्रोक्तमपि व्यक्तं तावत्सम्यक्तया न ज्ञायते यावद्बीजक्रिया नोपपाद्यते~। तत्किं व्यक्तज्ञानार्थमेवायमारम्भः~। नेत्याह~। यस्माच्च सुधीभिरप्यव्यक्तयुक्त्या विना प्रश्ना ज्ञातुं प्रायो न शक्या मन्दधीभिस्तु नितान्तं ज्ञातुमशक्या एवेत्यर्थः~। प्रश्नाश्चात्र सिद्धान्तशिरोमणौ त्रिप्रश्नाधिकारे वक्ष्यमाणा भाकर्णे खगुणाङ्गुले किल सखे याम्यो भुजस्त्र्यङ्गुल इत्यादयः~। परे प्रश्नाध्यायोक्ता इतरे पृच्छकवशादपि ते ज्ञेयाः~। यद्वा तस्माद्व्यक्तं पूर्वं प्रोक्तमिदानीं बीजक्रियां च वच्मि~। यस्माद्व्यक्तयुक्त्या विना प्रश्नाः प्रायो बहुधा ज्ञातुं नो शक्याः~। तेनैवमुपलभ्यते केचन प्रश्ना व्यक्तयुक्त्यापि ज्ञातुं शक्यन्ते~। वक्ष्यति च {\color{violet}प्रश्नाध्याये\textendash }

\begin{quote}
{\color{violet}पाट्या च बीजेन च कुट्टकेन वर्गप्रकृत्या च तथोत्तराणि~।\\
गोलेन यन्त्रैः कथितानि तेषां बालावबोधे कतिचिच्च वच्मि~॥} इति~।
\end{quote}

तथा च प्रश्नोत्तरार्थज्ञानसाधनमव्यक्तं च भवति~। यतस्तस्माद्व्यक्तं पूर्वं प्रोक्तमिदानीं बीजक्रियां वच्मीत्यर्थः~। ननु प्रश्नोत्तरार्थज्ञानसाधनं द्वयमपि भवत्यतस्तर्हि त्वयोक्तमेतत्कथं व्यक्तं पूर्वेप्रोक्तमित्यत आह\textendash \,अव्यक्तबीजमिति~। अव्यक्तस्य बीजं मूलं तथा च यावत् व्यक्तगणितोक्तभिन्नपरिकर्माष्टकत्रैराशिकादिकं न ज्ञायते तावदव्यक्ते प्रवेशो न भवतीति व्यक्तं पूर्वं प्रोक्तमिति भावः~। तदेवं व्यक्तसापेक्षतया व्यक्तानन्तरं~ग्रहगणितोपयुक्ततया ग्रहगणितात् प्रागव्यक्तस्यारम्भो युक्त इति सङ्गतिः प्रदर्शिता~। असङ्गतप्रलापो हि प्रेक्षाव-तामनवधेयवचनो भवति~। बीजक्रियां
\end{sloppypar}

\newpage

\begin{sloppypar}
\noindent वच्मीति वदतैकवर्णानेकवर्णसमीकरणमध्यमाहरणभावितरूपभेदचतुष्टयभिन्नं गणितं विषयः प्रदर्शितः~। तदुपयुक्ततया धनर्णषड्विधकरणीषड्विधकुट्टकवर्गप्रकृतिचक्रवालान्यपि विषयत्वेन प्रदर्शितानि~। विषयस्य शास्त्रस्य च प्रतिपाद्यप्रतिपादकभावः~सम्बन्धोऽपि बीजक्रियां वच्मीत्यनेनैव दर्शितः~। यद्वा ज्ञातेऽपि विषये प्रयोजने च~वेदबाह्यैरहेतु-कैराधुनिकैः कल्पितमिदमुत पारम्पर्यागतमिति संशयेन नूतनकल्पितमेवेदं~शास्त्रमिति भ्रमेण वा प्रेक्षावन्तः शिष्टा न प्रवर्तेरन्~। तदर्थं पारम्पर्यलक्षणसम्बन्धकथनम् आवश्यकम्~। तच्च बीजगणितस्य प्रश्नज्ञानसाधनत्वं वदताचार्येण कृतमेव~। तथा हि अव्यक्तगणितं प्रश्नज्ञानसाधनत्वाज्ज्योतिषत्वाद्वेदाङ्गत्वाद्ब्रह्मणः सकाशात् वसिष्ठादिद्वारा पारम्पर्येणागतमित्युक्तं भवति~। उक्तं च {\color{violet}नारदेन "अस्ति शास्त्रस्य सम्बन्धो वेदाङ्गमिति धातृत"} इति~। आचार्योऽपि {\color{violet}गोलाध्याये} स्पष्टीकृतवासनायां वक्ष्यति\textendash 

\begin{quote}
{\color{violet}दिव्यं ज्ञानमतीन्द्रियं यदृषिभिर्ब्राह्मं वसिष्ठादिभिः\\
पारम्पर्यवशाद्रहस्यमवनीं नीतं प्रकाश्यं ततः~।\\
नैतद्द्वेषिकृतघ्नदुर्जनदुराचाराचिरावासिनां \\
स्यादायुःसुकृतक्षयो मुनिकृतां सीमामिमामुज्झतः~॥} इति~।
\end{quote}

प्रयोजनं तु प्रश्नोत्तरार्थज्ञानं गोलज्ञानं चापरं परम्परया जगतः शुभाशुभफलादेश्च~। यतो वक्ष्यति {\color{violet}गोलाध्याये\textendash }

\begin{quote}
{\color{violet}ज्योतिःशास्त्रफलं पुराणगणकैरादेश इत्युच्यते \\
नूनं लग्नबलाश्रितः पुनरयं तत्स्पष्टखेटाश्रयम्~।\\
ते गोलाश्रयिणोऽन्तरेण गणितं गोलोऽपि न ज्ञायते \\
तस्माद्यो गणितं न वेत्ति स कथं गोलादिकं ज्ञास्यति~॥} इति~।
\end{quote}

{\color{violet}नारदो}ऽपि\textendash \,{\color{violet}प्रयोजनं तु जगतः शुभाशुभनिरूपणम्} इति~। मुख्यं च शास्त्रप्रयोजनमेवास्य प्रयोजनम्~। यो ज्योतिषं वेत्ति नरः स सम्यग्धर्मार्थकामाल्लँभते यशश्चेति~। इहाधिकारी तु प्रश्नादिजिज्ञासुः पठितव्यक्तश्च~। स च द्विज एव~। यद्वक्ष्यति {\color{violet}सिद्धान्तशिरोमणौ\textendash }

\begin{quote}
{\color{violet}तस्माद्द्विजैरध्ययनीयमेतत्पुण्यं रहस्यं परमं च तत्त्वम्~॥} इति~।
\end{quote}

अत्रैवकारस्य पाठक्रमेण योजने ज्योतिषस्यावश्याध्ययनीयता प्रतीयते~। द्विजैरेवेति योजने द्विजातिरिक्तैरनध्ययनीयता च प्रतीयते~। द्वे अप्यत्र युक्ते इति~। ननु यद्वेति व्याख्याने अव्यक्तबीजमित्यत्र तत्पुरुषसमासे व्यक्तस्य कृत्स्नस्य तदेकबीजमिति सर्वग्रन्थविरोधः~। कश्चिदव्यक्तभागो व्यक्तस्य बीजं कश्चिद्व्यक्तभागोऽव्यक्तस्य बीजमिति न विरोध इति चेत्~। न~। कृत्स्नपदस्योक्त-
\end{sloppypar}

\newpage

\begin{sloppypar}
\noindent त्वात्~। न च व्यक्तस्य कृत्स्नस्य तदेकबीजमिति बीजस्य व्यक्तमूलकत्वेऽप्यविरुद्धमिति वाच्यम्~। व्यक्तिज्ञानेऽव्यक्तज्ञानमव्यक्तज्ञाने च व्यक्तज्ञानमिति परस्पराश्रयस्य दुस्तरत्वात्~। मैवम्~। {\color{violet}गङ्गा गङ्गेति यो ब्रूयाद्योजनानां शतैरपि~। मुच्यते सर्वपापेभ्य} इत्यादौ सर्वशब्दस्येव प्रकृते कृत्स्नपदस्य बहुत्वपरत्वात्~। इतरथा व्यक्तानन्तरमव्यक्तारम्भानुपपत्तेः~। अत एव कश्चन व्यक्तभागोऽव्यक्तमूलं कश्चिदव्यक्तभागो व्यक्तमूलमिति विरोधपरिहारो युक्त एव कृत्स्नपदे सङ्कोचस्यावश्याभ्युपेयत्वान्न हि व्यक्तोक्तसङ्कलनव्यवकलनादिष्वप्यव्यक्तं मूलमिति केनाप्युररी क्रियते~। किं तु गुणघ्नमूलोनेत्यादावेव~। किञ्च कृत्स्नपदे सङ्कोचाभावेऽपि न कश्चिद्दोषः~। तथाहि यथा गुणघ्नमूलोनेत्यादिव्यक्तगणितस्याव्यक्तमूलकत्वेऽपि न स्वरूप-निर्वाहाय तदपेक्षा किं तूपपत्तावेव तद्वदखिलस्यापि व्यक्तस्याव्यक्तमूलकत्वे कुतस्त्यः परस्पराश्रय इत्यलं पल्लवितेन~॥~२~॥ \\

{\small अव्यक्तक्रिया तावदव्यक्तषड्विधाधीना तदपि धनर्णषड्विधाधीनमतः प्रथमतस्तदत्र प्रतिपादनीयं तत्रापि व्यवकलनादीनां सङ्कलनपूर्वकत्वाद्धनर्णसङ्कलनं तावदुपजातिकापूर्वार्धेनाह\textendash }

\phantomsection \label{1.3}
\begin{quote}
{\large \textbf{{\color{purple}योगे युतिः स्यात्क्षययोः स्वयोर्वा धनर्णयोरन्तरमेव योगः~॥~३~॥}}}
\end{quote}

\hyperref[1.3]{\textbf{क्षययो}}र्ऋणयोः \hyperref[1.3]{\textbf{स्वयो}}र्धनयोर्वा योगे कर्तव्ये युतिः स्यात्~। एतदुक्तं भवति~। ययोर्योगः कर्तव्योऽस्ति तौ रूपात्मकौ करण्यात्मकौ वा राशी यद्युभावप्यृणगतौ धनगतौ वा भवतस्तदा तयो राश्योर्योगः कार्यः~। क्रमादुत्क्रमतोऽथ वाङ्कयोग इति व्यक्तगणितोक्तयोगो विधेयः~। स एवात्र योगो भवति~। करण्योस्तु योगोऽन्तरं वा योगं करण्योर्महतीं प्रकल्प्येत्यादि वक्ष्यमाणप्रकारेण विधेयमिति द्रष्टव्यम्~। एवं बहूनामपि सजातीययोग उक्तः~। यत्र त्वेको राशिर्धनमितरश्चर्णं तयोर्योगे कर्तव्ये किं कर्तव्यं तदाह\textendash \,\hyperref[1.3]{\textbf{धनर्णयोरन्तरमेव योग}} इति~। व्यक्तरीत्या यदन्तरं सम्पद्यते स एव धनर्णयोर्योग इत्यर्थः~। शेषस्य धनर्णवशाद्योगस्यापि धनर्णत्वं ज्ञेयम्~॥~३~॥\\

{\small अथोक्तेऽर्थे शिष्यबोधार्थमुदाहरणचतुष्टयमुपजातिकयाह\textendash }

\phantomsection \label{1.4}
\begin{quote}
{\large \textbf{{\color{purple}रूपत्रयं रूपचतुष्टयं च क्षयं धनं वा सहितं वदाशु~।\\
स्वर्णं क्षयः स्वं च पृथक्पृथक्त्वे धनर्णयोः सङ्कलनामवैषि~॥~४~॥}}}
\end{quote}

\hyperref[1.4]{\textbf{रूपत्रयं रूपचतुष्टयं चे}}ति~। द्वयमप्यृणमित्येकम्~। द्वयमपि धनमिति द्वितीयम्~। आद्यं धनमपरमृणमिति तृतीयम्~। प्रथममृणमितरद्धनमिति चतुर्थमेवं चत्वार्युदाहरणानि~। \hyperref[1.4]{\textbf{धनर्णयोरि}}ति~। धने चर्णे च धनर्णम्~। धनं चर्णं धनर्णम्~। धनर्णं च धनर्णं
\end{sloppypar}

\newpage

\begin{sloppypar}
\noindent च धनर्णे~। तयोर्धनर्णयोः~। धनयोर्ऋणयोर्धनर्णयोश्चेत्यर्थः~। चतुर्थप्रश्नस्य तृतीयेऽन्तर्भूतत्वात् पक्षत्रयमेवोद्दिष्टमिति~॥~४~॥ \\

{\small नन्विदं धनमिदमृणमिति वेदं व्यक्तमिदमव्यक्तमित्यादि वा कथमवधेयमित्यत आह\textendash }

\phantomsection \label{1.5}
\begin{quote}
{\large \textbf{{\color{purple}अत्र रूपाणामव्यक्तानां चाद्याक्षराण्युपलणार्थं लेख्यानि~। \\
तथा यान्यूणगतानि तान्यूर्ध्वबिन्दूनि चेति~॥~५~॥ }}}
\end{quote}

अतिरोहितार्थमिदम्~। यद्यप्यृणत्वादिकमालापत एवावगन्तुं शक्यं तथाप्यालापबहुत्व ऋणत्वादौ भ्रान्तिः संशीतिर्वा स्यात्~। उपस्थितिलाघवं च न स्यादित्यूर्ध्वबिन्द्वादिलिखनं युक्ततरम्~। धनर्णत्वं तु व्यवकलनोपपत्तौ विचारयिष्यामः~। अत्र प्रथमोदाहरणे न्यासः~। ३ं~। ४ं~। योगे जातं ७ं~। द्वितीये न्यासः~। ३~। ४ योगे जातम् ७~। तृतीये न्यासः~। ३~। ४ं~। धनर्णयोरन्तरमेव योग इति जातम् १ं~। चतुर्थे न्यासः~। ३ं~। ४ अन्तरमेव योग इति जातम् १~। अत्रोपपत्तिर्लोकसिध्दैव~। तथाहि देवदत्तस्य मुद्रात्रयमृणमेकमितरदपि मुद्राचतुष्टयमृणमित्यभिहिते मुद्रासप्तकमृणमस्तीति प्रतीतिरस्त्यागोपालाविपालेभ्यो व्यव-हारसिद्धा~। एवं देवदत्तस्य मुद्रात्रयं धनमेकमन्यदपि मुद्राचतुष्टयं धनमस्तीत्युक्तेऽस्त्यस्य मुद्रासप्तकं धनमिति विलसति सार्वजनीनो व्यवहारः~। अत उक्तम्\textendash \,योगे युतिः स्यात्क्षययोः स्वयोर्वेति~। अथ देवदत्तस्य मुद्रात्रयं धनमस्ति मुद्राचतुष्टयमृणमप्यस्तीत्युक्ते नास्य धनमस्ति किं तूत्तमर्णस्य मुद्रात्रये दत्त एकैव मुद्रास्यर्णमस्तीति वरीवर्ति सकलजनसाधारणो व्यवहारः~। अत उक्तं\textendash \,धनर्णयोरन्तरमेव योग इति~॥~९~॥\\ 

{\small ननु व्यक्ते भिन्नानामभिन्नानां च सङ्कलनव्यवकलनादि पृथक्पृथगुक्तम्~। अत्र तु भिन्नानां सङ्कलनव्यवकलनाद्यं च न पृथगभिहितमस्ति तत्कथं कर्तव्यमिति तदाह\textendash }

\phantomsection \label{1.6}
\begin{quote}
\begin{center}
{\large \textbf{{\color{purple}एवं भिन्नेष्वपीति~॥~६~॥}}}
\end{center}
\end{quote}

अयमर्थः\textendash \,सच्छेदानामपि रूपाणां वर्णानां वा योगार्थं धनर्णत्ववशाद्योगेऽन्तरे वा प्राप्ते योगोऽन्तरं तुल्यहरांशकानामित्यादिना योगोऽन्तरं वा विधेयमिति एवं भिन्नव्यव-कलनादिष्वपि बोद्धव्यम्~॥~६~॥\\

{\small यद्यपि व्यवकलनादीनां सङ्कलनोपजीवकत्वात्तत्प्राथम्येन सङ्कलननिरूपणं युक्तं न तथा गुणनप्राथम्येन व्यवकलने निरूपणं युक्तमुपजीव्योपजीवकभावाभावात्तथापि धनर्णताव्यत्यास-मात्रविलक्षणस्य व्यवकलनस्य गुणनापेक्षया सङ्कलनान्तरङ्गत्वात्खण्डगुणन इष्टोनयुक्तेन गुणेन निघ्न इत्यस्मिन्नपि गुणने तस्योपजीव्यत्वाच्च गुणनप्राथम्येन तन्निरूपणं युक्तमित्युपजातकोत्तरार्धेन तदाह\textendash }

\phantomsection \label{1.7}
\begin{quote}
{\large \textbf{{\color{purple}संशोध्यमानं स्वमृणत्वमेति स्वत्वं क्षयस्तद्युतिरुक्तवच्च~॥~७~॥ }}}
\end{quote}

\end{sloppypar}

\newpage

\begin{sloppypar}
संशोध्यतेऽपनीयते \hyperref[1.7]{\textbf{तत्संशोध्यमानम्}}~। रूपं वर्णः करणी चेति त्रिलिङ्गसामान्यं~नपुंस-कत्वम्~। तद्यदि धनमस्ति तर्हि ऋणत्वमेति~। यदि क्षयोऽस्ति तर्हि धनत्वमेति~।~पश्चात् उक्तवत्तद्युतिश्च~। एतदुक्तं भवति\textendash \,ययोरन्तरं विधेयमस्ति तयोर्मध्ये संशोध्यमानस्य धनर्णताव्यत्यासं कृत्वा योगे युतिः स्यादित्यादिना तयोर्युतिः कर्तव्या~। तदेव व्यव-कलनं फलं भवतीत्यर्थः~। अत्रोपपत्तिः\textendash \,ऋणत्वमिह त्रिधा तावदस्ति~। देशतः~कालतो वस्तुतश्चेति~। तच्च वैपरीत्यमेव~। यत उक्तमाचार्यैः {\color{violet}लीलावत्यां क्षेत्रव्यवहारे\textendash \,दशसप्त-दशप्रमौ भुजावि}त्यस्मिन्नुदाहरणे~। ऋणगताबाधादिग्वैपरीत्येनेत्यर्थ इति~। तत्रैकरेखा स्थिता द्वितीया दिग्विपरीता दिगित्युच्यते~। यथा पूर्वदिग्विपरीता पश्चिमा दिक्~। यथा चोत्तरदिग्विपरीता दक्षिण दिगित्यादि~। तथा च पूर्वापरदेशयोर्मध्य एकतरस्य धनत्वे कल्पिते तं प्रति तदितरस्यर्णत्वम्~। यथा पूर्वगतेर्धनत्वकल्पने यदा ग्रहः पश्चिमगतिर्भवति तदा ग्रहे गतितुल्यकला ऋणं भवति~। अथवा पश्चिमभ्रमस्य धनत्वे यावद्ग्रहः पूर्वतो गच्छति तावत्पश्चिमभ्रम ऋणमिति दक्षिणोत्तरदेशादिष्वप्येवमेवर्णत्वं बोध्यम्~। एवं पूर्वोत्तरकालयोरप्यन्योन्यमृणत्वं वारप्रवृत्त्यादिषु प्रसिद्धम्~। एवं यस्मिन् वस्तुनि यस्य स्वस्वामिभावसम्बन्धस्तस्य तद्धनमिति व्यवह्रियते~। तस्मिन्वैपरीत्यं तु परस्य स्वस्वामिभावसम्बन्धः~। अतो देवदत्तस्वामिके धने यावद्यज्ञदत्तस्वामिकत्वं~तावद्देव-दत्तस्यर्णमिति व्यवह्रियते~। तत्र पूर्वदेशस्य धनत्वं पश्चिमदेशस्य चर्णत्वं प्रकल्प्योपपत्तिः उच्यते~। सा यथा\textendash \,श्रीविश्वेशितुः शम्भोरानन्दकाननात्पुरन्दरदिशि पञ्चदशसु योजनेषु स्वर्गतरङ्गिणीतीरविलासि वरीवर्ति किलैकं पत्तनम्~। वरुणदिशि चाष्टयोजनेष्विन्दी-वरदलश्यामलपतङ्गतनयातरङ्गचुम्बिभिः शरच्चन्द्रिकाधवलैः सुरनदीलोलकल्लोलैः~स्मृत-हरिहरमूर्तिरानन्दलहरीरनुभवञ्जागर्ति तीर्थराजः प्रयागः~। तयोस्तूच्चावचसकलजनव्यव-हारसिद्धमस्ति त्रयोविंशतियोजनात्मकमन्तरम्~। तच्च योगं विना नोपपद्यते~। अतो विजातीययोरन्तरे साध्ये योगः कर्तव्यः परन्तु स योगः पश्चिमः पूर्वो वा~। तत्र पत्तनात् प्रयागः कस्यां दिशीति विचारे तावदानन्दकाननात् प्रयागपर्यन्तमष्टयोजनात्मको देशो यथा पश्चिमस्तथा पत्तनादपि पश्चिमो भवति किन्त्वानन्दकाननात् पत्तनपर्यन्तं पञ्चदशयोजनात्मकमेकं शकलम्~। ततः प्रयागावधि द्वितीयमष्टयोजनात्मकम्~। शकल-द्वयस्य पश्चिमस्थत्वाज्जातस्त्रयोविंशतियोजनात्मकः पश्चिमो देशः~। एवं प्रयागात्पत्तनं कस्यां दिशीति विचारे प्रयागादानन्दवनपर्यन्तं देशशकलं विपरीतदिक्कं भवति~। तथा च यस्मादन्तरं साध्यते तदवधि शकलं विपरीतदिक्कं भवतीत्यत उक्तं संशोध्यमानं स्वमृणत्वमेति स्वत्वं क्षय इति~। एवं धनर्णयोरन्तरे प्रतिपादितम्~। एवं धनयोरपि~। तद्यथा~। एकः किल काशीतः पूर्वदिग्भागे दशयोजनानि गत इतरोऽपि तस्मिन्नेव सप्तयोजनानि गतस्तयोश्चान्तरं योजनत्रयं सर्वजनप्रसि-
\end{sloppypar}

\newpage

\begin{sloppypar}
\noindent द्धम्~। तच्च दशयोजनगात्पश्चिमम्~। सप्तयोजनगात्पूर्वम्~। इदमपि प्रथमावधिभूतस्य खण्डस्य व्यत्यासे कृते धनर्णयोरन्तरमेव योग इति योगे च कृते सिध्यति~। एवमृणयोरपि बोध्यम्~। अत उपपन्नं संशोध्यमानं स्वमृणत्वमेति स्वत्वं क्षयस्तद्युतिरुक्तवच्चेति~। अन्यदपि सुधीभिरूहनीयम्~॥~७~॥\\

{\small अत्रोदाहरणचतुष्टयमुपजातिकापूर्वार्धेनाह\textendash }

\phantomsection \label{1.8}
\begin{quote}
{\large \textbf{{\color{purple}त्रयाद्द्वयं स्वात्स्वमृणादृणं च व्यस्तं च संशोध्य वदाशु शेषम्~॥~८~॥}}}
\end{quote}

स्वात्त्रयात् स्वं द्वयमित्येकमृणात्त्रयादृणं द्वयमिति द्वितीयमित्युदाहरणद्वयम्~। व्यस्तत्वे च स्वात्त्रयादृणं द्वयमित्येकमृणात्त्रयात्स्वं द्वयमिति द्वितीयमेवं चत्वार्युदाहरणानि~। तत्र प्रथमे न्यासः ३~। २~। संशोध्यमानं २ स्वमृणत्वमेतीति जातम्~। ३~। २ं~। अनयोर्युतिरुक्तवत्~। धनर्णयोरन्तरमेव योग इति जातम्~। १~। द्वितीये न्यासः ३ं~। २ं~। जातमुक्तवदन्तरम्~। १ं~। तृतीये न्यासः ३~। २ं~। संशोध्यमानं क्षयः स्वत्वमेतीत्यादिना जातं ५~। चतुर्थे न्यासः ३ं~। २~। संशोध्यमानं स्वमृणत्वमेतीत्यादिना जातं ५ं~। इदमेव प्रतीत्यर्थं पूर्वपश्चिमदेशत्वेन योज्यते~। पू.\;३ पू.\;२ संशोध्यमानः पूर्वदेशः पश्चिमदेशो भवतीति ज्ञातं पू.\;३ प.\;२~। अनयोर्धनर्णयोरन्तरमेव योग इति शेषमन्तरं पू.\;१~। अत्रैकस्मादवधेः पूर्वतो योजनद्वयेन त्रयेण च नरौ तिष्ठतः~। तत्र योजनद्वयगतात् पुंसो योजनत्रयगो योजनमेकं पूर्वतस्तिष्ठतीत्यर्थः~। अत्रोदाहरणेषु द्वयस्य शोध्यतोक्तेर्योजनद्वयगान्नरादन्तरं ज्ञातव्यम्~। अथ द्वितीये प.\;३ प.\;२~। उक्तवदन्तरे जातं प.\;१~। पश्चिमतो योजनत्रयगतः पश्चिमतो योजनद्वयगतादेकेन योजनेन पश्चिमतस्तिष्ठतीत्यर्थः~। तृतीये न्यासः पू.\;३ प.\;२~। उक्तवदन्तरे जातं पू.\;५~। पश्चिमतो योजनद्वयगतात्पुंसः पूर्वतो योजनत्रयगः पञ्चभिर्योजनैः पूर्वतस्तिष्ठतीत्यर्थः~। चतुर्थे न्यासः प.\;३ पू.\;२ उक्तवज्जातमन्तरं प.\;५~। पूर्वतो योजनद्वयगतात्पश्चिमतो योजनत्रयगः पञ्चभिर्योजनैः पश्चिमतस्तिष्ठतीत्यर्थः~॥~८~॥\\

{\small अथ भागहारादीनां गुणनोपजीवकत्वाद्भुजङ्गप्रयातपूर्वार्धखण्डेन गुणनमाह\textendash }

\phantomsection \label{1.9}
\begin{quote}
{\large \textbf{{\color{purple}स्वयोरस्वयोः स्वं वधः स्वर्णघाते~। \\
क्षयः~॥~९~॥}}}
\end{quote}

\hyperref[1.9]{\textbf{स्वयोरस्वयो}}र्वा \hyperref[1.9]{\textbf{वधो}} गुणनम्~। एकस्यापरतुल्यावृत्तिरिति यावत्~। धनं भवति~। \hyperref[1.9]{\textbf{स्वर्णघाते क्षयो}} भवति~। एतदुक्तं भवति~। यदा गुण्यो गुणकश्चेति द्वावपि धनमृणं वा भवतस्तदा तदुत्थं गुणनफलं धनं भवति~। यदा त्वेकतरो धनमृणमितरस्तदा तदुत्थं गुणनफलमृणं भवतीति~। अत्र गुणनफलस्य धनर्णत्वमात्रं प्रतिपादितम्~। अङ्कतस्तु व्यक्तोक्ताः सर्वेऽपि गुणनप्रकारा द्रष्टव्याः~॥~९~॥
\end{sloppypar}

\newpage

\begin{sloppypar}
{\small अथ गुणनोदाहरणत्रयमुपजातिकोत्तरार्धेनाह\textendash }

\phantomsection \label{1.10}
\begin{quote}
{\large \textbf{{\color{purple}धनं धनेनर्णमृणेन निघ्नं द्वयं त्रयेण स्वमृणेन किं स्यात्~॥~१०~॥}}}
\end{quote}

ऋणं धनेनेति चतुर्थमप्युदाहरणं द्रष्टव्यम्~। अत्र गुणकः ३ गुण्यः २~। अथ प्रथमे न्यासः~। २~। ३~। उक्तवज्जातं गुणनफलं धनं ६~। द्वितीये न्यासः २ं~। ३ं~। अस्वयोर्वधः स्वमिति जातम् ६~। तृतीये न्यासः २ं~। ३~। स्वर्णघाते क्षय इति जातं ६ं~। चतुर्थे न्यासः २~। ३ं~। स्वर्णघाते क्षय इति ६ं~। गुण्येन हते गुणके च तदेवेति चूर्णिकया गुण्यत्वगुणकत्वयोः कामचारः प्रदर्शितः~। ननु स्वयोर्वधः स्वं भवितुमर्हति~। समजातीयत्वाद्दृष्टचरत्वाच्च~। परमृणयोर्वधः कथं धनं भवितुमर्हति विजातीयत्वात्~। एवं स्वर्णघातेऽपि क्षयः कथं भवति~। न च विजातीयत्वादिति वाच्यम्~। वैपरीत्यस्यापि सुवचत्वाद्धनमेव कथं न स्याद्विनिगमनाविरहात्~। अत्रोच्यते\textendash \,गुण्यस्य गुणकतुल्यावृत्तिर्हि गुणनफलमिति तावत्प्रसिद्धम्~। तत्र गुणको द्विविधः~। धनमृणं चेति~। तत्र धनगुणके सति धनस्य, ऋणस्य वा गुण्यस्यावर्तने क्रियमाणे क्रमेण धनमृणं च गुणनफलं स्यात्~। अतः स्वयोर्वधः स्वम्~। गुणकस्य धनत्वे गुण्यस्यर्णत्वे त्वृणमिति सिद्धम्~।\\

अथर्णगुणके विचारः\textendash \,तत्रर्णत्वं वैपरीत्यमिति प्रागेव प्रतिपादितम्~। तथा च ऋण-गुणको नाम विपरीतगुणकः~। गुण्यस्य विपरीतावर्तनकर इति यावत्~। तथा सति धने गुण्ये गुणनफलमृणम्~। ऋणे गुण्ये गुणनफलं धनमिति सिद्धम्~। अत्रान्तिम-पक्षेऽस्वयोर्वधः स्वमित्युपपन्नम्~। मध्यमपक्षयोस्तु गुण्यगुणकयोरेकतरस्य धनत्वेऽन्यस्य-र्णत्वे फलमृणमुत्पद्यत इति स्वर्णघाते क्षय इत्युक्तम्~। यद्वा गणितेनोपपत्तिः प्रदर्श्यते~। धनगुणने तावदविवाद एव~। ऋणगुणने तु विचारः~। अस्ति तावदिदं सुप्रसिद्धं गुण्यगुणकखण्डाभ्यां पृथग्गुणितः सहितश्च गुणनफलं भवतीति~। यथा गुण्यः १३५ गुणकः १२~। अस्य खण्डद्वयं ४~। ८~। एकमिष्टमिष्टोनो राशिरपरं च~। खण्डाभ्यां पृथग्गुणितो गुण्यः ५४०~। १०८०~। योगे जातं गुणनफलं १६२०~। एकमेव कल्पितमिष्टं ४ं~। एतदूनो राशिः १२ द्वितीयं खण्डं १६~। अत्रापि पृथक्खण्डद्वयगुणितेन सहितेन च गुण्येन गुणनफलेन च भवितव्यम्~। तत्र खण्डाभ्यां ४ं~। १६~। पृथग्गुणितो गुण्यः $\dot{\hbox{५४०}}$~। २१६०~। अनयोर्योगे गुणनफलं नोपपद्यत इति~। गुणनफलान्यथानुपपत्त्या स्वर्णघाते क्षयो भवतीत्यवगम्यते~। यतस्तथा कृतो $\dot{\hbox{५४०}}$~। २१६०~। धनर्णयोरन्तरमेव योग इति १६२० गुणनफलमुपपद्यते~। अत उक्तं स्वर्णघाते क्षय इति~। एवं गुण्यखण्डे प्रत्येकं
\end{sloppypar}

\newpage

\begin{sloppypar}
\noindent गुणकखण्डगुणिते सहिते च गुणनफलं भवति~। तद्यथा\textendash \,गुण्यः १३५ एतस्य खण्डद्वयं १३०।५~। गुणकस्यापि खण्डद्वयं ४।८~। गुणकखण्डाभ्यां प्रत्येकं गुणितं गुण्यपूर्वखण्डं १३० जातं ५२०।१०४०~। एवमेव प्रत्येकं गुणितं द्वितीयखण्डं ५ जातं २०।४०~। सर्वेषां योगे जातं गुणनफलं १६२०~। एवमेव कृतमभीष्टं खण्डद्वयं गुण्यस्य १४०।५ं~। गुणकस्यापि १६।४ं~। अत्रापि गुणकखण्डाभ्यां प्रत्येकं गुणितं पूर्वखण्डं १४० जातं २२४०।$\dot{\hbox{५६०}}$~। अनयोर्योगः १६८०~। एवमेव द्वितीयमपि ५ं गुणकखण्डाभ्यां~पृथक् गुणितं $\dot{\hbox{८०}}$।२० अत्रर्णगुणितमृणं सजातीयत्वादृणमेवेति कृते गुणनफलं नोपपद्यत इति गुणनफलान्यथानुपपत्त्या, ऋणमृणगुणितं धनं भवतीत्यवगम्यते~। यतस्तथा कृते $\dot{\hbox{८०}}$।२० । गुणनफलं १६२० उपपद्यत इत्यत उक्तमस्वयोर्वधः स्वमिति~। एवं बुद्धिमद्भिरन्यदप्यूह्यम्~। ननु वर्गस्य समाद्विघातरूपतया गुणनान्तरङ्गत्वाद्भजनानपेक्षत्वाच्च प्रथमतो निरूपणं युक्तम्~। न च {\color{violet}"भक्तो गुणः शुध्यति" [ली.\;५]} इत्यादिना गुणनप्रकारेण वर्गकरणे भजनस्योपजीव्यतया तस्यैव प्राथम्येन निरूपणं युक्तमिति वाच्यम्~। गुणनादपि पूर्वं तन्निरूपणप्रसङ्गादिति चेन्न~। वर्गकरणप्रकाराणामतिविलक्षणतया वर्गस्य गुणनं प्रति बहिरङ्गत्वात्~। प्रत्युत वर्गं प्रति पदस्येव गुणनं प्रति भजनस्यैवान्तरङ्गत्वाद्वर्गं प्रत्युपजीव्यत्वाच्च प्रथमतस्तन्निरूपणस्यैवावश्यकत्वात्~॥~१०~॥\\

{\small कस्यचिद्गुणनप्रकारस्य भजनसापेक्षत्वेऽपि भजननिरपेक्षतयापि गुणनस्य सिद्धत्वाद्भजनस्य तु सर्वथा गुणनसापेक्षत्वाद्गुणनानन्तरमेव तन्निरूपणं युक्तमिति भुजङ्गप्रयातपूर्वार्धशेषशकलेन तदाह\textendash }

\phantomsection \label{1.11}
\begin{quote}
{\large \textbf{{\color{purple}भागहारेऽपि चैवं निरुक्तम्~॥~११~॥}}}
\end{quote}

\hyperref[1.11]{\textbf{भागहारेऽपि}} गुणनवदेव \hyperref[1.11]{\textbf{निरुक्तम्}} इत्यर्थः~। एतदुक्तं भवति\textendash \,भाज्यभाजकयोरुभयोरपि धनत्वे, ऋणत्वे वा लब्धिर्धनमेव~। यदा त्वेकतरस्य धनत्वमृणत्वमितरस्य तदा लब्धम् ऋणमेवेति~। अत्राप्यङ्कतो भागप्रकारो व्यक्तोक्तो ज्ञेयः~॥~११~॥\\

{\small अत्रोदाहरणचतुष्टयमुपजातिकयाह\textendash }

\phantomsection \label{1.12}
\begin{quote}
{\large \textbf{{\color{purple}रूपाष्टकं रूपचतुष्टयेन धनं धनेनर्णमृणेन भक्तम्~।\\
ऋणं धनेन स्वमृणेन किं स्याद्द्रुतं वदेदं यदि बोबुधीषि~॥~१२~॥}}}
\end{quote}

स्पष्टोऽर्थः~। प्रथमे न्यासः\textendash \;{\scriptsize $\begin{matrix}
\mbox{{८}}\\
\vspace{-1.5mm}
\mbox{{४}}
\vspace{1mm}
\end{matrix}$} स्वयोर्भागहारः स्वमिति जाता लब्धिर्धनं २~। द्वितीये न्यासः {\scriptsize $\begin{matrix}
\mbox{{८ं}}\\
\vspace{-1.5mm}
\mbox{{४ं}}
\vspace{1mm}
\end{matrix}$} अस्वयोर्भागहारः स्वमिति जाता लब्धिर्धनमेव २~। तृतीये
\end{sloppypar}

\newpage

\begin{sloppypar}
\noindent न्यासः\textendash \;{\scriptsize $\begin{matrix}
\mbox{{८ं}}\\
\vspace{-1.5mm}
\mbox{{४}}
\vspace{1mm}
\end{matrix}$} स्वर्णभागहारे क्षय इति जाता लब्धिः, ऋणं २ं~। चतुर्थे न्यासः\textendash \,{\scriptsize $\begin{matrix}
\mbox{{८}}\\
\vspace{-1.5mm}
\mbox{{४ं}}
\vspace{1mm}
\end{matrix}$} स्वर्णभागहारे क्षय इति जाता लब्धिः, ऋणं २ं~। अत्रोपपत्तिः\textendash

\begin{quote}
{\color{violet}भाज्याद्धरः शुध्यति यद्गुणः स्यादन्त्यात्फलं तत्खलु भागहारे~।}
\end{quote}

\noindent इत्युक्तत्वाद्यस्मिन्नङ्के हरगुणिते भाज्यादपनीते शुद्धिर्भवति सा किल लब्धिः~। तत्र प्रथमे\textendash \,{\scriptsize $\begin{matrix}
\mbox{{८}}\\
\vspace{-1.5mm}
\mbox{{४}}
\vspace{1mm}
\end{matrix}$} धनेन द्वयेन हरे ४ गुणिते ८ भाज्यात्, ८ अपनीते शुद्धिर्भवतीति धनं द्वयं २ लब्धिः~। द्वितीयेऽपि {\scriptsize $\begin{matrix}
\mbox{{८ं}}\\
\vspace{-1.5mm}
\mbox{{४ं}}
\vspace{1mm}
\end{matrix}$} धनद्वयेन हरेऽ\textendash \,४ं\textendash \,स्मिन्गुणिते ८ं भाज्या\textendash \,८ं\textendash \,दस्मादपनीयमाने संशोध्यमानं क्षयः स्वत्वमेतीति \hyperref[1.3]{"धनर्णयोरन्तरमेव योगः"} इति च कृते बुद्धिर्भवतीति द्वयं धनमेव लब्धिः २~। एवं सिद्धम्~। स्वयोरस्वयोर्वा भागहारे स्वमिति~। तृतीये तु {\scriptsize $\begin{matrix}
\mbox{{८ं}}\\
\vspace{-1.5mm}
\mbox{{४}}
\vspace{1mm}
\end{matrix}$} धनद्वयेन हरे ४ गुणिते ८ भाज्यादस्मात्, ८ं अपनीते संशोध्यमानं स्वमृणत्वमेतीति ऋणयोर्योगे $\dot{\hbox{१६}}$ शुद्धिर्न स्यादृणगुणिते तु हरे ८ं शुद्धिर्भवतीत्यृणद्वयं लब्धिः २ं~। एवं चतुर्थेऽपि {\scriptsize $\begin{matrix}
\mbox{{८}}\\
\vspace{-1.5mm}
\mbox{{४ं}}
\vspace{1mm}
\end{matrix}$} ऋणगुणित एव हरः शुध्यतीति ऋणमेव लब्धिरिति सिद्धं स्वर्णभागहारे क्षय इति~। अत उक्तं भागहारेऽपि चैवं निरुक्तमिति~॥~१२~॥\\

{\small एवं सकलवर्गोपयुक्तमुक्त्वा वर्गं तन्मूलं च भुजङ्गप्रयातोत्तरार्धेनाह\textendash }

\phantomsection \label{1.13}
\begin{quote}
{\large \textbf{{\color{purple}कृतिः स्वर्णयोः स्वं स्वमूले धनर्णे \\
न मूलं क्षयस्यास्ति तस्याकृतित्वात्~॥~१३~॥}}}
\end{quote}

स्वस्य, ऋणस्य वा वर्गः स्वं भवति~। अङ्कतस्तु वर्गप्रकारा व्यक्तोक्ताः सर्वेऽपि द्रष्टव्याः~। अथ मूलमाह\textendash \,\hyperref[1.13]{\textbf{स्वमूले धनर्णे}} इति~। स्वस्य धनस्य मूले धनर्णे स्याताम्~। धनस्यैव वर्गस्य, ऋणमपि मूलं भवतीत्यर्थः~। अथात्र विशेषमाह\textendash \,\hyperref[1.13]{\textbf{न मूलं क्षयस्यास्ती}}ति~। तत्र हेतुमाह\textendash \,\hyperref[1.13]{\textbf{तस्याकृतित्वात्}} इति~। वर्गस्य हि मूलं लभ्यते~। ऋणाङ्कस्तु न वर्गः~। कथमतस्तस्य मूलं लभ्यते~। ननु, ऋणाङ्कः कुतो वर्गो न भवति~। न हि राजनिदेशः~। किञ्च यदि न वर्गस्तर्हि वर्गत्वं निषेद्धुमप्यनुचितमप्रसक्तेः~। सत्यम्~। ऋणाङ्कं वर्गं वदता भवता कस्य स वर्ग इति वक्तव्यम्~। न तावद्धनाङ्कस्य~। समद्विघातो हि वर्गः~। तत्र धनाङ्केन धनाङ्के गुणिते यो वर्गो भवेत् स धनमेव~। स्वयोर्वधः स्वमित्युक्तत्वात्~। नाप्यृणाङ्कस्य~। तत्रापि समद्विघातार्थमृणाङ्केनर्णाङ्के गुणिते धनमेव वर्गो भवेत्~। अस्वयोर्वधः स्वमित्युक्तत्वात्~। एवं सति कमपि तमङ्कं न पश्यामो यस्य वर्गः क्षयो भवेत्~। न चाप्रसक्तिः~। अङ्कसादृश्याद्भ्रान्त्या वर्गत्वप्रसक्तेः~। वर्गयुक्तिस्तु गुणनयुक्तिरेव~। मूले तु व्यस्तविधिरेवोपपत्तिः~॥~१३~॥
\end{sloppypar}

\newpage

\begin{sloppypar}
{\small अथ वर्गोदाहरणद्वयमुपजातिकापूर्वार्धेनाह\textendash }

\phantomsection \label{1.14}
\begin{quote}
{\large \textbf{{\color{purple}धनस्य रूपत्रितयस्य वर्गं क्षयस्य च ब्रूहि सखे ममाशु~॥~१४~॥}}}
\end{quote}

स्पष्टोऽर्थः~। प्रथमे न्यासः~। ३ जातो वर्गः ९ स्वम्~। द्वितीये न्यासः ३ं जातो वर्गः ९ स्वमेव~। कृतिः स्वर्णयोः स्वमित्युक्तत्वात्~॥~१४~॥ \\

{\small अथोत्तरार्धेन मूलोदाहरणद्वयमाह\textendash }

\phantomsection \label{1.15}
\begin{quote}
{\large \textbf{{\color{purple}धनात्मकानामधनात्मकानां मूलं नवानां च पृथग्वदाशु~॥~१५~॥}}}
\end{quote}

अतिरोहितार्थम्~। [प्रथमे] न्यासः ९ जातं मूलं ३ वा ३ं~। स्वमूले
धनर्णे इत्युक्तत्वात्~। द्वितीये न्यासः~। ९ं एषामवर्गत्वान्मूलं नास्ति~। धने धनपदे वा न कश्चिद्धनर्णत्वकृतो विशेषः~। किन्तु सजातीयत्वमेवेति नात्र तन्निरूपणमिति ध्येयम्~॥~१५~॥

\begin{quote}
{\color{violet}दैवज्ञवर्यगणसन्ततसेव्यपार्श्वबल्लालसञ्ज्ञगणकात्मजनिर्मितेऽस्मिन्~।\\
बीजक्रियाविवृतिकल्पलतावतारे स्वर्णोद्भवाः समभवन्निति षट्प्रकाराः~॥}
\end{quote}
\vspace{-1mm}

\begin{center}
इति श्रीसकलगणकसार्वभौमश्रीबल्लालदैवज्ञसुतकृष्णगणकविरचिते \\
बीजविवृतिकल्पलतावतारे धनर्णे (र्ण?) षड्विधविवरणम्~। \\
(अत्र मूलं मूलश्लोकैः सह ग्रन्थसङ्ख्या दशाधिकशतत्रयम्~।) \\
\vspace{6mm}

\rule{0.2\linewidth}{0.8pt}\\
\vspace{-4mm}

\rule{0.2\linewidth}{0.8pt}
\end{center}
\end{sloppypar}

\newpage
\thispagestyle{empty}

\begin{center}
\textbf{\large २\; शून्यषड्विधम्~।}\\
\rule{0.2\linewidth}{0.8pt}
\end{center}

\begin{sloppypar}
अथ यथा स्वरूपवर्णादिषड्विधोपयुक्ततया धनर्णषड्विधस्य प्रथमतो निरूपणं युक्तं तथा खषड्विधस्यापि तद्युक्तम्~। तच्च यद्यपि व्यक्तोक्तशून्यपरिकर्माष्टकेनात्र धनर्णषड्विधेन च गतार्थमिति नारम्भणीयं तथापि यद्यत्र नारभ्येत तर्हि शिष्यैर्व्यक्तोक्तशून्यपरिकर्ममार्गेणैव शून्यगणितं क्रियेत न तु धनर्णकृतो विशेषोऽनवधानाद्भ्रमाद्वेति तन्निरासार्थमिह तदारम्भणं युक्तमेव~। ननु खं हि शून्यमभाव इति यावत्~। तस्य सङ्कलनादिषड्विधं न सम्भवति~। सङ्कलनादिफलस्य सङ्ख्याधर्मत्वात्~। न च सङ्ख्यायाः शून्येन सह सङ्कलनाद्ये कर्तव्ये मा भूच्छून्ये सङ्कलनादिफलं किन्तु सङ्ख्यायामेव तदस्त्विति वाच्यम्~। एवमपि खचतुर्विधमेव सम्भवेन्न खषड्विधं वर्गमूलयोस्तदसम्भवात्~। वस्तुतस्तु द्वितीयसङ्ख्याया अभावात् सङ्कलनादेरप्यसम्भव एव~। तस्य सङ्ख्याद्वयसाध्यत्वादिति~। अत्रोच्यते\textendash \,अस्त्वेव शून्यस्यापि सङ्कलनादिसम्भवः~। न च द्वितीयसङ्ख्याया अभावात्तदसम्भव इति वाच्यम्~। शून्यसङ्कलनादावपि द्वितीयसङ्ख्यायाः सत्त्वात्~। तद्यथा\textendash \,पञ्चोत्तरशतस्य १०५ विंशत्या २० योगे कर्तव्ये स यथास्थानं कार्यः~। तत्रैकस्यां सङ्ख्यायां दशकस्थाने शून्यमेकस्थाने पञ्च~। इतरस्यां दशकस्थाने द्वयमेकस्थाने  शून्यमिति~। अस्त्यत्र शून्यसङ्कलनेऽपि सङ्ख्याद्वयम्~। एवं व्यवकलनादिष्वपि ज्ञेयम्~। एवं {\color{violet}"स्थाप्योऽन्त्यवर्ग"} इत्यादिना वर्गकरणे {\color{violet}"स्थाप्यो घनोऽन्त्यस्य ततोऽन्त्यवर्गः"} इत्यादिना घनकरणे च शून्यवर्गघनयोरपि सम्भवो द्रष्टव्यः~। ननु शून्यं किं सङ्ख्यान्तर्गतमभावो वेति व्युत्पादयन्त्यार्याः~। अस्ति ते जिज्ञासा यदि तच्छ्रूयताम्~। सविशेषमिदं सङ्ख्याव्युत्पादनम्~। तथा हि\textendash \,इह किल सकलचराचरनिर्माता भगवान्परमकारुणिकः स्वयंभूस्तत्क्रमविशेषविशिष्टवर्णमयानि शास्त्राणि सृष्ट्वाथाल्पमेधसां तदुपस्थितये मेधाविनां तु तदुपस्थितिलाघवाय सति विस्मरणेऽन्यनिरपेक्षं तत्स्मरणाय चाश्रुतपरकृतग्रन्थावगमाय च यथा वर्णज्ञापकलिपीः ससर्ज तथा सङ्ख्योपस्थितिलाघवाय तज्ज्ञापकानङ्कानप्यसृजत्~। तत्र प्रतिवर्णं लिपिसर्गे वर्णानामियत्तया तज्ज्ञापकलिपिष्वपि सास्तीति लिपिषु सङ्केतग्रहः सुशकः~। इह तु प्रतिसङ्ख्यमङ्कसर्गे सङ्ख्यानामानन्त्यात्तज्ज्ञापकाङ्केषु वर्षशतेनाप्यशक्यः सङ्केतग्रहः~। तथा हि\textendash \,इह कुशाग्रबुद्धेरपि प्रतिदिनं यथाकथञ्चिच्छतपर्यन्तमपि सङ्केतग्रहे तदेकचित्ततया शतवर्षपर्यन्तमभ्यासेन षट्त्रिंशल्लक्षपर्यन्तं सङ्केतग्रहः स्यान्मेधाविनः~। न तु तदधिक-सङ्ख्याज्ञापकाङ्केष्विति~। अतः परमकारुणिको भगवानतिचतुरो नवैवाङ्कान्ससर्ज~। यथा १~। २~। ३~। ४~। ५~। ६~। ७~। ८~। ९~। 
\end{sloppypar}

\newpage

\begin{sloppypar}
\noindent अथ चाभीष्टस्थानाद्वामक्रमेण द्वितीयतृतीयादिस्थानान्युत्तरोत्तरं दशगुणानां सङ्ख्यानां सञ्ज्ञाभिर्दशशतादिभिरसङ्केतयत्~। प्रथमस्थानं चैकगुणसङ्ख्यास्थानत्वादेकसञ्ज्ञया~।~तथा सति नवैवाङ्कास्तत्र स्थानसम्बन्धात्~। स्थानानि वा तत्तदङ्कसम्बन्धाद्यथा~स्वान्तान्तां सङ्ख्यां ज्ञापयेयुरिति सकलसङ्ख्यावगमः सुगम इति~। यथाभीष्टस्थाने निवेशितोऽयमङ्क ३ एकगुणायास्त्रित्वसङ्ख्याया ज्ञापको भवति~। ततो वामतो द्वितीयस्थाने निवे-शितः स्वसङ्ख्याया दशकज्ञापको भवति~। यथा दशकद्वयज्ञापकोऽयम् २३~। एवं वामतस्तृतीयचतुर्थपञ्चमादिस्थाननिवेशितोऽङ्क उत्तरोत्तरं दशगुणानां शतसहस्रायुता-दीनां यथास्वं ज्ञापको भवति~। तत्राभीष्टसङ्ख्याया यथासम्भवमेकदशकशताद्यभावे तत्स्थानपूरणार्थमभावद्योतकाङ्कः शून्यसञ्ज्ञको लिपिविशेषो निवेश्यते~।~यथाष्टो-त्तरशतसङ्ख्याया दशकाभावाद्द्वितीयस्थाने शून्यनिवेशनं १०८~। यथा वाष्टोत्तरसहस्रसङ्ख्यायां दशकशतकयोरभावाद्द्वितीयतृतीयस्थानयोस्तत् १००८~। अनयोरुदाहृतसङ्ख्ययोर्यथाक्रमम् अष्टकशतकयोरष्टकसहस्रकयोरेव वानिवेशे १८ द्वितीयस्थानानिवेशितस्य दशकज्ञापक-त्वादष्टादशत्वं प्रतीयेत नाभीष्टसङ्ख्या~। अत एवात्रायुतलक्षादीनामभावेऽपि~तत्स्थाने शून्यं निवेश्यते~। तेन विनाप्यभीष्टसङ्ख्याज्ञापकस्थानपूरणात्~। अतोऽभीष्टसङ्ख्यायाम् उत्तरावधिभूताङ्कस्थानाद्दक्षिणस्थानानां पूरकत्वात्तत्रोक्तरीत्या शून्यनिवेशनमावश्यकम्~। वामस्थानानां त्वपूरकत्वादानन्त्याच्च न तत्तथेति~। नन्वस्ति लिपिषु सव्यक्रमः शिष्टसंमतो माङ्गलिकत्वादादरणीयश्च तत्कथं तमपहायापसव्यक्रम आदृत इति चेत्~। न~। शतसहस्रायुतलक्षादिसङ्ख्याया उत्तरोत्तरमभ्यर्हितत्वात् तत्सव्यक्रमस्योचितत्वादेतत्क्रमस्य युक्तत्वात्~। न चाभ्यर्हितसङ्ख्यातः सव्यक्रमार्थमुत्तरावधितः प्रदक्षिणक्रमेणैव द्वितीया-दिस्थानानां सञ्ज्ञास्त्विति वाच्यम्~। उत्तरावधेरभावात्~। परिच्छिन्नसङ्ख्यासु तत्सत्त्वेऽपि तस्यानियतत्वात्~। प्रथमावधेस्तु नियतत्वात्तत्स्थानमारभ्य स्थानसञ्ज्ञायुक्ततरेत्यलं पल्लवितेन~। तदेवं शून्यस्याभावत्वेऽपि तत्सङ्कलनादेर्न सङ्ख्याद्वयसाध्यत्वहानिः~। न हि द्वितीयसङ्ख्याया उभयोर्वा सङ्ख्ययोर्दशकाद्यभावमात्रेण सर्वथा तदभाव इति~। वस्तुतस्तु सङ्ख्याया दशकाद्यभावे सर्वथाप्यभावे वेत्यभावमात्रे यत्षड्विधं तत्खषड्विधमुच्यते~। अन्यथानन्तस्य खहरराशेः खमूलस्य चासम्भवात्~। ननु द्वितीयसङ्ख्यायाः सर्वाथा-प्यभावे कथं सङ्कलनादेः सम्भवस्तस्य सङ्ख्याद्वयसाध्यत्वादित्युक्तमेवेति चेत् न~। खसङ्कलनादेरतथात्वात्~। ययोः सङ्ख्यासङ्कलनादिना यस्य सङ्ख्या सम्भवति तयोरन्य-तरस्योभयोर्वाभावे तस्य सङ्ख्यायाः सङ्ख्याभावस्य वा खसङ्कलनादिफलत्वात्~। यथा शरक्रान्तिसङ्ख्ययोर्यथासम्भवं सङ्कलनेन व्यवकलनेन वा स्फुटक्रान्तिसङ्ख्या भवतीति तयोरन्यतरस्योभयोर्वा भावे स्फुटक्रान्तेः सङ्ख्यायास्तदभावस्य वा यथास्वं खसङ्कलन-व्यवकलनफलत्वम्~। एवं गुणनादिष्वपि बोध्यम्~। न च वस्तुतः खषड्विधाभावे किमनेन 
\end{sloppypar}

\newpage

\begin{sloppypar}
\noindent परिभाषामात्रेणेति वाच्यम्~। अस्ति महत्प्रयोजनमेतस्याः परिभाषायाः~। तथा हि\textendash \,यदि परिभाषा न विधीयेत तदा क्रान्तिशरयोः सत्त्वे तयोरेकभिन्नदिक्वे तत्सङ्ख्यासङ्कलन-व्यवकलनाभ्यां स्फुटक्रान्तिसङ्ख्या भवति~। एकस्यैव सत्त्वे तत्सङ्ख्यातुल्या स्फुटक्रान्तिसङ्ख्या भवति~। द्वयोरभावे स्फुटक्रान्त्यभाव इति वक्तव्यं स्यात्~। एवं प्रतिपदं साधकसङ्ख्याया अभावे साध्यसङ्ख्यायाः साधनार्थं पृथग्वचनावश्यकतया ग्रन्थगौरवं स्यात्~। खषड्विध-परिभाषायां त्वेकभिन्नदिशोः क्रान्तिशरयोः सङ्ख्यासङ्कलनव्यवकलनाभ्यां स्फुटक्रान्तिसङ्ख्या भवतीत्येव वक्तव्यं स्यात्~। एवं प्रतिपदं तथा सति ग्रन्थलाघवं गणितपरिच्छेदश्च स्यादिति दिक्~।\\

{\small तदेवं खषड्विधस्यावश्यकत्वाद्भुजङ्गप्रयातेन तदाह~। तत्र पूर्वार्धेन खसङ्कलनव्यवकलने आह\textendash }

\phantomsection \label{2.16}
\begin{quote}
{\large \textbf{{\color{purple}स्वयोगे वियोगे धनर्णं तथैव च्युतं शून्यतस्तद्विपर्यासमेति~॥~१६~॥}}}
\end{quote}

अस्यार्थः~। रूपस्य यावत्तावदादिवर्णस्य करण्या वा शून्येन सह योगे वियोगे वा कर्तव्ये रूपादिकं धनमृणं वा तथैव भवेत्~। योगवियोगकृतो न कश्चिद्विशेष इत्यर्थः~। अत्र खयोगो द्विविधः~। खेन योगो रूपादेः खयोग इत्येकः~। खस्य योगो रूपादिना खयोग इति द्वितीयः~। एवं खवियोगोऽपि द्विविधः~। खेन वियोग इत्येकः~। खाद्वियोग इति द्वितीयः~। तत्र द्वैधेऽपि खयोगे पूर्वस्मिन्खवियोगे च रूपादिकं धनमृणं वा यथास्थितमेव~। खाद्वियोगे विशेषमाह~। \hyperref[2.16]{\textbf{च्युतं शून्यत}} इति~। धनमृणं वा रूपादिकं \hyperref[2.16]{\textbf{शून्यतः}} शोधितं \hyperref[2.16]{\textbf{सद्विपर्यासं}} वैपरीत्यं प्राप्नोति~। धनं चेच्छून्यतश्च्युतमृणं भवति~। ऋणं चेद्धनं भवतीत्यर्थः~॥~१६~॥\\

{\small अत्रोदाहरणानीन्द्रवज्रापूर्वाधेनाह\textendash }

\phantomsection \label{2.17}
\begin{quote}
{\large \textbf{{\color{purple}रूपत्रयं स्वं क्षयगं च खं च~।\\
किं स्यात् खयुक्तं वद खच्युतं च~॥~१७~॥}}}
\end{quote}

खाच्च्युतमिति पाठः~। धनं \hyperref[2.17]{\textbf{रूपत्रय}}मृणं रूपत्रयं \hyperref[2.17]{\textbf{खं}} चैतत्त्रयमपि पृथक् पृथक्~\hyperref[2.17]{\textbf{खयुक्तं किं स्याद्वद}}~। खेन युक्तं खयुक्तम्~। खे युक्तं खयुक्तमित्युदाहरणद्वयमपि द्रष्टव्यम्~। एवं खच्युतमित्यत्रापि तृतीयापञ्चमीतत्पुरुषाभ्यामुदाहरणद्वयं द्रष्टव्यम्~। खाद्वियोग उदाहरणमाह\textendash \,वद खाच्च्युतं चेति~। अत्र शून्यस्य धनत्व ऋणत्वे वा न कश्चिद्विशेष इति तस्य धनर्णत्वं नोद्दिष्टम्~। न्यासः ३~। ३ं~। ०~। एतानि खेन युक्तानि खे युक्तानि खेन च्युतानि चाविकृतान्येव~। ३~। ३ं~। ०~। अथ खाच्छोधनार्थं न्यासः ३~। ३ं~। ०~। एतानि खाच्छोधितानि जातानि विपर्यस्तानि ३ं~। ३~। ०~। शून्यस्य विपर्यासे न कश्चिद्विशेष इति स न कृतः~। वस्तुतस्तु खस्य धनर्णत्वं
\end{sloppypar}

\newpage

\begin{sloppypar}
\noindent नास्त्येवाभावत्वात्~। न च सङ्ख्यागतं योजकयोज्यत्वादिकं यथा तदभावे शून्य उपचर्यते तद्वद्धनर्णत्वमुपचर्यतामिति वाच्यम्~। योजकयोज्यवियोजकवियोज्यगुणकगुण्यभाजक-भाज्यत्वधर्माणां फले विशेषोपलम्भात्तदुपचारस्यावश्यकत्वात्~। सङ्ख्याभावे धनर्णत्वयोस्तु फले विशेषानुपलम्भात्तदुपचारस्य व्यर्थत्वादिति दिक्~।\\

अथ खसङ्कलनव्यवकलनयोरुपपत्तिः~। इह योज्ययोजकयोरुभयोरन्यतरस्य~वा~यावा-नुपचयोऽपचयो वा भवति तावानेव तत्सङ्कलनेऽपीति प्रसिद्धम्~। यथा योज्यः~३ योजकः ४ सङ्कलनफलम् ७~। अथवा योजकः ३ सङ्कलनफलम् ६~। अथवा~योजकः २ सङ्कलनफलम् ५~। योजकः १ फलम् ४~। एवं योजकः ० योज्यः~३~। अत्र योजकसङ्ख्यायां यावानपचयस्तावानेव सङ्कलनफलेऽप्युपलभ्यत इति योजकतुल्ये योजकापचये सङ्कलनफलेऽपि योजकतुल्ये-नापचयेन भाव्यम्~। तथा सति योज्यतुल्यमेव सङ्कलनफलं स्यादिति खेन योगेऽविकृतो राशिः~। एवं योज्यापचयवशादपि सङ्कलनफलापचयाद्योज्यतुल्ये योज्यापचये~सङ्कलन-फलेऽपि तावतैवापचयेन भाव्यमिति योजकसङ्ख्यातुल्यमेव सङ्कलनफलं स्यादिति~खस्य योगेऽप्यविकृतो राशिः~। एवमुभयापचयवशेन शून्ययोः सङ्कलनफलं शून्यमिति द्रष्ट-व्यम्~। अथ वियोज्यसङ्ख्यायां वियोजकसङ्ख्यातुल्येऽपचये व्यवकलनफलं भवति~।~तत्र वियोजकसङ्ख्याया यावानपचयस्तावानेवोपचयो व्यवकलनफले भवतीति वियोजक-तुल्ये वियोजकापचये सति व्यवकलनफले वियोजकतुल्येनोपचयेन भाव्यमिति वियोज्यसङ्ख्यातुल्यं व्यवकलनफलं स्यादतः खेन वियोगेऽविकृतो राशिः~। अथ वियोज्ये यथा यथापचयो भवति तथा तथा व्यवकलनफलेऽप्यस्तीति प्रसिद्धम्~। यथा वियोज्यः ५ वियोजकः ३ व्यवकलनफलम् २~। अथ वियोज्यः ४ व्यवकलनफलम् १~। वियोज्यः ३ व्यवकलनफलम् ०~। अथ वियोज्यः २ अत्रापि व्यवकलनफलेनैकोनेन भाव्यम्~। तथा सति व्यवकलनफलम् १~। अथ वियोज्यः १ उक्तवद्व्यवकलनफलम् २ं~। वियोज्यः ० उक्तवद्व्यवकलनफलेन ३ं भाज्यमित्युपपन्नम्~। \hyperref[2.16]{"च्युतं शून्यतस्तद्विपर्यासमेति"} इति~। एवं योज्ययोजकयोर्वियोज्यवियोजकयोश्च धनत्वं प्रकल्प्य यथा युक्तिरुक्ता तथोभयोः ऋणत्वमपि प्रकल्प्य द्रष्टव्या~। एकस्य धनत्वमितरस्यर्णत्वमिति कल्पने तूपचयाप-चययोरन्यथात्वेनोपपत्तिर्द्रष्टव्येत्यलं पल्लवितेन~॥~१७~॥\\

{\small अथोत्तरार्धेन खगुणनादिचतुष्टयमाह\textendash }

\phantomsection \label{2.18}
\begin{quote}
{\large \textbf{{\color{purple}वधादौ वियत् खस्य खं खेन घाते \\
खहारो भवेत् खेन भक्तश्च राशिः~॥~१८~॥}}}
\end{quote}
\end{sloppypar}

\newpage

\begin{sloppypar}
यथा पूर्वं खयोगवियोगयोर्द्वैविध्यमुक्तं तथा खगुणनभजनयोरपि द्वैविध्यमस्ति~। खस्येति खेनेति च~। वर्गादिषु तु खस्येत्येक एव प्रकारः सम्भवति वर्गादिकरणे द्वितीयसङ्ख्यानपेक्षणात्~। तत्र खस्येति प्रकारेष्वाह~। वधादौ वियत्\textendash \,खस्येति~। \hyperref[2.18]{\textbf{खस्य}} शून्यस्य \hyperref[2.18]{\textbf{वधादौ}} गुणनभजनवर्गतन्मूलादिषु कर्तव्येषु \hyperref[2.18]{\textbf{वियत्}} स्यात्~। गुणनफलादिकं शून्यं भवेदित्यर्थः~। \hyperref[2.18]{\textbf{खेने}}ति~। गुणनप्रकारेण फलमाह~। \hyperref[2.18]{\textbf{खं खेन घात}} इति~। खेन शून्येन घाते कस्यचिदङ्कस्य गुणने गुणनफलं खं स्यात्~। अत्र {\color{violet}"खगुणश्चिन्त्यश्च शेषविधौ"} इत्यादि पाटीस्थो विशेषो द्रष्टव्यः~। अन्यथा {\color{violet}"त्रिभज्यकोन्मण्डलशङ्कुघातात्"} इत्यादिना यष्ट्यानयनेन गोलसंधौ यष्ट्यभावापत्तेरिति दिक्~। \hyperref[2.18]{\textbf{खेने}}ति भजनप्रकारे फलमाह~। \hyperref[2.18]{\textbf{खहारो भवेत् खेन भक्तश्च राशिः}}~। इति~। खेन भक्तो राशिः खहारो भवेत्~। खं हारो यस्येति खहारोऽनन्त इत्यर्थः~। उदाहरणावसरे वक्ष्यति च~। अयमनन्तो राशिः खहर उच्यत इति~। अत्रोपपत्तिः~। गुण्यस्यापचयवशाद्गुणनफलस्यापचय इति तावत् प्रसिद्धम्~। यथा गुणकः १२ गुण्यः ४ गुणनफलं ४८~। अथवा गुण्यः ३ गुणनफलं ३६~। वा गुण्यः २ गुणनफलं २४~। वा गुण्यः १ गुणनफलं १२~। वा गुण्यः {\scriptsize $\begin{matrix}
\mbox{{१}}\\
\vspace{-1.5mm}
\mbox{{२}}
\vspace{1mm}
\end{matrix}$} गुणनफलं ६~। वा गुण्यः {\scriptsize $\begin{matrix}
\mbox{{१}}\\
\vspace{-1.5mm}
\mbox{{४}}
\vspace{1mm}
\end{matrix}$} गुणनफलं ३~। वा गुण्यः {\scriptsize $\begin{matrix}
\mbox{{१}}\\
\vspace{-1.5mm}
\mbox{{१२}}
\vspace{1mm}
\end{matrix}$} गुणनफलं १ इति~। अनयैव युक्त्या गुण्यस्य परमापचये गुणनफलस्यापि परमापचयेन भाव्यम्~। परमाचये च शून्यतैव पर्यवस्यतीति शून्ये गुण्ये गुणनफलं शून्यमेवेति सिद्धम्~। यद्वा गुण्य एकैकापचये गुणनफले गुणकतुल्योऽपचयो भवति~। यथा गुणकः ८ गुण्यः ४ गुणनफलम् ३२~। एकोनो गुण्यः ३ गुणनफलम् २४~। पुनरेकोनो गुण्यः २ गुणनफलम् १६~। पुनरेकोनो गुण्यः १ गुणनफलम् ८~। पुनरेकोनो गुण्यः ० अत्रापि गुणनफले गुणकतुल्येनापचयेन भाव्यम्~। तथा सति गुणनफले शून्यतैव सिद्धा~। एवं गुणकापचयवशादपि गुणनफलेऽपचयाद्गुणकस्यापि शून्यत्वे गुणनफलं शून्यमेवेति सिद्धम्~। ननु गुणकवैलक्षण्यादेकस्मिन्नपि गुण्ये गुणनफलवैचित्र्यमस्ति तत्कथं शून्ये गुण्ये गुणकवैलक्षण्येऽपि गुणनफलं शून्यमेवेति चेत्~। न~। अप्रयोजकत्वात्~। अन्यथैकातिरिक्तसङ्ख्याया वर्गवर्गमूलघनघनमूलादीनां वैलक्षण्यव्याप्तेरेकसङ्ख्याया अपि तेषां वैलक्षण्यापत्तेः~। वस्तुतस्तु गुणको ह्यावर्तकः~। सति गुण्ये गुण्यस्य गुणक-तुल्यावर्तनाद्गुणनफलं भवतीति गुणकवैचित्र्येऽस्ति गुणनफलवैचित्र्यम्~। इह त्वावर्त-नीयस्य गुण्यस्याभावाद्गुणकसहस्रमपि कमावर्तयेदिति गुणनफलस्याप्यभाव इति~। एवं भाज्यापचयवशाद्भजनफलापचयोऽस्तीति भाज्ये शून्ये भजनफलं शून्यमिति पूर्वयुक्त्यैव सिद्धम्~। वर्गादेश्च द्वितीयसङ्ख्यानिरपेक्षत्वाद्वर्गादिसङ्ख्याया अभावाच्चाभाव इति स्पष्टम्~। तदेवमुपपन्नम् "वधादौ वियत्खस्य खं खेन घाते" इति~। खहरोपपत्तिस्तूदाहरणे वक्ष्यते~॥~१८~॥
\end{sloppypar}

\newpage

\begin{sloppypar}
{\small अत्रोदाहरणानीन्द्रवज्रोत्तरार्धेनाह\textendash }

\phantomsection \label{2.19}
\begin{quote}
{\large \textbf{{\color{purple}द्विघ्नं त्रिहृत्खं खहृतं त्रयं च शून्यस्य वर्गं वद मे पदं च~॥~१९~॥}}}
\end{quote}

अत्र द्वाभ्यां हन्यते तद्द्विघ्नमिति व्युत्पत्त्या गुण्ये द्वौ हन्तीति व्युत्पत्त्या शून्ये गुणके च पृथगुदाहरणं द्रष्टव्यम्~। शेषं स्पष्टम्~। प्रथमे न्यासः~। गुणकः २ गुण्यः ० गुणनफलं वधादौ वियत्खस्येति जातम् ०~। द्वितीये न्यासः~। गुणकः ० गुण्यः २ खं खेन  घात इति जातम् ०~। अथ भागहारे प्रथमोदाहरणे न्यासः~। भाजकः ३ भाज्यः ० वधादौ वियत्खस्येति जातं भजनफलं ०~। द्वितीये न्यासः~। भाजकः ० भाज्यः ३ खहारो भवेत्खेन भक्तश्च राशिरिति जातः खहरः ३~। ननु यो राशिर्येन ह्रियते स तद्धरो भवतीति राशेः खेन हरणे खहरो भवेदेति स्पष्टमेवास्ति~। किन्तु खेन राशौ हृते का लब्धिरिति प्रश्नस्य किमुत्तरमित्यत आह~। अयमनन्तो राशिः खहर इत्युच्यत इति~। लब्धिरनन्तेत्युत्तरमिति भावः~। एतस्यानन्तत्वे ह्येषा युक्तिस्त्वस्ति~। यथा यथा भाजकापचयस्तथा तथा लब्धेरुपचयः~। तथा सति भाजकाङ्के परमापचिते लब्धेः परमोपचयेन भाव्यम्~। लब्धेश्चेदियत्तोच्येत तर्हि परमत्वं न स्यात्ततोऽप्याधिक्यसम्भवात्~। अतो लब्धेरियत्ताभाव एव परमत्वम्~। तदेवमुपपन्नं खहरो राशिरनन्त इति~॥~१९~॥\\

{\small अथानन्तपदसञ्जातभगवत्स्मृतिर्भागवतोत्तमः श्रीभास्कराचार्यः प्रसङ्गेनापि स्तुतो हरिः कृता-र्थतां सम्पादयतीति दृढनिश्चयः खहरराशेरविकारतादृष्टान्तप्रसङ्गेन श्रीभगवन्तमनन्तं स्तौति\textendash }

\phantomsection \label{2.20}
\begin{quote}
{\large \textbf{{\color{purple}अस्मिन्विकारः खहरेण राशावपि प्रविष्टेष्वपि निःसृतेषु~।\\
बहुष्वपि स्याल्लयसृष्टिकालेऽनन्तेऽच्युते भूतगणेषु यद्वत्~॥~२०~॥ }}}
\end{quote}

उपजातिकेयम्~। अस्यार्थः~। प्रलयकाले श्रीभगवत्यनन्तेऽच्युते बहुष्वपि भूतगणेषु प्रविष्टेषु लीनेष्वपि वा निःसृतेषु देहादिमत्तया भगवतोऽनन्तात् पृथग्भूतेष्वपि यद्वद्विकारो नास्ति न हि तेषु प्रविष्टेषु महान्भवति निःसृतेषु वा लघुर्भवति तथास्मिन्खहरे राशावपि बहुष्वपि राशिषु प्रविष्टेषु निःसृतेषु वा विकारो नास्तीति~। ननु कथं विकारो नास्ति~। न हीशनिदेशः~। योगे वियोगे वाविकारस्य व्याप्तिसिद्धिः स्यात्~। सत्यम्~। सर्वत्र योगोऽन्तरं वा समच्छेदत्वे भवति~। प्रकृतेऽपि समच्छेदतां विधायैव योगोऽन्तरं वा विधेयम्~। समच्छेदता च\textendash
\vspace{-4mm}

\begin{quote}
{\color{violet}"अन्योन्यहाराभिहतौ हरांशौ~।" }
\end{quote}
\vspace{-2mm}

इत्यनेन~। तथा च खहरस्य राशेर्हरेण शून्येनापरराशौ गुणिते शून्यमेव भवेत्~। शून्ययोगवियोगयोश्चाविकृतत्वं पूर्वमेवोक्तम्~। ननु यद्यप्यभिन्नराशिना योगान्तरयोर्भवति अविकृतत्वं तथापि भिन्नराशिना योगेऽन्तरे च त्वदुक्तरीत्या भवेदेव विकारः~। यथा
\end{sloppypar}

\newpage

\begin{sloppypar}
\noindent {\scriptsize $\begin{matrix}
\mbox{{३}}\\
\vspace{-1.5mm}
\mbox{{०}}
\vspace{1mm}
\end{matrix}$}~। {\scriptsize $\begin{matrix}
\mbox{{१}}\\
\vspace{-1.5mm}
\mbox{{३}}
\vspace{1mm}
\end{matrix}$}~। अन्योन्यहाराभिहतौ हरांशाविति जातौ तुल्यहरौ {\scriptsize $\begin{matrix}
\mbox{{९}}\\
\vspace{-1.5mm}
\mbox{{०}}
\vspace{1mm}
\end{matrix}$}~। {\scriptsize $\begin{matrix}
\mbox{{०}}\\
\vspace{-1.5mm}
\mbox{{०}}
\vspace{1mm}
\end{matrix}$}~। अनयोर्योगे जातं {\scriptsize $\begin{matrix}
\mbox{{९}}\\
\vspace{-1.5mm}
\mbox{{०}}
\vspace{1mm}
\end{matrix}$}~। अथ यद्युच्येतैकस्य हरेण येन केनचिदङ्केन वापरराशिहरांशगुणनमात्रेण तुल्यहरत्वे जाते परतः श्रमो व्यर्थ एव~। प्रकृतेऽपि खहरराशेर्हरेण शून्येनापरराशि\textendash \,{\scriptsize $\begin{matrix}
\mbox{{१}}\\
\vspace{-1.5mm}
\mbox{{३}}
\vspace{1mm}
\end{matrix}$}\textendash \,हरांशगुणनमात्रेण तुल्यहरस्य जातत्वाद्योगेऽन्तरे च नास्त्येव विकार इति~। तर्हि खहरस्य खहरेण योगेऽन्तरे च भवेदेव विकारः~। यथा राशी {\scriptsize $\begin{matrix}
\mbox{{३}}\\
\vspace{-1.5mm}
\mbox{{०}}
\vspace{1mm}
\end{matrix}$}~। {\scriptsize $\begin{matrix}
\mbox{{५}}\\
\vspace{-1.5mm}
\mbox{{०}}
\vspace{1mm}
\end{matrix}$}~। अनयोस्तुल्यहरत्वाद्योगे जातं {\scriptsize $\begin{matrix}
\mbox{{८}}\\
\vspace{-1.5mm}
\mbox{{०}}
\vspace{1mm}
\end{matrix}$}~। तत्कथं न विकार इति चेत्~। मैवम्~। अत्रापि फलतो विकाराभावात्~। न हि खेन भक्तेषु त्रिष्वन्यत्फलमष्टसु भक्तेष्वितरदिति किन्तूभयत्राप्यनन्तत्वे न व्यभिचार इति~। यथोदयकाले न्यूनाधिकपरिमाणयोरपि शड्कोश्छायानन्त्यं न व्यभिचरति तथा वर्तमानेऽस्मिन्काले भूते भविष्यति च गतकल्पसङ्ख्याया न्यूनाधिकभावेऽप्यनन्तत्वाव्यभिचारः~। किञ्चोन्न-तांशजीवास्वरूपे शङ्कौ यदि दृग्ज्याभुजस्तदेष्टे द्वादशाङ्गुलादिके शङ्कौ किमिति त्रैराशिकेन च्छाया सिध्यति~। तत्रोदयकाल उन्नतजीवाया अभावः~। दृग्ज्या च त्रिज्यामिता १२०~। अत्र द्वित्रिचतुरङ्गुलादीनां शङ्कूनामुक्तत्रैराशिकेन छायासाधने २४०~। ३६०~। ४८०~। एतदाद्याः सिध्यन्ति खहराश्छायाः~। न ह्येतासु फलतो वैलक्षण्यमस्ति~। यतस्तस्मिन्नपि काले न्यूनाधिकपरिमाणानामपि शङ्कूनां छायानन्त्यं न व्यभिचरति~। किञ्चोदयकाल एव ३४३८~। १२०~। १००~। ९० आभ्यस्त्रिज्याभ्यः प्राग्वदनुपातेन द्वादशाङ्गुलशङ्कोश्छायाः {\scriptsize $\begin{matrix}
\mbox{{४१२५६}}\\
\vspace{-1.5mm}
\mbox{{०}}
\vspace{1mm}
\end{matrix}$}~। {\scriptsize $\begin{matrix}
\mbox{{१४४०}}\\
\vspace{-1.5mm}
\mbox{{०}}
\vspace{1mm}
\end{matrix}$}~। {\scriptsize $\begin{matrix}
\mbox{{१२००}}\\
\vspace{-1.5mm}
\mbox{{०}}
\vspace{1mm}
\end{matrix}$}~। {\scriptsize $\begin{matrix}
\mbox{{१०८०}}\\
\vspace{-1.5mm}
\mbox{{०}}
\vspace{1mm}
\end{matrix}$}~। न ह्यासां भेदः सम्भाव्यते~। नहि त्रिज्याभेदप्रयुक्तश्छायाभेदः~। किन्तु नानात्रिज्याभ्योऽनुपातसिद्धा छाया तुल्यैवेति सकलगणकानामविवाद इति सर्वमवदातम्~। एवं मतिमद्भिरन्यदप्यूह्यम्~। शून्यस्य वर्गः ० वर्गमूलम् ०~। एवं घनादिष्वपि शून्यतैव~॥~२०~॥

\begin{quote}
{\color{violet}दैवज्ञवर्यगणसन्ततसेव्यपार्श्वबल्लालसञ्ज्ञगणकात्मजनिर्मितेऽस्मिन्~।\\
बीजक्रियाविवृतिकल्पलतावतारे व्यक्तिः क्रमादभवदम्बरषड्विधस्य~॥ }
\end{quote}
\vspace{-1mm}

\begin{center}
इति श्रीसकलगणकसार्वभौमश्रीबल्लालदैवज्ञसुतकृष्णदैवज्ञविरचिते\\
बीजविवृतिकल्पलतावतारे खषड्विधविवरणम्~॥\\
( अत्र ग्रन्थसङ्ख्या पादोनशतद्वयम् १७५ )~। \\
\vspace{6mm}

\rule{0.2\linewidth}{0.8pt}\\
\vspace{-4mm}

\rule{0.2\linewidth}{0.8pt}
\end{center}
\end{sloppypar}

\newpage
\thispagestyle{empty}

\begin{center}
\textbf{\large ३\; वर्णषड्विधम्~।}\\
\rule{0.2\linewidth}{0.8pt}
\end{center}

\begin{sloppypar}
{\small अथ यद्यपि करणीषड्विधं निरूपणीयमित्युक्तत्वादुक्तषड्विधस्यान्तरङ्गत्वमिति प्रथमतस्तन्निरूप्य बहिरङ्गमव्यक्तषड्विधं पश्चान्निरूपणीयमिति युक्तम्~। तथापि करणीषड्विधस्यातिकठिनतया तन्निरू-पणे प्रयासबाहुल्यादव्यक्तषड्विधनिरूपणे च प्रयासलाघवात् सूचीकटाहन्यायेनाव्यक्तषड्विधं प्रथमतो निरूपयति~। तत्र द्वित्र्यादीनां राशीनामव्यक्तत्वे सञ्ज्ञाभेदमन्तरेण तत्सङ्करः स्यादतस्तन्निरासा-र्थमव्यक्तसञ्ज्ञाः शालिन्याह\textendash }

\phantomsection \label{3.21}
\begin{quote}
{\large \textbf{{\color{purple}यावत्तावत्कालको नीलकोऽन्यो\\
{\color{white}अ} ~वर्णः पीतो लोहितश्चैतदाद्याः~।\\
अव्यक्तानां कल्पिता मानसञ्ज्ञाः\\
{\color{white}अ} ~तत्सङ्ख्यानं कर्तुमाचार्यवर्यैः~॥~२१~॥}}}
\end{quote}

\hyperref[3.21]{\textbf{यावत्तावत्}} इत्येकं नाम~। \hyperref[3.21]{\textbf{कालकः}} २~। \hyperref[3.21]{\textbf{नीलकः}} ३~। \hyperref[3.21]{\textbf{पीतः}} ४~। \hyperref[3.21]{\textbf{लोहितः}} ५~। \hyperref[3.21]{\textbf{एतदाद्या}} हरितश्वेतकचित्रकादयोऽनेकवर्णसमीकरणपठिता वर्णा \hyperref[3.21]{\textbf{अव्यक्तानाम्}} अज्ञातराशीनां \hyperref[3.21]{\textbf{मान-सञ्ज्ञा आचार्यवर्यैः कल्पिताः}}~। नामकल्पने प्रयोजनमाह\textendash \,\hyperref[3.21]{\textbf{तत्सङ्ख्यानं कर्तुम्}} इति~। तेषाम् अज्ञातराशीनां सङ्ख्यानं गणनां कर्तुं साधयितुं ज्ञातुमिति यावत्~॥~२१~॥\\

{\small एवमव्यक्तसञ्ज्ञा अभिधाय तत्सङ्कलनव्यवकलने उपजातिकापूर्वार्धेनाह\textendash }

\phantomsection \label{3.22}
\begin{quote}
{\large \textbf{{\color{purple}योगोऽन्तरं तेषु समानजात्योर्विभिन्नजात्योश्च पृथक् स्थितिश्च~॥~२२~॥}}}
\end{quote}

तेषु वर्णेषु मध्ये रूपेष्वित्यपि द्रष्टव्यम्~। \hyperref[3.22]{\textbf{समानजात्योः}} समानैका जातिर्ययोस्तौ तथा तयोः समानजात्योः पूर्वोक्तो योगोऽन्तरं च स्यात्~। अत्र स्यादिति पदमुत्तरदलस्थमन्वेति देहलीदीपन्यायेन~। "पृथक् स्थितिः स्यात्" इति वा पाठः~। समानजात्योरित्युपलक्षणं समानजातीनामित्यपि द्रष्टव्यम्~। यद्वा बहूनामपि योगे द्वयोर्योगस्यैव मुख्यत्वाद्युग-पत्सर्वयोगस्य कर्तुमशक्यत्वाद्द्विवचनम्~। जातिश्चात्र रूपत्वम्~। यावत्तावत्त्वम्~। काल-कत्वम्~। नीलकत्वम्~। यावत्तावद्वर्गत्वम्~। यावत्तावद्घनत्वम्~। यावत्तावद्वर्गवर्गत्वं च~। यावत्तावत्कालकभावितत्वमित्यादिर्योज्ययोजकनिष्ठसकलजातिव्याप्या योज्ययोजकनिष्ठा च~। न त्वङ्कत्वं वर्णत्वं वा~। अङ्कत्वोक्तौ विशेषणवैयर्थ्यापत्तिः~। व्यावर्त्याभावात्~। वर्णत्वोक्तौ वर्णकल्पनानर्थक्यप्रसङ्गः~। असङ्करार्थं हि वर्णकल्पना~। वर्णत्वजात्या साजात्ये विवक्षिते सङ्कर एव स्यात्~। तस्मादुक्तविधजात्यैव साजात्यं विवक्षितम्~। यद्वा समानशब्दस्य तुल्यार्थत्वाद्योज्ययोजकयोः स्वस्वनिष्ठसकलजातिभिः साजात्यं विवक्षितं विभिन्नजात्योश्च~। चस्त्वर्थे~। विभिन्ना जातिर्ययोस्त- 
\end{sloppypar}

\newpage

\begin{sloppypar}
\noindent योर्वा योगेऽन्तरे वा क्रियमाणे पृथक् स्थितिश्च~। चावधारणे~। पृथक्
स्थितिरेव स्यात् इत्यर्थः~। एतदुक्तं भवति~। रूपस्य रूपेण यावत्तावतो यावत्तावता कालकस्य कालकेन कालकवर्गस्य कालकवर्गेण कालकघनस्य कालकघनेन कालकनीलकभावितस्य तद्भावितेन~। एवं समानजात्योर्योगेऽन्तरे वा कर्तव्ये योगोऽन्तरं वोक्तवद्भवति~। रूपस्य यावत्तावता कालकादिभिर्वा यावत्तावतः कालकादिभिर्यावत्तावतो यावद्वर्गेण यावद्घनस्य यावता तद्वर्गेण वा भावितादिभिर्वा~। एवं विभिन्नजात्योर्योगेऽन्तरे वा कर्तव्ये पृथक् स्थितिरेव~। अत्रैकपङ्क्ताविति द्रष्टव्यम्~। अन्यथा योगान्तरज्ञापकाभावादिति~। अत्रोपपत्तिस्तु व्यक्ते प्रसिद्धैव~। अन्यथा समच्छेदविधानपूर्वकं योगान्तरकथनं न स्यात्~। किञ्च विभिन्नजात्योर्योगः किमात्मकः~। यथा राशिद्वयमंशपञ्चकं चेत्यनयोर्विभिन्नजात्योरपि योगः क्रियेत तर्हि सप्त स्युः~। ते सप्त न राशयो न वा लवाः~। नहि ग्रहेण राशिद्वयमंशपञ्चकं च भुक्तमित्युक्ते ग्रहेण सप्त राशयः सप्त लवा वा भुक्ता इति कस्यापि प्रतीतिरस्त्युपपद्यते वा किन्तु ग्रहेण कियद्भुक्तमस्तीति प्रश्ने राशिद्वयमंशपञ्चकं च भुक्तमित्युत्तरस्य सर्वसम्प्रतिपन्नत्वाद्युक्तत्वाच्च पृथक्स्थितिरेव युक्ता~। अत्रैव साजात्ये योगो भवत्येव~। यथा\textendash \,राशिद्वयस्य लवाः ६० पञ्चभिर्लवै\textendash \,५\textendash \,र्योगे जाताः पञ्चषष्टिर्लवाः ६५~। ग्रहेण राशिद्वयमंशपञ्चकं च भुक्तमित्युक्ते पञ्चषष्टिर्लवा भुक्ता इत्यस्त्येव प्रतीतिः सर्वसंमतेत्यादि सुधीभिरूह्यम्~॥~२२~॥\\

{\small नन्वेवं वर्णेष्वपि साजात्यं विधाय योगोऽस्त्विति चेन्न~। वर्णमानानामज्ञातत्वात् साजात्य-विधानस्याशक्यत्वात्~। अत एव तन्मानोत्थापनानन्तरं साजात्येन योगो भवत्येव~। एवमेव वियोगेऽप्युपपत्तिर्द्रष्टव्या~। अत्रोदाहरणानि भुजङ्गप्रयातेनाह\textendash }

\phantomsection \label{3.23}
\begin{quote}
{\large \textbf{{\color{purple}स्वमव्यक्तमेकं सखे सैकरूपं धनाव्यक्तयुग्मं विरूपाष्टकं च~।\\
युतौ पक्षयोरेतयोः किं धनर्णे विपर्यस्य चैक्ये भवेत् किं वदाशु~॥~२३~॥}}}
\end{quote}

एकस्य रूपसहितमेकं धनमव्यक्तमित्येकः~। रूपाष्टकरहितं धनमव्यक्तयुग्ममिति द्वितीयः~। एतयोः पक्षयोर्युतौ किं फलं स्यात्~। अथ च पक्षयोर्धनर्णे विपर्यस्यैक्ये किं फलं स्यादिति~। अत्र पूर्वपक्षमात्रव्यत्यासादुत्तरपक्षमात्रव्यत्यासादुभयपक्षव्यत्यासाच्च प्रश्नत्रयम्~। व्यत्यासाभावे चैकमित्युदाहरणचतुष्टयम्~। धनर्णे इत्यत्र भावप्रधानो निर्देशः~। यद्वाव्यक्ते रूपे इत्यध्याहार्य योजना द्रष्टव्या~। एकमव्यक्तमिदं १ या १ एकं रूपमिदम्~। रू १ अनयोर्योगे द्वयं न भवति~। भिन्नजातित्वात्~। किन्तु पङ्क्तौ पृथक् स्थितिरेवेति जात एकः पक्षः~। या १ रू १ एवं धनाव्यक्तयुग्मं या २ अस्माद्रूपाष्टके शोध्यमाने संशोध्यमानं स्वमृणत्वमेतीति जातमृणं रूपाष्टकं ८ं अनयोर्धनर्णयोरन्तर-
\end{sloppypar}

\newpage

\begin{sloppypar}
\noindent मेव योग इति ऋणगताः षट् ६ं न भवति~। किञ्चैकपङ्क्तौ पृथक् स्थितिरेव~। तथा कृते जातो द्वितीयः पक्षः~। या २ रू ८ं योगार्थमुभयोर्न्यासो {\scriptsize $\begin{matrix}
\mbox{{या \;१ \;रू \;१}}\\
\vspace{-1.5mm}
\mbox{{या \;२ \;रू \;८ं}}
\vspace{1mm}
\end{matrix}$}~। अनयोर्योगे कर्तव्ये समानजात्योरेव योग इति~। अव्यक्तमव्यक्तेन रूपं रूपेण च संयोज्यम्~। तथा कृते जातं या ३~। रू ७ं आद्यपक्षे धनर्णत्वे विपर्यस्य न्यासः {\scriptsize $\begin{matrix}
\mbox{{या \;१ं \;रू \;१ं}}\\
\vspace{-1.5mm}
\mbox{{या \;२ \;रू \;८ं}}
\vspace{1mm}
\end{matrix}$} अनयोरुक्तवद्योगे जातं या १ रू ९ं~। द्वितीयपक्षव्यत्यासे न्यासो {\scriptsize $\begin{matrix}
\mbox{{या \;१ \;रू \;१}}\\
\vspace{-1.5mm}
\mbox{{या \;२ं \;रू \;८}}
\vspace{1mm}
\end{matrix}$}~। योगे जातं या १ं रू ९ उभयपक्षधनर्णव्यत्यासे न्यासो \;{\scriptsize $\begin{matrix}
\mbox{{या \;१ं \;रू \;१ं}}\\
\vspace{-1.5mm}
\mbox{{या \;२ं \;रू \;८}}
\vspace{1mm}
\end{matrix}$}~। योगे जातं या ३ं रू ७~। एवं द्वयोर्भिन्नजातित्वे सत्युदाहरणान्युक्तानि~॥~२६~॥\\

{\small अथ त्रयाणां वैजात्ये सत्युदाहरणं भुजङ्गप्रयातपूर्वार्धेनाह\textendash }

\phantomsection \label{3.24}
\begin{quote}
{\large \textbf{{\color{purple}धनाव्यक्तवर्गत्रयं सत्रिरूपं क्षयाव्यक्तयुग्मेन युक्तं च किं स्यात्~॥२४~॥}}}
\end{quote}

त्रिभी रूपैः सहितं धनमव्यक्तवर्गत्रयं सत्रिरूपम्~। याव ३ रू ३ अयं पक्ष ऋणा-व्यक्तयुग्मेनानेन या २ं योज्यः~। इदमव्यक्तयुग्मं न वर्गैः संयुज्यते नापि रूपैः~।~भिन्न-जातित्वात्~। तस्मात्पङ्क्तौ पृथक् स्थितिरेेव~। तत्र क्रमस्तु\textendash \,आदौ वर्गधनस्य~। ततो वर्गवर्गस्य~। ततो धनस्य~। ततो वर्गस्य~। ततोऽव्यक्तस्य~। ततो रूपाणामित्यादिः~। तथा स्थितौ जातम्~। याव ३ या २ं रू ३ एवं कालकादिष्वपि बोद्धव्यम्~॥~२४~॥\\

{\small अथोत्तरार्धेन व्यवकलनोदाहरणमाह\textendash }

\phantomsection \label{3.25}
\begin{quote}
{\large \textbf{{\color{purple}धनाव्यक्तयुग्मादृणाव्यक्तषट्कं सरूपाष्टकं प्रोज्झ्य शेषं वदाशु~॥~२५~॥}}}
\end{quote}

स्पष्टोऽर्थः~। अथ न्यासः~। सरूपाष्टकमृणाव्यक्तषट्कमुक्तवज्जातम्~। य ६ं रू ८ एत-द्धनाव्यक्तयुग्मादस्मात् या २ विशोध्यम्~। तत्र संशोध्यमानं स्वमृणत्वमेतीत्यादिना जातः शोध्यपक्षः~। या ६ रू ८~। एतन्मध्ये व्यक्तमेव सजातीयत्वादव्यक्ते योज्यम्~। रूपाणां पृथक्स्थतिरेवेति तथा कृते जातम्~। या ८ रू ८ं~॥~२५~॥
\end{sloppypar}

\newpage

\begin{sloppypar}
{\small एवं सङ्कलनव्यवकलने अभिधायोपजातिकोत्तरार्धेनोपजातिकया च वर्णगुणनमाह\textendash }

\phantomsection \label{3.26}
\begin{quote}
{\large \textbf{{\color{purple}स्याद्रूपवर्णाभिहतौ तु वर्णो द्वित्र्यादिकानां समजातिकानाम्~॥\\
वधे तु तद्वर्गघनादयः स्युः तद्भावितं चासमजातिघाते~।\\
भागादिकं रूपवदेव शेषं व्यक्ते यदुक्तं गणिते तदत्र~॥~२६~॥}}}
\end{quote}

अस्यार्थः\textendash \,वर्णगुणनं त्रिधैव सम्भवति~। रूपेण सजातीयवर्णेन विजातीयवर्णेन वा~। तत्र रूपेण गुणने~। \hyperref[3.26]{\textbf{स्याद्रूपवर्णभिहतौ तु वर्णः}} इति~। रूपवर्णाभिहतौ तु वर्णः स्यात्~। अयमर्थः~। रूपेण वर्णे गुणनीये वर्णेन वा रूपे गुणनीयेऽङ्कतस्तु गुणनफलं भवति~। नाम तु वर्णस्यैव~। अथ सजातीयवर्णेन गुणने समजातिकानां द्वित्र्यादिकानां वर्णानां वधे तु तद्वर्गघनादयः स्युः~। एतदुक्तं भवति~। यावत्तावता यावत्तावति गुणिते समजात्योर्द्वयोर्घात इति यावत्तावद्वर्गः स्यात्~। स चेत्पुनर्यावत्तावता गुण्यते तदा समत्रिघातत्वाद्यावत्तावद्घनः स्यात्~। अयमपि चेत्तेन गुण्यते तदा समचतुर्घातत्वाद्यावत्तावद्वर्गवर्गो भवेत्~। असावपि तेन गुणितश्चेत्पञ्चघातत्वाद्यावद्वर्गघनयोर्घातः~।\\

एवं षड्घाते यावद्वर्गघनो यावद्घनवर्गो वा भवेत्~। इत्यादि~। कालकादीनामपि~सम-द्वित्र्यादिवधे कालकादिवर्गघनादयो ज्ञेयाः~। अथ विजातीयवर्णेन गुणने~। असमजातिघाते तद्भावितं स्यादिति~। विजातीयवर्णयोर्घाते तयोर्वर्णयोर्भावितं स्यात्~। यथा\textendash \;यावता कालके गुणिते यावत्कालकभावितं भवति~। कालकेन नीलके गुणिते कालकनीलकभावितं भवतीत्यादि~। यावत्कालकभावितं यदि कालकेन गुण्यते तदा यावत्कालकवर्गभावितं भवति~। इदमपि यदि यावत्तावता गुण्यते तदा यावद्वर्गकालकवर्गभावितं भवतीत्यादि सुधीभिरूह्यम्~। एवं गुणने विशेषमुक्त्वा भागादिकमाह\textendash \;शेषं भागादिकं भागवर्ग-मूलघनघनमूलादि यद्व्यक्ते गणित उक्तं तदत्र रूपवज्ज्ञेयम्~। {\color{violet}"भाज्याद्धरः शुध्यति यद्गुणः"} इत्यादिना भजनफलं ज्ञेयम्~। {\color{violet}"समद्विघातः कृतिरुच्यत"} इत्यादिना वर्गो ज्ञेय इत्यादि~। भागादिकानां गुणनपूर्वकत्वाद्गुणनसञ्ज्ञाविशेषस्य चोक्तत्वात्तत्र कोऽपि विशेषो वक्तव्यो नास्तीति भावः~। इदमुपलक्षणम्~। अत्रासङ्करार्थं गुणनफलसञ्ज्ञामात्रमुक्तम्~। अङ्कतस्तु गुणनादिकं व्यक्ते गणिते यदुक्तं तदत्र ज्ञेयमित्यपि द्रष्टव्यम्~॥~२६~॥ \\

{\small एवमत्र {\color{violet}"गुण्यान्त्यमङ्कं गुणकेन हन्यात्"~।} इत्यादिना गुणनफलसिद्धावपि शिष्यसौकर्यार्थं {\color{violet}"गुण्यस्त्वधोऽधो गुणखण्डतुल्य"} इत्यादिव्यक्तोक्तखण्डगुणनं वसन्ततिलकया विशदयति\textendash }

\phantomsection \label{3.27.1}
\begin{quote}
{\large \textbf{{\color{purple}गुण्यः पृथग्गुणकखण्डसमो निवेश्यः \\
तैः खण्डकैः क्रमहतः सहितो यथोक्त्या~।}}}
\end{quote}
\end{sloppypar}

\newpage

\begin{sloppypar}
\phantomsection \label{3.27}
\begin{quote}
{\large \textbf{{\color{purple}अव्यक्तवर्गकरणीगुणनासु चिन्त्यो \\
व्यक्तोक्तखण्डगुणनाविधिरेवमत्र~॥~२७~॥}}}
\end{quote}

गुणकस्य यावति खण्डानि तावत्सु स्थानेषु पृथक् गुण्यो निवेश्यः~। अत्र खण्डानि सञ्ज्ञाभेदेनावगन्तव्यानि~। यथा गुणको या ३ रू २~। अत्र सञ्ज्ञाद्वयाद्गुणकस्य खण्डद्वयम्~। यथा वा गुणको याव २ या ३ का ५~। अत्र सञ्ज्ञात्रयाद्गुणकस्य खण्डत्रयमित्यादि~। अथ पृथङ्निवेशितो गुण्यस्तैर्गुणकखण्डैः प्रथमस्थाने प्रथमखण्डेन द्वितीयस्थाने द्वितीयेन तृतीयस्थाने तृतीयेनेत्येवं क्रमेण \hyperref[3.26]{"स्याद्रूपवर्णाभिहतौ तु वर्णः"} इत्यादिना गुणितः सन् यथोक्त्या पूर्वोक्तप्रकारेण \hyperref[3.22]{"योगोऽन्तरं तेषु समानजात्योः"} इत्यादिना \hyperref[1.3]{"योगे युतिः स्यात्क्षययोः स्वयोर्वा"} इत्यादिना च सहितः~। अत्राव्यक्तगणितेऽ\hyperref[3.27]{\textbf{व्यक्तवर्गकरणीगुणनासु}} यथा तथा व्यक्तगुणनासु वर्गार्थगुणनासु करणीगुणनासु च \hyperref[3.27]{\textbf{व्यक्तोक्तखण्डगुणनाविधिरेवं चिन्त्यः}}~। एवमन्येऽपि गुणनाप्रकारा द्रष्टव्याः~॥~२७~॥\\

{\small अत्रोदाहरणानि शालिन्याह\textendash }

\phantomsection \label{3.28}
\begin{quote}
{\large \textbf{{\color{purple}यावत्तावत्पञ्चकं व्येकरूपं यावत्तावद्भिस्त्रिभिः सद्विरूपैः~। \\
सङ्गुण्य द्राग्ब्रूहि गुण्यं गुणं वा व्यस्तं स्वर्णं कल्पयित्वा च विद्वन्~॥२८~॥}}}
\end{quote}

गुण्ये गुणे वेति व्यस्तस्वर्णमिति च पाठभेदात्पाठत्रयं प्रसिद्धमस्ति~। तत्र पूर्वलिखितपाठे तावदियं व्याख्या~। स्वर्णं गुण्यं स्वर्णं गुणकं वा \hyperref[3.28]{\textbf{व्यस्तं कल्पयित्वे}}ति~। गुण्ये गुणे वेति पाठे गुण्ये विद्यमानं स्वर्णं यथासम्भवं स्वमृणं यावत्कालकरूपादिव्यस्तं कल्पयित्वेति~। एवं गुणेऽपि~। अथ व्यस्तस्वर्णमिति पाठे गुण्यं गुणं वा व्यस्तस्वर्णं कल्पयित्वा~।  व्यस्तं स्वर्णं यथासम्भवं स्वमृणं च यावदादि यत्र तं तादृशं कल्पयित्वेत्यर्थः~। अत्र सर्वत्र {\color{violet}"सविशेषणौ हि विधिनिषेधौ विशेषणमुपसङ्क्रामतो विशेष्ये बाधके सति"} इति न्यायेन स्वर्णत्वयोरेव व्यस्तत्वविधानं द्रष्टव्यम्~। शेषं स्पष्टम्~। अत्र यथास्थितगुण्यगुणकयोरेकमुदाहरणम्~। गुण्यमात्रव्यत्यासे द्वितीयम्~। गुणकमात्रव्यत्यासे तृतीयम्~। चकारादुभयव्यत्यासे चतुर्थमिति चत्वार्युदाहरणानि~। अत्र रूपोनं यावत्तावत्पञ्चकं गुण्यो या ५ रू १ं~। रूपद्वययुतं यावत्तावत्त्रयं गुणको या ३ रू २~। \hyperref[3.27.1]{"गुण्यः पृथग्गुणकखण्डसमो निवेश्यः"} इत्यादिना गुणनार्थं न्यासो \;{\scriptsize $\begin{matrix}
\mbox{{या \;३~। \;या \;५ \;रू \;१ं~।}}\\
\vspace{-1.5mm}
\mbox{{रू \;२~। \;या \;५ \;रू \;१ं~।}}
\vspace{1mm}
\end{matrix}$} \;अत्र यावत्त्रयेण यावत्तावत्पञ्चके गुणितेऽङ्कतः पञ्चदश १५~। अक्षरतस्तु \hyperref[3.26]{"द्वित्र्यादिकानां समजातिकानां वधे तु तद्वर्गघनादयः स्युः"} इत्यादिना जाता यावत्तावद्वर्गाः~। तत्र यावत्तावतो वर्गस्य चाद्याक्षरोपलक्षणपूर्वकं लिखने सम्पन्नं याव १५~। अथ
\end{sloppypar}

\newpage

\begin{sloppypar}
\noindent यावत्त्रयेण क्षयरूपे गुणिते \hyperref[1.9]{"स्वर्णघाते क्षय"} इत्यङ्कतः ३ं~। अक्षरतस्तु \hyperref[3.26]{"रूपवर्णाभिहतौ वर्णः स्यात्"} इति जातो वर्ण एव या ३ं~। एवं प्रथमपङ्क्तौ जातं याव १५ या ३ं~। अथ द्वितीयस्थाने द्वितीयेन गुणकखण्डेन रू २ यावत्पञ्चके गुणितेऽङ्कतो दश १०~। अक्षरतस्तु \hyperref[3.26]{"रूपवर्णाभिहतौ वर्णः"} इति जातो वर्णो या १०~। रूपद्वयेन क्षयरूपे गुणिते \hyperref[1.9]{"स्वर्णघाते क्षयः"} इति जातं २ं~। अत्राक्षरसञ्ज्ञा व्यक्ते प्रसिद्धैव~। नहि व्यक्ते द्वित्र्यादिघाते सञ्ज्ञाभेदोऽस्ति~। रूपं तु व्यक्तमेव~। अतो रूपस्य रूपेण गुणनेऽक्षरतो रूपमेव~। तथा सति जातं रू २ं~। एवं जातं द्वितीयपङ्क्तौ गुणनफलं या १० रू २ं~। एवमुभयपङ्क्त्योर्न्यासो \;{\scriptsize $\begin{matrix}
\mbox{{याव \;१५ \;या \;३~।}}\\
\vspace{-1.5mm}
\mbox{{या ~~१० \;रू \;२ं~।}}
\vspace{1mm}
\end{matrix}$} \;अत्र यथोक्त्या सहित इति \hyperref[3.22]{"योगोऽन्तरं तेषु समानजात्योः"} इत्यादिना तत्र प्रथमपङ्क्तौ यावत्त्रयमृणम्~। द्वितीयपङ्क्तौ यावद्दशकं धनम्~। अनयोः साजात्याद्योगे \hyperref[1.3]{"धनर्णयोरन्तरमेव योगः"} इति जातं या ७~। इतरयोर्भिन्नजातित्वात्पृथक् स्थितिरेव~। तथा कृते जातं गुणनफलं याव १५ या ७ रु २ं~। अथ गुण्येन धनर्णव्यत्यासं कृत्वा द्वितीयोदाहरणे न्यासो \;{\scriptsize $\begin{matrix}
\mbox{{या \;३~। \;या \;५ं \;रू \;१~।}}\\
\vspace{-1.5mm}
\mbox{{रू \;२~। \;या \;५ं \;रू \;१~।}}
\vspace{1mm}
\end{matrix}$} \;गुणकखण्डाभ्यां गुणिते जातं \;{\scriptsize $\begin{matrix}
\mbox{{याव \;१ं५ \;या \;३~।}}\\
\vspace{-1.5mm}
\mbox{{या ~~१ं० \;रू \;२~।}}
\vspace{1mm}
\end{matrix}$} \;यथोक्त्या योगे जातं गुणनफलं याव १५ं या ७ं रू २~। अथ गुणके धनर्णताव्यत्यासं कृत्वा तृतीयोदाहरणे न्यासो \;{\scriptsize $\begin{matrix}
\mbox{{या \;३ं~। \;या \;५ \;रू \;१ं~।}}\\
\vspace{-1.5mm}
\mbox{{रू \;२ं~। \;या \;५ \;रू \;१ं~।}}
\vspace{1mm}
\end{matrix}$} \;गुणने जातं \;{\scriptsize $\begin{matrix}
\mbox{{याव \;१ं५ \;या \;३~।}}\\
\vspace{-1.5mm}
\mbox{{या ~~१ं० \;रू \;२~।}}
\vspace{1mm}
\end{matrix}$} \;यथोक्तयोगे जातं गुणनफलं याव १५ं या ७ं रू २~। अथोभयोर्व्यत्यासे चतुर्थोदाहरणे न्यासो \;{\scriptsize $\begin{matrix}
\mbox{{या \;३ं~। \;या \;५ं \;रू \;१~।}}\\
\vspace{-1.5mm}
\mbox{{रू \;२ं~। \;या \;५ं \;रू \;१~।}}
\vspace{1mm}
\end{matrix}$} \;गुणिते जातं \;{\scriptsize $\begin{matrix}
\mbox{{याव \;१५ \;या \;३ं~।}}\\
\vspace{-1.5mm}
\mbox{{या ~~१० \;रू \;२~।}}
\vspace{1mm}
\end{matrix}$} \;यथोक्तयोगे जातं गुणनफलं याव १५ या ७ रू २ं~। अत्रोपपतिः~। रूपै रूपेषु गुणितेषु रूपाणि भवन्तीति प्रसिद्धम्~। रूपेण वर्णे गुणिते रूपं वा भवेद्वर्णो वा~। विनिगमनाविरहे सति कथं वर्ण एवेत्युक्तम्~। उच्यते~। अज्ञातराशि-
\end{sloppypar}

\newpage

\begin{sloppypar}
\noindent मानं तावच्चतुर्धैव सम्भवति~। रूपसमूहस्तदवयवो रूपं रूपावयवो वेति~। तत्र रूप-समूहत्वमज्ञातराशेरङ्गीकृत्य युक्तिरुच्यते~। अस्ति किञ्चिद्धान्यं सप्ताढकमानेनैकं मानम् १~। इदं सप्तगुणितं जातम् ७~। एतस्य गुणनफलस्य रूपात्मकत्वं समूहात्मकत्वं वेति विचार्यम्~। तत्रास्य रूपात्मकत्वे सप्ताढकधान्यमिदमिति स्यात्~। न चैतद्युक्तम्~। गुणनात्पूर्वमेव सप्ताढकस्य धान्यस्य विद्यमानत्वात्~। गुणनोत्तरं त्वेकोनपञ्चाशदाढका भाव्याः~। अतः समूहात्मकत्वं वक्तव्यम्~। तथा सति सप्ताढकधान्यसमूहाः सप्तेत्युपपन्नं \hyperref[3.26]{'स्याद्रूपवर्णाभिहतौ तु वर्णः'} इति~। अथाज्ञातराशौ रूपममूहावयवत्वमुररीकृत्य युक्तिरुच्यते~। अस्ति सप्ताढकस्य मानस्य त्र्यंशमितं मानम्~। अनेन मानेनास्ति धान्यमिति १~। इदं त्रिगुणितं ३~। अस्य रुपात्मकत्व आढकत्रयमेव स्यात्~। तच्चायुक्तम्~। आढकसप्तकस्य त्र्यंशे हि त्रिगुणित आढकसप्तकेन भाव्यम्~। अत एव तस्य समूहावयवात्मकत्वम्~। तथा सति त्रय आढकसप्तकत्र्यंशा इति स्यात्~। एवमप्युपपन्नं~। \hyperref[3.26]{'स्याद्रूपवर्णाभिहतौ तु वर्ण'} इति अथ रूपावयवत्वमज्ञातराशेरुररीकृत्योच्यते~। अस्त्याढकचतुर्थांशमितं मानम्~। एतन्मितं धान्यं प्रस्थमितं १ भवति~। इदं त्रिभिर्गुणितं ३~। नेदं रूपात्मकम्~। अस्य रूपात्मकत्व आढकत्रयं स्यात्~। न चैतद्युक्तम्~। तस्माद्रूपावयवात्मकत्वमस्य वक्तव्यम्~। तथा सत्याढकचतुर्थांशास्त्रय इति भवति प्रस्थत्रयम्~। एवमप्युपपन्नं \hyperref[3.26]{'रूपवर्णाभिहतौ वर्णः'} इति~। अथाज्ञातराशे रूपत्वे वर्णरूपयोरभेदाद्गुणनफले वर्णतापि युक्तैव~। न च गुणनफले रूपत्वमेवास्तु~। तस्यापि युक्तत्वादिति वाच्यम्~। अज्ञातराशे रूपत्वे वर्णरूपत्वेनावगमाभावात्~। अवधृते हि राशे रूपत्वे गुणनफले रूपत्वमपि युक्तम्~। अत्र तु राशेरज्ञानाद्रूपत्वानवधारणात्~। न चैवं गुणनफले वर्णत्वमपि कयं स्याद्रूपसमूहत्वादिना राशेरनवगमादिति वाच्यम्~। नहि रूपवर्णयोर्गुणनफलस्य वर्णत्वे रूपसमूहत्वादिनाप्यवगमो राशेरावश्यकः~। किन्तु तस्य चतुष्टयसाधारणत्वाच्चतुष्टयान्यतमत्वेनैव राशेरवगमोऽपेक्षितः~। स चास्त्येव~। चतुष्टयान्यस्य राशेरसंभवात्~। अत एव लाघवाद्वर्णत्वपुरस्कारेणैव प्रकृतगुणनफलस्य वर्णत्वमुक्तमाचार्यैरित्युपपन्नं \hyperref[3.26]{'स्याद्रूपवर्णाभिहतौ तु वर्णः'} इति~। किञ्च रूपं हि व्यक्त-सङ्ख्या~। तथा गुणनेऽङ्कत एव गुणनं स्यान्नाक्षरतः~। न च रूपव्यक्तसङ्ख्ययोरभेदे सङ्ख्याज्ञापकाङ्कलिखनमेवास्तु किं रूपप्रथमाक्षरलिखनेनेति वाच्यम्~। अङ्कस्य भेदकाभावे वर्णाङ्कसंनिधानेन कदाचित्सङ्करः स्यादित्यसङ्करार्थं रूपाक्षरलिखनात्~। अत एव सति रेखादिके भेदके नास्त्येवाक्षरलिखनोपयोगः~। किन्तु शीघ्रोपस्थितये तत्~। एवं यावत् वर्गादीनामपि रूपगुणनेऽक्षरतो न विकार इत्यादि सुधीभिरन्यदप्युह्यम्~। अथ समजाति-वर्णगुणने तत्र वर्णस्य रूपसमूहत्वमुररीकृत्य युक्तिरुच्यते~। यथाढसप्तकस्यैकः समूहः १ अनेनैवास्मिन्गुणिते जातम् १~। अस्याढकसप्तलक्षणसमूहात्मकत्व एक-
\end{sloppypar}

\newpage
 
\begin{sloppypar}
\noindent गुणितसमूहस्य समूहगुणितसमूहस्य चाभेदापत्तिः~। न चात्रेष्टापत्तिः~। एकस्मिन्गुण्ये गुणक-भेदाद्गुणनफलभेदस्यावश्यकत्वात्~। अतो गुणनफलस्य समूहवर्गात्मकत्वं वक्तव्यम्~।~तथा सत्येकोनपञ्चाशदाढकाः स्युः~। युक्तं चैतत्~। अतः समानजात्योर्द्वयोर्वर्णयोर्वधे~तद्वर्गो भवतीत्युपपन्नम्~। एवं समूहावयवत्वादिकमप्यङ्गीकृत्य युक्तिर्द्रष्टव्या~। एवं त्र्यादीनां~सम-जातिकानां वधे धनादित्वमप्यूह्यम्~। तदेवमुपपन्नं \hyperref[3.26]{'द्वित्र्यादिकानां समजातिकानां~वधे तु तद्वर्गघनादयः स्युः'} इति~। अथासमजातिघात आढकसप्तकात्मकः समूहः~१~आढ-कपञ्चात्मकोऽन्यः १~। अनयोर्वधे जातं १~। नायमाढकसप्तकात्मकः समूहः~।~तस्यैक-गुणस्य समूहगुणितस्य चाभेदापत्तेः~। नायं समूहवर्गः~। समूहस्य स्वेन गुणने समूहान्तरेण च गुणने गुणनफलस्याभेदापत्तेः~। अतः समूहयोर्वधोऽयमेकः~।~तथा सति पञ्चत्रिंशदाढकाः स्युः~। युक्तं चैतत्~। तस्मादसमजातिघाते तयोर्घात इत्यक्षरतो भवितुं युक्तम्~। तत्राद्यैर्घातस्य भावितमिति सञ्ज्ञा कृता~। वधशब्दस्याद्याक्षरलिखने यावदादिवर्गेण सङ्करः स्यात्~। घातशब्दस्याद्याक्षरलिखने कदाचिद्घ नेन सङ्करः स्यात्~। गुणनशब्दप्रथमाक्षरलिखनेऽश्लीलता स्यात्~। हतिशब्दप्रथमाक्षरलिखने कदा-चिद्धरितकवर्णभ्रमः स्यादिति~। अथ यद्यपरः कश्चिच्छब्दोऽस्ति तत्प्रथमाक्षरलिखने सङ्करादिदोषो न स्यात्~। अस्तु तर्हि तल्लिखनं न कदाचित्क्षतिः~। किन्त्वाचार्येणाद्या-क्षरानुरोधाद्भावितमिति सञ्ज्ञा कृतेत्युपपन्नं \hyperref[3.26]{'तद्भावितं चासमजातिघाते'} इति~। खण्ड-गुणनोपपत्तिः स्पष्टैव~॥~२८~॥\\

{\small अथ {\color{violet}"भाज्याद्धरः शुध्यति"} इत्यादिना भजनफलसिद्धावपि वर्णसञ्ज्ञावधानार्थं मन्दावबोधार्थं च पुनः शालिन्या विशदयति\textendash }

\phantomsection \label{3.29}
\begin{quote}
{\large \textbf{{\color{purple}भाज्याच्छेदः शुध्यति प्रच्युतः सन्स्वेषु स्वेषु स्थानकेषु क्रमेण~।\\
यैर्यैर्वर्णैः सङ्गुणो यैश्च रूपैर्भागाहारे लब्धयस्ताः स्युरत्र~॥~२९~॥}}}
\end{quote}

\hyperref[3.29]{\textbf{छेदो}} हरः स \hyperref[3.29]{\textbf{यैर्यैर्वर्णैर्यै रूपै}}श्च गुणितः सन् भाज्यात् \hyperref[3.29]{\textbf{स्वेषु स्वेषु स्थानेषु}} यथास्वं समानजातिषु \hyperref[3.29]{\textbf{प्रच्युतः सञ्छुद्ध्यति}} न शिष्यति \hyperref[3.29]{\textbf{ता}} अत्र \hyperref[3.29]{\textbf{लब्धयः स्युः}}~। ते वर्णास्तानि च रूपाणि लब्धयः स्युरित्यर्थः~। अत्र यैर्गुणितो हरो भाज्याच्छुध्यति तेष्वधिको लब्धिर्भवतीति द्रष्टव्यम्~। अन्यथा न्यूनगुणोऽपि हरः शुध्यतीति न्यूना अपि लब्धयः स्युः~। यद्वा भाज्योऽपि शुध्यतीति द्रष्टव्यम्~। ता लब्धय इत्यत्र तच्छब्दस्य विधीयमानलिङ्गता {\color{violet}'शैत्यं हि यत् सा प्रकृतिर्जलस्य'} इत्यादौ प्रसिद्धा~। {\color{violet}'दैवे युगसहस्रे द्वे ब्राह्मः कल्पौ तु तौ नृणाम्'} इत्यस्य व्याख्यावसरे लिखितं च {\color{violet}क्षीरस्वामिना 'सर्वनाम्नां विधीयमानानूद्यमानलिङ्गग्रहणे कामचारः'} इति~। अत्रोदाहरणार्थं पूर्वगुणनफलस्य स्वगुणच्छेदस्य न्यासः~। तत्र भाज्यो याव १५ या ७ 
\end{sloppypar}

\newpage

\begin{sloppypar}
\noindent रू २ं~। भाजको या ३ रू २~। अत्र भाज्ये प्रथमतो यावद्वर्गाः सन्ति तेभ्यो यावत् वर्गाणामेव शोधनं युक्तम्~। समजातित्वात्~। अत्र हरे तु प्रथमतो यावत्त्रयमस्ति~। तस्य रूपेण गुणने \hyperref[3.26]{'स्याद्रूपवर्णाभिहतौ तु वर्णः'} इति वर्ण एव स्यान्न तद्वर्गः~। यावता गुणनेऽपि समानजातिघातत्वाद्यद्यपि यावद्वर्गो भवेत्तथाप्यङ्कतस्त्रयमेवेति तच्छोधनेऽपि भाज्येन यावद्वर्गाणां न शुद्धिः~। अतो यावत्पञ्चकेन भाजके गुणिते पञ्चदश यावद्वर्गा भवेयुः~। तथा सति शुद्धिर्भवेदिति यावत्पञ्चकेन या ५ छेदोऽयं या ३ रू २ गुणितो याव १५ या १०~। अस्मिन्भाज्यादस्मात् याव १५ या ७ रू २ं यथास्थानमपनीते जातं या ३ं रू २ं~। यावत्पञ्चकेन गुणितश्छेदः शुद्ध इति यावत्पञ्चकं लब्धिः\textendash \,या ५~। अथ भाज्यशेषे यावत्तावत्त्रयमस्ति~। अतो हरे रूपेण गुणिते तस्माच्छोधिते तस्य शुद्धिः स्यात्~। परं धनरूपेण गुणेन \hyperref[1.7]{'संशोध्यमानं स्वमृणत्वमेति'} इति द्वयोरृणत्वाद्योगः स्यादिति न शुद्धिः~। तस्मादृणरूपेण गुणिते तस्माच्छोधितस्य शुद्धिः स्यादिति ऋणरूपेण रू १ं हरोऽयं या ३ रू २ गुणितो या ३ं रू २ं भाज्यशेषादस्मात् या ३ं रू २ं 'च्युतः शुध्यति' इति रूपमृणं लब्धी रू १ं~। एवं जाता लब्धिः\textendash \,या ५ रू १ं~। पूर्वगुण्योऽयम्~। अथ द्वितीयोदाहरणे भाज्यो याव १ं५ या ७ं रू २~। भाजको या ३ रू २~। उक्तवज्जाता लब्धिः\textendash \,या ५ रू १~। अथ तृतीयोदाहरणे भाज्यो याव १ं५ या ७ रू २~। भाजको या ३ं रू २ं~। उक्तवल्लब्धिः\textendash \,या ५ रू १ं~। अथ चतुर्थोदाहरणे भाज्यो याव १५ या ७ रू २ं~। भाजको या ३ं रू २ं~। उक्तवल्लब्धिः\textendash \,या ५ं रू १~। अत्रोपपत्तिः\textendash \,भाज्यराशिस्तावत्कयोश्चिद्गुण्यगुणकयोर्गुणनफलम्~। भाजकस्तु गुण्यगुणकयोरन्यतरः~। तदितरो लब्धिश्चेति स्थितिरस्ति~। तत्रास्मिन्गुणनफलेऽस्मिंश्च गुणके सति को गुण्य इत्यस्मिन्गुण्ये सति को वा गुणक इति लब्धिः प्रश्नार्थः~। तत्र गुणको येन गुणितः सन् गुणनफलसमो भवेत्स, गुण्यो वा येन गुणितः सन् गुणनफलसमः स्यात्स गुणक इति स्पष्टैव युक्तिः~। ननु तथाप्येतावदेव वक्तव्यं यद्गुणितो हरो भाज्यसमः स्यादिति न तु \hyperref[3.29]{'प्रच्युतः सञ्छुद्ध्यति'} इति~। गौरवात्~। सत्यम्~। असमे समताभ्रमनिबन्धनोऽलब्धौ लब्धिभ्रमः स्यादिति तन्निराप्तार्थं \hyperref[3.29]{"प्रच्युतः सञ्छुद्ध्यति"} इत्युक्तम्~। अन्यथा भाज्येऽस्मिन् याव १५ या ७ रू २ सति हरेऽस्मिन् या ३ रू २ अनेन या ५ रू १ गुणितो याव १ं५ या ७ं रू २ भाज्यसमताभ्रमेण लब्धिरियं या ५ं रू १ इति भ्रमः स्यात्~। शोधने तु \hyperref[1.7]{'संशोध्यमानं स्वमृणत्वमेति'} इत्युभयेषां यावद्वर्गाणां यावतां च धनत्वा-
\end{sloppypar}

\newpage

\begin{sloppypar}
\noindent द्रुपयोश्चर्णत्वाद्योगे वैगुण्यं स्यान्नतु शुद्धिरिति लब्धित्वभ्रमो न स्यात्~। ननु विशेषादर्शनं भ्रमं प्रति हेतुरिति यथा प्रकृते धनर्णत्वलक्षणविशेषादर्शनाद्भाज्यसमताभ्रमस्तथा शोधनेऽपि विशेषादर्शनस्य सत्त्वात्कथं न भ्रमः स्यादिति चेन्मैवम्~। तत्र धनर्णत्वलक्षणविशेषस्य दर्शनमदर्शनं च सम्भाव्यते~। शोधने तु \hyperref[1.7]{'संशोध्यमानं स्वमृणत्वमेति'} इति धनर्णत्वाक्षेपान्न विशेषादर्शनं सम्भवति~। किञ्च भाजको येन गुणितो भाज्यसमो भवेत्तस्य न शीघ्रमुपस्थितिः शोधने तु भाज्येऽत्र प्रथमतः पञ्चदश यावद्वर्गा दृश्यन्ते~। भाजके तु यावत्त्रयम्~। तद्यदि यदत्पञ्चकेन गुण्यते तर्हि पञ्चदश यावद्वर्गा भवेयुः~। तथा सति यावद्वर्गाणां शुद्धिः स्यादित्यस्ति शीघ्रोपस्थितिः~। एवं भाज्यशेषशुद्धावपीत्यलं पल्लवितेन~॥~२९~॥\\

{\small अथ यद्यपि वर्गसूत्रमन्तरा तदुदाहरणं वक्तुमनुचितं तथापि वर्गस्य का समद्विघात-रूपत्वाद्गुणनसूत्रेणैव तत्सिद्धेः \hyperref[3.27]{'अव्यक्तवर्गकरणीगुणनासु चिन्त्यः'} इति विशेषोक्तेश्च तदुचितमेवेति शालिन्यर्धेन तदाह\textendash }

\phantomsection \label{3.30}
\begin{quote}
{\large \textbf{{\color{purple}रूपैः षड्भिर्वर्जितानां चतुर्णामव्यक्तानां ब्रूहि वर्गं सखे मे~॥~३०~॥}}}
\end{quote}

स्पष्टोऽर्थः~। रूपषट्कोनमव्यक्तचतुष्टयमिदं या ४ रू ६ं~। वर्गार्थमयमेव गुण्यो गुणकश्चेति न्यासो \;{\scriptsize $\begin{matrix}
\mbox{{या \;४~। \;या \;४ \;रू \;६ं~।}}\\
\vspace{-1.5mm}
\mbox{{रू \;६ं~। \;या \;४ \;रू \;६ं~।}}
\vspace{1mm}
\end{matrix}$} \;स्थानद्वयेऽपि गुणने जातं \;{\scriptsize $\begin{matrix}
\mbox{{याव \;१६ \;या \;२ं४~।}}\\
\vspace{-1.5mm}
\mbox{{या ~~२ं४ \;रु \;३६~।}}
\vspace{2mm}
\end{matrix}$}\; योगे जातो वर्गो याव १६ या ४ं८ रू ३६~॥~३०~॥\\

{\small अथ वर्गे दृष्टे कस्यायं वर्ग इति मूलाङ्कज्ञानार्थमुपायमुपजातिकयाह\textendash }

\phantomsection \label{3.31}
\begin{quote}
{\large \textbf{{\color{purple}कृतिभ्य आदाय पदानि तेषां द्वयोर्द्वयोश्चाभिहतिं द्विनिघ्नीम्~।\\
शेषात्त्यजेद्रूपपदं गृहीत्वा चेत्सन्ति रूपाणि तथैव शेषम्~॥~३१~॥}}}
\end{quote}

तेषां वर्गराशिगताव्यक्तानां मध्ये \hyperref[3.31]{\textbf{कृतिभ्यः पदान्यादाय तेषां}} पदानां परस्परं \hyperref[3.31]{\textbf{द्वयोर्द्वयोरभिहतिं द्विनिघ्नीं शेषात्}} विशोधयेत्~। यदि शुद्धिर्भवेत्तदा तानि तस्य वर्गस्य पदानि स्युरित्यर्थादुक्तं भवति~। कृत्योरित्यपि द्रष्टव्यम्~। अथ यदि वर्गराशौ \hyperref[3.31]{\textbf{रूपाणि सन्ति}} तर्हि \hyperref[3.31]{\textbf{रूपपदं गृहीत्वा शेषं}} तथैव द्वयोर्द्वयोश्चाभिहतिं द्विनिघ्नीं शेषात्यजेदिति~। रूपेषु सत्सु यदि रूपपदं न लभ्यते तदा स वर्गराशिर्नेत्यर्थादुक्तं भवति~। अत्रोदाहरणम्~। पूर्वसिद्धवर्गस्य मूलार्थं न्यासो याव १६ या ४ं८ रू ३६~। अत्र वर्गराशौ षोडश यावद्वर्गाः षट्त्रिंशद्रूपाणि चेति वर्गद्वयमस्माद्गृहीते मूले या ४ रू ६~। अनयोर्द्वयोरभिहतिं या २४ द्विनिघ्नीं या ४८ शेषात्त्यजेदिति \hyperref[1.7]{'संशोध्यमानं स्वमृणत्वमेति'}~।
\end{sloppypar}

\newpage

\begin{sloppypar}
\noindent इत्यृणयोर्योगे शुद्धिर्न स्यादिति द्वयोरन्यतरस्यर्णत्वं कल्प्यते~।
तथा सति द्वयोरभिहतिर्द्विगुणे या ४ं८ \hyperref[1.7]{'संशोध्यमानं स्वमृणत्वमेति'} इति धनत्वे \hyperref[1.3]{'धनर्णयोरन्तरमेव योगः'} इति शुद्धिः स्यात्~। अतोऽस्य या ४ रू ६ं अस्य वा या ४ं रू ६ वर्गोऽयं यव १६ या ४ं८ रू ३६~। ननु रूपषट्कयुतस्य यावत्त्रयस्य या ३ रू ६ वर्गोऽयं~। याव ९ य ३६ रू ३६~। अत्र \hyperref[3.31]{'कृतिभ्य आदाय पदानि'} इत्यादिना सर्वेभ्योऽपि मूललाभाच्छेषाभावे द्वयोर्द्वयोरभिहतिं द्विनिघ्नीं कुतः शोध्येति चेन्न~। यावतां या ३६ मूलाभावात्~। नहि यावदात्मकः कस्यापि वर्गः सम्भवति यदस्य मूलं सम्भवेदिति सर्वमवदातम्~। अत्रोपपत्तिः~। समद्विघातो हि वर्गः~। तथा च यस्य वर्गः क्रियते स एव गुण्यो गुणकश्व~। तत्रैकखण्डात्मके वर्गे कस्यायं समद्विघात इति समद्विघातान्वेषणे मूलावगमः सुगमः~। अथ खण्डद्वयस्य वर्गार्थं न्यासो \;{\scriptsize $\begin{matrix}
\mbox{{या \;४~। \;या \;४ \;रू \;६~।}}\\
\vspace{-1.5mm}
\mbox{{रू \;६~। \;या \;४ \;रू \;६~।}}
\vspace{1mm}
\end{matrix}$}\; अत्र प्रथमपङ्क्तावेकस्यखण्डस्य वर्गः खण्डद्वयाभिहतिश्च~। द्वितीयपङ्क्तावपि खण्डद्वयाभिहतिर्द्वितीयखण्डवर्गश्च~। अत्र पङ्क्तिद्वयेऽपि खण्डाभिहतिरस्तीति योगे द्विगुणिताभिहतिः स्यात्~। अतः खण्डद्वयस्य वर्गे खण्डत्रयं भवति खण्डवर्गौ द्विगुणिता खण्डद्वयाभिहतिश्च याव १६ या ४८ रू ३६~। अथ खण्डत्रयवर्गे \;{\scriptsize $\begin{matrix}
\mbox{{या \;३~। \;या \;३ \;का \;४ \;नी \;५~।}}\\
\mbox{{का \;४~। \;या \;३ \;का \;४ \;नी \;५~।}}\\
\vspace{-1.5mm}
\mbox{{नी \;५~। \;या \;३ \;का \;\;४ \;नी \;५~।}}
\vspace{2mm}
\end{matrix}$}\; अत्र प्रथमपङ्क्तौ प्रथमखण्डवर्गः प्रथमद्वितीयखण्डाभिहतिः प्रथमतृतीयखण्डाभिहतिश्च~। द्वितीयपङ्क्तौ द्वितीयखण्डवर्गः प्रथमद्वितीयाभिहतिर्द्वितीयतृतीयाभिहतिश्च~। तृतीयपङ्क्तौ तृतीयखण्डवर्गः प्रथमतृतीयाभिहतिर्द्वितीयतृतीयाभिहतिश्चेति~। एवं चतुरादिखण्डवर्गेष्वपि~। तथा च वर्गे क्रियमाणे खण्डानां वर्गा द्वयोर्द्वयोर्द्विगुणाभिहतिश्च स्यात्~। तस्मात्सुष्ठूक्तं \hyperref[3.31]{'कृतिभ्य आदाय'} इत्यादि~। ननु वर्गराशाववश्यं खण्डवर्गा भवन्तीति कृतिभ्यः पदान्यादायेत्यनेनैव चरितार्थत्वाद्द्वयोर्द्वयोरित्यादि व्यर्थमिति चेन्न~। तथा सति यत्र राशौ याव ९ या ८ रू ९ एवमस्ति तस्यापि मूलं या ३ रू ३ स्यात्~। न चैतद्युक्तम्~। यतोऽस्य वर्गोऽयं याव ९ या १८ रू ९~। तस्माद्यदि मूलेषु गृहीतेषु द्वयोर्द्वयोर्द्विगुणाभिहतिः शिष्यते तर्ह्येव तस्य वर्गत्वमिति नियमार्थं द्वयोर्द्वयोश्चाभिहतिं द्विनिघ्नीं शेषात्त्यजेदित्युक्तम्~॥~३१~॥ \\

{\small एवमेवावर्णषड्विधोदाहरणान्युक्त्त्वानेकवर्णषड्विधयोदाहरणानि प्रदर्शयति~। तत्रार्ययानेकवर्ण-सङ्कलनव्यकलनयोरुदाहरणमाह\textendash }
\end{sloppypar}

\newpage

\begin{sloppypar}
\phantomsection \label{3.32}
\begin{quote}
{\large \textbf{{\color{purple}यावत्तावत्कालकनीलकवर्णास्त्रिपञ्चसप्तधनम्~।\\
द्वित्र्येकमितैः क्षयगैः सहिता रहिताः कति स्युस्तैः~॥~३२~॥}}}
\end{quote}

\hyperref[3.32]{\textbf{धनं त्रिपञ्चसप्त यावत्तावत्कालकनीलकवर्णाः क्षयगैर्द्वित्र्येकमितै}}स्तैर्यावत्तावत्कालक-नीलकवर्णैः \hyperref[3.32]{\textbf{सहिताः कति स्यू रहिताः}} च कति स्युरित्युदाहरणद्वयम्~। अत्र यावत् तावत्कालकनीलकवर्णानां भिन्नजातित्वात्पृथक्स्थितिरेव~। या ३ का ५ नी ७ एतैः क्षयगैर्द्वित्रयेकमितैरेतैः\textendash \,या २ं का ३ं नी १ं सहिताः \hyperref[1.3]{'धनर्णयोरन्तरमेव योगः'} इति \hyperref[3.22]{'योगोऽन्तरं तेषु समानजात्योः'} इति जाता या १ का २ नी ६~। रहिताश्चेत्तदा संशोध्यमानमृणं धनं भवतीति धनत्वे साजात्याद्योगे जाता या ५ का ८ नी ८~॥~३२~॥\\

{\small अथानेकवर्णगुणनादिचतुष्टयोदाहरणानि मन्दाक्रान्तयाह\textendash }

\phantomsection \label{3.33}
\begin{quote}
{\large \textbf{{\color{purple}यावत्तावत्त्रयमृणमृणं कालकौ नीलकः स्वं \\
रूपेणाढ्या द्विगुणितमितैस्तैस्तु तैरेव निघ्नाः~।\\
किं स्यात्तेषां गुणनजफलं गुण्यभक्तं च किं स्यात्\\
गुण्यस्याथ प्रकथय कृतिं मूलमस्याः कृतेश्च~॥~३३~॥ }}}
\end{quote}

स्फुटोऽर्थः~। ऋणं यावत्तावत्त्रयं या ३ं ऋणं कालकौ का २ं धनं नीलको नी १ एते रूपेणाढ्या जातो गुण्यो या ३ं का २ं न १ रू १~। एत एव द्विगुणा जातो गुणको या ६ं का ४ं नी २ रू २~। अयं गुणनार्थं न्यासः~।

\begin{center}
\begin{tabular}{rllll}
या ६ं~। & या ३ं & का २ं & नी १ & रू १ \\
का ४ं~। & या ३ं & का २ं & नी १ & रू १ \\
नी २~। & या ३ं & का २ं & नी १ & रू १ \\
रू २~। & या ३ं & का २ं & नी १ & रू १ 
\end{tabular}
\end{center}

\noindent \hyperref[3.26]{'स्याद्रूपवर्णाभिहतौ तु वर्णः'} इत्यादिना गुणनेन जातं पङ्क्तिचतुष्टये गुणनफलमक्षर-तोऽङ्कतश्च~। अत्र यावद्वर्गाधस्तिर्यक्स्थितानां च क्रमेण साजात्याद्योगे कालकवर्गादपि

\begin{center}
\begin{tabular}{lclclclc}
याव  & १८ & याकाभा & १२ & यानीभा & ६ं & या & ६ं\\
याकाभा & १२ & काव & ८ & कानीभा & ४ं & का & ४ं\\
यानीभा & ६ं & कानीभा & ४ंं & नीव & २ & नी & २\\
या & ६ं & का & ४ं & नी & २ & रू & २
\end{tabular}
\end{center}
\end{sloppypar}

\newpage

\begin{sloppypar}
\noindent तिर्यगधःस्थितानां कालकनीलक भा ४ं का ४ं क्रमेण साजात्याद्योगे नीलकवर्गादपि तिर्यगधःस्थितयोः नी २ साजात्याद्योगेऽन्येषां पृथक्स्थितौ च जातं गुणनफलं याव १८ याकाभा २४ यानीभा १२ं या १२ं काव ८ कानीभा ८ं का ८ं नीव २ नी ४ रू २~। अथेदं गुण्यभक्तं किं स्यादिति भागहारार्थं गुण्यच्छेदस्य गुणनफलस्य न्यासो याव १८ याकाभा २४ यानीभा १ं२ या १ं२ काव ८ कानीभा ८ं का ८ं नीव २ नी ४ रू २~। या ३ं का २ं नी १ रू १~। अत्र \hyperref[3.29]{'भाज्याच्छेदः शुध्यति प्रच्युतः सन्'} इत्यादिना लब्धिर्ग्राह्या~। अत्र भाज्ये प्रथमतोऽष्टादश यावद्वर्गाः सन्ति भाजके च यावत्त्रयं या ३ं~। अस्मिन्यावत्षट्केन गुणित ऋणमष्टादश यावद्वर्गा भवन्ति~। एते यदा शोध्यन्ते तदा धनं स्युरिति साजात्याद्योगः स्यान्न शुद्धिः~। ऋणयावत्षट्केन हरगुणने तु शुद्धिः स्यात्~। अतोऽनेन या ६ं हरो गुणितो जातो याव १८ याकाभा १२ यानीभा ६ं या ६ं~। अस्मिन्यथास्थानं भाज्यादपनीते शेषं लब्धिश्च याकाभा १२ यानीभा ६ं या ६ं काव ८ कानीभा ८ं का ८ं नीव २ नी ४ रू २~। या ३ं का २ं नी १ रू १~। लब्धिश्च या ६ं~। अथ भाज्ये यावत्कालकभावितमस्ति~। ऋणकालकैः का ४ं हरगुणने तस्य शुद्धिः स्यादिति लब्धिः का ४ं~। एतद्गुणो भाजको जातो याकाभा १२ काव ८ कानीभा ४ं का ४ं~। अस्मिन्भाज्यादपनीते शेषं यानीभा ६ं या ६ं कानीभा ४ं का ४ं नीव २ नी ४ं रू २~। अत्र भाज्ये यावन्नीलकभावितमस्ति~। नीलकद्वयेन भाजके गुणिते तस्मादपनीते शुद्धिः स्यादिति लब्धिः\textendash \,नी २~। एतद्गुणो भाजको यानीभा ६ं कानीभा ४ं नीव २ नी २~। अस्मिन्भाज्यादपनीते शेषं या ६ं का ४ं नी २ रू २~। अथ भाज्ये यावत्षट्कमस्ति~। हरे रूपद्वयगुणिते तस्य  शुद्धिः स्यादिति लब्धी रू २~। रूपद्वयगुणितो हरो या ६ं का ४ं नी २ रू २~। अस्मिन्भाज्यादपनीते सर्वशुद्धिरिति जाता सम्पूर्णा लब्धिः\textendash \,या ६ का ४ नी २ रू २~। \hyperref[3.33]{\textbf{अथ गुण्यस्य प्रकथये}}ति गुण्यस्य स्वेन गुणनार्थं न्यासः\textendash

\begin{center}
\begin{tabular}{rllll}
या ३ं~। & या ३ं & का २ं & नी १ & रू १~। \\
का २ं~। & या ३ं & का २ं & नी १ & रू १~।\\
नी १~। & या ३ं & का २ं & नी १ & रू १~।\\
रू १~। & या ३ं & का २ं & नी १ & रू १~।
\end{tabular}
\end{center}

\noindent उक्तवद्गुणने योगे च जातो वर्गो याव ९ याकाभा १२ यानीभा ६ं या ६ं काव
\end{sloppypar}

\newpage

\begin{sloppypar}
\noindent ४ कानीभा ४ं का ४ं नीव १ नी २ रू १~। अथास्याः कृतेर्मूलं कथयेति मूलोदाहरणम्~। अत्र \hyperref[3.31]{'कृतिभ्य आदाय पदानि'} इति गृहीतानि पदानि या ३ का २ नी १ रू १~। अत्र द्वयोर्द्वयोरभिहतिं द्विनिघ्नीं यथाक्रमं याकाभा १२ यानीभा ६ या ६~। इयं वर्गशेषाच्छोध्येति \hyperref[1.7]{'संशोध्यमानं स्वमृणत्वमेति'} इति यद्यपि यावत्कालकभावितानामृणत्वे \hyperref[1.3]{'धनर्णयोरन्तरमेव योगः'} इति भवति शुद्धिस्तथापि यावन्नीलकभावितानां यावतां चर्णत्वे साजात्याद्योगे द्वैगुण्यं स्यान्न शुद्धिः~। अतो यावत्तावत्त्रयमृणं मूलं कल्प्यते~। \hyperref[1.13]{'स्वमूले धनर्णे'} इत्युक्तत्वात्~। तथा सति द्वयोर्द्वयोरभिहतिर्द्विनिघ्नीं याकाभा १२ यानीभा ६ं या ६ं~। अत्र यद्यपि \hyperref[1.7]{'संशोध्यमानं स्वमृणत्वमेति'} इत्यादिना यावन्नीलकभावितानां यावतां च भवति शुद्धिस्तथापि यावत्कालकभावितानां द्वैगुण्यं स्यान्न शुद्धिः~। तस्मात्पूर्वस्यामभिहतौ यावन्नीलकभावितानां यावतां च व्यत्यासार्थं नीलकरूपयोर्ऋणत्वं कल्प्यमथवास्यामभिहतौ यावत्कालकभावितानां व्यत्यासार्थं कालकस्यर्णत्वं कल्प्यमिति द्विधैव गतिरस्ति~। तथा सति मूलान्येतानि या ३ं का २ं नी १ रू १~। एतानि वा या ३ का २ नी १ं रू १ं~। उभयेषामपि परस्परं द्वयोर्द्वयोरभिहतिर्द्विनिघ्नी तुल्यैव~। याकाभा १२ यानीभा ६ं या ६ं कानीभा ४ं का ४ं नी २~। अस्याः शोधनेन भवति सर्वशुद्धिरिति द्वयस्यापि पदत्वं सिद्धम्~॥~३३~॥ 

\begin{quote}
{\color{violet}दैवज्ञवर्यगणसन्ततसेव्यपार्श्वबल्लालसञ्ज्ञगणकात्मजनिर्मितेऽस्मिन्~। \\
बीजक्रियाविवृतिकल्पलतावतारे वर्णोद्भवाः समभवन्निति षट् प्रकाराः~।}
\end{quote}
\vspace{-1mm}

\begin{center}
इति श्रीसकलगणकसार्वभौमश्रीबल्लालदैवज्ञसुतकृष्णगणकविरचिते\\
बीजविवृतिकल्पलतावतारे वर्णषड्विधविवरणं समाप्तम्~॥\\
(अथ ग्रन्थसङ्ख्या विंशत्यधिकशतत्रयम् ३२०)~।
\vspace{6mm}

\rule{0.2\linewidth}{0.8pt}\\
\vspace{-4mm}

\rule{0.2\linewidth}{0.8pt}
\end{center}
\end{sloppypar}

\newpage
\thispagestyle{empty}

\begin{center}
\textbf{\large ४\; करणीषड्विधम्~।}\\
\rule{0.2\linewidth}{0.8pt}
\end{center}

\begin{sloppypar}
अथ करणीषड्विधं व्याख्यायते~। अत्रेदमवगन्तव्यम्~। मूलराश्योर्वर्गद्वारा यत्षड्विधं तत्करणीषड्विधमिति~। अस्य षड्विधस्य वर्गत्वपुरस्कारेणैव प्रवृत्तेः~। अत एवास्मिन्षड्विधे~। मूलदराशावपि करणीत्वव्यवहारः~। करणीत्वपुरस्कारेण गणितप्रवृत्तावयं न स्यात्~। करणीषड्विधमिति सञ्ज्ञा तु करणाराशावेतस्य गणितस्यावश्यकत्वाद्द्रष्टव्या~। तत्र यस्य राशेर्मूलेऽपेक्षिते निरग्रं मूलं न सम्भवति स करणी~। न त्वमूलदराशिमात्रम्~। तथा सति द्वित्रिपञ्चषडादिषु सर्वदा करणीत्वव्यवहारः स्यात्~। अस्तु स इति चेन्न~। तथा सति तत्प्रयुक्तं कार्यं स्यात्~। यथा\textendash \,अष्टौ द्विसंयुता अष्टादशैव स्युरित्यादि~। नन्वस्तु परिभाषामात्रमिदं तथापि किमनेन करणीषड्विधनिरूपणश्रमेण~। न ह्यस्ति लोके कर-णीभिर्व्यवहारः~। किं तु तदासन्नमूलैरेव~। तत्षड्विधं च रूपषड्विधेनैव गतार्थम्~। किं च कृतेऽपि करणीगणितेऽन्ततस्तदासन्नमूलेनैव न्यवहारस्तद्वरं प्रागेव तदादर इति चेन्मैवम्~। प्रागेव स्थूलमूलग्रहणे तद्गुणनादावतिस्थूलता स्यात्~। कृते तु सूक्ष्मे करणी-गणिते पश्चात्तदासन्नमूलग्रहणे किञ्चिदेवान्तरं स्यान्न महदित्यस्ति महान् विशेष इति करणीषड्विधमवश्यमारम्भणीयम्~। तद्यद्यपि व्यक्तषड्विधान्तरङ्गत्वाद्वर्णषड्विधात्~प्रागेवा-रब्धुं युक्तं तथाप्येतस्य निरूपणावगमयोः प्रयासगौरवात् सूचीकटाहन्यायेन वर्णषड्विधा-नन्तरमप्यारम्भो युक्त एव~।\\

{\small तत्र तावदिन्द्रवज्रोपजातिकाभ्यां प्रकारद्वयेन करणीसङ्कलनव्यवकलने गुणनभजनयोर्विशेषं च प्रतिपादयति\textendash }

\phantomsection \label{4.34}
\begin{quote}
{\large \textbf{{\color{purple}योगं करण्योर्महतीं प्रकल्प्य घातस्य मूलं द्विगुणं लघुं च~।\\
योगान्तरे रूपवदेतयोस्ते वर्गेण वर्गं गुणयेद्भजेच्च~॥\\
लघ्व्या हृतायास्तु पदं महत्या सैकं निरेकं स्वहतं लघुघ्नम्~।\\
योगान्तरे~स्तः~क्रमशस्तयोर्वा~पृथक्स्थितिः~स्याद्यदि~नास्ति~मूलम्~॥~३४~॥}}}
\end{quote}

करण्योर्योगेऽन्तरे वा कर्तव्ये रूपवत्कृतो यः करणीयोगः सा महती करणीति कल्पयेत्~। करण्योर्घातस्य मूलं द्विगुणं सा लघुः करणीति कल्पयेत्~। तयोर्लघुमहत्योः कल्पितकरण्यो रूपवत्कृते ये योगान्तरे ते प्रथमकरण्योर्योगान्तरे स्तः~। अथ \hyperref[3.27]{'अव्यक्तवर्गकरणीगुणनासु चिन्त्यः'} इत्यादिना {\color{violet}'भाज्याद्धरः शुध्यति'} इत्यादिना च करणीगुणनभजनयोः सिद्धावपि तत्र विशेषमाह\textendash \,\hyperref[4.34]{\textbf{'वर्गेण वर्गं गुणयेद्भजेच्च'}} इति~।
\end{sloppypar}

\newpage

\begin{sloppypar}
\noindent एतदुक्तं भवति~। करणीगुणने कर्तव्ये यदि रूपाणां गुण्यत्वं गुणकत्वं वा स्यात्करणीभजने वा कर्तव्ये यदि रूपाणां भाज्यत्वं भाजकत्वं वा स्यात्तदा रूपाणां वर्गं कृत्वा गुणनभजने कार्ये~। करण्या वर्गरूपत्वादिति~। वर्गस्यापि समद्विघाततया गुणनविशेषत्वादुक्तत्वत् सिद्धिः~। {\color{violet}'स्थाप्योऽन्त्यवर्गो द्विगुणान्त्यनिघ्नाः'} इत्यादिना व्यक्तोक्तप्रकारेण वा करणीवर्गस्यापि सिद्धिः स्यात्किं तु \hyperref[4.34]{\textbf{'वर्गेण वर्गं गुणयेद्भजेच्च'}} इत्युक्तत्वाद्द्विगुणान्त्यनिघ्ना इत्यत्र चतुर्गुणान्त्यनिघ्ना इति द्रष्टव्यम्~। मूलज्ञानार्थं तु सूत्रं वक्ष्यति~। अथ प्रकारान्तरेण योगान्तरे लघ्व्या इत्यादिना निरूपयति\textendash \,लघ्व्या करण्या हृताया महत्याः करण्या यत्पदं तदेकत्र \hyperref[4.34]{\textbf{सैकम्}} अपरत्र \hyperref[4.34]{\textbf{निरेकम्}} उभयमपि वर्गितं लघुकरणीगुणितं च क्रमेण करण्योर्योगान्तरे स्तः~। अत्र लघ्व्या महत्या भागे यदि भिन्नता स्यात्तदा मूलालाभे मूलार्थं यथासम्भवमपवर्तो द्रष्टव्यः~। अनया युक्त्या महत्या हृताया लघ्व्याः पदेन रूपं युतोनं वर्गितं च महतीघ्नं योगान्तरे स्त इति ज्ञेयम्~। अत्र करण्योर्मध्ये याङ्कतो लघुः सा लघुर्याङ्कतो महती सा महतीति ज्ञेयम्~। ननु पूर्वसूत्रोक्ता करण्योर्योगो महती घातस्य मूलं द्विगुणं लघुरिति~। अत्र लघ्व्या अमहत्येति व्याख्येयम्~। अथ च महत्या लघ्व्या इति व्याख्येयम्~। अन्यथा करण्योः साम्येऽनेन सूत्रेण योगान्तरसिद्धिर्न स्यादिति~। अत्र द्वयोर्मध्य एकया भक्तायाः परकरण्याः पदस्य रूपेण योगान्तरयोर्वर्गौ भाजककरणीघ्नौ योगान्तरे स्त इति वक्तुं साधीयः~। ननु पूर्वसूत्रे घातस्य मूलमित्यत्र पदग्रहणमुक्तम्~। द्वितीयसूत्रेऽपि लघ्व्या हृताया महत्याः पदमित्यत्र तदुक्तम्~। तत्र यदि पदं न लभ्यते तर्हि योगान्तरे कथं कर्तव्ये इत्यत आह\textendash \,\hyperref[4.34]{\textbf{'पृथक्स्थितिः स्याद्यदि नास्ति मूलम्'}} इति~। स्पष्टोऽर्थः~॥~६४~॥ \\

{\small अत्रोदाहरणान्युपजातिकयाह\textendash }

\phantomsection \label{4.35}
\begin{quote}
{\large \textbf{{\color{purple}द्विकाष्टमित्योस्त्रिभसङ्ख्ययोश्च योगान्तरे ब्रूहि सखे करण्योः~।\\
त्रिसप्तमित्योश्च चिरं विचिन्त्य चेत्षड्विधं वेत्सि सखे करण्याः~॥~३५~॥}}}
\end{quote}

स्पष्टोऽर्थः~। प्रथमोदाहरणे न्यासः\textendash \,क २ क ८~। अनयोर्योगो महती १०~। करण्योर्घातस्य १६ मूलम् ४~। द्विगुणं ८ लघुः~। क्रमेण लघुमहत्योर्न्यासः\textendash \,ल.\,क.\,८ म.\,क.\,१०~। अनयोः योगान्तरे रूपवत्कृते १८~। २ \hyperref[4.35]{\textbf{द्विकाष्टमित्योः}} करण्योर्योगोऽष्टादश १८~। अन्तरं द्वयम् २~। यो हि द्विकाष्टकयोर्मूलयोगः स एवाष्टादशानां मूलम्~। यत्तु द्विकाष्टकयोर्मूलान्तरं तदेव द्विकमूलमित्यर्थः~। अथात्र द्वितीयसूत्रेण योगान्तरे लघ्व्या २ हृताया महत्याः ८ लब्धम् ४~। अस्य पदं सैकं निरेकं च ३~। १~। द्वयोरपि वर्गो ९~। १~। लघु\textendash \,२\textendash \,घ्नौ च १८~। २ क्रमेण जाते ते एव योगान्तरे~। अथ द्वितीयोदाहरणे
\end{sloppypar}

\newpage

\begin{sloppypar}
\noindent न्यासः\textendash \,क ३ क २७~। अनयोर्योगो महती क ३०~। घातस्य ८१ मूलं ९ द्विगुणं १८ लघुः~। अनयोर्योगान्तरे ४८।१२~। अथ द्वितीयप्रकारेण~। लघ्व्या हृताया महत्या लब्धम् ९~। अस्य पदं ३ सैकं निरेकं च ४।२ स्वहतं १६।४ लघु\textendash \,३\textendash \,गुणं ४८।१२ जाते ते एव योगान्तरे~। अथ तृतीयोदाहरणे न्यासः\textendash \,क ३ क ७~। अनयोर्योगो महती १०~। करण्योर्घातः २१~। अस्य मूलाभावात् \hyperref[4.34]{'पृथक्स्थितिः स्याद्यदि नास्ति मूलम्'} इति जाता पृथक्स्थितिः~। योगे क ३ क ७~। अन्तरे क ३ क ७~। अत्रोपपत्तिः\textendash \,करण्योर्मूलयोगो यस्य मूलं स किल करणीयोगः~। स तु मूलयोर्युतिवर्ग एव~। कथमन्यथा तस्य मूलं मूलयुतिः~। एवं करण्योर्मूलान्तरं यस्य मूलं तत्किल  करण्यन्तरम्~। तत्तु मूलान्तरवर्ग एव~। कथमन्यथा तस्य मूलं मूलान्तरं स्यात्~। तत्र करण्यौ हि मूलवर्गौ~। अतः करण्योर्मूले गृहीत्वा तयोर्युतिवर्गः कर्तव्यः~। स एव करणीयोगः स्यात्~। एवं करणीमूलान्तरवर्गः करण्यन्तरं स्यात्~। परं करण्या मूलं न लभ्यते~। अतोऽन्यथा यतितव्यम्~। अत्र किल युतिवर्गोऽन्तरवर्गो वा साध्यः~। स तु वर्गयोगोपलम्भे सुबोधः~। वर्गयोगस्तु करणीयोग एव~। करण्योर्वर्गरूपत्वात्~।\\ 

ननु वर्गयोगावगमेऽपि कथं युतिवर्गोऽन्तरवर्गो वा सुबोधस्तयोर्वैलक्षण्यादिति चेदुच्यते~। वर्गयोगो द्विगुणितघातेन युक्तो युतिवर्गो भवति~। यथा राशी ३।५~। अनयोर्वर्गयोगः ३४~। द्विगुणितघातेनानेन ३० युतो ६४ जातो युतिवर्गः~। वा राशी ३।७~। अनयोर्वर्गयुतिः ५८~। द्विघ्नघातेन ४२ युतो १०० जातो युति\textendash \,१०\textendash \,वर्गः~। एवं सर्वत्र~। तथा वर्गयोगो द्विघ्नघातेन हीनोऽन्तरवर्गो भवति~। यथा राशी ४।२~। अनयोर्वर्गयोगः २०~। द्विघ्नघातेन १६ हीनो जातो\textendash \,४\textendash \,ऽन्तरं २ वर्गः ४~। वा राशी ३।८~। अनयोर्वर्गयोगः ७३~। द्विघ्नघातेन ४८ हीनो जातो\textendash \,२५\textendash \,ऽन्तरं ५ वर्गः २५~। एवं सर्वत्र~। तस्माद्वर्गयोगो द्विघ्नघातयुतो युतिवर्गो भवति द्विघ्नघातेन हीनोऽन्तरवर्गो भवतीति सिद्धम्~। अत्र मूलयोर्वर्गयोगः करणीयोग एव~। असौ करणीद्वयमूलघातेन द्विघ्नेन योज्यो युतिवर्गार्थं वियोज्यश्चान्तरवर्गार्थम्~। तत्र यः करणीमूलयोर्घातः स एव करणीघातमूलम्~। अतः सुष्ठूक्तं \hyperref[4.34]{'योगं करण्योर्महतीं प्रकल्प्य घातस्य मूलं द्विगुणं लघुं च~। योगान्तरे रूपवदेतयोस्ते'} इति~।\\

ननूपपत्त्या विना वर्गयोगो द्विघ्नघातेन युतो हीनो वा~। युतिवर्गोऽन्तरवर्गो वा भवतीत्येतदेव कथं[चितं] क्वचिद्दर्शनं त्वप्रयोजकम्~। अन्यथा चतुर्गुणो राशिघातो युतिवर्गो भवतीत्यपि सुवचम्~। तस्यापि क्वचित्तथा दर्शनात्~। तथाहि\textendash \,राशी २।२~। अनयोर्घातः ४ चतुर्गुणः १६~। अयं जातो युति\textendash \,४\textendash \,वर्गः १६~। वा राशी ३।३~। अनयोर्घातश्चतुर्गुणः ३६~। अयमेव युति\textendash \,६\textendash \,वर्गश्च ३६~। वा राशी ४।४~। अनयोर्घातः १६~। चतुर्गुणः ६४~। अयमेव युति\textendash \,८\textendash \,वर्गः ६४ इत्यादिषु~। तस्मात्क्वचिद्दर्शनमप्रयोजकं क्वचिद्व्यभि-
\end{sloppypar}

\newpage

\begin{sloppypar}
\noindent चारस्यापि सम्भवात्~। अतो वर्गयोगो द्विघ्नघातयुतोनो युतिवर्गोऽन्तरवर्गश्च भवतीत्यत्र युक्तिर्वक्तव्येति चेत्सत्यम्~। इयमुपपत्तिरेकवर्णमध्यमाहरणान्ते {\color{violet}'वर्गयोगस्य यद्राश्योर्युतिवर्गस्य चान्तरम्~। द्विघ्नघातसमानं स्यात्'} इत्यत्र~। तथा {\color{violet}'राश्योरन्तरवर्गेण द्विघ्नो घातः समन्वितः~। वर्गयोगसमः स स्यात्'} इत्यत्राप्याकर एव स्फुटी भविष्यति~। विवरिष्यते चास्माभिस्तत्रैवेति नेह निरूप्यते~। अथ वर्गयोर्य एव मूलघातः स एव घातमूलमित्यत्र युक्तिरुच्यते~। वर्गघातो हि चतुर्घातः~। वर्गस्य समद्विघातरूपत्वात्~। एवमेकस्य समराशिद्वयस्येतरस्य च समराशिद्वयस्य घातः इति चतुर्घातोः वर्गघातः~। यथा राशी ३।५~। अनयोर्वर्गघातार्थं घातवर्गार्थं वा राशिचतुष्टयेन भाव्यम्~। ३।३।५।५~। अत्र घातद्वयमेवं ९।२५ एवं वा १५।१५~। राश्योर्घातौ राशिवर्गौ वा~। अत्र वर्गयो\textendash \,९।२५\textendash \,र्घाते २२५ घातयोर्वा १५।१५ समयोर्घाते २२५ पूर्वचतुष्कस्य घातोऽस्ति~। अतो वर्गघातस्य घातवर्गस्य चाभेदाद्यदेव घातवर्गस्य मूलं तदेव वर्गघातस्यापि~। तत्र घातवर्गस्य मूलं घात एव भवेदिति वर्गघातस्यापि मूलं घात एव~। अत उपपन्नं य एव मूलघातः स एव घातमूलमिति~।\\

अथ द्वितीयसूत्रोपपत्तिः~। अत्रापि करण्योर्मूलयुतिवर्गो मूलान्तरवर्गो वा साध्योऽस्ति~। करण्योस्तु मूलं न लभ्यते~। अतः करणीद्वयं तथापवर्तनीयं यथा मूलं लभ्येत~। परं तथा मूललाभेऽपि तयोर्युतिवर्गोऽन्तरवर्गो वा करण्यपवर्तेनापवर्तितः स्यात्~। यतोऽप-वर्तितकरण्या मूलमपवर्ताङ्कमूलेनापवर्तितं स्यात्~। एवं द्वितीयकरण्या अपि~। तयोर्मूल-योर्युतिरप्यपवर्तमूलेनैवापवर्तिता स्यात्~। युतेर्वर्गस्त्वपवर्तमूलवर्गेणापवर्तितः स्यात्~। अप-वर्तमूलवर्गस्त्वपवर्त एव~। अतो युतिवर्गोऽन्तरवर्गो वापवर्ताङ्केन गुणनीय इति युक्तिरस्ति~। अथापवर्तो विचारणीयः~। करण्याः केनापवर्तेन मूललाभः स्यादिति~। तत्र करण्यङ्केनैव करण्या अपवर्ते रूपमेकं स्यात्तस्य चावश्यं मूललाभः~। तत्र यदि महत्याः करण्या अपवर्तः क्रियते तदा लघ्व्या करण्याः अपवर्तो न स्यात्~। अत आचार्येण, लघ्व्याः करण्या अपवर्तः कृतः~। तथा सति जातं लघुस्थाने रूपम्~। महत्यपि लघ्व्यापवर्त्य ततो मूलं च ग्राह्यमत उक्तं \hyperref[4.34]{'लघ्व्या हृतायास्तु पदं महत्याः'} इति~। इदमपवर्तितमहत्याः पदम्~। अपवर्तितलघ्व्यास्तु रूपमेव पदम्~। अनयोर्युतावन्तरे वा कर्तव्ये महतीपदं सैकं निरेकं वा भवेत्~। द्वितीयपदस्य रूपत्वात्~। अत उक्तम् \hyperref[4.34]{'सैकं निरेकम्'} इति~। एवं जाता मूलयुतिर्मूलान्तरं च~। अथानयोर्वर्गो विधेयः~। अत उक्तम् \hyperref[4.34]{'स्वहतम्'} इति~। एवं जातो युतिवर्गोऽन्तरवर्गश्च~। परमपवर्तित एव~। अतोऽपवर्तनेन लघुकरण्या द्वयमेतद्गुणनीयम्~। अत उक्तम् \hyperref[4.34]{'लघुघ्नम्'} इति~। इदमुपलक्षणम्~। येनापवर्ते करण्योमूले लभ्येते तेनापवर्त्य करण्योर्मूले ग्राह्ये~। तयोर्युतिवर्गोऽन्तरवर्गो वापवर्ताङ्केन गुणितः सन् भवेदेव करण्योर्योगा-
\end{sloppypar}

\newpage

\begin{sloppypar}
\noindent न्तरं चेत्यादि सुधीभिरुह्यम्~। अथ \hyperref[4.34]{'वर्गेण वर्गं गुणयेत्'} इत्यत्रोपपतिः~। इह हि करणीषड्विधेन तन्मूलयोरेव षड्विधं साध्यते~। यथा द्विकाष्टमित्योः करण्योर्योगस्य दशत्वे सत्यपि मूलयोगार्थं तस्याष्टादशत्वमेव निरूपितमित्यादि~। तद्वदिहापि करण्या द्व्यादिगुणत्वं तथा सम्पादनीयं यथा तत्पदं द्व्यादिगुणं भवति~। तत्र द्व्यादिभिरेव करणीगुणने तत्पदं द्व्यादिगुणं न भवति किन्तु द्व्यादिवर्गेण तद्गुणने~। यथा राशिः ४~। एतस्य द्विगुणत्वेऽभीप्सिते चेदस्य वर्गो १६ द्विगुणः ३२ क्रियते तर्ह्यस्य पदं द्विगुणो राशिर्न भवति~। राशि\textendash \,४\textendash \,वर्गे १६ द्विवर्गेण ४ गुणिते तु ६४ तत्पदं ८ भवति द्विगुणो राशिः~। एवं त्र्यादिगुणत्वेऽपि द्रष्टव्यम्~। अत उपपन्नम् \hyperref[4.34]{'वर्गेण वर्गं गुणयेत्'} इति~। एवं भजनेऽप्युपपत्तिर्द्रष्टव्या~। अस्ति चाचार्येण {\color{violet}पाट्याम्} उक्तं {\color{violet}'वर्गे कृती घनविधौ तु घनौ विधेयौ हारांशयोरपि पदे च पदप्रसिद्ध्यै'} इति~। उपपादितं चास्माभिस्तद्व्याख्यावसरे~॥३५~॥\\

{\small अथ गुणन उदाहरणद्वयमुपजातिकयाह\textendash }

\phantomsection \label{4.36}
\begin{quote}
{\large \textbf{{\color{purple}द्वित्र्यष्टसङ्ख्यागुणकः करण्योर्गुण्यस्त्रिसङ्ख्या च सपञ्चरूपा~।\\
वधं प्रचक्ष्वाशु विपञ्चरूपे गुणेऽथवा त्र्यर्कमिते करण्यौ~॥~३६~॥}}}
\end{quote}

अत्र पञ्चरूपसहिता \hyperref[4.36]{\textbf{त्रिसङ्ख्या}} करणीगुण्या~। \hyperref[4.36]{\textbf{गुणकः}} तु \hyperref[4.36]{\textbf{द्वित्र्यष्टसङ्ख्याः}} करण्याः पञ्च-रूपोने त्र्यर्कमिते करण्यौ वा~। अत्र गुणकद्वयादुदाहरणद्वयं ज्ञेयम्~। अथ प्रथमोदाहरणे न्यासो गुणकः क २ क ३ क ८~। गुण्यो रू ५ क ३ \hyperref[4.34]{'वर्गेण वर्गं गुणयेत्'} इति करण्या वर्ग-रूपत्वाद्रूपाणामपि वर्गे कृते जातो गुण्यः क २५ क ३~। यथा खण्डैः पृथग्गुणितः सहितश्च गुण्यो गुणनफलं भवति तथा खण्डयोगेनापि गुणितो भवत्येवेति प्रसिद्धम्~। अतो गुणके द्विकाष्टमित्योः करण्योर्योगे कृते जातो गुणकः क १८ क ३~। \hyperref[3.27.1]{'गुण्यः पृथग्गुणकखण्डसमो निवेश्यः'} इति गुणनार्थं न्यासः\textendash \;{\scriptsize $\begin{matrix}
\mbox{{क \;१८~। \;क \;२५ \;क \;३~।}}\\
\vspace{-1.5mm}
\mbox{{क \;~३~। \;क \;२५ \;क \;३~।}}
\vspace{1mm}
\end{matrix}$}\; गुणनेन जातं क ४५० क ५४ क ७५ क ९~। करणीनवकस्य मूलं लभ्यत इति मूले गृहीते जातं गुणनफलं रू ३ क ४५० क ५४ क ७५~। अथ द्वितीयोदाहरणे न्यासो गुणकः रू ५ं क ३ क १२~। गुण्यः क २५ क ३~। अत्र गुणके व्यर्कमितयोः करण्योर्योगे जातं क २७~॥~३६~॥ \\

{\small \hyperref[4.34]{'वर्गेण वर्गं गुणयेत्'} इति रूपवर्गे कर्तव्ये कृतिः स्वर्णयोः स्वमिति पञ्चविंशतिकरण्या धनत्वे प्राप्ते विशेषमुपजातिकयाह\textendash }

\phantomsection \label{4.37}
\begin{quote}
{\large \textbf{{\color{purple}क्षयो भवेच्च क्षयरूपवर्गश्चेत्साध्यतेऽसौ करणीत्वहेतोः~। \\
ऋणात्मिकायाश्च तथा करण्या मूलं क्षयो रूपविधानहेतोः~॥~३७~॥}}}
\end{quote}
\end{sloppypar}

\newpage

\begin{sloppypar}
क्षयरूपाणां वर्गस्तर्हि क्षयो भवेत्~। असौ क्षयरूपवर्गश्चेत्करणीत्वनिमित्तं साध्यते~। न मूलं क्षयस्यास्तीत्यस्यापवादमाह\textendash \,\hyperref[4.37]{\textbf{'ऋणात्मिकायाः'}} इति~। ऋणात्मिकायाः करण्या मूलं तर्हि क्षयो भवेच्चेन्मूलं रूपविधाननिमित्तं साध्यत इति~। अत्रोपपत्तिः~। अत्र किल रूपवर्गः करणीगुणनार्थं क्रियते~। स यद्यपि धनमेव तथापि तस्य मूलमृणमेव~। \hyperref[1.13]{'स्वमूले धनर्णे'} इत्युक्तत्वात्~। करणीयोगेन च मूलयुतिवर्गः साध्यते~। तत्र क्षयरूपवर्गकरण्या यदि धनत्वं कल्प्यते तदान्यया धनकरण्या सह योगः स्यात्~। तस्य च मूलं मूलयुतिरेव~। भाव्यं च मूलान्तरेण~। \hyperref[1.3]{'धनर्णयोरन्तरमेव योगः'} इत्युक्तत्वात्~। तस्मात् करण्या ऋणसञ्ज्ञा मूलस्यर्णत्वबोधार्थमेव \;कृता~। \;बालावबोधार्थमिदमुदाह्रियते\textendash \,रू \,३ रू \,७ं~। \;अनयोः युति\textendash \,४\textendash \,वर्ग\textendash \,१६\textendash \,स्तावदयम्~। स च करण्या धनत्वे कल्पिते सति न सिध्यति~। यथा\textendash \,उदाहृतरूपयोः करण्योः क ९ क ४९~। \hyperref[4.34]{'योगं करण्योर्महतीं प्रकल्प्य'} इत्यादिना जातो योगः क १००~। न ह्ययं युतिवर्गस्तस्मादृणत्वं कल्प्यते~। तस्माद्यदि करणीयोगादिकं न साध्यते तदा क्षयरूपवगों धनमेवं~। अत्र करणीत्युपलक्षणम्~। यत्र वर्गयोगात्करणीयोगवद्युतिवर्गा-दिकं साध्यते तत्र क्षयरूपवर्गः क्षय एव कल्पनीय इति ध्येयम्~। अलमतिविस्तरेण~। प्रकृतमनुसरामः~। गुणकः रू ५ं क ३ क १२~। करणीयोगः क २७~। रूपवर्गः क्षयः क २५~। एवं जातो गुणकः क २५ं क २७~। गुण्यः क २५ क ३~। गुणनार्थं न्यासः\textendash \;{\scriptsize $\begin{matrix}
\mbox{{क \;२५ं~। \;क \;२५ \;क \;३~।}}\\
\vspace{-1.5mm}
\mbox{{क \;२७~। \;क \;२५ \;क \;३~।}}
\vspace{1mm}
\end{matrix}$}\; गुणनाज्जातं क ६२ं५ क ७५ं क ६७५ क ८१~। प्रथमचतुर्थ्योः करण्योर्मूले रू २५ं रू ९~। अनयोर्योगो रू १६~। इतरकरण्योरन्तरं क ३००~। एवं जातं गुणनफलं रू १६ं क ३००~। अथ भजनोदाहरणम्~। पूर्वगुणनफलस्य स्वगुणच्छेदस्य न्यासः\textendash \;{\scriptsize $\begin{matrix}
\mbox{{क \;९ \;क \;४५० \;क \;७५ \;क \;५४~।}}\\
\vspace{-1.5mm}
\mbox{{क \;२ \;क \;३ \;क \;८~~~~~~~~~~~~~~~~~~~~~~}}
\vspace{1mm}
\end{matrix}$}\; भाजके द्विकाष्टमित्योः करण्योर्योगे जातो भाजकः क ३ क १८~। अथ \hyperref[3.29]{'भाज्याच्छेदः शुध्यति'} इत्यादिना लब्धिर्ग्राह्या~। अत्र भाज्ये प्रथमतः करणीनवकमस्ति~। भाजके त्रिगुणिते तच्छुध्येदिति भाजकस्त्रिभिर्गुणितः क ९ क ५४~। अस्य शोधनेन प्रथमचतुर्थ्योर्भाज्यकरण्योः शुद्धिः~। अतो लब्धिः क ३~। अथ भाज्यशेषं क ४५० क ७५ पुनर्भाजके पञ्चविंशतिगुणे क ७५ क ४५० भाज्यशेषाद्यथासम्भवमपनीते शुद्धिरस्तीति जाता लब्धिः क २५~। एतस्या मूलं लभ्यत इति गृहीतं मूलं रू ५~। एवं जाता लब्धिः रू ५ क ३~। अथ द्वितीयोदाहरणे
\end{sloppypar}

\newpage

\begin{sloppypar}
\noindent भाज्यः क २५ं६ क ३००~। भाजकः क २५ं क ३ क १२~। करण्योर्योगे जातो भाजकः क २५ं क २७~। अत्र पूर्वगुण्येनानेन क ३ क २५ लब्ध्या भाव्यम्~। अतस्त्रिगुणो भाजकः \hyperref[1.7]{'संशोध्यमानं स्वमृणत्वमेति'} इति ज्ञातः क ७५ क ८१ं~। अत्र भाज्यभाजकगतयोर्धनर्णकरण्योरन्तरं न भवति मूलाभावात्~। अतो भाज्यभाजकधनकरण्योः क ३०० क ७५ ऋणकरण्योश्च क २५ं६ क ८१ं योगे जातं भाज्यशेषं क ६७५ क ६२ं५~। अस्मात् पञ्चविंशतिगुणे भाजके क ६२ं५ क ६७५ अपनीते शुद्धिरस्तीति जाता लब्धिः क ३ क २५~। मूले गृहीते जाता लब्धिः त्रिसङ्ख्या च सपञ्चरूपेति रू ५ क ३~॥~३७~॥\\

{\small अत्र द्वितीयोदाहरणे भाजकः कियद्गुणो \hyperref[3.29]{'भाज्याच्छेदः शुध्यति'} इति दुःखबोधमतः परम-कारुणिकैराचार्यैः शिष्यबोधार्थमुपायान्तरमुपजातिकाद्वयेन निरूप्यते\textendash }

\phantomsection \label{4.38}
\begin{quote}
{\large \textbf{{\color{purple}धनर्णताव्यत्ययमीप्सितायाश्छेदे करण्या असकृद्विधाय~।\\
तादृक्छिदा भाज्यहरौ निहन्यादेकैव यावत्करणी हरे स्यात्~॥\\
भाज्यास्तया भाज्यगताः करण्यो लब्धाः करण्यो यदि योगजाः स्युः~।\\
विश्लेषसूत्रेण पृथक्च कार्या यथा तथा प्रष्टुरभीप्सिताः स्युः~॥३८~॥ }}}
\end{quote}

\hyperref[4.38]{\textbf{छेदे ईप्सिताया}} एकस्याः \hyperref[4.38]{\textbf{करण्या}} धनर्णताविपर्यासं कृत्वा तादृशेन च्छेदेन यथास्थितौ \hyperref[4.38]{\textbf{भाज्यहरौ}} गुणयेत्~। एवं कृते करणीनां यथोक्त्या योगे च कृते भाज्यभाजकौ स्तः~। अथास्मिन्नपि भाजके यदि द्व्यादीनि करणीखण्डानि स्युस्तदात्रापि एकस्याः करण्या धनर्णताविपर्यासं कृत्वा तादृशभाजकेन पूर्वगुणनसम्पन्नौ भाज्यभाजकौ गुणयेत्~। तत्रापि यथासम्भवं करणीयोगे कृते तौ भाज्यभाजकौ स्तः~। एवमसकृत्तावद्विधेयं यावद्भाजके एकैव करणी भवेत्~। अथ सम्पन्नया भाजककरण्या सम्पन्नभाज्यकरण्यो रूपवदेव भाज्या~। यल्लभ्यते ता लब्धिकरण्यो भवन्ति~। अथ यदि लब्धाः करण्यो योगजाः स्युर्न पुनः प्रष्टुरभीप्सितास्तदा वक्ष्यमाणविश्लेषसूत्रेण तथा पृथक्कार्या यथा प्रष्टुरभीप्सिताः स्युः~। द्वितीयोदाहरणे भाज्यः क २५ं६ क ३०० भाजकः क २ं५ क २७~। अत्र पञ्चविंशतिकरण्या ऋणत्वव्यत्यासं कृत्वा जातो हरः क २५ क २७ अनेन हरेण यथास्थितौ भाज्यहारौ गुणयेदिति गुणनार्थं न्यासः
\vspace{-3mm}

\begin{center}
क २५~। क २५ं६ क ३००~। ~~~क २५~। क २५ं~। क २७~। \\
क २७~। क २५ं६ क ३००~। ~~~क २७~। क २५ं~। क २७~। 
\end{center}
\vspace{-1mm}

\noindent भाज्ये गुणिते जातं क ६४ं०० क ७५०० क ६९ं१२ क ८१०० प्रथमचतुर्थ्योद्वितीयतृतीययोश्च योगे जातं भाज्ये करणद्वयं क १०० क १२~। भाजके गुणिते जातं क ६२ं५ क ६७५ क ६७ं५ क ७२९~। अत्रापि प्रथमचतुर्थ्योद्वितीयतृतीययोश्च
\end{sloppypar}

\newpage

\begin{sloppypar}
\noindent योगे जातं क ४ क ०~। एवं हरे जाता करण्येकैव क ४~। अनया भाज्यकरण्यौ क १०० क १२ भक्ते लब्धिः क २५ क ३~। एवं पूर्वोदाहरणेऽपि न्यासः~। भाज्यः क ९ क ४५० क ७५ क ५४~। भाजकः क १८ क ३~। अत्र च्छेदे त्रिमितकरण्या ऋणत्वं प्रकल्प्य तादृशच्छेदेनानेन क १८ क ३ं भाज्यभाजकयोर्गुणनार्थे न्यासः 
\vspace{-2mm}

\begin{center}
क ९ क ४५० क ७५ क ५४~। क १८~। ~~~क १८ क ३~। क १८~। \\
क ९ क ४५० क ७५ क ५४~। क ~३ं~। ~~~~क १८ क ३~। क ~३ं~। 
\end{center}
\vspace{-2mm}

\noindent भाज्ये गुणिते जातं क १६२ क ८१०० क १३५० क ९७२ क २७ं क १३ं५० क २२ं५ क १६ं२~। अत्र तुल्ययोर्धनर्णकरण्योर्योगेन शुद्धौ सत्यां शेषं करणीचतुष्टयं क ८१०० क २२ं५ क ९७२ क २७ं~। अत्र प्रथमद्वितीययोस्तृतीयचतुर्थ्योश्च योगे जातं भाज्ये करणीद्वयं क ५६२५ क ६७५~। एवं भाजके गुणिते जातं क ३२४ क ५४ क ५४ं क ९ं~। अत्रानयोः क ५४ं क ५४ योगे जाता शुद्धिः~। इतरयोः ३२४~। ९ं योगे जाता करणी २२५~। एवं हरकरण्येकैव जाता क २२५~। अनया भाजककरण्या हृते लब्धिः क २५ क ३~। एवं लब्धा करणी यदि योगजा स्यात्तदा विश्लेषसूत्रेण पृथक्कार्या~। तत्रोदाहरणम्~। भाज्यः क ९ क ४५० क ७५ क ५४ भाजकः क २५ क ३~। अत्र भाजके त्रिमितकरण्या ऋणत्वं प्रकल्प्य तादृशहरेण भाज्यहरयोर्गुणनार्थं न्यासः
\vspace{-2mm}

\begin{center}
क २५~। क ९ क ४५० क ७५ क ५४~। ~~~क २५~। क २५ क ३~। \\
क ~\;३~। क ९ क ४५० क ७५ क ५४~। ~~~~क ~३ं~। क २२ क ~३~। 
\end{center}
\vspace{-2mm}

\noindent भाज्ये गुणिते क २२५ क ११२५० क १८७५ क १३५० क २७ं क १३ं५० क २२ं५ क १६ं२~। अत्र धनर्णकरणीनां साम्यान्नाशे शेषकरण्यः क २७ं~। १८७५ क ११२५० क १६ं२~। आस्वनयोः क २७ं क १८७५ अनयोश्च क ११२५० क १६ं२ योगे जातं भाज्ये करणीद्वयं क १४५२ क ८७१२~। एवं हरे गुणिते जातं क ६२५ क ७५ क ७५ं क ९ं~। अत्रापि तुल्ययोर्धनर्णकरण्योर्नाशे परयोः क ६२५ क ९ योगे जातैकैव मानककरणी क ४८४~। अनया भाज्यकरण्योर्भजने जाता लब्धिः क ३ क १८~। अत्र किल द्वित्र्यष्टसङ्ख्यागुणकः करण्योर्गुण्यस्त्रिसङ्ख्या च सपञ्चरूपा~। अनयोर्वधो भाज्यत्वेनोदाहृतः~। तयोरेकतरेणास्य भजनेऽन्यतरो लब्धिः स्यात्~। प्रकृते तु सपञ्चरूपया त्रिसङ्ख्यया ह्रियतेऽतो द्वित्र्यष्टकरणीभिः फलेन भाव्यम्~। उक्तरीत्या त्वियं लब्धिः क १८ क३~। एतन्मध्ये इयं क ३ अभीष्टा~। इतरत्करणीद्वयमपेक्षितम्~। अत इयं योगकरणी क १८ पृथक्कार्या~॥~३८~॥
\end{sloppypar}

\newpage

\begin{sloppypar}

{\small अतः पृथक्करणं वसन्ततिलकया निरूपयति\textendash }

\phantomsection \label{4.39}
\begin{quote}
{\large \textbf{{\color{purple}{\color{white}अ} \hspace{-10mm} वर्गेण~योगकरणी~विहृता~विशुध्येत्~खण्डानि~तत्कृतिपदस्य~यथेप्सितानि~।\\
{\color{white}अ} \hspace{-9mm} कृत्वा~तदीयकृतयः~खलु~पूर्वलब्ध्या~क्षुण्णा~भवन्ति~पृथगेवमिमाः~करण्यः~॥~३९~॥}}}
\end{quote}

\hyperref[4.39]{\textbf{योगकरणी}} येन \hyperref[4.39]{\textbf{वर्गेण विहृता}} सती \hyperref[4.39]{\textbf{विशुध्येत्तत्कृतिपदस्य यथेप्सितानि खण्डानि कृत्वा तदीयकृतयः पूर्वलब्ध्या क्षुण्णाः पृथक्करण्यो भवन्ति~।}} सा चासौ कृतिश्चेति कर्मधारयो द्रष्टव्यः~। एतदुक्तं भवति~। योगकरणी येन वर्गेण विहृता सती निःशेषा भवेत्तस्य वर्गस्य मूलं ग्राह्यम्~। तस्य खण्डानि प्रष्टुर्यावन्त्यभीष्टानि तावन्ति कृत्वा तेषां खण्डानां वर्गाः कर्तव्याः~। ते वर्गाः पूर्वलब्ध्या क्षुण्णाः~। वर्गेण विहृतायां योगकरण्यां या लब्धिः सा पूर्वलब्धिः~। तया गुणितास्ते वर्गाः पृथक्करण्यो भवन्ति~। प्रकृतोदाहरणे योगकरणी क १८ इयमनेन वर्गेण ९ विहृता सती शुध्यति~। लब्धिश्च २~। वर्गस्य ९ पदम् ३~। अस्य खण्डे १~। २~। अनयोर्वगौ १~। ४~। पूर्वलब्ध्या २ गुणितौ २~। ८~। जाते करणीखण्डे क २ क ८~। एवं पूर्वकरण्या क ३ सह जाता द्वित्र्यष्टसङ्ख्या लब्धिकरण्यः~। एवं प्रष्टुर्यदि खण्डत्रयमभीष्टं स्यात्तर्हि वर्गपदस्यास्य ३ खण्डत्रयम् १~। १~। १~। एभ्यः पूर्ववज्जातानि करणीखण्डानि २~। २~। २~। एतासामपि करणीनां योगे करणी सैव भवति क १८~। एवं प्रष्टुरिच्छावशादन्यान्यपि खण्डानि कार्याणि~। एवमन्यत्रापि द्रष्टव्यम्~।\\

अथ \hyperref[4.38]{'धनर्णताव्यत्ययमीप्सितायाः'} इत्यत्र युक्तिः~। तुल्येनाङ्केनापवर्तितयोर्गुणितयोर्वा भाज्यभाजकयोः फले वैषम्याभाव इति तावत्प्रसिद्धम्~। तत्र हरकरणी यथैका भवति तथा भाज्यभाजकौ गुणनीयावपवर्त्यौ वा~। तथा सति भजनं सुगमं स्यात्~। तत्रापवर्ते विचारगौरवमस्ति~। यथा भाजककरण्योः केनापवर्ते कृत एकैव करणी स्यादिति विचारणीयम्~। पुनस्तेनाङ्केन भाज्यकरणीनामपर्वतः सम्भवति, न वेति विचारणीयमिति~। अतः केनचिद्भाज्यभाजकौ गुणनीयौ~। तत्र भाजकतुल्यो गुणकः कृतः~। तथा सति भाजकगुणने वर्गत्वात्खण्डवर्गौ खण्डाभिहतिद्वयं च स्यात्~। तत्र वर्गरूपयोः करणीखण्डयोः मूललाभादवश्यं तयोर्योगे एकैव करणी स्यात्~। परं खण्डाभिहतिद्वयमवशिष्टं स्यात्~। अत आचार्येणैकस्या गुणककरण्याः धनर्णताव्यत्यास उक्तः~। तथा सति खण्डवधयोर्मध्य एकस्य धनत्वमितरस्यर्णत्वमिति तयोर्योगे नाशः स्यात्~। एवं हरे त्वेकैव करणी स्यात्~। हरस्य गुणितत्वाद्भाज्यगुणनमावश्यकमित्युपपन्नं धनर्णताव्यत्ययमित्यादि~। एवं त्र्यादि-खण्डेष्वप्यूह्यम्~। तत्र खण्डबाहुल्याद्युगपत्तन्नाशो न भवतीत्यसकृदित्युक्तम्~। अथ विश्लेष-सूत्रोपपत्तिः~। सा च करणीयोगद्वितीयसूत्रव्यत्यासेन यथा करण्यौ करण्यो वा केनचित् अपवर्त्य तन्मूलयुतिवर्गोऽपवर्ताङ्केन गुणितः सन् योगकरणी भवति~।
\end{sloppypar}

\newpage

\begin{sloppypar}
\noindent तथा च या या योगकरणी सा सा युतिवर्गापवर्ताङ्कयोराहतिः~। अतः सा वर्गेण विहृता विशुध्येदेव~। लब्धिस्त्वपवर्ताङ्क एव स्यात्~। येन वर्गेण विहृता विशुध्येत्स युतिवर्ग एव~। तस्य पदं मूलयुतिः स्यात्~। युतेः खण्डान्यपवर्तितकरणीनां मूलानि स्युः~। तेषां वर्गा अपवर्तितकरण्यः स्युः~। एता अपवर्तगुणिता यथास्थितकरण्यः स्युः~। अपवर्ताङ्कस्तु पूर्वलब्धिरेव~। अतः सुष्ठूक्तम् {\color{violet}'वर्गेण योगकरणी विहृता विशुध्येत्'} इत्यादि~॥~३९~॥\\

{\small वर्गस्य गुणनसूत्रेणैवोक्तत्वात्तदुदाहरणानि सार्धोपजातिकयाह\textendash }

\phantomsection \label{4.40}
\begin{quote}
{\large \textbf{{\color{purple}द्विकत्रिपञ्चप्रमिताः करण्यस्तासां कृतिं द्वित्रिकसङ्ख्ययोश्च~।\\
षट्पञ्चकद्वित्रिकसंमितानां पृथक् पृथङ्मे कथयाशु विद्वन्~।\\
अष्टादशाष्टद्विकसंमितानां कृती कृतीनां च सखे पदानि~॥~४०~॥}}}
\end{quote}

स्पष्टोऽर्थः~। पूर्वोदाहरणे करण्यः क २ क ३ क ५~। वर्गस्य समद्विघातरूपत्वादयमेव गुण्यो गुणकश्चेति गुणनार्थं न्यासः
\vspace{-2mm}

\begin{center}
\begin{tabular}{ccc}
{क २ | क २ क ३ क ५~।} & {} & {क \;~४ क \;~६ क १०~।}\\
{क ३~। क २ क ३ क ५~।} & {गुणिते जातं} & {क \;~६ क \;~९ क १५~।}\\
{क ५~। क २ क ३ क ५~।} & {} & {क १० क १५ क २५~।}
\end{tabular}
\end{center}
\vspace{-2mm}

\noindent अत्रासां क ४ क ९ क २५ मूलानि २~। ३~। ५~। एषां योगः रू १०~। अन्यासां करणीनां मध्ये द्वयोर्द्वयोस्तुल्ययोर्योगे जाताश्चतुर्गुणाः करण्यः क २४ क ४० क ६०~। एवं जातो वर्गो रू १० क २४ क ४० क ६०~। अथवा {\color{violet}'स्थाप्योऽन्त्यवर्गो द्विगुणान्त्यनिघ्नाः'} इत्यादिना वर्गो विधेयः~। तत्र करणीवर्गे चतुर्गुणान्त्यनिघ्ना इति बोध्यम्~। \hyperref[4.34]{'वर्गेण वर्गं गुणयेत्'} इत्युक्तत्वात्~। न्यासः क २ क ३ क ५ {\color{violet}'स्थाप्योऽन्त्यवर्गः'} इत्यादिना जातानि वर्गखण्डानि क ४ क २४ क ४० क ९ क ६० क २५~। अत्र वर्गाणां मूलानि गृहीत्वा २~। ३~। ५ ऐक्यं च कृत्वा जातो वर्गो रू १० क २४ क ४० क ६०~। अथ द्वितीयोदाहरणे क २ क ३ {\color{violet}'स्थाप्योऽन्त्यवर्गः'} इत्यादिना क ४ क २४ क ९~। वर्गयोर्मूलैक्ये कृते जातो वर्गो रू ५ क २४~। अथ तृतीयोदाहरणे न्यासः क ६ क ५ क २ क ३~। उक्तवज्जातानि वर्गखण्डानि क ३६ क १२० क ४८ क ७२ क २५ क ४० क ६० क ४ क २४ क ९~। आसु वर्गरूपाभ्यः करणीभ्यो मूलानि गृहीत्वा योगं च कृत्वा जातो वर्गो रू १६ क १२० क ४८ क ७२ क ४० क ६० क २४~। अथ चतुर्थोदाहरणे न्यासः क १८ क ८ क २~। उक्तवज्जातानि वर्गखण्डानि क ३२४
\end{sloppypar}

\newpage

\begin{sloppypar}
\noindent क ५७६ क १४४ क ६४ क ६४ क ४~। सर्वेषां वर्गरूपत्वाज्जातानि मूलानि १८।२४।१२।८।८।२~। एषां योगे जातो वर्गो रू ७२~। यद्वा प्रथमत एव लाघवार्थं करणीयोगं कृत्वा पश्चाद्वर्गः कार्यः~। यथा\textendash \,क १८ क ८ क २ द्विकाष्टमित्योर्योगः क १८~। पुनरस्याः क १८ पूर्वकरण्या क १८ योगे जाता करणी क ७२~। अस्या वर्गे जाता करणी ५१८४~। अस्या मूलं जातो वर्गो रू ७२~। एवमुदाहृतकरणीनां खण्डगुणनेनापि वर्गाः साध्याः~। एवं खण्डद्वयस्याभिहतिरित्यादिप्रकारद्वयेनापि वर्गाः साध्याः~॥~४०~॥ \\

{\small अथ वर्गे दृष्टे कस्यायं वर्ग इति मूलज्ञानार्थमुपायमुपजातिकाद्वयेनाह\textendash }

\phantomsection \label{4.41}
\begin{quote}
{\large \textbf{{\color{purple}वर्गे करण्या यदि वा करण्योस्तुल्यानि रूपाण्यथवा बहूनाम्~।\\
विशोधयेद्रूपकृतेः पदेन शेषस्य रूपाणि युतोनितानि~॥\\
पृथक्तदर्धे करणीद्वयं स्यान्मूलेऽथ बह्वी करणी तयोर्या~।\\
रूपाणि तान्येवमतोऽपि भूयः शेषाः करण्यो यादॆ सन्ति वर्गे~॥~४१~॥}}}
\end{quote}

\hyperref[4.41]{\textbf{वर्गे करण्यास्तुल्यानि करण्योर्वा}} तुल्यानि \hyperref[4.41]{\textbf{बहूनां}} करणीनां वा तुल्यानि \hyperref[4.41]{\textbf{रूपाणि~रूप-कृतेः शोधयेत्}}~। अत्र रूपग्रहणं योगवियोगयोः \hyperref[4.34]{'योगं करण्योर्महतीं प्रकल्प्य'} इत्यादि प्रकारस्य व्यावृत्त्यर्थम्~। \hyperref[4.41]{\textbf{शेषस्य पदेन रूपाणि पृथक् युतोनितानि कृत्वा तदर्धे}} कार्ये~। \hyperref[4.41]{\textbf{मूले}} तत् \hyperref[4.41]{\textbf{करणीद्वयं}} भवति~। \hyperref[4.41]{\textbf{यदि}} पुनर्वर्गे \hyperref[4.41]{\textbf{शेषाः}} करण्यः \hyperref[4.41]{\textbf{सन्ति}} तर्हि तयोः मूलकरण्योर्मध्येऽल्पा मूलकरणी या महती तानि रूपाणि प्रकल्प्यातो रूपेभ्यो भूयोऽप्येवं करणीतुल्यानि रूपाणि रूपकृतेर्विशोधयेदित्यादिना पुनरपि मूलकरणीद्वयं स्यात्~। पुनः अपि यदि शेषाः करण्यो भवेयुस्तदैवमेव पुनः कुर्यात्~। अत्र महती रूपाणीत्युपल-क्षणम्~। क्वचिन्महती मूलकरण्यल्पा तु रूपाणीत्यपि द्रष्टव्यम्~। वक्ष्यति चाचार्यश्चत्वा-रिंशदशीतिरित्युदाहरणावसरे~। अथ च महती रूपाणीत्युपलक्षणं तेन क्वचिदल्पापीति~। अथ पूर्वसिद्धवर्गस्य मूलार्थं न्यासो रू १० क २४ क ४० क ६०~। अत्र रूपकृतेः १०० एककरणीतुल्यरूपशोधने शेषस्य पदाभावः~। करणीत्रितयस्य तुल्यरूपाणि तु न शुध्यन्ति~। अतः करणीद्वयतुल्यरूपाणि शोध्यानि~। करणीद्वयं त्वभीष्टम्~। इदं क २४ क ४० इदं वा क २४ क ६० इदं वा क ४० क ६०~। तत्र प्रथकरणीद्वयं विशोध्य मूलं साध्यते~। रूपकृतेः १०० करणीद्वय\textendash \,२४।४०\textendash \,तुल्यरूपाणि विशोध्य शेषं ३६ अस्य पदं ६ अनेन रूपाणि १० युतोनितानि १६।४ अर्धे ८।२~। वर्गेऽन्यापि करण्यस्ति क ६०~। अतो महती मूलकरणी रूपाणि ८~। एषां वर्गः ६४~। अस्माच्छेषकरणीतुल्यरूपाणि ६० विशोध्य शेषस्य ४ पदेन २ रूपाणि ८ युतोनि-
\end{sloppypar}

\newpage

\begin{sloppypar}
\noindent तानि कृत्वा १०।६ अर्धे ५।३~। एवं जाता मूलकरण्यः क २ क ३ क ५~। एवं द्वितीयतृतीयकरणीद्वययोः प्रथमशोधनेनाप्येता एव मूलकरण्यो भवन्ति~। अथ द्वितीयो-दाहरणे न्यासो रू ५ क २४~। रूपकृतेः २५ करणीतुल्यरूपाणि २४ विशोध्य शेषस्य १ मूलेन १ रूपाणि ५ युतोनितानि ६।४ तदर्धे ३।२ जाते मूलकरण्यौ क २ क ३~। अथ तृतीयोदाहरणे न्यासो रू १६ क १२० क ७२ क ६० क ४८ क ४० क २४~। रूपकृतेः २५६ करणीत्रितयस्यास्य १२०।७२।४८ तुल्यानि रूपाणि विशोध्य शेषस्यास्य १६ पदेन ४ रूपाणि १६ युतोनितानि २०।१२ तदर्धे १०।६~। अनयोरल्पा मूलकरणी क ६ महती रूपाणि १०~। एषां कृतेः १०० करणीद्वयं ६०।२४ अपास्य शेषस्य १६ पदेन ४ रूपाणि १० युतोनितानि १४।६ तदर्धे ७।३~। अनयोरल्पा ३ मूलकरणी~। महती ७ रूपाणि~। एषां कृतेः ४९ करणी\textendash \,४०\textendash \,तुल्यानि रूपाण्यपास्य शेषस्य ९ पदेन ३ रूपाणि ७ युतोनितानि १०।४ तदर्धे ५।२ जाते मूलकरण्यौ क ५ क २~। एवं जाताः सर्वा मूलकरण्यः क ६ क ३ क ५ क २~। अथ चतुर्थोदाहरणे न्यासः~। रू ७२ क ०~। रूपकृतेः ५१८४ करणीं ० विशोध्य शेषस्य ५१८४ पदेन ७२ रूपाणि ७२ युतोनितानि १४४।० तदर्धे ७२।०~। एवं जाता मूलकरणी क ७२~। नन्वियं कृतिः रू ७२ अष्टादशाष्टद्विकसंमितानां करणीनाम्~। तत्कथमस्या मूलं द्विसप्ततिकरण्य इति चेदुच्यते~। इयं तासामेव युतिकरणी क ७२~। अतः प्रतीत्यर्थं विश्लेषसूत्रेण पृथक्क्रियते~। यथा\textendash \,इयं योगकरणी ७२ वर्गेणानेन ३६ विहृता लब्धिः २~। कृतिपदं ६ पूर्वं खण्डत्रयमासीदिति खण्डत्रयं कृतम्~। ३।२।१~। एषां कृतयः ९।४।१ पूर्वलब्ध्या २ गुणिता जाताः पृथक्करण्यः १८।८।२~। अत्रोपपत्तिः\textendash \,करणीवर्गस्तावदेवं भवति {\color{violet}'स्थाप्योऽन्त्यवर्गश्चतुर्गुणान्त्यनिघ्ना'} इत्यादिना~। तत्र प्रथमस्थाने प्रथमकरणीवर्गः~। ततः प्रथमकरणीद्वितीयादिकरणीघाताश्चतुर्गुणाः~। ततो द्वितीयकरणीवर्गः~। ततो द्वितीय-करणीतृतीयादिकरणीघाताश्चतुर्गुणाः~। एवमग्रेऽपि तृतीयकरणीवर्गादि~। एवं यावन्ति करणीखण्डानि तावतामवश्यं वर्गाः स्युः~। वर्गत्वात्तेभ्योऽवश्यं मूललाभः~। तानि च मूलानि करणीतुल्यान्येव~। तथा च वर्गराशौ यो रूपगणः स एव मूलकरणीयोगः~। परं रूपरीत्या न करणीरीत्या~। यदि तु करणीरीत्यैव करणीयोगो ज्ञायेत तदा {\color{violet}'वर्गेण योगकरणी विहृता विशुध्येत्'} इत्यादिना पृथक्करणं सुलभम्~। प्रकृते तु रूपरीत्या करणीयोग इत्यन्यथा यतितव्यम्~। तत्रेदं प्रसिद्धम्~। \hyperref[8.131]{'चतुर्गुणस्य घातस्य युतिवर्गस्य चान्तरम्~। राश्यन्तरकृतेस्तुल्यम्'} इति~। इदमेकवर्णमध्यमाहरणे मूल एव स्फुटी भविष्यति~। विवरिष्यते चास्माभिस्तत्रैव~। अत्र तु यानि रूपाणि स करणीयोगः~। अतो रूपवर्गः करणीयुतिवर्गः~। वर्गराशौ कानिचित्करणीखण्डानि प्रथमकरणीद्वितीयादिकरणीघाताश्चतुर्गुणाः~। तेषां योगे प्रथमकरण्याः शेषकरणीयोगस्य
\end{sloppypar}

\newpage

\begin{sloppypar}
\noindent च घातश्चतुर्गुणः स्यात्~। युतिवर्गोऽपि प्रथमकरण्याः शेषकरणीयोगराशेश्चास्ति~।~अतः तयोरन्तरे प्रथमकरण्याः शेषकरणी योगस्य चान्तरवर्गः स्यात्~। अत उक्तम्\textendash \,\hyperref[4.41]{'वर्गे करण्या यदि वा करण्योस्तुल्यानि रूपाण्यथवा बहूनाम्~। विशोधयेद्रूपकृतेः'} इति~। एवं ज्ञातोऽन्तरवर्गः~। तस्य मूलं प्रथमकरण्याः शेषकरणीयोगस्य चान्तरम्~। रूपाणि तु तयोरेव योगः~। योगान्तरे च ज्ञाते {\color{violet}'योगोऽन्तरेणोनयुतोर्द्धितः'} इति सङ्क्रमणसूत्रेण तयोर्ज्ञानं सुलभम्~। तदिदमुक्तम्\textendash \,शेषस्य पदेन रूपाणि पृथग्युतोनितानि तदर्धे करणी-द्वयं स्यादिति~। एवं जाता प्रथमकरणी~। अवशिष्टकरणीयोगश्च~। अत्र मूले करणीद्वयम् आगतम्~। तत्र का वा प्रथमकरणी~। को वा शेषकरणीयोगः~। तत्र करणीयोगे मह-त्वस्यैककरण्यां स्वल्पत्वस्य चौचित्याल्लघुकरणी प्रथमा~। महती तु शेषकरणीयोगः~। अथ द्वितीयादिकरणयोगाद्द्वितीयकरणीतृतीयादिकरणीघाताच्चतुर्गुणाच्चोक्तवद्द्वितीयकरणी पृथक्कार्या~। अत उक्तं बह्वी करणी तयोर्यानि रूपाणि तानीति~। एवं तृतीयादिकरणीनामपि पृथक्करणम्~। इदमत्रावधेयम्~। मूले बह्वी करणी तयोर्यानि रूपाणि तानि त्वन्यत्र क्वचिल्लघुकरणीरूपाणि~। लघुकरण्या अपि शेषकरणीयोगत्वसम्भवात्~। यत्र ह्येका करणी महती इतरकरणीखण्डानि चातिलघूनि तत्र शेषकरणीयोगः पूर्वकरणीतो लघुरपि  स्यादेव~। यथा करण्यः क १० क ३ क २~। अत्रेतरकरणीयोगः पूर्वकरण्या लघुरस्ति~। अत्र प्रतीत्यर्थमुदाहरणं क १३ क ७ क ३ क २~। {\color{violet}'स्थाप्योऽन्त्यवर्गश्चतुर्गुणान्त्यनिघ्ना'} इत्यादिना जातो वर्गः क १६९ क ३६४ क १५६ क १०४ क ४९ क ८४ क ५६ क ९ क २४ क ४~। वर्गरूपाणां मूलानि १३।७।३।२~। एषां योगः २५~। एवं जातो वर्गो रू २५ क ३६४ क १५६ क १०४ क ८४ क ५६ क २४~। अत्रेयं रूपकृतिः ६२५~। अत्र {\color{violet}चतुर्गुणान्त्यनिघ्ना} इत्यादिना चतुर्गुणप्रथमकरणीगुणितं करणीत्रितयमेवास्तीति चतुर्गुणघातत्वात्तदेव शोध्यम्~। अतो रूपकृतेः ६२५ करणीत्रितयमेतत् ३६४।१५६।१०४ अपास्य शेषस्य १ पदेन १ रूपाणि युतोनितानि २६।२४ अर्धे १३।१२~। अत्र लघुः प्रथमकरणीति वक्तुमनुचितमुदाहृतकरणीषु तस्या अभावात्~। नापि महती रूपाणीति~। तस्याः शेषकरणीयोगत्वाभावात्~। अतोऽत्र लघुरेव रूपाणि १२~। एषां कृतिः १४४ उक्तवच्चतुर्घातरूपं करणीद्वयं ८४।५६ अपास्य शेषस्य ४ पदेन २ युतोनितानि रूपाणि १४।१० अर्धे ५।७~। अत्रापि पूर्ववन्महती मूलकरणी ७ लघ्वी ५ रूपाणि ५~। एषां कृतेः २५ करणीं २४ अपास्य शेषस्य पदेन १ युतोनितानि रूपाणि ६।४~। तदर्धे ३।२~। एवं जाताः सर्वा मूलकरण्यः क १३ क ७ क ३ क २~। तस्मान्महतीरूपाणीति न नियमः~। यत्तु महती रूपाणीत्युक्तं तद्बहूनामैक्ये
\end{sloppypar}

\newpage

\begin{sloppypar}
\noindent सङ्ख्याबाहुल्यस्योत्सर्गात्~। वस्तुतस्तु करण्याः प्राथमिकत्वं काल्पनिकमिति यैव करणी पृथक्कर्तुं शक्यते सैव कार्या~। तत्र लघुकरणीखण्डानां शोधनेन लघुः पृथग्भवति~। बृहत्खण्डशोधनेन महती पृथग्भवति~। तत्र यद्यपि बृहत्खण्डानां शोधनेन महती पृथग्भवति तथापि साधितमूलकरणीद्वयमध्येऽस्या न महत्त्वनियमः~। इतरकरणीयोगरूपाया~द्विती-यमूलकरण्या अपि महत्त्वसम्भवात्~। लघुखण्डशोधने तु लघुः पृथग्भवति~। साधित-करणीद्वयमध्येऽप्यस्ति तस्या लघुत्वनियमः~। इतरकरणीयोगरूपाया द्वितीयमूलकरण्या लघुत्वासम्भवात्~। अतो लघुखण्डकशोधनपूर्वकं मूलग्रहणे लघुर्मूलकरणी महती रूपाणीति नियमो द्रष्टव्यः~। बृहत्खण्डशोधनपूर्वकं मूलग्रहणे त्वनियमः~। अथ च महती रूपाणीत्युपलक्षणम्~। तेन क्वचिदल्पापीति प्रस्तुत्योदाह्रियते~। चत्वारिंशदशीतिद्वि-शतीतुल्याः करण्यश्चेत्~। सप्तदशरूपयुक्ता इति वर्गेऽपि लघुखण्डशोधनपूर्वकं मूलग्रहणे लघुर्मूलकरणी महती रूपाणीति नियमस्य न भङ्गोऽस्ति~। तथाहि\textendash \,उदाहृतवर्गन्यासः रू १७ क ४० क ८० क २००~। अत्र रूपकृतेः २८९ लघुकरणीद्वयं ४०।८० अपास्य शेषस्य १६९ पदेन १३ रूपाणि १७ युतोनितानि ३०।४ अर्धे १५।२~। अत्र लघुर्मूलकरणी २~। महती रूपाणि १५~। एषां कृतेः २२५ करणीम् २०० अपास्य शेषस्य २५ मूलेन ५ रूपाणि १५ युतोनितानि २०।१० अर्धे १०।५~। एवं जाता मूलकरण्यस्ता एव क १० क ५ क २~। अतः शिष्याणां गणितसौकर्यार्थं लघुखण्डशोधनपूर्वकं मूलं ग्राह्यमिति नियमो वक्तुमुचितः~। अन्यथा लघुर्महती वा मूलकरणीति व्याकुलता स्यादिति~। शोध्यकरणीनियमं त्वग्रे वक्ष्यति~। एकादिसङ्कलितमितकरणीखण्डानीत्यादिना~॥~४१~॥\\

{\small अथ यत्र वर्गराशावृणकरणी भवति तत्र मूलग्रहणे विशेषमुपजातिकयाह\textendash }

\phantomsection \label{4.42}
\begin{quote}
{\large \textbf{{\color{purple}ऋणात्मिका चेत्करणी कृतौ स्याद्धनात्मिकां तां परिकल्प्य साध्ये~।\\
मूले करण्यावनयोरभीष्टा क्षयात्मिकैका सुधियावगम्या~॥~४२~॥}}}
\end{quote}

यदि वर्गे \hyperref[4.42]{\textbf{करणी ऋणात्मिका स्यात्}} तर्हि \hyperref[4.42]{\textbf{तां धनात्मिकां परिकल्प्य मूले करण्यौ साध्ये}}~। \hyperref[4.42]{\textbf{अनयोः}} मूलकरण्योर्मध्येऽ\hyperref[4.42]{\textbf{भीष्टैका}} करणी \hyperref[4.42]{\textbf{सुधिया क्षयात्मिका}} ज्ञेया~। अत्र सुधियेति हेतुगर्भमुक्तम्~। तेन वर्गे यद्येकैव क्षयकरणी भवति तदैवैकस्या मूलकरण्याः क्षयत्वम्~। यदि द्व्यादयो भवन्ति तदैकस्या द्वयोर्बहूनां वा मूलकरणीनां युक्त्या यथा सम्भवति तथा क्षयत्वं कल्प्यम्~। यत्र वर्गे सर्वा अपि धनकरण्यस्तत्रापि सर्वासामपि मूलकरणीनां पक्षे क्षयत्वमवगन्तव्यमिति~। अत्रोपपत्तिः~। य एव ऋणकरणीवर्गः स एव धनकरणीवर्गः~। परमृणकरणीवर्गे करण्यृणात्मिका परत्र धनात्मिके-
\end{sloppypar}

\newpage

\begin{sloppypar}
\noindent त्येव विशेषः~। तथा सति वर्गे करणी ऋणात्मिका धनात्मिका वा भवतु मूलं त्वङ्कतः सममेवोचितम्~। उक्तविधिना रूपकृतेः क्षयकरणीशुद्धौ तु संशोध्यमानमृणं धनं स्यादिति योग एव स्यात्~। रूपवर्गाद्धनकरणीशुद्धौ संशोध्यमानं स्वमृणं स्यादित्यन्तरं स्यात्~। अन्तरे च मूलाङ्कसिद्धिरुक्तैव~। अतो धनामिकां तां परिकल्प्येत्युक्तम्~। परमेवं धनवर्गस्येव पदं स्यात्~। अत उक्तं क्षयात्मिकैकेति~॥~४२~॥\\

{\small अत्रोदाहरणानि सार्धोपजातिकयाह\textendash }

\phantomsection \label{4.43}
\begin{quote}
{\large \textbf{{\color{purple}त्रिसप्तमित्योर्वद मे करण्योर्विश्लेषवर्गं कृतितः पदं च~।\\
द्विकत्रिपञ्चप्रमिताः करण्यः स्वस्वर्णगा व्यस्तधनर्णगा वा~।\\
तासां कृतिं ब्रूहि कृतेः पदं च चेत्षड्विधं वेत्सि सखे करण्याः~॥~४३~॥}}}
\end{quote}

अत्र मूलग्रहण एव विशेषोक्तेर्यद्यपि सिद्धं वर्गमुद्दिश्य मूलप्रश्न एवोचितस्तथापि यदि कश्चिद्ब्रूयाद्वर्गे क्षयकरणी न सम्भवत्येवेति तं प्रति त्रिसप्तमित्योः करण्योर्विश्लेषवर्गं ब्रूही-त्यादिर्वर्गप्रश्नो द्रष्टव्यः~। शेषं स्पष्टम्~। न्यासः क ३ं क ७ वा न्यासः क ७ं क ३ अनयोर्वर्गः सम एव रू १० क ८४ं~। अत्र वर्गे ऋणकरण्या यथास्थितत्वे उक्तवद्वर्गपदाभावः~। तथा हि\textendash \,रूपकृतेः १०० करणीं ८४ अपास्य शेषं १८४~। अस्य पदाभावान्नोक्तवन्मूलसिद्धिः~। अतः क्षयकरणीं धनात्मिकां परिकल्प्य मूलं ग्राह्यम्~। तथा सति रूपकृतेः १०० करणीमपास्य शेषं १६ अस्य पदेन ४ रूपाणि १० युतोनितानि १४।६ अर्धे ७।३ जाते मूलकरण्यौ क ७ क ३ अनयोरेकाभीष्टा क्षयात्मिकेति जाते मूलकरण्यौ क ७ क ३ं वा क ७ं क ३~। अथ द्वितीयोदाहरणे न्यासः क २ क ३ क ५ं~। व्यस्तधनर्णत्वेन तृतीयोदाहरणे न्यासः~। क २ं क ३ं क ५~। अनयोः पक्षयोर्जातो वर्गः सम एव रू १० क २४ क ४ं० क ६ं०~। अत्राप्यृणत्वे यथास्थित उक्तवन्मूलाभावः~। तस्मात् \hyperref[4.42]{'ऋणात्मिका चेत्करणी कृतौ स्याद्धनात्मिकां तां परिकल्प्य साध्ये'} इति कृते करण्योः ४०।६० तुल्यानि रूपाणि १०० रूपकृतेः १०० अपास्य शेषम् ० अस्य पदेन ० रूपाणि युतोनितानि १०।१० अर्धे ५।५~। अनयोरेकस्यामृणत्वमवश्यं कल्प्यम्~। अन्यथा वर्गे क्षयकरणी न स्यादिति~। तत्र मूलकरण्याः क्षयत्वमितरस्या धनत्वं च प्रकल्प्य तावदुदाहरणं लिख्यते~। क ५ं इयं मूलकरणी~। शेषकरणीरूपाणि ५~। एतेषां कृतेः २५ करणीं २४ अपास्य शेषस्य १ पदेन रूपाणि ५ युतोनितानि ६।४ अर्धे जाते मूलकरण्यौ क ३ क २~। अत्रोभयोर्धनत्वमेव युक्तम्~। एकस्या ऋणत्वे वर्गे शेषकरण्याः क २४ धनत्वं न स्यात्~। तयोश्चतुर्गुणघातात्मकत्वात्~। अस्याः उभयोः
\end{sloppypar}

\newpage

\begin{sloppypar}
\noindent क्षयत्वे यद्यपि शेषकरण्याः सम्भवति धनत्वं तथापि पूर्वकरण्योः क्षयत्वं न स्यात्~। पूर्वमूलकरण्या क ५ं चतुर्गुणया क २ं० गुणितयोरेतयोर्मूलकरण्योः क ३ क २ धनत्वात् क ४० क ६० एवं जातं पदं क ५ं क ३ क २~। अथ मूलकरण्या धनत्वं प्रकल्प्योदाहरणम्~। मूलकरणी क ५ शेषा ५ रूपाणि रूपकृतेः २५ शेषकरणीं २४ अपास्य पूर्ववज्जाते मूलकरण्यौ क ३ क २ अत्रोभयोः क्षयत्वमेव युक्तम्~। एकस्या एव क्षयत्व उक्तयुक्त्या शेषकरण्याः क २४ धनत्वं न स्यात्~। उभयोर्धनत्वं उक्तयुक्त्या पूर्वकरण्योः क ४० क ६० क्षयत्वं न स्यात्~। एवं वा जातं पदं क ५ क ३ क २ तस्मादुक्तं सुधियेति~। एवमनयोः क २४ क ४० अनयोर्वा क २४ क ६० प्रथमतः शोधनेनापि पदद्वयं द्रष्टव्यम्~। नन्वृणकरण्या धनत्वकल्पनं विनैवास्ति मूलसिद्धिः~। यथाहि\textendash \,क २ क ३ क ५ं वा क २ं क ३ं क ५ {\color{violet}'स्थाप्योऽन्त्यवर्गः'} इत्यादिना जातो वर्गः क ४ क २४ क ४ं० क ९ क ६० क २५ \hyperref[1.13]{'स्वमूले धनर्णे'} इति वर्गकरणीनां मूलानि रू २ रू ३ रू ५ं वा रू २ं रू ३ं रू ५ उभयेषामपि योगः सम एव रू ० एवं जातो वर्गः रू ० क २४ क ४ं० क ६ं०~। अत्र रूपकृतेः ० करणीद्वयं क २४ क ४ं० अपास्य शेषस्य १६ पदेन ४ रूपाणि ० युतोनितानि ४।४ अर्धे २।२ एका मूलकरणी क २ अपरा २ रूपाणि~। एतत्कृतेः ४ शेषकरणीं क ६० अपास्य शेषस्य ६४ पदेन ८ रूपाणि २ युतोनितानि ६।१० अर्धे ३।५ जाता मूलकरण्यः क २ क ३ क ५~। अथ यदि परा मूलकरणी क २ आद्या क २ रूपाणि~। एतत्कृतेः ४ शेषकरणीं ६० अपास्य शेषस्य ६४ पदेन ८ रूपाणि २ युतोनितानि १०।६ अर्धे ५।३ एवं जाता मूलकरण्यः क २ क ३ क ५~। अथवा रूपकृतेः ० करणीद्वयं क ४ं० क ६ं० अपास्य शेषस्य १०० पदेन १० रूपाणि ० युतोनितानि १०।१० अर्धे ५।५ अनयोराद्या मूलकरणी क ५ परा ५ रूपाणि~। एतत्कृतेः २५ शेषकरणीं २४ अपास्य शेषस्य १ पदेन १ रूपाणि ५ युतोनितानि ६।४ अर्धे ३।२ एवं जाता मूलकरण्यः क ५ क २ क ३~। अथ यदि परा मूलकरणी क ५ आद्या ५ रूपाणि १ एतत्कृतेः २५ शेषकरणीं क २४ अपास्य शेषस्य १ पदेन १ रूपाणि ५ युतोनितानि ६।४ अर्धे ३।२ एवं जाता मूलकरण्यः क ५ क ३ क २~। एवमनयोरपि क २४ क ६० शोधने पदद्वयं द्रष्टव्यम्~। एवं वर्गकरण्या धनत्वकल्पनं विनैव मूलसिद्धावपि स्वयमशुद्धं वर्गं कृत्वा तस्योक्तवन्मूलं नायातीति \hyperref[4.42]{'ऋणात्मिका चेत्करणी कृतौ स्याद्धनात्मिकां तां परिकल्प्य साध्या'} इति विशेषमभिधाय पुनर्मूलकरणीनां मध्ये क्षयत्वकल्पनेऽनुगमाभावात्सुधियेत्यादि यदुक्तं तदयुक्तं विशेषज्ञानामाचार्याणा-
\end{sloppypar}

\newpage

\begin{sloppypar}
\noindent मिति चेदुच्यते~। विस्मृतगुडरसस्य पित्तोपहतरसनस्य गुडं भक्षयतस्तिक्तं रसमनुभवतो देवदत्तस्य मधुरोऽयं गुड इति यथार्थवादिनि सर्वज्ञेऽपि यथा भ्रान्तत्वनिश्चयस्तथाचार्थे तवापि स युक्त एव~। ननु कथमिदमवगन्तव्यम्~। शृणु तर्हि~। एता हि मूलकरण्यः क २ क ३ क ५ं एता वा क २ं क ३ं क ५ एतासामासन्नमूलानि गृहीत्वा तदैक्यं च कृत्वा कृते वर्गे वर्गकरणीनामेवादौ वर्गं कृत्वा पश्चादासन्नमूलानि गृहीत्वा कृते योगे तुल्यतयैव भाव्यम्~। करणीषड्विधस्य स्वमूलषड्विधार्थं प्रवृत्तेः~। अन्यथा योगं करण्योरित्यादिना कृतः करणीयोगो नोपपद्येत~। तत्रासामासन्नमूलानि {\scriptsize $\begin{matrix}
\mbox{{१}}\\
\vspace{-1.5mm}
\mbox{{२५}}
\vspace{1mm}
\end{matrix}$}~। {\scriptsize $\begin{matrix}
\mbox{{१}}\\
\vspace{-1.5mm}
\mbox{{४४}}
\vspace{1mm}
\end{matrix}$}~। {\scriptsize $\begin{matrix}
\mbox{{२ं}}\\
\vspace{-1.5mm}
\mbox{{१४}}
\vspace{1mm}
\end{matrix}$}~। वा \,{\scriptsize $\begin{matrix}
\mbox{{१ं}}\\
\vspace{-1.5mm}
\mbox{{२५}}
\vspace{1mm}
\end{matrix}$}~। {\scriptsize $\begin{matrix}
\mbox{{१ं}}\\
\vspace{-1.5mm}
\mbox{{४४}}
\vspace{1mm}
\end{matrix}$}~। {\scriptsize $\begin{matrix}
\mbox{{२}}\\
\vspace{-1.5mm}
\mbox{{१४}}
\vspace{1mm}
\end{matrix}$}~। एषां योगो धनम् ०।५५~। ऋणं वा ०।५५~। अनयोर्वर्गस्तुल्य एव ०।५० धनम्~। अथाचार्यैः प्रथमतः कृतस्य करणीवर्गस्यास्य रू १० क २४ क ४ं० क० ६ं० करणीनामासन्नमूलानि {\scriptsize $\begin{matrix}
\mbox{{४}}\\
\vspace{-1.5mm}
\mbox{{५४}}
\vspace{1mm}
\end{matrix}$}~। {\scriptsize $\begin{matrix}
\mbox{{६ं}}\\
\vspace{-1.5mm}
\mbox{{१९}}
\vspace{1mm}
\end{matrix}$}~। {\scriptsize $\begin{matrix}
\mbox{{७ं}}\\
\vspace{-1.5mm}
\mbox{{४५}}
\vspace{1mm}
\end{matrix}$}~। रुपेषु १० संयोज्य जातो वर्गः~। स एव ०।५०~। अथ यदि त्वत्कृतस्य वर्गस्य रू ० क २४ क ४ं० क ६ं० आसन्नमूलानां तेषामेव योगः क्रियते तदायं स्यात् ९ं~। अयमशुद्धो वर्गः~। ऋणत्वादप्यशुद्धिः~। न त्वृणं वर्गः सम्भवतीति निरूपितं न मूलं क्षयस्यास्ति तस्याकृतित्वादित्यत्र~। अथ यदि मूलस्य सावयवत्वादस्मिन्वर्गे तव न स्फुटा प्रतीतिरस्ति तर्हीदमुदाहरणं रू ३ रू ७ं~। एतेषां योगस्य रू ४ं वर्गेणानेन रू १६ भाव्यम्~। अथात्र त्वदुक्तरीत्या यदि करणीवर्गः क्रियते तदैतावान्न भवति~। तथा हि\textendash \,क ९ क ४ं९~। अत्र {\color{violet}'स्थाप्योऽन्त्यवर्गः'} इत्यादिना जातो वर्गः क ८१ क १७६ं४ क २४०१~। अत्राचार्योक्तमार्गेण मूलानि रू ९ रू ४ं२ रू ४९~। एषां योगे भवति वर्गः स एव रू १६~। त्वदुक्तमार्गेण मूलग्रहणे जातानि मूलानि रू ९ रू ४ं२ रू ४९ं~। एषां योगः रू ० ८२ं~। नह्ययं वर्गः सम्भवति~। ननु तर्हि \hyperref[1.13]{'स्वमूले धनर्णे'} इत्यस्य का गतिः~। शृणु तर्हि~। मूलग्रहणे हि \hyperref[1.13]{'स्वमूले धनर्णे'} इत्युक्तम्~। प्रकृते तु मूलकरणीवर्गे करणीनां रूपजातित्वेन स्थापनमस्ति~। न तु मूलं गृह्यते~। अत एव कृतेष्वपि रूपेषु करणीवर्ग इत्येव व्यवहारोऽस्ति न तु करणीवर्गमूलमिति~। एवं रूपाणामपि करणीजातित्वेन स्थापने सति वर्गविधानं नास्ति~। अत एव क्षयरूपाणां करणीत्वेन स्थापने क्षयकरण्य एव स्थाप्यन्ते~। वर्गविधाने तु क्षयत्वं
\end{sloppypar}

\newpage

\begin{sloppypar}
\noindent कथं स्यात्~। तस्माद्रूपकरण्योर्भिन्नजातित्वप्रयुक्तः सङ्ख्याभेदो न तु वास्तवः~। यथा~वरा-टकजात्या विंशतिः २० काकिणी जात्यैकः १ पणजात्या चतुर्थांशो $\dfrac{{\footnotesize{\hbox{१}}}}{{\footnotesize{\hbox{४}}}}$ द्रम्मजात्या चतुःषष्ट्यंशः $\dfrac{{\footnotesize{\hbox{१}}}}{{\footnotesize{\hbox{६४}}}}$ स्थाप्यते~। न ह्यासां सङ्ख्यानां फलतो भेदोऽस्ति~। अत एव रूपत्रयस्य करणीनवकस्य वा वर्गो रूपनवकमेव~। रूपकरण्योः फलतो भेदे सम एव वर्गः कथं स्यात्~। तस्मात्सुष्ठूक्तम् \hyperref[4.42]{'ऋणात्मिका चेत्करणी कृतौ स्यात्'} इत्यादि~। ननु मूलकरणीनां क्षयत्वकल्पने कोऽनुगमः~। शृणु~। \hyperref[4.44]{'वर्गे करणीत्रितये करणीद्वितयस्य तुल्यरूपाणि'} इत्यादि वक्ष्यमाणप्रकारेण शोध्यकरणीनां नियमे तासां धनत्वमेव प्रकल्प्य मूलकरण्यौ साध्ये~। तत्र या मूलकरणी तस्या धनत्वमृणत्वं वा प्रकल्प्य तया चतुर्गुणया यथास्थितधनर्णताकाः शोधितकरण्यो भाज्याः~। भजने यादृश्यः करण्यो धनमृणं वा लभ्यन्ते तादृश्यः शेषकरण्यो ज्ञेया इत्यादि मतिमद्भिरन्यदप्यूह्यमित्यलं पल्लवितेन~॥~४३~॥ \\

{\small अथ वर्गे करण्या यदि वा करण्योरित्याद्युक्तेरनियमेन करणीशोधने सति मूलाशुद्धिः स्यादिति करणीवर्गे करणीसङ्ख्यानियमपूर्वकं शोध्यकरणीनियमं गीतिद्वयेनार्याद्वितयेन च निरूपयति\textendash }

\phantomsection \label{4.44}
\begin{quote}
{\large \textbf{{\color{purple}एकादिसङ्कलितमितकरणीखण्डानि वर्गराशौ स्युः~।\\
वर्गे करणीत्रितये करणीद्वितयस्य तुल्यरूपाणि~।\\
करणीषट्के तिसृणां दशसु चतसृणां तिथिषु च पञ्चानाम्~।\\
रूपकृतेः प्रोज्झ्य पदं ग्राह्यं चेदन्यथा न सत्क्वापि~।\\
उत्पत्स्यमानयैवं मूलकरण्याल्पया चतुर्गुणया~।\\
यासामपवर्तः स्याद्रूपकृतेस्ता विशोध्याः स्युः~।\\
अपवर्ते या लब्धा मूलकरण्यो भवन्ति ताश्चापि~।\\
शेषविधिना न यदि ता भवन्ति मूलं तदा तदसत्~॥~४४~॥}}}
\end{quote}

अत्र द्वितीयगीतौ तिथिषु पञ्चानामिति बहवः पठन्ति तत्र तिथिषु च पञ्चानामिति पठ-नीयम्~। अन्यथा छन्दोभङ्गात्~। अत्रैकादिसङ्कलितमितकरणीखण्डानि वर्गराशौ स्युरित्य-नेनैककरण्या वर्ग एका करणी द्वयोः करण्योर्वर्गे करणीत्रितयं स्यादित्यादि निरूपितं तच्च प्रत्यक्षविरुद्धमतः स्वयमेव तदर्थं विवृणोति~। करणीवर्गराशौ रूपैरवश्यं भवितव्यम्~। एककरण्या वर्गे रूपाण्येव~। द्वयोः सरूपैका करणी~। तिसृणां तिस्रश्चतसृणां षट् पञ्चानां दश षण्णां पञ्चदश~। ततो द्व्यादीनां करणीनां वर्गेष्वेकादिसङ्कलित-
\end{sloppypar}

\newpage

\begin{sloppypar}
\noindent मितानि करणीखण्डानि यथाक्रमं स्युः~। अथ यद्युदाहरणे तावन्ति न भवन्ति तदा संयोज्य योगकरणीं विश्लेष्य वा तावन्ति कृत्वा मूलं ग्राह्यमित्यर्थः~। वर्गे करणीत्रितये करणीद्वितयस्य तुल्यरूपाणीत्यादि स्पष्टार्थमिति~। अत्र करणीवर्गे राशौ रूपैरवश्यं भवितव्यम्~। एककरण्या वर्गे रूपाण्येव~। द्वयोः सरूपैकेत्यार्यां कल्पयित्वा सूत्रमध्ये पठन्ति तदशुद्धम्~। करणी तिसृणां तिस्र इत्यादेरग्रिमग्रन्थस्यानन्वयात्~। नह्येकमेव वाक्यं श्लोकचूर्णिकात्मकमिति रीतिरस्ति~। पूर्वार्धे छन्दोभङ्गाच्च~। सङ्कलितं च {\color{violet}"सैकपदघ्नपदार्थमथैकाद्यङ्कयुतिः किल सङ्कलिताख्या"} इत्युक्तं {\color{violet}पाट्याम्}~। तस्मान्मूले यद्येतावत्प्रभृतीनि करणीखण्डानि 

\begin{center}
२~। ३~। ४~। ~५~। ~~६~। ~~७~। ~८~। ~~९~। ~१०~। ~११~। १२~। १३~। १४~। ~~१५~।\\
१~। ३~। ६~। १०~। १५~। २१~। २८~। ३६~। ४५~। ५५~। ६६~। ७८~। ९१~। १०५~।
\end{center}

\noindent तदा वर्गराशौ तावत्प्रभृतीनि करणीखण्डानि~। शेषे किञ्चिन्मया व्याख्यायते\textendash \,उत्पत्स्य-मानयेति~। अत्राल्पयेत्युपलक्षणम्~। यत्र महती मूलकरणी, अल्पा रूपाणि तत्र महत्या चतुर्गुणया यासामपवर्तः स्यात्ता एव विशोध्याः~। आचार्यमते त्वल्पत्वं पारिभाषिकम्~। यतोऽस्य सूत्रस्योदाहरणे यां मूलकरणीं रूपाणि प्रकल्प्यान्ये करणीखण्डे साध्येते सा महतीत्यर्थ इति व्याकरिष्यति~। पुनर्नियमान्तरमाह\textendash \,\hyperref[4.44]{\textbf{अपवर्त}} इति~। \hyperref[4.44]{\textbf{अल्पया}} क्वचित् महत्या वा \hyperref[4.44]{\textbf{चतुर्गुणया}}पवर्ते कृते याः करण्यो \hyperref[4.44]{\textbf{लब्धास्ता}} एव \hyperref[4.44]{\textbf{मूलकरण्यो भवन्ती}}ति वस्तु-स्थितिः~। अथ यदि \hyperref[4.44]{\textbf{शेषविधिना}} मूलेऽथ बह्वी करणी तयोर्येत्यादिना वा \hyperref[4.44]{\textbf{न भवन्ति तदा तन्मूलमसदि}}ति~। अत्राल्पयेत्युपलक्षणमिति यद्व्याख्यातं तद्बृहत्खण्डशोधनपूर्वकं मूलग्रहणे~। लघुखण्डशोधनपूर्वकं मूलग्रहणे त्वल्पयेत्येव~। अत्रोपपत्तिः~। यत्रैकैव करणी तत्र {\color{violet}'स्थाप्योऽन्त्यवर्गः'} इति वर्ग एव स्यात्~। तस्य च मूललाभाद्रूपाण्येव स्युः~। यत्र तु करणीद्वयं तत्र {\color{violet}'स्थाप्योऽन्त्यवर्गः'} इत्येककरण्या वर्गः~। तदुत्तरं {\color{violet}'चतुर्गुणान्त्यनिघ्नाः'} इति शेषमेकैव चतुर्गुणान्त्यनिघ्नीति~। एवं यत्र करणीत्रयं तत्र {\color{violet}'स्थाप्योऽन्त्यवर्गः'} इत्येककरण्या वर्गः~। तदुत्तरं {\color{violet}'चतुर्गुणान्त्यनिघ्नाः'} इति शेषकरणीद्वयं चतुर्गुणान्त्यनिघ्नम्~। ततोऽन्त्यं त्यक्त्वेति शेषं करणीद्वयम्~। तत्रापि {\color{violet}'स्थाप्योऽन्त्यवर्गः'} इति द्वितीयकरण्या वर्गः~। चतुर्गुणान्त्यनिघ्नी चापरा~। एवं यत्र करणीषट्कं तत्र {\color{violet}'स्थाप्योऽन्त्यवर्गः'} इति प्रथमकरण्या वर्गः~। ततः पञ्च शेषकरण्यश्चतुर्गुणान्त्यनिघ्न्य इति पञ्च करणीखण्डानि पुनरन्त्यत्यागे द्वितीयकरण्या वर्गः~। शेषाश्चतस्रश्चतुर्गुणान्त्यनिघ्न्य इति चत्वारि खण्डानि~। पुनरन्त्यत्यागे तृतीयकरण्या वर्गः~। शेषास्तिस्रश्चतुर्गुणान्त्यनिघ्न्य इति त्रीणि खण्डानि पुनरन्त्यत्यागे चतुर्थकरण्या वर्गः~। ततः शेषं करणीद्वयं चतुर्गुणान्त्यनिघ्नमिति खण्डद्वयम्~। पुनरन्त्यत्यागे
\end{sloppypar}

\newpage

\begin{sloppypar}
\noindent पञ्चमकरण्या वर्गः~। शेषा करणी चतुर्गुणान्त्यनिघ्नीत्येकं खण्डम्~। पुनरन्त्यं त्यक्त्वा षष्ठ्या वर्गः~। एवं वर्गे जाताः षण्णामपि करणीनां वर्गाः~। तेषां मूलानि मूलकरणी तुल्यानि रूपाणि स्युः~। अतस्तेषां योगः करणीवर्गे रूपाणि~। करणीखण्डानि तु प्रथमं पञ्च ततश्चत्वारि ततस्त्रीणि ततो द्वे तत एकमिति व्यस्तमेकाद्येकोत्तराणि भवन्ति~। तस्मात् एकोनपदसङ्कलितमितकरणीखण्डानि भवन्ति~। प्रथमखण्डस्य वर्गत्वेनैव स्थापनात्~। अतो द्व्यादीनां वर्ग एकादिसङ्कलितमितकरणीखण्डानीत्युक्तम्~। अनयैव युक्त्या वर्गे करणीत्रितय इत्यादि बोध्यम्~। यतो रूपाणि करणीयोगस्तस्य वर्गो युतिवर्गः~। तत्र प्रथमकरण्याः पृथक्करणे प्रथमकरणीशेषकरणीपञ्चकघातश्चतुर्गुणः शोध्योऽन्तरवर्गार्थम्~। अत्र तु युक्तिः प्रागेवोक्ता~। अतः षण्णां करणीनां वर्गे प्रथमकरण्याः पृथक्करणे प्रथम-करणीशेषकरणीपञ्चकघातश्चतुर्गुणोऽन्तरवर्गार्थं युतिवर्गः शोध्यो भवतीति करणीषट्कवर्गे पञ्चैव करण्यः शोध्याः~। तदिदमुक्तं तिथिषु पञ्चानामिति~। यतः करणीषट्कवर्गे पञ्चदशैव करणीखण्डानि भवन्ति~। एवं करणीपञ्चकवर्गे प्रथमकरण्याः पृथक्करणे प्रथमकरणीशेष-करणीचतुष्टयघातश्चतुर्गुणः शोध्य इति चतस्र एव करण्यः शोध्याः~। तदिदमुक्तं दशसु चतसृणामति~। एवं करणीषट्के तिसृणां वर्गे करणीत्रितये करणीद्वितयस्य तुल्यरूपाणी-त्याद्यपि बोध्यम्~। एवमपि यदा केनचिद्धृष्टेनोक्तनियमपूर्वकं यथामूलमायाति तथा रूपाणि करणीश्च कल्पयित्वा यदि पृच्छ्यते तदा तदुदाहरणं खिलमखिलं वेति ज्ञानार्थम् उक्तमल्पया चतुर्गुणया यासामपवर्तः स्यादिति~। अत्राल्पयेति प्रथमकरणी लक्ष्यते~। यतः प्रथमकरणीशेषकरणीघातश्चतुर्गुणः शोध्यतेऽतो याः शोधितास्तासामाद्यया चतुर्गु-णयापर्वतः स्यादेव~। यद्यपवर्तो न स्यात्तदोदाहरणस्य खिलत्वं स्फुटमेव~। अथ यदि धृष्टतरेण प्रथमशोध्यकरण्यस्ता एव तादृश्योऽन्या वा स्थापिताः परतस्तु याः काश्चन युक्त्या स्थापितास्तदा तदुदाहरणस्य खिलत्वाखिलत्वज्ञानार्थमुक्तमपवर्ते या लब्धा इत्यादि~। यतश्चतुर्गुणान्त्यनिघ्ना इत्यत्र चतुर्गुणप्रथमकरणीशेषं करणीघातोऽस्ति तत्र चतुर्गुणप्रथकरण्यापवर्ते मूलकरण्य एव लभ्या इति मूलकरण्योऽपवर्तादेव ज्ञाताः~। यदि तु शेषविधिना ता न भवन्ति तदा शेषकरणीनां दुष्टत्वात्तदुत्पन्नं मूलमपि दुष्टमित्युपपन्नम्~॥~४४~॥\\

{\small अथ वर्गे करणीत्रितय इत्यादि नियमं विना मूलग्रहणे मूलासत्त्वमित्यत्रोदाहरणमार्ययाह\textendash }

\phantomsection \label{4.45}
\begin{quote}
{\large \textbf{{\color{purple}वर्गे यत्र करण्यो दन्तैः ३२ सिद्धै\textendash \,२४\textendash \,र्गजै\textendash \,८\textendash \,र्मिता विद्वन्~।\\
रूपैर्दशाभिरुपेताः किं मूलं ब्रूहि तस्य स्यात्~॥~४५~॥}}}
\end{quote}

स्पष्टोऽर्थः~। न्यासो रू १० क ३२ क २४ क ८ अत्र वर्गे करण्या इत्यादिनैव
\end{sloppypar}

\newpage

\begin{sloppypar}
\noindent मूलग्रहणे करणीत्रितयशोधनं विना शेषस्य पदाभावात्~। रूपकृतेः १०० करणीत्रयतुल्य-रूपाणि विशोध्य शेषस्य ३६ पदेन ६ रूपाणि १० युतोनितानि १६।४ तदर्धे जाते मूलकरण्यौ क ८ क २ तदिदं पदमसत्~। यतोऽस्य वर्गोऽयं रू १० क ६४ अत उक्तं वर्गे करणीत्रितये करणीद्वितयस्य तुल्यरूपाणीत्यादि~। एवं येषां करणीखण्डानां योगे रूपकृतेः शोधिते शेषस्य पदं लभ्यते तादृशानि करणीखण्डानि कल्पयित्वोदाहरणानि द्रष्टव्यानि~॥~४९~॥ \\

{\small अथ वर्गे करणीत्रितय इत्यादि नियमेनापि मूलग्रहणेऽग्रिमनियमं विना मूलं दुष्टमित्यत्रोदा-हरणमार्ययाह\textendash }

\phantomsection \label{4.46}
\begin{quote}
{\large \textbf{{\color{purple}वर्गे यत्र करण्यस्तिथिविश्वहुताशनैश्चतुर्गुणितैः~।\\
तुल्या दशरूपाढ्याः किं मूलं ब्रूहि तस्य स्यात्~॥~४६~॥}}}
\end{quote}

स्पष्टोऽर्थः~। न्यासो रू १० क ६० क ५२ क १२~। रूपकृते १०० उक्तनियमेन करणीद्वयं ५२।१२ अपास्य शेषस्य ३६ पदेन ६ रूपाणि १० युतोनितानि १६।४ अर्धे ८।२ अनयोरल्पा मूलकरणी २ महती रूपाणि ८ तत्कृतेः ६४ करणीम् ६० अपास्य शेषस्य ४ पदेन २ रूपाणि ८ युतोनितानि १०।६ तदर्धे ५।३ एवं जातं मूलं क २ क ३ क ५ तदिदमसत्~। यतोऽस्य वर्गोऽयं रू १० क २४ क ४० क ६० अत उक्तमल्पया चतुर्गुणया यासामपवर्तः स्यादिति~। अत्राल्पया २ चतुर्गुणया ८ शोधितकरण्योः ५२।१२ अपवर्तो न भवतीत्यशुद्धं पदम्~॥~४६~॥\\

{\small एवमपि मूलग्रहणेऽग्रिमनियमाभावे मूलमसदित्यत्रोदाहरणमार्ययाह\textendash }

\phantomsection \label{4.47}
\begin{quote}
{\large \textbf{{\color{purple}अष्टौ षट् पञ्चाशत् षष्टिः करणीत्रयं कृतौ यत्र~।\\
रूपै\textendash \,६\textendash \,र्दशभिरुपेतं किं मूलं ब्रूहि तस्य स्यात्~॥~४७~॥}}}
\end{quote}

अत्र करणीत्रितयं कृतौ सखे यत्रेति केचित्पठन्ति तदशुद्धम्~। मात्राधिक्येन च्छन्दो-भङ्गात्~। स्फुटोऽर्थः~। न्यासो रू १० क ८ क ५६ क ६० अत्र करणीत्रितये करणीद्वितयस्येति नियमात्~। करणीद्वय\textendash \,८।५६\textendash \,शोधनेन जाते मूलकरण्यौ ८।२ अत्राल्पया २ चतुर्गुणया ८ शोधितकरण्योः ८।५६ अपवर्तः सम्भवतीत्यल्पा मूलकरणी २ महतीरूपाणि ८ पुनरेतेभ्य उक्तवज्जातं करणीद्वयं ५।३~। अत्राप्यल्पया ३ चतुर्गुणया १२ शोधितकरण्या ६० अपवर्तः सम्भवतीति जातं मूलं क २ क ३ क ५ तदिदमप्यसत्~। यतोऽस्य वर्गोऽयं रू १० क २४ क ४० क ६०~। अत उक्तमपवर्ते या लब्धा इत्यादि~। अत्राल्पया २ चतुर्गुणया ८ शोधितकरण्योः ८।५६ अपवर्तेन लब्धे १।७ शेषविधिना त्वन्ये मूलकरण्यौ ५।३~॥~४७~॥
\end{sloppypar}

\newpage

\begin{sloppypar}
{\small अथ वर्गे षट्प्रभृतिषु करणीखण्डेष्वप्येवमेवेति व्याप्तिं प्रदर्शयितुमुपजातिकयोदाहरणमाह\textendash }

\phantomsection \label{4.48}
\begin{quote}
{\large \textbf{{\color{purple}चतुर्गुणाः सूर्यतिथीषु रुद्रनागर्तवो यत्र कृतौ करण्यः~।\\
सविश्वरूपा वद तत्पदं ते यद्यस्ति बीजे पटुताभिमानः~॥४८~॥}}}
\end{quote}

अत्र रुद्रा इति पाठे नागर्तवश्चतुर्गुणा इति न प्रतीयतेऽतो रुद्रनागर्तव इति पाठः साधीयान्~। स्फुटोऽर्थः~। न्यासो रू १३ क ४८ क ६० क २० क ४४ क २४ क ३२~। अत्र करणीषट्के तिसृणामिति नियमपूर्वकं मूलं नायातीति नायं वर्गः~। यदि तु नियमं विहाय मूलं गृह्यते तर्ह्यसत्~। तथाहि\textendash \,रूपकृतेः १६९ करणीम् ४८ अपास्योक्तवज्जातं मूले करणीद्वयं १२।१ पुनर्महतीरूपाणीति तत्कृतेः १४४ क ६० क २० अपास्योक्तवज्जातं मूले करणीद्वयं १०।२~। पुनरपि महतीरूपाणीति तत्कृतेः १०० क ४४ क ३२ क २४ अपास्योक्तवज्जातं करणीद्वयं ५।५ एवं जातं मूलं क १ क २ क ५ क ५ तदिदमसत्~। यतोऽस्य वर्गोऽयं रू १३ क ८ क २० क २० क ४० क ४० क १००~। अत्र शतमितकरण्या मूललाभात्तन्मूलं १० रूपेषु १३ प्रक्षिप्य जातानि रूपाणि २३~। समकरण्योर्योगे जाता चतुर्गुणिता ८०।१६०~। एवं जातो वर्गो रू २३ क ८ क ८० क १६०~। अस्माद्वर्गान्मूलग्रहणे खण्डत्रयमेवायाति~। अस्ति च मूले करणीचतुष्टयमिति योगकरणी विश्लेष्या~। ननु प्रथमं वर्गे करण्या यदि वा करण्योरित्यादिना नियमं विनैव मूलग्रहणमुक्तमिदानीं तं तं नियमं विना मूलग्रहणे सदसदित्युच्यते तत्कथं प्रथमत एव नियमपूर्वकं मूलग्रहणं नोक्तमित्यत आह\textendash \,यैरस्य मूलानयनस्य नियमो न कृतस्तेषामिदं दूषणमिति प्रथमं सर्वसाधारण्येन मूलग्रहणमुक्तमिदानीं तावन्मात्रेण मूलग्रहणे मूलाशुद्धिरिति स्वयं विशेष उक्त इति भावः~। ननूद्दिष्टवर्गेभ्य उक्तविधिना तु मूलं न लभ्यतेऽथ यदि तादृशवर्गाणां मूलापेक्षा स्यात्तदा किं विधेयमित्यत आह\textendash \,एवंविधे वर्गे करणीनामासन्नमूलकरणेन मूलान्यानीय रूपेषु प्रक्षिप्य मूलं वाच्यमिति~। तद्रूपसङ्ख्याकाः करण्यो मूलमित्यर्थः~। शेषं स्पष्टम्~॥~४८~॥ \\

{\small क्वचिदल्पापि रूपाणीत्यत्रोदाहरणमुपगीत्याह\textendash }

\phantomsection \label{4.49}
\begin{quote}
{\large \textbf{{\color{purple}चत्वारिंशदशीतिद्विशतीतुल्याः करण्यश्चेत्~।\\
सप्तदशरूपयुक्तास्तत्र कृतौ किं पदं ब्रूहि~॥~४९~॥}}}
\end{quote}

अत्र चतुर्थचरणे 'यत्र कृतौ तत्र किं पदं ब्रूहि' इति पाठेऽसावुद्गीतिर्ज्ञेया~। अशीतिरिति रेफान्तः पाठो न युक्तः~। स्पष्टोऽर्थः~। न्यासो रू १७ क ४० क ८० क २०० अत्र लघुखण्डशोधनपूर्वकं मूलग्रहणे महत्येव रूपाणीति प्रागेव प्रति-
\end{sloppypar}

\newpage

\begin{sloppypar}
\noindent पादितम्~। अथ बृहत्खण्डशोधनपूर्वकमूलग्रहणे महती रूपाणीत्युक्तविधिना यद्यपि मूलं नायाति तथापि नासौ वर्ग इति वक्तुमनुचितम्~। किं त्वल्पा रूपाणीति प्रकल्पनेऽपि यदि मूलं न लभ्यते तदैवावर्गत्वं युक्तम्~। प्रकृते तु रूपकृतेः २८९ करणीद्वयम् २००।८० अपास्य शेषस्य ९ पदेन ३ रूपाणि १७ युतोनितानि २०।१४ अर्धे १०।७ जाते मूलकरण्यौ~। अत्राल्पया ७ चतुर्गुणया २८ शोधितकरण्योः २००।८० अपवर्तो न भवतीत्येतावता न मूलाशुद्धिः~। किन्त्वल्पा रूपाणीति  प्रकल्पने महत्या चतुर्गुणया अपवर्तासम्भवे~। तस्मात्पारिभाषिकेऽल्पमहत्त्वे न स्वरूपेण~। अत एवाचार्येण \hyperref[4.46]{'वर्गे यत्र करण्यस्तिथिविश्वहुताशनैश्चतुर्गुणितैः'} इत्यस्मिन्नुदाहरणे निरूपितम्~। या मूलकरणी रूपाणि प्रकल्प्यान्ये करणीखण्डे साध्येते सा महती प्रकल्प्येत्यर्थ इति~। एवं कृतपरिभाषया प्रकृतेऽल्पा १० मूलकरणी महती ७ रूपाणि~। एतत्कृतेः ४९ करणीम् ४० अपास्योक्तवज्जाते मूलकरण्यौ ५।२~। एवं जातं मूलं क १० क ५ क २~। अस्य सर्वनियमपूर्वकत्वाच्छुद्धता भवति~। तस्य वर्गः स एव रू १७ क २०० क ८० क ४०~। एवं मतिमद्भिरन्यदप्यूह्यम्~।

\begin{quote}
{\color{violet}दैवज्ञवर्यगणसन्ततसेव्यपार्श्वबल्लालसञ्ज्ञगणकात्मजनिर्मितेऽस्मिन्~।\\
बीजक्रियाविवृतिकल्पलतावतारे व्यक्तिः क्रमेण करणीभवषड्विधस्य~।}
\end{quote}
\vspace{-1mm}

\begin{center}
इति श्रीसकलगणकसार्वभौमश्रीबल्लाळदैवज्ञसुतकृष्णगणकविरचिते\\
बीजविवृतिकल्पलतावतारे करणीषड्विधविवरणम्~। \\
\vspace{1mm}

(अत्र मूलश्लोकैः सह ग्रन्थसङ्ख्या पञ्चनवत्यधिकपञ्चशतानि ५९५~। \\
एवं चतुर्षु षड्विधेषु जाता ग्रन्थसङ्ख्या पुरन्दरशतानि १४००~।)
\vspace{6mm}

\rule{0.2\linewidth}{0.8pt}\\
\vspace{-4mm}

\rule{0.2\linewidth}{0.8pt}
\end{center}
\end{sloppypar}

\newpage
\thispagestyle{empty}

\begin{center}
\textbf{\large ५\; कुट्टकविवरणम्~।}\\
\rule{0.2\linewidth}{0.8pt}
\end{center}

\begin{sloppypar}
एवं सामान्यतोऽव्यक्तक्रियोपयुक्तं षड्विधचतुष्टयमुक्त्वानेकवर्णसमीकरणप्रक्रियोपयुक्तं कुट्टकमाह\textendash \,भाज्यो हारः क्षेपक इत्यादिना~।\\

ननु नेह कुट्टकस्यारम्भो युक्तः~। पाट्यां तस्य निरूपितत्वात्~। न च \hyperref[9.134]{'अन्त्योन्मितौ कुट्टविधेर्गुणाप्ती~। ते भाज्यतद्भाजकवर्णमाने'} इत्यनेकवर्णप्रक्रियोपयुक्तत्वात्तस्यारम्भोऽत्र युक्त इति वाच्यम्~। उपयुक्तत्वाविशेषाद्भिन्नाभिन्नपरिकर्मादित्रैराशिकादिकमप्यत्रारभ्येत~। अथ {\color{violet}'पाट्या च बीजेन च कुट्टकेन वर्गप्रकृत्या च तथोत्तराणि'} इति प्रश्नाध्याये कुट्टकस्य पृथङ्निर्देशात् 

\begin{quote}
{\color{violet}परिकर्मविंशतिं यः सङ्कलिताद्यां पृथग्विजानाति~।\\
अष्टौ च व्यवहाराञ्छायान्तान्भवति गणकः सः~।}
\end{quote}

इति {\color{violet}ब्रह्मगुप्तादिपाटीगणितारम्भे} पाटीस्वरूपकथनेऽनिर्देशाच्च न तस्य व्यक्तान्तर्भूत-त्वमिति व्यक्ते तदारम्भो नावश्यक इति चेदिहाप्यनन्तर्भूतत्वाविशेषादनारम्भ एव युक्त इति~। अत्रोच्यते~। {\color{violet}'त्रुट्यादिप्रलयान्तकालकलनामानप्रभेदः क्रमाच्चारश्च द्युसदां द्विधा च गणितम्'} इति सिद्धान्तलक्षणकथने द्विविधगणितमुक्तं व्यक्तमव्यक्तसञ्ज्ञम् इति~। सिद्धान्तपाठाधिकारिनिरूपणे च गणितस्य द्वैविध्यश्रवणादभ्युपेयमेव तद्द्वैविध्यम्~।~परं कुट्टकस्य कुत्रान्तर्भाव इत्यस्ति संशयः~। तत्र पाटीस्वरूपकथने तदनिर्देशाद्व्यक्ते~तस्या-नावश्यकत्वाच्च न तत्रान्तर्भूतिः किं त्वव्यक्तेऽनेकवर्णप्रक्रियायां तस्यावश्यकत्वात्तत्रै-वान्तर्भावः~। अनेकवर्णमध्यमाहरणे \hyperref[10.149]{'वर्गाद्यं चेत्तुल्यशुद्धौ कृतायां पक्षस्यैकस्योक्तवद्वर्ग-मूलम्~। वर्गप्रकृत्या परपक्षमूलम्'} इत्यावश्यकत्वाद्वर्गप्रकृतेरिव व्यक्ते तदभिधानं त्वव्यक्त-मार्गानपेक्षत्वादव्यक्तगणितानभिज्ञानां तज्ज्ञानार्थं यथा {\color{violet}'बाले मरालकुलमूलदलानि सप्त'} इत्याद्युदाहरणजातस्यैकवर्णमध्यमाहरणविषयस्य विनैवाव्यक्तमार्गं सुखेन ज्ञानार्थं {\color{violet}'गुणघ्न-मूलोनयुतस्य'} इत्यादेः~। {\color{violet}'पाट्या च बीजेन च कुट्टकेन'} इति पृथङ्निर्देशस्तु तदतिशयार्थः~। यथा प्रमाणप्रमेयेत्यादिन्यायसूत्रे प्रमेयान्तर्गतत्वेऽपि प्रमाणादीनां पृथङ्निर्देशस्तथा वा बीज-चतुष्टयनिरपेक्षतयैव प्रश्नोत्तरार्थज्ञानहेतुत्वाद्वा~। तदेवं युक्तोऽत्र कुट्टकारम्भः~। तत्र कुट्टको नाम गुणकः~। हिंसावाचकशब्दैर्गुणनाभ्युपगमात्~। योगरूढ्या गुणकविशेषश्चायम्~। कश्चिद्राशिर्येन गुणित उद्दिष्टक्षेपयुतोन उद्दिष्टहरेण भक्तः सन्निःशेषो भवेत्स गुणकः कुट्टक इति पूर्वेषां व्यपदेशात्~। तत्र कुट्टकज्ञानार्थं प्रथमविधेयमुद्देशखिलत्वं च शालिन्या निरू-पयति\textendash
\end{sloppypar}

\newpage

\begin{sloppypar}
\phantomsection \label{5.50}
\begin{quote}
{\large \textbf{{\color{purple}भाज्यो हारः क्षेपकश्चापवर्त्यः केनाप्यादौ सम्भवे कुट्टकार्थम्~।\\
येन च्छिन्नौ भाज्यहारौ न तेन क्षेपश्चैतद्दुष्टमुद्दिष्टमेव~॥~५०~॥}}}
\end{quote}

कश्चिद्राशिर्येन गुणित उद्दिष्टक्षेपयुतोन उद्दिष्टहरेण भक्तः सन्निःशेषो भवति तस्य गुणकस्य कुट्टक इति सञ्ज्ञेत्युक्तं प्राक्~। अत्रागता लब्धिर्लब्धिसञ्ज्ञैव~। हरो हरसञ्ज्ञ एव~। क्षेपोऽपि क्षेपसञ्ज्ञ एव~। अन्वर्थसञ्ज्ञाश्चैताः~। यो राशिर्गुण्यते तस्य भाज्य इति सञ्ज्ञा~। भजनयोगात्~। अस्य कुट्टकस्य ज्ञानार्थमादौ स भाज्यो हारः क्षेपकश्च केनापि तुल्येनाङ्केनापवर्त्यः~। भाज्यहारक्षेपा एकेनैवापवर्त्या इत्यर्थः~। कस्मिन्सति~। अपवर्तनसम्भवे सति~। अपवर्तनं नाम निःशेषभजनम्~। तच्चैकातिरिक्तेनाभिन्नेन द्रष्टव्यम्~। अन्यथा सति सम्भव इति विरुध्येत~। एकेन भिन्नेन वा केनचिदङ्केन सर्वत्रापवर्तनसम्भवात्~। \hyperref[5.51]{'तौ भाज्यहारौ दृढसञ्ज्ञकौ स्तः'} इत्यस्य विवरणे 'दृढा' इत्यन्वर्थसञ्ज्ञा~। पुनर्नापवर्तन्ते न क्षीयन्त इत्यर्थ इति व्याख्यातवद्भिः श्रीगणेशदैवज्ञचरणैरप्युक्त एवायमर्थः~। यत्त्वर्धेनापवर्त्येत्यादि क्वचित् दृश्यते तद्द्विगुणत्वादिपरम्~। भाज्यहारक्षेपाणामपवर्तनसम्भवे सत्यवश्यमपवर्त्या एव~। अन्यथा कुट्टकसिद्धिर्न सम्भवतीत्यर्थसिद्धम्~। उद्देशस्य खिलत्वज्ञानार्थमाह\textendash \,येनेति~। येनाङ्केन भाज्यहारौ छिन्नावपवर्तितौ तेनैवाङ्केन क्षेपश्चेन्न च्छिन्नोऽपवर्तितो न स्यात्तदा तदुद्दिष्टं पृच्छकेनं पृष्टं दुष्टमेव~। अयं भाज्यो येन केनापि गुणितस्तेन क्षेपेण युतोनस्तेन हरेण भक्तः सन्कदाचिदपि निःशेषो न भवेदित्यर्थः~॥~५०~॥\\

{\small अथापवर्ताङ्कं कुट्टकेतिकर्तव्यतां चोपजातिकात्रयेणाह\textendash }

\phantomsection \label{5.51}
\begin{quote}
{\large \textbf{{\color{purple}परस्परं भाजितयोर्ययोर्यः शेषस्तयोः स्यादपवर्तनं सः~।\\
तेनापवर्तेन विभाजितौ यौ तौ भाज्यहारौ दृढसञ्ज्ञकौ स्तः~।\\
मिथो भजेत्तौ दृढभाज्यहारौ यावद्विभाज्ये भवतीह रूपम्~।\\
फलान्यधोऽधस्तदधो निवेश्यः क्षेपस्तथान्ते खमुपान्तिमेन~॥\\
स्वोर्ध्वे हतेऽन्त्येन युते तदन्त्यं त्यजेन्मुहुः स्यादिति राशियुग्मम्~।\\
ऊर्ध्वो विभाज्येन दृढेन तष्टः फलं गुणः स्यादपरो हरेण~॥~५१~॥}}}
\end{quote}

ययो राश्योः परस्परं भाजितयोः सतोर्यः शेषोऽङ्कः स तयोरपवर्तनं स्यात्~। तेन तौ निःशेषं भज्येते एव~। एतदुक्तं भवति~। हरेण भाज्ये भक्ते यच्छेषं तेनापि स हरो भाजनीयः~। तच्छेषेणापि भाज्यशेषं तेनापि हरशेषमिति पुनः पुनः परस्परभजने क्रियमाणे यद्यन्ते रूपं शेषं स्यात्तदा तौ नापवर्तेते एव~। रूपस्यैव शेषत्वात्~। तेनापवर्ते भाज्यहारक्षेपाणामविकार एव~। यदा तु शून्यं शेषं स्यात्तदा हरीभूतं यत्प्राक् शेषमधः स्थापितं तदेव भाज्यहरयोरपवर्तनं स्यात्~। शेषो ह्यपवर्ताङ्कः~। तस्मादन्तिमशेषोऽङ्क एवापवर्तनाङ्कः~। शून्यं शेषमिति तु शेषा-भावपरम्~। अन्यथापवर्तनं नाम निःशेषभ-
\end{sloppypar}

\newpage

\begin{sloppypar}
\noindent जनमिति विरुध्येत~। तत्रापि शून्यशेषत्वात्~। एवं ज्ञातेनापवर्ताङ्केन यौ भाज्यहारौ विभाजितौ दृढसञ्ज्ञकौ स्तः~। तेनैव क्षेपोऽप्यपवर्त्यः~। \hyperref[5.50]{'भाज्यो हारः क्षेपकश्चापवर्त्य'} इत्युक्तत्वात्~। सोऽपि दृढसञ्ज्ञः स्यात्~। 'दृढा' इत्यन्वर्थसञ्ज्ञा~। पुनर्नापवर्तन्ते न क्षीयन्त इत्यर्थः~। 'दृढौ' इति सञ्ज्ञां वदता कृतेऽप्यपवर्ते यावदन्यदपवर्तनं सम्भवति तावदपवर्तनीयाविति ज्ञापितम्~। पुनरपवर्तनं च स्वकल्पिताङ्केनापवर्ते कृते~। अन्यथा परस्परं भाजितयोरित्यादिना ज्ञातेनापवर्ताङ्केनापवर्ते पुनरपवर्तनासम्भवात्~। अथ तौ दृढभाज्यहारावुक्तवन्मिथः परस्परं तावद्भजेत्~। तावत्कथम्~। यावद्विभाज्ये भाज्यस्थाने रूपं भवति~। इहैतेषु परस्परभजनेष्वागतानि फलान्यधोऽधो निवेश्यानि~। फलं च फले च फलानि च फलानि~। द्वन्द्वैकशेषः~। एकमेव फलं लब्ध्वा यदि रूपं शेषं स्यात्तदा तदेकमेव फलं स्थाप्यम्~। द्वे चेत्तर्हि द्वे स्थाप्ये~। बहूनि चेत्तर्हि बहूनि स्थाप्यानीत्यर्थः~। तेषां फलानां वल्लीवदधोऽधः स्थापितानामधः क्षेपो निवेश्यः~। दृढ इति पूर्वानुवृत्तिः~। तथेति पदाद्वा दृढत्वं क्षेपस्यावगन्तव्यम्~। अस्मिन्पक्षे तथेति पदस्य नाग्रेऽन्वयः~। तथा तेषामप्यधोऽन्ते खं निवेश्यम्~। एवं वल्ली जायते~। तत उपान्तिमेनाङ्केन स्वोर्ध्वे स्वोर्ध्वस्थितेऽङ्के हतेऽन्त्येनाङ्केन युते च सति तदन्त्यं त्यजेत्~। इति मुहुरुपान्तिमेन स्वोर्ध्वे हतेऽन्त्येन युते तदन्त्यं त्यजेदिति पुनः पुनः कृते राशियुग्मं स्यात्~। तत्रोर्ध्वराशिर्दृढेन विभाज्येन तष्टः सन्फलं भवेत्~। फलं नाम लब्धिः~। अपरोऽधस्तनो राशिर्दृढेन हरेण तष्टः सन्गुणः स्यात्~। तक्षू त्वक्षू तनूकरणे~। कर्मणि क्तः~। तष्टस्तनूकृतः कृशीकृतोऽवशेषित इति यावत्~। भक्त्वावशेषितराशिर्ग्राह्यो न तु लब्धमित्यर्थः~। तेन गुणेन दृढभाज्ये गुणिते दृढक्षेपयुतोने दृढहरेण भक्ते शेषं न स्यादिति~। उद्दिष्टेष्वपि भाज्यहारक्षेपेषु ते एव गुणलब्धी स्त इत्यर्थसिद्धमविशेषात्~॥~५१~॥\\

{\small अथागतफलेषु विषमेषु सत्सु विशेषमुपजातिकयाह\textendash }

\phantomsection \label{5.52}
\begin{quote}
{\large \textbf{{\color{purple}एवं तदैवात्र यदा समास्ताः स्युर्लब्धयश्चेद्विषमास्तदानीम्~।\\
यथागतौ लब्धिगुणौ विशोध्यौ स्वतक्षणाच्छेषमितौ तु तौ स्तः~॥~५२~॥}}}
\end{quote}

एवं तदैव स्यात्~। यदात्र परस्परभजने ता आगता लब्धयः समाः स्युर्द्वे चतस्रः षडित्यादयः~। यदि तु ता लब्धयो विषमाः स्युरेका तिस्रः पञ्च वेत्यादयस्तदानीमुक्तप्रकारेण यथागतौ लब्धिगुणौ तौ स्वतक्षणाच्छोध्यौ शेषतुल्यौ तौ लब्धिगुणौ स्तः~। तक्ष्यते तनू-क्रियतेऽनेनेति तक्षणः~। तक्ष्णोतीति तक्षण इति वा~। स्वश्चासौ तक्षणश्च स्वतक्षणः तस्मात्~। गुणो दृढहाराच्छोध्यो लब्धिर्दृढभाज्याच्छोध्येत्यर्थः~।\\

\hyperref[5.50]{'भाज्यो हारः क्षेपकश्चापवर्त्यः'} इत्यत्र तावदियं युक्तिः\textendash \,अनपवर्तितयोर्ययोर्भाज्यभाज-कयोर्यादृशी लब्धिस्तयोः केनचिदेकेनाङ्केन गुणितयोरपवर्तितयोर्वा तादृगेव लब्धिरिति तु
\end{sloppypar}

\newpage

\begin{sloppypar}
\noindent प्रसिद्धम्~। प्रकृते तु कल्पितभाज्यः केनचिद्गुणकेन गुणितो धनर्णक्षेपयुतः सम्भाज्यः स्यात्~। भाजकस्तु यथास्थित एव~। तथा चात्र भाज्यस्य खण्डद्वयम्~। गुणगुणितकल्पित-भाज्य एकं क्षेपो द्वितीयम्~। अनयोर्योगे भाज्ये सिद्धे भाज्यभाजकयोरपवर्ते कृतेऽपि वा अस्ति लब्धिवैलक्षण्यम्~। तस्माद्येन भाजकोऽपवर्तितस्तेन खण्डद्वययोगलक्षणो~भाज्यः अप्यपवर्त्यः~। तत्र योगापवर्तनेऽपवर्तितखण्डकयोर्योगे वा तुल्यतैव स्यात्~। यथा भाज्यभाजकौ $\dfrac{{\footnotesize{\hbox{२७}}}}{{\footnotesize{\hbox{१५}}}}$ त्रिभिरपवर्ते जातौ $\dfrac{{\footnotesize{\hbox{९}}}}{{\footnotesize{\hbox{५}}}}$~। यद्वा भाज्यखण्डे ९।१८ अनयोस्त्रिभिः अपवर्ते ३।६ योगे च जातः स एवापवर्तितभाज्यः ९~। एवमन्यादृगपि खण्डद्वयं बहूनि वा खण्डानि विधायापवर्त्य तद्योगेऽपवर्तितभाज्य एव स्यात्~। तस्माद्भाजकस्यापवर्तने गुणगुणितकल्पितभाज्योऽपवर्त्यः क्षेपोऽप्यपवर्त्यः~। तत्र यद्यपि गुणस्याज्ञातत्वाद्गुणगुणित-भाज्यस्याप्यज्ञाने तस्यापवर्तनमशक्यं तथापि कल्पितभाज्येऽपवर्तिते पश्चाद्गुणकेन गुणिते गुणगुणितकल्पितभाज्यलक्षणो भाज्यखण्ड एवापवर्तितः~स्यात्~। गुणितस्यापवर्तनेऽप-वर्तितस्य वा गुणनेऽविशेषात्~। तथा च कल्पितभाज्यो येन गुणेन गुणितः सन् भाज्य-खण्डं भवत्यपवर्तितभाज्योऽपि तेनैव गुणेन गुणितः सन्नपवर्तितं भाज्यखण्डं भवेत्~। अपवर्तितक्षेपश्च द्वितीयम्~। तदेवं भाज्यहारक्षेपाणामनपवर्तितानामपवर्तितानां च गुण-लब्ध्योरविशेषाल्लाघवाच्च \hyperref[5.50]{'भाज्यो हारः क्षेपकश्चापवर्त्यः'} इत्युक्तम्~। अपवर्तनमावश्यकं न वेति \hyperref[5.51]{'मिथो भजेत्तौ दृढभाज्यहारौ'} इत्यादेरुपपत्तौ विचारयिष्यते~।\\

अथ खिलत्वोपपत्तिः~। इह भाज्यभाजकयोरपवर्ते यद्यपि न तल्लब्धेर्वैचित्र्यं तथापि शेषस्य तदस्त्येव~। अपवर्तितयोः शेषमपवर्ताङ्केन गुणितं सदनपवर्तितयोः शेषं स्यात्~। यथा भाज्यभाजकौ $\dfrac{{\footnotesize{\hbox{२१}}}}{{\footnotesize{\hbox{१५}}}}$ त्रिभिरपवर्तितौ $\dfrac{{\footnotesize{\hbox{७}}}}{{\footnotesize{\hbox{५}}}}$~। अत्रैकगुणे भाज्ये स्वस्वहरभक्ते सति शेषे ६।२ द्विगुणिते भाज्ये स्वस्वहरभक्ते सति शेषे १२।४ त्रिगुणितस्य शेषे ३।१ चतुर्गुणितस्य ९।३ पञ्चगुणितस्य ०।० षडादिभिर्गुणने पुनस्तान्येव शेषाणि स्युः~। तस्मात् अत्र गुणकमात्रेऽपवर्तितहरेऽस्मिन् ५ शेषं ०।१।२।३।४ एभ्योऽन्यन्न स्यात्~। अनपवर्तित-हरे १५ तु शेषं ०।३।६।९।१२ एभ्योऽन्यन्न स्यात्~। अत्र सर्वेषामपि शेषाणामेकादि-गुणितापवर्ताङ्करूपत्वादपवर्तः स्यादेव~। अथ क्षेपविचारः~। तत्र शून्यशेषे गुणके क्षेपा-भाव एकादिगुणितहरतुल्ये वा क्षेपे शून्यं शेषं स्यान्नान्यस्मिन्क्षेपे~। तथा च हरस्या-पवर्तनसम्भवे क्षेपस्य सुतरामपवर्तनसम्भवः~। अथान्यशेषेषु सकलगुणकेषु शेषतुल्य ऋण-क्षेपे शेषोनहरतुल्ये धनक्षेपे वैकादिगुणितहरयुतयोरुभयोर्वा शून्यं शेषं स्यात्~। नान्यस्मिन् क्षेपे~। अत्र शेषतुल्यस्य शेषोनहरतुल्यस्य वा क्षेपस्योक्तशेषेष्वेवान्तर्भावादपवर्तः स्यादेव~। एवं केवलस्यापवर्तसम्भवे हरयुतस्य क्षेपस्य सुतरामपवर्तसम्भवः~। तदेवं न कमपि तादृशं क्षेपं पश्यामो यो भाज्यहरापवर्ताङ्केन नाप-
\end{sloppypar}

\newpage

\begin{sloppypar}
\noindent वर्तेत~। तस्माद्यत्र क्षेपेऽपवर्तो न स्यात् तादृशक्षेपे शून्यशेषता कथमपि न स्यात्~। शून्यशेषक्षेपाणामुक्तरीत्या नियतत्वादित्यलं पल्लवितेन~। तस्मात् \hyperref[5.50]{'येन च्छिन्नौ भाज्यहारौ न तेन क्षेपश्चैतद्दुष्टमुद्दिष्टमेव'} इति सुष्ठूक्तम्~। अथापवर्ताङ्कज्ञानार्थं युक्तिः~। अपवर्ताङ्कश्चा-त्रापवर्ताङ्केषु महान् ज्ञातव्यो येनापवर्तितयोर्भाज्यभाजकयोः पुनर्नापवर्तः स्यात्~। अनेना-पवर्तितयोर्दृढत्वोक्तेः~। अथ तज्ज्ञानार्थमुपायः~। तत्र भाज्यभाजकयोस्तुल्यत्वे तन्मित एव महानपवर्ताङ्क इति मन्दैरप्यवगम्यते~। तयोर्वैलक्षण्ये तु स विचारपदवीमारोढुमर्हति~। तत्र द्वयोः $\dfrac{{\footnotesize{\hbox{२२१}}}}{{\footnotesize{\hbox{१९५}}}}$ मध्ये यः १९५ लघुस्ततोऽधिकोऽपवर्ताङ्को नैव स्यात्~। तेनाङ्केन लघोः अपवर्तनस्य बाधितत्वात्~। लुघुतुल्यस्तु स्यात्~। यदि लुघुना महति भक्ते शेषं न स्यात्~। निःशेषभजनरूपत्वात्तस्य~। यदि च शेषं २६ स्यात्तदा न लघुतुल्योऽपवर्ताङ्कः~। किं त्वधिकस्य बाधितत्वाल्लघोरपि लघुर्महानपर्वताङ्कः स्यात्~। तत्रापि विचारः~। अत्र हि महतो राशेः खण्डद्वयम्~। यावल्लघुना भक्तं तावदेकं १९५ शेषेतुल्यमपरं २६~। एवं सति लघुतो न्यूनाङ्केषु मध्ये यः शेषतः २६ अधिकस्तस्य नास्त्येवापवर्तकत्वम्~। तेन यथाकथञ्चिल्लघोः १९५ अपवर्ते लघुराशिभक्तस्याधिकराशिखण्डस्य १९५ अप्यपवर्तः स्यान्न तु शेषतुल्यद्वितीयखण्डस्य २६~। तथा च लघुतः १९५ न्यूनाङ्केषु यदि महानपवर्ताङ्कः स्यात्तर्हि शेषतुल्यः २६ तथा च लघुः स्यात्~। परं शेषेण २६ लघुराशौ १९५ भक्ते यदि शेषं न स्यात्तथा सति शेषतुल्याङ्केन लघोरपवर्तनस्य जातत्वाल्लघुभक्तस्याधिकराशिखण्डस्य १९५ शेषतुल्यद्वितीयखण्डस्य २६ अप्यपवर्तः स्यात्~। यदि तु शेषं स्यात्तर्हि पूर्वशेषतः २६ न्यून एव महानपवर्ताङ्कः स्यान्नाधिकः~। अधिकस्य बाधितत्वात्~। अथ तत्रापि विचारः~। लघुराशेर्हि खण्डद्वयं १८२।१३ यावत्पूर्वशेषेण भक्तं तावदेकं १८२ द्वितीयशेषतुल्यं द्वितीयम् १३~। एवं सति पूर्वशेषेान्न्यूनाङ्केषु यो द्वितीयशेषादधिकः स्यान्न स्यादयमपवर्ताङ्कः~। तेन यथाकथञ्चित्पूर्वशेषस्य २६ अपवर्ते शेषभक्तलघुखण्डकस्य १८२ अपवर्तः स्यान्न द्वितीयशेषतुल्यद्वितीयखण्डस्य १३~। तथा सति लघुराशेरनपवर्तनाल्लघुभक्तस्याधिकराशिखण्डस्य १९५ अप्यनपवर्ते कस्याप्यपवर्तो न स्यात्~। तस्मात्पूर्वशेषतः २६ न्यूनाङ्केषु यदि महानपवर्ताङ्कः स्यात्तर्हि द्वितीयशेषतुल्यः १३ एव स्यात्~। परं द्वितीयशेषेण १३ पूर्वशेषे २६ भक्ते यदि शेषं न स्यात्~। यतस्तथा सति पूर्वशेषस्य २६ अपवर्तस्य जातत्वात् २६ तद्भक्तस्य लघुराशिखण्डस्य १८२ अथ च द्वितीयशेषतुल्यद्वितीयखण्डस्य २६ अप्यपवर्तः स्यात्~। तथा सति लघुराशेः १९५ अपवर्तनस्य जातत्वाल्लघुभक्तस्याधिकराशिखण्डस्य १९५ अप्यपवर्तः स्यात्~। पूर्व-शेषतुल्यस्य द्वितीयखण्डस्य २६ अप्यपवर्तोऽनुपदमेव परं यदीति ग्रन्थेन प्रतिपादित इत्य-धिकराशेरप्यपवर्तः स्यादेव~। यदि च द्वितीयशेषेण पूर्वशेषे भक्ते शेषं स्यात्तर्ह्यनयैव युक्त्या तृतीय-
\end{sloppypar}

\newpage

\begin{sloppypar}
\noindent शेषतुल्यो महानपवर्ताङ्कः स्यात्~। एवमनयोपपत्त्या पूर्वपूर्वशेष उत्तरोत्तरेण येन शेषेण भक्ते शेषं न स्यात्तच्छेषं महानपवर्ताङ्कः स्यात्~। तदेवमुपपन्नं \hyperref[5.51]{'परस्परं भाजितयोर्ययोर्यः शेषस्तयोः स्यादपवर्तनं सः'} इति~।\\

अथ \hyperref[5.51]{'मिथो भजेत्तौ दृढभाज्यहारौ'} इत्यादावुपपत्तिः~। क्षेपाभावे तावच्छून्येन भाज्ये गुणिते हरभक्ते शेषं न स्यादिति शून्यमेव गुणो लब्धिश्च~। यदि वा हरतुल्ये गुणे गुणहरयोस्तुल्यत्वान्नाशे भाज्यतुल्या लब्धिः स्याच्छेषं च न स्यात्~। एवं द्व्यादिगुणितहरतुल्ये गुणे हरेण गुणहरयोरपवर्ते गुणस्थाने द्व्यादयः स्युरिति द्व्यादिगुणितभाज्यतुल्या लब्धिः स्याच्छेषं च न स्यात्~। तस्मात्क्षेपाभावे शून्यमिष्टाहतहरो वा गुणः~। लब्धिस्तु शून्यमिष्टाहतभाज्यो वेति~। एवमत्र हरतुल्यो गुणोपचयो भाज्यतुल्यो लब्ध्युपचयः सर्वत्र~। अत एव वक्ष्यति~। अथ \hyperref[5.59]{'इष्टाहतस्वस्वहरेण युक्ते ते वा भवेतां बहुधा गुणाप्ती'} इति~। अथ सत्यपि क्षेपे हरतुल्ये द्व्यादिगुणितहरतुल्ये वा तस्मिन्पूर्वोक्त एव शून्यादिको गुणः स्यात्~। सति हि पूर्वोक्तगुणके क्षेपवशादेव शेषं स्यात्~। क्षेपोऽपि यद्येकादिगुणितहरतुल्यः स्यात्तर्हि शेषं कुतः स्यात्~। तस्मादेतादृशे क्षेपे सत्यपि पूर्वोक्त एव गुणः~। लब्धौ तु हरभक्ते क्षेपे यल्लभ्यते तावदधिकं स्यात् धनक्षेपे~। ऋणक्षेपे तु तावन्न्यूनं स्यात्~। अत एव वक्ष्यति \hyperref[5.58]{'क्षेपाभावोऽथवा यत्र क्षेपः शुध्येद्धरोद्धृतः~। ज्ञेयः शून्यं गुणस्तत्र क्षेपो हरहृतः फलम्'} इति~। अथान्यथा क्षेपे भाज्यखण्डद्वयेनोपपत्तिः~। हरेण यावद्भाज्यं तावदेकं शेषमपरम्~। यथा भाज्यभाजकौ $\dfrac{{\footnotesize{\hbox{१६}}}}{{\footnotesize{\hbox{७}}}}$ उक्तवज्जाते भाज्यखण्डे १४।२~। अत्र पूर्वखण्डस्य हरेण निःशेषभजनाद्येन केनापि गुणकेन गुणितस्यापि तस्य निःशेषभजनं स्यादेव~। अथोद्दिष्टक्षेपः परखण्डेन भक्तः सन्यदि शुध्येत्तर्ह्यत्र या लब्धिः स एव गुणकः स्यात् परं वियोगे~। यतस्तेन गुणकेन गुणितस्य भाज्यापरखण्डस्य क्षेपसमत्वनियमात्क्षेपवियोगे नाशः स्यादेव~। अथ यदि न शुध्येत्तर्ह्यशक्यो गुणकावगमः~। अतोऽन्यथा यतितव्यम्~। भाजकेन भाज्ये भक्ते यदि रूपं शेषं स्यात्तर्हि द्वितीयखण्डमपि रूपं स्यात्~। तथा सति येन केनापि क्षेपेण तस्य गुणने क्षेपसमत्वनियमादुक्तयुक्त्या क्षेपसम एव गुणः परं वियोगे~। योगे तु क्षेपोनहरो गुणः~। यतस्तेन गुणितं भाज्यापरखण्डं क्षेपो न हरसमं स्यादस्य च क्षेपयोगे हरसमता स्यादिति हरेण निःशेषभजनं स्यादेव~। लब्धिस्तु केवलभाज्ये हरभक्ते या स्यात्सैव गुणगुणिता सती गुणितभाज्यं स्यात्~। परं वियोगे योगे तु तादृशी सैका~। परखण्डस्य शुद्ध्यभावाद्धरतुल्यशेषत्वाच्च~। अथ यदि भाज्ये हरेण भक्ते रूपं शेषं न स्यात्तर्हि गुणकावगमो दुर्गमः~। अतो भाज्यशेषेण हरं भजेत्~। अत्र च हरो भाज्यः~। भाज्यशेषं भाजकः~। अत्रापि यदि रूपं शेषं स्यात्तर्हि क्षेपतुल्यो गुणो वियोगे~। योगे तु क्षेपोनहरो गुणः पूर्ववल्लब्धिश्च~। उक्तयुक्ते-
\end{sloppypar}

\newpage

\begin{sloppypar}
\noindent रविशेषात्~। अत्रापि यदि रूपं शेषं न स्यात्तर्हि नास्ति गुणकानुगमः सुगमः~। तस्मात् अस्यापि शेषेण हरीभूतं शेषं भजेत्~। तत्र यदि रूपं शेषं स्यात्तर्हि तस्मिन्भाज्य उक्तयुक्त्या क्षेपाङ्कतुल्यः क्षेपोनहरतुल्यश्च गुणः स्याद्वियोगयोगयोः~। अत्रापि रुपाधिके शेषे गुणो दुर्गमः~। तस्मात्परस्परभजने सति कुत्रचिद्रूपं शेषमपेक्षितम्~। तच्च सत्य-पवर्तनसम्भवे भाज्यभाजकयोरनपवर्ते कथं स्यात्~। किं तु तत्रापवर्ताङ्कतुल्यं शेषं स्यात् परस्परभजनेऽन्त्यशेषस्यैवापवर्ताङ्कत्वात्~। कृते त्वपवर्ते शेषमप्यपवर्ताङ्केनापवर्तितं स्यात्~। अन्त्यशेषं त्वपवर्ताङ्कतुल्यम्~। तच्चेदपवर्ताङ्केनापवर्तितं स्याद्रूपमेवान्त्यशेषं स्यादिति जातं भाज्यभाजकयोरपवर्तस्यावश्यकत्वम्~। ननु यद्यप्युपान्तिमशेषतुल्ये भाज्ये पूर्वशेषेण भक्ते रूपं शेषं स्यादिति ज्ञातस्तस्मिन्गुणस्तथाप्युद्दिष्टभाज्ये कथं गुणकसिद्धिरिति चेत्~। व्यस्त-विधिना तमवगच्छ~। तथा हि\textendash \,भाज्यभाजकक्षेपाः
\vspace{-1mm}

\begin{center}
भा १२११ क्षे २१~।\\
ह ४९७ ~~~~~~~~~~~~
\end{center}
\vspace{-1mm}

\noindent अत्र परस्परं भाजितयोर्भाज्यभाजकयोरन्त्यशेषं ७~। अनेनापवर्तिता भाज्यहरक्षेपाः~। 
\vspace{-1mm}

\begin{center}
भा १७३ क्षे ३~।\\
ह ७१ ~~~~~~~~~~~
\end{center}
\vspace{-1mm}

\noindent अत्र दृढयोरेतयोर्भाज्यभाजकयोः परस्परभजनाल्लब्धिशेषयोर्वल्ल्यौ
\vspace{-1mm}

\begin{center}
ल \hspace{6mm} शे\\
२ \hspace{5mm} ३१\\
२ \hspace{7mm} ९\\
३ \hspace{7mm} ४\\
२ \hspace{7mm} १
\end{center}
\vspace{-1mm}

\noindent क्रमेण भाज्यभाजकाश्च जाताः

\begin{center}
\begin{tabular}{rrrr}
भा ~१७३~। & भा ~७१~। & भा ~३१~। & भा ~९\\
ह ~~~७१~। & ह ~~~३१~। & ह ~~~९~। & ह ~४
\end{tabular}
\end{center}

\noindent अत्रान्त्यभाज्ये खण्डद्वयम्~। यावद्धरभक्तं तावदेकं शेषमपरम्~। एवं खण्डे ८।१~। उक्तयुक्त्या वियोगे जातः क्षेपसमो गुणः ३ केवलभाज्यलब्धिर्गुणगुणिता सती लब्धिः स्यादिति प्रकृ-तेऽन्त्यभाज्यलब्धिः २ गुणेनानेन ३ गुणिता लब्धिश्च ६~। तदिदमुक्तं~। \hyperref[5.51]{'मिथो भजेत्तौ दृढ-भाज्यहारौ यावद्विभाज्येभवतीह रूपम्~। फलान्यधोऽधस्तदधो}
\end{sloppypar}

\newpage

\begin{sloppypar}
\noindent \hyperref[5.51]{निवेश्यः क्षेपः'} इति~।
\vspace{-1mm}

\begin{center}
फ~~~~~ \\
२~~~~~ \\
२~~~~~ \\
३~~~~~ \\
२~~~~~ \\
३ क्षे 
\end{center}
\vspace{-1mm}

एवमत्रान्त्यो जातो गुणः~। अन्त्येन हतः स्वोर्ध्वो लब्धिश्चेति जातम्~।
\vspace{-1mm}

\begin{center}
२ \\
२ \\
३ \\
ल ६~~~~~ \\
गु ३ क्षे 
\end{center}
\vspace{-1mm}

\noindent अथास्मिन्नेव क्षेपेऽस्मात्पूर्वभाज्येऽस्मिन्
\vspace{-1mm}

\begin{center}
भा ३१ \\
ह ~९
\end{center}
\vspace{-1mm}

\noindent गुणो विचार्यते~। अत्राप्युक्तवत्खण्डे २७।४ अत्र पूर्वखण्डं येन केनापि गुणितं हरभक्तं निःशेषं स्यादेव~। अतोऽपरखण्डादेव गुणविचारो युक्तः~। अतो जातौ भाज्यभाजकौ $\dfrac{{\footnotesize{\hbox{४}}}}{{\footnotesize{\hbox{९}}}}$~। अत्रान्त्यभाज्यभाजकयोर्व्यत्यासोऽस्तीति गुणलब्ध्योरपि व्यत्यासमात्रम्~। तत्र युक्तिः~। भाज्ये ९ गुणेन ३ गुणिते २७ क्षेपेण ३ वियुक्ते २४ हरेण ४ भक्ते सति लब्धिः ६ भवति~। अतो व्यस्तविधिना लब्ध्या ६ हरे ४ अस्मिन् गुणिते २४ क्षेप\textendash \,३\textendash \,युते २७ भाज्य\textendash \,९\textendash \,भक्ते लब्धो गुणः ३~। तदेवं पर्यवस्यति~। अयं भाज्यः ४ तस्य लब्ध्या ६ गुणितः २४ तेन क्षेपेण ६ युतः २७ स्वहरेणानेन ९ भक्तः सञ्छुध्यतीत्यन्त्यभाज्यलब्धिरेव ६ अत्र गुणका लब्धिश्वान्त्यभाज्यगुणः ३~। एवं वल्ल्यां जातं
\vspace{-2mm}

\begin{center}
~~~२ \\
~~~२ \\
~~~३ \\
गु ६ \\
ल ३
\end{center}
\end{sloppypar}

\newpage

\begin{sloppypar}
\noindent परमत्र भाज्ये पूर्वखण्डलब्धिर्गुणगुणिता सती स्यात्~। गुणश्चात्र वल्ल्यामुपान्तिमः ६ पूर्व-खण्डलब्धिश्च तदूर्ध्वं तिष्ठति ३~। अत उपान्तिमेन स्वोर्ध्वे हते जाता पूर्वखण्डलब्धिः १८ द्विती-यखण्डलब्धिश्च वल्ल्यामन्त्या ३ अतस्तया युता पूर्वखण्डलब्धिः १८ अस्मिन्भाज्ये सकला लब्धिः स्यात् २१~। एवं जातं वल्ल्याम्
\vspace{-1mm}

\begin{center}
~~~~२\\
~~~~२ \\
ल ~२१\\
गु ~~६\\
ल ~~३
\end{center}
\vspace{-1mm}

\noindent अस्मिन्भाज्ये गुणलब्ध्योः सिद्धत्वादधःस्थलब्धेः प्रयोजनाभावादपगमे जातं वल्ल्यां
\vspace{-1mm}

\begin{center}
~~~~२\\
~~~~२\\
ल ~२१ \\
गु ~~६
\end{center}
\vspace{-1mm}

\noindent तदिदमुक्तं \hyperref[5.51]{'उपान्तिमेन स्वोर्ध्वे हतेऽन्त्येन युते तदन्त्यं त्यजेत्'} इति~। एवमस्मिन्भाज्ये $\dfrac{{\footnotesize{\hbox{३१}}}}{{\footnotesize{\hbox{९}}}}$ व्यस्तविधिना जातौ लब्धिगुणौ २१।६ योगेऽथ तदूर्ध्वभाज्येऽस्मिन्
\vspace{-1mm}

\begin{center}
भा ~७१ \\
ह ~~३१
\end{center}
\vspace{-1mm}

\noindent तस्मिन्नेव क्षेप\textendash \,३\textendash \,गुणो विचार्यते~। अत्राप्युक्तवत्खण्डे ६२।९ पूर्वखण्डं पृथक्संस्थाप्य जातौ भाज्यहरौ $\dfrac{{\footnotesize{\hbox{९}}}}{{\footnotesize{\hbox{३१}}}}$ अत्राप्यनुपदं प्रदर्शितयोर्भाज्यभाजकयोर्व्यत्यासाल्लब्धिगुणव्यत्यासमात्रम्~। व्यस्तविधेस्तुल्यत्वात्~। तथा जातं वल्ल्यां
\vspace{-1mm}

\begin{center}
~~~~२\\
~~~~२ \\
गु ~२१\\ 
ल ~~६
\end{center}
\vspace{-1mm}

\noindent अत्रापि पूर्वखण्डलब्धिर्गुणगुणिता स्यात्~। गुणोऽत्राप्युपान्तिमः~। तदूर्ध्वे च पूर्वखण्डलब्धिः २~। अत उपान्तिमेन स्वोर्ध्वे हते जाता पूर्वखण्डलब्धिः ४२ इयं द्वितीयखण्ड-
\end{sloppypar}

\newpage

\begin{sloppypar}
\noindent लब्ध्यात्मकेनान्त्येन ६ युता जाता सम्पूर्णा लब्धिः ४८~। एवं जातं वल्ल्याम्
\vspace{-1mm}

\begin{center}
~~~~~२\\
ल ~४८\\
गु ~२१ \\
ल ~~६
\end{center}
\vspace{-1mm}

\noindent अत्राप्यधःस्थलब्धेः प्रयोजनाभावादपगमे जातम्
\vspace{-1mm}

\begin{center}
~~~~~२\\
ल ~४८\\
गु ~२१ 
\end{center}
\vspace{-1mm}

\noindent एवमस्मिन्भाज्ये $\dfrac{{\footnotesize{\hbox{७१}}}}{{\footnotesize{\hbox{३१}}}}$ व्यस्तविधिना जातौ वियोगे लब्धिगुणौ ४८।२१~। अथ तदूर्ध्वे भाज्ये मुख्येऽस्मिन् १७३ गुणविचारः~। अत्राप्युक्तवत्खण्डे १४२।३१ कृत्वा जातौ भाज्यभाजकौ $\dfrac{{\footnotesize{\hbox{३१}}}}{{\footnotesize{\hbox{७१}}}}$~। अत्राप्यनुपदं सिद्धगुणयोर्भाज्यभाजकयोर्व्यत्यासाल्लब्धिगुणयोः क्षेपस्य च व्यत्यासे जातौ क्षेपयोगे लब्धिगुणौ २१।४८~। जातं वल्ल्याम्
\vspace{-1mm}

\begin{center}
~~~~~२\\
गु ~४८ \\
ल ~२१
\end{center}
\vspace{-1mm}

\noindent अत्रापि पूर्वखण्डलब्ध्यर्थमुपान्तिमेन ४८ स्वोर्ध्वे २ हते ९६ सकललब्ध्यर्थमन्त्येन २१ युते ११७ जातं वल्ल्याम्
\vspace{-1mm}

\begin{center}
ल ~११७ \\
गु ~~४८ \\
ल ~~२१ 
\end{center}
\vspace{-1mm}

\noindent अधःस्थलब्धेः प्रयोजनाभावादपगमे जातं
\vspace{-1mm}

\begin{center}
ल ~११७\\
गु ~~४८
\end{center}
\vspace{-1mm}

\noindent तदेवं मुख्यभाज्येऽस्मिन्~।
\vspace{-1mm}

\begin{center}
भा ~१७३ ~क्षे ~३ \\
ह ~~~७१ ~~~~~~~~
\end{center}
\end{sloppypar}

\newpage

\begin{sloppypar}
\noindent क्षेपयुतौ जातौ लब्धिगुणौ ११७।४८~। तदिदमुक्तं \hyperref[5.51]{'मुहुः स्यादिति राशियुग्मम्'} इति~। अत्र विनान्त्यभाज्यं सर्वेषु भाज्येषु पूर्वखण्डलब्धिसाधने गुणस्योपान्तिमत्वादुपान्तिमेन स्वोर्ध्वे हत इति~। सकललब्धिसाधनार्थमुत्तरखण्डलब्ध्यात्मकेऽन्त्येन युते, इति च वक्तव्यम्~। अन्त्यभाज्ये तु गुणस्यान्तिमत्वादुत्तरखण्डलब्धेरभावाच्च~। अन्त्येन हते स्वोर्ध्वे, इत्येव वक्तव्यं स्यादत आचार्येण तदन्तेऽपि शून्यनिवेशनमुक्तम्~। यतस्तथा कृते सर्वत्रोपान्तिमेन \hyperref[5.51]{'स्वोर्ध्वे हतेऽन्त्येन युते तदन्त्यं त्यजेत्'} इत्यनुगमः स्यात्~। एवं सिद्धौ लब्धिगुणौ
\vspace{-1mm}

\begin{center}
ल ~११७~~ \\
गु ~~४८~।
\end{center}
\vspace{-1mm}

\noindent अत्र हरतुल्ये गुणोपचये भाज्यतुल्यो लब्धेरुपचयो भवतीत्युक्तं प्राक्~। तयैव युक्त्या हरतुल्ये गुणापचये भाज्यतुल्यो लब्धेरपचयः स्यात्~। अतो हराधिके गुणे यथासम्भवमेकादिगुणो हरस्तस्मादपनेयः~। स लघुतरो गुणः स्यात्~। एवमेव तल्लब्धिश्च~। अत उक्तम् \hyperref[5.51]{'ऊर्ध्वो विभाज्येन दृढेन तष्टः फलं गुणः स्यादपरो हरेण'} इति~। उक्तयुक्त्यैव वक्ष्यति~। \hyperref[5.55]{'गुणलब्ध्योः समं ग्राह्यं धीमता तक्षणे फलम्'} इति~। न हि गुणस्यैकगुणहरतुल्यापचये द्विगुणभाज्यतुल्यो लब्धेरपचयः सम्भवतीत्यादि~। नन्वेवं सिद्धयोर्मुख्यभाज्यस्य लब्धिगुणयोर्योगजत्वं वियोग-जत्वं वा कथमवगन्तव्यमन्त्योपान्तिमादिषु भाज्येषु गुणस्य योगजवियोगजत्वयोरननु-गमादिति चेदुच्यते~। अन्त्ये भाज्ये क्षेपतुल्यो वियोगजो गुण इत्युक्तमसकृत्~। अतो व्यस्त-विधिना योगजो गुणः स्यादुपान्तिमभाज्ये~। पुनरतो व्यस्तविधिना तृतीयभाज्ये वियोगजो गुणः स्यात्~। एवं चतुर्थे योगजः पञ्चमे वियोगज इत्यादिनान्त्यभाज्यादारभ्य समभाज्ये योगजो विषमभाज्ये तु वियोगजो गुणः स्यात्~। तत्र मुख्यभाज्यस्य विषमता समता वा परस्परभजनलब्धीनां विषमतया समतया वा नियता भवति~। तस्मात्परस्परभजने यदि लब्धयः समास्तदा योगजौ लब्धिगुणौ यदि विषमास्तदा वियोगजौ लब्धिगुणौ मुख्यभाज्ये स्याताम्~। तत्र वियोगजयोर्लब्धिगुणयोर्वक्ष्यमाणत्वादत्र योगजयोरेव प्रतिपादनं युक्तम्~। अत उक्तम् \hyperref[5.52]{'एवं तदैवात्र यदा समास्ताः स्युर्लब्धयः'} इति~। विषमलब्धिषु पुनर्वियोगजौ लब्धिगुणौ सिध्यतः~। अपेक्षितौ च योगजौ अत उक्तम् \hyperref[5.52]{'स्युर्लब्धयश्चेद्विषमास्तदानीं यथागतौ लब्धिगुणौ विशोध्यौ~। स्वतक्षणाच्छेषमितौ तु तौ स्तः'} इति~। वियोगजो गुणो हराच्छुद्धः सन्योगजो भवेदित्यत्र युक्तिः प्रागुक्ता~। अथवान्यथोच्यते~। यो भाज्यो येन गुणेन गुणितः स्वहरेण भक्तो निःशेषः स्यात्स तद्गुणखण्डाभ्यां पृथग्गुणितः पृथग्भाजकेन भक्तः शुध्येदेव~। लब्धियोगश्च लब्धिः स्यात्~। यदा तु पृथग्गुणितयोर्मध्य एकतरो हरेण भक्तः सशेषः स्यात्तदा परोऽपि हरभक्तस्तावतैव
\end{sloppypar}

\newpage

\begin{sloppypar}
\noindent शेषेण न्यूनः स्यात्~। कथमन्यथा पृथग्गुणितयोर्योगो हरभक्तः शुध्येत्~। तत्र भाज्यो हर-तुल्यगुणेन गुणितो हरभक्तः शुध्येदेव~। गुणहरयोस्तुल्यत्वात्तत्र भाज्यतुल्या लब्धिश्च~। अत्र गुणहरयोस्तुल्यत्वाद्भाजकखण्डे एवं गुणखण्डे~। तत्रैकखण्डेन भाज्ये गुणिते हरभक्ते यावत् शेषं तावदेवापरखण्डगुणे भाज्ये न्यूनं स्यात्~। यथा\textendash 
\vspace{-1mm}

\begin{center}
भा ~१७ \\
ह ~~१५
\end{center}
\vspace{-1mm}

\noindent हरतुल्यगुण\textendash \,१५\textendash \,गुणितो भाज्यः २५५ हरेण १५ भक्तो लब्धिश्च १७~। अथ गुणखण्डाभ्यां १।१४ पृथग्गुणितः १७।२३८~। अत्र प्रथमे हरभक्ते शेषं २~। अत्र द्वयमधिकमिति तावता क्षेपेण वियोगे निःशेषभजनं भवति लब्धिश्च~। अपरखण्डे तु तावति २ क्षिप्ते २४० हरेण भक्ते निःशेषभजनं भवति लब्धिश्च १६~। अथवा गुणखण्डाभ्यां २।१३ पृथग्गुणितः ३४।२२१ एको हरभक्तः शेषं ४ एतच्छुद्धौ ३० गुणः १२ लब्धिश्च २~। परत्र २२१ तावत्येव ४ क्षिप्त २२५ निःशेषभजनादपरखण्ड\textendash \,१३\textendash \,गुणो लब्धिश्च १५~। अथवा गुणखण्डाभ्यां ३।१२ पृथग्गुणितः ५१।२०४~। अत्राद्यः षडूनः परश्च षड्युतः शुध्यतीति षट्क्षेपे योगवियोगजौ गुणौ गुणखण्डे एव १२।३ भाज्यखण्डे एव तल्लब्धी च १४।३~। अत उपपन्नम्~। \hyperref[5.52]{'यथागतौ लब्धिगुणौ विशोध्यौ स्वतक्षणात्'} इति~। अत एव वक्ष्यति \hyperref[5.54]{'योगजे तक्षणाच्छुद्धे गुणाप्ती स्तो वियोगजे'} इति~। तदेवं \hyperref[5.51]{'मिथो भजेत्तौ दृढभाज्यहृारौ'} इत्यादिना \hyperref[5.52]{'स्वतक्षणाच्छेषमितौ तु तौ स्तः'} इत्यन्तेन गुणलब्धिसाधनमुपपन्नम्~। स्यादेतत्~। आचार्येण कुट्टकार्थं यदपवर्तनावश्यकत्वमुक्तं तत् कथम्~। अनपवर्ते तदसिद्धेरिति चेत्~। तथा हि यथापवर्तसम्भवे सत्यपवर्ते कृते परस्पर-भजने रूपं शेषं स्यादस्मिंश्च क्षेपगुणिते क्षेपसमतया वियोगे शुद्धिः स्यादिति~। यथा क्षेपतुल्यो गुणस्तथानपवर्ते परस्परभजनेऽपवर्ताङ्कमितेऽन्त्यशेषे क्षेपगुणिते क्षेपतुल्यता न स्यादिति न क्षेपतुल्यो गुणः~। सत्यम्~। तथाप्यन्त्यशेषेण क्षेपे भक्ते यल्लभ्यते तावति गुणे क्षेपतुल्यं शेषं स्यादिति~। तस्य गुणत्वे बाधकाभावात्~। न च यत्रान्त्यशेषेण क्षेपो न शुध्यति तत्र कथं गुणः स्यादिति वाच्यम्~। तत्र खिलत्वस्य निरूपितत्वादाचार्योक्तत्वाच्च~। न च यथापवर्ते \hyperref[5.51]{'यावद्विभाज्ये भवतीह रूपम्'} इत्यनुगमः सुवचोऽस्ति~। न तथानपवर्ते यावद्विभाज्येऽमुकं भवेदित्यनुगमः सुवचोऽस्तीति~। क्रियावतारो न स्यादिति वाच्यम्~। यावद्विभाज्ये शून्यं न भवेदित्यनुगमस्य सुवचत्वात्~। अथवा 'यावद्विभाज्ये भवतीह शून्यम्' इति वक्तव्यम्~। अन्त्यहरेण क्षेपे भक्ते यल्लभ्यते तदन्त्यफलादेशेन निवेश्यं तदधः शून्यं निवेश्यमिति च वक्तव्यम्~। यतोऽत्रान्त्यभाज्यः शून्यमन्त्यहरस्त्वपवर्ताङ्कः~। अतः शून्यमेव गुण इति तदधः
\end{sloppypar}

\newpage

\begin{sloppypar}
\noindent स्थाप्यम्~। शून्यगुणान्त्यलब्धिः क्षेपतक्षणलाभाढ्या लब्धिरिति सा लब्धिस्थाने स्थाप्येति युक्तं भवति~। न च लाघवार्थमपवर्त इति वाच्यम्~। अनपवर्तितयोरपवर्तितयोश्च हरभाज्ययोः परस्परभजने लब्धिसाम्यात्~। अपवर्तितयोर्लघुत्वाल्लाघवमिति चेन्न~। अनपवर्तितयोः पर-स्परभजनस्यापवर्ताङ्कज्ञानार्थमावश्यकतया प्रत्युतापवर्तितयोः परस्परभजनयोर्गौरवात्~। न च सकलगुणलाभार्थमपवर्तनावश्यकत्वम्~। तथा हि\textendash \,व्यस्तविधिना लब्धिगुणसिद्धौ \hyperref[5.51]{'ऊर्ध्वो विभाज्येन दृढेन तष्टः फलं गुणः स्यादपरो हरेण'} इत्यनेन भवति लघुर्गुणो लब्धिश्च~। अनपवर्तिताभ्यां तक्षणे तद्द्वयं न स्यात्~। \hyperref[5.59]{'इष्टाहतस्वस्वहरेण युक्ते'} इत्यत्र गुणेनेष्टाहतहरो लब्धाविष्टाहतभाज्यश्च क्षेपावुक्तौ~। तत्रानपवर्तितहरतुल्ये तादृशभाज्यतुल्ये च क्रमेण गुण-हरयोः क्षेपेऽवान्तरगुणलब्ध्यवगमश्च न स्यादिति वाच्यम्~। भवत्वपवर्तितयोस्तक्षणत्वं क्षेपत्वं च~। तथापि गुणलब्ध्योः प्रागेव सिद्धतया \hyperref[5.51]{'मिथो भजेत्तौ दृढभाज्यहारौ'} इति कुट्टका-र्थमपवर्तानावश्यकत्वात्~। न च नोक्तौ वापवर्तावश्यकतेति वाच्यम्~। \hyperref[5.50]{'भाज्यो हारः क्षेपकश्चा-पवर्त्यः केनाप्यादौ सम्भवे कुट्टकार्थम्'} इत्यत्र {\color{violet}'समेन केनाप्यपवर्त्य हारभाज्यौ भजेद्वा'} इत्यत्रेव {\color{violet}'मिथो हराभ्यामपवर्तिताभ्यां यद्वा'} इत्यत्रेव च वाकारश्रवणात्~। \hyperref[5.51]{'यावद्विभाज्ये भवतीह रूपम्'} इति रूपशेष एव कुट्टकविधानाच्च~। किं च भाज्यशेषेण क्षेपे निःशेषभक्ते या लब्धिः सा वियोगे गुण इत्यस्य क्षेपे परस्परभजनं सर्वत्र नावश्यकमित्यस्ति लाघवम्~। तथा हि\textendash
\vspace{-3mm}

\begin{center}
भा २१ क्षे १६ \\
ह ~१३ ~~~~~~~
\end{center}
\vspace{-1mm}

\noindent अत्र भाज्ये हरेण भक्ते शेषम् ८~। अनेन क्षेपे १६ भक्ते लब्धिजातो वियोगजो गुणः २~। {\color{violet}'गुणगुणिता भाज्यलब्धिर्लब्धि'}श्चेति जाता लब्धिः २~। आचार्योक्तप्रकारे तु \hyperref[5.51]{'मिथो भजेत्तौ'} इत्यादिना वल्लीयम् 
\vspace{-3mm}

\begin{center}
~१\\
~१\\
~१\\
~१\\
~१\\
१६\\
~०
\end{center}
\end{sloppypar}

\newpage

\begin{sloppypar}
\noindent \hyperref[5.51]{उपान्तिमेन स्वोर्ध्वे हतेऽन्त्येन युत} इत्यादिना जातं राशिर्द्वयम्
\vspace{-1mm}

\begin{center}
१२८\\
८०~~
\end{center}
\vspace{-1mm}

\noindent \hyperref[5.51]{'ऊर्ध्वो विभाज्येन दृढेन तष्टः'} इत्यादिना जातौ लब्धिगुणौ तावेव २।२ अथवा
\vspace{-1mm}

\begin{center}
भा २१ क्षे १५\\
ह ~१३ ~~~~~~~~~
\end{center}
\vspace{-1mm}

\noindent अत्र भाज्यशेषेण ८ भक्तः क्षेपो न शुध्यत्यतो भाज्यशेषेण भक्तो हरः~। एवं जातं लब्धिद्वयं १।१ द्वितीयशेषं च ५~। अनेन भक्तः क्षेपः शुध्यतीति लब्धं गुणं ३ अन्ते तदधः शून्यं च निवेश्य जाता वल्ली
\vspace{-1mm}

\begin{center}
१\\
१\\
३\\
०
\end{center}
\vspace{-1mm}

\noindent \hyperref[5.51]{'उपान्तमेन स्वोर्ध्वे हत'} इत्यादिना जातं रााशिद्वयं
\vspace{-1mm}

\begin{center}
ल ~~६ \\
गु ~~३
\end{center}
\vspace{-1mm}

\noindent लब्धिसमत्वाज्जातौ योगजौ लब्धिगुणावस्मत्पक्षे~। आचार्यप्रकारे तु वल्ली
\vspace{-1mm}

\begin{center}
~१\\
~१\\
~१\\
~१\\
~१\\
१५\\
~०
\end{center}
\vspace{-1mm}

\noindent उक्तवज्जातं राशिद्वयं~ {\scriptsize $\begin{matrix}
\mbox{{१२०}}\\
\vspace{-1.5mm}
\mbox{{७५}}
\vspace{1mm}
\end{matrix}$} ~तक्षणे जातं~ {\scriptsize $\begin{matrix}
\mbox{{१५}}\\
\vspace{-1.5mm}
\mbox{{१०}}
\vspace{1mm}
\end{matrix}$}

\end{sloppypar}

\newpage

\begin{sloppypar}
\noindent लब्धिविषमत्वात्स्वतक्षणाच्छोधने जातौ लब्धिगुणौ योगजौ तावेव ६।३~। एवमस्मत् पक्षेऽस्ति लाघवम्~। तदेवमपवर्तावश्यकत्वे गौरवमेवेति प्रतिभाति~। अत्रोच्यते~। प्रकारा-न्तरेणापवर्ताङ्कोपस्थितौ तेनापवर्ते कृते भाज्यभाजकयोर्लघुत्वादस्त्येव कुट्टके लाघवम्~। किं चाविदुषामाचार्योक्तप्रकारे यथास्ति गणितसौकर्यं न तथान्यप्रकारे~। अन्यप्रकारे हि अनपवर्तितयोर्भाज्यहरयोः परस्परभजनादिना गुणलब्धिसाधनमपवर्तितयोस्तु तक्षणत्वं क्षेपत्वं चेत्यनुसंधानेऽस्ति गौरवम्~। किं च नायमारम्भो लौकिकगणितफलकः~। किं तु ग्रहगणितफलकः~। तत्र हि विकलाशेषाद्ग्रहानयने विकलाशेषं शुद्धिः~। षष्टिर्भाज्यः कुदिनानि हार इति प्रकल्प्य या लब्धिस्ता विकला यो गुणस्तत्कलाशेषमित्यादिरस्ति प्रकारः~। वक्ष्यति च~। \hyperref[5.67]{'कल्प्याथ शुद्धिर्विकलावशेषं षष्टिश्च भाज्यः कुदिनानि हारः~। तज्जं फलं स्युर्विकला गुणस्तु लिप्ताग्रमस्माच्च कलालवाग्रम्~। एवं तदूर्ध्वं च'} इति~। तत्रर्णक्षेपस्य विकलाद्यग्रस्यानियतत्वात्प्रतिप्रश्नं ततस्ततो विकलाद्यग्रात्कुट्टकप्रकरणेऽस्ति भूयान्प्रयासः~। अतः सुखार्थं स्थिरकुट्टको वक्ष्यते~। \hyperref[5.66]{'क्षेपं विशुद्धिं परिकल्प्य रूपं पृथक्तयोर्ये गुणकारलब्धी~। अभीप्सितक्षेपविशुद्धिनिघ्ने स्वहारतष्टे भवतस्तयोस्ते'}~इति~। एतादृशः स्थिरकुट्टकस्त्वपवर्त एव सम्भवति~। अनपवर्ते रूपक्षेपस्याभावात्~। यदि अप्यनपवर्तेऽप्यपवर्ताङ्कतुल्यक्षेपेण सम्भवति स्थिरकुट्टकस्तथापि यद्यप्यपवर्ताङ्कक्षेप एते गुणाप्ती तर्ह्यभीष्टक्षेपे क इति त्रैराशिकेऽपवर्ताङ्को हारः स्यात्~। रूपक्षेपात्त्रैराशिके तु गुणन-मात्रमित्यस्ति लाघवम्~। यद्वा सुधियः साधयन्तु यथा कथञ्चित्~। अज्ञानुग्राहकैराचार्यैः अवधानलाघवायापवर्तावश्यकत्वमुक्तमिति न कोऽपि दोष इत्यलं पल्लवितेन~॥~१२~॥\\

{\small तदेवं भाज्यहारक्षेपाणामपवर्तसम्भवेऽपवर्तं कृत्वैव कुट्टकः कार्यो भाज्यहारयोरेवापवर्तसम्भवे खिलत्वं चेति प्रतिपादितम्~। अथ क्षेपभाज्ययोरेव क्षेपभाजकयोरेव वापवर्तसम्भवे किं कार्यं तदाह\textendash }

\phantomsection \label{5.53}
\begin{quote}
{\large \textbf{{\color{purple}भवति कुट्टविधेर्युतिभाज्ययोः समपवर्तितयोरपि वा गुणः~।\\
भवति यो युतिभाजकयोः पुनः स च भवेदपवर्तनसङ्गुणः~॥~५३~॥}}}
\end{quote}

\hyperref[5.53]{\textbf{युतिः}} क्षेपः~। \hyperref[5.53]{\textbf{युतिभाज्ययोः समपवर्तितयोः}} सतोरपि \hyperref[5.51]{'मिथो भजेत्तौ दृढभाज्यहारौ'} इति यथोक्तात्कुट्टकविधेर्वा \hyperref[5.53]{\textbf{गुणः}} स्यात्~। \hyperref[5.53]{\textbf{अपिः}} समुच्चये~। \hyperref[5.53]{\textbf{वा}} प्रकारान्तरे~। क्षेपभाज्ययोः अपवर्तनसम्भवेऽप्यपवर्तनमकृत्वापि गुणः सिध्यति~। यद्वा तयोरपवर्तितयोः सतोरपि यथो-क्तकुट्टकविधिना स एव गुणः स्यादित्यर्थः~। तेन गुणेन भाज्यं सङ्गुण्य क्षेपेण संयोज्य हरेण विभज्य लब्धिरत्र ज्ञेया~। \hyperref[5.53]{\textbf{भवति}} य इति पुनर्विशेषे~। \hyperref[5.53]{\textbf{युतिभाजकयो}}स्त्वपवर्तनसम्भवे सत्य-\hyperref[5.53]{\textbf{पवर्तितयोः}} सतोर्यथोक्तकुट्टकविधिना यो गुणो भवति \hyperref[5.53]{\textbf{स}}
\end{sloppypar}

\newpage

\begin{sloppypar}
\noindent \hyperref[5.53]{\textbf{च भवेत्}}~। परम\hyperref[5.53]{\textbf{पवर्तनसङ्गुणः}} सन्ननपवर्तितयोरपि गुणसिद्धिर्भवति चकारात्~। यद्वा~\hyperref[5.53]{\textbf{अपि वा}}-शब्दसामर्थ्यादध्याहारेण योजना~। सा यथा\textendash \,युतिभाज्ययोः समपवर्तितयोर्या लब्धिः भवति~। अपि वा युतिभाजकयोस्त्वपवर्तितयोर्यो गुणो भवति सा लब्धिः~। स च गुणोऽपवर्तनसङ्गुणः सम्भवेत्~। लिङ्गविपरिणामेन लब्धिरपवर्तनसङ्गुणा सती भवेदिति योज्यम्~। युतिभाज्ययोः समपवर्तितयोर्लब्धिरपवर्ताङ्केन गुण्या~। गुणकस्तु यथागत एव~। युतिभाजकयोस्त्वपवर्तितयोर्गुणोऽपवर्ताङ्केन गुण्यः~। लब्धिर्यथागता वेत्यर्थः~। अत्र यद्वा इत्यादिना व्याख्यातोऽर्थो युक्ततरोऽस्ति~। परं न तथायं शब्दलभ्यः~। आचार्याणामपि नायम् अर्थोऽभिप्रेतः किं तु प्रथमः~। यतः \hyperref[5.61]{'शतं हतं येन युतं नवत्या'} इत्युदाहरणे ते वक्ष्यन्ति~। अत्र लब्धिर्न ग्राह्येति~। गुणघ्नभाज्ये क्षेपयुते हरभक्ते लब्धिश्चेति च गुणनभजनाल्लब्धिश्चेति च~।\\

अत्रोपपतिः~। येभ्यो भाज्यहारक्षेपेभ्यो~। \hyperref[5.51]{'मिथो भजेत्तौ दृढभाज्यहारौ'} इत्यादिना~ये गुणाप्ती स्यातां तेषु भाज्यादिषु ते गुणाप्ती पूर्वोक्तयुक्त्योपपन्ने एव~। अपि च भाज्य-भाजकयोर्यथास्थितयोः केनाप्येकेन गुणितयोर्भक्तयोर्वा नास्ति फले भेद इति तु प्रसि-द्धतरम्~। प्रकृते तु भाज्यस्य खण्डद्वयम्~। गुणगुणितः कल्पितभाज्य एकं क्षेपोऽपरम्~। हर एव हरः~। एषु त्रिष्वेकस्यापि गुणनेऽभीष्टे त्रयाणामपि गुणनमावश्यकम्~। उक्तयुक्तेः एव~। तत्र गुणगुणितकल्पितभाज्यस्य गुणने प्रकारत्रयं सम्भवति~। गुणमेवादौ सङ्गुण्य तादृशेन गुणेन कल्पितभाज्यो गुण्य इत्येकः प्रकारः~। कल्पितभाज्यमेवादौ सङ्गुण्य पश्चात् यथास्थितेन गुणकेन तं गुणयेदिति द्वितीयः~। गुणगुणितं कल्पितभाज्यं गुणयेदिति तृतीयः प्रकारः~। अथ भाज्यादित्रयमपवर्त्य कुट्टकेन येन गुणाप्ती साधिते ते अपवर्तितेष्वेव भाज्यादिषु युक्ते अपेक्षिते तत्तूद्दिष्टभाज्यादिषु~। अतोऽपवर्तितभाज्यादिकमपवर्ताङ्केन गुणयेत्तदुद्दिष्टभाज्यादिकं भवति~। येभ्यः कुट्टकः कृतस्तेषु गुणितेषु भक्तेषु वा फलभेदो नास्तीति जाते ते एव गुणाप्ती उद्दिष्टभाज्यादिष्वपीति~। अथ यत्र भाज्यक्षेपावेवापवर्तितौ न हरस्तत्रापि तदुत्थे गुणाप्ती तेषु युक्ते एव~। अपेक्षिते तूद्दिष्टभाज्यादिषु~। तत्र हरस्तूद्दिष्ट एवास्ति~। भाज्यक्षेपौ त्वपवर्ताङ्कगुणितावुद्दिष्टौ भवतः~। परं हरोऽप्यपवर्ताङ्केन गुण्यः~। भाज्यस्य गुणितत्वात्~। गुणिते च हरे न स्यादुद्दिष्टहरः~। तथा सत्युद्दिष्टभाज्यक्षेपयोरेव गुणाप्तिसिद्धिर्नोद्दिष्टहरे~। अतोऽत्र हरो न गुणनीयः~। परं भाज्यशकलयोर्गुणनेन भाज्य-मात्रस्य गुणनाल्लब्धिरपि प्रकृतेऽपवर्ताङ्कगुणिता सती भवेत्~। अत उक्तं युतिभाज्ययोः समपवर्तितयोर्यो लब्धिः सापवर्तसङ्गुणा गुणस्तु यथागत एवेति~। अथ यत्र भाजकक्षेपौ एवापवर्त्य कुट्टकः कृतस्तत्रापि ये सिद्धे गुणलब्धी ते तेष्वेव भवतः~। अपेक्षिते तूद्दिष्ट-भाज्यादिषु~। प्रकृते
\end{sloppypar}

\newpage

\begin{sloppypar}
\noindent कल्पितभाज्यस्तूद्दिष्ट एवास्ति~। हरक्षेपौ त्वपवर्ताङ्केन गुणितावुद्दिष्टौ भवतः~। परं क्षेप-लक्षणभाज्यखण्डस्य गुणितत्वादपरमपि भाज्यखण्डं गुणनीयम्~। परखण्डं च गुणगुणितः कल्पितभाज्यः~। अतोऽसावपवर्ताङ्केन गुण्यः~। अस्य गुणनं तु त्रेधा सम्भवतीत्युक्तम्~। तत्र कल्पितभाज्यस्य गुणने उद्दिष्टकल्पितभाज्यो न स्यात्~। अतो गुण एव गुणनार्हो भवति~। अत उक्तम्\textendash \,\hyperref[5.53]{'भवति यो युतिभाजकयोः पुनः स च भवेदपवर्तनसङ्गुणः'} इति~। अथ यत्र क्षेपमात्रमपवर्त्य कुट्टकः क्रियते तत्रापि तस्मिन्क्षेपे ते गुणाप्ती युक्ते~। अथ स क्षेपस्तेनापवर्ताङ्केन गुणितः सन्नुद्दिष्टक्षेपो भवति~। परं भाज्यखण्डस्य गुणितत्वात् अपरं भाज्यखण्डं गुणनीयम्~। हरोऽपि गुणनीयः~। गुणिते च गुणे भाज्यखण्डमपि गुणितं भवतीति गुणकोऽपवर्ताङ्केन गुण्यः~। एवं जातं भाज्यखण्डयोर्गुणनम्~। हरस्य गुणने तु नोद्दिष्टहरसिद्धिरिति भाज्यमात्रस्य गुणनाल्लब्धिरपवर्ताङ्कगुणिता स्यात्~।~अतः क्षेपमात्रस्यापवर्तने ये गुणलब्धी तयोरपवर्ताङ्कगुणने सत्युद्दिष्टगुणाप्तिसिद्धिः~। अपवर्ताङ्कः चात्रोद्दिष्टक्षेपतुल्यः~। स्वेन स्वस्य सदापवर्तनसम्भवात्~। अतोऽपवर्तितक्षेपोऽपि रूपमेव~। अनयैवोपपत्त्या वक्ष्यति\textendash \,\hyperref[5.66]{'क्षेपं विशुद्धिं परिकल्प्य रूपं पृथक्तयोर्ये गुणकारलब्धी~। अभी-प्सितक्षेपविशुद्धिनिघ्ने स्वहारतष्टे भवतस्तयोस्ते'} इति~। अथ यत्र हारभाज्यावेवापवर्त्य कुट्टकः क्रियते तत्र सिद्धे ये गुणाप्ती ते अपवर्तितयोरेव युक्ते~। उद्दिष्टसिद्ध्यर्थं त्वपवर्ताङ्केन गुणने क्षेपगुणनस्याप्यावश्यकतया नोद्दिष्टक्षेपसिद्धिरत एव तत्र खिलत्वमुक्तम्~। अत एव त्रयाणामपवर्तनसम्भवेऽपि यदि हरभाज्यावेवापवर्त्य लब्धिगुणौ साध्येते तदा नोद्दिष्ट-सिद्धिः~। अत एव भाज्यमात्रस्य भाजकमात्रस्य वापवर्तनेन सिद्धाभ्यां लब्धिगुणाभ्यां नोद्दिष्टसिद्धिरित्यादि सुधीभिरूह्यम्~। \\

{\small अथ ऋणक्षेपे ऋणभाज्ये वा सति विशेषमनुष्टुभाह\textendash }

\phantomsection \label{5.54}
\begin{quote}
{\large \textbf{{\color{purple}योगजे तक्षणाच्छुद्धे गुणाप्ती स्तो वियोगजे~।\\
धनभाज्योद्भवे तद्वद्भवेतामृणभाज्यजे~॥~५४~॥}}}
\end{quote}

\hyperref[5.54]{\textbf{योगजे}} धनक्षेपजे ये गुणाप्ती ते स्वतक्षणाच्छुद्धे वियोगजे भवतः~। गुणो दृढहरात् शुद्धः सल्लँब्धिर्दृढभाज्याच्छुद्धा सती ऋणक्षेपे भवतीत्यर्थः~। एवं धनभाज्योद्भवे गुणाप्ती तद्वत्स्वतक्षणाच्छुद्धे ऋणभाज्यजे भवतः~। अत्रोत्तरार्धे 'ऋणभाज्योद्भवे तद्वद्भवेतामृण-भाजके' इत्यपि पाठः क्वचिदृश्यते~। अस्यार्थः\textendash \,योगजे गुणाप्ती स्वतक्षणाच्छुद्धे वियोगजे भवतः~। तद्वदृणभाज्योद्भवे भवतः~। तद्वदृणभाजकेऽपि गुणाप्ती भवतः~। क्षेपभाज्य-हाराणामन्यतमे ऋणे सति पूर्वसिद्धे गुणाप्ती स्वतक्षणाच्छोध्ये इत्यर्थः~। एवं द्वौ चेदृणगतौ तदा पुनरपि स्वतक्षणाच्छोध्ये इत्यर्थः~। एवं त्रयाणामप्यृणत्वे त्रिवारं स्वतक्षणाच्छोध्ये इत्यर्थः~। अयमपपाठः~। नहि भाजकस्य धनत्वे
\end{sloppypar}

\newpage

\begin{sloppypar}
\noindent ऋणत्वे वास्ति कश्चिदङ्कतो विशेषो येनोपायान्तरमारभ्येत~। किन्तु धनर्णताव्यत्यासमात्रं लब्धेः~। भाज्यस्य तु धनत्वे ऋणत्वे च क्षेपयोगे क्रियमाणेऽस्त्यङ्कतोऽपि विशेष इति तस्यर्णत्व उपायान्तरमारम्भणीयमेव~। आचार्यस्याप्यनभिमत एवायं पाठः~। यतो \hyperref[5.63]{'अष्टादश गुणाः केन दशाढ्या वा दशोनिताः~। शुद्धं भागं प्रयच्छन्ति क्षयगैका-दशोद्धृताः'} इत्युदाहृत्य \,{\scriptsize $\begin{matrix}
\mbox{{भा ~१८ ~क्षे ~१०}}\\
\vspace{-1.5mm}
\mbox{{ह ~~१ं१ ~~~~~~~~~}}
\vspace{1mm}
\end{matrix}$}~। अत्र भाजकस्य धनत्वे कृते गुणलब्धी ८।१४~। ऋणेऽपि भाजके एवं किन्तु लब्धिर्ऋणगता कल्प्या भाजकस्यर्णत्वात् ८।१४ं इति वक्ष्यति~। अस्मिन्पाठेऽर्थाशुद्धिरप्युदाहरणविवरणावसरे प्रतिपादयिष्यते~। वस्तुतस्तूत्त-रार्धमनपेक्षितमेव~। पूर्वार्धेनैव गतार्थत्वात्~। तथाहि\textendash \,योगजे गुणाप्ती वियोगजे भवत इति हि तदर्थः~। तत्र भाज्यक्षेपयोर्धनत्वे ऋणत्वे वा ये गुणाप्ती ते योगजे~। यत उभयोर्धनत्वे ऋणत्वे वा \hyperref[1.3]{'योगे युतिः स्यात् क्षययोः स्वयोर्वा'} इति नास्ति कश्चिदङ्कतो विशेषः~। यदा पुनर्भाज्यक्षेपयोरन्यतरस्यर्णत्वं तदा \hyperref[1.3]{'धनर्णयोरन्तरमेव योगः'} इत्युक्तत्वादन्तरे क्रियमाणे भवत्यङ्कतोऽपि विशेष इति तदर्थमुपायान्तरमारम्भणीयम्~। तदर्थमुक्तं 'स्वत-क्षणाच्छुद्धे वियोगजे भवतः' इति~। अस्मात्पूर्वार्धादतिरिक्तः को वार्थ उत्तरार्धेन प्रतिपाद्यते येन तदपेक्षितं स्यात्~। अयमर्थो \hyperref[5.62]{'यद्गुणाक्षयगषष्टिरन्विता'} इत्युदाहरणे \hyperref[5.54]{'धनभाज्यो-द्भवे तद्वद्भवेतामृणभाज्यजे'} इति मन्दावबोधनार्थं मयोक्तम्~। अन्यथा योगजे तक्ष-णाच्छुद्धेरित्यादिनैव तत्सिद्धेरिति वदताचार्येणैव प्रतिपादयिष्यते~। तस्मात्सिद्धान्तान्तर्गत-बीजमूलसूत्रे पूर्वार्धमात्रम्~। द्वितीयमर्धं तु तद्विवरणरूपेऽस्मिन्बीजगणिते बालावबो-धार्थमुक्तमतस्तत्पृथग्गणनां नार्हति~। अतः कुट्टकसूत्रेष्वनुष्टुभां चतुष्टयमेव न सार्धं तत्~। अनुष्टुप्-त्रयमेका च गाथेति कल्पनस्यान्यायत्वात्~। अनुपपत्तेरभावादित्यलं पल्लवितेन~। \\

{\small सूत्रोपपत्तिस्तु \hyperref[5.52]{'यथागतौ लब्धिगुणौ विशोध्यौ स्वतक्षणाच्छेषमितौ तु तौ स्तः'} इत्यस्योपप-त्तिनिरूपणावसर एव निरूपिता~। अथ क्षेपे हरमात्राद्भाज्यमात्राद्वा हरभाज्याभ्यां वान्यूने क्वचित् विशेषमुत्तरार्धेनाह\textendash }

\phantomsection \label{5.55}
\begin{quote}
{\large \textbf{{\color{purple}गुणलब्ध्योः समं ग्राह्यं धीमता तक्षणे फलम्~॥~५५~॥}}}
\end{quote}

\hyperref[5.51]{'ऊर्ध्वो विभाज्येन दृढेन तष्टः फलं गुणः स्यादपरो हरेण'} इत्यत्र गुणलब्धिसम्बन्धिनि तक्षणे क्रियमाणे सत्युभयत्र तक्षणस्य फलं तुल्यमेव ग्राह्यम्~। केन धीमता बुद्धिमता~। हेतुगर्भमिदम्~। तथाहि\textendash \,उभयत्र तक्षणे क्रियमाणे यत्राल्पं तक्षणफलं लभ्यते तत्तुल्यम् एवान्यत्रापि ग्राह्यम्~। नन्वधिकं प्राप्तमप्यस्योपपत्तिः \hyperref[5.51]{'ऊर्ध्वो विभाज्येन}
\end{sloppypar}

\newpage

\begin{sloppypar}
\noindent \hyperref[5.51]{दृढेन तष्टः फलं गुणः स्यादपरो हरेण'} इत्यस्य युक्तिनिरूपणे निरूपिता~। अत्र पुस्तकेषु \hyperref[5.55]{'गुणलब्ध्योः समं ग्राह्यम्'} इत्यादिश्लोकार्धस्य \hyperref[5.54]{'योगजे तक्षणाच्छुद्धे'} इत्यतः प्राक् पाठो दृश्यते~। स तु लेखकदोषज इति प्रतिभाति~। पुस्तकपाठक्रमस्वीकारे तु \hyperref[5.55]{'गुणलब्ध्योः समं ग्राह्यम्'} इत्यत्र प्रकारान्तरार्थं प्रवृत्तस्य \hyperref[5.56]{'हरतष्टे धनक्षेपे'} इत्येतस्य सूत्रस्य व्यवधानं स्यात्~। उदाहरणक्रमविरोधश्च स्यात्~। लीलावतीपुस्तकेषु पुनरस्मल्लिखितक्रम एवास्ति~। युक्तश्चायमिति प्रतिभाति~। \\

{\small अथात्र गुणलब्ध्योस्तक्षणे फलयोरतुल्यता यथा न भवति तथा प्रकारान्तरमनुष्टुभाह\textendash }

\phantomsection \label{5.56}
\begin{quote}
{\large \textbf{{\color{purple}हरतष्टे धनक्षेपे गुणलब्धी तु पूर्ववत्~।\\
क्षेपतक्षणलाभाढ्या लब्धिः शुद्धौ तु वर्जिता~॥~५६~॥}}}
\end{quote}

यत्र क्षेपो हरादधिकस्तत्र हरेण क्षेपस्तक्ष्यः~। तष्टक्षेपमेव क्षेपं प्रकल्प्य पूर्ववद्गुणलब्धी साध्ये~। तत्र गुणो यथागत एव लब्धिस्तु क्षेपतक्षणलाभाढ्या कार्या~। क्षेपस्य तक्षणमवशेषणं तत्र यो लाभः फलं तेनाढ्या युक्ता~। एवं धनक्षेपे शुद्धौ ऋणक्षेपे तु हरतष्टे कृते सति पूर्ववद्योगजे तक्षणाच्छुद्धे गुणाप्ती स्तो वियोगजे इत्युक्तप्रकारेण ये गुणाप्ती स्तस्तत्र लब्धिः क्षेपतक्षणलाभेन वर्जिता कार्या~। यदा तु भाज्यादन्यूने हरान्न्यूने क्षेपे गुणलब्ध्योस्तक्षणे क्वचित्फलवैषम्यं स्यात्तत्रैतस्य सूत्रस्याप्रवृतेः \hyperref[5.55]{'गुणलब्ध्योः समं ग्राह्यम्'} इत्यादिनैव तक्षणफलं ग्राह्यम्~। यथा \,{\scriptsize $\begin{matrix}
\mbox{{भा ~३ ~क्षे ~३}}\\
\vspace{-1.5mm}
\mbox{{ह ~~४ ~~~~~~~}}
\vspace{1mm}
\end{matrix}$}\, अत्रोक्तवज्जातं राशिद्वयं \,{\scriptsize $\begin{matrix}
\mbox{{ल ~~३}}\\
\vspace{-1.5mm}
\mbox{{गु ~~३}}
\vspace{1mm}
\end{matrix}$}\, अत्र गुणतक्षणे किञ्चिन्न लभ्यते~। लब्धितक्षणे त्वेकः प्राप्यते स न ग्राह्यः~। एवं क्षेपस्य हरेण तक्षणेऽपि भाज्यादन्यूनतया यदि क्वचित्फलवैषम्यं स्यात्तत्रापि गुणलब्ध्योः समं ग्राह्यमित्यादिनैव तक्षणफलं ग्राह्यम्~। यथात्र \,{\scriptsize $\begin{matrix}
\mbox{{भा ~३ ~क्षे ~७}}\\
\vspace{-1.5mm}
\mbox{{ह ~~४ ~~~~~~~}}
\vspace{1mm}
\end{matrix}$}\, एतादृशस्थले फलयोर्यथा वैषम्यं न भवति तथा प्रकारान्तरं न दृश्यते~। अत्रोपपत्तिः~। क्षेपस्यात्र खण्डद्वयं कृतम्~। एकादि-गुणहरतुल्यमेकं शेषमपरम्~। तत्र शेषमिते क्षेपे यः साधितो गुणस्तेन गुणेन भाज्ये गुणिते तेन क्षेपेण युते हरेण भक्ते च शेषं न स्यात्~। अथेद्दिष्टक्षेपार्थमपरखण्डमपि योज्यम्~। तेनापि युते तस्मिन्भाज्ये हरभक्ते शेषं नैव स्यात्~। तस्यैकादिगुणहरतुल्यत्वात्~। किन्तु हरेण तस्मिन्क्षेपखण्डे भक्ते यल्लभ्यते तावल्लब्धावधिकं स्यात्~। एवमृणक्षेपे तावदेव न्यूनं स्यात् इत्युपपन्नम्~।
\end{sloppypar}

\newpage

\begin{sloppypar}
{\small अथ भाज्येऽपि हरादधिके विशेषमाहानुष्टुभा\textendash }

\phantomsection \label{5.57}
\begin{quote}
{\large \textbf{{\color{purple}अथवा भागहारेण तष्टयोः क्षेपभाज्ययोः~।\\
गुणः प्राग्वत्ततो लब्धिर्भाज्याद्धतयुतोद्धृतात्~॥~५७~॥}}}
\end{quote}

यत्र भाज्यक्षेपौ हरादधिकौ तत्र पूर्ववद्वा क्षेपमात्रतक्षणेन वा गुणाप्ती साध्ये~। अथवा भाज्यक्षेपौ द्वावपि हरेण तक्ष्यौ~। तष्टयोः क्षेपभाज्ययोः प्राग्वदेव गुणाप्ती साध्ये~। तत्र गुण एव ग्राह्यो न लब्धिः~। कथं तर्हि लब्धिर्ज्ञेयेति~। तदाह\textendash \,\hyperref[5.57]{\textbf{'भाज्याद्धतयुतोद्धृतात्'}} इति~। हतश्चासौ युतश्च हतयुतः, स चासावुद्धृतश्चेति हतयुतोद्धृतस्तस्मात्~। गुणेन गुणितात् क्षेपेण युताद्भाजकेन भक्तादुद्दिष्टाद्भाज्याद्या लब्धिर्भवति सा ज्ञेयेत्यर्थः~। अस्त्यत्र लब्धिज्ञाने प्रकारान्तरमपि~। तथाहि\textendash \,भाज्यतक्षणलाभो गुणेन गुणनीयः~। पश्चात्क्षेपतक्षणलाभेन संस्कार्यः~। संस्कृतेन तेन गणितागता लब्धिः संस्कार्या सा लब्धिर्भवतीति~। गौरवात् इदमुपेक्षितमाचार्यैः~। अत्रोपपत्तिः~। यथा क्षेपस्य खण्डद्वयं कृत्वा पूर्वमुपपत्तिः प्रदर्शिता तथात्र भाज्यस्यापि खण्डद्वयेनोपपत्तिर्ज्ञेया~।\\

{\small अथ क्षेपाभाव एकादिगुणहरसमे वा क्षेपे विशेषमनुष्टुभाह\textendash }

\phantomsection \label{5.58}
\begin{quote}
{\large \textbf{{\color{purple}क्षेपाभावोऽथवा यत्र क्षेपः शुध्येद्धरोद्धृतः~।\\
ज्ञेयः शून्यं गुणस्तत्र क्षेपो हरहृतः फलम्~॥~५८~॥}}}
\end{quote}

स्पष्टोऽर्थः~। उपपत्तिरपि कुट्टकोपपत्तिप्रारम्भ एवोक्ता~। \\

{\small अथ गुणलब्ध्योरनेकत्वमुपजातिकापूर्वार्धेनाह\textendash }

\phantomsection \label{5.59}
\begin{quote}
{\large \textbf{{\color{purple}इष्टाहतस्वस्वहरेण युक्ते ते वा भवेतां बहुधा गुणाप्ती~॥~५९~॥}}}
\end{quote}

स्वस्य स्वस्य हरः स्वस्वहरः~। इष्टेनाहतश्चासौ स्वस्वहरश्चेष्टाहतस्वस्वहरः~। तेन युक्ते गुणाप्ती बहुधा भवेताम्~। इष्टेन गुणितं हरं गुणे प्रक्षिपेत्तेनैवेष्टेन गुणितं भाज्यं लब्धौ च प्रक्षिपेत्~। एवमेते गुणाप्ती इष्टवशाद्भवत इत्यर्थः~। अस्योपपत्तिः \hyperref[5.51]{'मिथो भजेत्तौ दृढभाज्य-हारौ'} इत्यस्योपपत्तिकथनोपक्रम एव प्रदर्शिता~।\\

{\small अथोक्तसूत्राणां क्रमेणोदाहरणानि शिष्यबोधार्थं निरूपयति~।\\
\vspace{-2mm}

तेषु यत्र त्रयाणामप्यपवर्तः सम्भवति लब्धयश्च समास्तादृशमुदाहरणं रथोद्धतया तावदाह\textendash }

\phantomsection \label{5.60}
\begin{quote}
{\large \textbf{{\color{purple}एकविंशतियुतं शतद्वयं यद्गुणं गणक पञ्चषष्टियुक्~।\\
पञ्चवर्जितशतद्वयोद्धृतं शुद्धिमेति गुणकं वदाशु तम्~॥~६०~॥}}}
\end{quote}
\end{sloppypar}

\newpage

\begin{sloppypar}
स्पष्टोऽर्थः~। न्यासः \,{\scriptsize $\begin{matrix}
\mbox{{भा ~२२१ ~क्षे ~६५}}\\
\vspace{-1.5mm}
\mbox{{ह ~~१९५ ~~~~~~~~~}}
\vspace{1mm}
\end{matrix}$}\, अत्रापवर्ताङ्कज्ञानार्थं भाज्ये २२१ हरेण १९५ भक्ते शेषं २६ अनेन पुनर्हरे भक्ते शेषम् १३~। अनेनापि पुनः पूर्वशेषे २६ भक्ते शेषाभावः ०~। अतः परस्परं भाजितयोरन्त्यशेषमिदम् १३~। इदमेव तयोरपवर्तनम्~। अनेन तौ निःशेषं भज्येते एव~। अनेनापवर्तिता भाज्यहारक्षेपा जाता दृढाः \,{\scriptsize $\begin{matrix}
\mbox{{भा ~१७ ~क्षे ~५}}\\
\vspace{-1.5mm}
\mbox{{ह ~~१५ ~~~~~~~~}}
\vspace{1mm}
\end{matrix}$}\, अनयोर्दृढभाज्यहारयोः परस्परं भक्तयोर्लब्धमधोऽधस्तदधः क्षेपस्तदधः शून्यं निवेश्यमिति जाता वल्ली \;{\scriptsize $\begin{matrix}
\mbox{{१}}\\
\mbox{{७}}\\
\mbox{{५}}\\
\vspace{-1.5mm}
\mbox{{०}}
\vspace{2mm}
\end{matrix}$}~। अत्रो-पान्तिमेन ५ स्वोर्ध्वे ७ हते ३५ अन्त्येन ० युते ३५ अन्त्यं ० त्यजेदिति जातं \;{\scriptsize $\begin{matrix}
\mbox{{१}}\\
\mbox{{३५}}\\
\vspace{-1.5mm}
\mbox{{५}}
\vspace{2mm}
\end{matrix}$}~। पुनरुपा-न्तिमेन ३५ स्वोर्ध्वे १ हते ३५ अन्त्येन ५ युते ४० अन्त्यं ५ त्यजेदिति जातं राशिद्वयं \,{\scriptsize $\begin{matrix}
\mbox{{४०}}\\
\vspace{-1.5mm}
\mbox{{३५}}
\vspace{1mm}
\end{matrix}$}\, एतौ दृढभाज्यहाराभ्यामाभ्यां \,{\scriptsize $\begin{matrix}
\mbox{{१७}}\\
\vspace{-1.5mm}
\mbox{{१५}}
\vspace{1mm}
\end{matrix}$}\, तष्टौ शेषे \,{\scriptsize $\begin{matrix}
\mbox{{६}}\\
\vspace{-1.5mm}
\mbox{{५}}
\vspace{1mm}
\end{matrix}$}\, जातौ क्रमेण लब्धिगुणौ~। \hyperref[5.59]{'इष्टाहतस्व-स्वहरेण युक्ते ते वा भवेतां बहुधा गुणाप्ती'} इत्युक्तत्वादनयोर्लब्धिगुणयोः स्वतक्षणात् इष्टगुणं क्षेप इत्येकमिष्टं प्रकल्प्य जातौ लब्धिगुणौ वा $\dfrac{{\footnotesize{\hbox{२३}}}}{{\footnotesize{\hbox{२०}}}}$ द्विकेनेष्टेन वा $\dfrac{{\footnotesize{\hbox{४०}}}}{{\footnotesize{\hbox{३५}}}}$ त्रिकेणेष्टेन वा $\dfrac{{\footnotesize{\hbox{५७}}}}{{\footnotesize{\hbox{५०}}}}$~।
एवमिष्टवशाल्लब्धिगुणयोरानन्त्यं ज्ञेयम्~। तेन तेन गुणेनोद्दिष्टभाज्ये गुणिते क्षेपेण युते च सति सा सा लब्धिः शेषाभावश्च भवतीत्यर्थः~। अत्रापि क्षेपभाज्यावेव क्षेपभाज-कावेव वापवर्त्य यद्वा प्रथमतः क्षेपभाज्यौ पश्चात्क्षेपभाजकौ चापवर्त्याथवादौ त्रीनपवर्त्य पश्चात्क्षेपभाजकावप्यपवर्त्य चतुर्धा कुट्टको द्रष्टव्यः~। तत्र त्रयाणामपवर्तसम्भवे सति यदि द्वावेवापवर्त्य कुट्टकः क्रियते तदा न सकलगुणलाभः~। अथ चतुर्षु प्रकारेषु भाज्यादीनां क्रमेण न्यासः~।

\begin{center}
\begin{tabular}{llll}
भा ~१७ ~क्षे ~५ & भा ~२२१ ~क्षे ~५ & भा ~१७ ~क्षे ~१ & भा ~१७ ~क्षे ~१ \\
~ह ~१९५ & ~ह ~~१५ & ~ह ~~३९ & ~ह ~~३ 
\end{tabular}
\end{center}

क्रमेण लब्धिगुणौ ~~$\begin{matrix}
\mbox{{७ ~~७४ ~~७ ~~~६}}\\
\vspace{-1.5mm}
\mbox{{८० ~~५ ~~१६ ~~१}}
\vspace{1mm}
\end{matrix}$
\end{sloppypar}

\newpage

\begin{sloppypar}
\noindent प्रथमे लब्धेः ७ अपवर्ताङ्केन १३ गुणने जातै लब्धिगुणौ ९१।८०~। द्वितीये गुणकं ५ अपवर्तेन १३ सङ्गुण्य जातौ लब्धिगुणौ ७४।६५~। तृतीये क्षेपभाज्यापवर्तेन १३ लब्धिं ७ सङ्गुण्य क्षेपभाजकापवर्तेन ५ गुणं १६ सङ्गुण्य जातौ लब्धिगुणौ ९१।८०~। चतुर्थे क्षेपभाजकापवर्तेन ५ गुणं १ सङ्गुण्य जातौ लब्धिगुणौ ६।५~। एवं यथासम्भवं सर्वत्र प्रकारा ऊह्याः~। वक्ष्यमाण-स्थिरकुट्टकप्रकारेण गुणलब्धिसाधनं सर्वत्र बोध्यम्~। \\

{\small अथ त्रयाणामनपवर्ते भवति कुट्टविधेरिति सूत्रस्य स्वतन्त्रमुदाहरणं योगजे तक्षणाच्छुद्ध इत्यस्य च क्रमेणोदाहरणद्वयमुपजातिकयाह\textendash }

\phantomsection \label{5.61}
\begin{quote}
{\large \textbf{{\color{purple}शतं हतं येन युतं नवत्या विवर्जितं वा विहृतं त्रिषष्ट्या~।\\
निरग्रकं स्याद्वद मे गुणं तं स्पष्टं पटीयान्यदि कुट्टकेऽसि~॥~६१~॥}}}
\end{quote}

शतं येन गुणेन हतं नवत्या युतं त्रिषष्ट्या विहृतं निरग्रकं स्यात्तं गुणमाशु वद~। अथ वियोग उदाहरणं विवर्जितं वेति~। शतं येन हतं नवत्या विवर्जितं त्रिषष्ट्या विहृतं निरग्रकं स्यात्तं गुणं च वद~। यदि त्वं कुट्टके पटीयान् पटुतरोऽसि~। \\

न्यासः \,{\scriptsize $\begin{matrix}
\mbox{{भा ~१०० ~क्षे ~९०}}\\
\vspace{-1.5mm}
\mbox{{ह ~~६३ ~~~~~~~~~~}}
\vspace{1mm}
\end{matrix}$}~। अत्र हरभाज्ययोः परस्परं भक्तयोः शेषम् १~। अत इदमपवर्तनम्~। अनेनापवर्तनेऽर्थादनपवर्त एव~। अत्र प्राग्वद्वल्ली \;{\scriptsize $\begin{matrix}
\mbox{{१}}\\
\mbox{{१}}\\
\mbox{{१}}\\
\mbox{{२}}\\
\mbox{{२}}\\
\mbox{{१}}\\
\mbox{{९०}}\\
\vspace{-1.5mm}
\mbox{{०}}
\vspace{2mm}
\end{matrix}$}\; जातं राशिद्वयं \,{\scriptsize $\begin{matrix}
\mbox{{२४३०}}\\
\vspace{-1.5mm}
\mbox{{१५३०}}
\vspace{1mm}
\end{matrix}$}\, स्वस्वहारेण तक्षणे कृते जातौ लब्धिगुणौ \,{\scriptsize $\begin{matrix}
\mbox{{३०}}\\
\vspace{-1.5mm}
\mbox{{१८}}
\vspace{1mm}
\end{matrix}$}~। यद्वा भाज्यक्षेपौ दशभिरपवर्त्य न्यासः \,{\scriptsize $\begin{matrix}
\mbox{{भा ~१० ~क्षे ~९}}\\
\vspace{-1.5mm}
\mbox{{ह ~~६३ ~~~~~~}}
\vspace{1mm}
\end{matrix}$}~। पूर्ववद्वल्ली \;{\scriptsize $\begin{matrix}
\mbox{{०}}\\
\mbox{{६}}\\
\mbox{{३}}\\
\mbox{{९}}\\
\vspace{-1.5mm}
\mbox{{०}}
\vspace{2mm}
\end{matrix}$}\; पूर्ववद्राशिद्वयं \,{\scriptsize $\begin{matrix}
\mbox{{२७}}\\
\vspace{-1.5mm}
\mbox{{१७१}}
\vspace{1mm}
\end{matrix}$}\, तक्षणे जातं \,{\scriptsize $\begin{matrix}
\mbox{{७}}\\
\vspace{-1.5mm}
\mbox{{४५}}
\vspace{1mm}
\end{matrix}$}

\end{sloppypar}

\newpage

\begin{sloppypar}
\noindent लब्धयो विषमा इति स्वतक्षणाभ्यामाभ्यां \,{\scriptsize $\begin{matrix}
\mbox{{१०}}\\
\vspace{-1.5mm}
\mbox{{६३}}
\vspace{1mm}
\end{matrix}$}\, शोधितौ जातौ लब्धिगुणौ \,{\scriptsize $\begin{matrix}
\mbox{{३}}\\
\vspace{-1.5mm}
\mbox{{१८}}
\vspace{1mm}
\end{matrix}$}~। अत्र लब्धिर्न ग्राह्या~। किन्तु गुणघ्नभाज्ये क्षेपयुते हरभक्ते लब्धिः ३०~। यद्वापवर्तेन १० गुणिता लब्धिः ३ इयं जाता ३०~। एवं जातौ तावेव लब्धिगुणौ \,{\scriptsize $\begin{matrix}
\mbox{{३०}}\\
\vspace{-1.5mm}
\mbox{{१८}}
\vspace{1mm}
\end{matrix}$}\, इदमत्रावधेयम्~। इष्टाहतस्वस्वहरेणेति क्षेपे कर्तव्ये यदि प्रथमत उत्पन्नयोर्लब्धिगुणयोः क्रियते तदा यादृशाभ्यां भाज्यहराभ्यामुत्पन्नौ लब्धिगुणौ तावेवेष्टगुणौ क्षेपौ भवतः~। यथात्रैकेनेष्टेन \,{\scriptsize $\begin{matrix}
\mbox{{१३}}\\
\vspace{-1.5mm}
\mbox{{८१}}
\vspace{1mm}
\end{matrix}$}\, पश्चाल्लब्धिरपवर्ताङ्केन १० गुण्या~। एवं जातावेकेनेष्टेन लब्धिगुणौ \,{\scriptsize $\begin{matrix}
\mbox{{१३०}}\\
\vspace{-1.5mm}
\mbox{{८१}}
\vspace{1mm}
\end{matrix}$}\,~। एवं युतिभाजकमात्रापवर्तेऽपि क्षेपानन्त-रमपवर्ताङ्केन गुणो गुणनीयः~। एवमेव युतिभाज्ययोर्युतिभाजकयोश्चापवर्ते क्षेपानन्तरमेव स्वस्वापवर्तेन लब्धिगुणौ गुणनीयौ~। यदि तु स्वोद्दिष्टसिद्धयोर्लब्धिगुणयोः क्षेपः क्रियते तदोद्दिष्टभाज्यहरावेवेष्टगुणौ क्षेपौ भवतः~। यथात्र लब्धिगुणौ \,{\scriptsize $\begin{matrix}
\mbox{{३०}}\\
\vspace{-1.5mm}
\mbox{{१८}}
\vspace{1mm}
\end{matrix}$}\, एकेनेष्टेन क्षेपौ \,{\scriptsize $\begin{matrix}
\mbox{{१००}}\\
\vspace{-1.5mm}
\mbox{{६३}}
\vspace{1mm}
\end{matrix}$}\, स्वस्वक्षेपयुतौ जातौ लब्धिगुणौ तावेव \,{\scriptsize $\begin{matrix}
\mbox{{१३०}}\\
\vspace{-1.5mm}
\mbox{{८१}}
\vspace{1mm}
\end{matrix}$}~। यत्र तु त्रयाणामप्यपवर्तः क्रियते तत्रेष्टगुणयोः दृढभाज्यहारयोरेव यदा कदापि क्षेपत्वं सम्भवतीत्यादि सुधीभिः सर्वत्रोह्यम्~। अथवा हरक्षेपौ नवभिरपवर्त्य न्यासः \,{\scriptsize $\begin{matrix}
\mbox{{भा ~१०० ~क्षे ~१०}}\\
\vspace{-1.5mm}
\mbox{{ह ~~~~७ ~~~~~~~~~~}}
\vspace{1mm}
\end{matrix}$}~। पूर्ववद्वल्ली \;{\scriptsize $\begin{matrix}
\mbox{{१४}}\\
\mbox{{३}}\\
\mbox{{१०}}\\
\vspace{-1.5mm}
\mbox{{०}}
\vspace{2mm}
\end{matrix}$}\; जातं राशिद्वयं \,{\scriptsize $\begin{matrix}
\mbox{{४३०}}\\
\vspace{-1.5mm}
\mbox{{३०}}
\vspace{1mm}
\end{matrix}$}\, तक्षणे जातं \,{\scriptsize $\begin{matrix}
\mbox{{३०}}\\
\vspace{-1.5mm}
\mbox{{२}}
\vspace{1mm}
\end{matrix}$}\, हरक्षेपापवर्ताङ्केन ९ गुणं सङ्गुण्य वा जातौ लब्धिगुणौ तावेव \,{\scriptsize $\begin{matrix}
\mbox{{३०}}\\
\vspace{-1.5mm}
\mbox{{१८}}
\vspace{1mm}
\end{matrix}$}~। अथवा भाज्यक्षेपौ हरक्षेपौ चापवर्त्य न्यासः \,{\scriptsize $\begin{matrix}
\mbox{{भा ~१० ~क्षे ~१}}\\
\vspace{-1.5mm}
\mbox{{ह ~~७ ~~~~~~}}
\vspace{1mm}
\end{matrix}$}~। पूर्ववद्वल्ली \;{\scriptsize $\begin{matrix}
\mbox{{१}}\\
\mbox{{२}}\\
\mbox{{१}}\\
\vspace{-1.5mm}
\mbox{{०}}
\vspace{2mm}
\end{matrix}$}\; जातं राशिद्वयं \,{\scriptsize $\begin{matrix}
\mbox{{३}}\\
\vspace{-1.5mm}
\mbox{{२}}
\vspace{1mm}
\end{matrix}$}~। अत्र भाज्यक्षेपापवर्तेन १० लब्धिं सङ्गुण्य हरक्षेपापवर्तेन ९ गुणं च सङ्गुण्य वा जातौ लब्धिगुणौ तावेव \,{\scriptsize $\begin{matrix}
\mbox{{३०}}\\
\vspace{-1.5mm}
\mbox{{१८}}
\vspace{1mm}
\end{matrix}$}

\end{sloppypar}

\newpage

\begin{sloppypar}
\noindent एकेनेष्टेनोक्तवल्लब्धिगुणौ \,{\scriptsize $\begin{matrix}
\mbox{{१३०}}\\
\vspace{-1.5mm}
\mbox{{८१}}
\vspace{1mm}
\end{matrix}$}\, द्विकेन वा \,{\scriptsize $\begin{matrix}
\mbox{{२३०}}\\
\vspace{-1.5mm}
\mbox{{१४४}}
\vspace{1mm}
\end{matrix}$}~। अत्र प्रथमन्यासे तृतीयन्यासे च हरतष्टे धनक्षेप इत्यपि प्रकारः सम्भवति~। अथवा भागहारेण तष्टयोः क्षेपभाज्ययोरित्यपि~। अथ द्वितीयोदाहरणे न्यासः \,{\scriptsize $\begin{matrix}
\mbox{{भा ~१०० ~क्षे ~९०}}\\
\vspace{-1.5mm}
\mbox{{ह ~~६३ ~~~~~~~~~}}
\vspace{3mm}
\end{matrix}$}\, \hyperref[5.54]{'योगजे तक्षणाच्छुद्धे गुणाप्ती स्तो वियोगजे'} इति उक्तत्वाद्योगजौ लब्धिगुणौ \,{\scriptsize $\begin{matrix}
\mbox{{३०}}\\
\vspace{-1.5mm}
\mbox{{१८}}
\vspace{1mm}
\end{matrix}$} स्वतक्षणाभ्यामाभ्यां शोधितौ जातौ नवतिवियोगे लब्धिगुणौ \,{\scriptsize $\begin{matrix}
\mbox{{७०}}\\
\vspace{-1.5mm}
\mbox{{४५}}
\vspace{1mm}
\end{matrix}$}~। एवं सर्वेष्वपि प्रकारेषु बोध्यम्~। अत्रापि क्षेपवशादानन्त्यम्~॥~६१~॥\\

{\small अथ \hyperref[5.54]{'धनभाज्योद्भवे तद्वत्'} इत्यस्योदाहरणद्वयं रथोद्धतयाह\textendash }

\phantomsection \label{5.62}
\begin{quote}
{\large \textbf{{\color{purple}यद्गुणाक्षयगषष्टिरन्विता वर्जिता च यदि वा त्रिभिस्ततः~।\\
स्यात्त्रयोदशहृता निरग्रका तं गुणं गणक मे पृथग्वद~॥~६२~॥}}}
\end{quote}

क्षेपस्य धनत्वेनैकमृणत्वेन द्वितीयमित्युदाहरणद्वयम्~। शेषं स्पष्टम्~।\\

न्यासः \,{\scriptsize $\begin{matrix}
\mbox{{भा ~६० ~क्षे ~३}}\\
\vspace{-1.5mm}
\mbox{{ह ~~१३ ~~~~~~}}
\vspace{1mm}
\end{matrix}$}\, वल्ली \;{\scriptsize $\begin{matrix}
\mbox{{४}}\\
\mbox{{१}}\\
\mbox{{१}}\\
\mbox{{१}}\\
\mbox{{१}}\\
\mbox{{३}}\\
\vspace{-1.5mm}
\mbox{{०}}
\vspace{2mm}
\end{matrix}$}\; जातं \,{\scriptsize $\begin{matrix}
\mbox{{६९}}\\
\vspace{-1.5mm}
\mbox{{१५}}
\vspace{1mm}
\end{matrix}$}\, तक्षणे जातं \,{\scriptsize $\begin{matrix}
\mbox{{९}}\\
\vspace{-1.5mm}
\mbox{{२}}
\vspace{1mm}
\end{matrix}$}~। लब्धयो विषमा इति स्वतक्षणाभ्यां \,{\scriptsize $\begin{matrix}
\mbox{{६०}}\\
\vspace{-1.5mm}
\mbox{{१३}}
\vspace{1mm}
\end{matrix}$}\, विशोध्य जातौ लब्धिगुणौ \,{\scriptsize $\begin{matrix}
\mbox{{५१}}\\
\vspace{-1.5mm}
\mbox{{११}}
\vspace{1mm}
\end{matrix}$}\, धनभाज्ये धनक्षेपे च~। धनभाज्योद्भवे तद्वदित्युक्तत्वात्स्वतक्षणशुद्धौ जातावृणभाज्ये धनक्षेपे च लब्धिगुणौ \,{\scriptsize $\begin{matrix}
\mbox{{९}}\\
\vspace{-1.5mm}
\mbox{{२}}
\vspace{1mm}
\end{matrix}$}~। अत्र भाज्य-भाजकयोर्विजातीययोर्भागहरेऽपि चैवं निरुक्तमित्युक्तत्वाल्लब्धेर्ऋणत्वं ज्ञेयम् \,{\scriptsize $\begin{matrix}
\mbox{{९}}\\
\vspace{-1.5mm}
\mbox{{२}}
\vspace{1mm}
\end{matrix}$}~। पुनरेतौ स्वतक्षणाभ्यामाभ्यां \,{\scriptsize $\begin{matrix}
\mbox{{६०}}\\
\vspace{-1.5mm}
\mbox{{१३}}
\vspace{1mm}
\end{matrix}$}\, शोधितौ जातावृणभाज्यक्षेपयोर्लब्धिगुणौ \,{\scriptsize $\begin{matrix}
\mbox{{५१}}\\
\vspace{-1.5mm}
\mbox{{११}}
\vspace{1mm}
\end{matrix}$}~। अत्रापि
\end{sloppypar}

\newpage

\begin{sloppypar}
\noindent हरभाज्ययोर्विजातीयत्वाल्लब्धेर्ऋणत्वमिति जातौ \,{\scriptsize $\begin{matrix}
\mbox{{५ं१}}\\
\vspace{-1.5mm}
\mbox{{११}}
\vspace{1mm}
\end{matrix}$}~।\\

अत्रेदमवधेयम्\textendash \,प्रथमतो भाज्यभाजकक्षेपाणां धनत्वमेव प्रकल्प्य लब्धिगुणौ साध्यौ~। अथ यद्युद्दिष्टभाज्यक्षेपयोर्धनत्वमृणत्वं वा स्यात्तदा साधितगुणाप्तिभ्यामेवोद्दिष्टसिद्धिः~। यदा तु भाज्यक्षेपयोरन्यतरस्य धनत्वमृणत्वमितरस्य तदा यथागतौ लब्धिगुणौ स्वतक्षणाभ्यां शोध्यौ ताभ्याम् उद्दिष्टसिद्धिः~। हरस्य धनत्व ऋणत्वे वा न कश्चित् कुट्टके विशेषः~। उक्तरीत्या गुणाप्त्योर्धनत्वम् एव~। भाज्यभाजकयोर्मध्य एकस्यैव ऋणत्वे लब्धिमात्रस्यर्णत्वं ज्ञेयम्~। भागहारेऽपि चैवं निरुक्तमित्युक्तत्वादिति सङ्क्षेपः~। एवमेकवारशोधनेनैवोद्दिष्ट-सिद्धिर्भवति~। यत्तु भाज्ये ऋणगते स्वतक्षणाच्छोधनमेकं क्षेप ऋणगते पुनर्द्वितीयमित्युक्तं तद्बालबोधार्थम्~। अयमर्थ आचार्येणैव विवृतः~। "\hyperref[5.54]{धनभाज्योद्भवे तद्वद्भवेतामृणभाज्यजे} इति मन्दावबोधार्थं मयोक्तमन्यथा \hyperref[5.54]{योगजे तक्षणाच्छुद्धे} इत्यादिनैव तत्सिद्धेः~। यतो धनर्णयोगो वियोग एव~। अत एव भाज्यभाजकक्षेपाणां धनत्वमेव प्रकल्प्य गुणाप्ती साध्ये~। ते योगजे भवतस्ते स्वतक्षणाभ्यां शुद्धे वियोगजे कार्ये" इत्यादिना~। एवमृण-भाज्येऽप्यप्रयासेनैव कुट्टकसिद्धौ सत्यामप्यन्यैर्वृथा प्रयासः कृत इत्याह "भाज्ये भाजके वा ऋणगते परस्परभजनाल्लब्धय ऋणगताः स्थाप्या इति किं प्रयासेन" इति~। अत्र क्षेपस्य-र्णत्वे धनत्वे वोपान्तिमेन स्वोर्ध्वे हत इत्यादिकरणे धनर्णत्वावधानेन प्रयासगौरवं द्रष्टव्यम्~। न केवलं प्रयासः~। अपि तु लब्धौ व्यभिचारोऽपि~। तथाहि\textendash \,प्रकृतोदाहरणे न्यासः \,{\scriptsize $\begin{matrix}
\mbox{{भा ~६ंं० ~क्षे ~३}}\\
\vspace{-1.5mm}
\mbox{{ह ~~१३ ~~~~~~}}
\vspace{1mm}
\end{matrix}$}\, उक्तवद्वल्ली \;{\scriptsize $\begin{matrix}
\mbox{{४ं}}\\
\mbox{{१ं}}\\
\mbox{{१ं}}\\
\mbox{{१ं}}\\
\mbox{{१ं}}\\
\mbox{{३}}\\
\vspace{-1.5mm}
\mbox{{०}}
\vspace{2mm}
\end{matrix}$}\; जातं राशिद्वयं \,{\scriptsize $\begin{matrix}
\mbox{{६ं९}}\\
\vspace{-1.5mm}
\mbox{{१५}}
\vspace{1mm}
\end{matrix}$}\, तक्षणे जातं \,{\scriptsize $\begin{matrix}
\mbox{{९ं}}\\
\vspace{-1.5mm}
\mbox{{२}}
\vspace{1mm}
\end{matrix}$}\, लब्धिवैषम्यात्स्वतक्षणशुद्धौ जातौ लब्धिगुणावृणभाज्ये धनक्षेपे च \,{\scriptsize $\begin{matrix}
\mbox{{५ं१}}\\
\vspace{-1.5mm}
\mbox{{११}}
\vspace{1mm}
\end{matrix}$}~। अत्र लब्धौ व्यभिचारः~। यतोऽनेन ११ भाज्येऽस्मिन् ६ं० गुणिते ६६ं० क्षेप\textendash \,३\textendash \,युते ६५ं७ हरभक्ते लब्धिः ५ं० शेषं च ७~। नन्वत्र शेषसत्वात् गुणोऽपि व्यभिचारी~। तत्कथ-
\end{sloppypar}

\newpage

\begin{sloppypar}
\noindent मुक्तं लब्धौ व्यभिचारः स्यादिति~। सत्यम्~। न ह्यत्र लब्धावेवेत्यवधारणमस्ति~। किं तु लब्धावित्युपलक्षणम्~। तेन गुणेऽपि व्यभिचारः स्यादित्यर्थः~। लब्धिकाले व्यभिचार-निश्चयाल्लब्धौ व्यभिचारः स्यादित्युक्तमिति~। नन्वत्र नास्ति व्यभिचारः~। तथाहि\textendash \,अत्र उक्तवज्जातं राशिद्वयं \,{\scriptsize $\begin{matrix}
\mbox{{६ं९}}\\
\vspace{-1.5mm}
\mbox{{१५}}
\vspace{1mm}
\end{matrix}$}\, तक्षणे जातौ लब्धिगुणौ \,{\scriptsize $\begin{matrix}
\mbox{{९ं}}\\
\vspace{-1.5mm}
\mbox{{२}}
\vspace{1mm}
\end{matrix}$}\, अनेन २ भाज्येऽस्मिन् ६ं० गुणिते १२ं० क्षेप\textendash \,३\textendash \,युते ११ं७ हरभक्ते लब्धिरियं ९ं इति चेन्न~। तत् किं विषमलब्धिष्वपि स्वतक्षणाच्छोधनमपाकतुर्मुद्यतोऽसि~। तथा सति भाज्यभाजकक्षेपाणां धनत्वे लब्धीनां विषमत्वे च व्यभिचारस्तावत् स्यात्~। यथास्मिन्नेवोदाहरण उक्तवल्लब्धिगुणौ \,{\scriptsize $\begin{matrix}
\mbox{{९}}\\
\vspace{-1.5mm}
\mbox{{२}}
\vspace{1mm}
\end{matrix}$}\, अनेन २ भाज्ये ६० गुणिते १२० क्षेप\textendash \,३\textendash \,युते १२३ हर\textendash \,१३\textendash \,भक्ते निःशेषता न स्यात्~। अथ यद्युच्येत धनविषमलब्धिषु स्वतक्षणाच्छोधनमावश्यकं न त्वृणलब्धिष्विति चेन्न~। व्यभिचारः तावत्स्यात्~। यथास्मिन्नेवोदाहरणे हरमात्रस्यर्णत्व उक्तवज्जातौ लब्धिगुणौ \,{\scriptsize $\begin{matrix}
\mbox{{९ं}}\\
\vspace{-1.5mm}
\mbox{{२}}
\vspace{1mm}
\end{matrix}$}\, अनेन २ भाज्ये ६० गुणिते १२० क्षेप\textendash \,३\textendash \,युते १२३ हरभक्ते निरग्रताया अभावात्~। किं च समलब्धिष्वप्यस्ति व्यभिचारसम्भवः~। यथाष्टादश गुणाः केनेत्यनुपदवक्ष्यमाणोदाहरणे~। तथाहि\textendash \;{\scriptsize $\begin{matrix}
\mbox{{भा ~१८ ~क्षे ~१०}}\\
\vspace{-1.5mm}
\mbox{{ह ~~११ं ~~~~~~}}
\vspace{1mm}
\end{matrix}$}\, अत्र वल्ली \;{\scriptsize $\begin{matrix}
\mbox{{१ं}}\\
\mbox{{१ं}}\\
\mbox{{१ं}}\\
\mbox{{१ं}}\\
\mbox{{१०}}\\
\vspace{-1.5mm}
\mbox{{०}}
\vspace{2mm}
\end{matrix}$}\; जातं राशिद्वयं \,{\scriptsize $\begin{matrix}
\mbox{{५०}}\\
\vspace{-1.5mm}
\mbox{{३ं०}}
\vspace{1mm}
\end{matrix}$}\, तक्षणे \,{\scriptsize $\begin{matrix}
\mbox{{१४}}\\
\vspace{-1.5mm}
\mbox{{८ं}}
\vspace{1mm}
\end{matrix}$}\, अत्र गुणेन ८ं भाज्ये १८ गुणिते १४ं४ क्षेप\textendash \,१०\textendash \,युते १३ं४ हर\textendash \,११ं\textendash \,भक्ते लब्धिः १२ शेषं २ं इत्यूह्यम्~। अत्र समलब्धिषु हरस्यर्णत्वे सति विषमलब्धिषु भाज्यस्यर्णत्वे सति वा पूर्वेषां कुट्टके व्यभिचार इति निष्कर्षः~॥~६२~॥
\end{sloppypar}

\newpage

\begin{sloppypar}
{\small अथ भाजकस्यर्णत्वेऽनुष्टुभोदाहरणमाह\textendash }

\phantomsection \label{5.63}
\begin{quote}
{\large \textbf{{\color{purple}अष्टादश गुणाः केन दशाढ्या वा दशोनिताः~।\\
शुद्धं भागं प्रयच्छन्ति क्षयगैकादशोद्धृताः~॥~६३~॥}}}
\end{quote}

अष्टादशेति च्छेदः~। स्पष्टमन्यत्~। न्यासः \;{\scriptsize $\begin{matrix}
\mbox{{भा ~१८ ~क्षे ~१०}}\\
\vspace{-1.5mm}
\mbox{{ह ~~११ं ~~~~~~~}}
\vspace{1mm}
\end{matrix}$}\, वल्ली \;{\scriptsize $\begin{matrix}
\mbox{{१}}\\
\mbox{{१}}\\
\mbox{{१}}\\
\mbox{{१}}\\
\mbox{{१०}}\\
\vspace{-1.5mm}
\mbox{{०}}
\vspace{2mm}
\end{matrix}$}\; राशिद्वयं \,{\scriptsize $\begin{matrix}
\mbox{{५०}}\\
\vspace{-1.5mm}
\mbox{{३०}}
\vspace{1mm}
\end{matrix}$}\, तक्षणे जातं \,{\scriptsize $\begin{matrix}
\mbox{{१४}}\\
\vspace{-1.5mm}
\mbox{{८}}
\vspace{1mm}
\end{matrix}$}\, त्रयाणां धनत्वे जातावेतौ लब्धिगुणौ~। हरमात्रस्यर्णत्वेऽप्येतावेव लब्धिगुणौ किन्तु लब्धिमात्रमृणं भागहारेऽपि चैवं निरुक्तमित्युक्तत्वात्~। एवमृणहरे जातौ लब्धिगुणौ \,{\scriptsize $\begin{matrix}
\mbox{{१ं४}}\\
\vspace{-1.5mm}
\mbox{{८}}
\vspace{1mm}
\end{matrix}$}~। अथर्णक्षेपे \hyperref[5.54]{'योगजे तक्षणाच्छुद्धे'} इत्यादिना जातौ \,{\scriptsize $\begin{matrix}
\mbox{{४}}\\
\vspace{-1.5mm}
\mbox{{३}}
\vspace{1mm}
\end{matrix}$}~। अत्र हरस्य धनत्वे ऋणत्वे वा लब्धिगुणावेतावेव~। किन्तु हरस्यर्णत्वे लब्धेर्ऋणत्वं ज्ञेयम्~। अत्र सर्वत्र ऋणत्वनिमित्तं यत्स्वतक्षणाच्छोधनं तद्भाज्यक्षेपयोरेकतरस्यैव ऋणत्वे~। नान्यथा~। तथा भाज्यभाजकयोरेकतरस्यैवर्णत्वे लब्धेर्ऋणत्वं न त्वन्यथेति निष्कर्षः~। केचित् 'ऋण-भाज्योद्भवे तद्वद्भवेतामृणभाजके' इति पाठं कल्पयित्वा भाजकर्णत्वेऽपि स्वतक्षणाच्छोधनं कुर्वन्ति~। तदसदिति प्रतिभाति~। यथास्मिन्नुदाहरणे त्रयाणां धनत्वे जातौ लब्धिगुणौ \,{\scriptsize $\begin{matrix}
\mbox{{१४}}\\
\vspace{-1.5mm}
\mbox{{८}}
\vspace{1mm}
\end{matrix}$}~। अथ हरमात्रस्यर्णत्वे स्वतक्षणाभ्यां शोधितौ जातौ \,{\scriptsize $\begin{matrix}
\mbox{{४}}\\
\vspace{-1.5mm}
\mbox{{३}}
\vspace{1mm}
\end{matrix}$}~। अनेन ३ भाज्येऽस्मिन् १८ गुणिते ५४ क्षेप\textendash \,१०\textendash \,युते ६४ हर\textendash \,१ं१\textendash \,भक्ते लब्धिरियं ५ं~। शेषं च ९~। तस्मादिदमसत्~। यद्युच्येत भाज्यो भाजको वा यादृश उद्दिष्टस्तादृशस्यैव तक्षणत्वमिति स्वतक्षणाभ्यामाभ्यां \,{\scriptsize $\begin{matrix}
\mbox{{१८}}\\
\vspace{-1.5mm}
\mbox{{११}}
\vspace{1mm}
\end{matrix}$}\, शोधितौ जातौ लब्धिगुणौ \,{\scriptsize $\begin{matrix}
\mbox{{४}}\\
\vspace{-1.5mm}
\mbox{{३}}
\vspace{1mm}
\end{matrix}$}~। नात्र कोऽपि दोष 
\end{sloppypar}

\newpage

\begin{sloppypar}
\noindent इति~। न~। \hyperref[1.7]{'संशोध्यमानं स्वमृणत्वमेति'} इत्यादिना शोधने कृते जातो गुणः १ं९~। सोऽयमसत्~। न च तक्षणस्यर्णत्वे तक्ष्यस्याप्यृणत्वमिति प्रथमतो गुणस्य ८ं ऋणत्वे संशोध्यमानमृणं धनं भवतीत्यादिना जातोऽस्मदुक्त एव गुणः ३~। न ह्यसावसन्निति वाच्यम्~। तत्किं बीजान्तरमधीतवानसि~। न ह्यस्मिन्बीज ईदृशोऽर्थः कस्मिन्नपि सूत्रे प्रतिपादितोऽस्ति~। अथास्त्वाचार्याभिप्रायज्ञः स्वतः कल्पको वा भवान्~। इदं तु पृच्छ्यते~। अधोराशिर्धनहरेण तष्टः सन् योगजो गुणो भवेदुत क्षयहरेण तष्टः सन्~। तत्र क्षयतक्षणे भवन्मते गुणस्यापि क्षयत्वम्~। न ह्यस्य ८ं योगजत्वमस्ति~। भजने निरग्रताया अभावात्~। अस्य ८ं अयोगजत्वे \hyperref[5.54]{'योगजे तक्षणाच्छुद्धे'} इति सूत्रं कथं प्रवर्तेत येन त्वदभिमतो गुणः ३ं सिध्येत्~। धनतक्षणे तु गुणस्य धनत्वे ८ संशोध्यमानं स्वमृणत्वमेतीति क्षयत्वे जाते ८ं तक्षणस्य ११ धनत्वे ऋणत्वे च शोधनेन जातौ क्रमेण गुणौ ३।१९ं~। अनयोर्दुष्टत्वं स्पष्टमेवेत्यलं पल्लवितेन~॥~६३~॥\\

{\small अथ \hyperref[5.55]{'गुणलब्ध्योः समं ग्राह्यम्'} इति \hyperref[5.56]{'हरतष्टे धनक्षेपे'} इति \hyperref[5.57]{'अथवा भागहारेण तष्टयोः'} इति चैतेषामुदाहरणमनुष्टुभाह\textendash }

\phantomsection \label{5.64}
\begin{quote}
{\large \textbf{{\color{purple}येन सङ्गुणिताः पञ्च त्रयोविंशतिसंयुताः~।\\
वर्जिता वा त्रिभिर्भक्ता निरग्राः स्युः स को गुणः~॥~६४~॥}}}
\end{quote}

स्पष्टोऽर्थः~। न्यासो \;{\scriptsize $\begin{matrix}
\mbox{{भा ~५ ~क्षे ~२३}}\\
\vspace{-1.5mm}
\mbox{{ह ~~३ ~~~~~~~}}
\vspace{1mm}
\end{matrix}$}\, प्राग्वद्वल्ली \;{\scriptsize $\begin{matrix}
\mbox{{१}}\\
\mbox{{१}}\\
\mbox{{२३}}\\
\vspace{-1.5mm}
\mbox{{०}}
\vspace{2mm}
\end{matrix}$}\; राशिद्वयं \,{\scriptsize $\begin{matrix}
\mbox{{४६}}\\
\vspace{-1.5mm}
\mbox{{२३}}
\vspace{1mm}
\end{matrix}$}~। अत्र तक्षणेऽधोराशौ सप्त लभ्यन्ते~। ऊर्ध्वराशौ तु नव~। ते नव न ग्राह्याः~। \hyperref[5.55]{'गुणलब्ध्योः समं ग्राह्यं धीमता तक्षणे फलम्'} इत्यतः सप्तैव ग्राह्या इति जातौ लब्धिगुणौ \,{\scriptsize $\begin{matrix}
\mbox{{११}}\\
\vspace{-1.5mm}
\mbox{{२}}
\vspace{1mm}
\end{matrix}$}\, योगजौ~। अनयोः स्वस्वतक्षणाच्छोधने जातौ वियोगजौ लब्धिगुणौ \,{\scriptsize $\begin{matrix}
\mbox{{६ं}}\\
\vspace{-1.5mm}
\mbox{{१}}
\vspace{1mm}
\end{matrix}$}~। वियोगे धनलब्ध्यपेक्षा चेत्तर्हि \hyperref[5.59]{'इष्टाहतस्वस्वहरेण युक्ते'} इत्यादिना द्विकेनेष्टेन जातौ लब्धिगुणौ \,{\scriptsize $\begin{matrix}
\mbox{{४}}\\
\vspace{-1.5mm}
\mbox{{७}}
\vspace{1mm}
\end{matrix}$}~। एवं सर्वत्र~। अथवा \hyperref[5.56]{'हरतष्टे धनक्षेपे'} इति न्यासः \;{\scriptsize $\begin{matrix}
\mbox{{भा ~५ ~क्षे ~२}}\\
\vspace{-1.5mm}
\mbox{{ह ~~३ ~~~~~~}}
\vspace{1mm}
\end{matrix}$}\, वल्ली \;{\scriptsize $\begin{matrix}
\mbox{{१ं}}\\
\mbox{{१}}\\
\mbox{{२}}\\
\vspace{-1.5mm}
\mbox{{०}}
\vspace{2mm}
\end{matrix}$}\; राशि-

\end{sloppypar}

\newpage

\begin{sloppypar}
\noindent द्वयं \,{\scriptsize $\begin{matrix}
\mbox{{४}}\\
\vspace{-1.5mm}
\mbox{{२}}
\vspace{1mm}
\end{matrix}$}\, एतौ योगजौ लब्धिगुणौ~। तक्षणशोधनेन जातौ वियोगजौ \,{\scriptsize $\begin{matrix}
\mbox{{१}}\\
\vspace{-1.5mm}
\mbox{{१}}
\vspace{1mm}
\end{matrix}$}~। अत्र \hyperref[5.57]{'क्षेपतक्षण-लाभाढ्या लब्धिः शुद्धौ तु वर्जिता'} इति क्षेपतक्षणलाभेन ७ योगजलब्धि\textendash \,४\textendash \,युता ११ शुद्धौ तु लब्धिः १ वर्जिता ६ जातौ तावेव लब्धिगुणौ \,{\scriptsize $\begin{matrix}
\mbox{{११~। ६ं~}}\\
\vspace{-1.5mm}
\mbox{{~२~। १~}}
\vspace{1mm}
\end{matrix}$}~। \hyperref[5.57]{'अथवा भागहारेण तष्टयोः'} इति न्यासो \;{\scriptsize $\begin{matrix}
\mbox{{भा ~२ ~क्षे ~२}}\\
\vspace{-1.5mm}
\mbox{{ह ~~३ ~~~~~~}}
\vspace{1mm}
\end{matrix}$}\, वल्ली \;{\scriptsize $\begin{matrix}
\mbox{{०}}\\
\mbox{{१}}\\
\mbox{{२}}\\
\vspace{-1.5mm}
\mbox{{०}}
\vspace{2mm}
\end{matrix}$}\; राशिद्वयं \,{\scriptsize $\begin{matrix}
\mbox{{२}}\\
\vspace{-1.5mm}
\mbox{{२}}
\vspace{1mm}
\end{matrix}$}~। अत्रापि जातः पूर्व एव गुणः~। लब्धिस्तु \hyperref[5.57]{'भाज्याद्धतयुतोद्धृतात्'} इति गुण\textendash \,२\textendash \,गुणितो भाज्यः ५ जातः १० क्षेप\textendash \,२३\textendash \,युतः ३३ हरेण ३ भक्तो जाता लब्धिः सैव ११~। अथवा मदुक्तप्रकारेण लब्धिः~। गुणेन २ भाज्यतक्षणलाभो १ गुणितः २ क्षेपतक्षणलाभेन ७ संस्कृतः ९ गणितागतलब्ध्या च संस्कृतः ११ जाता सैव लब्धिः~। एवं सर्वत्र~॥~९४~॥\\

{\small अथ \hyperref[5.58]{'क्षेपाभावोऽथवा यत्र क्षेपः शुध्येद्धरोद्धृतः'} इत्यनयोरुदाहरणे रथोद्धतयाह\textendash }

\phantomsection \label{5.65}
\begin{quote}
{\large \textbf{{\color{purple}येन पञ्च गुणिताः खसंयुताः पञ्चषष्टिसहिताश्च तेऽथवा~।\\
स्युस्त्रयोदश हृता निरग्रकास्तं गुणं गणक कीर्तयाशु मे~॥~६५~॥}}}
\end{quote}

स्पष्टोऽर्थेः~। उदाहरणद्वयेऽपि न्यासो \;{\scriptsize $\begin{matrix}
\mbox{{भा ~५ ~क्षे ~०}}\\
\vspace{-1.5mm}
\mbox{{ह ~१३ ~~~~~~}}
\vspace{1mm}
\end{matrix}$}~। \;{\scriptsize $\begin{matrix}
\mbox{{भा ~५ ~क्षे ~६५}}\\
\vspace{-1.5mm}
\mbox{{ह ~१३ ~~~~~~~}}
\vspace{1mm}
\end{matrix}$}~। प्रथमे क्षेपाभावोऽस्ति~। द्वितीये क्षेपो हरोद्धृतः शुध्यतीत्युभयत्रापि शून्यमेव गुणः~। क्षेपो हरहृतः फलमिति द्वयोरपि लब्धी ०।५~। एवं जातौ लब्धिगुणौ \,{\scriptsize $\begin{matrix}
\mbox{{०~। ५~}}\\
\vspace{-1.5mm}
\mbox{{०~। ०~}}
\vspace{1mm}
\end{matrix}$}~। \hyperref[5.59]{'इष्टाहतस्वस्वहरेण युक्ते'} इत्यादिनैकेनेष्टेन १ जातौ \,{\scriptsize $\begin{matrix}
\mbox{{~५~। १०~}}\\
\vspace{-1.5mm}
\mbox{{१३~। १३~}}
\vspace{1mm}
\end{matrix}$}~। एवमिष्टवशादानन्त्यम्~। अथवात्र प्रथमप्रकारेण \hyperref[5.56]{'हरतष्टे धनक्षेपे'} इत्यनेन च गुणाप्ती साध्ये~॥~६५~॥
\end{sloppypar}

\newpage

\begin{sloppypar}
{\small अथ ग्रहगणिते विशेषोपयुक्तं स्थिरकुट्टकमुपजातिकोत्तरपूर्वार्धाभ्यामाह\textendash }

\phantomsection \label{5.66}
\begin{quote}
{\large \textbf{{\color{purple}क्षेपं विशुद्धिं परिकल्प्य रूपं पृथक्तयोर्ये गुणकारलब्धी~।\\
अभीप्सितक्षेपविशुद्धिनिघ्ने स्वहारतष्टे भवतस्तयोस्ते~॥~६६~॥}}}
\end{quote}

\hyperref[5.66]{\textbf{क्षेपं}} धनक्षेपम्~। \hyperref[5.66]{\textbf{विशुद्धि}}मृणक्षेपं \hyperref[5.66]{\textbf{रूपं परिकल्प्य तयो}}र्धनर्णक्षेपयोः \hyperref[5.66]{\textbf{पृथग्गुणकारलब्धी ये}} स्यातां ते अभीप्सितक्षेपविशुद्धिगुणितेः \hyperref[5.66]{\textbf{स्वहारतष्टे च तयोः}} क्षेपविशुद्ध्योस्ते गुणाप्ती \hyperref[5.66]{\textbf{भवतः}}~। एतदुक्तं भवति~। \hyperref[5.51]{'मिथो भजेत्तौ दृढभाज्यहारौ'} इत्यदिना फलान्यधोऽधो निवेश्य तदधः क्षेपस्थाने रूपं निवेश्यान्ते खं च निवेश्योपान्तिमेन स्योर्ध्वे हत इत्यादिना धनक्षेपे ऋणक्षेपे च गुणलब्धी पृथक्पृथक्साध्ये~। अथाभीप्सितक्षेपो यदि धनमस्ति तर्हि धनक्षेपजे गुणाप्ती अभीप्सितक्षेपेण गुणनीये~। यदि स्वभीप्सितक्षेपः क्षयोऽस्ति तर्हि ऋणक्षेपजे गुणाप्ती अभीप्सितेनर्णक्षेपेण गुणनीये~। पश्चात् स्वस्वहारेण पूर्ववत्तक्ष्ये~। ते उद्दिष्टगुणाप्ती स्तः~। अत्र मन्दविश्वासार्थमुदाहरणं प्रदर्शयति~। प्रथमोदाहरणे दृढभाज्यहारयो रूपक्षेपयोर्न्यासो \;{\scriptsize $\begin{matrix}
\mbox{{भा ~१७ ~क्षे ~१}}\\
\vspace{-1.5mm}
\mbox{{ह ~१५ ~~~~~~~}}
\vspace{1mm}
\end{matrix}$}~। अत्रोक्तवद्गुणाप्ती ७\,।\,८~। एते अभीष्टपञ्चगुणिते ३५\,।\,४०~। स्वहारतष्टे जाते ५।६~। ते एव गुणाप्ती~। अथ रूपशुद्धौ गुणाप्ती ८।९ एते पञ्चगुणे ४०।४५~। स्वहारतष्टे प्रथमोदाहरणे शुद्धिजे गुणाप्ती १०।११~। एवं सर्वत्रेति~। स्पष्टोऽर्थः~। सविस्तरं तु {\color{violet}'लिप्ताग्रं शशिनः खखभ्रगगनप्राणर्तुभूमिर्हृतम्'} इत्यादिना निबद्धः स्थिरकुट्टको {\color{violet}गोलाध्याये} दर्शितः~। स्थिरकुट्टकोपपत्तिस्तु \hyperref[5.53]{'भवति कुट्टविधेर्युतिभाज्ययोः'} इत्यस्योपपत्तौ प्रदर्शिता~। अथवा रूपक्षेपे यद्येते गुणाप्ती तर्हि स्वाभीष्टक्षेपे के इति त्रैराशिकेनोपपत्तिर्द्रष्टव्या~। ननु किमर्थमयं स्थिरकुट्टक उक्तः~। नहि प्रतिप्रश्नं तावेव भाज्यभाजकौ येन कृते स्थिरकुट्टके लाघवं स्यात् इत्यत आह\textendash \,अस्य ग्रहगणिते महानुपयोग इति~। अयमर्थः~। यद्यपि लौकिकेषु कुट्टकप्रश्नेषु प्रतिप्रश्नं भाज्यभाजकभेदान्न स्थिरकुट्टकोपयोगोऽस्ति~। तथापि ग्रहगणिते विविधक्षेपेषु तावेव भाज्यभाजकौ भवत इति तत्रास्त्येव स्थिरकुट्टकोपयोग इति~॥~६६~॥\\

{\small अथ यदि कश्चिद्ब्रूयाद्ग्रहगणिते स्थिरकुट्टकोपयोगः कुत्रास्ति~। तदर्थमुपदेशव्याजेन तत्स्थलम् उपजातिकोत्तरार्धेनोपजातिकया च दर्शयति\textendash }

\phantomsection \label{5.67}
\begin{quote}
{\large \textbf{{\color{purple}कल्प्याथ शुद्धिर्विकलावशेषं षष्टिश्च भाज्यः कुदिनानि हारः~।\\
तज्जं फलं स्युर्विकला गुणस्तु लिप्ताग्रमस्माच्च कलालवाग्रम्~।\\
एवं तदूर्ध्वं च तथाधिमासावमाग्रकाभ्यो दिवसा रवीन्द्वोः~॥~६७~॥}}}
\end{quote}

अस्यार्थः स्वयमेव विवृणोति~। "ग्रहस्य विकलाशेषात्तद्ग्रहाहर्गणयोरानयनम्~। तत्र
\end{sloppypar}

\newpage

\begin{sloppypar}
\noindent षष्टिर्भाज्यः कुदिनानि हारः~। विकलावशेषं शुद्धिरिति प्रकल्प्य गुणाप्ती साध्ये~। तत्र लब्धिर्विकलाः स्युः~। गुणस्तु कलावशेषम्~। एवं कलावशेषं शुद्धिं प्रकल्प्य~। तत्र लब्धिः कलाः~। गुणो भागशेषम्~। भागशेषं शुद्धिस्त्रिंशद्भाज्यः कुदिनानि हारस्तत्र फलं भागा गुणो राशिशेषम्~। द्वादशभाज्यः कुदिनानि हारः~। राशिशेषं शुद्धिस्तत्र फलं गतराशयः~। गुणो भगणशेषम्~। कल्पभगणा भाज्यः कुदिनानि हारः~। भगणशेषं शुद्धिः~। तत्र फलं गतभगणाः~। गुणोऽहर्गणः स्यादिति~। अस्योदाहरणानि प्रश्नाध्याये~। एवं कल्पाधिमासा भाज्यः~। रविदिनानि हारः~। अधिमासशेषं शुद्धिः~। फलं गताधिमासाः~। गुणो गतरविदिवसाः~। एवं युगावमानि भाज्यः~। चन्द्रदिवसा हरः~। अवमशेषं शुद्धिः~। फलं गतावमानि~। गुणो गतचान्द्रदिवसाः" इति~। अत्राचार्यव्याख्याने युगावमानि भाज्य इत्यत्र कल्पशब्दस्थाने युगेति लिखनं लेखकभ्रमजं द्रष्टव्यम्~। यद्वा न केवलं कल्पजैर्भगणकुदिनाधिमासावमादिभिर्ग्रहाहर्गणाद्यानयने विकलाशेषादेस्तदानयनं किं तु युगजैरपि कुदिनाद्यैस्तत्साधने तदुत्पन्नाद्विकलाशेषाद्युगजभाज्यभाजकेभ्योऽपि तत्साधनं भवतीति सूचनाय युगावमानीत्युक्तम्~। एवं यथासम्भवं युगचरणजैरपि कुदिनादिभिः ग्रहादिसाधने तादृशभाज्यभाजकेभ्यः कुट्टको ज्ञेयः~। अत एव सूत्रे कुदिनानि हार इत्येवोक्तं न तु कल्पकुदिनानीति~। अत्र मन्दप्रतीत्यर्थं कल्पितानि कल्पकुदिनानि १९ ग्रहभगणाः कल्पे कल्पिताः ९~। अहर्गणः १३~। अत्र कल्पकुदिनैः कल्पभगणास्तदाहर्गणतुल्यैः किमिति त्रैराशिकेन १९।९।१३ {\color{violet}'द्युचरचक्रहतो दिनसञ्चयः क्वहहृतो भगणादि फलं ग्रहः'} इत्यनेन सिद्धो भगणादिग्रहः ६।१।२६।५०।३१ विकलापर्यन्तम्~। विकला शेषं च ११~। अस्माद्विलोमगत्या ग्रहोऽहर्गणश्चानीयते \hyperref[5.67]{'कल्प्याथ शुद्धिर्विकलावशेषम्'} इत्यादिना~। अत्र कुट्टकार्थं न्यासो\textendash \;{\scriptsize $\begin{matrix}
\mbox{{भा ~६० ~क्षे ~११ं}}\\
\vspace{-1.5mm}
\mbox{{ह ~१९ ~~~~~~~~}}
\vspace{1mm}
\end{matrix}$}\, वल्ली \;{\scriptsize $\begin{matrix}
\mbox{{३}}\\
\mbox{{६}}\\
\mbox{{११}}\\
\vspace{-1.5mm}
\mbox{{०}}
\vspace{2mm}
\end{matrix}$}\; जातं राशिद्वयं \,{\scriptsize $\begin{matrix}
\mbox{{२०९}}\\
\vspace{-1.5mm}
\mbox{{६६}}
\vspace{1mm}
\end{matrix}$} तक्षणे जातौ लब्धिगुणौ \,{\scriptsize $\begin{matrix}
\mbox{{२९}}\\
\vspace{-1.5mm}
\mbox{{९}}
\vspace{1mm}
\end{matrix}$} \hyperref[5.54]{'योगजे तक्षणाच्छुद्धे'} इति जातौ लब्धिगुणौ \,{\scriptsize $\begin{matrix}
\mbox{{३१}}\\
\vspace{-1.5mm}
\mbox{{१०}}
\vspace{1mm}
\end{matrix}$} ऋणक्षेपे अत्र लब्धिः ३१ विकलाः~। गुणाः कलाशेषम् १०~। इदमृणक्षेपम् १ं०~। अथ कलानयनार्थं कुट्टके न्यासो \;{\scriptsize $\begin{matrix}
\mbox{{भा ~६० ~क्षे ~१ं०}}\\
\vspace{-1.5mm}
\mbox{{ह ~१९ ~~~~~~~~}}
\vspace{1mm}
\end{matrix}$}~। उक्त-वज्जातौ
\end{sloppypar}

\newpage

\begin{sloppypar}
\noindent लब्धिगुणौ ५०।१६~। अत्र लब्धिः कलाः ५०~। गुणो भागशेषं १६~। पुनर्भागशेषं शुद्धिरिति लवार्थं कुट्टके न्यासो \;{\scriptsize $\begin{matrix}
\mbox{{भा ~३० ~क्षे ~१ं६}}\\
\vspace{-1.5mm}
\mbox{{ह ~१९ ~~~~~~~~}}
\vspace{1mm}
\end{matrix}$}~। अत्राप्युक्तवल्लब्धिगुणौ २६।१७~। अत्र लब्धिर्भागाः २६ गुणो राशिशेषम् १७~। राशिशेषं शुद्धिरिति राशिज्ञानार्थं न्यासो \;{\scriptsize $\begin{matrix}
\mbox{{भा ~१२ ~क्षे ~१ं७}}\\
\vspace{-1.5mm}
\mbox{{ह ~१९ ~~~~~~~~}}
\vspace{1mm}
\end{matrix}$}~। अत्राप्युक्तवल्लब्धिगुणौ १।३~। अत्र लब्धिमितो राशिः १~। गुणो भगणशेषं ३~। भगणशेषं शुद्धिः~। कल्पभगणाः ९ भाज्यः कल्पकुदिनानि १९ हर इति न्यासो \;{\scriptsize $\begin{matrix}
\mbox{{भा ~९ ~क्षे ~३ं}}\\
\vspace{-1.5mm}
\mbox{{ह ~१९ ~~~~~~~~}}
\vspace{1mm}
\end{matrix}$}~। अत्राप्युक्तवल्लब्धिगुणौ ६।१३~। अत्र लब्धिर्गतभगणाः ६~। गुणोऽहर्गणः १३~। एवं मन्द-प्रतीत्यर्थमिष्टान्कल्पसौरदिवसान्कल्पाधिमासांश्च प्रकल्प्याधिमासशेषाद्गताधिमाससौर-दिवसा दर्शनीयाः~। एवमवमाग्राद्गतावमचान्द्रदिवसाश्च~। अस्त्यत्र ग्रहगणिते स्थिरकुट्टकस्य महत् प्रयोजनम्~। तथाहि~। विकलाग्राद्ग्रहानयने षष्टिर्भाज्यः~। कल्पकुदिनानि हार इति भाज्यभाजकौ नियतावेव~। विकलाशेषमृणक्षेपः स त्वनियतः~। अत्र स्थिरकुट्टकाकरणे प्रतिप्रश्नं दीर्घवल्लीसम्भूतयोर्लब्धिगुणयोः साधनेऽस्ति गौरवम्~। स्थिरकुट्टके तु रूपमृणक्षेपं प्रकल्प्य लब्धिगुणौ स्थिरौ कृत्वा तत्तद्विकलाशेषेण तयोर्गुणने सति स्वस्वहारेण तक्षणे च सति स्वाभीप्सितलब्धिगुणसिद्धिरित्यतिलाघवमस्ति~। अत उक्तमस्य ग्रहगणिते महान् उपयोग इति~।\\

अथ \hyperref[5.67]{'कल्प्याथ शुद्धिर्विकलाविशेषम्'}~। इत्यादावुपपत्तिः~। अत्र {\color{violet}'द्युचरचक्रहतो दिनसञ्चयः'} इत्यादिना ग्रहानयनेऽहर्गणः १३ कल्पभगणैः ९ गुणितः ११७ कल्पकुदिनैः १९ भक्तो लब्धं गतभगणाः ६ शेषं भगणशेषं ३ तद्द्वादशगुणं ३६ कुदिनैर्भक्तं लब्धं १ राशयः~। शेषं राशिशेषं १७ तत्त्रिंशता सङ्गुण्य ५१० कुदिनैर्भक्तं लब्धमंशाः २६ शेषमंशशेषं १६ तत् षष्ट्या सङ्गुण्य ९६० कुदिनैर्भक्तं लब्धं कलाः ५० शेषं कलाशेषं १० तत्पुनः षष्ट्या सङ्गुण्य ६०० कुदिनैर्भक्तं लब्धं विकलाः ३१ शेषं विकलाशेषम् ११~।\\

अथ व्यस्तविधिना विकलाशेषाद्ग्रहानयनम्~। तत्र युक्तिः~। अत्र कलाशेषे १० षष्ट्या गुणिते ६०० कुदिनभक्ते यच्छेषं तद्विकलाशेषम् ११~। तच्चेत्षष्टिगुणितात्कलाशेषादपनीयते ५८९ तदा तत्कुदिनभक्तं निःशेषं स्यात्~। लब्धिश्च विकलाः स्युः~। परमत्र कलाशेषस्याज्ञाने षष्टिगुणितस्य सुतरामज्ञानादुक्तविधिर्न सिध्यति~। अत्र षष्ट्या गुणितं कलाशेषं कलाशेषेण वा गुणिता षष्टिः समैव~। गुण्यगुणकयोरभेदात्~। तस्मात्षष्टिः
\end{sloppypar}

\newpage

\begin{sloppypar}
\noindent कलाशेषेण गुणिता विकलाशेषेणोना कुदिनभक्ता निःशेषा स्याल्लब्धिस्तु विकलाः स्युः~। प्रकृते षष्टिर्विकलाशेषं च ज्ञायते~। केवलं कलाशेषं न ज्ञायते~। तज्ज्ञातार्थम् उपायः~। षष्टिर्येन गुणिता सती विकलाशेषेणोना कुदिनभक्ता निःशेषा भवेत्तदेव कलाशेषं स्यात्~। अयम् अर्थश्च कुट्टकस्य विषयः~। षष्टिः केन गुणिता विकलाशेषेण रहिता कुदिनभक्ता निःशेषा स्यादिति प्रश्ने पर्यवसानात्~। अत्र यो गुणस्तदेव कलाशेषमुक्तयुक्तेः~। या लब्धिस्ता विकला उक्तयुक्तेरेव~। अत उपपन्नम् \hyperref[5.67]{'कल्प्याथ शुद्धिर्विकलावशेषं षष्टिश्च भाज्यः कुदिनानि हारः~। तज्जं फलं स्युर्विकला गुणस्तु लिप्ताग्रम्'} इति~। अथ कलाशेषात्कलाज्ञानम्~। तत्र भागशेषे षष्ट्या गुणिते कुदिनैर्भक्ते लब्धिः कला भवन्ति~। शेषं च कलाशेषम्~। अत उक्तयुक्त्या षष्टिर्भागशेषेण गुणिता कलाशेषेणोना कुदिनभक्ता निःशेषा स्याल्लब्धिश्च कलाः स्युः~। तत्र भागशेषरूपस्य गुणकस्याज्ञानादयमर्थः कुट्टकस्यैव विषयः~। षष्टिः केन गुणिता कलाशेषेणोना कुदिनभक्ता निःशेषा स्यादिति प्रश्ने पर्यवसानात्~। अत्र यो गुणः स एव भागशेषम्~। या लब्धिस्ताः कलाः~। उक्तयुतेः~। अत उक्तम् अस्माच्च कला-लवाग्रमिति~। अथ भागशेषाद्भागज्ञानम्~। तत्र राशिशेषे त्रिंशता गुणिते कुदिनैर्भक्ते लब्धिरंशा भवन्ति~। शेषं च भागशेषम्~। अत्राप्युक्तयुक्त्यैव त्रिंशत्केन गुणिता भागशेषोनाः कुदिनैर्भक्ता निःशेषाः स्युरिति कुट्टकविषयतास्ति~। अत्र यो गुणः स एव राशिशेषं स्याद्या लब्धिस्त एव भागाः स्युः~। अथ राशिशेषाद्राशिज्ञानम्~। तत्र भगणशेषे द्वादशगुणिते कुदिनैर्भक्ते लब्धी राशयः~। शेषं च राशिशेषम्~। अत्रापि द्वादशकेन गुणिता राशिशेषोनाः कुदिनैर्भक्ता निःशेषाः स्युरिति कुट्टकविषयतास्ति~। अत्र यो गुणस्तदेव भगणशेषं या लब्धिस्त एव गतराशयः स्युः~। अथ भगणशेषाद्गतभगणाहर्गणयोर्ज्ञानम्~। तत्र कल्प-भगणा अहर्गणगुणिताः कुदिनैर्भक्ता लब्धिर्गतभगणा भवन्ति~। शेषं च भगणशेषम्~। अतोऽत्रापि कल्पभगणाः केन गुणिता भगणशेषोनाः कुदिनैर्भक्ता निरग्रकाः स्युरिति कुट्टकविषयतास्ति~। अत्र यो गुणः स एवाहर्गणः~। या लब्धिस्त एव गतभगणाः~। अत उक्तम् \hyperref[5.67]{'एवं तदूर्ध्वं च'} इति~। एवं कल्पसौरदिवसैः कल्पाधिमासास्तदेष्टसौरैः कियन्त इति त्रैराशिकेन कल्पाधिमासेष्विष्टसौरदिवसैर्गुणितेषु कल्पसौरैर्भक्तेषु या लब्धिस्ते गताधिमासाः~। यच्छेषं तदधिमासशेषम्~। अतोऽत्रापि कल्पाधिमासाः कैर्गुणिता अधि-मासशेषोनाः कलासौरदिनैर्भक्ता निःशेषाः स्युरित्यस्ति कुट्टकविषयता~। अत्र यो गुणस्त एवेष्टसौरदिवसाः~। या लब्धिस्त एव गताधिमासाः~। एवं कल्पचान्द्रैः कल्पावमानि तदेष्ट-चान्द्रैः कियन्तीत्यनुपातेन कल्पावमेष्विष्टचान्द्रैर्गुणितेषु कल्पचान्द्रैर्भक्तेषु या लब्धिस्तानि गतावमानि भवन्ति~। शेषं चावमाग्रम्~।
\end{sloppypar}

\newpage

\begin{sloppypar}
\noindent अतोऽवमाग्राद्व्यस्तविधिना गतावमचान्द्राणामानयनमुक्तयुक्त्या कुट्टकेन सिध्येदेव~। अत उक्तम् \hyperref[5.67]{'तथाधिमासावमाग्रकाभ्यां दिवसा रवीन्द्वोः'} इति~। अत्रेदमवधेयम्~। विकला-शेषाद्ग्रहानयने विकलाशेषमृणक्षेपः षष्टिर्भाज्यः कल्पकुदिनानि हार इति प्रकल्प्य कुट्टकेन यौ लब्धिगुणौ ताविष्टाहतस्वस्वहरेण युक्तौ न विधेयौ~। योजने हि लब्धिः षष्टितोऽधिका स्याद्गुणश्च कुदिनतोऽधिकः स्यात्~। न चैतत् सम्भवति~। यतो लब्धिः विकला गुणश्च कलाशेषम्~। न हि विकलाः षष्टितोऽधिकाः सम्भवन्ति~। न वा कलाशेषं कुदिनतोऽधिकं सम्भवति~। कुदिनानां हरत्वात्~। अनयैव युक्त्या भगणशेषपर्यन्तं गुणलब्ध्योः क्षेपो न देयः~। भगणशेषाद्गतभगणाहर्गणयोरानयने तु क्षेपदाने यत्र बाधकं न स्यात्तत्र तादृशः क्षेपो देयः~। तस्माद्विकलाशेषाद्ग्रहानयने राश्यादिर्ग्रहो नियत एव~। गतभगणाहर्गणयोः त्वनियतत्वमिति सिद्धम्~। एवमधिमासावमाग्राभ्यां सौरचान्द्रदिनानयनेऽप्यनियतत्वम्~। मतिमद्भिरन्यदप्यूह्यम्~। अलं पल्लवितेन~॥~६७~॥\\

{\small एवमेकस्मिन्गुणके सति राशिज्ञानमभिधायाथ द्व्यादिषु गुणेषु सत्सु राशिज्ञानमुपजात्याह\textendash }

\phantomsection \label{5.68}
\begin{quote}
{\large \textbf{{\color{purple}एको हरश्चेद्गुणकौ विभिन्नौ तदा गुणैक्यं परिकल्प्य भाज्यम्~।\\
अग्रैक्यमग्रं कृत उक्तवद्यः संश्लिष्टसञ्ज्ञः स्फुटकुट्टकोऽसौ~॥~६८~॥}}}
\end{quote}

चेदेको हरः स्याद्गुणकौ तु विभिन्नौ स्तः~। गुणकावित्युपलक्षम्~। तेन त्र्यादयो वा गुणकाः स्युः~। एकस्यैव राशेः पृथक्पृथग्द्वौ गुणकौ त्रयश्चतुरादयो वा गुणकाः स्युः~।~सर्वत्र हरस्त्वेक एव स्यात्तदा तेषां द्व्यादीनां गुणकानामैक्यं भाज्यं परिकल्प्योद्दिष्टं यदग्रैक्यं तद-ग्रमृणक्षेपं प्रकल्प्यार्थाद्धरमेव हरं प्रकल्प्योक्तवद्यः कृतः स्फुटकुट्टकोऽसौ संश्लिष्टसञ्ज्ञः स्यात्~। संश्लिष्टस्फुटकुट्टकः~। अन्वर्थसञ्ज्ञेयम्~। तथाहि\textendash \,कुट्टके गुणकः~। संश्लिष्टानामेकी-भूतानामग्राणां सम्बन्धी स्फुटो विविक्तः कुट्टकः संश्लिष्टकुट्टकः~। स एव राशिः स्यादित्य-र्थात्सिद्धम्~। अत्र लब्धिर्न ग्राह्या~। अत्र हि यथोद्दिष्टैर्गुणकैः पृथग्गुणिते राशौ हरतष्टे सति या आगता लब्धयस्तदग्राणां चैक्ये हरतष्टे सति या लब्धयस्तासामैक्यं तदत्र कुट्टके लब्धिरूपमुत्पद्यते~। प्रयोजनाभावात्तन्न ग्राह्यम्~। अत्रोपपत्तिः~। यथा गुण्यं भाज्यं कल्पयित्वा कुट्टकेन गुणकः सिध्यति तथा गुणकं भाज्यं प्रकल्प्य कुट्टकेन यो गुणः स गुण्य एव सिध्यति~। अत एव पूर्वसूत्रे \hyperref[5.67]{'षष्टिश्च भाज्यः'} इत्याद्युक्तम्~। तत्र यथैकेन गुणकेन गुणितो राशिर्हरभक्तो यच्छेषं तेनोनितः स हरभक्तः शुध्यति तथान्यैरपि गुणकैः पृथक्पृथग्गुणितो हरभक्तो यानि शेषाणि तैर्यथास्वं रहितो हरभक्तः शुध्येदेव~। युक्तेस्तुल्यत्वात्~। तत्र सर्वत्र यद्येक एव हरः स्यात्तर्हि यथा पृथग्गुणितः स्वस्वशेषोनो हरभक्तः
\end{sloppypar}

\newpage

\begin{sloppypar}
\noindent शुध्यति तथा पृथग्गुणितो युक्तश्च शेषैक्येनोनो हरभक्तः शुध्येदेव~। तत्र गुणकैः पृथग्गुणितो युक्तश्चेद्गुणकयोगेनैव गुणितः स्यात्~। अतो गुणकयोग एवात्र गुणः~। शेषयोग एव शेषम्~। यथा दश १० द्व्यादिभिः २।३।४ गुणिताः २०।३०।४०~। हर\textendash \,१९\textendash \,भक्ताः पृथक्पृथक् लब्धयः १।१।२ शेषाणि च १।११।२~। एतैर्यथास्वमूनाः १९।१९।३८ हरभक्ताः शुध्यन्ति~। एवं गुणैक्येन ९ गुणिता दश ९० शेषैक्येन १४ रहिता ७६ एकोनविंशत्या भक्ताः शुध्यन्ति~। लब्धिश्च लब्धियोग एव ४~। अतो गुणकयोगस्य गुणकत्वाद्गुणकयोगो भाज्यः~। अग्रैक्यं शुद्धिर्हर एव हरः~। अत्र कुट्टके यो गुणः सिध्येत्स गुण्यराशिरेवेत्युपपन्नम् \hyperref[5.68]{'एको हरश्चेद्गुणकौ विभिन्नौ'} इत्यादि~॥~६८~॥\\

{\small अत्रोदाहरणमुपजात्याह\textendash }

\phantomsection \label{5.69}
\begin{quote}
{\large \textbf{{\color{purple}कः पञ्चनिघ्नो विहृतस्त्रिषष्ट्या सप्तावशेषोऽथ स एव राशिः~।\\
दशाहतः स्याद्विहृतस्त्रिषष्ट्या चतुर्दशाग्रो वद राशिमेनम्~॥~६९~॥}}}
\end{quote}

स्पष्टोऽर्थः~। अत्रोक्तवन्न्यासो \;{\scriptsize $\begin{matrix}
\mbox{{भा ~१५ ~क्षे ~२ं१}}\\
\vspace{-1.5mm}
\mbox{{ह ~६३ ~~~~~~~~~}}
\vspace{1mm}
\end{matrix}$}~। पूर्ववज्जातो गुणः १४~। अयमेव राशिः~। अन्यदप्युदाहरणं {\color{violet}गोलाध्याये 'ये याताधिकमासहीनदिवसाः'} इति~। बहुगुणकोदाहरणमपि तत्रैव {\color{violet}'चक्राग्राणि ग्रहाग्रकाणि'} इत्यादिश्लोकद्वयेन~। अत्र भगणराश्यादीनां शेषेष्वहर्गणस्य क्रमेण गुणकाः~। कल्पभगणाः १ द्वादश गुणास्ते २ षष्ट्यधिकशतत्रय\textendash \,३६०\textendash \,गुणास्ते ३ खखनृपाक्षि\textendash \,२१६००\textendash \,गुणास्ते ४ खखखतर्कनन्दतरणि\textendash \,१२९६०००\textendash \,गुणास्ते ५~। एवम् अन्येऽपि गुणका ऊह्याः~। अत्र गुणैक्यं भाज्यं प्रकल्प्य यो गुणः सिध्येत्स एवाहर्गणः~॥~६९~॥

\begin{quote}
{\color{violet}दैवज्ञवर्यगणसन्ततसेव्यपार्श्वबल्लाळसञ्ज्ञगणकात्मजनिर्मितेऽस्मिन्~।\\
बीजक्रियाविवृतिकल्पलतावतारे युक्तेर्विविक्तिरिति कुट्टकसिद्धिहेतोः~॥}
\end{quote}
\vspace{-1mm}

\begin{center}
इति श्रीसकलगणकसार्वभौमश्रीबल्लाळदैवज्ञसुतश्रीकृष्णदैवज्ञविरचिते \\
बीजविवृतिकल्पलतावतारे कुट्टकविवरणम्~॥~५~॥\\
\vspace{1mm}

अत्र मूलश्लोकैः सह सङ्ख्या ८००~।
\vspace{6mm}

\rule{0.2\linewidth}{0.8pt}\\
\vspace{-4mm}

\rule{0.2\linewidth}{0.8pt}
\end{center}
\end{sloppypar}

\newpage
\thispagestyle{empty}

\begin{center}
\textbf{\large ६\; वर्गप्रकृतिः~।}\\
\rule{0.2\linewidth}{0.8pt}
\end{center}

\begin{sloppypar}
{\small एवमनेकवर्णप्रक्रियोपयुक्तं कुट्टकमुक्त्वेदानीमनेकवर्णमध्यमाहरणोपयुक्तां वर्गप्रकृतिं निरूप-यति~। तत्र प्रथमतस्तत्स्वरूपं शालिन्याह\textendash }

\phantomsection \label{6.70}
\begin{quote}
{\large \textbf{{\color{purple}इष्टं ह्रस्वं तस्य वर्गः प्रकृत्या क्षुण्णो युक्तो वर्जितो वा स येन~।\\
मूलं दद्यात्क्षेपकं तं धनर्णं मूलं तच्च ज्येष्ठमूलं वदन्ति~॥~७०~॥}}}
\end{quote}

अनेकवर्णमध्यमाहरणे पक्षयोः समीकरणानन्तरमेकपक्षस्य पदे गृहीते सति द्विती-यपक्षे यदि सरूपोऽव्यक्तवर्गः स्यात्~। यथा\textendash \,काव १२ रू १~। तत्र पूर्वपक्षतुल्यतया द्वितीयपक्षेणापि मूलदेन भाव्यम्~। अस्ति चात्र कालकवर्गो द्वादशगुणः सरूपश्च~। अतो यस्य वर्गो द्वादशगुणः सरूपः सन्वर्गो भवेत्तदेव कालकमानमित्यर्थात्सिध्यति~। यच्चात्र पदं तत्पूर्वपक्षपदसममुभयपक्षयोस्तुल्यत्वात्~। सविस्तरं तु तत्रैव प्रतिपादयिष्यते~। वर्गः प्रकृतिर्यत्रेति वर्गप्रकृतिः~। यतोऽस्य गणितस्य यावदादिवर्गः प्रकृतिः~। यद्वा यावदादिवर्गेषु प्रकृतिभूतादङ्कादिदं गणितं प्रवर्तत इति वर्गप्रकृतिः~। अत्र यावद्वर्गादिषु प्रकृतिभूतो योऽङ्कः स प्रकृतिशब्देनोच्यते~। स चाव्यक्तवर्गगुणक एव~। अतोऽत्र पदसाधने वर्गस्य यो गुणः स प्रकृतिशब्देन व्यवह्रियते~। आदाविष्टं पदं प्रकल्प्य तस्य वर्गः प्रकृतिगुणो येनाङ्केन युक्तो वर्जितो वा मूलं दद्यात्तमङ्कं क्रमेण धनमृणं च क्षेपकं वदन्त्याचार्याः~। तन्मूलं ज्येष्ठमूलमिति वदन्ति~। प्रथमतो यदिष्टं पदं प्रकल्पितं तच्च ह्रस्वमिति वदन्ति~। अन्वर्थाश्चैताः सञ्ज्ञाः~। यत्र तु क्षेपवियोगात्कुत्रचिज्ज्येष्ठपदं ह्रस्वपदादल्पं भवति तत्रापि भावनया ह्रस्वपदादाधिकमेव भवति~॥~७०~॥\\

{\small एवमेकेषु ह्रस्वज्येष्ठक्षेपकेषु ज्ञातेष्वनेकत्वार्थमुपायं शालिनीत्रयेणाह\textendash }

\phantomsection \label{6.71}
\begin{quote}
{\large \textbf{{\color{purple}ह्रस्वज्येष्ठक्षेपकान्न्यस्य तेषां तानन्यान्वाधो निवेश्य क्रमेण~।\\
साध्यान्येभ्यो भावनाभिर्बहूनि मूलान्येषां भावना प्रोच्यतेऽतः~॥\\
वज्राभ्यासौ ज्येष्ठलघ्वोस्तदैक्यं ह्रस्वं लघ्वोराहतिश्च प्रकृत्या~।\\
क्षुण्णा ज्येष्ठाभ्यासयुग्ज्येष्ठमूलं तत्राभ्यासः क्षेपयोः क्षेपकः स्यात्~॥\\
ह्रस्वं वज्राभ्यासयोरन्तरं वा लघ्वोर्घातो यः प्रकृत्या विनिघ्नः~।\\
घातो यश्च ज्येष्ठयोस्तद्वियोगो ज्येष्ठं क्षेपोऽत्रापि च क्षेपघातः~॥~७१~॥}}}
\end{quote}

प्रथमसिद्धान् \hyperref[6.71]{\textbf{ह्रस्वज्येष्ठक्षेपकान्}} पङ्क्तौ विन्यस्य \hyperref[6.71]{\textbf{तेषामधस्तानन्यान्वा}} ह्रस्वज्येष्ठक्षेप-कान् \hyperref[6.71]{\textbf{क्रमेण}} निवेश्यैतेभ्यः पङ्क्तिद्वयस्थापितेभ्यो ह्रस्वज्येष्ठक्षेपकेभ्यो यतो भावनाभिर्बहूनि मूलानि साध्यान्यत एषां भावना प्रोच्यते~। अन्यान्वेत्यत्र तस्यामेव
\end{sloppypar}

\newpage

\begin{sloppypar}
\noindent प्रकृताविति ज्ञेयम्~। यद्यपि भावनाभिः क्षेपा अपि बहवो भवन्ति तथापि नास्ति नियमः~। रूपक्षेपपदजासु भावनासु व्यभिचारात्~। अतः क्षेपा बहवः साध्या इति नोक्तम्~। इष्टक्षेपे सिद्धे तेषामनुद्देश्यत्वाच्च~। तत्र भावना द्विविधा~। समासभावनान्तरभावना चेति~। तत्र पदयोर्महत्त्वेऽपेक्षिते समासभावनामाह\textendash \,\hyperref[6.71]{'वज्राभ्यासौ ज्येष्ठलघ्वोः'} इत्यादिना~। \hyperref[6.71]{\textbf{ज्येष्ठलघ्वोः}} यौ \hyperref[6.71]{\textbf{वज्राभ्यासौ तदैक्यं ह्रस्वं}} स्यात्~। वज्राभ्यासो नाम तिर्यग्गुणनम्~। वज्रस्य तिर्यक्प्रहार-स्वभावत्वात्~। तस्मादूर्ध्वकनिष्ठेनाधःस्थं ज्येष्ठं गुणनीयम् अधःस्थकनिष्ठेनोर्ध्वस्थं ज्येष्ठं गुणनीयम्~। तयोरैक्यं ह्रस्वं स्यात्~। \hyperref[6.71]{\textbf{लघ्वोराहतिः प्रकृत्या}} गुणिता ज्येष्ठयोर्वधेन युक्ता ज्येष्ठमूलं स्यात्~। क्षेपयोरभ्यासः क्षेपकः स्यादिति~। \\

अथ पदयोर्लघुत्वेऽभीप्सितेऽन्तरभावनामाह\textendash \,\hyperref[6.71]{\textbf{'ह्रस्वं वज्राभ्यासयोरन्तरं वा'}} इति~। वज्राभ्यासयोरन्तरं वा ह्रस्वं स्यात्~। ऐक्यापेक्षया विकल्पः~। अत्र यः प्रकृत्या गुणितो लघ्वोर्घातो यश्च केवलो ज्येष्ठयोर्घातस्तद्वियोगो ज्येष्ठं स्यात्~। अत्रापि क्षेपघातः क्षेपकः स्यात्पूर्ववदेव~। अत्र प्रथमसूत्रोपपत्तिः स्पष्टतरा~। अथ भावनोपपत्तिरुच्यते~। तत्रासङ्करा-र्थमाद्यद्वितीयादिपदप्रथमाक्षरोपलक्षणपूर्वकं बीजक्रिया लिख्यते~। यथा\textendash \,कनिष्ठज्येष्ठक्षे-पाणां पङ्क्त्योर्न्यासः \;{\small $\begin{matrix}
\mbox{{आक ~१ ~~आज्ये ~१ ~~आक्षे ~१}}\\
\vspace{-1.5mm}
\mbox{{द्विक ~१ ~~द्विज्ये ~१ ~~द्विक्षे ~१}}
\vspace{1mm}
\end{matrix}$}\; अथ \hyperref[6.72]{'इष्टवर्गहृतः क्षेपः क्षेपः स्यात्'} इति वक्ष्यमाणसूत्रोक्तेन \hyperref[6.72]{'क्षेपः क्षुण्णः क्षुण्णे तदा पदे'} इत्यनेन प्रकारेण परस्पर-ज्येष्ठमिष्टं प्रकल्प्य पङ्क्त्योर्जाताः कनिष्ठज्येष्ठक्षेपाः~। 
\vspace{-1mm}

\begin{center}
\begin{tabular}{lll}
द्विज्ये ० आक १ & द्विज्ये ० आज्ये १ & द्विज्येव ० आक्षे १~।\\
आज्ये ० द्विक १ & द्विज्ये ० आज्ये १ & आज्येव ० द्विक्षे १~।
\end{tabular}
\end{center}
\vspace{-1mm}

अत्रोर्ध्वपङ्क्तौ द्वितीयज्येष्ठवर्गगुणित आद्यक्षेपोऽस्ति~। तत्र द्वितीयज्येष्ठवर्गोऽन्यथा साध्यते~। द्वितीयकनिष्ठवर्गः प्रकृतिगुणो द्वितीयक्षेपयुतो जातो द्वितीयज्येष्ठवर्गः~। द्विक ० प्र १ द्विक्षे १ अनेन गुणित आद्यक्षेपो जातः खण्डद्वयात्मकः क्षेपः~। द्विकव ० प्र ० आक्षे १ द्विक्षे ० आक्षे १~। अत्र प्रथमखण्ड आद्यक्षेपोऽन्यथा साध्यते~। ज्येष्ठवर्गे हि खण्डद्वयमस्ति~। प्रकृतिगुणः कनिष्ठवर्ग एकम्~। क्षेपोऽपरम्~। तत्र ज्येष्ठवर्गात्प्रकृतिगुणे कनिष्ठवर्गे शोधिते क्षेप एवावशिष्यते~। अत आद्यकनिष्ठवर्गः प्रकृतिगुण आद्यज्येष्ठवर्गादपनीतो जात आद्यः क्षेपः~। आकव ० प्र १ं आज्येव १~। अयं प्रकृतिगुणेन द्वितीयकनिष्ठवर्गेण गुणितः सन्प्रकृतक्षेपाद्यखण्डं भवेदिति जातमाद्यं खण्डं खण्डद्वयात्मकम्~।

\begin{center}
द्विकव ० प्र ० आकव ० प्र १ं ~~~~द्विकव ० प्र ० आज्येव १~।
\end{center}
\end{sloppypar}

\newpage

\begin{sloppypar}
अत्र प्रथमखण्डे प्रकृत्या वारद्वयं गुणनाज्जातं प्रकृतिवर्गेण गुणनम्~। तथा सति जातं प्रथमखण्डम्~। द्विकव ० आकव ० प्रव १ं~। एवमूर्ध्वपङ्क्तौ जातः खण्डत्रयात्मकः क्षेपः~।

\begin{center}
द्विकव ० आकव ० प्रव १ं ~~~द्विकव ० प्र ० आज्येव १ ~~~द्विक्षे ० आक्षे १~।
\end{center}

अनयैव युक्त्या द्वितीयपङ्क्तावपि जातः खण्डत्रयात्मकः क्षेपः~।

\begin{center}
द्विकव ० आकव ० प्रव १ं ~~~आकव ० प्र ० द्विज्येव १ं ~~~द्विक्षे ० आक्षे १~। 
\end{center}

एवं पङ्क्तिद्वये जाताः कनिष्ठज्येष्ठक्षेपाः~।

\begin{center}
द्विज्ये ० आक १ ~~~~द्विज्ये ० आज्ये १\\
आज्ये ० द्विक १ ~~~~द्विज्ये ० आज्ये १\\
द्विकव ० आकव ० प्रव १ं ~~~~द्विकव ० प्र ० आज्येव १ ~~~~द्विक्षे ० आक्षे १\\
द्विकव ० आकव ० प्रव १ं ~~~~आकव ० प्र ० द्विज्येव १ ~~~~द्विक्षे ० आक्षे १
\end{center}

अत्र ज्येष्ठलघ्वोरेकोऽभ्यास ऊर्ध्वपङ्क्तौ कनिष्ठम्~। अपरोऽभ्यासो द्वितीयपङ्क्तौ कनिष्ठम्~। ज्येष्ठं तूभयत्र ज्येष्ठाभ्यासरूपमेकमेव~। अत्र प्रत्येकं वज्राभ्यासस्य कनिष्ठत्वकल्पने क्षेपो महान् स्यादित्याचार्यैरन्यथा यतितम्~। तद्यथा~। वज्राभ्यासयोगः कनिष्ठं कल्पितम्~। द्विज्ये ० आक १ आज्ये ० द्विक १~। अस्य वर्गः~। 

\begin{center}
द्विज्येव ० आकव १ द्विज्ये ० आक ० आज्ये ० द्विक २ आज्येव द्विकव १~।
\end{center}

प्रकृतिगुणः~।

\begin{center}
द्विज्येव ० आकव ० प्र १ द्विज्ये ० आक ० आज्ये ० द्विक ० प्र २ आज्येव ० द्विकव ० प्र १~।
\end{center}

अयं केन क्षेपेण युतः सन्मूलदः स्यादिति विचार्यते~। तत्रास्य खण्डद्वयम्~। एकैक-वज्राभ्यासजज्येष्ठवर्गतुल्यमेकम्~। शेषमपरम्~। तत्र कनिष्ठवर्गः प्रकृतिगुणः क्षेपयुतो ज्येष्ठवर्गः स्यादिति जातौ पङ्क्तिद्वये ज्येष्ठवर्गौ~।

\begin{center}
{\small द्विज्येव ० आकव ० प्र १ ~द्विकव ० आकव ० प्रव १ं ~द्विकव ० प्र ० आज्येव १ ~द्विक्षे ० आक्षे १~।\\
आज्येव ० द्विकव ० प्र १ ~द्विकव ० आकव ० प्रव १ं ~आकव ० प्र ० द्विज्येव १ ~द्विक्षे ० आक्षे १~।}
\end{center}

पङ्क्तिद्वयेऽपि ज्येष्ठाभ्यासलक्षणस्य ज्येष्ठम्य तुल्यत्वादेतौ ज्येष्ठवर्गावपि तुल्यावेव~। तृती-योऽयमपि~। द्विज्येव ० आज्येव १~। अथ वज्राभ्यासयोगरूपकल्पितकनिष्ठस्य वर्गात्प्रकृति-गुणादस्मात्~। 

\begin{center}
{\small द्विज्येव ० आकव ० प्र १ ~द्विज्ये ० आक ० आज्ये ० ~द्विक ० प्र २ ~आज्येव ० द्विकव ० प्र १~।}
\end{center}

ज्येष्ठवर्गद्वयेऽपि पृथक्पृथगपनीते शेषं तुल्यमेव~।
\end{sloppypar}

\newpage

\begin{sloppypar}
\begin{center}
द्विज्ये ० आक ० आज्ये ० द्विक ० प्र २ आकव ० द्विकव ० प्रव १ आक्षे ० द्विक्षे १ं~।
\end{center}

इदं शोधितेन ज्येष्ठवर्गेण पुनर्यदि योज्यते तर्हि कल्पितकनिष्ठवर्गः प्रकृतिगुणो यथास्थितः स्यात्~। अथायमपि ज्येष्ठवर्गः~। द्विज्येव ० आज्येव १ शोधितेन सम इति~। अनेन योगे जातः कल्पितकनिष्ठवर्गप्रकृतिगुणः~। द्विज्येव ० आज्येव १ द्विज्ये ० आक ० आज्ये ० द्विक ० प्र २ आकव ० द्विकव ० प्रव १ आक्षे ० द्विक्षे १ं~।\\

अस्मात्क्षेपघातेन युक्तात् \hyperref[3.31]{'कृतिभ्य आदाय पदानि'} इत्यादिना पदमिदं द्विज्ये ० आज्ये १ आक ० द्विक ० प्र १ लभ्यत इत्युपपन्नं \hyperref[6.71]{'लघ्वोराहतिश्च प्रकृत्या क्षुण्णा ज्येष्ठाभ्यासयुग्ज्येष्ठमूलम्'} इत्यादि~। एवं वज्राभ्यासयोरन्तरं कनिष्ठं प्रकल्प्योक्तयुक्त्यान्तर-भावनोपपत्तिरपि द्रष्टव्या~। एवं खण्डक्षोदेन बहुविधा उपपत्तयः सन्ति~। ग्रन्थविस्तरभयान्न लिख्यन्ते~॥~७१~॥\\

{\small एवं भावनाभ्यामिष्टक्षेपजपदसिद्धौ तेभ्य एव क्षेपान्तरजपदानयनमथ च यत्र कुत्रापि क्षेपपद-सिद्धौ स चेदिष्टवर्गेण गुणितो भक्तो वोद्दिष्टक्षेपो भवेत्तदा तेभ्य एवोद्दिष्टक्षेपजपदानयनमनुष्टुभाह\textendash }

\phantomsection \label{6.72}
\begin{quote}
{\large \textbf{{\color{purple}इष्टवर्गहृतः क्षेपः क्षेपः स्यादिष्टभाजिते~।\\
मूले ते स्तोऽथ वा क्षेपः क्षुण्णः क्षुण्णे तदा पदे~॥~७२~॥}}}
\end{quote}

यस्मिन्क्षेपे कनिष्ठज्येष्ठपदे सिद्धे स क्षेप इष्टस्य वर्गेण भक्तः सन्यदि क्षेपो भवति तदा ते पदे इष्टभक्ते सती पदे स्तः~। यदि त्विष्टवर्गेण गुणितः सन्क्षेपो भवति तदा ते पदे इष्टगुणिते स्तः~। यस्येष्टस्य वर्गेण क्षेपो गुणितस्तेन पदे गुणनीये इत्यर्थः~। अत्रोपपत्तिः~। वर्गराशिर्वर्गेण गुणितो भक्तो वा वर्गत्वं न जहातीति सुप्रसिद्धम्~। प्रकृते कनिष्ठवर्गः कव १~। कनिष्ठवर्गः प्रकृतिगुणः क्षेपयुतो ज्येष्ठवर्गो भवतीति जातो ज्येष्ठवर्गः कव ० प्र १ क्षे १~। अथोभयोरपीष्टवर्गेण गुणितयोर्न्यासः~। 

\begin{center}
इव ० कव १ ~~~इव ० कव ० प्र ० १ ~~~इव ० क्षे १~।
\end{center}

अत्र कनिष्ठज्येष्ठवर्गयोरिष्टवर्गगुणनात्तत्पदयोरिष्टमेव गुणकः स्यात्~। यतो यैवेष्ट-वर्गकनिष्ठवर्गहतिः स एवेष्टकनिष्ठाहतिवर्गः~। एवं ज्येष्ठवर्गेऽपि~। इष्टकनिष्ठाहतिवर्गस्य पदं त्विष्टकनिष्ठाहतिरेव स्यात्~। एवं ज्येष्ठवर्गस्यापि~। अथात्र क्षेपविचारः~। प्रकृति-गुणस्य कनिष्ठवर्गस्य केवलस्य ज्येष्ठवर्गस्य च यदन्तरालं स हि क्षेपः~। प्रकृते च तदन्त-रालमिष्टवर्गहतः पूर्वक्षेपः~। एवमेवेष्टवर्गेण कनिष्ठज्येष्ठवर्गयोर्हरणेऽपि~। तदेवमुपपन्नम् \hyperref[6.72]{'इष्टवर्गहृतः क्षेपः'} इत्यादि~॥~७२~॥
\end{sloppypar}

\newpage

\begin{sloppypar}
{\small अथ यत्र कुत्राप्युद्दिष्टक्षेपे रूपक्षेपजपदाभ्यां भावनया पदानेकत्वं भवतीति रूपक्षेपजपदसाधनं प्रकारान्तरेण सार्धानुष्टुभाह\textendash }

\phantomsection \label{6.73}
\begin{quote}
{\large \textbf{{\color{purple}इष्टवर्गप्रकृत्योर्यद्विवरं तेन वा भजेत्~।\\
द्विघ्नमिष्टं कनिष्ठं तत्पदं स्यादेकसंयुतौ~।\\
ततो ज्येष्ठमिहानन्त्यं भावनातस्तथेष्टतः~॥~७३~॥}}}
\end{quote}

इष्टवर्गप्रकृत्योर्यद्विवरं तेन द्विघ्नमिष्टं भजेत्~। तदेकसंयुतौ रूपक्षेपे कनिष्ठं स्यात्ततः कनिष्ठाज्ज्येष्ठं स्यात् \hyperref[6.70]{'इष्टं ह्रस्वं तस्य वर्गः प्रकृत्या क्षुण्णः'} इत्यादिना~। इह कनिष्ठज्येष्ठ-योर्भावनावशात्तथेष्टवशादानन्त्यमस्ति~।\\

अत्रोपपत्तिः~। इष्टं ह्रस्वमित्यत्रैवेष्टं कनिष्ठमित्युक्तम्~। तच्चेद्द्विघ्नं तदा कनिष्ठवर्गश्चतुर्गुणः स्यात्~। असौ चतुर्गुणः कनिष्ठवर्गप्रकृत्योश्चतुर्गुणो घातः~। अयं केन युतो मूलप्रदो भव-तीति विचार्यते~। \hyperref[8.131]{'चतुर्गुणस्य घातस्य युतिवर्गस्य चान्तरम्~। राश्यन्तरकृतेस्तुल्यम्'} इति राश्यन्तरवर्गेण युतश्चतुर्गुणितो घातो युतिवर्गो भवति~। तस्य चावश्यं मूललाभः~। अत्र तु कनिष्ठवर्गप्रकृत्योश्चतुर्गुणो घातोऽस्ति~। कनिष्ठं त्विष्टमेव~। अत इष्टवर्गप्रकृत्योश्चतुर्गुणो घातोऽयम्~। असाविष्टवर्गप्रकृत्यन्तरवर्गेण योजितश्चेदवश्यं मूलदः स्यात्~। तथा च द्विघ्नमिष्टं कनिष्ठम्~। तस्मादिष्टवर्गप्रकृत्योरन्तरवर्गतुल्ये क्षेपे ज्येष्ठं पदमपि सिध्यति~। अपेक्षितं च रूपक्षेपे~। तत्र युक्तिः~। \hyperref[6.72]{'इष्टवर्गहृतः क्षेपः क्षेपः स्यादिष्टभाजिते~। मूले ते स्तः'} इत्यनेन~। अत्रेष्टवर्गप्रकृत्योर्विवरतुल्यमिष्टं कल्पितम्~। तद्वर्गेण क्षेपे भक्ते रूपमेव स्यात्~। कनिष्ठं त्विष्टवर्गप्रकृत्योर्विवरेणैव भाज्यम्~। प्रकृते कनिष्ठं तु द्विघ्नमिष्टम्~। अत उपपन्नम् \hyperref[6.73]{'इष्टवर्गप्रकृत्योर्यद्विवरं तेन वा भजेत्~। द्विघ्नमिष्टम्'} इति~॥~७३~॥\\

{\small अथ वर्गप्रकृतावुदाहरणद्वयमनुष्टुभाह\textendash }

\phantomsection \label{6.74}
\begin{quote}
{\large \textbf{{\color{purple}को वर्गोऽष्टहतः सैकः कृतिः स्याद्गणकोच्यताम्~।\\
एकादशगुणः को वा वर्गः सैकः कृतिः सखे~॥~७४~॥}}}
\end{quote}

स्पष्टोऽर्थः~। प्रथमे न्यासः प्र ८ क्षे १~। अत्रैकमिष्टं ह्रस्वं प्रकल्प्य जाते मूले~। क १ ज्ये ३ क्षे १~। एतेषां भावनार्थं न्यासः \;{\small $\begin{matrix}
\mbox{{प्र ~८ ~क ~१ ~ज्ये ~३ ~क्षे ~१}}\\
\vspace{-1.5mm}
\mbox{{~~~~~~~~क ~१ ~ज्ये ~३ ~क्षे ~१}}
\vspace{1mm}
\end{matrix}$}\; अत्र सूत्रं \hyperref[6.71]{'वज्राभ्यासौ ज्येष्ठलघ्वोः'} इत्यादि~। प्रथमकनिष्ठ\textendash \,१\textendash \,द्वितीयज्येष्टमूल\textendash \,३\textendash \,अभ्यासः ३~। द्वितीयकनिष्ठप्रथमज्येष्ठयोः १।३ अभ्यासः ३~। अनयोरैक्यं ६ कनिष्ठपदं
\end{sloppypar}

\newpage

\begin{sloppypar}
\noindent स्यात्~। कनिष्ठयोः १।१ आहतिः १ प्रकृति\textendash \,८\textendash \,गुणा ८ ज्येष्ठयोः ३।३ अभ्यासेनानेन ९ युता १७ ज्येष्ठपदं स्यात्~। क्षेपयोः १।१ आहतिः क्षेपः १ स्यात्~। प्राङ्मूलक्षेपाणामेभिः सह भावनार्थं न्यासः\textendash \;{\small $\begin{matrix}
\mbox{{प्र ~८ ~क ~१ ~ज्ये ~~३ ~क्षे ~१}}\\
\vspace{-1.5mm}
\mbox{{~~~~~क ~६ ~ज्ये ~१७ ~क्षे ~१}}
\vspace{1mm}
\end{matrix}$}~। अत्र वज्राभ्यासौ १७।१८ अनयोरैक्यं ह्रस्वं ३५~। लघ्वो-राहतिः ६ प्रकृत्या ८ क्षुण्णा ४८ ज्येष्ठाभ्यासेन ५१ युक् ९९ ज्येष्ठमूलम्~। क्षेपयोरभ्यासः क्षेपः १~। क ३५ ज्ये ९९ क्षे १~। एवं भावनावशादानन्त्यम्~। अथ द्वितीयोदाहरणे न्यासः प्र ११ क्षे १~। रूपमिष्टं कनिष्ठं प्रकल्प्य तद्वर्गात्प्रकृतिगुणाद्रूपद्वयमपास्य मूलं ज्येष्ठं ३ भावनार्थं न्यासः~। \;{\small $\begin{matrix}
\mbox{{प्र ~११ ~क ~१ ~ज्ये ~३ ~क्षे ~२ं}}\\
\vspace{-1.5mm}
\mbox{{~~~~~~~~~क ~१ ~ज्ये ~३ ~क्षे ~२ं}}
\vspace{1mm}
\end{matrix}$}\; ज्येष्ठलघ्वोर्वज्राभ्यासौ ३।३ अनयोरैक्यं ६ ह्रस्वम्~। लघ्वोराहतिः १ प्रकृत्या क्षुण्णा ११ ज्येष्ठाभ्यासेन ९ युक् २० ज्येष्ठमूलम्~। क्षेपयोः २।२ अभ्यासः ४ क्षेपः~। क ६ ज्ये २० क्षे ४ \hyperref[6.72]{'इष्टवर्गहृतः क्षेपः'} इत्यादिना रूपद्वयमिष्टं प्रकल्प्य जाते रूपक्षेपमूले क ३ ज्ये १० क्षे १~। समासभावनार्थं न्यासः \;{\small $\begin{matrix}
\mbox{{क ~३ ~ज्ये ~१० ~क्षे ~१}}\\
\vspace{-1.5mm}
\mbox{{क ~३ ~ज्ये ~१० ~क्षे ~१}}
\vspace{1mm}
\end{matrix}$}\; जाते मूले क ६० ज्ये १९९ क्षे १~। एवमत्र भावनावशादानन्त्यम्~। अथवा रूपमिष्टं कनिष्ठं प्रकल्प्य जाते पञ्चक्षेपपदे क १ ज्ये ४ क्षे ५~। अनयोस्तुल्यभावनया मूले क ८ ज्ये २७ क्षे २५ \hyperref[6.72]{'इष्टवर्गहृतः क्षेपः'} इति पञ्चकमिष्टं प्रकल्प्य जाते रूपक्षेपमूले क \,{\scriptsize $\begin{matrix}
\mbox{{८}}\\
\vspace{-1.5mm}
\mbox{{५}}
\vspace{1mm}
\end{matrix}$}\, ज्ये \,{\scriptsize $\begin{matrix}
\mbox{{२७}}\\
\vspace{-1.5mm}
\mbox{{५}}
\vspace{1mm}
\end{matrix}$}\, क्षे १~। अनयोः पूर्वमूलाभ्यां सह भावनार्थं न्यासः~। \;{\small $\begin{matrix}
\mbox{{प्र ~११ ~क ~{\scriptsize $\begin{matrix}
\mbox{{८}}\\
\vspace{-1.5mm}
\mbox{{५}}
\vspace{1mm}
\end{matrix}$} ~ज्ये ~{\scriptsize $\begin{matrix}
\mbox{{२७}}\\
\vspace{-1.5mm}
\mbox{{५}}
\vspace{1mm}
\end{matrix}$} ~क्षे ~१}}\\
\vspace{-1.5mm}
\mbox{{~~~~~~~~~~~क ~३ ~ज्ये ~१० ~क्षे ~१}}
\vspace{1mm}
\end{matrix}$}\; भावनया लब्धे मूले क \,{\scriptsize $\begin{matrix}
\mbox{{१६१}}\\
\vspace{-1.5mm}
\mbox{{५}}
\vspace{1mm}
\end{matrix}$}\, ज्ये \,{\scriptsize $\begin{matrix}
\mbox{{५३४}}\\
\vspace{-1.5mm}
\mbox{{५}}
\vspace{1mm}
\end{matrix}$}\, क्षे १~। अथवा \hyperref[6.71]{'ह्रस्वं वज्राभ्यासयोरन्तरम्'} इत्यादिना कृतयान्तरभावनया जाते मूले क \,{\scriptsize $\begin{matrix}
\mbox{{१}}\\
\vspace{-1.5mm}
\mbox{{५}}
\vspace{1mm}
\end{matrix}$}\, ज्ये \,{\scriptsize $\begin{matrix}
\mbox{{६}}\\
\vspace{-1.5mm}
\mbox{{५}}
\vspace{1mm}
\end{matrix}$}\, क्षे १~। एवमनेकधा~। \hyperref[6.73]{'इष्टवर्गप्रकृत्योर्यद्विवरं तेन वा भजेत्'} इत्यादिना पक्षान्तरेण पदे रूपक्षेपे प्रतिपाद्येते~। प्रथमोदाहरणे रूपत्रयमिष्टं प्रकल्प्य यथोक्तकरणे कनिष्ठं ६ अस्य वर्गः ३६ प्रकृति\textendash \,८\textendash \,गुणः २८८ सैको २८९ अस्य मूलं १७ ज्येष्ठपदम्~। एवं द्वितीयोदाहरणेऽपि रूपत्रयमिष्टं प्रकल्प्य जाते कनिष्ठज्येष्ठे क ३ ज्ये १० क्षे १~। एवमिष्टवशात्समासान्तरभावनाभ्यां च पदानामानन्त्यम्~॥~७४~॥
\end{sloppypar}

\newpage

\begin{sloppypar}
{\small अथ कनिष्ठज्येष्ठयोरभिन्नतार्थं चक्रवालाख्यां वर्गप्रकृतिमनुष्टुभ उत्तरार्धेनानुष्टुप्त्रितयेनानुष्टुप्-पूर्वार्धेन चाह\textendash }

\phantomsection \label{6.75}
\begin{quote}
{\large \textbf{{\color{purple}ह्रस्वज्येष्ठपदक्षेपान्भाज्यप्रक्षेपभाजकान्~।\\
कृत्वा कल्प्यो गुणस्तत्र तथा प्रकृतितश्च्युते~।\\
गुणवर्गे प्रकृत्योनेऽथवाल्पं शेषकं यथा~।\\
तत्तु क्षेपहृतं क्षेपो व्यस्तः प्रकृतितश्च्युते~।\\
गुणलब्धिः पदं ह्रस्वं ततो ज्येष्ठमतोऽसकृत्~।\\
त्यक्त्वा पूर्वपदक्षेपांश्चक्रवालमिदं जगुः~।\\
चतुर्द्व्येकयुतावेवमभिन्ने भवतः पदे~।\\
चतुर्द्विक्षेपमूलाभ्यां रूपक्षेपार्थभावना~॥~७५~॥}}}
\end{quote}

प्रथमतः \hyperref[6.70]{'इष्टं ह्रस्वं तस्य वर्गः'} इत्यादिना ह्रस्वज्येष्ठक्षेपान्कृत्वा पश्चात्तान् ह्रस्वज्येष्ठ-क्षेपान्क्रमेण भाज्यक्षेपभाजकान्कृत्वा कुट्टकेन तथा गुणः साध्यो यथा गुणस्य वर्गे प्रकृतितश्च्युते प्रकृत्योने वा शेषकमल्पं स्यात्~। तत्तु शेषं पूर्वक्षेपहृतं सत्क्षेपः स्यात्~। गुणवर्गे प्रकृतितश्च्युते सत्ययं क्षेपो व्यस्तः स्यात्~। धनं चेदृणमृणं चेद्धनं भवेदित्यर्थः~। यस्य गुणस्य वर्गेण प्रकृत्या सहान्तरं कृतं तस्य गुणस्य या लब्धिस्तत्कनिष्ठं पदं स्यात्~। ततः कनिष्ठाज्ज्येष्ठं पूर्ववत्स्यात्~। अथ प्रथमकनिष्ठज्येष्ठक्षेपांस्त्यक्त्वाधुना साधितेभ्यः कनिष्ठज्येष्ठपदक्षेपेभ्यः पुनः कुट्टकेन गुणाप्ती आनीयोक्तवत्कनिष्ठज्येष्ठपदक्षेपाः साध्याः~। एवमसकृत्~। आचार्या एतद्गणितं चक्रवालमिति जगुः~। एवं चक्रवालेन चतुर्द्व्येकयुतौ चतुःक्षेपे द्विक्षेपे एकक्षेपे चाभिन्ने पदे भवतः~। इदमुपलक्षणम्~। यत्र कुत्रापि क्षेपेऽभिन्ने पदे भवतः~। युतावित्युपलक्षणम्~। तेन शुद्धावपीति ज्ञेयम्~। अथ रूपक्षेपपदानयने प्रकारान्तरमप्यस्तीत्याह\textendash \,चतुर्द्विक्षेपमूलाभ्यामिति~। चतुःक्षेपमूलाभ्यां द्विक्षेपमूलाभ्यां च रूपक्षेपार्थं भावना रूपक्षेपार्थभावना~। कार्येति शेषः~। चतुःक्षेपे \hyperref[6.72]{'इष्टवर्गहृतः क्षेपः'} इत्यादिना द्विक्षेपे तु तुल्यभावनया चतुःक्षेपपदे प्रसाध्य पश्चात् \hyperref[6.72]{'इष्टवर्गहृतः क्षेपः'} इत्यादिना रूपक्षेपजे पदे वा भवत इत्यर्थः~। एवं नवत्र्यादिक्षेपमूलाभ्यामपि रूपक्षेपार्थभावना द्रष्टव्या~। अस्मिन् रूपक्षेपपदानयने तयोर्नाभिन्नत्वनियमः~। चतुर्द्विक्षेपमूलाभ्यां तु रूपक्षेपार्थभावनायां कनिष्ठज्येष्ठयोर्द्वयं हर इति प्रायो रूपक्षेपपदयोरभिन्नत्वं सिध्यतीति चतुर्द्विक्षेपमूलाभ्यामित्युक्तम्~। यदा तु भावनयाभिन्नत्वं न सिध्येत्तदा पुनश्चक्रवालेनैव पदे साध्ये इति परमपि सुधीभिरूह्यम्~। अत्रोपपत्तिः~। \hyperref[6.72]{'इष्टवर्गहृतः क्षेपः'} इत्यादियुक्त्या कनिष्ठमिष्टगुणितं चेदिष्टवर्गेण क्षेपोऽपि
\end{sloppypar}

\newpage

\begin{sloppypar}
\noindent गुणनीयः~। तथा सति जातो कनिष्ठक्षेपौ इ ० क १~। इव ० क्षे १~। अत्र क्षेपतुल्यमिष्टं प्रकल्प्य \hyperref[6.72]{'इष्टवर्गहृतः क्षेपः'} इत्यादिना जातौ कनिष्ठक्षेपौ \;{\small $\begin{matrix}
\mbox{{इ ~० ~क ~१~। ~इव ~० ~क्षे ~१~।}}\\
\vspace{-1.5mm}
\mbox{{\hspace{6mm} क्षे ~१~। \hspace{8mm} क्षेव १~।}}
\vspace{1mm}
\end{matrix}$}\; एवमत्र प्रथमकनिष्ठमिष्टगुणं क्षेपभक्तं कनिष्ठं स्यात्~। कनिष्ठवज्ज्येष्ठमपि~। प्रथमक्षेपः त्विष्टवर्गगुणितः क्षेपवर्गभक्तः सन्नत्र क्षेपः स्यात्~। अत्र क्षेपे हारभाज्ययोः पूर्वक्षेपेणापवर्ते जातः क्षेपः \;{\small $\begin{matrix}
\mbox{{इव ~१}}\\
\vspace{-1.5mm}
\mbox{{क्षे ~~१}}
\vspace{1mm}
\end{matrix}$}\; प्रथमक्षेपभक्त इष्टवर्गः~। तस्मादिष्टगुणं कनिष्ठं क्षेपभक्तं सद्यदि कनिष्ठं कल्प्यते तर्हि इष्टवर्गः क्षेपभक्तः सन् क्षेपो भवति~। अत्रेष्टं तादृशं कल्पनीयं येन गुणितं कनिष्ठं क्षेपभक्तं शुध्येत्~। अन्यथा कनिष्ठमभिन्नं कथं स्यात्~। तदर्थं कनिष्ठं केन गुणितं क्षेपभक्तं निःशेषं स्यादिति कनिष्ठं भाज्यं प्रकल्प्य क्षेपं हारं च प्रकल्प्य क्षेपाभावे गुणाप्ती साध्ये~। अत्र या लब्धिस्तत्कनिष्ठपदम्~। योऽत्र गुणस्तदेवेष्टमिति गुणकवर्गः पूर्वक्षेपभक्तः क्षेपः स्यात्~। ज्येष्ठमपि गुणगुणितं क्षेपभक्तं ज्येष्ठं स्यात्~। अत्र क्षेपो महान् भवतीत्याचार्येणान्यथा यतितम्~। कनिष्ठं भाज्यम्~। ज्येष्ठपदं क्षेपम्~। क्षेपं हरं च प्रकल्प्य गुणाप्ती साधिते~। पूर्वं तु गुणगुणितं कनिष्ठं क्षेपभक्तं सत्कनिष्ठं भवतीति स्थितम्~। इदानीं तु गुणगुणितं कनिष्ठं ज्येष्ठयुतं क्षेपभक्तं सत्कनिष्ठं स्यात्~। तस्माज्ज्येष्ठं क्षेपभक्तं सत् कनिष्ठेऽधिकं जातम्~। एवं सति प्रकृतिगुणे कनिष्ठवर्गे किमधिकं भवतीति विचार्यते~। तत्र पूर्वकनिष्ठं \;{\small $\begin{matrix}
\mbox{{इ ~० ~क ~१}}\\
\vspace{-1.5mm}
\mbox{{\hspace{6mm} क्षे ~१}}
\vspace{1mm}
\end{matrix}$}~। अस्य वर्गः \;{\small $\begin{matrix}
\mbox{{इव ~० ~कव ~१}}\\
\vspace{-1.5mm}
\mbox{{\hspace{9mm} क्षेव ~१}}
\vspace{1mm}
\end{matrix}$}~। प्रकृतिगुणः \;{\small $\begin{matrix}
\mbox{{इव ~० ~कव ~० ~प्र ~१}}\\
\vspace{-1.5mm}
\mbox{{\hspace{16mm} क्षेव ~१}}
\vspace{1mm}
\end{matrix}$}\; ज्येष्ठसाधनार्थं क्षेपश्चायं \;{\small $\begin{matrix}
\mbox{{इव ~१}}\\
\vspace{-1.5mm}
\mbox{{क्षे ~~१}}
\vspace{1mm}
\end{matrix}$}~। अथ क्षेपभक्तज्येष्ठाधिकं कनिष्ठं \;{\small $\begin{matrix}
\mbox{{इ ~० ~क ~१ ~ज्ये ~१}}\\
\vspace{-1.5mm}
\mbox{{\hspace{15mm} क्षे ~१}}
\vspace{1mm}
\end{matrix}$}~। अस्य वर्गः \;{\small $\begin{matrix}
\mbox{{इव ~० ~कव ~१ ~इ ~० ~क ~० ~ज्ये ~२ ~ज्येव ~१}}\\
\vspace{-1.5mm}
\mbox{{\hspace{43mm} क्षेव ~१}}
\vspace{1mm}
\end{matrix}$}~। प्रकृतिगुणः {\small $\begin{matrix}
\mbox{{प्र \,० \,इव \,० \,कव \,१ \,प्र \,० \,इ \,० \,क \,० \,ज्ये \,२ \,प्र \,० \,ज्येव \,१}}\\
\vspace{-1.5mm}
\mbox{{\hspace{61mm} क्षेव \,१}}
\vspace{1mm}
\end{matrix}$}~। अत्रान्त्यखण्डमन्यथा साध्यते~। कनिष्ठवर्गः प्रकृतिगुणः क्षेपयुतो ज्येष्ठवर्गो भवतीति जातः कव ० प्र १ क्षे १~। अयं प्रकृति-गुणः कव ० प्रव १ प्र ० क्षे १~। एवं जातं \\
{\color{white}अ} \hfill {\small $\begin{matrix}
\mbox{{प्र ~० ~इव ~० ~कव ~१ ~प्र ~० ~इ ~० ~क ~० ~ज्ये ~२ ~कव ~० ~प्रव ~१ ~प्र ~० ~क्षे ~१}}\\
\vspace{-1.5mm}
\mbox{{\hspace{82mm} क्षेव ~१}}
\vspace{1mm}
\end{matrix}$}~।

\end{sloppypar}

\newpage

\begin{sloppypar}
\noindent तस्मात् अत्राधिकं \;{\small $\begin{matrix}
\mbox{{प्र ~० ~इ ~० ~क ~० ~ज्ये ~२ ~कव ~० ~प्रव ~१ ~प्र ~० ~क्षे ~१}}\\
\vspace{-1.5mm}
\mbox{{\hspace{55mm} क्षेव ~१}}
\vspace{1mm}
\end{matrix}$}~। अतः प्रकृतिगुणे कनिष्ठवर्ग एतावत्क्षिप्तं स्यात्~। ज्येष्ठार्थं तु पूर्वं युक्त्या गुणवर्गः क्षेपभक्तः क्षेपणीयो भवति~। तदर्थमधिकस्य खण्डद्वयं कृतं \;{\small $\begin{matrix}
\mbox{{प्र ~० ~इ ~० ~क ~० ~ज्ये ~२ ~कव ~० ~प्रव ~१}}\\
\vspace{-1.5mm}
\mbox{{\hspace{40mm} क्षेव ~१}}
\vspace{1mm}
\end{matrix}$}\; इदमेकम्~। अन्यदिदं \;{\small $\begin{matrix}
\mbox{{क्षे ~० ~प्र ~१}}\\
\vspace{-1.5mm}
\mbox{{\hspace{5mm} क्षेव ~१}}
\vspace{1mm}
\end{matrix}$}~। अस्मिन्भाज्यभाजकयोः क्षेपेणापवर्ते जातं \;{\small $\begin{matrix}
\mbox{{प्र ~१}}\\
\vspace{-1.5mm}
\mbox{{क्षे ~१}}
\vspace{1mm}
\end{matrix}$}~। अनेनाधिकेन क्षेपभक्ता प्रकृतिः क्षिप्ता स्यात्~। क्षेपणीयस्तु क्षेपभक्तो गुणवर्गः~। तदत्र गुणवर्गप्रकृ-त्योरन्तरालमपि क्षेपभक्तं क्षेप्यम्~। तथा सति क्षेपभक्तो गुणवर्ग एव क्षिप्तो भवेत्~। अत उक्तम्\textendash \,\hyperref[6.75]{'तथा प्रकृतितश्च्युते~। गुणवर्गे प्रकृत्योनेऽथवाल्पं शेषकं यथा~। तत्तु क्षेपहृतं क्षेपः'} इति~। तत्र प्रकृतितश्चेद्गुणवर्गोऽधिको भवति तदैव क्षेपभक्तं गुणवर्गप्रकृत्यन्तरं योज्यम्~। क्षिप्तस्य न्यूनत्वात्~। यदा तु गुणवर्गो न्यूनस्तदा क्षेपभक्तं गुणवर्गप्रकृत्यन्तरं शोध्यम्~। क्षिप्तस्याधिकत्वात्~। अत उक्तम् \hyperref[6.75]{'व्यस्तः प्रकृतितश्च्युते'} इति~। यत्तु गुणवर्ग-प्रकृत्योरन्तरमल्पं यथा स्यात्तथा गुणः कल्प्य इत्युक्तं तत्क्षेपस्य लघुत्वार्थम्~। ननु तथापि ज्येष्ठवर्गेऽधिकमिदमस्त्येव \;{\small $\begin{matrix}
\mbox{{प्र ~० ~क ~० ~इ ~० ~ज्ये ~२ ~कव ~० ~प्रव ~१}}\\
\vspace{-1.5mm}
\mbox{{\hspace{40mm} क्षेव ~१}}
\vspace{1mm}
\end{matrix}$}~। यस्य ज्येष्ठस्य वर्गे इदमधिकं तज्ज्येष्ठं तु \;{\small $\begin{matrix}
\mbox{{इ ~० ~ज्ये ~१}}\\
\vspace{-1.5mm}
\mbox{{\hspace{7mm} क्षे ~१}}
\vspace{1mm}
\end{matrix}$}~। अस्य वर्गेऽस्मिन् \;{\small $\begin{matrix}
\mbox{{इव ~० ~ज्येव ~१}}\\
\vspace{-1.5mm}
\mbox{{\hspace{9mm} क्षेव ~~१}}
\vspace{1mm}
\end{matrix}$}~। अधिके क्षिप्ते सति जातं \;{\small $\begin{matrix}
\mbox{{इव ~० ~ज्येव ~१ ~प्र ~० ~क ~० ~इ ~० ~ज्ये ~२ ~कव ~० ~प्रव ~१}}\\
\vspace{-1.5mm}
\mbox{{\hspace{61mm} क्षेव ~१}}
\vspace{1mm}
\end{matrix}$}~। अत्राधिके जातेऽपि \hyperref[3.31]{'कृतिभ्य आदाय पदानि'} इत्यादिना पदमायाति \;{\small $\begin{matrix}
\mbox{{इ ~० ~ज्ये ~१ ~क ~० ~प्र ~१}}\\
\vspace{-1.5mm}
\mbox{{\hspace{23mm} क्षे ~१}}
\vspace{1mm}
\end{matrix}$}~। तस्मादयमपि ज्येष्ठवर्गो भवति~। एतावांस्तु विशेषः~। इष्टगुणं कनिष्ठं क्षेपभक्तं सद्यदि कनिष्ठं कल्प्यते तर्हीष्टवर्गः क्षेपभक्तः सन्क्षेपो भवति~। इष्टगुणं ज्येष्ठं क्षेपभक्तं सत्तत्र ज्येष्ठं भवति~। यदा त्विष्टगुणं कनिष्ठं ज्येष्ठयुतं क्षेपभक्तं सत्कनिष्ठं कल्प्यते तदा गुणवर्गप्रकृत्योरन्तरं क्षेपभक्तं सत्क्षेपो भवति~। इष्टगुणं ज्येष्ठं प्रकृतिगुणकनिष्ठेन युक्तं क्षेपभक्तं सत्तत्र ज्येष्ठं भवतीति~। अत्र यद्यपीष्टवशादेव पदसिद्धिरस्तीति कुट्टकस्य
\end{sloppypar}

\newpage

\begin{sloppypar}
\noindent नापेक्षा तथाप्यभिन्नत्वार्थं कुट्टकः कृतः~। अत उपपन्नं ह्रस्वज्येष्ठपदक्षेपानित्यादि~। अत्र तु ततः कनिष्ठाज्ज्येष्ठमिति पूर्ववत् ज्येष्ठमुक्तम्~। अन्यथापि ज्येष्ठापेक्षा चेत्तदा गुणगुणितं ज्येष्ठं प्रकृतिगुणेन कनिष्ठेन युतं क्षेपभक्तं ज्येष्ठं भवतीत्यस्मदुक्तमार्गेण ज्येष्ठं कुर्यात्~॥~७५~॥\\

{\small अत्रोदाहरणं वसन्ततिलकयाह\textendash }

\phantomsection \label{6.76}
\begin{quote}
{\large \textbf{{\color{purple}का सप्तषष्टिगुणिता कृतिरेकयुक्ता \\
का चैकषष्टिनिहता च सखे सरूपा~।\\
स्यान्मूलदा यदि कृतिप्रकृतिर्नितान्तं \\
त्वच्चेतसि प्रवद तात ततालतावत्~॥~७६~॥}}}
\end{quote}

स्पष्टोऽर्थः~। प्रथमोदाहरणे रूपं कनिष्ठं रूपत्रयमृणक्षेपं च प्रकल्प्य न्यासः प्र ६७ ह्र १ ज्ये ८ क्षे ३ं~। अत्र ह्रस्वं भाज्यं क्षेपं भाजकं ज्येष्ठं क्षेपं च प्रकल्प्य कुट्टकार्थं न्यासः \;{\small $\begin{matrix}
\mbox{{भा ~१ ~क्षे ~८}}\\
\vspace{-1.5mm}
\mbox{{\hspace{7mm} ह ~३}}
\vspace{1mm}
\end{matrix}$}~। अत्र \hyperref[5.56]{'हरतष्टे धनक्षेप'} इति कृते जाता वल्ली \,{\small $\begin{matrix}
\mbox{{०}}\\
\mbox{{२}}\\
\vspace{-1.5mm}
\mbox{{०}}
\vspace{1mm}
\end{matrix}$}\, लब्धिगुणौ \,{\small $\begin{matrix}
\mbox{{०}}\\
\vspace{-1.5mm}
\mbox{{२}}
\vspace{1mm}
\end{matrix}$}\, लब्धिवैषम्यात्स्वतक्षणशुद्धौ \,{\small $\begin{matrix}
\mbox{{१}}\\
\vspace{-1.5mm}
\mbox{{१}}
\vspace{1mm}
\end{matrix}$}\, क्षेपतक्षणलाभा\textendash \,२\textendash \,द्या लब्धिरिति लब्धिगुणौ \,{\small $\begin{matrix}
\mbox{{३}}\\
\vspace{-1.5mm}
\mbox{{१}}
\vspace{1mm}
\end{matrix}$}~। हरस्यर्णत्वाल्लब्धे ऋणत्वे कृते जातौ सक्षेपौ \;{\small $\begin{matrix}
\mbox{{क्षे ~१ ~ल ~३ं}}\\
\vspace{-1.5mm}
\mbox{{क्षे ~३ं ~गु ~१}}
\vspace{1mm}
\end{matrix}$}\; अस्य गुणस्य १ वर्गे प्रकृतेः ६७ विशोधिते शेषं ६६ अल्पं न स्यात्~। अतो रूपद्वयमृणमिष्टं २ प्रकल्प्येष्टाहतस्वस्वहरेणेत्यादिना वा जातौ लब्धिगुणौ \,{\small $\begin{matrix}
\mbox{{५ं}}\\
\vspace{-1.5mm}
\mbox{{७}}
\vspace{1mm}
\end{matrix}$}~। अस्य गुणस्य ७ वर्गे ४९ प्रकृतेः ६७ शोधिते शेषं १८ पूर्वक्षेपेणानेन ३ं हृते लब्धं ६ं अयं क्षेपः~। गुणवर्गे प्रकृतेः ७ विशोधिते व्यस्तः स्यादिति धनं क्षेपः ६ लब्धिस्तु ५ं कनिष्ठं पदम्~। अस्यर्णत्वे धनत्वे चेष्टं ह्रस्वं तस्य वर्ग इत्यादावुत्तरकर्मणि न विशेषोऽस्तीति जातं धनं कनिष्ठम् ५~। अस्य वर्गे प्रकृतिगुणे षड्युते जातं मूलं ज्येष्ठम् ४१~। अथवा मदुक्तप्रकारेण ज्येष्ठं ८ गुणक\textendash \,७\textendash \,गुणितं ५६ कनिष्ठेन १ प्रकृति\textendash \,६७\textendash \,गुणेन ६७ युतं १२३ क्षेपेण ३ं भक्तं ४ं१ जातं ज्येष्ठम्~। अस्यापि कनिष्ठस्येव धनत्वमिति जातं तदेव ज्येष्ठम् ४१~। एवं जाता ह्रस्वज्येष्ठक्षेपाः ह्र ५ ज्ये ६१ क्षे ६~। पुनरेषां कुट्टकार्थं न्यासः \;{\small $\begin{matrix}
\mbox{{भा ~५ ~क्षे ~४१}}\\
\vspace{-1.5mm}
\mbox{{\hspace{7mm} ह ~~६}}
\vspace{1mm}
\end{matrix}$}~। अत्र पूर्ववल्लब्धिगुणौ सक्षेपौ \;{\small $\begin{matrix}
\mbox{{क्षे ~५ ~ल ~११}}\\
\vspace{-1.5mm}
\mbox{{क्षे ~६ ~गु ~~५}}
\vspace{1mm}
\end{matrix}$}~। अस्यैव गुणस्य ५ वर्गे २५ प्रकृतेः शोधितेऽल्पमन्तरं ४२ भवति~। इदमन्तरं ४२ क्षेपेण ६
\end{sloppypar}

\newpage

\begin{sloppypar}
\noindent हृतं ७ जातः क्षेपः~। प्रकृतितश्च्युते व्यस्त इति जातः क्षेपः ७ं~। लब्धिः कनिष्ठम् ११~। अस्य वर्गे प्रकृतिगुणे सप्तहीने मूलं ज्येष्ठम् ९०~। अथवा पूर्वज्येष्ठं ४१ गुण\textendash \,५\textendash \,गुणितं २०५ कनिष्ठेन ५ प्रकृति\textendash \,६७\textendash \,गुणेन ३३५ युतं ५४० क्षेपेण ६ हृतं ९० जातं ज्येष्ठम्~। एवं जाताः कनिष्ठज्येष्ठक्षेपाः क ११ ज्ये ९० क्षे ७ं~। पुनरेषां कुट्टकार्थं न्यासः \;{\small $\begin{matrix}
\mbox{{भा ~११ ~क्षे ~९०}}\\
\vspace{-1.5mm}
\mbox{{\hspace{9mm} ह ~~७}}
\vspace{1mm}
\end{matrix}$}~। अत्र हरतष्टे धनक्षेप इति जाता वल्ली \,{\small $\begin{matrix}
\mbox{{१}}\\
\mbox{{१}}\\
\mbox{{१}}\\
\mbox{{६}}\\
\vspace{-1mm}
\mbox{{०}}
\vspace{1mm}
\end{matrix}$}\, राशिद्वयं \;{\small $\begin{matrix}
\mbox{{१८}}\\
\vspace{-1.5mm}
\mbox{{१२}}
\vspace{1mm}
\end{matrix}$}\; तष्टं \;{\small $\begin{matrix}
\mbox{{७}}\\
\vspace{-1.5mm}
\mbox{{५}}
\vspace{1mm}
\end{matrix}$} लब्धयो विषमा इति स्वतक्षणाच्छोधनेन जातौ लब्धिगुणौ \;{\small $\begin{matrix}
\mbox{{४}}\\
\vspace{-1.5mm}
\mbox{{२}}
\vspace{1mm}
\end{matrix}$}~। क्षेपतक्षणलाभा\textendash \,१२\textendash \,ढ्या लब्धिरिति जातौ \;{\small $\begin{matrix}
\mbox{{१६}}\\
\vspace{-1.5mm}
\mbox{{२}}
\vspace{1mm}
\end{matrix}$}\, हरस्यर्णत्वाल्लब्धेर्ऋणत्वमिति जातौ सक्षेपो लब्धिगुणौ \;{\small $\begin{matrix}
\mbox{{क्षे ~११ ~ल ~१ं६}}\\
\vspace{-1.5mm}
\mbox{{क्षे ~~७ं ~गु ~~२}}
\vspace{1.5mm}
\end{matrix}$}~। अस्य गुणस्य २ वर्गस्य ४ प्रकृते\textendash \,६७\textendash \,श्चान्तरं ६३ अल्पं न स्यादिति रूपमृणमिष्टं १ं प्रकल्प्य क्षेपे क्षिप्ते जातौ लब्धिगुणौ \;{\small $\begin{matrix}
\mbox{{ल ~२७ं}}\\
\vspace{-1.5mm}
\mbox{{गु ~~९}}
\vspace{1.5mm}
\end{matrix}$}~। अस्य गुणस्य ९ वर्गे ८१ प्रकृत्या ६७ हीने १४ शेषं क्षेपेण ७ं भक्तं जातः क्षेपः २ं लब्धिः २७ं कनिष्ठं पूर्वद्धनम् २७~। अस्य वर्गे प्रकृतिगुणे द्व्यूने मूलं ज्येष्ठम् २२१~। अथवा पूर्वज्येष्ठं ९० गुण\textendash \,९\textendash \,गुणितं ८१० कनिष्ठेन ११ प्रकृति\textendash \,६७\textendash \,गुणेन ७३७ युतं १५४७ क्षेपेण ७ भक्तं २२१ जातं धनं कनिष्ठवत् २२१~। एवं कनिष्ठज्येष्ठक्षेपाः क २७ ज्ये २२१ क्षे २~। अथानयोस्तुल्यभावनार्थं न्यासः \;{\small $\begin{matrix}
\mbox{{क ~२७ ~ज्ये ~२२१ ~क्षे ~२ं}}\\
\vspace{-1.5mm}
\mbox{{क ~२७ ~ज्ये ~२२१ ~क्षे ~२ं}}
\vspace{1mm}
\end{matrix}$}\; भावनया जाते चतुःक्षेपमूले क ११९३४ ज्ये ९७६८४ क्षे ४~। द्वयमिष्टं प्रकल्प्येष्टवर्गहृतः क्षेप इति जाते रूपक्षेपमूले क ५९६७ ज्ये ४८८४२ क्षे १~। अथ
\end{sloppypar}

\newpage

\begin{sloppypar}

\noindent द्वितीयोदाहरण एकमिष्टं कनिष्ठं प्रकल्प्य रूपत्रयं क्षेपं च प्रकल्प्य न्यासः प्र ६१ क १ ज्ये ८ क्षे ३~। कुट्टकार्थं न्यासः \;{\small $\begin{matrix}
\mbox{{भा ~१ ~क्षे ~८}}\\
\vspace{-1.5mm}
\mbox{{\hspace{7mm} ह ~३}}
\vspace{1mm}
\end{matrix}$}~। प्राग्वद्धरतष्टे धनक्षेप इति जातं राशिद्वयं २~। लब्धिवैषम्यात्स्वतक्षणशोधने क्षेपतक्षणलाभाढ्या लब्धिरिति च कृते जातौ \;{\small $\begin{matrix}
\mbox{{३ ~क्षे ~१}}\\
\vspace{-1.5mm}
\mbox{{१ ~क्षे ~३}}
\vspace{1mm}
\end{matrix}$}~। अस्य गुणस्य १ वर्गे १ प्रकृतेः शोधितेऽन्तरं ६० अल्पं न स्यादिति द्वयमिष्टं प्रकल्प्य वा जातौ लब्धिगुणौ \;{\small $\begin{matrix}
\mbox{{ल ~५}}\\
\vspace{-1.5mm}
\mbox{{गु ~७}}
\vspace{1mm}
\end{matrix}$}~। अस्य गुणस्य ७ वर्गे ४९ प्रकृतेः ६१ शोधिते शेषं १२ क्षेपेण ३ भक्तं जातः क्षेपः ४~। प्रकृतितश्च्युते गुणवर्गे व्यस्त इति जातं ४ं लब्धिः ५ कनिष्ठम्~। अस्य वर्गे २५ प्रकृतिगुणे १५२५ चतुरूने १५२१ पदं ज्येष्ठं ३९~। अथवा पूर्वज्येष्ठं ८ गुण\textendash \,७\textendash \,गुणं ५६ कनिष्ठेन १ प्रकृति\textendash \,६१\textendash \,गुणेन ६१ युतं ११७ क्षेपेण ३ भक्तं जातं तदेव ज्येष्ठम् ३९~। एवं कनिष्ठज्येष्ठक्षेपाः क ५ ज्ये ३९ क्षे ४ं~। इष्टवर्गहृतः क्षेप इत्युत्पन्नरूपशुद्धिमूलयोर्भावनार्थं न्यासः \;{\small $\begin{matrix}
\vspace{1mm}
\mbox{{क ~{\scriptsize $\begin{matrix}
\mbox{{५}}\\
\vspace{-1.5mm}
\mbox{{२}}
\vspace{1mm}
\end{matrix}$} ~ज्ये ~{\scriptsize $\begin{matrix}
\mbox{{३९}}\\
\vspace{-1.5mm}
\mbox{{२}}
\vspace{1mm}
\end{matrix}$} ~क्षे ~१ं}}\\
\vspace{-1.5mm}
\mbox{{क ~{\scriptsize $\begin{matrix}
\mbox{{५}}\\
\vspace{-1.5mm}
\mbox{{२}}
\vspace{1mm}
\end{matrix}$} ~ज्ये ~{\scriptsize $\begin{matrix}
\mbox{{३९}}\\
\vspace{-1.5mm}
\mbox{{२}}
\vspace{1mm}
\end{matrix}$} ~क्षे ~१ं}}
\vspace{2mm}
\end{matrix}$}~। भावनया जाते रूपक्षेपमूले क १९५ ज्ये १५२३ क्षे १~। अनयोः पुना रूपशुद्धिपदाभ्यां भावनार्थं न्यासः \;{\small $\begin{matrix}
\vspace{1mm}
\mbox{{क ~{\scriptsize $\begin{matrix}
\mbox{{१९५}}\\
\vspace{-1.5mm}
\mbox{{२}}
\vspace{1mm}
\end{matrix}$} ~ज्ये ~{\scriptsize $\begin{matrix}
\mbox{{१५२३}}\\
\vspace{-1.5mm}
\mbox{{२}}
\vspace{1mm}
\end{matrix}$} ~क्षे ~१}}\\
\vspace{-1.5mm}
\mbox{{क ~~{\scriptsize $\begin{matrix}
\mbox{{५}}\\
\vspace{-1.5mm}
\mbox{{२}}
\vspace{1mm}
\end{matrix}$} ~~ज्ये ~~{\scriptsize $\begin{matrix}
\mbox{{३९}}\\
\vspace{-1.5mm}
\mbox{{२}}
\vspace{1mm}
\end{matrix}$} ~~क्षे ~१ं}}
\vspace{2mm}
\end{matrix}$}~। अतो जाते रूपशुद्धौ मूले ३८०५~। २९७१८ क्षे १ं~। अनयोस्तुल्यभावनया जाते रूपक्षेपमूले क २२६१५३९८० ज्ये १७६६३१९०४९ क्षे १~। अतः पुनः पुनर्भावनावशादानन्त्यम्~॥~७६~॥\\

{\small अथ रूपशुद्धौ खिलत्वमनुष्टुभ उत्तरार्धेन निरूपयति\textendash }

\phantomsection \label{6.77}
\begin{quote}
{\large \textbf{{\color{purple}रूपशुद्धौ खिलोद्दिष्टं वर्गयोगो गुणो न चेत्~॥~७७~॥}}}
\end{quote}

यदि प्रकृतिर्वर्गयोगरूपा न भवेत्तर्हि रूपशुद्धावुद्दिष्टं खिलं ज्ञेयम्~। कस्यापि वर्गस्तया प्रकृत्या गुणितो रूपोनः सन्मूलदो नैव भवेदित्यर्थः~। अत्रोपपत्तिः~। यदि ऋणक्षेपो वर्गरूपः स्यात्तदा ऋणं रूपक्षेपोऽपि भवेत्~। इष्टवर्गहृतः क्षेप इत्यादिना~। ऋणक्षेपो वर्गरूपस्तु तदैव भवेद्यदि प्रकृतिगुणः कनिष्ठवर्गो वर्गयोगात्मकः स्यात्~। तथा सत्येकस्मिन्वर्गे शोधितेऽपरवर्गस्य मूलसम्भवात्~। प्रकृतिगुणः कनिष्ठवर्गो वर्गयोगात्मकस्तदैव स्यात्~। यदि प्रकृतिर्वर्गयोगात्मिका स्यात्~। यतो वर्गण गुणितो वर्गो
\end{sloppypar}

\newpage

\begin{sloppypar}
\noindent वर्ग एव भवतीति प्रकृतेः खण्डद्वयं यदि वर्गात्मकं स्यात्तदा ताभ्यां खण्डाभ्यां कनिष्ठवर्गस्य पृथग्गुणने खण्डद्वयमपि वर्गरूपं स्यात्~। तयोर्योगो वर्गयोगः स्यात्स एव सम्पूर्णप्रकृत्या गुणितः कनिष्ठवर्गो भवतीति प्रकृतेर्वर्गयोगरूपत्वे प्रकृतिगुणः कनिष्ठवर्गोऽपि वर्गयोगात्मकः स्यादित्युपपन्नम् \hyperref[6.77]{'रूपशुद्धौ खिलोद्दिष्टं वर्गयोगो गुणो न चेत्'} इति~॥~७७~॥\\

{\small अथाखिलत्वे रूपशुद्धौ प्रकारान्तरेण पदानयनमनुष्टुभानुष्टुप्पूर्वार्धेन चाह\textendash }

\phantomsection \label{6.78}
\begin{quote}
{\large \textbf{{\color{purple}अखिले कृतिमूलाभ्यां द्विधा रूपं विभाजितम्~।\\
द्विधा ह्रस्वपदं ज्येष्ठं ततो रूपविशोधने~।\\
पूर्ववद्वा प्रसाध्येते पदे रूपविशोधने~॥~७८~॥}}}
\end{quote}

अखिले सति ययोर्वर्गयोर्योगः प्रकृतिरस्ति तयोर्मूलाभ्यां द्विधा रूपं विभाजितं सद्रूपशुद्धौ द्विधा ह्रस्वपदं भवति~। ततस्ताभ्यां कनिष्ठाभ्यां तस्य वर्गः प्रकृत्या क्षुण्ण इत्यादिना ज्येष्ठपदमपि द्विधा भवति~। यद्वाखिलत्वे सति पूर्ववदिष्टं ह्रस्वमित्यादिना ऋणे चतुरादिक्षेपे पदे प्रसाध्य \hyperref[6.72]{'इष्टवर्गहृतः क्षेपः'} इत्यादिना रूपशुद्धौ पदे प्रसाध्ये~। अत्रोपपत्तिः~। ययोर्वर्गयोर्योगः प्रकृतिरस्ति ताभ्यां वर्गाभ्यां कनिष्ठवर्गः पृथग्गुणितो युतश्चेत्प्रकृत्यैव गुणितः स्यात्~। अस्मात्प्रकृतिगुणकनिष्ठवर्गात् प्रकृतिखण्डभूतयोर्वर्गयोरन्यतरेण गुणितः कनिष्ठवर्गश्चेच्छोध्यते तर्हीतरगुणितः कनिष्ठवर्गोऽवशिष्यत इति तस्यावश्यं मूललाभात् अन्यतरेण वर्गेण गुणितः कनिष्ठवर्ग एव ऋणं क्षेपः सम्भवति~। अथ रूपशुद्ध्यर्थम् अन्यतरवर्गस्य पदेन गुणितं कनिष्ठमिष्टं प्रकल्प्य \hyperref[6.72]{'इष्टवर्गहृतः क्षेपः'} इति कृते रूपमृणं क्षेपो भवति~। अथेष्टेन कनिष्ठं भाज्यम्~। इष्टं तु वर्गस्य पदेन गुणितं कनिष्ठम्~। अत्र भाज्यभाजकयोः कनिष्ठेनापवर्ते जातं भाज्यस्थाने रूपम्~। भाजकस्थाने तु प्रकृतिखण्ड-भूतस्य वर्गस्य पदमिति~। अत उपपन्नम् \hyperref[6.78]{'कृतिमूलाभ्यां द्विधा रूपं विभाजितम्~। द्विधा ह्रस्वपदम्'} इति~॥~७८~॥\\

{\small अत्रोदाहरणद्वयमनुष्टुभाह\textendash }

\phantomsection \label{6.79}
\begin{quote}
{\large \textbf{{\color{purple}त्रयोदशगुणो वर्गो निरेकः कः कृतिर्भवेत्~।\\
को वाष्टगुणितो वर्गो निरेको मूलदो वद~॥~७९~॥}}}
\end{quote}

अत्र प्रथमोदाहरणे प्रकृतिर्द्विकत्रिकयोर्वर्गयोगः १३~। अतो द्विकेन हृतं रूपं रूपशुद्धौ कनिष्ठपदं \;{\small $\begin{matrix}
\mbox{{१}}\\
\vspace{-1.5mm}
\mbox{{२}}
\vspace{1mm}
\end{matrix}$}\, स्यात्~। अस्य वर्गात्प्रकृति\textendash \,१३\textendash \,गुणात् \;{\small $\begin{matrix}
\mbox{{१३}}\\
\vspace{-1.5mm}
\mbox{{४}}
\vspace{1mm}
\end{matrix}$}\, एकोनात् \;{\small $\begin{matrix}
\mbox{{९}}\\
\vspace{-1.5mm}
\mbox{{४}}
\vspace{1mm}
\end{matrix}$}\, मूलं ज्येष्ठपदम् \;{\small $\begin{matrix}
\mbox{{३}}\\
\vspace{-1.5mm}
\mbox{{२}}
\vspace{1mm}
\end{matrix}$}~। अथवा त्रिकेण हृतं रूपं कनिष्ठं स्यात् \;{\small $\begin{matrix}
\mbox{{१}}\\
\vspace{-1.5mm}
\mbox{{३}}
\vspace{1mm}
\end{matrix}$}~। अतः पूर्ववज्ज्येष्ठम् \;{\small $\begin{matrix}
\mbox{{२}}\\
\vspace{-1.5mm}
\mbox{{३}}
\vspace{1mm}
\end{matrix}$}~। अथवा 
\end{sloppypar}

\newpage

\begin{sloppypar}
\noindent पूर्ववद्यथा इष्टं १ कनिष्ठम्~। अस्य वर्गात्प्रकृतिगुणात् १३ चतुरूनात् ९ मूलं ज्येष्ठम् ३~। क्रमेण न्यासः क १ ज्ये ३ क्षे ४ं~। \hyperref[6.72]{'इष्टवर्गहृतः क्षेपः'} इत्यादिना रूपद्वयमिष्टं प्रकल्प्य जाते रूपशुद्धौ पदे क \;{\small $\begin{matrix}
\mbox{{१}}\\
\vspace{-1.5mm}
\mbox{{२}}
\vspace{1mm}
\end{matrix}$}\, ज्ये \;{\small $\begin{matrix}
\mbox{{३}}\\
\vspace{-1.5mm}
\mbox{{२}}
\vspace{1mm}
\end{matrix}$}\, क्षे १ं~। अथवा प्रकृति\textendash \,१३\textendash \,गुणात्कनिष्ठवर्गात् १३ नव विशोध्य जातं ज्येष्ठम् २~। क्रयेण न्यासः क १ ज्ये २ क्षे ९ं~। \hyperref[6.72]{'इष्टवर्गहृतः क्षेपः'} इत्यादिना जाते रूपशुद्धौ मूले क {\small $\begin{matrix}
\mbox{{१}}\\
\vspace{-1.5mm}
\mbox{{३}}
\vspace{1mm}
\end{matrix}$} ज्ये \;{\small $\begin{matrix}
\mbox{{२}}\\
\vspace{-1.5mm}
\mbox{{३}}
\vspace{1mm}
\end{matrix}$} क्षे १ं~। चक्रवालेनाभिन्ने वा~। रूपशुद्धौ पूर्वपदयोर्न्यासः क \;{\small $\begin{matrix}
\mbox{{१}}\\
\vspace{-1.5mm}
\mbox{{२}}
\vspace{1mm}
\end{matrix}$}\, ज्ये \;{\small $\begin{matrix}
\mbox{{३}}\\
\vspace{-1.5mm}
\mbox{{२}}
\vspace{1mm}
\end{matrix}$}\, क्षे १ं~। ह्रस्वज्येष्ठपदक्षेपानित्यादिना कुट्टकार्थं न्यासः \;{\small $\begin{matrix}
\vspace{0.5mm}
\mbox{{भा ~{\scriptsize $\begin{matrix}
\mbox{{१}}\\
\vspace{-1.5mm}
\mbox{{२}}
\vspace{1mm}
\end{matrix}$} ~क्षे ~{\scriptsize $\begin{matrix}
\mbox{{३}}\\
\vspace{-1.5mm}
\mbox{{२}}
\vspace{1mm}
\end{matrix}$}}}\\
\vspace{-1.5mm}
\mbox{{\hspace{8mm} ह ~१ं}}
\vspace{1mm}
\end{matrix}$}~। अत्र भाज्यभाजकक्षेपानर्धेन \;{\small $\begin{matrix}
\mbox{{१}}\\
\vspace{-1.5mm}
\mbox{{२}}
\vspace{1mm}
\end{matrix}$}\, अपवर्त्य न्यासः \;{\small $\begin{matrix}
\mbox{{भा ~१ ~क्षे ~३}}\\
\vspace{-1.5mm}
\mbox{{\hspace{8mm} ह ~२ं}}
\vspace{1mm}
\end{matrix}$}~। अत्र \hyperref[5.56]{'हरतष्टे धनक्षेपः'} इत्यादिना जातं राशिद्वयम् \;{\small $\begin{matrix}
\mbox{{०}}\\
\vspace{-1.5mm}
\mbox{{१}}
\vspace{1mm}
\end{matrix}$}~। लब्धयो विषमा इति स्वतक्षणाच्छोधने कृते क्षेपतक्षणलाभाढ्या लब्धिरिति च कृते जातौ लब्धिगुणौ \;{\small $\begin{matrix}
\mbox{{ल ~२ं ~क्षे ~१}}\\
\vspace{-1.5mm}
\mbox{{गु ~१ ~क्षे ~२ं}}
\vspace{1mm}
\end{matrix}$}~। अत्रास्य गुणस्य १ वर्गे प्रकृतितश्च्युतेऽन्तरं १२ अल्पं न भवतीति रूपमृणमिष्टं प्रकल्प्य क्षेपे क्षिप्ते जातौ लब्धिगुणौ \;{\small $\begin{matrix}
\mbox{{ल ~३ं}}\\
\vspace{-1.5mm}
\mbox{{गु ~३}}
\vspace{1mm}
\end{matrix}$}~। अस्य गुणस्य वर्गे ९ प्रकृतितश्च्युते शेषं ४ क्षेप\textendash \,१ं\textendash \,भक्तं जातः क्षेपः ४ं \hyperref[6.75]{'व्यस्तः प्रकृतितश्चुत'} इति व्यस्तः ४~। लब्धिः ३ं कनिष्ठं प्राग्वद्धनं ३~। अतो ज्येष्ठम् ११~। क्रमेण न्यासः क ३ ज्ये ११ क्षे ४~। पुनः कुट्टकार्थं न्यासः \;{\small $\begin{matrix}
\mbox{{भा ~३ ~क्षे ~११}}\\
\vspace{-1.5mm}
\mbox{{\hspace{7mm} ह ~~४}}
\vspace{1mm}
\end{matrix}$}~। पूर्ववल्लब्धिगुणौ \;{\small $\begin{matrix}
\mbox{{ल ~९ ~क्षे ~३}}\\
\vspace{-1.5mm}
\mbox{{गु ~३ ~क्षे ~४}}
\vspace{1mm}
\end{matrix}$}~। अस्य गुणस्य वर्गे ९ प्रकृतितश्च्युते शेषं ४ क्षेप ४ भक्तं जातः क्षेपः १~। \hyperref[6.75]{'व्यस्तः प्रकृतितश्च्युत'} इति व्यस्तः १ं~। लब्धिः ५ कनिष्ठम्~। अतो ज्येष्ठं १८~। क्रमेण न्यासः क ५ ज्ये १८ क्षे १ं~। एवं रूपशुद्धौ जाते मूले अभिन्ने~। अत्र सर्वत्र रूपक्षेपजपदाभ्यां भावनया पदानामानन्त्यं ज्ञेयम्~। अथ द्वितीयोदाहरणे प्रकृतिः ८
\end{sloppypar}

\newpage

\begin{sloppypar}
\noindent अयं द्विकयोर्वर्गयोगः~। प्राग्वज्जाते ह्रस्वज्येष्ठे क \,{\small $\begin{matrix}
\mbox{{१}}\\
\vspace{-1.5mm}
\mbox{{२}}
\vspace{1mm}
\end{matrix}$}\, ज्ये १ क्षे १ं~। प्राग्वच्चक्रवालेनाभिन्ने कार्ये~॥~७९~॥\\

{\small अथवा \hyperref[6.72]{'क्षेपः क्षुण्णः क्षुण्णे तदा पदे'} इत्यस्योदाहरणमनुष्टुभाह\textendash }

\phantomsection \label{6.80}
\begin{quote}
{\large \textbf{{\color{purple}को वर्गः षड्गुणस्त्र्याढ्यो द्वादशाढ्योऽथवा कृतिः~।\\
युतो वा पञ्चसप्तत्या त्रिशत्या वा कृतिर्भवेत्~॥~८०~॥}}}
\end{quote}

स्पष्टोऽर्थः~। अत्र रूपमिष्टं कनिष्ठं प्रकल्प्य न्यासः प्र ६ क १ ज्ये ३ क्षे ३~। अत्र द्वादशक्षेपार्थमयं क्षेप इष्टवर्गेणानेन ४ क्षुण्णश्चेदिष्टेन २ पदे गुण्ये~। तथा सति जाते द्वादशक्षेपपदे क २ ज्ये ६ क्षे १२~। एवमनयैव युक्त्या पञ्चगुणे ते एव पदे जाते पञ्चसप्ततिक्षेपे क ५ ज्ये १५ क्षे ७५~। एवं दशगुणे जाते त्रिशतीक्षेपे क १० ज्ये ३० क्षे ३००~। इदमुपलक्षणम्~॥~८०~॥\\

{\small येम केनाप्युपायेनोद्दिष्टक्षेपे पदे प्रसाध्य पश्चाद्रूपक्षेपभावनयानन्त्यं तयोर्भवतीत्यनुष्टुभ उत्तरार्ध-पूर्वार्धाभ्यामाह\textendash }

\phantomsection \label{6.81}
\begin{quote}
{\large \textbf{{\color{purple}स्वबुद्ध्यैव पदे ज्ञेये बहुक्षेपविशोधने~।\\
तयोर्भावनयानन्त्यं रूपक्षेपपदोत्थया~॥~८१~॥}}}
\end{quote}

क्षेपाश्च विशोधनानि च क्षेपविशोधनानि~। बहूनां क्षेपविशोधनानां समाहारो बहुक्षेप-विशोधनम्~। तस्मिन्~। यस्मिन्कस्मिन्नपि क्षेपे धने ऋणे वा प्रथमतः स्वबुद्ध्यैव पदे ज्ञेये इत्यर्थः~। पश्चाद्रूपक्षेपपदोत्थया भावनया तयोरानन्त्यं सुलभम्~। यतस्तत्राभ्यासः क्षेपयोः क्षेपकः स्यादिति रूपक्षेपेण गुणितो यः कश्चिद्धनमृणं क्षेपो यथास्थित एव स्यादिति~॥~८१~॥\\

{\small स्वबुद्ध्यैव पदे ज्ञेये इत्युक्तम्~। तत्र कांश्चित्प्रकारान्दर्शयति~। तत्रापि वर्गभक्तायां प्रकृतौ पदानयने प्रकारान्तरमनुष्टुभ उत्तरार्धेनाह\textendash }

\phantomsection \label{6.82}
\begin{quote}
{\large \textbf{{\color{purple}वर्गच्छिन्ने गुणे ह्रस्वं तत्पदेन विभाजयेत्~॥~८२~॥}}}
\end{quote}

वर्गच्छिन्ने गुणे सति ह्रस्वं तत्पदेन विभाजयेत्~। एतदुक्तं भवति~। प्रकृतिं केनचित् वर्गेणापवर्त्यापवर्तितया प्रकृत्या कनिष्ठज्येष्ठे साध्ये~। तत्र येन वर्गेण प्रकृतेरपवर्तः कृतस्तस्य पदेन कनिष्ठं भाज्यम्~। ज्येष्ठं तु यथास्थितमेव~। उद्दिष्टप्रकृतावेते पदे भवत इत्यर्थः~। अत्रो-पपत्तिः~। प्रकृतौ केनचिद्वर्गेणापवर्तितायां ज्येष्ठवर्गोऽपि तेनैव वर्गेणाप-
\end{sloppypar}

\newpage

\begin{sloppypar}
\noindent वर्तितः स्यात्~। अतो ज्येष्ठं तन्मूलेनापवर्तितं स्यात्~। कनिष्ठं तु नापवर्तितं स्यात्~। नहि प्रकृतिकृतो विशेषः कनिष्ठेऽस्ति येन प्रकृतेर्गुणने भजने वा कनिष्ठं गुणितमपवर्तितं वा स्यात्~। अतस्तन्मूलेन कनिष्ठमेव भाज्यम्~। ज्येष्ठं तु भक्तमेवेति~। अनयैव युक्त्या प्रकृतिं केनचिद्वर्गेण सङ्गुण्य तादृश्या प्रकृत्या कनिष्ठज्येष्ठे प्रसाध्य कनिष्ठं तत्पदेन गुणयेदित्यपि बोध्यम्~॥~८२~॥\\

{\small अत्रोदाहरणमनुष्टुभोऽर्धेनाह\textendash }

\phantomsection \label{6.83}
\begin{quote}
{\large \textbf{{\color{purple}द्वात्रिंशद्गुणितो वर्गः कः सैको मूलदो वद~॥~८३~॥}}}
\end{quote}

स्पष्टोऽर्थः~। अर्धमिहेष्टं कनिष्ठं प्रकल्प्य प्राग्वज्जाते मूले प्र ३२ क \,{\scriptsize $\begin{matrix}
\mbox{{१}}\\
\vspace{-1.5mm}
\mbox{{२}}
\vspace{1mm}
\end{matrix}$}\, ज्ये ३ क्षे १~। अथवा प्रकृतिः ३२ चतुर्भिश्छिन्ना ८~। अनया प्रकृत्या कनिष्ठज्येष्ठे क १ ज्ये ३ क्षे १~। चतुर्णां पदेन कनिष्ठमेव विभज्य जाते द्वात्रिंशत्प्रकृतौ पदे क \,{\scriptsize $\begin{matrix}
\mbox{{१}}\\
\vspace{-1.5mm}
\mbox{{२}}
\vspace{1mm}
\end{matrix}$}\, ज्ये ३ क्षे १~। एवं षोडशभिरपि प्रकृतिं छित्वा प्र २ जाते कनिष्ठज्येष्ठे क २ ज्ये ३ क्षे १~। प्राग्वत्कनिष्ठं षोडशमूलेन ४ विभज्य जाते ते एव कनिष्ठज्येष्ठे क \,{\scriptsize $\begin{matrix}
\mbox{{१}}\\
\vspace{-1.5mm}
\mbox{{२}}
\vspace{1mm}
\end{matrix}$}\, ज्ये ३ क्षे १~। एवमन्यत्रापि~॥~८३~॥\\

{\small अथ प्रकृतौ वर्गरूपायां पदानयन उपायान्तरमनुष्टुभाह\textendash }

\phantomsection \label{6.84}
\begin{quote}
{\large \textbf{{\color{purple}इष्टभक्तो द्विधा क्षेप इष्टोनाढ्यो दलीकृतः~।\\
गुणमूलहृतश्चाद्यो ह्रस्वज्येष्ठे क्रमात्पदे~॥~८४~॥}}}
\end{quote}

उद्दिष्टक्षेप इष्टेन भक्तः सन्द्विधा स्थाप्यः~। स एकत्रेष्टेनोनः~। अपरत्रेष्टेन युतः~। उभयत्रापि दलीकृतोऽर्धितः~। आद्यस्तु गुणमूलहृतः प्रकृतिमूलहृत इत्यर्थः~। क्रमाद्ध्रस्वज्येष्ठपदे स्तः~। अत्र गुणमूलहृतस्त्वाद्य इत्युक्तेर्यत्र वर्गरूपा प्रकृतिर्भवति तत्रैवास्य सूत्रस्यावसर इति ज्ञेयम्~। अत्रोपपत्तिः~। यत्र वर्गरूपा प्रकृतिस्तत्र क्षेपाभाव एव ज्येष्ठमूलं लभ्यते~। यतः कनिष्ठवर्गे वर्गरूपप्रकृत्या गुणिते वर्ग एव स्यात्~। अथ क्षेपे क्षिप्तेऽपि चेदस्य मूलं लभ्यते~। नूनमयं युतिवर्गः~। यतोऽस्य मूलं प्रथमज्येष्ठात्किञ्चिदधिकं स्यात्~। तथा च यावदधिकं तावता युक्तस्य ज्येष्ठस्य वर्गोऽयमित्यधिकज्येष्ठयोर्युतिवर्गोऽयम्~। अत्र युतिवर्गे 'खण्डद्वयस्याभिहतिर्द्विनिघ्नी तत्खण्डवर्गैक्ययुता कृतिः' इति खण्डत्रयेण भाव्यम्~। ज्येष्ठवर्गो ज्येष्ठाधिकयोर्द्विघ्नो घातोऽधिकवर्गश्चेति~। अत्र क्षेपात्पूर्वं केवलज्येष्ठवर्गस्थितः क्षेपे क्षिप्ते तु युतिवर्गो भवतीति क्षेपेऽस्ति खण्डद्वयम्~। अधिकवर्गो ज्येष्ठाधिकघातो द्विघ्नश्चेति~। अत्र किमधिकमिति न ज्ञायते~। तदिष्टं प्रकल्प्य जातः क्षेपः इ ० ज्ये २ इव १~। अस्मिन्क्षेप इष्टहृते जातं ज्ये २ इ १~। द्विगुणज्येष्ठमिष्टयुक्तम्~। अत्र
\end{sloppypar}

\newpage

\begin{sloppypar}
\noindent चेदिष्टं क्षिप्यते तदा ज्येष्ठेष्टयोर्द्विगुणा युतिर्भवति ज्ये २ इ २~। अस्यार्धं ज्येष्ठेष्टयोर्युतिः स्यात् ज्ये १ इ १~। क्षेपानन्तरमिदमेव ज्येष्ठं भवति~। एवमुपपन्नमिष्टभक्तः क्षेप इष्टोनाढ्यो दलीकृतो ज्येष्ठं भवतीति~। अथ कनिष्ठज्ञानार्थमुपायः~। तदर्थं केवलज्येष्ठं साध्यते~। यतः कनिष्ठवर्गो वर्गरूपप्रकृतिगुणित एव ज्येष्ठवर्गः~। अतः प्रकृतिमूलगुणितं कनिष्ठमेव ज्येष्ठं स्यात्~। अतो विलोमविधिना ज्येष्ठं गुणमूलहृतं कनिष्ठं स्यादिति~। अत आदौ केवलज्येष्ठं साध्यते~। क्षेपः इज्ये २ इव १~। इष्टभक्तः ज्ये २ इ १~। इष्टोनः ज्ये २~। दलीकृतः ज्ये १~। जातं केवलज्येष्ठम्~। तथा च \hyperref[6.84]{'इष्टभक्तो द्विधा क्षेप इष्टोनाढ्यो दलीकृतः'} इत्यनेन केवलज्येष्ठमिष्टाधिकज्येष्ठं च साधितम्~। तत्र केवलज्येष्ठं गुणमूलभक्तं सत्कनिष्ठं भवतीत्यत उक्तम् \hyperref[6.84]{'गुणमूलहृतश्चाद्यः'} इति~। अथवान्यथोपपत्तिः~। वर्गरूपप्रकृत्या गुणितः कनिष्ठवर्गो वर्ग एव भवेत्~। अथ क्षेपेऽपि क्षिप्ते यदि वर्गः स्यात्तर्हि क्षेपो वर्गान्तरमेव स्यात्~। तस्मात्क्षेपाभावे यज्ज्येष्ठं क्षेपे च यज्ज्येष्ठं तयोर्वर्गान्तरं क्षेपः~। अथ {\color{violet}'वर्गान्तरं राशिवियोगभक्तं योगस्ततः प्रोक्तवदेव राशी'} इत्युक्तत्वादत्रान्तरमिष्टं कल्प्यते~। तेन क्षेपे रूपवर्गान्तरे भक्ते योगो लभ्येत~। ततः सङ्क्रमणसूत्रेण राशिज्ञानं सुलभम्~। तदेवमुपपन्नम् \hyperref[6.84]{'इष्टभक्तो द्विधा क्षेप इष्टोनाढ्यो दलीकृतः'} इति~। \hyperref[6.84]{'गुणमूलहृतश्चाद्यः'} इत्यत्र तु पूर्ववदेवोपपत्तिः~। अनयैव युक्त्या ऋणक्षेपेऽपि बोध्यम्~। एतावांस्तु विशेषः~। धनक्षेपे बृहद्राशिरुद्दिष्टज्येष्ठमृणक्षेपे तु लघुराशिरुद्दिष्टज्येष्ठम्~। अत ऋणक्षेपे इष्टाढ्योनो दलीकृत इति द्रष्टव्यम्~। यद्यपि क्षेपस्यर्णत्वाङ्कनेन यथाश्रुत एव पाठेऽयमर्थः सम्पद्यते तथापि कनिष्ठज्येष्ठयोर्ऋणत्वं स्यात्~। तस्मादृणत्वाङ्कनं विनैवेष्टाढ्योन इति पाठव्यत्ययेन पदसाधनमृणक्षेपे द्रष्टव्यम्~॥~८४~॥\\

{\small अत्रोदाहरणद्वयमनुष्टुभाह\textendash }

\phantomsection \label{6.85}
\begin{quote}
{\large \textbf{{\color{purple}का कृतिर्नवभिः क्षुण्णा द्विपञ्चाशद्युता कृतिः~।\\
को वा चतुर्गुणो वर्गस्त्रयस्त्रिंशद्युता कृतिः~॥~८५~॥}}}
\end{quote}

स्पष्टोऽर्थः~। अत्र प्रथमोदाहरणे क्षेपः ५२ द्विकेनेष्टेन २ हृतो द्विष्ठः २६।२६ इष्टोनाढ्यो २४।२८ दलीकृतो जातः १२।१४ अनयोराद्यः १२ प्रकृति\textendash \,९\textendash \,मूलेन ३ भक्तो ४ जाते ह्रस्व-ज्येष्ठे ४।१४~। अथवा क्षेपं चतुर्भिर्विभज्यैवमेव जाते ह्रस्वज्येष्ठे क \;{\small $\begin{matrix}
\mbox{{३}}\\
\vspace{-1.5mm}
\mbox{{२}}
\vspace{1mm}
\end{matrix}$}\, ज्ये \;{\small $\begin{matrix}
\mbox{{१७}}\\
\vspace{-1.5mm}
\mbox{{२}}
\vspace{1mm}
\end{matrix}$}\,~। एवम् इष्टवशादानन्त्यम्~। अथ द्वितीयोदाहरणे क्षेपः ३३ प्रकृतिः ४~। अत्रैकेनेष्टेन जाते ह्रस्वज्येष्ठे क ८ ज्ये १७~। त्रिकेण वा २।७~॥~८५~॥
\end{sloppypar}

\newpage

\begin{sloppypar}
{\small अथ प्रकृतिसमक्षेपे उदाहरणद्वारा युक्तिं प्रदर्शयितुमुदाहरणमनुष्टुभाह\textendash }

\phantomsection \label{6.86}
\begin{quote}
{\large \textbf{{\color{purple}त्रयोदशगुणो वर्गः कस्त्रयोदशवर्जितः~।\\
त्रयोदशयुतो वा स्याद्वर्ग एव निगद्यताम्~॥~८६~॥}}}
\end{quote}

स्पष्टोऽर्थः~। प्रथमोदाहरणे प्रकृतिः १३ रूपमिष्टं प्रकल्प्य प्राग्वत्त्रयोदशविशोधने पदे क १ ज्ये ० क्षे १३ं~। एवं प्रकृतिसमे यत्र कुत्रापि ऋणक्षेपे रूपमेवेष्टं प्रकल्प्य ज्येष्ठपदं साध्यमिति युक्तिः प्रदर्शिता भवति~। यतो रूपमिते कनिष्ठे तद्वर्गः प्रकृतिगुणः प्रकृतिसम एव स्यात्~। तत्र क्षेपस्यापि प्रकृतिसमत्वे तच्छोधनेन शून्यतया पदमपि शून्यं स्यादिति~। अथ ज्येष्ठस्य शून्यत्वे यदि लोकस्य प्रतीतिर्नास्ति तर्हि रूपक्षेपपदोत्थया भावनयानन्त्यमिति ज्ञापयितुमाह~। अत्र \hyperref[6.73]{'इष्टवर्गप्रकृत्योर्यद्विवरम्'} इत्यादिना रूपक्षेपमूले क \;{\small $\begin{matrix}
\mbox{{३}}\\
\vspace{-1.5mm}
\mbox{{२}}
\vspace{1mm}
\end{matrix}$}\, ज्ये \;{\small $\begin{matrix}
\mbox{{११}}\\
\vspace{-1.5mm}
\mbox{{२}}
\vspace{1mm}
\end{matrix}$}\, क्षे १~। आभ्यां भावनया त्रयोदशर्णक्षेपमूले क \;{\small $\begin{matrix}
\mbox{{११}}\\
\vspace{-1.5mm}
\mbox{{२}}
\vspace{1mm}
\end{matrix}$}\, ज्ये \;{\small $\begin{matrix}
\mbox{{३९}}\\
\vspace{-1.5mm}
\mbox{{२}}
\vspace{1mm}
\end{matrix}$}\, क्षे १३ं इति स्पष्टोऽर्थः~। एवं भावनावशादानन्त्यं द्रष्टव्यम् इत्यर्थः~। एवं प्रकृतिसमे ऋणक्षेपे पदसिद्धौ सति सम्भवे धनक्षेपेऽपि पदसिद्धिः सुलभा रूपशुद्धिभावनयेति प्रदर्शयितुमाह~। एषामृणक्षेपपदानां रूपशुद्धिपदाभ्यामाभ्यां \;{\small $\begin{matrix}
\mbox{{१}}\\
\vspace{-1.5mm}
\mbox{{२}}
\vspace{1mm}
\end{matrix}$}\;।\;{\small $\begin{matrix}
\mbox{{३}}\\
\vspace{-1.5mm}
\mbox{{२}}
\vspace{1mm}
\end{matrix}$}\, विशेषसमभावनया धनत्रयोदशक्षेपमूले \;{\small $\begin{matrix}
\mbox{{३}}\\
\vspace{-1.5mm}
\mbox{{२}}
\vspace{1mm}
\end{matrix}$}\;।\;{\small $\begin{matrix}
\mbox{{१३}}\\
\vspace{-1.5mm}
\mbox{{२}}
\vspace{1mm}
\end{matrix}$}\, वा १८।६५ इति~। अत्र रूपशुद्धौ पदानयनं तु रूपशुद्धौ खिलोद्दिष्टमित्यादिना प्रागेवोक्तम्~। विशेषभावनान्तरभावना समभावना समासभावना~। शेषं स्पष्टम्~॥~८६~॥ \\

{\small एवमृणप्रकृतावपि यथासम्भवं पदानयनं द्रष्टव्यमिति तदुदाहरणमनुष्टुभाह\textendash }

\phantomsection \label{6.87}
\begin{quote}
{\large \textbf{{\color{purple}ऋणगैः पञ्चभिः क्षुण्णः को वर्गः सैकविंशतिः~।\\
वर्गः स्याद्वद चेद्वेत्सि क्षयगप्रकृतौ विधिम्~॥~८७~॥}}}
\end{quote}

स्पष्टोऽर्थः~। न्यासः प्र ५ं क्षे २१~। अत्र रूपमिष्टं प्रकल्प्येष्टं ह्रस्वमित्यादिना जाते मूले क १ ज्ये ४ क्षे २१~। वा क २ ज्ये १ क्षे २१~। रूपक्षेपभावनया पदानन्त्यं प्राग्वत्~॥~८७~॥\\

{\small अत्र ग्रन्थारम्भे वच्मि बीजक्रियां चेति प्रतिज्ञाय तदुपयोगितया निरूपितस्य धनर्णषड्विधादेः चक्रवालान्तस्य गणितस्य बीजत्वं भ्रमादधिगच्छेयुरधिगम्य च बीजत्वं }
\end{sloppypar}

\newpage

\begin{sloppypar}
\noindent {\small बीजस्य नीरसतां चावगच्छेयुः शिष्यास्तन्निरासार्थमाहानुष्टुभा\textendash }

\phantomsection \label{6.88}
\begin{quote}
{\large \textbf{{\color{purple}उक्तं बीजोपयोगीदं सङ्क्षिप्तं गणितं किल~।\\
अतो बीजं प्रवक्ष्यामि गणकानन्दकारकम्~॥~८८~॥}}}
\end{quote}

स्पष्टोऽर्थः~॥~८८~॥
\vspace{2mm}

\begin{quote}
{\color{violet}दैवज्ञवर्यगणसन्ततसेव्यपार्श्वबल्लालसञ्ज्ञगणकात्मजनिर्मितेऽस्मिन्~।\\
बीजक्रियाविवृतिकल्पलतावतारे जाता कृतिः प्रकृतिरत्र तु चक्रवालम्~॥~६~॥}
\end{quote}

अत्र वर्गप्रकृतौ चक्रवालमपि वर्गप्रकृत्यन्तर्गतमित्यर्थः~॥

\begin{center}
इति श्रीसकलगणकसार्वभौमश्रीबल्लाळदैवज्ञसुतकृष्णदैवज्ञविरचिते \\
बीजविवृतिकल्पलतावतारे निजभेदचक्रवालयुक्तवर्गप्रकृतिविवरणं समाप्तिमगमत्~॥~६~॥\\
\vspace{1mm}

अत्र मूलश्लोकैः सह ग्रन्थसङ्ख्या ३८०~। आदितो ग्रन्थसङ्ख्या २५८०~।
\vspace{6mm}

\rule{0.2\linewidth}{0.8pt}\\
\vspace{-4mm}

\rule{0.2\linewidth}{0.8pt}

\end{center}
\end{sloppypar}

\newpage
\thispagestyle{empty}

\begin{center}
\textbf{\large ७\; एकवर्णसमीकरणम्~।}\\
\rule{0.2\linewidth}{0.8pt}
\end{center}

\begin{sloppypar}
ॐ नमोऽव्यक्तनिदानाय~। अत्रातो बीजं प्रवक्ष्यामीति बीजनिरूपणं प्रतिज्ञातमतः तन्निरूपणीयम्~। तच्चतुर्विधमस्तीति प्रवदन्त्याचार्याः~। तथा हि~। प्रथममेकवर्णसमीकर-णम्~। द्वितीयमनेकवर्णसमीकरणम्~। तृतीयं मध्यमाहरणम्~। चतुर्थं भावितमिति~।~तत्र समशोधनादिनाव्यक्तराशेर्मानमवगन्तुं यत्रैकमेव वर्णमधिकृत्य पक्षयोः साम्यं क्रियते~तदेक-वर्णसमीकरणमित्युच्यते~। यत्र त्वनेकान्वर्णानधिकृत्य पक्षसाम्यं क्रियते~तदनेकवर्णसमी-करणमुच्यते~। यत्र तु वर्णवर्गादिकमधिकृत्य पक्षसाम्यं कृत्वा मूलग्रहणपूर्वकं व्यक्तमानं साध्यते तन्मध्यमाहरणम्~। यतोऽस्मिन्वर्गराशेर्मूलग्रहणे द्वयोरभिहतिं द्विनिघ्नीं शेषात्त्यजेत् इत्यनेन मध्यमस्य खण्डस्याहरणमपनयनं प्रायो भवत्यतो मध्यमाहरणमित्युच्यते~। यत्र तु भावितमधिकृत्य साम्यं क्रियते तद्भावितमित्युच्यत इति~। नन्वेकवर्णसमीकरणस्य लक्षणं नैतद्युज्यते मध्यमाहरणविशेषेऽतिव्याप्तेः~। एवमनेकवर्णसमीकरणस्यापि यत्कृतं लक्षणं तन्न युज्यते मध्यमाहरणविशेषे भाविते चातिव्याप्तेरिति चेन्न~। प्रथमलक्षणे लक्षणाक्रान्तस्य मध्यमाहरणविशेषस्यापि लक्षणत्वात्~। द्वितीयलक्षणेऽपि लक्षणाक्रान्त-योर्मध्यमाहरणविशेषभावितयोरपि लक्ष्यत्वात्~। अत एव द्वेधा विभागो मुख्यः~। एकवर्ण-समीकरणमनेकवर्णसमीकरणं चेति~। अत एवैकवर्गसमीकरणान्तर्गतमध्यमाहरणमेक-वर्णसमीकरणखण्डादनुपदमेव लिखितमन्यत्त्वनेकवर्णसमीकरणखण्डादनुपदं लिखितम् आचार्यैः~। ननु तथापि विरूद्धधर्माक्रान्तयोरेकानेकवर्णसमीकरणविशेषयोर्विरुद्धधर्मा-व्यापकेन मध्यमाहरणत्वेन कथं क्रोडीकरणमिति चेत्~। पृथिवीत्वतेजस्त्वाक्रान्तयोः पार्थि-वतैजसशरीरयोः शरीरत्वेनैवावगच्छ~। तस्मान्मुख्यो विभागः तु द्वेधैव~। एकवर्ण-समीकरणमनेकवर्णसमीकरणं चेति~। तत्राद्यं द्विविधम्~। एकवर्णसमीकरणं मध्यमा-हरणं चेति~। द्वितीयं त्रिविधम्~। अनेकवर्णसमीकरणं मध्यमाहरणं भावितं चेति~। एवं पञ्चधापि विभागः सम्भवति~। अत्र मध्यमाहरणयोस्तत्त्वेनैकीकरणे चतुर्धापि विभागः सम्भवति~। अयमेवादृत आद्यैराचार्यैः~। स्वतन्त्रेच्छस्य नियोक्तुमशक्यत्वात्~। ननु सामा-न्यविशेषरूपयोरेकवर्णसमीकरणयोः कथम् एकशब्दाभिधेयत्वमेवमनेकवर्णसमीकरणयोः अपीति चेत्~। देशविशेषस्य तदन्तर्गतनगरविशेषस्य च काश्मीरशब्दाभिधेयत्ववदवगच्छ~। सिन्धुशब्दादिवच्च~। ननु तथापि लक्षणभेद आवश्यक इति चेच्छृणु तर्हि~। सामान्यलक्षणं प्रागेवोक्तम्~। विशेषलक्षणं तु यत्रैकमेकवर्णमधिकृत्य पक्षयोः समीकरणेनविनैव मूलग्रहणं व्यक्तं मानं सिध्यति तदेकवर्णसमीकरणमिति~। एवमनेकवर्णसमीकरणस्यापि ज्ञेयम्~। ननु साक्षाद्विभाजकोपाधीनामभावाद्बीजस्य पञ्चविधत्वं चातुर्विध्यं वा न सम्भवतीति चेत्~। न~। अवान्तरविभाजकोपाधिभिरपि विभागे बाधकाभावात्~। अत एव न्यायनये साक्षाद्विभाज-
\end{sloppypar}

\newpage

\begin{sloppypar}
\noindent कोपाधिभ्यामभावस्य द्वैविध्येऽप्यवान्तरविभाजकोपाधिभिश्चातुर्विध्याङ्गीकारः~। एवमेका-दश्याः शुद्धा विद्धेति भेदद्वय एव मुख्ये सत्यप्यवान्तरोपाधिभिरष्टादशभेदस्वीकारः~। एवम् अन्यत्राप्यस्ति~। श्रीमद्भास्कराचार्याणां तु बीजद्वैविध्यमेवाभिमतमस्तीति लक्ष्यते~। यतस्ते प्रथममेकवर्णसमीकरणं बीजं द्वितीयमनेकवर्णसमीकरणं बीजमिति प्रथमद्वितीयशब्द-पूर्वकं विभागमभिधाय तदनु यत्र वर्णस्य द्वयोर्बहूनां वा वर्गादिगतानां समीकरणं तन्मध्य-माहरणम्~। यत्र भावितस्य तद्भावितमिति बीजचतुष्टयं वदन्त्याचार्या इति वक्ष्यति~। अत्र हि यत्रेति बीजद्वयमनूद्य मध्यमाहरणत्वभावितत्वविधानप्रतीतेर्मुख्यं द्वैविध्यमेव प्रतीयते~। किं चाविशेषस्वरूपैकवर्णसमीकरणसमाप्तावित्येकवर्णसमीकरणं बीजमित्यनुक्त्वैव~।\\

{\small अथाव्यक्तवर्गादिसमीकरणं तच्च मध्यमाहरणमित्याद्युक्त्वा तत्करणसूत्रमप्यव्यक्तवर्गादिपदा-वशेषमित्यादि पूर्वशेषतयैव प्रतिपाद्य मध्यमाहरणविशेषसमाप्तावित्येकवर्णसमीकरणं बीजमथा-नेकवर्णसमीकरणं बीजमिति वक्ष्यन्ति~। युक्तं चैतदिति प्रतिभाति~। तत्रानेकवर्णानामेकवर्णपूर्व-कत्वादेकवर्णसमीकरणं प्रथमतः शालिनीत्रयेणाह\textendash }

\phantomsection \label{7.89}
\begin{quote}
{\large \textbf{{\color{purple}यावत्तावत्कल्प्यमव्यक्तराशेर्मानं तस्मिन्कुर्वतोद्दिष्टमेव~।\\
तुल्यौ पक्षौ साधनीयौ प्रयत्नात्त्यक्त्वा क्षिप्त्वा वापि सङ्गुण्य भक्त्वा~॥\\
एकाव्यक्तं शोधयेदन्यपक्षाद्रूपाण्यन्यस्येतरस्माच्च पक्षात्~।\\
शेषाव्यक्ते नोद्धरेद्रूपशेषं व्यक्तं मानं जायते व्यक्तराशेः~॥\\
अव्यक्तानां द्व्यादिकानामपीह यावत्तावद्द्व्यादिनिघ्नं हृतं वा~।\\
युक्तोनं वा कल्पयेदात्मबुद्ध्या मानं क्वापि व्यक्तमेवं विदित्वा~॥~८९~॥}}}
\end{quote}

पृच्छकेन पृष्टे सत्युदाहरणे योऽव्यक्तराशिस्तस्य मानं यावत्तावदेकं द्व्यादि वा प्रकल्प्य तस्मिन्नव्यक्तराशावुद्देशकालापवदेव सर्वं गुणनभजनत्रैराशिकश्रेढीक्षेत्रादि गणकेन कार्यम्~। तथा कुर्वता गणकेन द्वौ पक्षौ प्रयत्नेन समौ कार्यौ~। यद्यालापपक्षौ समौ न स्तस्तदैकतरे न्यूने पक्षे किञ्चित्प्रक्षिप्याधिकपक्षात्तावदेव विशोध्य वा न्यूनं पक्षं केनचित्सङ्गुण्य वाधिकं पक्षं तावतैव भक्त्वा वा समौ कार्यौ~। एवं गुणनक्षेपाभ्यां गुणनशुद्धिभ्यां वा भजनक्षेपाभ्यां भजनशुद्धिभ्यां वा पक्षयोः समता कार्या~। एवं वर्गादिकरणेनापि स्वबुद्ध्या पक्षौ समौ कार्यौ~। अत्रेदमप्यवगन्तव्यम्~। यद्युदाहरणे द्व्यादयोऽव्यक्ता राशयः स्युस्तदा यावत्तावत् कालकनीलकादीनि तेषां मानानि प्रकल्प्योक्तवत्पक्षौ पक्षा वा समाः कार्या इति~। अत्र प्रथमसूत्रं सकलबीजसाधारणम्~। अत्र प्रकृतसमीकरणे शोधनमाह\textendash \,एकाव्यक्तमिति~। कृतयोः समयोः पक्षयोरेकस्य पक्षस्याव्यक्तमन्यपक्षस्याव्यक्ताच्छोध्यम्~। अव्यक्तवर्गादिकं चेत्स्यात्तदा तदपि तस्मादेव पक्षाच्छोध्यम्~। एवं यदि करणी-
\end{sloppypar}

\newpage

\begin{sloppypar}
\noindent गुणितमव्यक्तवर्गादिकं वा स्यात्तदा तदपि शोध्यम्~। अथान्यस्य पक्षस्य रूपाणीतरपक्षस्य रूपेभ्यः शोध्यानि~। यदि करण्यः सन्ति तदा ता अप्युक्तप्रकारेण योगं करण्योरित्यादिना शोध्याः~। ततोऽव्यक्तशेषाङ्केन रूपशेषे भक्ते यल्लभ्यते तदेकस्याव्यक्तस्य मानं व्यक्तं जायते~। अथ यद्यव्यक्तशेषं यावत्करणीस्वरूपं स्यात्तदापि याकारस्य प्रयोजनाभावादपगमं कृत्वा तया करण्या रूपशेषे करणीशेषे वा वर्गेण वर्गं भजेदित्यादिना भक्ते यल्लभ्यते तन्मूलमेकस्याव्यक्तस्य मानं भवति~। यदि तु लब्धेर्मूलं न लभ्यते तदा करण्यात्मकं व्यक्तं मानं भवति~। तेन व्यक्तमानेन कल्पिताव्यक्तराशिरुत्थाप्यः~। यद्येकस्याव्यक्तस्य व्यक्तमानमिदं तदा कल्पिताव्यक्तस्य किमिति त्रैराशिकेन कल्पिताव्यक्तस्य यद्व्यक्तं मानं भवति तत्पूर्वाव्यक्तराशिं परिमृज्य स्थापनीयमित्यर्थः~। अथोत्थाप्यपक्षे यदि यावत् करण्यः स्युस्तदा वर्गेण वर्गं गुणयेदिति ता उत्थाप्याः~। तासां मूलमव्यक्तस्य मानं भवति~। एवं यावद्वर्गघनादिकमपि लब्धव्यक्तमानस्य वर्गघनादिभिरुत्थाप्यम्~। एवमनेकवर्णसमीकर-णेऽपि यादृशं यस्य मानं सिध्यति तादृशेन तस्योत्थापनं विधेयम्~। अथ यत्र द्व्यादयोऽव्य-क्तराशयो भवेयुस्तत्र यद्यप्यनेकवर्णसमीकरणेनोदाहरणसिद्धिरस्ति तथापि बुद्धिवैचि-त्र्यार्थमत्राप्याह\textendash \,अव्यक्तानां द्व्यादिकानामपीति~। इहैकवर्णसमीकरणे यद्यदाहरणे द्व्याद-योऽव्यक्ता राशयः स्युस्तदाप्येकस्याव्यक्तस्यैकं यावत्तावत्प्रकल्प्यान्येषां द्व्यादिभिरिष्टैर्गु-णितं भक्तं वेष्टै रूपैरूनं युतं वा यावत्तावदेव प्रकल्प्यम्~। अथवैकस्य यावत्तावद-न्येषां व्यक्तान्येव मानानीष्टानि कल्प्यानि~। एवं विदित्वेति~। यथा क्रिया निवर्हति तथा बुद्धिमता ज्ञात्वा शेषाणामव्यक्तानि व्यक्तानि वा मानानि कल्प्यानीत्यर्थः~। अत्रोपपत्तिः~। अज्ञातराशेर्मानं चतुर्धैव सम्भवति~। रूपसमूहस्तदवयवो वा रूपं रूपावयवो वेत्युक्तं प्राक्~। एतेष्वज्ञातराशेः किं मानमिति विशेषतोऽनवगमाद्राशेरव्यक्तमानमित्युच्यते~। अत एव विशेषतो ज्ञाने व्यक्तमित्येवोच्यते तत्रोदाहरणेऽव्यक्तराशेर्यथोक्तालापे कृते यदि केनापि प्रकारेणोद्देशकालापानुरोधेन पक्षद्वयस्य समता भवति तदाव्यक्तस्य व्यक्तं मानं सुबोधम्~। तथा हि~। यद्येकस्मिन्पक्षे रूपाण्येवान्यस्मिन्पक्षे त्वव्यक्तमेव तदोभयोस्तुल्य-त्वादव्यक्तसङ्ख्यायास्तानि रूपाणि व्यक्तमेव मानं सिद्धम्~। अतस्त्रैराशिकेनेष्टराशिसिद्धिः~। यथा यद्येतावतां यावत्तावतामेतावन्ति रूपाणि तदा कल्पितयावत्तावतः किमिति~। अथ यदि पक्षद्वयेऽपि किञ्चित्खण्डमव्यक्तं किञ्चित्तु व्यक्तं तदापि तथा यतितव्यं यथैकस्मिन्पक्षे व्यक्तराशिरेवान्यस्मिंस्तु रूपाण्येव स्युरिति~। तत्र युक्तिः~। समयोः समक्षेपे समशुद्धौ वा समेन गुणने भजने वा न समत्वहानिरस्तीति स्फुटम्~। तत्र यस्मिन्नेकतरे पक्षे यादृशोऽव्यक्तराशिरस्ति तादृशस्याव्यक्तराशेस्तस्मात्पक्षाच्छोधनेन तस्मिन्पक्षे रूपाण्येव स्युः~। परं समत्वार्थमितरपक्षादपि तादृशोऽव्यक्तराशिः शोध्यो भवति~। एतदेवोक्तम् \hyperref[7.89]{'एकाव्यक्तं शोधयेदन्यपक्षात्'} इति~। अथान्यस्मिन्पक्षे यादृशो रूपराशिरस्ति तादृशस्य शोधने तस्मि-
\end{sloppypar}

\newpage

\begin{sloppypar}
\noindent न्पक्षेऽव्यक्तराशिरेव स्यात्~। परं साम्यार्थं तादृशो रूपराशिर्द्वितीयपक्षरूपराशेः शोध्यो भवति~। एतदेवोक्तं \hyperref[7.89]{'रूपाण्यन्यस्येतरस्माच्च पक्षात्'} इति~। एवं कृते जात एकस्मिन् पक्षेऽव्यक्तराशिरेव परपक्षे रूपराशिरेव~।\\

अथ त्रैराशिकम्~। यद्यनेनाव्यक्तराशिनासौ रूपराशिस्तदा कल्पिताव्यक्तराशिना किम् इति~। शेषाव्यक्तराशिना रूपराशिर्भाज्यः कल्पिताव्यक्तेन गुण्यः~। तत्र 'शेषाव्यक्तेनोद्धरेत् रूपशेषम्' इति तूक्तमेव~। कल्पिताव्यक्तगुणनं तूत्थापनेऽन्तर्भूतम्~। यदि वा शेषाव्यक्त-राशिना यदि रूपशेषराशिर्लभ्यते तदैकेनाव्यक्तेन किमिति~। अत्र गुणकस्यैकत्वाच्छेषा-व्यक्तेनोद्धरेद्रूपशेषमित्येवोक्तम्~। एवमेकस्याव्यक्तस्य व्यक्तमाने सिद्धे कल्पिताव्यक्तराशेः अप्यनुपातेन स्यादेव~। अत्र प्रमाणस्यैकत्वात्कल्पिताव्यक्तराशिना व्यक्तमानस्य गुणन-मात्रं भवति~। इदमेवोत्थापनम्~। तस्माद्येन केनाप्युपायेन समपक्षयोः साम्याविरोधेन तथा यतितव्यं यथैकस्मिन्पक्षे रूपाण्येवान्यस्मिन्पक्षेऽव्यक्तमेव स्यात्~। अन्यथाव्यक्तस्य व्यक्तत्वेन ज्ञानम् असुलभम्~। \hyperref[7.89]{'एकाव्यक्तं शोधयेदन्यपक्षात्'} इत्यादिना तूक्तयुक्त्या तथा सिध्यतीत्युपपन्नम् \hyperref[7.89]{'एकाव्यक्तं शोधयेदन्यपक्षात्'} इत्यादि~। \hyperref[7.89]{'अव्यक्तानां द्व्यादिकानामपि'} इत्यत्रोपपत्तिः स्फुटैव~। यतो राशिवैलक्षण्यार्थं कालकादयः कल्प्यन्ते~। तत्रैकस्मिन्वर्णेऽपि सङ्ख्याभेदाद्वा रूपयोगवियोगवशाद्वा सम्भवतीति~। एवं द्व्यादिष्वज्ञातराशिष्वेकं विहाया-न्येषां मानानि व्यक्तान्येव तुल्यान्यतुल्यानि वा स्वेच्छया यदि कल्प्यन्ते तर्हि तदनुरोधेन जायमानादव्यक्तमानादुदाहरणसिद्धिर्भवेदेव~। तस्मात् पक्षयोः समत्वेन पूर्वयुक्त्या यथा राशिः सिध्यति तथा व्यक्ता अव्यक्ता वा राशयः कल्प्याः~॥ ८९~॥\\

{\small तत्रोद्देशकालापमात्रेण पक्षद्वयसाम्यसिद्धौ तावदुदाहरणमुपजातिकयाह\textendash }

\phantomsection \label{7.90}
\begin{quote}
{\large \textbf{{\color{purple}एकस्य रूपत्रिशती षडश्वा अश्वा दशान्यस्य तु तुल्यमौल्याः~।\\
ऋणं तथा रूपशतं च यस्य तौ तुल्यवित्तौ च किमश्वमौल्यम्~॥~९०~॥}}}
\end{quote}

स्पष्टोऽर्थः~। अत्राश्वमौल्यमज्ञातम्~। तस्य मानं कल्पितं या १~। अथ यद्येकस्याश्वस्येदं मौल्यं तदा षण्णां किमिति त्रैराशिकेन लब्धं षण्णामश्वानां मूल्यं या ६~। अत्र रूपशतत्रये ३०० क्षिप्ते जातमाद्यस्य सर्वधनं या ६ रू ३००~। एवं द्वितीयस्य दशानामश्वानां मूल्यं या १०~। अस्माद्रूपशतेऽपनीते जातं द्वितीयस्य सर्वधनं या १० रू १०ं०~। एतौ तुल्यवित्ताविति पक्षौ स्वत एव समौ जातौ \;{\small $\begin{matrix}
\mbox{{या ~~६ ~रू ~३००}}\\
\vspace{-1.5mm}
\mbox{{या ~१० ~रू ~१०ं०}}
\vspace{1mm}
\end{matrix}$}~। यदेव त्रिशतीत्युक्तस्य यावत्षट्कस्य मानं तदेव शतोनस्य यावद्दशकस्य मानमित्यर्थः~। अथानयोः
\end{sloppypar}

\newpage

\begin{sloppypar}
\noindent पक्षयोर्यदि यावत् षट्कं शोध्यते तदापि समयोः समक्षेपे समशुद्धौ वा समतैव स्यादिति यावत् षट्कशोधनेऽपि जातौ समौ \;{\small $\begin{matrix}
\mbox{{रू ~३०० ~~~~~~~~~}}\\
\vspace{-1.5mm}
\mbox{{या ~४ ~रू ~१०ं०}}
\vspace{1mm}
\end{matrix}$}~। यदेव शतत्रयं तदेव शतोनं यावत् तावच्चतुष्टयमित्यर्थः~। अथ यदि पक्षयोः शतं प्रक्षिप्यते तदाप्युक्तवत् समतैव स्यादिति शतप्रक्षेपे जातौ पक्षौ \;{\small $\begin{matrix}
\mbox{{रू ~४००}}\\
\vspace{-1.5mm}
\mbox{{या ~~४~~}}
\vspace{1mm}
\end{matrix}$}~। यदेव शतचतुष्टयं तदेव यावत् तावच्चतुष्टयमित्येकस्य यावत् तावतः शतं सङ्ख्येत्यवगतम् १००~। तस्मादस्मिन्नुदाहरणे यद्यावत्तावच्छतरूपसमूहा-त्मकमिति सिद्धम्~॥~९०~॥\\

{\small अथ त्यक्त्वा क्षिप्त्वेत्यादिना सङ्गुण्य भक्त्वेत्यादिना च यथा पक्षसाम्यं भवति तथोदाहरण-द्वयमुपजातिकयाह\textendash }

\phantomsection \label{7.91}
\begin{quote}
{\large \textbf{{\color{purple}यदाद्यवित्तस्य दलं द्वियुक्तं तत्तुल्यवित्तो यदि वा द्वितीयः~।\\
आद्यो धनेन त्रिगुणोऽन्यतो वा पृथक्पृथक् मे वद वाजिमौल्यम्~॥~९१~॥}}}
\end{quote}

अत्राप्येकस्य षडश्वा रूपशतत्रयं चास्ति परस्य दशाश्वा रूपशतमृणं चास्ति~। परम् अनयोर्वित्तं समं नास्ति~। किं तु प्रथमस्य वित्तार्धं द्वियुक्तं यावद्भवति तावद्द्वितीयस्य सर्व-धनमस्तीत्यश्वमौल्ये नान्यथा भाव्यम्~। अत्र पूर्ववदश्वमौल्यं यावत्तावत्प्रकल्प्य जाते द्वयोः सर्वधने \;{\small $\begin{matrix}
\mbox{{या ~~६ ~रू ~३००}}\\
\vspace{-1.5mm}
\mbox{{या ~१० ~रू ~१०ं०}}
\vspace{1mm}
\end{matrix}$}~। अत्र प्रथमस्य धनार्धं द्वियुतं सद्द्वितीयस्य सर्वधनसममिति जातौ पक्षौ समौ \;{\small $\begin{matrix}
\mbox{{या ~~३ ~रू ~१५२}}\\
\vspace{-1.5mm}
\mbox{{या ~१० ~रू ~१०ं०}}
\vspace{1mm}
\end{matrix}$}~। यद्वा विलोमविधिना द्वितीयधनं द्विहीनं द्विगुणं प्रथमवित्तेन समं स्यादिति जातौ \;{\small $\begin{matrix}
\mbox{{या ~~६ ~रू ~३००}}\\
\vspace{-1.5mm}
\mbox{{या ~२० ~रू ~२०ं४}}
\vspace{1mm}
\end{matrix}$}~। अथवा द्वितीयधनं द्विहीनं प्रथमधनार्धेन सममिति जातौ पक्षौ \;{\small $\begin{matrix}
\mbox{{या ~~३ ~रू ~१५०}}\\
\vspace{-1.5mm}
\mbox{{या ~१० ~रू ~१०२}}
\vspace{1mm}
\end{matrix}$}~। पक्षत्रयेऽपि पूर्वयुक्त्या लब्धं यावत्तावन्मानं ३६~। अस्मिन् उदाहरणे षट्त्रिंशद्रूपसमूहात्मकत्वं यावत्तावतः सिद्धम्~। एवं तृतीयोदाहरणे पञ्चविंशति-रूपसमूहात्मकत्वं यावत्तावतः~। एवं सर्वत्रालापानुरोधेन पक्षसाम्यं यथा तथा सम्पाद्योक्त-युक्त्या यावत्तावतो मानं व्यक्तं ज्ञेयम्~॥~९१~॥
\end{sloppypar}

\newpage

\begin{sloppypar}
{\small अथाव्यक्तानां द्व्यादिकानामपीत्यस्योदाहरणं शार्दूलविक्रीडितेनाह\textendash }

\phantomsection \label{7.92}
\begin{quote}
{\large \textbf{{\color{purple}माणिक्यामलनीलमौक्तिकमितिः पञ्चाष्ट सप्त क्रमात्\\
एकस्यान्यतरस्य सप्त नव षट् तद्रत्नसङ्ख्या सखे~।\\
रूपाणां नवतिर्द्विषष्टिरनयोस्तौ तुल्यवित्तौ तथा\\
बीजज्ञ प्रतिरत्नजानि सुमते मौल्यानि शीघ्रं वद~॥~९२~॥}}}
\end{quote}

उक्तयुक्त्या कल्पितानि माणिक्यादीनां मौल्यानि या ३ या २ या १~। उक्तवज्जातौ पक्षौ {\small $\begin{matrix}
\mbox{{या ~३८ ~रू ~९०}}\\
\vspace{-1.5mm}
\mbox{{या ~४५ ~रू ~६२}}
\vspace{1mm}
\end{matrix}$}~। उक्तवत् जातं व्यक्तं मानम् ४~। अनेनोत्थापने जातानि माणिक्यादीनां व्यक्तानि मानानि १२।८।४~। अथवा माणिक्यमानं या १~। नीलमुक्ताफलयोर्व्यक्ते एव कल्पिते ५।३~। उक्तवद्यावत्तावन्मानम् १३~। एवं जातानि मौल्यानि १३।५।३~। एवं कल्पनावशात् अनेकधा~॥~९२~॥\\

{\small अथ \hyperref[7.89]{'युक्तोनं वा कल्पयेदात्मबुद्ध्या'} इत्यस्योदाहरणं सिंहोद्धतयाह\textendash }

\phantomsection \label{7.93}
\begin{quote}
{\large \textbf{{\color{purple}एको ब्रवीति मम देहि शतं धनेन \\
त्वत्तो भवामि हि सखे द्विगुणस्ततोऽन्यः~।\\
ब्रूते दशार्पयसि चेन्मम षड्गुणोऽहं \\
त्वत्तस्तयोर्वद धने मम किं प्रमाणे~॥~९३~॥}}}
\end{quote}

स्पष्टोऽर्थः~। अत्राव्यक्तवैलक्षण्यमात्रं यदि धनं द्वयोः कल्प्यते तदालापद्वयं युगपत् कर्तुमशक्यम्~। एकैकालापमात्रेण यदि व्यक्तं मानं साध्यते तदैकैक एवालापः सम्भवेन्नोदा-हरणसिद्धिः~। अत आद्यान्ययोस्तथा धने कल्पनीये यथैक आलापः स्वत एव घटेत~। तथा कल्पिते \;{\small $\begin{matrix}
\mbox{{या ~२ ~रू ~१०ं०}}\\
\vspace{-1.5mm}
\mbox{{या ~१ ~रू ~१००}}
\vspace{1mm}
\end{matrix}$}~। अनयोः परस्य शते गृहीत आद्यो द्विगुणितः स्यादित्येक आलापो घटते~। अथाद्याद्दशापनीय दशभिः परधनं युतं षड्गुणं स्यादित्याद्यं षड्गुणीकृत्य परं वा षड्भिर्भक्त्वा न्यासः \;{\small $\begin{matrix}
\mbox{{या ~१२ ~रू ~६६ं०~। या ~२ ~रू ~११ं०}}\\
\vspace{-1.5mm}
\mbox{{या ~~१ ~रू ~११०~। या ~{\scriptsize $\begin{matrix}
\mbox{{१}}\\
\vspace{-1.5mm}
\mbox{{६}}
\vspace{1mm}
\end{matrix}$} ~~रू ~{\scriptsize $\begin{matrix}
\mbox{{११०}}\\
\vspace{-1.5mm}
\mbox{{६}}
\vspace{1mm}
\end{matrix}$}}}
\vspace{1mm}
\end{matrix}$}~। अथवा द्वितीयालापः सम्भवति~। तथा कल्पिते \;{\small $\begin{matrix}
\mbox{{या ~१ ~रू ~१०}}\\
\vspace{-1.5mm}
\mbox{{या ~६ ~रू ~१ं०}}
\vspace{1mm}
\end{matrix}$}~। अत्राद्याद्दशसु गृहीतेषु द्वितीयः स्वत एव षड्गुणो भवति~। अथ द्वितीयाच्छतमपनीय शतेन युतमाद्यं धनं द्विगुणं भवतीति परं द्विगुणीकृत्याद्यं दलीकृत्य वा न्यासः \;{\small $\begin{matrix}
\mbox{{या ~~१ ~रू ~११०~।}}\\
\vspace{-1.5mm}
\mbox{{या ~१२ ~रू ~२२ं०~।}}
\vspace{1mm}
\end{matrix}$}

\end{sloppypar}

\newpage

\begin{sloppypar}
\noindent {\small $\begin{matrix}
\vspace{0.5mm}
\mbox{{या ~{\scriptsize $\begin{matrix}
\mbox{{१}}\\
\vspace{-1.5mm}
\mbox{{२}}
\vspace{1mm}
\end{matrix}$} ~~रू ~{\scriptsize $\begin{matrix}
\mbox{{११०}}\\
\vspace{-1.5mm}
\mbox{{२}}
\vspace{1mm}
\end{matrix}$}}}\\
\vspace{-1.5mm}
\mbox{{या ~६ ~रू ~११ं०}}
\vspace{1mm}
\end{matrix}$}~। अत्र प्रथमपक्षद्वयाभ्यां यावत्तावन्मानम् ७०~। द्वितीयपक्षद्वयाभ्यां यावत् तावन्मानं ३०~। स्वस्वद्रव्ये उत्थाप्योभयत्रापि समे एव धने ४०।१७०~॥~९३~॥\\

{\small अथ शिष्याणां बुद्धिप्रसारार्थं विचित्राण्युदाहरणानि प्रदर्शयति~। तत्र शार्दूलविक्रीडितेनोदा-हरणमाह\textendash }

\phantomsection \label{7.94}
\begin{quote}
{\large \textbf{{\color{purple}माणिक्याष्टकमिन्द्रनीलदशकं मुक्ताफलानां शतं\\
यत्ते कर्णविभूषणे समधनं क्रीतं त्वदर्थे मया~।\\
तद्रत्नत्रयमौल्यसंयुतिमितिस्त्र्यूनं शतार्धं प्रिये\\
मौल्यं ब्रूहि पृथग्यदीह गणिते कल्पासि कल्याणिनि~॥~९४~॥}}}
\end{quote}

समधनमिति~। यदेव माणिक्याष्टकस्य मूल्यं तदेवेन्द्रनीलदशकस्य तदेव मुक्ताफल-शतस्येति~। कर्णविभूषणे~। कर्णविभूषणनिमित्तं तद्रत्नमौल्येत्यादि~। एकैकस्य माणिक्यादेः यन्मौल्यं तेषां युतिस्तु सप्तचत्वारिंशत्~। शेषं स्पष्टम्~। अत्र माणिक्यादीनां मौल्यकल्पने क्रिया न निर्वहतीति समधनमेव यावत्तावत्कल्पितं या १~। शेषं गणितमाकर एव स्फुटम्~। रत्नमौल्यानि जातानि २५।२०।२~। समधनम् २००~॥~९४~॥\\

{\small अथान्यदुदाहरणं पाटीस्थं प्रदर्शयति\textendash }

\phantomsection \label{7.95}
\begin{quote}
{\large \textbf{{\color{purple}पञ्चांशोऽलिकुलात्कदम्बमगमत्त्र्यंशः शिलीन्ध्रं तयोः\\
विश्लेषस्त्रिगुणो मृगाक्षि कुटजं दोलायमानोऽपरः~।\\
कान्ते केतकमालतीपरिमलप्राप्तैककालप्रियात्\\
दूताहूत इतस्ततो भ्रमति खे भृङ्गोऽलिसङ्ख्यां वद~॥~९५~॥}}}
\end{quote}

कदम्बस्य पुष्पं कदम्बम्~। {\color{violet}'अवयवे च प्राण्योषधिवृक्षेभ्यः'} इत्यण्~। {\color{violet}'पुष्पमूलेषु बहुलम्'} इति तस्य लुक्~। शिलीन्ध्र्याः पुष्पं शिलीन्ध्रम्~। {\color{violet}लुक्तद्धितलुकी}ति स्त्रीप्रत्ययलोपः~। शिली-न्ध्रीकं चोरसदृश ओषधिविशेषः~। कुटजो गिरिमल्लिका~। तस्य पुष्पं कुटजम्~। शेषस्य भ्रमरस्य दोलायमानत्वे हेतुगर्भं विशेषणं केतकेत्यादि~। केतक्याः पुष्पं केतकम्~। मालत्याः पुष्पं मालती~। मल्लिकायाः पुष्पं मल्लिकेतिवन्न स्त्रीप्रत्ययलोपः~। सुमना मालती जातिः इत्यभिधानान्मालती जातिः~। तयोः परिमलौ~। प्राप्त एककालो याभ्यां तौ प्राप्तैककालौ तौ च तौ प्रियादूतौ च प्राप्तैककालप्रियादूतौ केतकमालतीपरिमलौ प्राप्तैककालप्रियादूताविव केतकमालतीपरिमलप्राप्तैककालप्रियादूतौ~। ताभ्यामाहूतः स तथा~। यथा कश्चिन्नायको नायिकाद्वयदूताभ्यां युगपदाहूतः सन्
\end{sloppypar}

\newpage

\begin{sloppypar}
\noindent दोलायमानो भवति तथा परिमलद्वयग्रहणाद्भृङ्गोऽपि दोलायमान इत्यर्थः~। केतकीमाल-त्योर्भ्रमरोपभोग्यत्वेन तत्प्रियात्वम्~। तदुपगमनं पुष्पपरिमलग्रहणेनेति परिमलयोर्दूतत्वम्~। अत्रालिकुलमानं या १~। अतः कदम्बादिगतभ्रमरमानं या \;{\small $\begin{matrix}
\mbox{{१४}}\\
\vspace{-1.5mm}
\mbox{{१५}}
\vspace{1mm}
\end{matrix}$}~। एतद्दृष्टेन भ्रमरेण युतम् अलिकुलसममिति \;{\small $\begin{matrix}
\vspace{0.5mm}
\mbox{{या \;{\scriptsize $\begin{matrix}
\mbox{{१४}}\\
\vspace{-1.5mm}
\mbox{{१५}}
\vspace{1mm}
\end{matrix}$}\; रू ~१५}}\\
\vspace{-1.5mm}
\mbox{{या ~१ ~~रू ~~०}}
\vspace{1mm}
\end{matrix}$}\; वा एतद्या \;{\small $\begin{matrix}
\mbox{{१४}}\\
\vspace{-1.5mm}
\mbox{{१५}}
\vspace{1mm}
\end{matrix}$}\; राशेः या १ अपास्य रूपसममिति वा \;{\small $\begin{matrix}
\vspace{0.5mm}
\mbox{{या \;{\scriptsize $\begin{matrix}
\mbox{{१}}\\
\vspace{-1.5mm}
\mbox{{१५}}
\vspace{1mm}
\end{matrix}$}\; रू ~०}}\\
\vspace{-1.5mm}
\mbox{{या ~० ~~रू ~१}}
\vspace{1mm}
\end{matrix}$}\; रूपं राशेरपास्यैतत्सममिति वा \;{\small $\begin{matrix}
\vspace{0.5mm}
\mbox{{या \;{\scriptsize $\begin{matrix}
\mbox{{१४}}\\
\vspace{-1.5mm}
\mbox{{१५}}
\vspace{1mm}
\end{matrix}$}\; रू ~०}}\\
\vspace{-1.5mm}
\mbox{{या ~१ ~~रू ~१ं}}
\vspace{1mm}
\end{matrix}$}~। पक्षयोः साम्यं कृत्वा जातं तुल्यमेव यावत्तावन्मानम् १५~॥~९५~॥\\

{\small अथान्योक्तमप्युदाहरणं क्रियालाघवार्थं प्रदर्शयति\textendash }

\phantomsection \label{7.96}
\begin{quote}
{\large \textbf{{\color{purple}पञ्चकशतदत्तधनात्फलस्य वर्गं विशोध्य परिशिष्टम्~।\\
दत्तं दशकशतेन तुल्यः कालः फलं च तयोः~॥~९६~॥}}}
\end{quote}

गीतिरियम्~। प्रतिमासं पञ्च वृद्धिर्यस्येति पञ्चकमिति विज्ञानेश्वरेण व्यवहाराध्याये विवृतम्~। सञ्ज्ञायां कप्रत्ययविधानात्~। तादृशं यच्छतं तेन प्रमाणेन दत्तं यद्धनं तस्य किञ्चित् कालजं यत्फलं कलान्तरं तस्य वर्गमूलधनाद्विशोध्य यदवशिष्टं धनं तद्दशकशतेन प्रतिमासं दश वृद्धिर्यस्येति दशकम्~। तच्च तच्छतं च तेन प्रमाणेन दत्तं तयोः प्रथम-द्वितीययोर्मूलद्रव्ययोस्तुल्ये काले तुल्यमेव फलं भवति~। एवं सति ते के धने इति वदेति शेषः~। अत्र काले यावत्तावत्कल्पिते क्रिया न निर्वहतीत्यतः काल इष्टः कल्पनीयः~। अतोऽत्र कल्पिताः पञ्च मासाः ५~। मूलधनं यावत्तावत् या १~। अतः फलार्थं पञ्चराशिके न्यासः \;{\small $\begin{matrix}
\mbox{{~१~~~। ~५~~}}\\
\mbox{{१००~। या ~१}}\\
\vspace{-1.5mm}
\mbox{{५ \hspace{8mm}}}
\vspace{1mm}
\end{matrix}$}~। लब्धं फलं या \;{\small $\begin{matrix}
\mbox{{१}}\\
\vspace{-1.5mm}
\mbox{{४}}
\vspace{1mm}
\end{matrix}$}~। अस्य वर्गो याव \;{\small $\begin{matrix}
\mbox{{१}}\\
\vspace{-1.5mm}
\mbox{{१६}}
\vspace{1mm}
\end{matrix}$}\; मूलधनात् या १ समच्छेदेन शोधिते जातं द्वितीयमूलधनं याव \;{\small $\begin{matrix}
\mbox{{१ं}}\\
\vspace{-1.5mm}
\mbox{{१६}}
\vspace{1mm}
\end{matrix}$}\; या १६~। अत्रापि मासपञ्चकेन पञ्चराशिकेन न्यासः \;{\small $\begin{matrix}
\mbox{{१~। ~~५ \hspace{10mm} }}\\
\vspace{0.5mm}
\mbox{{१००~। याव \;{\scriptsize $\begin{matrix}
\mbox{{१ं}}\\
\vspace{-1.5mm}
\mbox{{१६}}
\vspace{1mm}
\end{matrix}$}\; या ~१६}}\\
\mbox{{१०~। \hspace{16mm} }}
\vspace{1mm}
\end{matrix}$}~। लब्धं फलं याव \;{\small $\begin{matrix}
\mbox{{१ं}}\\
\vspace{-1.5mm}
\mbox{{३२}}
\vspace{1mm}
\end{matrix}$}\; या १६

\end{sloppypar}

\newpage

\begin{sloppypar}
\noindent एतत्पूर्वफलस्यास्य या \;{\small $\begin{matrix}
\mbox{{१}}\\
\vspace{-1.5mm}
\mbox{{४}}
\vspace{1mm}
\end{matrix}$}\; सममिति न्यासः \;{\small $\begin{matrix}
\vspace{1mm}
\mbox{{याव ~० ~या \;{\scriptsize $\begin{matrix}
\mbox{{१}}\\
\vspace{-1.5mm}
\mbox{{४}}
\vspace{1mm}
\end{matrix}$}\;}}\\
\vspace{-1.5mm}
\mbox{{याव ~१ं ~या \;{\scriptsize $\begin{matrix}
\mbox{{१६}}\\
\vspace{-1.5mm}
\mbox{{३२}}
\vspace{1mm}
\end{matrix}$}\;}}
\vspace{1mm}
\end{matrix}$}~। पक्षौ यावत्तावतापवर्त्य सम-च्छेदीकृत्य च्छेदगमे जातौ \;{\small $\begin{matrix}
\mbox{{या ~० ~~रू ~८}}\\
\vspace{-1.5mm}
\mbox{{या ~१ं ~रू ~१६}}
\vspace{1mm}
\end{matrix}$}~। प्राग्वल्लब्धं यावत्तावन्मानम् ८~। एतन्मूलधनम्~। अनेनोत्थापितं द्वितीयं कलान्तरे च ४।२।२~। अथास्यानयनेऽव्यक्तकल्पनां विनैव क्रियालाघवार्थं निरूपयति "अथवा प्रथमप्रमाणफलेन द्वितीयप्रमाणफले भक्ते यल्लभ्यते तद्गुणितेन द्वितीयमूलधनेन तुल्यमेव प्रथममूलधनं स्यात्कथमन्यथा समे काले समं फलं स्यात्~। अतो द्वितीयस्यायं २ गुणः~। एकगुणं द्वितीयमेकोनगुणगुणितं फलवर्गे वर्तते~। अत एकोनगुणेनेष्टकल्पितकलान्तरस्य वर्गे भक्ते द्वितीयमूलधनं स्यात् तत्फलवर्गयुतं प्रथमं स्यात्" इत्यन्तेन~। अयमर्थः~। एकशतप्रमाणेन शतस्य मूलधनस्य यत्कलान्तरं भवति तदेव द्विकशतेन पञ्चाशत एव स्याच्चतुष्कशतेन पञ्चविंशतेरेव स्यात्पञ्चकशतेन विंशतेरेव स्याद्दशकशतेन दशानामेव स्यात्~। अतः प्रथमप्रमाणफलं येन गुणितं सद्द्वितीयप्रमाणफलं भवति तेनैव प्रथममूलधनं भक्तं सद्द्वितीयधनं स्यात्~। द्वितीयं वा गुणितं सत्प्रथमं स्यात्~। उक्तविलक्षणयोस्तु मूलधनयोः समे काले समं कलान्तरं कथमपि न स्यात्~। गुणकस्तु प्रथमप्रमाणफलेन द्वितीयप्रमाणफले भक्ते यल्लभ्यते स एव~। यतः प्रथमप्रमाणफलं गुण्यो द्वितीयप्रमाणफलं गुणनफलमिति~। तस्मात्प्रथमप्रमाणफलेन द्वितीयप्रमाणफले भक्ते यल्लभ्यते तेन गुणितं द्वितीयधनं प्रथमधनं स्यात्~। किन्तु द्वितीयधनं न ज्ञायते~। तदर्थमुपायः~। यद्यपि द्वितीयधनमिष्टं प्रकल्प्य तदेव गुणेन सङ्गुण्य प्रथममपि भवति~। अनयोः समे काले समं कलान्तरं च भवति तथापि फलवर्गतुल्यमन्तरं न स्यात्~। उक्तयुक्तेः~। फलवर्गपुरस्कारेणाप्रवृत्तेः~। उद्दिष्टं तु फलवर्गतुल्यमन्तरमतो नोद्दिष्टसिद्धिः~। अतोऽन्यथा यतितव्यम्~। इह किल फलस्य वर्गे प्रथमधनाच्छोधिते यच्छेषं तद्द्वितीयधनं भवतीति व्यस्तविधिना द्वितीयधनं फलवर्गयुतं सत्प्रथमधनं स्यात्~। तथा च प्रथमधनं ज्ञातुं द्वितीयधनं फलवर्गेण योज्यमथवा गुणेन गुणनीयम्~। गुणनं च खण्डाभ्यामपि सम्भवति~। तत्र यदि रूपमेकं खण्डं कल्प्यते तर्ह्येकोनगुणोऽपरं खण्डं स्यात्~। अत्र प्रथमखण्डेन रूपेण द्वितीयधनस्य गुणने द्वितीयधनमेकगुणं स्यात्~। अपरखण्डेन गुणन एकोनगुणगुणितं द्वितीयधनं स्यात्~। अनयोर्योगे सम्पूर्णगुणगुणितं स्यादित्येकगुणं द्वितीयमेकोनगुणगुणितेन द्वितीयेन योज्यम्~। तदेवं द्वितीयधनमेकोनगुणगुणितद्वितीयधनेन वा फलवर्गेण वा युक्तं सत्प्रथमधनं भवतीति य एव फलवर्गस्तदेवैकोनगुणगुणितं द्वितीयधनम्~। अत एवोक्तमाचार्येण\textendash \,'एकगुणं द्वितीयमेकोनगुणगुणितं फलवर्गे वर्तत' इति~। अतः
\end{sloppypar}

\newpage

\begin{sloppypar}
\noindent फलवर्ग एकोनगुणेन भक्ते यल्लभ्यते तदेव द्वितीयधनं स्यात्~। यद्यपि फलवर्गो न ज्ञातोऽस्ति तथापीष्टकल्पनेन तत्सिद्धेः सुखेनोदाहरणसिद्धिः~। तदेवं सिद्धम्~। कलान्तरमिष्टं प्रकल्प्य तस्य वर्गे एकोनगुणेन भक्ते यल्लभ्यते तद्द्वितीयधनम्~। इदं कलान्तरवर्गेण युक्तं सत्प्रथमधनं भवति~। मूलकलान्तराभ्यां पञ्चराशिकेन कालोऽपि सिध्यतीति यावत्तावत्कल्पनां विनैवास्ति क्रियालाघवमिति~। अथ प्रकृतोदाहरणे प्रथमप्रमाणफलेनानेन ५ द्वितीयप्रमाणफलेऽस्मिन् १० भक्ते द्वयं लभ्यत इति द्वितीयस्यायं २ गुणः~। अत्र कल्पितः फलवर्गः ४ अयमेकोनेन गुणेन १ भक्तो जातं द्वितीयधनं ४ इदं गुणेन २ गुणितमथवा फलवर्गेण ४ युतं जातं प्रथममूलधनं ८ फलं च २~। यदि शतस्य पञ्च फलं तदाष्टानां किमिति लब्धमेकमासेऽष्टानां फलम् \;{\small $\begin{matrix}
\mbox{{२}}\\
\vspace{-1.5mm}
\mbox{{५}}
\vspace{1mm}
\end{matrix}$}~। यद्यनेनैको मासस्तदा द्विकेन किमिति लब्धं मासाः ५~। एवं द्वितीयमूलधनादप्येत एव मासाः ५~॥~९६~॥\\

{\small अथ स्वप्रदर्शितक्रियालाघवस्य व्याप्तिं प्रदर्शयितुमुदाहरणान्तरमाह\textendash }

\phantomsection \label{7.97}
\begin{quote}
{\large \textbf{{\color{purple}एकशतदत्तधनात्फलस्य वर्गं विशोध्य परिशिष्टम्~।\\
पञ्चकशतेन दत्तं तुल्यः कालः फलं च तयोः~॥~९७~॥}}}
\end{quote}

गीतिरियम्~। अत्रोक्तवद्द्वितीयस्य गुणः ५ एकोनगुणेन ४ इष्टफलस्यास्य ४ वर्गे १६ भक्ते जातं द्वितीयधनम् ४~। इदं गुणेन ५ गुणितं फलवर्ग\textendash \,१६\textendash \,युतं वा जातं प्रथमम् २०~। उभाभ्यामपि मूलफलाभ्यां पञ्चराशिकेन त्रैराशिकद्वयेन वा जातः कालः २०~॥~९७~॥\\

{\small अथ शार्दूलविक्रीडितेनोदाहरणमाह\textendash }

\phantomsection \label{7.98}
\begin{quote}
{\large \textbf{{\color{purple}माणिक्याष्टकमिन्द्रनीलदशकं मुक्ताफलानां शतं\\
सद्वज्राणि च पञ्च रत्नवणिजां येषां चतुर्णां धनम्~।\\
सङ्गस्नेहवशेन ते निजधनाद्दत्त्वैकमेकं मिथो\\
जातास्तुल्यधनाः पृथग्वद सखे तद्रत्नमौल्यानि मे~॥~९८~॥}}}
\end{quote}

स्पष्टोऽर्थः~। अत्र माणिक्यादिरत्नानाम् अङ्कपुरस्कारेण यथोक्तालापं कृत्वा समानां समशुद्धौ समतैवेत्येकैकरत्नं प्रत्येकम् अपनीय च माणिक्यादिमौल्यानयनम् आकर एव स्फुटम्~॥~९८~॥\\

{\small अथार्ययोदाहरणमाह\textendash }

\phantomsection \label{7.99}
\begin{quote}
{\large \textbf{{\color{purple}पञ्चकशतेन दत्तं मूलं सकलान्तरं गते वर्षे~।\\
द्विगुणं षोडशहीनं लब्धं किं मूलमाचक्ष्व~॥~९९~॥}}}
\end{quote}

\end{sloppypar}

\newpage

\begin{sloppypar}
स्पष्टोऽर्थः~। अस्य गणितमाकर एव स्फुटम्~॥~९९~॥\\

{\small अथ वसन्ततिलकयोदाहरणमाह\textendash }

\phantomsection \label{7.100}
\begin{quote}
{\large \textbf{{\color{purple}यत्पञ्चकद्विकचतुष्कशतेन दत्तं \\
खण्डैस्त्रिभिर्नवतियुक् त्रिशती धनं तत्~।\\
मासेषु सप्तदशपञ्चसु तुल्यमाप्तं \\
खण्डत्रयेऽपि सफलं वद खण्डसङ्ख्याम्~॥~१००~॥}}}
\end{quote}

यन्नवतियुक्त्रिशतीरूपं धनं ३९० त्रिभिः खण्डैः पञ्चकद्विकचतुष्कशतेन दत्तं तत्सप्त-दशपञ्चसु मासेषु क्रमेण खण्डत्रयेऽपि सफलं तुल्यं प्राप्तं चेत्खण्डसङ्ख्यां वद~। एतदुक्तं भवति~। मूलधनं नवतियुक्शतत्रयम् अस्ति ३९०~। अस्य त्रीणि खण्डानि कृत्वैकं खण्डं पञ्चकशतप्रमाणेन दत्तं द्वितीयं द्विकशतेन दत्तं तृतीयं चतुष्कशतेन दत्तम्~। तत्र प्रथमं खण्डं माससप्तके गते सकलान्तरं यावद्भवति तावदेव द्वितीयं सकलान्तरं मासदशके गते भवति~। तृतीयमपि मासपञ्चके गते सकलान्तरं तावदेव भवति~। यद्येवं तर्हि कानि खण्डानि सम्भवन्ति तद्वद~। अत्र समधनस्य सफलखण्डस्य प्रमाणं यावत्तावत्प्रकल्प्यम्~। या १~। अतोऽनुपातेन पृथक्पृथङ्मूलधनानि या \;{\small $\begin{matrix}
\mbox{{२०}}\\
\vspace{-1.5mm}
\mbox{{२७}}
\vspace{1mm}
\end{matrix}$}\, या \;{\small $\begin{matrix}
\mbox{{५}}\\
\vspace{-1.5mm}
\mbox{{६}}
\vspace{1mm}
\end{matrix}$}\, या \;{\small $\begin{matrix}
\mbox{{५}}\\
\vspace{-1.5mm}
\mbox{{६}}
\vspace{1mm}
\end{matrix}$}\, आनीय तेषामैक्यं या \;{\small $\begin{matrix}
\mbox{{६५}}\\
\vspace{-1.5mm}
\mbox{{२७}}
\vspace{1mm}
\end{matrix}$}~। सर्वधनस्यास्य रू ३९० समं कृत्वाप्तयावत्तावन्मानेन १६२ उत्थापितानि खण्डानि १२०।१३५।१३५~। शेषमाकरे स्फुटतरम्~॥~१००~॥\\

{\small अथोदाहरणं वंशस्थवृत्तेनाह\textendash }

\phantomsection \label{7.101}
\begin{quote}
{\large \textbf{{\color{purple}पुरप्रवेशे दशदो द्विसङ्गुणं विधाय शेषं दशभुक् च निर्गमे~।\\
ददौ दशैवं नगरत्रयेऽभवत्त्रिनिघ्नमाद्यं वद तत्कियद्धनम्~॥~१०१~॥}}}
\end{quote}

कश्चिद्वणिक् किञ्चिद्धनं गृहीत्वा व्यापारार्थं किञ्चित्पुरं प्रति गतवान्~। तत्र पुर-प्रवेशनिमित्तं शुल्कं दश दत्त्वा पुरं प्रविश्य शेषधनं व्यापारेण द्विगुणं विधाय तन्मध्ये दश भुक्त्वा निर्गमनिमित्तं पुनर्दश दत्तवान्~। अथ तच्छेषधनं गृहीत्वा पुरान्तरं गतवान्~। तत्रापि दश दत्त्वा द्विगुणीकृत्य दश भुक्त्वा दश दत्त्वा च ततस्तृतीयं नगरं गतवान्~। तत्रापि दश दत्त्वा द्विगुणीकृत्य दश भुक्त्वा दश दत्त्वा च स्वगृहं प्रत्यागतवान्~। एवं सति यत्प्रथमं धनं तत्त्रिगुणमभवत्~। तर्हि तत्प्रथमं धनं कियदिति वदेति प्रश्नार्थः~। कल्पितो राशिः या १~। अस्यालापवत्सर्वं कृत्वा पुरत्रयनिवृत्तौ जातं धनम्~। या ८ रू २८ं०~। एतदाद्यस्य त्रिगुणस्य या ३ समं कृत्वाप्तं यावत्तावन्मानम् ५६~॥~१०१~॥
\end{sloppypar}

\newpage

\begin{sloppypar}
{\small अथ शार्दूलविक्रीडितेनोदाहरणमाह\textendash }

\phantomsection \label{7.102}
\begin{quote}
{\large \textbf{{\color{purple}सार्धं तन्दुलमानकत्रयमहो द्रम्मेण मानाष्टकं\\
मुद्गानां च यदि त्रयोदशमिता एता वणिक्काकिणीः~।\\
आदायार्पय तन्दुलांशयुगुलं मुद्गैकभागान्वितं\\
क्षिप्रं क्षिप्रभुजो व्रजेम हि युतः सार्थोऽग्रतो यास्यति~॥~१०२~॥}}}
\end{quote}

स्पर्ष्टोऽर्थो व्याख्यातश्च लीलावतीविवृतौ~। अत्र तन्दुलमानमानं या २ मुद्गमानप्रमाणं च या १ प्रकल्प्य गणितमाकर एव स्पष्टम्~॥~१०२~॥\\

{\small अथानुष्टुभोदाहरणमाह\textendash }

\phantomsection \label{7.103}
\begin{quote}
{\large \textbf{{\color{purple}स्वार्धपञ्चांशनवमैर्युक्ताः के स्युः समास्त्रयः~।\\
अन्यांशद्वयहीना ये षष्टिशेषाश्च तान्वद~॥~१०३~॥}}}
\end{quote}

ये त्रयो राशयः स्वार्धपञ्चांशनवमैर्युक्ताः सन्तः समाः स्युः~। अथ चान्यांशद्वयहीनाः सन्तः षष्टिशेषाः स्युस्ते के तान्वद~। एतदुक्तं भवति~। अस्ति राशित्रयम्~। तत्राद्यः स्वार्धेन द्वितीयः स्वपञ्चमांशेन तृतीयः स्वनवमांशेन युक्तः सर्वेऽपि समा एव भवन्ति~। अथ चाद्यराशिर्द्वितीयस्य पञ्चमांशेन तृतीयस्य नवमांशेन च हीनः सन् षष्टिर्भवति~। द्वितीयराशिराद्यस्थार्धेन तृतीयस्य नवमांशेन च हीनः सन् षष्टिरेव भवति~। तृतीयराशिरपि प्रथमस्यार्धेन द्वितीयस्य पञ्चांशेन च हीनः सन् षष्टिरेव भवति~। तर्हि ते के राशय इति तान्वद~। अत्र समराशिमानं या १~। अतो विलोमविधिना ज्ञाता राशयः~। या \;{\small $\begin{matrix}
\vspace{-0.5mm}
\mbox{{२}}\\
\vspace{-1.5mm}
\mbox{{३}}
\vspace{1mm}
\end{matrix}$}\, या \;{\small $\begin{matrix}
\vspace{-0.5mm}
\mbox{{५}}\\
\vspace{-1.5mm}
\mbox{{६}}
\vspace{1mm}
\end{matrix}$}\, या \;{\small $\begin{matrix}
\vspace{-0.5mm}
\mbox{{९}}\\
\vspace{-1.5mm}
\mbox{{१०}}
\vspace{1mm}
\end{matrix}$}~। इहान्यभागद्वयोनाः सर्वेऽप्येवं या \;{\small $\begin{matrix}
\vspace{-0.5mm}
\mbox{{२}}\\
\vspace{-1.5mm}
\mbox{{५}}
\vspace{1mm}
\end{matrix}$}\, शेषाः स्युः~। एतत्षष्टिसमं कृत्वाप्तयावत् तावन्मानेनोत्थापिता जाता राशयः १००।१२५।१३५~॥~१०३~॥\\

{\small अथान्यदुदाहरणमनुष्टुभाह\textendash }

\phantomsection \label{7.104}
\begin{quote}
{\large \textbf{{\color{purple}त्रयोदश तथा पञ्च करण्यौ भुजयोर्मिती~।\\
भूरज्ञातात्र चत्वारः फलं भूमिं वदाशु मे~॥~१०४~॥}}}
\end{quote}

फलं क्षेत्रफलम्~। भूमिं वदेति प्रश्नादेव भूमेरज्ञाने सिद्धे भूरज्ञातेति पुनर्वचनमस्मिन् गणिते
\end{sloppypar}

\newpage

\begin{sloppypar}
\noindent भूमेर्यावत्तावत्त्वेनापि ज्ञानं नापेक्षितमिति सूचनार्थम्~। स्पष्टमन्यत्~। न्यासः \\
\vspace{4mm}

क्षेत्रफलं रू ४ 

\begin{center}
\vspace{-13mm}
\hspace{20mm} \includegraphics[scale=0.6]{Diagram_p2pg26.png}
\end{center}
\vspace{-3mm}

\noindent अत्र भूमेर्यावत्तावत्कल्पने क्रिया प्रसरति मध्यमाहरणं विना न निर्वहति च~। तथाहि\textendash \,भूमिः या १~। अथ 'त्रिभुजे भुजयोर्योगः' इत्यादिनाबाधे यथा~। भुजौ क १३~। क ५ं~। अनयोर्योगः क १३ क ५ भुजयोरन्तरेणानेन क १३ क ५ं~।\\

गुणनार्थं न्यस्तः ~~{\small $\begin{matrix}
\mbox{{क ~१३~। क ~१३ ~क ~५}}\\
\vspace{-1.5mm}
\mbox{{क ~~५ं~। क ~१३ ~क ~५}}
\vspace{1mm}
\end{matrix}$}\\

गुणने जातानि करणीखण्डानि क १६९~। क ६५~। क ६५ं~। क २५ं~। अत्र मध्ममकरण्योर्धनर्णयोस्तुल्यत्वान्नाशः~। आद्यान्त्यकरण्योर्मूले रू १३ रू ५ं अनयोर्योगे जातं गुणनफलं रू ८~। अयं भुवा हृतः \;{\small $\begin{matrix}
\mbox{{रू ~८}}\\
\vspace{-1.5mm}
\mbox{{या ~१}}
\vspace{1mm}
\end{matrix}$}\; लब्ध्या समच्छेदेन भूरूनयुता दलिता च जाते आबाधे \;{\small $\begin{matrix}
\mbox{{याव ~१ ~रू ~८ं}}\\
\vspace{-1.5mm}
\mbox{{या ~२ \hspace{7mm}}}
\vspace{1mm}
\end{matrix}$}~। \;{\small $\begin{matrix}
\mbox{{याव ~१ ~रू ~८}}\\
\vspace{-1.5mm}
\mbox{{या ~२ \hspace{7mm}}}
\vspace{1mm}
\end{matrix}$}~। लघोराबाधाया वर्गं \;{\small $\begin{matrix}
\mbox{{यावव ~१ ~याव ~१ं६ ~रू ~६४}}\\
\vspace{-1.5mm}
\mbox{{याव ~४ \hspace{20mm}}}
\vspace{1mm}
\end{matrix}$}\; लघुभुजस्य क ५ वर्गात् रू ५ समच्छेदेनापास्य \;{\small $\begin{matrix}
\mbox{{यावव ~१ं ~याव ~३६ ~रू ~६ं४}}\\
\vspace{-1.5mm}
\mbox{{याव ~४ \hspace{20mm}}}
\vspace{1mm}
\end{matrix}$}\; जातो लम्ब-वर्गः~। एवं द्वितीयाबाधावर्गं \;{\small $\begin{matrix}
\mbox{{यावव ~१ ~याव ~१६ ~रू ~६४}}\\
\vspace{-1.5mm}
\mbox{{याव ~४ \hspace{20mm}}}
\vspace{1mm}
\end{matrix}$}~। द्वितीयभुज\textendash \,क १३\textendash \,वर्गात् रू १३ समच्छेदेनापास्य वा जातो लम्बवर्गः स एव~। \;{\small $\begin{matrix}
\mbox{{यावव ~१ं ~याव ~३६ ~रू ~६ं४}}\\
\vspace{-1.5mm}
\mbox{{याव ~४ \hspace{20mm}}}
\vspace{1mm}
\end{matrix}$}~। \\

अथ प्रकारान्तरेण लम्बगुणं भूम्यर्धं क्षेत्रफलं भवतीति व्यस्तविधिना भूम्यर्धेन
\end{sloppypar}

\newpage

\begin{sloppypar}
\noindent या \;{\small $\begin{matrix}
\vspace{-0.5mm}
\mbox{{१}}\\
\vspace{-1.5mm}
\mbox{{२}}
\vspace{1mm}
\end{matrix}$}\, क्षेत्रफलं ४ भक्तं जातो लम्बः \;{\small $\begin{matrix}
\mbox{{रू ~८}}\\
\vspace{-1.5mm}
\mbox{{या ~१}}
\vspace{1mm}
\end{matrix}$}\, अस्य वर्गः \;{\small $\begin{matrix}
\mbox{{रू ~६४}}\\
\vspace{-1.5mm}
\mbox{{याव ~१}}
\vspace{1mm}
\end{matrix}$}~। लम्बवर्गयोर्न्यासः {\small $\begin{matrix}
\mbox{{यावव ~१ं ~याव ~३६ ~रू ~६ं४}}\\
\mbox{{याव ~४ \hspace{22mm}}}\\
\mbox{{यावव ~० ~याव ~० ~रू ~६४}}\\
\vspace{-1.5mm}
\mbox{{याव ~१ \hspace{22mm}}}
\vspace{1mm}
\end{matrix}$} \\
\vspace{2mm}

पक्षौ समच्छेदीकृत्य च्छेदगमे न्यासः \;{\small $\begin{matrix}
\mbox{{यावव ~१ं ~याव ~३६ ~रू ~६ं४}}\\
\vspace{-1.5mm}
\mbox{{यावव ~० ~याव ~० ~रू ~२५६}}
\vspace{1mm}
\end{matrix}$} \\
\vspace{2mm}

समशोधने जातं \;{\small $\begin{matrix}
\mbox{{रू ~३२ं० \hspace{11mm}}}\\
\vspace{-1.5mm}
\mbox{{यावव ~१ ~याव ~३ं६}}
\vspace{1mm}
\end{matrix}$}~।\\
\vspace{2mm}

अथाव्यक्तवर्गादि यदावशेषमित्यादिवक्ष्यमाणमध्यमाहरणविधिना पक्षयोरष्टादशवर्गं ३२४ प्रक्षिप्य गृहीते मूले~। \;{\small $\begin{matrix}
\mbox{{रू ~२ \hspace{10mm}}}\\
\vspace{-1.5mm}
\mbox{{याव ~१ ~रू ~१ं८}}
\vspace{1mm}
\end{matrix}$}\, अव्यक्तपक्षर्णगरूपतोऽल्पमित्यादिना जातं द्विविधं यावत्तावद्वर्गमानम् २०।१६~। अत्राद्यमनुपपन्नत्वान्न ग्राह्यम्~। अनुपपत्तावुपपत्तिं तु मध्यमाहरणविवरणे वक्ष्यामः~। यावत्तावद्वर्गमानस्य १६ पदं ४ जातं यावत्तावन्मानम्~। इयमेव भूः ४~। अथ पूर्वसिद्धलम्बवर्गं \;{\small $\begin{matrix}
\mbox{{यावव ~१ ~याव ~३६ ~रू ~६ं४}}\\
\vspace{-1mm}
\mbox{{\hspace{22mm} याव ~४}}
\vspace{1mm}
\end{matrix}$}\, भूम्यर्धवर्गेण याव \;{\small $\begin{matrix}
\mbox{{१}}\\
\vspace{-1mm}
\mbox{{४}}
\vspace{1mm}
\end{matrix}$}\, सङ्गुण्य जातः क्षेत्रफलवर्गः \;{\small $\begin{matrix}
\mbox{{यावव ~१ं ~याव ~३६ ~रू ~६ं}}\\
\vspace{-1mm}
\mbox{{\hspace{19mm} १६}}
\vspace{1mm}
\end{matrix}$}\, अयं क्षेत्रफलस्यास्य ४ वर्गेण सम इति समशोधनार्थं न्यासः \;{\small $\begin{matrix}
\mbox{{यावव ~१ं ~याव ~३६ ~रू ~६ं४}}\\
\mbox{{\hspace{19mm} १६}}\\
\vspace{-1mm}
\mbox{{यावव ~० ~याव ~~० ~रू ~१६}}
\vspace{1mm}
\end{matrix}$}\, पक्षौ समच्छेदीकृत्य च्छेदगमे प्राग्वल्लब्धं यावत्तावन्मानम् ४~। तदेवं भूमेर्यावत्तावत्कल्पने क्रिया प्रसरति~। अत आचार्येणाव्यक्तकल्पनानिरपेक्षमेव यथोदाहरणसिद्धिर्भवेत्तथा
\end{sloppypar}

\newpage

\begin{sloppypar}
\noindent स्वेच्छयैको भुजो क १३ भूमिः कल्पिता फले विशेषाभावात्~। दर्शनम्~।

\begin{center}
\includegraphics[scale=0.5]{Diagram_p2pg28.png}
\end{center}

\noindent लम्बगुणं भूम्यर्धं क्षेत्रफलं भवतीति क्षेत्रफलं भूम्यर्धभक्तं लम्बः स्यात्~। तत्र यद्यपि द्वाभ्यां भागेऽर्धं भवतीति भूमेरर्धार्थं द्वाभ्यां भाग उचितस्तथापि वर्गेण वर्गं भजेदित्युक्त-त्वात्प्रकृते वर्गरूपाया भूमेरर्धार्थं चतुर्भिरेव भाग उचितः~। एवं जातं भूम्यर्धं \;{\small $\begin{matrix}
\mbox{{क ~१३}}\\
\vspace{-1mm}
\mbox{{~~~~४}}
\vspace{1mm}
\end{matrix}$}~। उक्तवत्क्षेत्रफलमपि वर्गीकृतं क १६~। क्षेत्रफलेऽस्मिन् क १६ भूम्यर्धेनानेन \;{\small $\begin{matrix}
\mbox{{क ~१३}}\\
\vspace{-1mm}
\mbox{{~~~~४}}
\vspace{1mm}
\end{matrix}$}\, भक्ते जातो लम्बः \;{\small $\begin{matrix}
\mbox{{क ~६४}}\\
\vspace{-1mm}
\mbox{{~~~~१३}}
\vspace{1mm}
\end{matrix}$}~। अस्य कोटिरूपवर्गं \;{\small $\begin{matrix}
\mbox{{रू ~६४}}\\
\vspace{-1mm}
\mbox{{~~~~१३}}
\vspace{1mm}
\end{matrix}$}\, ज्ञातभुजस्य कर्णरूपस्य क ५ वर्गात् रू ५ अपास्य \;{\small $\begin{matrix}
\mbox{{रू ~१}}\\
\vspace{-1mm}
\mbox{{~~~१३}}
\vspace{1mm}
\end{matrix}$}\, मूलं \;{\small $\begin{matrix}
\mbox{{क ~१}}\\
\vspace{-1mm}
\mbox{{~~~१३}}
\vspace{1mm}
\end{matrix}$}\, जाता लघुराबाधा~। यथा करण्या वर्गे तत्तुल्यानि रूपाणि भवन्ति तथा रूपाणां मूले रूपतुल्या करणी भवितुमर्हति~। यतो यस्य राशेर्यो वर्गस्तस्य वर्गस्य स राशिर्मूलमिति~। अथाबाधां \;{\small $\begin{matrix}
\mbox{{क ~१}}\\
\vspace{-1mm}
\mbox{{~~~१३}}
\vspace{1mm}
\end{matrix}$}\, भूमेः क १३ अपास्य योगं करण्योरित्यादिना लघ्व्या हृतायास्त्वित्यादिना वा जातान्याबाधा \;{\small $\begin{matrix}
\mbox{{क ~१४४}}\\
\vspace{-1mm}
\mbox{{~~~~१३}}
\vspace{1mm}
\end{matrix}$}~। इयमाबाधा भुजः~। लम्बः कोटिः~। अज्ञातभुजः कर्णः~। अत्र भुजकोट्योर्ज्ञाने तत्कृत्योर्योगपदं कर्ण इति कर्णः सुलभः~। द्वितीया-बाधायाः \;{\small $\begin{matrix}
\mbox{{क ~१४४}}\\
\vspace{-1mm}
\mbox{{~~~~१३}}
\vspace{1mm}
\end{matrix}$}\, वर्गः \;{\small $\begin{matrix}
\mbox{{रू ~१४४}}\\
\vspace{-1mm}
\mbox{{~~~~१३}}
\vspace{1mm}
\end{matrix}$}\, लम्बस्य \;{\small $\begin{matrix}
\mbox{{क ~६४}}\\
\vspace{-1mm}
\mbox{{~~~~१३}}
\vspace{1mm}
\end{matrix}$}\, वर्गेण \;{\small $\begin{matrix}
\mbox{{रू ~६४}}\\
\vspace{-1mm}
\mbox{{~~~~१३}}
\vspace{1mm}
\end{matrix}$}\, युतः रू १६~। अस्य पदं रू ४ ज्ञातोऽज्ञातभुजः~। प्रष्ट्रा या भूमिः पृष्टा सैवाचार्येण भुजत्वेन कल्पिता~। तस्मादत्र यो भुजोऽवगतः रू ४ इय-
\end{sloppypar}

\newpage

\begin{sloppypar}
\noindent मेव सा भूः~। एवमन्यं भुजं क ५ भूमिं प्रकल्प्य न्यासः~।

\begin{center}
\includegraphics[scale=0.45]{Diagram_p2p31-1.png}
\end{center}

\noindent अत्रापि पूर्ववत्फलाल्लम्बः \;{\small $\begin{matrix}
\mbox{{क ~६४}}\\
\vspace{-1mm}
\mbox{{~~~~५}}
\vspace{1mm}
\end{matrix}$}~। लम्बवर्गं \;{\small $\begin{matrix}
\mbox{{रू ~६४}}\\
\vspace{-1mm}
\mbox{{~~~~५}}
\vspace{1mm}
\end{matrix}$}\, भुजवर्गात् रू १३ अपास्य \;{\small $\begin{matrix}
\mbox{{रू ~१}}\\
\vspace{-1mm}
\mbox{{~~~५}}
\vspace{1mm}
\end{matrix}$}\, मूलं \;{\small $\begin{matrix}
\mbox{{क ~१}}\\
\vspace{-1mm}
\mbox{{~~~५}}
\vspace{1mm}
\end{matrix}$}\, जाताबाधा~। इमां योगं करण्योरित्यादिना भूमेः क ५ अपास्य जातान्या \;{\small $\begin{matrix}
\mbox{{क ~१६}}\\
\vspace{-1mm}
\mbox{{~~~~५}}
\vspace{1mm}
\end{matrix}$}~। अस्या वर्गात् \;{\small $\begin{matrix}
\mbox{{रू ~१६}}\\
\vspace{-1mm}
\mbox{{~~~~५}}
\vspace{1mm}
\end{matrix}$}\, लम्बवर्गेण \;{\small $\begin{matrix}
\mbox{{रू ~६४}}\\
\vspace{-1mm}
\mbox{{~~~~५}}
\vspace{1mm}
\end{matrix}$}\, युतात् १६ मूलं ज्ञातोऽज्ञातभुजः ४~। एवमन्यथापि सुधीभिरुह्यम्~॥~१०४~॥\\

{\small अथान्यदुदाहरणमार्ययाह\textendash }

\phantomsection \label{7.105}
\begin{quote}
{\large \textbf{{\color{purple}दशपञ्चकरण्यन्तरमेको बाहुः परश्च षट् करणी~।\\
भूरष्टादश करणी रूपोना लम्बमाचक्ष्व~॥~१०५~॥}}}
\end{quote}

स्पष्टोऽर्थः~। अत्राबाधाज्ञाने लम्बज्ञानमिति लघुराबाधा कल्पिता या १~। एतदूना भूरन्या-बाधेति तथा न्यासः

\begin{center}
\includegraphics[scale=0.5]{Diagram_p2pg31-2.png}
\end{center}
\end{sloppypar}

\newpage

\begin{sloppypar}
\noindent अत्राबाधे भुजौ~। भुजौ तु कर्णौ~। कोटिस्तूभयत्र लम्ब एवेति स्वाबाधावर्गं स्वभुजवर्गादपास्य लम्बवर्गौ भवतः~। तत्र लघोराबाधाया वर्गः याव १~। लघुभुजस्यास्य क ५ं क १० स्थाप्योऽन्त्यवर्गश्चतुर्गुणान्त्यनिघ्ना इत्यादिना क २५ क २०ं० क १०० आद्यान्त्यकरण्योर्योगे कृते क २२५ मूले च गृहीते रू १५ जातो लघुभुजवर्गः रू १५ क २०ं० अयमाबाधावर्गोनः सञ्जातो लम्बवर्गः याव १ं रू १५ क २०ं०~। एवं द्वितीयाबाधायाः या १ं रू १ं क १८~। अत्र {\color{violet}'स्थाप्योऽन्त्यवर्गः'} इत्यादिना यथासम्भवं द्विगुणान्त्यनिघ्नाश्चतुर्गुणान्त्यनिघ्नाश्चेति कृते जातो वर्गः याव १ या २ याक ७२ं रू १ क ७२ं क ३२४~। अन्त्यकरण्या मूलं रू १८ रुपेण संयोज्य परखण्डानां भिन्नजातित्वात् पृथक्स्थितौ च जातः याव १ या २ याक ७२ं रू १९ क ७२ं~। एवमाबाधावर्गं स्वभुजस्यास्य क ६ वर्गादस्मात् रू ६ विशोध्य वा जातो लम्बवर्गः याव १ं या २ं याक ७२ रू १३ं क ७२ लम्बवर्गौ समाविति समशोधनार्थं न्यासः
\vspace{-1mm}

\begin{center}
याव ~१ं ~या ~० ~याक ~~० ~रू ~१५ ~क ~२०ं०\\
याव ~१ं ~या ~२ं ~याक ~७२ ~रू ~१ं३ ~क ~~७२
\end{center}
\vspace{-1mm}

\noindent अत्राद्यपक्षादव्यक्तमात्रे शोधित इतरस्माच्च व्यक्तमात्रे शोधिते योगं करण्योरित्यादिना करण्योर्योगे च कृते जाते शेषे \;{\small $\begin{matrix}
\mbox{{या ~२ ~याक ~७ं२}}\\
\vspace{-1mm}
\mbox{{रू ~२ं८ ~क ~५१२}}
\vspace{1mm}
\end{matrix}$}~। \\

\noindent अथाव्यक्तशेषेण व्यक्तशेषस्य भागार्थं न्यासः~। \;{\small $\begin{matrix}
\mbox{{रू ~२ं८ ~क ~५१२}}\\
\vspace{-1mm}
\mbox{{या ~२ ~याक ~७ं२}}
\vspace{1mm}
\end{matrix}$}\, अत्राव्यक्तशेषेण व्यक्तशेषं कथं भाज्यमित्याह 'अत्र याकारस्य प्रयोजनाभावात्तदपगमे कृते समभाज्यभाजकौ' इति \;{\small $\begin{matrix}
\mbox{{रू ~२ं८ ~क ~५१२}}\\
\vspace{-1mm}
\mbox{{रू ~~२ ~क ~~७२ं}}
\vspace{1mm}
\end{matrix}$}~। वस्तुतस्त्वव्यक्तशेषतुल्येनाव्यक्तेन यदि व्यक्तशेषं तुल्यं व्यक्तं लभ्यते तदैकेनाव्यक्तेन किमिति त्रैराशिकेन \\
\vspace{-2mm}

या २ याक ७ं२~। रू २ं८ क ५१२~। या १ \\
\vspace{-2mm}

\noindent अव्यक्तस्य व्यक्तं मानं भवतीतीच्छाप्रमाणयोर्यावत्तावतापवर्ते भवतीष्टो हरः रू २ क ७ं२~। अन्यथान्यत्राप्यव्यक्तशेषेण रूपशेषे भक्ते रूपात्मकं फलं कथं स्यात्~। आचार्यैस्त्वन्यत्र याकारस्यापगमेऽप्यज्ञानां गणितसिद्धिर्भवतीति तत्र याकारापगमो नोक्तः~।
\end{sloppypar}

\newpage

\begin{sloppypar}
\noindent प्रकृते तु याकारानपगमे~। \hyperref[4.38]{'धनर्णताव्यत्ययमीप्सितायाश्छेदे करण्याः'} इत्यादिना भाज्य-भाजकयोर्गुणने भूयाननर्थः स्यादिति याकारापगम उक्तः~। अथ द्विसप्ततिमिताया भाजक-करण्या धनत्वं प्रकल्प्य तादृक्छिदा क ४ क ७२ भाज्यभाजकयोर्गुणनार्थे न्यासः
\vspace{-1mm}

\begin{center}
क ~~४~। क ~७ं८४ ~क ~५१२~। ~~~~क ~~४~। \;क ~४ ~क ~७ं२\\
क ~७२~। क ~७ं८४ ~क ~५१२~। ~~~~क ~७२~। क ~४ ~क ~७ं२
\end{center}
\vspace{-1mm}

\noindent भाज्ये गुणिते जातानि खण्डानि
\vspace{-1mm}

\begin{center}
क ~३१ं३६ ~क ~२०४८ ~क ~५६ं४४८ ~क ~३६८६४~।
\end{center}
\vspace{-1mm}

\noindent अत्राद्यान्त्ययोर्द्वितीयतृतीययोश्च करण्योर्लघ्व्या हृतायास्तु पदमित्यादिनान्तरे कृते जाते भाज्यकरण्यौ क १८४९६ क २६ं९९२~।\\

एवं भाजके करणीखण्डानि क १६ क २८ं८ क २८८ क ५१ं८४~।\\

अत्र द्वितीयतृतीयकरण्योरन्तरे नाशः~। आद्यान्त्ययोरन्तरे कृते जाता भाजककरणी क ४६ं२४~। अनया भाज्ये हृते लब्धं यावत्तावन्मानं क ४ं क ८~। प्रथमकरण्या मूले गृहीते जातं रू २ं क ८~। इमयेव लघुराबाधा~। एतदूना भूः रू १ं क १८ योगं करण्योरित्यन्तरे कृते जाता द्वितीयाबाधा रू १ क २~।\\

अथ प्रथमलम्बवर्गस्योत्थापनार्थं न्यासः याव १ं रू १५ क २०ं०~। अत्राद्यमेव खण्डम् अव्यक्तं स च यावद्वर्गोऽस्ति~। अतो यावत्तावन्मानस्यास्य क ४ं क ८ वर्गो रू १२ क १२ं८ जातं यावत्तावद्वर्गमानम्~। यावद्वर्गस्य ऋणगतत्वादिदं रू १२ क १२ं८ उत्तरखण्डद्वयात् अस्मात् रू १५ क २०ं० विशोध्य जातो लम्बवर्गः रू ३ क ८ं~। एवं द्वितीयस्य लम्बवर्गस्यो-त्थापनार्थं न्यासः
\vspace{-2mm}

\begin{center}
\hspace{-25mm} याव १ं या २ं याक ७२ रू १३ं क ७२~।
\end{center}
\vspace{-1mm}

अत्राद्य खण्डत्रयमव्यक्तम्~। तत्र प्रथमखण्डस्य पूर्ववन्मानं रू १२ क १२ं८~। द्वितीयखण्डे यावत्तावद्द्वयमस्तीति यावत्तावन्मानं रू २ं क ८ द्वाभ्यां सङ्गुण्य वर्गेण वर्गं गुणयेदिति करणीं चतुर्भिः सङ्गुण्य जातं द्वितीयखण्डमानं रू ४ं क ३२~। अथ तृतीयस्य~। यद्येकेन यावत्तावता व्यक्तमानमिदं क ४ं क ८ तदाभीष्टेनानेन याक ७२ किमिति त्रैराशिकार्थं न्यासः~। या १~। क ४ं क ८~। याक ७२~।

\end{sloppypar}

\newpage

\begin{sloppypar}
अत्र प्रमाणेच्छयोः प्रमाणेनापवर्ते कृतेऽपवर्तितेच्छया क ७२ फले गुणिते जातं तृतीयखण्डमानं क २८ं८ क ५७६~। द्वितीयकरण्या मूले गृहीते जातं रू २४ क २८ं८~। एवं जातान्यव्यक्तखण्डत्रयस्य व्यक्तमानानि\\
\vspace{-2mm}

रू १२ क १२ं८~। रू ४ं क ३२~। रू २४ क २८ं८~।\\
\vspace{-2mm}

अत्र लम्बवर्गे आद्ययोरव्यक्तखण्डयोर्ऋणत्वेन शोध्यत्वात्तदुत्थव्यक्तयोरपि शोध्यत्वेन संशोध्यमानं स्वमृणत्वमेतीत्यादिना जातं रू १ं२ क १२८~। रू ४ क ३ं२~। रू २४ क २८ं८~। एवमग्रिमव्यक्तद्वयेन सह जातानि पञ्च खण्डानि लम्बवर्गे\\
\vspace{-2mm}

रू १ं२ क १२८~। रू ४ क ३ं२~। रू २४ क २८ं८~। रू १ं३~। क ७२~।\\ 
\vspace{-2mm}

अत्र रूपाणां यथोक्तयोगे कृते जातं रू ३~। आद्ययोः करण्योः क १२८ क ३ं२ अन्तरे जातं क ३ं२~। अस्या तृतीयकरण्या सह २८ं८ अन्तरे जातं क १२ं८~। अस्याः पुनरन्त्यया क ७२ अन्तरे जातं क ८ं~। अथवा ऋणकरण्योरनयोः क ३२ क २८८ धनकरण्योरनयोश्च क १२८ क ७२ योगे जातं करणीद्वयं क ५१२~। क ३९२~। अनयोरन्तरे जाता सैव करणी क ८ं~। एवं जातो लम्बवर्गः स एव रू ३ क ८ं~। अथवाबाधा क ४ं क ८ वर्गं रू १२ क १ं२८ स्वभुजस्य क ५ं क १० वर्गात् रू १५ क २०ं० उक्तवदपास्य जातो लम्बवर्गः स एव रू ३ क ८ं~। एवं द्वितीयाबाधा क १ क २ वर्गं रू ३ क ८ स्वभुज\textendash \,क ६\textendash \,वर्गात् रू ६ अपास्य जातो लम्बवर्गः स एव रू ३ क ८ं~।\\

अथास्य पदम्~। तत्र ऋणात्मिका चेत्करणी कृतौ स्याद्धनात्मिकां तां परिकल्प्येति कृते रूपकृतेः ९ करणीतुल्यानि रूपाणि ८ अपास्य शेषस्य १ पदेन १ रूपाणि ३ युतोनितानि ४।२~। अर्धे २।१~। ऋणात्मिकैका सुधियावगम्येऽत्यल्पकरण्या ऋणत्वे कृते पदे च गृहीते जातो लम्बः रू १ं क २~। इदमुदाहरणं व्यक्तमार्गेणापि सिध्यति~। तद्यथा\textendash \,त्रिभुजे भुजयोर्योग इत्यादिना भुजयोरनयोः क ५ं क १०~। क ६~। योगः क ५ं क १० क ६~। लघुभुजं क ५ं क १० महतो भुजात् क ६ अपास्य जातं भुजयोरन्तरं क ५ क १ं० क ६~। अन्तरेण योगस्य गुणनार्थं न्यासः~।
\vspace{-2mm}

\begin{center}
 क ५~। ~क ५ं क १० क ६\\
 क १ं०~। क ५ं क १० क ६\\
 क ६~। ~क ५ं क १० क ६
\end{center}
\end{sloppypar}

\newpage

\begin{sloppypar}
गुणिते जातं खण्डनवकं\\

क \,२ं५ \,क \,५० \,क \,३० \,क \,५० \,क \,१०ं० \,क \,६ं० \,क \,३ं० \,क \,६० \,क \,३६\\

अत्र त्रिंशन्मितकरण्योः षष्टिमितकरण्योश्च धनर्णत्वान्नाशे पञ्चाशन्मितकरण्योर्योगे च कृते क २०० शेषकरणीमूलानां ५ं।१ं०।६ योगे च कृते ९ं जातं गुणनफलं रू ९ं क २०० इदं भूम्यानया रू १ं क १८ भाज्यम्~। अत्र वर्गेण वर्गे भजेदित्युक्तेः क्षयो भवेच्च क्षयरूपवर्ग इति रूपवर्गे कृते जातौ भाज्यभाजकौ \;{\small $\begin{matrix}
\mbox{{क ~८ं१ ~क ~२००}}\\
\vspace{-1mm}
\mbox{{क ~~१ं ~~क ~~१८}}
\vspace{1mm}
\end{matrix}$}~।\\

अथ भाजकस्यैकीकरणार्थं धनर्णताव्यत्ययमीप्सिताया इत्यादिना भाजककरण्याः क १ं धनत्वं प्रकल्प्य तादृक्छिदा क १ क १८ भाज्यभाजकयोर्गुणनार्थं न्यासः
\vspace{-1mm}

\begin{center}
क ~१~। ~क ८१ं क २०० ~~~~क ~१~। ~क १ं क १८\\
क १८~। क ८ं१ क २०० ~~~~क १८~। क १ं क १८
\end{center}
\vspace{-1mm}

भाज्ये गुणिते जातानि करणीखण्डानि क ८ं१ क २०० क १४ं५८ क ३६०० आद्यान्त्य-करण्योर्मध्यमकरण्योश्चान्तरे जातो भाज्यः क २६०१ क ५७ं८~। भाजके गुणिते जातं क १ं क १८ क १ं८ क ३२४~। मध्यमकरण्योर्नाश आद्यान्त्यकरण्योरन्तरे कृते जाता भाजक एकैव करणी क २८९~। अनया भाज्ये भक्ते लब्धिः क ९ क २ं~। प्रथमकरण्या पदे जाता लब्धिः रू ३ क २ं~। अनया भूरेषा रू १ं क १८~। यथावदूनयुता~। रू ४ं क ३२~। रू २ क ८~। यथावदर्द्धिता रू २ं क ८~। रू १ क २ जाते आबाधे~। आभ्यां पूववल्लम्बः रू १ं क २~। आसन्नमूलग्रहणेन जाताः क्षेत्रभुजाद्याः~। दर्शनम्

\begin{center}
\includegraphics[scale=0.5]{Diagram_p2pg33.png}
\end{center}
\end{sloppypar}

\newpage

\begin{sloppypar}
अत्र दशपञ्चकरण्योरासन्नमूले~। ३।१०॥२।१४~। अनयोरन्तरमेको भुजः ०।५६~। एवं सर्वत्र द्रष्टव्यम्~। अत्रापि प्रतीत्यर्थं गणितं लिख्यते~। भुजयोः \;{\small $\begin{matrix}
\mbox{{०}}\\
\vspace{-1.5mm}
\mbox{{५६}}
\vspace{1mm}
\end{matrix}$}\;।\;{\small $\begin{matrix}
\mbox{{२}}\\
\vspace{-1.5mm}
\mbox{{२७}}
\vspace{1mm}
\end{matrix}$}~। योगः \;{\small $\begin{matrix}
\mbox{{३}}\\
\vspace{-1.5mm}
\mbox{{२३}}
\vspace{1mm}
\end{matrix}$}~। भुजयोरन्तरेण \;{\small $\begin{matrix}
\mbox{{१}}\\
\vspace{-1.5mm}
\mbox{{३१}}
\vspace{1mm}
\end{matrix}$}\, गुणितः \;{\small $\begin{matrix}
\mbox{{५}}\\
\vspace{-1.5mm}
\mbox{{८}}
\vspace{1mm}
\end{matrix}$}\, भुवा \;{\small $\begin{matrix}
\mbox{{३}}\\
\vspace{-1.5mm}
\mbox{{१५}}
\vspace{1mm}
\end{matrix}$}\, हृतो लब्धिः \;{\small $\begin{matrix}
\mbox{{१}}\\
\vspace{-1.5mm}
\mbox{{३५}}
\vspace{1mm}
\end{matrix}$}\, अनया द्विष्ठा भूरूनयुता \;{\small $\begin{matrix}
\mbox{{१}}\\
\vspace{-1.5mm}
\mbox{{४०}}
\vspace{1mm}
\end{matrix}$}\;। {\small $\begin{matrix}
\mbox{{४}}\\
\vspace{-1.5mm}
\mbox{{५०}}
\vspace{1mm}
\end{matrix}$}~। दलिता जाते आबाधे \;{\small $\begin{matrix}
\mbox{{०}}\\
\vspace{-1.5mm}
\mbox{{५०}}
\vspace{1mm}
\end{matrix}$}\;।\;{\small $\begin{matrix}
\mbox{{२}}\\
\vspace{-1.5mm}
\mbox{{५०}}
\vspace{1mm}
\end{matrix}$}\, अथाबाधा \;{\small $\begin{matrix}
\mbox{{०}}\\
\vspace{-1.5mm}
\mbox{{५०}}
\vspace{1mm}
\end{matrix}$}\, वर्गं \;{\small $\begin{matrix}
\mbox{{०}}\\
\vspace{-1.5mm}
\mbox{{४२}}
\vspace{1mm}
\end{matrix}$}\, स्वभुज\textendash \,{\small $\begin{matrix}
\mbox{{०}}\\
\vspace{-1.5mm}
\mbox{{५६}}
\vspace{1mm}
\end{matrix}$}\textendash \,वर्गात् \;{\small $\begin{matrix}
\mbox{{०}}\\
\vspace{-1.5mm}
\mbox{{५२}}
\vspace{1mm}
\end{matrix}$}\, अपास्य शेषस्य \;{\small $\begin{matrix}
\mbox{{०}}\\
\vspace{-1.5mm}
\mbox{{१०}}
\vspace{1mm}
\end{matrix}$}\, मूलं \;{\small $\begin{matrix}
\mbox{{०}}\\
\vspace{-1.5mm}
\mbox{{२५}}
\vspace{1mm}
\end{matrix}$}\, जातो लम्बः~। एवं द्वितीयाबाधा \;{\small $\begin{matrix}
\mbox{{२}}\\
\vspace{-1.5mm}
\mbox{{२५}}
\vspace{1mm}
\end{matrix}$}\, वर्गं \;{\small $\begin{matrix}
\mbox{{५}}\\
\vspace{-1.5mm}
\mbox{{५०}}
\vspace{1mm}
\end{matrix}$}\,~स्वभुज\textendash \,{\small $\begin{matrix}
\mbox{{२}}\\
\vspace{-1.5mm}
\mbox{{२७}}
\vspace{1mm}
\end{matrix}$}\textendash \,वर्गात् ६ अपास्य शेषस्य \;{\small $\begin{matrix}
\mbox{{०}}\\
\vspace{-1.5mm}
\mbox{{१०}}
\vspace{1mm}
\end{matrix}$}\, मूलं {\small $\begin{matrix}
\mbox{{०}}\\
\vspace{-1.5mm}
\mbox{{२५}}
\vspace{1mm}
\end{matrix}$}\, जातो लम्बः स एव \;{\small $\begin{matrix}
\mbox{{०}}\\
\vspace{-1.5mm}
\mbox{{२५}}
\vspace{1mm}
\end{matrix}$}~। एवमन्यत्रापि सुधीभिः ऊह्यम्~॥~१०५~॥\\

{\small अथ पक्षयोः समशोधनानन्तरमव्यक्तवर्गघनादिकेऽपि शेषे यथासम्भवमपवर्तेन मध्यमाहरणं विनैवोदाहरणसिद्धिरस्तीति प्रदर्शयितुमुदाहरणषट्कमाह~। तत्रोदाहरणद्वयमनुष्टुभाह\textendash }

\phantomsection \label{7.106}
\begin{quote}
{\large \textbf{{\color{purple}असमानसमच्छेदान् राशींस्तांश्चतुरो वद~।\\
यदैक्यं यद्धनैक्यं वा येषां वर्गैक्यसंमितम्~॥~१०६~॥}}}
\end{quote}

असमानाश्च ते समच्छेदाश्च तान्~। यदैक्यं येषां वर्गैक्यसंमितमित्येकं यद्धनैक्यं येषां वर्गैक्यसंमितमिति द्वितीयमित्युदाहरणद्वयम्~। असमानसमप्रज्ञेति पाठे तु हे असमप्रज्ञ निरुपमबुद्धे समास्तांश्चतुरो राशीन्वदेति योजनीयम्~। प्रथमपाठस्त्वसाधुरिति प्रतिभाति~। न हि समच्छेदत्वपुरस्कारेणोदाहरणमिह साध्यते किं तु समच्छेदत्वं सम्पातायातम्~। असमान् इति त्वपेक्षितमेव~। अन्यथा रूपमितैश्चतुर्भिरुदाहरणसिद्धेः~। अत्र राशीनामसमानत्वेनो-द्देशात्कल्पिता अतुल्या राशयः~। या १ या २ या ३ या ४ उदाहरणद्वयस्यापि गणितं त्वाकर एव स्फुटम्~॥~१०६~॥ \\

{\small अन्यदुदाहरणद्वयमनुष्टुभाह\textendash }

\phantomsection \label{7.107}
\begin{quote}
{\large \textbf{{\color{purple}त्र्यस्रक्षेत्रस्य यस्य स्यात्फलं कर्णेन संमितम्~।\\
दोःकोटिश्रुतिघातेन समं यस्य च तद्वद~॥~१०७~॥}}}
\end{quote}
\end{sloppypar}

\newpage

\begin{sloppypar}
स्पष्टोऽर्थः~। अत्र दोःकोटिकर्णानामव्यक्तकल्पने विशेषोऽस्ति~। जात्यत्र्यस्रे नियतानां तेषां बाधितत्वात्~। अत इष्टजात्यस्य भुजकोटिकर्णैः पृथग्गुणितं यावत्तावत्तेषां मानानि प्रकल्प्योदाहरणद्वयमपि साध्यम्~। आकर एव स्पष्टमन्यत्~॥~१०७~॥\\

{\small अन्यदुदाहरणमनुष्टुभाह\textendash }

\phantomsection \label{7.108}
\begin{quote}
{\large \textbf{{\color{purple}युतौ वर्गोऽन्तरे वर्गो ययोर्घाते घनो भवेत्~।\\
तौ राशी शीघ्रमाचक्ष्व दक्षोऽसि गणिते यदि~॥~१०८~॥}}}
\end{quote}

ययो राश्योर्युतावन्तरे च वर्गो भवेद्घाते तु घनो भवेत्तौ राशी शीघ्रं वद~। अत्र क्रिया-सङ्कोचार्थं तथा राशी कल्प्यौ यथा युतावन्तरे च वर्गः स्यात्~। तथा कल्पितौ याव ४ याव ५~। अनयोर्घातो यावव २०~। एष घन इतीष्टयावत्तावद्दशकस्य घनेन याघ १००० समीकरणे पक्षौ यावत्तावद्घनेनापवर्त्य प्राग्वज्जातौ राशी १००००।१२५००~॥~१०८~॥\\

{\small अथान्यदुदाहरणमनुष्टुभाह\textendash }

\phantomsection \label{7.109}
\begin{quote}
{\large \textbf{{\color{purple}घनैक्यं जायते वर्गो वर्गैक्यं च ययोर्घनः~।\\
तौ चेद्वेत्सि तदाहं त्वां मन्ये बीजविदां वरम्~॥~१०९~॥}}}
\end{quote}

स्पष्टोऽर्थः~। अत्र यथैक आलापः स्वतः सम्भवति तथा राशी कल्पितौ याव १ याव २~। अनयोर्घनयोगः यावघ ९ एष स्वयमेव वर्गो जातः~। यतोऽस्य वर्गमूलमिदं याघ ३ अस्मिन्नर्थे आकर एवाक्षिप्य समाहितम्~। अयमर्थः~। यावद्वर्गघनो राशिः षट्घातात्मकोऽस्ति~। सम-द्विघातस्य समत्रिघातो भवतीति यथा द्विघातस्य घनस्तथा त्रिघातस्य समद्विघातो भव-तीति त्रिघातस्य वर्गोऽपि भवितुं युक्त एवेति~। अथ तयोरेव राश्योः याव १ याव २ वर्ग-योगः यावव ५~। अयं घन इतीष्टं यावत्तावत्पञ्चकघनं याघ १२५ समं कृत्वा पक्षौ यावत् तावद्घनेनापवर्त्य प्रावग्वज्जातौ राशी ६२५।१२५०~। अथवान्यथा मया कल्पितौ राशी याघ ५ याघ १० अनयोर्वर्गैक्यं स्वत एव घनो जायते याघव १२५~। अस्य षड्घातात्मकत्वाद्द्विघात-रूपं घनमूलं यतः सम्भवति याव ५~। अथानयो राश्योः याघ ५ याघ १० घनैक्यं याघघ ११२५~। एतद्वर्ग इति यावत्तावद्वर्गवर्गपञ्चसप्ततिः~। यावव ७५~। वर्गेण याववव ५६२५ समं कृत्वा पक्षौ यावत्तावद्वर्गवर्गेणापवर्त्य पक्षयोर्न्यासः या ११२५ रू ० पूर्ववद्यावत्तावन्मानं ५~। अनेनोत्थापितौ राशी तावेव ६२५।१२५०~। या ० रू ५६२५
\end{sloppypar}

\newpage

\begin{sloppypar}
\noindent अथावायं याघघ ११२५ वर्ग इति यावत्तावद्वर्गवर्गवर्गवर्गपञ्चकस्य यावववव ५ तत्पञ्च-दशकस्य वा यावववव १५ वर्गेण याववववव २५ अनेन वा याववववव २२५ समं कृत्वा पक्षौ याघघ १ अनेनापवर्त्य प्राग्वद्यावत्तावन्मानं ४५ वा ५~। एवमनेकधा~। एवमव्यक्तापवर्तनं यथा सम्भवति तथान्यदपि चिन्त्यम्~॥~१०९~॥\\

{\small अथान्यदुदाहरणं गीत्याह\textendash }

\phantomsection \label{7.110}
\begin{quote}
{\large \textbf{{\color{purple}यत्र त्र्यस्रे क्षेत्रे धात्री मनुसंमिता सखे बाहू~।\\
एकः पञ्चदशान्यस्त्रयोदश वदावलम्बकं तत्र~॥~११०~॥}}}
\end{quote}

स्पष्टोऽर्थः~। आबाधां या १ प्रकल्प्य गणितमप्याकर एव स्फुटम्~। अनतिप्रयोजनम् एतदुदाहरणम्~॥~११०~॥\\

{\small अथ भुजे कोटिकर्णयोगे च ज्ञाते तयोः पृथक्करणं दर्शयितुमुदाहरणं मालिन्याह\textendash }

\phantomsection \label{7.111}
\begin{quote}
{\large \textbf{{\color{purple}यदि समभुवि वेणुर्द्वित्रिपाणिप्रमाणो\\
गणक पवनवेगादेकदेशे सुभग्नः~।\\
भुवि नृपमितहस्तेष्वङ्ग लग्नं तदग्रं\\
कथय कतिषु मूलादेष भग्नः करेषु~॥~१११~॥}}}
\end{quote}

स्पष्टोऽर्थः~। अत्र वंशाधरखण्डं कोटिस्तत्प्रमाणं या १ प्रकल्प्य गणितमाकरे स्फुटम्~। एवमूर्ध्वखण्डमपि या १ प्रकल्प्य गणितं द्रष्टव्यम्~। एवं कोटौ भुजकर्णयोगे च ज्ञाते तत् पृथक्करणमपि द्रष्टव्यम्~।\\

तदुदाहरणं {\color{violet}पाट्या}मुक्तम्~। यथा\textendash

\begin{quote}
{\color{violet}'अस्ति स्तम्भतले बिलं तदुपरि क्रीडाशिखण्डी स्थितः\\
स्तम्भे हस्तनवोच्छ्रिते त्रिगुणितस्तम्भप्रमाणान्तरे~।\\
दृष्ट्वाहिं बिलमाव्रजन्तमपतत्तिर्यक्स तस्योपरि\\
क्षिप्रं ब्रूहि तयोर्बिलात्कतिमितैः साम्येन गत्योर्युतिः~॥'}
\end{quote}

अत्रापि भुजं कर्ण वा या १ प्रकल्प्य प्राग्वद्गणितं द्रष्टव्यम्~॥~१११~॥\\

{\small अथ कोटिकर्णान्तरे भुजे च ज्ञाते कोटिकर्णज्ञानं भवतीति प्रदर्शयितुमुदाहरणं मन्दाक्रान्तयाह\textendash }

\phantomsection \label{7.112}
\begin{quote}
{\large \textbf{{\color{purple}चक्रक्रौञ्चाकुलितसलिले क्वापि दृष्टं तडागे\\
तोयादूर्ध्वं कमलकलिकाग्रं वितस्तिप्रमाणम्~।\\
मन्दं मन्दं चलितमनिलेनाहतं हस्तयुग्मे\\
तस्मिन्मग्नं गणक कथय क्षिप्रमम्बुप्रमाणम्~॥~११२~॥}}}
\end{quote}
\end{sloppypar}

\newpage

\begin{sloppypar}
स्पष्टोऽर्थः~। एतत्क्षेत्रसंस्थानं {\color{violet}पाट्यां} पाठनिबद्धम्~। यथा\textendash

\begin{quote}
{\color{violet}'सखे पद्मतन्मज्जनस्थानमध्यं भुजः कोटिकर्णान्तरं पद्मदृश्यम्~।\\
नलः कोटिरेतन्मितं स्याद्यदम्भो वदैवं समानीय पानीयमानम्~॥'}
\end{quote}

अत्र नलिनीनलप्रमाणं जलगाम्भीर्यम् इति तत्प्रमाणं या १ प्रकल्प्य गणितम् आकरे स्फुटम्~॥~११२~॥\\

{\small अथान्यदुदाहरणं शार्दूलविक्रीडितेनाह\textendash }

\phantomsection \label{7.113}
\begin{quote}
{\large \textbf{{\color{purple}वृक्षाद्धस्तशतोच्छ्रयाच्छतयुगे वापीं कपिः कोऽप्यगात्\\
उत्तीर्याथ परो द्रुतं श्रुतिपथात्प्रोड्डीय किञ्चिद्रुमात्~।\\
जातैवं समता तयोर्यदि गतावुड्डीयमानं कियत्\\
विद्वंश्चेत्सुपरिश्रमोऽस्ति गणिते क्षिप्रं तदाचक्ष्व मे~॥~११३~॥}}}
\end{quote}

परः कपिर्द्रुमात् किञ्चित् प्रोड्डीय श्रुतिपथाद्वापीमगादिति योजनीयम्~। श्रुतिपथादिति ल्यब्लोपे पञ्चमी~। श्रुतिपथमाश्रित्येति तदर्थः~। शेषं स्पष्टम्~। अत्रोड्डीयमानं या १ प्रकल्प्य गणितमाकरे स्पष्टम्~॥~११३~॥\\

{\small अथान्यदुदाहरणमार्ययाह\textendash }

\phantomsection \label{7.114}
\begin{quote}
{\large \textbf{{\color{purple}पञ्चदशदशकरोच्छ्रायवेण्वोरज्ञातमध्यभूमिकयोः~।\\
इतरेतरमूलाग्रगसूत्रयुतेर्लम्बमानमाचक्ष्व~॥~११४~॥}}}
\end{quote}

अत्र लम्बज्ञानार्थं वेण्वन्तरालभूमिज्ञानं नावश्यकमिति सूचयितुमज्ञातमध्यभूमिकयोः इति वेणुविशेषणं न तु प्रश्नपूरणार्थम्~। तेन विनापि प्रश्नपूरणात्~। शेषं स्पष्टम्~। क्षेत्रदर्शनम्\textendash 

\begin{center}
\includegraphics[scale=0.3]{Diagram_p2pg37.png}
\end{center}
\end{sloppypar}

\newpage

\begin{sloppypar}
अत्र क्रियावतारार्थं वेण्वन्तरालभूमिमिष्टां विंशतिमितां प्रकल्प्य सूत्रसम्पाताल्लम्बमानं यावत्तावत्प्रकल्प्य गणितमाकरे स्फुटम्~। अथ यावत्तावत्कल्पनां विनापि लम्बज्ञानार्थमाह 'अथवा वंशसम्बन्धिन्यावाबाधे तद्युतिर्भूमिः' इत्यादि~। अयमर्थः~। यथा यथा वंशो महाल्लँघुर्वा भवति तथा तथा तदाश्रिताबाधापि महती लघुर्वा भवति~। अतस्त्रैराशिकेनैवाबाधे ज्ञातुं शक्ये~। यथा यदि वंशयोगेन सकला भूर्लभ्यते तदैकेन वंशेन किमिति पृथगाबाधे १२।८~। अथ भूमि\textendash \,२०\textendash \,तुल्ये भुजे लघुवंशः १० कोटिस्तदा बृहदाबाधाभुजे १२ केति लब्धो लम्बः ६ यतो लघुवंशः कोटिर्भूमिर्भुजो लघुवंशाग्रादितरवंशमूलगामि सूत्रं कर्ण इत्येतत्क्षेत्रवशादेव बृहदाबाधा भुजो लम्बः कोटिरिति भवति~। एवं लघ्वाबाधाबृहद्वंशाभ्यामप्यनुपातो द्रष्टव्यः~। अथ भूमिकल्पनं विनापि लम्बसिद्धिमाह 'अथवा वंशयोर्वधो योगहृतो यत्र तत्रापि वंशान्तरे लम्बः स्यादिति किं भूमिकल्पनयापि' इति~। अत्रोपपत्तिः~। यदि वंशयोगेन भूर्लभ्यते तदा बृहद्वंशेन किमिति लब्धा बृहद्वंशाश्रिताबाधा~। \;{\small $\begin{matrix}
\mbox{{भू \,० \,बृ \,१}}\\
\vspace{-1mm}
\mbox{{व \,० \,यो \,१}}
\vspace{1mm}
\end{matrix}$}~। अथ भूमितुल्ये भुजे लघुवंशः कोटिस्तदा बृहदाबाधया किम् इति जातो लम्बः \;{\small $\begin{matrix}
\mbox{{भू \,० \,बृ \,० \,ल \,१}}\\
\vspace{-1mm}
\mbox{{~~वं ~यो ~० ~भू ~१}}
\vspace{1mm}
\end{matrix}$}~। अत्र भाज्यभाजकयोर्भूम्यापवर्ते जातं \;{\small $\begin{matrix}
\mbox{{बृ \,० \,ल \,१}}\\
\vspace{-1mm}
\mbox{{~~~वं \,यो \,१}}
\vspace{1mm}
\end{matrix}$}~। एवम् उपपन्नं वंशयोर्वधो योगहृतो लम्बः स्यादिति~॥~११४~॥
\vspace{2mm}

\begin{quote}
{\color{violet}दैवज्ञवर्यगणसन्ततसेव्यपार्श्वबल्लालसञ्ज्ञगणकात्मजनिर्मितेऽस्मिन्~।\\
बीजक्रियाविवृतिकल्पलतावतारेऽभूदेकवर्णजसमीकरणैकखण्डः~॥~७~॥}
\end{quote}
\vspace{-1mm}

\begin{center}

इति श्रीसकलगणकसार्वभौमश्रीबल्लाळदैवज्ञसुतकृष्णदैवज्ञविरचिते \\
बीजविवृतिकल्पलतावतार एकवर्णसमीकरणखण्डविवरणम्~।
\vspace{2mm}

अत्र ग्रन्थसङ्ख्या ४९०~।
\vspace{6mm}

\rule{0.2\linewidth}{0.8pt}\\
\vspace{-4mm}

\rule{0.2\linewidth}{0.8pt}
\end{center}
\end{sloppypar}

\newpage
\thispagestyle{empty}

\begin{center}
\textbf{\large ८\; मध्यमाहरणम्~।}\\
\rule{0.2\linewidth}{0.8pt}
\end{center}

\begin{sloppypar}
{\small तदेवं समशोधनादिना यथैकस्मिन्पक्ष एकजातीयमव्यक्तमेव परपक्षे च व्यक्तमेव भवति तथा-पवर्तादिनोपायेन सम्पाद्य प्रश्नमङ्ग उक्तः~। अथ यद्यप्यपवर्तेनापि तथा न भवति तत्र मध्यमाहरण-लक्षणमुपायान्तरमिन्द्रवज्रयोपजातिकाभ्यां चाह\textendash }

\phantomsection \label{8.115}
\begin{quote}
{\large \textbf{{\color{purple}अव्यक्तवर्गादि यदावशेषं पक्षौ तदेष्टेन निहत्य किञ्चित्~।\\
क्षेप्यं तयोर्येन पदप्रदः स्यादव्यक्तपक्षस्य पदेन भूयः~॥\\
व्यक्तस्य पक्षस्य समक्रियैवमव्यक्तमानं खलु लभ्यते तत्~।\\
न निर्वहश्चेद्धनवर्गवर्गेष्वेवं तदा ज्ञेयमिदं स्वबुद्ध्या~।\\
अव्यक्तमूलर्णगरूपतोऽल्पं व्यक्तस्य पक्षस्य पदं यदि स्यात्~।\\
ऋणं धनं तच्च विधाय साध्यमव्यक्तमानं द्विविधं क्वचित्तत्~॥~११५~॥}}}
\end{quote}

एतानि सूत्राण्याचार्य एव व्याख्यातवान्~। अत्रोपपत्तिः~। एकस्मिन्पक्षेऽव्यक्तमेव परपक्षे च व्यक्तमेव यदि भवति तर्हि तयोः समत्वात्तस्याव्यक्तस्य तद्व्यक्तं मानं भवतीति पूर्वमेवो-क्तम्~। किन्तु व्यक्तशेषस्य हरणार्थमव्यक्तशेषमपृथक्स्थमपेक्षितमतस्तादृशं यथा भवति तथा यतितव्यम्~। तत्र समयोः पक्षयोः समक्षेपे समशुद्धौ वा समगुणके वा समहरे वा मूलग्रहणे वा वर्गकरणे वा धनादिकरणे वा न समत्वहानिरिति तु स्पष्टम्~। अथ यत्रा-व्यक्तवर्गादिकं स्यादेकपक्षे परपक्षे च रूपाण्येव तत्र मूलेन विना कदापि नाव्यक्तस्या-पृथक्स्थितिः~। अतः पक्षयोः साम्याविरोधेन मूले ग्राह्ये~। तथा सति मूलयोरपि समत्वं स्यात्~। अत उक्तं \hyperref[8.115]{'पक्षौ तदेष्टेन निहत्य किञ्चित्क्षेप्यं तयोर्येन पदप्रदः स्यात्'} इति~। अत्रेष्टेन निहत्येत्युपलक्षणम्~। क्वचिदिष्टेन पक्षावपवर्तनीयौ क्वचिदिष्टं पक्षयोः शोध्यमित्याद्यपि ध्येयम्~। शेषोपपत्तिस्तु पूर्ववत्~। द्विविधमाने तु तत्रोदाहरणं \hyperref[8.124]{'वनान्तराले प्लवगाष्टभागः'} इति वक्ष्यमाणम्~। अत्र कपियूथं या १ अस्याष्टांशवर्गो द्वादशयुतो यूथसम इति समशोधने कृते जातौ पक्षौ \;{\small $\begin{matrix}
\mbox{{याव ~१ ~या ~६ं४ ~~रू ~~०}}\\
\vspace{-1mm}
\mbox{{याव ~० ~या ~~० ~रू ~७६ं८}}
\vspace{1mm}
\end{matrix}$}~। अत्र पक्षयोर्द्वात्रिंशद्वर्गं १०२४ प्रक्षिप्य जातौ पक्षौ \;{\small $\begin{matrix}
\mbox{{याव ~१ ~या ~६ं४ ~रू ~१०२४}}\\
\vspace{-1mm}
\mbox{{याव ~० ~या ~~० ~रू ~~२५६}}
\vspace{1mm}
\end{matrix}$}~। अत्रोर्ध्वपक्षस्य पदमिदं या १ रू \;{\small $\begin{matrix}
\mbox{{३ं२}}\\
\vspace{-1mm}
\mbox{{१८}}
\vspace{1mm}
\end{matrix}$}\, इदं वा या १ं रू ३२~। द्वितीयपक्षस्य पदमिदं रू १६~। पदयोः समशो-
\end{sloppypar}

\newpage

\begin{sloppypar}
\noindent धनार्थं न्यासः \;{\small $\begin{matrix}
\mbox{{या ~१ ~रू ~३ं२}}\\
\vspace{-1mm}
\mbox{{या ~० ~रू ~१६}}
\vspace{1mm}
\end{matrix}$}~। अयं वा \;{\small $\begin{matrix}
\mbox{{या ~१ं ~रू ~३२}}\\
\vspace{-1mm}
\mbox{{या ~० ~रू ~१६}}
\vspace{1mm}
\end{matrix}$}~। अतो द्विविधमपि मानमुपपद्यते ४८।१६~। नन्वव्यक्तपदरूपेभ्यो व्यक्तपदेऽधिकेऽपि द्विविधं मानमनया युक्त्या कथं न स्यात्~। शृणु तर्हि~। अव्यक्तपक्षजरूपाणामृणत्वे व्यक्तस्य धनत्वमेव~। अस्मिन्प्रकारेऽव्यक्तशेषस्य धनत्वार्थमव्यक्तपक्षरूपाण्येव व्यक्तपक्षाच्छोध्यानि~। तानि च धनं भवतीति नास्त्यनु-पपत्तिः~। अथ रूपाणां धनत्वे व्यक्तस्यर्णत्वमेवेति द्वितीयप्रकारे व्यक्तमेव धनत्वार्थमितर-पक्षाच्छोध्यम्~। व्यक्तरूपाणि त्वव्यक्तपक्षजपदरूपेभ्यः शोध्यत्वादृणं भवति~। तानि~यदि अधिकानि तदा ऋणं मानं स्यादिति द्वितीयं सर्वथाप्यनुपपन्नम्~। अत उक्तं \hyperref[8.115]{'अव्यक्तमूल-र्णगरूपतोऽल्पं व्यक्तस्य पक्षस्य पदं यदि स्यात्'} इति~। अथ यत्रालापे रूपोनमव्यक्तम् अस्ति तस्य वर्गे कर्तव्ये रूपाणामृणत्वादव्यक्तस्यर्णत्वमुत्पद्यते~। तत्र पदग्रहणे रूपाणामेव ऋणत्वं नाव्यक्तस्य~। आलापे रूपाणाम् ऋणत्वनिश्चयात्~। अव्यक्तस्यर्णत्वे कल्पिते ऋणं पक्षः स्यात्~। न ह्यधिकस्य शोध्यत्वे धनं पक्षः सम्भवति~। भवतु वा क्वचिदस्य धनत्वम्~। तथाप्यालापसिद्धपक्षादन्यथात्वं तु स्यादेव~। एवं सत्यालापसिद्धपक्षसमेन द्वितीयपक्षेण कथमस्य साम्यं स्यात्~। अतस्तत्समीकरणेनागतं मानमुपपन्नमेव स्यात्~। ऋणत्वात्~। न हि व्यक्ते ऋणगते लोकस्य प्रतीतिरस्ति~। तस्मादेतादृश उदाहरणे व्यक्त-पदे व्यक्तमूलर्णगरूपतोऽल्पेऽपि द्विविधं मानं न सम्भवति~। रूपाणां धनत्वकल्पनेन सिद्धस्य मानस्यानुपपन्नत्वात्~। एवमव्यक्तोनरूपवर्ग उद्दिष्टे सति तन्मूले व्यक्तस्यैव ऋणत्वं न रूपाणाम्~। उक्तयुक्तेरविशेषात्~। अतस्तत्रापि द्विविधं मानं न सम्भवति~। रूपाणाम् ऋणत्वकल्पनेन सिद्धस्य मानस्यानुपपन्नत्वात्~। इत्येवं बहुधा भवति~। क्वचित्क्षेपशोधना-दिना शेषविधानां विपरीतमपि भवति~। क्वचिदव्यक्तस्य स्वतोऽप्यृणत्वे द्विविधमूलसम्भ-वेऽपि द्वितीयमनुपपन्नं भवति~। अत एवाचार्यैर्द्विविधं क्वचित्तदित्यनियमेनैवोक्तम्~। अथ द्वितीयमानस्यानुपपत्तौ वक्ष्यमाणमुदाहरणं प्रतीत्यर्थं प्रदर्श्यते~। 

\begin{quote}
\hyperref[8.125]{यूथात्पञ्चांशकस्त्र्यूनो वर्गितो गह्वरं गतः~।\\
दृष्टः शाखामृगः शाखामारूढो वद ते कति~॥}
\end{quote}

अत्र यूथं या ५ अस्य पञ्चांशः या १~। त्र्यूनः या १ रू ३ं~। वर्गितः याव १ या ६ं रू ९~। दृष्टेन युतो याव १ या ६ं रू १०~। यूथसम इति शोधने कृते जातं \;{\small $\begin{matrix}
\mbox{{याव ~१ ~या ~१ं१ ~रू ~०~।}}\\
\vspace{-1mm}
\mbox{{\hspace{14mm} १ं० ~रू ~०~।}}
\vspace{1mm}
\end{matrix}$}

\end{sloppypar}

\newpage

\begin{sloppypar}
\noindent पक्षौ चतुर्भिः सङ्गुण्य तयोरेकादशवर्गं क्षिप्त्वा जातौ \;{\small $\begin{matrix}
\mbox{{याव ~४ ~या ~४ं४ ~रू ~१२१}}\\
\vspace{-1mm}
\mbox{{\hspace{20mm} रू ~८१}}
\vspace{1mm}
\end{matrix}$}~।\\

अत्र रूपाणामेव ऋणत्वोद्देशादुक्तयुक्त्या पदमिदमेव या २ रू १ं१ नेदं या २ं रू ११~। द्वितीयपक्षस्य पदं रू ९~। पुनः समीकरणेन लब्धं यावत्तावन्मानेन १० उत्थापितो जातो राशिः ५०~। रूपाणां धनत्वे तु यावत्तावन्मानमिदं १~। राशिश्च ५~। नह्यस्य पञ्चांशः ५ त्रिभिरूनः सम्भवति~। एवमस्मिन्नेवोदाहरणे यूथात्पञ्चांशकस्त्रिच्युत इति यथालापः स्यात्तदा द्वितीयमानमेव युक्तं न तु पूर्वम्~। नहि पूर्वराशेः पञ्चांशः १० त्रिच्युतः सम्भवति~। अत एव\textendash
\vspace{-1mm}

\begin{quote}
{\color{violet}'द्युज्यकापमगुणार्कदोर्ज्यका स्वं युतिं खखखबाणसंमिताम्~।\\
वीक्ष्य भास्करमवेहि मध्यमं मध्यमाहरणमस्ति चेद्ध्रुवम्~॥'}
\end{quote}
\vspace{-1mm}

इत्यस्मिंस्त्रिप्रश्नोदाहरणे क्रान्तिज्यां यावत्तावन्मितां प्रकल्प्य ततोऽनुपातेन दोर्ज्यां चानीय तयोर्योगमुद्दिष्टयुतेर्विशोध्य तद्वर्गं क्रान्तिज्यापवर्गोनत्रिज्यावर्गात्मकेन द्युज्यावर्गेण समं कृत्वा समशोधने कृते पक्षयोः पदग्रहणावसरे व्यक्तमृणं रूपाणि धनमित्येव गृह्यते~। अत एव तदानयनसूत्रेऽपि तेनाढ्य ऊनो भवेदित्येवोक्तम्~। रूपाणामृणत्वे तु तेनाढ्य आढ्यो भवेदित्यप्युच्येत~। एवं मदुक्तयुक्त्या द्विविधमानोपपत्त्यनुपपत्ती सर्वत्रावधार्ये~। तदेवमुपपन्नं द्विविधं क्वचित्तदिति~। पदग्रहणार्थं \hyperref[8.115]{'पक्षौ तदेष्टेन निहत्य किञ्चित्~। क्षेप्यं तयोः'} इति उक्तम्~॥~११५~॥\\

{\small तत्र केन पक्षौ गुणनीयौ किं वा तयोः क्षेप्यमिति बालावबोधार्थं {\color{violet}श्रीधराचार्य}कृतमुपायं दर्शयति\textendash }

\phantomsection \label{8.116}
\begin{quote}
{\large \textbf{{\color{purple}'चतुराहतवर्गसमै रूपैः पक्षद्वयं गुणयेत्~।\\
पूर्वाव्यक्तस्य कृतेः समरूपाणि क्षिपेत्तयोरेव'} इति~{\color{purple}॥~११६~॥}}}
\end{quote}

अस्यार्थः~। चतुर्गुणितेनाव्यक्तवर्गाङ्केन पक्षद्वयं गुणयेत्~। गुणनात्प्राग्यो व्यक्ताङ्कस्तद्वर्ग-तुल्यानि रूपाणि पक्षयोः क्षिपेत्~। एवं कृतेऽवश्यमव्यक्तपक्षस्य मूलं लभ्यते~। द्वितीयपक्ष-स्याप्येतत्समत्वान्मूलेन भाव्यम्~। एवं सति व्यक्तपक्षस्य यदि मूलं न लभ्यते तदा तत्खिलम् एवेत्यर्थात्सिद्धम्~। अत्र श्रीधराचार्यसूत्रे मूलोपायस्याव्यक्तवर्गाव्यक्तसापेक्षतयोक्तत्वाद्यत्रै-कस्मिन्पक्षेऽव्यक्तवर्गोऽव्यक्तं च भवेत्तत्रैवास्य प्रवृत्तिरन्यत्र तु पदोपायः सुधिया स्वधिया चिन्त्यः~। अथ श्रीधराचार्यसूत्रोपपत्तिः~। यत्र किल समशोधने कृत एकपक्षेऽव्यक्तवर्गः अव्यक्तं चास्ति, इतरस्मिन्पक्षे रूपाण्येव सन्ति तत्र प्रथमपक्षे रूपयोगेन विना
\end{sloppypar}

\newpage

\begin{sloppypar}
\noindent कथमपि न मूललाभः~। यतः केवलाव्यक्तस्य वर्गकरणेऽव्यक्तवर्ग एव स्यात्~। रूपयुता-व्यक्तस्य वर्गकरणेऽव्यक्तवर्गोऽव्यक्तं रूपाणि च स्युः~। प्रकृते त्वव्यक्तवर्गोऽव्यक्तं च तिष्ठति स न कस्यापि वर्गः~। अतोऽवश्यं रूपाणि क्षेप्याणि~। यद्यप्यव्यक्तशोधनेनाप्यव्यक्तमात्रस्य शेषत्वादव्यक्तपक्षस्य मूलं लभ्यते तथापि द्वितीयपक्षे तथा सति साव्यक्तानि रूपाणि स्युरिति नास्य मूललाभ इति पक्षयो रूपाण्येव क्षेप्याणि~। तत्र यदाव्यक्तवर्गस्य~मूलं लभ्यते तदा केवलं रूपाण्येव क्षेप्याणि~। यदा त्वव्यक्तवर्गस्य मूलं न लभ्यते तदाव्यक्त-वर्गोऽपि तथा केनचिद्योज्यो गुणनीयो वा यथा मूलं लभ्येत~। तत्राव्यक्तवर्गयोगे यद्यपि अव्यक्तपक्षस्य मूलं लभ्यते तथापि द्वितीयपक्षे साव्यक्तवर्गाणि रूपाणि स्युरित्यव्यक्ता-भावान्न मूललाभः~। न च पक्षयोरव्यक्तमपि क्षेप्यमिति वाच्यम्~। गौरवात्~। किं च यदाव्यक्तपक्षेऽव्यक्तवर्गद्वयमस्ति तदा पक्षयोः किं क्षेप्यम्~। द्विसप्तचतुर्दशत्रयोविंशति-चतुस्त्रिंशत्सप्तचत्वारिंशद्द्विषष्ट्याद्यव्यक्तवर्गक्षेपे प्रथमपक्षस्यैव मूलं लभ्येत नेतरस्य~।~एक-चतुराद्यव्यक्तवर्गक्षेपे तु प्रथमपक्षस्य मूलं न लभ्येत~। न च यत्राव्यक्तवर्गद्वयमस्ति तत्र पक्षयोः एकस्याव्यक्तवर्गस्य शोधनेनोभयोरपि मूलं लभ्येत इति वाच्यम्~। द्वितीयपक्ष ऋण-स्याव्यक्तवर्गस्य मूलाभावात्~। न च त्रिपञ्चादिष्वव्यक्तवर्गेषु सत्स्वेकचतुरादयो व्यक्तवर्गाः पक्षयोः क्षेप्या द्विषडादिष्वव्यक्तवर्गेषु सत्सु पक्षौ द्विषडादिभिर्गुणनीयाविति वाच्यम्~।~अनु-गमे सत्यननुगमस्यान्याय्यत्वात्~। क्रियानिर्वाहस्यानियतत्वाच्च~। अतिगौरवाच्च~। यतोऽव्यक्त-वर्गाव्यक्तरूपाणि तथा क्षेप्याणि यथोभयपक्षयोरपि मूलं लभ्येत~। किं च मन्दबोधार्थं ह्युपायकथनम्~। एतादृशस्य तु क्षेपस्य मन्ददुर्ज्ञेयतयोपायकथनं व्यर्थमेव स्यात्~। तदेवं व्यक्तवर्गः केनचिद्गुणनीय एवेति सिद्धम्~। तत्र यदाव्यक्तवर्गस्य मूलं लभ्यते तदा रूपा-ण्येव क्षेप्याणि~। तानि कियन्तीति विचार्यते~। तत्र यद्यव्यक्तवर्गस्यैकमव्यक्तं मूलं लभ्यते तर्ह्यव्यक्तार्धवर्गक्षेपेऽव्यक्तपक्षस्यावश्यं मूललाभः~। यतः \hyperref[3.31]{'कृतिभ्य आदाय पदानि'} इत्यादि-नाव्यक्तवर्गस्यैकमव्यक्तं मूलं रूपाणां त्वव्यक्तार्धतुल्यरूपाणि द्वयोरभिहतिरव्यक्तार्धतुल्या स्यात्सा द्विनिघ्नी अव्यक्ततुल्या स्यादिति तच्छोधनेन निःशेषता स्यात्~। एवं यत्राव्यक्त-वर्गस्याव्यक्तद्वयं मूलं लभ्यते तत्राप्यनयैव युक्त्या यथास्थिताव्यक्तचतुर्थांशवर्गतुल्य-रूपक्षेपेऽवश्यं मूललाभः~। एवं यत्राव्यक्तत्रयं मूलं लभ्यते तत्र पक्षस्थिताव्यक्तषडंशवर्ग-तुल्यरूपक्षेपेऽवश्यं मूललाभः~। तथा च यत्राव्यक्तवर्गस्य मूलं लभ्यते तत्र तेन मूलाङ्केन द्विगुणेनाव्यक्ताङ्के भक्ते यल्लभ्यते तद्वर्गतुल्यानि रूपाणि क्षेप्याणीति सिद्धम्~।\\

अथ यत्राव्यक्तवर्गाङ्कस्य न मूलं लभ्यते तत्र तेनैवाङ्केन गुणने सत्यवश्यं मूललाभ इति अव्यक्तवर्गाङ्केन पक्षौ गुणनीयौ~।
\end{sloppypar}

\newpage

\begin{sloppypar}
अथात्र पूर्वयुक्त्या रूपक्षेपः~। तदर्थमव्यक्तवर्गमूलाङ्केन द्विगुणेनाव्यक्ताङ्को भाज्यः~। अत्रा-व्यक्तवर्गमूलाङ्कस्त्वगुणितोऽव्यक्तवर्गाङ्कः~। तथा चागुणितेनाव्यक्तवर्गाङ्केन द्विगुणेनाव्यक्ताङ्को भाज्यः~। पक्षगुणकेनागुणिताव्यक्तवर्गाङ्केन गुण्यश्च~। अत्र गुणहरयोरगुणिताव्यक्तवर्गाङ्केना-पवर्ते कृते जातः पूर्वाव्यक्ताङ्कस्य द्वयं भाजकः~। अतः पूर्वाव्यक्तार्धवर्गतुल्यानि रूपाणि क्षेप्याणीति सिद्धम्~। एवं यत्र विनैव गुणनमव्यक्तवर्गाङ्कस्य मूलं लभ्यते तत्राप्युक्तयुक्त्या पक्षावव्यक्तवर्गाङ्केन सङ्गुण्य पूर्वाव्यक्तार्धवर्गतुल्यानि रूपाणि प्रक्षिप्य च मूलं लभ्येतैव~। युक्तेरविशेषात्~। तदेवं पक्षावव्यक्तवर्गाङ्केन गुण्यौ पूर्वाव्यक्तार्धवर्गतुल्यानि रूपाणि तयोः क्षेप्यानि चेति सिद्धम्~। एतावतैव पक्षयोर्मूललाभे सिद्धेऽप्यभिन्नत्वार्थं पुनश्चतुर्भिर्गुणनम् उक्तम्~। यतो वर्गेण वर्गगुणने कृते नास्ति वर्गत्वहानिः~। अथात्र पूर्वयुक्त्या क्षेपः~। अत्राव्यक्त-वर्गे चतुर्भिर्गुणिते तन्मूलाङ्को द्विगुणितः स्यात्~। तेन च द्विगुणेनाव्यक्ताङ्को भाज्य इति जातः पूर्वाव्यक्तस्य पूर्वाव्यक्तवर्गाश्चतुर्गुणो भाजकः~। पक्षगुणकोऽपि तावानेवास्तीति गुणहरयोः तुल्यत्वान्नाशे पूर्वाव्यक्ततुल्यानि रूपाणि क्षेप्याणीति सिद्धम्~। तदेवमुपपन्नम्\textendash
\vspace{-1mm}

\begin{quote}
\hyperref[8.116]{'चतुराहतवर्गसमै रूपैः पक्षद्वयं गुणयेत्~।\\
पूर्वाव्यक्तस्य कृतेः समरूपाणि क्षिपेत्तयोरेव~॥'} इति~।
\end{quote}
\vspace{-1mm}

एवं कृतेऽपि यदि व्यक्तपक्षस्य मूलं न लभ्यते तदा करण्यात्मकं मूलं ग्राह्यम्~॥~११६~॥\\

{\small अथात्र शिष्यबुद्धिप्रसारार्थं विविधान्युदाहरणानि निरूपयन्नेकमुदाहरणं मालिन्याह\textendash }

\phantomsection \label{8.117}
\begin{quote}
{\large \textbf{{\color{purple}अलिकुलदलमूलं मालतीं यातमष्टौ\\
निखिलनवमभागाश्चालिनी भृङ्गमेकम्~।\\
निशि परिमललुब्धं पद्ममध्ये निरुद्धं\\
प्रतिरणति रणन्तं ब्रूहि कान्तेऽलिसङ्ख्याम्~॥~११७~॥}}}
\end{quote}

स्पष्टोऽर्थः~। अत्रालिकुलप्रमाणं द्विगुणवर्गात्मकं कल्प्यं यतोऽस्यैव दलमूलं सम्भवति~। अतस्तथा कल्पितमाचार्यैः याव २~। गणितमाकरे स्फुटम्~। जातालिकुलसङ्ख्या ७२~॥~११७~॥\\

{\small अथान्यदुदाहरणं शार्दूलविक्रीडितेनाह\textendash }

\phantomsection \label{8.118.1}
\begin{quote}
{\large \textbf{{\color{purple}पार्थः कर्णवधाय मार्गणगणं क्रुद्धो रणे सन्दधे\\
तस्यार्धेन निवार्य तच्छरगणं मूलैश्चतुर्भिर्हयान्~।}}}
\end{quote}
\end{sloppypar}

\newpage

\begin{sloppypar}
\phantomsection \label{8.118}
\begin{quote}
{\large \textbf{{\color{purple}शल्यं षड्भिरथेषुभिस्त्रिभिरपि च्छत्रं ध्वजं कार्मुकं\\
चिच्छेदास्य शिरः शरेण कति ते यानर्जुनः सन्दधे~॥~११८~॥}}}
\end{quote}

स्पष्टोऽर्थः~। अत्र कल्पितं बाणमानं याव १~। अस्यार्धं याव \;{\small $\begin{matrix}
\mbox{{१}}\\
\vspace{-1mm}
\mbox{{२}}
\vspace{1mm}
\end{matrix}$}~। चत्वारि मूलानि या ४~। दृश्यबाणगणश्च रू १०~। एषामैक्यं याव \;{\small $\begin{matrix}
\mbox{{१}}\\
\vspace{-1mm}
\mbox{{२}}
\vspace{1mm}
\end{matrix}$}\, या ८ रू २०~। राशिः याव १~। समं कृत्वा पक्षौ समच्छेदीकृत्य च्छेदगमे शोधने च कृते पक्षयोः षोडश रूपाणि प्रक्षिप्य मूले गृहीत्वा पुनः समीकरणेन लब्धं यावत्तावन्मानम् १०~। जाता बाणसङ्ख्या १००~॥~११८~॥\\

{\small अथान्यदुदाहरणमुपजातिकयाह\textendash }

\phantomsection \label{8.119}
\begin{quote}
{\large \textbf{{\color{purple}व्येकस्य गच्छस्य दलं किलादिरादेर्दलं तत्प्रचयः फलं च~।\\
चयादिगच्छाभिहतिः स्वसप्तभागाधिका ब्रूहि चयादिगच्छान्~॥~११९~॥}}}
\end{quote}

फलं चेति~। चस्त्वर्थे~। तथा सति फलशब्दस्योत्तरार्धेनान्वयः सुबोधः~। शेषं स्पष्टम्~। अत्र गच्छमानं यावत्तावच्चतुष्टयरूपाधिकं या ४ रू १ प्रकल्प्य गणितमाकरे स्फुटम्~। द्वितीय-प्रकारेण फलसाधनार्थं पाटीस्थं सूत्रमिदम्\textendash

\begin{quote}
{\color{violet}'व्येकपदघ्नचयो मुखयुक् स्यादन्त्यधनं मुखयुग्दलितं तत्~।\\
मध्यधनं पदसङ्गुणितं तत्सर्वधनं गणितं च तदुक्तम्'} इति~॥~११९~॥
\end{quote}

{\small अथान्यदुदाहरणमनुष्टुभाह\textendash }

\phantomsection \label{8.120}
\begin{quote}
{\large \textbf{{\color{purple}कः खेन विहृतो राशिः कोट्या युक्तोऽथवोनितः~।\\
वर्गितः स्वपदेनाढ्यः खगुणो नवतिर्भवेत्~॥~१२०~॥}}}
\end{quote}

स्पष्टोऽर्थः~। अत्र राशिः या १~। अयं खहृतः या \;{\small $\begin{matrix}
\mbox{{१}}\\
\vspace{-1mm}
\mbox{{०}}
\vspace{1mm}
\end{matrix}$}~। अयं कोट्या युक्त ऊनितो वाविकृत एव~। खहरत्वात्~। अथायं या \;{\small $\begin{matrix}
\mbox{{१}}\\
\vspace{-1mm}
\mbox{{०}}
\vspace{1mm}
\end{matrix}$}\, वर्गितो याव \;{\small $\begin{matrix}
\mbox{{१}}\\
\vspace{-1mm}
\mbox{{०}}
\vspace{1mm}
\end{matrix}$}~। स्वपदेन या \;{\small $\begin{matrix}
\mbox{{१}}\\
\vspace{-1mm}
\mbox{{०}}
\vspace{1mm}
\end{matrix}$}\, युक्तः याव १ या \;{\small $\begin{matrix}
\mbox{{१}}\\
\vspace{-1mm}
\mbox{{०}}
\vspace{1mm}
\end{matrix}$}~। अयं खगुणो जातः याव १ या १~। गुणहरयोस्तुल्यत्वेन
\end{sloppypar}

\newpage

\begin{sloppypar}
\noindent नाशात्~। अथामुं नवतिसमं कृत्वा समशोधने कृते पक्षौ चतुर्भिः सङ्गुण्य रूपं प्रक्षिप्य प्राग्वज्जातो राशिः ९~॥~१२०~॥\\

{\small अन्यदुदाहरणमनुष्टुभाह\textendash }

\phantomsection \label{8.121}
\begin{quote}
{\large \textbf{{\color{purple}कः स्वार्धसहितो राशिः खगुणो वर्गितो युतः~।\\
स्वपदाभ्यां खभक्तश्च जातः पञ्चदशोच्यताम्~॥~१२१~॥}}}
\end{quote}

स्पष्टोऽर्थः~। अत्र राशिः या १~। गणितमाकरे स्फुटम्~। मूलार्थं रूपचतुष्टयं क्षेपः~॥~१२१~॥\\

{\small अन्यदुदाहरणमार्ययाह\textendash }

\phantomsection \label{8.122}
\begin{quote}
{\large \textbf{{\color{purple}राशिर्द्वादशनिघ्नो राशिघनाढ्यश्च कः समो यस्य~।\\
राशिकृतिः षड्गुणिता पञ्चत्रिंशद्युता विद्वन्~॥~१२२~॥}}}
\end{quote}

स्पष्टोऽर्थः~। गणितमाकरे स्फुटम्~॥~१२२~॥\\

{\small अथान्यदुदाहरणं सार्धानुष्टुभाह\textendash }

\phantomsection \label{8.123}
\begin{quote}
{\large \textbf{{\color{purple}को राशिर्द्विशतीक्षुण्णो राशिवर्गयुतो हतः~।\\
द्वाभ्यां तेनोनितो राशिवर्गवर्गोऽयुतं १०००० भवेत्~।\\
रूपोनं वद तं राशिं वेत्सि बीजक्रियां यदि~॥~१२३~॥}}}
\end{quote}

स्पष्टोऽर्थः~। रूपोनमयुतं भवेदिति योजनीयम्~। राशिः या १~। अस्य यथोक्ते समशोधने कृते पक्षयोः याव ४ या ४०० रू १~। एतावत्क्षिप्त्वा गणितमाकरे स्फुटम्~॥~१२३~॥\\

{\small अथ \hyperref[8.115]{'अव्यक्तमूलर्णगरूपतोऽल्पम्'} इत्यस्य सूत्रस्योदाहरणमुपजातिकयाह\textendash }

\phantomsection \label{8.124}
\begin{quote}
{\large \textbf{{\color{purple}वनान्तराले प्लवगाष्टभागः संवर्गितो वल्गति जातरागः~।\\
ब्रूत्कारनादप्रतिनादहृष्टा दृष्टा गिरौ द्वादश ते कियन्तः~॥~१२४~॥}}}
\end{quote}

प्लवगा वानराः~। ब्रूदिति तन्नादानुकृतिः~। शेषं स्पष्टम्~। गणितमाकरे स्फुटम्~। द्विधा मानं चैतत् ४८।१६~॥~१२४~॥\\

{\small अथ द्विधामानस्य क्वाचित्कत्वप्रदर्शनार्थमुदाहरणद्वयमनुष्टुब्द्वयेनाह\textendash }

\phantomsection \label{8.125}
\begin{quote}
{\large \textbf{{\color{purple}यूथात्पञ्चांशकस्त्र्यूनो वर्गितो गह्वरं गतः~।\\
दृष्टः शाखामृगः शाखामारूढो वद ते कति~॥\\
कर्णस्य त्रिलवेनोना द्वादशाङ्गुलशङ्कुभा~।\\
चतुर्दशाङ्गुला जाता गणक ब्रूहि तां द्रुतम्~॥~१२५~॥}}}
\end{quote}

त्रिभिरूनस्त्र्यूनः~। शाखामृगो वानरः~। स्पष्टमन्यत्~। गणितमाकरे स्पष्टम्~॥~१२५~॥
\end{sloppypar}

\newpage

\begin{sloppypar}
{\small अथान्यदुदाहरणमनुष्टुब्द्वयेनाह\textendash }

\phantomsection \label{8.126}
\begin{quote}
{\large \textbf{{\color{purple}चत्वारो राशयः के ते मूलदा ये द्विसंयुताः~।\\
द्वयोर्द्वयोर्यथासन्नघाताश्चाष्टादशान्विताः~॥\\
मूलदाः सर्वमूलैक्यादेकादशयुतात्पदम्~।\\
त्रयोदश सखे जातं बीजज्ञ वद तान्मम~॥~१२६~॥}}}
\end{quote}

स्पष्टोऽर्थः~। अत्रोदाहरणे राशीनामव्यक्तकल्पने क्रिया न निर्वहति~। अत एकं मूलं यावत्तावत्प्रकल्प्य यथा सर्वमूलानि सिध्यन्ति तथा निरूपयति~। तत्र राशिमूलकल्प-नार्थमाह\textendash \,'अत्र राशी येन युतौ मूलदौ भवतः स किल राशिक्षेपः~। मूलयोरन्तरवर्गेण हतो राशिक्षेपो वधक्षेपो भवति~। तयो राश्योर्वधस्तेन युतोऽवश्यं मूलदः स्यादित्यर्थः' इति~।\\

नन्वनेन ग्रन्थेन राशिमूलकल्पनं कथमुक्तम्~। शृणु~। मूलयोरन्तरवर्गेण हतो राशिक्षेपो वधक्षेपो भवतीत्यनेन राशिक्षेपस्य मूलान्तरवर्गस्य च वधो वधक्षेपोऽस्तीति स्पष्टीकृतम्~। तथा च राशिक्षेपेण वधक्षेपे भक्ते यल्लभ्यते स एव मूलान्तरवर्गः~। अतस्तस्य पदं मूलान्तरं स्यात्~। अतो यावत्तावदात्मकं प्रथममूलं तेन मूलान्तरेण युक्तं सद्द्वितीयमूलं स्यात्~। तदपि पुनस्तेनैव युक्तं सत्तृतीयं स्यादित्यादि~॥~१२६~॥\\

{\small इदमेवोक्तमाद्यपरिभाषायामपि\textendash }

\phantomsection \label{8.127}
\begin{quote}
{\large \textbf{{\color{purple}राशिक्षेपाद्वधक्षेपो यद्गुणस्तत्पदोत्तरम्~।\\
अव्यक्तराशयः कल्प्या वर्गिताः क्षेपवर्जिताः~॥~१२७~॥}}}
\end{quote}

अत्राव्यक्तराशयो राशिमूलान्येव~। अत एवैभ्यो राशिज्ञानमुक्तं चतुर्थचरणेन \hyperref[8.127]{'वर्गिताः क्षेपवर्जिताः'} इत्यनेन~। यतो राशिः क्षेपेण योज्यस्तस्य मूलं राशिमूलं भवत्यतो व्यस्तविधिना राशिमूलं वर्गितं क्षेपोनं सद्राशिर्भवेदित्यर्थः~। अथैतेभ्यो वधमूलान्याह\textendash
\vspace{-1mm}

\begin{quote}
{\color{violet}'राशिमूलानां यथासन्नं द्वयोर्द्वयोर्वधाः~।\\
राशिक्षेपोना राशिवधमूलानि भवन्ति~॥'} इति~।
\end{quote}
\vspace{-1mm}

स्पष्टोऽर्थः~। अत्रोभयोपपत्तिरुच्यते~। अत्र क्षेपयुतराशेर्मूले ज्ञाते व्यस्तविधिना मूलवर्गे क्षेपोने राशिर्भवेदिति जातः प्रथममूलात्प्रथमराशिः प्रमूव १ क्षे १ं~। एवं द्वितीयमूलात् द्वितीयोऽपि द्विमूव १ क्षे १ं~। अनयोर्वधो येन युक्तः सन्मूलदो भवेत्स एव वधक्षेपः~। तदर्थम् अनयोर्गुणनार्थं न्यासः \;{\small $\begin{matrix}
\mbox{{प्रमूव ~१~। ~द्विमूव ~१ं ~क्षे ~१ं~।}}\\
\vspace{-1mm}
\mbox{{~~~~क्षे ~१ं~। ~द्विमूव ~१ं ~क्षे ~१ं~।}}
\vspace{1mm}
\end{matrix}$} \\

गुणनाज्जातं प्रमूव ० द्विमूव १ प्रमूव ० क्षे १ं द्विमूव ० क्षे १ं क्षेव १~।\\
\vspace{-1mm}

अत्र द्वितीयखण्डे क्षेपगुणः प्रथममूलवर्ग ऋणमस्ति~। तृतीयखण्डे क्षेपगुणो द्वितीय-मूलवर्ग ऋणमस्ति~। अत्र लाघवान्मूलवर्गयोगः क्षेपगुण ऋणमिति
\end{sloppypar}

\newpage

\begin{sloppypar}
न्यासः\textendash \,प्रमूव ० द्विमूव १ मूवयो ० क्षे १ं क्षेव १\\

अत्राद्यखण्डे मूलवर्गघातोऽस्ति~। य एव मूलवर्गघातः स एव मूलघातवर्ग इति तथा न्यासः\textendash \,मूघाव १ मूवयो ० क्षे १ं क्षेव १~।\\

अत्र द्वितीयखण्डे मूलवर्गयोगः क्षेपगुणोऽस्ति~। तत्र मूलवर्गयोगस्य खण्डद्वयम्~। एकं मूलान्तरवर्गः~। अपरं मूलयोर्द्विघ्नो घातः~। {\color{violet}'राश्योरन्तरवर्गेण द्विघ्ने घाते युते तयोः~। वर्गयोगो भवेत्~।'} इत्युक्तत्वात्~। अतो जाते वर्गयोगस्य खण्डे मूअंव १ मूघा २~। अनयोः क्षेपेण गुणने जातं द्वितीयखण्डं खण्डद्वयात्मकं मूअंव ० क्षे १ं मूघा ० क्षे २ं~। सर्वेषां खण्डानां क्रमेण न्यासः\textendash \,मूघाव १ मूअंव ० क्षे १ं मूघा ० क्षे २ं क्षेव १~।\\

अयं हि राशिवधः~। अयं येन युतः सन्मूलदः स्यात्स एव वधक्षेपः~। अत्र क्षेपगुणे मूलान्तरवर्गे क्षिप्ते शेषस्यास्य\textendash \,मूघाव १ मूधा ० क्षे २ं क्षेव १~।\textendash \,\hyperref[3.31]{'कृतिभ्य आदाय पदानि'} इत्यादिना पदमिदमायाति~। मूघा १ क्षे १ं~। इदं हि वधमूलम्~। अत उपपन्नं मूलयोरन्तरवर्गेण हतो राशिक्षेपो वधक्षेपो भवतीति~। राशिमूलानां यथासन्नं द्वयोर्द्वयोर्वधा राशिक्षेपोना वधमूलानि भवन्तीत्यपि~। अनयैव युक्त्या द्वितीयतृतीययोस्तृतीयचतुर्थयोरपि राश्योर्वधमूलोपपत्तिर्द्रष्टव्या~। एवमेकराशिमूलं यावत्तावत्प्रकल्प्य ततः सर्वमूलसिद्धिरुक्ता~। अथ प्रकृतोदाहरणे योजयति~। अत्रोदाहरणे राशिक्षेपाद्वधक्षेपो नवगुणो नवानां च मूलं त्रयोऽतस्त्र्युत्तराणि राशिमूलानीत्यादिना~। शेषं स्पष्टम्~॥~१२७~॥\\

{\small अथान्यदुदाहरणमनुष्टुभाह\textendash }

\phantomsection \label{8.128}
\begin{quote}
{\large \textbf{{\color{purple}क्षेत्रे तिथिनखैस्तुल्ये दोःकोटी तत्र का श्रुतिः~।\\
उपपत्तिश्च रूढस्य गणितस्यास्य कथ्यताम्~॥~१२८~॥}}}
\end{quote}

{\color{violet}'तत्कृत्योर्योगपदं कर्णः'} इति रूढस्य प्रसिद्धस्य गणितस्योपपत्तिः कथ्यताम्~। उपपत्तिम् एव प्रष्टुमत्र श्रुतिप्रश्नो द्रष्टव्यः~। अत्र कर्णः या १~। क्षेत्रदर्शनम्

\begin{center}
\includegraphics[scale=0.5]{Diagram_p2pg47.png}
\end{center}
\end{sloppypar}

\newpage

\begin{sloppypar}
अत्र कर्णस्य भूमित्वकल्पने दर्शनम्

\begin{center}
\includegraphics[scale=0.5]{Diagram_p2pg48-1.png}
\end{center}

क्षेत्रं परिवर्त्य दर्शनम्

\begin{center}
\includegraphics[scale=0.5]{Diagram_p2pg48-2.png}
\end{center}

अत्र लम्बादुभयतो ये त्र्यस्रे तयोरपि भुजकोटी पूर्वभुजकोट्यनुरूपे भवतः~। तत्र भुजाश्रिताबाधा भुजो लम्बः कोटिः पूर्वभुजः कर्ण इत्येकं त्र्यस्रम्~। लम्बो भुजो द्वितीयाबाधा कोटिः पूर्वकोटिः २० कर्ण इत्यपरम्~। नन्वत्र त्र्यस्रद्वयेऽपि लम्ब एव कथं न कोटिः~। सत्यम्~। दोःकोट्योर्नामभेदो न स्वरूपभेद इति यद्यप्यस्ति तथापि प्रकृते भुजकोट्योः पूर्वभुजकोट्यनुरूपत्वविवक्षया न तथा~। पूर्वं हि भुजात् कोटिर्महतीति प्रकृतेऽपि तथैव भाव्यम्~। किं च प्रकृतभुजकोट्योः पूर्वभुजकोट्यनुरूपत्वे विवक्षिते सति भुजतुल्ये कर्णे यदि लम्बः कोटिस्तदा कोटितुल्ये कर्णे केति त्रैराशिकेन कोटिभेदेन भाव्यम्~। यद्वा परस्परस्पर्धिदिशोर्भुजयोरेकतरस्य कोटिरिति सञ्ज्ञा स्वेच्छया क्रियताम्~। परं यावत्तावति कर्णे यदि विंशतिमिता कोटिस्तदा विंशतिमिते कर्णे का कोटिरिति त्रैराशिकेन विंशतिमिते कर्णे परस्परस्पर्धिदिशोर्भुजयोर्मध्ये महानेव भुज आबाधारूपः सिध्येन्न लम्बरूपो लघुभुजः~। प्रमाणभुजस्य महत्त्वात्~। एवं यावत्तावति कर्णे यदि पञ्चदशमितो भुजस्तर्हि पञ्चदशमिते कर्णे को भुज इति पञ्चदशमिते कर्णे परस्परस्पर्धिदिशोर्भुजयोर्मध्ये लघुरेव भुज आबाधारूपः सिध्येन्न तु लम्बरूपो महान्भुजः~। प्रमाणभुजस्य लघुत्वात्~। तदेवं यत्र कुत्रापि जात्ये त्र्यस्रे यदि यावत्तावत्कर्णो भूः कल्प्यते तर्हि यावत्तावति कर्णे भुजो भुजस्तदा भुजतुल्ये कर्णे क इति त्रैराशिकेन या १~। भु १~। भु १ भुजाश्रिताबाधा सिध्येत् \;{\small $\begin{matrix}
\mbox{{भुव ~१}}\\
\vspace{-1mm}
\mbox{{~या ~१}}
\vspace{1mm}
\end{matrix}$}~। एवं यावत्तावति कर्णे यदि कोटिः कोटिस्तदा कोटितुल्ये कर्णे केति त्रैराशिकेन या १~। को १~। को १~। कोट्याश्रिताबाधा सिध्येत् \;{\small $\begin{matrix}
\mbox{{कोव ~१}}\\
\vspace{-1mm}
\mbox{{~या ~~१}}
\vspace{1mm}
\end{matrix}$}~। आबाधयोर्योगोऽयं \;{\small $\begin{matrix}
\mbox{{भुव ~१ ~कोव ~१}}\\
\vspace{-1mm}
\mbox{{~~~~~~~~~~ या ~~१}}
\vspace{1mm}
\end{matrix}$}~। अयं भूम्यानया या १ सम इति पक्षौ समच्छेदीकृत्य

\end{sloppypar}

\newpage

\begin{sloppypar}
\noindent च्छेदगमे जातौ \;{\small $\begin{matrix}
\mbox{{~~~~~~~~~~ याव ~१}}\\
\vspace{-1mm}
\mbox{{भुव ~१ ~कोव ~१}}
\vspace{1mm}
\end{matrix}$}~। अत्र पक्षयोः समत्वाद्य एव यावद्वर्गः स एव भुजकोटि-वर्गयोग इति सिद्धम्~। प्रकृते कर्णो यावत्तावदात्मकोऽस्तीति यावत्तावद्वर्गः कर्णवर्ग एव~। तस्मात्सिद्धं य एव कर्णवर्गः स एव भुजकोटिवर्गयोग इति~। अतोऽस्य पदं कर्णो भवितुम् अर्हति~। अत उपपन्नं तत्कृत्योर्योगपदं कर्ण इति~। \\

अथवान्यथोपपत्तिः~। उद्दिष्टक्षेत्रमिदम्

\begin{center}
    \includegraphics[scale=0.4]{Diagram_p2pg49-1.png}
\end{center}

कर्णो यथा बहिर्भवति तथैतत्सममन्यत्क्षेत्रं योज्यते~। दर्शनम्

\begin{center}
    \includegraphics[scale=0.4]{Diagram_p2pg49-2.png}
\end{center}

अथैवमेव तृतीयक्षेत्रं योज्यते

\begin{center}
    \includegraphics[scale=0.4]{Diagram_p2pg49-3.png}
\end{center}

एवं चतुर्थक्षेत्रयोगे दर्शनम्

\begin{center}
    \includegraphics[scale=0.4]{Diagram_p2pg49-4.png}
\end{center}
\end{sloppypar}

\newpage

\begin{sloppypar}
एवं समजात्यचतुष्टयेन तद्भुजकोट्यन्तरसमचतुर्भुजेन समकर्णेन पञ्चमेन चेति पञ्चभिः क्षेत्रैरेकं समकर्णं समचतुर्भुजं क्षेत्रं भवति~।\\

यत्तु समजात्यचतुष्टयमात्रेण समचतुर्भुजं भवति तद्विषमकर्णमेव दर्शनम्~।

\begin{center}
    \includegraphics[scale=0.5]{Diagram_p2pg50.png}
\end{center}

अत्र द्विगुणो भुज एकः कर्णो द्विगुणा कोटिरपरः~। यत्र तु भुजकोट्योः समत्वं तत्रान्तरा-भावात्प्रकारद्वयेनापि क्षेत्रचतुष्टयमात्रेण समकर्णं भवति~। अथ प्रकृते समकर्णे विषमकर्णे च समचतुर्भुजे त्र्यस्रकर्णतुल्या एव भुजाः~। परं समकर्णे चतुर्भुजे भुजकोट्यन्तरसमचतुर्भुजं क्षेत्रमधिकमस्ति~। अत एव भुजसमत्वेऽपि यथा यथा कर्णवैषम्यं भवति तथा तथा क्षेत्रसङ्कोचात्क्षेत्रफलमल्पं भवतीति प्रतिपादितमाचार्यैर्लीलावत्याम्~।\\

अथ प्रकृतम् अनुसरामः~। अत्र समकर्णे समचतुर्भुजे क्षेत्रे समश्रुतौ तुल्यचतुर्भुजे च {\color{violet}'तथायते तद्भुजकोटिघातः'} इत्यनेन भुजकोटिघातः फलं भवति~। अत्र भुजकोट्योः समतया भुजकोटिघातः समद्विघातो भवतीति भुजवर्ग एव क्षेत्रफलम्~। अतः क्षेत्रफले ज्ञाते सति तन्मूलं भुजमानं स्यात्~। चतुर्भुजे यो भुजः स एव त्र्यस्त्रे कर्णोऽस्तीति कर्णोऽपि ज्ञातः स्यात्~। अतः क्षेत्रफलं खण्डैः साध्यते~। तत्र त्र्यस्त्रे भुजकोटिघातार्धं फलं भवतीति जातमेकस्मिंस्त्र्यस्रे क्षेत्रफलं भु ० को \;{\small $\begin{matrix}
\mbox{{१}}\\
\vspace{-1mm}
\mbox{{२}}
\vspace{1mm}
\end{matrix}$}~। इदं चतुर्गुणं सत् त्र्यस्त्रचतुष्टयस्य फलं स्यादिति जातं भु ० को २~। अथ भुजकोट्यन्तरसमचतुर्भुजस्य समकर्णस्य क्षेत्रफलस्योक्तयुक्त्या भुजकोट्यन्तरवर्गः फलं स्यात्~। तत्र भुजकोट्यन्तरमिदं भु १ं को १~। अस्य वर्गः {\color{violet}'स्थाप्योऽन्त्यवर्गः'} इत्यादिना~। यद्वा खण्डगुणनेन जातः~। ~~भुव १ भु ० को २ं कोव १~।\\

इदमन्तर्लघुचतुर्भुजस्य क्षेत्रस्य फलं त्र्यस्रचतुष्टयफलेनानेन भु ० को २ युतं सज्जातं प्रकृतचतुर्भुजस्य फलं भुव १ कोव १~। एवं भुजकोट्योर्द्विघ्नो घातो भुजकोट्यन्तरवर्गेण युतः सन्भुजकोटिवर्गयोगो भवति~॥~१२८~॥\\

{\small एतदेवाहानुष्टुभा\textendash }
\end{sloppypar}

\newpage

\begin{sloppypar}
\phantomsection \label{8.129}
\begin{quote}
{\large \textbf{{\color{purple}दोःकोट्यन्तरवर्गेण द्विघ्नो घातः समन्वितः~।\\
वर्गयोगसमः स स्याद्द्वयोरव्यक्तयोर्यथा~॥~१२९~॥}}}
\end{quote}

तत्र दोःकोटी उपलक्षणम्~। अत एव पाट्यामुक्तम्\textendash \,राश्योरन्तरवर्गेणेत्यादि~। द्वयोः अव्यक्तयोर्यथेति~। राशी या १ का १~। अनयोरन्तरवर्गः ~याव १ याकाभा २ं काव १~।\\

अस्य द्विघ्नघातेनानेन याकाभा २ योगे जातो वर्गयोग एव याव १ काव १~। अथवा तान्येव क्षेत्रखण्डान्यन्यथा विन्यस्य क्षेत्रफलं साध्यते~। यथा\textendash \\

अत्र लघुचतुर्भुजस्य बाह्यभुजेन स्वमार्गवृद्धेन क्षेत्रं विच्छिद्य दर्शनम्

\begin{center}
    \includegraphics[scale=0.5]{Diagram_p2pg51-1.png}
\end{center}

अतो मध्ये रेखामपनीय दर्शनम्~।

\begin{center}
    \includegraphics[scale=0.5]{Diagram_p2pg51-2.png}
\end{center}

एवं जातं समचतुर्भुजद्वयम्~। एकं कोटितुल्यचतुर्भुजमपरं भुजतुल्यचतुर्भुजम्~। द्वयमपि समकर्णम्~। अत उक्तवदेकत्र कोटिवर्गः क्षेत्रफलमपरत्र भुजवर्गः क्षेत्रफलमित्युभयोर्योगे जातो भुजकोटिवर्गयोगः प्रथमचतुर्भुजे क्षेत्रफलम्~। अस्य पदं चतुर्भुजे भुजः स्यात्~। स एव त्र्यस्रे कर्ण इति वोपपन्नं {\color{violet}'तत्कृत्योर्योगपदं कर्णः'} इति~॥~१२९~॥\\

{\small अथान्यदुदाहरणमनुष्टुभाह\textendash }

\phantomsection \label{8.130}
\begin{quote}
{\large \textbf{{\color{purple}भुजात्त्र्यूनात्पदं व्येकं कोटिकर्णान्तरं सखे~।\\
यत्र तत्र वद क्षेत्रे दोःकोटिश्रवणान्मम~॥~१३०~॥}}}
\end{quote}

स्पष्टोऽर्थः~। अत्र कोटिकर्णान्तरमिष्टं कल्पितम् २~। भुजात्त्र्यूनात्पदं व्येकं सत्कोटि-कर्णान्तरं भवत्यतो विलोमविधिना कोटिकर्णान्तरं २ सैकं ३ वर्गितं ९ त्रियुतं जातो
\end{sloppypar}

\newpage

\begin{sloppypar}
\noindent भुजः १२~। अस्य वर्गः कोटिकर्णयोर्वर्गान्तरम् १४४~। यतो भुजकोट्योर्वर्गयोगः कर्ण-वर्गोऽस्त्यतः कर्णवर्गात्कोटिवर्गेऽपनीते भुजवर्ग एवावशिष्यते~। अतो योऽयं भुजवर्गः १४४ तत्कोटिकर्णयोर्वर्गान्तरं कल्पितकोटिकर्णान्तरमिदम् २~। अतो वर्गान्तरं राशिवियोगभक्तं योग इति जातः कोटिकर्णयोगः ७२~। {\color{violet}'योगान्तराभ्यां योगोऽन्तरेणोनयुतोऽर्द्धितः'} इति सङ्क्रमणसूत्रेण जातौ कोटिकर्णौ ३५।३७ एवं कोटिकर्णान्तरमेकं १ प्रकल्प्योक्तवज्जाता भुजकोटिकर्णाः ७।२४।२५~। एवमनेकधा~। अथ {\color{violet}'वर्गान्तरं राशिवियोगभक्तं योगः'} इति अत्रोपपत्तिः~। वर्गान्तरं हि योगान्तरघातोऽस्ति~। अतोऽस्मिन्नन्तरेण भक्ते योगो लभ्येतैव~। योगेन वा भक्तेऽन्तरं लभ्येतेति किं चित्रम्~। वर्गान्तरं योगान्तरघातोऽस्तीत्यत्र का युक्तिरिति चेच्छृणु~। समकर्णे समचतुर्भुजे क्षेत्रे भुजवर्ग एव क्षेत्रफलं भवति~। अत उक्तविधक्षेत्रे भुज-तुल्यो राशिः क्षेत्रफलतुल्यस्तद्वर्गश्च~। यथा राशी ७।५~। अनयोरुक्तवद्वर्गौ

\begin{center}
    \includegraphics[scale=0.4]{Diagram_p2pg52-1.png}
\end{center}

\noindent सप्तवर्गात्पञ्चवर्गं विशोध्यम्~। इदं वर्गान्तरम्

\begin{center}
    \includegraphics[scale=0.5]{Diagram_p2pg52-2.png}
\end{center}

\noindent अत्र पार्श्वद्वयेऽपि क्षेत्रशेषस्य विस्तारो भुजान्तरतुल्य एव स्यात्~। भुजावेव राशी इति राश्यन्तरतुल्य एव विस्तारः स्यात्~। दैर्घ्यं त्वेकतरपार्श्वे बृहद्भुजतुल्यमन्यस्मिन्पार्श्वे लघुभुज-तुल्यं यथैवं
\begin{center}
    \includegraphics[scale=0.4]{Diagram_p2pg52-3.png}
\end{center}
\end{sloppypar}

\newpage

\begin{sloppypar}
एवं वा

\begin{center}
    \includegraphics[scale=0.4]{Diagram_p2pg53-1.png}
\end{center}

अनर्योगे जातं क्षेत्रशेषमेवम्

\begin{center}
    \includegraphics[scale=0.45]{Diagram_p2pg53-2.png}
\end{center}

अस्य क्षेत्रस्य राशियोगतुल्यं दैर्घ्यं राश्यन्तरतुल्यो विस्तारश्चायते भुजकोटिघातः फलम् इति योगान्तरघातस्य फलम्~। इदं क्षेत्रशेषं हि पूर्वकल्पितराश्योर्वर्गान्तरं योगान्तरघात-रूपमुपपन्नम्~॥~१३०~॥\\

{\small अथ वक्ष्यमाणोदाहरणोपयुक्तमन्यदनुष्टुब्द्वयेनाह\textendash }

\phantomsection \label{8.131}
\begin{quote}
{\large \textbf{{\color{purple}वर्गयोगस्य यद्राश्योर्युतिवर्गस्य चान्तरम्~।\\
द्विघ्नघातसमानं स्याद्द्वयोरव्यक्तयोर्यथा~॥\\
चतुर्गुणस्य घातस्य युतिवर्गस्य चान्तरम्~।\\
राश्यन्तरकृतेस्तुल्यं द्वयोरव्यक्तयोर्यथा~॥~१३१~॥}}}
\end{quote}

अत्र प्रथमसूत्रे वर्गयोगस्य युतिवर्गस्य चान्तरे कृते द्विघ्नो घातो भवतीति प्रतिपादितम्~। तत्र युक्तिर्द्वयोरव्यक्तयोर्यथेति~। यथा राशी या १ का १~। अनयोर्वर्गयोगोऽयं याव १ काव १~। युतिवर्गोऽयं याव १ याकाभा २ काव १~। वर्गयोगयुतिवर्गयोरन्तरमिदं याकाभा २~। राश्योर्द्विघ्नघातोऽस्ति~। पूर्ववत्क्षेत्रद्वारा वा युक्तिः~। यथा राशी ३।५~। अनयोर्वर्गौ~। \\

युतिर्वर्गोऽयम्
\vspace{-1mm}

\begin{center}
    \includegraphics[scale=0.5]{Diagram_p2pg53-3.png}
\end{center}
\end{sloppypar}

\newpage

\begin{sloppypar}
युतिवर्गाद्वर्गद्वयं विशोध्य शेषम्

\begin{center}
    \includegraphics[scale=0.5]{Diagram_p2pg54.png}
\end{center}

अत्र क्षेत्रशेषखण्डयोरेकराशितुल्यो विस्तारः~।\\

परराशितुल्यं दैर्घ्यमिति प्रत्येकं राशिघातः फलम्~। अत उपपन्नं 'वर्गयोगयुतिवर्गयोः अन्तरं द्विघ्नघातसमम्' इति~। द्वितीयसूत्रे तु चतुर्गुणस्य घातस्य युतिवर्गस्य चान्तरम्~। राश्यन्तरवर्गो भवति इति प्रतिपादितम्~। तत्र युक्तिर्द्वयोरव्यक्तयोर्यथेति~। यथा राशी या १ का १~। अनयोर्घातश्चतुर्गुणोऽयं याकाभा ४ युतिवर्गश्चायम्
\vspace{-2mm}
 
\begin{center}
याव ~१ ~याकाभा ~२ ~काव ~१~।
\end{center}
\vspace{-2mm}

युतिवर्गाच्चतुर्गुणघातेऽपनीते शेषमिदम्
\vspace{-2mm}

\begin{center}
याव ~१ ~याकाभा ~२ं ~काव ~१~।
\end{center}
\vspace{-2mm}

इदं राश्यन्तरवर्ग एव~। यद्वा क्षेत्रगतोपपत्तिः सा तु मूल एव स्फुटास्ति~॥~१३१~॥\\

{\small अथोदाहरणमनुष्टुभाह\textendash }

\phantomsection \label{8.132}
\begin{quote}
{\large \textbf{{\color{purple}चत्वारिंशद्युतिर्येषां दोःकोटिश्रवसां वद~।\\
भुजकोटिवधो येषु शतं विंशतिसंयुतम्~॥~१३२~॥}}}
\end{quote}

अत्र किल भुजकोटिवधोऽयं १२०~। अयं द्विघ्नः सन् २४० भुजकोटियुतिवर्गस्य भुज-कोटिवर्गयोगस्य चान्तरं स्यात्~। \hyperref[8.131]{'वर्गयोगस्य यद्राश्योर्युतिवर्गस्य चान्तरम्~। द्विघ्नघात-समानं स्यात्'} इत्युक्तत्वात्~। तत्र यो हि भुजकोटिवर्गयोगः स एव कर्णवर्गः~। अतो भुजकोटि-युतिवर्गस्य कर्णवर्गस्य चान्तरमिदम् २४०~। अत्र भुजकोटियुतिरेको राशिः~। कर्णोऽपरः~। अनयोर्वगान्तरमिदम् २४०~। तत्र योगान्तरघातसममित्युक्तत्वाद्भुजकोटियुतिकर्णयोगस्य भुजकोटियुतिकर्णान्तरस्य च घातो भवति २४०~। तत्र भुजकोटियुतिकर्णयोगस्तु त्रयाणां योगो भवति~। स चात्र चत्वारिंशन्मित उद्दिष्ट एवास्ति ४०~। अतोऽनेन योगेन ४० योगा-न्तरघातेऽस्मिन् २४० भक्ते लब्धं भुजकोटियुतिकर्णान्तरम् ६~। अथ योगान्तराभ्यामेताभ्यां ४०।६ सङ्क्रमणेन जातौ राशी २३।१७~। भुजकोटियुतिरेकः २३ कर्णोऽपरः १७~। अत्र लघु-राशिरेव कर्णो ज्ञेयः~। भुजकोटियुतितस्तस्याधिक्यासम्भवात्~। उक्तमप्याचार्यैः {\color{violet}लीलाव-त्याम्}~।
\end{sloppypar}

\newpage

\begin{sloppypar}
\begin{quote}
{\color{violet}'धृष्टोद्दिष्टमृजुभुजं क्षेत्रं यत्रैकबाहुतः स्वल्पा~।\\
तदितरभुजयुतिरथवा तुल्या ज्ञेयं तदक्षेत्रम्~॥'} इति
\end{quote}

अथ भुजकोटिवधे १२० चतुर्गुणे ४८० भुजकोटियुति\textendash \,२३\textendash \,वर्गादस्मात् ५२९ शोधिते शेषम् ४९~। इदं भुजकोट्योरन्तरवर्गः~। \hyperref[8.131]{'चतुर्गुणस्य घातस्य युतिवर्गस्य चान्तरम्~। राश्यन्तरकृतेस्तुल्यम्'} इत्युक्तत्वात्~। अतोऽस्य ४९ मूलं ७ भुजकोट्योरन्तरम्~। भुजकोटि-योगश्चायं २३~। आभ्यां सङ्क्रमणेन जाते भुजकोटी ८।१५~॥~१३२~॥\\

{\small अथान्यदुदाहरणमनुष्टुभाह\textendash }

\phantomsection \label{8.133}
\begin{quote}
{\large \textbf{{\color{purple}योगो दोःकोटिकर्णानां षट्पञ्चाश\textendash \,५६\textendash \,द्वधस्तथा~।\\
षट्शती सप्तभिः क्षुण्णा ४२०० येषां तान्मे पृथग्वद~॥~१३३~॥}}}
\end{quote}

स्पष्टोऽर्थः~। अत्र कर्णं यावत्तावन्मितं प्रकल्प्य गणितमाकरे स्फुटम्~॥~१३३~॥
\vspace{2mm}

\begin{quote}
{\color{violet}दैवज्ञवर्यगणसन्ततसेव्यपार्श्वबल्लालसञ्ज्ञगणकात्मजनिर्मितेऽस्मिन्~।\\
बीजक्रियाविवृतिकल्पलतावतारेऽभूदेकवर्णजसमीकरणं सभेदम्~॥}
\end{quote}
\vspace{-4mm}

\begin{center}
इति श्रीसकलगणकसार्वभौमश्रीबल्लालदेवज्ञसुतकृष्णगणकविरचिते~~~\\
बीजविवृतिकल्पलतावतारे निजभेदमध्यमाहरणसहितमेकवर्णसमीकरणम्~॥~८~॥
\vspace{2mm}

\rule{0.2\linewidth}{0.8pt}\\
\end{center}

अत्र खण्डयोर्ग्रन्थसङ्ख्ये ४९०।३२५~। एवमेकवर्णसमीकरणे ग्रन्थसङ्ख्या ८१५~। एवमादितो जाता ग्रन्थसङ्ख्या ३३९५~।

\begin{center}
\rule{0.2\linewidth}{0.8pt}\\
\vspace{-4mm}

\rule{0.2\linewidth}{0.8pt}
\end{center}

\end{sloppypar}

\newpage
\thispagestyle{empty}

\begin{center}
\textbf{\large ९\; अनेकवर्णसमीकरणम्~।}\\
\rule{0.2\linewidth}{0.8pt}
\end{center}

\begin{sloppypar}
{\small एवमनेकवर्णानामेकवर्णपूर्वकत्वादादावेकवर्णसमीकरणमुक्त्वेदानीं क्रमप्राप्तमनेकवर्णसमीक-रणं शालिन्योपजातिकाद्वयेन शालिनीपूर्वार्धेन चाह\textendash }

\phantomsection \label{9.134}
\begin{quote}
{\large \textbf{{\color{purple}आद्यं वर्णं शोधयेदन्यपक्षादन्यान्रूपाण्यन्यतश्चाद्यभक्ते~।\\
पक्षेऽन्यस्मिन्नाद्यवर्णोन्मितिः स्याद्वर्णस्यैकस्योन्मितीनां बहुत्वे~॥\\
समीकृतच्छेदगमे तु ताभ्यस्तदन्यवर्णोन्मितयः प्रसाध्याः~।\\
अन्त्योन्मितौ कुट्टविधेर्गुणाप्ती ते भाज्यतद्भाजकवर्णमाने~।\\
अन्येऽपि भाज्ये यदि सन्ति वर्णास्तन्मानमिष्टं परिकल्प्य साध्ये~।\\
विलोमकोत्थापनतोऽन्यवर्णमानानि भिन्नं यदि मानमेवम्~।\\
भूयः कार्यः कुट्टकोऽत्रान्त्यवर्णं तेनोत्थाप्योत्थापयेद्व्यस्तमाद्यात्~॥~१३४~॥}}}
\end{quote}

अस्मिञ्छालिनीपूर्वार्धेऽन्यत्पाठद्वयं दृश्यते~। 'भूयः कार्यः कुट्टकादन्यवर्णः~।' इति~। 'भूयः कार्यः कुट्टकादन्यवर्णस्तेनोत्थाप्योत्थापयेदन्तिमाद्यान्' इति च~। एतानि सूत्राण्याचार्यैः एव सम्यग्व्याख्यातानीति नास्माभिर्व्याक्रियन्ते~।\\

अथ शालिनीपूर्वार्धे व्याख्या~। यद्युत्थापने कृतेऽन्यवर्णमानं भिन्नं लभ्यते तदात्र भूयः कुट्टकः कार्यः~। तेन कुट्टकेनान्त्यवर्णम् उत्थाप्याद्याद्व्यस्तमुत्थापयेत्~। कुट्टको गुणविशेष इति प्रागेव निरूपितम्~। तेन कुट्टकेन सक्षेपेण गुणेनान्त्ययोरन्त्येषु वा वर्णमानेषु यो वर्णस्तमुत्थाप्याद्याद्व्यस्तं पुनरुत्थापयेत्~। यस्योन्मानस्य पूर्वमुत्थापने भिन्नं मानमभवत् तदुन्मानमाद्यम्~। तत आरभ्य पुनरपि विलोमोत्थापनं कर्तव्यमित्यर्थः~। अयमेव पाठो मुख्यः~। आचार्यैः सूत्रविवरणावसरेऽस्यैव विवरणात्~। तद्विवरणं यथा\textendash \,'अथ यदि विलोमोत्थापने क्रियमाणे पूर्ववर्णोन्मितौ तन्मितिर्भिन्ना लभ्यते तदा कुट्टकविधिना यो गुणः सक्षेप उत्पद्यते स भाज्यवर्णस्य मानम्~। तेनान्त्यवर्णमानेषु तं वर्णमुत्थाप्य पूर्वोन्मितिषु विलोमोत्थापन-प्रकारेणान्यमानानि' इति~। 'भूयः कार्यः कुट्टकादन्यवर्णः' इति पाठस्त्वसाधुः~। उत्थाप्य इति पदस्यानन्वयात्~। अत्र यद्यप्याद्यमित्यध्याहृत्याद्यमुत्थाप्येति तदन्वयः स्यात्तथाप्यन्त्य-वर्णोत्थापनस्यानुक्तेर्न्यूनतादोषः स्यादेव~। अथ यदि न्यूनतादोषपरिहारार्थमन्त्यवर्णमिति अध्याह्रियते तथा सति तेनान्त्यवर्णेनान्त्यवर्णमुत्थाप्येति तदन्वयः स्यात्~। इह हि यद्यन्य-वर्णमानं भिन्नं स्यात्तदा भूयः कुट्टकादन्यवर्णः कार्य इत्युक्तेरन्यवर्णो भाजकवर्ण एव~। एवं सति भाजकवर्णमानेनान्त्यवर्णमुत्थाप्येत्यर्थः पर्यवस्यति~। न चासौ युक्तः~। भाजकवर्णान्त्य-वर्णयोर्भेदात्~। किं तु
\end{sloppypar}

\newpage

\begin{sloppypar}
\noindent भाज्यवर्णस्यान्त्यवर्णस्य चाभेदाद्भाज्यवर्णमानेनैवान्त्यवर्णोत्थापनं युक्तम्~। तदेवं द्वितीय-पाठो न साधुः~। एवं तृतीयपाठोऽप्यसाधुः~। \\

अथैतस्यार्थस्य स्पष्टत्वार्थं वक्ष्यमाणमुदाहरणं लिख्यते~।

\begin{quote}
\hyperref[9.139]{'षड्भक्तः पञ्चाग्रः पञ्चविभक्तो भवेच्चतुष्काग्रः~।\\
चतुरुद्धृतस्त्रिकाग्रो द्व्यग्रस्त्रिसमुद्धृतः कः स्यात्~॥'} इति~।
\end{quote}

अत्र राशिः या १~। अयं \hyperref[9.139]{'षड्भक्तः पञ्चाग्रः'} इति लब्धिप्रमाणं कालकं प्रकल्प्य कालक-गुणितो हरः स्वाग्रेण पञ्चकेन युक्तः का ६ रू ५~। यावत्तावत्सम इति साम्यकरणेन जाता यावत्तावदुन्मितिः \;{\small $\begin{matrix}
\mbox{{का ~६ ~रू ~५~।}}\\
\vspace{-1mm}
\mbox{{~~~~~~~~~ या ~१~।}}
\vspace{1mm}
\end{matrix}$} \\

\noindent एवं पञ्चादिहरेषु नीलकादयो लभ्यन्त इति जाता यावत्तावदुन्मितयः \\
 
{\small $\begin{matrix}
\mbox{{नी ~५ ~रू ~४}}\\
\vspace{-1mm}
\mbox{{~~~~~~~ या ~१}}
\vspace{1mm}
\end{matrix}$}~। \hspace{4mm} {\small $\begin{matrix}
\mbox{{पी ~४ ~रू ~३}}\\
\vspace{-1mm}
\mbox{{~~~~~~~ या ~१}}
\vspace{1mm}
\end{matrix}$}~। \hspace{4mm} {\small $\begin{matrix}
\mbox{{लो ~३ ~रू ~२}}\\
\vspace{-1mm}
\mbox{{~~~~~~~ या ~१}}
\vspace{1mm}
\end{matrix}$}~।\\

\noindent आसां प्रथमद्वितीययोः समीकरणेन लब्धा कालकोन्मितिः \;{\small $\begin{matrix}
\mbox{{नी ~५ ~रू ~१ं~।}}\\
\vspace{-1mm}
\mbox{{~~~~~~~ का ~६~।}}
\vspace{1mm}
\end{matrix}$} \\

\noindent द्वितीयतृतीययोर्नीलकोन्मितिः \;{\small $\begin{matrix}
\mbox{{पी ~४ ~रू ~१ं}}\\
\vspace{-1mm}
\mbox{{~~~~~~~ नी ~५}}
\vspace{1mm}
\end{matrix}$}~।\\

\noindent एवं तृतीयचतुर्थयोः पीतकोन्मितिः \;{\small $\begin{matrix}
\mbox{{लो ~३ ~रू ~१}}\\
\vspace{-1mm}
\mbox{{~~~~~~~ पी ~४}}
\vspace{1mm}
\end{matrix}$}~।\\

\noindent इयमन्त्या~। \hyperref[9.134]{'अन्त्योन्मितौ कुट्टविधेर्गुणाप्ती'} इत्यादिना जाते लोहितपीतकयोर्माने सक्षेपे \\
\vspace{-2mm}

{\small $\begin{matrix}
\mbox{{ह ~३ ~रू ~२ ~पी}}\\
\vspace{-1mm}
\mbox{{ह ~४ ~रू ~३ ~लो}}
\vspace{1mm}
\end{matrix}$}~।\\

\noindent अथ नीलकोन्मानमिदम् \;{\small $\begin{matrix}
\mbox{{पी ~४ ~रू ~१ं}}\\
\vspace{-1mm}
\mbox{{~~~~~~~~ नी ~५}}
\vspace{1mm}
\end{matrix}$}~।

\end{sloppypar}

\newpage

\begin{sloppypar}
\noindent अत्र नीलकपञ्चकस्य रूपोनपीतकचतुष्टयं मानमस्ति~। तत्र पीतकस्य कुट्टकसिद्धं मानमिदं ह ३ रू २~। अतो यद्येकस्य पीतकस्येदं मानं तदा पीतकचतुष्टयस्य किमिति पी १~। ह ३ रू २~। पी ४ त्रैराशिकेन जातं पीतकचतुष्टयस्य मानं ह १२ रू ८~। इदं रूपोनं जातं नीलकपञ्चकस्य मानं ह १२ रू ७~। यदि नीलकपञ्चकस्येदं तदैकस्य नीलकस्य किमिति नी ५~। ह १२ रू ७~। नी १~। त्रैराशिकेन नीलकस्य मानं भिन्नं लभ्यते ~ह \,{\small $\begin{matrix}
\mbox{{१२}}\\
\vspace{-1mm}
\mbox{{५}}
\vspace{1mm}
\end{matrix}$}\, रू ७~॥\\
\vspace{-2mm}

\noindent अतोऽत्र भूयः कुट्टकः कार्यः~। कुट्टको गुणकविशेषः~। स चोक्तविधिना जातः सक्षेपः~श्वे ५ रू ४~। \hyperref[9.134]{'गुणाप्ती ते भाज्यतद्भाजकवर्णमाने'} इत्युक्तत्वादसौ कुट्टको श्वे ५ रू ४ भाज्यवर्णस्य हरितकस्य मानं श्वे ५ रू ४ ह~। अनेनान्त्ययोः पीतकलोहितकमानयोरनयोः \;{\small $\begin{matrix}
\mbox{{ह ~३ ~रू ~२ ~पी}}\\
\vspace{-1mm}
\mbox{{ह ~४ ~रु ~३ ~लो}}
\vspace{1mm}
\end{matrix}$}~। वर्णं हरितकमुत्थाप्याद्याद्व्यस्तं पुनरुत्थापयेत्~। तदुत्थापनं यथा\textendash \,इह हरितकत्रयं रूपद्वययुतमेकस्य पीतकस्य मानमस्ति~। हरितकमानं च कुट्टकसिद्धमिदं श्वे ५ रू ४~। यद्येकस्य हरितकस्येदं मानं तर्हि हरितकत्रयस्य किमिति ह १~। श्वे ५ रू ४~। ह ३ त्रैराशिकेन जातं हरितकत्रयमानं श्वे १५ रु १२~। इदं रूपद्वययुतं जातं पीतकमानं श्वे १५ रू १४ पी~। अनयैव युक्त्या लोहितकमानमपि श्वे २० रू १९ लो~। एवं जाते पीतकलोहितकयोर्माने \;{\small $\begin{matrix}
\mbox{{श्वे ~१५ ~रू ~१४ ~पी~।}}\\
\vspace{-1mm}
\mbox{{श्वे ~२० ~रू ~१९ ~लो~।}}
\vspace{1mm}
\end{matrix}$}\; एवं जातं कुट्टकेनान्त्यवर्णोत्थापनम्~।\\
\vspace{1mm}

अथ लोहितपीतकयोराद्यान्नीलकादारभ्य व्यस्तम् उत्थापयेत्~। तत्र नीलकमानम् इदं \;{\small $\begin{matrix}
\mbox{{पी ~४ ~रू ~१ं~।}}\\
\vspace{-1mm}
\mbox{{~~~~~~~ नी ~५~।}}
\vspace{1mm}
\end{matrix}$}\; इह रूपोनं पीतकचतुष्टयं नीलकपञ्चकस्य मानमस्ति~। तत्र पीतकमानमिदं श्वे १५ रू १४ पी~। यद्येकस्य पीतकस्येदं तदा पीतकचतुष्टयस्य किमिति पी १~। श्वे १५ रू १४~। पी ४ त्रैराशिकेन पीतकचतुष्टयमानं श्वे ६० रू ५६~। इदं रूपोनं सज्जातं नीलकपञ्चकस्य मानं श्वे ६० रू ५५~। यदि नीलकपञ्चकस्येदं तदैकस्य किमिति नी ५~। श्वे ६० रू ५५~। नी १ त्रैराशिकेन जातं नीलकमानं श्वे १२ रू ११~। नी~। अथ नीलकादाद्यः कालकस्तस्य मानमिदं \;{\small $\begin{matrix}
\mbox{{नी ~५ ~रू ~१ं}}\\
\vspace{-1mm}
\mbox{{~~~~~~~ का ~६}}
\vspace{1mm}
\end{matrix}$}~। इह रूपोनं नीलकपञ्चकं कालकषट्कस्य मानमस्ति~। तत्र प्राग्वत्त्रैराशिकेन जातं नीलकपञ्चकमानं श्वे ६० रू ५५~। इदं रूपोनं सज्जातं कालकषट्क-
\end{sloppypar}

\newpage

\begin{sloppypar}
\noindent मानं श्वे ६० रू ५४~। अतोऽनुपाताज्जातमेकस्य कालकस्य मानं श्वे १० रू ९ का~। अथ कालकादाद्यो यावत्तावत्तस्य मानमिदं \;{\small $\begin{matrix}
\mbox{{का ६ रू ५}}\\
\vspace{-1mm}
\mbox{{या १ ~~~~~~}}
\vspace{1mm}
\end{matrix}$}~। इह कालकषट्कं रूपपञ्चकयुतं यावत्तावतो मानमस्ति~। तत्र कालकषट्कस्यानुपातसिद्धं मानमिदं श्वे ६० रू ५४~। इदं रूपपञ्चकयुतं जातं यावत्तावन्मानं श्वे ६० रू ५९ या~। एवमन्यास्वपि यावत्तावदुन्मितिषूत्थापनेनेदमेव मानं सिध्यति~। तदेवं जातानि यावत्तावदादीनां मानानि व्यक्ताव्यक्तानि \;{\small $\begin{matrix}
\mbox{{श्वे \,६० \,रू \,५९ \,या}}\\
\mbox{{श्वे \,१० \,रू ~\,९ \,का}}\\
\mbox{{श्वे \,१२ \,रू \,११ \,नी}}\\
\mbox{{श्वे \,१५ \,रू \,१४ \,पी}}\\
\vspace{-1mm}
\mbox{{श्वे \,२० \,रू \,१९ \,लो}}
\vspace{1mm}
\end{matrix}$}~। अत्र श्वेतकस्य शून्ये माने कल्पिते जातो राशिः ५९~। कालकादयस्तु षडादिभाजकानां लब्धयः कल्पिता अतस्तन्मानानि जाताः क्रमाल्लब्धयः ९।११।१४।१९~। एवं श्वेतकस्य मानं रूपमिष्टं प्रकल्प्य जातो राशिः ११९~। लब्धयश्च १९।२३।२९।३९~। एवमिष्टवशादानन्त्यम्~। अथोपपत्तिरुच्यते~। अत्र किल बहूनां मानान्यव्यक्तानि सन्ति~। तत्र पूर्वयुक्त्यैकस्मिन्पक्षे यद्येकमेवाव्यक्तं स्यादन्यत्र च रूपाण्येव स्युस्तदा तस्याव्यक्तस्य मानं सुबोधम्~। अतस्तथा यतितव्यं यथैकस्मिन्पक्ष एकमेवाव्यक्तं स्यात्समत्वाविरोधेन~। तत्र \hyperref[9.137]{'अश्वाः पञ्चगुणाङ्गमङ्गल-मिताः'} इति वक्ष्यमाणमुदाहरणमधिकृत्य युक्तिरुच्यते~। अत्राश्वादीनां मूल्यान्यज्ञातानीति यावत्तावदादीनि कल्पितानि या १~। का १~। नी १~। पी १~। अतोऽनुपातेन निजनिजाश्चा-दीनां धनान्येकीकृत्य जातानि चतुर्णां समधनानि \;{\small $\begin{matrix}
\mbox{{या \,५ \,का \,२ \,नी \,८ \,पी \,७}}\\
\mbox{{या \,३ \,का \,७ \,नी \,२ \,पी \,१}}\\
\mbox{{या \,६ \,का \,४ \,नी \,१ \,पी \,२}}\\
\vspace{-1mm}
\mbox{{या \,८ \,का \,१ \,नी \,३ \,पी \,१}}
\vspace{1mm}
\end{matrix}$}~। अत्र चतुर्णामपि धनानि समानीति प्रथमद्वितीयधने अपि समे एव~। अत्रैकपक्षे यथैकमेवाव्यक्तं भवति तथा यतितव्यम्~। तत्रैकतरपक्ष एकं वर्णं विहाय यदवशिष्यते तत्तुल्यं चेदुभयोः पक्षयोः शोध्येत तर्ह्येकस्मिन्पक्ष एकमेवाव्यक्तं स्यात्~। यं विहायावशिष्टं शोध्यते तस्मिन्पक्षे तस्यैव वर्णस्य शेषत्वात्~। तत्र कं वर्णमपहाय शेषं पक्षयोः शोध्यमिति यद्यपि नास्ति नियमः तथापि प्रथमातिक्रमे कारणाभावात्प्रथमवर्णमपहाय शेषं पक्षयोः शोध्यम्~। अथ प्रकृते प्रथमवर्णमपहाय शेषमिदं का २ नी ८ पी ७~। अस्मिन्पक्षयोः शोधिते जातमाद्यपक्षे या ५~। द्वितीयपक्षे तु जातं या ३ का ५ नी ६ं पी ६ं~। अस्ति
\end{sloppypar}

\newpage

\begin{sloppypar}
\noindent चानयोः समत्वम्~। समयोः समक्षेपे समशुद्धौ वा समत्वाहानेः~। तथा सति यदेव यावत्तावत् पञ्चकस्य मानं तदेव नीलकषट्कपीतकषट्करहितस्य यावत् त्रयकालकपञ्चयोगस्यापीति सिद्धम्~। तथा च यावत्पञ्चकस्य मानं ज्ञातुं यावत्त्रयस्यापि ज्ञानमपेक्षितम्~। तत्र यदि स्वमानज्ञाने स्वमानज्ञानापेक्षा स्यात्तदात्माश्रयात् कल्पकोटिशतैरपि मानज्ञानं न स्यात्~। अतः सा यथा न भवति तथा यतितव्यम्~। इतरपक्षे यः सजातीयो वर्णस्तत्तुल्यं पक्षयोः शोध्यम्~। प्रकृत इतरपक्षे सजातीयो वर्णोऽयं या ३~। एतस्मिन्पक्षयोः शोधिते जातम् आद्यपक्षे या २~। द्वितीयपक्षे का ५ नी ६ं पी ६ं~। एवं कृते यदेव यावत् तावद्द्वयस्य मानं तदेव नीलकषट्कपीतकषट्करहितस्य कालकपञ्चकस्य मानमिति नास्ति स्वमानज्ञाने स्वमानज्ञानापेक्षा~। अत उक्तं \hyperref[9.134]{'आद्यं वर्णं शोधयेदन्यपक्षादन्यान्रूपाण्यन्यतश्च'} इति~।\\

अथातस्त्रैराशिकम्~। यदि यावत्तावद्द्वयस्येदं मानं तदैकस्य यावत्तावतः किमिति या २~। का ५ नी ६ं पी ६ं~। या १ त्रैराशिकेन जातं यावत्तावन्मानं \;{\small $\begin{matrix}
\mbox{{का ~५ं ~नी ~६ं ~पी ~६}}\\
\vspace{-1mm}
\mbox{{\hspace{16mm} या ~२}}
\vspace{1mm}
\end{matrix}$}~। अत्र हरे याकारलिखनं यावत्तावन्मानमिदमित्युपस्थित्यर्थं न तु यावत्तावद्द्वयं हरः~। प्रमाणेच्छयोः यावत्तावतापवर्तनात्~। अनपवर्ते त्विच्छया गुणने क्रियमाणे भावितं स्यात्~। तदेवमुक्त-प्रकारेण प्रथमद्वितीययोर्द्वितीयतृतीययोस्तृतीयचतुर्थयोश्च धनयोः २ समशोधनेन जाताः प्रथमवर्णोन्मितयः\\
\vspace{-2mm}

\;{\small $\begin{matrix}
\mbox{{का ~५ ~नी ~६ं ~पी ~६}}\\
\vspace{-1mm}
\mbox{{\hspace{16mm} या ~२}}
\vspace{1mm}
\end{matrix}$} \hspace{6mm} \;{\small $\begin{matrix}
\mbox{{का ~३ ~नी ~१ ~पी ~१ं}}\\
\vspace{-1mm}
\mbox{{\hspace{16mm} या ~३}}
\vspace{1mm}
\end{matrix}$} \hspace{6mm} \;{\small $\begin{matrix}
\mbox{{का ~३ ~नी ~२ं ~पी ~१}}\\
\vspace{-1mm}
\mbox{{\hspace{16mm} या ~२}}
\vspace{1mm}
\end{matrix}$}~।\\

तदेतत् उक्तम् आद्यभक्ते पक्षेऽन्यस्मिन्नाद्यवर्णोन्मितिः स्यात् इति~। एवं प्रथमतृतीययोः प्रथमचतुर्थयोर्द्वितीयचतुर्थयोश्च समशोधनेनान्या अपि यावत्तावदुन्मितयः सम्भवन्ति~। परं प्रयोजनाभावान्न कृताः~। अथ भाज्यवर्णानां कालकादीनामिष्टानि मानानि प्रकल्प्यैक्यं कृत्वा यदि स्वहरेण ह्रियते तदा भिन्नमभिन्नं वा प्रथमवर्णमानं स्यात्~। इतरेषां तु कल्पितान्येव~। तथा सति सुखेनोद्दिष्टसिद्धिः~। अथ यद्यभिन्नमेव मानमपेक्षितं तर्हि यं कञ्चिदेकं वर्णं विहाय परेषां मानानीष्टानि कल्प्यानि~। तथा सति भाज्य एको वर्णः कानिचिद्रूपाणि च स्युः~। अथ तस्य वर्णस्य मानं तथेष्टं कल्प्यं यथा तेनेष्टेन गुणितो वर्णाङ्कस्तै रूपैर्युतो हरभक्तो निःशेषः स्यात्~। एवं कृते प्रथमवर्णमानमभिन्नमेव स्यात्~।\\

अथ तादृशस्येष्टस्य ज्ञानार्थमुपायः~। इह हि वर्णाङ्कः केन गुणितस्तै रूपैर्युतः स्वहरहृतो निःशेषः स्यादिति विचारः कुट्टके पर्यवस्यति~। अथ कुट्टकविधिना यो गुणः
\end{sloppypar}

\newpage

\begin{sloppypar}
\noindent स्यात्तेन गुणितो वर्णाङ्कस्तै रूपैर्युतः स्वहरभक्तो निःशेषः स्यादेवेति भाज्यवर्णस्य गुणतुल्ये माने कल्पिते भाजकवर्णस्य मानं लब्धितुल्यमभिन्नमेव स्यात्~। अत उक्तं \hyperref[9.134]{'कुट्टकविधेः गुणाप्ती ते भाज्यतद्भाजकवर्णमाने' 'अन्येऽपि भाज्ये यदि सन्ति वर्णास्तन्मानमिष्टं परिकल्प्य साध्ये'} इति~। अत्र भाज्यवर्णमानानां यदिष्टकल्पनम् उक्तं तत्तेषां माने नियते सत्येव ज्ञेयम्~। यदि तु केनापि प्रकारेण तन्मानं नियतं सिध्येत् तदा नियतेष्ट-कल्पनेन व्यभिचार एव स्यात्~। यथास्मिन्नेवोदाहरणे यथा चतुर्णां समधनत्वम् उद्दिष्टं तथा यदि \;{\small $\begin{matrix}
\mbox{{या ~५ ~का ~२ ~नी ~८ ~पी ~७~।}}\\
\vspace{-1mm}
\mbox{{या ~३ ~का ~७ ~नी ~२ ~पी ~१~।}}
\vspace{1mm}
\end{matrix}$}\; द्वयोरेवोद्दिष्टं स्यात् तदा तदुत्पन्नोन्मितौ \;{\small $\begin{matrix}
\mbox{{का ~५ ~नी ~६ं ~पी ~६ं}}\\
\vspace{-1mm}
\mbox{{\hspace{16mm} या ~२}}
\vspace{1mm}
\end{matrix}$}\; भाज्यवर्णमानानामनियतत्वात् तदिष्टकल्पनेनोद्दिष्टसिद्धिः स्यात्~। यथात्र कालकादीनाम् इष्टानि कल्पितानि ४।२।१~। एभ्यो जातं यावत्तावन्मानं १ जातान्यश्वादिमूल्यानि १।४।२।१~। यद्वा कल्पितानि ४।१।२~। जातं यावत् तावन्मानम् १~। जातान्यश्वादिमूल्यानि १।४।१।२~। यद्वा कल्पितानि ६।२।१~। जातं यावत् तावन्मानम् ६~। जातान्यश्वादिमूल्यानि ६।५।२१~। यद्वा कल्पितानि ६।१।२ जातानि मूल्यानि ६।६।२।१~। यद्वा कल्पितानि ४।१।१~। जातानि मूल्यानि ४।४।१।१~। एवमिष्टवशादनेकधा~। यदि त्वनयोर्धनयोरन्यधनेनापि समतोद्दिष्टा स्यात् तदा भाज्यवर्णमानानामिष्टकल्पने व्यभिचारः स्यादेव~। नह्यन्यधनानुरोधेन काचित् क्रियात्र कृतास्ति येनान्यधनसमता सिध्येत्~। यदि तु तदनुरोधिक्रियां विनापि तत्समता सिध्येत्तदा किमीदृग्धनं स्याद्यत्समं न स्यात्~। तस्मादेतादृश्युदाहरणे भाज्यवर्णमानानाम् इष्टकल्पनं न युक्तम्~। किं तु नियतम् एव तन्मानं साध्यम्~। तच्च यथा समपक्षेभ्य आद्यवर्णमानं साधितं तथा भाज्याद्यवर्णस्यापि साध्यम्~। अत उक्तं \hyperref[9.134]{'वर्णस्यैकस्योन्मितीनां बहुत्वे समीकृतच्छेदगमे तु ताभ्यस्तदन्यवर्णोन्मितयः प्रसाध्याः'} इति~। अत्र बहुत्वम् अनेकत्वम्~। उन्मितिद्वयादप्यन्यवर्णोन्मितिसम्भवात्~। समीकृतच्छेदगम इत्यत्रोपपत्तिः त्वेकवर्णसमीकरण आचार्येणैव स्पष्टीकृता~। अथ प्रकृतोदाहरणे यावत्तावत् उन्मितयः {\small $\begin{matrix}
\mbox{{का ~५ ~नी ~६ं ~पी ~६ं}}\\
\vspace{-1mm}
\mbox{{\hspace{16mm} या ~२}}
\vspace{1mm}
\end{matrix}$}~। \hspace{1mm} {\small $\begin{matrix}
\mbox{{का ~३ ~नी ~१ ~पी ~१ं}}\\
\vspace{-1mm}
\mbox{{\hspace{16mm} या ~३}}
\vspace{1mm}
\end{matrix}$}~। \hspace{1mm} {\small $\begin{matrix}
\mbox{{का ~३ ~नी ~२ं ~पी ~१}}\\
\vspace{-1mm}
\mbox{{\hspace{16mm} या ~२}}
\vspace{1mm}
\end{matrix}$}~। अत्र हरे याकारस्या-वास्तवत्वात्~। {\color{violet}'अन्योन्यहाराभिहतौ हरांशौ'} इत्यादिना भावितं न भवति~। याकारस्य वास्तवत्वेऽपि हराभ्यामपवर्तिताभ्यां यद्वा हरांशौ गुण्यावित्युक्तत्वाद्धरयोर्यावत्तावतापवर्त-नाद्भावितं न भवति~। अत्र प्रथमद्वितीययोर्द्वितीयतृतीययोः प्रथमतृतीययोश्च जाताः कालको-न्मितयः \;{\small $\begin{matrix}
\mbox{{नी ~२० ~पी ~१६}}\\
\vspace{-1mm}
\mbox{{\hspace{9mm} का ~९}}
\vspace{1mm}
\end{matrix}$}~। \hspace{1mm} {\small $\begin{matrix}
\mbox{{नी ~८ ~पी ~५ं}}\\
\vspace{-1mm}
\mbox{{\hspace{7mm} का ~३}}
\vspace{1mm}
\end{matrix}$}~। \hspace{1mm} {\small $\begin{matrix}
\mbox{{नी ~४ ~पी ~७}}\\
\vspace{-1mm}
\mbox{{\hspace{7mm} का ~२}}
\vspace{1mm}
\end{matrix}$}~। अत्राप्ये-

\end{sloppypar}

\newpage

\begin{sloppypar}
\noindent कतरस्येष्टं मानं प्रकल्प्य परस्य तथेष्टं कल्प्यं यथा कालकमानमभिन्नं भवेत्~। परं भाज्य-वर्णमाननियतत्वे तदिष्टकल्पनमयुक्तम्~। भवति चात्र कालकोन्मितिभ्यां समीकृतच्छेदाभ्यां छेदगमादिना नीलकोन्मानं नियतं \;{\small $\begin{matrix}
\mbox{{पी ~३१}}\\
\vspace{-1mm}
\mbox{{नी ~~४}}
\vspace{1mm}
\end{matrix}$}~। अत्र यदेवैकत्रिंशत्पीतकमानं तदेव नीलक-चतुष्टयस्य~। अत्राप्यभिन्नत्वार्थं पीतकस्य तथेष्टं मानं कल्प्यं यथा तद्गुणितः पीतकाङ्कश्चतुर्भिः भक्तः शुध्येत्~। अस्ति चायं कुट्टकविषयः~। अत्र भाज्यवर्णाङ्को भाज्यः~। भाजकवर्णाङ्को भाजकः~। यत्र तु भाज्ये रूपाण्यपि स्युस्तत्र रूपाणि क्षेपः~। इह तु पीतकाङ्कः केन गुणितश्चतुर्भक्तः शुध्येदित्येवास्तीति क्षेपाभावः~।\\

अथ कुट्टकार्थं न्यासः \;{\small $\begin{matrix}
\mbox{{भा ~३१ ~क्षे ~०}}\\
\vspace{-1mm}
\mbox{{हा ~४ \hspace{8mm}}}
\vspace{1mm}
\end{matrix}$}~। \hyperref[5.58]{'क्षेपाभावोऽथवा यत्र क्षेपः शुद्धो हरोद्धृतः~। ज्ञेयः शून्यं गुणस्तत्र क्षेपो हरहृतः फलम्'} इति जातौ लब्धिगुणौ \;{\small $\begin{matrix}
\mbox{{ल ~०}}\\
\vspace{-1mm}
\mbox{{गु ~०}}
\vspace{1mm}
\end{matrix}$}~। अत्र \hyperref[5.59]{'इष्टाहतस्वस्वहरेण युक्ते ते वा भवेतां बहुधा गुणाप्ती'} इत्युक्तत्वादिष्टगुणा एकत्रिंशल्लब्धौ क्षेप्या इष्टगुणाश्चत्वारो गुणे क्षेप्याः~। तत्रेष्टस्येच्छाधीनत्वेनानियतत्वाद्वर्णस्वरूपमिष्टं कल्पनीयम्~। वर्णस्य हि यत् यन्मानं कल्प्यते तत्तत्सम्भवतीति सर्वेष्टानामनुगमः स्यात्~। यदि तु व्यक्तमिष्टं कल्प्यते तदा न सर्वेष्टसिद्धिः~। अथ प्रकृते यावत्तावदादीनां पीतकपर्यन्तानां मानानि नियतानि सन्तीति तेषामन्यतमस्येष्टकल्पने सर्वेष्टानुगमो न स्यादत एभ्योऽन्यवर्ण इष्टः कल्पितः लो १~। अनेन गुणिते स्वस्वहरे क्षिप्ते सति जातौ लब्धिगुणौ \;{\small $\begin{matrix}
\mbox{{लो ~३१ ~रू ~० ~ल}}\\
\vspace{-1mm}
\mbox{{लो ~~४ ~रू ~० ~गु}}
\vspace{1mm}
\end{matrix}$}~। अत्र पीतकाङ्को येन गुणितः स्वस्वहरभक्तो निःशेषः स्यात् स गुण एव पीतकस्येष्टं मानं स्यात्~। यल्लभ्यते तदेव नीलकमानमभिन्नं स्यादिति गुणो भाज्यवर्णमानं लब्धिस्तु भाजकवर्णमानमिति~। तथा सति जाते नीलकपीतकयोर्माने \;{\small $\begin{matrix}
\mbox{{लो ~३१ ~रू ~० ~नी}}\\
\vspace{-1mm}
\mbox{{लो ~~४ ~रू ~० ~पी}}
\vspace{1mm}
\end{matrix}$}~। तदेवमन्त्योन्मितौ भाज्यवर्णमानं नियतं नास्तीति तस्य मानम् इष्टं कल्प्यम्~। तत्रापि कुट्टकसिद्धगुणतुल्य इष्टे कल्पिते भाजकवर्णमानम् अभिन्नं भवतीति गुणतुल्यं भाज्यवर्णमानं कल्प्यते~। पूर्वोन्मितिषु तु भाज्यवर्णमानानां नियतत्वादिष्टकल्पनम् अयुक्तम्~। अत उक्तं \hyperref[9.134]{'अन्त्योन्मितौ कुट्टविधेः'} इत्यादि~। अथ पूर्वपूर्ववर्णोन्मितिषूत्तरोत्तरवर्णा भाज्यतया तिष्ठन्तीत्युत्तरोत्तरवर्णमानज्ञानं विना पूर्वपूर्ववर्णमानं न सिध्येदत उक्तं \hyperref[9.134]{'विलोमकोत्थापनतोऽन्यवर्णमानानि'} इति~। अथ प्रकृते कालकोन्मितिरियं \;{\small $\begin{matrix}
\mbox{{नी ~२० ~पी ~१६}}\\
\vspace{-1mm}
\mbox{{\hspace{9mm} का ~९}}
\vspace{1mm}
\end{matrix}$}~। अत्र विंशतिनीलकषोडशपीतकयोगो नवभक्तः कालक-मानमस्ति~।
\end{sloppypar}

\newpage

\begin{sloppypar}
\noindent तत्र यद्येकस्य नीलकस्येदं मानं तदा विंशतिनीलकानां किमिति नी १~। लो ३१ रू ० नी २० त्रैराशिकेन जातं नीलकविंशतेर्मानं लो ६२० रू ०~। अथैकस्य पीतकस्येदं तदा षोडशपीतकानां किमिति पी १~। लो ४ रू ०~। पी १६ त्रैराशिकेन जातं षोडशपीतकमानं लो ६४ रू ०~। अनयोर्योगोऽयं लो ६८४ रू ०~। नवभक्तो जातं कालकमानं लो ७६ रू ० का~। एवमन्ययोरपि कालकोन्मित्योरिदमेव मानं सिध्यति~। अथवा यावत्तावदुन्मितिरियं \;{\small $\begin{matrix}
\mbox{{का ~५ ~नी ~६ं ~पी ~६ं}}\\
\vspace{-1mm}
\mbox{{या ~२ \hspace{15mm}}}
\vspace{1mm}
\end{matrix}$}~। अत्रापि पूर्ववदनुपातेन जातानि कालकपञ्चकादीनां मानानि लो ३८० रू ०~। लो १८ं६ रू ०~। लो २ं४ रू ०~। एषां योगः लो १७० रू ०~। स्वहरेण द्विकेन भक्तो जातं यावत्तावन्मानं लो ८५ रू ० या~। एवमन्यास्वप्युन्मितिष्विदमेव मानं सिध्यति~। एवं सर्वत्र~। यस्य वर्णस्य व्यक्तमव्यक्तं वा व्यक्ताव्यक्तं वा मानं सिध्यति तस्य वर्णस्यान्यत्र विद्यमानस्यापि त्रैराशिकेनोत्थापनं द्रष्टव्यम्~। एतदेवोक्तमाचार्यैः सूत्रव्याख्यानान्ते 'इह यस्य वर्णस्य यन्मानमागतं व्यक्तमव्यक्तं व्यक्ताव्यक्तं वा तस्य मानस्याव्यक्ताङ्केन गुणने कृते तद्वर्णाक्षरस्य निरसनमुत्थापनमुच्यते' इति~। अथ यदि विलोमोत्थापने क्रियमाणे मानं भिन्नमायाति तदा भिन्नत्वार्थं भूयः कुट्टकः कार्यः~। उक्तयुक्तेरविशेषात्~। तदेवं सर्वमुपपन्नम्~। प्रकृते जातानि यावत्तावदादीनां मानानि \;{\small $\begin{matrix}
\mbox{{लो ~८५ ~रू ~० ~या}}\\
\mbox{{लो ~७६ ~रू ~० ~का}}\\
\mbox{{लो ~३१ ~रू ~० ~नी}}\\
\vspace{-1mm}
\mbox{{लो ~~४ ~रू ~० ~पी}}
\vspace{1mm}
\end{matrix}$}~। अत्र सर्वेष्टानुगमार्थे लोहित इष्टः कल्पितोऽस्ति~। तत्र यद्येकमिष्टं कल्प्यते तर्हि जातानि यावत्तावदादिमानानि ८५।७६। ३१।४~। द्विकमिष्टं चेदेतानि १७०।१५२।६२।८~। एवमिष्टवशादनेकधा~। एतान्येवाश्वादि मूल्यानि~॥~१३४~॥\\

{\small अथ शिष्यबुद्धिप्रसारार्थमुदाहरणानि निरूपयन्प्रथमं तावदेकवर्णपठितमुदाहरणद्वयं निरूप-यति~। पूर्वं बीजाद्धि कल्पनागौरवेण तत्सिध्यति~। इह तु कल्पनालाघवेनेत्यस्ति विशेषः~।\\

तदुदाहरणद्वयं च\textendash }

\phantomsection \label{9.135}
\begin{quote}
{\large \textbf{{\color{purple}माणिक्यामलनीलमौक्तिकमितिः पञ्चाष्ट सप्त क्रमात्\\
एकस्यान्यतरस्य सप्त नव षट् तद्रत्नसङ्ख्या सखे~।\\
रूपाणां नवतिर्द्विषष्टिरनयोस्तौ तुल्यवित्तौ तथा\\
बीजज्ञ प्रतिरत्नजानि सुमते मौल्यानि शीघ्रं वद~॥~१३५~॥}}}
\end{quote}

\end{sloppypar}

\newpage

\begin{sloppypar}
\phantomsection \label{9.136}
\begin{quote}
{\large \textbf{{\color{purple}एको ब्रवीति मम देहि शतं धनेन\\
त्वत्तो भवामि हि सखे द्विगुणस्ततोऽन्यः~।\\
ब्रूते दशार्पयसि चेन्मम षड्गुणोऽहं\\
त्वत्तस्तयोर्वद धने मम किं प्रमाणे~॥~१३६~॥}}}
\end{quote}

माणिक्यामलनीलमौक्तिकमितिरित्येकम्~। एको ब्रवीतीत्यपरम्~। उदाहरणद्वयस्यापि गणितमाकर एवं स्फुटम्~।\\

{\small एको ब्रवीतीत्यादिसजातीयोदाहरणेष्वव्यक्तक्रिया सङ्क्षिप्य तत्परिपाकजेन मार्गेण तदानयनम् उक्तमस्मद्गुरुभिः {\color{violet}श्रीविष्णुदैवज्ञैः}~। तद्यथा\textendash }

\begin{quote}
\textbf{{\color{violet}स्वस्वैकयुक्तगुणदानजघातयोऽर्योऽनल्पः परः परगुणाभिहतस्तदैक्यम्~।\\
तत्स्यान्निरेकगुणघातहृतं हि राशिस्तत्सङ्गुणाधिकगुणः परवर्जितः सन्~॥\\
{\color{white}अ} \hspace{10mm} द्वितीयराशिमानं स्यादव्यक्तक्रियया विना~।\\
{\color{white}अ} \hspace{10mm} व्यक्तमव्यक्तयुक्तं यद्ये न बुध्यन्ति ते जडाः~॥} इति~।}
\end{quote}

अत्र परगुणाभिहत इत्यत्र परशब्दोऽन्यघातवाचको न तु पारिभाषिकः~। अन्यगुणका-भिहत इति वा पठनीयम्~। तत्सङ्गुणाधिकगुण इत्यत्राधिकोऽनल्पः पर इति यावत्~। तस्य गुणोऽधिकगुणो न त्वधिकश्चासौ गुणश्चेति कर्मधारयः~। तत्सङ्गुणः परगुण इति वा पठनीयम्~। शेषं स्पष्टम्~। अत्र प्रथमो गुणः २ दानं च १००~। द्वितीयो गुणः ६ दानं च १०~। एकयुक्तेन गुणेन स्वस्वदाने गुणिते जातौ स्वस्वैकयुक्तगुणदानजघातौ ३००।७०~। अत्रानल्पः परः ३००~। अयमन्यस्य गुणेन ६ गुणितः १८००~। द्वितीयस्तु यथास्थित एव ७०~। अनयोरैक्यं १८७०~। इदं गुणघातेन १२ निरेकेण ११ हृतं जातो राशिः १७०~। अनेनाधिकस्य गुणो २ गुणितः ३४० परेणानेन ३०० वर्जितो जातो द्वितीयराशिरिति ४०~॥~१३५~॥~१३६~॥\\

{\small अथ शार्दूलविक्रीडितेनोदाहरणमाह\textendash }

\phantomsection \label{9.137}
\begin{quote}
{\large \textbf{{\color{purple}अश्वाः पञ्चगुणाङ्गमङ्गलमिता येषां चतुर्णां धना-\\
न्युष्ट्राश्च द्विमुनिश्रुतिक्षितिमिता अष्टद्विभूपावकाः~।\\
तेषामश्वतरा वृषा मुनिमहीनेत्रेन्दुसङ्ख्याः क्रमात्\\
सर्वे तुल्यधनाश्च ते वद सपद्यश्वादिमौल्यानि मे~॥~१३७~॥}}}
\end{quote}

मङ्गलान्यष्टौ~। अश्वतरा वाम्यः~। महाराष्ट्रभाषया वेसरशब्दवाच्याः~। शेषं स्पष्टम्~। गणितं तूपपत्तिविवरणावसरे स्पष्टीकृतम्~॥~१७~॥
\end{sloppypar}

\newpage

\begin{sloppypar}
{\small अथ वैचित्र्यार्थमाद्योदाहरणं प्रदर्शयति\textendash }

\phantomsection \label{9.138}
\begin{quote}
{\large \textbf{{\color{purple}त्रिभिः पारावताः पञ्च पञ्चभिः सप्त सारसाः~।\\
सप्तभिर्नव हंसाश्च नवभिर्बर्हिणस्त्रयः~॥\\
द्रम्मैरवाप्यते द्रम्मशतेन शतमानय~।\\
एषां पारावतादीनां विनोदार्थं महीपतेः~॥~१३८~॥}}}
\end{quote}

पूर्वश्लोकोक्तं पारावतसारसादिकं प्राणिजातं त्रिपञ्चादिभिर्द्रम्मैरवाप्यते~। एवं सति द्रम्मशतेनैषां पारावतादीनां शतमानयेति व्याख्येयम्~। बर्हिणस्त्रय इत्यत्र बर्हिणां त्रयमिति पाठश्चेत्साधुः~। यद्वा द्रम्मैरवाप्यत इति स्थाने द्रम्मैरवाप्यास्तदिति पाठश्चेत् साधुः~। शेषं स्पष्टम्~। अत्र प्रमाणे मौल्ययोगो जीवयोगश्च चतुर्विंशतिरस्ति~। अपेक्षितश्च शतम्~। अतः किञ्चिद्गुणैः प्रमाणद्रव्यैर्जीवा ग्राह्याः~। तत्र तुल्यगुणकगुणितैः प्रमाणद्रव्यैर्जीवग्रहण उभयेषामपि योगः शतं न स्यात्~। यतश्चतुर्गुणितानां प्रमाणद्रव्याणां योगः षण्णवतिः तत्क्रीतजीवानामपि~। पञ्चगुणितानां तु योगो विंशत्युत्तरशतं स्यात्~। यद्यपि चतुर्विंशतितुल्ये योगे यद्येको गुणस्तदा शतमिते योगे क इति लब्धेन गुणकेन पञ्चविंशतिषडंशेन \;{\small $\begin{matrix}
\mbox{{२५}}\\
\vspace{-1mm}
\mbox{{६}}
\vspace{1mm}
\end{matrix}$}\, गुणने तद्योगः शतं स्यात्तथापि पारावतादयोऽखण्डा न लभ्येरन्~। तस्मादतुल्येन गुणकेन भाव्यम्~। कश्चिद्गुणः पारावतप्रमाणमौल्यस्य~। अपरः सारसमौल्यस्य~। अन्यो हंसमौल्यस्य~। इतरो मयूरमौल्यस्येति~। ते च गुणका न ज्ञायन्ते~। अतो यावत्तावदादयः कल्पिताः या १ का १ नी १ पी १~। एतैर्गुणितानि जातानि मूल्यानि या ३ का ५ नी ७ पी ९~। अथ द्रम्मत्रयेण मूल्येन यदि पञ्च पारावता लभ्यन्ते तदा यावत्तावत्त्रयेण मूल्येन कियन्त इति ३।५~। या ३ त्रैराशिकेन लब्धाः पारावता या ५~। एवं सारसादयोऽपि कालकपञ्चकादिमौल्यैर्लब्धाः का ७ नी ९ पी ३~। अथवा यद्गुणितानि द्रव्याणि स्युर्जीवा अपि तद्गुणिताः स्युरिति जातानि द्रव्याणि जीवाश्च \;{\small $\begin{matrix}
\mbox{{या ~३ ~का ~५ ~नी ~७ ~पी ~९}}\\
\vspace{-1mm}
\mbox{{या ~५ ~का ~७ ~नी ~९ ~पी ~३}}
\vspace{1mm}
\end{matrix}$}~। अथ मौल्ययोगं जीवयोगं च पृथक्पृथक्शतसमं कृत्वा लब्धयावत्तावदुन्मानाभ्यां कालकोन्मानं विधाय शेषं गणितमाकरे स्फुटम्~। यद्वा केषां मौल्यानां योगः शतमस्तीति न ज्ञायते~। अतो मौल्यान्येव यावत्तावदादीनि प्रकल्प्य या १ का १ नी १ पी १~। ततोऽनुपातेन पारावतादीनानीय या \;{\small $\begin{matrix}
\mbox{{५}}\\
\vspace{-1mm}
\mbox{{३}}
\vspace{1mm}
\end{matrix}$}\, का \;{\small $\begin{matrix}
\mbox{{७}}\\
\vspace{-1mm}
\mbox{{५}}
\vspace{1mm}
\end{matrix}$}\, नी \;{\small $\begin{matrix}
\mbox{{९}}\\
\vspace{-1mm}
\mbox{{७}}
\vspace{1mm}
\end{matrix}$}\, पी \;{\small $\begin{matrix}
\mbox{{३}}\\
\vspace{-1mm}
\mbox{{९}}
\vspace{1mm}
\end{matrix}$}~। पूर्वविधिनैव गणितं विधेयम्~। इयांस्तु विशेषः~।
अत्र जीवानां योगः समच्छेदतया विधेयः~। शेषं पूर्ववत्~॥~१३८~॥
\end{sloppypar}

\newpage

\begin{sloppypar}

{\small अथ \hyperref[9.134]{'भूयः कार्यः कुट्टकः'} इत्यस्योदाहरणमार्ययाह\textendash }

\phantomsection \label{9.139}
\begin{quote}
{\large \textbf{{\color{purple}षड्भक्तः पञ्चाग्रः पञ्चविभक्तो भवेच्चतुष्काग्रः~।\\
चतुरुद्धृतस्त्रिकाग्रो द्व्यग्रस्त्रिसमुद्धृतः कः स्यात्~॥~१३९~॥}}}
\end{quote}

स्पष्टोऽर्थः~। अस्य गणितं सूत्रव्याख्यावसर एव स्पष्टीकृतम्~। आकरेऽपि स्पष्टमस्ति~।\\

अथ द्वितीयप्रकारेण कल्पितो राशिः या १~। अयं \hyperref[9.139]{'षड्भक्तः पञ्चाग्रः'} इति लब्धिं कालकं प्रकल्प्य तद्गुणितहरं का ६ स्वाग्रेण ५ युतं का ६ रू ५ राशिसमं कृत्वा लब्धं यावत्तावन्मानं का ६ रू ५~। अनेन राशिमुत्थाप्य जातो राशिः का ६ रू ५~। एक आलापोऽस्य घटते~। पुनरयं 'पञ्चहृतश्चतुरग्रः' इति लब्धिं नीलकं प्रकल्प्य तद्गुणितहरं नी ५ स्वाग्रेण ४ युतमस्य का ६ रू ५ समं कृत्वा लब्धं कालकमानं भिन्नं \;{\small $\begin{matrix}
\mbox{{नी ~५ ~रू ~१ं}}\\
\vspace{-1mm}
\mbox{{\hspace{7mm} का ~६}}
\vspace{1mm}
\end{matrix}$}~। कुट्टकेनाभिन्नं कालकमानं जातं पी ५ रू ४~। अथ कालकषट्कं पञ्चयुतं पूर्वराशिरस्ति~। तत्रैकस्य कालकस्य मानमिदं पी ५ रू ४~। इदं षड्गुणितं जातं कालकषट्कस्य पी ३० रू २४~। इदं पञ्चयुतं जात उत्थापितः पूर्वराशिः पी ३० रू २९~। अस्यालापद्वयं घटते~। एवमग्रेऽपि~। आकरेऽपि स्पष्टमिदम्~। एवमुत्थापनं सर्वत्र द्रष्टव्यम्~॥~१३९~॥\\

{\small अन्यदुदाहरणमार्ययाह\textendash }

\phantomsection \label{9.140}
\begin{quote}
{\large \textbf{{\color{purple}स्युः पञ्चसप्तनवभिः क्षुण्णेषु हृतेषु केषु विंशत्या~।\\
रूपोत्तराणि शेषाण्यवाप्तयश्चापि शेषसमाः~॥~१४०~॥}}}
\end{quote}

स्पष्टोऽर्थः~। अत्र शेषाण्येतानि या १~। या १ रू १~। या १ रू २~। रूपोत्तराणि प्रकल्प्य कालकादीन्राशींश्च प्रकल्प्य गणितमाकरे स्पष्टम्~॥~१४०~॥\\

{\small अन्यदुदाहरणमनुष्टुभाह\textendash }

\phantomsection \label{9.141}
\begin{quote}
{\large \textbf{{\color{purple}एकाग्रो द्विहृतः कः स्याद्द्विकाग्रस्त्रिसमुद्धृतः~।\\
त्रिकाग्रः पञ्चभिर्भक्तस्तद्वदेव हि लब्धयः~॥~१४१~॥}}}
\end{quote}

अत्र यावत्तावन्मितं राशिं प्रकल्प्य लब्धिर्द्विहृता सत्येकाग्रा यथा भवति तथैतादृशी का २ रू १ प्रकल्प्य गणितमाकरे स्फुटम्~।\\

राशिर्द्विहृत एकाग्रः स्यात् तल्लब्धिरपि द्विहृतैकाग्रा स्यात्~। एवम् अग्रेऽपि व्याख्ये-यम्~॥~१४१~॥
\end{sloppypar}

\newpage

\begin{sloppypar}
{\small अन्यदुदाहरणं शार्दूलविक्रीडितेनाह\textendash }

\phantomsection \label{9.142}
\begin{quote}
{\large \textbf{{\color{purple}कौ राशी वद पञ्चषट्कविहृतावेकद्विकाग्रौ ययोः\\
द्व्यग्रं त्र्युद्धृतमन्तरं नवहृता पञ्चाग्रका स्याद्युतिः~।\\
घातः सप्तहृतः षडग्र इति तौ षट्काष्टकाभ्यां विना\\
विद्वन्कुट्टकवेदिकुञ्जरघटासंघट्टसिंहोऽसि चेत्~॥~१४२~॥}}}
\end{quote}

स्पष्टोऽर्थः~। अत्र पञ्चहृतः सन्नेकाग्रो लघुराशिः षट्कमेव सम्भवति~। एवं षट्कहृतो द्विकाग्रो लघुराशिरष्टावेव सम्भवति~। अतः कौ राशी पञ्चषट्कविहृतावेकद्विकाग्रौ भवत इति प्रश्ने षट्काष्टकयोरेव प्रथममुपस्थितिर्भवति~। यदृच्छया तयोः सर्वोऽप्यालापः सम्भवति~। तदत्र कल्पनां विनैव प्रश्नभङ्गः स्यादित्यत उक्तं षट्काष्टकाभ्यां विनेति~। अत्र राशी पञ्चषट्कविहृतौ यथैकद्विकाग्रौ भवतस्तथैतादृशौ या ५ रू १~। या ६ रू २ प्रकल्प्य घातालापकरणावसरे वर्गत्वान्महती क्रिया भवेदिति पीतकमेकेनोत्थाप्य प्रथमराशिं व्यक्तमेव कृत्वा गणितमाकरे स्पष्टम्~॥~१४२~॥\\

{\small अन्यदुदाहरणमनुष्टुभाह\textendash }

\phantomsection \label{9.143}
\begin{quote}
{\large \textbf{{\color{purple}नवभिः सप्तभिः क्षुण्णः को राशिस्त्रिंशता हृतः~।\\
यदग्रैक्यं फलैक्याढ्यं भवेत्षड्विंशतेर्मितम्~॥~१४३~॥}}}
\end{quote}

राशिर्नवभिः सप्तभिः पृथग्गुणितः~। उभयत्रापि त्रिंशता हृतः~। शेषं स्पष्टम्~। अत्र राशौ नवभिः सप्तभिः पृथग्गुणिते त्रिंशता भक्ते च लब्धिद्वयं शेषद्वयं च पृथक्पृथक्स्यात्~। यदि तु गुणयोगेन राशिरेकत्रैव गुण्यते त्रिंशता च ह्रियते तदा तत्र फलं फलैक्यं स्याच्छेषं च शेषैक्यं स्यात्~। यथा राशिः ५ नवभिः सप्तभिः पृथग्गुणितः ४५।३५ त्रिंशता हृतः फले १।१~। शेषे च १५।५~। अथ स एव राशिः ५ गुणयोगेन १६ गुणितः ८० त्रिंशता हृतः फलं २ शेषं च २०~। अत्र हि फलं पूर्वफलैक्यमेव~। शेषं च पूर्वशेषैक्यमेव~। अत्रोदाहरणे फलैक्यशेषैक्ययोरेवावश्यकतया लाघवाद्गुणयोगं गुणं प्रकल्प्य गणितमाकरे स्पष्टम्~। नन्वत्र गुणयोगेन राशौ गुणिते हरेण भक्ते शेषैक्यमपि हरतष्टं स्यात्~। तत्र यद्यपि हरान्न्यूने शेषैक्ये सति तस्य यथास्थितस्य हरतष्टस्य चाविशेषान्न काचित्क्षतिस्तथापि हरादधिके शेषैक्ये तस्य यथास्थितस्य हरतष्टस्य च हरतुल्यमन्तरं स्यात्~। फलैक्यं च सैकं स्यात्~। यथा राशिः ६ अयं नवभिः सप्तभिश्च पृथग्गुणितः ५४।४२ त्रिंशता हृतः फले १।१ शेषे च २४।१२~। अथ स एव राशिः ६ गुणयोगेन १६ गुणितः ९६ त्रिंशता हृतः फलं ३ शेषं च ६~। अत्र हि फलं पूर्वफलैक्यं सैकमस्ति~। शेषं च सैकं
\end{sloppypar}

\newpage

\begin{sloppypar}
\noindent हरतष्टम् अस्ति~। अतो गुणयोगे गुणे कल्पिते सति फलैक्यशेषैक्ययोरन्यथात्वेन क्रिया व्यभिचरेदिति चेन्मैवम्~। गुणयोगे गुणे कल्पिते सति यदि फलप्रमाणं कालकः कल्प्येत तर्हि त्वदुक्तयुक्त्या क्वचित्पूर्वफलैक्यशेषैक्ययोरन्यथात्वेन क्रिया व्यभिचरेत्~। इह तु फलैक्यप्रमाणमेव कालकः कल्प्यते~। तथा सति हरगुणेऽस्मिन्भाज्यादपनीते शेषैक्यमपि यथास्थितं स्यान्न हरतष्टमिति नास्ति फलैक्यशेषैक्ययोरन्यथात्वम्~। किन्तु गुणयोग-सम्बन्धिनोः फलशेषयोः क्वचिदन्यथात्वं स्यात्~। परं तस्यानपेक्षितत्वादन्यथात्वेऽपि न काचित्क्षतिः~। अत एव लब्धैक्यप्रमाणं कालक इत्येवोक्तमाचार्यैरपि~।\\

अथात्र प्रतीत्यर्थमस्मिन्नेवोदाहरणे \hyperref[9.143]{'यदग्रैक्यं फलैक्याढ्यमष्टत्रिंशन्मितं भवेत्'} इति प्रकल्प्य गणितं लिख्यते~। राशिः या १~। गुणयोगेन १६ गुणितः या १६~। अयं त्रिंशद्भक्तः फलैक्यप्रमाणं कालकः का १~। अस्मिन्हरगुणे का ३० भाज्यात् या १६ अपनीते जातं शेषैक्यं या १६ का ३ं०~। इदं फलैक्येन कालकेन युतं या १६ का २ं९~। अष्टत्रिंशत्समं कृत्वा कुट्टकेन लब्धे यावत्तावत्कालकमाने \;{\small $\begin{matrix}
\mbox{{नी ~२९ ~रू ~६ ~या}}\\
\vspace{-1mm}
\mbox{{नी ~१६ ~रू ~२ ~का}}
\vspace{1mm}
\end{matrix}$}~। अत्र फलैक्यशेषैक्ययोग-स्येयत्तानिर्देशात्क्षेपोऽनुचित इति जाते यावत्तावत्कालकमाने ६।२~। तत्र यावत्तावन्मानं राशिः ६~। अयं नवभिः सप्तभिः पृथग्गुणितः ५४।४२~। त्रिंशद्भक्तः फले १।१ शेषे च २४।१२~। अत्र यदेव फलैक्यं तदेव कालकमानं न तु गुणयोगसबन्धि फलं शेषैक्येऽपि या १६ का ३ं०~। यावत्तावत्कालकौ स्वस्वमानेनोत्थाप्य जातं शेषैक्यं यथास्थितमेव ३६ न तु हरतष्टम्~। गुणयोगसम्बन्धि तु शेषमिदं ६ फलं च ३~। अथात्रैवोदाहरणे यदि फलप्रमाणं कालकः कल्प्यते तदा पूर्ववज्जातं फलं का १ शेषं च या १६ का ३ं०~। इदं फलं फलैक्यं सैकमस्तीति फलं रूपोनं सज्जातं फलैक्यं का १ रू १ं~। अथ शेषमपि शेषैक्यं हरतष्टमस्तीति शेषं हरयुक्तं सज्जातं शेषैक्यं या १६ का ३ं० रू ३०~। अथ फलैक्यशेषैक्ययोगं या १६ का २ं९ रू २९ अष्टत्रिंशत्समं कृत्वा कुट्टकेन प्राग्वद्यावत्तावत्कालकमाने ६।३~। अत्र हि गुणयोगसम्बन्धि फलमेव कालकः कल्पितोऽस्तीति कालकमानं तादृगेव सिद्धम्~। तदेवं कालकस्य फलत्वकल्पनेऽप्युदाहरणसिद्धिरस्ति~। इयांस्तु विशेषः~। फलप्रमाणे कालके कल्पिते यदि फलैक्यशेषैक्ययोरन्यथात्वं निश्चितं स्यात्तर्ह्येव फलं निरेकं शेषं च सहरं कर्तुं युज्यते नान्यथा~। फलैक्ये तु कालके कल्पिते न कोऽपि विचारोऽस्तीति लाघवात्फलैक्यमेव कालकः कल्प्यत इति सर्वमवदातम्~॥~१४३~॥
\end{sloppypar}

\newpage

\begin{sloppypar}
{\small अथान्यदुदाहरणमनुष्टुभाह\textendash }

\phantomsection \label{9.144}
\begin{quote}
{\large \textbf{{\color{purple}कस्त्रिसप्तनवक्षुण्णो राशिस्त्रिंशद्विभाजितः~।\\
यदग्रैक्यमपि त्रिंशद्धृतमेकादशाग्रकम्~॥~१४४~॥}}}
\end{quote}

स्पष्टोऽर्थः~। अत्रापि गुणयोगो गुणः प्राग्वत् राशिः या १ गुणयोगेन १९ गुणितः या १९~। त्रिंशता हृतो लब्धप्रमाणं कालकः का १~। अत्र यदग्रैक्यमपि त्रिंशद्धृतम् इति शेषैक्यस्य हरतष्टस्यैवावश्यकतया फलप्रमाणम् एव कालकः कल्प्यते~। फलैक्यप्रमाणे कालके कल्पिते सति पूर्वोदाहरणोक्तयुक्त्या शेषैक्यं यथास्थितमेव स्यान्न हरतष्टम्~। अत एवाचार्यैरत्र लब्धं कालक इत्येवोक्तम्~। अथ लब्धिगुणं हरं भाज्यादपनीय जातमुक्तयुक्त्या त्रिंशत्तष्टं शेषैक्यं या १९ का ३ं०~। तदेवं यदग्रैक्यमपि त्रिंशद्धृतमिति द्वितीयालापस्य प्रथमालाप एवान्तर्भूतत्वादिदमेवैकादशसमं कृत्वा प्राग्वज्जातो राशिः नी ३० रू २९~॥~१४४~॥\\

{\small अथान्यदुदाहरणमनुष्टुभाह\textendash }

\phantomsection \label{9.145}
\begin{quote}
{\large \textbf{{\color{purple}कस्त्रयोविंशतिक्षुण्णः षष्ट्याशीत्या हृतः पृथक्~।\\
यदग्रैक्यं शतं दृष्टं कुट्टकज्ञ वदाशु तम्~॥~१४५~॥}}}
\end{quote}

स्पष्टोऽर्थः~। अत्र राशिः या १ त्रयोविंशतिगुणितः या २३~। अमुं षष्ट्याशीत्या च पृथक् भक्त्वा कालकनीलकौ फले प्रकल्प्य यथास्वं लब्धिगुणं हरं भाज्यादपनीय जाते पृथक्शेषे या २३ का ६ं०~। या २३ नी ८ं०~। अनयोरैक्यं या ४६ का ६ं० नी ८ं०~। शतसमं कृत्वा लब्धा यावत्तावदुन्मितिः \;{\small $\begin{matrix}
\mbox{{का ~६० ~नी ~८० ~रू ~१००}}\\
\vspace{-1mm}
\mbox{{\hspace{20mm} या ~४६}}
\vspace{1mm}
\end{matrix}$}~। भाज्यभाजकौ द्वाभ्याम् अपवर्त्य जाता \;{\small $\begin{matrix}
\mbox{{का ~३० ~नी ~४० ~रू ~५०}}\\
\vspace{-1mm}
\mbox{{\hspace{20mm} या ~२३}}
\vspace{1mm}
\end{matrix}$}~। अत्र यावत्तावन्मानं भिन्नं लभ्यत इति कुट्टके-नाभिन्नं कार्यम्~। तत्र \hyperref[9.134]{'अन्येऽपि भाज्ये यदि सन्ति वर्णास्तन्मानमिष्टं परिकल्प्य साध्ये'} इत्युक्तत्वात्कालकनीलकयोरन्यतरस्येष्टं मानं कल्प्यम्~। परं तदिह न युक्तम्~। यतोऽत्र कालकनीलकावेकस्मादेव भाज्यात्षष्ट्यशीत्योर्लब्धे~। तत्र षष्टिलब्धस्य कालकस्य व्यक्ति-कल्पने तदेव चरणोनमशीतिलब्धं बलात्स्यादिति नीलकस्यापि व्यक्तमेव मानं स्यात्~। एवम् अशीतिलब्धस्य नीलकस्य व्यक्तत्वकल्पने त्रैराशिकेन तदेव सत्र्यंशं बलात्षष्टिलब्धं स्यादिति कालकमानमपि व्यक्तमेव स्यात्~। तथा सति शेषयोगस्य नास्ति शतानुरोधिनी क्रियेति नोदाहरणसिद्धिः~।
\end{sloppypar}

\newpage

\begin{sloppypar}
अथ यद्येकतरवर्णस्येष्टं मानं प्रकल्प्य ततस्त्रैराशिकेन द्वितीयवर्णमानं व्यक्तमकृत्वैव कुट्टकेन  तन्मानं साध्येत तर्हि तदुक्तविधानादन्यथोत्पन्नम् अपि बाधितम् एव स्यात्~। न हि चरणोनात् षष्टिलब्धादन्यदशीतिलब्धं सम्भवति~। सत्र्यंशादशीतिलब्धादन्यत् षष्टिलब्धं वा सम्भवति~॥~१४५~॥\\

{\small एतदेवानुष्टुभाह\textendash }

\phantomsection \label{9.146}
\begin{quote}
{\large \textbf{{\color{purple}अत्राधिकस्य वर्णस्य भाज्यस्थस्येप्सिता मितिः~।\\
भागलब्धस्य नो कल्प्या क्रिया व्यभिचरेत्तथा~॥~१४६~॥}}}
\end{quote}

अस्यार्थः~। अत्र भाज्यस्थस्य भागलब्धस्याधिकवर्णस्य मितिरिष्टा न कल्पनीया~। अधिकवर्णस्य कुट्टकोपयुक्तवर्णादतिरिक्तस्येत्यर्थः~। अथ तदिष्टकल्पनेऽनिष्टमाह\textendash \,तथा सति क्रिया व्यभिचरेदिति~। अत्रोपपत्तिरुक्तैव~। तदेवमुक्तविधकल्पनया नोदाहरणसिद्धिः अस्तीत्यन्यथा यतितमाचार्यैः~। अत्र स्वस्वभागहारान्न्यूने शेषे यथा भवतो यथा च तद्योगः शतं स्यात्तथा शेषे प्रकल्प्य गणितमाकरे स्पष्टम्~।\\

ननु षष्ट्या यदि कालको लभ्यते तदाशीत्या किमिति त्रैराशिकेनाशीतिलब्धिमानीय का \,{\small $\begin{matrix}
\mbox{{३}}\\
\vspace{-1mm}
\mbox{{४}}
\vspace{1mm}
\end{matrix}$}\, प्राग्वच्छेषक्रियास्तु~। नह्येवं सति भाज्ये वर्णद्वयं भवति येन द्वितीयवर्णेष्टकल्पनजो दोषः स्यादिति चेन्न~। न ह्यत्र लब्ध्यनुपातो युक्तः~। अनुपातेन लब्धिसाधने हि यावतो भाज्यखण्डस्य षष्टिजा लब्धिरस्ति तावत एवाशीतिजा लब्धिः सावयवा स्यात्~। सा च न युक्ता~। न हि शेष उद्देश्ये सावयवा लब्धिः सम्भवति~। यत्तु पूर्वमुक्तं कालकमानस्य व्यक्तकल्पने ततोऽनुपातेन नीलकमानमपि व्यक्तं स्यादिति तत्र व्यक्तत्वेनात्यल्पमन्तरं भवतीति न कोऽपि दोषः~। अथ यद्यनुपातजा लब्धिः सावयवा न स्यात्तदा त्वदुक्तरीत्यापि भवेदेवोदाहरणसिद्धिः~। तथा हि~। राशिः या १~। अस्मात्त्रयोविंशतिगुणात्षष्टया लब्धि-कल्पितं का १~। अतोऽनुपातजाशीतिलब्धिः का ३~। अथ यथास्वं हरगुणां लब्धिं भाज्यात् अपनीय जाते शेषे \;{\small $\begin{matrix}
\mbox{{या ~२३ ~का ~२४ं०~।}}\\
\vspace{-1mm}
\mbox{{या ~२३ ~का ~२४०~।}}
\vspace{1mm}
\end{matrix}$}\, अत्र यावतो भाज्यखण्डात्षष्ट्या लब्धिस्तावत एवाशीत्या अपीति शेषे समे एव भवतः~। अतः शेषयोगं शतसममथवा शेषं पञ्चाशत्समं कृत्वा कुट्टकेन लब्धे यावत्कालकमाने नी २४० रू १९० या~। नी २३ रू १८ का~॥~१४६~॥\\

{\small अथान्यदुदाहरणमनुष्टुभाह\textendash }

\phantomsection \label{9.147}
\begin{quote}
{\large \textbf{{\color{purple}कः पञ्चगुणितो राशिस्त्रयोदशविभाजितः~।\\
यल्लब्धं राशिना युक्तं त्रिंशज्जातं वदाशु तम्~॥~१४७~॥}}}
\end{quote}
\end{sloppypar}

\newpage

\begin{sloppypar}
स्पष्टोऽर्थः~। अत्राव्यक्ते राशौ कल्पिते तत्रोद्देशकालापे च कृत उद्दिष्टगुणहरानुरोधिनी न काचित्क्रियास्तीति नोदाहरणसिद्धिर्भवतीत्यत्राचार्यैरिष्टकर्मणैव राशिरानीतः \,{\small $\begin{matrix}
\mbox{{६५}}\\
\vspace{-1mm}
\mbox{{३}}
\vspace{1mm}
\end{matrix}$}~॥~१४७~॥\\

{\small अथ सार्धानुष्टुभोक्तमाद्योदाहरणं प्रदर्शयति\textendash }

\phantomsection \label{9.148}
\begin{quote}
{\large \textbf{{\color{purple}षडष्टशतकाः क्रीत्वा समार्घेन फलानि ये~।\\
विक्रीय च पुनः शेषमेकैकं पञ्चभिः पणैः~॥\\
जाताः समपणास्तेषां कः क्रयो विक्रयश्च कः~॥~१४८~॥}}}
\end{quote}

अस्यार्थः~। षट् अष्टौ शतं च धनं विद्यते येषां ते षडष्टशताः~। अर्श आदिभ्योऽजिति मत्वर्थीयोऽच्प्रत्ययः~। त एव षडष्टशतका इत्यत्र स्वार्थे कन्प्रत्ययः~। धनं चात्र पणाः~। जाताः समपणा इत्युक्तेः~। तादृशा ये फलव्यापारिणः समेनैव मूल्येन स्वस्वद्रव्यानुपातेन फलानि क्रीत्वा तानि समेनैव केनचिन्मूल्येन विक्रीय च यच्छेषं पणविक्रयान्न्यूनं तद्यदृच्छया पान्थबाहुल्येन फलाल्पतया चैकैकं फलं पञ्चभिः पञ्चभिः पणैर्विक्रीय च समपणाः समाः पणा येषां ते तथा जाताः~। एवं चेत्तर्हि तेषां फलव्यापारिणां क्रयः पणलभ्यफलप्रमाणं विक्रयः पणदेयफलप्रमाणं किमिति प्रश्नः~। दलानीति पाठे ताम्बूलवल्लीपर्णानि कदल्यादिपर्णानि वा ज्ञेयानि~। अथ तावदस्योदाहरणस्य गणितमाकरस्थं लिख्यते~। अत्र क्रयः या १~। विक्रय इष्टं दशाधिकं शतं ११०~। क्रयः षड्गुणितो विक्रयेण हृतो लब्धं कालकः का १~। लब्धिगुणं हरं षड्गुणिताद्राशेरपनीय शेषं या ६ का ११ं०~। इदं पञ्चगुणं लब्धियुतं जाताः प्रथमस्य पणाः या ३० का ५४ं९~। एवं द्वितीयतृतीययोरपि पणाः साध्याः~। तत्र लब्धिः अनुपातेन यदि षण्णां कालकस्तदाष्टानां शतस्य च किमिति लब्धिरष्टानां का \,{\small $\begin{matrix}
\mbox{{४}}\\
\vspace{-1mm}
\mbox{{३}}
\vspace{1mm}
\end{matrix}$}~। शतस्य च का \,{\small $\begin{matrix}
\mbox{{५०}}\\
\vspace{-1mm}
\mbox{{३}}
\vspace{1mm}
\end{matrix}$}~। लब्धिगुणं हरं भाज्यादपास्य ततः प्राग्वज्जाता द्वितीयस्य पणाः या १२० का \,{\small $\begin{matrix}
\mbox{{२१९६}}\\
\vspace{-1mm}
\mbox{{३}}
\vspace{1mm}
\end{matrix}$}~। एवं तृतीयस्य पणाः या १५०० का \,{\small $\begin{matrix}
\mbox{{२७४ं५०}}\\
\vspace{-1mm}
\mbox{{३}}
\vspace{1mm}
\end{matrix}$}~। एते सर्वे समा इति समच्छेदीकृत्य च्छेदगमे प्रथमद्वितीयपक्षयोर्द्वितीयतृतीययोः प्रथमतृतीययोश्च समीकरणेन लब्धा यावत् तावदुन्मितिस्तुल्यैव \,{\small $\begin{matrix}
\mbox{{का ~५४९}}\\
\vspace{-1mm}
\mbox{{या ~~३०}}
\vspace{1mm}
\end{matrix}$}~। अत्र कुट्टकलब्धं यावत्तावन्मानं नी \,{\small $\begin{matrix}
\mbox{{५४९}}\\
\vspace{-1mm}
\mbox{{२२}}
\vspace{1mm}
\end{matrix}$}\, रू ०~। नीलकम् एकेनोत्थाप्य जातः क्रयः ५४९ इति~।

\end{sloppypar}

\newpage
 
\begin{sloppypar}
 अथात्र किञ्चित् विचार्यते~। इह हि षड्गुणितात् क्रयाद्विक्रयहृताद्यदि कालको लभ्यते तदाष्टगुणिताच्छतगुणिताच्च किम् इति त्रैराशिकेन लब्धिसाधनं कृतम् आचार्यैः~। तत्र पृच्छ्यते~। षड्गुणितस्य क्रयस्य येह लब्धिः कल्पिता सा किम् अशेषा सशेषा वा~। आद्ये शेषाभावाच्छेषम् एकैकं पञ्चभिः पणैरित्यालापविरोधः~। द्वितीये तु तादृशलब्धेरनुपातेन गुणान्तरलब्धिसाधनम् अयुक्तम्~। गुणान्तरलब्धौ हि शेषोत्थलब्धितुल्यम् अन्तरं स्यादिति व्यभिचारः स्यात्~। तद्यथा\textendash \,भाज्यभाजकौ \,{\small $\begin{matrix}
\mbox{{१५}}\\
\vspace{-1mm}
\mbox{{१३}}
\vspace{1mm}
\end{matrix}$}~। अत्र षड्गुणितभाज्या ९० ल्लब्धिरियं ६ शेषमिदम् १२~। अथ षड्गुणितभाज्याच्चेदियं लब्धि\textendash \,६\textendash \,स्तदाष्टगुणिताच्छतगुणिताच्च केति त्रैराशिकेन जाते अष्टगुणितशतगुणितभाज्ययोः क्रमेण लब्धी ८।१००~। न चैते युक्ते~। यतोऽष्टगुणितभाज्यादस्मा\textendash \,१२०\textendash \,च्छतगुणितभाज्यादस्माच्च १५०० क्रमेण लब्धी ९।११५~। अतो लब्ध्यनुपातो न युक्तः~। ननु केवलभाज्ये हरभक्ते यच्छेषं तद्गुणितगुणकादधिके हरे शेषोत्था लब्धिर्नैव सम्भवति~। तथा सति व्यभिचारः क्वत्यः~। तथाहि~। केवलभाज्यस्य हि खण्डद्वयम् अस्ति~। यावद्धरभक्तं तावदेकम्~। शेषतुल्यम् अपरम्~। तत्र प्रथमखण्डं केवलम्  अपि हरभक्तं शुध्यतीति गुणकेन गुणितं सत् सुतरां शुध्येत्~। तस्य लब्धिस्तु केवलभाज्यस्य या लब्धिः सैव गुणकगुणिता सती स्यात्~। अतस्तत्रानुपातो युक्त एव~। अथ द्वितीयखण्डं गुणकेन गुणितं सद्गुणकगुणितशेषतुल्यं स्यात्~। ततोऽधिको यदि हरः स्यात् तर्हि द्वितीयखण्डोत्थलब्धिः कथं सम्भवेत्~। अतः पूर्वानुपातसिद्धैव लब्धिर्गुणितभाज्यजा स्यात्~। एवं केवलभाज्ये हरेण भक्ते यदि रूपं शेषं स्यात्तदा गुणितभाज्यस्य द्वितीयखण्डं गुणतुल्यमेव स्यादिति गुणाधिके हरे शेषोत्थलब्धेरभावाल्लब्ध्यनुपातो युक्त एव~। अत एवाचार्यैर्गुणाधिक एवेष्टविक्रयः कल्पितः ११०~। यदि गुणान्न्यून इष्टविक्रयः कल्प्येत~तदा-नुपातजलब्धौ त्वदुक्तयुक्त्या व्यभिचारः स्यात्~। किं तु प्रकृते न तथास्तीति न कोऽपि दोष इति चेन्मैवम्~। यद्यपि भवदुक्तयुक्त्या लब्धौ व्यभिचारो नास्ति तथापि यस्य गुणकस्य लब्धिरल्पा स्यात्तस्य शेषमप्यल्पम्~। यस्य च लब्धिरधिका तस्य शेषमप्यधिकं स्यादिति पणसाम्यं कथमपि न स्यात्~। तदेवमाचार्यविचारितः पन्था न तर्कसहकृत इति प्रतिभाति~।\\

अत्रोच्यते~। सशेषा लब्धिस्तावद्द्विविधा~। धनशेषा ऋणशेषा चेति~। शेषमपि द्विविधम्~। धनमृणं चेति~। तत्र हरादल्पेन येन रहितः सन्भाज्यो हरभक्तः शुध्येत्तच्छेषं धनम्~। तत्र या लब्धिः सा धनशेषा~। अथ हरादल्पेन येन सहितः सन्भाज्यो हरभक्तः शुध्येत्तच्छेषमृणम्~। तत्र या लब्धिः सा ऋणशेषा~। अत्र रहितसहितभाज्ययोरन्तरं शेषयोगतुल्यमेव स्यात्~। तच्च हरतुल्यमेव~। अन्यथा द्वयोरपि हरभक्तयोः शुद्धिः कथं
\end{sloppypar}

\newpage

\begin{sloppypar}
\noindent स्यात्~। यद्यपि द्व्यादिगुणितहरतुल्येऽप्यन्तर उभयोः शुद्धिः सम्भवति तथापि नेह तथा~। इह हि शेषयोगतुल्यमन्तरम्~। एवं सति हरादल्पयोः शेषयोर्योगो द्व्यादिगुणितहरतुल्यः कथं स्यात्~। तस्माद्रहितसहितभाज्ययोर्हरतुल्यमन्तरं भवतीति तल्लब्ध्यो रूपमन्तरं स्यात्~। तत्र रहितभाज्यजा लब्धिर्धनशेषा~। अपरा ऋणशेषा~। अतो धनशेषा लब्धिः सैका सति ऋणशेषा लब्धिः स्यात्~। इयं वा निरेका सती धनशेषा लब्धिः स्यात्~। एवं धनर्णशेषयोगो हरतुल्योऽस्तीति धनशेषं हराच्छोधितं सदृणशेषं स्यात्~। इदं वा हराच्छोधितं सद्धनशेषं स्यात्~।\\

प्रतीत्यर्थमङ्कतोऽपि लिख्यते~। भाज्यभाजकौ \,{\small $\begin{matrix}
\mbox{{२९}}\\
\vspace{-1mm}
\mbox{{१३}}
\vspace{1mm}
\end{matrix}$}~। अत्र भाज्यस्त्र्यूनः सन् २६ हरभक्तः शुध्यतीति धनशेषमिदं ३~। धनशेषाल्लब्धिश्च २~। अथायमेव भाज्यो २९ दशसहितः सन् ३९ हरभक्तः शुध्यतीति ऋणशेषमिदम् १०~। ऋणशेषालब्धिश्च ३~। अत्र सर्वं यथोक्तम् अस्ति~। एवं सर्वत्र~। इत्येवं स्थितिरस्ति~। अथ प्रकृते यथा केवलभाज्यस्य रूपमिते धनशेषे सति गुणितभाज्यस्य गुणतुल्यं धनशेषं भवतीति गुणाधिके हरे शेषोत्थलब्धेरभावाल्लब्ध्यनुपातो युक्तस्तथा केवलभाज्यस्य रूपमित ऋणशेषे सति गुणितभाज्यस्य गुणकतुल्यमृणशेषं स्यादिति गुणाधिके हरे शेषोत्थलब्धेरभावादत्रापि लब्ध्यनुपातो युक्तः~। अत्र शेषाणि ऋणं सन्तीति धनत्वार्थे तानि हराच्छोध्यानि~। तथा सति गुणकोनहरः शेषं स्यात् इति यस्य गुणकस्य लब्धिरधिका तस्य शेषमल्पं यस्य च लब्धिरल्पा तस्य शेषमधिकं स्यादिति पणसाम्यं भवेत्~। अत आचार्यैर्ऋणशेषा लब्धिः कालकमिता कल्पितास्तीति न कोऽपि दोषः~। अत एवात्र कालकमानं सैकलब्धिसमं दृश्यते~।\\

ननु तर्हि ऋणशेषा लब्धयो निरेकाः सत्यो धनलब्धयः स्युरित्यनुपातजलब्धीर्निरेका कृत्वा कर्म कर्तुं युज्यते~। आचार्यैस्तु न तथा कृतमिति कथं दोषो न स्यादिति चेन्न~। तथा कृतेऽपि पक्षसाम्यमस्तीति फलतो दोषाभावात्~। अतस्तथा करणे पक्षेषु सामान्येन रूपाणि अधिकानि स्युरकरणे तु रूपाभाव एवेत्याचार्यकृतपक्षास्तुल्यैरेव रूपैरूना जाता इति ते साम्यं न त्यजन्तीति~।\\

नन्वत्र यावत्तावदुन्मानमिदं \,{\small $\begin{matrix}
\mbox{{का ~५४९}}\\
\vspace{-1mm}
\mbox{{या ~~~३०}}
\vspace{1mm}
\end{matrix}$}~। अत्र भाज्यभाजकयोस्त्रिभिरपवर्तः सम्भवति~। \hyperref[5.50]{'भाज्यो हारः क्षेपकश्चापवर्त्यः'} इति कुट्टकार्थमावश्यकश्च सः~। तत्कथं कृतेऽपवर्ते मानम् असदागच्छति~। अनपवर्ते च सदिति चेच्छृणु तर्हि~। इह हि शेषमावश्यकम्~। कृते त्वपवर्ते शेषाण्यपवर्तितानि स्युरिति नोद्दिष्टसिद्धिः~। तदुक्तम् {\color{violet}आचार्यैर्गोले प्रश्नाध्याये\textendash }
\end{sloppypar}

\newpage

\begin{sloppypar}
\begin{quote}
{\color{violet}'उद्दिष्टं कुट्टके तज्ज्ञैर्ज्ञेयं निरपवर्तनम्~।\\
व्यभिचारः क्वचित्क्वापि खिलत्वापत्तिरन्यथा~॥'} इति~।
\end{quote}

अथ यथापवर्तादिसंशयो न भवति तथा सोपपत्तिकं लिख्यते~। क्रयः या १~। विक्रय इष्टः ११०~। केवलक्रये विक्रयेण हृते ऋणशेषा लब्धिरियं का १~। एकगुणक्रयस्य चेदियं लब्धिस्तदा षडादिगुणितस्य केति त्रैराशिकेन जाताः षडष्टशतगुणितक्रयस्य पृथक्पृथग्लब्धयः का ६~। का ८~। का १००~। एता निरेका जाता धनशेषा लब्धयः का ६ रू १ं~। का ८ रू १ं~। का १०० रू १ं~। अथ यथास्वं लब्धिगुणं हरं गुणितभाज्यादपनीय जातानि धनशेषाणि
\vspace{-2mm}

\begin{center}
\begin{tabular}{rrrrrr}
या & ६ & का & ६६ं० & रू & ११०\\
या & ८ & का & ८८ं० & रू & ११०\\
या & १०० & का & ११०ं०० & रू & ११०
\end{tabular}
\end{center}
\vspace{-2mm}

\noindent अथैकस्य फलस्य यदि पञ्च पणास्तदा शेषमितफलानां किमिति जाताः पृथक्शेषफलपणाः
\vspace{-2mm}

\begin{center}
\begin{tabular}{rrrrrr}
या & ३० & का & ३३ं०० & रू & ५५० \\
या & ४० & का & ४४ं०० & रू & ५५० \\
या & ५०० & का & ५५ं००० & रू & ५५०
\end{tabular}
\end{center}
\vspace{-2mm}

एते स्वस्वलब्धपणैर्युता जाताः~।
\vspace{-2mm}

\begin{center}
\begin{tabular}{rrrrrr}
या & ३० & का & ३२ं९४ & रू & ५४९ \\
या & ४० & का & ४३ं९२ & रू & ५४९ \\
या & ५०० & का & ५४ं९०० & रू & ५४९
\end{tabular}
\end{center}
\vspace{-2mm}

\noindent एते समा इति प्रथमद्वितीययोर्द्वितीयतृतीययोः प्रथमतृतीययोश्च समशोधने कृते यथासम्भ-वमपवर्ते च कृते जाता यावत्तावदुन्मितिस्तुल्यैव \;{\small $\begin{matrix}
\mbox{{का ~५४९}}\\
\vspace{-1mm}
\mbox{{या ~~~५}}
\vspace{1mm}
\end{matrix}$}~। अतः कुट्टकेन जाते यावत् तावत्कालकमाने \;{\small $\begin{matrix}
\mbox{{नी ~५४९ ~रू ~० ~या}}\\
\vspace{-1mm}
\mbox{{नी ~~~५ ~~रू ~० ~का}}
\vspace{1mm}
\end{matrix}$}~। लब्धिषु कालकं स्वमानेनोत्थाप्य जाता लब्धयः \;{\small $\begin{matrix}
\mbox{{नी ~~३० ~रू ~१ं}}\\
\mbox{{नी ~~४० ~रू ~१ं}}\\
\vspace{-1mm}
\mbox{{नी ~५०० ~रू ~१ं}}
\vspace{1mm}
\end{matrix}$}~। अत्र नीलकमेकेनैवोत्थापयेत्~। अन्यथा क्रये विक्रयेण हृते रूपाधि-
\end{sloppypar}

\newpage

\begin{sloppypar}
\noindent कमृणशेषं स्यादिति शेषोत्थलब्धिसम्भवेन लब्धिव्यभिचारान्मानमसत्स्यात्~। षडष्टदशका इति पाठे तु नीलकमानं दशपर्यन्तं सम्भवति~। यतस्तत्र क्रये कल्पितविक्रयेण हृते~दश-पर्यन्तमृणशेषं स्यात्~। तथा सति गुणघ्नशेषादधिक एव हरोऽस्तीति शेषोत्थलब्धेरभावेन व्यभिचाराभावात्~। एवं षडष्टशतका इति पाठेऽपि यदि द्व्यादिगुणाच्छताधिको विक्रयः कल्प्यते तदा तत्रापि द्व्यादिकं नीलकमानं सम्भवत्येव~।\\

अथान्यथा साध्यते~। इहाधिकगुणकाच्छतादेकगुणादेव विक्रयोऽधिकोऽस्तीति केवल-क्रयस्य रूपमेव वर्णशेषं सम्भवति नान्यत्~। द्व्यादिके हि शेषे गुणघ्नादस्माद्धरो न्यूनः स्यादिति शेषोत्थलब्धिसम्भवेन व्यभिचारः स्यात्~। अतो ज्ञातं व्यक्तम् एव केवलक्रयस्यर्ण-शेषम्~१ं~। इदं गुणकगुणितं सज्जातं पृथग्गुणघ्नभाज्यशेषम् ६।८।१००~। एतानि~हरा\textendash \,११०\textendash \,दपास्य जातानि धनशेषाणि १०४।१०२।१०~। अथैतानि प्राग्वत्पञ्चगुणानि जाताः~शेष-फलपणाः \;{\small $\begin{matrix}
\mbox{{रू ~५२०}}\\
\mbox{{रू ~५१०}}\\
\vspace{-1mm}
\mbox{{रू ~~५०}}
\vspace{1mm}
\end{matrix}$}~। अथर्णशेषलब्धिं कालकमितां प्रकल्प्य प्राग्वज्जाता धनलब्धयः \;{\small $\begin{matrix}
\mbox{{का ~~~६ ~~रू ~१ं}}\\
\mbox{{का ~~~८ ~~रू ~१ं}}\\
\vspace{-1mm}
\mbox{{का ~१०० ~रू ~१ं}}
\vspace{1mm}
\end{matrix}$}~। शेषफलपणा लब्धपणयुता जाताः \;{\small $\begin{matrix}
\mbox{{का ~~~६ ~रू ~५१९}}\\
\mbox{{का ~~~८ ~रू ~५०९}}\\
\vspace{-1mm}
\mbox{{का ~१०० ~रू ~४९}}
\vspace{1mm}
\end{matrix}$}~। एते समा इति समशोधने कृते प्रथमबीजेनैव लब्धं कालकमानम् ५~। अनेन लब्धिषु कालकम् उत्थाप्य जाता लब्धयः २९।३९।४९९~। केवलक्रयलब्धिरप्युत्थापिता जाता ५~। इयं निरेका जाता केवलक्रयस्य धनलब्धिः ४~। केवलक्रयस्यर्णशेषमिदं १ हराच्च्युतं जातम् १०९~। लब्धिर्हर\textendash \,११०\textendash \,गुणा ४४० शेषयुता जातः क्रयः ५४९~। एवं यत्र द्व्यादिकमपि शेषं सम्भवति तत्र तदपि प्रकल्प्य क्रयः साध्यः~। यद्वा रूपं शेषं प्रकल्प्य साधितो यः क्रयः स एव द्व्यादिगुणोऽपि विधेयः~। एवमन्येऽपि प्रकाराः सन्ति ते ग्रन्थविस्तरभयान्न लिख्यन्ते~। एवं सर्वत्र यथा यथोपपन्नं भवति तथा तथा सुधीभिरूह्यम्~॥\\

अस्यानयनार्थं व्यक्तरीत्यैव सूत्रं कृतमस्मद्गुरुचरणैः {\color{violet}श्रीविष्णुदैवज्ञैः}\textendash

\begin{quote}
{\color{violet}'शेषविक्रयहतेष्टविक्रयः शीतरश्मिरहितो भवेत्क्रयः~।\\
पुंधनादधिक इष्टविक्रयः कल्प्य इत्थमवगम्य धीमता~॥'}
\end{quote}
\end{sloppypar}

\newpage

\begin{sloppypar}
एकस्य शेषफलस्य विक्रयलभ्याः पणा इह शेषविक्रयो विवक्षितः~। स चात्र पञ्च~। यदि तु शेषस्य विक्रयः पणदेयफलप्रमाणं शेषविक्रय इति विवक्षितं तदात्र पञ्चमांशः शेषविक्रयः~। अस्मिन्विवक्षिते शेषविक्रयहतेष्टविक्रय इति पठनीयम्~। पुंधनादित्यत्र~जात्ये-कवचनम्~। पुंसोर्धनं पुंधनम्~। शेषं स्पष्टम्~।\\

अथात्र प्रसङ्गात्स्वकृतमुदाहरणं लिख्यते\textendash

\begin{quote}
{\color{violet}'सप्तासन्मणिवणिजोऽत्र योऽधिकश्रीः स प्रादात्परधनसंमितं परेभ्यः~।\\
प्रत्येकं परसममेवमेव दत्वा ये जाताः सममणयोऽङ्ग किं धनास्ते~॥}
\end{quote}

अत्र मणिप्रमाणानि यावत्तावदादीनि प्रकल्प्यानेकवर्णसमीकरणेन साध्यानि~। अस्या-नयनार्थं व्यक्तरीत्या मत्सूत्रमप्यस्ति~। तद्यथा\textendash

\begin{quote}
{\color{violet}'वद सैकनरैर्मितमेकधनं द्विगुणं विधुहीनमिदं तु परम्~।\\
अमुना विधिना परतोऽपि परं द्विगुणं द्विगुणं द्वयमेव समम्~॥}
\end{quote}

उक्तवत्कृते जातानि धनानि ८।१५।२९।५७।११३।२२५।४४९~। द्वयमेव द्विगुणं द्विगुणं सत्समं समधनं भवति~। एतदुक्तं भवति~। नरद्वयं २ चेत्तर्हि द्वयं २ द्विगुणं सत् ४ समधनं भवति~। नरत्रयं चेत्पुनरेतत् ४ द्विगुणं ८ समधनं भवति~। नरचतुष्टयं चेत्पुनरिदं ८ द्विगुणं १६ समधनं भवतीत्यादि~। एवमत्र जातं समधनं १२८~। अन्यदिदं मत्कृतमुदाहरणम्\textendash

\begin{quote}
{\color{violet}'श्रीकृष्णेन यदिन्द्रनीलपटलं क्रीतं प्रियार्थं ततो\\
भागं भीष्मसुताष्टमं यदधिकं रूपं तदप्याददे~।\\
सत्याद्याः पुनरेवमेव विदधुः सप्ताप्यनालोकिताः\\
पत्युः प्रापुरिमाः पुनः समबलं सानन्दमादिं वद~॥}
\end{quote}

अत्र राशिः या १~। अयमष्टहृतो लब्धः कालकः का १~। कालकगुणं हरमग्रयुतं राशिसमं कृत्वाप्तं यावत्तावन्मानं का ८ रू १~। एक आलापोऽस्य घटते~। अथ राशेः सकाशादष्टमांशे रूपे चापनीते शेषं का ७~। पुनरिदमष्टहृतं लब्धं नीलकस्तगुणितहरमग्रयुतं नी ८ रू १~। राशिः का ७~। समं कृत्वा कुट्टकेन लब्धं कालकमानं सक्षेपं पी ८ रू ७~। अनेन राशिमुत्थाप्य जातो राशिः पी ६४ रू ५७~। अस्यालापद्वयं घटते~। शेषराशावुत्थापिते जातः शेषराशिः पी ५६ रू ४९~। अथ मुख्यराशेरालापद्वये कृते शेषराशेरेकालापे [च] कृते जातो द्वितीयशेषराशिः पी ४९ रू ४२~। पुनरयमष्टहृतो लब्धो लोहितस्तद्गुणं हरमग्रयुतं राशिसमं कृत्वा कुट्टकेन
\end{sloppypar}

\newpage

\begin{sloppypar}
\noindent लब्धं कालकमानं ह ८ रू ७~। अनेनोत्थापितो जातो राशिः ह ५१२ रू ५०५~। एवमग्रेऽपि~। नवमालापे त्वग्राभावाल्लब्धिगुणहर एव शेषराशिसमः कार्यः~॥~१४८~॥
\vspace{2mm}

\begin{quote}
{\color{violet}दैवज्ञवर्यगणसन्ततसेव्यपार्श्वबल्लाळसञ्ज्ञगणकात्मजनिर्मितेऽस्मिन्~।\\
बीजक्रियाविवृतिकल्पलतावतारे द्वित्र्यादिवर्णजसमीकृतिखण्डमेतत्~॥}
\end{quote}
\vspace{-4mm}

\begin{center}
इति श्रीसकलगणकसार्वभौमश्रीबल्लाळदैवज्ञसुतकृष्णदैवज्ञाविरचिते\\ बीजविवृतिकल्पलतावतारेऽनेकवर्णसमीकरणप्रथमखण्डविवरणम्~॥~९~॥
\vspace{2mm}

\rule{0.2\linewidth}{0.8pt}\\
\end{center}

अत्र ग्रन्थसङ्ख्या ४७३~। एवमादितो जाता ग्रन्थसङ्ख्या ३८६८~।

\begin{center}
\rule{0.2\linewidth}{0.8pt}\\
\vspace{-4mm}

\rule{0.2\linewidth}{0.8pt}
\end{center}
\end{sloppypar}

\newpage
\thispagestyle{empty}

\begin{center}
\textbf{\large १०\; अनेकवर्णसमीकरणान्तर्गतं मध्यमाहरणम्~।}\\
\rule{0.4\linewidth}{0.8pt}
\end{center}

\begin{sloppypar}
{\small एवमनेवर्णसमीकरणखण्डं प्रतिपाद्य मध्यमाहरणसञ्ज्ञं तद्विशेषं निरूपयितुं तदारम्भं प्रति-जानीते\textendash \,अथ मध्यमाहरणभेदा इति~। स्पष्टोऽर्थः~। वक्ष्यमाणसूत्रे पूर्वोत्तरार्धयोश्छन्दोभेदोऽस्तीति कस्यचिद्भ्रमः स्यात्तन्निरासार्थम् आह\textendash \,तत्र श्लोकोत्तरार्धादारभ्येति~। यदिह प्रथमतोऽर्धं पृच्छ्यते न तत्पूर्वार्धम्~। किं तु \hyperref[9.134]{'भूयः कार्यः कुट्टकः'} इति प्राक्पठितपूर्वार्धस्य श्लोकस्योत्तरार्धमित्यर्थः~। अथ शालिन्युत्तरार्धेनोपजातिकाद्वयेन च मध्यमाहरणस्येतिकर्तव्यतामाह\textendash }

\phantomsection \label{10.149}
\begin{quote}
{\large \textbf{{\color{purple}'वर्गाद्यं चेत्तुल्यशुद्धौ कृतायां पक्षस्यैकस्योक्तवद्वर्गमूलम्'~।\\
वर्गप्रकृत्या परपक्षमूलं तयोः समीकारविधिः पुनश्च~।\\
वर्गप्रकृत्या विषयो न चेत्स्यात्तदान्यवर्णस्य कृतेः समं तम्~।\\
कृत्वापरं पक्षमथान्यमानं कृतिप्रकृत्याद्यमितिस्तथा च~।\\
वर्गप्रकृत्या विषयो यथा स्यात्तथा सुधीभिर्बहुधा विचिन्त्यम्~॥~१४९~॥}}}
\end{quote}

एतत्सार्धसूत्रद्वयमाचार्यैरेव व्याख्यातम्~। \hyperref[10.149]{'वर्गप्रकृत्या विषयो यथा स्यात्तथा सुधीभिः बहुधा विचिन्त्यम्'} इत्युक्तम्~॥~१४९~॥\\

{\small तत्र यदि बुद्ध्यैव विचिन्त्यं तर्हि किं बीजेनेत्याशङ्कायामुत्तरं वसन्ततिलकयाह\textendash }

\phantomsection \label{10.150}
\begin{quote}
{\large \textbf{{\color{purple}बीजं मतिर्विविधवर्णसहायिनी हि मन्दावबोधविधये विबुधैर्निजाद्यैः~।\\
विस्तारिता~गणकतामरसांशुमद्भिर्या~सैव~बीजगणिताह्वयतामुपेता~॥~१५०~॥}}}
\end{quote}

अस्यार्थ आचार्यैरेव विवृतः~। \hyperref[10.149]{'पक्षस्यैकस्योक्तवद्वर्गमूलम्' 'वर्गप्रकृत्या परपक्षमूलम्'} इत्यादि पूर्वमुक्तं तत्र परपक्षः कीदृशः सन्वर्गप्रकृतेविषयो भवति~॥~१५०~॥\\

{\small अथ यदि विषयस्तर्हि विविधवर्गप्रकृत्या परपक्षमूले गृहीतेऽपि केन पदेन पूर्वपदसमीकरणं कार्यमित्यादि मन्दावबोधार्थमुपजातिकासिंहोद्धताभ्यां विशदयति\textendash }

\phantomsection \label{10.151}
\begin{quote}
{\large \textbf{{\color{purple}एकस्य पक्षस्य पदे गृहीते द्वितीयपक्षे यदि रूपयुक्तः~।\\
अव्यक्तवर्गोऽत्र कृतिप्रकृत्या साध्ये तदा ज्येष्ठकनिष्ठमूले~॥\\
ज्येष्ठं तयोः प्रथमपक्षपदेन तुल्यं कृत्वोक्तवत्प्रथमवर्णमितिः प्रसाध्या~।\\
ह्रस्वं~भवेत्प्रकृतिवर्णमितिः~सुधीभिरेवं~कृतिप्रकृतिरत्र~नियोजनीया~॥~१५१~॥}}}
\end{quote}
\end{sloppypar}

\newpage

\begin{sloppypar}
यत्र पक्षयोः समशोधने कृते सत्यव्यक्तवर्गादिकमवशेषं भवति तत्र पूर्ववत्पक्षौ तदेष्टेन निहत्य किञ्चित्क्षेप्यमित्यादिनैकपक्षस्य मूले गृहीते सति यदि द्वितीयपक्षेऽव्यक्तवर्गः सरूपः स्यात् तदासौ पक्षो वर्गप्रकृतेर्विषय इति वर्गप्रकृत्या मूले साध्ये~। तत्र वर्णवर्गे योऽङ्कः सा प्रकृतिः कल्प्या रूपाणि क्षेपः कल्प्यः~। एवं कनिष्ठज्येष्ठे साध्ये~। अथ तयोः ज्येष्ठकनिष्ठयोर्मध्ये ज्येष्ठं प्रथमपक्षपदेन समं कृत्वोक्तवत् \hyperref[7.89]{'एकाव्यक्तं शोधयेदन्यपक्षात्'} इत्यादिनैकवर्णसमीकरणेन प्रथमवर्णमितिः साध्या~। यस्य पक्षस्य पूर्वं पदं गृहीतं स प्रथमः~। तत्र यो वर्णः स प्रथमवर्णः~। प्रथमश्चासौ वर्णश्च प्रथमवर्ण इति कर्मधारये द्वितीय-वर्णाङ्कितपक्षस्य यदि प्रथमतः पदं गृह्यते तदा व्यभिचारः स्यात्~। अथ तयोर्मध्ये यत्कनिष्ठं तत्प्रकृतिवर्णमानं भवेत्~। अत्रोपपत्तिः~। \hyperref[8.115]{'अव्यक्तवर्गादि यदावशेषम्'} इत्यादिना यद्येकस्य पक्षस्य पदं लभ्यते तदावश्यं द्वितीयपक्षस्यापि पदेन भाव्यम्~। उभयोः समत्वात् तथा च समत्वेन जातो यः सरूपो व्यक्तवर्गः स वर्गराशिरेव~। अथ तज्ज्ञानार्थमुपायः~। स यथा~। वक्ष्यमाणोदाहरणे \hyperref[8.115]{'पक्षौ तदेष्टेन निहत्य'} इत्यादिना जातौ समौ पक्षौ \;{\small $\begin{matrix}
\mbox{{याव ~३६ ~या ~१२ ~रू ~१}}\\
\vspace{-1mm}
\mbox{{काव ~६ \hspace{10mm} रू ~१}}
\vspace{1mm}
\end{matrix}$}~। अत्राद्यपक्षस्य पदमिदं या ६ रू १~। समत्वाद्द्वितीयपक्षस्यापि पदेन भाव्यम्~। अत्र द्वितीयपक्षे कालकवर्गः षड्गुणितो रूपयुतोऽस्ति~। तस्माद्यस्य वर्गः षड्गुणितो रूपयुतो वर्गः स्यात्तदेव कालकमानं स्यात्~। अयं तु वर्गप्रकृतेर्विषयः~। को वर्गः षड्गुणो रूपयुतो वर्गः स्यादिति पर्य-वसानात्~। यस्य वर्गः षड्गुणो रूपयुतो वर्गः स्यात्तदिह कालकमानं तदेव कनिष्ठपदमपि~। अत उक्तं \hyperref[10.151]{'ह्रस्वं भवेत्प्रकृतिवर्णमितिः'} इति~। द्विकस्य २ वर्गः ४ षड्गुणो २४ रूपयुतो २५ वर्गो भवतीति द्वयं कालकमानम्~। यो जातो वर्गः २५ स एव द्वितीयपक्षः~। अस्य यत्पदं ५ तत्पूर्वपक्षपदेन तुल्यमेव~। पक्षयोस्तुल्यत्वात्~। अस्य वर्गस्य २५ यत्पदं ५ तज्ज्येष्ठमेव~। \hyperref[6.70]{'इष्टं ह्रस्वं तस्य वर्गः प्रकृत्या क्षुण्णो युक्तो वर्जितो वा स येन~। मूलं दद्यात्क्षेपकं तं धनर्णं मूलं तच्च ज्येष्ठमूलं वदन्ति'} इति प्रागुक्तेः~। अत उपपन्नं \hyperref[10.151]{'ज्येष्ठं तयोः प्रथमपक्षपदेन तुल्यम्'} इत्यादि~॥~१५१~॥\\

{\small अत्रोदाहरणमनुष्टुभाह\textendash }

\phantomsection \label{10.152}
\begin{quote}
{\large \textbf{{\color{purple}को राशिर्द्विगुणो राशिवर्गैः षङ्भिः समन्वितः~।\\
मूलदो जायते बीजगणितज्ञ वदाशु तम्~॥~१५२~॥}}}
\end{quote}

स्पष्टोऽर्थः~। गणितमाकरे स्पष्टम्~॥~१५२~॥

\end{sloppypar}

\newpage

\begin{sloppypar}
{\small अथानुष्टुभा रचितमाद्योदाहरणं शिष्यबुद्धिप्रसारार्थं लिखति\textendash }

\phantomsection \label{10.153}
\begin{quote}
{\large \textbf{{\color{purple}राशियोगकृतिर्मिश्रा राश्योर्योगघनेन च~।\\
द्विघ्नस्य घनयोगस्य सा तुल्या गणकोच्यताम्~॥~१५३~॥}}}
\end{quote}

स्पष्टोऽर्थः~। अत्र क्रिया यथा विस्तारं नैति तथैतौ या १ का १ं~। या १ का १ राशी प्रकल्प्य गणितं स्फुटमाकरे~। द्वितीयपक्षस्य वर्गप्रकृत्या पदं ग्राह्यमित्युक्तम्~॥~१५३~॥\\

{\small अथ यदि द्वितीयपक्षे साव्यक्तवर्गोऽव्यक्तवर्गवर्गः स्याद्यदि वा साव्यक्तवर्गवर्गो व्यक्तवर्गवर्गः स्यात्तदा नासौ वर्गप्रकृतेर्विषयस्तत्कथं पदं ग्राह्यमिति शङ्कायां मन्दावबोधार्थं सार्धोपजातिकयाह\textendash }

\phantomsection \label{10.154}
\begin{quote}
{\large \textbf{{\color{purple}द्वितीयपक्षे सति सम्भवे तु कृत्यापवर्त्यात्र पदे प्रसाध्ये~।\\
ज्येष्ठं कनिष्ठेन तथा निहन्याच्चेद्वर्गवर्गेण कृतोऽपवर्तः~॥\\
कनिष्ठवर्गेण तदा निहन्याज्ज्येष्ठं ततः पूर्ववदेव शेषम्~॥~१५४~॥}}}
\end{quote}

अत्र द्वितीयपक्षमिति पाठश्चेत्साधीयान्~। अथ सूत्रार्थः~। सम्भवे सति द्वितीयपक्षे कृत्यापवर्त्य पदे प्रसाध्ये~। एवं वर्गवर्गेणापवर्तसम्भवे सति वर्गवर्गेणापवर्त्य पदे प्रसाध्ये~। एतदुक्तं भवति~। द्वितीयपक्षे यदि साव्यक्तवर्गोऽव्यक्तवर्गवर्गोऽस्ति तदाव्यक्तवर्गेणापवर्ते कृते सरूपोऽव्यक्तवर्गः स्यादिति वर्गप्रकृतेर्विषयः स्यात्~। एवं द्वितीयपक्षे यदि साव्यक्त-वर्गवर्गोऽव्यक्तवर्गवर्गवर्गोऽस्ति तत्राव्यक्तवर्गवर्गेणापवर्ते कृते सति सरूपो व्यक्तवर्गः स्यादिति वर्गप्रकृतेर्विषयः स्यात्~। अतः प्राग्वत्पदे साध्ये~। इयांस्तु विशेषः~। अव्यक्त-वर्गेणापवर्ते कृते सति यज्ज्येष्ठमागतं तत्कनिष्ठेन गुणयेत्~। अव्यक्तवर्गवर्गेणापवर्ते तु यज्ज्येष्ठमागतं तत्कनिष्ठवर्गेण गुणयेत्~। कनिष्ठं तूभयत्र यथास्थितमेव~। एवं त्र्यादिगत-वर्गेणापवर्ते कनिष्ठवर्गवर्गादिना ज्येष्ठगुणनं द्रष्टव्यम्~। शेषं पूर्ववत्~। \hyperref[10.151]{'ज्येष्ठं तयोः~प्रथ-मपक्षपदेन तुल्यम्'} इत्यादि~।\\

अत्रोपपत्तिः~। यदा द्वितीयपक्षेऽव्यक्तवर्गवर्गोऽव्यक्तवर्गश्च स्यात्तदाव्यक्तवर्गेणापवर्ते कृते सरूपो व्यक्तवर्गः स्यात्~। अनेनापि वर्गेणैव भाव्यम्~। न हि वर्गराशिर्वर्गेण गुणितो भक्तो वा वर्गत्वं जहाति~। तदयं पक्षो येन वर्णमानेन कल्पितेन वर्गरूपः स्यात्तदेव प्रकृति-वर्णमानं प्राग्वत्~। अत्र जातो यो वर्गः स पूर्वोक्तयुक्त्या ज्येष्ठवर्ग एव~। परमेतस्य पदं न पूर्वपक्षपदसमम्~। अस्य पक्षस्याव्यक्तवर्गेणापवर्तनात्~। अतोऽसावपवर्तितपक्षो ज्येष्ठवर्ग-रूपोऽपवर्तेनाव्यक्तवर्गेण गुणितः सन्यथास्थितः स्यादिति पूर्वपक्षसमः स्यात्~। अव्यक्तस्य तु मानं व्यक्तमेव कनिष्ठरूपं ज्ञातमस्ति~। अतः कनि-
\end{sloppypar}

\newpage

\begin{sloppypar}
\noindent ष्ठवर्गेण गुणितो ज्येष्ठवर्गः पूर्वपक्षसमः स्यात्~। अतोऽस्य पदं पूर्वपक्षपदसममेव स्यात्~। अस्य पदं तु कनिष्ठगुणितं ज्येष्ठमेव~। अत उपपन्नं \hyperref[10.154]{'ज्येष्ठं कनिष्ठेन तदा निहन्यात्'} इति~। एवं वर्गवर्गेणापवर्ते कृते ज्येष्ठवर्गः प्रथमपक्षसाम्यार्थं कनिष्ठवर्गवर्गेण गुणनीयस्तस्य च पदं कनिष्ठवर्गगुणितं ज्येष्ठमेव~। अत उपपन्नं \hyperref[10.154]{'चेद्वर्गवर्गेण कृतोऽपवर्तः कनिष्ठवर्गेण तदा निहन्यात्~। ज्येष्ठम्'} इति~। एवं त्र्यादिगतवर्गेणापवर्तेऽप्युपपत्तिर्द्रष्टव्या~॥~१५५~॥\\

{\small अथ वर्गेणापवर्ते तावदुदाहरणमनुष्टुभाह\textendash }

\phantomsection \label{10.155}
\begin{quote}
{\large \textbf{{\color{purple}यस्य वर्गकृतिः पञ्चगुणा वर्गशतोनिता~।\\
मूलदा जायते राशिं गणितज्ञ वदाशु तम्~॥~१५५~॥}}}
\end{quote}

स्पष्टोऽर्थः~। गणितमाकरे स्पष्टम्~॥~१५५~॥\\

{\small अथ यत्र वर्गवर्गेणापवर्तः सम्भवति तादृशमुदाहरणमनुष्टुभाह\textendash }

\phantomsection \label{10.156}
\begin{quote}
{\large \textbf{{\color{purple}कयोः स्यादन्तरे वर्गो वर्गयोगो ययोर्घनः~।\\
तौ राशी कथयाभिन्नौ बहुधा बीजवित्तम~॥~१५६~॥}}}
\end{quote}

स्पष्टोऽर्थः~। गणिमाकरे व्यक्तम्~॥~१५६~॥\\

{\small अत्र यत्रैकस्य पक्षस्य पदे गृहीते सति द्वितीयपक्षे साव्यक्तोऽव्यक्तवर्गः सरूपोऽरूपो वा भवति तदासौ पक्षो वर्गप्रकृतेर्न विषयः~। अतस्तत्रोपायमुपजातिकोत्तरार्धेनोपजातिकया चाह\textendash }

\phantomsection \label{10.157}
\begin{quote}
{\large \textbf{{\color{purple}साव्यक्तरूपो यदि वर्णवर्गस्तदान्यवर्णस्य कृतेः समं तम्~।\\
कृत्वा पदं तस्य तदन्यपक्षे वर्गप्रकृत्योक्तवदेव मूले~॥\\
कनिष्ठमाद्येन पदेन तुल्यं ज्येष्ठं द्वितीयेन समं विदध्यात्~॥~१५७~॥}}}
\end{quote}

अत्र यदि द्वितीयपक्षे साव्यक्तो वर्णवर्ग इत्येव विवक्षितम्~। रूपेषु पुनरनास्था~। तानि भवन्तु मा वा~। शेषं स्पष्टम्~। व्याख्यातमप्याचार्यैः~। \\

अत्रोपपत्तिः~। एकस्य पक्षस्य पदे गृहीते सति यो द्वितीयपक्षे साव्यक्तोऽव्यक्तवर्गः सरूपोऽरूपो वा स्यात्स च वर्गराशिरेव~। अत उक्तं \hyperref[10.157]{'तदान्यवर्णस्य कृतेः समं तम्'} इति~। अत्र द्वितीयपक्षस्य प्रथमपक्षेणापि साम्यमस्ति कल्पिततृतीयवर्णवर्गेणापि साम्यमस्तीति प्रथमपक्षस्य तृतीयवर्णवर्गेण साम्यं बलाद्भाव्यम्~। तृतीयवर्णवर्गस्य यत्पदं स तृतीयवर्ण एव~। स एवान्यवर्ण इत्युच्यते~। अतः प्रथमपक्षपदस्यान्यवर्णेन साम्यं
\end{sloppypar}

\newpage

\begin{sloppypar}
\noindent स्यादित्यन्यवर्णमानस्य पूर्वपक्षपदेन साम्यम् उचितम्~। अथ द्वितीयपक्षस्यान्यवर्णवर्गेण समीकरणे कृते सत्यन्यवर्णपक्षोऽवश्यं वर्गप्रकृतेर्विषयः स्यात्~। तथा हि\textendash \,इह द्वितीयपक्षे यदि साव्यक्तोऽव्यक्तवर्गोऽस्ति तदान्यवर्णवर्गसमीकरणे \hyperref[9.134]{'आद्यं वर्णं शोधयेत्'} इत्यादिना शोधने कृतेऽपि पक्षौ यथास्थितावेव स्याताम्~। अथ \hyperref[8.116]{'चतुराहतवर्गसमैः'} इत्यादिना द्वितीय-पक्षेऽव्यक्तवर्गोऽव्यक्तं रूपाणि च तथा स्युर्यथा मूलं लभ्येत~। तृतीये तु सरूपोऽव्यक्तवर्गः स्यादित्ययं वर्गप्रकृतेर्विषयः~।\\

अथ यदि द्वितीयपक्षे साव्यक्तोऽव्यक्तवर्गः सरूपोऽस्ति तदान्यवर्णवर्गेण समीकरणे द्वितीयपक्षे साव्यक्तोऽव्यक्तवर्ग एव स्यात्~। तृतीये तु सरूपोऽव्यक्तवर्गः~। अत्रापि~\hyperref[8.116]{'चतु-राहतवर्गसमै रूपैः'} इत्यादिकरणे तृतीयपक्षे सरूपोऽव्यक्तवर्ग एव स्यादित्यवश्यं~वर्गप्रकृतेः विषयः~। इह \hyperref[8.116]{'चतुराहतवर्गसमै रूपैः'} इत्यादिकरणेऽपि समगुणक्षेपतया द्वितीयतृतीयपक्षौ साम्यं न त्यजतः~। प्रथमस्तु साम्यं त्यजति~। तत्र तथाकरणात्~। अतस्तृतीयपक्षस्य ज्येष्ठवर्गात्मकस्य यत् पदं ज्येष्ठस्वरूपं तद्द्वितीयपक्षपदेनैव समं भवितुम् अर्हति न प्रथम-पक्षपदेन~। अत उपपन्नं \hyperref[10.157]{'ज्येष्ठं द्वितीयेन समं विदध्यात्'} इति~। अथ तृतीयपक्षे वर्गप्रकृत्या पदे गृह्यमाणे यत्कनिष्ठं तदेव प्रागुक्तयुक्त्या तृतीयवर्णमानम्~। तच्च प्रथमपक्षपदेन तुल्यं भवितुमर्हति~। तृतीयवर्णवर्गस्य प्रथमपक्षसमत्वात्~। अत उक्तं \hyperref[10.157]{'कनिष्ठमाद्येन पदेन तुल्यम्'} इति~॥~१५७~॥\\

{\small अत्रोदाहरणमनुष्टुभाह\textendash }

\phantomsection \label{10.158}
\begin{quote}
{\large \textbf{{\color{purple}त्रिकादिद्व्युत्तरश्रेढ्यां गच्छे क्वापि च यत्फलम्~।\\
तदेव त्रिगुणं कस्मिन्नन्यगच्छे भवेद्वद~॥~१५८~॥}}}
\end{quote}

अतिरोहितार्थम्~। गणितं स्पष्टमाकरे~॥~१५८~॥\\

{\small अथ यद्येकस्य पक्षस्य पदे गृहीते सति द्वितीयपक्षे द्वित्र्यादयो वर्णवर्गा भवेयुस्तत्रोपायम् उपजातिकयाह\textendash }

\phantomsection \label{10.159}
\begin{quote}
{\large \textbf{{\color{purple}सरूपके वर्णकृती तु यत्र तत्रेच्छयैकां प्रकृतिं प्रकल्प्य~।\\
शेषं ततः क्षेपकमुक्तवच्च मूले विदध्यादसकृत्समत्वे~॥~१५९~॥}}}
\end{quote}

सरूपके इत्यत्रानियमः~। यदि रूपाणि भवेयुस्तर्हि तान्यपि क्षेपपक्षे प्रकल्प्यानि~। वर्ण-कृती इति द्विवचनोपादानाद्यत्र त्र्यादयो वर्णवर्गा भवेयुस्तत्र त्र्यादिवर्णानामिष्टानि व्यक्तानि मानानि प्रकल्प्य तैस्तानुत्थाप्य स्थापयेत्~। यदि तु रूपाण्यपि सन्ति तदा तेषु प्रक्षिपेत्~। एवं कृते सति सरूपके वर्णकृती एव भवतः~। अथात्र स्वेच्छयैकां वर्णकृतिं प्रकृतिं प्रकल्प्य यत्पक्षशेषं वर्णवर्गमात्रं सरूपं वा तत्क्षेपकं प्रकल्प्योक्तव-
\end{sloppypar}

\newpage

\begin{sloppypar}
\noindent न्मूले विदध्यात्~। अत्रापि प्रागुक्तयुक्त्या वर्णवर्गे योऽङ्कः सा प्रकृतिः~। अत्रेष्टं ह्रस्वमित्यादि-करणे कनिष्ठं व्यक्तं न कल्पनीयम्~। यतस्तथा सति शेषविधिना सरूपो वर्णवर्ग एव स्यादिति कथमपि न ज्येष्ठपदलाभः~। किं तु क्षेपसजातीयो वर्णः कनिष्ठं कल्प्यम्~। यतस्तथा सति तस्य वर्गः प्रकृत्या गुणितः क्षेपसजातीयो वर्णवर्गः स्यादित्युभयोः साजात्याद्योगे सति वर्णवर्ग एव स्यादतोऽस्य पदं सम्भवेत्~। क्षेपसजातीयवर्णोऽप्येकादिगुणितस्तथा कल्प्यो यथा शेषविधिनाङ्कतोऽपि मूलं लभ्येत~।\\

ननु यत्र सरूपो वर्णवर्गः क्षेपः स्यात्तत्र क्षेपसजातीयवर्णे कनिष्ठे कल्पितेऽपि~शेष-विधिना सरूपो वर्णवर्ग एव स्यादिति कथं ज्येष्ठपदलाभ इति चेत् सत्यम्~। तत्र क्षेप-सजातीयवर्णः सरूपः कनिष्ठं कल्पनीयम्~। तथा सति शेषविधिनाव्यक्तवर्गोऽव्यक्तं रूपाणि च स्युरिति ज्येष्ठपदं लभ्येत~। परं वर्णाङ्को रूपाङ्कश्च युक्त्या तथा कल्पनीयो यथा शेषविधौ कृते सत्यङ्कतोऽपि मूलं लभ्येत~। अथ यदि वर्गगता प्रकृतिरस्ति तदेष्टभक्तो द्विधा क्षेप इत्यादिना मूले साध्ये~। नन्वेवं कृतेऽपि कनिष्ठज्येष्ठयोरव्यक्तस्वरूपत्वाद्राशिमानमव्यक्तमेव स्यात्तत्किमनेनेत्यत आह\textendash \,असकृत्समत्व इति~।\\

अयमर्थः~। शेषालापविधिना यदि पुनः समीकरणं कर्तव्यमस्ति तदा राशिमानमव्यक्तं युक्तमेव~। यदि तु शेषालापविधिर्नास्ति तदा त्र्यादिवर्णानामिव द्वितीयवर्णस्यापि व्यक्तमेव मानं कल्पनीयम्~। तथा सति सरूपोऽव्यक्तवर्ग एव स्यादिति प्राग्वद्वर्गप्रकृत्या राशिमानं व्यक्तमेव सिध्येत्~। अत्रोपपत्तिः प्राग्वदेव~। इयांस्तु विशेषः~। तत्र प्रकृतिवर्णमानं व्यक्तं कल्पितमिह पुनरव्यक्तं व्यक्ताव्यक्तं वा कल्प्यत इति~॥~१५९~॥\\

{\small अत्रोदाहरणमनुष्टुभाह\textendash }

\phantomsection \label{10.160}
\begin{quote}
{\large \textbf{{\color{purple}तौ राशी वद यत्कृत्योः सप्ताष्टगुणयोर्युतिः~।\\
मूलदा स्याद्वियोगस्तु मूलदो रूपसंयुतः~॥~१६०~॥}}}
\end{quote}

अतिरोहितार्थः~। गणितमाकरे व्यक्तम्~॥~१६०~॥\\

{\small अथ यत्र प्रकृतिर्वर्गगता स्यात्तादृशमुदाहरणमनुष्टुभाह\textendash }

\phantomsection \label{10.161}
\begin{quote}
{\large \textbf{{\color{purple}घनवर्गयुतिवर्गो ययो राश्योः प्रजायते~।\\
समासोऽपि ययोर्वर्गस्तौ राशी शीघ्रमानय~॥~१६१~॥}}}
\end{quote}

स्पष्टोऽर्थः~। गणितमाकरे स्पष्टम्~॥~१६१~॥\\

{\small अथ 'यत्रैकपक्षस्य पदे गृहीते द्वितीयपक्षे यदि वर्णवर्गौ~। भावितं च स्यात्'}
\end{sloppypar}

\newpage

\begin{sloppypar}
\noindent {\small तत्रोपायमुपजातिकया\textendash }

\phantomsection \label{10.162}
\begin{quote}
{\large \textbf{{\color{purple}सभाविते वर्णकृती तु यत्र तन्मूलमादाय तु शेषकस्य~।\\
इष्टोद्धृतस्येष्टविवर्जितस्य दलेन तुल्यं हि तदेव कार्यम्~॥~१६२~॥}}}
\end{quote}

यत्र द्वितीयपक्षे वर्णवर्गौ सभावितौ स्यातां तत्र तदन्तर्वर्तिनो यावतो मूलं लभ्यते तावतो ग्राह्यम्~। अथ यच्छेषं तदिष्टेन भाज्यम्~। यल्लभ्यते तत्तेनैवेष्टेन वर्जितं च कार्यम्~। अथास्य दलेन पूर्वगृहीतस्य खण्डमूलस्य समीकरणं कार्यम्~। अत्र यद्यपि कियतः पक्षखण्डस्य मूलं ग्राह्यमिति नियमो न कृतोऽस्ति तथापि यथैकस्य वर्णवर्गस्य खण्डमात्रमवशिष्येत तथा पदं ग्राह्यमिति द्रष्टव्यम्~। अन्यथा क्रिया न निर्वहेत्~। इह शेषस्य वर्णवर्गस्य सजातीय-वर्णात्मकमिष्टं कल्पनीयम्~। अत्रापि राशिमानमव्यक्तमेव सिध्यतीति प्राग्वदसकृत्समत्वे सतीति द्रष्टव्यम्~। यदा तु शेषालापविधिर्नास्ति तदैकं राशिं व्यक्तमेव प्रकल्प्य क्रिया कार्या~।\\

अत्रोपपत्तिः~। एकस्य पक्षस्य पदे गृहीते सति यो द्वितीयः पक्षः समावितवर्णवर्गद्वया-त्मकोऽस्ति स वर्ग एव~। पक्षयोः समत्वात्~। अथ यावतस्तत्खण्डस्य मूलं लभ्यते तत्खण्डम् अपि वर्गराशिरेव~। कथमन्यथा तन्मूलं लभ्येत~। अतो बृहद्राशिवर्गात् सम्पूर्णपक्षाल्लघु-राशिवर्गात्मके पक्षखण्डेऽपनीते यच्छेषं तल्लघुबृहद्राश्योर्वर्गान्तरम् एव~। अतोऽन्तरम् इष्टं प्रकल्प्य {\color{violet}'वर्गान्तरं राशिवियोगभक्तम्'} इत्यादिना योगः स्यात्~। अतः शेषमिष्टोद्धृतं~जातो योगः~। अथाभ्यां योगान्तराभ्यां {\color{violet}'योगोऽन्तरेणोनयुतोऽर्धितः'} इति सङ्क्रमणेन राशी~स्याताम्~। तत्र {\color{violet}'योगोऽन्तरेणोनयुतोऽर्धितश्च'} बृहद्राशिः स्यात्स तु प्रयोजनाभावान्नोक्तः~। एवं~योगोऽन्त-रेण विवर्जितोऽस्य दलं लघुराशिः स्यात्~। तत्र शेषमिष्टोद्धृतं योगोऽस्ति~। अत इष्टकल्पि-तेनान्तरेण विवर्जितस्यास्य यद्दलं स लघुराशिरिति जातम्~। अथ प्राक्पृथक्कृतं पक्षखण्डं लघुराशिवर्गात्मकमस्तीति तत्पदमपि लघुराशिरेव~। अत एतयोरुभयोः समीकरणं कर्तुं युक्तमेव~। अत उपपन्नं \hyperref[10.162]{'शेषकस्य~। इष्टोद्धृतस्येष्टविवर्जितस्य दलेन तुल्यं हि तदेव कार्यम्'} इति~॥~१६२~॥\\

{\small अत्रोदाहरणमनुष्टुभाह\textendash }

\phantomsection \label{10.163}
\begin{quote}
{\large \textbf{{\color{purple}ययोर्वर्गयुतिर्घातयुता मूलप्रदा भवेत्~।\\
तन्मूलगुणितो योगः सरूपञ्चाशु तौ वद~॥~१६३~॥}}}
\end{quote}

स्पष्टोऽर्थः~। गणितमाकरे स्पष्टम्~॥~१६३~॥\\

{\small अत्र क्रियालाघवं प्रदर्शयितुं कस्यचिदुदाहरणं प्रदर्शयति\textendash }

\phantomsection \label{10.164.1}
\begin{quote}
{\large \textbf{{\color{purple}यत्स्यात्साल्पवधार्धतो घनपदं यद्वर्गयोगात्पदं\\
ये योगान्तरयोर्द्विकाभ्यधिकयोर्वर्गान्तरात्साष्टकात्~।}}}
\end{quote}
\end{sloppypar}

\newpage

\begin{sloppypar}
\phantomsection \label{10.164}
\begin{quote}
{\large \textbf{{\color{purple}यच्चैतत्पदपञ्चकं च मिलितं स्याद्वर्गमूलप्रदं\\
तौ राशी कथयाशु निश्चलमते षट्काष्टकाभ्यां विना~॥~१६४~॥}}}
\end{quote}

शार्दूलविक्रीडितमेतत्~। अत्र साल्पहतेर्दलादिति पाठश्चेत्साधीयान्~। यतोऽस्मिन्पाठे साल्पेति हतिविशेषणमसंशयं प्रतीयते~। शेषं स्पष्टम्~। अत्रालापानां बहुत्वे सकृत्क्रिया न निर्वहति~। अतो बुद्धिमता तथा राशी कल्प्यौ यथैकेनैव वर्णेन सर्वेऽप्यालापा घटेरन्~। तथाचार्यैः कल्पितौ याव १ रू १ं~। या २~। वा याव १ या २~। या २ रू २~। वा याव १ या २ं~। या २ रू २ं~। वा याव १ या ४ं रू ३~। या २ रू ४~। गणितं स्पष्टमाकरे~। एवमेवाचार्यैः स्वबुद्ध्या राशी प्रकल्प्य गणितं प्रदर्शितम्~॥~१६४~॥\\

{\small अथ मन्दार्थं राशिकल्पनोपाय आवश्यकः~। तत्र प्रतिपादकं सूत्रमेव यदि पठ्यते तदा कावेतौ राशी यदर्थमिदं सूत्रं प्रवृत्तमिति कस्यचिदनवबोधः स्यात्तन्निरासार्थमादौ प्रतिजानीतेऽनुष्टुभा\textendash }

\phantomsection \label{10.165}
\begin{quote}
{\large \textbf{{\color{purple}एवं सहस्रधा गूढा मूढानां कल्पना यतः~।\\
क्रियया कल्पनोपायस्तदर्थमथ कथ्यते~॥~१६५~॥}}}
\end{quote}

यथेह चतुर्धा राशिकल्पना कृतैवं राशिकल्पना सहस्रधास्ति~। सा यतो मूढानां गूढा अतस्तदर्थं मन्दार्थं क्रियया कल्पनोपायः कथ्यते~॥~१६५~॥\\

{\small अथ प्रतिज्ञातमुपायमुपजातिकेन्द्रवज्राभ्यामाह\textendash }

\phantomsection \label{10.166}
\begin{quote}
{\large \textbf{{\color{purple}सरूपमव्यक्तमरूपकं वा वियोगमूलं प्रथमं प्रकल्प्यम्~।\\
योगान्तरक्षेपकभाजिताद्यद्वर्गान्तरक्षेपकतः पदं स्यात्~॥\\
तेनाधिकं तत्तु वियोगमूलं स्याद्योगमूलं तु तयोस्तु वर्गौ~।\\
स्वक्षेपकोनौ हि वियोगयोगौ स्यातां ततः सङ्क्रमणेन राशी~॥~१६६~॥}}}
\end{quote}

स्पष्टोऽर्थः~। योगान्तरक्षेपकभाजितादित्युक्तेर्यत्र योगान्तरयोस्तुल्यः क्षेपकस्तत्रैवानेन सूत्रेण राशिकल्पनं न त्वतुल्ये क्षेप इति द्रष्टव्यम्~।\\

अत्रोपपत्तिः~। इह तावदिदं विचार्यते~। ययोर्योगान्तरे स्वक्षेपेण युते मूलदे स्यातां तयोर्वर्गान्तरं केन युतं मूलदं स्यादिति~। तत्रेदं सुप्रसिद्धं वर्गयोर्घातो घातवर्गो भवतीति~। क्षेपयुते च योगान्तरे योगवियोगमूलयोर्वर्गौ~। अतोऽनयोर्घातो योगवियोगमूलयोर्घातवर्गः स्यात्~। वर्गान्तरं तु केवलयोगान्तरघातः~। अतः केवलयोगान्तरयोर्घातस्य क्षेपयुतयोगा-न्तरघातस्य च यदन्तरं स वर्गान्तरक्षेपो भवितुमर्हति~। यतो वर्गान्तरे तेन क्षेपेण युते सति योग-
\end{sloppypar}

\newpage

\begin{sloppypar}
\noindent वियोगमूलयोर्घातवर्गः स्यादित्यतो मूलं लभ्येत~। तदन्तरं यथा~। तत्र क्षेपयुतयोगान्तरे यो १ क्षे १~। अं १ क्षे १~। अनयोर्घातार्थं न्यासः \;{\small $\begin{matrix}
\mbox{{यो ~१~। अं ~१ ~क्षे ~१}}\\
\vspace{-1mm}
\mbox{{क्षे ~१~। अं ~१ ~क्षे ~१}}
\vspace{1mm}
\end{matrix}$}~। घाते कृते जातो योगवियोगमूलयोर्घातवर्गः यो ० अं १ यो ० क्षे १ अं ० क्षे १ क्षेव १~। अत्र द्वितीयखण्डे क्षेपगुणितो योगोऽस्ति~। तत्र योगोऽन्यथा साध्यते~। योगमूलवर्गः क्षेपोनः सञ्जातो योगः योमूव १ क्षे १ं~। अयं क्षेपेण गुणितो जातं द्वितीयखण्डं योमूव ० क्षे १ क्षेव १ं~। अनयैव युक्त्या जातं तृतीयखण्डमपि अंमूव ० क्षे १ क्षेव १ं~। अत्रोभयत्र प्रथमखण्डे मूलवर्गः क्षेपगुणोऽस्ति~। अतोऽनयोर्योगे जातो मूलवर्गयोगः क्षेपगुणः योमूअंमूवयो ० क्षे १~। द्वितीयखण्डयोर्योगे जातं क्षेव २ं~। एवं जातो द्वितीयतृतीयखण्डयोगः योमूअंमूवयो ० क्षे १ क्षेव २ं~। एवं जातानि चत्वारि खण्डानि यो ० अं १ योमूअंमूवयो ० क्षे १ क्षेव २ं क्षेव १~। अत्रान्त्यखण्डयोर्योगे जातानि त्रीणि खण्डानि यो ० अं १ योमूअंमूवयो ० क्षे १ क्षेव १ं~। एवं जातो योगवियोगमूलयोर्घातवर्गः खण्डत्रयात्मकः~। तत्राद्यखण्डं वर्गान्तरम्~। इतरत्खण्डद्वयं वर्गान्तरक्षेपः~। तदेवं योगवियोगमूलयोर्घातवर्गे वर्गान्तरात्साध्यमाने खण्डद्वयात्मकः क्षेपो महान्भवति~। अथ योगवियोगमूलघातवर्गादल्पो वर्गो यदि वर्गान्तरात्साध्यते तदा क्षेपोऽपि लघीयान्स्यादतः क्षेपोनघातवर्गः साध्यते~। तत्र क्षेपोनो मूलघातोऽयं योमू ० अंमू १ क्षे १ं~। अस्य वर्गः {\color{violet}'स्थाप्योऽन्त्यवर्गः'} इत्यादिना जातः योमूव ० अंमूव १ योमू ० अंमू ० क्षे २ं क्षेव १~। अत्र प्रथमखण्डमूलघातवर्गोऽस्ति~। अतो मूलघातवर्गाद्यदि मूलयोर्द्विघ्नो घातः क्षेपगुणितः शोध्यते क्षेपवर्गश्च योज्यते तदा क्षेपोनघातवर्गस्य वर्गो भवतीति सिद्धम्~। तत्र पूर्वसिद्धोऽयमपि यो ० अं १ योमूअंमूवयो ० क्षे १ं क्षेव १ मूलघातवर्गः~। अथात्र क्षेपोनमूलघातस्य वर्गार्थं प्रागुक्तं शोध्यमिदं योमू ० अंमू ० क्षे २ं~। योज्यं चेदं क्षेव १~। योज्ये योजितेऽन्त्यखण्डनाशाज्जातं खण्डद्वयं यो ० अं १ योमूअंमूवयो ० क्षे १~। अत्र द्वितीयखण्डे मूलवर्गयोगः क्षेपगुणोऽस्ति~। शोध्यश्च द्विघ्नो मूलघातः क्षेपगुणः~। अत्रोभयत्र क्षेपो गुणोऽस्ति~। तत्र गुणितयोर्वियोगे वियुक्तयोर्वा गुणने कश्चिद्विशेषो नास्तीति प्रथमत एव वर्गयोगाद्द्विघ्ने घातेऽपनीते {\color{violet}'राश्योरन्तरवर्गेण द्विघ्ने घाते युते तयोः~। वर्गयोगो भवेत्'} इत्युक्तत्वाद्विलोमविधिना जातो मूलान्तरवर्गः~। स च क्षेपगुणः सञ्जातं द्वितीयखण्डं मूअंव ० क्षे १~। एवं जातः क्षेपोनघातस्य वर्गः खण्डद्वयात्मकः यो ० अं १ मूअंव ० क्षे १~। अत्र प्रथमखण्डं वर्गान्तरम्~। द्वितीयखण्डं वर्गान्तरक्षेपः~। अतः सिद्धमिदं योगान्तरक्षेपो मूलान्तर-
\end{sloppypar}

\newpage

\begin{sloppypar}
\noindent वर्गगुणितः सन्वर्गान्तरक्षेपो भवतीति~। अतो योगान्तरक्षेपेण वर्गान्तरक्षेपे भक्ते यल्लभ्यते स योगवियोगमूलान्तरवर्ग एव~। अस्य मूलं योगवियोगमूलयोरन्तरम् एव~स्यात्~। अतो वियोगमूलम् अनेन युक्तं सद्योगमूलं स्यात्~। इदं वा वियुक्तं सद्वियोगमूलं स्यात्~। अतः सुष्ठूक्तं \hyperref[10.166]{'योगान्तरक्षेपकभाजिताद्यद्वर्गान्तरक्षेपकतः पदं स्यात्~। तेनाधिकं तत्तु~वियोगमूलं स्याद्योगमूलम्'} इति~। एवं योगमूलं प्रथमतः सरूपमरूपं वा व्यक्तं प्रकल्प्य तत उक्तयुक्त्या वियोगमूलं साध्यम्~। एवं सिद्धाभ्यां योगवियोगमूलाभ्यां विलोमविधिना योगवियोगौ साध्यौ~। तत्र योगः सक्षेपोऽस्य मूलं योगमूलं भवतीति योगमूलं वर्गितं क्षेपोनं सद्योगः स्यात्~। एवं वियोगमूलाद्वियोगोऽपि स्यात्~। अत उक्तं \hyperref[10.166]{'तयोस्तु वर्गौ~। स्वक्षेपकोनौ हि वियोगयोगौ'} इति~। एवं योगवियोगसिद्धौ सङ्क्रमणेन राशिज्ञानं सुगमम्~। एतयो राश्योः मूलत्रयानुरोधेन सिद्धत्वादवश्यं मूलत्रयलाभः~। अवशिष्टपदद्वयलाभे तु न नियमोऽस्ति~। तदनुरोधेन राश्योरसिद्धेः~। अत एव वक्ष्यमाणोदाहरणे मूलत्रयानुरोधेन सिद्धयोरव्यक्त-राश्योः साल्पवधस्यार्धाद्घनपदं वर्गैक्याद्वा पदं न लभ्यते~।\\

ननु तर्हि प्रकृतोदाहरणे कथं पदद्वयलाभोऽस्तीति चेदुच्यते~। प्रकृते मूलत्रयानुरोधेन सिद्धयोरव्यक्तराश्योर्यादृशेन विधिना पदलाभोऽस्ति तादृशविधेरेवोद्दिष्टत्वात्~। तथा हि~। प्रकृते मूलत्रयानुरोधसिद्धावव्यक्तराशी याव १ रू १ं~। या २~। अनयोर्वधः याघ २ या २ं~। अयमल्पराश्यूनो द्विगुणो घनोऽस्ति~। अतोऽयं यदि साल्पोऽर्धितश्च क्रियते तदा घनो भवतीति घनपदं लभ्यते~। अतः प्रश्नविदा गणकेनायमेव विधिरुदाहरणे निबद्धः~। एवमत्र साल्पवधाच्चतुर्गुणादपि घनपदं लभ्यते~। अतोऽसावपि विधिर्यद्युदाहरणे निबध्यते चेत्तदा प्रकृतवदुद्दिष्टसिद्धिः स्यात्~। एवं वर्गैक्यपदेऽपि द्रष्टव्यम्~।\\

यदि पुनरव्यक्तराश्यनुरोधमपहाय स्वेच्छयैवोद्देशकालापः स्याद्यथात्रैवोदाहरणे~साल्प-वधाद्दशयुक्ताद्घनपदमिति तदा तु मूलत्रयानुरोधसिद्धाभ्यामव्यक्तराशिभ्यां नोद्दिष्टसिद्धिः~। न चेदं खिलम्~। षट्काष्टकयोर्वधात्साल्पात् ५४ दशयुतात् ६४ घनपदसम्भवात्~। तदेवं \hyperref[10.166]{'सरूपमव्यक्तमरूपकं वा'} इत्यादिना सिद्धयोरव्यक्तराश्योर्वियोगमूलयोगमूलवर्गान्तरमूला-न्येव नियतानि न तु पदपञ्चकमपि नियतमिति सिद्धम्~॥~१६६~॥\\

{\small अथास्य सूत्रस्य व्याप्तिं प्रदर्शयितुमुदाहरणं शार्दूलविक्रीडितेनाह\textendash }

\phantomsection \label{10.167}
\begin{quote}
{\large \textbf{{\color{purple}राश्योर्योगवियोगकौ त्रिसहितौ वर्गौ भवेतां तयोः \\
वर्गैक्यं चतुरूनितं रवियुतं वर्गान्तरं स्यात्कृतिः~।\\
साल्पं घातदलं घनः पदयुतिस्तेषां द्वियुक्ता कृतिः\\
तौ राशी वद कोमलामलमते षट् सप्त हित्वा परौ~॥~१६७~॥}}}
\end{quote}

स्फुटोऽर्थः~। अत्र कयो राश्योर्योगवियोगौ त्रिसहितौ वर्गौ भवेतामिति विचारे षट्क-

\end{sloppypar}

\newpage

\begin{sloppypar}
\noindent सप्तकयोः शीघ्रम् उपस्थितिर्भवति~। यदृच्छया चानयोः सर्वेऽप्यालापा घटन्त इत्यनभि-ज्ञोऽप्यस्य प्रश्नस्योत्तरं वदेदिति तन्निरासार्थमुक्तं षट् सप्त हित्वेति~। अत्र प्रथमं रूपोनम् अव्यक्तं या १ रू १ं वियोगमूलं प्रकल्प्योक्तसूत्रोक्तयुक्त्या राशी आनीय याव १ रू २ं~। या २ गणितमाकरे स्पष्टम्~॥~१६७~॥\\

{\small अथार्यया निबद्धमाद्योदाहरणं प्रदर्शयति\textendash }

\phantomsection \label{10.168}
\begin{quote}
{\large \textbf{{\color{purple}राश्योर्ययोः कृतिवियुती चैकेन संयुतौ वर्गौ~।\\
रहिते वा तौ राशी गणयित्वा कथय यदि वेत्सि~॥~१६८~॥}}}
\end{quote}

स्फुटोऽर्थः~। अत्र प्रथमोदाहरणे कल्पितौ राशी याव ४~। याव ५ रू १ं द्वितीयोदाहरणे राशी याव ४~। याव ५ रू १~। गणितं राशिकल्पने युक्तिश्चाकर एव स्फुटा~॥~१६८~॥\\

{\small अथैकस्य पक्षस्य पदे गृहीते सति द्वितीयपक्षे यदि सरूपमरूपं वाव्यक्तं भवति तत्रोपायम् अनुष्टुब्द्वयेनाह\textendash }

\phantomsection \label{10.169}
\begin{quote}
{\large \textbf{{\color{purple}यत्राव्यक्तं सरूपं हि तत्र तन्मानमानयेत्~।\\
सरूपस्यान्यवर्णस्य कृत्वा कृत्यादिना समम्~॥\\
राशिं तेन समुत्थाप्य कुर्याद्भूयोऽपरां क्रियाम्~।\\
सरूपेणान्यवर्णेन कृत्वा पूर्वपदं समम्~॥~१६९~॥}}}
\end{quote}

यत्राद्यपक्षमूले गृहीतेऽन्यपक्षेऽव्यक्तं सरूपमरूपं वा स्यात्तत्रान्यवर्णस्य सरूपस्य वर्गेण साम्यं कृत्वा तस्याव्यक्तस्य मानमानयेत्~। यत्र तु प्रथमपक्षस्य घनपदे गृहीतेऽन्यपक्षेऽव्यक्तं सरूपमरूपं वा स्यात्तत्रान्यवर्णस्य सरूपस्य घनेन साम्यं कृत्वा व्यक्तस्य मानमानयेत्~। कृत्यादिनेत्यादिपदोपादानात्~। अथागतेन वर्णात्मकेनाव्यक्तमानेन राशिमुत्थाप्य सरूपेण कल्पितेनान्यवर्णेनाद्यपक्षपदसाम्यं च कृत्वा पुनरन्यां क्रियां कुर्यात्~। यदि पुनः क्रिया नास्ति तदा सरूपस्यान्यवर्णस्य वर्गादिना समीकरणं न कार्यम्~। यतस्तथा कृते राशिमानमव्यक्तमेव स्यात्~। किं तु व्यक्तेनैव वर्गादिना समीकरणं कार्यम्~। यत एवं कृते राशिमानं व्यक्तमेव स्यात्~। अत्र व्यक्तवर्गो व्यक्तघनो वा तथा कल्प्यो यथा मानमभिन्नं स्यादिति~।\\

अत्रोपपत्तिः~। आद्यपक्षे पदे गृहीते द्वितीयपक्षे यदव्यक्तं केवलं सरूपं वा तदपि वर्ग एव~। आद्यपक्षतुल्यत्वात्~। अतः केनचिद्वर्गेण समीकरणमुचितम्~। तत्तु यदि
\end{sloppypar}

\newpage

\begin{sloppypar}
\noindent व्यक्तेनैव वर्गेण क्रियते तदा राशिर्व्यक्त एव स्यादिति शेषालापक्रियावतारो न स्यात्~। अत एव शेषक्रियाया अभाव इदमुचितमेव~। तस्माच्छेषविधौ कर्तव्येऽन्यवर्णस्य केवलस्य सरूपस्य वा वर्गेण समीकरणमुचितम्~। एवं सति सरूपस्यान्यवर्णस्येति यदुक्तं तस्यायम् आशयः~। यत्र द्वितीयपक्षे केवलमव्यक्तमस्ति तत्राव्यक्ताङ्कगुणितस्य केवलस्यान्यवर्णस्य वर्गेण~समी-करणे व्यक्तमानमभिन्नं स्यादिति यद्यपि तत्र केवलान्यवर्णवर्गेण समीकरणमुचितम्~। यत्रापि द्वितीयपक्षे सरूपमव्यक्तं तत्रापि यदि रूपाणामव्यक्ताङ्केनापवर्तसम्भवस्तर्ह्युक्तविधान्य-वर्णवर्गसमीकरणमुचितमेव~। यतः समशोधनेनान्यवर्णवर्गपक्षे पूर्वाव्यक्तपक्षजानि रूपाणि भवेयुः~। तथा सत्याद्यभक्ते पक्षेऽन्यस्मिन्निति कृते मानमभिन्नं स्यात्~। तथापि यत्राव्यक्ताङ्केन रूपाणां नापवर्तः सम्भवति तत्र केवलस्यान्यवर्णस्य वर्गेण समीकरणे मानं भिन्नमेव स्यात्~। अत उक्तं सरूपस्येति~। अत्र रूपाणि तथा कल्प्यानि यथा समशोधनेन रूपनाशो भवेदथवा व्यक्ताङ्केन तेषाम् अपवर्तः स्यात्~। रूपकल्पनोपायश्च \hyperref[10.178]{'हरभक्ता यस्य कृतिः'} इत्यादिर्वक्ष्यमाण उह्यः~। मन्दैस्तु हरभक्तेति स्थाने व्यक्ताङ्कभक्तेति पठित्वा वक्ष्यमाणविधिना रूपाणि~कल्प्यानि~। शेषं स्पष्टम्~॥~१६९~॥\\

{\small अत्रोदाहरणमनुष्टुभाह\textendash }

\phantomsection \label{10.170}
\begin{quote}
{\large \textbf{{\color{purple}यस्त्रिपञ्चगुणो राशिः पृथक्सैकः कृतिर्भवेत्~।\\
वद तं बीजमध्येऽसि मध्यमाहरणे पटुः~॥~१७०~॥}}}
\end{quote}

स्पष्टोऽर्थः~। गणितमाकरे स्पष्टम्~॥~१७०~॥\\

{\small पूर्वपक्षस्य घनपदे गृहीते सत्यन्यवर्णस्य घनेन समीकरणं कार्यमित्युक्तम्~। तत्रोदाहरणमाद्यैः अनुष्टुभा निबद्धं दर्शयति\textendash }

\phantomsection \label{10.171}
\begin{quote}
{\large \textbf{{\color{purple}को राशिस्त्रिभिरभ्यस्तः सरूपो जायते घनः~।\\
घनमूलं कृतीभूतं त्र्यभ्यस्तं कृतिरेकयुक्~॥~१७१~॥}}}
\end{quote}

स्पष्टोऽर्थः~। गणितमाकरे स्पष्टम्~॥~१७१~॥\\

{\small अथ विशेषप्रदर्शनाय परमुदाहरणमनुष्टुभाह\textendash }

\phantomsection \label{10.172}
\begin{quote}
{\large \textbf{{\color{purple}वर्गान्तरं कयो राश्योः पृथग्द्वित्रिगुणं त्रियुक्~।\\
वर्गौ स्यातां वद क्षिप्रं षट्कपञ्चकयोरिव~॥~१७२~॥}}}
\end{quote}

आपातविचारेणापि षट्कपञ्चकयोरुपस्थितिर्भवतीत्यनभिज्ञोऽप्यस्य प्रश्नस्योत्तरं वदेत्~। अत उक्तं षट्कपञ्चकयोरिवेति~। षट्कपञ्चकयोर्वर्गान्तरमुक्तविधमस्तीति प्रसिद्धमे-
\end{sloppypar}

\newpage

\begin{sloppypar}
\noindent वास्ति~। किं त्वेतयोः वर्गान्तरं यथोक्तविधम् अस्ति तथान्ययोः कयो राश्योः अस्तीति प्रश्नार्थः~॥~१७२~॥\\

{\small अत्र राश्योरव्यक्तकल्पने क्रिया न निर्वहतीत्यतो वर्गान्तरमेवाव्यक्तं कल्प्यमिति प्रदर्शयन्ननु-ष्टुभाह\textendash }

\phantomsection \label{10.173}
\begin{quote}
{\large \textbf{{\color{purple}क्वचिदादेः क्वचिन्मध्यात्क्वचिदन्त्यात्क्रिया बुधैः~।\\
आरभ्यते यथा लघ्वी निर्वहेच्च यथा तथा~॥~१७३~॥}}}
\end{quote}

स्पष्टोऽर्थः~। अत्र वर्गान्तरस्यैवं यद्यव्यक्तमानं कल्प्यते तर्हि~। \hyperref[10.170]{'यस्त्रिपञ्चगुणो राशिः'}~इति प्रागुक्तोदाहरणवत्सुखेनोदाहरणसिद्धिः स्यात्~। परमियान्विशेषः~। तत्र राशिरव्यक्तः कल्पित इति राशिमानमेव सिद्धम्~। इह तु वर्गान्तरमव्यक्तं कल्पितमिति राश्योर्वर्गान्तरमेव सिध्येत्~। अतोऽन्तरमिष्टं प्रकल्प्य {\color{violet}'वर्गान्तरं राशिवियोगभक्तम्'} इत्यादिना वर्गान्तराद्राशी साध्या-विति गणितं व्यक्तमाकरे~॥~१७३~॥\\

{\small अथ \hyperref[10.169]{'यत्राव्यक्तं सरूपं हि'} इत्यत्र विशेषमाह सार्धानुष्टुभा\textendash }

\phantomsection \label{10.174}
\begin{quote}
{\large \textbf{{\color{purple}वर्गादेर्यो हरस्तेन गुणितं यदि जायते~।\\
अव्यक्तं तत्र तन्मानमभिन्नं स्याद्यथा तथा~॥\\
कल्प्योऽन्यवर्णवर्गादिस्तुल्यं शेषं यथोक्तवत्~॥~१७४~॥}}}
\end{quote}

एतदाचार्यैरेव व्याख्यातम्~। 'यद्यप्येकस्य पक्षस्य पदे गृहीते सति द्वितीयपक्षे~यत् अव्यक्तम् अस्ति तस्मिन्वर्गादेर्हरेणान्येन वा केनचिद्गुणकेन गुणिते जाते सति न कश्चित् विशेषोऽस्तीति' पूर्वसूत्र एव \hyperref[10.174]{'तन्मानम् अभिन्नं स्याद्यथा तथा~। कल्प्योऽन्यवर्णवर्गादिः'} इति विशेषो यद्युक्तः स्यात् तदेवं सूत्रं व्यर्थम् एव~। तथाप्यन्यत्र राशिमाने भिन्नेऽप्यागते शेषविधिना राशिमानमभिन्नमेव स्यादिति तत्रायं विशेषो नोक्तः~। इह तु वर्गकुट्टके~शेष-विधेरभावादन्यवर्णवर्गसमीकरणमात्रेण यथा राशिरभिन्नः स्यात्तथा यतितव्यमिति~विशेष-स्यावश्यकत्वात्पृथक् सूत्रमपेक्षितमेव~। एवं घनकुट्टकेऽपि~।\\

ननु तथापि यत्र शेषविधेरभावस्तत्र तन्मानमभिन्नं स्याद्यथा तथान्यवर्णवर्गादिः कल्प्य इत्येतदर्थकं सूत्रमपेक्षितं न तु वर्गादेर्यो हर इत्यादीति चेन्न~। अन्यत्र राशिमाने भिन्नेऽप्यागते भवत्युद्दिष्टसिद्धिः~। इह तु न तथा~। न हि भिन्नराशिवर्ग उद्दिष्टक्षेपयुतोनो भिन्नेनोद्दिष्टहरेण भक्तः शुध्यति~। एवं घनोऽपि~। तस्मादत्र राशिमानस्याभिन्नतावश्यकतया \hyperref[10.174]{'वर्गादेर्यो हरस्तेन गुणितम्'} इत्याद्युक्तम्~।\\

ननु शेषविधेरभावश्चेत्तर्हि व्यक्तेनैव वर्गादिना समीकरणमस्त्विति चेदुच्यते~। तत्रापि व्यक्ताङ्कस्तथा कल्प्यो यथास्य वर्गेण समीकरणे राशिमानमभिन्नं स्यात्~।
\end{sloppypar}

\newpage

\begin{sloppypar}
\noindent इहापि व्यक्ताङ्ककल्पनमेव गरीयोऽस्ति~। न ह्यन्यवर्णकल्पने किञ्चित्काठिन्यमस्ति~। किं तु पूर्वाव्यक्ताङ्केन गुणित एव स कल्प्यते~। किं च व्यक्तवर्गादिना समीकरणे तदुत्पन्नो राशिः एक एव स्यात्~। इह तु क्षेपवशादनेके राशयः स्युरित्यस्ति महान्विशेष इत्यादि सुधीभिः ऊह्मम्~॥~१७४~॥\\

{\small अत्रोदाहरणद्वयमनुष्टुभाह\textendash }

\phantomsection \label{10.175}
\begin{quote}
{\large \textbf{{\color{purple}को वर्गश्चतुरूनः सन्सप्तभक्तो विशुध्यति~।\\
त्रिंशदूनोऽथवा कस्तं यदि वेत्सि वद द्रुतम्~॥~१७५~॥}}}
\end{quote}

स्पष्टोऽर्थः~। इदमुदाहरणद्वयं वर्गकुट्टकस्य~। कुट्टको हि गुणविशेषः प्रागुक्तः~। स इह वर्गरूपोऽस्ति~। यतोऽस्य प्रश्नस्यैकः केन वर्गेण गुणितश्चतुरूनः सन्सप्तभक्तो~विशुध्यती-त्यत्र पर्यवसानमस्ति~। एवं द्वितीयप्रश्नस्यापि~। एवमयमङ्कः केन घनेन गुणित उद्दिष्टक्षेप-युतोन~उद्दिष्टहरेण भक्तः शुध्यतीत्यत्र यः प्रश्नः पर्यवस्येत्स घनकुट्टकप्रश्नः~। गणितं स्पष्टम् आकरे~॥~१७५~॥\\

{\small तन्मानमभिन्नं यथा स्यात्तथान्यवर्णवर्गादिः कल्प्य इत्युक्तम्~। तत्र मन्दावबोधार्थमार्यया गीतिभ्यां च पूर्वैः पठितमुपायं प्रदर्शयति\textendash }

\phantomsection \label{10.176}
\begin{quote}
{\large \textbf{{\color{purple}हरभक्ता यस्य कृतिः शुध्यति सोऽपि द्विरूपपदगुणितः~।\\
तेनाहतोऽन्यवर्णो रूपपदेनान्वितः कल्प्यः~॥~१७६~॥\\
न यदि पदं रूपाणां क्षिपेद्धरं तेषु हारतष्टेषु~।\\ 
तावद्यावद्वर्गो भवति न चेदेवमपि खिलं तर्हि~॥~१७७~॥\\
हत्वा क्षिप्त्वा च पदं यत्राद्यस्येह भवति तत्रापि~।\\
आलापित एव हरो रूपाणि तु शोधनादिसिद्धानि~॥~१७८~॥}}}
\end{quote}

अस्यार्थः सोपपत्तिक उच्यते~। इह वर्गकुट्टके को वर्ग उद्दिष्टक्षेपेण युत ऊनो वोद्दिष्ट-हरभक्तः शुध्यतीत्यालापोऽस्ति~। तत्र राशौ यावत्तावदात्मके कल्पिते तस्य वर्गे यथासम्भवं क्षेपेण युत ऊने च कृते हरेण ह्रियमाणेऽस्मिल्लँब्धिर्न ज्ञायत इति लब्धिप्रमाणं कालकः कल्प्यते~। अथ हरगुणा लब्धिः स्वाग्रेण युता भाज्यसमा भवतीति सर्वत्र प्रसिद्धमस्ति~। इह त्वग्राभावाद्धरगुणितैव लब्धिर्भाज्यसमा भवितुमर्हति~। लब्धिश्चात्र कालकात्मकभक्तव्यम्~। अतो वर्गादेर्यो हरस्तेन गुणितमव्यक्तं द्वितीयपक्षो भवति~। पूर्वपक्षे तु यावत्तावद्वर्गः क्षेप-तुल्यानि रूपाणि च भवन्ति~। अथानयोः समशोधनेन पूर्वपक्षरूपाणि द्वितीयपक्षे भवन्ति~। एवमत्र द्वितीयपक्षे हरतुल्यो वर्णाङ्कः क्षेपतुल्यानि रूपाणि धनमृणं वा भवतीति सिद्धम्~।
\end{sloppypar}

\newpage

\begin{sloppypar}
अथ पूर्वपक्षस्य वर्गात्मकत्वात्पदे गृहीते द्वितीयवर्णाङ्केनोद्दिष्टपक्षोऽपि पूर्वपक्षसमत्वात् वर्ग एवेति कस्यचिदन्यवर्णस्य वर्गेण समः कर्तुं युज्यते~। परमन्यवर्णस्तथा कल्प्यो यथास्य वर्गो द्वितीयवर्णाङ्केनोद्दिष्टहरात्मकेन हृतः शुध्येत्~। तथा सति द्वितीयवर्णमानमभिन्नं स्यात्~।\\

ननु यस्य वर्णस्य सरूपस्यारूपस्य वा वर्गः प्रथमद्वितीयपक्षाभ्यां तुल्यतया कल्प्यते स तादृशो वर्णः पूर्वपक्षपदसमो भवितुम् अर्हतीति तयोः समीकरणेन राशिमानं सिध्येत्~। तद्यदि कदाचिद्भिन्नं स्यात् तदा कुट्टकेनाभिन्नं कर्तुं युज्यते~। द्वितीयवर्णस्तु न राशिः~। एवं सति तन्मानस्याभिन्नत्वार्थम् इयान्क्लेशो निरर्थक इति चेदुच्यते~। इह हि द्वितीयवर्णो निःशेषलब्धिः कल्पितास्ति~। सा यदि भिन्नापि स्यात् तदा स को राशिरस्ति यस्य वर्गः क्षेपयुतोनो हरभक्तो न शुध्येत्~। अपि तु सर्वस्यापि वर्ग उक्तविधः शुध्येदेव~। अतः प्रश्नो व्यर्थ एव स्यात्~। तस्माद्द्वितीयवर्णमानमभिन्नम् एव यथा भवति तथा यतितव्यम्~। तदर्थं \hyperref[10.176]{'हरभक्ता यस्य कृतिः'} इत्यादिसूत्रस्य प्रवृत्तिः~। तत्र द्वितीयपक्षे हरतुल्यो वर्णाङ्कः क्षेपतुल्यानि रूपाणि च भवन्तीति स्थितम्~। क्षेपाभावे तु हरगुणितो वर्ण एवैष भवति नतु रूपाणि~। तत्र रूपाभावे तावदुच्यते~। यस्य कृतिर्हरभक्ता सती शुध्यति तेनाङ्केन गुणितोऽन्यवर्णः कल्प्यः~। तथा सति तस्य वर्णस्य वर्गो हरभक्तः शुध्येदेव~। अत एतादृशेऽन्यवर्णवर्गे कल्पिते द्वितीयवर्णमानमभिन्नं स्यात्~। अत्र यद्यपि हरगुणितेऽन्यवर्णे कल्पिते तस्य वर्गो हरभक्तः शुध्येदेवेति हरगुणितोऽन्यवर्णः कल्प्य इत्येव वक्तुमुचितं लाघवात्तथापि योऽत्र कल्पितोऽन्यवर्णः स एव राशौ क्षेपः पर्यवस्यति~। एवं सति हरान्न्यूने तदङ्के सम्भवति सति यदि हरतुल्यस्तदङ्कः कल्प्यते तदा क्षेपो महान् स्यादिति न सकलराशिलाभः~। यथा कुट्टकेऽनपवर्तितहरभाज्ययोः क्षेपत्वे कल्पिते न सकलगुणलब्धिलाभः किं तु दृढयोस्तयोः क्षेपत्वे सकलगुणलब्धिलाभोऽस्ति तद्वदिहापि~। अतः सकलराशिलाभार्थं \hyperref[10.176]{'हरभक्ता यस्य कृतिः'} इत्याद्युक्तम्~। अत्र यस्य न्यूनतमस्येति द्रष्टव्यम्~। अन्यथा क्षेपमहत्त्वेन दोषतादवस्थ्यं गौरवं च स्यादिति~।\\

अथ यदि द्वितीयपक्षे रूपाणि सन्ति तदा तानि रूपाणि हरभक्तानि शुध्यन्ति न वेति विचारणीयम्~। यद्येतानि शुध्यन्ति तदा प्राग्वदेव हरभक्ता यस्य कृतिः शुध्यति~तेना-हतोऽन्यवर्णः कल्प्यः~। उक्तयुक्तेरविशेषात्~। किं तु समशोधनेन द्वितीयपक्षरूपाण्यन्य-वर्णवर्गपक्षे भवन्ति~। तान्यपि यदि हरभक्तानि शुध्यन्ति तदा वर्णमानमभिन्नं सिद्धमेव~। \\

अथ यदि द्वितीयपक्षगतानि रूपाणि हरभक्तानि न शुध्यन्ति तदा प्राग्वदन्यवर्ण-कल्पनेऽपि समशेाधनेन द्वितीयपक्षरूपाणां तृतीयपक्षे गमने तेषां हरेणाशुद्धेर्द्वितीयवर्ण-
\end{sloppypar}

\newpage

\begin{sloppypar}
\noindent मानमभिन्नं स्यात्~। तदर्थं तृतीयपक्षस्तथा कल्प्यो यथा तत्र द्वितीयपक्षरूपतुल्यानि रूपाणि स्युः~। यतस्तथा सति समशोधनेन रूपाभावः स्यादिति प्रागुक्तयुक्त्या द्वितीयवर्णमानमभिन्नं स्यात्~। परं द्वितीयपक्षरूपतुल्यानि तृतीयपक्षरूपाणि तदैव स्युर्यदि द्वितीयपक्षरूपपदेन युत ऊनो वान्यवर्णः कल्पे(ल्प्ये)त~। यतस्तस्य वर्गे यथापूर्वं रूपाणि स्युः~। अत उक्तं~\hyperref[10.176]{'तेना-हतोऽन्यवर्णो रूपपदेनान्वितः कल्प्यः'} इति~। अन्वित इत्युपलक्षणम्~। ऊनोऽपि कल्प्यः~। युक्तेरविशेषात्~।\\

ननु रूपयुते रूपोने वान्यवर्णे कल्पिते तस्य वर्गे क्रियमाणेऽन्यवर्णवर्गोऽन्यवर्णो रूपाणि चेति खण्डत्रयं स्यात्~। तत्र समशोधनेन रूपनाशे खण्डद्वयमवशिष्यते~। तत्र यद्यपि वर्गात्मकं प्रथमखण्डं प्रागुक्तयुक्त्या हरभक्तं शुध्यति तथापि वर्णात्मकं द्वितीयखण्डं शुध्येदेवेति कथम् अवमन्तव्यम् इति चेदुच्यते~। इह प्रथमखण्डे {\color{violet}'स्थाप्योऽन्त्यवर्गः'} इति कल्पिताङ्कः कस्य कृतिर्भवति द्वितीयखण्डे तु {\color{violet}'द्विगुणान्त्यनिघ्ना अपरेऽङ्काः'} इत्यनेन कल्पिताङ्को द्वाभ्यां रूपपदेन च गुणितो भवति~। इदं खण्डद्वयमपि यथा हरभक्तं शुध्यति तथाङ्कः कल्प्यः~। अत एवोक्तं \hyperref[10.176]{'यस्य कृतिर्हरभक्ता शुध्यति'}~। अपि च सोऽङ्को द्विरूपपदगुणितोऽपि शुध्यति तदा तेनाङ्केनाहतोऽन्यवर्णः कल्प्य इति हरगुणितान्यवर्णकल्पने तु न कोऽपि विचारोऽस्ति~। यतः स स्वत एव हरभक्तः शुध्यतीति स्वगुणितो द्विरूपपदगुणितो वा सुतरां शुध्येत्~। सोऽपीति स्थाने योऽपीति पाठश्चेत्साधीयानिति प्रतिभाति~।\\

अथ यदि द्वितीयपक्षरूपाणां पदं न लभ्यते तदा तृतीयपक्षरूपाणां द्वितीयपक्ष-रूपसाम्यं कथमपि न स्यात्~। तृतीयपक्षो हि मूलदः कल्पनीयः~। यतोऽस्य पदेन~प्रथम-पक्षपदसाम्यं विधेयमस्ति~। अतोऽत्र रूपैर्मूलदैरेव भाव्यम्~। द्वितीयपक्षे तु रूपाण्यमूल-दानि सन्तीति कथमुभयोः पक्षयो रूपसाम्यं स्यात्~। अत एतादृशे स्थले समशोधनोत्तरं रूपशेषेणावश्यं भाव्यम्~। अतस्तृतीयपक्षे रूपवर्गस्तथा कल्प्यो यथा तस्य द्वितीयपक्षरूपैः सहान्तरमेकादिगुणितहरतुल्यं स्यात्~। यतस्तथा सति तच्छेषं हरभक्तं शुध्येदेवेति द्वितीय-वर्णमानमभिन्नं स्यात्~।\\

अथ तादृशवर्गज्ञानार्थमुपायः~। द्वितीयपक्षरूपेष्वेकादिगुणितहरे योजिते शोधिते वा यो वर्गः स्यात्तस्य तैः सहान्तरं गुणितहरतुल्यमेव स्यादतस्तादृशवर्गार्थं द्वितीयपक्षरूपेषु तावद्धरं क्षिपेद्यावद्वर्गः स्यात्~। तत्र रूपेषु हरतष्टेषु हरयोजनेनैव शोधनजं योगजं च फलं सिध्यतीति लाघवादिदमेव वक्तुमुचितम्~। अत उक्तं \hyperref[10.176]{'न यदि पदं रूपाणां क्षिपेद्धरं तेषु हारतष्टेषु~। तावद्यावद्वर्गः'} इति~। अस्य वर्गस्य पदेनान्वितोऽन्यवर्णः कल्प्य इत्यर्थतः सिद्धम्~।
\end{sloppypar}

\newpage

\begin{sloppypar}
अत्रेदमपि द्रष्टव्यम्~। यदि रूपाणि हरतष्टानि मूलदानि स्युस्तदा तत्पदेनान्वितोऽन्यवर्णः कल्प्य इति~। उक्तयुक्तेरविशेषात्~। अथैवं कृतेऽपि यदि वर्गो न स्यात्तदा नास्त्येव तादृशो वर्गो यस्य द्वितीयपक्षरूपैः सहान्तरमेकादिगुणितहरतुल्यं स्यादिति सिद्धमुद्दिष्टस्य खिलत्वम्~। अत उक्तं \hyperref[10.176]{'भवति न चेदेवमपि खिलं तर्हि'} इति~। अथ यत्र द्वित्रिपञ्चादिगुणितो वर्ग उद्दिष्टः स्यात्तत्र समशोधनमात्रेण पूर्वपक्षपदलाभात्~। \hyperref[8.115]{'पक्षौ तदेष्टेन निहत्य'} इत्यादिना प्रथमपक्षपदे गृहीते द्वितीयपक्षे वर्णाङ्को हरतुल्यो न स्यात्किं त्वष्टगुणितः स्यात्~। रूपाण्यपि क्षेपतुल्यानि न स्युः किं तु गुणितानि स्युः~। अतस्तत्रापि प्रागुक्तयुक्त्या यस्य कृतिर्गुणितहरतुल्येन द्वितीयवर्णाङ्केन भक्ता सती शुध्यतीत्यादिनान्यवर्णकल्पनं युक्तं भवति~। एवं सति \hyperref[10.176]{'हत्वा क्षिप्त्वा च पदं यत्राद्यस्येह भवति तत्रापि~। आलापित एव हरः'} इति यदुक्तं तल्लाघवार्थं द्रष्टव्यम्~।\\

ननु गुणितहरस्थाने केवलहरे कृते पक्षसाम्यं कथं तिष्ठेत्~। साम्याभावे च साम्य-प्रयुक्तः शेषविधिः कथं स्यात्~। \hyperref[10.176]{'आलापित एव हरः'} इति यदुक्तं तदयुक्तम्~। अथ चेत् अप्रामाणिकमपि लाघवमूरी क्रियते तर्हि हरार्धाधिकमपि कथं न गृह्यते~। कथं वा~रूपाणि अप्यगुणितान्येव न गृह्यन्त इति चेदुच्यते~। आलापितहरेऽपि गृहीते पक्षसाम्यं न हीयते~। तथा हि~। वर्गे द्वित्र्यादिगुणित उद्दिष्टे सति लब्धिप्रमाणं गुणकभक्तकालकः कल्प्यते~। अथायं हरगुणः सन्द्वितीयपक्षो भवतीति प्राग्वद्द्वितीयपक्षे हर एव वर्णाङ्कः स्यात्~। परमुद्दिष्ट-गुणकस्तस्य च्छेदः स्यात्~। अथ समच्छेदीकरणायानेन च्छेदेन पूर्वपक्षस्य गुणने कर्तव्य उद्दिष्टगुणेन यावद्वर्गस्य भूयो गुणनं भवतीति गुणवर्गगुणितो यावद्वर्गो भवति~। क्षेपकस्तु समच्छेदीकरणावसर एव गुण्यत इति क्षेपतुल्यानि रूपाणि गुणकगुणितानि भवन्ति~। अथ च्छेदगमे कृते समशोधनेन तादृशरूपाणां द्वितीपक्षगमने सति प्रथमपक्षस्य मूलत्वात् पदे गृहीते सति द्वितीयपक्षे केवलहरो वर्णाङ्को भवति~। रूपाणि तु गुणितानि भवन्ति~। अत आद्यैरमुं कल्पनश्रमं परित्यज्य कालकमेव लब्धिप्रमाणं प्रकल्प्य शेषविधिना सिद्धे द्वितीयपक्षे गुणितहरस्थाने केवलहरग्रहणमात्रमुक्तम्~। लाघवात्~।\\

ननु तथाप्यालापित एव हर इत्यवधारणमयुक्तम्~। गुणितहरग्रहणेऽप्युद्दिष्टसिद्धेरिति चेत्सत्यम्~। यत्राद्यस्य पक्षस्य हत्वा क्षिप्त्वा च पदं भवति तत्राप्यालापित एव हरो ग्राह्यः~। किं गुणितहरेणेति वाक्यपर्यवसानस्य विवक्षितत्वादवधारणं नास्त्येव~। अवधारणे तु विवक्षित आलापित एव हरो ग्राह्यो न तु गणित इति वाक्यपर्यवसानं स्यात्~। अत्र क्षिप्त्वेति यदुक्तं तत्प्रथमराशौ सरूपे कल्पिते सतीति द्रष्टव्यम्~। यद्वा \hyperref[8.115]{'पक्षौ तदेष्टेन निहत्य किञ्चित्क्षेप्यं तयोः'} इत्येतदर्थकस्याद्यसूत्रस्य स्मारकं हत्वा क्षिप्त्वेति~। तथा चायमर्थः~। यस्मिन्सूत्रे हत्वा क्षिप्त्वा चेत्यादिना पदग्रहणमुक्तं
\end{sloppypar}

\newpage

\begin{sloppypar}
\noindent तत्सूत्रप्रवृत्तिपूर्वकं यत्राद्यस्य पदं भवतीति~। एवं घनकुट्टकेऽपि योज्यम्~। तद्यथा\textendash \,तत्रा-प्युक्तवद्द्वितीयपक्षे हर एव वर्णाङ्को भवति~। तत्र रूपाणामभावे हरभक्तानां तेषां शुद्धौ वा यस्य घनो हरभक्तः शुध्यति तेनाङ्केनाहतोऽन्यवर्णः कल्प्यः~। यदि तु रूपाणां हरेण न शुद्धि-स्तदा रूपाणां घनपदेनान्वित ऊनो वान्यवर्णः कल्प्यः~। यदि तु रूपाणां घनमूलं न लभ्यते तदा तेषु रूपेषु हरतष्टेषु तावद्धरं क्षिपेद्यावद्घनो भवेत्~। एवमपि कृते यदि घनो न भवेत्तदा तदुद्दिष्टं खिलं ज्ञेयम्~।\\ 

अथ रूपपदेनान्वितस्य कल्प्यमानवर्णस्य घने स्थाप्यो घनोऽन्त्यस्येत्यादि चत्वारि खण्डानि भवन्ति~। तत्र रूपात्मकस्य चतुर्थखण्डस्य प्रागुक्तयुक्त्या शुद्धिर्भवति~। अथ त्रयाणां शेषखण्डानां हरभक्तानां यथा शुद्धिर्भवति तथाङ्कः कल्प्यः~। तत्र प्रथमखण्डे कल्प्यमानाङ्कस्य घनो भवेत्~। द्वितीये तस्य वर्गो रूपघनपदेन त्रिभिश्च गुणितो भवेत्~। तृतीये रूपघनपदस्य वर्गेण त्रिभिश्च गुणितो भवेत्~। अतो यस्य घनो हरभक्तः शुध्यत्यपि च यस्य वर्गस्त्रिरूपपदगुणितो हरभक्तः शुध्यत्यपि च यो रूपपदवर्गेण त्रिभिश्च गुणितो हरभक्तः शुध्यति तेनाङ्केनाहतोऽन्यवर्णः कल्प्यः~। हरगुणितान्यवर्णकल्पने तु न कोऽपि विचारः~। वर्गकुट्टके तु यदि लब्धिप्रमाणं कालकवर्गः कल्प्यते तदान्यवर्णकल्पनं विनैव सुखेनोदा-हरणसिद्धिरस्ति~। यतस्तत्राद्यपक्षपदे गृहीते द्वितीयपक्षस्य वर्गप्रकृत्या पदमायाति~। एवं सत्यपि यदन्यथा यतितम् आचार्यैस्तदवर्गगतलब्धावप्युदाहरणसिद्ध्यर्थम् इत्यादि सुधीभिः ऊह्यम्~॥~१७८~॥\\

{\small अथ घनकुट्टकोदाहरणमनुष्टभाह\textendash }

\phantomsection \label{10.179}
\begin{quote}
{\large \textbf{{\color{purple}षड्भिरूनो घनः कस्य पञ्चभक्तो विशुध्यति~।\\
तं वदास्ति तवालं चेदभ्यासो घनकुट्टके~॥~१७९~॥}}}
\end{quote}

स्पष्टोऽर्थः~। गणितमाकरे स्पष्टम्~॥~१७९~॥ \\

{\small अथ हत्वा क्षिप्त्वेत्यस्योदाहरणमनुष्टुभाह\textendash }

\phantomsection \label{10.180}
\begin{quote}
{\large \textbf{{\color{purple}यद्वर्गः पञ्चभिः क्षुण्णस्त्रियुक्तः षोडशोद्धृतः~।\\
शुद्धिमेति समाचक्ष्व दक्षोऽसि गणिते यदि~॥~१८०~॥}}}
\end{quote}

स्पष्टोऽर्थः~। गणितमाकरे व्यक्तम्~॥~१८०~॥ 
\end{sloppypar}

\newpage

\begin{quote}
{\color{violet}दैवज्ञवर्यगणसन्ततसेव्यपार्श्वबल्लालसञ्ज्ञगणकात्मजनिर्मितेऽस्मिन्~।\\
बीजक्रियाविवृतिकल्पलतावतारेऽभून्मध्यमाहरणमेतदनेकवर्णे~॥}
\end{quote}
\vspace{-4mm}

\begin{center}
इति श्रीसकलगणकसार्वभौमश्रीबल्लाळदैवज्ञसुतकृष्णगणकविरचिते \\
बीजविवृतिकल्पलतावतारेऽनेकवर्णसमीकरणभेदस्य मध्यमाहरणस्य विवरणम्~॥~१०~॥
\vspace{2mm}

\rule{0.2\linewidth}{0.8pt}\\
\end{center}

अत्र ग्रन्थसङ्ख्या ४५० पञ्चाशदधिकचतुःशतानि~। एवमादितो ग्रन्थसङ्ख्या ४३१८~।

\begin{center}
\rule{0.2\linewidth}{0.8pt}\\
\vspace{-4mm}

\rule{0.2\linewidth}{0.8pt}
\end{center}

\newpage
\thispagestyle{empty}

\begin{center}
\textbf{\large ११\; भावितम्~।}\\
\rule{0.2\linewidth}{0.8pt}
\end{center}

\begin{sloppypar}
{\small अथ क्रमप्राप्तं भावितसञ्ज्ञमनेकवर्णविशेषमुपजातिकयाह\textendash }

\phantomsection \label{11.181}
\begin{quote}
{\large \textbf{{\color{purple}मुक्त्वेष्टवर्णं सुधिया परेषां कल्प्यानि मानानि यथेप्सितानि~।\\
तथा भवेद्भावितभङ्ग एवं स्यादाद्यबीजक्रिययेष्टसिद्धिः~॥~१८१~॥}}}
\end{quote}

स्पष्टार्थमिदम्~। विवृतं चाचार्यैः~। द्वितीयादिराशीनां व्यक्तकल्पनेनास्य विषयस्यैक-वर्णसमीकरणान्तर्गतत्वादुपपत्तिरत्र तदुपपत्तिरेव~॥~१८१~॥ 
\vspace{2mm}

{\small अत्रोदाहरणमनुष्टुभाह\textendash }

\phantomsection \label{11.182}
\begin{quote}
{\large \textbf{{\color{purple}चतुस्त्रिगुणयो राश्योः संयुतिर्द्वियुता तयोः~।\\
राशिघातेन तुल्या स्यात्तौ राशी वेत्सि चेद्वद~॥~१८२~॥}}}
\end{quote}

स्पष्टोऽर्थः~। गणितमाकरे स्पष्टम्~॥~१८२~॥ 
\vspace{2mm}

{\small उदाहरणान्तरमनुष्टुभाह\textendash }

\phantomsection \label{11.183}
\begin{quote}
{\large \textbf{{\color{purple}चत्वारो राशयः के ते यद्योगो नखसङ्गुणः~।\\
सर्वराशिहतेस्तुल्यो भावितज्ञ निगद्यताम्~॥~१८३~॥}}}
\end{quote}

स्पष्टोऽर्थः~। गणितमाकरे स्पष्टम्~॥~१८३~॥ 
\vspace{2mm}

{\small शिष्यबुद्धिप्रसारार्थमन्यदुदाहरणद्वयं शार्दूलविक्रीडितेनाह\textendash }

\phantomsection \label{11.184}
\begin{quote}
{\large \textbf{{\color{purple}यौ राशी किल या च राशिनिहतिर्यौ राशिवर्गौ तथा~।\\
तेषामैक्यपदं सराशियुगुलं जातं त्रयोविंशतिः~।\\
पञ्चाशत्त्रियुताथवा वद कियत्तद्राशियुग्मं पृथक्\\
कृत्वाभिन्नमवेहि वत्स गणकः कस्त्वत्समोऽस्ति क्षितौ~॥~१८४~॥}}}
\end{quote}

स्पष्टोऽर्थः~। गणितमाकरे स्पष्टम्~॥~१८४~॥ 
\vspace{2mm}

{\small अत्रैकस्मिन्राशौ व्यक्ते कल्पिते द्वितीयो राशिर्बहुधा भिन्न एवायाति~। कदाचिदभिन्नोऽपि~। अतोऽभिन्नराशिसिद्धिर्महतायासेन भवति~। तदर्धं यथाल्पायासेन राशिमानमभिन्नं सिध्यति तथा सार्धानुष्टुब्द्वयेनाह\textendash }

\phantomsection \label{11.185}
\begin{quote}
{\large \textbf{{\color{purple}भावितं पक्षतोऽभीष्टात्त्यक्त्वा वर्णौ सरूपकौ~।\\
अन्यतो भाविताङ्केन ततः पक्षौ विभज्य च~॥\\
वर्णाङ्काहतिरूपैक्यं भक्त्वेष्टेनेष्टतत्फले~।\\
एताभ्यां संयुतावूनौ कर्तव्यौ स्वेच्छया च तौ~।\\
वर्णाङ्कौ वर्णयोर्माने ज्ञातव्ये ते विपर्ययात्~॥~१८५~॥}}}
\end{quote}
\end{sloppypar}

\newpage

\begin{sloppypar}
स्पष्टोऽर्थः~। आचार्यैर्व्याख्यातश्च~। अत्रोपपत्तिराचायैर्लिखितास्ति~। किं तु लेखकादि-दोषादुपदेशविच्छित्त्या च सम्प्रति सा न स्वकार्यक्षमा~। अत इयं भावितोपपत्तिर्विविच्यो-च्यते~। तत्र \hyperref[11.182]{'चतुस्त्रिगुणयो राश्योः'} इति प्रथमोदाहरणे यथोक्ते समशोधने कृते जातौ पक्षौ \;{\small $\begin{matrix}
\mbox{{या ~४ ~का ~३ ~रू ~२~।}}\\
\vspace{-1mm}
\mbox{{या ~का ~भा ~१~। ~~~~}}
\vspace{1mm}
\end{matrix}$} अनयोः पक्षयोस्तुल्यत्वाद्यदेव यावत्तावत्कालकभावितस्य मानं तदेव यावत्तावच्चतुष्टयकालकत्रयरूपद्वययोगमानम्~। भावितं च समकर्णायतचतुर्भुजक्षेत्र-फलम्~। तत्र वर्णौ भुजकोटी~। दर्शनम्

\begin{center}
\includegraphics[scale=0.45]{Diagram_page98-1.png}
\end{center}

{\color{violet}'समश्रुतौ तुल्यचतुर्भुजे च तथायते तद्भुजकोटिघातः'} इति जातं क्षेत्रफलं याका भा १~। इदं क्षेत्रगतसमकोष्ठमानम्~। एतेन सममिदं या ४ का ३ रू २~। तथा च क्षेत्रान्तः यावत्तावच्चतुष्टयं कालकत्रयं रूपद्वयं चास्ति~। तत्र क्षेत्रमध्ये यावत्तावच्चतुष्टयस्य दर्शनम् इदम्

\begin{center}
\includegraphics[scale=0.45]{Diagram_page98-2.png}
\end{center}

अथ शेषक्षेत्रे सम्पूर्णः कालको दर्शयितुम् अशक्यः~। यतो दीर्घभुजोऽत्र कालकमानम्~। स च यावत्तावच्चतुष्टयापनयनेन रूपचतुष्टयोनो दृश्यते~। अतो रूपचतुष्टयोनं कालकत्रयं प्रदर्श्यते~।

\begin{center}
\includegraphics[scale=0.45]{Diagram_page98-3.png}
\end{center}
\end{sloppypar}

\newpage

\begin{sloppypar}
इह कालकेषु प्रत्येकं यावत्तावदङ्कतुल्यानि रूपाणि ४ न्यूनानि सन्तीति कालकत्रयस्य जातानि कालकाङ्कगुणितानि तानि न्यूनानि १२~। अथ यदि भावितक्षेत्रात्प्रथमतः कालक-त्रयमपनीयते तर्हि कालकाङ्कतुल्यरूपै\textendash \,३\textendash \,रूनं यावत्तावतो लघुभुजस्य मानं दृश्यते~। अतो रूपत्रयोनस्य यावत्तावतश्चतुष्टयं प्रदर्श्यते~।

\begin{center}
\includegraphics[scale=0.45]{Diagram_page99-1.png}
\end{center}

इह यावत्तावत्सु प्रत्येकं कालकाङ्क\textendash \,३\textendash \,तुल्यानि रूपाणि न्यूनानि सन्तीति यावत् तावच्चतुष्टयस्य जातानि चतुर्गुणितानि न्यूनानि १२~। उभयथापि वर्णाङ्काहतितुल्यै रूपैरूनं यावत्तावच्चतुष्टयं कालकत्रयं च क्षेत्रमध्ये प्रदर्शितं भवति~। अथ यदि सङ्कीर्णम् एव यावत् तावच्चतुष्टयं कालकत्रयं च प्रदर्श्यते तदैवं दर्शनं भवति~।

\begin{center}
\includegraphics[scale=0.45]{Diagram_page99-2.png}
\end{center}

इह ये कोणे कोष्ठका उत्पद्यन्ते सा वर्णाङ्काहतिरेव~। अथ वर्णाङ्काहतितुल्यास्ते कोण-कोष्ठका यदि कालकत्रयमध्ये गुण्यन्ते तदा यावत्तावच्चतुष्टयार्थं तावन्त एव कोष्ठका अपे-क्षिताः यदि तु यावत्तावच्चतुष्टयमध्ये गण्यन्ते तदा कालकत्रयार्थं तावन्त एव कोष्ठका अपे-क्षिताः~। उभयथापि क्षेत्रशेषखण्डे यदि वर्णाङ्काहतितुल्याः कोष्ठका गृह्यन्ते तदा सम्पूर्णं यावत् तावच्चतुष्टयं सम्पूर्णं कालकत्रयं च भवति~। भावितसमपक्षे च यावच्चतुष्टयं कालक-त्रयं रूपद्वयं च वर्तते~। अतः क्षेत्रशेषे वर्णाङ्काहत्या रूपद्वयेन च भाव्यम्~। कथमन्यथा द्वितीयपक्षो भावितसमः स्यात्~। तस्माद्भावितक्षेत्रान्तर्गते कोणस्थे लघुक्षेत्रे वर्णाङ्काहति-रूपैक्यतुल्याः कोष्ठकाः सन्तीति सिद्धम्~। ते च तस्य लघुक्षेत्रस्य फलम्~। तद्भुजयोर्वधा-ज्जातम्~। अत इष्टमेकभुजं प्रकल्प्य तेन क्षेत्रफले भक्ते यल्लभ्यते तद्द्वितीयो भुजः स्यात्~। अथाभ्यां भुजाभ्यां यावत्तावत्कालकयोर्माने ज्ञातुं न किञ्चित्काठिन्यमस्ति~। तथा हि यतो यावत्तावदङ्कतुल्यै रूपैरूनः कालकोऽस्य लघुक्षेत्रस्यैको भुजोऽस्त्यतोऽसौ यावत्तावदङ्कतुल्यै रूपैर्युतः सन्कालकमानं
\end{sloppypar}

\newpage

\begin{sloppypar}
\noindent स्यात्~। एवं कालकाङ्कतुल्यै रूपैरूनो यावत्तावद्वर्णो लघुक्षेत्रस्य द्वितीयो भुजोऽस्त्यतोऽसौ कालकाङ्कतुल्यै रूर्पैर्युतः सन्यावत्तावन्मानं स्यात्~। अत्रेष्टं यदि कालकखण्डात्मकस्य भुजस्य मानं कल्प्यते तदानेन क्षेत्रफले भक्ते यत्फलं तद्यावत्तावत्खण्डात्मकस्य द्वितीयभुजस्य मानं स्यात्~। अत इष्टं यावत्तावदङ्कयुतं कालकमानं स्यात्~। (फलं कालकाङ्कयुतं यावत् तावन्मानं स्यात्~।) यदि त्विष्टं यावत्खण्डात्मकस्य भुजस्य मानं कल्प्यते तदा फलं कालक-खण्डात्मकस्य भुजस्य मानं स्यात्~। अत इष्टं कालकाङ्कयुतं यावत्तावन्मानं स्यात्~। फलं यावत्तावदङ्कयुतं कालकमानं स्यादिति~। अत उपपन्नमिष्टफलाभ्यां स्वेच्छया संयुतौ वर्णाङ्कौ व्यत्ययाद्वर्णयोर्माने ज्ञातव्ये इति~।\\ 

अथवान्यथोपपत्तिः~। भावितक्षेत्रान्तर्गतक्षेत्रस्य भुजयोर्माने अन्यवर्णौ कल्पिते दर्शनम्

\begin{center}
\includegraphics[scale=0.45]{Diagram_page100.png}
\end{center}

इह नीलको यावत्तावदङ्कतुल्यै रूपैर्युतो जातं कालकमानं नी १ रू ४~। एवं पीतकाङ्कः कालकाङ्कतुल्यै रूपैर्युतो जातं यावत्तावन्मानं पी १ रू ३~। एवं क्रमेण जाते यावत्तावत्काल-कमाने पी १ रू ३~। नी १ रू ४~। आभ्यां पक्षयोरनयोः \;{\small $\begin{matrix}
\mbox{{या ~४ ~का ~३ ~रू ~२}}\\
\vspace{-1mm}
\mbox{{याकाभा ~१ ~~~~~~}}
\vspace{1mm}
\end{matrix}$}\, यावत्तावत्काल-कावुत्थाप्य जातमुपरिगपक्षे पी ४ रू १२ नी ३ रू १२ रू २~। द्वितीयपक्षे तु यावत्कालकयोः वधोऽस्तीति गुणनार्थं न्यासः \;{\small $\begin{matrix}
\mbox{{पी ~१~। नी ~१ ~रू ~४}}\\
\vspace{-1mm}
\mbox{{रू ~३~। नी ~१ ~रू ~४}}
\vspace{1mm}
\end{matrix}$}\, गुणनाज्जातो द्वितीयपक्षः पीनीभा १ पी ४ नी ३ रू १२~। एवं पक्षौ 
\vspace{-1mm}

\begin{center}
पी ~४ ~रू ~१२ ~नी ~३ ~रू ~१२ ~रू ~२ \\
पीनीभा ~१ ~~पी ~४ ~~नी ~३ ~~रू ~१२ 
\end{center}
\vspace{-1mm}

\noindent अथ नीलकयोः पीतकयोश्च तुल्यत्वात्समशोधनेन नाशे जातौ पक्षौ \;{\small $\begin{matrix}
\mbox{{रू ~१२ ~रू ~१२ ~रू ~२}}\\
\vspace{-1mm}
\mbox{{नीपीभा ~~१ ~~~~रू ~१२}}
\vspace{1mm}
\end{matrix}$}

\end{sloppypar}

\newpage

\begin{sloppypar}
अथोभयपक्षयोर्वर्णाङ्काहतितुल्यरूपाणां समशोधनेन नाशे जातौ रू १२ रू २~।~अत्रो-र्ध्वपक्षे वर्णाङ्काहतितुल्यानि रूपाणि सन्ति यथास्थितरूपाणि नीपीभा १ च सन्ति~।~अतो वर्णाङ्काहतिरूपैक्यमुपरिगपक्षे रू १४~। अधःपक्षे तु नीपीभा १~। पक्षयोः समत्वात् यदेव नीलकपीतकभावितं तदेव वर्णाङ्काहतिरूपैक्यमिति सिद्धम्~। अतो नीलकपीतकयोरेक-तरस्येष्टं मानं प्रकल्प्य तेन वर्णाङ्काहतिरूपैक्ये भक्ते यल्लभ्यते तद्द्वितीयस्य मानं स्यात्~। एवं सिद्धमिष्टतत्फले अन्तःक्षेत्रभुजयोर्माने इति~।\\

अथ यावत्कालकमानयोः पीतकनीलकौ स्वस्वमानेनोत्थाप्य वा प्राग्वद्वेष्टतत्फलाभ्यां स्वेच्छया संयुतौ वर्णाङ्कौ व्यत्ययाद्वर्णयोर्माने भवत इत्युपपद्यते~। तदेवं भावितसमे द्विती-यपक्षे वर्णाङ्कयो रूपाणां च धनत्वे प्रतिपादितम्~। यत्र तु वर्णाङ्कावृणं रूपाणि तु धनं तत्रान्यथा संस्था भवति~। तथा हि कल्पितौ पक्षौ \;{\small $\begin{matrix}
\mbox{{या ~४ं ~का ~३ं ~रू ~३०}}\\
\vspace{-1mm}
\mbox{{याकाभा ~१ ~~~~~~}}
\vspace{1mm}
\end{matrix}$}~। अत्र पक्षयोः यावच्चतुष्टये कालकत्रये च क्षिप्ते जातौ \;{\small $\begin{matrix}
\mbox{{या ~० ~का ~० ~रू ~३० ~~}}\\
\vspace{-1mm}
\mbox{{या का भा ~१ ~या ~४ ~का ~३}}
\vspace{1mm}
\end{matrix}$}~। अत्र स्वाङ्कगुणाभ्यां वर्णाभ्यां युक्तस्य भावितस्य यन्मानं तदेव रूपाणामपीति सिद्धम्~। तस्य दर्शनम् 

\begin{center}
\includegraphics[scale=0.45]{Diagram_page101-1.png}
\end{center}

एतद्द्वितीयपक्षस्य रूपात्मकस्य मानम्~। अत्र रिक्तकोणे वर्णाङ्काहतितुल्याः कोष्ठका यदि क्षिप्यन्ते तदैवं भवति~।

\begin{center}
\includegraphics[scale=0.45]{Diagram_page101-2.png}
\end{center}

अस्य महतः क्षेत्रस्य वर्णाङ्काहतिरूपैक्यफलम् अस्ति~। पूर्वं यस्य क्षेत्रस्य वर्णाङ्काहति-रूपैक्यं फलं तत्क्षेत्रं भावितक्षेत्रान्तर्गतं कोणस्थमासीत्~। इदानीं तु भावितक्षेत्रमेव तदन्तर्गतं कोणस्थं भवतीति विशेषः~। महतः क्षेत्रस्यैकं भुजमिष्टं प्रकल्प्यानेन क्षेत्रफले
\end{sloppypar}

\newpage

\begin{sloppypar}
\noindent भक्ते प्राग्वद्द्वितीयभुजमानं भवेत्~। इहेष्टं तथा कल्पनीयं यथा स्वयमेकतरवर्णाङ्कादधिकं भवेत्तत्फलं चान्यवर्णाङ्कादधिकं भवेत्~। अथाभ्यां भुजाभ्यां वर्णमानं साध्यम्~। तद्यथा\textendash \,इह कालकाङ्कयुतो यावत्तावद्वर्ण एको भुजोऽस्ति~। अतोऽसौ कालकाङ्केनोनो यावत्तावन्मानं स्यात्~। एवं यावत्तावदङ्कयुतः कालकोऽस्य क्षेत्रस्य द्वितीयभुजोऽस्ति~। अतोऽसौ यावत् तावदङ्कोनः कालकमानं स्यात्~। अत्र भुजौ त्विष्टतत्फले~। अत इष्टतत्फले वर्णाङ्कोने~व्यत्य-यान्माने भवत इति यद्यपि वक्तुमुचितं तथापि प्रकृते वर्णाङ्कावृणगताविति तद्योग एव कृते सतीष्टतत्फले वर्णाङ्कोने भवत इति तथा नोक्तम्~।\\

अथ यत्र वर्णाङ्कौ धनं रूपाणि त्वृणं तत्र द्वैविध्यमस्ति~। अन्योन्यभुजतो न्यूनौ वर्णा-ङ्कावित्येकः प्रकारः~। अन्योन्यभुजतोऽधिकौ वर्णाङ्काविति द्वितीयः~। तत्र प्रथमे  प्रागुक्तयुक्त्या भावितक्षेत्रान्तर्गतलघुक्षेत्रे वर्णाङ्काहत्या रूपोनया भाव्यम्~। सा च वर्णाङ्काहती रूपयुता सती रूपोना भवति~। रूपाणामृणत्वात्~। अतोऽत्रापि वर्णाङ्काहतिरूपैक्यमेव भावितक्षेत्रा-न्तर्गतक्षेत्रस्य फलं भवति~। अतः प्रथमप्रकारे प्राग्वदेवोपपद्यते~। द्वितीयप्रकारे त्वन्यभुज-मानाद्वर्णाङ्कोऽधिकोऽस्तीति स्वाङ्कगुणवर्णस्य मानं भावितक्षेत्रमतिक्रम्य बहिरपि भवति~। यतो भावितक्षेत्रे कालकमानतुल्या एव यावद्वर्णाः सम्भवन्ति नाधिकाः~। एवं यावत्तावत् मानतुल्या एव कालकाः सम्भवन्ति नाधिकाः~। अथ तत्र स्वाङ्कगुणवर्णयोर्दर्शनम्~।\\

\begin{center}
\includegraphics[scale=0.5]{Diagram_page102.png}
\end{center}

अत्र भावितक्षेत्रं यदि स्वाङ्कगुणयावत्तावन्मध्ये गण्यते तर्हि स्वाङ्कगुणकालकमानार्थम् अन्यद्भावितक्षेत्रमपेक्षितम्~। यदि तु स्वाङ्कगुणकालकमानमध्ये गण्यते तर्हि स्वाङ्कगुणया-वत्तावन्मानार्थमन्यद्भावितक्षेत्रमपेक्षितम्~। उभयथापि भावितक्षेत्रलिखितक्षेत्रयोर्योगे~स्वाङ्क-गुणवर्णौ भवतः~। अतो रूपैर्लिखितक्षेत्रसमैर्भाव्यम्~। कथमन्यथा स्वाङ्कगुणवर्णौ रूपैरूनौ भावितसमौ भवतः~। अथ लिखितं रूपात्मकं क्षेत्रं रिक्तकोणे यदि पूर्यते तदैवं भवति~।
\end{sloppypar}

\newpage

\begin{sloppypar}
\begin{center}
\includegraphics[scale=0.5]{Diagram_page103.png}
\end{center}

अत्र वर्णाङ्काहतिः क्षेत्रफलम् अस्ति~। पूर्वलिखितक्षेत्रे तु रूपाण्येव~। अतो वर्णाङ्काहती रूपैरूना सती भावितक्षेत्रबहिःकोणस्थस्य लघुक्षेत्रस्य फलं भवति~। तच्च वर्णाङ्काहतिरूपै-क्यकरणादेव सम्पद्यते~। यतोऽत्र रूपाणामृणत्वाद्वर्णाङ्काहतेश्च धनत्वात् 'धनर्णयोरन्तरम् एव योगः' इति योगे कृते रूपैरूनैव वर्णाङ्काहतिर्भवति~। अथ लघुक्षेत्रस्यैकं भुजम् इष्टं प्रकल्प्यानेन क्षेत्रफले भक्ते द्वितीयभुजमानं स्यात्~। अथाभ्यां भुजाभ्यां वर्णमाने साध्ये~। ते यथा\textendash \;इह यावत्तावन्मानोनः कालकाङ्कोऽस्य लघुक्षेत्रस्यैको भुजोऽस्ति~। अतोऽनेन कालकाङ्क ऊनः सन्यावत्तावन्मानं भवेत्~। एवं कालकमानेनोनो यावत्तावदङ्कोऽस्य लघु-क्षेत्रस्य द्वितीयभुजोऽस्ति~। अतोऽनेन यावत्तावदङ्क ऊनः सन्कालकमानं भवेत्~। भुजौ त्विष्टतत्फले~। तत्फलेष्टे वा~। अत इष्टतत्फलाभ्यां स्वेच्छयोनौ वर्णाङ्कौ व्यत्ययान्माने भवत इत्युपपन्नम्~। तदेवमयं निष्कृष्टोऽर्थः~। यदि भावितसमे पक्षे रूपाणि धनं स्युस्तदे-ष्टतत्फलाभ्यां वर्णाङ्को धनमृणं वा यथावत्संयुक्तावेव व्यत्ययान्माने भवतः~। यदि तु रूपाण्यृणं स्युस्तदेष्टतत्फलाभ्यां स्वेच्छया संयुतावूनौ च वर्णाङ्कौ व्यत्ययान्माने भवतः~। अस्मिन्पक्षे वर्णाङ्कयोर्धनत्वम् एव~। न हि त्रयाणामृणत्वे वर्णमानं धनं सम्भवति~। ऋणे वा वर्णमाने लोकानां प्रतीतिरस्ति~। अत्रापरो विशेषः~। यत्र संयुक्तवर्णाङ्कने ऊनवर्णाङ्कने च माने उपपन्ने भवतस्तत्रोभे अपि ग्राह्ये~। अन्यत्र तु ये उपपन्ने ते एव ग्राह्ये इति~। इति भावितोपपत्तिः~। अत्र त्रयाणामपि धनत्वे चतुस्त्रिगुणयो राश्योरित्युदाहरणं प्रदर्शितम्~॥~१८५~॥\\

{\small अथ यत्र वर्णाङ्कौ धनं रूपाणि त्वृणं स्युस्तादृशमुदाहरणमनुष्टुभाह\textendash }

\phantomsection \label{11.186}
\begin{quote}
{\large \textbf{{\color{purple}द्विगुणेन कयो राश्योर्घातेन सदृशं भवेत्~।\\
दशेन्द्राहतराश्यैक्यं द्व्यूनषष्टिविवर्जितम्~॥~१८६~॥}}}
\end{quote}

स्पष्टोऽर्थः~। गणितमाकरे स्पष्टम्~॥~१८६~॥ \\

{\small अथ यत्र वर्णाङ्कावृणं रूपाणि तु धनं स्युस्तादृशमुदाहरणमनुष्टुभाह\textendash }

\phantomsection \label{11.187}
\begin{quote}
{\large \textbf{{\color{purple}त्रिपञ्चगुणराशिभ्यां युक्तो राश्योर्वधः कयोः~।\\
द्विषष्टिप्रमितो जातो राशी त्वं वेत्सि चेद्वद~॥~१८७~॥}}}
\end{quote}

\end{sloppypar}

\newpage

\begin{sloppypar}
स्पष्टोऽर्थः~। गणितमाकरे स्पष्टम्~। \\

अथ यत्र रूपाणामृणत्वे प्रकारद्वयेनोत्पन्नमानयोरेकतरे एवोपपन्ने भवतस्तादृशम् उदाहरणं पूर्वचतुर्थमस्तीति तदेव प्रदर्शयति\textendash \,\hyperref[11.184]{'यौ राशी किल या च राशिनिहतिः'} इत्यादि~। गणितं स्पष्टमाकरे~॥~१८७~॥ 

\begin{quote}
{\color{violet}दैवज्ञवर्यगणसन्ततसेव्यपार्श्वबल्लाळसञ्ज्ञगणकात्मजनिर्मितेऽस्मिन्~। \\
बीजक्रियाविवृतिकल्पलतावतारेऽभूद्भावितं सकलमेतदनेकवर्णे~॥ }
\end{quote}

इति श्रीसकलगणकसार्वभौमश्रीबल्लाळदैवज्ञसुतकृष्णगणकविरचिते बीजक्रियावि-वृतिकल्पलतावतारेऽनेकवर्णे भावितविवरणम्~। अत्र ग्रन्थसङ्ख्या १४०~। एवमादितो ग्रन्थ-सङ्ख्या ४४५८ इत्यनेकवर्णसमीकरणविवरणं समाप्तम्~॥~११~॥

\begin{center}
\rule{0.2\linewidth}{0.8pt}\\
\vspace{-4mm}

\rule{0.2\linewidth}{0.8pt}
\end{center}
\end{sloppypar}

\newpage
\thispagestyle{empty}

\begin{center}
\textbf{\large ग्रन्थसमाप्तिः~।}\\
\rule{0.2\linewidth}{0.8pt}
\end{center}

\begin{sloppypar}
{\small अथास्य ग्रन्थस्य प्रचारार्थं गुरूत्कर्षकथनरूपं मङ्गलमाचरन्ग्रन्थसमाप्तिं वसन्ततिलकयाह\textendash }

\phantomsection \label{12.1}
\begin{quote}
{\large \textbf{{\color{purple}आसीन्महेश्वर इति प्रथितः पृथिव्याम् \\
आचार्यवर्यपदवीं विदुषां प्रयातः~।\\
लब्ध्वावबोधकलिकां तत एव चक्रे\\
तज्जेन बीजगणितं लघु भास्करेण~॥~१~॥}}}
\end{quote}

लघ्विति च्छेदः स्पष्टोऽर्थः~॥~१~॥ 
\vspace{2mm}

{\small ननु बीजगणितानि ब्रह्मगुप्तादिभिः प्रतिपादितानि सन्ति तत्किमर्थमाचार्यैर्यतितमिति शङ्का-यामिन्द्रवज्रयोत्तरमाह\textendash }

\phantomsection \label{12.2}
\begin{quote}
{\large \textbf{{\color{purple}ब्रह्माह्वयश्रीधरपद्मनाभबीजानि यस्मादतिविस्तृतानि~।\\
आदाय तत्सारमकारि नूनं सद्युक्तियुक्तं लघु शिष्यतुष्ट्यै~॥~२~॥}}}
\end{quote}

अत्रापि लघ्विति च्छेदः~। शेषं स्पष्टम्~॥~२~॥
\vspace{2mm}

{\small कथमिदं लघ्वित्याशङ्कायामाहानुष्टुप्पूर्वार्धेन\textendash }

\phantomsection \label{12.3}
\begin{quote}
{\large \textbf{{\color{purple}अत्रानुष्टुप्सहस्रं हि ससूत्रोद्देशके मितिः~॥~३~॥}}}
\end{quote}

हि यतोऽत्र ससूत्रोद्देशके बीजगणितेऽनुष्टुप्सहस्रमितिः पूर्वबीजगणितेषु तु सहस्र-द्वयत्रयादिमितिरस्ति अतो लघ्विदमित्यर्थः~। 
\vspace{2mm}

{\small नन्विदमपि विस्तृतमस्ति~। क्वचित्क्वचिदेकस्मिन्नेवं विषय उदाहरणबाहुल्योक्तेरिति शङ्कायाम् अनुष्टुभोत्तरपूर्वार्धाभ्यामाह\textendash }

\phantomsection \label{12.4}
\begin{quote}
{\large \textbf{{\color{purple}क्वचित्सूत्रार्थविषयं व्याप्तिं दर्शयितुं क्वचित्~।\\
क्कचिच्च कल्पनाभेदं क्कचिद्युक्तिमुदाहृतम्~॥\\
क्वचित्सूत्रार्थविषयं दर्शयितुमुदाहृतम्~॥~४~॥}}}
\end{quote}

यथा भाविते \hyperref[11.182]{'चतुस्त्रिगुणयो राश्योः'} इति \hyperref[11.186]{'द्विगुणेन कयो राश्योः'} इति \hyperref[11.187]{'त्रिपञ्च-गुणराशिभ्याम्'} इति \hyperref[11.184]{'यौ राशी किल या च राशिनिहतिः'} इत्युदाहरणचतुष्टयमुदाहृतम्~। न ह्येकस्मिन्नुदाहृते \hyperref[11.185]{'भावितं पक्षतोऽभीष्टात्'} इति सूत्रस्यार्थः सर्वोऽपि विषयी भवति~। तस्मादशेषं सूत्रार्थं दर्शयितुम् उदाहणचतुष्टयम् अप्यावश्यकम्~। एवं क्कचिद्व्याप्तिं दर्शयितुम् उदाहृतम्~। यथा\textendash \,\hyperref[7.96]{'पञ्चकशतदत्तधनात्'} इत्युदाहृत्य \hyperref[7.97]{'एकशत[दत्त]धनतः'} इति तादृशमेव पुनरुदाहृतम्~। इदं यदि नोदाह्रियते तर्हि स्वकृते प्रकारविशेषे मन्दानां विश्वासो न भवेत् इत्येतदावश्यकम्~। एवं कल्पनाभेदं दर्शयितुं \hyperref[9.136]{'एको ब्रवीति मम देहि'} इत्युदाहरणमेकवर्ण-समीकरण उदाहृतम्~। एवं विविधयुक्तिप्रदर्शनार्थमपि बहुषु स्थलेषूदाहृतमस्ति~। तस्मादयं विस्तारो न दोषाय~॥~४~॥
\end{sloppypar}

\newpage

\begin{sloppypar}
{\small ननु पूर्वबीजेषूदाहरणानि बहूनि सन्तीह तु स्वल्पान्येवोक्तानीति न सकलोदाहरणावगमः स्यादत आह\textendash }

\phantomsection \label{12.5}
\begin{quote}
{\large \textbf{{\color{purple}न ह्युदाहरणान्तोऽस्ति स्तोकमुक्तमिदं यतः~॥~५~॥}}}
\end{quote}

हि यत उदाहरणान्तो नास्ति~। अत इदं स्तोकं स्वल्पमुक्तम्~। पूर्वबीजेष्वपि सकलानि उदाहरणानि नैवोक्तानि~। तेषामनन्तत्वेन वक्तुमशक्यत्वात्~। अतोऽल्पैरप्युदाहरणैर्विविध-युक्तिषु प्रदर्शितासु शेषं व्यर्थमिति भावः~॥~५~॥\\

{\small नन्वत्र स्वल्पमुक्तं पूर्वबीजानि त्वतिविस्तृतान्यतस्तान्येव मन्दप्रयोजनायालमित्याशङ्कायामाह~। यतः\textendash }

\phantomsection \label{12.6}
\begin{quote}
{\large \textbf{{\color{purple}दुस्तरः स्तोकबुद्धीनां शास्त्रविस्तारवारिधिः~।\\
अथवा शास्त्रविस्तृत्या किं कार्यं सुधियामपि~॥~६~॥}}}
\end{quote}

यो हि विस्तारः स मन्दार्थं सुध्यर्थं वा~। नाद्यः~। यतः शास्त्रविस्तारवारिधिः स्तोक-बुद्धीनां मन्दानां दुस्तरः~। दुर्बोध इति यावत्~। यतो महति ग्रन्थे प्रत्युत किं कुत्रास्ति किमत्र कर्तव्यमित्यनवबोधेनेतिकर्तव्यतामूढा एव ते स्युः~। नान्त्यः~। सुधियामपि शास्त्र-विस्तृत्या किं कार्यम्~। यतस्ते कल्पनासमर्थाः~। ननु लघ्वपि बीजं मन्दार्थं सुध्यर्थं वा~। नाद्यः~। तैर्ज्ञातुमशक्यत्वात्~। नान्त्यः~। तेषां कल्पकत्वादिति चेन्न~। स्वल्पग्रन्थस्य मन्दा-नामभ्याससाध्यत्वान्न तावत्प्रथमपक्षे दोषः~॥~६~॥\\

{\small द्वितीयेऽपि न दूषणमित्याह\textendash }

\phantomsection \label{12.7}
\begin{quote}
{\large \textbf{{\color{purple}उपदेशलवं शास्त्रं कुरुते धीमतो यतः~।\\
तत्तु प्राप्यैव विस्तारं स्वयमेवोपगच्छति~॥~७~॥}}}
\end{quote}

यतः शास्त्रं धीमत उपदेशलवं कुरुते~। तत्तु शास्त्रं सुधियं प्राप्य स्वयमेव विस्ता-रमुपगच्छति~। न हि सुधियोऽपि किञ्चिदप्यनधीत्य जानन्ति~। अत इदं मदुक्तं सुधीमन्द-साधारणप्रयोजनायेति सवैरपि पठनीयमिति भावः~॥~७~॥\\ 

{\small ननु शास्त्रं सुधियं प्राप्य स्वयमेव विस्तारमुपगच्छतीति कथमित्याशङ्कायां सदृष्टान्तमाह\textendash }

\phantomsection \label{12.8}
\begin{quote}
{\large \textbf{{\color{purple}जले तैलं खले गुह्यं पात्रे दानं मनागपि~।\\
प्राज्ञे शास्त्रं स्वयं याति विस्तारं वस्तुशक्तितः~॥~८~॥}}}
\end{quote}

स्पष्टोऽर्थः~॥~८~॥
\end{sloppypar}

\newpage

\begin{sloppypar}
{\small एवं स्वकृतस्यास्य बीजस्य गुणान्युक्क्त्या संस्थाप्योपसंहरति\textendash }

\phantomsection \label{12.9}
\begin{quote}
{\large \textbf{{\color{purple}गणक भणतिरम्यं बाललीलावगम्यं~।\\ 
सकलगणितसारं सोपपत्तिप्रकारम्~।\\
इति बहुगुणयुक्तं सर्वदोषैर्विमुक्तं~। \\
पठ पठ मतिवृद्ध्यै लघ्विदं प्रौढसिद्ध्यै~॥~९~॥ }}}
\end{quote}

गणकेति सम्बोधनम्~। भणतयः शब्दास्तै रम्यम्~। पदलालित्ययुक्तमित्यर्थः~। शेषं स्पष्टम्~॥~९~॥

\begin{quote}
{\small {\color{violet}अभूत्पृथिव्यां प्रथितो गुणौघैश्चिन्तामणिर्दैवविदां वरिष्ठः~। \\
सम्पूजनानेहसि यस्य गौरी स्मृता स्तुता प्रत्यहमाविरासीत्~॥~१~॥}
\vspace{1mm}

{\color{violet}तत्सूनवः पञ्च बभूवुरेषां ज्येष्ठोऽभिरामः किल रामनामा~। \\
भविष्यदर्थज्ञतया हि यस्य विदर्भराजोऽपि निदेशवर्ती~॥~२~॥}
\vspace{1mm}

{\color{violet}रामादभूतां सीतायां पुत्रौ कुशलवाविव~। \\
त्रिमल्लो गोपिराजश्च गुणैः सर्वैः समन्वितौ~॥~३~॥ }
\vspace{1mm}

{\color{violet}त्रिमल्लसूनुर्जयति द्विजेन्द्रो बल्लाळसञ्ज्ञः शितिकण्ठभक्तः~। \\
यः सन्ततं रुद्रजपातिसङ्गाद्ब्राह्मं महो मूर्तमिवावभाति~॥~४~॥ }
\vspace{1mm}

{\color{violet}दैवज्ञवर्यगणसन्ततसेव्यपार्श्वबल्लालसञ्ज्ञगणकस्य सुतोऽस्ति कृष्णः~। \\
रामानुजः स परमेश्वरतुष्टिहेतोर्बीजक्रियाविवृतिकल्पलतामकार्षीत्~॥~५~॥ }
\vspace{1mm}

{\color{violet}यद्भास्करेण निजधामगुणातिरेकात् सम्पादितं सगुणवर्गघनं हि बीजम्~। \\
तत्कृष्णभूमिमधिगम्य विचारवारिसंसिक्तमङ्कुरजनुष्यभवत्समर्थम्~॥~६~॥ }
\vspace{1mm}

{\color{violet}यैर्यैः श्रमैर्विरचितोऽस्ति नवाङ्कुरोऽसौ \\
तेषामभिज्ञ इह कः परमात्मनोऽन्यः~। \\
इत्थं विचिन्त्य जगदीश तवैव तुष्ट्यै \\
सर्वज्ञ ते चरणयोर्निहितस्ततोऽयम्~॥~७~॥ }}
\end{quote}

इति श्रीवक्रतुण्डार्पणमस्तु~। ग्रन्थसङ्ख्या ४५००~।

\begin{center}
\rule{0.2\linewidth}{0.8pt}\\
\vspace{-4mm}

\rule{0.2\linewidth}{0.8pt}
\end{center}
\end{sloppypar}

\newpage
\thispagestyle{empty}

\begin{center}
\textbf{\large हस्तलिखितप्रतीनां समाप्तिः~।}\\
\rule{0.3\linewidth}{0.8pt}
\end{center}

\begin{sloppypar}
क. इति सञ्ज्ञित आनन्दाश्रमग्रन्थसङ्ग्रहालयस्थे ग्रन्थसमाप्तिरेवं विद्यते\textendash

\begin{quote}
कश्चित्किञ्चित्स्वं गृहीत्वा प्रयागाद्यातः काशीं तत्र तत्पञ्चनिघ्नम्~।\\
कृत्वा पञ्चाशल्लवं विंशदंशनिघ्नं तस्य ब्राह्मणेभ्यः प्रदत्वा~॥~८~॥\\
शम्भोः पूजां चैकमूलेन कृत्वा राशेर्धृत्यंशेन पञ्चाहतेन~।\\
कौशेयादीन् सङ्गृहीत्वा दशस्वो जातस्तत्स्वं ब्रूहि बीजज्ञ तूर्णम्~॥
\end{quote}

नगगजभूपैः १६८७ प्रमिते शाके लिलेख यादवोऽभिज्ञः~। 
\vspace{2mm}

मल्लारिजः शिवपूर्याम्~॥ 

\begin{center}
श्रीसाम्बसदाशिवार्पणमस्तु~। 
\end{center}
\vspace{-2mm}

\begin{quote}
यादृशं पुस्तकं दृष्टं तादृशं लिखितं मया~। \\
यदि शुद्धमशुद्धं वा मम दोषो न विद्यते~॥ 
\end{quote}

स्वार्थं परार्थं च शके १७६७ विश्वावसुनामसंवत्सरे पौषशुद्धद्वितीयायां भौमवासरे पुस्तकं समाप्तम्~। 

\begin{center}
श्रीसीतारामचन्द्रार्पणमस्तु~। \\
\vspace{1mm}

श्रीगजाननः प्रसन्नः~। 
\end{center}

ख. इति सञ्ज्ञित आनन्दाश्रमग्रन्थसङ्ग्रहालयस्थे ग्रन्थसमाप्तिरेवं विद्यते\textendash 

\begin{quote}
श्रीकृष्ण राम मधुसूदन दानवारे शौरे त्रिविक्रम गदाधर पद्मनाभ~। \\
श्रीवत्स भक्तजनपालक विश्ववन्द्य लक्ष्मीपतेऽस्तु मम ते सततं प्रणामः~॥~१~॥ \\
शम्भो शशाङ्कधर भस्मविभूषिताङ्ग नन्द्यादिवन्दितपदद्वयभूतनाथ~। \\
मूर्ध्ना धृतत्रिपथगार्द्रजटासमूह गौरीपतेऽस्तु सततं मम ते प्रणामः~॥~२~॥ \\
हेरम्ब शङ्कर तनूद्भव वारणास्य बालार्कदीधितिसदृक्षशरीरकान्ते~। \\
लम्बोदरैकरद विघ्ननिवारणैकहेतो मम प्रतिदिनं नतिराविरस्तु~॥~३~॥ \\
आदित्य भास्कर दिवाकर लोकबन्धो सप्ताश्व विश्वनयनान्ध्यहरारुणेश~। \\
सिन्दूरधूसरकरप्रकरप्रभैकराशेऽस्तु ते मम मतिप्रकरः सदैव~॥~४~॥ \\
क्षीरोद्भवे कमलवासिनि विश्ववन्द्ये पद्मे रमे कमलशोभितहस्तपद्मे~। \\
ब्रह्माच्युतेशहरसूनुदिवाकरादिवन्द्येऽस्तु ते मम प्रणामततिः सदैव~॥~५~॥ 
\end{quote}

शके १८१२ विकृतिनामसंवत्सरे मार्गशीर्ष शुक्ल\textendash \,१४\textendash \,गुरौ हर्डीकरोपनामकविनायकेन लिखितमिदम्~। 

\begin{center}
ग्रन्थसङ्ख्या ४५००
\end{center}
\end{sloppypar}

\newpage

\begin{sloppypar}
ग. इति सञ्ज्ञिते भाण्डारकरप्राच्यविद्यासंशोधनमन्दिरस्थे ग्रन्थसमाप्तिरेवं विद्यते\textendash 
\vspace{2mm}

शके १७४७ पार्थिवनामाब्द उत्तरायणे शशिऋतौ (शिशिरर्तौ) माघमासे शुक्लपक्षे १ भौमवासरे धनिष्ठानक्षत्रे वर्यान्योग एतच्छुभदिन इदं पुस्तकं समाप्तम्~। श्रीजगदम्बार्पणमस्तु शुभं भवतु~॥ श्रीरस्तु~॥ श्रीराम~॥ श्रीकृष्ण श्रीहरि~॥ \\

घ. इति सञ्ज्ञिते ग्रन्थसमाप्तिर्न विद्यते त्रुटितत्वात्~। \\

ङ. इति सञ्ज्ञिते कोल्ब्रूक् इत्यभिधेनाङ्ग्लेन लन्दनस्थग्रन्थसङ्ग्रहालयायार्पिते ग्रन्थसमा-प्तिरेवं विद्यते\textendash 
\vspace{2mm}

'इति श्रीभास्कराचार्यविरचिते सिद्धान्तशिरोमणौ बीजगणिताध्यायः समाप्तः~॥
\vspace{2mm}

ज्यारसभूधरचन्द्रे वर्षे रसद्विनृपे च शाके~॥ संवत् १७६१ वरषे अगहन सूदि नवमी ९ शूक्रवासरे~॥ प्रयागमध्य अनूपसीधलेखक इदं पूस्तकं लीख्यते~॥~छ~॥ छश्लोक १२२४~। \\

च. इति सञ्ज्ञिते काशीस्थकाॅलेजस्थेऽपि कसञ्ज्ञितस्थं 'कश्चित्किञ्चित्' इदं पद्यं विद्यते~। 

\begin{center}
\rule{0.2\linewidth}{0.8pt}\\
\vspace{-4mm}

\rule{0.2\linewidth}{0.8pt}
\end{center}

\end{sloppypar}

\end{document}
